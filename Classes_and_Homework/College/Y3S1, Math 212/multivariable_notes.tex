\documentclass[10pt]{extarticle}
\title{}
\author{Avinash Iyer}
\date{}
\usepackage[shortlabels]{enumitem}

%font setup
%
\usepackage{newpxtext,eulerpx}

%paper setup
\usepackage{geometry}
\geometry{letterpaper, portrait, margin=1in}
\usepackage{fancyhdr}

%symbols
\usepackage{amsmath}
\usepackage{mathtools}
\usepackage{hyperref}
\usepackage{gensymb}

\usepackage[T1]{fontenc}
\usepackage[utf8]{inputenc}

%chemistry stuff
\usepackage[version=4]{mhchem}
\usepackage{chemfig}

%plotting
\usepackage{pgfplots}
\usepackage{tikz}
\tikzset{middleweight/.style={pos = 0.5, fill=white}}
\tikzset{weight/.style={pos = 0.5, fill = white}}
\tikzset{lateweight/.style={pos = 0.75, fill = white}}
\tikzset{earlyweight/.style={pos = 0.25, fill=white}}

%\usepackage{natbib}

%graphics stuff
\usepackage{graphicx}
\graphicspath{ {./images/} }

%code stuff
%when using minted, make sure to add the -shell-escape flag
%you can use lstlisting if you don't want to use minted
%\usepackage{minted}
%\usemintedstyle{pastie}
%\newminted[javacode]{java}{frame=lines,framesep=2mm,linenos=true,fontsize=\footnotesize,tabsize=3,autogobble,}
%\newminted[cppcode]{cpp}{frame=lines,framesep=2mm,linenos=true,fontsize=\footnotesize,tabsize=3,autogobble,}

\usepackage{listings}
\usepackage{color}
\definecolor{dkgreen}{rgb}{0,0.6,0}
\definecolor{gray}{rgb}{0.5,0.5,0.5}
\definecolor{mauve}{rgb}{0.58,0,0.82}

\lstset{frame=tb,
	language=Java,
	aboveskip=3mm,
	belowskip=3mm,
	showstringspaces=false,
	columns=flexible,
	basicstyle={\small\ttfamily},
	numbers=none,
	numberstyle=\tiny\color{gray},
	keywordstyle=\color{blue},
	commentstyle=\color{dkgreen},
	stringstyle=\color{mauve},
	breaklines=true,
	breakatwhitespace=true,
	tabsize=3
}
% text + color boxes
\usepackage{tcolorbox}
\tcbuselibrary{breakable}
\newtcolorbox{problem}[1]{colback = white, title = {#1}, breakable}
\newtcolorbox{solution}{colback = white, colframe = black!75!white, title = Solution, breakable}
%including PDFs
\usepackage{pdfpages}
\setlength{\parindent}{0pt}

\pagestyle{fancy}
\fancyhf{}
\rhead{Avinash Iyer}
\lhead{Math 212: Class Notes}
\begin{document}{
  \begin{problem}{The basis of Multivariable Calculus}
    If a function is continuous and differentiable, on a small enough interval, the function will approximate a line (i.e., a function of $x$).\\

    A similar intuition applies to functions of more than one variable (but with a plane, cube, hypercube, etc.). However, in multivariable functions, we will have to sacrifice the ability to visualize it.\\

    For example, in multiple dimensions, it is possible for there to be a function that is both strictly decreasing (in one dimension) and strictly increasing (in another dimension).
  \end{problem}
  \begin{problem}{Some Functions and Sets}
    \[
      f(x,y) = x^2-y^2
    \] 
    \begin{description}[font=\normalfont\scshape]
      \item[Domain:]  $\{(x,y)\mid \exists f(x,y)\}$
      \item[Range:] $\{f(x,y) \mid (x,y)\in \textrm{Dom}(f)\} = \mathbb{R}$
      \item[Graph:] $\textrm{Graph}(f) = \{x,y.f(x,y) \mid x,y\in \textrm{Dom}(f)\}$. For example, $(1,3,4)\notin \textrm{Graph}(f)$ since $1^2-3^2 \neq 4$.
    \end{description}
  \end{problem}
  \begin{problem}{Examples}
    In $\mathbb{R}^3$, in $x,y,z$ coordinates, $z=3$ is a plane defined as follows:
    \begin{itemize}
      \item Parallel to the $xy$ plane.
      \item Passes through the point $(0.0,3)$.
    \end{itemize}
    \begin{center}
      \begin{tikzpicture}
        \begin{axis}[grid=major]
          \addplot3[mesh,domain=-2:2,y domain=-2:2]{3};
          \addlegendentry{$z = 3$}
        \end{axis}
      \end{tikzpicture}
    \end{center}
    Meanwhile, $y=0$ would be a ``wall'' that passes through the origin that contains the line $y=0$ in the $xy$ plane.\\

    Finally, $z = x+y+1$ is a plane, as we can see below.
    \begin{center}
      \begin{tikzpicture}
        \begin{axis}[grid=major]
          \addplot3[mesh,domain=-2:2,y domain=-2:2]{x+y+1};
          \addlegendentry{$z = x+y+1$}
        \end{axis}
      \end{tikzpicture}
    \end{center}
  \end{problem}
  \begin{problem}{Visualizing a function of multiple variables}
    Consider the function $f(x,y) = x^2-y^2$. We can try visualizing slices as follows:
    \begin{itemize}
      \item $f(-2,y) = 4-y^2$
      \item $f(0,y) = -y^2$
      \item $f(2,y) = 4-y^2$
      \item $f(x,-2) = x^2+4$
      \item $f(x,0) = x^2$
      \item $f(x,2) = x^2+4$
    \end{itemize}
    Alternatively, we can visualize via contour diagrams.
  \end{problem}
}\end{document}
