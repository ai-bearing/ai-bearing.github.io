\documentclass[8pt]{extarticle}
\title{}
\author{Avinash Iyer}
\date{}
\usepackage[shortlabels]{enumitem}


%paper setup
\usepackage{geometry}
\geometry{letterpaper, portrait, margin=1in}
\usepackage{fancyhdr}

%symbols
\usepackage{amsmath}
\usepackage{amssymb}
\usepackage{amsthm}
\usepackage{mathtools}
\usepackage{hyperref}
\usepackage{gensymb}
\usepackage{multirow,array}

\newtheorem*{remark}{Remark}
\usepackage[T1]{fontenc}
\usepackage[utf8]{inputenc}

%chemistry stuff
%\usepackage[version=4]{mhchem}
%\usepackage{chemfig}

%plotting
\usepackage{pgfplots}
\usepackage{tikz}
\tikzset{middleweight/.style={pos = 0.5, fill=white}}
\tikzset{weight/.style={pos = 0.5, fill = white}}
\tikzset{lateweight/.style={pos = 0.75, fill = white}}
\tikzset{earlyweight/.style={pos = 0.25, fill=white}}

%\usepackage{natbib}

%graphics stuff
\usepackage{graphicx}
\graphicspath{ {./images/} }
\usepackage[style=numeric, backend=biber]{biblatex} % Use the numeric style for Vancouver
\addbibresource{the_bibliography.bib}
%code stuff
%when using minted, make sure to add the -shell-escape flag
%you can use lstlisting if you don't want to use minted
%\usepackage{minted}
%\usemintedstyle{pastie}
%\newminted[javacode]{java}{frame=lines,framesep=2mm,linenos=true,fontsize=\footnotesize,tabsize=3,autogobble,}
%\newminted[cppcode]{cpp}{frame=lines,framesep=2mm,linenos=true,fontsize=\footnotesize,tabsize=3,autogobble,}

%\usepackage{listings}
%\usepackage{color}
%\definecolor{dkgreen}{rgb}{0,0.6,0}
%\definecolor{gray}{rgb}{0.5,0.5,0.5}
%\definecolor{mauve}{rgb}{0.58,0,0.82}
%
%\lstset{frame=tb,
%	language=Java,
%	aboveskip=3mm,
%	belowskip=3mm,
%	showstringspaces=false,
%	columns=flexible,
%	basicstyle={\small\ttfamily},
%	numbers=none,
%	numberstyle=\tiny\color{gray},
%	keywordstyle=\color{blue},
%	commentstyle=\color{dkgreen},
%	stringstyle=\color{mauve},
%	breaklines=true,
%	breakatwhitespace=true,
%	tabsize=3
%}
% text + color boxes
\usepackage[most]{tcolorbox}
\tcbuselibrary{breakable}
\newtcolorbox{problem}[1]{colback = white, title = {#1}, breakable}
\newtcolorbox{solution}{colback = white, colframe = black!75!white, title = Solution, breakable}
%including PDFs
%\usepackage{pdfpages}
\setlength{\parindent}{0pt}
\usepackage{cancel}
\pagestyle{fancy}
\fancyhf{}
\rhead{Avinash Iyer}
\lhead{Math 212: Homework 5}
\newcommand{\card}{\text{card}}
\newcommand{\ran}{\text{ran}}
\newcommand{\N}{\mathbb{N}}
\newcommand{\Q}{\mathbb{Q}}
\newcommand{\Z}{\mathbb{Z}}
\newcommand{\R}{\mathbb{R}}
\begin{document}
  \begin{problem}{14.6}
    \begin{description}[font=\normalfont]
      \item[4:]
        \begin{align*}
          \frac{dz}{dt} &= \frac{\partial z}{\partial x}\frac{dx}{dt} + \frac{\partial z}{\partial y}\frac{dy}{dt}\\
                        &= \frac{2\frac{1}{t}}{\frac{1}{t^2} + t}\left(-\frac{1}{t^2}\right) + \frac{2\sqrt{t}}{t + \frac{1}{t^2}}
        \end{align*}
      \item[6:]
        \begin{align*}
          \frac{dz}{dt} &= \frac{\partial z}{\partial x}\frac{dx}{dt} + \frac{\partial z}{\partial y}\frac{dy}{dt}\\
                        &= 2e^{1-t^2} -2\left((2-t^2)e^{1-t^2}\right)
        \end{align*}
      \item[8:]
        \begin{align*}
          \frac{\partial z}{\partial u} &= \frac{\partial z}{\partial x}\frac{\partial x}{\partial u} + \frac{\partial z}{\partial y}\frac{\partial y}{\partial u}\\
                                        &=\frac{\left(u^3 + v^3\right)^2\left(4u^3 + 4v^2u\right)}{\left(\left(u^2 + v^2\right)\left(u^2 + v^3\right)\right)^2} + \frac{\left(u^2 + v^2\right)^2\left(6u^5 + 6u^2v^3\right)}{\left(\left(u^2 + v^2\right)\left(u^2 + v^3\right)\right)^2}\\
          \frac{\partial z}{\partial v} &= \frac{\partial z}{\partial x}\frac{\partial x}{\partial v} + \frac{\partial z}{\partial y}\frac{\partial y}{\partial v}\\
                                        &=\frac{\left(u^3 + v^3\right)^2\left(4v^3 + 4u^2v\right)}{\left(\left(u^2 + v^2\right)\left(u^2 + v^3\right)\right)^2} + \frac{\left(u^2 + v^2\right)^2\left(6v^5 + 6v^2u^3\right)}{\left(\left(u^2 + v^2\right)\left(u^2 + v^3\right)\right)^2}\\
        \end{align*}
      \item[14:]
        \begin{align*}
          \frac{\partial z}{\partial u} &= \frac{\partial z}{\partial x}\frac{\partial x}{\partial u} + \frac{\partial z}{\partial y}\frac{\partial y}{\partial u}\\
                                        &= -2\left(\cos(v)\sin(u^2) + \sin(v)\sin(u^2)\right)\\
          \frac{\partial z}{\partial v} &= \frac{\partial z}{\partial x}\frac{\partial x}{\partial v} + \frac{\partial z}{\partial y}\frac{\partial y}{\partial v}\\
                                        &= 2u\sin(v)\sin(u^2) - 2u\cos(v)\sin(u^2)
        \end{align*}
      \item[16:]
        \begin{align*}
          \frac{dz}{dt} &= \frac{\partial z}{\partial x}\frac{dx}{dt} + \frac{\partial z}{\partial y}\frac{dy}{dt}\\
                        &= 3t^{10}(3t^2) + 2t^{11}(2t)\\
                        &= 13t^{12}\\
          z &= t^{13}\\
          \frac{dz}{dt} &= 13t^{12}
        \end{align*}
      \item[38:] I don't know how to do this problem.
    \end{description}
  \end{problem}
  \begin{problem}{14.7}
    \begin{description}[font=\normalfont]
      \item[2:]
        \begin{align*}
          f_{xx} &= 2\\
          f_{xy} &= 2\\
          f_{yx} &= 2\\
          f_{yy} &= 2
        \end{align*}
      \item[6:]
        \begin{align*}
          f_{xx} &= 0\\
          f_{xy} &= e^{y}\\
          f_{yx} &= e^{y}\\
          f_{yy} &= xe^{y}\\
        \end{align*}
      \item[12:]
        \begin{align*}
          \ell(x,y) &= -1 + (1)x + (-1)y\\
          q(x,y) &= \ell(x,y) + (-1)x^2 + (1)(xy)\\
                 &= -1 + x - y - x^2 + xy
        \end{align*}
      \item[14:]
        \begin{align*}
          \ell(x,y) &= 1\\
          q(x,y) &= 1 - 2x^2 - y^2
        \end{align*}
      \item[42:]
        \begin{align*}
          \ell(x,y) &= 1 + \frac{1}{2}x + y\\
          q(x,y) &= \ell(x,y) - \frac{1}{8}x^2 - xy - \frac{1}{4}y^2\\
          f(0.9,0.2) &= 1.14\\
          \ell(0.9,0.2) &= 1.65\\
          q(0.9,0.2) &= 1.36
        \end{align*}
      \item[44:]
        \begin{align*}
          \ell(x,y) &= 1 + x - y\\
          q(x,y) &= \ell(x,y) - xy + y^2\\
          f(0.9,0.2) &= 0.737\\
          \ell(0.9,0.2) &= 1.7\\
          q(0.9,0.2) &= 1.56
        \end{align*}
      \item[48:]
        \begin{align*}
          \frac{\partial^2f}{\partial x^2} &= \frac{2xy}{(x^2 + y^2)^2}\\
          \frac{\partial^2 f}{\partial y^2} &= \frac{-2xy}{(x^2 + y^2)^2}\\
          0 &= \frac{\partial^2 f}{\partial x^2} + \frac{\partial^2 f}{\partial y^2}
        \end{align*}
      \item[50 (a):]
        \begin{align*}
          \frac{\partial u}{\partial t} &= \frac{e^{-\frac{x^2}{4t}}(x^2 - 2t)}{8t^2\sqrt{\pi t}}\\
          \frac{\partial^2 u}{\partial x^2} &= \frac{e^{-\frac{x^2}{4t}}(x^2 - 2t)}{8t^2\sqrt{\pi t}}
        \end{align*}
      \item[52:] We can assume that $z_{yy} = 0$, as $y$ is only of degree $1$ in $z$.
    \end{description}
  \end{problem}
  \begin{problem}{15.1}
    \begin{description}[font=\normalfont]
      \item[2:]
        \begin{itemize}
          \item $D$: Saddle Point
          \item $B$: Local Maximum
          \item $C$: Saddle Point
          \item $G$: Local Minimum
          \item $F$: Saddle Point
          \item $E$: Local Minimum
        \end{itemize}
      \item[4:] Because neither $f_{x}$ nor $f_{y}$ are equal to zero, $(1,2)$ cannot be a critical point.
      \item[6:]
        \begin{align*}
          f_{xx} &= 2\\
          f_{yy} &= \cos y\\
          f_{xy} &= 0\\
          D(0,0) &= 4 > 0
        \end{align*}
        Therefore, the point is a local minimum.
      \item[8:]
        \begin{align*}
          f_{xx} &= 12x^2\\
          f_{yy} &= 6y\\
          D(0,0) &= 0
        \end{align*}
        The critical point is indeterminate.
      \item[10:]
        \begin{align*}
          f_{xx} &= 0\\
          f_{yy} &= 0\\
          f_{xy} &= 1\\
          D(0,0) &= -1 < 0
        \end{align*}
        The critical point is a saddle point.
      \item[14:]
        \begin{align*}
          \frac{\partial f}{\partial x} &= 2x - 2y\\
          \frac{\partial f}{\partial y} &= 6y - 8 - 2x\\
          \frac{\partial f}{\partial x} &= 0\\
          y &= 2\\
          \frac{\partial f}{\partial y} &= 0\\
          x &= 2\\
          \frac{\partial^2 f}{\partial x^2} &= 2\\
          \frac{\partial^2f}{\partial y^2} &= 6\\
          \frac{\partial^2 f}{\partial x \partial y} &= -2\\
          D &= 8 > 0
        \end{align*}
        Therefore, the point $f(2,2)$ is a local minimum.
      \item[20:]
        \begin{align*}
          \frac{\partial f}{\partial x} &= 6x^2 - 6xy + 12x\\
          \frac{\partial f}{\partial y} &= -3x^2 - 12y\\
          \frac{\partial f}{\partial x} &= 0\\
          \frac{\partial f}{\partial y} &= 0\\
          y &= -\frac{1}{4}x^2\\
          0 &= 6x^2 + \frac{3}{2}x^3 + 12x\\
          0 &= \frac{3}{2}x\left(x^2 + 4x + 8\right)\\
          (x,y) &= \{(0,0)\}\\
          \frac{\partial^2 f}{\partial x^2}\biggr\vert_{(0,0)} &= 12\\
          \frac{\partial^2f}{\partial y^2}\biggr\vert_{(0,0)} &= -12\\
          \frac{\partial^2f}{\partial x \partial y} &= 0\\
          D(0,0) &= -144 < 0
        \end{align*}
        Therefore, the point $f(0,0)$ is a saddle point
      \item[24:]
        \begin{align*}
          \frac{\partial f}{\partial x} &= 2xy + 1\\
          \frac{\partial f}{\partial y} &= 2xy + 1\\
          y &= -\frac{1}{2x}\\
          (x,y) &= \left\{\left(t,-\frac{1}{2t}\right) \mid t\neq 0 \right\}\\
          \frac{\partial^2 f}{\partial x^2} &= 2y\\
          \frac{\partial^2f}{\partial y^2} &= 2x\\
          \frac{\partial^2f}{\partial x \partial y} &= 2y\\
          D &= 4xy-4y^2\\
            &= -2 - \frac{1}{4t^2}\\
            &< 0
        \end{align*}
        Therefore, the points $f(t,-1/2t)$ are saddle points.
      \item[28:]
        \begin{align*}
          f_{xx} &= 6x\\
          f_{yy} &= 2k\\
          f_{xy} &= 9\\
          D &= 12tk - 81
        \end{align*}
        If $k > 81/12$, then the point is a local minimum, and if $k < 81/12$, then the point is a saddle. It is not possible for the point to be a local maximum.
      \item[32:]
        \begin{enumerate}[(a)]
          \item 
            \begin{align*}
              \frac{\partial f}{\partial x} &= -2(x-a)e^{-(x-a)^2 - (y-b)^2}\\
              \frac{\partial f}{\partial y} &= -2(y-b)e^{-(x-a)^2 - (y-b)^2}\\
              \frac{\partial^2f}{\partial x^2} &= 2\left(2x^2-4ax+(2a^2-1)\right)e^{-(x-a)^2 - (y-b)^2}\\
              \frac{\partial^2f}{\partial y^2} &= 2\left(2y^2-4by+(2b^2-1)\right)e^{-(x-a)^2 - (y-b)^2}\\
              \frac{\partial^2f}{\partial x\partial y} &= -4(x-a)(y-b)e^{-(x-a)^2 - (y-b)^2}\\
              D(a,b) &>0
            \end{align*}
            Therefore, the critical point is at $(a,b)$.
          \item $a=-1$, $b=5$
          \item The point at $(a,b)$ is a local maximum.
        \end{enumerate}
      \item[34:]
        \begin{enumerate}[(a)]
          \item Local Maximum.
          \item Saddle Point
          \item Local Minimum
          \item None.
        \end{enumerate}
      \item[38:]
      \item[42:]
    \end{description}
  \end{problem}
\end{document}
