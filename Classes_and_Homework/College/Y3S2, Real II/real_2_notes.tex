\documentclass[10pt]{extarticle}
\title{}
\author{}
\date{}
\usepackage[shortlabels]{enumitem}


%paper setup
\usepackage{geometry}
\geometry{letterpaper, portrait, margin=1in}
\usepackage{fancyhdr}
% sans serif font:
\usepackage{cmbright}
%symbols
\usepackage{amsmath}
\usepackage{bigints}
\usepackage{amssymb}
\usepackage{amsthm}
\usepackage{mathtools}
\usepackage{bbm}
\usepackage{bbold}
\usepackage[hidelinks]{hyperref}
\usepackage{gensymb}
\usepackage{multirow,array}
\usepackage{multicol}

\newtheorem*{remark}{Remark}
\usepackage[T1]{fontenc}
\usepackage[utf8]{inputenc}

%chemistry stuff
%\usepackage[version=4]{mhchem}
%\usepackage{chemfig}

%plotting
\usepackage{pgfplots}
\usepackage{tikz}
\usetikzlibrary{cd}
\tikzset{middleweight/.style={pos = 0.5}}
%\tikzset{weight/.style={pos = 0.5, fill = white}}
%\tikzset{lateweight/.style={pos = 0.75, fill = white}}
%\tikzset{earlyweight/.style={pos = 0.25, fill=white}}

%\usepackage{natbib}

%graphics stuff
\usepackage{graphicx}
\graphicspath{ {./images/} }
\usepackage[style=numeric, backend=biber]{biblatex} % Use the numeric style for Vancouver
\addbibresource{the_bibliography.bib}
%code stuff
%when using minted, make sure to add the -shell-escape flag
%you can use lstlisting if you don't want to use minted
%\usepackage{minted}
%\usemintedstyle{pastie}
%\newminted[javacode]{java}{frame=lines,framesep=2mm,linenos=true,fontsize=\footnotesize,tabsize=3,autogobble,}
%\newminted[cppcode]{cpp}{frame=lines,framesep=2mm,linenos=true,fontsize=\footnotesize,tabsize=3,autogobble,}

%\usepackage{listings}
%\usepackage{color}
%\definecolor{dkgreen}{rgb}{0,0.6,0}
%\definecolor{gray}{rgb}{0.5,0.5,0.5}
%\definecolor{mauve}{rgb}{0.58,0,0.82}
%
%\lstset{frame=tb,
%	language=Java,
%	aboveskip=3mm,
%	belowskip=3mm,
%	showstringspaces=false,
%	columns=flexible,
%	basicstyle={\small\ttfamily},
%	numbers=none,
%	numberstyle=\tiny\color{gray},
%	keywordstyle=\color{blue},
%	commentstyle=\color{dkgreen},
%	stringstyle=\color{mauve},
%	breaklines=true,
%	breakatwhitespace=true,
%	tabsize=3
%}
% text + color boxes
\usepackage[most]{tcolorbox}
\tcbuselibrary{breakable}
\tcbuselibrary{skins}
\newtcolorbox{problem}[1]{colback=white,enhanced,title={\small #1},
          attach boxed title to top center=
{yshift=-\tcboxedtitleheight/2},
boxed title style={size=small,colback=black!60!white}, sharp corners, breakable}
%including PDFs
%\usepackage{pdfpages}
\setlength{\parindent}{0pt}
\usepackage{cancel}
\pagestyle{fancy}
\fancyhf{}
\rhead{Avinash Iyer}
\lhead{Real Analysis II: Class Notes}
\newcommand{\card}{\text{card}}
\newcommand{\ran}{\text{ran}}
\newcommand{\N}{\mathbb{N}}
\newcommand{\Q}{\mathbb{Q}}
\newcommand{\Z}{\mathbb{Z}}
\newcommand{\R}{\mathbb{R}}
\newcommand{\iprod}[2]{\left\langle #1,#2\right\rangle}
\newcommand{\norm}[1]{\left\Vert #1 \right\Vert}
\setcounter{secnumdepth}{0}
\begin{document}
\section{Normed Vector Spaces}%
\subsection{Vector Spaces}%
Throughout, $\mathbb{F} = \R$ or $\mathbb{C}$. A \textbf{vector space} over $\mathbb{F}$ is a nonempty set $V$ equipped with two operations: vector addition and scalar multiplication.
\begin{align*}
  V\times V \xrightarrow{+} V\\
  (v,w)\mapsto v+w \tag*{Vector Addition}\\
  F\times V \rightarrow V\\
  (\alpha,v)\mapsto \alpha v \tag*{Scalar Multiplication}
\end{align*}
The vector space is an Abelian group, where $u,v,w\in V$ and $\alpha,\beta\in \mathbb{F}$, we have:
\begin{enumerate}[(i)]
  \item $u+(v+w) = (u+v)+w$
  \item $\exists 0_v\in V$ with $\forall v\in V,~0_v + v = v + 0_v = v$
  \item $(\forall v\in V)(\exists w\in V)$ with $v+w = 0_v$
  \item $\forall v,w\in V,~v+w = w+v$
  \item $\alpha(v+w) = \alpha v + \alpha w,~(\alpha + \beta)v = \alpha v + \beta v$
  \item $\alpha(\beta w) = (\alpha\beta)w$
  \item $1\cdot v = v$
\end{enumerate}
\textbf{Remarks:}
\begin{enumerate}[(a)]
  \item $0_v$ is unique and known as the zero vector.
  \item The vector $w$ in (iii) is unique, and denoted $-v$.
  \item $0\cdot v = 0_v$
  \item $(-1)\cdot v = -v$
  \item Property (iv) follows from all the other axioms.
  \item For $n\in\N$, $n\cdot v = \underbrace{v+v+\cdots+v}_{n~\text{times}}$
\end{enumerate}
\subsection{Subspaces}%
Let $V$ be a vector space over $\mathbb{F}$. A \textbf{subspace} is a nonempty subset $W\subseteq V$ satisfying the following:
\begin{enumerate}[(i)]
  \item $w\in W,\alpha\in \mathbb{F}\rightarrow \alpha w \in W$.
  \item $w_1,w_2\in W \Rightarrow w_1 + w_2\in W$.
\end{enumerate}
\begin{description}
  \item[Remark:] $0_v$ is always a member of any subspace; a subspace is also a vector space.
\end{description}
\subsubsection{Proposition: Intersection of Subspaces}%
If $\{W_i\}_{i\in I}$ is a family of subspaces of $V$, then, $\bigcap W_i$ is a subspace of $V$.
\subsubsection{Proposition: Union of Subspaces}%
It is not the case that the union of subspaces of $V$ also a subspace. For example, consider $\R^2$ with the traditional vector space operations:
\begin{align*}
  \begin{pmatrix}x\\y\end{pmatrix} + \begin{pmatrix}x'\\y'\end{pmatrix} &= \begin{pmatrix}x+x'\\y+y'\end{pmatrix}\\
  \alpha \begin{pmatrix}x\\y\end{pmatrix} &= \begin{pmatrix}\alpha x\\\alpha y\end{pmatrix}
\end{align*}
If $W_1,W_2\in V$ are subspaces such that $W_1 \cup W_2$ is a subspace, then $W_1\subseteq W_2$ or $W_2\subseteq W_1$.
\subsubsection{Generated Subspaces}%
Let $S\subseteq V$ be any subset of a vector space $V$. Then,
\begin{align*}
  \text{span}(S) &= \left\{\sum_{j=1}^{n}\alpha_jv_j\mid \alpha_1,\dots,\alpha_n\in \mathbb{F},v_1,\dots,v_n\in S\right\}
\end{align*}
\begin{description}
  \item[Remarks:]\hfill
    \begin{itemize}
      \item $\text{span}(S)\subseteq V$ is a subspace.
      \item $\text{span}(S) = \bigcap W$, where $S\subseteq W$ and $W\subseteq V$ is a subspace. Thus, $\text{span}(S)$ is the ``smallest'' subspace containing $S$, or the subspace generated by $S$.
    \end{itemize}
\end{description}
\subsubsection{Proposition: Quotient Group on Vector Space}%
Let $V$ be a vector space, and let $W\subseteq V$ is a subspace. Define $u\sim_{W} v\leftrightarrow u-v\in W$.
\begin{enumerate}[(1)]
  \item $\sim_W$ is an equivalence relation.
  \item If $[v]_W$ denotes the equivalence class of $v$, then $[v]_W = v+W = \{v+w|w\in W\}$.
  \item $V/W:= \{[v]_W|v\in V\}$ is a vector space with $[v_1]_W + [v_2]_W = [v_1+v_2]_W$ and $\alpha[v]_W = [\alpha v]_W$.
\end{enumerate}
\begin{description}
  \item[Proof of (1):]\hfill
    \begin{itemize}
      \item Reflexive: $u\sim_W u$, since $u-u = 0\in W$.
      \item Transitive: Suppose $u\sim_W v$, and $v\sim_W z$. Then, $u-v\in W$, and $v-z\in W$. So, $(u-v) + (v-z)\in W$, so $u-z\in W$. Whence, $u\sim_W z$.
      \item Symmetric: If $u\sim_W v$, then $u-v\in W$, so $-1\cdot(u-v)\in W$, so $v-u\in W$. Whence, $v\sim_W u$.
    \end{itemize}
  \item[Proof of (2):]
    \begin{align*}
      [v]_W &= \{u\in V\mid u\sim_W v\}\\
            &= \{u\in V\mid u-v\in W\}\\
            &= \{u\in V\mid u = v+w~\text{some $w\in W$}\}\\
            &= \{v+w\mid w\in W\}\\
            &= v+W
    \end{align*}
  \item[Proof of (3):] Prove that the operation is well-defined.
\end{description}
\subsection{Bases}%
Let $V$ be a vector space and $S\subseteq V$ be a subset.
\begin{enumerate}[(1)]
  \item $S$ is said to be spanning for $V$ if $\text{span}(S) = V$.
  \item $S$ is linearly independent if, for $\sum_{j=1}^{n}\alpha_jv_j = 0_v$ with $\alpha_1,\dots,\alpha_n\in \mathbb{F}$, $v_1,\dots,v_n\in S$, then $\alpha_1=\alpha_2 = \cdots = \alpha_n = 0$.
  \item $S$ is a basis for $V$ if $S$ is linearly independent and spanning for $V$.
\end{enumerate}
\subsubsection{Proposition: Existence of Basis}%
Every vector space admits a basis. If $B_0\subseteq V$ is linearly independent, $\exists B\subseteq V$ such that $B$ is a basis and $B \supseteq B_0$.
\begin{description}
  \item[Background:] A relation on a set $X$ is a subset $R\subseteq X\times X$. If $R$ is reflexive ($x\sim x$), transitive ($x\sim y,y\sim z \rightarrow x\sim z$), and antisymmetric ($x\sim y, y\sim x \rightarrow x=y$), then $R$ is an ordering, and we write $x\leq y$.\\

    If $\leq$ is an ordering of $X$ such that $\forall x,y\in X,~x\leq y$ or $y\leq x$, then $\leq$ is a total (or linear) ordering.\\

    Let $\leq$ be an ordering of $X$, let $Y\subseteq X$. An upper bound for $Y$ is an element $u\in X$ such that $y\leq u$ $\forall y\in Y$. A maximal element in $X$ is an element $m\in X$ such that $x\in X,~x \geq m \rightarrow x=m$.
  \item[Example:] $\N$ under the division ordering defines $a\leq b \Leftrightarrow a|b$. If we want to find the maximal elements of $A = \{2,6,9,12\}$, we would see that they are $9$ and $12$ (since no element of $A$ can be divided by $9$ and $12$). Meanwhile, $\N$ itself has no maximal elements.\\

    This leads us to ask: given an ordered set, $(X,\leq)$, does $X$ admit maximal elements.
  \item[Zorn's Lemma (or Axiom):] Let $(X,\leq)$ be an ordered set. Suppose that every totally ordered subset, $Y\subseteq X$ has an upper bound in $X$. Then, $X$ admits at least one maximal element.\\

    The proof of Zorn's Lemma relies on the Axiom of Choice (and Zorn's Lemma is equivalent to the Axiom of Choice).
  \item[Proof:] Let $X = \{D\mid B_0 \subseteq D \subseteq V\}$ with $D$ linearly independent. Since $B_0\subseteq X$, $X \neq \emptyset$. Define $D,E\in X$, $D\leq E \Leftrightarrow D\subseteq E$. We will show that $X$ has a maximal element.\\

    Consider any totally ordered subset, $Y = \{D_i\}_{i\in I}$. Consider $D = \bigcup D_i$. Clearly, $B_0\subseteq D\subseteq V$. Suppose $\sum \alpha_kv_k = 0_v$ with $v_1,\dots,v_n \in D$. Therefore, $\exists D_j$ with $v_1,\dots,v_n\in D_j$ because $Y$ is totally ordered. However, by definition, $D_j$ is a linearly independent set --- therefore, $\alpha_k = 0$. Thus, $D$ is linearly independent.\\

    Since $D$ is linearly independent, and $B_0\subseteq D$, it must be the case that $D\in X$. $D$ is also an upper bound for $Y$. So, by Zorn's Lemma, $X$ has a maximal element, $B$.\\

    So, $B_0\subseteq B \subseteq V$, $B$ is independent, and $B$ is maximal in $X$. We claim that $B$ is a basis for $V$. Suppose toward contradiction that $\exists v \in V$ such that $v\notin \text{span}(B)$. Consider $B' = B \cup \{v\}$.\\

    Then, $B_0\subseteq B'$, and $B'$ is linearly independent --- if $\sum \alpha_kv_k + \alpha v = 0$, where $v_1,\dots,v_n\in B$, then either:
    \begin{itemize}
      \item If $\alpha = 0$, then $\alpha_kv_k = 0 \Rightarrow \alpha_k = 0$.
      \item If $\alpha \neq 0$, then $\sum\alpha_kv_k = -\alpha v$, which means $v\in \text{span}(B)$. $\bot$
    \end{itemize}
    Thus, we have a linearly independent set, $B'$, with $B\subseteq B'$, and $B_0\subseteq B'$. Therefore, $B'\in X$. However, this contradicts the maximality of $B$. Therefore, $\text{span}(B) = V$, and $B$ is a basis for $V$.
\end{description}
\subsection{Examples: Vector Spaces}%
\begin{enumerate}[(1)]
  \item $n$-Dimensional Vectors:
    \begin{align*}
      \mathbb{F}^n &= \left\{ \begin{pmatrix}x_1\\\vdots\\x_n\end{pmatrix} \mid x_j\in \mathbb{F}\right\}\\
      \begin{pmatrix}x_1\\\vdots\\x_n\end{pmatrix} + \begin{pmatrix}y_1\\\vdots\\y_n\end{pmatrix} &= \begin{pmatrix}x_1+y_1\\\vdots\\x_n + y_n\end{pmatrix}\\
      \alpha \begin{pmatrix}x_1\\\vdots\\x_n\end{pmatrix} &= \begin{pmatrix}\alpha x_1\\\vdots\\\alpha x_n\end{pmatrix}\\
      B &= \{e_1,\dots,e_n\}\\
      \shortintertext{where $e_i$ denotes the unit vector at position $i$.}
    \end{align*}
  \item $m\times n$ Matrices:
    \begin{align*}
      \mathbb{M}_{m,n}(\mathbb{F}) &= \left\{ \begin{pmatrix}a_{11} & \cdots & a_{1n}\\\vdots & \ddots & \vdots \\ a_{m1} & \cdots & a_{mn}\end{pmatrix}\mid a_{ij}\in \mathbb{F}\right\}\\
      (a_{ij}) + (b_{ij}) &= (a_{ij} + b_{ij})\\
      \alpha(a_{ij}) &= (\alpha a_{ij})\\
      B &= \{e_{ij}\}\\
      \shortintertext{where $e_{ij}$ denotes a matrix of $0$ everywhere except column $i$ and row $j$.}
    \end{align*}
  \item Functions with domain $\Omega$:
    \begin{align*}
      \mathcal{F}(\Omega,\mathbb{F}) &= \{f\mid f: \Omega \rightarrow \mathbb{F}\}\\
      (f+g)(x) &= f(x) + g(x)\\
      (\alpha f)(x) &= \alpha f(x)
    \end{align*}
  \item Bounded functions with domain $\Omega$:
    \begin{align*}
      \ell_{\infty}(\Omega,\mathbb{F}) &= \{f\in \mathcal{F}(\Omega,\mathbb{F}) \mid \Vert f\Vert_u \leq \infty\}\\
      \Vert f\Vert_u &= \sup_{x\in\Omega}|f(x)|
    \end{align*}
    Exercises:
    \begin{itemize}
      \item Triangle Inequality: $\Vert f+g\Vert_u \leq \Vert f\Vert_u + \Vert g\Vert_u$
      \item Scalar Multiplication/Absolute Homogeneity: $\Vert \alpha f\Vert_u = |\alpha| \Vert f\Vert_u$
      \item Positive Definite: $\Vert f\Vert_u = 0 \Rightarrow f = \mathbb{0}$
    \end{itemize}
    \begin{description}
      \item[Proof of Triangle Inequality:] Given $x\in\Omega$,
        \begin{align*}
          |(f+g)(x)| &= |f(x) + g(x)|\\
                     &\leq |f(x)| + |g(x)|\\
                     &\leq \Vert f\Vert_u + \Vert g \Vert_u\\
                     \shortintertext{Therefore,}
          \sup|(f+g)(x)| &\leq \Vert f\Vert_u + \Vert g\Vert_u\\
          \Vert f+g\Vert_u &\leq \Vert f\Vert_u + \Vert g\Vert_u
        \end{align*}
    \end{description}
  \item Continuous functions on closed and bounded intervals:
    \begin{align*}
      C([a,b],\mathbb{F}) &= \{f: [a,b]\rightarrow \mathbb{F}\mid f~\text{continuous}\}
    \end{align*}
    Check that $C([a,b],\mathbb{F}) \subseteq \ell_{\infty}([a,b],\mathbb{F})$ is a subspace.
  \item Let $f:[a,b]\rightarrow \R$ be any function. Let $\mathcal{P}: a = x_0 < x_1< x_2 < \cdots < x_n = b$.
    \begin{align*}
      \text{var}(f;\mathcal{P}) &:= \sum_{k=1}^{n} |f(x_k)-f(x_{k-1})|\\
      \text{var}(f) &= \sup_{\mathcal{P}} \text{var}(f;\mathcal{P})\\
      \text{BV}([a,b]) &= \{f:[a,b]\rightarrow \R\mid \text{var}(f) < \infty\}\\
      \Vert f\Vert_{\text{BV}} &= |f(a)| + \text{var}(f)
    \end{align*}
    $\text{BV}([a,b])$ is a vector space. 
    \begin{description}
      \item[Question:] Is $\mathbb{1}_{\Q} \in \text{BV}([0,1])$?
    \end{description}
  \item Suppose $K\subseteq V$ is a \textit{convex} subset of a vector space: $v,w\in K,t\in [0,1]\Rightarrow (1-t)v + tw\in K$. Let $\text{Aff}(K) = \{f: K\rightarrow \R\mid f\text{ is affine}\}$, where $f$ is affine if $\forall v,w\in K,t\in[0,1],f((1-t)v + tw) = (1-t)f(v) + tf(w)$.
    \begin{description}
      \item[Exercise:] Show that $\text{Aff}(K)\subseteq \mathcal{F}(K,\R)$ is a subspace.
    \end{description}
  \item Let $S$ be defined as
    \begin{align*}
      S = \{(a_k)_{k=1}^{\infty}\mid a_k\in\mathbb{F}\}.
    \end{align*}
    Under pointwise operations, $S$ is a vector space.
    \begin{align*}
      (a_k)_k + (b_k)_k &= (a_k + b_k)_k\\
      \alpha(a_k)_k &= (\alpha a_k)_k
    \end{align*}
    \begin{description}
      \item[Note 1:] $S = \mathcal{F}(\N,\mathbb{F})$.
      \item[Note 2:] $c_{00} \subseteq \ell_{1}\subseteq c_{0} \subseteq c \subseteq \ell_{\infty}\subseteq S$.
        \begin{itemize}
          \item $c_{00} = \left\{(a_k)_k\mid \text{finitely many }a_k\neq 0\right\}$
          \item $c_0 = \left\{(a_k)k\mid (a_k)_k\rightarrow 0\right\}$
          \item $c = \left\{(a_k)_k\mid (a_k)_k\rightarrow a < \infty\right\}$
          \item $\ell_{\infty} = \left\{(a_k)_k\mid \Vert(a_k)_k\Vert_u<\infty\right\}$
          \item $\ell_1 = \left\{(a_k)_k\mid \sum_{k=1}^{\infty}|a_k| = a < \infty\right\}$
        \end{itemize}
    \end{description}
  \item $C_C(\R) \subseteq C_0(\R) \subseteq \ell_{\infty}(\R)$ are all subspaces.
    \begin{itemize}
      \item $C_C(\R) = \left\{f:\R\rightarrow \mathbb{F}\mid f\text{ compactly supported}\right\}$: $f:\R\rightarrow \mathbb{F}$ is compactly supported if $\exists [a,b]$ such that $x\notin[a,b]\Rightarrow f(x) = 0$.
      \item $C_0(\R) = \left\{f:\R\rightarrow \mathbb{F}\mid f\text{ continuous, }\lim_{x\rightarrow \pm\infty} f(x) = 0\right\}$
    \end{itemize}
  \item Let $S$ be any non-empty set.
    \begin{align*}
      \mathbb{F}(S)&:=\left\{f: S\rightarrow \mathbb{F}\mid f\text{ finitely supported}\right\}\\
      \text{supp}(f)&=\left\{x\in S\mid f(x)\neq 0\right\}
    \end{align*}
    We claim that $\mathbb{F}(S)\subseteq \mathcal{F}(S,\mathbb{F})$ is a subspace. Consider $e_t:S\rightarrow \mathbb{F}$ defined as follows:
    \begin{align*}
      e_t(s) &= \begin{cases}
        1 & s = t\\
        0 & s\neq t
      \end{cases}.
    \end{align*}
    We claim that $\xi = \{e_t\}_{t\in S}$ is a basis for $\mathbb{F}(S)$.\\

    Indeed, given $f\in \mathbb{F}(S)$, we know that $\text{supp}(f) = \{t_1,\dots,t_n\}\subseteq S$. Therefore, $f = \sum_{k=1}^{n}f(t_k)e_{t_k} \in \text{span}(\xi)$. Therefore, $\xi$ is spanning for $\mathbb{F}(S)$. Suppose $\sum_{k=1}^{n}\alpha_{t_k}e_{t_k} = \mathbb{0}$ for some $\alpha_k\in \mathbb{F}$, $t_k\in S$.
    \begin{align*}
      \left(\sum_{k=1}^{\alpha_{t_k}}e_{t_k}\right) &= \mathbb{0}(t_1)\\
      \alpha_{t_1} &= 0.
    \end{align*}
    Similarly, $\alpha_{t_j} = 0$ for $j = 1,\dots,n$. Therefore, $\xi$ is linearly independent. Since $\xi$ is linearly independent and spanning, $\xi$ forms a basis for $\mathbb{F}(S)$.
    \begin{description}
      \item[Note:] The free vector space, $\mathbb{F}(S)$, displays the universal property.\\

        There are functions $\iota: S\rightarrow \mathbb{F}(S)$, where $\iota(t) = e_{t}$, and given any map $\varphi: S\rightarrow V$ for $V$ a vector space over $\mathbb{F}$, $\exists!$ linear map $T_{\varphi}: \mathbb{F}(S)\rightarrow V$ such that $\iota\circ T_{\varphi} = \varphi$.
        \begin{center}
      % https://tikzcd.yichuanshen.de/#N4Igdg9gJgpgziAXAbVABwnAlgFyxMJZABgBpiBdUkANwEMAbAVxiRAGUQBfU9TXfIRQBGclVqMWbADrSAtnRwALAEYrgAMS4AKdgEpuvEBmx4CRUcPH1mrRCABq3cTCgBzeEVAAzAE4Q5JDIQHAgkUQlbGWl8HDpDH39AxGDQpAAmagY6FRgGAAV+MyEQXyw3JRwQahspe1l6XzQlLASQPwDw6jTETMi6kAAVAH1gBromlq5nLiA
      \begin{tikzcd}
      S \arrow[r, "\iota"] \arrow[rd, "\varphi"'] & \mathbb{F}(S) \arrow[d, "T_{\varphi}"] \\
                                                  & V                                     
      \end{tikzcd}
        \end{center}
      \item[Proof:] Every $f\in \mathbb{F}(S)$ has a unique expression $f = \sum_{k=1}^{n}f(t_k)e_{t_k}$, where $\text{supp}(f) = \{t_1,\dots,t_n\}$. Therefore,
        \begin{align*}
          T_{\varphi}(f) := \sum_{k=1}^{n}f(t_k)\varphi(t_k)
        \end{align*}
      \item[Exercise:] Show $T_{\varphi}$ is linear and unique.
      \item[Exercise 2:] Suppose $V$ is a vector space over $\mathbb{F}$ with basis $B$. Show that $\mathbb{F}(B) \cong V$. Remember that $V\cong W$ if $\exists$ $T:V\rightarrow W$ such that $T$ is bijective and linear.
    \end{description}
\end{enumerate}
\subsection{Normed Spaces}%
To every vector $v\in V$, we want to assign a length to $v$, $\Vert v \Vert$.\\

A \textbf{norm} on a vector space $V$ is a map
\begin{align*}
  \Vert \cdot \Vert: V\rightarrow \R^{+}\\
  v\mapsto \Vert v\Vert \geq 0
\end{align*}
such that
\begin{enumerate}[(i)]
  \item Homogeneity: $\Vert \alpha v \Vert = |\alpha|\Vert v \Vert$
  \item Triangle Inequality: $\Vert v + w\Vert \leq \Vert v \Vert + \Vert w \Vert$
  \item Positive definiteness: $\Vert v \Vert = 0 \Rightarrow v = \mathbb{0}_V$.
\end{enumerate}
If $p: V\rightarrow \R^{+}$ satisfies (i) and (ii), then $p$ is a \textbf{seminorm}.\\

The pair $(V,\Vert \cdot \Vert)$ is called a normed space.\\

Two norms, $\Vert \cdot \Vert$ and $\Vert \cdot \Vert'$ are called \textbf{equivalent} if $\exists c_1,c_2 \geq 0$ with, $\forall v\in V$,
\begin{align*}
  \Vert v \Vert &\leq c_1\Vert v \Vert'\\
  \Vert v \Vert' &\leq c_2\Vert v \Vert
\end{align*}
\begin{description}
  \item[Note:] On $\R^n$, all norms are equivalent.
  \item[Exercise:] If $p$ is any seminorm on $V$, then $|p(v)-p(w)| \leq p(v-w)$.
  \item[Notation:] If $V$ is a normed space, then $B_V = \{v\in V\mid \Vert v \Vert\leq 1\}$, and $U_V = \{v\in V\mid \Vert v \Vert < 1\}$ are the closed and open unit ball respectively.
\end{description}
\subsubsection{Examples of Normed Spaces}%
\begin{enumerate}[(1)]
  \item Given $V = \mathbb{F}^n$ and $\displaystyle x = \begin{bmatrix}x_1\\x_2\\\vdots\\x_n\end{bmatrix}$, we have different norms:
    \begin{align*}
      \Vert x \Vert_1 &= \sum_{j=1}^{n}|x_j|\\
      \Vert x \Vert_{\infty} &= \max_{1\leq j \leq n}|x_j|\\
      \Vert x \Vert_2 &= \left(\sum_{j=1}^{n}|x_j|^2\right)^{1/2}.
    \end{align*}
    In general, for $1\leq p < \infty$,
    \begin{align*}
      \Vert x\Vert_{p} &= \left(\sum_{j=1}^{\infty}|x_j|^{p}\right)^{1/p}.
    \end{align*}
    \begin{description}
      \tiny
      \item[Exercise:] Show that $\Vert \cdot \Vert_1$ and $\Vert \cdot \Vert_{\infty}$ are norms. Show that $\lim_{p\rightarrow\infty}\Vert x \Vert_p = \Vert x \Vert_{\infty}$.
    \end{description}
    We want to show that $\Vert \cdot \Vert_{p}$ defines a norm for $1\leq p < \infty$. If $ 1\leq p < \infty$, its conjugate index $q\in [1,\infty]$ whereby $\frac{1}{p} + \frac{1}{q} = 1$. For example, if $p = 1$, then $q = \infty$, and if $p = \infty$, then $q = 1$.
    \begin{description}
      \item[Lemma 1:] For $1 < p < \infty$, $p^{-1} + q^{-1} = 1$, $f: [0,\infty)\rightarrow \R$, $f(t) = \frac{1}{p}t^p - t + \frac{1}{q}$. Then, $f(t) \geq 0$ for all $t \geq 0$.
      \item[Proof 1:] We can see that $f'(t) = t^{p-1}-1$. Then, $f'(t) = 0$ at $t = 1$; $f'(t) > 0$ for $t > 1$ and $f'(t) < 0$ for $t \in [0,1)$.\\
        
        So, since $f(t) \geq f(1)$ for all $t \geq 0$, and $f(1) = 0$, $f(t) \geq 0$ for all $t\geq 0$.
      \item[Lemma 2:] For $1 < p < \infty$, $p^{-1} + q^{-1} = 1$, $z,y\geq 0$, $xy \leq \frac{1}{p}x^p + \frac{1}{q}y^q$.
      \item[Proof 2:] We know from Lemma 1, $t \leq \frac{1}{p}t^p + \frac{1}{q}$. Multiply by $y^q$ to get
        \begin{align*}
          ty^{q} \leq \frac{1}{p}t^py^q + \frac{1}{q}y^q.
        \end{align*}
        Set $t = xy^{1-q}$. Then,
        \begin{align*}
          xy^{1-q}y^q &\leq \frac{1}{p}x^py^{p-pq}y^{q} + \frac{1}{q}y^q.
        \end{align*}
        Since $\frac{1}{p} + \frac{1}{q} = 1$, $p-pq = -q$, so
        \begin{align*}
          xy &\leq \frac{1}{p}x^p + \frac{1}{q}y^q.
        \end{align*}
    \end{description}
    With these two lemmas in mind, we get two important inequalities.
    \begin{description}
      \item[Hölder's Inequality:] For $1\leq p \leq \infty$, $p^{-1} + q^{-1} = 1$. Then, for $x,y\in \mathbb{F}^{n}$,
        \begin{align*}
          \left|\sum_{j=1}^{n}x_jy_j\right| &\leq \Vert x \Vert_{p} \Vert y \Vert_{q}. 
        \end{align*}
      \item[Proof of Hölder's Inequality:] For $p = 1$, the solution is as follows:
        \begin{align*}
          \left|\sum_{j=1}^{n}x_jy_j\right| &\leq \sum_{j=1}^{n}|x_j||y_j|\\
                                            &\leq \sum_{j=1}^{n}|x_j|\Vert y\Vert_{\infty}\\
                                            &= \Vert x \Vert_q \Vert y \Vert_{\infty},
        \end{align*}
        and similarly for $p=\infty, q = 1$.\\

        For $1 < p < \infty$, assume $\Vert x \Vert_p = \Vert y \Vert_q = 1$.
        \begin{align*}
          \left|\sum_{j=1}^{n}x_jy_j\right| &\leq \sum_{j=1}^{\infty}|x_j||y_j|\\
                                            &\leq \sum_{j=1}^{n}\left(\frac{1}{p}|x_j|^p + \frac{1}{q}|y_j|^q\right)\\
                                            &= \frac{1}{p} \left(\sum_{j=1}^{n}|x_j|^p\right) + \frac{1}{q}\left(\sum_{j=1}^{n}|y_j|^q\right)\\
                                            &= \frac{1}{p} + \frac{1}{q}\\
                                            &= 1.
        \end{align*}
        If $\Vert x \Vert_p = 0$ or $\Vert y \Vert_q = 0$, then $x = \mathbb{0}_{\mathbb{F}}$ or $y = \mathbb{0}_{\mathbb{F}}$, the inequality still holds.\\

        Assume $\Vert x \Vert_p \neq 0$, $\Vert y \Vert_{p} \neq 0 $. Set
        \begin{align*}
          x' &= \frac{x}{\Vert x \Vert_p}\\
          y' &= \frac{y}{\Vert y \Vert_p}.
        \end{align*}
        It can be verified that $\Vert x'\Vert_p = 1 = \Vert y'\Vert_q$. Therefore,
        \begin{align*}
          \left|\sum_{j=1}^{n}x_j'y_j'\right| &\leq 1\\
          \left|\sum_{j=1}^{n}\frac{x_j}{\Vert x \Vert_p}\frac{y_j}{\Vert y \Vert_q}\right| &\leq 1\\
          \left|\sum_{j=1}^{n}x_jy_j\right| &\leq \Vert x\Vert_p\Vert y\Vert_q
        \end{align*}
      \item[Minkowski's Inequality:] Given $x,y\in\mathbb{F}^n$, $1\leq p \leq \infty$, $\frac{1}{p} = \frac{1}{q} = 1$,
        \begin{align*}
          \Vert x + y\Vert_p &\leq \Vert x \Vert_p + \Vert y \Vert_p
        \end{align*}
      \item[Proof of Minkowski's Inequality:] We can verify for $p = 1, q = \infty$, and vice versa.\\

        Assume $1 < p < \infty$. Then,
        \begin{align*}
          \Vert x + y\Vert_p^p &= \sum_{j=1}^{n}|x_j + y_j|^p\\
                               &=\sum_{j=1}^{\infty}|x_j + y_j||x_j + y_j|^{p-1}\\
                               &\leq \sum_{j=1}^{\infty}|x_j||x_j + y_j|^{p-1} + \sum_{j=1}^{n}|y_j||x_j + y_j|^{p-1}\\
                               &\leq \left(\sum_{j=1}^{n}|x_j|^p\right)^{1/p}\left(\sum_{j=1}^{n}|x_j + y_j|^{pq-q}\right)^{1/q} + \left(\sum_{j=1}^{n}|y_j|^{p}\right)^{1/p}\left(\sum_{j=1}^{n}|x_j + y_j|^{pq-q}\right)^{1/q}\tag*{Hölder's Inequality}\\
                               &= \Vert x \Vert_p \Vert x+y\Vert_{p}^{p/q} + \Vert y \Vert_p\Vert x+y\Vert_{p}^{p/q}\\
                               &= \left(\Vert x \Vert_p + \Vert y \Vert_p\right) \Vert x + y \Vert_{p}^{p-1}
        \end{align*}
        Divide by $\Vert x  + y \Vert_{p}^{p-1}$ to get desired inequality.
    \end{description}
  \item $\ell_{\infty}(\Omega,\mathbb{F})$ with $\Vert \cdot \Vert_u$. This includes subspaces that inherit the norm, such as
    \begin{align*}
      C([a,b]) \subseteq \ell_{\infty}(\Omega)\\
      \ell_{\infty}(\R) \supseteq C_0(\R) \supseteq C_{C}(\R)
    \end{align*}
    \begin{description}
      \tiny
      \item[Exercise:] Show that $C_0(\R) \subseteq \ell_{\infty}(\R)$ is a subspace.
    \end{description}
  \item $\Omega = \N$, $\ell_{\infty} = \ell_{\infty}(\N)$ with $\Vert \cdot \Vert_{\infty}$. Subspaces that inherit the norm are
    \begin{align*}
      c_{00}\subseteq c_0 \leq \ell_{\infty}.
    \end{align*}
  \item $\ell_1$ with $\Vert \cdot \Vert_1$,
    \begin{align*}
      \Vert (a_k)_k\Vert_1 &= \sum_{k=1}^{n}|a_k|.
    \end{align*}
  \item $C([a,b])$ with 
    \begin{align*}
      \Vert f \Vert_1 &= \int_{a}^{b}|f(x)|dx.
    \end{align*}
  \item Let $1\leq p < \infty$.
    \begin{align*}
      \ell_{p} &= \left\{(a_k)_{k=1}^{\infty} \mid \sum_{k=1}^{\infty}|a_k|^{p} < \infty\right\}
    \end{align*}
    is a normed space with
    \begin{align*}
      \Vert (a_k)_k\Vert_{p} &= \left(\sum_{k=1}^{\infty}|a_k|^{p}\right)^{1/p}
    \end{align*}
    We will show that the triangle inequality holds for this norm.
    \begin{align*}
      \left(\sum_{k=1}^{n}|a_k + b_k|^{p}\right)^{1/p} &= \left\Vert \begin{bmatrix}a_1 + b_1\\\vdots\\a_n + b_n\end{bmatrix}\right\Vert_{\ell_{p}^{n}}\\
                                                       &= \left\Vert \begin{bmatrix}a_1 \\\vdots\\a_n\end{bmatrix} + \begin{bmatrix}b_1\\\vdots\\b_n\end{bmatrix}\right\Vert_{\ell_{p}^{n}}\\
                                                       &\leq \left\Vert \begin{bmatrix}a_1\\\vdots\\a_n\end{bmatrix}\right\Vert + \left\Vert \begin{bmatrix}b_1\\\vdots\\b_n\end{bmatrix}\right\Vert_{\ell_{p}^{\infty}}\\
                                                       &\leq \Vert (a_k)_k\Vert_p + \Vert(b_k)_k\Vert_{p}.
    \end{align*}
    Taking the limit as $n\rightarrow\infty$ (by the definition of an infinite series), we find that $\Vert (a_k)_k + (b_k)_k\Vert_p \leq \Vert (a_k)_k\Vert_p + \Vert (b_k)_k\Vert_p$.
  \item $\text{BV}([a,b]) = \{f: [a,b]\rightarrow \R\mid \text{Var}(f) < \infty\}$ with the norm $\Vert f \Vert_{\text{BV}} = |f(a)| + \text{Var}(f)$ is a normed space:
    \begin{align*}
      \Vert f \Vert_{\text{BV}} &= 0\\
      |f(a)| &= 0\\
      \text{Var}(f) &= 0\\
      \shortintertext{given $t\in (a,b]$, look at the partition $a < t \leq b$. Then,}
      \text{Var}(f) &\geq |f(t)-f(a)| + |f(b)-f(t)|\\
      f(t) &= 0\\
      f &= \mathbb{0}_f.
    \end{align*}
  \item $\mathbb{M}_{m,n}(\mathbb{F})$ with
    \begin{align*}
      \Vert a \Vert_{\text{op}} &= \sup_{\Vert \xi \Vert_{\ell_{2}^{n}} \leq 1} \Vert a\xi \Vert_{\ell_{2}^{m}}
    \end{align*}
    is a normed vector space. If $\Vert a \Vert_{\text{op}} = 0$, then
    \begin{align*}
      ae_j &= 0 \tag*{$\forall j \in \{1,\dots,n\}$.}\\
      \shortintertext{take the dot product with $i\neq j$}
      ae_{j}\cdot e_{i} &= a_{ij}\\
                        &= 0\\
                        \shortintertext{so $a_{ij} = 0$ for all $a_{ij}$, so $a$ is the $\mathbb{0}$ matrix.}
    \end{align*}
  \item Let $V,W$ be vector spaces over $\mathbb{F}$. Then, $\mathcal{L}(V,W) = \{T\mid T:V\rightarrow W\text{ linear}\}$, where $T(\alpha v_1 + \beta v_2) = \alpha T(v_1) + \beta T(v_2)$.\\

    $\mathcal{L}(V,W)$ is a vector space with operations
    \begin{align*}
      (T+S)(v) &= T(v) + S(v)\\
      (\alpha T)(v) &= \alpha T(v).
    \end{align*}
    \begin{description}
      \item[Notation:] $\mathcal{L}(V) := \mathcal{L}(V,V)$ is all linear operators on $V$. $\mathcal{L}(V,\mathbb{F}) = V'$ is all linear functionals.
    \end{description}
    Suppose $V$ and $W$ are normed vector spaces. If $T: V\rightarrow W$, set
    \begin{align*}
      \Vert T\Vert_{\text{op}} &:= \sup_{\Vert v \Vert_{v}\leq 1}\Vert T(v)\Vert_{W},\\
      \mathbb{B}(V,W) &= \{T\in \mathcal{L}(V,W) \mid \Vert T \Vert_{op}\leq \infty\},
    \end{align*}
    where $\mathbb{B}(V,W)$ is referred to as the set of all bounded linear maps from $V$ to $W$. $\mathbb{B}(V,W)$ with $\Vert \cdot \Vert_{\text{op}}$ is a normed space.
    \begin{itemize}
      \item Homogeneity:
        \begin{align*}
          \Vert \alpha T \Vert_{\text[op]} &= \sup_{\Vert v \Vert_{V} \leq 1} \Vert \alpha T(v)\Vert_{W}\\
                                           &= \sup_{\Vert v \Vert_{V}\leq 1} |\alpha|\Vert T(v)\Vert_{W}\\
                                           &= |\alpha|\sup_{\Vert v \Vert_{V}\leq 1}\Vert T(v)\Vert_{W}\\
                                           &= |\alpha|\Vert T\Vert_{\text{op}}.
        \end{align*}
      \item Triangle Inequality: for $\Vert v \Vert_{V}\leq 1$,
        \begin{align*}
          \Vert \left(T + S\right)(v)\Vert_{W} &= \Vert T(v) + S(v)\Vert_{W}\\
                                               &\leq \Vert T(v)\Vert_{W} + \Vert S(v)\Vert_{W}\\
                                               &\leq \Vert T \Vert_{\text{op}} + \Vert S \Vert_{\text{op}}\\
                                               \shortintertext{so}
          \Vert T+S\Vert_{\text{op}} &= \sup_{\Vert v \Vert\leq 1}\Vert T+S(v)\Vert\\
                                     &\leq \Vert T \Vert_{\text{op}} + \Vert S \Vert_{\text{op}}
        \end{align*}
      \item Positive Definite: If $\Vert T \Vert_{\text{op}} = 0$, then $T(v) = 0$ for all $v\in V$, $\Vert v \Vert\leq 1$.\\

        Let $v\in V$, $v\neq 0$. Then, $\frac{v}{\Vert v \Vert}\in B_V.$
        \begin{align*}
          T\left(\frac{v}{\Vert v\Vert}\right) &= 0\\
          \frac{1}{\Vert v \Vert}T(v) &= 0\\
          T(v) &= 0
        \end{align*}
    \end{itemize}
    \begin{description}
      \item[Special Cases:] $\mathbb{B}(V) = \mathbb{B}(V,V)$, $V^{\ast} = \mathbb{B}(V,\mathbb{F})$.
      \item[Exercise:] $\mathcal{L}(\mathbb{F}^n,\mathbb{F}^m) = \mathbb{B}(\ell_{2}^{n},\ell_{2}^{m}).$
    \end{description}
    \item Inner Product Spaces (expanded upon below).
\end{enumerate}
  \subsubsection{Inner Product Spaces}%
  An inner product on a vector space $V$ is a pairing 
  \begin{align*}
    V \times V \xrightarrow{\langle\cdot,\cdot\rangle} \mathbb{F}
  \end{align*}
  that satisfies
  \begin{enumerate}[(i)]
    \item $\displaystyle \langle v_1+v_2,w\rangle = \langle v_1,w\rangle + \langle v_2,w\rangle$, $\langle \alpha v,w\rangle = \alpha \langle v,w\rangle$.
    \item $\langle v,w\rangle = \overline{\langle w,v\rangle}$
    \item $\langle v,v\rangle \geq 0$.
    \item If $\langle v,v\rangle = 0$, then $v = 0$.
  \end{enumerate}
  The pair $(V,\langle\cdot,\cdot\rangle)$ is known as an inner product space.
  \begin{description}
    \item[Remarks:] $\langle v,w_1+w_2\rangle = \langle v,w_1\rangle + \langle v,w_2\rangle$, $\langle v,\alpha w \rangle = \overline{\alpha}\langle v,w\rangle$.
  \end{description}
  If $\langle\cdot,\cdot\rangle$ is an inner product on a linear space $V$, then set 
  \begin{align*}
    \Vert v\Vert_{2} &:= \langle v,v\rangle^{1/2}.
  \end{align*}
  \begin{description}
    \item[Exercise:] $\Vert \alpha v \Vert_{2} = |\alpha|\Vert v_2\Vert$, $\Vert v \Vert_2 = 0 \Rightarrow v = 0$.
  \end{description}
  $v,w\in (V,\langle,\cdot,\cdot\rangle)$ are \textit{orthogonal} if $\langle v,w\rangle = 0$.\\

  The Pythagoran theorem states that for $v_1,\dots,v_n\in V$ mutually orthogonal, then
  \begin{align*}
    \left\Vert\sum_{i=1}^{n}v_i\right\Vert^{2} &= \sum_{j=1}^{n}\Vert v_j\Vert^2.
  \end{align*}
  For two vectors $v,w\in V$, $P_{w}(v) = \frac{\langle v,w\rangle}{\langle w,w\rangle}w$.
  \begin{description}
    \item[Exercise:] Check that $\iprod{P_{w}(v)}{v-P_{w}(v)}$, meaning
      \begin{align*}
        \Vert v \Vert^2 &= \Vert P_w(v)\Vert^2 + \Vert v-P_{w}(v)\Vert ^2
      \end{align*}
    \item[Cauchy-Schwarz Inequality:] In any inner product space,
      \begin{align*}
        |\iprod{v}{w}| &\leq \Vert v \Vert \cdot \Vert w \Vert.
      \end{align*}
    \item[Proof of Cauchy-Schwarz:] From the exercise,
      \begin{align*}
        \Vert v \Vert &\geq \Vert P_w(v)\Vert\\
        \Vert v \Vert &\geq \left\Vert \frac{\iprod{v}{w}}{\iprod{w}{w}}w\right\Vert\\
                      &= \frac{|\iprod{v}{w}|}{\Vert w \Vert^2}\Vert w \Vert\\
                      \shortintertext{therefore,}
        \Vert v \Vert \Vert w \Vert &\geq |\iprod{v}{w}|
      \end{align*}
  \end{description}
  The triangle inequality follows from the Cauchy-Schwarz inequality.
  \begin{description}
    \item[Proof of Triangle Inequality:]
      \begin{align*}
        \Vert v + w \Vert_2^2 &= \iprod{v+w}{v+w}\\
                              &= \iprod{v}{v} + \iprod{v}{w} + \iprod{w}{v} + \iprod{w}{w}\\
                              &= \Vert v \Vert^2 + \Vert w \Vert^2 + \iprod{v}{w} + \overline{\iprod{v}{w}}\\
                              &= \Vert v \Vert^2 + \Vert w \Vert^2 + 2\text{Re}\iprod{v}{w}\\
                              &\leq \Vert v\Vert^2 + \Vert w \Vert^2 + 2|\iprod{v}{w}|\\
                              &\leq \Vert v \Vert^2 + \Vert w \Vert^2 + 2\Vert v \Vert\Vert w \Vert\tag*{Cauchy-Schwarz Inequality}\\
                              &= \left(\Vert v \Vert + \Vert w \Vert\right)^2.
      \end{align*}
      Take square roots on both sides.
  \end{description}
  \begin{enumerate}[(1)]
    \item $\ell_{2}^{n}=\mathbb{F}^n$ with
      \begin{align*}
        \iprod{ \begin{bmatrix}x_1\\\vdots\\x_n\end{bmatrix} }{ \begin{bmatrix}y_1\\\vdots\\y_n\end{bmatrix} } &= \sum_{i=1}^{n}x_i\overline{y_i}.
      \end{align*}
      Cauchy-Schwarz is found as
      \begin{align*}
        \left|\sum_{j=1}^{n}x_j\overline{y_j}\right| &\leq \left(\sum_{j=1}^{n}|x_j|^2\right)^{1/2}\left(\sum_{j=1}^{n}|y_j|^2\right)^{1/2}.
      \end{align*}
    \item $\ell_2$ with
      \begin{align*}
        \iprod{(a_j)_j}{(b_j)_{j}} &= \sum_{j=1}^{\infty}a_j\overline{b}_j.
      \end{align*}
      We can see that for any finite $n$, the Cauchy-Schwarz inequality in $\ell_{2}^{n}$ states
      \begin{align*}
        \left|\sum_{j=1}^{n} a_j\overline{b_j}\right| &\leq \left(\sum_{j=1}^{n}|a_j|^2\right)^{1/2} \left(\sum_{j=1}^{n}|b_j|^2\right)^{1/2}\\
                                                      &\leq \left(\sum_{j=1}^{\infty}|a_j|^2\right)^{1/2}\left(\sum_{j=1}^{\infty}|b_j|^2\right)^{1/2}.
      \end{align*}
    Taking the limit as $n\rightarrow\infty$, we see that $\iprod{(a_j)_j}{(b_j)_j}$ is convergent.
    \item $C([a,b])$ with
      \begin{align*}
        \iprod{f}{g} &= \int_{a}^{b}f(x)\overline{g(x)}dx.
      \end{align*}
    \item Let $V = \mathbb{M}_{n}(\mathbb{C})$.\\

      Recall that if
      \begin{align*}
        a &= (a_{ij})_{i,j},
      \end{align*}
      then
      \begin{align*}
        a^{\ast} &= (\overline{a_{ji}})_{i,j}.
      \end{align*}
      Let $\text{Tr}:\mathbb{M}_{n}(\mathbb{C}) \rightarrow \mathbb{C}$, $\text{Tr}((a_{ij})) = \sum_{i=1}^{n}a_{ii}$.
      \begin{itemize}
        \item $\text{Tr}(I_{n}) = n$
        \item $\text{Tr}(a + \alpha b) = \text{Tr}(a) + \alpha\text{Tr}(b)$
        \item $\text{Tr}(ab) = \text{Tr}(ba)$
      \end{itemize}
      Then, if $\text{Tr}(a^{\ast}a) = 0$, then $a = \mathbb{0}_{\mathbb{M}_n}$.
      \begin{align*}
        a^{\ast}a &= (\overline{a_{ji}})_{i,j}(a_{ij})_{i,j}\\
                  &= \left(\sum_{k=1}^{n}\overline{ki}a_{kj}\right)_{i,j}\\
        \text{Tr}(a^{\ast}a) &= \sum_{i=1}^{n}\sum_{k=1}^{n}\overline{a_{ki}}a_{ki}\\
                             &= \sum_{i,k=1}^{n}|a_{ki}|^{2}\\
                             &= \sum_{i,j=1}^{n}|a_{ij}|^{2}.
      \end{align*}
      If $\text{Tr}(a^{\ast}a) = 0$, then $a_{ij} = 0$ for all $i,j$.\\

      We define
      \begin{align*}
        \iprod{a}{b}_{\text{HS}} &= \text{Tr}(b^{\ast}a).
      \end{align*}
      \begin{enumerate}[(i)]
        \item $(b_1 + b_2)^{\ast} = b_1^{\ast} + b_2^{\ast}$
        \item $(\alpha b)^{\ast} = \overline{\alpha}b^{\ast}$
        \item $(b_1b_2)^{\ast} = b_2^{\ast}b_1^{\ast}$
        \item $b^{\ast\ast} = b$
      \end{enumerate}
      The norm is defined as
      \begin{align*}
        \norm{a}_{\text{HS}} &= \iprod{a}{a}^{1/2}\\
                             &= \text{Tr}(a^{\ast}a)^{1/2}\\
                             &= \left(\sum_{i,j=1}^{n}|a_{ij}|^2\right)^{1/2}
      \end{align*}
  \end{enumerate}
  \section{Metric Spaces}%
  We looked at normed spaces, where we attach a length $\norm{v}$ to very vector $v$. We can also speak of the distance between two vectors, defined as $d(v,w) = \norm{v-w}$.\\

  Notice that the following hold:
  \begin{itemize}
    \item $d(v,w) \geq 0$
    \item 
      \begin{align*}
        d(v,w) &= \norm{v-w}\\
               &= \norm{(-1)(w-v)}\\
               &= |-1|\norm{w-v}\\
               &= \norm{w-v}
      \end{align*}
    \item 
      \begin{align*}
        d(u,w) &= \norm{u-w}\\
               &= \norm{u-v+v-w}\\
               &\leq \norm{u-v} + \norm{v-w}\\
               &= d(u,v) + d(v,w).
      \end{align*}
    \item $d(v,v) = \norm{v-v} = 0$. If $d(v,w) = 0$, then $\norm{v-w} = 0$, so $v-w = \mathbb{0}$, so $v = w$.
  \end{itemize}
  In Real Analysis I, we studied the properties (such as convergence, limits, and continuity) of a particular normed vector space, namely $(\R,|\cdot|)$. We will expand these concepts to all metric spaces.\\

  \subsection{Definition of a Metric Space}%
  Let $X$ be a non-empty set. A \textbf{metric} on $X$ is a map
  \begin{align*}
    d: X\times X \rightarrow \R^{+}\\
    (x,y)\mapsto d(x,y)\geq 0
  \end{align*}
  such that
  \begin{enumerate}[(i)]
    \item Symmetry: $d(x,y) = d(y,x)$ for all $x,y\in X$.
    \item Triangle Inequality: $d(x,z) \leq d(x,y) + d(y,z)$ for all $x,y,z\in X$.
    \item Zero Distance: $d(x,x) = 0$
    \item Definite: $d(x,y) = 0 \Rightarrow x = y$
  \end{enumerate}
  If $d$ satisfies (i), (ii), and (iii), then $d$ is called a semi-metric. If $d$ satisfies (iv) as well, then $d$ is a metric.\\

  If $d$ is a (semi-)metric on $X$, the pair $(X,d)$ is called a (semi-)metric space.\\

  Two metrics, $d$ and $\rho$, on $X$, are equivalent if $\exists c_1,c_2 \geq 0$ such that $d(x,y) \leq c_1\rho(x,y)$ and $\rho(x,y) \leq c_2 d(x,y)$ for all $x,y$.
  \subsection{Examples of Metric Spaces}%
  \begin{enumerate}[(1)]
    \item Discrete Metric:
      \begin{align*}
        d(x,y) &= \begin{cases}
          1 & x\neq y\\
          0 & x=y
        \end{cases}
      \end{align*}
      for $X$ any set.
    \item Hamming distance: between two bit strings of equal length. Let
      \begin{align*}
        X &= \{0,1\}^{n}\\
          &= \{0,1\} \underbrace{\times \cdots \times }_{\text{$n$ times}}\{0,1\}\\
        d_{H}((x_j)_{1}^{n},(y_j)_{1}^{n}) &= \left|\{j\mid x_j\neq y_j\}\right|.
      \end{align*}
    \item Any normed space $(V,\norm{\cdot})$ is a metric space.
      \begin{align*}
        d(v,w) &= \norm{v-w}.
      \end{align*}
      \begin{description}
        \item[Exercise:] Show that if two norms are equivalent, their induced metrics are equivalent.
      \end{description}
    \item Subset of Metric Space: If $(X,d)$ is a metric space, and $Y\subseteq X$ is non-empty. Then, $(Y,d)$ is a metric space.
    \item Paris metric: let $(X,\rho)$ be a metric space. Let $p\in X$ be a fixed point.
      \begin{align*}
        \rho(x,y) &:= \begin{cases}
          0 & x=y\\
          \rho(x,p) + \rho(p,y) & x\neq y
        \end{cases}
      \end{align*}
    \item Bounded metric: Let $\rho$ be a (semi-)metric on $X$. Set
      \begin{align*}
        d(x,y) &= \frac{\rho(x,y)}{1 + \rho(x,y)}.
      \end{align*}
      We claim that $d$ is a (semi-)metric. Notice that $0\leq d(x,y) \leq 1$.
      \begin{description}
        \item[Proof:] Clearly, $d(x,y) = d(y,x)$. Additionally, $d(x,x) = 0$. If $d(x,y) = 0$ and $\rho$ is a metric, then $\rho(x,y) = 0$, so $x = y$.\\

          To show the triangle inequality, we examine the function
          \begin{align*}
            f(t) &= \frac{t}{1+t}\\
            f'(t) &= \frac{1}{(1+t)^2} > 0.
          \end{align*}
          Since $\rho$ satisfies the triangle inequality, $\rho(x,z) \leq \rho(x,y) + \rho(y,z)$. Apply $f$ on both sides. Then,
          \begin{align*}
            \underbrace{\frac{\rho(x,z)}{1 + \rho(x,z)}}_{d(x,z)} &\leq \frac{\rho(x,y) + \rho(y,z)}{1 + (\rho(x,y) + \rho(y,z))}\\
                                            &= \frac{\rho(x,y)}{1+\rho(x,y) + \rho(y,z)} + \frac{\rho(y,z)}{1+\rho(x,y) + \rho(y,z)}\\
                                            &\leq \underbrace{\frac{\rho(x,y)}{1 + \rho(x,y)}}_{d(x,y)} + \underbrace{\frac{\rho(y,z)}{1 + \rho(y,z)}}_{d(y,z)}.
          \end{align*}
      \end{description}
    \item If $d_1,\dots,d_n$ are metrics on $X$, $c_1,\dots,c_n \geq 0$. Then, 
      \begin{align*}
        d(x,y) &= \sum_{k=1}^{n}c_kd_k(x,y)
      \end{align*}
       is a metric.
     \item Let $\{\rho_k\}_{k=1}^{\infty}$ be a family of semi-metrics. Assume the family is separating --- for all $x\neq y$, there exists $k$ such that $\rho_k(x,y) \neq 0$.\\

       Let $d_k$ be defined as
       \begin{align*}
         d_k(x,y) &= \frac{\rho_k(x,y)}{1 + \rho_k(x,y)}.
       \end{align*}
       Note that $\{d_k\}_{k=1}^{\infty}$ is also separating.\\

       Then, 
       \begin{align*}
         d(x,y) &= \sum_{k=1}^{\infty}2^{-k}d_k(x,y)
       \end{align*}
      is a metric.\\

    We will now define the Frechet Metric using this method. Let $X = C(\R)$. For each $k = 1,2,3,\dots$, set $p_k(f) = \sup_{x\in[-k,k]}|f(x)|$.\\

      We can verify that $p_k$ defines a seminorm. We can then check $\rho_k(f,g) = p_k(f-g)$ is a semi-metric. \\

      We claim that $\{\rho_k\}$ is separating: if $f\neq g$, then there exists $x_0\in \R$ with $f(x_0)\neq g(x_0)$. Since $f$ and $g$ are continuous, there is a neighborhood $[x_0-\delta,x_0+\delta]$ such that $f(x)\neq g(x)$ for all $x\in [x_0-\delta,x_0+\delta]$. Find $k$ such that $[x_0-\delta,x_0+\delta]\subseteq [-k,k]$. Then, $\rho_k(f-g) > 0$.\\

      Construct $d_k$ as above, and then $d$ as follows:
      \begin{align*}
        d_{\text{F}} &= \sum \frac{2^{-k}p_k(f-g)}{1 + p_k(f-g)}
      \end{align*}
    \item Product of metric spaces: let $(X_k,\rho_k)_{k=1}^{\infty}$ be a countable family of metric spaces. For each $k$, let
      \begin{align*}
        d_k(x,y) &= \frac{\rho_k(x,y)}{1 + \rho_k(x,y)}.
      \end{align*}
      \begin{description}
        \item[Remark:] If the $\rho_k$ are already uniformly bounded, let $d_k = \rho_k$.
      \end{description}
      Let 
      \begin{align*}
        X &= \prod_{k=1}^{\infty}X_k\\
          &= \left\{(x_k)_k\mid x_k\in X_k\right\}\\
          &= \left\{f: \N\rightarrow \bigsqcup_{k=1}^{\infty}X_k\mid f(k)\in X_k\right\}.
      \end{align*}
      Define $D: X\times X \rightarrow [0,\infty)$ as
      \begin{align*}
        D(x,y) &= \sum_{k=1}^{\infty}2^{-k}\rho_k(x_k,y_k),\\
        D(f,g) &= \sum_{k=1}^{\infty}2^{-k}\rho(f(k),g(k)).
      \end{align*}
      For example, for each $k$, let $X_k = \{0,1\}$ with the discrete metric. Let 
      \begin{align*}
        \Delta &= \prod_{k\in\N}\{0,1\}\\
               &= \left\{(x_k)_k\mid x_k\in \{0,1\}\right\}\\
        D(x,y) &= \sum_{k=1}^{\infty}2^{-k}|x_k-y_k| \tag*{$(x_k)_k,(y_k)_k\in \Delta$.}
      \end{align*}
      $\Delta$ is known as the abstract Cantor set; every compact metric space is a surjective image of the abstract Cantor set.
    \item Geodesic Distance: let $\iprod{\cdot}{\cdot}$ be the standard dot product on $\R^3(\R^n)$, then
      \begin{align*}
        S^{2} &= \left\{x\in\R^3\mid \norm{x}_2 = 1\right\}\\
        S^{n-1} &= \left\{x\in\R^{n}\mid \norm{x}_2 = 1\right\}.
      \end{align*}
      To find the geodesic distance, we take $d(x,y) = \arccos(\iprod{x}{y})$. We claim $d$ is a metric.
      \begin{itemize}
        \item Symmetry: self-evident.
        \item $d(x,x) = \arccos(1) = 0$. Suppose $d(x,y) = 0$. Then, $\langle x,y\rangle = 1$, meaning $\norm{x-y}^2 = 0$, so $x = y$.
        \item Let $\theta = \arccos(\iprod{x}{y})$, $\varphi = \arccos(\iprod{y}{z})$, where $\theta,\varphi \in [0,\pi]$.
          \begin{align*}
            p_x &= \frac{\iprod{x}{y}}{\iprod{y}{y}} y\\
              &= \cos(\theta) y\\
            x &= \cos(\theta) y + \sin(\theta) u\\
            \intertext{where}
            u &= \frac{x-p_x}{\norm{x-p_x}}.\\
            \intertext{Similarly, we can take}
            z &= \cos(\varphi) y + \sin(\varphi) v\\
            \intertext{where}
            v &= \frac{z-p_{z}}{\norm{z-p_{z}}}.
            \intertext{So,}
            \iprod{x}{z} &= \cos(\theta)\cos(\varphi) + \sin(\theta)\sin(\varphi)\iprod{u}{v}\\
                         &\geq \cos(\theta)\cos(\varphi) - \sin(\theta)\sin(\varphi) \tag*{$\iprod{u}{v} \geq -1$}\\
                         &= \cos(\theta + \varphi).
                         \intertext{Since $\arccos$ is decreasing,}
            \arccos(\iprod{x}{z}) &\leq \arccos(\cos(\theta + \varphi))\\
                                  &= \theta + \varphi\\
                                  &= \arccos(\iprod{x}{y}) + \arccos(\iprod{y}{z}).
          \end{align*}
          Therefore, $d(x,y) \leq d(x,y) + d(y,z)$.
        \item Let $\Gamma = (V,E)$ be a simple connected graph. We define $d: V\times V\rightarrow [0,\infty)$ to be the length of the shortest path between vertices $u$ and $v$.
          \begin{description}
            \item[Exercise:] Show this is a metric.
          \end{description}
      \end{itemize}
    \item Let $(X,d)$ be any metric space. If $E\subseteq X$, define $\text{diam}(E) = \sup_{x,y\in E}d(x,y)$. $E$ is bounded if $\text{diam}(E) < \infty$.
      \begin{description}
        \item[Exercise:] If $(V,\norm{\cdot})$ is a normed space, $E\subseteq V$ is a subset, show the following are equivalent:
          \begin{enumerate}[(i)]
            \item $E$ is bounded (in the metric sense)
            \item $\sup_{v\in E}\norm{v} < \infty$
            \item $\exists r > 0$ such that $E\subseteq r B_V$.
          \end{enumerate}
      \end{description}
      Let $\Omega$ be any set. The function $f:\Omega\rightarrow X$ is bounded if $f(\Omega)\subseteq X$ is bounded. We let.
      \begin{align*}
        \text{Bd}(\Omega,X) = \{f:\Omega\rightarrow X\mid f\text{ is bounded}\}.
      \end{align*}
      \begin{description}
        \item[Remark:] $\text{Bd}(\Omega,\mathbb{F}) = \ell_{\infty}(\Omega,\mathbb{F})$.
      \end{description}
    \item $\text{Bd}(\Omega,X)$ with 
      \begin{align*}
        D_u(f,g) &= \sup_{x\in\Omega}d(f(x),g(x)).
      \end{align*}
      \begin{description}
        \item[Exercise:] Show that $D_u$ defines a metric.
      \end{description}
      Consider $\text{Bd}(\Omega,\mathbb{F}) = \ell_{\infty}$. Look at the subset
      \begin{align*}
        E &= \{f\in \text{Bd}(\Omega,\mathbb{F})\mid f(x) \in \{0,1\}\}.
      \end{align*}
      Then,
      \begin{align*}
        D_u(f,g) &= \sup_{x\in\Omega}|f(x)-g(x)|.\\
                 &= \begin{cases}
                   1 & f\neq g\\
                   0 & f = g
                 \end{cases}.
      \end{align*}
      When we take a particular subset of $D_u(f,g)$, we find that we get the discrete metric.
  \end{enumerate}
  Taking an overview of the concepts we have learned so far, we see
  \begin{align*}
    \text{Inner Product Spaces}\subseteq \text{Normed Vector Spaces} \subseteq \text{Metric Spaces}
  \end{align*}
  \section{Topology of Metric Spaces}%
  Throughout this section, let $(X,d)$ be a metric space.
  \begin{enumerate}[(1)]
    \item Let $x_0\in X$, $\delta > 0$. 
      \begin{enumerate}[(i)]
          \item We say
            \begin{align*}
              U(x_0,\delta) &= \left\{x\in X\mid d(x,x_0)<\delta\right\}
            \end{align*}
            is the open ball centered at $x_0$ with radius $\delta$.
          \item We say
            \begin{align*}
              B(x_0,\delta) &= \left\{x\in X\mid d(x,x_0)\leq \delta\right\}
            \end{align*}
            is the closed ball.
          \item We say
            \begin{align*}
              S(x_0,\delta) &= \left\{x\in X\mid d(x,x_0) = \delta\right\}
            \end{align*}
            is the sphere.
      \end{enumerate}
    \item $U\subseteq X$ is open if
      \begin{align*}
        (\forall x\in U)(\exists \delta > 0) \text{ such that } U(x,\delta)\subseteq U.
      \end{align*}
      Let
      \begin{align*}
        \tau_{X} &= \left\{U\subseteq X\mid U\text{ open}\right\}\\
                 &\subseteq \mathcal{P}(X).
      \end{align*}
    \item $D\subseteq X$ is closed if $D^{c}$ is open.
    \item If $x\in U\in \tau_X$, then $U$ is called an open neighborhood of $x$. If $x\in U\subseteq N$, where $U\in \tau_X$, then $N$ is a neighborhood of $x$.
      \begin{align*}
        \mathcal{N}_x &= \left\{N\mid N\text{ is a neighborhood of }x\right\}
      \end{align*}
    \item Let $A\subseteq X$. The interior of $A$ is
      \begin{align*}
        A^{\circ} &= \bigcup \left\{V\mid V\subseteq A, V\text{ open}\right\}.
      \end{align*}
      The closure of $A$ is
      \begin{align*}
        \overline{A} &= \bigcap \left\{D\mid A\subseteq D, D\text{ closed}\right\}.
      \end{align*}
      The boundary of $A$ is
      \begin{align*}
        \partial A &= \overline{A} \setminus A^{0}.
      \end{align*}
      \begin{description}
        \item[Exercise:] $\overline{A^{c}} = (A^{\circ})^{c}$, $(\overline{A})^{c} = (A^{c})^{\circ}$.
      \end{description}
  \end{enumerate}
  \begin{description}
    \item[Remarks:] $A^{\circ}$ is the largest open set contained in $A$. So, if $V$ is open and $V\subseteq A$, then $V\subseteq A^{\circ}$. Similarly, $\overline{D}$ is the smallest closed set containing $D$. If $C$ is closed and $D\subseteq C$, then $\overline{D}\subseteq C$.
  \end{description}
  \begin{itemize}
    \item For example, $(a,b]^{\circ} = (a,b)$. This is because $(a,b)$ is open and contained in $(a,b]$, so $(a,b)\subseteq (a,b]^{\circ}$.
    \item We will show that $\overline{A^{c}} \subseteq (A^{\circ})^c$.
      \begin{align*}
        A^{\circ}&\subseteq A\\
        (A^{\circ})^{c} &\supseteq A^{c}
        \intertext{The union of open sets is open, so $A^{\circ}$ is open, so $(A^{\circ})^{c}$ is closed by definition. Therefore,}
        (A^{\circ})^{c} \supseteq \overline{A^{c}}.
      \end{align*}
  \end{itemize}
  \subsection{Topology of Open Sets in a Metric Space}%
  The open sets $\tau_X$ form a topology:
  \begin{enumerate}[(i)]
    \item $\emptyset,X \in \tau_X$.
    \item If $\{V_i\}_{i\in I}\subseteq \tau_x$, then
      \begin{align*}
        \bigcup_{i\in I}V_i \in \tau_X.
      \end{align*}
    \item If $V_i,\dots,V_n\in \tau_X$, then
      \begin{align*}
        \bigcap_{i=1}^{n}V_i\in \tau_X.
      \end{align*}
      \begin{description}
        \item[Remark:] This is only true of finite intersections. For a counterexample, if $V_n = (-1/n,1/n)\subseteq \R$ with the Euclidean metric, then the infinite intersection yields $\{0\}$, which is closed in $\R$ with the Euclidean metric.
      \end{description}
  \end{enumerate}
  \begin{description}
    \item[Proof:]\hfill
      \begin{enumerate}[(1)]
        \item Clearly, $\emptyset$ (by vacuous truth) and $X$ are open.
        \item Let $x\in \bigcup_{i\in I}V_i$. Then, $\exists i_0\in I$ with $x\in V_{i_0}$. Since $V_{i_0}$ is open, $\exists \varepsilon > 0$ such that $U(x,\varepsilon)\subseteq V_{i_0}\subseteq \bigcup V_i$.
        \item Let $x\in \bigcap_{i=1}^{n} V_i$. Then, $x\in V_i$ for all $i\in 1,\dots,n$. Since each $V_i$ is open, $\exists \varepsilon_1,\dots,\varepsilon_n$ with $U(x,\varepsilon_i)\subseteq V_i$ for each $i=1,\dots,n$. Set $\varepsilon = \min\{\varepsilon_i\}_{i=1}^{n}$. Then, $U(x,\varepsilon)\subseteq U(x,\varepsilon_i)\subseteq V_i$ for all $i$. Therefore, $U(x,\varepsilon)\subseteq \bigcap_{i=1}^{n}V_i$.
      \end{enumerate}
  \end{description}
  \begin{description}
    \item[Exercise:] Show all open balls are open. In particular, show all open intervals are open.
    \item[Exercise:] Show the following:
      \begin{enumerate}[(1)]
        \item $X,\emptyset$ are closed.
        \item If $\{C_i\}_{i\in I}$ is a family of closed sets, then $\bigcap_{i\in I} C_i$ is closed.
        \item For $C_1,\dots,C_n$ closed, then $\bigcup_{i=1}^{n}C_i$ is closed.
        \item Closed balls are closed. Spheres are closed.
      \end{enumerate}
  \end{description}
  Let $x\in X$. Recall that $\mathcal{N}_{x}$ is the set of all neighborhoods of $x$.
  \begin{enumerate}[(i)]
    \item $N\in \mathcal{N}_x \Leftrightarrow \exists \delta > 0: U(x,\delta)\in N$
    \item $N\in \mathcal{N}_x,N\subseteq M \Rightarrow M\in \mathcal{N}_x$
    \item $N_1,N_2\in \mathcal{N}_x \Rightarrow N_1\cap N_2 \in \mathcal{N}_x$
  \end{enumerate}
  In this sense, $\mathcal{N}_x$ is a directed set with reverse inclusion.
  \subsection{Pointwise Characterization of Subsets}%
  Let $A\subseteq X$.
  \begin{enumerate}[(i)]
    \item $x\in A^{\circ}\Leftrightarrow \exists \delta > 0: U(x,\delta)\subseteq A$.
    \item $x\in \overline{A} \Leftrightarrow \forall \delta > 0: U(x,\delta)\cap A\neq\emptyset$.
    \item $x\in \partial A \Leftrightarrow \forall \delta > 0: U(x,\delta)\cap A \neq \emptyset$ and $U(x,\delta)\cap A^{c} \neq \emptyset$.
  \end{enumerate}
  \begin{description}
    \item[Proof:] Let $A\subseteq X$
      \begin{enumerate}[(i)]
        \item
          \begin{align*}
            x\in A^{\circ} &\Leftrightarrow x\in \bigcup_{\substack{V\in \tau_X\\V\subseteq A}}V\\
                       &\Leftrightarrow \exists V\in \tau_X, V\subseteq A, x\in V\\
                       &\Leftrightarrow \exists \delta > 0: U(x,\delta)\subseteq A.
          \end{align*}
        \item 
          \begin{align*}
            x\notin \overline{A} &\Leftrightarrow x\in (\overline{A})^{c}\\
                                 &\Leftrightarrow x\in (A^{c})^{\circ}\\
                                 &\Leftrightarrow \exists \delta > 0: U(x,\delta)\subseteq A^{c}\\
                                 &\Leftrightarrow \exists \delta > 0: U(x,\delta)\cap A = \emptyset.
          \end{align*}
          We negate both sides.
        \item
          \begin{align*}
            x\in \partial A &\Leftrightarrow x\in \overline{A}\setminus A^{\circ}\\
                            &\Leftrightarrow x\in \overline{A} \cap (A^0)^{c}\\
                            &\Leftrightarrow x\in \overline{A} \cap \overline{A}^{c}\\
                            &\Leftrightarrow x\in \overline{A} \text{ and } x\in \overline{A}^{c}\\
                            &\Leftrightarrow \forall \delta > 0: U(x,\delta) \cap A \neq \emptyset,U(x,\delta)\cap A^{c} \neq \emptyset\\
          \end{align*}
      \end{enumerate}
  \end{description}
  \begin{description}
    \item[Remark:] $\overline{U(v,\delta)} = B(v,\delta)$ in a normed space. $\partial U(v,\delta) = \partial B(v,\delta) = S(v,\delta)$ in a normed space. Also, $B(v,\delta)^{\circ} = U(v,\delta)$.
    \item[Proof:] We show that $\overline{U}(v,\delta) = B(v,\delta)$. Since $B(v,\delta)$ is closed, and $U(v,\delta) \subseteq B(v,\delta)$, we know $\overline{U(v,\delta)}\subseteq B(v,\delta)$.\\

      Let $w\in B(v,\delta)$. If $\norm{w-v} < \delta$, then $w\in U(v,\delta)$. Assume $\norm{w-v} = \delta$. Let $u_t = (1-t)v + tw$, where $t\in [0,1]$.
      \begin{align*}
        \norm{w-u_t} &= \norm{w - (1-t)v - tw}\\
                     &= \norm{(1-t)(w-v)}\\
                     &= (1-t)\norm{w-v}\\
                     &= (1-t)\delta.
      \end{align*}
      Let $\varepsilon > 0$. Let $t\in (0,1)$ such that $(1-t)\delta < \varepsilon$. Then, $u_t\in U(w,\varepsilon)\cap U(v,\delta)$. Therefore, $w\in \overline{U(v,\delta)}$.
  \end{description}
  \subsection{Unions and Intersections of Closure/Interior}%
  Let $(X,d)$ be a metric space.
  \begin{enumerate}[(i)]
    \item 
      \begin{align*}
        \left(\bigcup_{i\in I} A_i\right)^{\circ} \supseteq \bigcup_{i\in I}A_i^{\circ} \tag*{may be strict}
      \end{align*}
    \item
      \begin{align*}
        \overline{\bigcap_{i\in I} A_i} \subseteq \bigcap_{i\in I}\overline{A_i}
      \end{align*}
    \item 
      \begin{align*}
        \bigcap_{k=1}^{n}A_k^{\circ} &= \left(\bigcap_{k=1}^{n}A_k\right)^0
      \end{align*}
    \item 
      \begin{align*}
        \overline{\bigcup_{k=1}^{n}D_k} &= \bigcup_{k=1}^{n}\overline{D_k}
      \end{align*}
  \end{enumerate}
  \begin{description}
    \item[Proof:]\hfill
      \begin{enumerate}[(i)]
        \item 
          \begin{align*}
            A_i^{\circ} &\subseteq A_i\\
            \bigcup_{i\in I}A_i^{\circ}\subseteq \bigcup_{i\in I}A_i\\
            \bigcup_{i\in I}A_i^{\circ}\subseteq \left(\bigcup_{i\in I}A_i\right)^{\circ}
          \end{align*}
      \end{enumerate}
  \end{description}
  \begin{description}
    \item[Remark:] We claim $\overline{\Q} = \R$ under the absolute value metric. We know that $\Q \subseteq \R$, $\R$ is closed, meaning $\overline{\Q} \subseteq \R$. Let $t\in\R$, $\delta > 0$. We know that $(t-\delta,t+\delta) \cap \Q \neq \emptyset$. Therefore, $t\in \overline{\Q}$. Thus, $\overline{\Q} = \R$.
  \end{description}
  \subsection{Properties of Boundary}%
  Let $A\subseteq X$.
  \begin{enumerate}[(1)]
    \item $\partial A$ is closed.
    \item $\partial A = \partial A^{c}$
    \item $\overline{A} = A \cup \partial A$
    \item $A\setminus \partial A = A^{\circ}$
  \end{enumerate}
  \begin{description}
    \item[Proof:]\hfill
      \begin{enumerate}[(1)]
        \item 
          \begin{align*}
            \partial A &= \overline{A} \setminus A^{\circ}\\
                       &= \overline{A} \cap (A^{\circ})^{c}.
          \end{align*}
        \item Follows from pointwise characterization.
        \item Clearly, $A\cup \partial A \subseteq \overline{A}$. Let $x\in \overline{A}$. If $x\in A$, we're done. Otherwise, $x\in \overline{A} \setminus A \subseteq \overline{A} \setminus A^{\circ} = \partial A$.
        \item 
          \begin{align*}
            A\setminus \partial A &= A\cap (\partial A)^{c}\\
                                  &= A\cap (\overline{A} \setminus A^{\circ})^{c}\\
                                  &= A \cap \left(\overline{A} \cap (A^{\circ})^{c}\right)^{c}\\
                                  &= A\cap \left(\overline{A}^{c} \cup A^{\circ}\right)\\
                                  &= \left(A\cap \overline{A}^{c}\right) \cup \left(A\cap A^{\circ}\right)\\
                                  &= A^{\circ}
          \end{align*}
      \end{enumerate}
  \end{description}
  \subsection{Density and Separability}%
  Let $(X,d)$ be a metric space.
  \begin{enumerate}[(1)]
    \item $A\subseteq X$ is $d$-dense if $\overline{A} = X$.
    \item $N\subseteq X$ is nowhere dense if $(\overline{N})^{\circ} = \emptyset$.
    \item $(X,d)$ is separable if there is a countable dense subset.
  \end{enumerate}
  \begin{description}
    \item[Exercise:] If $N\subseteq X$ is closed, then $N$ is nowhere dense if and only if $N^{c}$ is dense.
    \item[Exercise:] The following are equivalent.
      \begin{enumerate}[(1)]
        \item $A\subseteq X$ is dense.
        \item $\forall \emptyset \neq U\in \tau_X$, $U\cap A\neq \emptyset$.
        \item $\forall x\in X,\forall \varepsilon > 0$, $U(x,\varepsilon)\cap A \neq \emptyset$.
        \item $\forall x\in X, \forall \varepsilon > 0, \exists a\in A$ such that $d(x,a) < \varepsilon$.
      \end{enumerate}
  \end{description}
  Let $X$ be a metric space.
  \begin{enumerate}[(1)]
    \item A base for $\tau_X$ is a family of open subsets $\mathcal{B}$ such that:
      \begin{align*}
        \left(\forall U\in \tau_X\right)(\forall x\in U) \exists B\in \mathcal{B} \text{ such that } x\in B\subseteq U.
      \end{align*}
      Equivalently,
      \begin{align*}
        \forall U\in \tau_X, U= \bigcup_{i\in I}B_i. \tag*{$B_i\in \mathcal{B}$}
      \end{align*}
    \item We say that $(X,d)$ is second countable if $\tau_X$ admits a countable base.
  \end{enumerate}
  \begin{itemize}
    \item For any $(X,d)$ a metric space, $\mathcal{B} = \{U(x,\varepsilon)\mid x\in X, \varepsilon > 0\}$ is a base. Indeed, given any $x\in U\subseteq \tau_X$, by definition, $\exists \varepsilon > 0$ such that $U(x,\varepsilon)\subseteq U$. Alternatively, $\mathcal{B}' = \{U(x,1/n)\mid x\in X, n\geq 1\}$ is a topological base.
    \item Let $X = \R^{d}$ with the Euclidean metric. Then, for $\mathcal{B} = \{U(q,1/n)\mid n\geq 1, q\in \Q^{d}\}$, we claim this is a base.\\

      Let $V\subseteq \R^{d}$ be open, $r\in V$. Since $V$ is open, $\exists \delta > 0$ with $U(r,\delta)\subseteq V$. Find $n$ large such that $1/n < \delta$. Find $q\in \Q^{d}$ with $\norm{r-q} < 1/2n$. This is always possible as $\Q^{d}$ is dense in $\R^d$.\\

      Consider $U(q,1/2n)$. Then, $r\subseteq U(q,1/2n)\subseteq U(r,\delta)\subseteq V$ because $\norm{r-q} < 1/2n$, and if $t\in U(q,1/2n)$, then 
      \begin{align*}
        \norm{t-r} &\leq \norm{t-q} + \norm{q-r}\\
                   &< 1/2n + 1/2n \\
                   &= 1/n\\
                   &< \delta.
      \end{align*}
  \end{itemize}
  \subsection{Separable, Non-Separable, Dense, and Non-Dense Sets}%
  \begin{enumerate}[(1)]
    \item $\left(R^{d},\norm{\cdot}_{p}\right)$ is separable for any $p\in [1,\infty]$. Indeed, $\Q^{d}\subseteq \R^d$ is the countable dense subset of $\R^{d}$.\\

      Let $r= \begin{bmatrix}r_1\\\vdots\\r_d\end{bmatrix}\in \R^{d}$. Find $q= \begin{bmatrix}q_1\\\vdots\\q_d\end{bmatrix}\in \Q^{d}$ with $|r_j-q_j| < \varepsilon/d$. Then, 
      \begin{align*}
        \norm{r-q}_{1} &= \sum_{j=1}^{d}|r_j-q_j|\\
                       &< d.
      \end{align*}
      We know that for any vector $r\in \R^{d}$, we can find a vector $q$ such that
      \begin{align*}
        \norm{q-r}_{p} &\leq c\norm{q-r}_{1},
      \end{align*}
      so for arbitrary $p$, find $q$ such that $\norm{q-r}_{1} < \varepsilon/c$.
    \item Similarly, $\mathbb{C}_{\Q} = \{a+bi\mid a,b\in\Q\}$ is also countable, meaning $\mathbb{C}_{\Q}^{d}\subseteq \mathbb{C}^{d}$ is dense and $\mathbb{C}^{d}$ is dense.
  \end{enumerate}
  \subsection{Proposition: Separable Subsets}%
  If $(X,d)$ is separable, and $Y\subseteq X$, then $(Y,d)$ is also separable.\\

  Let $\{a_k\}$ be a countable dense subset in $X$. Let $N = \{(m,n)\mid U(a_m,1/n)\cap Y \neq \emptyset\}$. Clearly, $N$ is nonempty. For each $(m,n)\in N$, choose $b_{(m,n)}\in Y\cap U(a_m,1/n)$. We claim $\{b_{(m,n)}\mid m,n\geq 1\}$ is dense in $Y$.\\

  Let $y\in Y$, $\varepsilon > 0$. Find $N$ large so that $\frac{1}{n} < \varepsilon/2$. Since $A\subseteq X$ is dense, find $U(y,1/n)\cap A \neq \emptyset$. Suppose $d(a_m,y) < 1/n$. Then,
  \begin{align*}
    d(b_{(m,n)},y) &\leq d(b_{(m,n)},a_m) + d(a_m,y)\\
                   &< \frac{1}{n} + \frac{1}{n}\\
                   &= \frac{2}{n}\\
                   &< \varepsilon.
  \end{align*}
  \begin{enumerate}[(1)]
    \item $\ell_p^{n}$ is separable.
    \item $c_{00} = \{(a_k)_{k=1}^{n}\mid \text{finitely many $a_k \neq 0$}\}$ with $\norm{\cdot}_u$ is separable.\\

      Recall that $e_k = (0,0,\dots,1,0,0,\dots)$ where $1$ is at position $k$. Consider $E = \Q\text{-span}\{e_k\mid k\geq 1\}$,
      \begin{align*}
        E &= \left\{\sum_{k=1}^{n}\alpha_ke_k \mid \alpha_k\in \Q, n\geq 1\right\}.
      \end{align*}
      The set $E$ is countable. If we fix $n\geq 1$, we have
      \begin{align*}
        E_n &= \left\{\sum_{k=1}^{n}\alpha_ke_k\mid \alpha_k\in\Q\right\}.
      \end{align*}
      Then, $E = \bigcup E_n$. Note
      \begin{align*}
        \underbrace{\Q\times\Q\times\cdots\times\Q}_{n} &\rightarrow E_n\\
        (\alpha_1,\dots,\alpha_n) &\mapsto \sum_{k=1}^{n}\alpha_ke_k.
      \end{align*}
      Thus, $E_n$ is countable, and $E$ is a countable union of countable sets.\\

      We claim that $E$ is dense. Given $z\in c_{00}$, $\varepsilon > 0$, we know that $z = \sum_{k=1}^{n}a_ke_k$ for some $n$ and $a_k\in \R$. Find $\alpha_k \in \Q$ such that $|\alpha_k - a_k| < \varepsilon$. Set $w = \sum_{k=1}^{n}\alpha_ke_k$. Then, $\norm{z-w}_{u} = \sup |\alpha_k-a_k| < \varepsilon$.\\

    \item $c_0$ with $\norm{\cdot}_u$ is separable.
    \item $\ell_{\infty}$ is not separable.\\

      Suppose $\ell_{\infty}$ were separable. Consider $E = \{(a_k)_k\in \ell_{\infty}\mid a_k \in \{0,1\}\}$. Then, $E$ is separable. Recall that $(E,\norm{\cdot}_u)$ has the discrete metric.\\

      In the discrete metric, every subset is open, meaning every subset is closed. Therefore, if $X$ is separable and discrete, then $X$ is countable.\\

      However, $E$ is not countable by Cantor's theorem. $\card(E) = 2^{\aleph_0}$.\\

      Alternatively, we can show that
      \begin{align*}
        (a_k)_k \mapsto \sum_{k=1}^{\infty}2^{-k}a_k
      \end{align*}
      is onto.
      \begin{description}
        \item[Exercise:] $\ell_p$ is separable for $1 \leq p < \infty$.
      \end{description}
    \item We will show that 
      \begin{align*}
        \mathbb{P}[0,1]\left\{\sum_{k=1}^{n}a_kx^k\mid a_k\in \R,n\geq 1\right\}
      \end{align*}
      is $\norm{\cdot}_u$-dense in $C([0,1])$ (see: Stone-Weierstrass Theorem). Using this, we can show that $(C([0,1]),\norm{\cdot}_u)$ is separable.
  \end{enumerate}
  \subsection{The Cantor Set}%
  \begin{align*}
    C_0 &= [0,1]\\
    C_1 &= [0,1/3] \cup [2/3,1]\\
    C_2 &= [0,1/9]\cup [2/9,1/3] \cup [2/3,7/9] \cup [8/9,1]\\
    C_3 &= [0,1/27] \cup [2/27,1/9] \cup \cdots \cup [26/27,1]\\
        &\vdots
  \end{align*}
  In each step, we delete the middle third of each interval. This process repeated ad infinitum yields the Cantor set.
  \begin{align*}
    \mathcal{C} &= \bigcap_{n=1}^{\infty}\bigcup_{k=0}^{3^{n-1}-1}\left(\left[\frac{3k+0}{3^n},\frac{3k+1}{3^n}\right]\cup \left[\frac{3k+2}{3^n},\frac{3k+3}{3^n}\right]\right).
  \end{align*}
  \begin{enumerate}[(i)]
    \item $\mathcal{C}$ is closed as it is the intersection of closed sets.
    \item $\text{length}(\mathcal{C}) = 0$. Look at the total length of the removed intervals,
      \begin{align*}
        l &= \frac{1}{3} + \frac{2}{9} + \frac{4}{27} + \frac{8}{81} + \cdots \\
          &= \sum_{k=1}^{\infty}\left(\frac{2^{k-1}}{3^k}\right)\\
          &= \frac{1}{2}\sum_{k=1}^{n}\left(\frac{2}{3}\right)^{k}\\
          &= 1.
      \end{align*}
      Thus, $\text{length}(\mathcal{C})$ = 0.
    \item $\mathcal{C}$ is nowhere dense --- $(\overline{\mathcal{C}})^{\circ} = \emptyset$. Since $\mathcal{C}$ is closed, $\mathcal{C}^{\circ} = \emptyset$.\\

      Suppose $\mathcal{C}^{\circ} \neq \emptyset$. Then, $\exists x\in \mathcal{C}, \varepsilon > 0$ such that $(x-\varepsilon,x+\varepsilon)\subseteq \mathcal{C}$. So, $(x-\varepsilon,x+\varepsilon) \subseteq \mathcal{C}_n$ for all $n$.\\

      Note $\mathcal{C}_n$ is the disjoint union of $2^n$ subintervals, each with length $1/3^n$. Find $m$ so large such that $3^{-m} < \varepsilon$. We know that $(x-\varepsilon,x+\varepsilon)\subseteq C_m$.\\

      However, $(x-\varepsilon,x+\varepsilon)$ has length $2\varepsilon > \frac{2}{3^{m}}$. Each subinterval in $C_m$ has length $1/3^m$. This implies $C_m$ contains an interval of length greater than $\frac{2}{3^m}$. $\bot$
    \item $\card(\mathcal{C}) = \card(\R)$
      \begin{description}
        \item[Claim 1:] Given $n\geq 1$,
          \begin{align*}
            E_n = \left\{\sum_{k=1}^{n}\frac{w_k}{3^k}\mid w_k\in \{0,2\}\right\}
          \end{align*}
          is precisely the set of \textit{left} endpoints of the subintervals of $C_n$.\\

          For $n=1$, if $w_1 = 0$, then we get $0$, and $w_1 = 2$ yields $2/3$. Meanwhile, if $n =2$, then we have
          \begin{align*}
            w_1=0,w_2=0 \mapsto 0\\
            w_1=0,w_2=2 \mapsto 2/9\\
            w_1=2,w_2=0 \mapsto 2/3\\
            w_1=2,w_2=2 \mapsto 8/9.
          \end{align*}
          By induction, we have shown for $n=1,2$. Assume this is true for $n$.
          \begin{align*}
            \sum_{k=1}^{n+1}w_k3^{-k} &= \underbrace{\sum_{k=1}^{n}w_k3^{-k}}_{(1)} + \underbrace{w_{n+1}3^{-(n+1)}}_{(2)}
          \end{align*}
          Part (1) denotes one of the left endpoints of $C_n$, called $C_{n,k}$ for some $1\leq k \leq 2^n$. Then, if $w_{n+1} = 0$, we get the left endpoint of $C_{n+1,2k-1}$, and if $w_n = 2$, we get the left endpoint of $C_{n+1,2k}$.
        \item[Claim 2:]
          \begin{align*}
            C &= \left\{\sum_{k=1}^{\infty}w_k3^{-k}\mid w_k \in \{0,2\}\right\}
          \end{align*}
          is precisely the Cantor set.\\

          Let $x = \sum_{k=1}^{\infty}w_k3^{-k}$. We will show that $x\in C_n$ for all $n$. Fix $n\geq 1$. Then,
          \begin{align*}
            x = \underbrace{\sum_{k=1}^{n}w_k3^{-k}}_{y} + \underbrace{\sum_{k>n}^{}w_k3^{-k}}_{z}.
          \end{align*}
          From our previous claim, $y$ is the left endpoint of some subinterval of $C_n$. Additionally,
          \begin{align*}
            z &= \sum_{k>n}w_k3^{-k}\\
              &\leq 2\sum_{k>n}3^{-k}\\
              &= \frac{2}{3^{n+1}} \left(1 + \frac{1}{3} + \frac{1}{9} + \cdots\right)\\
              &= \frac{1}{3^n}.
          \end{align*}
          Since the length of a subinterval in $C_n$ is exactly $3^{-n}$, it is the case that $x = y+z$ remains an element of $C_{n,k}$.\\

          Let $x\in \mathcal{C}$. Then, $x\in C_n$ for all $n$. Then, $x \in C_1$, so let $x_1$ be the left endpoint of the interval $C_{1,j}$ that contains $x$. Then, $|x-x_1| < \frac{1}{3}$, and $x_1 = w_13^{-1}$ for some $w_1\in \{0,2\}$.\\

          Let $x_2$ be the left endpoint of the subinterval $C_{2,j}$ that contains $x$. Then, $|x-x_2| < \frac{1}{3^2}$. Therefore, 
          \begin{align*}
            x_2 &= x_1 + w_23^{-2}\\
                &= w_13^{-1} + w_23^{-2}.
          \end{align*}
          Iterating, we have $x_n$, the left endpoint of the subinterval $C_{n,j}$ that contains $x$.
          \begin{align*}
            x_n &= \sum_{k=1}^{n} w_k3^{-k}.
          \end{align*}
          We have that $|x-x_n| < 3^{-n}$.\\

          Therefore, $(x_n)_n \rightarrow x$. Also,
          \begin{align*}
            x_n &= \sum_{k=1}^{n}w_k3^{-k}\\
                &\rightarrow \sum_{k=1}^{n}w_k3^{-k}.
          \end{align*}
          Thus,
          \begin{align*}
            x &= \sum_{k=1}^{\infty}w_k3^{-k}.
          \end{align*}
      \end{description}
      To prove $\card(\mathcal{C}) = \card(\R)$, we will show that $\card\left(\{0,1\}^{\mathbb{N}}\right) = \card(\mathcal{C})$.
      \begin{align*}
        (a_k)_k \mapsto \sum_{k=1}^{\infty}2a_k3^{-k}.
      \end{align*}
  \end{enumerate}
  \subsection{Relative (or Subspace) Topology}%
  We know that if $(X,d)$ is a metric space, and $Y\subseteq X$ is any subset, then $(Y,d)$ is a metric space. The question now is: what are the open sets of $Y$?\\

  For example, let $X = \R$, $Y = [0,1]$. Consider $U = [0,1/2)$. $U$ is not open in $\R$, as if $x = 0$, then there is no open ball completely contained in $U$. However, in $Y$, $U$ is open.\\

  Let $(X,d)$ be a metric space, $Y\subseteq X$ any subset. $V\subseteq Y$ is open if and only if $\exists U\subseteq X$ open such that $V = U\cap Y$. That is, $\tau_Y = \{U\cap Y\mid U\in \tau_X\}$.\\

  Let $V$ be open in $Y$. Then, $\forall x\in V$, $\exists \delta_x > 0$ such that $U_{Y}(x,\delta_x)\subseteq V$. We have $U_{Y}(x,\delta_x) = \{y\in Y\mid d(y,x) < \delta_x\}$. Let
  \begin{align*}
    U &= \bigcup_{x\in V} U_{X}(x,\delta_x)\\
    U\cap Y &= \left(\bigcup_{x\in V}U_{X}(x,\delta_x)\right)\cap Y\\
            &= \bigcup_{x\in V}U_X(x,\delta_x)\cap Y\\
            &= \bigcup_{x\in V} U_{Y} (x,\delta_x).
  \end{align*}
  Let $U$ be open in $X$. Then, for $x\in U\cap Y$, $\exists \delta_x$ such that $U(x,\delta_x)\subseteq U$. 
  \begin{enumerate}[(1)]
    \item $\ell_{\infty}$ is not a discrete metric space. However, $E = \{(a_k)_k\mid a_k \in \{0,1\}\}$ with the induced metric. Then, $E$ is a discrete metric space.
  \end{enumerate}
  \section{Convergent Sequences}%
  Fix a metric space $(X,d)$. A sequence in $X$ is a map $x: \N\rightarrow X$, $n\mapsto x(n) = x_n$.\\

  A natural sequence $(n_k)_k$ is a sequence in $\N$ with $n_k \geq k$ for all $k$. A subsequence of $(x_n)_n$ is a sequence $(x_{n_k})_k$, where $(n_k)_k$ is a natural sequence.\\

  A sequence $(x_n)_n$ converges to $x\in X$ if $\forall \varepsilon > 0$, $\exists N_{\varepsilon}\in \N$ such that $n\geq N_{\varepsilon}$ implies $d(x_n,x) < \varepsilon$. We write $(x_n)_n\xrightarrow{d} x$.
  \begin{description}
    \item[Exercise:] A sequence can have at most one limit, as metric spaces are Hausdorff.
  \end{description}
  \subsection{Proposition: Equivalent Definitions of Convergence}%
  Given $(x_n)_n \in X$, $x\in X$, the following are equivalent.
  \begin{enumerate}[(i)]
    \item $(x_n)_n\rightarrow x$ in $X$
    \item $(d(x_n,x))_n\rightarrow 0$ in $\R$
    \item $\forall V\in \mathcal{N}_x$, $\exists N\in \N$ with $n\geq N \Rightarrow x_n \in V$.
  \end{enumerate}
  \begin{description}
    \item[Exercise:] Let $(X,\rho)$ be a metric space, let $d(x,y) = \frac{\rho(x,y)}{1 + \rho(x,y)}$. A sequence $(x_n)_n\xrightarrow{d} x$ if and only if $(x_n)_n\xrightarrow{\rho} x$.
  \end{description}
  \subsection{Proposition: Convergent Sequences are Bounded}%
  Let $(x_n)_n\rightarrow x$ in $(X,d)$. Let $\varepsilon = 1$. Then, $\exists N\in \N$ large such that for $n\geq N$, $d(x_n,x) < 1$.\\

  If $m,n\geq N$, then $d(x_n,x_m) \leq d(x_n,x) + d(x,x_m) < 2$. Let $c = \max_{1\leq n,m\leq N} d(x_n,x_m)$. Then,
  \begin{align*}
    d(x_n,x_m) &\leq d(x_n,x_N) + d(x_n,x_m)\\
               &\leq 1 + c.
  \end{align*}
  Let $k = \max\{1+c,2\}$. Then, $\text{diam}(\{x_n\}) \leq k$.
  \subsection{Convergence in Different Metric Spaces}%
  \begin{description}
    \item[Convergence for Bounded Functions:] Recall that for $(Y,d)$ a metric space is
  \begin{align*}
    \text{Bd}(\Omega,Y) &= \{f:\Omega \rightarrow Y\mid f \text{ bounded}\}\\
    D_{u}(f,g) &= \sup_{x\in \Omega}d(f(x),g(x)).
  \end{align*}
  Then, $(f_n)_n\rightarrow f$ in $\text{Bd}(\Omega,Y)$ if and only if $D_{u}(f_n,f)\rightarrow 0$ in $\R$.
  \begin{align*}
    (\forall \varepsilon > 0)(\exists N\in \N)\text{ such that } n\geq N&\Rightarrow D_{u}(f_n,f) < \varepsilon\\
                                                         &\Leftrightarrow\\
    (\forall \varepsilon > 0)(\exists N\in \N)\text{ such that } n\geq N&\Rightarrow \sup_{x\in\Omega}d(f_n(x),f(x)) < \varepsilon\\
                                                         &\Leftrightarrow\\
    (\forall \varepsilon > 0)(\exists N\in \N)\text{ such that } n\geq N&\Rightarrow \forall x,~d(f_n(x),f(x)) < \varepsilon.
  \end{align*}
  This is exactly the definition of uniform convergence.\\

  Since $\ell_{\infty}(\Omega) = \text{Bd}(\Omega,\mathbb{F})$, convergence in $\ell_{\infty}(\Omega)$ is uniform convergence. This is also the case for subspaces, such as $c$, $c_{0}$, and $c_{00}$.
    \item[Convergence in the Frechet Metric:] Consider a separating family of semimetrics $\rho_k$ on a set $X$. Set $d_k = \frac{\rho_k}{1 + \rho_k}$. We saw that
      \begin{align*}
        d(x,y) &= \sum_{k=1}^{\infty} 2^{-k}d_k(x,y)
      \end{align*}
      is a metric on $X$.\\

      We claim that $(x_n)_n\rightarrow x$ in $(X,d)$ if and only if for all $k\geq 1$, $\rho_k(x_n,x)\rightarrow 0$.\\

      In the forward direction, we know that $(x_n)_n\rightarrow x$ with respect to $d$ if and only if $d(x_n,x)\rightarrow 0$ in $\R$. Since $0\leq 2^{-k}d_k(x_n,x) \leq d(x_n)$ for fixed $k$, we have that 
      \begin{align*}
        0 \leq d_k(x_n,x) \leq 2^{k}d(x_n,x),
      \end{align*}
      and as $n\rightarrow \infty$, $d(x_n,x)\rightarrow 0$, meaning $d_k(x_n,x)\rightarrow 0$. Therefore, $\rho_k(x_n,x)\rightarrow 0$.\\

      In the reverse direction, suppose $\rho_k(x_n,x)\rightarrow 0$ in $\R$ as $n\rightarrow\infty$ for all $k\geq 1$. Thus, $d_k(x_n,x)\rightarrow 0$ as $n\rightarrow\infty$ for all $k\geq 1$.\\

      Let $\varepsilon > 0$. Let $K$ be so large such that 
      \begin{align*}
        \sum_{k\geq K} 2^{-k}< \varepsilon/2.
      \end{align*}
      Therefore, $d_k(x_n,x)\rightarrow 0$ for all $k = 1,\dots,K$. Therefore, $\exists N_1,\dots,N_{K}$ such that for $n\geq N_{k}$,
      \begin{align*}
         d_k(x_n,x) < \frac{\varepsilon}{2}.
      \end{align*}
      Let $N = \max\{N_1,\dots,N_K\}$. Therefore, for $n\geq N$,
      \begin{align*}
        d_k(x_n,x) < \frac{\varepsilon}{2}
      \end{align*}
      for all $k = 1,\dots,K$.\\

      Thus, for all $n\geq N$,
      \begin{align*}
        d(x_n,x) &= \sum_{k=1}^{\infty}2^{-k}d_k(x_n,x)\\
                 &= \sum_{k=1}^{K}2^{-k}d_k(x_n,x) + \sum_{k=K+1}^{\infty} 2^{-k}d_k(x_n,x)\\
                 &\leq \frac{\varepsilon}{2}\sum_{k=1}^{K}2^{-k} + \frac{\varepsilon}{2}\\
                 &< \varepsilon
      \end{align*}
      Therefore, $(x_n)_n\rightarrow x$.\\

      Recall that, for the Frechet metric, our set was $X = C(\R)$. For $k = 1,2,3,\dots$, we had
      \begin{align*}
        p_k(f) &= \sup_{[-k,k]}|f(x)|
      \end{align*}
      as our seminorm, and our semimetric was
      \begin{align*}
        \rho_k(f,g) &= p_k(f-g).
      \end{align*}
      We also showed that the $\rho_k$ family is separating. We make $d_k(f,g) = \frac{\rho_k(f,g)}{1+\rho_k(f,g)}$ as the bounded family of separating metrics, and
      \begin{align*}
        d_F(f,g) &= \sum_{k=1}^{\infty}\frac{2^{-k}\rho_k(f-g)}{1 + \rho_k(f-g)}.
      \end{align*}
      In $\left(C(\R),d_F\right)$, $(f_n)_n\rightarrow f$ if and only if $\rho_k(f_n,f)\rightarrow 0$ for all $k$, meaning $(f_n)_n\rightarrow f$ uniformly on $[-k,k]$ for all $k$.\\

      This is known as convergence on compact subsets.
    \item[Convergence in a Product Space:] Let $(X,d)$ and $(Y,\rho)$ be metric spaces. Then,
      \begin{align*}
        X\times Y &= \left\{(x,y)\mid x\in X,y\in Y\right\},\\
        D_{1}((x,y),(x',y')) &= d(x,x') + \rho(y,y')\\
        D_{\infty} ((x,y),(x',y'))&= \max\{d(x,x'),\rho(y,y')\}.
      \end{align*}
      Both $D_1$ and $D_{\infty}$ are equivalent metrics.
      \begin{description}
        \item[Exercise:] $((x_n,y_n))_n \rightarrow (x,y)$ if and only if $(x_n)_n \xrightarrow{d}x$ and $(y_n)_n\xrightarrow{\rho} y$.
      \end{description}
  \end{description}
  \subsection{Series in a Normed Vector Space}%
  Let $(V,\norm{\cdot})$ be a normed vector space. Consider a sequence $(v_k)_k$ of vectors.
  \begin{align*}
    s_1 &= v_1\\
    s_2 &= v_1 + v_2\\
        &\vdots\\
    s_n &= \sum_{k=1}^{n}v_k.
  \end{align*}
  If $s_n\rightarrow s$ in $(V,\norm{\cdot})$, meaning $\norm{s_n-s}\rightarrow 0$, then we say the series $\sum_{k=1}^{\infty}v_k$ converges to $s$. We write
  \begin{align*}
    \sum_{k=1}^{\infty}v_k &= s.
  \end{align*}
  The series converges absolutely if
  \begin{align*}
    \sum_{k=1}^{\infty} \norm{v_k}
  \end{align*}
  converges in $\R$.
  \subsection{Proposition: Sequential Characterization of Closure}%
  Let $(X,d)$ be a metric space with $A\subseteq X$. $x\in \overline{A}$ if and only if $\exists (a_n)_n$ in $A$ with $(a_n)_n\rightarrow X$.\\

  In the forward direction, recall that $x\in \overline{A}$ if and only if $\forall \delta > 0$, $U(x,\delta)\cap A \neq \emptyset$. If $x\in \overline{A}$, then set $\varepsilon_n = 1/n$, and since $U(x,1/n) \cap A \neq \emptyset$. Let $a_n\in U(x,1/n)\cap A$. Then, $d(a_n,x) < 1/n \rightarrow 0$, meaning $a_n\rightarrow x$ and $a_n\in A$.\\

  In the reverse direction, if $(a_n)_n\rightarrow x$ and $\varepsilon > 0$, $\exists N$ with $n\geq N \Rightarrow a_n\in U(x,\varepsilon)\cap A$. Thus, $x\in \overline{A}$.
  \subsection{Proposition: Sequential Characterization of Closed Sets}%
  If $(X,d)$ is a metric space, $A\subseteq X$, then the following are equivalent:
  \begin{enumerate}[(i)]
    \item $A$ is closed.
    \item Whenever $(a_n)_n$ in $A$ with $(a_n)_n\xrightarrow{d} x$ in $X$, then $x\in A$.
  \end{enumerate}
  \begin{description}
    \item[Continuous Bounded Functions:] $C([a,b])\subseteq \ell_{\infty}([a,b])$ is closed under $\norm{\cdot}_{u}$, since if $(f_n)_n\rightarrow f$ uniformly, and $f_n$ is continuous, then $f$ is continuous.
    \item[Sequence Closure:] $c_{0}\subseteq \ell_{\infty}$ is closed under $\norm{\cdot}_u$. Let $(f_n)_n$ be a sequence
      \begin{align*}
        f_1 &= (f_1(1),f_1(2),\dots)\\
        f_2 &= (f_2(1),f_2(2),\dots)\\
        \lim_{k\rightarrow\infty}f_n(k) &= 0 \tag*{$\forall n$}
      \end{align*}
      Suppose $(f_n)_n \xrightarrow{\norm{\cdot}_{\infty}} f \in \ell_{\infty}$.\\

      Let $\varepsilon > 0$. Then, $\exists n\in \N$ such that for $n\geq N$, $\norm{f-f_n}_{\infty}< \varepsilon/2$. Also, $\lim_{k\rightarrow \infty}f_{N}(k) = 0$. Then, $\exists K\in \N$ such that for $k\geq K$, $|f_{N}(k)| < \varepsilon/2$. Thus, for $k\geq K$,
      \begin{align*}
        |f(k)| &= |f(k) - f_{N}(k) + f_{N}(k)|\\
               &\leq |f(k)-f_{N}(k)| + |f_{N}(k)|\\
               &\leq \norm{f-f_{N}}_{\infty} + |f_{N}(k)|\\
               &< \frac{\varepsilon}{2} + \frac{\varepsilon}{2}\\
               &= \varepsilon.
      \end{align*}
      Thus, $f\in c_{0}$.
  \end{description}
  \subsection{Distance to a Set}%
  Let $(X,d)$ be a metric space, $A\subseteq X$. Then, $\text{dist}_{A}: X\rightarrow [0,\infty)$ is defined as
  \begin{align*}
    \text{dist}_{A}(x) &= \inf_{a\in A}d(x,a).
  \end{align*}
  \begin{enumerate}[(1)]
    \item $\overline{A} = \{x\mid \text{dist}_{A} = 0\}$
    \item $\text{dist}_{A}(\cdot) = \text{dist}_{\overline{A}}(\cdot)$
    \item $|\text{dist}_{A}(x) - \text{dist}_{A}(y)| \leq d(x,y)$
  \end{enumerate}
  \begin{description}
    \item[Proof of (1):] Let $x\in \overline{A}$. Then, $\exists (a_n)_n$ such that $(a_n)_{n}\rightarrow x$. Then, $d(a_n,x) \rightarrow 0$. Since $0 \leq \text{dist}_{A}(x) \leq d(x,a_n)$, $\text{dist}_{A}(x) = 0$.\\

      Let $x$ be such that $\text{dist}_{A}(x) = 0$. By the definition of $\inf$, we construct $a_n$ by finding $a_n\in U(x,1/n)\cap A$. Thus, $d(a_n,x)\rightarrow 0$, meaning $(a_n)_n\rightarrow x$, so $x\in \overline{A}$.
    \item[Proof of (2):] Exercise; use (1).
    \item[Proof of (3):] For all $a\in A$,
      \begin{align*}
        \text{dist}_{A}(x) &\leq d(x,a)\\
                           &\leq d(x,y) + d(y,a).\\
        \intertext{Therefore,}
        \text{dist}_{A}(x) - d(x,y) &\leq d(y,a)\\
        \text{dist}_{A}(x) - d(x,y) &\leq \inf_{a\in A}d(y,a)\\
                                    &= \text{dist}_{A}(y)\\
        \text{dist}_{A}(x) - \text{dist}_{A}(y) &\leq d(x,y).
        \intertext{Similarly,}
        \text{dist}_{A}(y) - \text{dist}_{A}(x) &\leq d(y,x) = d(x,y)\\
        \intertext{meaning}
        |\text{dist}_{A}(y) - \text{dist}_{A}(x)| &\leq d(x,y).
      \end{align*}
  \end{description}
  \subsection{Continuity}%
  Let $(X,d)$ and $(Y,\rho)$ be metric spaces. A map $f: X\rightarrow Y$
  \begin{enumerate}[(1)]
    \item is continuous at $x_0\in X$ if
      \begin{align*}
        (\forall \varepsilon > 0)(\exists \delta > 0) \text{ such that } d(x,x_0)<\delta \Rightarrow \rho(f(x),f(x_0)) < \varepsilon\\
        (\forall \varepsilon > 0)(\exists \delta > 0) \text{ such that } x\in U_{X}(x_0,\delta) \Rightarrow f(x)\in U_Y(f(x_0),\varepsilon)\\
        (\forall \varepsilon > 0)(\exists \delta > 0) \text{ such that } f(U_X(x_0,\delta))\subseteq U_Y(f(x_0),\varepsilon).
      \end{align*}
    \item is continuous if $f$ is continuous at every $x_0\in X$.
  \end{enumerate}
  \subsection{Proposition: Equivalent Continuity Criteria}%
  Let $f: (X,d)\rightarrow (Y,\rho)$, $x_0\in X$. The following are equivalent:
  \begin{enumerate}[(1)]
    \item $f$ is continuous at $x_0$;
    \item $(\forall V\in \mathcal{N}_{f(x_0)})(U\in \mathcal{N}_{x_0})$ such that $f(U)\subseteq V$.
    \item $\forall (x_n)_n\rightarrow x_{0}$, $(f(x_n))_n\rightarrow f(x_0)$.
  \end{enumerate}
  \begin{description}[font=\normalfont]
    \item[(1) $\Leftrightarrow$ (2):] Clearly follows from definitions.
    \item[(1) $\Rightarrow$ (3):] Let $(x_n)_n\rightarrow x_0$. Let $\varepsilon > 0$. Then, $\exists \delta > 0$ such that $d(x,x_0) < \delta$ implies $\rho(f(x),f(x_0)) < \varepsilon$.\\

      Thus, $\exists N\in \N$ such that $n\geq N$ implies $d(x_n,x_0) < \delta$. So, if $n\geq N$, $d(x_n,x_0) < \delta$, implying $\rho(f(x_n),f(x_0)) < \varepsilon$. So, $(f(x_n))_n \rightarrow f(x_0)$.
    \item[(3) $\Rightarrow$ (1):] Suppose toward contradiction that $\exists \varepsilon_0 > 0$  such that for $\delta = 1/n$ where $n\in \N$, $\exists (x_n)_n: d(x_n,x_0) < \delta$ and $\rho(f(x_n),f(x_0)) \geq \varepsilon_0$. Then, $(x_n)_n\rightarrow x_0$, but $f(x_n)_n\nrightarrow f(x_0)$. $\bot$
  \end{description}
  \subsection{Proposition: Topological Criterion for Continuity}%
  Let $f: (X,d)\rightarrow (Y,\rho)$. The following are equivalent:
  \begin{enumerate}[(1)]
    \item $f$ is continuous.
    \item $\forall V\in \tau_Y$, $f^{-1}(V) \in \tau_{X}$.
    \item $\forall x\in X,\forall (x_n)_n\rightarrow x$, we have $(f(x_n))_n\rightarrow f(x)$.
  \end{enumerate}
  \begin{description}
    \item[Proof:] Exercise.
  \end{description}
  \subsection{Proposition: Composition of Functions}%
  Let $(X,d)\xrightarrow{f} (Y,\rho)\xrightarrow{g}(Z,p)$. If $f$ and $g$ are continuous, then $g\circ f$ is continuous.
  \begin{description}
    \item[Proof:] Exercise.
  \end{description}
  \subsection{Uniform Continuity}%
  Let $f: (X,d)\rightarrow (Y,\rho)$.
  \begin{enumerate}[(1)]
    \item $f$ is uniformly continuous if 
      \begin{align*}
        (\forall \varepsilon > 0)(\exists \delta > 0) \text{ such that }\forall x,x'\in X,d(x,x')< \delta \Rightarrow d(f(x),f(x')) < \varepsilon
      \end{align*}
    \item $f$ is Lipschitz if $\exists c > 0$ with
      \begin{align*}
        \rho(f(x),f(x')) \leq c d(x,x')
      \end{align*}
      for all $x,x'\in X$.
    \item If $\rho(f(x),f(x')) = d(x,x')$, then $f$ is an isometry. Isometries are always injective.
  \end{enumerate}
  \begin{description}
    \item[Exercise:]
      \begin{align*}
        \text{Isometry} \Rightarrow \text{Lipschitz} \Rightarrow \text{Uniformly Continuous} \Rightarrow \text{Continuous}.
      \end{align*}
      For example, $f(x) = x^2$ on $[0,\infty)$ is continuous but not uniformly continuous, and $\sqrt{x}$ on $[0,1]$ is uniformly continuous but not Lipschitz.
  \end{description}
  If $(V,\norm{\cdot})$ is a normed space, we might want to care that the following operations are continuous:
  \begin{itemize}
    \item $a: V\times V\rightarrow V$, $a(v,w) = v+w$:
  \begin{align*}
    \norm{a(v,w) - a(v',w')} &= \norm{v+w-(v'+  w')}\\
                             &= \norm{v-v' + w-w'}\\
                             &\leq \norm{v-v'} + \norm{w-w'}\\
                             &= d(v,v') + d(w,w')\\
                             &= d_1((v,w),(v',w')),
  \end{align*}
  meaning $a$ is Lipschitz.
    \item $m: \mathbb{F}\times V \rightarrow V$, $m(\alpha,v) = \alpha v$;
  \begin{align*}
    \norm{m(\alpha,v)-m(\beta,w)} &= \norm{\alpha v - \beta w}\\
                                  &= \norm{\alpha v - \alpha w + \alpha w - \beta w}\\
                                  &\leq |\alpha|\norm{v-w} + |\alpha - \beta| \norm{w}\\
  \intertext{If $(\alpha_n)_n\rightarrow \beta$ and $(v_n)_n\rightarrow w$, then}
    \norm{\alpha_nv_n - \beta w} &\leq |\alpha_n|\norm{v_n-w} + |\alpha_n - \beta|\norm{w}\\
                                 &\rightarrow 0.
  \end{align*}
    \item $\norm: V\rightarrow \mathbb{F}$:
      \begin{align*}
        \left|\norm{v}-\norm{w}\right| &\leq \norm{v-w},
      \end{align*}
      meaning $\norm{\cdot}$ is Lipschitz.
  \end{itemize}
  Let $(X,d)$ be a metric space. Then, $\text{dist}_A: X\rightarrow [0,\infty)$, $\text{dist}_A(x) = \inf_{a\in A}d(x,a)$ is continuous. We had shown
  \begin{align*}
    |\text{dist}_{A}(x) - \text{dist}_{A}(y)| &\leq d(x,y),
  \end{align*}
  meaning $\text{dist}_{A}$ is Lipschitz.
  \subsection{Proposition: Normal Property of Metric Spaces}%
  Given $A,B\subseteq X$ with $A\cap B = \emptyset$, then $\exists U,V\in \tau_X$ with $A\subseteq U$, $B\subseteq V$, and $U\cap V = \emptyset$.
  \begin{description}
    \item[Proof:] Set
      \begin{align*}
        f(x) &= \frac{\text{dist}_A(x)}{\text{dist}_A(x) + \text{dist}_B(x)}.
      \end{align*}
      Note that $\text{dist}_A(x) + \text{dist}_B(x)=0$ if and only if $x\in \overline{A}=A$ and $x\in \overline{B}=B$. Therefore, the denominator in $f(x)$ is always positive.\\

      Additionally, $f: X\rightarrow [0,1]$ is continuous. Note that $f(a) = 0$ for all $a\in A$ and $f(b) = 1$ for all $b\in B$.\\

      Let $U = f^{-1}((-1/2,1/2)) = f^{-1}([0,1/2))$, and $V = f^{-1}((1/2,3/2)) = f((1/2,1])$. Obviously, $U\subseteq A$ and $V\subseteq B$, and $U\cap V = \emptyset$.
  \end{description}
  \subsection{Proposition: Quotient Space}%
  Let $(V,\norm{\cdot})$ be a normed space, and let $W\subseteq V$ be a closed subspace. Then, $V/W$ is a normed space with
  \begin{align*}
    \norm{v+W} &= \text{dist}_W(v)\\
               &= \inf_{w\in W} \norm{v-w}.
  \end{align*}
  \subsection{Proposition: Uniform Continuity of Linear Transformations}%
  Let $T:V\rightarrow W$ be a linear transformation between two normed spaces. The following are equivalent:
  \begin{enumerate}[(1)]
    \item $T$ is continuous at $0_{V}$.
    \item $T$ is continuous.
    \item $T$ is uniformly continuous.
    \item $T$ is Lipschitz.
    \item $\exists c \geq 0$ such that $\norm{T(v)} \leq c\norm{v}$ for all $v\in V$.
    \item $\norm{T}_{\text{op}} = \sup_{\norm{v}\leq 1}\norm{T(v)} < \infty$. In other words, $T$ is bounded linear.
  \end{enumerate}
  \begin{description}
    \item[Proof:]\hfill
      \begin{description}[font=\normalfont]
        \item[(4) $\Rightarrow$ (3) $\Rightarrow$ (2) $\Rightarrow$ (1):] Obvious.
        \item[(6) $\Rightarrow$ (5)] Let $v\in V$. If $v = 0_V$, then $T(v) = 0_W$. Suppose $v\neq 0_V$. We know 
          \begin{align*}
            \norm{T\left(\frac{v}{\norm{v}}\right)} &\leq \norm{T}_{\text{op}}\\
            \frac{1}{\norm{v}}\norm{T(v)} &\leq \norm{T}_{\text{op}}\\
            \norm{T(v)} &\leq \norm{T}_{\text{op}} \norm{v}.
          \end{align*}
          Therefore, $c = \norm{T}_{\text{op}}$.
        \item[(5) $\Rightarrow$ (6):] We will have $\norm{T(v)} \leq c$ for all $v\in B_V$. Thus, $\norm{T}_{\text{op}} \leq c$ for such $c$.
        \item[(5) $\Rightarrow$ (4):] Let $v,w\in V$. Then,
          \begin{align*}
            \norm{T(v) - T(w)} &= \norm{T(v-w)}\\
                               &\leq c\norm{v-w},
          \end{align*}
          meaning $T$ is Lipschitz.
        \item[(1) $\Rightarrow$ (5):] Let $\varepsilon = 1$. Then, $\exists \delta$ such that
          \begin{align*}
            T(U_{V}(0,\delta)) &\subseteq U(T(0),1).\\
            \intertext{Since $T$ is linear,}
            T(U_{V}(0,\delta)) &\subseteq U_{W}(0,1).\\
          \end{align*}
          Let $v\in V\neq 0_V$. We know $\frac{\delta v}{2\norm{v}}\in U_V(0,\delta)$. Then, 
          \begin{align*}
            \norm{T\left(\frac{\delta v}{2\norm{v}}\right)} &\leq 1,\\
            \frac{\delta}{2\norm{v}} \norm{T(v)} &\leq 1\\
            \norm{T(v)} &\leq \frac{2}{\delta}\norm{v}.
          \end{align*}
          Set $c = \frac{2}{\delta}$. Clearly, $\norm{T(0)}\leq \frac{2}{\delta}\norm{0}$.
      \end{description}
  \end{description}
  A corollary to this is that any linear map $T: \ell_{p}^{n}\rightarrow W$ for $W$ a normed space is uniformly continuous.
  \subsection{Proposition: Continuous Functions on Dense Sets}%
  Let $(X,d)$, $(Y,\rho)$ be metric spaces, and $A\subseteq X$ dense. If $f,g: X\rightarrow Y$ and $f(A) = g(A)$, then $f(X) = g(X)$.
  \begin{description}
    \item[Proof:] Given $x\in X$, there exists $(a_n)_n\rightarrow x$. We know that $(g(a_n))_n\rightarrow g(x)$ and $(f(a_n))_n\rightarrow f(x)$. Since $f(a_n) = g(a_n)$ for all $a_n$, it is the case that $f(x) = g(x)$.
  \end{description}
  \subsection{Morphisms in the Category of Metric Spaces}%
  Let $(X,d)$ and $(Y,\rho)$ be metric spaces, $f: X\rightarrow Y$ a map.
  \begin{enumerate}[(1)]
    \item $f$ is a homeomorphism if $f$ is bijective, continuous, and has a continuous inverse. We write $X\cong Y$ are homeomorphic.
    \item $f$ is a uniformism if $f$ is bijective, uniformly continuous, and has a uniformly continuous inverse. We write $X\cong Y$ are uniformly isomorphic.
    \item $f$ is a metric isomorphism if $f$ is bijective, Lipschitz, and has a Lipschitz inverse. We write $X\cong Y$ are metrically isomorphic.
    \item $f$ is an isometric isomorphism if $f$ is bijective and isometric. We write $X\cong Y$ are isometrically isomorphic.
  \end{enumerate}
  For example, $R\cong (-\pi/2,\pi/2)$ are homeomorphic (using $\tan: (-\pi/2,\pi/2)\rightarrow \R$). However, $\R$ is not uniformly isomorphic to $(-\pi/2,\pi/2)$.\\

  Suppose $f: (-\pi/2,\pi/2)\rightarrow \R$ is a uniformism. Let $(x_n)_n = \pi/2-1/n$. Then, $(x_n)_n$ is Cauchy. Therefore, $(f(x_n))_n$ is Cauchy. Since $\R$ is complete, $(f(x_n))_n \rightarrow y$ for some $y\in \R$. Then, $f^{-1}(f(x_n))_n\rightarrow f^{-1}(y)$, meaning $(x_n)_n\rightarrow f^{-1}(y)\in (-\pi/2,\pi/2)$. However, $(x_n)_n\rightarrow \pi/2\notin (-\pi/2,\pi/2)$.

  \section{Completeness}%
  \subsection{Proposition: Weierstrass $M$-Test}%
  Let $V$ be a Banach space (complete normed vector space). Suppose $(v_k)_k$ is such that $\sum \norm{v_k}$ is convergent. Then, $(s_n)_n = \sum_{k=1}^{n}v_k$ converges in $V$. Additionally,
  \begin{align*}
    \norm{\sum_{k=1}^{\infty}v_k} &\leq \sum_{k=1}^{\infty}\norm{v_k}.
  \end{align*}
  \begin{description}
    \item[Proof:] Let $s_n = \sum_{k=1}^{n} v_k$, and $t_n = \sum_{k=1}^{n}\norm{v_k}$. Let $n > m$. Then,
      \begin{align*}
        \norm{s_n - s_m} &= \norm{\sum_{k=m+1}^{n}v_k}\\
                         &\leq \sum_{k=m+1}^{n}\norm{v_k}\\
                         &= |t_n-t_m|.
      \end{align*}
      Since $(t_n)_n$ converges, it is Cauchy, and thus $s_n$ is Cauchy. Since $V$ is complete, $(s_n)_n$ converges.
      \begin{align*}
        \norm{s_n} &= \norm{\sum_{k=1}^{n}v_k}\\
                   &\leq \sum_{k=1}^{n}\norm{v_k}\\
                   &\leq \sum_{k=1}^{\infty}\norm{v_k}.
      \end{align*}
      Let $n\rightarrow\infty$. Using the continuity of the norm, we get
      \begin{align*}
        \norm{\sum_{k=1}^{\infty}v_k} &\leq \sum_{k=1}^{\infty}\norm{v_k}.
      \end{align*}
  \end{description}
  \subsection{Proposition: Convergence in Hilbert Space}%
  Let $H$ be a Hilbert space (inner product space with a complete norm). Let $(e_n)_n$ be an orthonormal sequence in $H$. Let $(t_k)_k$ be a sequence in $\ell_2$. Then, $\sum_{k=1}^{\infty}t_ke_k$ converges in $H$, but not absolutely.
  \begin{description}
    \item[Proof:] Let $s_n = \sum_{k=1}^{n}t_ke_k$. For $n > m$,
      \begin{align*}
        \norm{s_n - s_m}^2 &= \norm{\sum_{k=m+1}^{n}t_ke_k}^2\\
                           &= \sum_{k=m+1}^{n}\norm{t_ke_k}^2 \tag*{Pythagorean Theorem}\\
                           &= \sum_{k=m+1}^{n}|t_k|^2
      \end{align*}
      Since $(t_k)_k\in \ell_2$, we know that $(t_k)_k$ is convergent and thus Cauchy. Thus, $(s_n)_n$ is Cauchy.\\

      Note that for $t_k = \frac{1}{k}$, $(t_k)_k$ is square-summable, but not summable in absolute value.
    \item[Exercise:] Show that
      \begin{align*}
        \norm{\sum_{k=1}^{\infty}t_ke_k}^2 &= \sum_{k=1}^{\infty}|t_k|^2.
      \end{align*}
      This result is known as Parseval's Theorem.
  \end{description}
  \section{Extensions of Continuous Functions}%
  \subsection{Lemma: Cauchy Sequences in Uniformly Continuous Functions}%
  Let $f: (X,d)\rightarrow (Y,\rho)$ be uniformly continuous. If $(x_n)_n$ is Cauchy, then $(f(x_n))_n$ is Cauchy.
  \begin{description}
    \item[Proof:] Let $\varepsilon > 0$. Then, $\exists \delta > 0$ such that
      \begin{align*}
        d(x,x') < \delta \Rightarrow d(f(x),f(x')) < \varepsilon.
      \end{align*}
      Similarly, there exists $N\in \N$ such that for $p,q \geq N$, $d(x_p,x_q) < \delta$. So, for $p,q \geq N$, $d(f(x_p),f(x_q)) < \varepsilon$.
    \item[Remark:] This is not true for continuous functions. For example, if $f(t) = 1/t$ on $(0,1)$, $x_n = 1/n$ is Cauchy but not convergent.
  \end{description}
  \subsection{Theorem: Extension on a Dense Subset}%
  Let $(X,d)$ be a metric space with $A\subseteq X$ dense. Suppose $f: A\rightarrow Y$ is uniformly continuous with $(Y,\rho)$ complete. Then, $\exists !$ uniformly continuous extension, $\tilde{f}: X\rightarrow Y$ that agrees with $f$ on $A$.
  \begin{description}
    \item[Proof:] Let $x\in X$. Then, $\exists (a_n)_n\in A$ with $(a_n)_n\rightarrow x$. Therefore, $(a_n)_n$ is Cauchy, and since $f$ is uniformly continuous, we know that $(f(a_n))_n$ is Cauchy. Thus, $\lim_{n\rightarrow\infty}(f(a_n))_n = \tilde{f}(x)$ exists.\\

      To show $\tilde{f}$ is well-defined, suppose $(b_n)_n$ is another sequence in $A$ with $(b_n)_n\rightarrow x$. Consider $(c_n)_n = (a_1,b_1,a_2,b_2,\dots)$. It must be the case that $(c_n)_n\rightarrow x$. Thus, $(f(c_n))_n$ converges to $y\in Y$. The subsequence of $(f(a_n))_n\rightarrow y$ and $(f(b_n))_n\rightarrow y$. So, we must have $\lim f(a_n) = \lim f(b_n)$.\\

      Note that $\tilde{f}(a) = f(a)$ for all $a\in A$, by choosing the sequence $(a,a,a,\dots)$.\\

      We claim that $\tilde{f}$ is uniformly continuous. Let $\varepsilon > 0$. We know $\exists \delta > 0$ such that for any $a,b\in A$, with $d(a,b) < \delta$, then $\rho(f(a),f(b)) < \varepsilon / 2$. Now, let $x,x'\in X$ with $d(x,x') < \delta / 4$. Find sequences $(a_n)_n\rightarrow x$ and $(b_n)_n\rightarrow x'$ with $(a_n)_n, (b_n)_n\in A$. Find $N$ large such that $n \geq N$ implies $d(a_n,x) < \delta/4$ and $d(b_n,x') < \delta/4$. For $n\geq N$, we have
      \begin{align*}
        d(a_n,b_n) &\leq d(a_n,x) + d(x,x') + d(x',b_n)\\
                   &< \frac{3\delta}{4}\\
                   &< \delta
      \end{align*}
      Thus, for $n\geq N$, $\rho(f(a_n),f(b_n)) < \varepsilon/2$. By continuity of $\rho$, taking $n\rightarrow\infty$, we get $\rho(\tilde{f}(x),\tilde{f}(x')) < \varepsilon/2$. Therefore, we have $d(x,x') < \delta/4 \Rightarrow d(\tilde{f}(x),\tilde{f}(x')) < \varepsilon$. Therefore, $\tilde{f}$ is uniformly continuous.\\

      Suppose $g: X\rightarrow Y$ is another continuous extension of $f$. Therefore, $g(a) = \tilde{f}(a)$ for all $a\in A$. However, $A$ is dense. Therefore, $g = \tilde{f}$.
  \end{description}
  \subsection{Completion of a Metric Space}%
  Let $(X,d)$ be a fixed metric space. A completion of $X$ is a pair $((Z,\rho),i)$.
  \begin{enumerate}[(i)]
    \item $(Z,\rho)$ is a complete metric space.
    \item $i: X\rightarrow Z$ is an isometry.
    \item $\overline{i(X)}^{\rho} = Z$.
  \end{enumerate}
  For example, the completion of $(0,1)$ is $\left(([0,1],|\cdot|),i(t) = t\right)$.
  \subsection{Isometric Isomorphism of Completions}%
  Given $\left((Z,\rho),i\right)$ and $\left((Z',\rho'),j\right)$ completions of $X$, then there exists a unique isometric isomorphism $\varphi: Z\rightarrow Z'$ such that the following diagram commutes.
  \begin{center}
    % https://tikzcd.yichuanshen.de/#N4Igdg9gJgpgziAXAbVABwnAlgFyxMJZABgBoBGAXVJADcBDAGwFcYkQANEAX1PU1z5CKcqWLU6TVuwBaPPiAzY8BIqIBMEhizaIQMgOQ8JMKAHN4RUADMAThAC2SMiBwQkoyTvZZ5N+06ILm5I6jSM9ABGMIwACgIqwiC2WGYAFjggNNrSegBWfiB2jh40IYhhXrkgADo1DLZoab7clNxAA
    \begin{tikzcd}
                                       & Z \arrow[dd, "\varphi"] \\
    X \arrow[ru, "i"] \arrow[rd, "j"'] &                         \\
                                       & Z'                     
    \end{tikzcd}
  \end{center}
  \subsection{Corollary: Isometric Map and Completion of Metric Space}%
  If $(X,d)$ is a metric space, and $i: (X,d)\rightarrow (Y,\rho)$ is an isometry into a complete metric space, then $\left((\overline{i(X)},\rho),i\right)$ is the completion of $X$.
  \subsection{Theorem: Every Metric Space has a Completion}%
  Consider the Banach space $(C_{b}(X),\norm{\cdot}_u)$. We embed $X \hookrightarrow C_b(X)$ as follows. Fix $x_0 \in X$. Given $x\in X$, $i(x) = X\rightarrow \mathbb{F}$ where $i(x)(t) = d(t,x) - d(t,x_0)$.\\

  Clearly, $i(x)$ is continuous for all $x$ as the distance function is continuous. Also,
  \begin{align*}
    |i(x)(t)| &= |d(t,x) - d(t,x_0)|\\
              &\leq d(x,x_0)\\
    \norm{i(x)}_{u} &\leq d(x,x_0).
  \end{align*}
  We need only show that $i(x)$ is an isometry. 
  \begin{align*}
    \norm{i(x)-i(y)}_{u} &= \sup_{t\in X}|i(x)(t)-i(y)(t)|\\
                         &= \sup_{t\in X}|d(t,x) - d(t,y)|\\
                         &= d(x,y).
  \end{align*}
  \subsection{Nowhere Dense Sets}%
  Let $(X,d)$ be a metric space. Recall that a subset $A$ if $(\overline{A})^{\circ} = \emptyset$. For example, $G = \{(x,y)\in \R^2\mid y = x^2\}$ is nowhere dense.
  \subsubsection{Proposition: Equivalent Conditions for Nowhere Dense Sets}%
  For a $A\subseteq X$, the following are equivalent:
  \begin{enumerate}[(i)]
    \item $A$ is nowhere dense.
    \item $\exists F\subseteq X$ closed with $F^{\circ} = \emptyset$, $A\subseteq F$.
    \item $\exists U\subseteq X$ open and dense with $U\subseteq A^{c}$.
  \end{enumerate}
  \begin{description}
    \item[Proof:]\hfill
      \begin{description}[font=\normalfont]
        \item[(i) $\Rightarrow$ (ii):] Take $F = \overline{A}$.
        \item[(ii) $\Rightarrow$ (i):] $\overline{A}\subseteq\overline{F}$, so $\overline{A}^{\circ}\subseteq \overline{F}^{\circ} = \emptyset$
        \item[(ii) $\Rightarrow$ (iii):] Take $U = F^{c}$. Note that $U = F^{c}\subseteq A^{c}$. Then, $\overline{U} = \overline{F^c} = (F^{\circ})^{c} = X$. Therefore, $U$ is dense and open, and $U$ is contained in $A^{c}$.
        \item[(iii) $\Rightarrow$ (ii):] Take $F = U^c$.
      \end{description}
  \end{description}
  A point $x\in X$ is isolated if $\exists \varepsilon > 0$ such that $U(x,\varepsilon) = \{x\}$.
  \subsubsection{Proposition: Extension of Nowhere Dense Sets}%
  Let $(X,d)$ be a metric space.
  \begin{enumerate}[(i)]
    \item If $A\subseteq X$ is nowhere dense and $B\subseteq A$, then $B$ is nowhere dense.
    \item If $A\subseteq X$ is nowhere dense, then $\overline{A}$ is nowhere dense.
    \item Let $A_1,\dots,A_n$ be nowhere dense. Then, $\bigcup A_i$ is nowhere dense.
    \item If $X$ has no isolated points, then every finite set is nowhere dense.
  \end{enumerate}
  \begin{description}
    \item[Proof:]\hfill
      \begin{enumerate}[(i)]
        \item $B\subseteq A$ implies $\overline{B}\subseteq \overline{A}$, so $\overline{B}^{\circ} = \emptyset$, so $B$ is nowhere dense.
        \item If $A$ is nowhere dense, then $\overline{\overline{A}}^{\circ} = \overline{A}^{\circ} = \emptyset$.
        \item Let $A_1$ and $A_2$ be nowhere dense. By the alternate characterization, $U_1\subseteq A_1^c$, where $U_1$ is open and dense. Similarly, $U_2 \subseteq A_2^c$, where $U_2$ is open and dense.
          \begin{align*}
            \left(A_1 \cup A_2\right)^{c} &= A_1^{c}\cap A_2^{c}\\
                                          &\supseteq U_1\cap U_2
          \end{align*}
          We know $U_1\cap U_2$ is open. We claim that $U_1\cap U_2$ is dense.\\

          Let $x\in X$, $\varepsilon > 0$. We want to show that $U(x,\varepsilon) \cap (U_1\cap U_2) \neq \emptyset$. Since $U_1$ is dense, we know $U_1\cap U(x,\varepsilon)\neq \emptyset$. Let $z\in U_1\cap U(x,\varepsilon)$. Therefore, $\exists \delta > 0$ such that $U(z,\delta)\subseteq U_1\cap U(x,\varepsilon)$. Since $U_2$ is dense, $U(z,\delta)\cap U_2 \neq \emptyset$. Therefore, $\emptyset \neq U(z,\delta)\cap U_2\subseteq U(x,\varepsilon)\cap (U_1\cap U_2)$.\\

          By induction, assuming $A_1\cup\cdots\cup A_{n-1}$ are nowhere dense, then $(A_1\cup \cdots \cup A_{n-1})\cup A_n$ is nowhere dense.
        \item Since $X$ has no isolated points, $\{x\}$ is closed but not open. Therefore, $(\overline{\{x\}})^{\circ} = \emptyset$. Use (iii).
      \end{enumerate}
  \end{description}
  \begin{description}
    \item[Remark:] Note that $\Q$ is not nowhere dense, but $\Q$ is the countable union of nowhere dense sets.
  \end{description}
  \subsection{Meager Sets}%
  Let $(X,d)$ be a metric space.
  \begin{enumerate}[(i)]
    \item $A\subseteq X$ is meager if $A$ is the countable union of nowhere dense sets. Or, $A$ is of the first category.
    \item $B\subseteq X$ is called residual if $B^c$ is meager.
      \begin{description}
        \item[Examples:] $\Q\subseteq \R$ is meager, so $\R\setminus \Q\subseteq \R$ is residual. $\Z\subseteq \R$ is meager, but $\Z\subseteq \Z$ is not meager.
      \end{description}
  \end{enumerate}
  \subsubsection{Proposition: Extension of Meager Sets}%
  \begin{enumerate}[(i)]
    \item If $A$ is meager, and $B\subseteq A$, then $B$ is meager.
    \item If $A_k$ is meager for $k=1,\dots$, then $A_k$ is meager.
    \item If $X$ has no isolated points, then every countable set is meager.
  \end{enumerate}
  \begin{description}
    \item[Proof:]\hfill
      \begin{enumerate}[(i)]
        \item $A = \bigcup A_k$, with $A_k$ nowhere dense. Then, $B = B \cap A = \bigcup B\cap A_k$.
        \item Each $A_k$ is meager, meaning $A_k = \bigcup A_{k_j}$ with $A_{k_j}$ nowhere dense. Thus, $A = \bigcup A_k$ is the countable union of $A_{k_j}$. Thus, $A$ is meager.
        \item Since singleton sets are nowhere dense, we write the countable set as the union of singleton sets.
      \end{enumerate}
  \end{description}
  \subsection{Proposition: Cantor's Intersection Theorem}%
  Let $(X,d)$ be a complete metric space, and $F_1\supseteq F_2\supseteq \cdots$ be a sequence of closed, nonempty sets with $\left(\text{diam}(F_n)\right)_n \rightarrow 0$. Then, $\bigcap F_n = \{x\}$ for some $x\in X$.
  \begin{description}
    \item[Proof:] Let $x_n\in F_n$ for $n\geq 1$. Note that $(x_n)_n$ is Cauchy. For $\varepsilon > 0$, let $N$ be large such that $n\geq N \Rightarrow \text{diam}(F_n) < \varepsilon$. For $m,n \geq N$, $d(x_n,x_m) < \varepsilon$ because $x_n,x_m\in F_N$. Therefore, $(x_n)_n\rightarrow x$ for $x\in X$.\\

      We claim that $\{x\} = \bigcap F_n$. To see this, fix $m\in \N$, and consider $(x_{m+k})_{k}\in F_m$. The tail sequence $(x_{m+k})_k \rightarrow x$. Since $F_m$ is closed, we know $x\in F_m$. Therefore, since $m$ is arbitrary, $x\in \bigcap F_n$.\\

      Now, suppose $\exists x,x'\in \bigcap F_n$ distinct. Then, $d(x,x') > 0$. However, $\exists N\in \N$ large with $\text{diam}(F_N) < d(x,x')$. However, $x,x'\in F_N$, which is a contradiction. Therefore, $\bigcap F_n = \{x\}$.
  \end{description}
  \subsection{Baire's Theorem}%
  Let $(X,d)$ be a complete metric space.
  \begin{enumerate}[(i)]
    \item If $\{V_k\}_{k\geq 1}$ is a countable family of open and dense subsets, then $\bigcap V_k$ is dense.
    \item $X$ is not meager.
  \end{enumerate}
  \begin{description}
    \item[Proof:] \hfill
      \begin{enumerate}[(i)]
        \item Let $U_0$ be any open ball. Since $V_1$ is open and dense, $U_0\cap V_1$ is open and nonempty. So, $\exists U_1$ with $B_1 = \overline{U_1} \subseteq U_0\cap V_1$. We can assure that $\text{diam}(B_1) < 1$.\\

          Consider $U_1\cap V_2$. Since $V_2$ is dense and open, $U_1\cap V_2$ is open and nomempty. Therefore, there must be $B_2 = \overline{U_2} \subseteq U_1\cap V_2$. We can insure that $\text{diam}(B_2) < 1/2$.\\

          Now, with $U_2\cap V_3$, we have $B_3 = \overline{U_3}\subseteq U_2\cap V_3$, with $\text{diam}(B_3) < 1/3$.\\

          Inductively, we have $U_1,\dots,U_{n-1}$ and $B_1,\dots,B_{n-1}$, we see that $U_{n-1}\cap V_n$ is open and nonempty, so we have $U_n$ with $B_n = \overline{U_n}\subseteq U_{n-1}\cap V_n$, with $\text{diam}(B_n) < 1/n$.\\

          Observe that we have $B_1\supseteq U_1 \supseteq B_2\supseteq U_2\cdots$. In particular, $\{B_n\}_{n\geq 1}$ is a nested sequence of closed sets with $\text{diam}(B_n) \rightarrow 0$. Therefore, $\bigcap B_n = \{x\}$.\\

          We claim that $x\in U_{0}\cap \left(\bigcap V_k\right)$. Note that $B_n \subseteq U_{n-1}\cap V_n \subseteq V_n$, Therefore, $x\in \bigcap B_n$ implies $x\in \bigcap V_n$. Also, $x\in B_1=\overline{U_1} \subseteq U_0\cap V_n \subseteq U_0$. Therefore, $\bigcap V_k$ is dense.
        \item Suppose $X = \bigcup A_k$ for $A_k$ nowhere dense. Therefore, $\exists V_k$ open and dense with $V_k\subseteq A_k^c$. Then,
          \begin{align*}
            \emptyset &= X^c\\
                      &= \left(\bigcup A_k\right)^c\\
                      &= \bigcap A_k^c\\
                      &\supseteq \bigcap V_k.
          \end{align*}
          Therefore, by the previous result, $\bigcap V_k$ is open and dense, which is a contradiction. Therefore, $X$ is not meager.
      \end{enumerate}
    \item[Question:] Is $\Q\subseteq \R$ meager? Yes, $\Q$ is the countable union of singleton sets. Is $\R\setminus \Q$ meager? The answer is no --- otherwise, we would write $\R = \Q \cup (\R\setminus \Q)$ would be a union of meager sets, but $\R$ is complete.
  \end{description}
  \subsection{Applying Baire's Theorem}%
  Let $(X,d)$ be a metric space.
  \begin{enumerate}[(i)]
    \item $G\subseteq X$ is a $G_{\delta}$-set if
      \begin{align*}
        G = \bigcap_{k\geq 1} V_k
      \end{align*}
      with $V_k$ open.
    \item $F\subseteq X$ is a $F_{\sigma}$-set if
      \begin{align*}
        F = \bigcup_{k\geq 1}C_k
      \end{align*}
      with $C_k$ closed.
  \end{enumerate}
  For example, $\Q\subseteq \R$ is $F_{\sigma}$, since $\Q$ is the countable union of singleton sets (which are closed in $\R$). It can be shown that $A$ is $F_{\sigma}$ if and only if $A^{c}$ is $G_{\delta}$.\\ 

  We claim that $\Q$ is not $G_{\delta}$.
  \begin{description}
    \item[Proof:] If $\Q$ is $G_{\delta}$, then $\R\setminus \Q$ is $F_{\sigma}$, so 
      \begin{align*}
        \R\setminus \Q &= \bigcup F_k
      \end{align*}
      for $F_k$ closed. Thus,
      \begin{align*}
        \R &= \Q\setminus \R\setminus\Q\\
           &= \bigcup \{q_k\} \cup \bigcup F_k.
      \end{align*}
      Therefore, $\R$ is the countable union of closed sets. Since $\R$ is complete, by Baire's Theorem, we must have $\{q_k\}^{\circ} \neq \emptyset$, or that $F_k^{\circ} \neq \emptyset$ for some $k$. However, $\{q_k\}^{\circ} = \emptyset$, and $F_k^{\circ} = \emptyset$ since $F_{k}\subseteq \R\setminus \Q$, and $\R\setminus \Q$ cannot contain an interval. Therefore, $\Q$ is not $G_{\delta}$
  \end{description}

  Let $(X,d)$ be a metric space. If $A$ is closed, then $A$ is $G_{\delta}$.
  \begin{description}
    \item[Proof:] Recall $\text{dist}_{A}: X\rightarrow \R$ is continuous. Therefore, $\text{dist}_{A}^{-1}((-1/n,1/n)) = \{x\mid \text{dist}_{A}(x) < 1/n\}$ is open. Recall that $x\in A$ if and only if $\text{dist}_{A}(x) = 0$.\\

      Therefore, we can write 
      \begin{align*}
        A &= \bigcap_{n\geq 1}\{x\mid \text{dist}_{A}(x) < 1/n\}.
      \end{align*}
      Therefore, $A$ is $G_{\delta}$.
  \end{description}
  It follows that if $A$ is open, then $A$ is $F_{\sigma}$.
  \subsection{Theorem: Set of Continuities}%
  Let $f: (X,d)\rightarrow (Y,\rho)$ be a map. Then, $C_f := \{x\in X\mid f\text{ is continuous at }x\}$ is a $G_{\delta}$ set.
  \subsubsection{Oscillation of a Function}%
  Let $f: (X,d) \rightarrow (Y,\rho)$. Fix $x_0\in X$. The oscillation $\omega_{f}(x_0) = \text{inf}_{\delta > 0}\text{diam}(f(U(x,\delta)))$, or
  \begin{align*}
    \omega_f(x_0) &= \inf_{\delta > 0}\left(\sup_{x,x'\in U(x,\delta)}\rho(f(x),f(x'))\right).
  \end{align*}
  Note that $\omega_f(x_0) \in [0,\infty]$.
  \begin{enumerate}[(i)]
    \item $f$ is continuous at $x_0$ if and only if $\omega_f(x_0) = 0$.
    \item Given $c > 0$, $\{x\mid \omega_f(x_0) < c\}\subseteq X$ is open.
  \end{enumerate}
  \begin{description}
    \item[Proof:]\hfill
      \begin{enumerate}[(i)]
        \item Suppose $f$ is continuous at $x_0$. Let $\varepsilon > 0$. Then, $\exists \delta > 0$ such that $d(x,x_0) < \delta \Rightarrow \rho(f(x),f(x_0)) < \varepsilon/2$. Therefore,
          \begin{align*}
            \text{diam}(f(U(x_0,\delta))) \leq \varepsilon,
          \end{align*}
          since for $x,x'\in U(x_0,\delta)$, we have
          \begin{align*}
            \rho(f(x),f(x')) &\leq \rho(f(x),f(x_0)) + \rho(f(x_0),f(x'))\\
                             &< \varepsilon.
          \end{align*}
          In particular, $\omega(f(x_0)) \leq \varepsilon$. Since $\varepsilon$ was arbitrary, we have $\omega_f(x_0) = 0$.\\

          Suppose $\omega_f(x_0) = 0$. Let $\varepsilon > 0$. By the property of infimum, then $\exists \delta > 0$ such that
          \begin{align*}
            \text{diam}(f(U(x_0,\delta))) < \varepsilon.
          \end{align*}
          In particular, if $d(x,x_0) < \delta$, then $\rho(f(x),f(x_0)) < \varepsilon$. Thus, $f$ is continuous at $x_0$.
        \item Let $V = \{x\mid \omega_f(x_0) < c\}$. Let $x_0\in V$. Since $x_0\in V$, $\omega_f(x_0) < c$. By the property of infimum, $\exists \delta > 0$ such that $\text{diam}(f(U(x_0,\delta))) < c$. Let $\varepsilon = \delta/2$. We claim that $U(x_0,\varepsilon) \subseteq V$.\\

          Let $z\in U(x_0,\varepsilon)$. Note that $U(z,\delta/2) \subseteq U(x_0,\delta)$. Therefore, $f(U(z,\delta/2)) \subseteq f(U(x_0,\delta))$. Thus, $\text{diam}(f(U(z,\delta))) \leq \text{diam}(f(U(x_0,\delta))) < c$.\\

          By property of oscillation, $\omega_f(z) < c$. So, $U(x_0,\varepsilon)\subseteq V$.
      \end{enumerate}
  \end{description}
  \begin{description}
    \item[Proof of Theorem:]
      \begin{align*}
        C_f &= \{x\mid f\text{ is continuous at }x\}\\
            &= \bigcap_{n\geq 1}\underbrace{\{x\mid \omega_f(x) < 1/n\}}_{\text{open sets}}
      \end{align*}
      meaning $x\in C_f \leftrightarrow \omega_f(x) = 0 \leftrightarrow \omega_f(x) < 1/n$ for all $n$.
  \end{description}
  \subsection{Applying Set of Continuities}%
  There does exist a function continuous at every irrational point and discontinuous at every rational point. Recall from Real Analysis that such $f$ is
  \begin{align*}
    f(x) &= \begin{cases}
      0 & x\in \R\setminus \Q\\
      \frac{1}{q} & x = \frac{p}{q}\text{ in lowest terms}
    \end{cases}
  \end{align*}
  However, there does not exist $f: \R\rightarrow \R$ with $C_f = \Q$, since the set of continuities is always a $G_{\delta}$ set.
  \subsection{Nowhere Differentiable Functions}%
  Does there exist a function $f: [0,1]\rightarrow \R$ such that $f$ is continuous on $[0,1]$ but differentiable nowhere? The answer is yes.
  \begin{align*}
    f(x) &= \sum_{n\geq 1} a^n\cos(b^n x),
  \end{align*}
  where $0 < a < 1$ and $ab > 1$ is such a function. This is known as the Weierstrass function.\\

  Such functions are not rare at all.\\

  In the complete normed vector space $X = (C[0,1],\norm{\cdot}_u)$, $\{f\in X\mid f\text{ differentiable nowhere}\}$ is the complement of a meager set (meaning it is topologically ``big'').
  \section{Compactness}%
  Compactness can best be analogized to finite dimensionality in a metric space.\\

  Let $(X,d)$ be a metric space, and let $K\subseteq X$. 
  \begin{enumerate}[(1)]
    \item A cover for $K$ is a family of subsets $\mathcal{U} = \{U_i\}_{i\in I}\subseteq \mathcal{P}(X)$ with $K \subseteq \bigcup U_i$.\\

      The cover $\mathcal{U}$ is called an open cover if each $U_i \subseteq X$ is open. The cover $\mathcal{U}$ is called finite if $I$ is finite. If $\mathcal{U}$ is a cover for $K$, a subcover of $\mathcal{U}$ is a subfamily $\mathcal{V} = \{U_i\}_{i\in J}$, with $J\subseteq I$, and $K \subseteq \bigcup_{j\in J}U_j$.
    \item $K$ is called compact if every open cover of $K$ admits a finite subcover. If $\{U_i\}_{i\in I}$ is any family that covers $K$, then there exists a finite $F\subseteq I$ such that $\{U_i\}_{i\in F}$ covers $K$.
  \end{enumerate}
  For example, the set $(0,1]\subseteq \R$ is not compact, because
  \begin{align*}
    (0,1] \subseteq \bigcup_{n\in \N}(1/n,3/2)
  \end{align*}
  does not admit a finite subcover.\\

  Any finite set is compact.\\

  A discrete metric space is $X$ is compact if and only if $X$ is finite.\\

  Let $(X,d)$ be a metric space, and $Y\subseteq X$. Let $K\subseteq Y$; $K$ is compact in $X$ if and only if $K$ is compact in $Y$. This can be shown by taking the relative topology of $Y$ on every open cover of $K$ in $X$.
  \subsection{Proposition: Properties of Compactness}%
  Let $(X,d)$ be a metric space.
  \begin{enumerate}[(1)]
    \item If $K\subseteq X$ is compact, then $K$ is closed and bounded.
    \item If $X$ is a compact metric space, and $K\subseteq X$ is closed, then $K$ is compact.
  \end{enumerate}
  \begin{description}
    \item[Proof of (2):] Let $K\subseteq \bigcup U_i$, with $U_i\subseteq X$ open. Then, $X = (X\setminus K) \cup \left(\bigcup_{i\in I} U_i\right)$. This is an open cover for $X$, meaning it admits a finite subcover $F\subseteq I$ such that $X = (X\setminus K) \cup \bigcup_{i\in F}U_i$. Clearly, $K\subseteq \bigcup_{i\in F}U_i$. Thus, $K$ is compact.
    \item[Proof of (1):] Let $K\subseteq X$ be compact. Then,
      \begin{align*}
        K &\subseteq \bigcup_{x\in K}\bigcup U(x,1).
      \end{align*}
      Since $K$ is compact, there exist $\{x_1,\dots,x_n\}$ with $K\subseteq \bigcup_{j=1}^{n}U(x_j,1)$. Let $c = \max d(x_i,x_j)$. If $x,y\in K$, then $x\in U(x_i,1)$ and $y\in U(x_j,1)$ for some $x_i,x_j$. Then,
      \begin{align*}
        d(x,y) &\leq d(x,x_i) + d(x_i,x_j) + d(x_j,y)\\
               &< 1 + c + 1 = 2+c.
      \end{align*}
      Thus, $\text{diam}(K) < \infty$.\\

      We will show that $K^{c}$ is open. Let $x_0 \notin K$. For each $x\in K$, there exist $\delta_x > 0$ with $U(x,\delta_x)\cap U(x_0,\delta_x) = \emptyset$. Then,
      \begin{align*}
        K\subseteq \bigcup_{x\in K}U(x,\delta_x).
      \end{align*}
      Since $K$ is compact, there exist $\{x_1,\dots,x_n\}$ with $K\subseteq \bigcup U(x_j,\delta_{x_j})$. Let $\delta = \min\{\delta_{x_j}\} > 0$. Then, $U(x_0,\delta) \subseteq K^{c}$.
  \end{description}
  \subsection{Proposition: Compactness and Intersections of Closed Sets}%
  Let $(X,d)$ be a metric space. The following are equivalent.
  \begin{enumerate}[(1)]
    \item $X$ is compact;
    \item If $\{C_i\}_{i\in I}$ is a family of closed sets with the finite intersection property (i.e., the intersection of finitely many elements of $\{C_i\}$ is non-empty), then $\bigcap_{i\in I}C_i \neq \emptyset$.
  \end{enumerate}
  \subsection{Proposition: Separability of Compact Metric Spaces}%
  Let $(X,d)$ be a compact metric space. Then, $(X,d)$ is separable.
  \begin{description}
    \item[Proof:] For fixed $n\geq 1$, consider the cover 
      \begin{align*}
        X = \bigcup U(x,1/n).
      \end{align*}
      By compactness, there exist $\{x_{n,1},\dots,x_{n,m_n}\}$ with
      \begin{align*}
        X = \bigcup_{j=1}^{m_n}U(x_{n,j},1/n).
      \end{align*}
      Let $S = \left\{x_{n,j}\mid n\in \N, 1\leq j \leq m_n\right\}$. Then, $S$ is countable.\\

      Let $x\in X$, $\varepsilon > 0$. Let $N$ be large such that $N^{-1} < \varepsilon$. So,
      \begin{align*}
        x\in \bigcup_{j=1}^{m_N} U(x_{N,j},1/N),
      \end{align*}
      so $x\in U(x_{N,j},1/N)$ for some $j$, whence $d(x,x_{N,j}) < 1/N < \varepsilon$, so $x_{N,j} \in U(x,\varepsilon)$. So, $\overline{S} = X$.
  \end{description}
  \subsection{Proposition: Sequential Compactness}%
  Let $(X,d)$ be a metric space, $K\subseteq X$. We say $K$ is sequentially compact if every sequence in $K$ admits a convergent subsequence in $K$.\\

  From Bolzano-Weierstrass, we know that $[a,b]\subseteq \R$ is sequentially compact.\\

  If $K$ is compact, then $K$ is sequentially compact.
  \begin{description}
    \item[Proof:] Let $(x_k)_k\in K$. Let $C_0 = \overline{\{x_1,x_2,\dots\}}$, $C_1 = \overline{\{x_2,x_3,\dots\}}$, etc. such that $C_n = \overline{\{x_{n+1},x_{n+2},\dots\}}$.\\

      Observe that $C_0 \supseteq C_1 \supseteq C_2 \supseteq \cdots$. Additionally, $\{C_n\}$ has the finite intersection property. Since $K$ is compact, the previous proposition states that $\bigcap C_n \neq \emptyset$. Let $x\in \bigcap C_n$.\\

      $x\in C_1$, meaning $\exists k_1 > 1$ with $d(x,x_{k_1}) < 1$. $x\in C_{k_1}$, meaning $\exists k_2 > k_1$ with $d(x,x_{k_2}) < 1/2$. $x\in C_{k_2}$, meaning $\exists k_3 > k_2$ with $d(x,x_{k_3}) < 1/3$. Continuing, we have $(x_{k_j})_j\in K$ with $d(x,x_{k_j}) < 1/j$. Thus, $(x_{k_j})_j\rightarrow x$.
  \end{description}
  If $(X,d)$ is sequentially compact, then $X$ is complete.
  \begin{description}
    \item[Lemma:] If $(x_n)_n$ is Cauchy, and $(x_n)_n$ admits a convergent subsequence, then $(x_n)_n$ is convergent.
    \item[Proof of Lemma:] Given $\varepsilon > 0$, $\exists N\in \N$ such that for $p,q\geq N$, $d(x_p,x_q) < \varepsilon/2$.\\

      Also, suppose $(x_{n_k})_k \rightarrow x$. Then, $\exists K\in \N$ large such that for $k\geq K$, $d(x_{n_k},x) < \varepsilon/2$.\\

      Therefore, for $ n\geq N$, find $k\geq \max\{N,K\}$, we have
      \begin{align*}
        d(x_n,x) &\leq d(x_n,x_{n_k}) + d(x_{n_k},x)\\
                 &< \varepsilon/2 + \varepsilon/2\\
                 &= \varepsilon.
      \end{align*}
    \item[Proof:] If $(X,d)$ is sequentially compact, for $(x_n)_n$ a Cauchy sequence in $(X,d)$, we have that $(x_n)_n$ admits a convergent subsequence. Then, we use the lemma.
  \end{description}
  \subsection{Total Boundedness}%
  Let $(X,d)$ be a metric space. $K\subseteq X$ is totally bounded if $\forall \delta > 0$, $\exists x_1,\dots,x_n\in K$ such that $K \subseteq \bigcup_{i=1}^{n} U(x_i,\delta)$.
  \begin{description}
    \item[Exercise:] If $K$ is totally bounded, then $K$ is bounded. If $L\subseteq K$, and $K$ is totally bounded, then $L$ is totally bounded.
  \end{description}
  \subsection{Sequential Compactness and Total Boundedness}%
  Let $(X,d)$ be a metric space. Let $K\subseteq X$ be sequentially compact. Then, $K$ is totally bounded.
  \begin{description}
    \item[Proof:] Suppose $K$ is not totally bounded. Then, $\exists \delta_0 > 0$ such that $K\nsubseteq \bigcup_{x\in F} U(x,\delta_0)$ for any finite $F$.\\

      Let $x_1 \in K$. Since $K\nsubseteq U(x_1,\delta_0)$, so let $x_2 \in K\setminus U(x_1,\delta_0)$. Since $K\nsubseteq U(x_1,\delta_0) \cup U(x_2,\delta_0)$, let $x_3 \in K\setminus \left(U(x_1,\delta_0) \cup U(x_2,\delta_0)\right)$. Continuing, we find $x_n\in K\setminus \bigcup_{j=1}^{n-1}U(x_j,\delta_0)$.\\

      Thus, we have a sequence $(x_n)_n$. By sequential compactness, $(x_n)_n$ admits $(x_{n_k})_k \rightarrow x \in K$. Since $(x_{n_k})_k$ is convergent, $(x_{n_k})_k$ is Cauchy. But, $d(x_p,x_q) \geq \delta_0$, since, without loss of generality, for  $p>q$, $x_p\notin U(x_q,\delta_0)$. $\bot$
  \end{description}
  \subsection{Corollary: Compact Subsets of Real Numbers}%
  If $K\subseteq \R$ is compact, $\sup K\in K$ and $\inf K \in K$.
  \begin{description}
    \item[Proof:] We can always construct sequences $(x_n)_n\rightarrow \sup K$ and $(y_n)_n\rightarrow \inf K$ in $K$. Since $\sup K < \infty$ and $\inf K < \infty$, since $K$ is compact, and thus bounded.\\

      Since $K$ is also closed, $\sup K\in K $ and $\inf K \in K$.
  \end{description}
  \subsection{Theorem: Equivalence of Compactness Definitions}%
  Let $(X,d)$ be a metric space. The following are equivalent.
  \begin{enumerate}[(1)]
    \item $X$ is compact.
    \item $X$ is sequentially compact.
    \item $X$ is complete and totally bounded.
  \end{enumerate}
  \begin{description}
    \item[Proof:] We proved that $(1)\Rightarrow (2) \Rightarrow (3)$. We will now prove $(3) \Rightarrow (1)$.\\

      Suppose $\mathcal{V}$ is an open cover of $X$ that fails to admit a finite subcover. Let $\varepsilon = 1$. Since $X$ is totally bounded $X = \bigcup_{j=1}^{m_1} U_{1,j}$, where $U_{1,j}$ are open balls of radius $1$.\\

      There must be some open ball among the $U_{1,j}$ not covered by finitely many members of $\mathcal{V}$. Call this ball $U(x_1,1)$. Let $\varepsilon = 1/2$. By total boundedness, $X = \bigcup_{j=1}^{m_2}U_{2,j}$, where $U_{2,j}$ are open balls of radius $1/2$. Then, $U(x_1,1) = \bigcup \left(U(x_1,1)\cap U_{2,j}\right)$. So, there must be an open ball of radius $1/2$, $U(x_2,1/2)$, such that $U(x_1,1)\cap U(x_2,1/2)$ cannot be covered by finitely many members of $\mathcal{V}$.\\

      Continuing, we have a sequence $(x_n)_n$, where $F_n = U(x_1,1)\cap U(x_2,1/2)\cap \cdots \cap U(x_n,1/n)$ cannot be covered by finitely many members of $\mathcal{V}$.\\

      Let $C_n = \overline{F_n}$. Notice that $F_1 \supseteq F_2 \supseteq \dots$, meaning $C_1 \supseteq C_2 \supseteq \dots$. We see that $\text{diam}(C_n) = \text{diam}(F_n) \leq 2/n$. Applying Cantor's intersection theorem, we have $\bigcap C_n = \{x\}$.\\

      Since $\mathcal{V}$ is an open cover, locate $V\in \mathcal{V}$ such that $x\in V$. Since $V$ is open, there exists $\varepsilon > 0$ such that $U(x,\varepsilon) \subseteq V$. Choose $N$ large such that $2/N < \varepsilon$. Since $x\in C_N$, $d(z,x) \leq 2/N < \varepsilon$ for all $z\in C_N$, meaning $F_N \subseteq C_N \subseteq U(x,\varepsilon)\subseteq V$.\\

      Therefore, $\{V\}$ is a cover for $F_N$. $\bot$
  \end{description}
  \subsection{Proposition: Multi-dimensional Bolzano-Weierstrass Theorem}%
  Let $\mathcal{R} = [a_1,b_1]\times [a_2,b_2]\times \cdots \times [a_d,b_d] = \prod_{j=1}^{d}[a_j,b_j]\subseteq \ell_p^{d}$. Then, $\mathcal{R}$ is sequentially compact, so $\mathcal{R}$ is compact.
  \begin{description}
    \item[Proof:] The proof in $\R^{d}$ works similarly to the proof in $\R^2$. Consider $\pi_x: \R^2 \rightarrow \R$ and $\pi_y: \R^2 \rightarrow \R$. We saw that $(v_{n})_n\rightarrow v$ in $\ell_{p}^{2}$ if and only if $(\pi_x(v_n))_n \rightarrow \pi_x(v)$ and $(\pi_y(v_n))_n \rightarrow \pi_y(v)$.\\

      If $(v_n)_n\in \mathcal{R}$, then $(\pi_x(v_n))_n\in [a_1,b_1]$. By Bolzano-Weierstrass, there is a convergent subsequence $(\pi_x(v_{n_k}))_k \rightarrow x\in [a_1,b_1]$.\\

      Now, consider $\left(\pi_y(v_{n_k})\right)_{k} \in [a_2,b_2]$. By Bolzano-Weierstrass, there is a convergent subsequence $\left(\pi_y(v_{n_{k_j}})\right)_j\rightarrow y\in [a_2,b_2]$. Thus, $(v_{n_{k_j}})_j\rightarrow (x,y)$ in $\mathcal{R}$.
  \end{description}
  \subsection{Heine-Borel Theorem}%
  Let $K\subseteq \R^{d}$. The following are equivalent:
  \begin{enumerate}[(i)]
    \item $K$ is compact;
    \item $K$ is sequentially compact;
    \item $K$ is closed and bounded.
  \end{enumerate}
  \begin{description}
    \item[Proof:] We have (i) $\Leftrightarrow$ (ii), and (i) $\Rightarrow$ (iii). We will show (iii) $\Rightarrow$ (ii).\\

      If $K$ is bounded, then $K\subseteq \mathcal{R} = \prod_{j=1}^{d}[a_j,b_j]$. Let $(v_n)_n$ be a sequence in $K$. By the previous proposition, there exists a subsequence $(v_{n_k})_k\rightarrow v \in \mathcal{R}$. Since $K$ is closed, $v\in K$. Therefore, $K$ is sequentially compact.
  \end{description}
  There are many examples of closed and bounded sets that are not compact (in infinite-dimensional vector spaces).\\

  For example, in $\ell_1 = \left\{a=(a_k)_k\mid \sum_{k=1}^{\infty}|a_k|<\infty\right\}$, we have $e_n = (0,0,\dots,0,1,0,\dots)$, with $1$ at the $n$th coordinate. For the sequence $(e_n)_n$, $\norm{e_k}_1 = 1$ for all $e_k$, so $(e_n)_n\in B_{\ell_1}$, which is closed and bounded. Observe that $\norm{e_n - e_m} = 2$ for all $m\neq n$, so there does not exist a convergent subsequence. Thus, $\ell_1$ is not sequentially compact.
  \begin{description}
    \item[Remark:] We will show that for a normed space, $(V,\norm{\cdot})$, $B_{V}$ is compact if and only if $\text{dim}(V) < \infty$.
  \end{description}
  \subsection{Proposition: Continuous Image of Compact Sets}%
  If $f: (X,d)\rightarrow (Y,\rho)$ is continuous, and $K\subseteq X$ is compact, then $f(K)\subseteq Y$ is compact.
  \begin{description}
    \item[Proof:] Let $\bigcup_{i\in I} V_i$ be an open cover for $f(K)$, where $V_i\subseteq Y$ open. Taking the preimage, we have
      \begin{align*}
        K&\subseteq f^{-1}(f(K))\\
         &\subseteq f^{-1}\left(\bigcup_{i\in I}V_i\right)\\
         &= \bigcup_{i\in I}f^{-1}(V_i)\\
         \intertext{since $f$ is continuous, $f^{-1}(V_i)\subseteq X$ are open. By compactness, there exists $F\subseteq I$ finite such that}
        K&\subseteq \bigcup_{i\in F}f^{-1}(V_i).\\
        \intertext{Taking the image, we have}
        f(K) &\subseteq f\left(\bigcup_{i\in F}f^{-1}(V_i)\right)\\
             &= \bigcup_{i\in F}f(f^{-1}(V_i))\\
             &= \bigcup_{i\in F}V_i.
      \end{align*}
      Thus, $f(K)$ has a finite subcover.
  \end{description}
  \subsection{Corollary: Compactness under Topologically Equivalent Metrics}%
  Let $d_1$ and $d_2$ be topologically equivalent ($\text{id}_{X}: (X,d_1) \rightarrow (X,d_2)$ is a homeomorphism). Then, $K\subseteq X$ is $d_1$-compact if and only if $K$ is $d_2$-compact.
  \subsection{Corollary: Heine-Borel Theorem Extension}%
  For $K\subseteq \ell_{p}^{n}$, $K$ is compact if and only if $K$ is closed and bounded.
  \subsection{Extreme Value Theorem}%
  Let $(X,d)$ be a metric space, $K\subseteq X$ compact, and $f: X\rightarrow \R$ continuous. Then, $\sup_{x\in X}f(x)=f(x_{M})$ and $\inf_{x\in X}f(x) = f(x_{m})$ for some $x_M,x_m\in K$.
  \begin{description}
    \item[Proof:] We know that $f(K)\subseteq \R$ is compact. Then, $\inf f(K)$ and $\sup f(K)$ are elements of $f(K)$.
  \end{description}
  \subsection{Proposition: Compactness of Closed Unit Ball}%
  Let $V$ be a finite-dimensional vector space over $\mathbb{F}$.
  \begin{enumerate}[(1)]
    \item All norms on $V$ are equivalent.
    \item For any norm, $\norm{\cdot}$ on $V$, $B_{(V,\norm{\cdot})} = \{v\in V\mid \norm{v}\leq 1\}$ is compact.
  \end{enumerate}
  \begin{description}
    \item[Proof of (1):] Let $\{v_1,\dots,v_n\}$ be a linear basis for $V$. Define
      \begin{align*}
        \norm{\sum_{j=1}^{n}t_jv_j}_{1} &= \sum_{j=1}^{n}|t_j|.
      \end{align*}
      This is a norm on $V$.\\

      Then, $\varphi: \ell_{1}^{n}\rightarrow V$
      \begin{align*}
        \varphi\left(\sum_{j=1}^{n}t_je_j\right) &= \sum_{j=1}^{n}t_jv_j
      \end{align*}
      is a linear isometric isomorphism. Since $B_{\ell_1^{n}}$ is compact, so too is $\varphi(B_{\ell_1^{n}})$, so $B_{\left(V,\norm{\cdot}\right)}$ is compact.\\

      Then, $S_1 := \{v\in V\mid \norm{v}_1 = 1\}$ is compact since $S_1\subseteq B_{(V,\norm{\cdot})}$ is closed.\\

      Let $\norm{\cdot}$ be any norm on $V$. We will show that $\norm{\cdot}$ is equivalent to $\norm{\cdot}_1$. Note that 
      \begin{align*}
        \norm{\sum_{j=1}^{n}t_jv_j} &\leq \sum_{j=1}^{n}|t_j|\norm{v_j}\\
                              &\leq c\sum_{j=1}^{n}|t_j|\\
                              &= c\norm{\sum_{j=1}^{n}t_jv_j}_{1}\\
        \intertext{where $c = \max\norm{v_j}$. Consider $g: (V,\norm{\cdot}_1)\rightarrow \R$, with $g(v) = \norm{v}$.}
        |g(v)-g(w)| &= |\norm{v}-\norm{w}|\\
                    &\leq \norm{v-w}\\
                    &\leq c\norm{v-w}_{1}\\
                    \intertext{so $g$ is Lipschitz, and thus continuous. $S_1$ is compact in $(V,\norm{\cdot})$, so by the extreme value theorem, $\inf_{v\in S_1}g(v) = g(v_0) = \norm{v_0}$ for some $v_0\in S_1$. Note that $D := \norm{v_0} > 0$, else $v_0 = 0$. Thus, $g(v)\geq D$ for all $v\in S_1$}
        \norm{v} &\geq D \tag*{$\forall v\in S_1$}\\
        \intertext{Let $0\neq v$. Then,}
        \frac{v}{\norm{v}_1} &\in S_1\\
        \norm{\frac{v}{\norm{v}_1}} &\geq D\\
        \intertext{so}
        \norm{v} &\geq D\norm{v}_1.
      \end{align*}
      Therefore, we have $\norm{v}_1\leq \frac{1}{D}\norm{v}$. Thus, any two norms on $V$ are equivalent.
    \item[Proof of (2):] Exercise.
  \end{description}
  \subsection{Corollary: Finite-Dimensional Subspaces}%
  Let $V$ be a normed space, and $W\subseteq V$ finite-dimensional. Then, $W\subseteq V$ is closed.
  \begin{description}
    \item[Proof:] We know there is a linear uniformism $\varphi: W\rightarrow \ell_{1}^{n}$, for $\text{dim}(W) = n$. If $(w_n)_n\rightarrow v\in V$, where $(w_n)_n\in W$, then $(w_n)_n$ is Cauchy. Therefore, $\left(\varphi(w_n)\right)_n$ is Cauchy in $\ell_1^{n}$. Since $\ell_1^{n}$ is complete, $\left(\varphi(w_n)\right)_n\rightarrow z\in \ell_1^{n}$. Since $\varphi^{-1}$ is uniformly continuous, $(w_n)_n = \left(\varphi^{-1}\left(\varphi(w_n)\right)\right)_n \rightarrow \varphi^{-1}(z)\in W$. Thus, $\varphi^{-1}(z) = v$, so $v\in W$.
  \end{description}
  \subsection{Proposition: Uncountable Basis of Banach Space}%
  If $V$ is an infinite-dimensional Banach space, then $\text{dim}(V)$ is uncountable.
  \begin{description}
    \item[Proof:] Let $\{e_n\}$ be a linearly independent set. Let $W_n = \text{span}\{e_1,\dots,e_n\}$. So, $W_n$ is closed, and $W_n\neq V$. We can see that $W_1\subseteq W_2 \subseteq \cdots$.\\

      We claim that $W_n^{\circ} = \emptyset$. Suppose $\exists U(x,\varepsilon) \subseteq W_n$ for some $\varepsilon > 0$. Given any $v\in V$ with $v\neq 0$, we take $\frac{\varepsilon}{2}\frac{v}{\norm{v}} + x \in W_n$. Thus, we have $\frac{\varepsilon}{2}\frac{v}{\norm{v}}\in W_n$, so $v\in W_n$, meaning $V\subseteq W_n$.\\

      By Baire's Theorem, $\bigcup W_n \neq V$.
  \end{description}
  \subsection{Proposition: Compact Unit Ball and Finite Dimensions}%
  Let $V$ be a normed space, and $B_V := \{v\mid \norm{v}\leq 1\}$. The following are equivalent:
  \begin{enumerate}[(i)]
    \item $B_V$ is compact;
    \item $\text{dim}(V) < \infty$.
  \end{enumerate}
  \begin{description}
    \item[Riesz's Lemma:] Let $V$ be a normed space, and $W$ a proper closed subspace. For every $t\in (0,1)$, there exists $v_t\in V$ with $\norm{v_t} = 1$ and $\text{dist}_W(v_t)\geq t$.
    \item[Proof of Riesz's Lemma:] Find $v_0\in V\setminus W$. We know $\text{dist}_W(v_0) := \delta > 0$. Recall that $\text{dist}_W(v_0) = \inf_{w\in W}\norm{v_0-w}$. Note that $t\delta < \delta$. So, $\delta < \frac{\delta}{t}$. Find $w_0 \in W$ with $\delta \leq \norm{v_0-w_0} < \frac{\delta}{t}$. Let $v_t = \frac{v_0-w_0}{\norm{v_0-w_0}}$. Then, $\norm{v_t} = 1$. We claim that $v_t$ satisfies the lemma.\\

      If $w\in W$ arbitrary, then
      \begin{align*}
        \norm{v_t - w} &= \norm{\frac{v_0-w_0}{\norm{v_0-w_0}}-w}\\
                       &= \frac{1}{\norm{v_0-w_0}}\norm{v_0-\underbrace{\left(w_0 + w\norm{v_0-w_0}\right)}_{\in W}}\\
                       &> \frac{t}{\delta}\cdot \delta\\
                       &= t.
      \end{align*}
      Thus, $\text{dist}_W(v_t) \geq t$.
    \item[Proof:] To show (i) $\Rightarrow$ (ii), we need Riesz's Lemma. Let $B_V$ be compact. Suppose toward contradiction that $\text{dim}(V) = \infty$.\\

      Choose $v_1\in V$ with $\norm{v_1} = 1$. Let $W_1 = \text{span}\{v_1\}\subset V$. Then, $W$ is closed and proper, meaning $\exists v_2\in V$ with $\norm{v_2} = 1$ with $\text{dist}_{W_1}(v_2) \geq 1/2$. Let $W_2 = \text{span}\{v_1,v_2\}$. Then, $W_2$ is a proper, closed subspace, meaning $\exists v_3\in V$ with $\norm{v_3} = 1$ and $\text{dist}_{W_2}(v_3)\geq 1/2$.\\
      
      Continuing, we find $\exists v_n\in V$ with $\norm{v_n} = 1$ and $\text{dist}_{W_{n-1}}(v_n) \geq 1/2$, where $W_{n-1} = \text{span}\{v_1,\dots,v_{n_1}\}$. We have a sequence $(v_n)_n\in B_V$. Since $B_V$ is compact, $\exists (v_{n_k})_k \rightarrow v\in B_V$, meaning $B_V$ is Cauchy. However, since $\norm{v_n-v_m}\geq 1/2$ for all $n$ and $m$. $\bot$
  \end{description}
  \subsection{Proposition: Compact Domain and Uniform Continuity}%
  If $f: (X,d)\rightarrow (Y,\rho)$ is continuous, and $X$ is compact, then $f$ is uniformly continuous.
  \begin{description}
    \item[Proof:] Let $\varepsilon > 0$. For each $x\in X$, we have $\exists \delta_{x} > 0$ such that for $d(z,x) < \delta_x \Rightarrow \rho(f(z),f(x)) < \varepsilon/2$.\\

      Since $X = \bigcup_{x\in X} U(x,\delta_x/2)$, by compactness, we have $x_1,\dots,x_n$ with $X = \bigcup_{j=1}^{n}U(x_j,\delta_{x_j}/2)$. Take $\delta = \min\{\delta_{x_j}/2\}$.\\

      Let $x,x'\in X$ arbitrary with $d(x,x') < \delta$. Locate $x\in U(x_j,\delta_{x_j}/2)$ for some $j$. Then,
      \begin{align*}
        d(x',x_j) &\leq d(x',x) + d(x,x_j)\\
                  &< \delta + \delta_{x_j}/2\\
                  &\leq \delta_{x_j}.
      \end{align*}
      Therefore,
      \begin{align*}
        \rho(f(x),f(x')) &\leq \rho(f(x),f(x_j)) + \rho(f(x_j),f(x'))\\
                         &< \varepsilon/2 + \varepsilon/2\\
                         &= \varepsilon.
      \end{align*}
  \end{description}
  \subsection{Compactness and Uniform Convergence}%
  \begin{enumerate}[(1)]
    \item Let $f_n: (0,1)\rightarrow \R$ with $f_n(t) = t^n$. Pointwise, $(f_n)_n \rightarrow \mathbb{0}$, meaning for $(f_n(t))_n \rightarrow \mathbb{0}(t) = 0$ for all $t\in (0,1)$. However, the convergence is not uniform. We have $\norm{f_n-\mathbb{0}}_{u} = \norm{f_n}_u = 1$.\\

      Note that $f_n(t)$ decreases pointwise to $0$ for all $t\in (0,1)$, meaning $f_1(t) \geq f_2(t) \geq f_3(t)\geq \cdots$.
    \item Consider the sequence of functions defined by
      \begin{align*}
        f_n(x) &= \begin{cases}
          0 & x\in (-\infty,n)\\
          x-n & x\in [n,n+1]\\
          1 & x\in (n_1,\infty)
        \end{cases}.
      \end{align*}
      Notice that $f_n(t)$ is decreasing in $n$ for all $t$ and $(f_n)_n \rightarrow \mathbb{0}$ pointwise, but convergence is not uniform, as $\norm{f_n}_u = 1$ for all $n$.
  \end{enumerate}
  \subsubsection{Dini's Theorem}%
  If $(X,d)$ is a compact metric space, and $\left(f_n: X\rightarrow \R\right)_n$ is a sequence of continuous real-valued functions with $\forall x\in X$, $(f_n(x))_n \rightarrow 0$ is decreasing. Then, $(f_n)_n \rightarrow \mathbb{0}$ uniformly.
  \begin{description}
    \item[Proof:] Let $\varepsilon > 0$. For each $n\geq 1$, take $U_n = \{x\mid f_n(x) < \varepsilon/2\}$. Then $U_n = f_n^{-1}((-\infty,\varepsilon/2))$. Since $f_n$ is continuous, and $(-\infty,\varepsilon/2)$, so too is $U_n$ in $X$.\\

      Notice that $U_1\subseteq U_2\subseteq \cdots$, as if $x\in U_n$, then $f_{n+1}(x) \leq f_n(x) < \varepsilon/2$, meaning $x\in U_{n+1}$. Then, we have that $\bigcup U_n = X$, as for all $x$, $f_n(x)\rightarrow 0$. Since $X$ is compact, we have $X = \bigcup U_{n_k} = U_{n_K}$. For any $x\in X$, $f_{n_K}(x) < \varepsilon/2$. Thus, $\norm{f_{n_K}} \leq \varepsilon/2 < \varepsilon$, so we have uniform convergence.
  \end{description}
  \subsection{Compactness in $C(X)$}%
  If $X$ is a compact metric space, then, by the Extreme Value Theorem, $C(X) = C_b(X)$. We can see that $C_b(X)$ is complete under $\norm{\cdot}_u$. We may ask when $\mathcal{F}\subseteq C(X)$ is compact.\\

  A family $\mathcal{F}\subseteq C(X)$ is equicontinuous if and only if $\forall \varepsilon > 0, \exists \delta > 0$ such that $\forall x,y\in X$ with $d(x,y) < \delta$, then $|f(x)-f(y)| < \varepsilon$ for all $f\in \mathcal{F}$.
  \begin{description}
    \item[Exercise:] For $\mathcal{F}\subseteq C(X)$ with $\mathcal{F}$ finite, then $\mathcal{F}$ is always equicontinuous.\\

      Since every $f\in \mathcal{F}$ is uniformly continuous, take the minimum value of $\delta$.
  \end{description}
  \subsection{Arzelà-Ascoli Theorem}%
  Let $(X,d)$ be a compact metric space. The family $\mathcal{F}\subseteq C(X)$ is compact if and only if $\mathcal{F}$ is closed, bounded, and equicontinuous.
  \begin{description}
    \item[Proof:] Let $\mathcal{F}$ be compact. Then, $\mathcal{F}$ is complete, and thus closed and totally bounded, meaning $\mathcal{F}$ is bounded. Thus, we need to show $\mathcal{F}$ is equicontinuous.\\

      Let $\varepsilon > 0$. By total boundedness, $\exists f_1,\dots,f_n\in \mathcal{F}$ with $\mathcal{F}\subseteq \bigcup_{j=1}^{n} U(f_j,\varepsilon/3)$. Each $f_j$ is uniformly continuous since $X$ is compact. Thus, $\exists \delta_j$ with $x,y\in X$ and $d(x,y)\leq \delta_j$, then $|f_j(x)-f_j(y)| < \varepsilon/3$.\\

      Let $\delta = \min\{\delta_j\}$. Given any $f\in \mathcal{F}$, we have $\mathcal{F}\in U(f_j,\varepsilon/3)$ for some $j$. For any $x,y\in X$ with $d(x,y) < \delta$, we have
      \begin{align*}
        |f(x)-f(y)| &\leq |f(x)-f_j(x)| + |f_j(x)-f_j(y)| + |f_j(y)-f(y)|\\
                    &\leq \norm{f-f_j}_u + |f_j(x)-f_j(y)| + \norm{f-f_j}_u\\
                    &< 2\varepsilon/3 + \varepsilon/3\\
                    &= \varepsilon
      \end{align*}
      Let $\mathcal{F}$ be closed, bounded, and equicontinuous. Since $\mathcal{F} \subseteq C(X)$ is closed, $\mathcal{F}$ is complete. We need only show $\mathcal{F}$ is totally bounded.\\

      Let $\varepsilon > 0$. Since $\mathcal{F}$ is equicontinuous, $\exists \delta > 0$ such that for all $x,y\in X$ with $d(x,y) < \delta$, then $|f(x)-f(y)| < \varepsilon/4$ for any $f\in \mathcal{F}$.\\

      Since $X$ is compact, $X$ is totally bounded, so $\exists x_1,\dots,x_n\in X$ with $X\subseteq \bigcup_{j=1}^{n}U(x_j,\delta)$. Consider the set $C_{\mathcal{F}} := \left\{\left(f(x_1),\dots,f(x_n)\right)|f\in\mathcal{F}\right\}\subseteq \R^n$.\\

      Since $\mathcal{F}$ is bounded, we have that $\norm{f}_u \leq M$ for all $f\in \mathcal{F}$ for some $M>0$. Thus, $|f(x_j)|\leq \norm{f}_u \leq M$ for $j = 1,\dots,n$. Thus, $C_{\mathcal{F}}$ is bounded in $\R^n$.
      \begin{description}
        \item[Exercise:] $S\subseteq \R^n$ is bounded if and only if $S$ is totally bounded.
      \end{description}
      Thus, $C_{\mathcal{F}}$ is totally bounded. Therefore, $\exists f_1,\dots,f_m\in \mathcal{F}$ with $C_{\mathcal{F}}\subseteq \bigcup_{i=1}^{m}U\left(\left(f_i(x_1),\dots,f_i(x_n)\right),\varepsilon/4\right)$.\\

      If $f\in \mathcal{F}$, then $\exists i = 1,\dots,m$ (\textasteriskcentered) such that $\norm{\left(f(x_1),\dots,f(x_n)\right)-\left(f_i(x_1),\dots,f_i(x_n)\right)}_1 < \varepsilon/4$. Thus,
      \begin{align*}
        \sum_{j=1}^{n}|f(x_j)-f_i(x_j)| < \varepsilon/4.
      \end{align*}
      We claim that $F\subseteq \bigcup_{i=1}^{m}U(f_i,\varepsilon)$. Let $f\in \mathcal{F}$ and $x\in X$. Pick $i$ as in (\textasteriskcentered), and $j$ with $x\in U(x_j,\delta)$. Then,
      \begin{align*}
        |f(x)-f_i(x)| &\leq |f(x)-f(x_j)| + |f(x_j)-f_i(x_j)| + |f_i(x_j)-f_i(x)|\\
                      &< 3\varepsilon/4\\
                      \intertext{so}
        \norm{f-f_i} &\leq 3\varepsilon/4\\
                     &< \varepsilon.
      \end{align*}
  \end{description}
  \subsection{Stone-Weierstrass Theorem}%
  Let $(X,d)$ be a compact metric space. Suppose $A\subseteq C(X;\R)$ with
  \begin{itemize}
    \item $f,g\in A \Rightarrow f + g\in A$;
    \item $f\in A,\alpha\in \mathbb{F}\Rightarrow \alpha f \in A$;
    \item $f,g\in A\Rightarrow fg\in A$;
    \item $\mathbb{1}_X \in A$;
    \item $A$ is separating --- if $x\neq y$ in $X$, then $\exists f\in A$ with $f(x)\neq f(y)$.
  \end{itemize}
  We say $A$ is a unital separating subalgebra of $C(X)$.\\

  Then, $\overline{A}^{\norm{\cdot}_u} = C(X;\R)$ ($A$ is uniformly dense).
  \subsection{Uniform Approximation by Polynomials}%
  For example, considering $\mathcal{P} = \left\{x\mapsto \sum_{k=0}^{n}a_kx^k\mid a_k\in \R\right\}\subseteq C([0,1])$. We can see that $\mathcal{P}$ is a separating unital subalgebra. Thus, $\mathcal{P}$ is dense.\\

  Let $f(x) = |x|$ on $[-1,1]$. Consider the sequence $P_n(x)$ given by
  \begin{align*}
    P_0(x) &= 0\\
    P_{n+1}(x) &= P_n(x) + \frac{x^2 - (P_n(x))^2}{2}.
  \end{align*}
  For example, $P_1(x) = x^2/2$, $P_2(x) = \frac{x^2}{2} + \frac{x^2 - x^4/4}{2}$. Then, $(P_n)_n \xrightarrow{\norm{\cdot}_u} f$.
  \begin{description}
    \item[Proof:] We claim that $0\leq P_n(x)\leq f(x)$ for all $x\in [-1,1]$. Clearly, $0\leq P_0(x) \leq |x|$, and $0 \leq P_1(x)\leq |x|$. Assume it is the case that $0\leq P_n(x) \leq |x|$. Then,
      \begin{align*}
        0\leq P_n(x) \leq |x|\\
        0\leq P_n^2(x) \leq x^2\\
        x^2 - P_n^2(x) \geq 0\\
        P_{n+1}(x) = P_{n}(x) + \frac{x^2-P_n^2(x)}{2} \geq 0\\
        \intertext{and}
        |x| - P_{n+1}(x) &= |x| - P_n(x) - \frac{|x|^2 - P_n(x)}{2}\\
                         &= |x| - P_n(x) - \frac{\left(|x|-P_n(x)\right)\left(|x| + P_n(x)\right)}{2}\\
                         &= \left(|x|-P_n(x)\right)\left(1-\frac{|x| + P_n(x)}{2}\right)\\
                         &\geq 0
      \end{align*}
      Observe that $P_n(x) \leq P_{n+1}(x)$. For every $x$, $(P_n(x))_n$ is increasing and bounded above by $|x|$. So, $P_n(x) \rightarrow L_x$.
      \begin{align*}
        P_{n+1}(x) &= P_{n}(x) + \frac{x^2 - P_n^2(x)}{2}\\
        L_x &= L_x + \frac{x^2 - L_x^2}{2}\\
        L_x &= \sqrt{x^2} = |x|.
      \end{align*}
      Thus, $(P_n)_n$ converges pointwise on $[-1,1]$. So, $(f-P_n)\rightarrow 0$ is decreasing pointwise. Whence, by Dini's Theorem, $\norm{f-P_n}_u \rightarrow 0$.
  \end{description}
  \section{Connectedness}%
  Let $(X,d)$ be a metric space. 
  \begin{enumerate}[(1)]
    \item Let $Y\subseteq X$. A splitting for $Y$ in $X$ is an inclusion $Y\subseteq U\cup V$, where $U,V\in \tau_X$ with $Y\cap U \cap V = \emptyset$.
      \begin{description}
        \item[Remark:] If we set $U_1 = U\cap Y$ and $V_1 = V\cap Y$, then $U_1$ and $V_1$ are open in $Y$ with the relative topology. We have $Y = U_1\sqcup V_1$. Also note that $U_1$ and $V_1$ are clopen in $Y$.
      \end{description}
    \item A splitting for $Y$ is called trivial if either $Y\cap U = \emptyset$ or $Y\cap V = \emptyset$.
    \item $Y$ is connected in $X$ if every splitting for $Y$ in $X$ is trivial. Otherwise, we say $Y$ is disconnected.
    \begin{description}
      \item[Exercise:] Suppose $C\subseteq Y\subseteq X$. $C$ is connected in $Y$ if and only if $C$ is connected in $X$.
    \end{description}
  \end{enumerate}
  \subsection{Connectedness of Subsets in $\R$}%
  We have $[a,b]\subseteq \R$ is connected.
  \begin{description}
    \item[Proof:] Suppose $[a,b]\subseteq U\cup V$ is a splitting.
      \begin{itemize}
        \item If $a=b$ or $a>b$, clearly the splitting is trivial.
        \item Assume $a< b$. Without loss of generality, $a\in U$. Suppose toward contradiction that $[a,b]\cap V \neq \emptyset$. Set $c = \inf [a,b]\cap V$.\\

          We claim that $a<c$; since $U$ is open, $\exists \varepsilon > 0$ such that $(a-\varepsilon,a+\varepsilon) \subseteq U$. So, $V\cap [a,b]\subseteq [a+\varepsilon,b]$. Therefore, $c \geq a+\varepsilon$. Thus, $[a,c)\subseteq U$.\\

          We claim $c\in V$. Since $U$ is open, we cannot have $c<b$ and $c\in U$. Also, if $c\in U$ and $c = b$, then $[a,b]\cap V = \emptyset$.\\

          Since $V$ is open, $\exists \delta > 0$ with $(c-\delta,c+\delta)\subseteq V$. However, this means $c\neq \inf V\cap [a,b]$.\\

          Thus, $V\cap [a,b] = \emptyset$.
      \end{itemize}
  \end{description}
  We have that $\Q\subseteq \R$ is disconnected.
  \begin{description}
    \item[Proof:] We have $\Q\subseteq (-\infty,\pi)\cup(\pi,\infty)$ is a non-trivial splitting.
  \end{description}
  \subsection{Proposition: Intervals in $\R$}%
  Every interval $I\subseteq \R$ is connected.
  \begin{description}
    \item[Proof:] Let $I\subseteq U\cup V$ be a non-trivial splitting. Therefore, $U\cap I \neq \emptyset$, and $V\cap I \neq \emptyset$. Let $a\in I\cap U$ and $b\in I\cap V$. Without loss of generality, $a < b$.Then, by the definition of an interval, $[a,b]\subseteq I\subseteq U\cup V$.\\

      However, at the same time, $[a,b]\cap U\cap V \subseteq I\cap U \cap V = \emptyset$. So, we have a splitting for $[a,b]$. This splitting for $[a,b]$ is non-trivial, since $[a,b]\cap U \neq \emptyset$ and $[a,b]\cap V \neq \emptyset$. However, we had shown that $[a,b]$ is connected.
  \end{description}
  If $I\subseteq \R$ is connected, then $I$ is an interval.
  \begin{description}
    \item[Proof:] Let $a = \inf I$ and $b = \sup I$. It is possible for $a$ to equal $-\infty$ and $b$ to equal $+\infty$. We claim that $(a,b)\subseteq I$.\\

      If $\exists c\in I$ with $c\notin (a,b)$, then we have a non-trivial splitting $I\subseteq (-\infty,c)\cup (c,\infty)$, which would contradict the assumption that $I$ is connected. Thus, $(a,b)\subseteq I$.\\

      If $s,t\in I$ with $s \leq t$, then $s\geq a$ or $s > a$, or $t\leq b$ or $t < b$. By cases, we find$[s,t]\subseteq I$, meaning $I$ is an interval.
    \item[Exercise:] If $Y\subseteq X$ is connected, then $\overline{Y}$ is connected.
  \end{description}
  \subsection{Connected Components and Clopen Sets}%
  Let $(X,d)$ be a metric space. We define $\sim_{X}$ on $X$ as $x\sim_{X}y$ if there is a connected $C\subseteq X$ with $x,y\in C$. This is an equivalence relation.\\

  We have that $x\sim_{X}x$ by taking $C = \{x\}$, so the relation is reflexive. Clearly, the relation is symmetric. To show transitivity, we need the following lemma:
  \begin{description}
    \item[Lemma:] If $Y_1,Y_2\subseteq X$ are connected with $Y_1\cap Y_2\neq \emptyset$, then $Y_1\cup Y_2$ is connected.
    \item[Proof of Lemma:] Let $Y_1\cup Y_2\subseteq U\cup V$ be a splitting. Note that $Y_i \subseteq U\cup V$, and $Y_i \cap U\cap V = \subseteq (Y_1\cup Y_2)\cap U \cap V =\emptyset$. For $i=1,2$, since $Y_i$ are connected, so we have splittings for $Y_i$. Since the $Y_i$ are connected, these splittings are trivial.\\

      Since the splitting for $Y_1$ is trivial, $Y_1\subseteq U$, or $Y_1\subseteq V$. Similarly, since the splitting for $Y_2$ is trivial, $Y_2\subseteq U$ or $Y_2\subseteq V$.\\

      Suppose $Y_1\subseteq U$ and $Y_2\subseteq U$. Then, $Y_1\cup Y_2\subseteq U$, and our original splitting is trivial.\\

      Suppose $Y_1\subseteq U$ and $Y_1\subseteq V$. Then, $\emptyset \neq Y_1 \cap Y_2 = (Y_1\cap U)\cap (Y_2\cap V) = (Y_1\cap Y_2)\cap (U\cap V)\subseteq (Y_1\cup Y_2)\cap U\cap V = \emptyset$.\\

      Other cases follow similarly.
  \end{description}
  If $x\sim_{X}y\sim_{X}z$, then there exist connected subsets $C,D\subseteq X$ with $x,y\in C$ and $y,z\in D$. Since $y\in C\cap D$, we have that $C\cup D$ is connected, so $x,z\in C\cup D$, which is connected.\\

  The equivalence classes of $X$ under $\sim_{X}$ are called components.
  \begin{description}
    \item[Remark:] $[x]_{\sim} = \{y\in X\mid y\sim_{X}x\} = \bigcup_{x\in C} C $ with $C$ connected. This is the largest connected subset of $X$ containing $x$. We have that $X = \bigsqcup_{i\in I}[x_i]_{\sim}$.
  \end{description}
  If $(X,d)$ is a metric space, and $C\subseteq X$ is clopen and connected, then $C$ is a component in $X$.
  \begin{description}
    \item[Proof:] Let $x\in C$. We claim that $C = [x]_{\sim}$.\\

      Clearly, $C\subseteq [x]_{\sim}$. Suppose $y\in [x]_{\sim}$ and $y\notin C$.\\

      Since $y\in [x]_{\sim}$, there is a connected $D\subseteq X$ with $x,y\in D$. We have that $D\subseteq C\cup (X\setminus C)$. This is a non-trivial splitting for $D$, meaning $D$ is disconnected. $\bot$
  \end{description}
  \subsection{Totally Disconnected Metric Spaces}%
  Consider the set $X = \{0\} \cup \{1/n\mid n\geq 1\}$ with the topology inherited from $\R$. We want to find the connected components.
  \begin{description}
    \item[Solution:] The set $\{1/n\}$ for each $n$ is connected in $\R$, meaning it is connected in $X$. Since $\{1/n\}$ is closed in $\R$, it is also closed in $X$. We also have that $\{1/n\} = X\cap (1/n - \delta_n,1/n + \delta_n)$, with $\delta_n = \frac{1}{n(n+1)}$.\\

      Since each $\{1/n\}$ is clopen and connected, each $\{1/n\}$ is a component. Additionally, $\{0\}$ is necessarily a component of $X$ since it is left over after we take $X \setminus \{1/n\mid n\geq 1\}$. We see that every connected component of $X$ is a singleton.
  \end{description}
  For $X=\Z$, we see that the components are singletons.\\

  For $X = \Q$, we need a little bit more machinery to find the components.
  \begin{description}
    \item[Solution:] Suppose $q,r\in\Q$ with $r\sim_{\Q}q$. Then, $\exists D\subseteq \Q$ connected with $r,q\in D$. If $r\neq q$, then let $x\in \R\setminus \Q$ with $x$ strictly between $r$ and $q$. Without loss of generality, $r < q$. Then, $D\subseteq \left((-\infty,x)\cap \Q\right) \cup \left((x,\infty)\cap \Q\right)$ is a non-trivial splitting, meaning $D$ is not connected.\\

      Therefore, $r = q$, meaning the components of $\Q$ are singletons.
  \end{description}
  If $(X,d)$ is a metric space where every connected component is a singleton, then $X$ is totally disconnected.
  \begin{description}
    \item[Exercise:] The Cantor set is totally disconnected.
  \end{description}
  \subsection{Proposition: Open Sets in $\R$}%
  If $U\subseteq \R$ is open, then $U = \bigsqcup_{i\in I}V_i$, where each $V_i\subseteq \R$ is an open interval and $I$ is countable.
  \begin{description}
    \item[Proof:] Let $U$ be the metric space with the topology inherited from $\R$. Then, $U = \bigsqcup_{i\in I}V_i$, with $V_i \subseteq U$ are the connected components in $U$.\\

      Since $V_i$ is connected in $U$, $V_i$ is connected in $\R$. Thus, $V_i$ is an interval. We will show that each $V_i$ is open in $\R$.\\

      Let $x\in V_i$. Since $U$ is open, $\exists \varepsilon > 0$ with $(x-\varepsilon,x+\varepsilon)\subseteq U$. Since $x\in (x-\varepsilon,x+\varepsilon)$, and $(x-\varepsilon,x+\varepsilon)$, it is the case that $(x-\varepsilon,x+\varepsilon)\subseteq [x]_{\sim_U} = V_i$. Thus, $V_i$ is open.\\

      Now, we need to show that $I$ is countable. Consider $N: I\rightarrow \Q$; $N(i) = q_i\in V_i$, with $q_i\in \Q$. If $i\neq j$, then $N(i) = N(j)$ since $V_i\cap V_j \neq \emptyset$. Hence, $N$ is injective, so $I$ is countable.
  \end{description}
  \subsection{Proposition: Connectedness and Continuity}%
  If $f: X_1\rightarrow X_2$ is continuous and $Y\subseteq X_1$ is connected, then $f(Y)\subseteq X_2$ is connected.
  \begin{description}
    \item[Proof:] Let $f(Y)\subseteq U\cup V$ is a splitting of $f(Y)\subseteq X_2$.\\

      Taking the preimage, we have $Y\subseteq f^{-1}(f(Y))\subseteq f^{-1}(U\cup V) = f^{-1}(U)\cup f^{-1}(V)$. We have that $f^{-1}(U)$ and $f^{-1}(V)$ are open in $X_1$. Additionally,
      \begin{align*}
        Y\cap f^{-1}(U)\cap f^{-1}(V) &= Y\cap f^{-1}(U\cap V)\\
                                      &\subseteq f^{-1}\left(f(Y)\right)\cap f^{-1}(U\cap V)\\
                                      &\subseteq f^{-1}\left(f(Y)\cap f^{-1}(U\cap V)\right)\\
                                      &= \emptyset.
      \end{align*}
      Thus, $Y\subseteq f^{-1}(U)\cup f^{-1}(V)$ is a splitting. Since $Y$ is connected, the splitting is trivial, meaning without loss of generality, $Y\subseteq f^{-1}(U)$. So, $f(Y)\subseteq U$.
  \end{description}
  \subsection{Intermediate Value Theorem}%
  Let $f: [a,b]\rightarrow \R$ is continuous. If $f(a)\leq \lambda \leq f(b)$, then $\lambda \in f([a,b])$.
  \begin{description}
    \item[Proof:] Since $[a,b]$ is compact and connected, and $f$ is continuous, $f([a,b])\subseteq \R$ is also connected. So, $f([a,b])$ is a compact and connected interval.\\

      Since $f(a),f(b)\in f([a,b])$, and $f([a,b])$ is an interval, $\lambda \in f([a,b])$.
  \end{description}
  \subsection{Proposition: Continuous Map to Totally Disconnected Set}%
  Let $X$ be connected, $Y$ totally disconnected, and $f: X\rightarrow Y$ continuous. Then, $f$ is a constant map.
  \begin{description}
    \item[Proof:] The continuous image of a connected set is connected, and the only connected sets in $Y$ are singletons, meaning the image of $X$ is a singleton.
  \end{description}
  \subsection{Path-Connectedness}%
  Let $(X,d)$ be a metric space.
  \begin{enumerate}[(i)]
    \item A path in $X$ is a continuous map $\gamma: [0,1]\rightarrow X$. If $\gamma(0) = x_0$ and $\gamma(1) = x_1$, we say the path connects $x_0$ to $x_1$.
    \item $X$ is said to be path-connected if for any two points $x_0$ and $x_1$, there exists a path. $Y\subseteq X$ is path connected if $Y$ is connected.
  \end{enumerate}
  \begin{enumerate}[(1)]
    \item Let $V$ be any normed space, and $C\subseteq V$ convex. By definition, $C$ is path-connected. Indeed, $\gamma(t) = (1-t)x_0 + x_1$.
    \item The metric space $\R^2\setminus \{0\}$ is path-connected.
  \end{enumerate}
  \subsection{Proposition: Composition of Paths}%
  Let $\gamma: [0,1] \rightarrow X$ is a path from $x_0$ to $x_1$, and $\sigma: [0,1]\rightarrow X$ is a path from $x_1$ to $x_2$. Then, the following are all true.
  \begin{enumerate}[(1)]
    \item $\gamma^{-1}: [0,1]\rightarrow X$, with $\gamma^{-1}(t) = \gamma(1-t)$, is a path from $x_1$ to $x_0$.
    \item $\sigma \cdot \gamma: [0,1]\rightarrow X$ is a path from $x_0$ to $x_2$, with $\sigma \cdot \gamma(t)$ defined as follows:
      \begin{align*}
        \sigma \cdot \gamma(t) &= \begin{cases}
          \gamma(2t)&0\leq t \leq 1/2\\
          \sigma(2t-1) & 1/2 \leq t \leq 1
        \end{cases}.
      \end{align*}
  \end{enumerate}
  \subsection{Lemma: Base Point and Path-Connectedness}%
  Let $(X,d)$ be a metric space, and $x_0\in X$ fixed. Suppose $\forall x$, $\exists $ a path from $x_0$ to $x$. Then, $X$ is path-connected.
  \begin{enumerate}[(1)]
    \item The unitary group is path-connected.
      \begin{align*}
        U_n(\mathbb{C}) &= \{U\in \mathbb{M}_n(\mathbb{C})\mid U^{\ast}U = I_n = UU^{\ast}\}\\
        d(U,V) &= \norm{U-V}_{\text{op}}
      \end{align*}
      Let $U\in U_n(\mathbb{C})$. By the spectral theorem via a unitary; there exists $V\in U_n(\mathbb{C})$ with $V^{\ast} U V = \text{diag}(\lambda_1,\lambda_2,\dots,\lambda_n)$, with $|\lambda_j| = 1$. Write $\lambda_j = e^{i\theta_j}$, with $\theta_j\in [0,2\pi)$.\\

      Consider $U_t = V\text{diag}\left(e^{it\theta_1},\dots,e^{it\theta_n}\right)V^{\ast}$. Clearly, $U_t\in \mathbb{M}_n(\mathbb{C})$. Additionally, $U_0 = I_n$, and $U_1 = U$. We have
      \begin{align*}
        \norm{U_s - U_t} &= \norm{V^{\ast}\Lambda_s V - V\Lambda_tV^{\ast}}\\
                         &= \norm{V(\Lambda_s - \Lambda_t)V^{\ast}}\\
                         &\leq \norm{V}\norm{\Lambda_s - \Lambda_t}\norm{V^{\ast}}\\
                         &= \norm{\Lambda_s - \Lambda_t}\\
                         &\rightarrow 0.
      \end{align*}
      Thus, $U_t$ is continuous, meaning we have a path from $I_n$ to $U$. Thus, $U_n(\mathbb{C})$ is path-connected.
  \end{enumerate}
  \subsection{Proposition: Path-Connectedness implies Connectedness}%
  If $(X,d)$ is a path-connected metric space, then $X$ is connected.
  \begin{description}
    \item[Proof:] Let $X = U\sqcup V$ be a splitting. Suppose $\exists x_0\in U$ and $x_1\in V$. We know $\exists \gamma: [0,1]\rightarrow X$ with $\gamma(0) = x_0$ and $\gamma(1) = x_1$. Since $[0,1]$ is connected and $\gamma$ is continuous, $\gamma([0,1])\subseteq X$ is connected. However, $\gamma([0,1])\subseteq U\cup V$ is a non-trivial splitting. $\bot$
    \item[Exercise:] If $f: X_1\rightarrow X_2$ is continuous, and $Y\subseteq X_1$ is path-connected, then $f(Y)\subseteq X_2$ is path-connected.
      \begin{description}
        \item[Proof of Exercise:] Let $f(y_1),f(y_2)\in f(Y)$. We have that $\gamma: [0,1]\rightarrow Y$ is a path. Thus, $f\circ \gamma: [0,1]\rightarrow f(Y)$ is a path.
      \end{description}
  \end{description}
  \subsection{A Connected Space that is not Path-Connected}%
  Set $Y_0 = \{0\}\times [-1,1]\subseteq \R^2$, and $Y_1 = \{(x,\sin(1/x))\mid x\in (0,1]\}$. Let $Y = Y_0 \cup Y_1$. This space is known as the topologist's sine curve, and it is connected but not path-connected.
  \begin{description}
    \item[Proof:] We can see that $Y_1$ is the continuous image of a connected set, so $Y_1$ is connected.\\

      We also see that $Y$ is connected, as $Y = \overline{Y_1}$.\\

      We claim that $Y$ is not path-connected. There does not exist a path $\gamma: [0,1]\rightarrow Y$ with $\gamma(0)\in Y_0$ and $\gamma(1)\in Y_1$. Suppose toward contradiction that such a path existed. Let $\gamma^{-1}(Y_0) := F$, with $\gamma^{-1}$ being the inverse image (not inverse path). Since $Y_0$ is closed, we have $F\subseteq [0,1]$ is closed, so $u = \sup F \in F$, and $u < 1$.\\

      By replacing $[0,1]$ by $[u,1]$, we may assume a new path $\gamma': [0,1]\rightarrow Y$ is a path with $\gamma_1(t) \in (0,1]$, for $\gamma'(t) = (\gamma'_1(t),\gamma'_2(t))$.\\

      Let $r>0$ be small such that $[-1,1] \supset [\gamma'_2(0)-r,\gamma'_2(0)+r]$. Since $\gamma'_2$ is continuous at $t=0$, we know $\exists \varepsilon > 0$ with $\gamma'_2([0,\varepsilon])\subseteq (\gamma'_2(0)-r,\gamma'_2(0) + r)$.\\

      Since $\gamma'_1([0,\varepsilon])$ is connected, and hence an interval, and $\gamma'_1(t) > 0$ for all $t\in (0,1]$, we can find $\delta$ small such that $[0,\delta]\subseteq \gamma'_1([0,\varepsilon))$.\\

      We have that $\gamma'_2(t) = \sin \left(\frac{1}{\gamma'_1(t)}\right)$ for $t > 0$. Therefore,
      \begin{align*}
        [-1,1] &= \left\{\sin\left(\frac{1}{x}\right) \mid 0 < x < \delta\right\}\\
               &\subseteq \left\{\sin \left(\frac{1}{\gamma'_1(t)}\right)\mid 0 < t < \varepsilon\right\}\\
               &= \gamma'_2((0,\varepsilon))\\
               &\subseteq (\gamma'_2(0)-r,\gamma'_2(0)+r)\\
               &\subset [-1,1].
      \end{align*}
  \end{description}
  \subsection{Proposition: Connectedness in a Normed Space}%
  Let $V$ be a normed space, and $Y\subseteq V$ is open and connected, then $Y$ is path-connected.
  \begin{description}
    \item[Proof:] Fix $y_0\in Y$. Consider the set $W = \{y\in Y\mid \exists \gamma\text{ from $y_0$ to $y$}\}$. We claim that $W$ is open in $Y$.\\

      Let $y\in W$. Since $Y$ is open, $\exists \delta > 0$ with $U(y,\delta)\subseteq Y$. If $w\in U(y,\delta)$, $\exists \gamma$ from $y$ to $w$. Concatenating, we get a path from $y_0$ to $w$. Thus, $U(y,\delta)\subseteq W$.\\

      We also claim $W$ is closed in $Y$.
  \end{description}
  \section{Measure Theory}%
  The theory of integration is tied to notions of length, area, volume, etc. The Riemann integral
  \begin{align*}
    \int_{0}^{1} f(x)dx &= \lim_{n\rightarrow\infty}\frac{1}{n}\sum_{k=1}^{n}f\left(\frac{k}{n}\right),
  \end{align*}
  is defined through the length of a subinterval. We took the interval $[0,1]$, calculated base multiplied by height, and found the area of the rectangle.\\

  It's easy to compute the length of an interval. However, Lebesgue integration does the opposite; it subdivides the range of $f$ into subintervals $I_k$, and calculates the ``length'' of $f^{-1}(I_k)$.\\

  We need a more rigorous treatment of length (or area, or volume) to deal with Lebesgue integration.\\

  Given $E\subseteq \R^n$, with $E$ ``sufficiently nice,'' we want to assign an extended positive real number $\lambda(E)\in [0,\infty]$, such that certain natural properties are satisfied.
  \begin{itemize}
    \item $\lambda(\emptyset) = 0$
    \item $\displaystyle \lambda \left(\prod_{j=1}^{n}[a_j,b_j]\right) = \prod_{j=1}^{n}(b_j-a_j)$
    \item $\lambda(x + E) = \lambda(E)$
    \item $\displaystyle \lambda \left(\bigsqcup_{k=1}^{\infty}E_k\right) = \sum_{k=1}^{n}E_k$
    \item if $E\subseteq F$, then $\lambda(E)\leq \lambda(F)$
  \end{itemize}
  \subsection{Proposition: Non-existence of $\lambda$}%
  There is no $\lambda: \mathcal{P}(\R)\rightarrow [0,\infty]$ that satisfies the properties above.
  \begin{description}
    \item[Proof:] Consider the equivalence relation on $[0,1]$, with $x\sim y \Leftrightarrow x-y\in \Q$.\\

      So, $[0,1] = \bigsqcup_{i\in I} [x_i]$, with $x_i\in [0,1]$. Let $\{r_k\}_{k=1}^{\infty}$ be an enumeration of $\Q \cap [-1,1]$. Let $N = \{x_i\}_{i\in I}$ (possible with the axiom of choice).\\

      Consider the set $E_k = r_k + N$.
      \begin{itemize}
        \item $E_k$ are pairwise disjoint; if $r_k + x_i = r_{\ell} + x_j$, then $x_j - x_i = r_k - r_{\ell}\in \Q$, meaning $x_i \sim x_j$.
        \item $E_k \subseteq [-1,2]$.
      \end{itemize}
      If $t\in [0,1]$, then $t\sim x_i$ for some $i\in I$. So, $t-x_i \in \Q$, and $t-x_i\in [-1,1]$, so $t-x_i = r_k$ for some $k$. Thus, $t\in E_k$. Thus, we have shown that $[0,1]\subseteq \bigsqcup E_k\subseteq [-1,2]$.\\

      If $\lambda$ were such a mapping, we have
      \begin{align*}
        1 &= \lambda([0,1])\\
          &\leq \lambda(\bigsqcup E_k)\\
          &= \sum \lambda(E_k)\\
          &= \sum \lambda(r_k + N)\\
          &= \sum \lambda(N).
      \end{align*}
      If $E = \bigsqcup E_k$, then $\lambda(E) \leq 3$ and $\lambda(E) = \sum \lambda(N)$. $\bot$.
  \end{description}
  Thus, we conclude that some sets are not measurable. We might then ask what sets are able to be measured.
  \begin{itemize}
    \item Intervals;
    \item open sets;
    \item closed sets.
  \end{itemize}
  We will eventually define a class of measurable sets, $\mathcal{L}$, and we will also construct a measure $\lambda: \mathcal{L}\rightarrow [0,\infty]$ satisfying the above properties.
  \subsection{Measurable Spaces and $\sigma$-Algebras}%
  Let $\Omega \neq \emptyset$.
  \begin{enumerate}[(1)]
    \item An algebra of subsets of $\Omega$ is a nonempty family $\mathcal{M}\subseteq \mathcal{P}(\Omega)$ such that
      \begin{itemize}
        \item If $E\in \mathcal{M}$, then $E^c\in \mathcal{M}$;
        \item If $E,F\in \mathcal{M}$, then $E\cup F\in \mathcal{M}$
      \end{itemize}
    \item A nonempty collection $\mathcal{M}\subseteq \mathcal{P}(\Omega)$ is a $\sigma$-algebra of subsets of $\Omega$ if
      \begin{enumerate}[(i)]
        \item If $E\in \mathcal{M}$, then $E^{c}\in \mathcal{M}$;
        \item If $\{E_k\}_{k=1}^{\infty}\in \mathcal{M}$, then $\bigcup E_k \in \mathcal{M}$.
      \end{enumerate}
    \item A measurable space is a pair $(\Omega,\mathcal{M})$ with $\Omega \neq \emptyset$ a set and $\mathcal{M}$ is a $\sigma$-algebra.
  \end{enumerate}
  Let $\mathcal{M}$ be an algebra of subsets of $\Omega$. Then, the following are true.
  \begin{enumerate}[(i)]
    \item $\emptyset,\Omega \in \mathcal{M}$;
    \item If $E_1,\dots,E_n \in \mathcal{M}$, then $\bigcup E_k \in \mathcal{M}$;
    \item If $E_1,\dots,E_n \in \mathcal{M}$, then $\bigcap E_k \in \mathcal{M}$;
    \item If $E,F\in \mathcal{M}$, then $E\setminus F\in \mathcal{M}$.
  \end{enumerate}
  \begin{description}
    \item[Proof:] \hfill
      \begin{enumerate}[(i)]
        \item Since $\mathcal{M}$ is not empty, there is an $E\in \mathcal{M}$, so $E^c\in \mathcal{M}$, so $E\cup E^{c} = \Omega \in \mathcal{M}$ ,and $(E\cup E^{c})^{c} = \emptyset \in \mathcal{M}$.
        \item Induction.
        \item We have $\bigcap E_k = \left(\bigcup_{i=1}^{\infty}E_k^c\right)^c \in \mathcal{M}$.
        \item We have $E\setminus F = E\cap F^{c}\in \mathcal{M}$.
      \end{enumerate}
  \end{description}
  If $\mathcal{M}$ is a $\sigma$-algebra, then (1) through (4) hold for countable families as well.
  \begin{enumerate}[(1)]
    \item $(\Omega,\mathcal{P}(\Omega))$ is a measurable space.
    \item $(\Omega,\{\emptyset,\Omega\})$ is a measurable space.
    \item For $\Omega$ uncountable, let $\mathcal{M} = \{E\subseteq \Omega\mid E\text{ countable or }E^c\text{ countable}\}$. Then, $(\Omega,\mathcal{M})$ is a measurable space.
    \item If $\{\mathcal{M}_i\}_{i\in I}$ is a family of $\sigma$-algebras on $\Omega$, then $\bigcap \mathcal{M}_i$ is a $\sigma$-algebra on $\Omega$.
  \end{enumerate}
  If $0\neq \mathcal{E}\subseteq \mathcal{P}(\Omega)$, the $\sigma$-algebra generated by $\mathcal{E}$ is
  \begin{align*}
    \sigma(\mathcal{E}) &= \bigcap_{\substack{\mathcal{M}_i\text{ $\sigma$-algebra}\\\mathcal{E}\subseteq \mathcal{M}_i}}\mathcal{M}_i.
  \end{align*}
  \subsection{Borel $\sigma$-Algebra}%
  Let $(X,d)$ be a metric space. Let $\tau_d = \{U\mid U\subseteq X\text{ open}\}$. The Borel $\sigma$-algebra on $X$ is
  \begin{align*}
    \mathcal{B}_X = \sigma(\tau_d).
  \end{align*}
  \begin{description}
    \item[Remark:] $\mathcal{B}_X$ contains all open sets, closed sets, F$_{\sigma}$ sets, G$_{\delta}$ sets, etc.
  \end{description}
  \subsection{Proposition: Borel $\sigma$-Algebra on $\R$}%
  Consider the families of $\mathcal{P}(\R)$,
  \begin{align*}
    \mathcal{E}_1 &= \{(a,b)\mid a < b\}\\
    \mathcal{E}_2 &= \{[a,b]\mid a<b \}\\
    \mathcal{E}_3 &= \{(a,b]\mid a < b\}\\
    \mathcal{E}_4 &= \{[a,b)\mid a < b\}\\
    \mathcal{E}_5 &= \{(-\infty,b)\mid b\in \R\}\\
    \mathcal{E}_6 &= \{(-\infty,b]\mid b\in \R\}\\
    \mathcal{E}_7 &= \{(a,\infty)\mid a\in \R\}\\
    \mathcal{E}_8 &= \{[a,\infty)\mid a\in \R\}.
  \end{align*}
  For $i = 1,\dots,8$, we have $\sigma(\mathcal{E}_i) = \mathcal{B}_{\R}$.
  \begin{description}
    \item[Proof:] Note that $\mathcal{E}_1\subseteq \tau_d\subseteq \sigma(\tau_d)\subseteq \mathcal{B}_{\R}$. Thus, $\sigma(\mathcal{E}_1)\subseteq \mathcal{B}_{\R}$. Let $U\in \R$ be open. Then, $U = \bigsqcup I_j$, with $I_j$ open. Consider any open interval $I$. If $I$ is bounded, then $I\in \mathcal{E}_1$. If $I$ is not bounded, then $I = \bigcup_{k=1}^{\infty} J_{k}$ with $J_k$ bounded open intervals. Since each $J_k\in \mathcal{E}_1$, then $I\in \sigma(\mathcal{E}_1)$. Therefore, each $I_j\in \sigma(\mathcal{E}_1)$, so $U\in \sigma(\mathcal{E})_1$. Thus, $\tau_d \subseteq \sigma(\mathcal{E}_1)$, so $\mathcal{B}_{\R}\subseteq \sigma(\mathcal{E}_1)$.\\

      Thus, $\mathcal{B}_{\R} = \sigma(\mathcal{E}_1)$.\\

      We have that $[a,b) = \bigcap_{n=1}^{\infty} \left(a-\frac{1}{n},b\right) \in \sigma(\mathcal{E}_1)$. Therefore, $\mathcal{E}_4\in \sigma(\mathcal{E}_1)$, thus $\sigma(\mathcal{E}_4)\subseteq \sigma(\mathcal{E}_1)$. Additionally, $(a,b) = \bigcup_{n=1}^{\infty}\left[a+\frac{1}{n},b\right]\in \sigma(\mathcal{E}_4)$. So, $\sigma(\mathcal{E}_1) = \sigma(\mathcal{E}_4) = \mathcal{B}_{\R}$.
  \end{description}
\end{document}
