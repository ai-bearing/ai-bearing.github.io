\documentclass[8pt]{extarticle}
\title{}
\author{}
\date{}
\usepackage[shortlabels]{enumitem}


%paper setup
\usepackage{geometry}
\geometry{letterpaper, portrait, margin=1in}
\usepackage{fancyhdr}
% sans serif font:
\usepackage{cmbright}
%symbols
\usepackage{amsmath}
\usepackage{bigints}
\usepackage{amssymb}
\usepackage{amsthm}
\usepackage{mathtools}
\usepackage{bbm}
\usepackage[hidelinks]{hyperref}
\usepackage{gensymb}
\usepackage{multirow,array}
\usepackage{multicol}

\newtheorem*{remark}{Remark}
\usepackage[T1]{fontenc}
\usepackage[utf8]{inputenc}

%chemistry stuff
%\usepackage[version=4]{mhchem}
%\usepackage{chemfig}

%plotting
\usepackage{pgfplots}
\usepackage{tikz}
\tikzset{middleweight/.style={pos = 0.5}}
%\tikzset{weight/.style={pos = 0.5, fill = white}}
%\tikzset{lateweight/.style={pos = 0.75, fill = white}}
%\tikzset{earlyweight/.style={pos = 0.25, fill=white}}

%\usepackage{natbib}

%graphics stuff
\usepackage{graphicx}
\graphicspath{ {./images/} }
\usepackage[style=numeric, backend=biber]{biblatex} % Use the numeric style for Vancouver
\addbibresource{the_bibliography.bib}
%code stuff
%when using minted, make sure to add the -shell-escape flag
%you can use lstlisting if you don't want to use minted
%\usepackage{minted}
%\usemintedstyle{pastie}
%\newminted[javacode]{java}{frame=lines,framesep=2mm,linenos=true,fontsize=\footnotesize,tabsize=3,autogobble,}
%\newminted[cppcode]{cpp}{frame=lines,framesep=2mm,linenos=true,fontsize=\footnotesize,tabsize=3,autogobble,}

%\usepackage{listings}
%\usepackage{color}
%\definecolor{dkgreen}{rgb}{0,0.6,0}
%\definecolor{gray}{rgb}{0.5,0.5,0.5}
%\definecolor{mauve}{rgb}{0.58,0,0.82}
%
%\lstset{frame=tb,
%	language=Java,
%	aboveskip=3mm,
%	belowskip=3mm,
%	showstringspaces=false,
%	columns=flexible,
%	basicstyle={\small\ttfamily},
%	numbers=none,
%	numberstyle=\tiny\color{gray},
%	keywordstyle=\color{blue},
%	commentstyle=\color{dkgreen},
%	stringstyle=\color{mauve},
%	breaklines=true,
%	breakatwhitespace=true,
%	tabsize=3
%}
% text + color boxes
\usepackage[most]{tcolorbox}
\tcbuselibrary{breakable}
\tcbuselibrary{skins}
\newtcolorbox{problem}[1]{colback=white,enhanced,title={\small #1},
          attach boxed title to top center=
{yshift=-\tcboxedtitleheight/2},
boxed title style={size=small,colback=black!60!white}, sharp corners, breakable}
%including PDFs
%\usepackage{pdfpages}
\setlength{\parindent}{0pt}
\usepackage{cancel}
\pagestyle{fancy}
\fancyhf{}
\rhead{Avinash Iyer}
\lhead{Real Analysis II: Class Notes}
\newcommand{\card}{\text{card}}
\newcommand{\ran}{\text{ran}}
\newcommand{\N}{\mathbbm{N}}
\newcommand{\Q}{\mathbbm{Q}}
\newcommand{\Z}{\mathbbm{Z}}
\newcommand{\R}{\mathbbm{R}}
\setcounter{secnumdepth}{0}
\begin{document}
\section{Normed Vector Spaces}%
\subsection{Vector Spaces}%
Throughout, $\mathbbm{F} = \R$ or $\mathbbm{C}$. A \textbf{vector space} over $\mathbbm{F}$ is a nonempty set $V$ equipped with two operations: vector addition and scalar multiplication.
\begin{align*}
  V\times V \xrightarrow{+} V\\
  (v,w)\mapsto v+w
  F\times V \rightarrow V\\
  (\alpha,v)\mapsto \alpha v
\end{align*}
The vector space is an Abelian group, where $u,v,w\in V$ and $\alpha,\beta\in \mathbbm{F}$, we have:
\begin{enumerate}[(i)]
  \item $u+(v+w) = (u+v)+w$
  \item $\exists 0_v\in V$ with $\forall v\in V,~0_v + v = v + 0_v = v$
  \item $(\forall v\in V)(\exists w\in V)$ with $v+w = 0_v$ --- 
  \item $\forall v,w\in V,~v+w = w+v$
  \item $\alpha(v+w) = \alpha v + \alpha w,~(\alpha + \beta)v = \alpha v + \beta v$
  \item $\alpha(\beta w) = (\alpha\beta)w$
  \item $1\cdot v = v$
\end{enumerate}
\textbf{Remarks:}
\begin{enumerate}[(a)]
  \item $0_v$ is unique and known as the zero vector.
  \item The vector $w$ in (iii) is unique, and denoted $-v$.
  \item $0\cdot v = 0_v$
  \item $(-1)\cdot v = -v$
  \item Property (iv) follows from all the other axioms.
  \item For $n\in\N$, $n\cdot v = \underbrace{v+v+\cdots+v}_{n~\text{times}}$
\end{enumerate}
\subsection{Subspaces}%
Let $V$ be a vector space over $\mathbbm{F}$. A \textbf{subspace} is a nonempty subset $W\subseteq V$ satisfying the following:
\begin{enumerate}[(i)]
  \item $w\in W,\alpha\in \mathbbm{F}\rightarrow \alpha w \in W$.
  \item $w_1,w_2\in W \Rightarrow w_1 + w_2\in W$.
\end{enumerate}
\begin{description}
  \item[Remark:] $0_v$ is always a member of any subspace; a subspace is also a vector space.
\end{description}
\subsubsection{Proposition: Intersection of Subspaces}%
If $\{W_i\}_{i\in I}$ is a family of subspaces of $V$, then, $\bigcap W_i$ is a subspace of $V$.
\subsubsection{Proposition: Union of Subspaces}%
It is not the case that the union of subspaces of $V$ also a subspace. For example, consider $\R^2$ with the traditional vector space operations:
\begin{align*}
  \begin{pmatrix}x\\y\end{pmatrix} + \begin{pmatrix}x'\\y'\end{pmatrix} &= \begin{pmatrix}x+x'\\y+y'\end{pmatrix}\\
  \alpha \begin{pmatrix}x\\y\end{pmatrix} &= \begin{pmatrix}\alpha x\\\alpha y\end{pmatrix}
\end{align*}
If $W_1,W_2\in V$ are subspaces such that $W_1 \cup W_2$ is a subspace, then $W_1\subseteq W_2$ or $W_2\subseteq W_1$.
\subsubsection{Generated Subspaces}%
Let $S\subseteq V$ be any subset of a vector space $V$. Then,
\begin{align*}
  \text{span}(S) &= \left\{\sum_{j=1}^{n}\alpha_jv_j\mid \alpha_1,\dots,\alpha_n\in \mathbbm{F},v_1,\dots,v_n\in S\right\}
\end{align*}
\begin{description}
  \item[Remarks:]\hfill
    \begin{itemize}
      \item $\text{span}(S)\subseteq V$ is a subspace.
      \item $\text{span}(S) = \bigcap W$, where $S\subseteq W$ and $W\subseteq V$ is a subspace. Thus, $\text{span}(S)$ is the ``smallest'' subspace containing $S$, or the subspace generated by $S$.
    \end{itemize}
\end{description}
\subsubsection{Proposition: Quotient Group on Vector Space}%
Let $V$ be a vector space, and let $W\subseteq V$ is a subspace. Define $u\sim_{W} v\leftrightarrow u-v\in W$.
\begin{enumerate}[(1)]
  \item $\sim_W$ is an equivalence relation.
  \item If $[v]_W$ denotes the equivalence class of $v$, then $[v]_W = v+W = \{v+w|w\in W\}$.
  \item $V/W:= \{[v]_W|v\in V\}$ is a vector space with $[v_1]_W + [v_2]_W = [v_1+v_2]_W$ and $\alpha[v]_W = [\alpha v]_W$.
\end{enumerate}
\begin{description}
  \item[Proof of (1):]\hfill
    \begin{itemize}
      \item Reflexive: $u\sim_W u$, since $u-u = 0\in W$.
      \item Transitive: Suppose $u\sim_W v$, and $v\sim_W z$. Then, $u-v\in W$, and $v-z\in W$. So, $(u-v) + (v-z)\in W$, so $u-z\in W$. Whence, $u\sim_W z$.
      \item Symmetric: If $u\sim_W v$, then $u-v\in W$, so $-1\cdot(u-v)\in W$, so $v-u\in W$. Whence, $v\sim_W u$.
    \end{itemize}
  \item[Proof of (2):]
    \begin{align*}
      [v]_W &= \{u\in V\mid u\sim_W v\}\\
            &= \{u\in V\mid u-v\in W\}\\
            &= \{u\in V\mid u = v+w~\text{some $w\in W$}\}\\
            &= \{v+w\mid w\in W\}\\
            &= v+W
    \end{align*}
  \item[Proof of (3):] Prove that the operation is well-defined.
\end{description}
\subsection{Bases}%
Let $V$ be a vector space and $S\subseteq V$ be a subset.
\begin{enumerate}[(1)]
  \item $S$ is said to be spanning for $V$ if $\text{span}(S) = V$.
  \item $S$ is linearly independent if, for $\sum_{j=1}^{n}\alpha_jv_j = 0_v$ with $\alpha_1,\dots,\alpha_n\in \mathbbm{F}$, $v_1,\dots,v_n\in S$, then $\alpha_1=\alpha_2 = \cdots = \alpha_n = 0$.
  \item $S$ is a basis for $V$ if $S$ is linearly independent and spanning for $V$.
\end{enumerate}
\subsubsection{Proposition: Existence of Basis}%
Every vector space admits a basis. If $S\subseteq V$ is linearly independent, $\exists B\subseteq V$ such that $B$ is a basis and $S\subseteq V$.
\begin{description}
  \item[Zorn's Lemma:]
\end{description}
\end{document}
