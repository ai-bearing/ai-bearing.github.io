\documentclass[8pt]{extarticle}
\title{}
\author{}
\date{}
\usepackage[shortlabels]{enumitem}


%paper setup
\usepackage{geometry}
\geometry{letterpaper, portrait, margin=1in}
\usepackage{fancyhdr}
% sans serif font:
\usepackage{cmbright}
%symbols
\usepackage{amsmath}
\usepackage{bigints}
\usepackage{amssymb}
\usepackage{amsthm}
\usepackage{mathtools}
\usepackage{bbold}
\usepackage[hidelinks]{hyperref}
\usepackage{gensymb}
\usepackage{multirow,array}
\usepackage{multicol}

\newtheorem*{remark}{Remark}
\usepackage[T1]{fontenc}
\usepackage[utf8]{inputenc}

%chemistry stuff
%\usepackage[version=4]{mhchem}
%\usepackage{chemfig}

%plotting
\usepackage{pgfplots}
\usepackage{tikz}
\usetikzlibrary{cd}
\tikzset{middleweight/.style={pos = 0.5}}
%\tikzset{weight/.style={pos = 0.5, fill = white}}
%\tikzset{lateweight/.style={pos = 0.75, fill = white}}
%\tikzset{earlyweight/.style={pos = 0.25, fill=white}}

%\usepackage{natbib}

%graphics stuff
\usepackage{graphicx}
\graphicspath{ {./images/} }
\usepackage[style=numeric, backend=biber]{biblatex} % Use the numeric style for Vancouver
\addbibresource{the_bibliography.bib}
%code stuff
%when using minted, make sure to add the -shell-escape flag
%you can use lstlisting if you don't want to use minted
%\usepackage{minted}
%\usemintedstyle{pastie}
%\newminted[javacode]{java}{frame=lines,framesep=2mm,linenos=true,fontsize=\footnotesize,tabsize=3,autogobble,}
%\newminted[cppcode]{cpp}{frame=lines,framesep=2mm,linenos=true,fontsize=\footnotesize,tabsize=3,autogobble,}

%\usepackage{listings}
%\usepackage{color}
%\definecolor{dkgreen}{rgb}{0,0.6,0}
%\definecolor{gray}{rgb}{0.5,0.5,0.5}
%\definecolor{mauve}{rgb}{0.58,0,0.82}
%
%\lstset{frame=tb,
%	language=Java,
%	aboveskip=3mm,
%	belowskip=3mm,
%	showstringspaces=false,
%	columns=flexible,
%	basicstyle={\small\ttfamily},
%	numbers=none,
%	numberstyle=\tiny\color{gray},
%	keywordstyle=\color{blue},
%	commentstyle=\color{dkgreen},
%	stringstyle=\color{mauve},
%	breaklines=true,
%	breakatwhitespace=true,
%	tabsize=3
%}
% text + color boxes
\renewcommand{\mathbf}[1]{\mathbb{#1}}
\usepackage[most]{tcolorbox}
\tcbuselibrary{breakable}
\tcbuselibrary{skins}
\newtcolorbox{problem}[1]{colback=white,enhanced,title={\small #1},
          attach boxed title to top center=
{yshift=-\tcboxedtitleheight/2},
boxed title style={size=small,colback=black!60!white}, sharp corners, breakable}
%including PDFs
%\usepackage{pdfpages}
\setlength{\parindent}{0pt}
\usepackage{cancel}
\pagestyle{fancy}
\fancyhf{}
\rhead{Avinash Iyer}
\lhead{Real Analysis II: Problem Set 5}
\newcommand{\card}{\text{card}}
\newcommand{\ran}{\text{ran}}
\newcommand{\N}{\mathbb{N}}
\newcommand{\Q}{\mathbb{Q}}
\newcommand{\Z}{\mathbb{Z}}
\newcommand{\R}{\mathbb{R}}
\newcommand{\C}{\mathbb{C}}
\newcommand{\iprod}[2]{\left\langle #1,#2\right\rangle}
\newcommand{\norm}[1]{\left\Vert #1\right\Vert}
\setcounter{secnumdepth}{0}
\begin{document}
  \section{Problem 1}%
  Show that $C_0(\R)$ is a Banach space.
  \begin{description}
    \item[Proof:] Let $(f_n)_n$ be a Cauchy sequence in $C_0(\R)$. Since each $f_k\in C_0(\R)$, it must be the case that each $f_k$ is uniformly continuous. For each $x\in \R$, it is thus the case that $(f_n(x))_n$ is Cauchy in $\R$. Since $\R$ is complete, $(f_n(x))_n\rightarrow f(x)$ for each $x\in \R$, and since each $f_k$ is uniformly continuous, it must be the case that $f(x)$ is continuous.\\

      For $\varepsilon > 0$, there must be $N$ large such that for $m,n\geq N$ and $m\geq n$, it must be the case that $|f_m(x)-f_n(x)| < \varepsilon$ for all $x\in \R$. Letting $m\rightarrow\infty$, we have $|f_n(x)-f(x)| < \varepsilon$, meaning $(f_n)_n\rightarrow f$. Thus, $f\in C_0(\R)$.
  \end{description}
  \section{Problem 2}%
  Show that $\ell_2$ is a Hilbert space.
  \begin{description}
    \item[Proof:] Let $(x_n)_n$ be a Cauchy sequence in $\ell_2$. Let $x_n(k)$ denote the index $k$ of the sequence $x_n\in \ell_2$. Then, by the equivalence of norms, $\exists c\in \R$ such that
      \begin{align*}
        |x_n(k) - x_m(k)| &\leq c\norm{x_m(k) - x_n(k)}_2\\
                          &\rightarrow 0 \tag*{since $(x_n)_n$ is Cauchy in $\ell_2$.}
      \end{align*}
      Since $\R$ (or $\C$) is complete, $x_n(k) \rightarrow x(k)$ as $k\rightarrow \infty$. We set $(x(k))_k$ to be the sequence such that $x_n(k) \rightarrow x(k)$ for each $n$.\\

      We must show that $\norm{x_n - x}_2\rightarrow 0$ as $n\rightarrow \infty$. This is equivalent to
      \begin{align*}
        \lim_{N\rightarrow\infty}\sum_{k=1}^{N}\lim_{m\rightarrow\infty}|x_n(k) - x_m(k)|^2 &\leq \lim_{m\rightarrow\infty}\sup_{m\geq M}\norm{x_n - x_m}^2\\
                                                                                            &\leq \varepsilon^2 \tag*{since $(x_n)_n$ is Cauchy.}
      \end{align*}
      Thus, $\norm{x_n-x_m} \rightarrow 0$ as $m\rightarrow \infty$ and $n\rightarrow\infty$, so $\norm{x_n - x}\rightarrow 0$ as $n\rightarrow\infty$.
  \end{description}
  \section{Problem 3}%
  Suppose $(X,d)$ is a complete metric space and $(x_n)_n$ is a sequence in $X$ such that there is a $\theta \in (0,1)$ with $d(x_{n+1},x_n) \leq \theta d(x_n,x_{n-1})$. Show that $(x_n)_n$ is convergent.
  \begin{description}
    \item[Proof:] We will show that $(x_n)_n$ is convergent by showing that $(x_n)_n$ Cauchy. Let $m,n$ be such that $m\geq n$. Notice that $d(x_n,x_{n-1}) \leq \theta^{n-2}d(x_2,x_1)$. Thus,
      \begin{align*}
        d(x_m,x_n) &\leq d(x_m,x_{m-1}) + d(x_{m-1},x_{m-2}) + \cdots + d(x_{n+1},x_n)\\
                   &\leq d(x_2,x_1)\left(\theta^{m-2} + \theta^{m-3} + \cdots + \theta^{n-1}\right)\\
                   &= d(x_2,x_1) \theta^{n-1}\left(1 + \theta + \theta^2 + \cdots + \theta^{p-q-1}\right)\\
                   &\leq d(x_2,x_1)\frac{\theta^{n-1}}{1-\theta}.
      \end{align*}
      Notice that the sequence $\left(\frac{\theta^{n-1}}{1-\theta}\right)_{n}\rightarrow 0$ in $\R$, meaning $(x_n)_n$ is Cauchy. Since $X$ is complete, $(x_n)_n$ is convergent.
  \end{description}
  \section{Problem 4}%
  Let $(X,d)$ be a complete metric space, and suppose $f: X\rightarrow X$ is a contractive map --- i.e., there is a $\theta \in (0,1)$ with
  \begin{align*}
    d(f(x),f(y)) \leq \theta d(x,y).
  \end{align*}
  Prove that $f$ has a unique fixed point.
  \begin{description}
    \item[Proof:] Let $x_0\in X$, and define $x_{n} = f(x_{n-1})$. For all $n$, we have
      \begin{align*}
        d(x_{n},x_{n-1}) &\leq \theta^{n}d(x_1,x_0).
      \end{align*}
      Therefore, for $x_m,x_n$ arbitrary in $X$ with $m > n$, we have
      \begin{align*}
        d(x_m,x_n) &\leq d(x_m,x_{m-1}) + \cdots + d(x_{n+1},x_n)\\
                   &\leq \theta^{m}d(x_1,x_0) + \cdots + \theta^{n+1}d(x_1,x_0)\\
                   &= d(x_1,x_0) \theta^{n+1}\left(1 + \theta + \cdots + \theta^{m-n-1}\right)\\
                   &\leq d(x_1,x_0)\frac{\theta^{n+1}}{1-\theta}.
      \end{align*}
      Since $\left(\frac{\theta^{n+1}}{1-\theta}\right)_n\rightarrow 0$ in $\R$ as $n\rightarrow\infty$, it must be the case that $d(x_m,x_n)\rightarrow 0$ as $m,n\rightarrow \infty$. Since $X$ is complete, this means $(x_n)_n\rightarrow x$ for some $x\in X$, meaning $f(x) = x$.\\

      Suppose it were the case that there existed $s,t$ distinct with $f(s) = s$ and $f(t) = t$. Then, $d(f(s),f(t)) = d(s,t) \leq \theta d(s,t)$, but $\theta < 1$. Thus, $x$ is unique.
  \end{description}
  \section{Problem 5}%
  If $(f_k)_k$ is an orthonormal sequence in a Hilbert space $\mathcal{H}$, show that the map
  \begin{align*}
    \varphi&: \ell_2 \rightarrow \mathcal{H}\\
    \varphi(\xi) &= \sum_{k=1}^{\infty}\xi(k)f_k
  \end{align*}
  is a well-defined isometry.
  \begin{description}
    \item[Proof:] Let $\xi,\eta\in \ell_2$. Then,
      \begin{align*}
        d(\xi,\eta) &= \norm{\xi-\eta}_2\\
        \varphi(\xi) &= \sum_{k=1}^{\infty}\xi(k)f_k\\
        \varphi(\eta) &= \sum_{k=1}^{\infty}\eta(k)f_k\\
        d(\varphi(\xi),\varphi(\eta)) &= \left(\sum_{k=1}^{\infty}\iprod{\xi(k)-\eta(k)}{f_k}\right)^{1/2}\\
                                      &=\norm{\xi - \eta}_2 \tag*{Parseval's Identity.}
      \end{align*}
  \end{description}
\end{document}
