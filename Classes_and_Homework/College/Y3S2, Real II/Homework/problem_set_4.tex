\documentclass[8pt]{extarticle}
\title{}
\author{}
\date{}
\usepackage[shortlabels]{enumitem}


%paper setup
\usepackage{geometry}
\geometry{letterpaper, portrait, margin=1in}
\usepackage{fancyhdr}
% sans serif font:
\usepackage{cmbright}
%symbols
\usepackage{amsmath}
\usepackage{bigints}
\usepackage{amssymb}
\usepackage{amsthm}
\usepackage{mathtools}
\usepackage{bbold}
\usepackage[hidelinks]{hyperref}
\usepackage{gensymb}
\usepackage{multirow,array}
\usepackage{multicol}

\newtheorem*{remark}{Remark}
\usepackage[T1]{fontenc}
\usepackage[utf8]{inputenc}

%chemistry stuff
%\usepackage[version=4]{mhchem}
%\usepackage{chemfig}

%plotting
\usepackage{pgfplots}
\usepackage{tikz}
\usetikzlibrary{cd}
\tikzset{middleweight/.style={pos = 0.5}}
%\tikzset{weight/.style={pos = 0.5, fill = white}}
%\tikzset{lateweight/.style={pos = 0.75, fill = white}}
%\tikzset{earlyweight/.style={pos = 0.25, fill=white}}

%\usepackage{natbib}

%graphics stuff
\usepackage{graphicx}
\graphicspath{ {./images/} }
\usepackage[style=numeric, backend=biber]{biblatex} % Use the numeric style for Vancouver
\addbibresource{the_bibliography.bib}
%code stuff
%when using minted, make sure to add the -shell-escape flag
%you can use lstlisting if you don't want to use minted
%\usepackage{minted}
%\usemintedstyle{pastie}
%\newminted[javacode]{java}{frame=lines,framesep=2mm,linenos=true,fontsize=\footnotesize,tabsize=3,autogobble,}
%\newminted[cppcode]{cpp}{frame=lines,framesep=2mm,linenos=true,fontsize=\footnotesize,tabsize=3,autogobble,}

%\usepackage{listings}
%\usepackage{color}
%\definecolor{dkgreen}{rgb}{0,0.6,0}
%\definecolor{gray}{rgb}{0.5,0.5,0.5}
%\definecolor{mauve}{rgb}{0.58,0,0.82}
%
%\lstset{frame=tb,
%	language=Java,
%	aboveskip=3mm,
%	belowskip=3mm,
%	showstringspaces=false,
%	columns=flexible,
%	basicstyle={\small\ttfamily},
%	numbers=none,
%	numberstyle=\tiny\color{gray},
%	keywordstyle=\color{blue},
%	commentstyle=\color{dkgreen},
%	stringstyle=\color{mauve},
%	breaklines=true,
%	breakatwhitespace=true,
%	tabsize=3
%}
% text + color boxes
\renewcommand{\mathbf}[1]{\mathbb{#1}}
\usepackage[most]{tcolorbox}
\tcbuselibrary{breakable}
\tcbuselibrary{skins}
\newtcolorbox{problem}[1]{colback=white,enhanced,title={\small #1},
          attach boxed title to top center=
{yshift=-\tcboxedtitleheight/2},
boxed title style={size=small,colback=black!60!white}, sharp corners, breakable}
%including PDFs
%\usepackage{pdfpages}
\setlength{\parindent}{0pt}
\usepackage{cancel}
\pagestyle{fancy}
\fancyhf{}
\rhead{Avinash Iyer}
\lhead{Real Analysis II: Problem Set 4}
\newcommand{\card}{\text{card}}
\newcommand{\ran}{\text{ran}}
\newcommand{\N}{\mathbb{N}}
\newcommand{\Q}{\mathbb{Q}}
\newcommand{\Z}{\mathbb{Z}}
\newcommand{\R}{\mathbb{R}}
\newcommand{\C}{\mathbb{C}}
\newcommand{\iprod}[2]{\left\langle #1,#2\right\rangle}
\newcommand{\norm}[1]{\left\Vert #1\right\Vert}
\setcounter{secnumdepth}{0}
\begin{document}
  \section{Problem 1}%
  Let $X$ be a metric space. Show that $X$ is second countable if and only if $X$ is separable. Conclude that if $X$ is a separable metric space, then every open set is the union of countably many open balls.
  \begin{description}
    \item[Proof:]\hfill
      \begin{description}[font=\normalfont]
        \item[$(\Rightarrow)$:] Let $X$ be second countable. Then, $X$ contains base $U_1,U_2,\dots\in \mathcal{B}$ such that each $U_i$ is nonempty. Let $x_1\in U_1,x_2\in U_2,\dots$.\\

          The set $\{x_i\}_{i\geq1}$ is countable, as each $x_i\in U_i$. For any $U\in \tau_X$ where $U\neq \emptyset$, $U = \bigcup_{i\in I}U_i$, meaning that $U\cap \{x_i\}_{i\geq 1} \neq \emptyset$. Thus, $\{x_i\}_{i\geq 1}$ is dense in $X$, meaning $X$ is separable.
        \item[$(\Leftarrow)$:] Let $X$ be separable, with countable dense subset $\{x_{i}\}_{i\geq 1}$. Let 
          \begin{align*}
            \mathcal{B} &= \{U(x_i,1/n)\mid x_i\in \{x_i\}_{i\geq 1},n\in\N\}.
          \end{align*}
          Then, for every $U\in \tau_X$, since $U\cap \{x_i\}_{i\geq 1}\neq \emptyset$, and $\exists n$ such that $U(x_k,1/n)\subseteq U$, it must be the case that $\mathcal{B}$ is a base for $\tau_X$. Thus, $X$ is second countable.
      \end{description}
      If $X$ is a separable metric space, then it admits a countable base, and any element of $\tau_X$ is a union of the elements of the base, implying that any element of $\tau_X$ is a union of countably many open balls.
  \end{description}
  \section{Problem 2}%
  Let $(X,d)$ be a metric space, $(x_n)_n$ a sequence in $x$, and $x\in X$. The following are equivalent:
  \begin{enumerate}[(i)]
    \item $(x_n)_n \rightarrow x$ in $X$;
    \item $(d(x_n,x))_n\rightarrow 0$ in $\R$;
    \item For every neighborhood $V\in \mathcal{N}_x$, there is an $N\in \N$ with $n\geq N \Rightarrow x_n\in V$.
  \end{enumerate}
  \begin{description}
    \item[Proof:] Let $(x_n)_n\rightarrow x$ in $X$. Then, for any $\varepsilon > 0$, $\exists N$ large such that $n\geq N \Rightarrow d(x_n,x) < \varepsilon$. However, this is precisely the same as $|d(x_n,x) - 0| < \varepsilon$, which is true if and only if $(d(x_n,x))\rightarrow 0$.
  \end{description}
  \section{Problem 3}%
  Let $X$ be a metric space. Show that a sequence $(x_n)_n$ converges in $X$ if and only if every subsequence $\left(x_{n_k}\right)_k$ admits a convergent subsequence $\left(x_{n_{k_j}}\right)_j$.
  \begin{description}
    \item[Proof:] I don't know how to do this.
  \end{description}
  \section{Problem 4}%
  Let $\{(X_k,d_k)\}$ be a family of metric spaces. Assume that for every $k\geq 1$, we have $d_k(x,y)\leq 1$ for all $x,y\in X_k$. Let 
  \begin{align*}
    X &:= \prod_{k\geq 1}X_k\\
    \intertext{denote the product with metric}
    d(f,g) &:= \sum_{k=1}^{\infty}2^{-k}d_k(f(k),g(k)).
  \end{align*}
  Show that a sequence $(f_n)_n$ converges to $f$ in $X$ if and only if $(f_n(k))_n\rightarrow f(k)$ for every $k\geq 1$.
    \begin{description}
      \item[Proof:] Let $(f_n)_n\rightarrow f$. Then, $(d(f_n,f))_n\rightarrow 0$. Therefore, for $\varepsilon > 0$, there exists an $N$ large such that
        \begin{align*}
          \sum_{k=1}^{\infty}2^{-k}d_k(f_n(k),f(k)) < \varepsilon
        \end{align*}
        for $n\geq N$. 
    \end{description}
  \section{Problem 5}%
  Let $V$ be a normed space. Show that the operations
  \begin{align*}
    a: V\times V \rightarrow V;\\
    a(v,w) = v+w\\
    \intertext{and}
    \mu: \mathbb{F}\times V\rightarrow V;\\
    \mu(\alpha,v) = \alpha v
  \end{align*}
  are continuous.
  \begin{description}
    \item[Proof:]\hfill
    \begin{itemize}
        \item $a: V\times V\rightarrow V$, $a(v,w) = v+w$:
          \begin{align*}
            \norm{a(v,w) - a(v',w')} &= \norm{v+w-(v'+  w')}\\
                                     &= \norm{v-v' + w-w'}\\
                                     &\leq \norm{v-v'} + \norm{w-w'}\\
                                     &= d(v,v') + d(w,w')\\
                                     &= d_1((v,w),(v',w')),
          \end{align*}
          meaning $a$ is Lipschitz.
        \item $\mu: \mathbb{F}\times V \rightarrow V$, $\mu(\alpha,v) = \alpha v$;
          \begin{align*}
            \norm{\mu(\alpha,v)-\mu(\beta,w)} &= \norm{\alpha v - \beta w}\\
                                          &= \norm{\alpha v - \alpha w + \alpha w - \beta w}\\
                                          &\leq |\alpha|\norm{v-w} + |\alpha - \beta| \norm{w}\\
          \intertext{If $(\alpha_n)_n\rightarrow \beta$ and $(v_n)_n\rightarrow w$, then}
            \norm{\alpha_nv_n - \beta w} &\leq |\alpha_n|\norm{v_n-w} + |\alpha_n - \beta|\norm{w}\\
                                         &\rightarrow 0.
          \end{align*}
    \end{itemize}
  \end{description}
  \section{Problem 6}%
  Let $(X,d)$ be a metric space, $f,g: X\rightarrow \mathbb{F}$ continuous maps, and $\alpha\in \mathbb{F}$. Show that $f+g$, $fg$, and $\alpha f$ are continuous.
  \begin{description}
    \item[Proof:] Let $(x_n)_n\rightarrow x\in X$. Then, we know that $|f(x_n)-f(x)| \rightarrow 0$ and $|g(x_n)-g(x)|\rightarrow 0$ (where $|\cdot|$ denotes absolute value in $\mathbb{F}$). Let $\varepsilon > 0$. Therefore, for $N$ large, we know that
      \begin{align*}
        |f(x_n) + g(x_n) - (f(x) + g(x))| &\leq |f(x_n) - f(x)| + |g(x_n)-g(x)|\\
                                          & < \varepsilon/2 + \varepsilon/2\\
                                          &= \varepsilon,
      \end{align*}
      meaning $|f(x_n) + g(x_n) - (f(x) + g(x))|\rightarrow 0$, so $(f(x_n) + g(x_n))_n\rightarrow f(x) + g(x)$. Thus, $f + g$ is continuous.\\

      Similarly,
      \begin{align*}
        |f(x_{n})g(x_n) - f(x)g(x)| &= |f(x_n)g(x_n) - f(x_n)g(x) + f(x_n)g(x) - f(x)g(x)|\\
                                    &= |f(x_n)(g(x_n)-g(x)) + g(x)(f(x_n)-f(x))|\\
                                    &\leq |f(x_n)||g(x_n)-g(x)| + |g(x)||f(x_n)-f(x)|\\
                                    &\leq c|g(x_n)-g(x)| + g(x)|f(x_n)-f(x)|\tag*{convergent sequences are bounded}\\
                                    &< \varepsilon
      \end{align*}
      so $(f(x_n)g(x_n))_n \rightarrow f(x)g(x)$.
  \end{description}
  \section{Problem 8}%
  Let $h: X\rightarrow Y$ be a homeomorphism of metric spaces. Show that the map
  \begin{align*}
    T_h: (C(X),\norm{\cdot}_u) \rightarrow (C(Y),\norm{\cdot}_u)\\
    T_h(f) = f\circ h
  \end{align*}
  is an isometric isomorphism of normed spaces.
  \begin{description}
    \item[Proof:] We will show that $T$ is linear, bijective, and isometric. 
      \begin{align*}
        T_{h}(f + g) &= (f+g)\circ h\\
                     &= f\circ h + g\circ h \\
                     &= T_h(f) + T_h(g).
      \end{align*}
      Let $T_h(f) = T_h(g)$. Then,
      \begin{align*}
        f\circ h &= g\circ h\\
        (f\circ h)\circ h^{-1} &= (g\circ h)\circ h^{-1}\\
        f\circ (h\circ h^{-1}) &= g\circ (h\circ h^{-1})\\
        f &= g.
      \end{align*}
  \end{description}
  
  \section{Problem 9}%
  Suppose $T: V\rightarrow W$ is a bijective linear map between normed spaces with $\norm{T}_{\text{op}} \leq 1$ and $\norm{T^{-1}}_{\text{op}}\leq 1$. Show that $T$ is an isometry.
  \begin{description}
    \item[Proof:] Since the operator norm for $T$ is less than or equal to $1$, we know that for $v,w\in V$,
      \begin{align*}
        \norm{T(v)-T(w)}_{W} &\leq \norm{v-w}_{V}\\
        \intertext{and}
        \norm{T^{-1}(T(v)) - T^{-1}(T(w))}_{V} &\leq \norm{T(v)-T(w)}_{W}\\
        \intertext{so, since $T$ is bijective,}
        \norm{v-w}_{V} &\leq \norm{T(v)-T(w)}_{W}\\
        \intertext{meaning}
        \norm{T(v)-T(w)}_{W} &= \norm{v-w}_{V}\\
        \intertext{so $T$ is an isometry.}
      \end{align*}
  \end{description}
  \section{Problem 10}%
  For each $\lambda = (\lambda_k)_k$ in $\ell_{\infty}$, define
  \begin{align*}
    \varphi_{\lambda}: \ell_1 \rightarrow \mathbb{F};\\
    \varphi_{\lambda}((a_k)_k) &= \sum_{k=1}^{\infty}\lambda_ka_k.
  \end{align*}
  \begin{enumerate}[(i)]
    \item Show that $\varphi_{\lambda}$ is well-defined and bounded linear.
      \begin{description}
        \item[Proof:] We will show that $\varphi_{\lambda}$ is linear, then well-defined, and we will show it is bounded in part (ii).
          \begin{align*}
            \varphi_{\lambda}((a_k)_k + (b_k)_k) &= \sum_{k=1}^{\infty}\lambda_k(a_k + b_k)\\
                                                 &= \sum_{k=1}^{\infty}\left(\lambda_ka_k + \lambda_kb_k\right)\\
                                                 &= \sum_{k=1}^{\infty}\lambda_ka_k + \sum_{k=1}^{\infty}\lambda_kb_k\\
                                                 &= \varphi_{\lambda}((a_k)_k) +\varphi_{\lambda}((b_k)_k)\\
            \varphi_{\lambda}(\alpha(a_k)_k) &= \sum_{k=1}^{\infty}\lambda_k(\alpha a_k)\\
                                             &= \sum_{k=1}^{\infty}\alpha\lambda_ka_k\\
                                             &= \alpha\sum_{k=1}^{\infty}\lambda_ka_k\\
                                             &= \alpha\varphi_{\lambda}((a_k)_k).
          \end{align*}
          Additionally, it is the case that $\varphi_{\lambda}((a_k)_k) = 0$ if and only if $a_k = 0$ for all $k$, meaning $\varphi_{\lambda}$ is linear.
      \end{description}
    \item Show that $\norm{\varphi_{\lambda}}_{\text{op}} = \norm{\lambda}_{\infty}$.
      \begin{description}
        \item[Proof:]
          \begin{align*}
            \norm{\varphi_{\lambda}((a_k)_k)}_{1} &= \sum_{k=1}^{\infty}|\lambda_ka_k|\\
                                                  &\leq \sum_{k=1}^{\infty}\norm{\lambda}_{\infty}|a_k|\\
                                                  &= \norm{\lambda}_{\infty}\sum_{k=1}^{\infty}|a_k|\\
                                                  &= \norm{\lambda}_{\infty}\norm{(a_k)_k}_1
          \end{align*}
          Therefore, $\norm{\varphi_{\lambda}}_{\text{op}} = \norm{\lambda}_{\infty}$.
      \end{description}
    \item Show that $\varphi: \ell_{\infty}\rightarrow \ell_{1}^{\ast}$, $\lambda \mapsto \varphi_{\lambda}$ is a linear isometry.
      \begin{description}
        \item[Proof:] 
      \end{description}
  \end{enumerate}
\end{document}
