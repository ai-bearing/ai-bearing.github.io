\documentclass[8pt]{extarticle}
\title{}
\author{}
\date{}
\usepackage[shortlabels]{enumitem}


%paper setup
\usepackage{geometry}
\geometry{letterpaper, portrait, margin=1in}
\usepackage{fancyhdr}
% sans serif font:
\usepackage{cmbright}
%symbols
\usepackage{amsmath}
\usepackage{bigints}
\usepackage{amssymb}
\usepackage{amsthm}
\usepackage{mathtools}
\usepackage{bbold}
\usepackage[hidelinks]{hyperref}
\usepackage{gensymb}
\usepackage{multirow,array}
\usepackage{multicol}

\newtheorem*{remark}{Remark}
\usepackage[T1]{fontenc}
\usepackage[utf8]{inputenc}

%chemistry stuff
%\usepackage[version=4]{mhchem}
%\usepackage{chemfig}

%plotting
\usepackage{pgfplots}
\usepackage{tikz}
\usetikzlibrary{cd}
\tikzset{middleweight/.style={pos = 0.5}}
%\tikzset{weight/.style={pos = 0.5, fill = white}}
%\tikzset{lateweight/.style={pos = 0.75, fill = white}}
%\tikzset{earlyweight/.style={pos = 0.25, fill=white}}

%\usepackage{natbib}

%graphics stuff
\usepackage{graphicx}
\graphicspath{ {./images/} }
\usepackage[style=numeric, backend=biber]{biblatex} % Use the numeric style for Vancouver
\addbibresource{the_bibliography.bib}
%code stuff
%when using minted, make sure to add the -shell-escape flag
%you can use lstlisting if you don't want to use minted
%\usepackage{minted}
%\usemintedstyle{pastie}
%\newminted[javacode]{java}{frame=lines,framesep=2mm,linenos=true,fontsize=\footnotesize,tabsize=3,autogobble,}
%\newminted[cppcode]{cpp}{frame=lines,framesep=2mm,linenos=true,fontsize=\footnotesize,tabsize=3,autogobble,}

%\usepackage{listings}
%\usepackage{color}
%\definecolor{dkgreen}{rgb}{0,0.6,0}
%\definecolor{gray}{rgb}{0.5,0.5,0.5}
%\definecolor{mauve}{rgb}{0.58,0,0.82}
%
%\lstset{frame=tb,
%	language=Java,
%	aboveskip=3mm,
%	belowskip=3mm,
%	showstringspaces=false,
%	columns=flexible,
%	basicstyle={\small\ttfamily},
%	numbers=none,
%	numberstyle=\tiny\color{gray},
%	keywordstyle=\color{blue},
%	commentstyle=\color{dkgreen},
%	stringstyle=\color{mauve},
%	breaklines=true,
%	breakatwhitespace=true,
%	tabsize=3
%}
% text + color boxes
\renewcommand{\mathbf}[1]{\mathbb{#1}}
\usepackage[most]{tcolorbox}
\tcbuselibrary{breakable}
\tcbuselibrary{skins}
\newtcolorbox{problem}[1]{colback=white,enhanced,title={\small #1},
          attach boxed title to top center=
{yshift=-\tcboxedtitleheight/2},
boxed title style={size=small,colback=black!60!white}, sharp corners, breakable}
%including PDFs
%\usepackage{pdfpages}
\setlength{\parindent}{0pt}
\usepackage{cancel}
\pagestyle{fancy}
\fancyhf{}
\rhead{Avinash Iyer}
\lhead{Real Analysis II: Problem Set 3}
\newcommand{\card}{\text{card}}
\newcommand{\ran}{\text{ran}}
\newcommand{\N}{\mathbb{N}}
\newcommand{\Q}{\mathbb{Q}}
\newcommand{\Z}{\mathbb{Z}}
\newcommand{\R}{\mathbb{R}}
\newcommand{\C}{\mathbb{C}}
\newcommand{\iprod}[2]{\left\langle #1,#2\right\rangle}
\newcommand{\norm}[1]{\left\Vert #1\right\Vert}
\setcounter{secnumdepth}{0}
\begin{document}
  \section{Problem 1}%
  Let $X = \{0,1\}^n$. Show that the Hamming distance:
  \begin{align*}
    d_H: X\times X \rightarrow [0,\infty)\\
    d_H\left((x_j)_{j=1}^{n},(y_j)_{j=1}^{n}\right) = \left|\left\{j\mid x_j\neq y_j\right\}\right|
  \end{align*}
  defines a metric on $X$.
  \begin{description}
    \item[Proof:]\hfill
      \begin{itemize}
        \item Symmetry:
          \begin{align*}
            d_H\left((x_j)_{j=1}^{n},(y_j)_{j=1}^{n}\right) &= \left|\left\{j\mid x_j \neq y_j\right\}\right|\\
                                                            &= \left|\left\{j\mid y_j \neq x_j\right\}\right|\\
                                                            &= d_H\left((y_j)_{j=1}^{n},(x_j)_{j=1}^{n}\right)
          \end{align*}
        \item Definiteness: it is only the case that $d_H(x_j,y_j) = 0$ if $x_j = y_j$ for all $j$, by the definition of the distance.
        \item Similarly, since $x_j = x_j$ for all $j$, $d_H(x_j,x_j) = 0$.
        \item Let $(x_j)_j$, $(y_j)_j$, and $(z_j)_j$ be sequences of bits. The set $\{j\mid x_j\neq z_j\}$ is formed by taking all the values $\{j\mid x_j\neq y_j\}$ along with $\{j\mid y_j\neq z_j\}$, net of particular indices where $x_j = z_j$, but $x_j\neq y_j$. Therefore,
          \begin{align*}
            d(x,z) &\leq d(x,y) + d(y,z).
          \end{align*}
      \end{itemize}
  \end{description}
  \section{Problem 2}%
  If $\norm{\cdot}$ and $\norm{\cdot}'$ are equivalent norms on a vector space $V$, show that the induced metrics $d$ and $d'$ are equivalent.
  \begin{description}
    \item[Proof:] Let $\norm{\cdot}$ and $\norm{\cdot}'$ be equivalent norms. Then, $\exists c_1,c_2\in \R$ such that $\norm{v-w}' \leq c_1\norm{v-w}$ and $\norm{v-w} \leq c_2\norm{v-w}'$. However, this is the exact same statement as $d(v,w)\leq c_1d'(v,w)$ and $d'(v,w)\leq c_2d'(v,w)$. Thus, $d$ and $d'$ are equivalent metrics.
  \end{description}
  \section{Problem 3}%
  Let $\{X_k,d_k\}$ be a sequence of metric spaces with uniformly bounded metrics. Let
  \begin{align*}
    X := \prod_{k\geq 1} X_k
  \end{align*}
  denote the product.
  \begin{enumerate}[(a)]
    \item Show that
      \begin{align*}
        D: X\times X \rightarrow [0,\infty)\\
        D(x,y) := \sum_{k\geq1}2^{-k}d_k(x_k,y_k)
      \end{align*}
      defines a metric on $X$.
    \item Consider the case where $\{X_k\} = \{0,2\}$ and $d_k(a,b) = |a-b|$ for every $k\geq 1$. We get the abstract Cantor set
      \begin{align*}
        \Delta := \prod_{k\geq 1} \{0,2\};\\
        D(x,y) := \sum_{k=1}^{\infty}3^{-k}|x_k - y_k|.
      \end{align*}
      Prove that $D(x,z) = D(y,z)$ implies $x=y$.
  \end{enumerate}
  \section{Problem 4}%
  Let $(V,\norm{\cdot})$ be a normed space, and suppose $E\subseteq V$. Show that the following are equivalent:
  \begin{enumerate}[(1)]
    \item $E$ is bounded --- $\text{diam}(E) < \infty$;
    \item $\sup_{v\in E}\norm{v} < \infty$;
    \item there is an $r > 0$ such that $E\subseteq B(0,r)$.
  \end{enumerate}
  \begin{description}
    \item[Proof:] We will start by showing (i) implies (ii). Let $E$ be a bounded subset of $V$. Thus, for all $v,w\in E$, $\norm{v-w} \leq c$ for some $c\in \R^{+}$.
  \end{description}
  \section{Problem 5}%
  Let $(X,d)$ be a metric space and suppose $A\subseteq X$. Show:
  \begin{enumerate}[(i)]
    \item $\overline{A^c} = (A^{\circ})^{c}$
    \item $(\overline{A})^{c} = (A^{c})^{\circ}$
  \end{enumerate}
  \begin{description}
    \item[Proof:]\hfill
      \begin{enumerate}[(i)]
        \item Let $x\in \overline{A^{c}}$. Then, $\exists \delta > 0 $ such that $U(x,\delta)\cap A^{c} \neq \emptyset$. Therefore, $\exists \delta > 0$ such that $U(x,\delta)\cap A$
      \end{enumerate}
  \end{description}
  \section{Problem 9}%
  Show that $c_0$ with $\norm{\cdot}_u$ is separable.
  \begin{description}
    \item[Proof:] Let $z\in c_0$. Set $\varepsilon_1 > 0$, then finding $N_1$ large such that for all $n > N_1$, $z_n < \varepsilon_1$. Set $z'\in c_{00}$ to be equal to $z$ on $1,\dots,N_1$ and equal to $0$ for all $n > N_1$.\\

      Recall that for
      \begin{align*}
        E_n &= \left\{\sum_{k=1}^{n}\alpha_ke_k\mid \alpha_k\in \Q\right\},\\
        E &= \bigcup E_n,
      \end{align*}
      $E$ is dense in $c_{00}$, meaning that there exists some  $w\in c_{00}$ such that $\norm{z'-w} < \varepsilon$ for any $\varepsilon > 0$. However, since $z' = z$ for all $n$ from $1,\dots,N_1$, and the index of $\norm{z}_{u}$ is contained in $1,\dots,N_1$, this means $\norm{z-w} < \varepsilon$, meaning $E$ is dense in $c_{0}$.\\

      Since $E$ is countable, this means $c_0$ is countable.
  \end{description}
  \section{Problem 10}%
  Let $\mathcal{C}$ denote the Cantor set. Show that $\mathcal{C}$ is nowhere dense.
  \begin{description}
    \item[Proof:] We know that $\mathcal{C}$ is closed, meaning all we need show is that $\mathcal{C}^{0} = \emptyset$.\\

      Suppose toward contradiction that $\mathcal{C}^{0}$ is not empty. Then, $\exists x\in \mathcal{C}$ and $\varepsilon > 0$ such that $(x-\varepsilon, x+\varepsilon) \subseteq \mathcal{C}$.\\

      Find $m$ so large such that $3^{-m} < \varepsilon$. Then, $(x-\varepsilon, x + \varepsilon)$ must be contained in a subinterval with length $\frac{1}{3^m}$. However, $2\varepsilon > \frac{1}{3^m}$, and every subinterval in the element $\mathcal{C}_m$ has length $\frac{1}{3^m}$.
  \end{description}
\end{document}
