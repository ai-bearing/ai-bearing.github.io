\documentclass[8pt]{extarticle}
\title{}
\author{}
\date{}
\usepackage[shortlabels]{enumitem}


%paper setup
\usepackage{geometry}
\geometry{letterpaper, portrait, margin=1in}
\usepackage{fancyhdr}
% sans serif font:
\usepackage{cmbright}
%symbols
\usepackage{amsmath}
\usepackage{bigints}
\usepackage{amssymb}
\usepackage{amsthm}
\usepackage{mathtools}
\usepackage{bbold}
\usepackage[hidelinks]{hyperref}
\usepackage{gensymb}
\usepackage{multirow,array}
\usepackage{multicol}

\newtheorem*{remark}{Remark}
\usepackage[T1]{fontenc}
\usepackage[utf8]{inputenc}

%chemistry stuff
%\usepackage[version=4]{mhchem}
%\usepackage{chemfig}

%plotting
\usepackage{pgfplots}
\usepackage{tikz}
\usetikzlibrary{cd}
\tikzset{middleweight/.style={pos = 0.5}}
%\tikzset{weight/.style={pos = 0.5, fill = white}}
%\tikzset{lateweight/.style={pos = 0.75, fill = white}}
%\tikzset{earlyweight/.style={pos = 0.25, fill=white}}

%\usepackage{natbib}

%graphics stuff
\usepackage{graphicx}
\graphicspath{ {./images/} }
\usepackage[style=numeric, backend=biber]{biblatex} % Use the numeric style for Vancouver
\addbibresource{the_bibliography.bib}
%code stuff
%when using minted, make sure to add the -shell-escape flag
%you can use lstlisting if you don't want to use minted
%\usepackage{minted}
%\usemintedstyle{pastie}
%\newminted[javacode]{java}{frame=lines,framesep=2mm,linenos=true,fontsize=\footnotesize,tabsize=3,autogobble,}
%\newminted[cppcode]{cpp}{frame=lines,framesep=2mm,linenos=true,fontsize=\footnotesize,tabsize=3,autogobble,}

%\usepackage{listings}
%\usepackage{color}
%\definecolor{dkgreen}{rgb}{0,0.6,0}
%\definecolor{gray}{rgb}{0.5,0.5,0.5}
%\definecolor{mauve}{rgb}{0.58,0,0.82}
%
%\lstset{frame=tb,
%	language=Java,
%	aboveskip=3mm,
%	belowskip=3mm,
%	showstringspaces=false,
%	columns=flexible,
%	basicstyle={\small\ttfamily},
%	numbers=none,
%	numberstyle=\tiny\color{gray},
%	keywordstyle=\color{blue},
%	commentstyle=\color{dkgreen},
%	stringstyle=\color{mauve},
%	breaklines=true,
%	breakatwhitespace=true,
%	tabsize=3
%}
% text + color boxes
\renewcommand{\mathbf}[1]{\mathbb{#1}}
\usepackage[most]{tcolorbox}
\tcbuselibrary{breakable}
\tcbuselibrary{skins}
\newtcolorbox{problem}[1]{colback=white,enhanced,title={\small #1},
          attach boxed title to top center=
{yshift=-\tcboxedtitleheight/2},
boxed title style={size=small,colback=black!60!white}, sharp corners, breakable}
%including PDFs
%\usepackage{pdfpages}
\setlength{\parindent}{0pt}
\usepackage{cancel}
\pagestyle{fancy}
\fancyhf{}
\rhead{Avinash Iyer}
\lhead{Real Analysis II: Problem Set 3}
\newcommand{\card}{\text{card}}
\newcommand{\ran}{\text{ran}}
\newcommand{\N}{\mathbb{N}}
\newcommand{\Q}{\mathbb{Q}}
\newcommand{\Z}{\mathbb{Z}}
\newcommand{\R}{\mathbb{R}}
\newcommand{\C}{\mathbb{C}}
\newcommand{\iprod}[2]{\left\langle #1,#2\right\rangle}
\newcommand{\norm}[1]{\left\Vert #1\right\Vert}
\setcounter{secnumdepth}{0}
\begin{document}
  \section{Problem 1}%
  Let $X = \{0,1\}^n$. Show that the Hamming distance:
  \begin{align*}
    d_H: X\times X \rightarrow [0,\infty)\\
    d_H\left((x_j)_{j=1}^{n},(y_j)_{j=1}^{n}\right) = \left|\left\{j\mid x_j\neq y_j\right\}\right|
  \end{align*}
  defines a metric on $X$.
  \begin{description}
    \item[Proof:]\hfill
      \begin{itemize}
        \item Symmetry:
          \begin{align*}
            d_H\left((x_j)_{j=1}^{n},(y_j)_{j=1}^{n}\right) &= \left|\left\{j\mid x_j \neq y_j\right\}\right|\\
                                                            &= \left|\left\{j\mid y_j \neq x_j\right\}\right|\\
                                                            &= d_H\left((y_j)_{j=1}^{n},(x_j)_{j=1}^{n}\right)
          \end{align*}
        \item Definiteness: it is only the case that $d_H(x_j,y_j) = 0$ if $x_j = y_j$ for all $j$, by the definition of the distance.
        \item Similarly, since $x_j = x_j$ for all $j$, $d_H(x_j,x_j) = 0$.
        \item Let $(x_j)_j$, $(y_j)_j$, and $(z_j)_j$ be sequences of bits. The set $\{j\mid x_j\neq z_j\}$ is formed by taking all the values $\{j\mid x_j\neq y_j\}$ along with $\{j\mid y_j\neq z_j\}$, net of particular indices where $x_j = z_j$, but $x_j\neq y_j$. Therefore,
          \begin{align*}
            d(x,z) &\leq d(x,y) + d(y,z).
          \end{align*}
      \end{itemize}
  \end{description}
  \section{Problem 2}%
  If $\norm{\cdot}$ and $\norm{\cdot}'$ are equivalent norms on a vector space $V$, show that the induced metrics $d$ and $d'$ are equivalent.
  \begin{description}
    \item[Proof:] Let $\norm{\cdot}$ and $\norm{\cdot}'$ be equivalent norms. Then, $\exists c_1,c_2\in \R$ such that $\norm{v-w}' \leq c_1\norm{v-w}$ and $\norm{v-w} \leq c_2\norm{v-w}'$. However, this is the exact same statement as $d(v,w)\leq c_1d'(v,w)$ and $d'(v,w)\leq c_2d'(v,w)$. Thus, $d$ and $d'$ are equivalent metrics.
  \end{description}
  \section{Problem 3}%
  Let $\{X_k,d_k\}$ be a sequence of metric spaces with uniformly bounded metrics. Let
  \begin{align*}
    X := \prod_{k\geq 1} X_k
  \end{align*}
  denote the product.
  \begin{enumerate}[(a)]
    \item Show that
      \begin{align*}
        D: X\times X \rightarrow [0,\infty)\\
        D(x,y) := \sum_{k\geq1}2^{-k}d_k(x_k,y_k)
      \end{align*}
      defines a metric on $X$.
    \item Consider the case where $\{X_k\} = \{0,2\}$ and $d_k(a,b) = |a-b|$ for every $k\geq 1$. We get the abstract Cantor set
      \begin{align*}
        \Delta := \prod_{k\geq 1} \{0,2\};\\
        D(x,y) := \sum_{k=1}^{\infty}3^{-k}|x_k - y_k|.
      \end{align*}
      Prove that $D(x,z) = D(y,z)$ implies $x=y$.
  \end{enumerate}
  \begin{description}
    \item[Proof:]\hfill
      \begin{enumerate}[(i)]
        \item Let $D$ be defined as above. Then, $D((x_k)_k,(x_k)_k)$ is a sum of $d_k(x_k,x_k)$, all uniformly zero, meaning $D((x_k)_k,(x_k)_k)= 0$.\\

          Similarly, $D((x_k)_k,(y_k)_k) = 0$ implies that $d_k(x_k,y_k) = 0$ for all $x_k,y_k$. Since $d_k$ is a metric, this means $x_k = y_k$ for all $k$, implying that $(x_k)_k = (y_k)_k$.\\

          Additionally, $d_k(x_k,y_k) = d_k(y_k,x_k)$, it is the case that $D((x_k)_k,(y_k)_k) = D((y_k)_k,(x_k)_k)$.\\

          Finally, we must show the triangle inequality:
          \begin{align*}
            D((x_k)_k,(z_k)_k) &= \sum_{k=1}^{\infty}2^{-k}d_k(x_k,z_k)\\
                               &\leq \sum_{k=1}^{\infty}2^{-k}(d_k(x_k,y_k) + d(y_k,z_k))\\
                               &= \sum_{k=1}^{\infty}2^{-k}d_k(x_k,y_k) + \sum_{k=1}^{\infty}d(y_k,z_k)\\
                               &= D((x_k)_k,(y_k)_k) + D((y_k)_k,(z_k)_k).
          \end{align*}
        \item Suppose $x \neq y$. Let $\ell$ denote the smallest index where $x_{\ell}\neq y_{\ell}$. Suppose without loss of generality that $x_{\ell} = 2$ and $y_{\ell} = 0$. Then, $||x_\ell-z_{\ell}| - |y_{\ell}-z_{\ell}|| = 2\cdot 3^{-\ell}$. Additionally,
          \begin{align*}
            0 &\leq \sum_{k=\ell + 1}^{\infty}3^{-k}|x_{k}-z_{k}|\\
              &\leq \sum_{k=\ell + 1}^{\infty}3^{-k}(2)\\
              &= \frac{2}{3^{\ell + 1}}\\
              &< \frac{2}{3^{\ell}}.
          \end{align*}
          Thus, $D(x,z) \neq D(y,z)$.
      \end{enumerate}
  \end{description}
  \section{Problem 4}%
  Let $(V,\norm{\cdot})$ be a normed space, and suppose $E\subseteq V$. Show that the following are equivalent:
  \begin{enumerate}[(1)]
    \item $E$ is bounded --- $\text{diam}(E) < \infty$;
    \item $\sup_{v\in E}\norm{v} < \infty$;
    \item there is an $r > 0$ such that $E\subseteq B(0,r)$.
  \end{enumerate}
  \begin{description}
    \item[Proof:] 
      \begin{description}[font=\normalfont]
        \item[(i) $\Rightarrow$ (ii):]Let $E$ be bounded. Then,
        \begin{align*}
          \left|\norm{v}-\norm{w}\right| &\leq \norm{v-w}\\
          \sup_{v,w\in E}\left|\norm{v}-\norm{w}\right| &\leq \sup_{v,w\in E}\norm{v-w}\\
          \sup_{v\in E}\norm{v} - \inf_{w\in E}\norm{w} &\leq c\\
          \sup_{v\in E}\norm{v} &\leq c + \inf_{w\in E}\norm{w}.
        \end{align*}
        \item[(ii) $\Rightarrow$ (iii):] Since, for $v\in E$, $\sup\norm{v} < \infty$, if we set $r = \sup\norm{v} + 1$, then $v \in B(0,r)$, meaning $E\subseteq B(0,r)$.
        \item[(iii) $\Rightarrow$ (i):] Let $E$ be such that $E\subseteq B(0,r)$ for some $r$. Then, $\forall v,w\in B(0,r)$, $\norm{v-w} \leq 2r$, meaning that $\forall v,w\in E,\norm{v-w} \leq 2r$, meaning $\text{diam}(E) < \infty$.
      \end{description}
  \end{description}
  \section{Problem 5}%
  Let $(X,d)$ be a metric space and suppose $A\subseteq X$. Show:
  \begin{enumerate}[(i)]
    \item $\overline{A^c} = (A^{\circ})^{c}$
    \item $(\overline{A})^{c} = (A^{c})^{\circ}$
  \end{enumerate}
  \begin{description}
    \item[Proof:]\hfill
      \begin{enumerate}[(i)]
        \item We have previously established that $\overline{A^{c}}\subseteq (A^{\circ})^c$. Let $x\in (A^{\circ})^{c}$. Then, $x\notin A^{\circ}$, meaning $\forall \delta > 0$, $U(x,\delta)\cap A^{c} \neq \emptyset$. Thus, $x\in \overline{A^{c}}$.
        \item Let $x\in \overline{A}^{c}$. Then, $x\notin \overline{A}$, meaning $\exists \delta > 0$ such that $U(x,\delta) \cap A = \emptyset$. Thus, $U(x,\delta)\subseteq A^{c}$, meaning $x\in (A^{c})^{\circ}$.\\

          Let $x\in (A^{c})^{\circ}$. Then, $\exists \delta > 0$ such that $U(x,\delta)\subseteq A^{c}$. Therefore, $U(x,\delta)\cap A = \emptyset$, meaning $x\notin \overline{A}$, so $x\in \overline{A}^{c}$.
      \end{enumerate}
  \end{description}
  \section{Problem 6}%
  In any metric space, show that open balls are open, closed balls are closed, and spheres are closed. Moreover, in a normed space, show that $\partial U(v,r) = \partial B(v,r) = S(v,r)$.
  \begin{description}
    \item[Proof:]\hfill
      \begin{enumerate}[(i)]
        \item Let $\delta > 0$, and $A = U(x,\delta)$ for some $x\in X$. Then, for any $y\in A$, set $\varepsilon = \min\{d(x,y),\delta-d(x,y)\}$. Then, $U(y,\varepsilon)\subseteq A$.
        \item Let $M = B(x,\delta)$. Let $y\in M^{c}$. Set $\varepsilon = d(x,y) - \delta$. Then, $U(y,\varepsilon)\subseteq M^{c}$, meaning $M^{c}$ is open, and $M$ is thus closed.
        \item Let $A = S(x,\delta)$ for some $\delta > 0$. Then, $A^{c} = U(x,\delta) \cup (B(x,\delta))^{c}$, meaning $A^{c}$ is a union of open sets, which is open. Thus, $A$ is closed.
        \item We have previously established that, in a normed space, $\overline{U(v,r)} = B(v,r)$. Therefore, 
          \begin{align*}
            \partial U(v,r) &= \overline{U(v,r)}\setminus U(v,r)^{\circ}\\
                            &= \{x\mid d(x,v) \leq r\} \setminus \{x\mid d(x,v) < r\}\\
                            &= \{x\mid d(x,v) = r\}\\
                            &= S(v,r).
          \end{align*}
          Similarly, in a normed vector space, $B(v,r)^{\circ} = U(v,r)$. Therefore,
          \begin{align*}
            \partial B(v,r) &= \overline{B(v,r)}\setminus B(v,r)^{\circ}\\
                            &= \{x\mid d(x,v) \leq r\} \setminus \{x\mid d(x,v) < r\}\\
                            &= \{x\mid d(x,v) = r\}\\
                            &= S(v,r).
          \end{align*}
      \end{enumerate}
  \end{description}
  \section{Problem 7}%
  Let $(X,d)$ be a metric space, and suppose $A\subseteq X$. Show that the following are equivalent:
  \begin{enumerate}[(i)]
    \item $A$ is dense in $X$;
    \item For all $U\in \tau_X$, $U\cap A \neq \emptyset$;
    \item For all $x\in X$ and for all $\varepsilon > 0$, $U(x,\varepsilon) \cap A \neq \emptyset$;
    \item For all $x\in X$ and for all $\varepsilon > 0$, there is an $a\in A$ with $d(x,a) < \varepsilon$.
  \end{enumerate}
  \begin{description}
    \item[Proof:]\hfill
      \begin{description}[font=\normalfont]
        \item[(i) $\Leftrightarrow$ (iii):] $\overline{A} = X$ $\Leftrightarrow$ $\forall x\in \overline{A}, \forall \varepsilon > 0,U(x,\varepsilon)\cap A \neq \emptyset$ $\Leftrightarrow$ $\forall x\in X,\forall \varepsilon > 0,U(x,\varepsilon) \cap A \neq \emptyset$.
        \item[(iii) $\Leftrightarrow$ (iv):] $\forall x\in X, \forall \varepsilon > 0, U(x,\varepsilon) \cap A \neq \emptyset$ $\Leftrightarrow$ $\forall x\in X,\exists a\in A \ni a\in U(x,\varepsilon)$ $\Leftrightarrow$ $\forall x\in x,\exists a\in a\ni d(x,a) < \varepsilon$.
        \item[(iii) $\Rightarrow$ (ii):] Suppose $\forall x\in x, U(x,\varepsilon) \cap A \neq \emptyset$. Then, since $U(x,\varepsilon)\subseteq U$ for some $U\in \tau_X$, it is the case that $U\cap A \neq \emptyset$.
        \item[(ii) $\Rightarrow$ (iii):] Since $U(x,\varepsilon) \in \tau_X$, it is the case that for any $x\in X$ and any $\varepsilon > 0$, $U(x,\varepsilon) \cap A \neq \emptyset$.
      \end{description}
  \end{description}
  \section{Problem 8}%
  Let $U\subseteq \R$ be an open set. For each $x\in U$, we put
  \begin{align*}
    I_x := \bigcup \left\{(a,b)\mid -\infty \leq a < b \leq \infty,x\in (a,b)\subseteq U\right\}.
  \end{align*}
  \begin{enumerate}[(i)]
    \item Show that $I_x$ is an open interval contained in $U$ with $x\in I_x$.
    \item Given $x,y\in U$, show that $I_x = I_y$ or $I_x \cap I_y = \emptyset$.
    \item Prove that $U = \bigsqcup_{j\in J}I_j$, where $J$ is a countable set and $I_j$ are open intervals.
  \end{enumerate}
  \begin{description}
    \item[Proof of (i):] By construction, $x\in I_x$, as $x\in (a,b)$ for all $(a,b)$ that satisfy inclusion into $I_x$. Additionally, since $x\in U$, $\exists \delta_x$ such that $(x-\delta_x,x+\delta_x)\subseteq U$. Since $I_x$ is the union of all such intervals, it must be the case that $(x-\delta_x,x+\delta_x)\subseteq I_x$, meaning $I_x$ is an interval by the characterization of intervals.
    \item[Proof of (ii):] Suppose that $\exists a_1,b_1$ such that $x\in (a_1,b_1)$ and $y\in (a_1,b_1)$. Then, it must be the case that $(a_1,b_1)\subseteq I_x$ and $(a_1,b_1)\subseteq I_y$. Thus, $I_x\subseteq I_y$ and $I_y\subseteq I_x$, meaning $I_x = I_y$.\\

      If $\nexists (a,b)$ that satisfy the condition, then it must be the case that $\exists \delta_1$ and $\delta_2$ such that $(x-\delta_1,x+\delta_1)\subseteq U$ and $(y-\delta_2,y+\delta_2)\subseteq U$, but $(x-\delta_1,x+\delta_1)\cap (y-\delta_2,y+\delta_2) = \emptyset$. Therefore, $I_y \cap I_x = \emptyset$.
  \end{description}
  \section{Problem 9}%
  Show that $c_0$ with $\norm{\cdot}_u$ is separable.
  \begin{description}
    \item[Proof:] Let $z\in c_0$. Set $\varepsilon_1 > 0$, then finding $N_1$ large such that for all $n > N_1$, $z_n < \varepsilon_1$. Set $z'\in c_{00}$ to be equal to $z$ on $1,\dots,N_1$ and equal to $0$ for all $n > N_1$.\\

      Recall that for
      \begin{align*}
        E_n &= \left\{\sum_{k=1}^{n}\alpha_ke_k\mid \alpha_k\in \Q\right\},\\
        E &= \bigcup E_n,
      \end{align*}
      $E$ is dense in $c_{00}$, meaning that there exists some  $w\in c_{00}$ such that $\norm{z'-w} < \varepsilon$ for any $\varepsilon > 0$. However, since $z' = z$ for all $n$ from $1,\dots,N_1$, and the index of $\norm{z}_{u}$ is contained in $1,\dots,N_1$, this means $\norm{z-w} < \varepsilon$, meaning $E$ is dense in $c_{0}$.\\

      Since $E$ is countable, this means $c_0$ is countable.
  \end{description}
  \section{Problem 10}%
  Let $\mathcal{C}$ denote the Cantor set. Show that $\mathcal{C}$ is nowhere dense.
  \begin{description}
    \item[Proof:] We know that $\mathcal{C}$ is closed, meaning all we need show is that $\mathcal{C}^{0} = \emptyset$.\\

      Suppose toward contradiction that $\mathcal{C}^{0}$ is not empty. Then, $\exists x\in \mathcal{C}$ and $\varepsilon > 0$ such that $(x-\varepsilon, x+\varepsilon) \subseteq \mathcal{C}$.\\

      Find $m$ so large such that $3^{-m} < \varepsilon$. Then, $(x-\varepsilon, x + \varepsilon)$ must be contained in a subinterval with length $\frac{1}{3^m}$. However, $2\varepsilon > \frac{1}{3^m}$, and every subinterval in the element $\mathcal{C}_m$ has length $\frac{1}{3^m}$.
  \end{description}
\end{document}
