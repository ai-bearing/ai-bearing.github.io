\documentclass[10pt]{extarticle}
\title{}
\author{}
\date{}
\usepackage[shortlabels]{enumitem}


%paper setup
\usepackage{geometry}
\geometry{letterpaper, portrait, margin=1in}
\usepackage{fancyhdr}
% sans serif font:
\usepackage{cmbright}
%symbols
\usepackage{amsmath}
\usepackage{bigints}
\usepackage{amssymb}
\usepackage{amsthm}
\usepackage{mathtools}
\usepackage{bbold}
\usepackage[hidelinks]{hyperref}
\usepackage{gensymb}
\usepackage{multirow,array}
\usepackage{multicol}

\newtheorem*{remark}{Remark}
\usepackage[T1]{fontenc}
\usepackage[utf8]{inputenc}

%chemistry stuff
%\usepackage[version=4]{mhchem}
%\usepackage{chemfig}

%plotting
\usepackage{pgfplots}
\usepackage{tikz}
\tikzset{middleweight/.style={pos = 0.5}}
%\tikzset{weight/.style={pos = 0.5, fill = white}}
%\tikzset{lateweight/.style={pos = 0.75, fill = white}}
%\tikzset{earlyweight/.style={pos = 0.25, fill=white}}

%\usepackage{natbib}

%graphics stuff
\usepackage{graphicx}
\graphicspath{ {./images/} }
\usepackage[style=numeric, backend=biber]{biblatex} % Use the numeric style for Vancouver
\addbibresource{the_bibliography.bib}
%code stuff
%when using minted, make sure to add the -shell-escape flag
%you can use lstlisting if you don't want to use minted
%\usepackage{minted}
%\usemintedstyle{pastie}
%\newminted[javacode]{java}{frame=lines,framesep=2mm,linenos=true,fontsize=\footnotesize,tabsize=3,autogobble,}
%\newminted[cppcode]{cpp}{frame=lines,framesep=2mm,linenos=true,fontsize=\footnotesize,tabsize=3,autogobble,}

%\usepackage{listings}
%\usepackage{color}
%\definecolor{dkgreen}{rgb}{0,0.6,0}
%\definecolor{gray}{rgb}{0.5,0.5,0.5}
%\definecolor{mauve}{rgb}{0.58,0,0.82}
%
%\lstset{frame=tb,
%	language=Java,
%	aboveskip=3mm,
%	belowskip=3mm,
%	showstringspaces=false,
%	columns=flexible,
%	basicstyle={\small\ttfamily},
%	numbers=none,
%	numberstyle=\tiny\color{gray},
%	keywordstyle=\color{blue},
%	commentstyle=\color{dkgreen},
%	stringstyle=\color{mauve},
%	breaklines=true,
%	breakatwhitespace=true,
%	tabsize=3
%}
% text + color boxes
\usepackage[most]{tcolorbox}
\tcbuselibrary{breakable}
\tcbuselibrary{skins}
\newtcolorbox{problem}[1]{colback=white,enhanced,title={\small #1},
          attach boxed title to top center=
{yshift=-\tcboxedtitleheight/2},
boxed title style={size=small,colback=black!60!white}, sharp corners, breakable}
%including PDFs
%\usepackage{pdfpages}
\setlength{\parindent}{0pt}
\usepackage{cancel}
\pagestyle{fancy}
\fancyhf{}
\rhead{Avinash Iyer}
\lhead{Real Analysis II: Problem Set 1}
\newcommand{\card}{\text{card}}
\newcommand{\ran}{\text{ran}}
\newcommand{\N}{\mathbb{N}}
\newcommand{\Q}{\mathbb{Q}}
\newcommand{\Z}{\mathbb{Z}}
\newcommand{\R}{\mathbb{R}}
\setcounter{secnumdepth}{0}
\begin{document}
  \section{Problem 1}%
  Let $V$ be a vector space and suppose $\{W_i\}$ is a family of subspaces of $V$.
  \begin{enumerate}[(i)]
    \item Show that $\bigcap_{i\in I} W_i$ is the largest subspace of $V$ contained in every $W_i$.
    \begin{description}
      \item[Proof:] We will show that (a) $\bigcap_{i\in I} W_i$ is a subspace of $V$, and (b) any subspace of every $W_i$ is contained in $\bigcap_{i\in I} W_i$.
        \begin{enumerate}[(a)]
          \item Let $v_i, v_j\in \bigcap_{i\in I} W_i$, $\alpha,\beta \in \mathbb{F}$. We want to show that $\alpha v_i + \beta v_j \in \bigcap_{i\in I}W_i$. Since $v_i\in \bigcap_{i\in I}W_i$, $v_i \in W_i$ for some $W_i$, and $v_j\in W_j$ for some $W_j$. Additionally, WLOG, $v_j\in W_i$, as both $v_i$ and $v_j$ are contained within their intersection. Therefore, $\alpha v_i + \beta v_j\in W_i$, so $\alpha v_i + \beta v_j\in \bigcap_{i\in I}W_i$.
          \item Let $P\subseteq W_i$ for each $i\in I$, where $P$ is a subspace. Since $W_i \cap W_j$ is a subspace, $P\subseteq W_i$, and $P\subseteq W_j$, this means $P\subseteq W_i \cap W_j$. Similarly, $P\subseteq \bigcap_{i\in I}W_i$ by the same argument.
        \end{enumerate}
    \end{description}
    \item Show that
      \begin{align*}
        \sum_{i\in I}W_i := \left\{\sum_{i\in F}w_i \mid w_i\in W_i,~F\subseteq I \text{ finite}\right\}
      \end{align*}
      is the smallest subspace containing each $W_i$.
    \begin{description}
      \item[Proof:] We will show that $\sum_{i\in I}W_i$ is a subspace, and show that it is a subset of any subspace containing every $W_i$.\\

        Let $u,v\in \sum_{i\in I}W_i$. Then, for $\alpha,\beta\in \mathbb{F}$,
        \begin{align*}
          \alpha u + \beta v &= \alpha \sum_{i\in F_{1}}\zeta_{i}w_i + \beta\sum_{i\in F_{2}}\xi_{i}w_i\\
                             &= \sum_{i\in F_{1}}(\alpha\zeta_{i})x_i + \sum_{i\in F_{2}}(\alpha\xi_i)w_i\\
                             &\in \sum_{i\in I}W_i.
        \end{align*}
        Additionally, if $P$ is a subspace such that every $W_i$ is contained in $P$, then every linear combination of every element in every $W_i$ is contained in $P$, meaning $\sum_{i\in I}W_i$ is contained in $P$.
    \end{description}
  \item We say that $V$ is the internal direct sum of the family $\{W_i\}_{i}$ and we write $V = \bigoplus_{i\in I}W_i$ if
    \begin{enumerate}[(a)]
      \item $\displaystyle\sum_{i\in I}W_i = V$
      \item For every $\displaystyle j\in I,W_{j}\cap \left(\sum_{i\neq j}W_i\right) = \{0\}$.
    \end{enumerate}
    If $V = \bigoplus_{i\in I}W_i$, show that every $v\in V$ has a unique expression
    \begin{align*}
      v = \sum_{i\in F} w_i \tag*{$F\subseteq I$ finite, $0\neq w_i\in W_i$.}
    \end{align*}
    \begin{description}
      \item[Proof:] Let $V = \bigoplus_{i\in I} W_i$, where $W_i \cap W_j = \{0\}$ for $i\neq j$.\\

        Let $v = v' + w_i$ for some $v\neq 0\in V$. Then, $v-w_i = (v' + w_i - w_i) \in \sum_{j\neq i}W_j$. Applying this process for $w_k$ until $v^{(n)} - w_k = 0$, we find a unique expression for $v$ as $\sum_{i\in F}w_i$.
    \end{description}
  \end{enumerate}
  \section{Problem 2}%
  Let $V$ be a vector space and suppose $S\subseteq V$ is any subset. Show that
  \begin{align*}
    \text{span}(S) &= \bigcap \{W\mid S\subseteq W,~W\subseteq V \text{ subspace}\}
  \end{align*}
  Deduce that $\text{span}(S)$ is the smallest subspace of $V$ containing $S$.
  \begin{description}
    \item[Proof:] Let $W$ be a subspace containing $S$. Since $W$ is a subspace, every linear combination of every element of $S$ is inside $W$, as every element of $S$ is an element of $W$. Therefore, for \textit{every} subspace $W$ such that $S\subseteq W$, any linear combination of every element in $S$ is also in $W$ --- thus, $\text{span}(S) = W$.\\

      From this, we can see that $\text{span}(S)$ can be no smaller than any subspace containing $S$, meaning $\text{span}(S)$ is the smallest subspace of $V$ containing $S$.
  \end{description}
  \section{Problem 3}%
  Let $V$ be a vector space with subspaces $W_i\subseteq V$ for $i=1,2$. If $W_1\cup W_2\subseteq V$ is a subspace, show that $W_1\subseteq W_2$ or $W_2\subseteq W_1$.
  \begin{description}
    \item[Proof:] Suppose $W_1\nsubseteq W_2$. Thus, $\exists u\in W_1$ such that $u\notin W_2$. Since $W_1$ is a subspace, it is not empty, and thus $\exists w\in W_1$. Since $W_1\cup W_2$ is a subspace, and $u,w\in W_1\cup W_2$, $u+w\in W_1\cup W_2$. Additionally, $u+w\notin W_2$, as if it were the case, then $u+w-w\in W_2$, meaning $u\in W_2$, violating one of our assumptions. Therefore, $u+w\in W_1$, meaning $W_2\subseteq W_1$. 
  \end{description}
  \section{Problem 4}%
    Let $V$ be a vector space over $\mathbb{F}$ and suppose $W\subseteq V$ is a subspace.
    \begin{enumerate}[(i)]
      \item Show that the quotient space $V/W = \{[v]_W\mid v\in V\}$ is a vector space with operations
        \begin{align*}
          [u]_W + [v]_W &:=[u+v]_W\\
          \alpha[v]_W &:= [\alpha v]_W
        \end{align*}
        \begin{description}
          \item[Proof of Addition:] Let $u_1\sim u_2$, $v_1 \sim v_2$ under the provided equivalence relation. Then, $w_1 + u_2 = u_1$, $w_2 + v_2 = v_1$ for some $w_1,w_2\in W$. So,
            \begin{align*}
              [u_1]_W + [v_1]_W &= (u_1 + W) + (v_1 + W)\\
                                &= (u_2 + w_1 + W) + (v_2 + w_2 +W)\\
                                &= (u_2 +v_2) + W\\
                                &= [u_2+v_2]_W.
            \end{align*}
          \item[Proof of Scalar Multiplication:] Let $\alpha \in \mathbb{F}$. Let $v_1\sim v_2$, meaning $v_1 = v_2 + w$ for some $w\in W$. Then,
            \begin{align*}
              \alpha[v_1]_W &= \alpha \left(v_1 + W\right)\\
                            &= \alpha \left(v_2 + w + W\right)\\
                            &= \alpha v_2 + W\\
                            &= [\alpha v_2]_W.
            \end{align*}
        \end{description}
      \item Suppose $\Vert \cdot \Vert$ is a norm on $V$. Show that
        \begin{align*}
          \Vert [v]_W\Vert_{V/W} := \inf_{w\in W}\Vert v-w\Vert
        \end{align*}
        is a seminorm on $V/W$.
        \begin{description}
          \item[Absolute Homogeneity:] Let $\alpha\in \mathbb{F}$. Then,
            \begin{align*}
              \Vert \alpha[v]_{W}\Vert_{V/W} &= \Vert [\alpha v]_{W}\Vert_{V/W}\\
                                            &= \inf_{w\in W}\Vert \alpha v - w\Vert\\
                                            &= \inf_{w\in W}\Vert \alpha (v-w)\Vert \tag*{$W$ subspace}\\
                                            &= |\alpha|\inf_{w\in W}\Vert v-w\Vert\\
                                            &= |\alpha|\Vert [v]_{W}\Vert_{V/W}.
            \end{align*}
          \item[Triangle Inequality:] Let $u,v\in V$. Then,
            \begin{align*}
              \Vert [u]_{W} + [v]_{W}\Vert_{V/W} &= \Vert [u+v]_{W}\Vert_{V/W}\\
                                                 &= \inf_{w\in W}\Vert (u+v) - w \Vert\\
                                                 &= \inf_{w\in W}\Vert (u-w) + (v-w)\Vert \tag*{$W$ subspace}\\
                                                 &\leq \inf_{w\in W}\left(\Vert u-w\Vert + \Vert v-w\Vert\right)\\
                                                 &= \Vert[u]_{W}\Vert_{V/W} + \Vert[v]_{W}\Vert_{V/W}.
            \end{align*}
        \end{description}
    \end{enumerate}
  \section{Problem 5}%
  Show that the quantity
  \begin{align*}
    \Vert f \Vert_1 := \int_{0}^{1}|f(t)| dt
  \end{align*}
  defines a norm on $C([0,1])$ with $\Vert f\Vert_1 \leq \Vert f \Vert_u$. Are $\Vert \cdot \Vert_1$ and $\Vert \cdot \Vert_u$ equivalent norms?
  \begin{description}
    \item[Non-Negativity:] Since $|f(t)| \geq 0$ for $t\in [0,1]$ by the definition of absolute value, it is the case that $\int_{0}^{1}|f(t)|dt\geq 0$.
    \item[Positive Definite:] If $\exists x\in (0,1)$ such that $|f(x)| > 0$, then $\exists \delta$ such that for $t\in [x-\delta/2,x+\delta/2]$, $|f(t)| > 0$, meaning $\displaystyle \int_{x-\delta/2}^{x+\delta/2}|f(t)| dt> 0$. Therefore, if $\Vert f \Vert_{1} = 0$, then $f = \mathbb{0}_f$.
    \item[Absolute Homogeneity:] Let $\alpha\in\R$
      \begin{align*}
        \Vert \alpha f \Vert_1 &= \int_{0}^{1}|\alpha f(t)|dt\\
                               &= \int_{0}^{1}|\alpha||f(t)|dt\\
                               &= |\alpha|\int_{0}^{1}|f(t)|dt\\
                               &= |\alpha|\Vert f\Vert_1
      \end{align*}
    \item[Triangle Inequality:]
      \begin{align*}
        \Vert f + g \Vert_1 &= \int_{0}^{1}|f(t) + g(t)|dt\\
                            &\leq \int_{0}^{1}\left(|f(t)| + |g(t)|\right)dt\\
                            &= \int_{0}^{1}|f(t)|dt + \int_{0}^{1}|g(t)|dt\\
                            &= \Vert f \Vert_1 + \Vert g \Vert_1
      \end{align*}
    \item[Norm Comparison:]
      \begin{align*}
        |f(x)| &\leq \Vert f \Vert_u \tag*{Definition of Supremum}\\
        \int_{0}^{1}|f(x)| &\leq \int_{0}^{1}\Vert f \Vert_{u}\\
        \Vert f \Vert_{1} &\leq \Vert f \Vert_{u}.
      \end{align*}
    \item[Equivalence (or lack thereof):] Suppose toward contradiction that $\exists c\in \R^{+}$ such that $\Vert f \Vert_{u} \leq c\Vert f \Vert_{1}$ for all $f\in C([0,1])$.\\

      Find $N\in \N$ large such that $N > c$. Then, for  $g = x^{N}$, $\Vert g \Vert_{u} = 1 > c\Vert g \Vert_{1} = c(1/N)$. $\bot$\\

      Therefore, $\Vert f \Vert_{1}$ and $\Vert f \Vert_{u}$ are not equivalent norms.
  \end{description}
  \section{Problem 6}%
  Show that all the $p$-norms, $\Vert \cdot \Vert_p~(1\leq p \leq \infty)$ on $\mathbb{F}^n$ are equivalent. Also, show that if $1\leq p \leq q \leq \infty$, then $\ell_p \subseteq \ell_q$.
  \begin{description}
    \item[Proof:] We will show that for $x\in \mathbb{F}^n$, $\Vert \cdot \Vert_1$ and $\Vert \cdot \Vert_{\infty}$ are equivalent norms.
      \begin{align*}
        \Vert x \Vert_{\infty} &= \max_{1\leq j \leq n}|x_j|\\
                               &\leq \sum_{i=1}^{n}|x_i|\\
                               &= \Vert x \Vert_1.\\
        \Vert x \Vert_{1} &= \sum_{i=1}^{n}|x_i|\\
                          &\leq \sum_{i=1}^{n}\max_{1\leq j \leq n}|x_j|\\
                          &= n\Vert x \Vert_{\infty}.
      \end{align*}
      Now, we will show that any $p$ norm is equivalent to the $\infty$ norm.
      \begin{align*}
        \sum_{i=1}^{n}|x_j|^p&\leq \sum_{i=1}^{n}\left(\max_{1\leq j\leq n}|x_j|\right)^p\\
                             &= n\Vert x\Vert_{\infty}^{p},\\
                             \shortintertext{so}
        \Vert x\Vert_{p} &\leq n^{1/p}\Vert x \Vert_{\infty}.
      \end{align*}
      Since every $p\in (1,\infty)$ norm is equivalent to the $\infty$ norm, and the $\infty$ norm is equivalent to the $1$ norm, every $p$ norm is equivalent to every other $p$ norm.\\

      Let $1\leq p \leq q \leq \infty$. Then, for $x\in \ell_{p}$, $\sum_{j=1}^{\infty} |x_j|^{p}$ is convergent.\\

      Therefore, $\lim_{j\rightarrow\infty} |x_j|^{p} = 0$. So, $\exists J\in \N$ such that for $j\geq J$, $|x_j|^{p} \leq 1$.\\

      For $j\geq J$, $|x_j|^q \leq |x_j|^{p}$, $\sum_{j=J}^{\infty}|x_j|^{q} \leq \sum_{j=J}^{\infty}|x_j|^{p}$. So, $\sum_{j=1}^{\infty}|x_j|^{q}$ is convergent, meaning $x\in \ell_{q}$, and $\ell_{p} \subseteq \ell_{q}$.
  \end{description}
  \section{Problem 7}%
  Let $\mathbb{M}_{m,n}(\mathbb{C})$ denote the linear space of all $m\times n$ matrices with coefficients from $\mathbb{C}$. For $a\in \mathbb{M}_{m,n}(\mathbb{C})$, set
  \begin{align*}
    \Vert a\Vert_{\text{op}} &:= \sup_{\xi\in B_{\ell_2}^n}\Vert a\xi\Vert_{\ell_2^{m}}.
  \end{align*}
  Show that $\Vert \cdot \Vert_{\text{op}}$ is a norm on $\mathbb{M}_{m,n}(\mathbb{C})$. This is the operator norm.
  \begin{description}
    \item[Positive Definite:] Since $\ell_{2}^{m}$ is a norm on $\R^m$, it must be the case that $\Vert a\Vert_{\text{op}} = 0$ if and only if the least upper bound on $\Vert a\xi\Vert$ is zero, occurring only when $a$ is the zero operator.
    \item[Absolute Homogeneity:] Let $\alpha\in \mathbb{C}$. Then,
      \begin{align*}
        \Vert \alpha a \Vert_{\text{op}} &= \sup_{\xi\in B_{\ell_2^{n}}}\Vert (\alpha a)\xi\Vert_{\ell_{2}^{m}}\\
                                         &= \sup_{\xi\in B_{\ell_{2}^{n}}}|\alpha|\Vert a \xi \Vert_{\ell_{2}^{m}}\tag*{$\ell_{2}^{m}$ is a norm}\\
                                         &= |\alpha|\sup_{\xi\in B_{\ell_2^{n}}}\Vert a \xi\Vert_{\ell_{2}^{m}}\\
                                         &= |\alpha|\Vert a \Vert_{\text{op}}
      \end{align*}
    \item[Triangle Inequality:] Let $a,b \in \mathbb{M}_{m,n}(\mathbb{C})$. Then,
      \begin{align*}
        \Vert a + b \Vert_{\text{op}} &= \sup_{\xi\in B_{\ell_{2}^{n}}} \Vert (a+b)\xi\Vert_{\ell_{2}^{m}}\\
                                      &= \sup_{\xi \in B_{\ell_{2}^{n}}} \Vert a\xi + b\xi\Vert_{\ell_2^{m}}\\
                                      &\leq \sup_{\xi\in B_{\ell_2^{n}}} \left(\Vert a\xi\Vert_{\ell_{2}^{m}} + \Vert b\xi\Vert_{\ell_{2}^{m}}\right)\tag*{$\ell_{2}^{m}$ is a norm}\\
                                      &\leq \sup_{\xi\in B_{\ell_{2}^{n}}}\Vert a\xi\Vert_{\ell_{2}^{m}} + \sup_{\xi\in B_{\ell_{2}^{n}}} \Vert b\xi \Vert_{\ell_{2}^{m}}\\
                                      &= \Vert a\Vert_{\text{op}} + \Vert b \Vert_{\text{op}}
      \end{align*}
  \end{description}
  Let $x\in \ell_{p}$, meaning $\Vert x \Vert_p < \infty$. Since $\Vert \cdot \Vert_{p}$ and $\Vert \cdot \Vert_q$ are equivalent norms, $\exists c\in \mathbb{F}$ such that $\Vert x \Vert_{q} \leq c\Vert x \Vert_{p}$. Therefore, $\Vert x \Vert_{q} < \infty$, meaning $x\in \ell_{q}$. Therefore, $\ell_{p} \subseteq \ell_{q}$.
  \section{Problem 8}%
  If $f: [a,b]\rightarrow\R$ is any function and $\mathcal{P} = \{a=x_0<x_1<x_2 < \cdots < x_{n-1} < x_n = b\}$ is a partition of $[a,b]$, we define the variation of $f$ on $\mathcal{P}$ as
  \begin{align*}
    \text{Var}(f;\mathcal{P}) = \sum_{k=1}^{n}|f(x_k)-f(x_{k-1})|.
  \end{align*}
  We say that $f$ is of bounded variation if
  \begin{align*}
    \text{Var}(f) = \sup_{\mathcal{P}}\text{Var}(f;\mathcal{P}) < \infty,
  \end{align*}
  where the supremum runs over all partitions of $[a,b]$. We define the space of all functions of bounded variation
  \begin{align*}
    \text{BV}([a,b]) := \{f:[a,b]\rightarrow\R\mid \text{Var}(f)<\infty\}
  \end{align*}
  \begin{enumerate}[(i)]
    \item Is the function $\mathbb{1}_{\Q}: [0,1]\rightarrow \R$ of bounded variation?
      \begin{description}
        \item[Proof:] The answer is no, $\mathbb{1}_{\Q}$ is not of bounded variation. Define $\mathcal{P}$ to be a partition where each alternating member of the partition is, respectfully, a member of $\Q$ and $\R\setminus\Q$. Since both $\Q$ and $\R\setminus\Q$ are dense in $[0,1]$, this partition is valid. However, the variation of $\mathbb{1}_{\Q}$ over this partition is infinite, as there are infinitely many rational and irrational numbers in the space, and the difference between the image of each element of the infinite partition is $1$, meaning the sum is infinite.
      \end{description}
    \item Show that $\text{BV}([a,b])\subseteq \ell_{\infty}([a,b])$ is a subspace.
      \begin{description}
        \item[Proof:] We will show that $\text{BV}\subseteq \ell_{\infty}$, and that any linear combination $f,g\in \text{BV}([a,b])$ is an element of $\text{BV}([a,b])$.\\

          To show $\text{BV}([a,b])\subseteq \ell_{\infty}([a,b])$, observe that for $x\in [a,b]$,
          \begin{align*}
            |f(x)| &= |f(x)-f(a) + f(a)|\\
                   &\leq |f(x)-f(a)| + |f(a)|\\
                   &\leq \text{Var}(f) + |f(a)|,\\
           \shortintertext{meaning $|f(x)|$ is bounded above, so}
           \sup_{x\in[a,b]}|f(x)| < \infty.
          \end{align*}
          Let $f,g\in \text{BV}([a,b])$ and let $\alpha,\beta \in \R$. Then,
          \begin{align*}
            \text{Var}(\alpha f + \beta g;\mathcal{P}) &= \sum_{k=1}^{n}|(\alpha f(x_k) + \beta g(x_k)) - (\alpha f(x_{k-1}) + \beta g(x_{k-1}))|\\
                                                       &\leq\sum_{k=1}^{n}|\alpha f(x_k)-\alpha f(x)| + \sum_{k=1}^{n}|\beta g(x_k)-\beta g(x_{k-1})|\\
                                                       &= |\alpha|\sum_{k=1}^{n}|f(x_k)-f(x_{k-1})| + |\beta|\sum_{k=1}^{n}|g(x_k) - g(x_{k-1})|\\
                                                       &\leq |\alpha|\text{Var}(f) + |\beta|\text{Var}(g),\\
                                                       \shortintertext{meaning}
            \text{Var}(\alpha f + \beta g) &\leq |\alpha|\text{Var}(f) + |\beta|\text{Var}(g),
          \end{align*}
          meaning $\alpha f + \beta g\in \text{BV}([a,b])$.
      \end{description}
    \item Show that $\Vert f\Vert_{\text{BV}} := |f(a)| + \text{Var}(f)$ defines a norm on $\text{BV}([a,b])$.
      \begin{description}
        \item[Proof:] We will show that $\Vert f \Vert_{\text{BV}}$ defines a norm on $\text{BV}([a,b])$.
          \begin{description}
            \item[Positive Definite:] Let $\Vert f \Vert_{\text{BV}} = 0$. Then, since $\text{Var}(f) \geq 0$, it must be the case that $\text{Var}(f) = 0$. So, $\text{Var}(f)$ is a constant function, $f(x) = c$ for $x\in [a,b]$.\\

              Additionally, since $\Vert f \Vert_{\text{BV}} = 0$, $|f(a)| = 0$, so $f(a) = 0$. Therefore, since $f(a) = 0$ and $f$ is constant for $x\in [a,b]$, $f = \mathbb{0}$.
            \item[Absolute Homogeneity:] Let $f\in \text{BV}([a,b])$, and $\alpha\in\R$. Then,
              \begin{align*}
                \Vert f \Vert_{\text{BV}} &= |\alpha f(a)| + \text{Var}(\alpha f)\\
                                          &= |\alpha||f(a)| + \sup_{\mathcal{P}}\sum_{k=1}^{n}|\alpha f(x_k) - \alpha f(x_{k-1})|\\
                                          &= |\alpha||f(a)| + |\alpha|\sup_{\mathcal{P}}\sum_{k=1}^{n}|f(x_k)-f(x_{k-1})|\\
                                          &= |\alpha|\left(|f(a)| + \text{Var}(f)\right)\\
                                          &= |\alpha|\Vert f \Vert_{\text{BV}}.
              \end{align*}
            \item[Triangle Inequality:] Let $f,g\in\text{BV}([a,b])$. Then,
              \begin{align*}
                \Vert f + g \Vert_{\text{BV}} &= |f(a) + g(a)| + \text{Var}(f + g)\\
                \shortintertext{$\alpha,\beta = 1$ from (ii), we have}
                \Vert f + g \Vert_{\text{BV}} &\leq |f(a)| + \text{Var}(f) + |g(a)| + \text{Var}(g)\\
                                              &= \Vert f \Vert_{\text{BV}} + \Vert g \Vert_{\text{BV}}
              \end{align*}
          \end{description}
      \end{description}
  \end{enumerate}
  \section{Problem 9}%
  Given any function $f: [0,1]\rightarrow \mathbb{C}$, we define
  \begin{align*}
    N(f) &:= \sup_{\substack{x\neq y\\ x,y\in[0,1]}}\frac{|f(x)-f(y)|}{|x-y|}\\
    \shortintertext{and}
    \Vert f\Vert_{\Lambda} &:= |f(0)| + N(f).\\
    \shortintertext{Moreover, set}
    \Lambda[0,1] &:= \left\{f: [0,1]\rightarrow \mathbb{C}\mid \Vert f\Vert_{\Lambda} < \infty\right\}
  \end{align*}
  \begin{enumerate}[(i)]
    \item Show that $\Lambda[0,1]$ is precisely the set of Lipschitz continuous functions on $[0,1]$.
      \begin{description}
        \item[Proof:] Let $f\in \Lambda[0,1]$. Then, $\Vert f\Vert_{\Lambda} = c$ for some finite $c$. Then, for $x,y\in [0,1]$
          \begin{align*}
            \frac{|f(x)-f(y)|}{|x-y|}&\leq N(f)\\
                                     &\leq \Vert f\Vert_{\Lambda}\\
                                     &= c.\\
                                     \shortintertext{So,}
            |f(x)-f(y)| &\leq c|x-y|,
          \end{align*}
          which defines a Lipschitz continuous function.
      \end{description}
    \item Verify that $\Lambda[0,1]$ is a vector space with norm $\Vert f\Vert_{\Lambda}$, which is the Lipschitz norm.
      \begin{description}
        \item[Proof of Vector Space:] Let $f,g\in \Lambda[0,1]$. Then, $f$ and $g$ are Lipschitz continuous. Let $\alpha\in \mathbb{C}$. Then,
          \begin{align*}
            |(\alpha f)(x) - (\alpha f)(y)| &= |\alpha||f(x)-f(y)|\\
                                            &\leq |alpha|c|x-y|\\
                                            &= h|x-y|,\\
                                            \shortintertext{and}
            |(f+g)(x) - (f+g)(y)| &= |f(x)-f(y) + g(x)-g(y)|\\
                                  &\leq |f(x)-f(y)| + |g(x)-g(y)|\\
                                  &\leq c|x-y| + d|x-y|\\
                                  &= \ell|x-y|,
          \end{align*}
          meaning that $\Lambda[0,1]$ is closed under addition and scalar multiplication.
        \item[Proof of Norm:]\hfill
          \begin{description}
            \item[Positive Definiteness:]
              \begin{align*}
                \Vert f\Vert_{\Lambda} &= 0\\
                |f(0)| + \sup_{\substack{x\neq y\\ x,y\in[0,1]}}\frac{|f(x)-f(y)|}{|x-y|} &= 0,\\
                \shortintertext{meaning that for $x,y\in [0,1]$ and $x\neq y$}
                f(x) &= f(y)\\
                \shortintertext{and}
                f(0) = 0\\
                \shortintertext{so $f = \mathbb{0}_f$.}
              \end{align*}
            \item[Absolute Homogeneity:] Let $\alpha\in\mathbb{C}$.
              \begin{align*}
                \Vert \alpha f \Vert &= |\alpha f(0)| + N(\alpha f)\\
                                     &= |\alpha||f(0)| + \sup_{\substack{x\neq y\\ x,y\in[0,1]}}\frac{|\alpha f(x)-\alpha f(y)|}{|x-y|}\\
                                     &= |\alpha|\left(|f(0)| + \sup_{\substack{x\neq y\\ x,y\in[0,1]}}\frac{|f(x)-f(y)|}{|x-y|}\right)\\
                                     &= |\alpha|\Vert f \Vert_{\Lambda}
              \end{align*}
            \item[Triangle Inequality:] Let $f,g\in \Lambda[0,1]$. Then,
              \begin{align*}
                \Vert f + g \Vert &= |f(0) + g(0)| + \sup_{\substack{x\neq y\\ x,y\in[0,1]}}\frac{\left|f(x)+g(x)-(f(y)+g(y))\right|}{|x-y|}\\
                                  &\leq \left(|f(0)| + \sup_{\substack{x\neq y\\ x,y\in[0,1]}}\frac{|f(x)-f(y)|}{|x-y|}\right)+ \left(|g(0)| + \sup_{\substack{x\neq y\\ x,y\in[0,1]}}\frac{|g(x)-g(y)|}{|x-y|}\right)\\
                                  &= \Vert f\Vert_{\Lambda} + \Vert g \Vert_{\Lambda}
              \end{align*}
          \end{description}
      \end{description}
      Therefore, $\Lambda[0,1]$ is a normed vector space with $\Vert \cdot \Vert_{\Lambda}$ as the Lipschitz norm.
    \item Show that $\Vert f\Vert_u \leq \Vert f \Vert_{\Lambda}$ for every $f:[0,1]\rightarrow \R$.
      \begin{description}
        \item[Proof:] I don't know how to show this proof.
      \end{description}
  \end{enumerate}
  \section{Problem 10}%
  Let $p$ be a seminorm on a vector space $V$.
  \begin{enumerate}[(i)]
    \item Show that $N_p := \{w\in V\mid p(w)=0\}$ is a subspace of $V$.
      \begin{description}
        \item[Proof:] Let $v,w\in N_p$. Then, $p(v) = 0$ and $p(w) = 0$. Since $p$ is a seminorm, for $\alpha,\beta\in \mathbb{F}$, we have:
          \begin{align*}
            p(\alpha v + \beta w) &\leq p(\alpha v) + p(\beta w)\\
                                  &= |\alpha| p(v) + |\beta|p(w)\\
                                  &= 0.
          \end{align*}
          Since $p$ is definitionally non-negative, $p(\alpha v + \beta w) = 0$. Therefore, $N_p$ is a vector space.
      \end{description}
    \item We form the quotient vector space $V/N_p$. Show that
      \begin{align*}
        \Vert [v]_{N_p}\Vert_{p} := p(v)
      \end{align*}
      defines a norm on $V/N_p$.
      \begin{description}
        \item[Proof:] Since $p$ is a seminorm, we know that absolute homogeneity and the triangle inequality already hold for $p$. Therefore, what we need to show is that $\Vert \cdot \Vert_{p}$ is positive definite.\\

          Let $\Vert [v]_{N_p}\Vert_{p} = 0$. Since $p(v) = 0$, $v\in N_{p}$, meaning $[v]_{N_p} = [\mathbb{0}]_{N_p}$.
      \end{description}
    \item If $(E,\Vert\cdot\Vert)$ is a normed space and $T: V\rightarrow E$ is a linear map, show that $p(v) := \Vert T(v)\Vert$ is a seminorm on $V$. In this case, what is $N_p$.
      \begin{description}
        \item[Absolute Homogeneity:] Let $\alpha\in \mathbb{F}$. Then,
          \begin{align*}
            p(\alpha v) &= \Vert T(\alpha v)\Vert\\
                        &= \Vert \alpha T(v)\Vert\\
                        &= |\alpha|\Vert T(v)\Vert\\
                        &= |\alpha| p(v).
          \end{align*}
        \item[Triangle Inequality:] Let $v,w\in V$. Then,
          \begin{align*}
            p(v + w) &= \Vert T(v+w)\Vert\\
                     &= \Vert T(v) + T(w)\Vert\\
                     &\leq \Vert T(v)\Vert + \Vert T(w)\Vert\\
                     &= p(v) + p(w).
          \end{align*}
          The subspace $N_p$ denotes the kernel of $T$.
      \end{description}
  \end{enumerate}
\end{document}
