\documentclass[8pt]{extarticle}
\title{}
\author{}
\date{}
\usepackage[shortlabels]{enumitem}


%paper setup
\usepackage{geometry}
\geometry{letterpaper, portrait, margin=1in}
\usepackage{fancyhdr}
% sans serif font:
\usepackage{cmbright}
%symbols
\usepackage{amsmath}
\usepackage{bigints}
\usepackage{amssymb}
\usepackage{amsthm}
\usepackage{mathtools}
\usepackage{bbm}
\usepackage[hidelinks]{hyperref}
\usepackage{gensymb}
\usepackage{multirow,array}
\usepackage{multicol}

\newtheorem*{remark}{Remark}
\usepackage[T1]{fontenc}
\usepackage[utf8]{inputenc}

%chemistry stuff
%\usepackage[version=4]{mhchem}
%\usepackage{chemfig}

%plotting
\usepackage{pgfplots}
\usepackage{tikz}
\tikzset{middleweight/.style={pos = 0.5}}
%\tikzset{weight/.style={pos = 0.5, fill = white}}
%\tikzset{lateweight/.style={pos = 0.75, fill = white}}
%\tikzset{earlyweight/.style={pos = 0.25, fill=white}}

%\usepackage{natbib}

%graphics stuff
\usepackage{graphicx}
\graphicspath{ {./images/} }
\usepackage[style=numeric, backend=biber]{biblatex} % Use the numeric style for Vancouver
\addbibresource{the_bibliography.bib}
%code stuff
%when using minted, make sure to add the -shell-escape flag
%you can use lstlisting if you don't want to use minted
%\usepackage{minted}
%\usemintedstyle{pastie}
%\newminted[javacode]{java}{frame=lines,framesep=2mm,linenos=true,fontsize=\footnotesize,tabsize=3,autogobble,}
%\newminted[cppcode]{cpp}{frame=lines,framesep=2mm,linenos=true,fontsize=\footnotesize,tabsize=3,autogobble,}

%\usepackage{listings}
%\usepackage{color}
%\definecolor{dkgreen}{rgb}{0,0.6,0}
%\definecolor{gray}{rgb}{0.5,0.5,0.5}
%\definecolor{mauve}{rgb}{0.58,0,0.82}
%
%\lstset{frame=tb,
%	language=Java,
%	aboveskip=3mm,
%	belowskip=3mm,
%	showstringspaces=false,
%	columns=flexible,
%	basicstyle={\small\ttfamily},
%	numbers=none,
%	numberstyle=\tiny\color{gray},
%	keywordstyle=\color{blue},
%	commentstyle=\color{dkgreen},
%	stringstyle=\color{mauve},
%	breaklines=true,
%	breakatwhitespace=true,
%	tabsize=3
%}
% text + color boxes
\renewcommand{\mathbf}[1]{\mathbbm{#1}}
\usepackage[most]{tcolorbox}
\tcbuselibrary{breakable}
\tcbuselibrary{skins}
\newtcolorbox{problem}[1]{colback=white,enhanced,title={\small #1},
          attach boxed title to top center=
{yshift=-\tcboxedtitleheight/2},
boxed title style={size=small,colback=black!60!white}, sharp corners, breakable}
%including PDFs
%\usepackage{pdfpages}
\setlength{\parindent}{0pt}
\usepackage{cancel}
\pagestyle{fancy}
\fancyhf{}
\rhead{Avinash Iyer}
\lhead{Real Analysis II: Problem Set 1}
\newcommand{\card}{\text{card}}
\newcommand{\ran}{\text{ran}}
\newcommand{\N}{\mathbbm{N}}
\newcommand{\Q}{\mathbbm{Q}}
\newcommand{\Z}{\mathbbm{Z}}
\newcommand{\R}{\mathbbm{R}}
\setcounter{secnumdepth}{0}
\begin{document}
  \section{Problem 1}%
  Let $V$ be a vector space and suppose $\{W_i\}$ is a family of subspaces of $V$.
  \begin{enumerate}[(i)]
    \item Show that $\bigcap_{i\in I} W_i$ is the largest subspace of $V$ contained in every $W_i$.
    \begin{description}
      \item[Proof:] We will show that (a) $\bigcap_{i\in I} W_i$ is a subspace of $V$, and (b) there is is no larger subspace of $V$ contained within every $W_i$.
        \begin{enumerate}[(a)]
          \item Let $v_i, v_j\in \bigcap_{i\in I} W_i$, $\alpha,\beta \in \mathbbm{F}$. We want to show that $\alpha v_i + \beta v_j \in \bigcap_{i\in I}W_i$. Since $v_i\in \bigcap_{i\in I}W_i$, $v_i \in W_i$ for some $W_i$, and $v_j\in W_j$ for some $W_j$. Additionally, WLOG, $v_j\in W_i$, as both $v_i$ and $v_j$ are contained within their intersection. Therefore, $\alpha v_i + \beta v_j\in W_i$, so $\alpha v_i + \beta v_j\in \bigcap_{i\in I}W_i$.
          \item Suppose there is a subspace $U$ of $V$ such that every $W_i$ is contained in $U$, and $U \supset \bigcap_{i\in I}W_i$. Since $U$ is a vector space, $U$ has a basis $B_u$; additionally, since we have shown that $\bigcap_{i\in I}W_i$ is a subspace, $\bigcap_{i\in I} W_i$ has a basis, $B_w$.
        \end{enumerate}
    \end{description}
    \item Show that
      \begin{align*}
        \sum_{i\in I}W_i := \left\{\sum_{i\in F}w_i \mid w_i\in W_i,~F\subseteq I \text{ finite}\right\}
      \end{align*}
      is the smallest subspace containing each $W_i$.
  \end{enumerate}
\end{document}
