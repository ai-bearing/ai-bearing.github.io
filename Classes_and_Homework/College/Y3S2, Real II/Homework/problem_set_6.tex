\documentclass[10pt]{extarticle}
\title{}
\author{}
\date{}
\usepackage[shortlabels]{enumitem}


%paper setup
\usepackage{geometry}
\geometry{letterpaper, portrait, margin=1in}
\usepackage{fancyhdr}
% sans serif font:
\usepackage{cmbright}
%symbols
\usepackage{amsmath}
\usepackage{bigints}
\usepackage{amssymb}
\usepackage{amsthm}
\usepackage{mathtools}
\usepackage{bbold}
\usepackage[hidelinks]{hyperref}
\usepackage{gensymb}
\usepackage{multirow,array}
\usepackage{multicol}

\newtheorem*{remark}{Remark}
\usepackage[T1]{fontenc}
\usepackage[utf8]{inputenc}

%chemistry stuff
%\usepackage[version=4]{mhchem}
%\usepackage{chemfig}

%plotting
\usepackage{pgfplots}
\usepackage{tikz}
\usetikzlibrary{cd}
\tikzset{middleweight/.style={pos = 0.5}}
%\tikzset{weight/.style={pos = 0.5, fill = white}}
%\tikzset{lateweight/.style={pos = 0.75, fill = white}}
%\tikzset{earlyweight/.style={pos = 0.25, fill=white}}

%\usepackage{natbib}

%graphics stuff
\usepackage{graphicx}
\graphicspath{ {./images/} }
%\usepackage[style=numeric, backend=biber]{biblatex} % Use the numeric style for Vancouver
%\addbibresource{the_bibliography.bib}
%code stuff
%when using minted, make sure to add the -shell-escape flag
%you can use lstlisting if you don't want to use minted
%\usepackage{minted}
%\usemintedstyle{pastie}
%\newminted[javacode]{java}{frame=lines,framesep=2mm,linenos=true,fontsize=\footnotesize,tabsize=3,autogobble,}
%\newminted[cppcode]{cpp}{frame=lines,framesep=2mm,linenos=true,fontsize=\footnotesize,tabsize=3,autogobble,}

%\usepackage{listings}
%\usepackage{color}
%\definecolor{dkgreen}{rgb}{0,0.6,0}
%\definecolor{gray}{rgb}{0.5,0.5,0.5}
%\definecolor{mauve}{rgb}{0.58,0,0.82}
%
%\lstset{frame=tb,
%	language=Java,
%	aboveskip=3mm,
%	belowskip=3mm,
%	showstringspaces=false,
%	columns=flexible,
%	basicstyle={\small\ttfamily},
%	numbers=none,
%	numberstyle=\tiny\color{gray},
%	keywordstyle=\color{blue},
%	commentstyle=\color{dkgreen},
%	stringstyle=\color{mauve},
%	breaklines=true,
%	breakatwhitespace=true,
%	tabsize=3
%}
% text + color boxes
%\renewcommand{\mathbf}[1]{\mathbb{#1}}
%\usepackage[most]{tcolorbox}
%\tcbuselibrary{breakable}
%\tcbuselibrary{skins}
%\newtcolorbox{problem}[1]{colback=white,enhanced,title={\small #1},
%          attach boxed title to top center=
%{yshift=-\tcboxedtitleheight/2},
%boxed title style={size=small,colback=black!60!white}, sharp corners, breakable}
%including PDFs
%\usepackage{pdfpages}
\setlength{\parindent}{0pt}
\usepackage{cancel}
\pagestyle{fancy}
\fancyhf{}
\rhead{Avinash Iyer}
\lhead{Real Analysis II: Problem Set 6}
\newcommand{\card}{\text{card}}
\newcommand{\ran}{\text{ran}}
\newcommand{\N}{\mathbb{N}}
\newcommand{\Q}{\mathbb{Q}}
\newcommand{\Z}{\mathbb{Z}}
\newcommand{\R}{\mathbb{R}}
\newcommand{\C}{\mathbb{C}}
\newcommand{\iprod}[2]{\left\langle #1,#2\right\rangle}
\newcommand{\norm}[1]{\left\Vert #1\right\Vert}
\setcounter{secnumdepth}{0}
\begin{document}
  \section{Problem 1}%
  Show that a discrete metric space is compact if and only if it is finite.
  \begin{description}
    \item[Proof:] Let $(X,d)$ be a discrete metric space. Suppose $(X,d)$ is not finite. Then, we can create an open cover of $X$ defined by
      \begin{align*}
        X &= \bigcup_{x\in X}\{x\}.
      \end{align*}
      Since every subset of $X$ is finite, this is an open cover, but this does not contain a finite subcover as $X$ is infinite.\\

      Suppose $(X,d)$ is not compact. Then, there is an open cover of $X$
      \begin{align*}
        X \subseteq \bigcup_{i\in I} U_i
      \end{align*}
      with no finite subcover. Specifically this means that for each $i\in I$, there is some $x_{i}\in U_i$ such that $x_{i}\notin \bigcup U_{-i}$. Therefore, we have $\{x_i\}_{i=1}^{\infty}\subseteq X$, so $X$ is infinite.
  \end{description}
  \section{Problem 2}%
  Let $X$ be a metric space and suppose $Y\subseteq X$. Show that $K\subseteq Y$ is compact in $Y$ with the relative topology if and only if $K$ is compact in $X$.
  \section{Problem 3}%
  Let $X$ be a metric space. Let $(x_n)_n$ be a sequence in $X$ which converges to a point $x_0\in X$. Show that $\{x_0,x_1,\dots\}$ is compact.
  \begin{description}
    \item[Proof:] Since $(x_n)_n\rightarrow x_0\in \{x_0,x_1,x_2,\dots\} = A$ is a bounded sequence, the set $A$ is bounded. Thus, all sequences in $A$ are bounded; since we can extract a convergent subsequence in $A$ by selecting a natural sequence by recursively selecting the smallest following index that contains $x_{i}$, $i$ greater than the index of the current point. If no such $i$ exists, then the sequence converges necessarily nonetheless.\\

      Since every sequence in $\{x_0,x_1,\dots\}$ admits a convergent subsequence, $\{x_0,x_1,\dots\}$ is sequentially compact, hence compact in $X$.
  \end{description}
  \section{Problem 4}%
  Let $(X,d)$ be a metric space. If $C,K\subseteq X$, we define $\text{dist}(C,K):= \inf_{x\in C,y\in K}d(x,y)$.
  \begin{enumerate}[(i)]
    \item If $K$ is compact and $C$ is closed, show that 
      \begin{align*}
        K\cap C = \emptyset \Leftrightarrow \text{dist}(C,K)> 0
      \end{align*}
      Can we remove the requirement that $K$ is compact and only require it to be closed?
    \item If both $K$ and $C$ are compact, show that there is $x\in C$ and $y\in K$ with $\text{dist}(C,K) = d(x,y)$.
  \end{enumerate}
  \begin{description}
    \item[Proof:]\hfill
      \begin{enumerate}[(i)]
        \item Let $K\cap C = \emptyset$. Then, by the normal property, $\exists U,V\in \tau_X$ with $K\subset U$ and $C\subset V$ and $U\cap V = \emptyset$. Choose $x\in U\setminus K$ and $y\in V\setminus C$. Then, $\exists \varepsilon_x,\varepsilon_y>0$ with $U(x,\varepsilon_x)\subseteq U$ and $U(y,\varepsilon_y)\subseteq V$. Thus, $d(x,y) > \varepsilon_x + \varepsilon_y > 0$, meaning $\text{dist}(C,K) > \varepsilon_x + \varepsilon_y > 0$. This direction of the proof did not require compactness.
      \end{enumerate}
  \end{description}
  \section{Problem 5}%
  Let $V$ be a finite-dimensional normed space. Show that the unit ball $B:= \{v\in V\mid \norm{v} \leq 1\}$ is compact.
  \begin{description}
    \item[Proof:] Having shown that all norms on $V$ are equivalent, we can create a homeomorphism $f: \ell_{2}^{n}\rightarrow V$, where $\text{dim}(V) = n$. Consider $f^{-1}(B_V)$. Since $B_V$ is bounded and closed, its continuous image under $f^{-1}$ is bounded and closed. Thus, $f^{-1}(B_V)$ is compact in $\ell_{2}^{n}$. So, $f(f^{-1}(B_V)) = B_V$ is a continuous image of a compact set, which is compact. Thus, $B_V$ is compact in $V$.
  \end{description}
  \section{Problem 6}%
  Prove that the unit ball in $C([0,1])$ is not compact.
  \begin{description}
    \item[Proof:] We have shown that $B_V$ is compact if and only if $V$ is finite-dimensional. Since $C([0,1])$ is infinite-dimensional, it must be the case that $B_V$ is not compact.
  \end{description}
  \section{Problem 7}%
  Let $V$ be a normed space and let $K,L\subseteq V$ be compact. Show that
  \begin{align*}
    K+L := \{x+y\mid x\in K,y\in L\}
  \end{align*}
  is also compact.
  \begin{description}
    \item[Proof:] We will show that $K+L$ is complete and totally bounded.\\

      Let $(a_n)_n$ be a Cauchy sequence in $K+L$. Then, $a_n = \chi_n + \sigma_n$  for $\chi_n\in K$ and $\sigma_n\in L$, both Cauchy. For large $m,n$, we have
      \begin{align*}
        |a_m - a_n| &= |(\chi_m + \sigma_m) - (\chi_m + \sigma_n)|\\
                    &\leq |\chi_m - \chi_n| + |\sigma_m - \sigma_n|\\
                    &< \varepsilon,
      \end{align*}
      and since $(\chi_n)_n \rightarrow \chi\in K$ and $(\sigma_n)_n\rightarrow \sigma \in L$, it must be the case that $(a_n)_n \rightarrow \chi + \sigma \in K+L$. Therefore, $K+L$ is complete.\\

      Let $\varepsilon > 0$. Since $K$ is totally bounded, $\exists x_1,\dots,x_n\in K$ such that $K\subseteq \bigcup_{i=1}^{n} U(x_i,\varepsilon)$. Similarly, since $L$ is totally bounded, $\exists y_1,\dots,y_m\in L$ such that $L\subseteq \bigcup_{j=1}^{m}U(y_i,\varepsilon)$.\\

      Let $x\in K+L$. Then, $x = x_K + y_L$ for $x_K\in K$ and $y_L\in L$. We have that $d(x_K,x_i) < \varepsilon$ for some $x_i$, and $d(y_L,y_j) < \varepsilon$ for some $y_j$.
  \end{description}
  \section{Problem 8}%
  Let $(f_n: [0,1]\rightarrow \R)_{n\geq 1}$ be a sequence of differentiable functions with $\sup\norm{f_n}_u < \infty$ and $\sup\norm{f'_n}_u < \infty$. Show that there is a subsequence $\left(f_{n_k}\right)_k$ that converges uniformly to a continuous function $f: [0,1]\rightarrow \R$.
  \begin{description}
    \item[Proof:] Let $(f_n)_n$ be the sequence defined as above.\\

      Let $K = \sup_{n\geq 1}\norm{f'_n}_u$. By the Mean Value Theorem, for all $x,y\in [0,1]$, we have that $|f_n(x)-f_n(y)| \leq K|x-y|$. Letting $\delta = \frac{\varepsilon}{2K}$, we have that $(f_n)_n$ is an equicontinuous family of functions.\\

      Since $\sup_{n\geq 1}\norm{f_n}_u < \infty$, the family $(f_n)_n$ is also bounded.\\

      By Arzelà-Ascoli, $\exists n_k$ such that $\left(f_{n_k}\right)_k \rightarrow f$ uniformly.
  \end{description}
  \section{Problem 9}%
  Let $(X_n,d_n)_n$ be a sequence of compact metric spaces. Show that the product $\prod X_n$ with the product metric is also compact.
  \begin{description}
    \item[Proof:] Let $(X_n,d_n)$ be a sequence of compact metric spaces with the distance between $x=(x_k)_k,y=(y_k)_k\in \prod X_n$ defined by $\sum_{k=1}^{\infty}2^{-k}d_{k}(x_k,y_k)$.
  \end{description}
  \section{Problem 10}%
  Let $(X,d)$ be a compact metric space and let $\mathcal{V}$ be an open cover of $X$. Show that there is a number $L(\mathcal{V})$ satisfying that given any nonempty $E\subseteq X$ with $\text{diam}(E) < L(\mathcal{V})$, there exists $V\in \mathcal{V}$ with $E\subseteq V$.
\end{document}
