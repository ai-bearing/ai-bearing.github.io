\documentclass[10pt]{extarticle}
\title{}
\author{}
\date{}
\usepackage[shortlabels]{enumitem}


%paper setup
\usepackage{geometry}
\geometry{letterpaper, portrait, margin=1in}
\usepackage{fancyhdr}
% sans serif font:
\usepackage{cmbright}
%symbols
\usepackage{amsmath}
\usepackage{bigints}
\usepackage{amssymb}
\usepackage{amsthm}
\usepackage{mathtools}
\usepackage{bbold}
\usepackage[hidelinks]{hyperref}
\usepackage{gensymb}
\usepackage{multirow,array}
\usepackage{multicol}

\newtheorem*{remark}{Remark}
\usepackage[T1]{fontenc}
\usepackage[utf8]{inputenc}

%chemistry stuff
%\usepackage[version=4]{mhchem}
%\usepackage{chemfig}

%plotting
\usepackage{pgfplots}
\usepackage{tikz}
\usetikzlibrary{cd}
\tikzset{middleweight/.style={pos = 0.5}}
%\tikzset{weight/.style={pos = 0.5, fill = white}}
%\tikzset{lateweight/.style={pos = 0.75, fill = white}}
%\tikzset{earlyweight/.style={pos = 0.25, fill=white}}

%\usepackage{natbib}

%graphics stuff
\usepackage{graphicx}
\graphicspath{ {./images/} }
%\usepackage[style=numeric, backend=biber]{biblatex} % Use the numeric style for Vancouver
%\addbibresource{the_bibliography.bib}
%code stuff
%when using minted, make sure to add the -shell-escape flag
%you can use lstlisting if you don't want to use minted
%\usepackage{minted}
%\usemintedstyle{pastie}
%\newminted[javacode]{java}{frame=lines,framesep=2mm,linenos=true,fontsize=\footnotesize,tabsize=3,autogobble,}
%\newminted[cppcode]{cpp}{frame=lines,framesep=2mm,linenos=true,fontsize=\footnotesize,tabsize=3,autogobble,}

%\usepackage{listings}
%\usepackage{color}
%\definecolor{dkgreen}{rgb}{0,0.6,0}
%\definecolor{gray}{rgb}{0.5,0.5,0.5}
%\definecolor{mauve}{rgb}{0.58,0,0.82}
%
%\lstset{frame=tb,
%	language=Java,
%	aboveskip=3mm,
%	belowskip=3mm,
%	showstringspaces=false,
%	columns=flexible,
%	basicstyle={\small\ttfamily},
%	numbers=none,
%	numberstyle=\tiny\color{gray},
%	keywordstyle=\color{blue},
%	commentstyle=\color{dkgreen},
%	stringstyle=\color{mauve},
%	breaklines=true,
%	breakatwhitespace=true,
%	tabsize=3
%}
% text + color boxes
%\renewcommand{\mathbf}[1]{\mathbb{#1}}
%\usepackage[most]{tcolorbox}
%\tcbuselibrary{breakable}
%\tcbuselibrary{skins}
%\newtcolorbox{problem}[1]{colback=white,enhanced,title={\small #1},
%          attach boxed title to top center=
%{yshift=-\tcboxedtitleheight/2},
%boxed title style={size=small,colback=black!60!white}, sharp corners, breakable}
%including PDFs
%\usepackage{pdfpages}
\setlength{\parindent}{0pt}
\usepackage{cancel}
\pagestyle{fancy}
\fancyhf{}
\rhead{Avinash Iyer}
\lhead{Real Analysis II: Problem Set 8}
\newcommand{\card}{\text{card}}
\newcommand{\ran}{\text{ran}}
\newcommand{\N}{\mathbb{N}}
\newcommand{\Q}{\mathbb{Q}}
\newcommand{\Z}{\mathbb{Z}}
\newcommand{\R}{\mathbb{R}}
\newcommand{\C}{\mathbb{C}}
\newcommand{\iprod}[2]{\left\langle #1,#2\right\rangle}
\newcommand{\norm}[1]{\left\Vert #1\right\Vert}
\setcounter{secnumdepth}{0}
\begin{document}
  \section{Problem 1}%
  Let $\mathcal{A}\subseteq \mathcal{P}(\Omega)$ be a family of subsets satisfying
  \begin{enumerate}[(i)]
    \item if $A\in \mathcal{A}$, then $A^{c}\in \mathcal{A}$;
    \item If $\{A_k\}_{k\geq 1}$ is a countable family of pairwise disjoint members of $\mathcal{A}$, then $\bigsqcup_{k\geq 1}A_k \in \mathcal{A}$.
  \end{enumerate}
  Prove that $\mathcal{A}$ is a $\sigma$-algebra on $\Omega$.
  \begin{description}
    \item[Proof:] We will show that if $\bigsqcup_{k\geq 1}A_k\in \mathcal{A}$ for $\{A_k\}_{k\geq 1}$ pairwise disjoint, then $\bigcup_{n\geq 1}B_n\in \mathcal{A}$ for $\{B_n\}_{n\geq 1}$ any family of elements of $\mathcal{A}$. Without loss of generality, let $\bigsqcup A_k \supseteq \bigcup B_n$.\\

      Define $B^{\ast}_i = \left(\bigcup_{n\geq 1}B_n\right)\cap A_i$. Then, the $B^{\ast}_i$ are pairwise disjoint, meaning $\bigsqcup_{n\geq 1}B^{\ast}_n \in \mathcal{A}$. Notice that
      \begin{align*}
        \bigsqcup_{i\geq 1} B^{\ast}_i &= \bigcup_{n\geq 1} B_n.
      \end{align*}
      Thus, $\bigcup B_n\in \mathcal{A}$.
  \end{description}
  \section{Problem 2}%
  Consider the family $\mathcal{E}: \{(-\infty,b)\mid b\in\R\}$. Show that $\sigma(\mathcal{E}) = \mathcal{B}_{\R}$.
  \begin{description}
    \item[Proof:] Consider the family $\mathcal{E}' := \{[a,b)\mid a,b\in\R\}$. We have established that $\sigma(\mathcal{E}') = \mathcal{B}_{\R}$.\\

      We see that for any element of $\mathcal{E}$, $(-\infty,b) = \bigcup_{n=1}^{\infty}[a-n,b)$, meaning $\mathcal{E}\in \sigma(\mathcal{E}')$, so $\sigma\left(\mathcal{E}\right)\subseteq \sigma(\mathcal{E}') = \mathcal{B}_{\R}$.\\

      Additionally, $[a,b) = (-\infty,b)\setminus (-\infty,a)$, meaning $\mathcal{E}' \in \sigma(\mathcal{E})$, so $\sigma(\mathcal{E}')\subseteq \sigma(\mathcal{E})$, so $\sigma\left(\mathcal{E}\right) = \sigma(\mathcal{E}') = \mathcal{B}_{\R}$.
  \end{description}
  \section{Problem 3}%
  Let $(\Omega,\mathcal{M})$ and $(\Lambda,\mathcal{N})$ be measurable spaces. We define the product $\sigma$-algebra on $\Omega \times \Lambda$ as
  \begin{align*}
    \mathcal{M}\otimes \mathcal{N} &:= \sigma\left(\left\{E\times F\mid E\in\mathcal{M},F\in\mathcal{N}\right\}\right).
  \end{align*}
  Prove that $\mathcal{B}_{\R}\otimes \mathcal{B}_{\R} = \mathcal{B}_{\R^2}$.
  \begin{description}
    \item[Proof:] For $a < b$ and $c < d$, it is the case that $(a,b)\times (c,d)\subseteq \R^2$ is open, meaning
      \begin{align*}
        \sigma\left(\{(a,b)\times (c,d)\mid a,b,c,d\in \R\}\right) &= \mathcal{B}_{\R}\otimes \mathcal{B}_{\R}\\
                                                                               &\subseteq \mathcal{B}_{\R^2}.
      \end{align*}
      Letting $U \in \mathcal{B}_{\R^2}$, it is the case that $U = \bigcup_{j=1}^{\infty}U(x_j,r_j)$. For each $U(x_j,r_j)$, take $I_j = (x_{jx}-r_j,x_{jx}+r_j)\times (x_{jy}-r_j,x_{jy}+r_j)$, so $U\subseteq \bigcup_{j=1}^{\infty}I_j$. Thus, $U\in \mathcal{B}_{\R}\otimes \mathcal{B}_{\R}$, so $\mathcal{B}_{\R^2}\subseteq \mathcal{B}_{\R}\otimes \mathcal{B}_{\R}$.
  \end{description}
  \section{Problem 4}%
  Let $(\Omega,\mathcal{M})$ and $(\Lambda,\mathcal{N})$ be measurable spaces. A map $f: \Omega \rightarrow \Lambda$ is $\mathcal{M}$-$\mathcal{N}$-measurable if $E\in\mathcal{N}\Rightarrow f^{-1}(E)\in \mathcal{M}$.\\

  Let $(\Omega,\mathcal{M})$ be a measurable space and suppose $E\in\mathcal{M}$. Show that $\mathcal{M}_{E} = \{M\cap E\mid M\in\mathcal{M}\}$ is a $\sigma$-algebra on $E$ and the inclusion map $\iota: E\rightarrow \Omega$ is $\mathcal{M}_{E}$-$\mathcal{M}$-measurable.
  \begin{description}
    \item[Proof:] Let $M\in \mathcal{M}$. Then, $\iota^{-1}(M) = E\cap M \in \mathcal{M}_{E}$. Thus, $f$ is $\mathcal{M}_{E}$-$\mathcal{M}$-measurable.
  \end{description}
  \section{Problem 5}%
  Let $(\Omega,\mathcal{M})$ and $(\Lambda,\mathcal{N})$ be measurable spaces. Suppose $\mathcal{N}$ is generated as a $\sigma$-algebra by a family of subsets $\mathcal{E}\subseteq \mathcal{P}(\Lambda)$. Prove that a map $f:\Omega \rightarrow \Lambda$ is $\mathcal{M}$-$\mathcal{N}$-measurable if and only if $f^{-1}(E)\in \mathcal{M}$ for all $E\in \mathcal{E}$. Conclude that a continuous function $f: X\rightarrow Y$ between metric spaces is $\mathcal{B}_{X}$-$\mathcal{B}_{Y}$-measurable.
  \begin{description}
    \item[Proof:] Let $\mathcal{N}$ be generated by $\mathcal{E}$. Then, for any $E_1,E_2\in \mathcal{E}$, it is the case that $E_1^{c}\in \mathcal{N}$ or $E_1\cup E_2\in \mathcal{N}$.\\

      Let $f$ be measurable. Then, since $\mathcal{E}\subseteq \mathcal{N}$, and for any $E\in \mathcal{N}$, $f^{-1}(E)\in \mathcal{M}$, it is the case that for any $E\in \mathcal{E}$, $f^{-1}(E)\in \mathcal{M}$.\\

      Let $f$ be a function such that for any $E\in \mathcal{E}$, $f^{-1}(E)\in \mathcal{M}$. So, $f^{-1}(E^{c}) = \left(f^{-1}(E)\right)^c \in \mathcal{M}$, and $f^{-1}(E_1\cup E_2) = f^{-1}(E_1)\cup f^{-1}(E_2) \in \mathcal{M}$. Therefore, for any $E\in \mathcal{N}$, it must be the case that $f^{-1}(E)\in \mathcal{M}$.\\

      Since the preimage of any element of the topology on $Y$ is the topology on $X$ if $f$ is continuous, it is the case that such a continuous function is $\mathcal{B}_{X}$-$\mathcal{B}_{Y}$-measurable.
  \end{description}
  \section{Problem 6}%
  Suppose $(\Omega,\mathcal{M})$ is a measurable space and $f: \Omega \rightarrow \Lambda$ is a map. Show that $\mathcal{N}:= \{E\subseteq \Lambda\mid f^{-1}(E)\in \mathcal{M}\}$ is a $\sigma$-algebra on $\Lambda$ and $f$ is $\mathcal{M}$-$\mathcal{N}$-measurable. $\mathcal{N}$ is called the $\sigma$-algebra produced by $f$.
  \begin{description}
    \item[Proof:] Let $E\in \mathcal{N}$. Then, $\left(f^{-1}(E)\right)^{c} \in \mathcal{M}$ (since $f$ is $\mathcal{M}$-$\mathcal{N}$-measurable), meaning $f^{-1}(E^{c})\in \mathcal{M}$, so $E^{c}\in \mathcal{N}$.\\

      Let $E_1,E_2\in \mathcal{N}$. Then, $f^{-1}(E_1)\cup f^{-1}(E_2) \in \mathcal{M}$, so$ f^{-1}(E_1\cup E_2)\in \mathcal{M}$, so $E_1\cup E_2\in \mathcal{N}$.\\

      Since $\mathcal{M}$ is a $\sigma$-algebra, the above holds for countable unions, meaning $\mathcal{N}$ is a $\sigma$-algebra.
  \end{description}
  \section{Problem 7}%
  Let $(\Omega,\mathcal{M},\mu)$ be a measure space, and suppose $\{E_k\}_{k\geq 1}$ is a decreasing sequence of measurable sets with $\mu(E_1) < \infty$. Show that
  \begin{align*}
    \mu\left(\bigcap_{k\geq 1}E_k\right) &= \lim_{k\rightarrow\infty}\mu(E_k)\\
                                         &= \inf_{k\geq 1}\mu(E_k).
  \end{align*}
  \begin{description}
    \item[Proof:] We see that for $n$, $\bigcap_{k=1}^{n}E_k = E_n$. Therefore, $\mu\left(\bigcap_{k=1}^{n}E_k\right) = \mu(E_n)$, meaning
      \begin{align*}
        \mu\left(\bigcap_{k=1}^{\infty}E_k\right) &= \lim_{n\rightarrow\infty}\mu\left(\bigcap_{k=1}^{n}E_k\right)\\
                                                  &= \lim_{n\rightarrow\infty}\mu(E_n).
      \end{align*}
  \end{description}
  \section{Problem 8}%
  Let $(\Omega,\mathcal{M})$ and $(\Lambda,\mathcal{N})$ be measurable spaces and suppose $f: \Omega \rightarrow \Lambda$ is measurable. If $\mu$ is a measure on $\mathcal{M}$, show that
  \begin{align*}
    f_{\ast}\mu: \mathcal{N}\rightarrow [0,\infty];~~f_{\ast}\mu(E) := \mu(f^{-1}(E))
  \end{align*}
  defines a measure on $(\Lambda,\mathcal{N})$. This is called the pushforward measure.
  \begin{description}
    \item[Proof:] Clearly, $f_{\ast}\mu(\emptyset) = 0$. Let $E_1,E_2\in \mathcal{N}$ be disjoint and nonempty. Note that $E_1 \sqcup E_2\in \mathcal{N}$. Thus,
      \begin{align*}
        f_{\ast}\mu(E_1\sqcup E_2) &= \mu\left(f^{-1}(E_1\sqcup E_2)\right)\\
                                   &= \mu\left(f^{-1}(E_1)\sqcup f^{-1}(E_2)\right)\\
                                   &= \mu(f^{-1}(E_1)) + \mu(f^{-1}(E_2))\\
                                   &= f_{\ast}\mu(E_1) + f_{\ast}\mu(E_2),
      \end{align*}
      meaning $f_{\ast}\mu$ is a measure on $(\Lambda,\mathcal{N})$.
  \end{description}
  \section{Problem 9}%
  A group $G$ is paradoxical if there are pairwise disjoint subsets of $G$; $E_1,\dots,E_n,F_1,\dots,F_m$ and group elements $t_1,\dots,t_n,s_1,\dots,s_m$ such that
  \begin{align*}
    G &= \bigsqcup_{j=1}^{n}t_jE_j\\
      &= \bigsqcup_{k=1}^{m}s_kF_k.
  \end{align*}
  A mean on a group $G$ is a finitely additive probability measure $\nu: \mathcal{P}(G)\rightarrow [0,1]$ that is translation invariant; that is, $\nu(tE) = \nu(E)$ for all $E\subseteq G$ and $t\in G$. A group is said to be amenable if it admits a mean.\\

  Show that a paradoxical group is nonamenable.
  \begin{description}
    \item[Proof:] Let $G$ be paradoxical. Suppose toward contradiction that there existed such a $\nu$. Then, $\nu(G)$, and
      \begin{align*}
        \nu(G) &= \nu\left(\bigsqcup_{j=1}^{n}t_jE_j\right)\\
               &= \sum_{j=1}^{n}\nu(t_jE_j)\\
               &= \sum_{j=1}^{n}\nu(E_j).
               \intertext{We know that $G \cup s_1F_1 = G$, meaning $\nu(G) = \nu(G \cup s_1F_1)$. However,}
        \nu(G \cup s_1F_1) &= \nu\left(\bigsqcup_{j=1}^{n}t_jE_j \sqcup s_1F_1\right)\\
                           &= \sum_{j=1}^{n}\nu(t_jE_j) + \nu(s_1F_1)\\
                           &= \nu(G) + \nu(s_1F_1)\\
                           &= \nu(G) + \nu(F_1)\\
                           &> \nu(G).
      \end{align*}
  \end{description}
  \section{Problem 10}%
  Let $\Delta$ be a totally disconnected compact metric space (for example, the Cantor set). Suppose $\varphi: C(\Delta)\rightarrow \R$ is a state --- $\varphi$ is linear, continuous, positive, and $\varphi\left(\mathbb{1}_{\Delta}\right) = 1$.
  \begin{enumerate}[(i)]
    \item Show that $\mathcal{C} := \{E\mid E\subseteq \Delta\}$ is an algebra of subsets on $\Delta$.
    \item Show that
      \begin{align*}
        \mu_0: \mathcal{C}\rightarrow [0,1];~~\mu_0(E) = \varphi\left(\mathbb{1}_{E}\right)
      \end{align*}
      is a well-defined finitely additive measure.
    \item If $\{E_k\}_{k\geq 1}$ is a countable family of members of $\mathcal{C}$ such that $\bigsqcup_{k\geq 1}E_k \in \mathcal{C}$, show that
      \begin{align*}
        \mu_0\left(\bigsqcup_{k\geq 1}E_k\right) &= \sum_{k=1}^{\infty}\mu_0(E_k).
      \end{align*}
  \end{enumerate}
  \begin{description}
    \item[Proof:]\hfill
      \begin{enumerate}[(i)]
        \item If $E\in \mathcal{C}$, then $E\subseteq \Delta$, so $E^{c}\subseteq \Delta$, and for $E_1,E_2\in \mathcal{C}$, $E_1\cup E_2\in \Delta$.
        \item Let $E,F\in \mathcal{C}$ with $E \cap F = \emptyset$. Then,
          \begin{align*}
            \mu_0(E\sqcup F) &= \varphi\left(\mathbb{1}_{E\sqcup F}\right)\\
                           &= \varphi\left(\mathbb{1}_{E} + \mathbb{1}_{F}\right)\\
                           &= \varphi\left(\mathbb{1}_{E}\right) + \varphi\left(\mathbb{1}_{F}\right)\\
                           &= \mu_0(E) + \mu_0(F).
          \end{align*}
        \item Let $\{E_k\}_{k\geq 1}$ be a countable family of members of $\mathcal{C}$ with $\displaystyle \bigsqcup_{k\geq 1}E_k\in \mathcal{C}$. We see that for any $n\in N$, $\displaystyle \bigsqcup_{k=1}^{n}E_k\in \mathcal{C}$, since $\mathcal{C}$ is an algebra of subsets.\\

          Therefore, 
          \begin{align*}
            \mu_0\left(\bigsqcup_{k=1}^{n}E_k\right) &= \sum_{k=1}^{n}\mu_0(E_k),
            \intertext{for any $n\in \N$, as $\mu_0$ is finitely additive. Since $\displaystyle \bigsqcup_{k\geq 1}E_k \in \mathcal{C}$, it is then the case that}
            \mu_0\left(\bigsqcup_{k=1}^{\infty}E_k\right) &= \lim_{n\rightarrow\infty}\mu_0\left(\bigsqcup_{k=1}^{n}E_k\right)\\
                                                       &= \lim_{n\rightarrow\infty}\sum_{k=1}^{n}\mu_0(E_k)\\
                                                       &= \sum_{k=1}^{\infty}\mu_0(E_k).
          \end{align*}
      \end{enumerate}
  \end{description}
\end{document}
