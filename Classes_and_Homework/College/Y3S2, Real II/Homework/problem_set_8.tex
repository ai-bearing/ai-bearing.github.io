\documentclass[10pt]{extarticle}
\title{}
\author{}
\date{}
\usepackage[shortlabels]{enumitem}


%paper setup
\usepackage{geometry}
\geometry{letterpaper, portrait, margin=1in}
\usepackage{fancyhdr}
% sans serif font:
\usepackage{cmbright}
%symbols
\usepackage{amsmath}
\usepackage{bigints}
\usepackage{amssymb}
\usepackage{amsthm}
\usepackage{mathtools}
\usepackage{bbold}
\usepackage[hidelinks]{hyperref}
\usepackage{gensymb}
\usepackage{multirow,array}
\usepackage{multicol}

\newtheorem*{remark}{Remark}
\usepackage[T1]{fontenc}
\usepackage[utf8]{inputenc}

%chemistry stuff
%\usepackage[version=4]{mhchem}
%\usepackage{chemfig}

%plotting
\usepackage{pgfplots}
\usepackage{tikz}
\usetikzlibrary{cd}
\tikzset{middleweight/.style={pos = 0.5}}
%\tikzset{weight/.style={pos = 0.5, fill = white}}
%\tikzset{lateweight/.style={pos = 0.75, fill = white}}
%\tikzset{earlyweight/.style={pos = 0.25, fill=white}}

%\usepackage{natbib}

%graphics stuff
\usepackage{graphicx}
\graphicspath{ {./images/} }
%\usepackage[style=numeric, backend=biber]{biblatex} % Use the numeric style for Vancouver
%\addbibresource{the_bibliography.bib}
%code stuff
%when using minted, make sure to add the -shell-escape flag
%you can use lstlisting if you don't want to use minted
%\usepackage{minted}
%\usemintedstyle{pastie}
%\newminted[javacode]{java}{frame=lines,framesep=2mm,linenos=true,fontsize=\footnotesize,tabsize=3,autogobble,}
%\newminted[cppcode]{cpp}{frame=lines,framesep=2mm,linenos=true,fontsize=\footnotesize,tabsize=3,autogobble,}

%\usepackage{listings}
%\usepackage{color}
%\definecolor{dkgreen}{rgb}{0,0.6,0}
%\definecolor{gray}{rgb}{0.5,0.5,0.5}
%\definecolor{mauve}{rgb}{0.58,0,0.82}
%
%\lstset{frame=tb,
%	language=Java,
%	aboveskip=3mm,
%	belowskip=3mm,
%	showstringspaces=false,
%	columns=flexible,
%	basicstyle={\small\ttfamily},
%	numbers=none,
%	numberstyle=\tiny\color{gray},
%	keywordstyle=\color{blue},
%	commentstyle=\color{dkgreen},
%	stringstyle=\color{mauve},
%	breaklines=true,
%	breakatwhitespace=true,
%	tabsize=3
%}
% text + color boxes
%\renewcommand{\mathbf}[1]{\mathbb{#1}}
%\usepackage[most]{tcolorbox}
%\tcbuselibrary{breakable}
%\tcbuselibrary{skins}
%\newtcolorbox{problem}[1]{colback=white,enhanced,title={\small #1},
%          attach boxed title to top center=
%{yshift=-\tcboxedtitleheight/2},
%boxed title style={size=small,colback=black!60!white}, sharp corners, breakable}
%including PDFs
%\usepackage{pdfpages}
\setlength{\parindent}{0pt}
\usepackage{cancel}
\pagestyle{fancy}
\fancyhf{}
\rhead{Avinash Iyer}
\lhead{Real Analysis II: Problem Set 8}
\newcommand{\card}{\text{card}}
\newcommand{\ran}{\text{ran}}
\newcommand{\N}{\mathbb{N}}
\newcommand{\Q}{\mathbb{Q}}
\newcommand{\Z}{\mathbb{Z}}
\newcommand{\R}{\mathbb{R}}
\newcommand{\C}{\mathbb{C}}
\newcommand{\iprod}[2]{\left\langle #1,#2\right\rangle}
\newcommand{\norm}[1]{\left\Vert #1\right\Vert}
\setcounter{secnumdepth}{0}
\begin{document}
  \section{Problem 1}%
  Let $\mathcal{A}\subseteq \mathcal{P}(\Omega)$ be a family of subsets satisfying
  \begin{enumerate}[(i)]
    \item if $A\in \mathcal{A}$, then $A^{c}\in \mathcal{A}$;
    \item If $\{A_k\}_{k\geq 1}$ is a countable family of pairwise disjoint members of $\mathcal{A}$, then $\bigsqcup_{k\geq 1}A_k \in \mathcal{A}$.
  \end{enumerate}
  Prove that $\mathcal{A}$ is a $\sigma$-algebra on $\Omega$.
  \section{Problem 2}%
  Consider the family $\mathcal{E}: \{(-\infty,b)\mid b\in\R\}$. Show that $\sigma(\mathcal{E}) = \mathcal{B}_{\R}$.
  \begin{description}
    \item[Proof:] Consider the family $\mathcal{E}' := \{[a,b)\mid a,b\in\R\}$. We have established that $\sigma(\mathcal{E}') = \mathcal{B}_{\R}$.\\

      We see that for any element of $\mathcal{E}$, $(-\infty,b) = \bigcup_{n=1}^{\infty}[a-n,b)$, meaning $\mathcal{E}\in \sigma(\mathcal{E}')$, so $\sigma\left(\mathcal{E}\right)\subseteq \sigma(\mathcal{E}') = \mathcal{B}_{\R}$.\\

      Additionally, $[a,b) = (-\infty,b)\setminus (-\infty,a)$, meaning $\mathcal{E}' \in \sigma(\mathcal{E})$, so $\sigma(\mathcal{E}')\subseteq \sigma(\mathcal{E})$, so $\sigma\left(\mathcal{E}\right) = \sigma(\mathcal{E}') = \mathcal{B}_{\R}$.
  \end{description}
  \section{Problem 3}%
  Let $(\Omega,\mathcal{M})$ and $(\Lambda,\mathcal{N})$ be measurable spaces. We define the product $\sigma$-algebra on $\Omega \times \Lambda$ as
  \begin{align*}
    \mathcal{M}\otimes \mathcal{N} &:= \sigma\left(\left\{E\times F\mid E\in\mathcal{M},F\in\mathcal{N}\right\}\right).
  \end{align*}
  Prove that $\mathcal{B}_{\R}\otimes \mathcal{B}_{\R} = \mathcal{B}_{\R^2}$.
  \section{Problem 4}%
  Let $(\Omega,\mathcal{M})$ and $(\Lambda,\mathcal{N})$ be measurable spaces. A map $f: \Omega \rightarrow \Lambda$ is $\mathcal{M}$-$\mathcal{N}$-measurable if $E\in\mathcal{N}\Rightarrow f^{-1}(E)\in \mathcal{M}$.\\

  Let $(\Omega,\mathcal{M})$ be a measurable space and suppose $E\in\mathcal{M}$. Show that $\mathcal{M}_{E} = \{M\cap E\mid M\in\mathcal{M}\}$ is a $\sigma$-algebra on $E$ and the inclusion map $\iota: E\rightarrow \Omega$ is $\mathcal{M}_{E}$-$\mathcal{M}$-measurable.
  \section{Problem 5}%
  Let $(\Omega,\mathcal{M})$ and $(\Lambda,\mathcal{N})$ be measurable spaces. Suppose $\mathcal{N}$ is generated as a $\sigma$-algebra by a family of subsets $\mathcal{E}\subseteq \mathcal{P}(\Lambda)$. Prove that a map $f:\Omega \rightarrow \Lambda$ is $\mathcal{M}$-$\mathcal{N}$-measurable if and only if $f^{-1}(E)\in \mathcal{M}$ for all $E\in \mathcal{E}$. Conclude that a continuous function $f: X\rightarrow Y$ between metric spaces is $\mathcal{B}_{X}$-$\mathcal{B}_{Y}$-measurable.
  \section{Problem 6}%
  Suppose $(\Omega,\mathcal{M})$ is a measurable space and $f: \Omega \rightarrow \Lambda$ is a map. Show that $\mathcal{N}:= \{E\subseteq \Lambda\mid f^{-1}(E)\in \mathcal{M}\}$ is a $\sigma$-algebra on $\Lambda$ and $f$ is $\mathcal{M}$-$\mathcal{N}$-measurable. $\mathcal{N}$ is called the $\sigma$-algebra produced by $f$.
  \section{Problem 7}%
  Let $(\Omega,\mathcal{M},\mu)$ be a measure space, and suppose $\{E_k\}_{k\geq 1}$ is a decreasing sequence of measurable sets with $\mu(E_1) < \infty$. Show that
  \begin{align*}
    \mu\left(\bigcap_{k\geq 1}E_k\right) &= \lim_{k\rightarrow\infty}\mu(E_k)\\
                                         &= \inf_{k\geq 1}\mu(E_k).
  \end{align*}
  \section{Problem 8}%
  Let $(\Omega,\mathcal{M})$ and $(\Lambda,\mathcal{N})$ be measurable spaces and suppose $f: \Omega \rightarrow \Lambda$ is measurable. If $\mu$ is a measure on $\mathcal{M}$, show that
  \begin{align*}
    f\ast \mu: \mathcal{N}\rightarrow [0,\infty];~~f\ast \mu(E) := \mu(f^{-1}(E))
  \end{align*}
  defines a measure on $(\Lambda,\mathcal{N})$. This is called the push-forward measure.
  \section{Problem 9}%
  A group $G$ is paradoxical if there are pairwise disjoint subsets of $G$; $E_1,\dots,E_n,F_1,\dots,F_m$ and group elements $t_1,\dots,t_n,s_1,\dots,s_m$ such that
  \begin{align*}
    G &= \bigsqcup_{j=1}^{n}t_jE_j\\
      &= \bigsqcup_{k=1}^{m}s_kF_k.
  \end{align*}
  A mean on a group $G$ is a finitely additive probability measure $\nu: \mathcal{P}(G)\rightarrow [0,1]$ that is translation invariant; that is, $\nu(tE) = \nu(E)$ for all $E\subseteq G$ and $t\in G$. A group is said to be amenable if it admits a mean.\\

  Show that a paradoxical group is nonamenable.
  \section{Problem 10}%
  Let $\Delta$ be a totally disconnected compact metric space (for example, the cantor set). Suppose $\varphi: C(\Delta)\rightarrow \R$ is a state --- $\varphi$ is linear, continuous, positive, and $\varphi\left(\mathbb{1}_{\Delta}\right) = 1$.
  \begin{enumerate}[(i)]
    \item Show that $\mathcal{C} := \{E\mid E\subseteq \Delta\}$ is an algebra of subsets on $\Delta$.
    \item Show that
      \begin{align*}
        \mu_0: \mathcal{C}\rightarrow [0,1];~~\mu_0(E) = \varphi\left(\mathbb{1}_{E}\right)
      \end{align*}
      is a well-defined finitely additive measure.
    \item If $\{E_k\}_{k\geq 1}$ is a countable family of members of $\mathcal{C}$ such that $\bigsqcup_{k\geq 1}E_k \in \mathcal{C}$, show that
      \begin{align*}
        \mu_0\left(\bigsqcup_{k\geq 1}E_k\right) &= \sum_{k=1}^{\infty}\mu_0(E_k).
      \end{align*}
  \end{enumerate}
\end{document}
