\documentclass[10pt]{extarticle}
\title{}
\author{}
\date{}
\usepackage[shortlabels]{enumitem}


%paper setup
\usepackage{geometry}
\geometry{letterpaper, portrait, margin=1in}
\usepackage{fancyhdr}
% sans serif font:
\usepackage{cmbright}
%symbols
\usepackage{amsmath}
\usepackage{bigints}
\usepackage{amssymb}
\usepackage{amsthm}
\usepackage{mathtools}
\usepackage{bbold}
\usepackage[hidelinks]{hyperref}
\usepackage{gensymb}
\usepackage{multirow,array}
\usepackage{multicol}

\newtheorem*{remark}{Remark}
\usepackage[T1]{fontenc}
\usepackage[utf8]{inputenc}

%chemistry stuff
%\usepackage[version=4]{mhchem}
%\usepackage{chemfig}

%plotting
\usepackage{pgfplots}
\usepackage{tikz}
\usetikzlibrary{cd}
\tikzset{middleweight/.style={pos = 0.5}}
%\tikzset{weight/.style={pos = 0.5, fill = white}}
%\tikzset{lateweight/.style={pos = 0.75, fill = white}}
%\tikzset{earlyweight/.style={pos = 0.25, fill=white}}

%\usepackage{natbib}

%graphics stuff
\usepackage{graphicx}
\graphicspath{ {./images/} }
%\usepackage[style=numeric, backend=biber]{biblatex} % Use the numeric style for Vancouver
%\addbibresource{the_bibliography.bib}
%code stuff
%when using minted, make sure to add the -shell-escape flag
%you can use lstlisting if you don't want to use minted
%\usepackage{minted}
%\usemintedstyle{pastie}
%\newminted[javacode]{java}{frame=lines,framesep=2mm,linenos=true,fontsize=\footnotesize,tabsize=3,autogobble,}
%\newminted[cppcode]{cpp}{frame=lines,framesep=2mm,linenos=true,fontsize=\footnotesize,tabsize=3,autogobble,}

%\usepackage{listings}
%\usepackage{color}
%\definecolor{dkgreen}{rgb}{0,0.6,0}
%\definecolor{gray}{rgb}{0.5,0.5,0.5}
%\definecolor{mauve}{rgb}{0.58,0,0.82}
%
%\lstset{frame=tb,
%	language=Java,
%	aboveskip=3mm,
%	belowskip=3mm,
%	showstringspaces=false,
%	columns=flexible,
%	basicstyle={\small\ttfamily},
%	numbers=none,
%	numberstyle=\tiny\color{gray},
%	keywordstyle=\color{blue},
%	commentstyle=\color{dkgreen},
%	stringstyle=\color{mauve},
%	breaklines=true,
%	breakatwhitespace=true,
%	tabsize=3
%}
% text + color boxes
%\renewcommand{\mathbf}[1]{\mathbb{#1}}
%\usepackage[most]{tcolorbox}
%\tcbuselibrary{breakable}
%\tcbuselibrary{skins}
%\newtcolorbox{problem}[1]{colback=white,enhanced,title={\small #1},
%          attach boxed title to top center=
%{yshift=-\tcboxedtitleheight/2},
%boxed title style={size=small,colback=black!60!white}, sharp corners, breakable}
%including PDFs
%\usepackage{pdfpages}
\setlength{\parindent}{0pt}
\usepackage{cancel}
\pagestyle{fancy}
\fancyhf{}
\rhead{Avinash Iyer}
\lhead{Real Analysis II: Final Exam}
\newcommand{\card}{\text{card}}
\newcommand{\ran}{\text{ran}}
\newcommand{\N}{\mathbb{N}}
\newcommand{\Q}{\mathbb{Q}}
\newcommand{\Z}{\mathbb{Z}}
\newcommand{\R}{\mathbb{R}}
\newcommand{\C}{\mathbb{C}}
\newcommand{\iprod}[2]{\left\langle #1,#2\right\rangle}
\newcommand{\norm}[1]{\left\Vert #1\right\Vert}
\setcounter{secnumdepth}{0}
\begin{document}
  \section{Problem 1}%
  Fix a measure space $(\Omega,\mathcal{M},\mu)$. If $\phi: \Omega \rightarrow [0,\infty)$ is a simple, positive, measurable function given by
  \begin{align*}
    \phi = \sum_{i=1}^{n}a_i\mathbb{1}_{A_i},~~a_i\geq 0;A_i\in \mathcal{M}
  \end{align*}
  we define
  \begin{align*}
    \int_{\Omega}\phi~d\mu &= \sum_{i=1}^{n}a_i\mu(A_i).
  \end{align*}
  Show that this is well-defined. That is, if there is another expression of $\phi$
  \begin{align*}
    \phi = \sum_{j=1}^{m}b_j\mathbb{1}_{B_j},~~b_j\geq 0;B_j\in \mathcal{M}
  \end{align*}
  then
  \begin{align*}
    \sum_{i=1}^{n}a_i\mu(A_i) &= \sum_{j=1}^{m}b_j\mu(B_j).
  \end{align*}
  \begin{description}
    \item[Proof:] Let $\{F_k\}_{k=1}^{\ell}$ be a refinement of disjoint subsets of $\Omega$ such that $\displaystyle A_i = \bigsqcup_{k\in I_i}F_k$ and $\displaystyle B_j = \bigsqcup_{j\in J_j}F_j$, where $I_i,J_j\subseteq \{1,\dots,\ell\}$.\\

      Let $M_k = \{i\mid F_k\subseteq A_i\}$ and $N_k = \{j\mid F_k\subseteq B_j\}$. Then,
      \begin{align*}
        \sum_{i=1}^{n}a_i\mathbb{1}_{A_i} &= \sum_{k=1}^{\ell}\sum_{i\in M_k}a_i\mathbb{1}_{F_k}\\
                                          &= \sum_{k=1}^{\ell}\sum_{j\in N_k} b_j\mathbb{1}_{F_k},\\
                                          &= \sum_{j=1}^{m}b_j\mathbb{1}_{B_j}\\
                                          \intertext{so}
        \sum_{i=1}^{n}a_i\mu(A_i) &= \sum_{k=1}^{\ell}\sum_{i\in M_k}a_k\mu(F_k)\\
                                  &= \sum_{k=1}^{\ell}\sum_{j\in N_k}b_j\mu(F_k)\\
                                  &= \sum_{j=1}^{m}b_j\mu(B_j).
      \end{align*}
  \end{description}
  \section{Problem 2}%
  Let $\Delta$ be a totally disconnected compact metric space (for example, the Cantor set). Also, suppose $\varphi: C(\Delta)\rightarrow \R$ is a state --- $\varphi$ is linear, continuous, positive $(f \geq 0 \Rightarrow \varphi(f) \geq 0)$, and $\varphi(\mathbb{1}_{\Delta}) = 1$.
  \begin{enumerate}[(i)]
    \item Show that $\mathcal{C} := \{E\mid E\subseteq \Delta~\text{is clopen}\}$ is an algebra of subsets of $\Delta$.
    \item Show that
      \begin{align*}
        \mu_0: \mathcal{C} \rightarrow [0,1];~~\mu_0(E) = \varphi(\mathbb{1}_{E})
      \end{align*}
      is a well-defined finitely additive measure.
    \item Show that $\mu_0$ is a pre-measure on $(\Delta,\mathcal{C})$.
    \item Prove that there is a unique Borel probability measure $\mu$ on $(\Delta,\mathcal{B}_{\Delta})$ such that
      \begin{align*}
        \int_{\Delta}f~d\mu = \varphi(f)~~\forall f\in C(\Delta).
      \end{align*}
  \end{enumerate}
  \begin{description}
    \item[Proof:]\hfill
      \begin{enumerate}[(i)]
        \item Since the complement of any clopen set is clopen, and the finite union of clopen sets is clopen, $\mathcal{C}$ is an algebra of subsets of $\Delta$.
        \item We can see that $\varphi(\mathbb{1}_{\emptyset}) = 0$, meaning $\mu_0(\emptyset) = 0$, and for $E,F\in \mathcal{C}$ disjoint,
          \begin{align*}
            \mu_0(E\sqcup F) &= \varphi(\mathbb{1}_{E\sqcup F})\\
                             &= \varphi(\mathbb{1}_{E} + \mathbb{1}_{F})\\
                             &= \varphi(\mathbb{1}_{E}) + \varphi(\mathbb{1}_{F})\\
                             &= \mu_0(E) + \mu_0(F).
          \end{align*}
        \item Let $\{E_k\}_{k\geq 1}\subseteq \mathcal{C}$ with $\bigsqcup_{k\geq 1}E_k \in \mathcal{C}$. Then,
          \begin{align*}
            \mu_0\left(\bigsqcup_{k\geq 1}E_k\right) &= \varphi\left(\mathbb{1}_{\bigsqcup_{k\geq 1}E_k}\right)\\
                                                     &= \sum_{k = 1}^{\infty}\varphi(\mathbb{1}_{E_k})\\
                                                     &= \sum_{k=1}^{\infty}\mu_0(E_k).
          \end{align*}
          Thus, $\mu_0$ is a pre-measure.
        \item Let $f\in C(\Delta)$. It is known that $\text{span}\left\{\mathbb{1}_{E_k}\mid E_k\subseteq \Delta\text{ clopen}\right\}$ is uniformly dense in $C(\Delta)$. Define
          \begin{align*}
            \varphi(f) &= \sup\left\{\sum_{k=1}^{n}\alpha_k\varphi\left(\mathbb{1}_{E_k}\right)\right\},
          \end{align*}
          where $\displaystyle\sum_{k=1}^{n}\alpha_k\mathbb{1}_{E_k}$ is an approximation of $f$ in $C(\Delta)$.
      \end{enumerate}
  \end{description}
\end{document}
