\documentclass[8pt]{extarticle}
\title{}
\author{}
\date{}
\usepackage[shortlabels]{enumitem}


%paper setup
\usepackage{geometry}
\geometry{letterpaper, portrait, margin=1in}
\usepackage{fancyhdr}
% sans serif font:
\usepackage{cmbright}
%symbols
\usepackage{amsmath}
\usepackage{amssymb}
\usepackage{amsthm}
\usepackage{mathtools}
\usepackage[hidelinks]{hyperref}
\usepackage{gensymb}
\usepackage{multirow,array}
\usepackage{multicol}

\newtheorem*{remark}{Remark}
\usepackage[T1]{fontenc}
\usepackage[utf8]{inputenc}

%chemistry stuff
%\usepackage[version=4]{mhchem}
%\usepackage{chemfig}

%plotting
\usepackage{pgfplots}
\usepackage{tikz}
\tikzset{middleweight/.style={pos = 0.5}}
%\tikzset{weight/.style={pos = 0.5, fill = white}}
%\tikzset{lateweight/.style={pos = 0.75, fill = white}}
%\tikzset{earlyweight/.style={pos = 0.25, fill=white}}

%\usepackage{natbib}

%graphics stuff
\usepackage{graphicx}
\graphicspath{ {./images/} }
\usepackage[style=numeric, backend=biber]{biblatex} % Use the numeric style for Vancouver
\addbibresource{the_bibliography.bib}
%code stuff
%when using minted, make sure to add the -shell-escape flag
%you can use lstlisting if you don't want to use minted
%\usepackage{minted}
%\usemintedstyle{pastie}
%\newminted[javacode]{java}{frame=lines,framesep=2mm,linenos=true,fontsize=\footnotesize,tabsize=3,autogobble,}
%\newminted[cppcode]{cpp}{frame=lines,framesep=2mm,linenos=true,fontsize=\footnotesize,tabsize=3,autogobble,}

%\usepackage{listings}
%\usepackage{color}
%\definecolor{dkgreen}{rgb}{0,0.6,0}
%\definecolor{gray}{rgb}{0.5,0.5,0.5}
%\definecolor{mauve}{rgb}{0.58,0,0.82}
%
%\lstset{frame=tb,
%	language=Java,
%	aboveskip=3mm,
%	belowskip=3mm,
%	showstringspaces=false,
%	columns=flexible,
%	basicstyle={\small\ttfamily},
%	numbers=none,
%	numberstyle=\tiny\color{gray},
%	keywordstyle=\color{blue},
%	commentstyle=\color{dkgreen},
%	stringstyle=\color{mauve},
%	breaklines=true,
%	breakatwhitespace=true,
%	tabsize=3
%}
% text + color boxes
\renewcommand{\mathbf}[1]{\mathbold{#1}}
\usepackage[most]{tcolorbox}
\tcbuselibrary{breakable}
\tcbuselibrary{skins}
\newtcolorbox{problem}[1]{colback=white,enhanced,title={\small #1},
          attach boxed title to top center=
{yshift=-\tcboxedtitleheight/2},
boxed title style={size=small,colback=black!60!white}, sharp corners, breakable}
%including PDFs
%\usepackage{pdfpages}
\setlength{\parindent}{0pt}
\usepackage{cancel}
\pagestyle{fancy}
\fancyhf{}
\rhead{Avinash Iyer}
\lhead{Econ 308: Problem Set 7}
\newcommand{\card}{\text{card}}
\newcommand{\ran}{\text{ran}}
\newcommand{\N}{\mathbb{N}}
\newcommand{\Q}{\mathbb{Q}}
\newcommand{\Z}{\mathbb{Z}}
\newcommand{\R}{\mathbb{R}}
\begin{document}
  \begin{problem}{Elastic Incidence}
    Consider the following model for the soda market. Suppose the aggregate demand for soda is given by $Q^D = 800 - 100P$ where $P$ denotes the price and $Q$ denotes the quantity of bottles of soda demanded. The aggregate supply of soda is given by $Q^S = 300P$.
    \begin{enumerate}[(a)]
      \item Compute the soda market equilibrium. What are the equilibrium price and quantity?
      \item Calculate the price elasticity of supply and the price elasticity of demand at the equilibrium. Compare the values and explain which side you would expect to face a higher incidence if a tax is levied on soda.
      \item Use your price elasticities to compute the marginal effect of a tax $t$ on the producer price $p$.
      \item Now suppose a tax of $t = \$1$ is imposed on each soda that is purchased. Compute the soda market equilibrium with the tax. What are the equilibrium producer price, consumer price, and quantity? Check that the change in producer price aligns with your answer to part (c).
    \end{enumerate}
  \end{problem}
  \begin{problem}{Gas Tax Incidence}
    Taxing gasoline is one potential way to reduce consumption and CO$_2$ emissions. But are gas taxes actually passed onto consumers in the form of higher gas prices? Or do they simply lower the profits of oil companies and not change the gas consumption and consequently emissions? To answer this, Doyle and Sampatharank (2008) leveraged a natural experiment in which IN and IL temporarily suspended their 5\% gas taxes on July $1$, 2000 in response to spiking gas prices.
    \begin{enumerate}[(a)]
      \item Suppose that you compare the difference in gas prices in Indiana and Illinois just before and after the tax repeal in Summer 2000. Explain why this simple difference is unlikely to be a causal estimate of the impact of the tax repeal on gas prices.
      \item Suppose instead that you use a difference-in-differences estimator by comparing the gas price changes in Indiana and Illinois (the treatment group) to the gas price changes in neighboring states that did not change their gas tax (the control group). Explain why this can help address some of the concerns in part (a).
    \end{enumerate}
  \end{problem}
\end{document}
