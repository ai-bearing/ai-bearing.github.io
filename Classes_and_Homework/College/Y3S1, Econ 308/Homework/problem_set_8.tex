\documentclass[10pt]{extarticle}
\title{}
\author{}
\date{}
\usepackage[shortlabels]{enumitem}


%paper setup
\usepackage{geometry}
\geometry{letterpaper, portrait, margin=1in}
\usepackage{fancyhdr}
% sans serif font:
\usepackage{cmbright}
%symbols
\usepackage{amsmath}
\usepackage{bigints}
\usepackage{amssymb}
\usepackage{amsthm}
\usepackage{mathtools}
\usepackage[hidelinks]{hyperref}
\usepackage{gensymb}
\usepackage{multirow,array}
\usepackage{multicol}

\newtheorem*{remark}{Remark}
\usepackage[T1]{fontenc}
\usepackage[utf8]{inputenc}

%chemistry stuff
%\usepackage[version=4]{mhchem}
%\usepackage{chemfig}

%plotting
\usepackage{pgfplots}
\usepackage{tikz}
\tikzset{middleweight/.style={pos = 0.5}}
%\tikzset{weight/.style={pos = 0.5, fill = white}}
%\tikzset{lateweight/.style={pos = 0.75, fill = white}}
%\tikzset{earlyweight/.style={pos = 0.25, fill=white}}

%\usepackage{natbib}

%graphics stuff
\usepackage{graphicx}
\graphicspath{ {./images/} }
\usepackage[style=numeric, backend=biber]{biblatex} % Use the numeric style for Vancouver
\addbibresource{the_bibliography.bib}
%code stuff
%when using minted, make sure to add the -shell-escape flag
%you can use lstlisting if you don't want to use minted
%\usepackage{minted}
%\usemintedstyle{pastie}
%\newminted[javacode]{java}{frame=lines,framesep=2mm,linenos=true,fontsize=\footnotesize,tabsize=3,autogobble,}
%\newminted[cppcode]{cpp}{frame=lines,framesep=2mm,linenos=true,fontsize=\footnotesize,tabsize=3,autogobble,}

%\usepackage{listings}
%\usepackage{color}
%\definecolor{dkgreen}{rgb}{0,0.6,0}
%\definecolor{gray}{rgb}{0.5,0.5,0.5}
%\definecolor{mauve}{rgb}{0.58,0,0.82}
%
%\lstset{frame=tb,
%	language=Java,
%	aboveskip=3mm,
%	belowskip=3mm,
%	showstringspaces=false,
%	columns=flexible,
%	basicstyle={\small\ttfamily},
%	numbers=none,
%	numberstyle=\tiny\color{gray},
%	keywordstyle=\color{blue},
%	commentstyle=\color{dkgreen},
%	stringstyle=\color{mauve},
%	breaklines=true,
%	breakatwhitespace=true,
%	tabsize=3
%}
% text + color boxes
\renewcommand{\mathbf}[1]{\mathbold{#1}}
\usepackage[most]{tcolorbox}
\tcbuselibrary{breakable}
\tcbuselibrary{skins}
\newtcolorbox{problem}[1]{colback=white,enhanced,title={\small #1},
          attach boxed title to top center=
{yshift=-\tcboxedtitleheight/2},
boxed title style={size=small,colback=black!60!white}, sharp corners, breakable}
%including PDFs
%\usepackage{pdfpages}
\setlength{\parindent}{0pt}
\usepackage{cancel}
\pagestyle{fancy}
\fancyhf{}
\rhead{Avinash Iyer}
\lhead{Econ 308: Problem Set 8}
\newcommand{\card}{\text{card}}
\newcommand{\ran}{\text{ran}}
\newcommand{\N}{\mathbb{N}}
\newcommand{\Q}{\mathbb{Q}}
\newcommand{\Z}{\mathbb{Z}}
\newcommand{\R}{\mathbb{R}}
\begin{document}
  \begin{problem}{Marginal and Average Tax Rates}
    Sebastian is single and has an adjusted gross income of \$50,000 in 2023. Vivian and Lauren are married and jointly have an adjusted gross income of \$200,000 in 2023. Assume that both households take the standard deduction and receive no tax credits.
    \begin{problem}{(a)}
      What is the highest marginal tax rate that each household pays?
      \tcblower
      \begin{description}[font=\normalfont]
        \item[Sebastian:] 12\%
        \item[Vivian and Lauren:] 22\%
      \end{description}
    \end{problem}
    \begin{problem}{(b)}
      What is each household's average tax rate?
      \tcblower
      \begin{description}
        \item[Sebastian:] 8.2\%
        \item[Vivian and Lauren:] 14.3\%
      \end{description}
    \end{problem}
    \begin{problem}{(c)}
      Senator Flattax recommends simplifying the income tax code by applying a single 15\% marginal tax rate on all taxable income (keeping the standard deductions the same). What would each household's average tax rate be under this tax system? Is this proposed tax system more or less progressive than the current US tax code?
      \tcblower
      \begin{description}
        \item[Sebastian:] 10.8\%
        \item[Vivian and Lauren:] 12.9\%
      \end{description}
      The proposed tax system is less progressive than the current US tax code.
    \end{problem}
  \end{problem}
  \begin{problem}{Tax Code Fairness}
    Oregon has an income tax but no state sales tax, while Washington has no state income tax but does have a state sales tax. Oregon residents can deduct the state taxes they pay from their federal income taxes, while Washington residents cannot deduct the state taxes they pay.
    \begin{enumerate}[(a)]
      \item Residents of which state will pay less in federal tax, all else equal?
      \item Is this a violation of the principle of vertical equity or horizontal equity?
      \item Explain how the equity violation in part (b) might be partly offset by the fact that people can choose to live in Oregon or Washington. And how might the cost-of-living adjust in response to this location choice?
    \end{enumerate}
    \tcblower
    \begin{enumerate}[(a)]
      \item Residents of Oregon will pay less in federal taxes than residents of Washington.
      \item This is a violation of the principle of horizontal equity — residents of higher income in Washington with similar consumption will pay more in taxes than residents in Oregon.
      \item The equity violation is adjusted by the fact that people will choose to live in states depending on their preference for consumption — residents who prioritize consumption will choose to live in Oregon, while residents who do not will live in Oregon. All else equal, this means that cost of living will increase in Oregon to equalize to the cost of state taxation and inability to deduct such taxation in Washington.
    \end{enumerate}
  \end{problem}
  \begin{problem}{Laffer Logic}
    Sundari takes over a small country and is trying to figure out how to maximize tax revenue to spend on herself. She hires you as an analyst to compute the revenue-maximizing tax rate on worker incomes.\\

    Suppose all agents in the economy have the following utility function over annual consumption $c$ and annual labor supply hours $l$:
    \begin{align*}
      u(c,l) &= c - \frac{l^{k+1}}{k+1}
    \end{align*}
    where $k > 0$ is a fixed parameter. Each worker earns an hourly wage $w$ that is taxed at constant rate $\tau$. Note that \textit{no} tax revenue is rebated to workers.
    \begin{problem}{(a)}
      Solve for the utility-maximizing labor supply $l^{\ast}$ in terms of $w(1-\tau)$ and $k$.
      \tcblower
      \begin{align*}
        u &= w(1-\tau)(l) - \frac{l^{k+1}}{k+1}\\
        u' &= w(1-\tau) - l^k\\
        l^{\ast} &= \left(w(1-\tau)\right)^{1/k}
      \end{align*}
    \end{problem}
    \begin{problem}{(b)}
      Use your answer in (a) to determine an individual's elasticity of labor supply with respect to the net-of-tax rate. Your answer should depend only on $k$.
      \tcblower
      \begin{align*}
        e &= \frac{dl^{\ast}}{d(1-\tau)}\frac{(1-\tau)}{l^{\ast}}\\
          &= \frac{w^{1/k}}{k\left(w(1-\tau)\right)^{1/k}(1-\tau)^{1-1/k}}\\
          &= \frac{1}{k}
      \end{align*}
    \end{problem}
    \begin{problem}{(c)}
      Let's now use the following data to credibly estimate the parameter $k$.
      \begin{center}
        \renewcommand{\arraystretch}{1.5}
        \begin{tabular}{c|cc|cc}
          \hline
          Data &\multicolumn{2}{c}{Sundariland (treatment)} & \multicolumn{2}{c}{Canada (control)}\\
          \hline
          Year & Tax Rate & Labor Supply Hours & Tax Rate & Labor Supply Hours\\
          \hline
          2016 & 20\% & 2000 & 20\% & 1950\\
          2018 & 12\% & 2300 & 20\% & 2050\\
          \hline
        \end{tabular}
      \end{center}
      \tcblower
      \begin{problem}{(i)}
        From 2016 to 2018 Sundariland cut its tax rate while Canada did not change tax rates, making it a compelling control for the treated Sundariland. What is the percentage change in the net-of-tax rate in Sundariland over this period? Use 2016 as the base period.
        \tcblower
        The change in net-of-tax rate was $+10\%$
      \end{problem}
      \begin{problem}{(ii)}
        What is the difference-in-difference estimate for the change in labor hours in Sundariland from the tax change? Convert this estimate into the percentage change in labor hours in Sundariland over this period. Use 2016 as the base period.
        \tcblower
        The DD estimate is $+200$ hours, which is a $+10\%$ increase in hours worked relative to base.
      \end{problem}
      \begin{problem}{(iii)}
        Use parts (c.i) and (c.ii) to estimate the elasticity $e$ of labor supply with respect to the net-of-tax rate. Note that we can write the elasticity as $e = \frac{\%\Delta l^{\ast}}{\%\Delta(1-\tau)}$.
        \tcblower
        \begin{align*}
          e &= \frac{10\%}{10\%}\\
          &= 1
        \end{align*}
      \end{problem}
      \begin{problem}{(iv)}
        Finally, use the empirical estimate of $e$ in part (c.iii) and the theoretical formula for $e$ in part (b) to estimate the parameter $k$.
        \tcblower
        We find that $k\approx 1$.
      \end{problem}
    \end{problem}
    \begin{problem}{(d)}
      Determine the revenue maximizing tax rate given your estimate of $k$ in part (c.iv). Explain to Sundari why the revenue maximizing tax rate is strictly less than 100\%, even though she would like to take away all of the workers' income.
      \tcblower
      \begin{align*}
        R &= \left(w\tau\right)\left(w(1-\tau)\right)\\
        \max_{\tau}(R) &\Rightarrow \tau = 0.5.
      \end{align*}
      The revenue maximizing rate is strictly less than 100\% because as taxes increase beyond the revenue maximizing rate, workers pull back their hours more quickly than the benefits from increased tax revenue appear in the treasury.
    \end{problem}
  \end{problem}
  \begin{problem}{Electing Taxes}
    In the lecture, it was found that the optimal linear tax rate with all revenues equally rebated to individuals is
    \begin{align*}
      \tau &= \frac{1-\overline{g}}{1-\overline{g} + e}\\
      e &= \frac{1-\tau}{Z}\frac{dZ}{d(1-\tau)}
    \end{align*}
    where $Z$ is average pre-tax earnings and $0\leq \overline{g} \leq 1$ measures the degree of pre-tax earnings equality. That is, $\overline{g} = 0$ corresponds to no pre-tax equality --- one person earns everything. And $\overline{g} = 1$ corresponds to maximal pre-tax equality --- everyone earns the same amount.\\

    Suppose that the linear tax rate is chosen not by a social planner aiming to maximize utilitarian social welfare, but instead chosen via majority voting. Because each individual in our model has single-peaked preferences over the tax rate $\tau$, the median voter theorem guarantees that the tax rate preferred by the voter with the median pre-tax earnings, $z_{\text{med}}$ will win in a pairwise majority vote against any other tax rate. The median voter equilibrium generates the following linear income tax rate:
    \begin{align*}
      \tau_{\text{med}} &= \frac{1-z_{\text{med}}/Z}{1-z_{\text{med}}/Z + e}
    \end{align*}
    Empirically, income distributions are skewed to the right, implying that $z_{\text{med}}< Z$.
    \tcblower
    \begin{problem}{(a)}
      According to the U.S. Census Bureau, the 2022 median personal income was \$40,480 and the mean personal income was \$59,430. Use these numbers, and $e = 0.25$, to compute $\tau_{\text{med}}$.
      \tcblower
      \begin{align*}
        \tau_{\text{med}} &= 56\%
      \end{align*}
    \end{problem}
    \begin{problem}{(b)}
      What happens to $\tau_{\text{med}}$ if median income $z_{\text{med}}$ is close to average income $Z$? Provide intuition for this result.
      \tcblower
      As $z_{\text{med}} \rightarrow Z$, the effective tax rate reduces, as the benefits to the median voter from the rebate also reduces commensurately. For example, if $z_{\text{med}} = Z$, then the median voter does not emerge any better off from a tax + rebate scheme, so they will not vote for it.
    \end{problem}
    \begin{problem}{(c)}
      What happens to $\tau_{\text{med}}$ if median income $z_{\text{med}}$ is much smaller than average income $Z$. Provide intuition for this result.
      \tcblower
      If $z_{\text{med}}$ is much smaller than $Z$, then the median voter benefits greatly from taxes and transfers, so will maximize them to the degree that efficiency is maintained --- this is akin to the result from the Rawlsian social welfare function's ideal linear tax rate.
    \end{problem}
  \end{problem}
\end{document}
