\documentclass[8pt]{extarticle}
\title{}
\author{Avinash Iyer}
\date{}
\usepackage[shortlabels]{enumitem}


%paper setup
\usepackage{geometry}
\geometry{letterpaper, portrait, margin=1in}
\usepackage{fancyhdr}

%symbols
\usepackage{amsmath}
\usepackage{amssymb}
\usepackage{amsthm}
\usepackage{mathtools}
\usepackage{hyperref}
\usepackage{gensymb}
\usepackage{multirow,array}

\newtheorem*{remark}{Remark}
\usepackage[T1]{fontenc}
\usepackage[utf8]{inputenc}

%chemistry stuff
%\usepackage[version=4]{mhchem}
%\usepackage{chemfig}

%plotting
\usepackage{pgfplots}
\usepackage{tikz}
\tikzset{middleweight/.style={pos = 0.5, fill=white}}
\tikzset{weight/.style={pos = 0.5, fill = white}}
\tikzset{lateweight/.style={pos = 0.75, fill = white}}
\tikzset{earlyweight/.style={pos = 0.25, fill=white}}

%\usepackage{natbib}

%graphics stuff
\usepackage{graphicx}
\graphicspath{ {./images/} }
\usepackage[style=numeric, backend=biber]{biblatex} % Use the numeric style for Vancouver
\addbibresource{the_bibliography.bib}
%code stuff
%when using minted, make sure to add the -shell-escape flag
%you can use lstlisting if you don't want to use minted
%\usepackage{minted}
%\usemintedstyle{pastie}
%\newminted[javacode]{java}{frame=lines,framesep=2mm,linenos=true,fontsize=\footnotesize,tabsize=3,autogobble,}
%\newminted[cppcode]{cpp}{frame=lines,framesep=2mm,linenos=true,fontsize=\footnotesize,tabsize=3,autogobble,}

%\usepackage{listings}
%\usepackage{color}
%\definecolor{dkgreen}{rgb}{0,0.6,0}
%\definecolor{gray}{rgb}{0.5,0.5,0.5}
%\definecolor{mauve}{rgb}{0.58,0,0.82}
%
%\lstset{frame=tb,
%	language=Java,
%	aboveskip=3mm,
%	belowskip=3mm,
%	showstringspaces=false,
%	columns=flexible,
%	basicstyle={\small\ttfamily},
%	numbers=none,
%	numberstyle=\tiny\color{gray},
%	keywordstyle=\color{blue},
%	commentstyle=\color{dkgreen},
%	stringstyle=\color{mauve},
%	breaklines=true,
%	breakatwhitespace=true,
%	tabsize=3
%}
% text + color boxes
\usepackage[most]{tcolorbox}
\tcbuselibrary{breakable}
\newtcolorbox{problem}[1]{colback = white, title = {#1}, breakable}
\newtcolorbox{solution}{colback = white, colframe = black!75!white, title = Solution, breakable}
%including PDFs
%\usepackage{pdfpages}
\setlength{\parindent}{0pt}
\usepackage{cancel}
\pagestyle{fancy}
\fancyhf{}
\rhead{Avinash Iyer}
\lhead{Econ 308: Problem Set 5}
\newcommand{\card}{\text{card}}
\newcommand{\ran}{\text{ran}}
\newcommand{\N}{\mathbb{N}}
\newcommand{\Q}{\mathbb{Q}}
\newcommand{\Z}{\mathbb{Z}}
\newcommand{\R}{\mathbb{R}}
\begin{document}
  \begin{problem}{Direct Provision vs. Mandates}
    The text identifies five reasons for government intervention in insurance markets: adverse selection, externalities, administrative costs, redistribution, and paternalism. Which two reasons can explain why the government would directly provide insurance to people --- such as health insurance --- instead of mandating individuals to purchase their own insurance --- such as automobile liability insurance. Explain.
    \tcblower
    The government would most likely use direct provision of insurance for the purposes of redistribution (from the healthy to the sick) and for reducing administrative costs (as one public insurer would have lower per capita administrative burden than a private insurance market).
  \end{problem}
  \begin{problem}{Distinguishing Deductibles}
    Chimnesia has two equal-sized groups of people: smokers and nonsmokers. Both types of people have utility $U = \sqrt{C}$, where $C$ is the amount of consumption that people have in any period. So long as they are healthy, individuals will consume their entire income of \$16,000. If they need medical attention (and have no insurance), they will have to spend \$12,000 to get healthy again, leaving them with only \$4,000 to consume. Smokers have a 10\% chance of requiring major medical attention, while nonsmokers have a 2\% chance.\\

    Insurance companies in Chimnesia can sell two types of policy. The ``low deductible'' policy covers all medical costs above \$2,000, while the ``high-deductible'' policy covers only medical costs above \$10,000.
    \begin{problem}{(a)}
      What is the actuarially fair premium for each type of policy and for each group?
      \tcblower
      \begin{description}
        \item[Low Deductible, Non-Smoker:]
          \begin{align*}
            b &= 10000\\
            p&= dq\\
             &= (10000)(0.02)\\
             &= 200
          \end{align*}
        \item[Low Deductible, Smoker:]
          \begin{align*}
            b &= 10000\\
            p&= dq\\
             &= (10000)(0.1)\\
             &= 1000
          \end{align*}
        \item[High Deductible, Non-Smoker:]
          \begin{align*}
            b &= 2000\\
            p&= dq\\
             &= (2000)(0.02)\\
             &= 40
          \end{align*}
        \item[High Deductible, Smoker:]
          \begin{align*}
            b &= 2000\\
            p&= dq\\
             &= (2000)(0.10)\\
             &= 200
          \end{align*}
      \end{description}
    \end{problem}
    \begin{problem}{(b)}
      If insurance companies can tell who is a smoker and who is a nonsmoker and charge the actuarially fair premiums for each policy and group, show that both groups will purchase the low deductible policy.
      \tcblower
      Since each group is risk averse, each group will purchase the policy that reduces the most variance in their income, which comes as a result of maximizing the amount of their income that is insured (i.e., taking the policy with the lowest deductible).
    \end{problem}
    Suppose that smoking status represents asymmetric information: each individual knows if they are a smoker, but the insurance company does not. With asymmetric information, companies can't offer one price to smokers and another to nonsmokers: if, for example, the price offered to nonsmokers is lower, smokers will pretend to be nonsmokers.
    \begin{problem}{(c)}
      \begin{problem}{(i)}
        Guess that both groups purchase low deductible policies in this setting. What is the actuarially fair price for this policy?
        \tcblower
        \begin{align*}
          p &= bq\\
            &= (10000)(0.5\cdot 0.1 + 0.5\cdot 0.02)\\
            &= 600
        \end{align*}
      \end{problem}
      \begin{problem}{(ii)}
        There is not a pooling equilibrium in which both smokers and nonsmokers buy the low deductible policy at the actuarially fair price. Which type of customer does not purchase the low deductible policy? Show why this is the case mathematically.
        \tcblower
        The expected utility of the nonsmokers under a no insurance setting is as follows:
        \begin{align*}
          EU^{\text{no ins}} &= 0.98\sqrt{16000} + 0.02\sqrt{4000}\\
                             &= 125.22\\
                             \shortintertext{and under the given policy:}
          EU^{\text{ins policy}} &= 0.98\sqrt{16000-600} + 0.02\sqrt{16000-2000-600}\\
                                 &= 123.93\\
                                 &< EU^{\text{no ins}}
        \end{align*}
        So, the risk-averse nonsmokers will avoid taking the insurance policy where everyone chooses to use the low deductible plan.
      \end{problem}
      \begin{problem}{(iii)}
        Given that only one type of customer will buy the low deductible policy, what will be the actuarially fair price for the low deductible policy?
        \tcblower
        Since only smokers will buy the low deductible policy, the actuarially fair price as calculated earlier is $p = 1000$.
      \end{problem}
    \end{problem}
    \begin{problem}{(d)}
      Show that it is possible for both groups to purchase insurance, with one group preferring to buy low deductible policies and one group preferring to buy high deductible policies. In particular, which group buys which policy and what is the premium for each type of policy?
      \tcblower
      As established earlier, nonsmokers do not buy the low deductible policy, meaning that the low deductible policy is taken entirely by smokers, whose actuarially fair premium is $p = 1000$.\newline

      Therefore, the nonsmokers take the high deductible policy, which is inefficient, but they pay $p = 40$ as their actuarially fair price.
    \end{problem}
  \end{problem}
  \begin{problem}{Retirement Reforms}
    The government of Weslovakia has just reformed its social security system. This reform changed two aspects of the system: (1) it abolished its actuarial reduction for early retirement, and (2) it reduced the payroll tax by half for workers who continued to work beyond the early retirement age. Will the average retirement age for Weslovakian workers increase or decrease in response to these two changes, or can you not tell?
    \tcblower
    We cannot tell whether or not the average retirement age increases or decreases as there are two countervailing factors at play: reduced payroll taxes decrease the cost of working past the early retirement age, but removing the actuarial reduction reduces the cost of retiring early. 
  \end{problem}
  \begin{problem}{Funding the Future}
    Consider an economy that is composed of identical individuals who live for two periods. These individuals have preferences over consumption in periods $1$ and $2$ given by $U = \sqrt{C_1} + \sqrt{C_2}$. They receive an income of $\$45$ in period $1$ and an income of $\$20$ in period 2. They can save as much of their income as they like in bank accounts, earning an interest rate of 10 per period. They do not care about their children, so they spend all their money before the end of period 2.\newline

    Each individual's lifetime budget constraint is given by $C_1 + C_2/(1+r) = Y_1 + Y_2/(1+r)$. Individuals choose consumption in each period by maximizing lifetime utility subject to this lifetime constraint.
    \tcblower
    \begin{problem}{(a)}
      What is the individual's optimal consumption in each period? How much saving does he or she do in the first period?
      \tcblower
      \begin{align*}
        C_1^* &= C_2^* \tag*{Consumption Smoothing Equilibrium}\\
        C_1^* + \frac{C_1^*}{1+r} &= Y_1 + Y_2/(1+r)\\
        (2+r)C_1^* &= Y_1(1+r) + Y_2\\
        C_1^* &= \frac{Y_1(1+r) + Y_2}{2+r}\\
        C_1^* &= 33.1\\
        S_1^* &= 11.9\\
        C_2^* &= 33.1
      \end{align*}
    \end{problem}
    \begin{problem}{(b)}
      The government has decided to set up a social security system. This system will take \$10 from each individual in the first period, put it in the bank, and transfer it to each person with interest in the second period. Write out the new lifetime budget constraint. How does the system affect the amount of private savings? How does the system affect national savings? Is this an example of a funded or unfunded social security system?
      \tcblower
      \begin{align*}
        C_1 + C_2/(1+r) &= \left(Y_1-10\right) + Y_2/(1+r) + 10(1+r)
      \end{align*}
      This system reduces private savings by one to one crowdout, but does not affect national savings (as the \$10 taxed today is equal to \$10 plus interest of future consumption). It is an example of a funded system.
    \end{problem}
    \begin{problem}{(c)}
      Suppose instead that the government uses the \$10 contribution from each individual to start paying out benefits to current retirees (who did not pay into a social security system when they were working). It still promises to pay current workers their \$10 (plus interest) back when they retire using contributions from future workers. Similarly, it will pay back future workers interest on their contributions using the contributions of the next generation of workers. An influential politician says: ``This is a free lunch: we help out current retirees, and current and future workers will make the same contributions and receive the same benefits, so it doesn't harm them either.'' But you, a savvy economist, realize that there is something wrong with this argument. Think through the same questions as in part (b) --- i.e., the effects on private and national savings of this system. Explain the key cost to this system relative to part (b).
      \tcblower
      The budget constraint in this model is as follows:
      \begin{align*}
        C_1 + C_2/(1+r) &= \left(Y_1-10\right) + Y_2/(1+r)
      \end{align*}
      The \$10 in this example is not used for future consumption (i.e., saved) on the part of workers (which would mean it has one to one crowdout with private savings and does not affect national savings), but is used to finance current consumption, meaning it has a negative effect on both private and national savings. This is the primary difference between a funded system (as was detailed previously) and an unfunded system.
    \end{problem}
  \end{problem}
\end{document}
