\documentclass[8pt]{extarticle}
\title{}
\author{Avinash Iyer}
\date{}
\usepackage[shortlabels]{enumitem}


%paper setup
\usepackage{geometry}
\geometry{letterpaper, portrait, margin=1in}
\usepackage{fancyhdr}

%symbols
\usepackage{amsmath}
\usepackage{amssymb}
\usepackage{amsthm}
\usepackage{mathtools}
\usepackage{hyperref}
\usepackage{gensymb}
\usepackage{multirow,array}

\newtheorem*{remark}{Remark}
\usepackage[T1]{fontenc}
\usepackage[utf8]{inputenc}

%chemistry stuff
%\usepackage[version=4]{mhchem}
%\usepackage{chemfig}

%plotting
\usepackage{pgfplots}
\usepackage{tikz}
\tikzset{middleweight/.style={pos = 0.5, fill=white}}
\tikzset{weight/.style={pos = 0.5, fill = white}}
\tikzset{lateweight/.style={pos = 0.75, fill = white}}
\tikzset{earlyweight/.style={pos = 0.25, fill=white}}

%\usepackage{natbib}

%graphics stuff
\usepackage{graphicx}
\graphicspath{ {./images/} }
\usepackage[style=numeric, backend=biber]{biblatex} % Use the numeric style for Vancouver
\addbibresource{the_bibliography.bib}
%code stuff
%when using minted, make sure to add the -shell-escape flag
%you can use lstlisting if you don't want to use minted
%\usepackage{minted}
%\usemintedstyle{pastie}
%\newminted[javacode]{java}{frame=lines,framesep=2mm,linenos=true,fontsize=\footnotesize,tabsize=3,autogobble,}
%\newminted[cppcode]{cpp}{frame=lines,framesep=2mm,linenos=true,fontsize=\footnotesize,tabsize=3,autogobble,}

%\usepackage{listings}
%\usepackage{color}
%\definecolor{dkgreen}{rgb}{0,0.6,0}
%\definecolor{gray}{rgb}{0.5,0.5,0.5}
%\definecolor{mauve}{rgb}{0.58,0,0.82}
%
%\lstset{frame=tb,
%	language=Java,
%	aboveskip=3mm,
%	belowskip=3mm,
%	showstringspaces=false,
%	columns=flexible,
%	basicstyle={\small\ttfamily},
%	numbers=none,
%	numberstyle=\tiny\color{gray},
%	keywordstyle=\color{blue},
%	commentstyle=\color{dkgreen},
%	stringstyle=\color{mauve},
%	breaklines=true,
%	breakatwhitespace=true,
%	tabsize=3
%}
% text + color boxes
\usepackage[most]{tcolorbox}
\tcbuselibrary{breakable}
\newtcolorbox{problem}[1]{colback = white, title = {#1}, breakable}
\newtcolorbox{solution}{colback = white, colframe = black!75!white, title = Solution, breakable}
%including PDFs
%\usepackage{pdfpages}
\setlength{\parindent}{0pt}
\usepackage{cancel}
\pagestyle{fancy}
\fancyhf{}
\rhead{Avinash Iyer}
\lhead{Econ 308: Problem Set 5}
\newcommand{\card}{\text{card}}
\newcommand{\ran}{\text{ran}}
\newcommand{\N}{\mathbb{N}}
\newcommand{\Q}{\mathbb{Q}}
\newcommand{\Z}{\mathbb{Z}}
\newcommand{\R}{\mathbb{R}}
\begin{document}
  \begin{problem}{Direct Provision vs. Mandates}
    The text identifies five reasons for government intervention in insurance markets: adverse selection, externalities, administrative costs, redistribution, and paternalism. Which two reasons can explain why the government would directly provide insurance to people --- such as health insurance --- instead of mandating individuals to purchase their own insurance --- such as automobile liability insurance. Explain.
    \tcblower
    The government would most likely use direct provision of insurance for the purposes of redistribution (from the healthy to the sick) and for reducing administrative costs (as one public insurer would have lower per capita administrative burden than a private insurance market).
  \end{problem}
  \begin{problem}{Distinguishing Deductibles}
    Chimnesia has two equal-sized groups of people: smokers and nonsmokers. Both types of people have utility $U = \sqrt{C}$, where $C$ is the amount of consumption that people have in any period. So long as they are healthy, individuals will consume their entire income of \$16,000. If they need medical attention (and have no insurance), they will have to spend \$12,000 to get healthy again, leaving them with only \$4,000 to consume. Smokers have a 10\% chance of requiring major medical attention, while nonsmokers have a 2\% chance.\\

    Insurance companies in Chimnesia can sell two types of policy. The ``low deductible'' policy covers all medical costs above \$2,000, while the ``high-deductible'' policy covers only medical costs above \$10,000.
    \begin{problem}{(a)}
      What is the actuarially fair premium for each type of policy and for each group?
      \tcblower
      \begin{description}
        \item[Low Deductible, Non-Smoker:]
          \begin{align*}
            b &= 10000\\
            p&= dq\\
             &= (10000)(0.02)\\
             &= 200
          \end{align*}
        \item[Low Deductible, Smoker:]
          \begin{align*}
            b &= 10000\\
            p&= dq\\
             &= (10000)(0.1)\\
             &= 1000
          \end{align*}
        \item[High Deductible, Non-Smoker:]
          \begin{align*}
            b &= 2000\\
            p&= dq\\
             &= (2000)(0.02)\\
             &= 40
          \end{align*}
        \item[High Deductible, Smoker:]
          \begin{align*}
            b &= 2000\\
            p&= dq\\
             &= (2000)(0.10)\\
             &= 200
          \end{align*}
      \end{description}
    \end{problem}
    \begin{problem}{(b)}
      If insurance companies can tell who is a smoker and who is a nonsmoker and charge the actuarially fair premiums for each policy and group, show that both groups will purchase the low deductible policy.
      \tcblower
      Since each group is risk averse, each group will purchase the policy that reduces the most variance in their income, which comes as a result of maximizing the amount of their income that is insured (i.e., taking the policy with the lowest deductible).
    \end{problem}
  \end{problem}
\end{document}
