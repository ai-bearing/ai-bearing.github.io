\documentclass[8pt]{extarticle}
\title{}
\author{Avinash Iyer}
\date{}
\usepackage[shortlabels]{enumitem}

%font setup
%
%\usepackage{newpxtext,eulerpx}

%paper setup
\usepackage{geometry}
\geometry{letterpaper, portrait, margin=1in}
\usepackage{fancyhdr}

%symbols
\usepackage{amsmath}
\usepackage{amssymb}
\usepackage{mathtools}
\usepackage{hyperref}
\usepackage{gensymb}
\usepackage{multirow,array}

\usepackage[T1]{fontenc}
\usepackage[utf8]{inputenc}

%chemistry stuff
\usepackage[version=4]{mhchem}
\usepackage{chemfig}

%plotting
\usepackage{pgfplots}
\usepackage{tikz}
\tikzset{middleweight/.style={pos = 0.5, fill=white}}
\tikzset{weight/.style={pos = 0.5, fill = white}}
\tikzset{lateweight/.style={pos = 0.75, fill = white}}
\tikzset{earlyweight/.style={pos = 0.25, fill=white}}

%\usepackage{natbib}

%graphics stuff
\usepackage{graphicx}
\graphicspath{ {./images/} }

%code stuff
%when using minted, make sure to add the -shell-escape flag
%you can use lstlisting if you don't want to use minted
%\usepackage{minted}
%\usemintedstyle{pastie}
%\newminted[javacode]{java}{frame=lines,framesep=2mm,linenos=true,fontsize=\footnotesize,tabsize=3,autogobble,}
%\newminted[cppcode]{cpp}{frame=lines,framesep=2mm,linenos=true,fontsize=\footnotesize,tabsize=3,autogobble,}

%\usepackage{listings}
%\usepackage{color}
%\definecolor{dkgreen}{rgb}{0,0.6,0}
%\definecolor{gray}{rgb}{0.5,0.5,0.5}
%\definecolor{mauve}{rgb}{0.58,0,0.82}
%
%\lstset{frame=tb,
%	language=Java,
%	aboveskip=3mm,
%	belowskip=3mm,
%	showstringspaces=false,
%	columns=flexible,
%	basicstyle={\small\ttfamily},
%	numbers=none,
%	numberstyle=\tiny\color{gray},
%	keywordstyle=\color{blue},
%	commentstyle=\color{dkgreen},
%	stringstyle=\color{mauve},
%	breaklines=true,
%	breakatwhitespace=true,
%	tabsize=3
%}
% text + color boxes
\usepackage[most]{tcolorbox}
\tcbuselibrary{breakable}
\newtcolorbox{problem}[1]{colback = white, title = {#1}, breakable}
\newtcolorbox{solution}{colback = white, colframe = black!75!white, title = Solution, breakable}
%including PDFs
%\usepackage{pdfpages}
\setlength{\parindent}{0pt}

\pagestyle{fancy}
\fancyhf{}
\rhead{Avinash Iyer}
\lhead{Econ 308: Problem Set 2}
\newcommand{\card}{\text{card}}
\newcommand{\ran}{\text{ran}}
\newcommand{\N}{\mathbb{N}}
\newcommand{\Q}{\mathbb{Q}}
\newcommand{\Z}{\mathbb{Z}}
\newcommand{\R}{\mathbb{R}}
\begin{document}
  \begin{problem}{Optimal Pigouvian Taxation}
    A coal-fired power plant releases air pollution into the atmosphere for every unit of electricity produced. The inverse demand function is $P_d = 20 - 0.5Q$, which represents the marginal benefit where $Q$ is the quantity consumed when consumers pay price $P_d$. The inverse supply curve for coal-fired electricity is $P_s = 5 + Q$, which represents the marginal private cost curve when the power plant produces $Q$ units. The marginal damage from emissions is given by $MD = 3.5Q$, which describes the cost of greenhouse gas emissions and local air pollution when the industry generates $Q$ units of coal-fired electricity.
    \begin{enumerate}[(a)]
      \item Illustrate the market for the coal-fired electricity with a supply/demand graph. Be sure to draw the curves for demand, supply, marginal damage, and social marginal cost.
      \item What is the private market equilibrium?
      \item What is the socially optimal/efficient quantity of coal-fired electricity?
      \item How large is the deadweight loss from this negative production externality?
      \item A corrective tax has the effect of ``internalizing the externality.'' How large should the (constant) per-unit corrective tax be in order to induce the market quantity to be socially optimal/efficient? Draw the firm's supply curve with the new tax on your graph in part (a). What is the consumer price paid and the supplier price received with the optimal corrective tax?
    \end{enumerate}
    \tcblower
    \begin{problem}{(b)}
      \begin{align*}
        20-0.5Q &= 5 + Q\\
        15 &= 1.5Q\\
        Q &= 10\\
        P &= 15
      \end{align*}
    \end{problem}
    \begin{problem}{(c)}
      \begin{align*}
        20-0.5Q &= 5+Q+3.5Q\\
        15 &= 5Q\\
        Q &= 3\\
        P &= 17.5
      \end{align*}
    \end{problem}
  \end{problem}
  \begin{problem}{Pollution Standards}
    Another way to reduce pollution is to mandate a maximum quantity of pollution (instead of taxing pollution-creating activities). 
  \end{problem}
\end{document}
