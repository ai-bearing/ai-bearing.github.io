\documentclass[12pt]{extarticle}
\title{}
\author{}
\date{}
\usepackage[shortlabels]{enumitem}


%paper setup
\usepackage{geometry}
\geometry{letterpaper, portrait, margin=1in}
\usepackage{fancyhdr}
% sans serif font:
\usepackage{cmbright}
%symbols
\usepackage{amsmath}
\usepackage{bigints}
\usepackage{amssymb}
\usepackage{amsthm}
\usepackage{mathtools}
\usepackage[hidelinks]{hyperref}
\usepackage{gensymb}
\usepackage{multirow,array}
\usepackage{multicol}

\newtheorem*{remark}{Remark}
\usepackage[T1]{fontenc}
\usepackage[utf8]{inputenc}

%chemistry stuff
%\usepackage[version=4]{mhchem}
%\usepackage{chemfig}

%plotting
\usepackage{pgfplots}
\usepackage{tikz}
\tikzset{middleweight/.style={pos = 0.5}}
%\tikzset{weight/.style={pos = 0.5, fill = white}}
%\tikzset{lateweight/.style={pos = 0.75, fill = white}}
%\tikzset{earlyweight/.style={pos = 0.25, fill=white}}

%\usepackage{natbib}

%graphics stuff
\usepackage{graphicx}
\graphicspath{ {./images/} }
\usepackage[style=numeric, backend=biber]{biblatex} % Use the numeric style for Vancouver
\addbibresource{the_bibliography.bib}
%code stuff
%when using minted, make sure to add the -shell-escape flag
%you can use lstlisting if you don't want to use minted
%\usepackage{minted}
%\usemintedstyle{pastie}
%\newminted[javacode]{java}{frame=lines,framesep=2mm,linenos=true,fontsize=\footnotesize,tabsize=3,autogobble,}
%\newminted[cppcode]{cpp}{frame=lines,framesep=2mm,linenos=true,fontsize=\footnotesize,tabsize=3,autogobble,}

%\usepackage{listings}
%\usepackage{color}
%\definecolor{dkgreen}{rgb}{0,0.6,0}
%\definecolor{gray}{rgb}{0.5,0.5,0.5}
%\definecolor{mauve}{rgb}{0.58,0,0.82}
%
%\lstset{frame=tb,
%	language=Java,
%	aboveskip=3mm,
%	belowskip=3mm,
%	showstringspaces=false,
%	columns=flexible,
%	basicstyle={\small\ttfamily},
%	numbers=none,
%	numberstyle=\tiny\color{gray},
%	keywordstyle=\color{blue},
%	commentstyle=\color{dkgreen},
%	stringstyle=\color{mauve},
%	breaklines=true,
%	breakatwhitespace=true,
%	tabsize=3
%}
% text + color boxes
\renewcommand{\mathbf}[1]{\mathbold{#1}}
\usepackage[most]{tcolorbox}
\tcbuselibrary{breakable}
\tcbuselibrary{skins}
\newtcolorbox{problem}[1]{colback=white,enhanced,title={\small #1},
          attach boxed title to top center=
{yshift=-\tcboxedtitleheight/2},
boxed title style={size=small,colback=black!60!white}, sharp corners, breakable}
%including PDFs
%\usepackage{pdfpages}
\setlength{\parindent}{0pt}
\usepackage{cancel}
\pagestyle{fancy}
\fancyhf{}
\rhead{Avinash Iyer}
\lhead{Econ 308: Paper 2}
\chead{SSWR}
\usepackage{setspace}
\newcommand{\card}{\text{card}}
\newcommand{\ran}{\text{ran}}
\newcommand{\N}{\mathbb{N}}
\newcommand{\Q}{\mathbb{Q}}
\newcommand{\Z}{\mathbb{Z}}
\newcommand{\R}{\mathbb{R}}
\begin{document}
\doublespacing
  \section*{Abstract}%
  \section*{Policy Outline}%
  As it currently stands, owner-occupied housing in the United States receives certain tax preferences that other asset classes do not receive. For instance, while owner-occupier households are able to list mortgage interest as an expense (reducing their taxable income), owner-occupier households are not taxed on the revenue that comes from taking on a mortgage --- known as ``net imputed rent,'' or the economic benefit from owner-occupancy. The exclusion of net imputed rent from the tax base is the second largest largest tax expenditure --- an untaxed part of the national income base --- after the exclusion for employer contributions for health insurance. The tax expenditure for the exclusion for net imputed rent totals to approximately \$1.7 trillion over the ten year period from fiscal year 2023 to fiscal year 2032.\\

  The next largest housing-related tax expenditure after the exclusion of net imputed rent is the exclusion of the first \$500,000 of capital gains on the sale of owner-occupied housing (for married couples filing jointly); according to the Treasury Department, the value of this tax exclusion over the ten year period from fiscal year 2023 to fiscal year 2032 will be approximately \$650 billion. An exclusion of this magnitude is not afforded to other common asset classes such as stocks, bonds, or mutual funds.\\

  The proposal I am outlining is to remove the tax preferences that the owner-occupied housing asset class is afforded --- this means that, while the mortgage interest deduction should not be removed, the income from net imputed rent should be included in the income tax base, and capital gains from house sales should not be given a tax exclusion of such a magnitude. This is meant to be a revenue-neutral reform of the tax code; the increased tax revenue from the adoption of this policy should be offset by a commensurate increase in the size of the standard deduction.
  \section*{Effects on Distribution of Income}%
  \section*{Effects on Distribution of Investment}%
  \section*{Recommendation}%
\end{document}
