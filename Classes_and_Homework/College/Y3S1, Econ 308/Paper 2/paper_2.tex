\documentclass[12pt]{extarticle}
\title{}
\author{}
\date{}
\usepackage[shortlabels]{enumitem}


%paper setup
\usepackage{geometry}
\geometry{letterpaper, portrait, margin=1in}
\usepackage{fancyhdr}
% sans serif font:
\usepackage{cmbright}
%symbols
\usepackage{amsmath}
\usepackage{bigints}
\usepackage{amssymb}
\usepackage{amsthm}
\usepackage{mathtools}
\usepackage[hidelinks]{hyperref}
\usepackage{gensymb}
\usepackage{multirow,array}
\usepackage{multicol}

\newtheorem*{remark}{Remark}
\usepackage[T1]{fontenc}
\usepackage[utf8]{inputenc}

%chemistry stuff
%\usepackage[version=4]{mhchem}
%\usepackage{chemfig}

%plotting
\usepackage{pgfplots}
\usepackage{tikz}
\tikzset{middleweight/.style={pos = 0.5}}
%\tikzset{weight/.style={pos = 0.5, fill = white}}
%\tikzset{lateweight/.style={pos = 0.75, fill = white}}
%\tikzset{earlyweight/.style={pos = 0.25, fill=white}}

%\usepackage{natbib}

%graphics stuff
\usepackage{graphicx}
\graphicspath{ {./images/} }
\usepackage[style=numeric, backend=biber]{biblatex} % Use the numeric style for Vancouver
\addbibresource{the_bibliography.bib}
%code stuff
%when using minted, make sure to add the -shell-escape flag
%you can use lstlisting if you don't want to use minted
%\usepackage{minted}
%\usemintedstyle{pastie}
%\newminted[javacode]{java}{frame=lines,framesep=2mm,linenos=true,fontsize=\footnotesize,tabsize=3,autogobble,}
%\newminted[cppcode]{cpp}{frame=lines,framesep=2mm,linenos=true,fontsize=\footnotesize,tabsize=3,autogobble,}

%\usepackage{listings}
%\usepackage{color}
%\definecolor{dkgreen}{rgb}{0,0.6,0}
%\definecolor{gray}{rgb}{0.5,0.5,0.5}
%\definecolor{mauve}{rgb}{0.58,0,0.82}
%
%\lstset{frame=tb,
%	language=Java,
%	aboveskip=3mm,
%	belowskip=3mm,
%	showstringspaces=false,
%	columns=flexible,
%	basicstyle={\small\ttfamily},
%	numbers=none,
%	numberstyle=\tiny\color{gray},
%	keywordstyle=\color{blue},
%	commentstyle=\color{dkgreen},
%	stringstyle=\color{mauve},
%	breaklines=true,
%	breakatwhitespace=true,
%	tabsize=3
%}
% text + color boxes
\renewcommand{\mathbf}[1]{\mathbold{#1}}
\usepackage[most]{tcolorbox}
\tcbuselibrary{breakable}
\tcbuselibrary{skins}
\newtcolorbox{problem}[1]{colback=white,enhanced,title={\small #1},
          attach boxed title to top center=
{yshift=-\tcboxedtitleheight/2},
boxed title style={size=small,colback=black!60!white}, sharp corners, breakable}
%including PDFs
%\usepackage{pdfpages}
\setlength{\parindent}{0pt}
\usepackage{cancel}
\pagestyle{fancy}
\fancyhf{}
\rhead{Avinash Iyer}
\lhead{Econ 308: Paper 2}
\chead{SSWR}
\usepackage{setspace}
\newcommand{\card}{\text{card}}
\newcommand{\ran}{\text{ran}}
\newcommand{\N}{\mathbb{N}}
\newcommand{\Q}{\mathbb{Q}}
\newcommand{\Z}{\mathbb{Z}}
\newcommand{\R}{\mathbb{R}}
\begin{document}
\doublespacing
  \section*{Abstract}%
  In this paper, I examine the effects of a revenue-neutral income tax base shift that incorporates net imputed rental income into the income tax base, while the revenue from such an inclusion is offset by a commensurate increase in the standard deduction. In the analysis, I find that such an inclusion will likely yield a positive effect on labor supply, reduce pre-tax labor income inequality, improve business capital formation, increasing utilitarian social welfare.
  \section*{Policy Outline}%
  As it currently stands, owner-occupied housing in the United States receives certain tax preferences that other asset classes do not receive. For instance, while owner-occupier households are able to list mortgage interest as an expense (reducing their taxable income), owner-occupier households are not taxed on the revenue that comes from taking on a mortgage --- known as ``net imputed rent,'' or the economic benefit from owner-occupancy.\supercite{yglesias_2016} The exclusion of net imputed rent from the tax base is the second largest largest tax expenditure --- an untaxed part of the national income base --- after the exclusion for employer contributions for health insurance. The tax expenditure for the exclusion for net imputed rent totals to approximately \$1.7 trillion over the ten year period from fiscal year 2023 to fiscal year 2032.\supercite{tax_expenditures}\\

  The next largest housing-related tax expenditure after the exclusion of net imputed rent is the exclusion of the first \$500,000 of capital gains on the sale of owner-occupied housing (for married couples filing jointly); according to the Treasury Department, the value of this tax exclusion over the ten year period from fiscal year 2023 to fiscal year 2032 will be approximately \$650 billion. An exclusion of this magnitude is not afforded to other common asset classes such as stocks, bonds, or mutual funds.\\

  The policy I am going to examine is to remove the tax preferences that the owner-occupied housing asset class is afforded --- this means that, while the mortgage interest deduction should not be removed (as business interest expenses are able to be deducted in other capital asset classes), the income from net imputed rent should be included in the income tax base, and capital gains from house sales should not be given a tax exclusion. This is meant to be a revenue-neutral reform of the tax code; the increased tax revenue from the adoption of this policy should be offset by a commensurate increase in the size of the standard deduction.
  \section*{Effects on Labor Income and Distribution}%
  By increasing the standard deduction to make this proposed tax reform revenue neutral, we see that the base of taxable labor income would decrease; this is because the national income tax base increases from the inclusion of net imputed rent, while at the same time, the standard deduction has been increased to offset the revenue from inclusion of net imputed rent into the tax base. Thus, the revenue of the federal government shifts to include more capital income in the tax base, while labor income is removed from the tax base so as to maintain revenue neutrality.\\

  The reduced effective income tax rate on labor takes the form of a reduction in the base of taxable income --- taxable income, which is adjusted gross income net of deductions, would be reduced due to the increased standard deduction. This means that the net-of-tax wage would increase relative to the status quo, seeing as that at a given gross labor income level, less of the income is subject to taxes. Since $w(1-\tau)$ increases, this would induce a substitution effect towards more labor, while the income effect would run counter to the increased labor.\\

  Assuming that individuals in the economy have quasilinear preferences over consumption and leisure given by
  \begin{align*}
    u(c,l) &= c - \frac{l^2}{2},
  \end{align*}
  we would expect that $l^{\ast} = w(1-\tau)$ increases, seeing as there would only be substitution effects in such a scenario.\\

  The largest increases in net-of-tax wage would occur with lower-income workers on the margin between the current standard deduction and the increased standard deduction that comes from including net imputed rent in the income tax base. Therefore, we would expect the substitution effect towards increased labor to run strongest among those with lower level incomes, meaning that we would expect a relative compression in pre-tax incomes as a result of lower income workers' increased labor supply relative to higher income workers' increased labor supply. In addition to these labor supply effects, we would also expect to see increases in disposable income from those lower in the income distribution due to simple reduced effective tax rates.\\

  Figari et al. (2017) modeled the effects of including imputed rental income in the tax base, and found similar results to those predicted in this paper. They saw a relative compression in the income distribution due to the increased tax exemption yielding greater disposable income for those towards the middle and bottom of the income distribution.\supercite{figari__2017}
  \section*{Effects on Capital Income and Distribution}%
  %substitution effect away from owner-occupied housing toward renter-occupied housing (due to equalizing effective tax rates between the two asset classes)
  %individual substitution effect of investment away from owner-occupied housing towards other asset classes, likely those based on physical capital
  %negative income effect from increased tax rates --- may be counterbalanced by substitution effect, but uncertain --- Switzerland, for example, has taxed net imputed rental income for a while, and has a much higher savings rate than the United States
  Upon including net imputed rental income in the tax base, we should expect various forms of substitution and income effects in the realm of investment. By removing the owner-occupancy privileges of the tax code, and bringing the resulting capital income taxation in line with those of other forms of capital income such as business income, we would expect a substitution effect away from owner-occupied housing as a form of capital income and towards other forms of capital income, thanks to the establishment of effective neutrality due to this policy implementation.\\

  In particular, removing these privileges for owner-occupied housing, and treating capital income from owner-occupied housing as akin to pass-through entities or C-corporations, would yield effective tax rates on owner-occupied housing capital income as those close to rental housing.\supercite{effective_rates} We would expect the substitution effect to yield increased construction of rental housing and investment in capital via corporations or pass-through entities. The substitution effect would occur at the individual level, as those who, on the margin, would invest in owner-occupied housing instead invest in other asset classes due to this equalized taxation.\\

  In addition to individual level investment decisions, we would see certain substitution and income effects from the implementation of taxation in owner-occupied housing. For a model, consider a homeowner's two-period intertemporal budget constraint.
  \begin{align*}
    C_1 + \frac{C_2}{1+r} &= Y_1 + \frac{Y_2}{1+r}
  \end{align*}
  After including net imputed rental income into the tax base, the new budget constraint is as follows.
  \begin{align*}
    C_1 + \frac{C_2}{1+r(1-\tau)} &= Y_1 + \frac{Y_2}{1+r}
  \end{align*}
  As a result, we would expect to see a substitution effect towards consuming more in period 1 (due to the effective increase in tax on investing in owner-occupied housing), and an income effect towards consuming less in period 1 (due to reduced total consumption ability thanks to the implementation of the tax). The primary policy question is as to the effective size of these two effects.\\

  Data from Switzerland, which has taxed net imputed rental income since 1934,\supercite{switzerland_imputed} suggest that it is possible that the income effect would be larger in magnitude than the substitution effect. The United States, which does not include net imputed rental income in the tax base, has a lower gross savings rate than Switzerland --- approximately 18\% in the United States\supercite{switzerland_savings} compared with approximately 34\% in Switzerland.\supercite{switzerland_savings}\\

  If we take the assumption that the income effect outweighs the substitution effect at the individual savings level and the substitution effect towards different capital asset classes at the investment level, we would expect that the effect of including net imputed rent in the income tax base would be an increase in business capital formation, yielding greater productivity for workers.
  \section*{Recommendation}%
  Due to the increased productivity of workers, and the commensurate increase in wages, as well as the reduced tax burden as a result of revenue neutrality, we can say with reasonable certainty that shifting the income tax base to include net imputed rent would yield a net increase in total utility. However, this improvement in the utilitarian social welfare function would not be a Pareto improvement --- those for whom imputed rental income consists of a sizable portion of their income would likely be negatively affected from inclusion into the tax base. However, given the utilitarian benefits, I would provide a positive recommendation to such a policy.
  \printbibliography
\end{document}
