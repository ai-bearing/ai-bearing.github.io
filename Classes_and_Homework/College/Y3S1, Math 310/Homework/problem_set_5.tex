\documentclass[10pt]{extarticle}
\title{}
\author{Avinash Iyer}
\date{}
\usepackage[shortlabels]{enumitem}


%paper setup
\usepackage{geometry}
\geometry{letterpaper, portrait, margin=1in}
\usepackage{fancyhdr}

%symbols
\usepackage{amsmath}
\usepackage{amssymb}
\usepackage{amsthm}
\usepackage{mathtools}
\usepackage{hyperref}
\usepackage{gensymb}
\usepackage{multirow,array}

\newtheorem*{remark}{Remark}
\usepackage[T1]{fontenc}
\usepackage[utf8]{inputenc}

%chemistry stuff
%\usepackage[version=4]{mhchem}
%\usepackage{chemfig}

%plotting
\usepackage{pgfplots}
\usepackage{tikz}
\tikzset{middleweight/.style={pos = 0.5, fill=white}}
\tikzset{weight/.style={pos = 0.5, fill = white}}
\tikzset{lateweight/.style={pos = 0.75, fill = white}}
\tikzset{earlyweight/.style={pos = 0.25, fill=white}}

%\usepackage{natbib}

%graphics stuff
\usepackage{graphicx}
\graphicspath{ {./images/} }
\usepackage[style=numeric, backend=biber]{biblatex} % Use the numeric style for Vancouver
\addbibresource{the_bibliography.bib}
%code stuff
%when using minted, make sure to add the -shell-escape flag
%you can use lstlisting if you don't want to use minted
%\usepackage{minted}
%\usemintedstyle{pastie}
%\newminted[javacode]{java}{frame=lines,framesep=2mm,linenos=true,fontsize=\footnotesize,tabsize=3,autogobble,}
%\newminted[cppcode]{cpp}{frame=lines,framesep=2mm,linenos=true,fontsize=\footnotesize,tabsize=3,autogobble,}

%\usepackage{listings}
%\usepackage{color}
%\definecolor{dkgreen}{rgb}{0,0.6,0}
%\definecolor{gray}{rgb}{0.5,0.5,0.5}
%\definecolor{mauve}{rgb}{0.58,0,0.82}
%
%\lstset{frame=tb,
%	language=Java,
%	aboveskip=3mm,
%	belowskip=3mm,
%	showstringspaces=false,
%	columns=flexible,
%	basicstyle={\small\ttfamily},
%	numbers=none,
%	numberstyle=\tiny\color{gray},
%	keywordstyle=\color{blue},
%	commentstyle=\color{dkgreen},
%	stringstyle=\color{mauve},
%	breaklines=true,
%	breakatwhitespace=true,
%	tabsize=3
%}
% text + color boxes
\usepackage[most]{tcolorbox}
\tcbuselibrary{breakable}
\tcbuselibrary{skins}
\newtcolorbox{problem}[1]{colback=white,enhanced,title={\small #1},
          attach boxed title to top center=
{yshift=-\tcboxedtitleheight/2},
boxed title style={size=small,colback=black!60!white}, breakable}
%including PDFs
%\usepackage{pdfpages}
\setlength{\parindent}{0pt}
\usepackage{cancel}
\pagestyle{fancy}
\fancyhf{}
\rhead{Avinash Iyer}
\lhead{Math 310: Problem Set 5}
\newcommand{\card}{\text{card}}
\newcommand{\ran}{\text{ran}}
\newcommand{\N}{\mathbb{N}}
\newcommand{\Q}{\mathbb{Q}}
\newcommand{\Z}{\mathbb{Z}}
\newcommand{\R}{\mathbb{R}}
\begin{document}
  \begin{problem}{Problem 1}
    Let $x_1 = 1$ and inductively set $x_{n+1} = \sqrt{2 + x_n}$. Show that $(x_n)_n$ converges and find its limit.
    \tcblower
    We claim that $(x_n)_n$ is bounded above by $2$ and monotone increasing.\\

    Base case, we have $x_1 = 1 \leq 2$. Inductively assume that $x_n \leq 2$.
    \begin{align*}
      x_{n+1} &= \sqrt{2 + x_n}\\
              &\leq \sqrt{2 + 2}\\
              &\leq 2
    \end{align*}
    Similarly, we know that $\forall 0 \leq x \leq 2$, it is the case that $x^2 \leq 2 + x$, as $x^2 - x - 2 \leq 0 \Rightarrow -1 \leq 0 \leq x \leq 2$. Therefore, $\forall n\in\N$,
    \begin{align*}
      x_{n} &\leq \sqrt{2 + x_n}\\
            &= x_{n+1}
    \end{align*}
    Therefore, since the sequence is monotone increasing and bounded above, the sequence is convergent. Specifically, we know that $(x_{n})_n \rightarrow x$ and $(x_{n+1})_n \rightarrow x$, so
    \begin{align*}
      x &= \sqrt{2 + x}\\
      x^2 - x - 2 &= 0\\
      x &= 2
    \end{align*}
  \end{problem}
  \begin{problem}{Problem 2}
    Does the following sequence converge?
    \begin{align*}
      x_n := \sum_{k=n+1}^{2n}\frac{1}{k}
    \end{align*}
    \tcblower
    Listing out terms, we have:
    \begin{align*}
      x_1 &= \frac{1}{2} \geq \frac{1}{2}\\
      x_2 &= \frac{1}{3} + \frac{1}{4} \geq x_1\\
      x_3 &= \frac{1}{4} + \frac{1}{5} + \frac{1}{6} \geq x_2\\
      x_4 &= \frac{1}{5} + \frac{1}{6} + \frac{1}{7} + \frac{1}{8} \geq x_3
    \end{align*}
    Using the contractive test, we have:
    \begin{align*}
      \left|x_{n+1} - x+n\right| &\leq \rho \left|x_{n} - x_{n-1}\right|\\
      \left|\frac{1}{(2n+2)(2n+1)}\right| &\leq \rho \left|\frac{1}{(2n)(2n-1)}\right|\\
      \frac{(2n)(2n-1)}{(2n+2)(2n+1)} &\leq \rho
    \end{align*}
    Since $\rho$ cannot be constant, we have no constant of contraction, meaning the sequence is not Cauchy and thus not convergent.
  \end{problem}
  \begin{problem}{Problem 3}
    Let $(f_n)_n$ denote the Fibonacci sequence and let
    \begin{align*}
      x_n := \frac{f_{n+1}}{f_n}.
    \end{align*}
    Given that $(x_n)_n$ converges, find its limit.
    \tcblower
    \begin{align*}
      x_n &= \frac{f_{n+1}}{f_n}\\
          &= \frac{f_{n} + f_{n-1}}{f_n}\\
          &= 1 + \frac{f_{n-1}}{f_n}\\
          &= 1 + \frac{1}{\frac{f_{n}}{f_{n-1}}}\\
          &= 1 + \frac{1}{x_{n-1}}\\
          \shortintertext{As $(x_n)_n \rightarrow x$, $(x_{n-1})_n\rightarrow x$, so}
      x &= 1 + \frac{1}{x}\\
      x^2 - x - 1 &= 0\\
      x &= \frac{1 + \sqrt{5}}{2}
    \end{align*}
  \end{problem}
  \begin{problem}{Problem 4}
    If $(x_n)_n$ is an unbounded sequence of real numbers, show that there is a sequence $(x_{n_k})_k$ such that
    \begin{align*}
      \left(\frac{1}{x_{n_k}}\right)_k \xrightarrow{x\rightarrow\infty} 0.
    \end{align*}
    \tcblower
    Since $(x_n)_n$ is not bounded, it is not convergent --- as $(x_n)_n$ is not convergent, it is not Cauchy. Set $\varepsilon_0 = 2$. Then, $\exists n_1 > 1$ such that
    \begin{align*}
      |x_{n_1} - x_1| &\geq \varepsilon_0\\
      \shortintertext{Similarly, $\exists n_2 > 0$ such that}
      |x_{n_2} - x_{n_1}| &\geq \varepsilon_0\\
      \shortintertext{Inductively, we have that $\forall k\in\N,\exists n_k > n_{k-1}$ such that}
      |x_{n_k} - x_{n_{k-1}}| &\geq \varepsilon_0
      \shortintertext{Therefore, $\left|x_{n_k}\right| > \left|x_{n_{k-1}}\right|$, so}
      \left(\frac{1}{\left|x_{n_k}\right|}\right)_{k} &\rightarrow 0
    \end{align*}
    as the given subsequence is monotone decreasing bounded below by zero.
  \end{problem}
  \begin{problem}{Problem 5}
    Suppose that every subsequence of a sequence $(x_n)_n$ has a subsequence that converges to 0. Show that $(x_n)_n \rightarrow 0$.
    \tcblower
    Every subsequence of $(x_n)_n$ has a subsequence that converges to zero --- in particular, this means the sequence $(x_n)_n$ has a subsequence that converges to zero.
    \begin{align*}
      (x_{n_{k}})_k &= (x_{n_1},x_{n_2},\dots)\\
      \shortintertext{consider the subsequence excluding this sequence:}
      (x_{n})_n \setminus (x_{n_k})_k &= (x_1,x_2,\dots,x_{n_1-1},x_{n_1+1},\dots)\\
      \shortintertext{This subsequence admits a subsequence that converges to zero:}
      (x_{m_k})_k &= (x_{m_1},x_{m_2},\dots)
    \end{align*}
    Inductively, via this process, we can assign every element of $(x_n)_n$ to a subsequence that converges to zero; therefore, $(x_n)_n$ must converge to zero.
  \end{problem}
  \begin{problem}{Problem 6}
    Let $(I_n)_n$ be a nested sequence of closed and bounded intervals. For each $n\in\N$ let $x_n\in I_n$. Use the Bolzano-Weierstrass Theorem for the sequence $(x_n)_n$ to give a proof of the Nested Intervals Property.
    \tcblower
    We have that
    \begin{align*}
      x_1 &\in [a_1,b_1]\\
      x_2 &\in [a_2,b_2]\subseteq [a_1,b_1]\\
       &\vdots \\
      x_n&\in [a_n,b_n] \subseteq [a_1,b_1]
    \end{align*}
    So, as $a_1\leq x_n\leq b_1$, we have that $\exists n_k$ such that
    \begin{align*}
      \left(x_{n_k}\right)_k \rightarrow x
    \end{align*}
    for some $x$. However, as $k\rightarrow\infty$, this means $n\rightarrow\infty$, or equivalently, we have that
    \begin{align*}
      x\in\bigcap_{n=1}^{\infty} I_n,
    \end{align*}
    meaning $\bigcap_{n=1}^{\infty}I_n\neq\emptyset$.
  \end{problem}
  \begin{problem}{Problem 7}
    If $(x_n)_n$ is a bounded sequence and $s:= \sup_{n}{x_n}$ such that $s\notin \{x_n\mid n\geq 1\}$, show that there is a subsequence $\left(x_{n_k}\right)_k$ that converges to $s$.
    \tcblower
    We know that since $(x_n)_n$ is bounded, $\exists (n_k)_k$ such that $\left(x_{n_k}\right)_k$ is monotone.\\

    Suppose that $\left(x_{n_k}\right)_k$ is monotone decreasing. This means $\left(x_{n_k}\right)_k\rightarrow \inf\left(x_{n_k}\right)$. However, this means $x_{n_1} \leq x_{n_2}\leq \cdots$, meaning any upper bound on this sequence must be in the set of $x_n$, including the supremum.\\

    Therefore, the monotone subsequence must be a monotone increasing subsequence. Let $\varepsilon > 0$. By the definition of supremum, $\exists n'\in \N$ such that $x_{n'} > s - \varepsilon$, and 
    \begin{align*}
      s \geq \cdots \geq x_{n'+1} \geq x_{n'} \geq s - \varepsilon
    \end{align*}
    We let our subsequence be the $n'$ tail of $x_n$, and we have found a subsequence that converges to $s$.
  \end{problem}
  \begin{problem}{Problem 8}
    Let $(x_n)_n$ and $(y_n)_n$ be bounded sequences. Show that
    \begin{align*}
      \limsup \left(\left(x_n + y_n\right)_n\right) \leq \limsup\left(x_n\right)_n + \limsup\left(y_n\right)_n
    \end{align*}
    \tcblower
    We have shown that
    \begin{align*}
      \sup(x_n + y_n)_n &\leq \sup(x_n)_n + \sup(y_n)_n
    \end{align*}
    Additionally, we know that the sequences of these suprema are related by the following:
    \begin{align*}
      \left(\sup_{n\geq m}\left(x_n + y_n\right)_n\right)_m &\leq \left(\sup_{n\geq m}\left(x_n\right)_n + \sup_{n\geq m}\left(y_n)_n\right)\right)_m\\
                                                            &= \left(\sup_{n\geq m}\left(x_n\right)_n\right)_m + \left(\sup_{n\geq m}\left(y_n\right)_n\right)_m
    \end{align*}
    Therefore,
    \begin{align*}
      \inf_m\left(\sup_{n\geq m}\left(x_n + y_n\right)_n\right)_m &\leq \inf_m\left(\sup_{n\geq m}\left(x_n\right)_n\right)_m + \inf_m\left(\sup_{n\geq m}\left(y_n\right)_n\right)_m
    \end{align*}
    So
    \begin{align*}
      \limsup\left(x_n + y_n\right)_n &\leq \limsup\left(x_n\right)_n + \limsup\left(y_n\right)_n
    \end{align*}
  \end{problem}
  \begin{problem}{Problem 9}
    Let $(x_n)_n$ be a bounded sequence. Show that
    \begin{align*}
      \liminf(x_n)_n &= \sup \left\{t\mid t\in\R,~\{n\mid x_{n}<t\}\text{ is finite} \right\}
    \end{align*}
    \tcblower
    By the definition of $\liminf(x_n)_n$, we have
    \begin{align*}
      \liminf(x_n)_n &= \sup{m\geq 1}\left(\inf_{n\geq m}(x_n)_n\right)_m,\\
      \shortintertext{The parenthetical on the right is equivalent to:}
      \inf_{n\geq m}(x_n)_n &= \max\left\{t \mid x_{n} \geq t\right\} \tag*{$\forall n\geq m$}\\
      \shortintertext{or, alternatively}
                            &= \sup\left\{t \mid x_{n} < t\right\} \tag*{$\forall n < m$}\\
                            \shortintertext{which is equivalent to the following statement}
      \liminf(x_n)_n &= \sup\left\{t\mid \{n\mid x_{n} < t\}\text{ is finite}\right\}
    \end{align*}
  \end{problem}
  \begin{problem}{Problem 10}
    Let $(x_n)_n$ be a bounded sequence. Show that
    \begin{align*}
      \liminf(-x_n)_n &= -\limsup(x_n)_n
    \end{align*}
    \tcblower
    \begin{align*}
      \liminf(-x_n)_n &= \sup_{m\geq 1}\left(\inf_{n\geq m}(-x_n)_n\right)_m\\
                      &= \sup_{m\geq 1}\left(-\sup_{n\geq m}(x_n)_n\right)_m\\
                      &= -\inf_{m\geq 1}\left(\sup_{n\geq m}(x_n)_n\right)_m\\
                      &=-\limsup(x_n)_n
    \end{align*}
  \end{problem}
\end{document}
