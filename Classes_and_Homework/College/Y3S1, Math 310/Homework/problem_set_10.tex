\documentclass[8pt]{extarticle}
\title{}
\author{}
\date{}
\usepackage[shortlabels]{enumitem}


%paper setup
\usepackage{geometry}
\geometry{letterpaper, portrait, margin=1in}
\usepackage{fancyhdr}
% sans serif font:
\usepackage{cmbright}
%symbols
\usepackage{amsmath}
\usepackage{bigints}
\usepackage{amssymb}
\usepackage{amsthm}
\usepackage{mathtools}
\usepackage{bbm}
\usepackage[hidelinks]{hyperref}
\usepackage{gensymb}
\usepackage{multirow,array}
\usepackage{multicol}

\newtheorem*{remark}{Remark}
\usepackage[T1]{fontenc}
\usepackage[utf8]{inputenc}

%chemistry stuff
%\usepackage[version=4]{mhchem}
%\usepackage{chemfig}

%plotting
\usepackage{pgfplots}
\usepackage{tikz}
\tikzset{middleweight/.style={pos = 0.5}}
%\tikzset{weight/.style={pos = 0.5, fill = white}}
%\tikzset{lateweight/.style={pos = 0.75, fill = white}}
%\tikzset{earlyweight/.style={pos = 0.25, fill=white}}

%\usepackage{natbib}

%graphics stuff
\usepackage{graphicx}
\graphicspath{ {./images/} }
\usepackage[style=numeric, backend=biber]{biblatex} % Use the numeric style for Vancouver
\addbibresource{the_bibliography.bib}
%code stuff
%when using minted, make sure to add the -shell-escape flag
%you can use lstlisting if you don't want to use minted
%\usepackage{minted}
%\usemintedstyle{pastie}
%\newminted[javacode]{java}{frame=lines,framesep=2mm,linenos=true,fontsize=\footnotesize,tabsize=3,autogobble,}
%\newminted[cppcode]{cpp}{frame=lines,framesep=2mm,linenos=true,fontsize=\footnotesize,tabsize=3,autogobble,}

%\usepackage{listings}
%\usepackage{color}
%\definecolor{dkgreen}{rgb}{0,0.6,0}
%\definecolor{gray}{rgb}{0.5,0.5,0.5}
%\definecolor{mauve}{rgb}{0.58,0,0.82}
%
%\lstset{frame=tb,
%	language=Java,
%	aboveskip=3mm,
%	belowskip=3mm,
%	showstringspaces=false,
%	columns=flexible,
%	basicstyle={\small\ttfamily},
%	numbers=none,
%	numberstyle=\tiny\color{gray},
%	keywordstyle=\color{blue},
%	commentstyle=\color{dkgreen},
%	stringstyle=\color{mauve},
%	breaklines=true,
%	breakatwhitespace=true,
%	tabsize=3
%}
% text + color boxes
\renewcommand{\mathbf}[1]{\mathbbm{#1}}
\usepackage[most]{tcolorbox}
\tcbuselibrary{breakable}
\tcbuselibrary{skins}
\newtcolorbox{problem}[1]{colback=white,enhanced,title={\small #1},
          attach boxed title to top center=
{yshift=-\tcboxedtitleheight/2},
boxed title style={size=small,colback=black!60!white}, sharp corners, breakable}
%including PDFs
%\usepackage{pdfpages}
\setlength{\parindent}{0pt}
\usepackage{cancel}
\pagestyle{fancy}
\fancyhf{}
\rhead{Avinash Iyer}
\lhead{Math 310: Problem Set 10}
\newcommand{\card}{\text{card}}
\newcommand{\ran}{\text{ran}}
\newcommand{\N}{\mathbbm{N}}
\newcommand{\Q}{\mathbbm{Q}}
\newcommand{\Z}{\mathbbm{Z}}
\newcommand{\R}{\mathbbm{R}}
\setcounter{secnumdepth}{0}
\begin{document}
  \begin{problem}{Problem 1}
    Using the definition of the derivative find $f'(c)$ where $c\in\R$ and $f(x) = \frac{1}{x}$.
    \tcblower
    \begin{align*}
      f'(c) &= \lim_{x\rightarrow c}\frac{\frac{1}{x}-\frac{1}{c}}{x-c}\\
            &= \lim_{x\rightarrow c}\frac{c-x}{(xc)(x-c)}\\
            &= \lim_{x\rightarrow c}\frac{-1}{xc}\\
            &= -\frac{1}{x^2}\tag*{$c\neq 0$}
    \end{align*}
  \end{problem}
  \begin{problem}{Problem 2}
    Let $n\in\N$ and consider the function
    \begin{align*}
      f(x) &= \begin{cases}
        x^n,&x > 0\\
        0,&x \leq 0
      \end{cases}.
    \end{align*}
    For which values of $n$ is $f$ differentiable at $x=0$.
    \tcblower
    We have that on $(0,\infty)$, $f(x) = x^n$, meaning $f'(x)$ on $(0,\infty)$ is $nx^{n-1}$. Therefore, as $(x_n)_n\rightarrow 0$ for $x_n\in (0,\infty)$, $\left(\frac{f(x_n) - f(0)}{x_n - 0}\right)_n \rightarrow 0$, taking $f(0)$ as given above, assuming $n > 1$ --- otherwise, $\lim_{x\rightarrow 0^{+}}\frac{f(x)-f(0)}{x-0} = 1$.
  \end{problem}
  \begin{problem}{Problem 3}
    Consider the function
    \begin{align*}
      f(x) &= \begin{cases}
        x^2,&x\in \Q\\
        0,&x\notin \Q
      \end{cases}.
    \end{align*}
    Show that $f$ is differentiable at $x=0$ and find $f'(0)$.
    \tcblower
    Let $(x_n)_n\rightarrow 0$, $x_n\neq 0$. Let $(x_{n_k})_k$ denote the sequence of irrational values of $x_n$, and let $(x_{m_l})_l$ denote the sequence of rational values of $x_n$. Then, $(f(x_n))_n \rightarrow 0$, regardless of whether $x_n \in (x_{m_l})_l$ or $x_n \in (x_{n_k})_k$. So, having established that the limit exists, we find that
    \begin{align*}
      f'(0) &= \lim_{x\rightarrow 0}\frac{x^2 - 0^2}{x-0}\\
            &= \lim_{x\rightarrow 0}x \\
            &= 0
    \end{align*}
  \end{problem}
\end{document}
