\documentclass[8pt]{extarticle}
\title{}
\author{}
\date{}
\usepackage[shortlabels]{enumitem}


%paper setup
\usepackage{geometry}
\geometry{letterpaper, portrait, margin=1in}
\usepackage{fancyhdr}
% sans serif font:
\usepackage{cmbright}
%symbols
\usepackage{amsmath}
\usepackage{bigints}
\usepackage{amssymb}
\usepackage{amsthm}
\usepackage{mathtools}
\usepackage{bbm}
\usepackage[hidelinks]{hyperref}
\usepackage{gensymb}
\usepackage{multirow,array}
\usepackage{multicol}

\newtheorem*{remark}{Remark}
\usepackage[T1]{fontenc}
\usepackage[utf8]{inputenc}

%chemistry stuff
%\usepackage[version=4]{mhchem}
%\usepackage{chemfig}

%plotting
\usepackage{pgfplots}
\usepackage{tikz}
\tikzset{middleweight/.style={pos = 0.5}}
%\tikzset{weight/.style={pos = 0.5, fill = white}}
%\tikzset{lateweight/.style={pos = 0.75, fill = white}}
%\tikzset{earlyweight/.style={pos = 0.25, fill=white}}

%\usepackage{natbib}

%graphics stuff
\usepackage{graphicx}
\graphicspath{ {./images/} }
\usepackage[style=numeric, backend=biber]{biblatex} % Use the numeric style for Vancouver
\addbibresource{the_bibliography.bib}
%code stuff
%when using minted, make sure to add the -shell-escape flag
%you can use lstlisting if you don't want to use minted
%\usepackage{minted}
%\usemintedstyle{pastie}
%\newminted[javacode]{java}{frame=lines,framesep=2mm,linenos=true,fontsize=\footnotesize,tabsize=3,autogobble,}
%\newminted[cppcode]{cpp}{frame=lines,framesep=2mm,linenos=true,fontsize=\footnotesize,tabsize=3,autogobble,}

%\usepackage{listings}
%\usepackage{color}
%\definecolor{dkgreen}{rgb}{0,0.6,0}
%\definecolor{gray}{rgb}{0.5,0.5,0.5}
%\definecolor{mauve}{rgb}{0.58,0,0.82}
%
%\lstset{frame=tb,
%	language=Java,
%	aboveskip=3mm,
%	belowskip=3mm,
%	showstringspaces=false,
%	columns=flexible,
%	basicstyle={\small\ttfamily},
%	numbers=none,
%	numberstyle=\tiny\color{gray},
%	keywordstyle=\color{blue},
%	commentstyle=\color{dkgreen},
%	stringstyle=\color{mauve},
%	breaklines=true,
%	breakatwhitespace=true,
%	tabsize=3
%}
% text + color boxes
\renewcommand{\mathbf}[1]{\mathbbm{#1}}
\usepackage[most]{tcolorbox}
\tcbuselibrary{breakable}
\tcbuselibrary{skins}
\newtcolorbox{problem}[1]{colback=white,enhanced,title={\small #1},
          attach boxed title to top center=
{yshift=-\tcboxedtitleheight/2},
boxed title style={size=small,colback=black!60!white}, sharp corners, breakable}
%including PDFs
%\usepackage{pdfpages}
\setlength{\parindent}{0pt}
\usepackage{cancel}
\pagestyle{fancy}
\fancyhf{}
\rhead{Avinash Iyer}
\lhead{Math 310: Problem Set 10}
\newcommand{\card}{\text{card}}
\newcommand{\ran}{\text{ran}}
\newcommand{\N}{\mathbbm{N}}
\newcommand{\Q}{\mathbbm{Q}}
\newcommand{\Z}{\mathbbm{Z}}
\newcommand{\R}{\mathbbm{R}}
\setcounter{secnumdepth}{0}
\begin{document}
  \begin{problem}{Problem 1}
    Using the definition of the derivative find $f'(c)$ where $c\in\R$ and $f(x) = \frac{1}{x}$.
    \tcblower
    \begin{align*}
      f'(c) &= \lim_{x\rightarrow c}\frac{\frac{1}{x}-\frac{1}{c}}{x-c}\\
            &= \lim_{x\rightarrow c}\frac{c-x}{(xc)(x-c)}\\
            &= \lim_{x\rightarrow c}\frac{-1}{xc}\\
            &= -\frac{1}{x^2}\tag*{$c\neq 0$}
    \end{align*}
  \end{problem}
  \begin{problem}{Problem 2}
    Let $n\in\N$ and consider the function
    \begin{align*}
      f(x) &= \begin{cases}
        x^n,&x > 0\\
        0,&x \leq 0
      \end{cases}.
    \end{align*}
    For which values of $n$ is $f$ differentiable at $x=0$.
    \tcblower
    We have that on $(0,\infty)$, $f(x) = x^n$, meaning $f'(x)$ on $(0,\infty)$ is $nx^{n-1}$. Therefore, as $(x_n)_n\rightarrow 0$ for $x_n\in (0,\infty)$, $\left(\frac{f(x_n) - f(0)}{x_n - 0}\right)_n \rightarrow 0$, taking $f(0)$ as given above, assuming $n > 1$ --- otherwise, $\lim_{x\rightarrow 0^{+}}\frac{f(x)-f(0)}{x-0} = 1$.
  \end{problem}
  \begin{problem}{Problem 3}
    Consider the function
    \begin{align*}
      f(x) &= \begin{cases}
        x^2,&x\in \Q\\
        0,&x\notin \Q
      \end{cases}.
    \end{align*}
    Show that $f$ is differentiable at $x=0$ and find $f'(0)$.
    \tcblower
    Let $(x_n)_n\rightarrow 0$, $x_n\neq 0$. Let $(x_{n_k})_k$ denote the sequence of irrational values of $x_n$, and let $(x_{m_l})_l$ denote the sequence of rational values of $x_n$. Then, $(f(x_n))_n \rightarrow 0$, regardless of whether $x_n \in (x_{m_l})_l$ or $x_n \in (x_{n_k})_k$. So, having established that the limit exists, we find that
    \begin{align*}
      f'(0) &= \lim_{x\rightarrow 0}\frac{x^2 - 0^2}{x-0}\\
            &= \lim_{x\rightarrow 0}x \\
            &= 0
    \end{align*}
  \end{problem}
  \begin{problem}{Problem 4}
    Determine the values of $x$ where $f(x) = x|x|$ is differentiable.
    \tcblower
    We can see that $f(x) = x|x|$ is equivalent to
    \begin{align*}
      f(x) &= \begin{cases}
        x^2,&x \geq 0\\
        -x^2,&x < 0
      \end{cases}.
    \end{align*}
    Since $x^2$ and $-x^2$ are polynomials, we have that for $c < 0$, $f$ is differentiable, as we evaluate $\frac{d}{dx}(-x^2)\big\vert_{c}$ and for $c > 0$, $f$ is also differentiable by evaluating $\frac{d}{dx}(x^2)\big\vert_{c}$.\\

    At $x=0$, we have to evaluate the left-hand and right-hand limits
    \begin{align*}
      f'(0)^{+} &= \lim_{x\rightarrow 0^{+}}\frac{f(x)-f(0)}{x-0}\\
                &= 0
      f'(0)^{-} &= \lim_{x\rightarrow 0^{-}}\frac{f(x)-f(0)}{x-0}\\
                &= 0.
    \end{align*}
    Since the left and right-hand derivatives agree with each other, it is the case that $f$ is differentiable at $x=0$, meaning $f(x) = x|x|$ is differentiable on $\R$.
  \end{problem}
  \begin{problem}{Problem 5}
    Let $I$ be an interval and suppose $f: I\rightarrow \R$ is differentiable with $f'(x) < 0$ for all $x\in I$. Show that $f$ is strictly decreasing on $I$.
    \tcblower
    By a lemma, we know that for $c\in I$ and $f'(c) < 0$, it must be the case that $\exists \delta$ such that for all $x\in (c-\delta,c)$, $f(c) < f(x)$. Since this is the case for all $c\in I$, $f$ is strictly decreasing.
  \end{problem}
  \begin{problem}{Problem 6}
    Prove that $f(x) = x^3 + e^x$ has a unique real root.
    \tcblower
    We know that for $x = -1$, $f(x) < 0$, and for $x = 1$, $f(x) > 0$. By the Intermediate Value Theorem, it must be the case that $\exists c\in [-1,1]$ such that $f(c) =0$. Additionally, it is also the case that $f'(x) = 3x^2 + e^x > 0~\forall x$, meaning that $f(x)$ is strictly increasing on its domain, so $f$ cannot take the value of $0$ at any other point $d^{\ast}$, otherwise there would be a point where $f'(k) = 0$ for some $k$ between $c$ and $d$.
  \end{problem}
  \begin{problem}{Problem 7}
    Suppose $f: [0,2]\rightarrow \R$ is continuous on $[0,2]$ and differentiable on $(0,2)$, and satisfies $f(0) = 0$, $f(1) = 1$, and $f(2) = 1$.
    \tcblower
    \begin{problem}{(i)}
      Show that there is a $c_1\in (0,1)$ with $f'(c_1) = 1$.
      \tcblower
      Since $f$ is continuous on $[0,2]$, $f$ is continuous on $[0,1]$, and since $f$ is differentiable on $(0,2)$, $f$ is differentiable on $(0,1)$. We apply the mean value theorem on $[0,1]$ to find $c_1^{\ast}$. 
    \end{problem}
    \begin{problem}{(ii)}
      Show that there is a $c_2\in (1,2)$ with $f'(c_2) = 0$.
      \tcblower
      Since $f$ is continuous on $[0,2]$, $f$ is continuous on $[1,2]$, and since $f$ is differentiable on $(0,2)$, $f$ is differentiable on $(1,2)$. Apply Rolle's Theorem on $[1,2]$ to find $c_2$.
    \end{problem}
    \begin{problem}{(iii)}
      Show that there is a $c_3\in (0,2)$ with $f'(c_3) = 1/3$.
      \tcblower
      Letting $c_1\in (0,1)$ and $c_2\in (1,2)$ be defined as above, we apply Darboux's Theorem on $[c_1,c_2]$ to find $c_3$ such that $f'(c_3) = 1/3$.
    \end{problem}
  \end{problem}
  \begin{problem}{Problem 8}
    Suppose $f,g: \R\rightarrow (0,\infty)$ are everywhere differentiable with $f' = f$ and $g' = g$. Prove that $f = \alpha g$ for some constant $\alpha > 0$.
    \tcblower
    \begin{align*}
      f &= \alpha g\\
      f' &= (\alpha g)'\\
         &= \alpha g'\\
         &= \alpha g\\
         &= f
    \end{align*}
  \end{problem}
  \begin{problem}{Problem 9}
    Let $h = \mathbbm{1}_{[0,\infty)}$. Prove that there does not exist a function $f: \R\rightarrow\R$ for which $f' = h$ on $\R$.
    \tcblower
    Since $h$ is discontinuous at $x=0$, $f$ must be non-differentiable at $x-0$; however, since $h$ takes a value at $x=0$, it must also be the case that $f$ is differentiable at $x=0$. $\bot$
  \end{problem}
  \begin{problem}{Problem 10}
    Let $s > t > 0$ and $n \geq 2$. By analyzing the function $f(x) = x^{1/n} - (x-1)^{1/n}$ on $[1,\infty)$, show that
    \begin{align*}
      s^{1/n} - t^{1/n} &< (s-t)^{1/n}
    \end{align*}
    \tcblower
    \begin{align*}
      s^{1/n} - t^{1/n} &< (s-t)^{1/n}\\
      \left(\frac{s}{t}\right)^{1/n}-1 &< \left(\frac{s}{t}-1\right)^{1/n}\\
      \left(\frac{s}{t}\right)^{1/n}-\left(\frac{s}{t}-1\right)^{1/n} &< 1,\\
      \shortintertext{and}
      f'(x) &= \frac{1}{n}\left(\frac{1}{x^{1/n}}-\frac{1}{(x-1)^{1/n}}\right)\\
            &= \frac{1}{n}\left(\frac{(x-1)^{1/n}-x^{1/n}}{x^{1/n}(x-1)^{1/n}}\right)\\
            &< 0,\\
            \shortintertext{and}
      f(1) &= 1,\\
      \shortintertext{so,}
      f\left(\frac{s}{t}\right)&< 1
    \end{align*}
  \end{problem}
  \begin{problem}{Problem 11}
    Show that for all $x > 0$,
    \begin{align*}
      1 + \frac{1}{2}x - \frac{1}{8}x^2 \leq \sqrt{1+x} \leq 1 + \frac{1}{2}x
    \end{align*}
    \tcblower
    Apply the Mean value theorem on $[0,x]$: $\exists c \in (0,x)$ such that
    \begin{align*}
      \frac{\sqrt{1+x}-1}{x} &=\frac{1}{2\sqrt{1+c}}\\
      \sqrt{1+x}-1 &= \frac{1}{2\sqrt{1+c}}x\\
                   &\leq \frac{1}{2}x\tag*{$c \geq 0$}\\
      \sqrt{1+x} &\leq 1 + \frac{1}{2}x.
    \end{align*}
    I don't know how to show the second part.
  \end{problem}
\end{document}
