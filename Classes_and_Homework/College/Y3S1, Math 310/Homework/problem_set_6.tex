\documentclass[8pt]{extarticle}
\title{}
\author{Avinash Iyer}
\date{}
\usepackage[shortlabels]{enumitem}


%paper setup
\usepackage{geometry}
\geometry{letterpaper, portrait, margin=1in}
\usepackage{fancyhdr}
% sans serif font:
\usepackage{cmbright}
%symbols
\usepackage{amsmath}
\usepackage{amssymb}
\usepackage{amsthm}
\usepackage{mathtools}
\usepackage{hyperref}
\usepackage{gensymb}
\usepackage{multirow,array}
\usepackage{multicol}

\newtheorem*{remark}{Remark}
\usepackage[T1]{fontenc}
\usepackage[utf8]{inputenc}

%chemistry stuff
%\usepackage[version=4]{mhchem}
%\usepackage{chemfig}

%plotting
\usepackage{pgfplots}
\usepackage{tikz}
\tikzset{middleweight/.style={pos = 0.5, fill=white}}
\tikzset{weight/.style={pos = 0.5, fill = white}}
\tikzset{lateweight/.style={pos = 0.75, fill = white}}
\tikzset{earlyweight/.style={pos = 0.25, fill=white}}

%\usepackage{natbib}

%graphics stuff
\usepackage{graphicx}
\graphicspath{ {./images/} }
\usepackage[style=numeric, backend=biber]{biblatex} % Use the numeric style for Vancouver
\addbibresource{the_bibliography.bib}
%code stuff
%when using minted, make sure to add the -shell-escape flag
%you can use lstlisting if you don't want to use minted
%\usepackage{minted}
%\usemintedstyle{pastie}
%\newminted[javacode]{java}{frame=lines,framesep=2mm,linenos=true,fontsize=\footnotesize,tabsize=3,autogobble,}
%\newminted[cppcode]{cpp}{frame=lines,framesep=2mm,linenos=true,fontsize=\footnotesize,tabsize=3,autogobble,}

%\usepackage{listings}
%\usepackage{color}
%\definecolor{dkgreen}{rgb}{0,0.6,0}
%\definecolor{gray}{rgb}{0.5,0.5,0.5}
%\definecolor{mauve}{rgb}{0.58,0,0.82}
%
%\lstset{frame=tb,
%	language=Java,
%	aboveskip=3mm,
%	belowskip=3mm,
%	showstringspaces=false,
%	columns=flexible,
%	basicstyle={\small\ttfamily},
%	numbers=none,
%	numberstyle=\tiny\color{gray},
%	keywordstyle=\color{blue},
%	commentstyle=\color{dkgreen},
%	stringstyle=\color{mauve},
%	breaklines=true,
%	breakatwhitespace=true,
%	tabsize=3
%}
% text + color boxes
\usepackage[most]{tcolorbox}
\tcbuselibrary{breakable}
\tcbuselibrary{skins}
\newtcolorbox{problem}[1]{colback=white,enhanced,title={\small #1},
          attach boxed title to top center=
{yshift=-\tcboxedtitleheight/2},
boxed title style={size=small,colback=black!60!white}, breakable}
%including PDFs
%\usepackage{pdfpages}
\setlength{\parindent}{0pt}
\usepackage{cancel}
\pagestyle{fancy}
\fancyhf{}
\rhead{Avinash Iyer}
\lhead{Math 310: Problem Set 6}
\newcommand{\card}{\text{card}}
\newcommand{\ran}{\text{ran}}
\newcommand{\N}{\mathbb{N}}
\newcommand{\Q}{\mathbb{Q}}
\newcommand{\Z}{\mathbb{Z}}
\newcommand{\R}{\mathbb{R}}
\begin{document}
  \begin{problem}{Problem 1}
    Let $(x_k)_k$ be a sequence of strictly positive numbers such that
    \begin{align*}
      (kx_k)_k \rightarrow L > 0.
    \end{align*}
    Show that $\sum_k x_k$ diverges.
    \tcblower
    Since $(kx_k)_k\rightarrow L$, every subsequence of $(kx_k)_k$ converges to $L$. Let $n_k = 2^k$. Then,
    \begin{align*}
      (2^kx_{2^k})_k \rightarrow L > 0,\\
      \shortintertext{implying that}
      \sum_{k}2^kx_{2^k} = \infty.
    \end{align*}
    By the Cauchy Condensation test, this implies that $\sum_k x_k$ diverges.
  \end{problem}
  \begin{problem}{Problem 2}
    Let $(x_k)_k$ be a sequence of strictly positive numbers. Show the following:
    \begin{enumerate}[(i)]
      \item If $\limsup_{n\rightarrow\infty}\frac{x_{k+1}}{x_k} < 1$, then $\sum_{k}x_k$ converges.
      \item If $\liminf_{n\rightarrow\infty}\frac{x_{k+1}}{x_k} > 1$, then $\sum_k x_k$ diverges.
    \end{enumerate}
  \end{problem}
  \begin{problem}{Problem 3}
    Consider the sequence of functions
    \begin{align*}
      f_n: R\rightarrow \R;\\
      f_n(x) = \arctan(nx)
    \end{align*}
    \begin{enumerate}[(i)]
      \item Show that $(f_n)_n \rightarrow \frac{\pi}{2}$
    \end{enumerate}
  \end{problem}
\end{document}
