\documentclass[8pt]{extarticle}
\title{}
\author{Avinash Iyer}
\date{}
\usepackage[shortlabels]{enumitem}


%paper setup
\usepackage{geometry}
\geometry{letterpaper, portrait, margin=1in}
\usepackage{fancyhdr}
% sans serif font:
\usepackage{cmbright}
%symbols
\usepackage{amsmath}
\usepackage{amssymb}
\usepackage{amsthm}
\usepackage{mathtools}
\usepackage{hyperref}
\usepackage{gensymb}
\usepackage{multirow,array}
\usepackage{multicol}

\newtheorem*{remark}{Remark}
\usepackage[T1]{fontenc}
\usepackage[utf8]{inputenc}

%chemistry stuff
%\usepackage[version=4]{mhchem}
%\usepackage{chemfig}

%plotting
\usepackage{pgfplots}
\usepackage{tikz}
\tikzset{middleweight/.style={pos = 0.5, fill=white}}
\tikzset{weight/.style={pos = 0.5, fill = white}}
\tikzset{lateweight/.style={pos = 0.75, fill = white}}
\tikzset{earlyweight/.style={pos = 0.25, fill=white}}

%\usepackage{natbib}

%graphics stuff
\usepackage{graphicx}
\graphicspath{ {./images/} }
\usepackage[style=numeric, backend=biber]{biblatex} % Use the numeric style for Vancouver
\addbibresource{the_bibliography.bib}
%code stuff
%when using minted, make sure to add the -shell-escape flag
%you can use lstlisting if you don't want to use minted
%\usepackage{minted}
%\usemintedstyle{pastie}
%\newminted[javacode]{java}{frame=lines,framesep=2mm,linenos=true,fontsize=\footnotesize,tabsize=3,autogobble,}
%\newminted[cppcode]{cpp}{frame=lines,framesep=2mm,linenos=true,fontsize=\footnotesize,tabsize=3,autogobble,}

%\usepackage{listings}
%\usepackage{color}
%\definecolor{dkgreen}{rgb}{0,0.6,0}
%\definecolor{gray}{rgb}{0.5,0.5,0.5}
%\definecolor{mauve}{rgb}{0.58,0,0.82}
%
%\lstset{frame=tb,
%	language=Java,
%	aboveskip=3mm,
%	belowskip=3mm,
%	showstringspaces=false,
%	columns=flexible,
%	basicstyle={\small\ttfamily},
%	numbers=none,
%	numberstyle=\tiny\color{gray},
%	keywordstyle=\color{blue},
%	commentstyle=\color{dkgreen},
%	stringstyle=\color{mauve},
%	breaklines=true,
%	breakatwhitespace=true,
%	tabsize=3
%}
% text + color boxes
\renewcommand{\mathbf}[1]{\mathbold{#1}}
\usepackage[most]{tcolorbox}
\tcbuselibrary{breakable}
\tcbuselibrary{skins}
\newtcolorbox{problem}[1]{colback=white,enhanced,title={\small #1},
          attach boxed title to top center=
{yshift=-\tcboxedtitleheight/2},
boxed title style={size=small,colback=black!60!white}, breakable}
%including PDFs
%\usepackage{pdfpages}
\setlength{\parindent}{0pt}
\usepackage{cancel}
\pagestyle{fancy}
\fancyhf{}
\rhead{Avinash Iyer}
\lhead{Math 310: Problem Set 6}
\newcommand{\card}{\text{card}}
\newcommand{\ran}{\text{ran}}
\newcommand{\N}{\mathbb{N}}
\newcommand{\Q}{\mathbb{Q}}
\newcommand{\Z}{\mathbb{Z}}
\newcommand{\R}{\mathbb{R}}
\begin{document}
  \begin{problem}{Problem 1}
    Let $(x_k)_k$ be a sequence of strictly positive numbers such that
    \begin{align*}
      (kx_k)_k \rightarrow L > 0.
    \end{align*}
    Show that $\sum_k x_k$ diverges.
    \tcblower
    Since $(kx_k)_k\rightarrow L$, every subsequence of $(kx_k)_k$ converges to $L$. Let $n_k = 2^k$. Then,
    \begin{align*}
      (2^kx_{2^k})_k \rightarrow L > 0,\\
      \shortintertext{implying that}
      \sum_{k}2^kx_{2^k} = \infty.
    \end{align*}
    By the Cauchy Condensation test, this implies that $\sum_k x_k$ diverges.
  \end{problem}
  \begin{problem}{Problem 2}
    Let $(x_k)_k$ be a sequence of strictly positive numbers. Show the following:
    \begin{enumerate}[(i)]
      \item If $\limsup_{k\rightarrow\infty}\frac{x_{k+1}}{x_k} < 1$, then $\sum_{k}x_k$ converges.
      \item If $\liminf_{k\rightarrow\infty}\frac{x_{k+1}}{x_k} > 1$, then $\sum_k x_k$ diverges.
    \end{enumerate}
    \tcblower
    \begin{problem}{(a)}
      Let $\varepsilon > 0$.
      \begin{align*}
        \limsup_{k\rightarrow\infty} \frac{x_{k+1}}{x_k} &:= u < 1\\
                                                         &= \inf_{n\geq 1}\left(\sup_{k\geq n}\frac{x_{k+1}}{x_k}\right)\\
        \shortintertext{By the definition of $\inf$, we have that $\exists N\in N$ large such that}
        \sup_{k\geq N}\frac{x_{k+1}}{x_k} &< u + \varepsilon.\\
        \shortintertext{By the definition of $\sup$, we have that $\forall k \geq N$,}
        \frac{x_{k+1}}{x_k} &< u + \varepsilon\\
        x_{k+1} &< (u + \varepsilon) x_k.\\
        \shortintertext{Inductively on $x_k$, we have that}
        x_{k + m} &< (L + \varepsilon)^{m}x_k,\\
        \shortintertext{and series-wise, we have}
        \sum_{k=N}^{\infty}x_k &< x_N \sum_{m=1}^{\infty}(u + \varepsilon)^m.\\
        \shortintertext{For sufficiently small $\varepsilon$, the sum on the right-hand side converges, implying that the sum on the left-hand side must converge. Therefore,}
        \sum_{k=1}^{\infty}x_k &= \sum_{k=1}^{N-1}x_k + \sum_{k=N}^{\infty}x_k\\
                               &< \sum_{k=1}^{N-1}x_k + x_n\sum_{m=1}^{\infty}(u + \varepsilon)^m,
        \shortintertext{meaning that $\sum_k x_k$ is bounded above by a convergent series, so it is convergent.}
      \end{align*}
    \end{problem}
    \begin{problem}{(b)}
      Let $\varepsilon > 0$.
      \begin{align*}
        \liminf_{k\rightarrow\infty}\frac{x_{k+1}}{x_k} &:= \ell > 1\\
                                                        &= \sup_{n\geq 1}\left(\inf_{k\geq n}\frac{x_{k+1}}{x_k}\right)\\
        \shortintertext{By the definition of $\sup$, we have that for large $N\in\N$, and for $k\geq N$,}
        \inf_{k\geq n} \frac{x_{k+1}}{x_k} &> \ell - \varepsilon.\\
        \shortintertext{By the definition of $\inf$, we also have that}
        \frac{x_{k+1}}{x_k} &>\ell - \varepsilon\\
        x_{k+1} &> (\ell - \varepsilon)x_k\\
        \shortintertext{Inductively, we have that}
        x_{k+m} &> (\ell - \varepsilon)^m x_k,\\
        \shortintertext{and via series, we have}
        \sum_{k=N}^{\infty}x_k &>x_N \sum_{m=1}^{\infty}(\ell - \varepsilon)^m.
        \shortintertext{For sufficiently small $\varepsilon$, the sum on the right-hand side diverges. Therefore,}
        \sum_{k=1}^{\infty}x_k &= \sum_{k=1}^{N-1}x_k + \sum_{k=N}^{\infty}x_k\\
                               &> x_N\sum_{k=1}^{\infty}(\ell - \varepsilon)^m + \sum_{k=1}^{N-1}x_k,\\
        \shortintertext{and since $\sum_k x_k$ is bounded below by a divergent series, the sum diverges.}
      \end{align*}
    \end{problem}
  \end{problem}
  \begin{problem}{Problem 3}
    Consider the sequence of functions
    \begin{align*}
      f_n&: \R\rightarrow \R;\\
      f_n(x) &= \arctan(nx)
    \end{align*}
    \begin{enumerate}[(i)]
      \item Show that $(f_n)_n \rightarrow \frac{\pi}{2}\text{sgn}$ point-wise.
      \item Show that the convergence in (i) is nonuniform on $(0,\infty)$.
      \item Show that the convergence in (i) is uniform on $[a,\infty)$ for a fixed $a > 0$.
    \end{enumerate}
    \tcblower
    \begin{problem}{(i)}
      Let $\varepsilon > 0$. We know that, $\exists N \in N$ such that $\forall n \geq N,~|\arctan(n) - \pi/2| < \varepsilon$.
      \begin{description}
        \item[Case 1:] Let $x = 0$. Then,
          \begin{align*}
            \arctan(nx) &= 0\tag*{$\forall n \geq 1$}\\
          \end{align*}
        \item[Case 2:] Let $x > 0$. Then, set $N' = \lceil N/x\rceil$. So, for $n'\geq N'$, we have 
          \begin{align*}
            \left|\arctan(nx) - \pi/2\right| &= \left|\arctan(n') - \pi/2\right|\\
                                  &< \varepsilon
          \end{align*}
          implying that $\arctan(nx) \rightarrow \pi/2$ when $x > 0$.
        \item[Case 3:] Let $x < 0$. Then, set $x^{\ast} = -x$, and we have the same result as in Case 2, where $\arctan(nx^{\ast}) \rightarrow \pi/2$.\\

          Since $\arctan(nx^{\ast}) = \arctan(n(-x)) = -\arctan(nx)$, we have that $\arctan(nx) \rightarrow -\pi/2$.
      \end{description}
    \end{problem}
    \begin{problem}{(ii)}
      Let $(x_k)_k = \frac{1}{k}$ and $n_k = k$. Set $\varepsilon_0 = \frac{\pi}{4}$. Then, we have that
      \begin{align*}
        \left|\arctan(n_kx_k) - \pi/2\right| &= \left|\arctan\left(k\frac{1}{k}\right) - \frac{\pi}{2}\right|\\
                                             &= \left|\arctan(1)-\frac{\pi}{2}\right|\\
                                             &= \left|\frac{\pi}{4}-\frac{\pi}{2}\right|\\
                                             &=\frac{\pi}{4}\\
                                             &\geq \varepsilon_0.
      \end{align*}
    \end{problem}
    \begin{problem}{(iii)}
      Since $\forall x\in [a,\infty)$, $x > 0$, and $\arctan(na) \rightarrow \frac{\pi}{2}$, it must be the case that $\arctan(nx) \rightarrow \frac{\pi}{2}\text{sgn}$ for $x\in [a,\infty)$.
    \end{problem}
  \end{problem}
  \begin{problem}{Problem 4}
    Consider the sequence of functions
    \begin{align*}
      f_n&:[0,\infty) \rightarrow \R;\\
      f_n(x) &= \frac{\sin(nx)}{1 + nx}.
    \end{align*}
    \begin{enumerate}[(i)]
      \item Show that $(f_n)_n\rightarrow 0$ pointwise.
      \item Show that the convergence in (i) is nonuniform on $[0,\infty)$.
      \item Show that the convergence in (i) is uniform on $[a,\infty)$ for a fixed $a > 0$.
    \end{enumerate}
    \tcblower
    \begin{problem}{(i)}
      We know that $f_n(0) = 0~\forall n\in\N$. For all $x > 0$, we have:
      \begin{align*}
        \left|\frac{\sin(nx)}{1+nx} - \mathbf{0}(x)\right| &\leq \frac{1}{1 + nx}\\
                                                           &< \frac{1}{nx}\\
                                                           &\rightarrow 0.
      \end{align*}
      So,
      \begin{align*}
        f_n \xrightarrow{\text{p.w.}} \mathbf{0}.
      \end{align*}
    \end{problem}
    \begin{problem}{(ii)}
      Let $n_k = k$ and $x_k = \frac{\pi}{2k}$. Set $\varepsilon_0 = 1/4$. Then,
      \begin{align*}
        \left|f_{n_k}(x_k) - \mathbf{0}(x_k)\right| &= \frac{\sin\left(k\frac{\pi}{2k}\right)}{1 + k\frac{\pi}{2k}}\\
                                                    &= \frac{1}{1 + \frac{\pi}{2}}\\
                                                    &\geq \varepsilon_0
      \end{align*}
    \end{problem}
    \begin{problem}{(iii)}
      On $[a,\infty)$, we have
      \begin{align*}
        \left|\frac{\sin(nx)}{1 + nx} - \mathbf{0}(x)\right| &\leq \frac{1}{1 + nx}\\
                                                             &\leq \frac{1}{na}\\
        \sup\left|\frac{\sin(nx)}{1 + nx} - \mathbf{0}(x)\right| &\leq \frac{1}{na}\\
                                                             &\rightarrow 0
      \end{align*}
      So, $\frac{\sin(nx)}{1 + nx} \rightarrow \mathbf{0}$ on $[a,\infty)$ uniformly.
    \end{problem}
  \end{problem}
  \begin{problem}{Problem 5}
    Show that the sequence of functions
    \begin{align*}
      f_n&: [0,\infty) \rightarrow \R;\\
      f_n(x) &= x^2e^{-nx}
    \end{align*}
    converges uniformly to $0$.
    \tcblower
    We know that $\forall n\in\N$, $f_n(0) = 0$. Otherwise, we have that
    \begin{align*}
      \sup\left(x^2e^{-nx}\right) &\Rightarrow \frac{df_n}{dx} = 0\\
      2xe^{-nx} - nx^2e^{-nx} &= 0\\
      xe^{-nx}\left(2-nx\right) &= 0\\
      x &= \frac{2}{n}\\
      f(x) &= \frac{4}{n^2e^2}.\\
      \shortintertext{Additionally, we have}
      n^2 &\geq n\\
      \frac{e^2n^2}{4} &\geq \frac{e^2n}{4}\\
      \frac{4}{e^2n^2} &\leq \frac{4}{e^2n},\\
      \shortintertext{so,}
      \sup(x^2e^{-nx}) \rightarrow 0.
    \end{align*}
    Therefore, $f_n(x)$ converges to $0$ uniformly.
  \end{problem}
  \begin{problem}{Problem 6}
    Let $f_n = \mathbf{1}_{n,n+1}$. Show that $(f_n)_n \rightarrow \mathbf{0}$ pointwise on $\R$. Is the convergence uniform?
    \tcblower
    $\forall x\in\R$, find $N\in\N$ so large such that $x < N$, which is always true by the Archimedean property. Then, $|f_n(x) - \mathbf{0}(x)| = 0 < \varepsilon$.\\

    However, since $\sup(f_n) = 1~\forall n$, it must be the case that $(f_n)_n$ does not converge to $\mathbf{0}$ uniformly.
  \end{problem}
  \begin{problem}{Problem 7}
    Let $(f_n)_n$ and $(g_n)_n$ be sequences in $\ell_{\infty}(\Omega)$ with $(f_n)_n \rightarrow f$ and $(g_n)_n \rightarrow g$ uniformly on $\Omega$. Prove that $(f_ng_n)_n \rightarrow fg$ uniformly on $\Omega$.
    \tcblower
    \begin{align*}
      \left\Vert f_n(x)g_n(x) - f(x)g(x)\right\Vert_u &= \left\Vert f_n(x)g_n(x) - f_n(x)g(x) + f_n(x)g(x) - f(x)g(x)\right\Vert_u\\
                                                      &= \left\Vert f_n(x)\left(g_n(x) - g(x)\right) + g(x)\left(f_n(x)-f(x)\right)\right\Vert_u\\
                                                      &\leq \left\Vert f_n(x)\right\Vert_u \cdot \left\Vert g_n(x) - g(x)\right\Vert_u + \left\Vert g(x)\right\Vert_u\left\Vert f_n(x) - f(x)\right\Vert_u \tag*{Triangle Inequality}\\
                                                      &\leq c\left\Vert f_n(x) - f(x)\right\Vert_u + d\left\Vert g_n(x) - g(x)\right\Vert \tag*{Definition of Supremum}\\
                                                      &\rightarrow 0
    \end{align*}
  \end{problem}
  \begin{problem}{Problem 8}
    Find a sequence of functions with $(f_n)_n$ defined on $[0,\infty)$ such that $\left|f_n\right|_u \geq n$, but $(f_n)_n \rightarrow 0$ pointwise.
    \tcblower
    Let $f_n$ be defined as $\delta_n$, where $\delta_n$ is defined as follows:
    \begin{align*}
      \delta_n(x) &= \begin{cases}
        n & x=n\\
        0 & \text{otherwise}
      \end{cases}
    \end{align*}
  \end{problem}
\end{document}
