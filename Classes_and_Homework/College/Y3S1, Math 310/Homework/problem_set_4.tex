\documentclass[10pt]{extarticle}
\title{}
\author{Avinash Iyer}
\date{}
\usepackage[shortlabels]{enumitem}


%paper setup
\usepackage{geometry}
\geometry{letterpaper, portrait, margin=1in}
\usepackage{fancyhdr}

%symbols
\usepackage{amsmath}
\usepackage{amssymb}
\usepackage{amsthm}
\usepackage{mathtools}
\usepackage{hyperref}
\usepackage{gensymb}
\usepackage{multirow,array}

\newtheorem*{remark}{Remark}
\usepackage[T1]{fontenc}
\usepackage[utf8]{inputenc}

%chemistry stuff
%\usepackage[version=4]{mhchem}
%\usepackage{chemfig}

%plotting
\usepackage{pgfplots}
\usepackage{tikz}
\tikzset{middleweight/.style={pos = 0.5, fill=white}}
\tikzset{weight/.style={pos = 0.5, fill = white}}
\tikzset{lateweight/.style={pos = 0.75, fill = white}}
\tikzset{earlyweight/.style={pos = 0.25, fill=white}}

%\usepackage{natbib}

%graphics stuff
\usepackage{graphicx}
\graphicspath{ {./images/} }
\usepackage[style=numeric, backend=biber]{biblatex} % Use the numeric style for Vancouver
\addbibresource{the_bibliography.bib}
%code stuff
%when using minted, make sure to add the -shell-escape flag
%you can use lstlisting if you don't want to use minted
%\usepackage{minted}
%\usemintedstyle{pastie}
%\newminted[javacode]{java}{frame=lines,framesep=2mm,linenos=true,fontsize=\footnotesize,tabsize=3,autogobble,}
%\newminted[cppcode]{cpp}{frame=lines,framesep=2mm,linenos=true,fontsize=\footnotesize,tabsize=3,autogobble,}

%\usepackage{listings}
%\usepackage{color}
%\definecolor{dkgreen}{rgb}{0,0.6,0}
%\definecolor{gray}{rgb}{0.5,0.5,0.5}
%\definecolor{mauve}{rgb}{0.58,0,0.82}
%
%\lstset{frame=tb,
%	language=Java,
%	aboveskip=3mm,
%	belowskip=3mm,
%	showstringspaces=false,
%	columns=flexible,
%	basicstyle={\small\ttfamily},
%	numbers=none,
%	numberstyle=\tiny\color{gray},
%	keywordstyle=\color{blue},
%	commentstyle=\color{dkgreen},
%	stringstyle=\color{mauve},
%	breaklines=true,
%	breakatwhitespace=true,
%	tabsize=3
%}
% text + color boxes
\usepackage[most]{tcolorbox}
\tcbuselibrary{breakable}
\newtcolorbox{problem}[1]{colback = white, title = {#1}, breakable}
\newtcolorbox{solution}{colback = white, colframe = black!75!white, title = Solution, breakable}
%including PDFs
%\usepackage{pdfpages}
\setlength{\parindent}{0pt}
\usepackage{cancel}
\pagestyle{fancy}
\fancyhf{}
\rhead{Avinash Iyer}
\lhead{Math 310: Problem Set 4}
\newcommand{\card}{\text{card}}
\newcommand{\ran}{\text{ran}}
\newcommand{\N}{\mathbb{N}}
\newcommand{\Q}{\mathbb{Q}}
\newcommand{\Z}{\mathbb{Z}}
\newcommand{\R}{\mathbb{R}}
\begin{document}
  \begin{problem}{Problem 1}
    Prove the following limits:
    \begin{enumerate}[(i)]
      \item $\displaystyle \left(\frac{2n}{n+2}\right)_n \rightarrow 2$
      \item $\displaystyle \left(\frac{\sqrt{n}}{n+1}\right)_n \rightarrow 0$
      \item $\displaystyle \left(\frac{(-1)^n}{\sqrt{n+7}}\right)_n \rightarrow 0$
      \item $\displaystyle \left(n^kb^n\right)_n \rightarrow 0$ where $0\leq b < 1$ and $k\in\N$
      \item $\displaystyle \left(\frac{2^{n+1}+3^{n+1}}{2^n + 3^n}\right)_n \rightarrow 1/3$
    \end{enumerate}
    \tcblower
    \begin{problem}{(i)}
     We need to show that
      \[(\forall \varepsilon > 0)(\exists N\in\N) \ni n > N \Rightarrow \left|\frac{2n}{n+2} - 2\right| < \varepsilon\]
      \begin{description}
        \item[Preliminary Work] 
          \begin{align*}
            \frac{2n}{n+2} &> 2-\varepsilon\\
            2n &> (2n - \varepsilon n) - 2\varepsilon + 4\\
            n &> \frac{4-2\varepsilon}{\varepsilon}
          \end{align*}
        \item[Proof] Let $\displaystyle N = \left\lceil\frac{4-2\varepsilon}{\varepsilon}\right\rceil$. Then,
          \begin{align*}
            n &> \frac{4-2\varepsilon}{\varepsilon}\\
            \varepsilon n &> 4-2\varepsilon\\
            0 &> 4-2\varepsilon-\varepsilon n\\
            2n &> 2n + 4 - \varepsilon (n+2)\\
            2n &> (2-\varepsilon)(n+2)\\
            \frac{2n}{n+2} - 2 &> -\varepsilon\\
            \left|\frac{2n}{n+2}-2\right| &< \varepsilon \tag*{$\displaystyle \frac{2n}{n+2}< 2~\forall n\in\N$} 
          \end{align*}
      \end{description}
    \end{problem}
    \begin{problem}{(ii)}
      We need to show that
      \begin{align*}
        (\forall \varepsilon > 0)(\exists N\in \N) \ni n > N \rightarrow \left|\left(\frac{\sqrt{n}}{n + 1}\right)\right| < \varepsilon
      \end{align*}
      \begin{description}
        \item[Preliminary Work]
          \begin{align*}
            \frac{\sqrt{n}}{n+1}&<\frac{\sqrt{n}}{n} \\
                                &< \varepsilon\\
            \frac{1}{\sqrt{n}} &< \varepsilon\\
            n &> \frac{1}{\varepsilon^2}
          \end{align*}
        \item[Proof] Let $\displaystyle N = \left\lceil\frac{1}{\varepsilon^2}\right\rceil$. Then,
          \begin{align*}
            n &> \frac{1}{\varepsilon^2}\\
            \frac{1}{\sqrt{n}} &< \varepsilon\\
            \frac{\sqrt{n}}{n+1} &< \frac{\sqrt{n}}{n}\\
                                 &< \varepsilon
          \end{align*}
      \end{description}
    \end{problem}
    \begin{problem}{(iii)}
      We need to show that
      \begin{align*}
        (\forall \varepsilon > 0)(\exists N \in \N) \ni n > N \Rightarrow \left|\frac{(-1)^n}{\sqrt{n+7}}\right| < \varepsilon
      \end{align*}
      \begin{description}
        \item[Preliminary Work]
          \begin{align*}
            \frac{1}{\sqrt{n+7}} &< \varepsilon\\
            \frac{1}{\varepsilon} & < \sqrt{n+7}\\
            n &> \frac{1}{\varepsilon^2} - 7 
          \end{align*}
        \item[Proof] Let $\displaystyle N =  \left\lceil\frac{1}{\varepsilon^2}\right\rceil-7$. Then,
          \begin{align*}
            n &> \frac{1}{\varepsilon^2} - 7\\
            n + 7 &> \frac{1}{\varepsilon^2}\\
            \frac{1}{\sqrt{n+7}} & < \varepsilon\\
            -\varepsilon &< \frac{-1}{\sqrt{n+7}}\\
            \left|\frac{(-1)^{n}}{\sqrt{n+7}}\right| & < \varepsilon
          \end{align*}
      \end{description}
    \end{problem}
    \begin{problem}{(iv)}
      We need to show that
      \begin{align*}
        (\forall \varepsilon > 0)(\exists N\in\N) \ni n > N \rightarrow \left|n^kb^n\right| < \varepsilon
      \end{align*}
      \begin{description}
        \item[Preliminary Work]
          \begin{align*}
            n^kb^n & < \varepsilon \\
            b^n &< \frac{\varepsilon}{n^k}\\
            n &< \frac{\ln \varepsilon - k\ln n}{\ln b}\\
            n\ln b &> \ln \varepsilon - k\ln n\\
            k\ln n + n\ln b &> \ln\varepsilon\\
            kn + n\ln b &> k\ln n + n\ln b\\
                        &> \ln\varepsilon\\
            n &> \frac{\ln \varepsilon}{k + \ln b}
          \end{align*}
        \item[Proof] Let $k\in\N$ and $0\leq b < 1$. For the trivial case $b = 0$, let $N = 1$. Then, $|(1)(0)| < \varepsilon~\forall \varepsilon > 0$. Otherwise, let $\displaystyle N = \left\lceil\frac{\ln\varepsilon}{k + \ln b}\right\rceil$
          \begin{align*}
            n &> \frac{\ln\varepsilon}{k + \ln b}\\
            kn + n\ln b &> \ln\varepsilon\\
            kn &> \ln\varepsilon - n\ln b\\
            \frac{k\ln n}{\ln b}<\frac{kn}{\ln b} &< \frac{\ln\varepsilon - n\ln b}{\ln b}\tag*{since $\ln b < 0$}\\
            \log_b\left(n^k\right) &< \log_b\left(\frac{\varepsilon}{b^n}\right)\\
            n^k &< \frac{\varepsilon}{b^n}\\
            b^n n^k &< \varepsilon
          \end{align*}
      \end{description}
    \end{problem}
  \end{problem}
\end{document}
