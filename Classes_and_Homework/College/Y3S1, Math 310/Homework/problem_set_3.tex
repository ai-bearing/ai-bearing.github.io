\documentclass[10pt]{extarticle}
\title{}
\author{Avinash Iyer}
\date{}
\usepackage[shortlabels]{enumitem}


%paper setup
\usepackage{geometry}
\geometry{letterpaper, portrait, margin=1in}
\usepackage{fancyhdr}

%symbols
\usepackage{amsmath}
\usepackage{amssymb}
\usepackage{amsthm}
\usepackage{mathtools}
\usepackage{hyperref}
\usepackage{gensymb}
\usepackage{multirow,array}

\newtheorem*{remark}{Remark}
\usepackage[T1]{fontenc}
\usepackage[utf8]{inputenc}

%chemistry stuff
%\usepackage[version=4]{mhchem}
%\usepackage{chemfig}

%plotting
\usepackage{pgfplots}
\usepackage{tikz}
\tikzset{middleweight/.style={pos = 0.5, fill=white}}
\tikzset{weight/.style={pos = 0.5, fill = white}}
\tikzset{lateweight/.style={pos = 0.75, fill = white}}
\tikzset{earlyweight/.style={pos = 0.25, fill=white}}

%\usepackage{natbib}

%graphics stuff
\usepackage{graphicx}
\graphicspath{ {./images/} }
\usepackage[style=numeric, backend=biber]{biblatex} % Use the numeric style for Vancouver
\addbibresource{the_bibliography.bib}
%code stuff
%when using minted, make sure to add the -shell-escape flag
%you can use lstlisting if you don't want to use minted
%\usepackage{minted}
%\usemintedstyle{pastie}
%\newminted[javacode]{java}{frame=lines,framesep=2mm,linenos=true,fontsize=\footnotesize,tabsize=3,autogobble,}
%\newminted[cppcode]{cpp}{frame=lines,framesep=2mm,linenos=true,fontsize=\footnotesize,tabsize=3,autogobble,}

%\usepackage{listings}
%\usepackage{color}
%\definecolor{dkgreen}{rgb}{0,0.6,0}
%\definecolor{gray}{rgb}{0.5,0.5,0.5}
%\definecolor{mauve}{rgb}{0.58,0,0.82}
%
%\lstset{frame=tb,
%	language=Java,
%	aboveskip=3mm,
%	belowskip=3mm,
%	showstringspaces=false,
%	columns=flexible,
%	basicstyle={\small\ttfamily},
%	numbers=none,
%	numberstyle=\tiny\color{gray},
%	keywordstyle=\color{blue},
%	commentstyle=\color{dkgreen},
%	stringstyle=\color{mauve},
%	breaklines=true,
%	breakatwhitespace=true,
%	tabsize=3
%}
% text + color boxes
\usepackage[most]{tcolorbox}
\tcbuselibrary{breakable}
\newtcolorbox{problem}[1]{colback = white, title = {#1}, breakable}
\newtcolorbox{solution}{colback = white, colframe = black!75!white, title = Solution, breakable}
%including PDFs
%\usepackage{pdfpages}
\setlength{\parindent}{0pt}
\usepackage{cancel}
\pagestyle{fancy}
\fancyhf{}
\rhead{Avinash Iyer}
\lhead{Math 310: Problem Set 3}
\newcommand{\card}{\text{card}}
\newcommand{\ran}{\text{ran}}
\newcommand{\N}{\mathbb{N}}
\newcommand{\Q}{\mathbb{Q}}
\newcommand{\Z}{\mathbb{Z}}
\newcommand{\R}{\mathbb{R}}
\begin{document}
  \begin{problem}{Problem 1}
    Find $\sup(A)$ and $\inf(A)$ where
    \begin{enumerate}[(a)]
      \item $A := \left\{1-\frac{(-1)^n}{n}\mid n\in\N\right\}$
      \item $A:= \left\{\frac{1}{n}-\frac{1}{m}\mid m,n\in\N\right\}$
      \item $A:= \left\{\frac{m}{n}\mid m,n\in\N,~m+n\leq 10\right\}$
    \end{enumerate}
    \tcblower
    \begin{problem}{(a)}
      \begin{description}[font=\normalfont]
        \item[$\sup(A) = 2$:] For any $t\in A,~t < 2$, we can find $a_t$ as follows:
          \[
            a_t := \begin{cases}
              1,~t<1\\
              \frac{4}{3},~1\leq t < \frac{4}{3}\\
              2,~t=\frac{4}{3}
            \end{cases}
          \] 
        \item[$\inf(A) = \frac{1}{2}$:] For any $t\in A$, $t>\frac{1}{2}$, we can find $a_t$ as follows:
          \[
            a_t := \begin{cases}
              1,~t > 1\\
              \frac{3}{4},~\frac{3}{4} < t \leq 1\\
              \frac{1}{2},~t < \frac{3}{4}
            \end{cases}
          \] 
      \end{description}
    \end{problem}
    \begin{problem}{(b)}
      \begin{description}[font=\normalfont]
        \item[$\sup(A) = 1$:] For any $t\in A$, $t < 1$, we can find $a_t > t$ as follows:
          \begin{enumerate}[(1)]
            \item Take $|t| \geq t$.
            \item If $|t| < \frac{1}{2}$, find $m$ such that $\frac{1}{m}<|t|$ (which exists by the Archimedean Property corollary). Set $a_t = 1-\frac{1}{m}$.
            \item If $|t| > \frac{1}{2}$, then find $m$ such that $\frac{1}{m} < 1 - |t|$, and set $a_t = 1-\frac{1}{m}$.
          \end{enumerate}
          In all three cases, $a_t > t$, meaning $\sup(A) = 1$
        \item[$\inf(A) = -1$]
      \end{description}
    \end{problem}
    \begin{problem}{(c)}
      Since $A$ is finite, $\sup(A) = \max(A) = 9$ and $\inf(A) = \min(A) = \frac{1}{9}$
    \end{problem}
  \end{problem}
  \begin{problem}{Problem 2}
    Suppose $u = \sup(A)$ such that $u\notin A$. Show that there is a strictly increasing sequence
    \[
      t_1 < t_2 < t_3 < \dots
    \] 
    With $t_n \in A$ and $t_n + \frac{1}{n} > u$ for all $n \geq 1$
    \tcblower
    Let $t_n = u-\frac{1}{2n}$. $t_n$ must be a strictly increasing sequence because $t_{n+1} = u - \frac{1}{2n+2} > u-\frac{1}{2n} = t_n$.\\

    Additionally, $t_n + \frac{1}{n} = u-\frac{1}{n} < u$, meaning $t_n\in A$.
  \end{problem}
  \begin{problem}{Problem 3}
    If $m$ is a lower bound for $A\subseteq \R$, show that the following are equivalent:
    \begin{enumerate}[(i)]
      \item $m = \inf(A)$
      \item $\forall t > m,~\exists a_t\in A \ni a_t < t$
      \item $\forall \varepsilon > 0,~\exists a_{\varepsilon} \ni m+\varepsilon > a_{\varepsilon}$
    \end{enumerate}
  \end{problem}
  \begin{problem}{Problem 4}
    Let $A,B\in\R$ be bounded subsets.
    \begin{enumerate}[(a)]
      \item Show that
        \begin{align*}
          \sup(A+B) &= \sup(A) + \sup(B)\\
          \inf(A+B) &= \inf(A) + \inf(B)\\
        \end{align*}
      \item If $t > 0$, show that
        \begin{align*}
          \sup(tA) &= t\sup(A)\\
          \inf(tA) &= t\inf(A)
        \end{align*}
    \end{enumerate}
    \tcblower
    \begin{problem}{(a)}
      Let $a = \sup(A)$ and $b = \sup(B)$, and $x_a\in A$ and $x_b\in B$. Then
      \begin{align*}
        a &\geq x_a\\
        a + x_b &\geq x_a + x_b \tag*{by the ordering of $\R$}\\
        a+b &\geq a + x_b \tag*{by the definition of $\sup(B)$}\\
        a+b &\geq x_a + x_b \tag*{by the ordering of $\R$}\\
        \sup(A) + \sup(B) &= \sup(A+B)
      \end{align*}
      Let $a' = \inf(A)$ and $b' = \inf(B)$, with $x_a$ and $x_b$ defined as above. Then
      \begin{align*}
        a' &\leq x_a\\
        a' + x_b &\leq x_a + x_b \tag*{by the ordering of $\R$}\\
        a' + b' &\leq a' + x_b' \tag*{by the definition of $\inf(B)$}\\
        a' + b' &\leq x_a + x_b \tag*{by the ordering of $\R$}\\
        \inf(A) + \inf(B) &= \inf(A+B)
      \end{align*}
    \end{problem}
    \begin{problem}{(b)}
      Let $a = \sup(A)$, $x_a\in A$, and $t > 0$. Then
      \begin{align*}
        a &\geq x_a \\
        ta &\geq tx_a \tag*{by the ordering of $\R$}\\
        t\sup(A) &= \sup(tA)
      \end{align*}
      Let $a' = \inf(A)$, with $x_a$ and $t$ defined as above.
      \begin{align*}
        a' &\leq x_a \\
        ta' &\leq tx_a \tag*{by the ordering of $\R$}\\
        t\inf(A) &= \inf(tA)
      \end{align*}
    \end{problem}
  \end{problem}
  \begin{problem}{Problem 5}
    Let $I = (0,1)$ denote the open unit interval and consider $F: I\times I \rightarrow \R$, $F(x,y) = 2x+y$. 
    \begin{align*}
      \shortintertext{Compute}
      \sup_{y\in I}\left(\inf_{x\in I}F(x,y)\right)\\
      \shortintertext{and}
      \inf_{x\in I}\left(\sup_{y\in I} F(x,y)\right)
    \end{align*}
    \tcblower
    We start by finding $\inf_{x\in I} F(x,y)$, which is equal to $F(x,y) = y$ (as the infimum is the greatest lower bound on $2x$, which is $2(0) = 0$). So, $\sup_{y\in I} y = 1$.\\

    We start by finding $\sup_{y\in I}F(x,y)$, which is $\sup_{y\in I} 2x + y$, which is $2x + 1$, as $\sup I = 1$. So, by similar reasoning, $\inf_{x\in I} 2x+1 = 1$.\\

    These values are the same.
  \end{problem}
  \begin{problem}{Problem 6}
    Let $D$ be a nomempty set and consider the vector space
    \begin{align*}
      \ell_{\infty}(D) := \{f\mid f: D\rightarrow \R~\text{is bounded}\}
    \end{align*}
    with point-wise addition and scalar multiplication. Show that
    \begin{align*}
      \Vert f \Vert_u := \sup_{x\in D}|f(x)|
    \end{align*}
    defines a norm on $\ell_{\infty}(D)$.
    \tcblower
    \begin{enumerate}[(1)]
      \item Because $\forall x\in\R,~|x| \geq 0$, $\Vert \cdot \Vert_u\geq 0$.
      \item $\Vert f + g \Vert_u = \sup_{x\in D} |f(x) + g(x)| \leq \sup_{x\in D}|f(x)| + \sup_{x\in D} |g(x)|$ (by the Triangle Inequality) $ = \Vert f \Vert_u + \Vert g \Vert_u $.
      \item $\Vert \mathbf{0} \Vert = \sup_{x\in D}|\mathbf{0}| = 0$.
      \item Let $\Vert f \Vert_u = 0$. Then, $\sup_{x\in D} |f(x)| = 0$, meaning that $\nexists x'\in D$ such that $f(x') \neq 0$ (or else $\sup_{x\in D}|f(x)| = f(x')$), so $f(x) = \mathbf{0}$.
      \item $\Vert tf\Vert_u = \sup_{x\in D}|tf(x)| = |t|\sup_{x\in D}|f(x)| = |t|\Vert f \Vert_u$.
    \end{enumerate}
    Therefore, $\Vert \cdot \Vert_u$ is a norm on $\ell_{\infty}$.
  \end{problem}
  \begin{problem}{Problem 7}
    Let $f,g: D\rightarrow\R$ be bounded functions. Show that
    \begin{enumerate}[(a)]
      \item $\sup_{x\in D}(f+g)(x) \leq \sup_{x\in D}f(x) + \sup_{x\in D}g(x)$
      \item $\inf_{x\in D}(f+g)(x) \geq \inf_{x\in D}f(x) + \inf_{x\in D}g(x)$
      \item $\vert \sup_{x\in D}f(x) - \sup_{x\in D}g(x) \vert \leq \sup_{x\in D}\vert f(x) - g(x) \vert$
    \end{enumerate}
  \end{problem}
  \begin{problem}{Problem 8}
    Find $\bigcap_{n = 1}^{\infty} I_n$ where
    \begin{enumerate}[(a)]
      \item $I_n = [0,1/n]$
      \item $I_n = (0,1/n)$
      \item $I_n = [n,\infty)$
    \end{enumerate}
    \tcblower
    \begin{problem}{(a)}
      For all $k > 1$, $\bigcap_{n=1}^{k} = [0,1/k]$, meaning that $\bigcap_{n=1}^{\infty} = \lim_{k\rightarrow\infty}[0,1/k] = \{0\}$.
    \end{problem}
    \begin{problem}{(b)}
      We will show that $\bigcap_{n=1}^{\infty} = \emptyset$.\\

      Suppose toward contradiction $\exists k\in \bigcap_{n=1}^{\infty}$. Then, $k > 0$, but $\forall n\in \N$, $k < 1/n$. However, by the Archimedean property, $k < 1/n~\forall n \Rightarrow k=0$. So $k > 0$ and $k = 0$. $\bot$
    \end{problem}
    \begin{problem}{(c)}
      We will show that $\bigcap_{n=1}^{\infty} [n,\infty) = \emptyset$.\\

      Suppose toward contradiction that $\exists k \in \bigcap_{n=1}^{\infty}$. Then, $k \geq n \forall n$. However, since $\N$ is unbounded, $\nexists$ such a $k$. $\bot$
    \end{problem}
  \end{problem}
\end{document}
