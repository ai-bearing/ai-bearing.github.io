\documentclass[10pt]{extarticle}
\title{}
\author{}
\date{}
\usepackage[shortlabels]{enumitem}


%paper setup
\usepackage{geometry}
\geometry{letterpaper, portrait, margin=1in}
\usepackage{fancyhdr}
% sans serif font:
\usepackage{cmbright}
%symbols
\usepackage{amsmath}
\usepackage{bigints}
\usepackage{amssymb}
\usepackage{amsthm}
\usepackage{mathtools}
\usepackage[hidelinks]{hyperref}
\usepackage{gensymb}
\usepackage{multirow,array}
\usepackage{multicol}

\newtheorem*{remark}{Remark}
\usepackage[T1]{fontenc}
\usepackage[utf8]{inputenc}

%chemistry stuff
%\usepackage[version=4]{mhchem}
%\usepackage{chemfig}

%plotting
\usepackage{pgfplots}
\usepackage{tikz}
\tikzset{middleweight/.style={pos = 0.5}}
%\tikzset{weight/.style={pos = 0.5, fill = white}}
%\tikzset{lateweight/.style={pos = 0.75, fill = white}}
%\tikzset{earlyweight/.style={pos = 0.25, fill=white}}

%\usepackage{natbib}

%graphics stuff
\usepackage{graphicx}
\graphicspath{ {./images/} }
\usepackage[style=numeric, backend=biber]{biblatex} % Use the numeric style for Vancouver
\addbibresource{the_bibliography.bib}
%code stuff
%when using minted, make sure to add the -shell-escape flag
%you can use lstlisting if you don't want to use minted
%\usepackage{minted}
%\usemintedstyle{pastie}
%\newminted[javacode]{java}{frame=lines,framesep=2mm,linenos=true,fontsize=\footnotesize,tabsize=3,autogobble,}
%\newminted[cppcode]{cpp}{frame=lines,framesep=2mm,linenos=true,fontsize=\footnotesize,tabsize=3,autogobble,}

%\usepackage{listings}
%\usepackage{color}
%\definecolor{dkgreen}{rgb}{0,0.6,0}
%\definecolor{gray}{rgb}{0.5,0.5,0.5}
%\definecolor{mauve}{rgb}{0.58,0,0.82}
%
%\lstset{frame=tb,
%	language=Java,
%	aboveskip=3mm,
%	belowskip=3mm,
%	showstringspaces=false,
%	columns=flexible,
%	basicstyle={\small\ttfamily},
%	numbers=none,
%	numberstyle=\tiny\color{gray},
%	keywordstyle=\color{blue},
%	commentstyle=\color{dkgreen},
%	stringstyle=\color{mauve},
%	breaklines=true,
%	breakatwhitespace=true,
%	tabsize=3
%}
% text + color boxes
\renewcommand{\mathbf}[1]{\mathbold{#1}}
\usepackage[most]{tcolorbox}
\tcbuselibrary{breakable}
\tcbuselibrary{skins}
\newtcolorbox{problem}[1]{colback=white,enhanced,title={\small #1},
          attach boxed title to top center=
{yshift=-\tcboxedtitleheight/2},
boxed title style={size=small,colback=black!60!white}, sharp corners, breakable}
%including PDFs
%\usepackage{pdfpages}
\setlength{\parindent}{0pt}
\usepackage{cancel}
\pagestyle{fancy}
\fancyhf{}
\rhead{Avinash Iyer}
\lhead{Math 310: Problem Set 8}
\newcommand{\card}{\text{card}}
\newcommand{\ran}{\text{Ran}}
\newcommand{\N}{\mathbb{N}}
\newcommand{\Q}{\mathbb{Q}}
\newcommand{\Z}{\mathbb{Z}}
\newcommand{\R}{\mathbb{R}}
\begin{document}
  \begin{problem}{Problem 1}
    Recall that a subset $U\subseteq \R$ is \textbf{open} if
    \begin{align*}
      (\forall x\in U)(\exists \varepsilon > 0) \ni V_{\varepsilon}(x) \subseteq U.
    \end{align*}
    Prove that a mapping $f: \R \rightarrow \R$ is continuous if and only if $f^{-1}(U) \subseteq \R$ is open for every open $U\subseteq \R$. 
    \tcblower
    \begin{description}
      \item[$(\Rightarrow)$] Let $f: \R \rightarrow \R$ be continuous. Then, $\forall \varepsilon > 0$, $\exists \delta > 0$ such that  $\forall c\in\R$, $x\in V_{\delta}(c) \Rightarrow f(x) \in V_{\varepsilon}(f(c))$. Let $U$ be an open set such that $f(c)\in U$. Then, $\exists \varepsilon_{0}$ such that $V_{\varepsilon_0}(f(c))\subseteq U$. So, $\exists \delta_0$ such that $V_{\delta_0}(c) \subseteq f^{-1}(V_{\varepsilon_0}(f(c)))\subseteq f^{-1}(U)$. So, $f^{-1}(U)$ is open.
      \item[$(\Leftarrow)$] Let $f: \R \rightarrow \R$ be such that for every open set $U\subseteq \R$, $f^{-1}(U)$ is open in $\R$.\\

        Since $U$ is open in $\R$, it must be the case that for every $f(c)\in U$, $\exists \varepsilon > 0$ such that $V_{\varepsilon}(f(c))\subseteq U$. Since $f^{-1}(U) = \{c\mid f(c)\in U\}$, it must be the case that $\exists \delta > 0$ such that $V_{\delta}(c) \subseteq f^{-1}(U)$.\\

        Therefore, $x\in V_{\delta}(c) \Rightarrow f(x)\in V_{\varepsilon}(f(c))$ for sufficiently small $\delta$. Thus, $f: \R \rightarrow \R$ is continuous.
    \end{description}
  \end{problem}
  \begin{problem}{Problem 2}
    Let $f,g: D\rightarrow \R$ be continuous. Show that $f\cdot g$ is continuous.
    \tcblower
    Since $f: D\rightarrow \R$ is continuous, then $\forall (x_n)_n,c\in D$ such that $(x_n)_n \rightarrow c$, $\left(f(x_n)\right)_n \rightarrow f(c)$. Similarly, since $g: D\rightarrow \R$ is continuous, then $\forall (x_n)_n,c\in D$ such that $(x_n)_n\rightarrow c$, $(g(x_n))_n \rightarrow g(c)$.\\

    So, $\forall (x_n)_n,c\in D$ such that $(x_n)_n \rightarrow c$, $(f(x_n)g(x_n))_n \rightarrow f(c)g(c)$ by the properties of sequences. Thus, $f\cdot g$ is continuous.
  \end{problem}
  \begin{problem}{Problem 3}
    Let $f: D\rightarrow \R$ and $g: E\rightarrow \R$ be continuous mappings with $\ran(f)\subseteq E$. Show that $g\circ f$ is continuous.
    \tcblower
    Every sequence $(x_n)_n\in D$ with $(x_n)_n \rightarrow c\in D$ has $(f(x_n))_n \rightarrow f(c)$. Since $(f(x_n))_n \in E$ and $f(c)\in E$, it must be the case that $(g(f(x_n)))_n \rightarrow g(f(c))$. So, $g\circ f: D\rightarrow \R$ is continuous.
  \end{problem}
  \begin{problem}{Problem 4}
    Show that the following functions are Lipschitz:
    \begin{enumerate}[(i)]
      \item $f: [-M,M] \rightarrow \R$ given by $f(x) = x^2$
      \item $g: [1,\infty)\rightarrow \R$ given by $g(x) = \frac{1}{x}$
      \item $g: \R \rightarrow \R$ given by $g(x) = \sqrt{x^2 + 4}$
    \end{enumerate}
    \tcblower
    \begin{problem}{(a)}
      Let $x,y\in [-M,M]$. Then,
      \begin{align*}
        |f(x) - f(y)| &= |x^2 - y^2|\\
                      &= |x-y||x+y|\\
                      &\leq \left(|x| + |y|\right)|x-y|\\
                      &\leq 2|M||x-y|
      \end{align*}
    \end{problem}
    \begin{problem}{(b)}
      Let $x,y\in [1,\infty)$. Then,
      \begin{align*}
        |f(x) - f(y)| &= \left|\frac{1}{x} - \frac{1}{y}\right|\\
                      &= \frac{1}{xy}|x-y|\\
                      &\leq |x-y|
      \end{align*}
    \end{problem}
    \begin{problem}{(c)}
      Let $x,y\in \R$. Then,
      \begin{align*}
        |f(x) - f(y)| &= |\sqrt{x^2 + 4} - \sqrt{y^2 + 4}|\\
                      &= \frac{|x^2 - y^2|}{\sqrt{x^2 + 4} + \sqrt{x^2 + 4}}\\
                      &= \frac{|x+y||x-y|}{\sqrt{x^2 + 4} + \sqrt{x^2 + 4}}\\
                      &\leq \frac{(|x| + |y|)|x-y|}{\sqrt{x^2+4} + \sqrt{y^2+4}}\\
                      &\leq \frac{(|x| + |y|)|x-y|}{\sqrt{x^2} + \sqrt{y^2}}\\
                      &= \frac{\left(|x| + |y|\right)|x-y|}{|x| + |y|} \\
                      &= |x-y|
      \end{align*}
    \end{problem}
  \end{problem}
  \begin{problem}{Problem 5}
    Show that the following functions are not Lipschitz:
    \begin{enumerate}[(a)]
      \item $f: \R \rightarrow \R$ given by $f(x) = x^2$
      \item $g: (0,\infty)$ given by $g(x) = \frac{1}{x}$
    \end{enumerate}
    \tcblower
    \begin{problem}{(a)}
      Let $x,y\in \R$. Then,
      \begin{align*}
        \left|f(x) - f(y)\right| &= \left|x^2 - y^2\right|\\
                                 &= |x-y||x+y|\\
                                 &\leq \left(|x| + |y|\right)|x-y|
      \end{align*}
      but since $|x| + |y|$ is unbounded, it must be the case that $\nexists c$ such that $|f(x) - f(y)| \leq c|x-y|$.
    \end{problem}
    \begin{problem}{(b)}
      Let $x,y\in (0,\infty)$. Then,
      \begin{align*}
        |f(x) - f(y)| &= \left|\frac{1}{x} - \frac{1}{y}\right|\\
                      &= \frac{|x-y|}{xy}
      \end{align*}
      but since $\frac{1}{xy}$ is unbounded on $(0,\infty)$, it must be the case that $\nexists c$ such that $|f(x) - f(y)| \leq c|x-y|$.
    \end{problem}
  \end{problem}
  \begin{problem}{Problem 6}
    Suppose $f: \R \rightarrow \R$ and for some $C \geq 0$, we have $|f(q)| \leq C$ for all rationals $q\in \Q$. Show that $\Vert f \Vert_{\R} \leq C$.
    \tcblower
    Let $t\in \R$. Then, $\exists (q_n)_n \in \Q$ such that $(q_n)_n \rightarrow t$, as the rationals are dense.\\

    Since $f$ is continuous, $\left(f(q_n)\right)_n \rightarrow f(t)$.\\

    Since $|f(q_n)| \leq C$ for all $q_n$, it must be the case that $f(t) \leq C$.
  \end{problem}
  \begin{problem}{Problem 7}
    Suppose $f: \R \rightarrow \R$ is an additive map, that is,
    \begin{align*}
      f(x+y) &= f(x) + f(y) \tag*{$\forall x,y\in \R$.}
    \end{align*}
    If $f$ is continuous at some point, say $x=c$, show that $f$ is continuous everywhere and that $f(x) = ax$ for some $a\in\R$.
    \tcblower
    Let $t\in\R$. Let $(x_n)_n \in \R$ with $(x_n)_n \rightarrow c$. Then, for the sequence $(x_n - c + t)_n\in\R$, with $(x_n - c + t)_n \rightarrow t$, we have
    \begin{align*}
      f(x_n - c + t) &= f(x_n) + - f(c) + f(t)\\
                 &\rightarrow f(c) - f(c) + f(t)\\
                 &= f(t)
    \end{align*}
    so $f$ must be continuous at $x = t$.
  \end{problem}
  \begin{problem}{Problem 8}
    Assume $g: \R\rightarrow \R$ satisfies
    \begin{align*}
      g(x+y) &= g(x)g(y) \tag*{$\forall x,y\in\R$.}
    \end{align*}
    If $g$ is continuous at $x=0$, show that $g$ is continuous everywhere. Then show that there is a $b\geq 0$ with $g(x) = b^x$.
  \end{problem}
  \begin{problem}{Problem 9}
    Let $p$ be a polynomial of odd degree. Show that $p$ has a real root.
    \tcblower
    Let $p(x) = a_{2n+1}x^{2n+1} + \cdots + a_{1}x + a_0$. Then, $\lim_{x\rightarrow\infty}p(x) = \pm \infty$, and $\lim_{x\rightarrow -\infty} p(x) = \mp \infty$. Without loss of generality, suppose $\lim_{x\rightarrow\infty}p(x) = +\infty$, and $\lim_{x\rightarrow-\infty} = -\infty$.\\

    Then, for any $N > 0$, $\exists x_1 > 0$ such that $p(x) > N$ for all $x > x_1$. So, $p(x_1) > 0$. Similarly, for any $M < 0$, $\exists x_2 < 0$ such that $p(x) < M$ for all $x < x_2$. So, $p(x_2) < 0$.\\

    By the intermediate value theorem on $[x_2,x_1]$, there must be a point where $p(x) = 0$ where $x\in [x_2,x_1]$.
  \end{problem}
  \begin{problem}{Problem 10}
    Let $f: \R\rightarrow \R$ be a continuous function that vanishes at infinity, that is,
    \begin{align*}
      \lim_{x\rightarrow \pm\infty}f = 0.
    \end{align*}
    Show that $f$ is bounded.
    \tcblower
    Let $\varepsilon > 0$. Then, $\exists N > 0$ and $M < 0$ such that $|f(x)| < \varepsilon$ for all $x > N$ an $x < M$.\\

    So, on $(-\infty,M)$ and $(N,\infty)$, $|f|$ is bounded by $\varepsilon$. Finally, on $[M,N]$, $|f|$ must be bounded by the Extreme Value Theorem.\\

    Therefore, $|f|$ is bounded on $\R$, and thus $f$ is bounded on $\R$.
  \end{problem}
  \begin{problem}{Problem 11}
    A function $f: D\rightarrow \R$ is said to be lower semicontinuous (LSC) at $x=c$ if
    \begin{align*}
      (\forall \varepsilon > 0)(\exists \delta > 0) \ni x\in D\cap V_{\delta}(c) \Rightarrow f(c) - \varepsilon < f(x).
    \end{align*}
    A function $f: D\rightarrow \R$ is said to be upper semicontinuous (USC) at $x=c$ if
    \begin{align*}
      (\forall \varepsilon > 0)(\exists \delta > 0) \ni x\in D\cap V_{\delta}(c) \Rightarrow f(x) < f(c) + \varepsilon
    \end{align*}
    \begin{enumerate}[(i)]
      \item Show that $f$ is continuous at $c$ if and only if $f$ is USC and LSC at $c$.
      \item Show that $f$ is LSC at $c$ if and only if
        \begin{align*}
          \liminf_{n\rightarrow\infty}f(x_n) \geq f(c),
        \end{align*}
        for every sequence $(x_n)_n$ in $D$ that converges to $c$.
      \item Show that $f$ is USC at $c$ if and only if
        \begin{align*}
          \limsup_{n\rightarrow\infty}f(x_n) \leq f(c)
        \end{align*}
        for every sequence $(x_n)_n$ in $D$ that converges to $c$.
      \item Show that a USC function $f: [a,b]\rightarrow \R$ admits an absolute maximum on $[a,b]$.
    \end{enumerate}
  \end{problem}
  \begin{problem}{Problem 12}
    Let $f: [a,b] \rightarrow \R$ be a continuous function satisfying the following property:
    \begin{align*}
      \forall x\in [a,b],~\exists y\in[a,b] \ni |f(y)| \leq \frac{1}{2}|f(x)|.
    \end{align*}
    Show that there is a $c\in [a,b]$ with $f(c) = 0$.
  \end{problem}
\end{document}
