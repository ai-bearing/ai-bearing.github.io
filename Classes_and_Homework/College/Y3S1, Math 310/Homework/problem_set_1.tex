\documentclass[10pt]{extarticle}
\title{}
\author{Avinash Iyer}
\date{}
\usepackage[shortlabels]{enumitem}

%font setup
%
%\usepackage{newpxtext,eulerpx}

%paper setup
\usepackage{geometry}
\geometry{letterpaper, portrait, margin=1in}
\usepackage{fancyhdr}

%symbols
\usepackage{amsmath}
\usepackage{amsfonts}
\usepackage{mathtools}
\usepackage{hyperref}
\usepackage{gensymb}

\usepackage[OT1]{fontenc}
\usepackage[utf8]{inputenc}

%chemistry stuff
\usepackage[version=4]{mhchem}
\usepackage{chemfig}

%plotting
\usepackage{pgfplots}
\usepackage{tikz}
\tikzset{middleweight/.style={pos = 0.5, fill=white}}
\tikzset{weight/.style={pos = 0.5, fill = white}}
\tikzset{lateweight/.style={pos = 0.75, fill = white}}
\tikzset{earlyweight/.style={pos = 0.25, fill=white}}

%\usepackage{natbib}

%graphics stuff
\usepackage{graphicx}
\graphicspath{ {./images/} }

%code stuff
%when using minted, make sure to add the -shell-escape flag
%you can use lstlisting if you don't want to use minted
%\usepackage{minted}
%\usemintedstyle{pastie}
%\newminted[javacode]{java}{frame=lines,framesep=2mm,linenos=true,fontsize=\footnotesize,tabsize=3,autogobble,}
%\newminted[cppcode]{cpp}{frame=lines,framesep=2mm,linenos=true,fontsize=\footnotesize,tabsize=3,autogobble,}

\usepackage{listings}
\usepackage{color}
\definecolor{dkgreen}{rgb}{0,0.6,0}
\definecolor{gray}{rgb}{0.5,0.5,0.5}
\definecolor{mauve}{rgb}{0.58,0,0.82}

\lstset{frame=tb,
	language=Java,
	aboveskip=3mm,
	belowskip=3mm,
	showstringspaces=false,
	columns=flexible,
	basicstyle={\small\ttfamily},
	numbers=none,
	numberstyle=\tiny\color{gray},
	keywordstyle=\color{blue},
	commentstyle=\color{dkgreen},
	stringstyle=\color{mauve},
	breaklines=true,
	breakatwhitespace=true,
	tabsize=3
}
% text + color boxes
\usepackage{tcolorbox}
\tcbuselibrary{breakable}
\newtcolorbox{problem}[1]{colback = white, title = {#1}, breakable}
\newtcolorbox{solution}{colback = white, colframe = black!75!white, title = Solution, breakable}
%including PDFs
\usepackage{pdfpages}
\setlength{\parindent}{0pt}

\pagestyle{fancy}
\fancyhf{}
\rhead{Avinash Iyer}
\lhead{Problem Set 1}
\begin{document}{
  \begin{problem}{Problem 1}
    If $F$ is a finite set and $k:F\rightarrow F$ is a self-map, prove that $k$ is injective if and only if $k$ is surjective.
    \tcblower
    Let $k$ be injective.
    \begin{align*}
      \textrm{card}(F) &= \textrm{card}(k(F))\tag*{definition of injection}\\
      k(F) &\subseteq F \tag*{definition of function}\\
      k(F) &= F
    \end{align*}
    \rule{\textwidth}{0.4pt}
    Let $k$ be surjective.
    \begin{align*}
      k(F) &= F \tag*{definition of surjection}\\
      \textrm{card}(k(F)) &= \textrm{card}(F) \tag*{cardinality definition}
    \end{align*}
    Therefore, $k$ is injective, as $F$ is finite.
  \end{problem}
  \begin{problem}{Problem 2}
    Prove that a set $A$ is infinite if and only if there is a non-surjective injection $f:A\xhookrightarrow{} A$.
    \tcblower
    \begin{description}[font=\normalfont]
      \item[$(\Rightarrow)$] Let $A$ be infinite. Then, $\exists i: \mathbb{N}\xhookrightarrow{} A$; $\forall n\in \mathbb{N}, a_n:=i(n)$. Let $f: A\rightarrow A$, $f(a_i) = a_{i+1}$. Then, for $a_{i_1} \neq a_{i_2}$, $f(a_{i_1}) = a_{i_1 + 1} \neq f(a_{i_2}) = a_{i_2 + 1}$. Therefore, $f$ is injective.
      \item[$(\Leftarrow)$] Suppose $A$ is finite. Then, by the result in Problem $1$, $\forall f: A\xhookrightarrow{} A$, $f$ must be surjective.
    \end{description}
  \end{problem}
  \begin{problem}{Problem 3}
    Let $A$, $B$, and $C$ be sets and suppose $\textrm{card}(A) < \textrm{card}(B) \leq \textrm{card}(C)$. Prove that $\textrm{card}(A) < \textrm{card}(C)$.
  \end{problem}
  \begin{problem}{Problem 4}
    If $A\subseteq B$ is an inclusion of sets with $A$ countable and $B$ uncountable, show that $B\setminus A$ is uncountable.
    \tcblower
    Suppose toward contradiction that $B\setminus A$ is countable.\\

    Then, $A\cup (B\setminus A)$ must be countable, by union of countable sets.\\

    However, $A\cup (B\setminus A) = B$, and $B$ is uncountable, meaning that $B\setminus A$ must be uncountable.
  \end{problem}
  \begin{problem}{Problem 5}
    Is the set $\{x\in\mathbb{R} \mid x>0~\textrm{and}~x^2\in\mathbb{Q}\}$ countable?
    \tcblower
    Let $q: \mathbb{Q} \rightarrow \mathbb{N}$ be the enumeration of the rationals; let $f:\{x\in\mathbb{R} \mid x>0~\textrm{and}~x^2\in\mathbb{Q}\}$ be defined as $f(x) = q\left(x^2\right)$.\\

    $x>0 \longrightarrow t(x) = x^2$ is a bijection; $\mathbb{Q}$ countable $\longrightarrow$ $q$ is a bijection; $f = q\circ t \longrightarrow f$ is a bijection, and thus $\{x\in\mathbb{R}\mid x>0~\textrm{and}~x^2\in\mathbb{Q}\}$ is countable.
  \end{problem}
  \begin{problem}{Problem 6}
    Consider the set $\mathcal{F}(\mathbb{N})$ of all finite subsets of $\mathbb{N}$. Is $\mathcal{F}(\mathbb{N})$ countable?
    \tcblower
    Let $f: \mathcal{F} \rightarrow \mathbb{N}$ be defined as follows, where $p_n$ denotes the $n$th prime number.
    \[
      f(\{a_1,a_2,\dots,a_n\}) = p_1^{a_1}\cdot p_2^{a_2}\cdots p_{n}^{a_n}
    \] 
    By the fundamental theorem of arithmetic, every natural number is equal to a unique product of powers of prime numbers, meaning that $f$ is injective, so $\mathcal{F}$ is countable.
  \end{problem}
  \begin{problem}{Problem 7}
    Let $k\in\mathbb{N}$.
    \begin{enumerate}[(i)]
      \item Prove that $\mathbb{N}^k = \underbrace{\mathbb{N}\times\mathbb{N}\times\cdots\mathbb{N}}_{k~\textrm{times}}$ is countable.
      \item Show that the set $\mathbb{N}^{\infty} := \{(n_k)_{k\geq 1}\mid n_k\in \mathbb{N}\}$ consisting of all sequences of natural numbers is uncountable.
      \item Prove that the set of \textbf{finitely-supported} natural sequences $c_c(\mathbb{N}) := \{(n_k)_{k\geq 1} \mid n_k\in\mathbb{N}, n_k=0~\text{for all but finitely many }k\}$ is countable.
    \end{enumerate}
    \tcblower
    \begin{problem}{(i)}
      Let $f: \mathbb{N}^k \rightarrow \mathbb{N}$ be defined as follows, where $p_n$ denotes the $n$th prime number in the sequence $\{2,3,5,\dots\}$
      \[
        f((a_1,a_2,\dots,a_k)) = p_1^{a_1}\cdot p_2^{a_2}\cdots p_k^{a_k}
      \] 
      By the fundamental theorem of arithmetic, every natural number is equal to a unique product of powers of prime numbers, so $f: \mathbb{N}^k \rightarrow \mathbb{N}$ is an injection, meaning $\mathbb{N}^k$ is countable
    \end{problem}
    \begin{problem}{(ii)}
      Suppose toward contradiction that the set of all sequences of natural numbers is countable: $f:A_n \rightarrow \mathbb{N}$ is surjective.
      \begin{align*}
        A_1 &= \{a_{11},a_{12},a_{13},\dots\}\\
        A_2 &= \{a_{21},a_{22},a_{23},\dots\}\\
            &\vdots
      \end{align*}
      Create a new sequence $N$ defined as follows:
      \begin{align*}
        n_{k} &= a_{kk} + 1
      \end{align*}
      Since $f$ is surjective, $\exists A_m = \{a_{m1},a_{m2},\dots,a_{mm},\dots\} = \{n_{1},n_{2},\dots,n_{m}\}$. However, since by definition, $n_m \neq a_{mm}$, $f$ must not be surjective. Thus, $\mathbb{N}^{\infty}$ is not countable.
    \end{problem}
    \begin{problem}{(iii)}
      Let $f: c_c(\mathbb{N}) \rightarrow \mathbb{N}$ be defined as follows, where $p_n$ denotes the $n$th prime number:
      \[
        f(\{n_1,n_2,\dots,n_k\}) = p_1^{n_1}\cdot p_2^{n_2}\cdots p_{k}^{n_k}
      \] 
      Since every natural number is represented uniquely by a product of powers of primes by the fundamental theorem of arithmetic, $f$ is injective, meaning $c_c(\mathbb{N})$ is countable.
    \end{problem}
  \end{problem}
  \begin{problem}{Problem 8}
    Let $f:\mathbb{R} \rightarrow \mathbb{R}$ be a function that sends rational numbers to irrational numbers and irrational numbers to rational numbers. Prove that the range $\textrm{ran}(f)$ cannot contain any interval.
  \end{problem}
  \begin{problem}{Problem 9}
    Prove that the set
    \[
      \mathcal{P} := \left\{\sum_{k=0}^{n}a_kx^k \mid n\subseteq \mathbb{N}_0,a_k\in\mathbb{Q}\right\}
    \] 
    consisting of all polynomials with rational coefficients, is countable.
    \tcblower
    Let $q: \mathbb{Q} \rightarrow \mathbb{N}$ be the enumeration of the rationals, and let $p_n$ denote the $n$th element in the sequence of prime numbers, where $p_1 = 2, p_2 = 3$, etc.\\

    Let $f: \mathcal{P} \rightarrow \mathbb{N}^k$ be defined as follows:

    \[
      f(a_0 + a_1x + a_2x^2 + \cdots + a_kx^k + \cdots) = (q(a_0),q(a_1),\dots,q(a_k),\dots)
    \] 
    Since $\mathbb{Q}$ is countable, $\forall a\in \mathbb{Q},~q(a)\in \mathbb{N}$, so the output of $f$ is a bijection to $\mathbb{N}^k$, meaning $\mathcal{P}$ is countable.
  \end{problem}
  \begin{problem}{Problem 10}
    A real number $t$ is called \textbf{algebraic} if there is a nonzero polynomial $p$ with rational coefficients such that $p(t) = 0$. If $t\in \mathbb{R}$ is not algebraic, then it is called \textbf{transcendental}. For example, $\sqrt{2}$ is algebraic, but $\pi$ is transcendental. Show that the set of algebraic numbers is countable, and conclude that there are uncountably many transcendental numbers.
    \tcblower
    Because $\mathcal{P}$ is countable, and there are $k$ roots in a $k$-degree polynomial, there are a countable number of polynomial roots, so the algebraic numbers are rational.\\

    Since $\mathbb{R}$ is uncountable, $\mathbb{R}\setminus\mathbb{A}$ is uncountable as shown by a previous result.
  \end{problem}
}\end{document}
