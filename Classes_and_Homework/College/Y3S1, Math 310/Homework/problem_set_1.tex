\documentclass[10pt]{extarticle}
\title{}
\author{Avinash Iyer}
\date{}
\usepackage[shortlabels]{enumitem}

%font setup
%
%\usepackage{newpxtext,eulerpx}

%paper setup
\usepackage{geometry}
\geometry{letterpaper, portrait, margin=1in}
\usepackage{fancyhdr}

%symbols
\usepackage{amsmath}
\usepackage{amsfonts}
\usepackage{mathtools}
\usepackage{hyperref}
\usepackage{gensymb}

\usepackage[OT1]{fontenc}
\usepackage[utf8]{inputenc}

%chemistry stuff
\usepackage[version=4]{mhchem}
\usepackage{chemfig}

%plotting
\usepackage{pgfplots}
\usepackage{tikz}
\tikzset{middleweight/.style={pos = 0.5, fill=white}}
\tikzset{weight/.style={pos = 0.5, fill = white}}
\tikzset{lateweight/.style={pos = 0.75, fill = white}}
\tikzset{earlyweight/.style={pos = 0.25, fill=white}}

%\usepackage{natbib}

%graphics stuff
\usepackage{graphicx}
\graphicspath{ {./images/} }

%code stuff
%when using minted, make sure to add the -shell-escape flag
%you can use lstlisting if you don't want to use minted
%\usepackage{minted}
%\usemintedstyle{pastie}
%\newminted[javacode]{java}{frame=lines,framesep=2mm,linenos=true,fontsize=\footnotesize,tabsize=3,autogobble,}
%\newminted[cppcode]{cpp}{frame=lines,framesep=2mm,linenos=true,fontsize=\footnotesize,tabsize=3,autogobble,}

\usepackage{listings}
\usepackage{color}
\definecolor{dkgreen}{rgb}{0,0.6,0}
\definecolor{gray}{rgb}{0.5,0.5,0.5}
\definecolor{mauve}{rgb}{0.58,0,0.82}

\lstset{frame=tb,
	language=Java,
	aboveskip=3mm,
	belowskip=3mm,
	showstringspaces=false,
	columns=flexible,
	basicstyle={\small\ttfamily},
	numbers=none,
	numberstyle=\tiny\color{gray},
	keywordstyle=\color{blue},
	commentstyle=\color{dkgreen},
	stringstyle=\color{mauve},
	breaklines=true,
	breakatwhitespace=true,
	tabsize=3
}
% text + color boxes
\usepackage{tcolorbox}
\tcbuselibrary{breakable}
\newtcolorbox{problem}[1]{colback = white, title = {#1}, breakable}
\newtcolorbox{solution}{colback = white, colframe = black!75!white, title = Solution, breakable}
%including PDFs
\usepackage{pdfpages}
\setlength{\parindent}{0pt}

\pagestyle{fancy}
\fancyhf{}
\rhead{Avinash Iyer}
\lhead{Problem Set 1}
\begin{document}{
  \begin{problem}{Problem 1}
    If $F$ is a finite set and $k:F\rightarrow F$ is a self-map, prove that $k$ is injective if and only if $k$ is surjective.
    \tcblower
    Let $k$ be injective.
    \begin{align*}
      \textrm{card}(F) &= \textrm{card}(k(F))\tag*{definition of injection}\\
      k(F) &\subseteq F \tag*{definition of function}\\
      k(F) &= F
    \end{align*}
    \rule{\textwidth}{0.4pt}
    Let $k$ be surjective.
    \begin{align*}
      k\circ k^{-1}(F) &= F \tag*{definition of surjection}\\
      (k\circ k^{-1}) \circ k(F) &= k(F) \tag*{apply $k$ on the right} \\
      k\circ (k^{-1} \circ k) (F) &= k(F) \tag*{associative property}
    \end{align*}
    Therefore, $k^{-1}\circ k = \textrm{id}_F$, meaning $k$ is injective.
  \end{problem}
  \begin{problem}{Problem 2}
    Prove that a set $A$ is infinite if and only if there is a non-surjective injection $f:A\rightarrow A$.
  \end{problem}
  \begin{problem}{Problem 3}
    Let $A$, $B$, and $C$ be sets and suppose $\textrm{card}(A) < \textrm{card}(B) \leq \textrm{card}(C)$. Prove that $\textrm{card}(A) < \textrm{card}(C)$.
  \end{problem}
  \begin{problem}{Problem 4}
    If $A\subseteq B$ is an inclusion of sets with $A$ countable and $B$ uncountable, show that $B\setminus A$ is uncountable.
    \tcblower
    Let $A$ be countable. Then, $A = \emptyset$, $A$ is finite, or $\exists f: \mathbb{N}\mapsto A$.\\

    Let $g: B\rightarrow \mathbb{N}$. By the definition of countability, $g$ is not injective.\\

    Let $k: B\setminus A \rightarrow \mathbb{N}$. We will show that for the above three cases, $k$ is not injective.
    \begin{description}
      \item[Case 1] If $A$ is the empty set, then we know that $B\setminus A = B$, and since $g$ is not injective, $k$ must not be injective.
      \item[Case 2] Let $L = \{1,2,\dots,n\}$ where $n$ is the cardinality of $A$. Since $A$ is finite, $\exists h:A\rightarrow L$. Let $s: \mathbb{N}\setminus L \rightarrow \mathbb{N},~s(l) = l + (n+1)$. Finally, let $t: B\setminus A \rightarrow \mathbb{N}\setminus L$. $s$ is a bijection (as it is a linear function), and $t$ is not injective, as $\textrm{card}(B\setminus A) = \textrm{card}(B) - n$. Thus, $s\circ t$ is not injective, so $B\setminus A$ is not countable.
      \item[Case 3] 
    \end{description}
  \end{problem}
  \begin{problem}{Problem 5}
    Is the set $\{x\in\mathbb{R} \mid x>0~\textrm{and}~x^2\in\mathbb{Q}\}$ countable?
  \end{problem}
  \begin{problem}{Problem 6}
    Consider the set $\mathcal{F}(\mathbb{N})$ of all finite subsets of $\mathbb{N}$. Is $\mathcal{F}(\mathbb{N})$ countable?
  \end{problem}
  \begin{problem}{Problem 7}
    Let $k\in\mathbb{N}$.
    \begin{enumerate}[(i)]
      \item Prove that $\mathbb{N}^k = \underbrace{\mathbb{N}\times\mathbb{N}\times\cdots\mathbb{N}}_{k~\textrm{times}}$ is countable.
      \item Show that the set $\mathbb{N}^{\infty} := \{(n_k)_{k\geq 1}\mid n_k\in \mathbb{N}\}$ consisting of all sequences of natural numbers is uncountable.
      \item Prove that the set of \textbf{finitely-supported} natural sequences $c_c(\mathbb{N}) := \{(n_k)_{k\geq 1} \mid n_k\in\mathbb{N}, n_k=0~\text{for all but finitely many }k\}$ is countable.
    \end{enumerate}
  \end{problem}
  \begin{problem}{Problem 8}
    Let $f:\mathbb{R} \rightarrow \mathbb{R}$ be a function that sends rational numbers to irrational numbers and irrational numbers to rational numbers. Prove that the range $\textrm{ran}(f)$ cannot contain any interval.
  \end{problem}
  \begin{problem}{Problem 9}
    Prove that the set
    \[
      \mathcal{P} := \left\{\sum_{k=0}^{n}a_kx^k \mid n\subseteq \mathbb{N}_0,a_k\in\mathbb{Q}\right\}
    \] 
    consisting of all polynomials with rational coefficients, is countable.
  \end{problem}
  \begin{problem}{Problem 10}
    A real number $t$ is called \textbf{algebraic} if there is a nonzero polynomial $p$ with rational coefficients such that $p(t) = 0$. If $t\in \mathbb{R}$ is not algebraic, then it is called \textbf{transcendental}. For example, $\sqrt{2}$ is algebraic, but $\pi$ is transcendental. Show that the set of algebraic numbers is countable, and conclude that there are uncountably many transcendental numbers.
  \end{problem}
}\end{document}
