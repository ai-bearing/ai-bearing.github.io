\documentclass[10pt]{extarticle}
\title{}
\author{Avinash Iyer}
\date{}
\usepackage[shortlabels]{enumitem}

%font setup
%
%\usepackage{newpxtext,eulerpx}

%paper setup
\usepackage{geometry}
\geometry{letterpaper, portrait, margin=1in}
\usepackage{fancyhdr}

%symbols
\usepackage{amsmath}
\usepackage{amssymb}
\usepackage{mathtools}
\usepackage{hyperref}
\usepackage{gensymb}
\usepackage{multirow,array}

\usepackage[T1]{fontenc}
\usepackage[utf8]{inputenc}

%chemistry stuff
\usepackage[version=4]{mhchem}
\usepackage{chemfig}

%plotting
\usepackage{pgfplots}
\usepackage{tikz}
\tikzset{middleweight/.style={pos = 0.5, fill=white}}
\tikzset{weight/.style={pos = 0.5, fill = white}}
\tikzset{lateweight/.style={pos = 0.75, fill = white}}
\tikzset{earlyweight/.style={pos = 0.25, fill=white}}

%\usepackage{natbib}

%graphics stuff
\usepackage{graphicx}
\graphicspath{ {./images/} }

%code stuff
%when using minted, make sure to add the -shell-escape flag
%you can use lstlisting if you don't want to use minted
%\usepackage{minted}
%\usemintedstyle{pastie}
%\newminted[javacode]{java}{frame=lines,framesep=2mm,linenos=true,fontsize=\footnotesize,tabsize=3,autogobble,}
%\newminted[cppcode]{cpp}{frame=lines,framesep=2mm,linenos=true,fontsize=\footnotesize,tabsize=3,autogobble,}

%\usepackage{listings}
%\usepackage{color}
%\definecolor{dkgreen}{rgb}{0,0.6,0}
%\definecolor{gray}{rgb}{0.5,0.5,0.5}
%\definecolor{mauve}{rgb}{0.58,0,0.82}
%
%\lstset{frame=tb,
%	language=Java,
%	aboveskip=3mm,
%	belowskip=3mm,
%	showstringspaces=false,
%	columns=flexible,
%	basicstyle={\small\ttfamily},
%	numbers=none,
%	numberstyle=\tiny\color{gray},
%	keywordstyle=\color{blue},
%	commentstyle=\color{dkgreen},
%	stringstyle=\color{mauve},
%	breaklines=true,
%	breakatwhitespace=true,
%	tabsize=3
%}
% text + color boxes
\usepackage[most]{tcolorbox}
\tcbuselibrary{breakable}
\newtcolorbox{problem}[1]{colback = white, title = {#1}, breakable}
\newtcolorbox{solution}{colback = white, colframe = black!75!white, title = Solution, breakable}
%including PDFs
%\usepackage{pdfpages}
\setlength{\parindent}{0pt}

\pagestyle{fancy}
\fancyhf{}
\rhead{Avinash Iyer}
\lhead{Problem Set 2}
\newcommand{\card}{\text{card}}
\newcommand{\ran}{\text{ran}}
\newcommand{\N}{\mathbb{N}}
\newcommand{\Q}{\mathbb{Q}}
\newcommand{\Z}{\mathbb{Z}}
\newcommand{\R}{\mathbb{R}}
\begin{document}
  \begin{problem}{Problem 1}
    Let $\mathbb{F}$ be a field. Show that the following hold:
    \begin{enumerate}[(i)]
      \item $-1(a) = -a$
      \item $-(-a) = a$
      \item $-(a+b) = (-a) + (-b)$
      \item $(-a)^{-1} = -(a^{-1})$
      \item $(ab)^{-1} = a^{-1}b^{-1}$
    \end{enumerate}
    \tcblower
    \begin{problem}{(i)}
      \begin{align*}
        0 &= (1 + (-1))\\
        0(a) &= (1 + (-1))a\\
        0 &= 1(a) + (-1)(a)\\
        0 &= a + (-1)(a)\\
        -a &= (-1)(a)
      \end{align*}
    \end{problem}
    \begin{problem}{(ii)}
      \begin{align*}
        0 &= -(-a) + (-a) \\
        a &= -(-a) + ((-a) + a)\\
        a &= -(-a)
      \end{align*}
    \end{problem}
    \begin{problem}{(iii)}
      \begin{align*}
        0 &= -(a+b) + (a+b)\\
        -b &= -(a+b) + a + (b-b)\\
        -a + (-b) &= -(a+b) + (a-a)\\
        (-a) + (-b) &= -(a+b)
      \end{align*}
    \end{problem}
    \begin{problem}{(iv)}
      \begin{align*}
        1 &= (-a)^{-1} (-a)\\
        -1 &= (-a)^{-1}(a)\\
        -1(a^{-1}) &= (-a)^{-1}\\
        -(a^{-1}) &= (-a)^{-1}
      \end{align*}
    \end{problem}
    \begin{problem}{(v)}
      \begin{align*}
        1 &= (ab)^{-1} (ab)\\
        b^{-1} &= (ab)^{-1} (a)\\
        a^{-1}b^{-1} &= (ab)^{-1}
      \end{align*}
    \end{problem}
  \end{problem}
  \begin{problem}{Problem 2}
    Consider the set
    \[
      K:= \{a + b\sqrt{2} \mid a,b\in\Q\}
    \] 
    Show that:
    \begin{enumerate}[(i)]
      \item $x,y\in K \Rightarrow x+y\in K \hat xy\in K$
      \item $x\neq 0 \Rightarrow x^{-1}\in K$
    \end{enumerate}
    \tcblower
    \begin{problem}{(i)}
      Let $x,y\in K$. Then, $x = a+b\sqrt{2}$ and $y = c+d\sqrt{2}$, where $a,b,c,d\in \Q$.\\

      $x+y = (a+c) + (b+d)\sqrt{2}\in K$, as $\Q$ is closed under addition.\\

      $xy = (ac + 2bd) + (ad + bc)\sqrt{2}\in\Q$, as $\Q$ is closed under multiplication.
    \end{problem}
    \begin{problem}{(ii)}
      Let $x=a+b\sqrt{2}\neq 0\in K$. Thus, at least one of $a,b\neq 0$.
      \begin{align*}
        x^{-1} &= \frac{1}{a+b\sqrt{2}}\\
               &= \frac{a-b\sqrt{2}}{a^2-2b^2}\\
               &= \frac{a}{a^2-2b^2} + \frac{-b\sqrt{2}}{a^2-2b^2}
      \end{align*}
      Since $a/(a^2 - 2b^2)$ and $(-b)/(a^2-2b^2)$ are both in $\Q$, $x^{-1}\in K$.
    \end{problem}
  \end{problem}
  \begin{problem}{Problem 3}
    Suppose $\mathbb{F}$ is a field admitting $P\subseteq F$ with the following properties:
    \begin{enumerate}[(C1)]
      \item If $x,y\in P$, then $x+y\in P$ and $xy\in P$
      \item For all $x\in \mathbb{F}$, $x\in P$ or $-x\in P$
      \item If $x,-x\in P$, then $x = 0$.
    \end{enumerate}
    Show that there is an ordering on $\mathbb{F}$ making it into an ordered field.
  \end{problem}
  \begin{problem}{Problem 4}
    Let $a,b\in \R$. Prove the following:
    \begin{enumerate}[(i)]
      \item If $0\leq a\leq \varepsilon$ for all $\varepsilon > 0$, then $a = 0$.
      \item If $a \leq b+\varepsilon$ for all $\varepsilon > 0$, then $a\leq b$.
    \end{enumerate}
    \tcblower
    \begin{problem}{(i)}
      Suppose toward contradiction that $a\neq 0$. Since $a \geq 0$, it must be that $a > 0$, so $\frac{1}{2}a > 0$. Let $\varepsilon = \frac{1}{2}a$. Therefore, $0 < \frac{1}{2}a < a$, which can't be true as $a \leq \varepsilon$ for all $\varepsilon > 0$. $\bot$
    \end{problem}
    \begin{problem}{(ii)}
      Let $a > b$. Then, $\exists \varepsilon > 0$ such that $a \geq b + \varepsilon$, where $0\leq \varepsilon \leq b-a$. Therefore, $a \not\leq b+\varepsilon$ for all $\varepsilon \geq 0$.
    \end{problem}
  \end{problem}
  \begin{problem}{Problem 5}
    If $a,b\in \R$, show that
    \[
      \left(\frac{1}{2}(a+b)\right)^2 \leq \frac{1}{2}(a^2 + b^2)
    \] 
    \tcblower
    \begin{align*}
      \left(\frac{1}{2}(a+b)\right)^2 &= \frac{1}{4}a^2 + \frac{1}{4}b^2 + \frac{1}{2}ab
    \end{align*}
    WLOG, let $a \geq b$. There are three cases: $a,b\in\R^+$, $a\in\R^+$, $-b\in \R^+$, or $-a,-b\in \R^+$.
    \begin{description}[font=\scshape]
      \item[Case 1:] If $a,b\in \R^+$, then $\frac{1}{2}ab \leq \frac{1}{2}a^2$. Since $a^2 \geq b^2$ (as $a \geq b$), it must be that $\frac{1}{2}a^2 \geq \frac{1}{4}a^2 + \frac{1}{4}b^2$.
    \end{description} 
    \begin{align*}
      \left(\frac{1}{2}(a+b)\right)^2 &= \frac{1}{4}a^2 + \frac{1}{4}b^2 + \frac{1}{2}ab\\
                                      &\leq \frac{1}{2}a^2 + \frac{1}{2}b^2\\
                                      &= \frac{1}{2}(a^2 + b^2)
    \end{align*}
    \begin{description}[font=\scshape]
      \item[Case 2:] If $a\in \R^+$ and $-b\in\R^+$, then $-\frac{1}{2}ab\in\R^+$, or $\frac{1}{2}ab < 0$.
    \end{description}
    \begin{align*}
      \left(\frac{1}{2}(a+b)\right)^2 &= \frac{1}{4}a^2 + \frac{1}{4}b^2 + \frac{1}{2}ab\\
                                      &\leq \frac{1}{4}a^2 + \frac{1}{4}b^2\\
                                      &\leq \frac{1}{2}a^2 + \frac{1}{2}b^2\\
                                      &= \frac{1}{2}(a^2 + b^2)
    \end{align*}
    \begin{description}[font=\scshape]
      \item[Case 3:] If $-a,-b\in \R^+$, then $\frac{1}{2}ab \in\R^+$, so we use similar logic to Case 1.
    \end{description}
  \end{problem}
  \begin{problem}{Problem 6}
    For $x\in\R$, show that $\sqrt{x^2} = |x|$.
    \tcblower
    Recall:
    \begin{align*}
      |x| &= \begin{cases}
        x,&x \in \R^+\\
        -x,& x\notin\R^+
      \end{cases}
    \end{align*}
    Suppose $x\in\R^+$. Then, since $\sqrt{x^2}\in\R^+$, and $y^2 = x^2 \Rightarrow y = \pm x$, it must be the case that $\sqrt{x^2} = x$.\\

    Suppose $x\notin\R^+$. Then, $x^2\in\R^+$, so $\sqrt{x^2} \in \R^+$, so $\sqrt{x^2} = -x$.\\

    Thus, $\sqrt{x^2} = |x|$.
  \end{problem}
  \begin{problem}{Problem 7}
    Let $x,y,a,b\in\R$ and $\varepsilon > 0$.
    \begin{enumerate}[(i)]
      \item Show that $|x-a| < \varepsilon$ if and only if $a-\varepsilon < x < a+\varepsilon$
      \item If $a < x < b$ and $a < y < b$, show that $|x-y| < b-a$. What does this mean geometrically?
    \end{enumerate}
    \tcblower
    \begin{problem}{(i)}
      \begin{description}
        \item[$(\Rightarrow)$] Let $|x-a| < \varepsilon$. Then, $x-a < \varepsilon$ and $-(x-a) < \varepsilon$. Thus, $x < a + \varepsilon$ and $-x < \varepsilon - a$, so $a-\varepsilon < x < a+ \varepsilon$.
        \item[$(\Leftarrow)$] Let $a-\varepsilon < x < a+\varepsilon$. Then, $-\varepsilon < (x-a) < \varepsilon$. Therefore, $|x-a| < \varepsilon$.
      \end{description}
    \end{problem}
    \begin{problem}{(ii)}
      Let $a < x < b$ and $a < y < b$. In the second case, we have that $-b < -y < -a$ (by multiplying all the inequalities by $-1$). Adding, we get $a-b < x-y < b-a$, or $-(b-a) < x-y < b-a$. Therefore, $|x-y| < b-a$.
    \end{problem}
  \end{problem}
  \begin{problem}{Problem 8}
    Find all $x\in\R$ that satisfy:
    \[
      4 < |x+2| + |x-1| < 5
    \] 
    \tcblower
    \begin{description}[font=\scshape]
      \item[Case 1: $x < -2$]
    \end{description}
    \begin{align*}
      4 < -(x+2) + -(x-1) &<5\\
      -5<(x+2) + (x-1) &< -4\\
      -5 < 2x+1 &< -4 \\
      -6 < 2x &< -5 \\
      -3 < x &< -2.5
    \end{align*}
    \begin{description}[font=\scshape]
      \item[Case 2: $-2\leq x < 1$]
    \end{description}
    \begin{align*}
      4 < (x+2) + -(x-1) &<5\\
      4 < 2 &< 5 \tag*{$\bot$}
    \end{align*}
    \begin{description}[font=\scshape]
      \item[Case 3: $1 \leq x$]
    \end{description}
    \begin{align*}
      4 < (x+2) + (x-1) &<5\\
      4 < 2x+1 &< 5\\
      1.5 < x &< 2
    \end{align*}
    So the solution is:
    \[
      x\in (-3,-2.5) \cup (1.5,2)
    \] 
  \end{problem}
  \begin{problem}{Problem 9}
    Let $a,b\in\R$. Show that
    \begin{align*}
      \max(a,b) &= \frac{1}{2}(a+b+|a-b|)\\
      \min(a,b) &= \frac{1}{2}(a+b-|a-b|)
    \end{align*}
    \tcblower
    WLOG, let $a > b$. Then:
    \begin{align*}
      \frac{1}{2}(a + b + |a-b|) &= \frac{1}{2}(a + b + (a-b)) \\
                                 &= a\\
      \frac{1}{2}(a+b-|a-b|) &= \frac{1}{2}(a+b-(a-b))\\
                             &= b
    \end{align*}
    Similarly, if $a = b$, then we have that $\max(a,b) = \min(a,b) = a = b$.
  \end{problem}
  \begin{problem}{Problem 10}
    If $x\neq y$ in $\R$, show that there is a $\delta > 0$ such that $V_{\delta}(x) \cap V_{\delta}(y) = \emptyset$.
    \tcblower
  Let $\delta = \frac{1}{2}|x-y|$. Then

  \[V_{\delta}(x) \cap V_{\delta}(y) = \left(x-\frac{1}{2}|x-y|,x+\frac{1}{2}|x-y|\right)\cap \left(y-\frac{1}{2}|x-y|,y+\frac{1}{2}|x-y|\right) = \emptyset\]
  \end{problem}
\end{document}
