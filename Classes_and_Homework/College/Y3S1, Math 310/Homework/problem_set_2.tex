\documentclass[10pt]{extarticle}
\title{}
\author{Avinash Iyer}
\date{}
\usepackage[shortlabels]{enumitem}

%font setup
%
%\usepackage{newpxtext,eulerpx}

%paper setup
\usepackage{geometry}
\geometry{letterpaper, portrait, margin=1in}
\usepackage{fancyhdr}

%symbols
\usepackage{amsmath}
\usepackage{amssymb}
\usepackage{mathtools}
\usepackage{hyperref}
\usepackage{gensymb}
\usepackage{multirow,array}

\usepackage[T1]{fontenc}
\usepackage[utf8]{inputenc}

%chemistry stuff
\usepackage[version=4]{mhchem}
\usepackage{chemfig}

%plotting
\usepackage{pgfplots}
\usepackage{tikz}
\tikzset{middleweight/.style={pos = 0.5, fill=white}}
\tikzset{weight/.style={pos = 0.5, fill = white}}
\tikzset{lateweight/.style={pos = 0.75, fill = white}}
\tikzset{earlyweight/.style={pos = 0.25, fill=white}}

%\usepackage{natbib}

%graphics stuff
\usepackage{graphicx}
\graphicspath{ {./images/} }

%code stuff
%when using minted, make sure to add the -shell-escape flag
%you can use lstlisting if you don't want to use minted
%\usepackage{minted}
%\usemintedstyle{pastie}
%\newminted[javacode]{java}{frame=lines,framesep=2mm,linenos=true,fontsize=\footnotesize,tabsize=3,autogobble,}
%\newminted[cppcode]{cpp}{frame=lines,framesep=2mm,linenos=true,fontsize=\footnotesize,tabsize=3,autogobble,}

%\usepackage{listings}
%\usepackage{color}
%\definecolor{dkgreen}{rgb}{0,0.6,0}
%\definecolor{gray}{rgb}{0.5,0.5,0.5}
%\definecolor{mauve}{rgb}{0.58,0,0.82}
%
%\lstset{frame=tb,
%	language=Java,
%	aboveskip=3mm,
%	belowskip=3mm,
%	showstringspaces=false,
%	columns=flexible,
%	basicstyle={\small\ttfamily},
%	numbers=none,
%	numberstyle=\tiny\color{gray},
%	keywordstyle=\color{blue},
%	commentstyle=\color{dkgreen},
%	stringstyle=\color{mauve},
%	breaklines=true,
%	breakatwhitespace=true,
%	tabsize=3
%}
% text + color boxes
\usepackage[most]{tcolorbox}
\tcbuselibrary{breakable}
\newtcolorbox{problem}[1]{colback = white, title = {#1}, breakable}
\newtcolorbox{solution}{colback = white, colframe = black!75!white, title = Solution, breakable}
%including PDFs
%\usepackage{pdfpages}
\setlength{\parindent}{0pt}

\pagestyle{fancy}
\fancyhf{}
\rhead{Avinash Iyer}
\lhead{Problem Set 2}
\newcommand{\card}{\text{card}}
\newcommand{\ran}{\text{ran}}
\newcommand{\N}{\mathbb{N}}
\newcommand{\Q}{\mathbb{Q}}
\newcommand{\Z}{\mathbb{Z}}
\newcommand{\R}{\mathbb{R}}
\begin{document}
  \begin{problem}{Problem 1}
    Let $\mathbb{F}$ be a field. Show that the following hold:
    \begin{enumerate}[(i)]
      \item $-1(a) = -a$
      \item $-(-a) = a$
      \item $-(a+b) = (-a) + (-b)$
      \item $(-a)^{-1} = -(a^{-1})$
      \item $(ab)^{-1} = a^{-1}b^{-1}$
    \end{enumerate}
    \tcblower
    \begin{problem}{(i)}
      \begin{align*}
        0 &= (1 + (-1))\\
        0(a) &= (1 + (-1))a\\
        0 &= 1(a) + (-1)(a)\\
        0 &= a + (-1)(a)\\
        -a &= (-1)(a)
      \end{align*}
    \end{problem}
    \begin{problem}{(ii)}
      \begin{align*}
        0 &= -(-a) + (-a) \\
        a &= -(-a) + ((-a) + a)\\
        a &= -(-a)
      \end{align*}
    \end{problem}
    \begin{problem}{(iii)}
      \begin{align*}
        0 &= -(a+b) + (a+b)\\
        -b &= -(a+b) + a + (b-b)\\
        -a + (-b) &= -(a+b) + (a-a)\\
        (-a) + (-b) &= -(a+b)
      \end{align*}
    \end{problem}
    \begin{problem}{(iv)}
      \begin{align*}
        1 &= (-a)^{-1} (-a)\\
        -1 &= (-a)^{-1}(a)\\
        -1(a^{-1}) &= (-a)^{-1}\\
        -(a^{-1}) &= (-a)^{-1}
      \end{align*}
    \end{problem}
    \begin{problem}{(v)}
      \begin{align*}
        1 &= (ab)^{-1} (ab)\\
        b^{-1} &= (ab)^{-1} (a)\\
        a^{-1}b^{-1} &= (ab)^{-1}
      \end{align*}
    \end{problem}
  \end{problem}
  \begin{problem}{Problem 2}
    Consider the set
    \[
      K:= \{a + b\sqrt{2} \mid a,b\in\Q\}
    \] 
    Show that:
    \begin{enumerate}[(i)]
      \item $x,y\in K \Rightarrow x+y\in K \hat xy\in K$
      \item $x\neq 0 \Rightarrow x^{-1}\in K$
    \end{enumerate}
    \tcblower
    \begin{problem}{(i)}
      Let $x,y\in K$. Then, $x = a+b\sqrt{2}$ and $y = c+d\sqrt{2}$, where $a,b,c,d\in \Q$.\\

      $x+y = (a+c) + (b+d)\sqrt{2}\in K$, as $\Q$ is closed under addition.\\

      $xy = (ac + 2bd) + (ad + bc)\sqrt{2}\in\Q$, as $\Q$ is closed under multiplication.
    \end{problem}
    \begin{problem}{(ii)}
      Let $x=a+b\sqrt{2}\neq 0\in K$. Thus, at least one of $a,b\neq 0$.
      \begin{align*}
        x^{-1} &= \frac{1}{a+b\sqrt{2}}\\
               &= \frac{a-b\sqrt{2}}{a^2-2b^2}\\
               &= \frac{a}{a^2-2b^2} + \frac{-b\sqrt{2}}{a^2-2b^2}
      \end{align*}
      Since $a/(a^2 - 2b^2)$ and $(-b)/(a^2-2b^2)$ are both in $\Q$, $x^{-1}\in K$.
    \end{problem}
  \end{problem}
\end{document}
