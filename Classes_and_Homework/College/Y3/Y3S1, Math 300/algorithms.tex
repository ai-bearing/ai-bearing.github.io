\documentclass[8pt]{extarticle}
\title{}
\author{Avinash Iyer}
\date{}
\usepackage[shortlabels]{enumitem}


%paper setup
\usepackage{geometry}
\geometry{letterpaper, portrait, margin=1in}
\usepackage{fancyhdr}

%symbols
\usepackage{amsmath}
\usepackage{amssymb}
\usepackage{amsthm}
\usepackage{mathtools}
\usepackage{hyperref}
\usepackage{gensymb}
\usepackage{multirow,array}

\newtheorem*{remark}{Remark}
\usepackage[T1]{fontenc}
\usepackage[utf8]{inputenc}

%chemistry stuff
%\usepackage[version=4]{mhchem}
%\usepackage{chemfig}

%plotting
\usepackage{pgfplots}
\usepackage{tikz}
\tikzset{middleweight/.style={pos = 0.5, fill=white}}
\tikzset{weight/.style={pos = 0.5, fill = white}}
\tikzset{lateweight/.style={pos = 0.75, fill = white}}
\tikzset{earlyweight/.style={pos = 0.25, fill=white}}

%\usepackage{natbib}

%graphics stuff
\usepackage{graphicx}
\graphicspath{ {./images/} }
\usepackage[style=numeric, backend=biber]{biblatex} % Use the numeric style for Vancouver
\addbibresource{the_bibliography.bib}
%code stuff
%when using minted, make sure to add the -shell-escape flag
%you can use lstlisting if you don't want to use minted
%\usepackage{minted}
%\usemintedstyle{pastie}
%\newminted[javacode]{java}{frame=lines,framesep=2mm,linenos=true,fontsize=\footnotesize,tabsize=3,autogobble,}
%\newminted[cppcode]{cpp}{frame=lines,framesep=2mm,linenos=true,fontsize=\footnotesize,tabsize=3,autogobble,}

%\usepackage{listings}
%\usepackage{color}
%\definecolor{dkgreen}{rgb}{0,0.6,0}
%\definecolor{gray}{rgb}{0.5,0.5,0.5}
%\definecolor{mauve}{rgb}{0.58,0,0.82}
%
%\lstset{frame=tb,
%	language=Java,
%	aboveskip=3mm,
%	belowskip=3mm,
%	showstringspaces=false,
%	columns=flexible,
%	basicstyle={\small\ttfamily},
%	numbers=none,
%	numberstyle=\tiny\color{gray},
%	keywordstyle=\color{blue},
%	commentstyle=\color{dkgreen},
%	stringstyle=\color{mauve},
%	breaklines=true,
%	breakatwhitespace=true,
%	tabsize=3
%}
% text + color boxes
\usepackage[most]{tcolorbox}
\tcbuselibrary{breakable}
\newtcolorbox{problem}[1]{colback = white, title = {#1}, breakable}
\newtcolorbox{solution}{colback = white, colframe = black!75!white, title = Solution, breakable}
%including PDFs
%\usepackage{pdfpages}
\setlength{\parindent}{0pt}
\usepackage{cancel}
\pagestyle{fancy}
\fancyhf{}
\rhead{Avinash Iyer}
\lhead{Newton's Method and other Algorithms}
\newcommand{\card}{\text{card}}
\newcommand{\ran}{\text{ran}}
\newcommand{\N}{\mathbb{N}}
\newcommand{\Q}{\mathbb{Q}}
\newcommand{\Z}{\mathbb{Z}}
\newcommand{\R}{\mathbb{R}}
\begin{document}
  Given $f(x)$, we want to find a value $x'$ such that $f(x') = 0$.
  \begin{enumerate}[(1)]
    \item Pick a value $x_0$ such that $x_0\in[f(a),f(b)]$, where $f(a) < 0$ and $f(b) > 0$.
    \item Take
      \begin{align*}
        x_{n+1} &= x_n - \frac{f(x_n)}{f'(x_n)}
      \end{align*}
  \end{enumerate}
  For example, take $f(x) = x^2 - 2$. We know that $f(1) < 0$ and $f(2) > 0$. Take $x_0 = 0$.
  \begin{align*}
    x_1 &= 1 - \frac{1^2 - 2}{2}\\
        &= \frac{3}{2} = 1.5
    x_2 &= \frac{3}{2} - \frac{\left(\frac{3}{2}\right)^2 - 2}{3}\\
        &= \frac{17}{12} = 1.41\overline{6}
  \end{align*}
  Newton's method has quadratic convergence. However, we can look at an even better algorithm.\\
  \vspace{4pt}
  \rule{\textwidth}{0.4pt}
  \vspace{4pt}
  Via the Taylor series, we know that
  \begin{align*}
    f(x_{n+1}) &\approx f(x_n) + f'(x_n)(x_{n+1} - x_n)\\
    0 &\approx f(x_n) + f'(x_n)(x_{n+1} - x_n)
  \end{align*}
  However, we can make this better by adding another term, creating cubic convergence.
  \begin{align*}
    0 &= f(x_n) + f'(x_n)(x_{n+1}-x_n) + \frac{f''(x_n)}{2}\left(x_{n+1}-x_n\right)^2\\
    x_{n+1}-x_n &= \frac{-f'(x_n) \pm \sqrt{f'(x_n)^2 - 2f''(x_n)f(x_n)}}{f''(x_n)}\\
    x_{n+1} &= x_n -\frac{f'(x_n)}{f''(x_n)} \mp \frac{\sqrt{f'(x_n)^2 - 2f''(x_n)f(x_n)}}{f''(x_n)}\\
    \shortintertext{after tedious algebra,}
    x_{n+1} &= x_n - \frac{f(x_n)}{f'(x_n)}-\frac{f(x_n)^2f''(x_n)}{2f'(x_n)^3} \tag*{cubic convergence formula}
  \end{align*}
\end{document}
