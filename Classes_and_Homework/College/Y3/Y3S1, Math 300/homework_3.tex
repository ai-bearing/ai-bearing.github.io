\documentclass[9pt]{extarticle}
\title{}
\author{Avinash Iyer}
\date{}
\usepackage[shortlabels]{enumitem}


%paper setup
\usepackage{geometry}
\geometry{letterpaper, portrait, margin=1in}
\usepackage{fancyhdr}

%symbols
\usepackage{amsmath}
\usepackage{amssymb}
\usepackage{amsthm}
\usepackage{mathtools}
\usepackage{hyperref}
\usepackage{gensymb}
\usepackage{multirow,array}
\newtheorem{theorem}{Theorem}
\usepackage[T1]{fontenc}
\usepackage[utf8]{inputenc}

%chemistry stuff
%\usepackage[version=4]{mhchem}
%\usepackage{chemfig}

%plotting
\usepackage{pgfplots}
\usepackage{tikz}
\tikzset{middleweight/.style={pos = 0.5, fill=white}}
\tikzset{weight/.style={pos = 0.5, fill = white}}
\tikzset{lateweight/.style={pos = 0.75, fill = white}}
\tikzset{earlyweight/.style={pos = 0.25, fill=white}}

%\usepackage{natbib}

%graphics stuff
\usepackage{graphicx}
\graphicspath{ {./images/} }
\usepackage[style=numeric, backend=biber]{biblatex} % Use the numeric style for Vancouver
\addbibresource{the_bibliography.bib}
%code stuff
%when using minted, make sure to add the -shell-escape flag
%you can use lstlisting if you don't want to use minted
%\usepackage{minted}
%\usemintedstyle{pastie}
%\newminted[javacode]{java}{frame=lines,framesep=2mm,linenos=true,fontsize=\footnotesize,tabsize=3,autogobble,}
%\newminted[cppcode]{cpp}{frame=lines,framesep=2mm,linenos=true,fontsize=\footnotesize,tabsize=3,autogobble,}

\usepackage{listings}
%\usepackage{color}
%\definecolor{dkgreen}{rgb}{0,0.6,0}
%\definecolor{gray}{rgb}{0.5,0.5,0.5}
%\definecolor{mauve}{rgb}{0.58,0,0.82}
%

\lstset{
  breaklines = true,
  basicstyle = \small\ttfamily,
}
%\lstset{frame=tb,
%	language=Java,
%	aboveskip=3mm,
%	belowskip=3mm,
%	showstringspaces=false,
%	columns=flexible,
%	basicstyle={\small\ttfamily},
%	numbers=none,
%	numberstyle=\tiny\color{gray},
%	keywordstyle=\color{blue},
%	commentstyle=\color{dkgreen},
%	stringstyle=\color{mauve},
%	breaklines=true,
%	breakatwhitespace=true,
%	tabsize=3
%}
% text + color boxes
\usepackage[most]{tcolorbox}
\tcbuselibrary{breakable}
\newtcolorbox{problem}[1]{colback = white, title = {#1}, breakable}
\newtcolorbox{solution}{colback = white, colframe = black!75!white, title = Solution, breakable}
%including PDFs
%\usepackage{pdfpages}
\setlength{\parindent}{0pt}
\usepackage{cancel}
\pagestyle{fancy}
\fancyhf{}
\rhead{Avinash Iyer}
\lhead{Editing Assignments}
\newcommand{\card}{\text{card}}
\newcommand{\ran}{\text{ran}}
\newcommand{\N}{\mathbb{N}}
\newcommand{\Q}{\mathbb{Q}}
\newcommand{\Z}{\mathbb{Z}}
\newcommand{\R}{\mathbb{R}}
\begin{document}
  \begin{lstlisting}
  \begin{description}
    \item[Theorem 1] \hfill
      \begin{itemize}
        \item The theorem statement is incorrect: for example, if $a=6,b=3,c=4$, then $a|(bc)$ but $a\not| b$ and $a\not|c$.
        \item The proof only looks at one case and generalizes to the entire integers.
      \end{itemize}
  \end{description}
  \begin{problem}{Corrected Theorem and Proof}
    \begin{theorem}
      Let $a,~b,~c\in \Z$ such that $a<b<c$. If $a|(bc)$, then $a|b$ or $a|c$
    \end{theorem}
    \begin{proof}
      Suppose toward contradiction that for  $a,b,c\in\Z$, $a \vert (bc)$, $a\not\vert b$, and $a\not\vert c$. Then $\forall x,y\in\Z$, $b\neq xa$ and $c \neq ya$. Then, $bc \neq (xy)a$. However, this means $a \not\vert bc$, as $xy\in\Z$. $\bot$
    \end{proof}
  \end{problem}
  %%%%%%%%%%%%%%%%%%%%%%%%%%%%%%%%%%%%%%%%%%%%%%
  %%%%%%%%%%%%%%%%%%%%%%%%%%%%%%%%%%%%%%%%%%%%%%
  \begin{description}
    \item[Theorem 2] \hfill
      \begin{itemize}
        \item The proof states the wrong assumption.
        \item The proof states the wrong conclusion, it is supposed to be that $1 > \frac{1}{a}$, not $a > \frac{1}{a}$.
      \end{itemize}
  \end{description}
  \begin{problem}{Corrected Theorem and Proof}
    \begin{theorem}
      If $a\in\R$ and $a > 1$, then $0 < \frac{1}{a} < 1$.
    \end{theorem}
    \begin{proof}
      Let $a\in\R,~a>1$. Then, $a > 0$, so $\frac{1}{a} > 0$.\\

      By the order properties of $\R$, multiplication by $\frac{1}{a}$ must preserve the sign in the inequality $a > 1$. So, $\frac{a}{a} > \frac{1}{a}$.\\

      Thus, we have $0 < \frac{1}{a} < 1$.
    \end{proof}
  \end{problem}

  %%%%%%%%%%%%%%%%%%%%%%%%%%%%%%%%%%%%%%%%%%%%%%
  %%%%%%%%%%%%%%%%%%%%%%%%%%%%%%%%%%%%%%%%%%%%%%
  \begin{description}
    \item[Theorem 3] \hfill
      \begin{itemize}
        \item Instead of \verb|\abs{x}|, the command for absolute value is $|x|$.
        \item Instead of \verb|\epsilon|, the proof writer should have used \verb|\varepsilon|.
        \item The proof does not state that it is toward contradiction.
      \end{itemize}
  \end{description}
  \begin{problem}{Corrected Theorem and Proof}
    \begin{theorem}
      If $|x| < \varepsilon$ for every real number $\varepsilon > 0$, then $x = 0$.
    \end{theorem}
    \begin{proof}
      Suppose toward contradiction that $\exists x \neq 0 0$ such that $|x| < \varepsilon$ for every $\varepsilon > 0$.

      Then, $\frac{|x|}{2} > 0$, as $|x| \neq 0$. This means $|x| < \frac{|x|}{2}$ by the theorem hypothesis.\\

      Dividing by $|x|$, we get $1 < \frac{1}{2}$. $\bot$
    \end{proof}
  \end{problem}

  %%%%%%%%%%%%%%%%%%%%%%%%%%%%%%%%%%%%%%%%%%%%%%
  %%%%%%%%%%%%%%%%%%%%%%%%%%%%%%%%%%%%%%%%%%%%%%
  \begin{description}
    \item[Theorem 4]\hfill
      \begin{itemize}
        \item The theorem's proof uses $k$ to denote the values of both $a$ and $b$.
      \end{itemize}
  \end{description}
  \begin{problem}{Corrected Theorem and Proof}
    \begin{theorem}
      Let $a,b\in\Z$ where $a\equiv 1 \mod 3$ and $b\equiv 2 \mod 3$. Then, $(a+b)\equiv 0 \mod 3$.
    \end{theorem}
    \begin{proof}
      Let $a\equiv 1 \mod 3$ and $b\equiv 2 \mod 3$. Then, for some $k,\ell\in\Z$, $a = 3k+1$ and $b = 3\ell + 2$.\\

      Then, $a+b = (3k+1) + (3\ell + 2) = 3(k + \ell + 1)$, so $a + b \equiv 0 \mod 3$.
    \end{proof}
  \end{problem}

  %%%%%%%%%%%%%%%%%%%%%%%%%%%%%%%%%%%%%%%%%%%%%%
  %%%%%%%%%%%%%%%%%%%%%%%%%%%%%%%%%%%%%%%%%%%%%%
  \begin{description}
    \item[Theorem 5]\hfill
      \begin{itemize}
        \item The proof of the theorem is often imprecise, using words such as ``impossible,'' and does not state that it is toward contradiction.
        \item In the third sentence of the proof, math mode is used even though the word ``and'' should not be in math mode.
      \end{itemize}
  \end{description}
  \begin{problem}{Corrected Theorem and Proof}
    \begin{theorem}
      There are no integers $a,b$ for which $2a + 4b = 1$.
    \end{theorem}
    \begin{proof}
      Suppose toward contradiction that $\exists a,b\in \Z$ such that $2a + 4b = 1$. Then, $a + 2b = \frac{1}{2}$. However, since $a,b\in\Z$, and $a + 2b$ are all operations 
    \end{proof}
  \end{problem}

  %%%%%%%%%%%%%%%%%%%%%%%%%%%%%%%%%%%%%%%%%%%%%%
  %%%%%%%%%%%%%%%%%%%%%%%%%%%%%%%%%%%%%%%%%%%%%%
  \begin{description}
    \item[Theorem 6]\hfill
      \begin{itemize}
        \item The proof of the theorem uses $k$ in reference to both $n$ and $n^2 + 5$.
      \end{itemize}
  \end{description}
  \begin{problem}{Corrected Theorem and Proof}
    \begin{theorem}
      Let $n$ be an integer. If $n^2 + 5$ is odd, then $n$ is even.
    \end{theorem}
    \begin{proof}
      Let $n$ be odd. Then, $n = 2k + 1$ for some $k\in\Z$. So, $n^2 + 5 = (2k+1)^2 + 5$, or $(4k^2 + 4k + 1) + 5$. So, $n^2 + 5 = 2(2k^2 + 2k + 3)$.
    \end{proof}
  \end{problem}
  \begin{description}
    \item[Theorem 7] \hfill
      \begin{itemize}
        \item The proof states that it is ``to the contrary,'' rather than by contradiction.
        \item $n^2$ cannot be less than $n$ by the ordering properties of $\N$.
      \end{itemize}
  \end{description}
  \begin{description}
    \item[Theorem 8]\hfill
      \begin{itemize}
        \item The series $1 - \frac{1}{2} + \frac{1}{3} - \frac{1}{4} + \cdots$ is conditionally convergent, meaning that its terms can be rearranged to satisfy any condition, implying that this proof cannot hold.
        \item On the first line of the second \verb|align| section, there are two equal signs.
        \item On the third line, a $-$ is put in place of a $2$.
      \end{itemize}
  \end{description}
  \end{lstlisting}
  \begin{description}
    \item[Theorem 1] \hfill
      \begin{itemize}
        \item The theorem statement is incorrect: for example, if $a=6,b=3,c=4$, then $a|(bc)$ but $a\not| b$ and $a\not|c$.
        \item The proof only looks at one case and generalizes to the entire integers.
      \end{itemize}
  \end{description}
  \begin{problem}{Corrected Theorem and Proof}
    \begin{theorem}
      Let $a,~b,~c\in \Z$ such that $a<b<c$. If $a|(bc)$, then $a|b$ or $a|c$
    \end{theorem}
    \begin{proof}
      Suppose toward contradiction that for  $a,b,c\in\Z$, $a \vert (bc)$, $a\not\vert b$, and $a\not\vert c$. Then $\forall x,y\in\Z$, $b\neq xa$ and $c \neq ya$. Then, $bc \neq (xy)a$. However, this means $a \not\vert bc$, as $xy\in\Z$. $\bot$
    \end{proof}
  \end{problem}
  %%%%%%%%%%%%%%%%%%%%%%%%%%%%%%%%%%%%%%%%%%%%%%
  %%%%%%%%%%%%%%%%%%%%%%%%%%%%%%%%%%%%%%%%%%%%%%
  \begin{description}
    \item[Theorem 2] \hfill
      \begin{itemize}
        \item The proof states the wrong assumption.
        \item The proof states the wrong conclusion, it is supposed to be that $1 > \frac{1}{a}$, not $a > \frac{1}{a}$.
      \end{itemize}
  \end{description}
  \begin{problem}{Corrected Theorem and Proof}
    \begin{theorem}
      If $a\in\R$ and $a > 1$, then $0 < \frac{1}{a} < 1$.
    \end{theorem}
    \begin{proof}
      Let $a\in\R,~a>1$. Then, $a > 0$, so $\frac{1}{a} > 0$.\\

      By the order properties of $\R$, multiplication by $\frac{1}{a}$ must preserve the sign in the inequality $a > 1$. So, $\frac{a}{a} > \frac{1}{a}$.\\

      Thus, we have $0 < \frac{1}{a} < 1$.
    \end{proof}
  \end{problem}

  %%%%%%%%%%%%%%%%%%%%%%%%%%%%%%%%%%%%%%%%%%%%%%
  %%%%%%%%%%%%%%%%%%%%%%%%%%%%%%%%%%%%%%%%%%%%%%
  \begin{description}
    \item[Theorem 3] \hfill
      \begin{itemize}
        \item Instead of \verb|\abs{x}|, the command for absolute value is $|x|$.
        \item Instead of \verb|\epsilon|, the proof writer should have used \verb|\varepsilon|.
        \item The proof does not state that it is toward contradiction.
      \end{itemize}
  \end{description}
  \begin{problem}{Corrected Theorem and Proof}
    \begin{theorem}
      If $|x| < \varepsilon$ for every real number $\varepsilon > 0$, then $x = 0$.
    \end{theorem}
    \begin{proof}
      Suppose toward contradiction that $\exists x \neq 0 0$ such that $|x| < \varepsilon$ for every $\varepsilon > 0$.

      Then, $\frac{|x|}{2} > 0$, as $|x| \neq 0$. This means $|x| < \frac{|x|}{2}$ by the theorem hypothesis.\\

      Dividing by $|x|$, we get $1 < \frac{1}{2}$. $\bot$
    \end{proof}
  \end{problem}

  %%%%%%%%%%%%%%%%%%%%%%%%%%%%%%%%%%%%%%%%%%%%%%
  %%%%%%%%%%%%%%%%%%%%%%%%%%%%%%%%%%%%%%%%%%%%%%
  \begin{description}
    \item[Theorem 4]\hfill
      \begin{itemize}
        \item The theorem's proof uses $k$ to denote the values of both $a$ and $b$.
      \end{itemize}
  \end{description}
  \begin{problem}{Corrected Theorem and Proof}
    \begin{theorem}
      Let $a,b\in\Z$ where $a\equiv 1 \mod 3$ and $b\equiv 2 \mod 3$. Then, $(a+b)\equiv 0 \mod 3$.
    \end{theorem}
    \begin{proof}
      Let $a\equiv 1 \mod 3$ and $b\equiv 2 \mod 3$. Then, for some $k,\ell\in\Z$, $a = 3k+1$ and $b = 3\ell + 2$.\\

      Then, $a+b = (3k+1) + (3\ell + 2) = 3(k + \ell + 1)$, so $a + b \equiv 0 \mod 3$.
    \end{proof}
  \end{problem}

  %%%%%%%%%%%%%%%%%%%%%%%%%%%%%%%%%%%%%%%%%%%%%%
  %%%%%%%%%%%%%%%%%%%%%%%%%%%%%%%%%%%%%%%%%%%%%%
  \begin{description}
    \item[Theorem 5]\hfill
      \begin{itemize}
        \item The proof of the theorem is often imprecise, using words such as ``impossible,'' and does not state that it is toward contradiction.
        \item In the third sentence of the proof, math mode is used even though the word ``and'' should not be in math mode.
      \end{itemize}
  \end{description}
  \begin{problem}{Corrected Theorem and Proof}
    \begin{theorem}
      There are no integers $a,b$ for which $2a + 4b = 1$.
    \end{theorem}
    \begin{proof}
      Suppose toward contradiction that $\exists a,b\in \Z$ such that $2a + 4b = 1$. Then, $a + 2b = \frac{1}{2}$. However, since $a,b\in\Z$, and $a + 2b$ are all operations 
    \end{proof}
  \end{problem}

  %%%%%%%%%%%%%%%%%%%%%%%%%%%%%%%%%%%%%%%%%%%%%%
  %%%%%%%%%%%%%%%%%%%%%%%%%%%%%%%%%%%%%%%%%%%%%%
  \begin{description}
    \item[Theorem 6]\hfill
      \begin{itemize}
        \item The proof of the theorem uses $k$ in reference to both $n$ and $n^2 + 5$.
      \end{itemize}
  \end{description}
  \begin{problem}{Corrected Theorem and Proof}
    \begin{theorem}
      Let $n$ be an integer. If $n^2 + 5$ is odd, then $n$ is even.
    \end{theorem}
    \begin{proof}
      Let $n$ be odd. Then, $n = 2k + 1$ for some $k\in\Z$. So, $n^2 + 5 = (2k+1)^2 + 5$, or $(4k^2 + 4k + 1) + 5$. So, $n^2 + 5 = 2(2k^2 + 2k + 3)$.
    \end{proof}
  \end{problem}
  \begin{description}
    \item[Theorem 7] \hfill
      \begin{itemize}
        \item The proof states that it is ``to the contrary,'' rather than by contradiction.
        \item $n^2$ cannot be less than $n$ by the ordering properties of $\N$.
      \end{itemize}
  \end{description}
  \begin{description}
    \item[Theorem 8]\hfill
      \begin{itemize}
        \item The series $1 - \frac{1}{2} + \frac{1}{3} - \frac{1}{4} + \cdots$ is conditionally convergent, meaning that its terms can be rearranged to satisfy any condition, implying that this proof cannot hold.
        \item On the first line of the second \verb|align| section, there are two equal signs.
        \item On the third line, a $-$ is put in place of a $2$.
      \end{itemize}
  \end{description}
\end{document}
