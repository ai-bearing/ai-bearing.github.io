\documentclass[9pt]{extarticle}
\title{}
\author{}
\date{}
\usepackage[shortlabels]{enumitem}


%paper setup
\usepackage{geometry}
\geometry{letterpaper, portrait, margin=1in}
\usepackage{fancyhdr}
% sans serif font:
%\usepackage{cmbright}
%symbols
\usepackage{amsmath}
\usepackage{bigints}
\usepackage{amssymb}
\usepackage{amsthm}
\usepackage{mathtools}
\usepackage{bbold}
\usepackage[hidelinks]{hyperref}
\usepackage{gensymb}
\usepackage{multirow,array}
\usepackage{multicol}

\newtheorem*{remark}{Remark}
\usepackage[T1]{fontenc}
\usepackage[utf8]{inputenc}

%chemistry stuff
%\usepackage[version=4]{mhchem}
%\usepackage{chemfig}

%plotting
\usepackage{pgfplots}
\usepackage{tikz}
\tikzset{middleweight/.style={pos = 0.5}}
%\tikzset{weight/.style={pos = 0.5, fill = white}}
%\tikzset{lateweight/.style={pos = 0.75, fill = white}}
%\tikzset{earlyweight/.style={pos = 0.25, fill=white}}

%\usepackage{natbib}

%graphics stuff
\usepackage{graphicx}
\graphicspath{ {./images/} }
\usepackage[style=numeric, backend=biber]{biblatex} % Use the numeric style for Vancouver
\addbibresource{the_bibliography.bib}
%code stuff
%when using minted, make sure to add the -shell-escape flag
%you can use lstlisting if you don't want to use minted
%\usepackage{minted}
%\usemintedstyle{pastie}
%\newminted[javacode]{java}{frame=lines,framesep=2mm,linenos=true,fontsize=\footnotesize,tabsize=3,autogobble,}
%\newminted[cppcode]{cpp}{frame=lines,framesep=2mm,linenos=true,fontsize=\footnotesize,tabsize=3,autogobble,}

%\usepackage{listings}
%\usepackage{color}
%\definecolor{dkgreen}{rgb}{0,0.6,0}
%\definecolor{gray}{rgb}{0.5,0.5,0.5}
%\definecolor{mauve}{rgb}{0.58,0,0.82}
%
%\lstset{frame=tb,
%	language=Java,
%	aboveskip=3mm,
%	belowskip=3mm,
%	showstringspaces=false,
%	columns=flexible,
%	basicstyle={\small\ttfamily},
%	numbers=none,
%	numberstyle=\tiny\color{gray},
%	keywordstyle=\color{blue},
%	commentstyle=\color{dkgreen},
%	stringstyle=\color{mauve},
%	breaklines=true,
%	breakatwhitespace=true,
%	tabsize=3
%}
% text + color boxes
\usepackage[most]{tcolorbox}
\tcbuselibrary{breakable}
\tcbuselibrary{skins}
\newtcolorbox{problem}[1]{colback=white,enhanced,title={\small #1},
          attach boxed title to top center=
{yshift=-\tcboxedtitleheight/2},
boxed title style={size=small,colback=black!60!white}, sharp corners, breakable}
%including PDFs
%\usepackage{pdfpages}
\setlength{\parindent}{0pt}
\usepackage{cancel}
%\pagestyle{fancy}
%\fancyhf{}
%\rhead{Avinash Iyer}
%\lhead{}
\newcommand{\card}{\text{card}}
\newcommand{\ran}{\text{ran}}
\newcommand{\N}{\mathbb{N}}
\newcommand{\Q}{\mathbb{Q}}
\newcommand{\Z}{\mathbb{Z}}
\newcommand{\R}{\mathbb{R}}
\newcommand{\C}{\mathbb{C}}
\newcommand{\iprod}[2]{\left\langle #1,#2\right\rangle}
\newcommand{\norm}[1]{\left\Vert #1\right\Vert}
\setcounter{secnumdepth}{0}
\begin{document}
  \begin{center}
    {\bf \Large Math 395 \\[0.1in]Homework 2 \\[0.1in]
    Due: 2/8/2024}\\[.25in]
    {\bf Name:} {Avinash Iyer}\\[0.15in]
    {\bf Collaborators:} {Nate Hall, Nora Manukyan, Gianluca Crescenzo, Isaac Ochoa, Jamie Perez-Schere, Timothy Rainone} \\
  \end{center}
  \section{Problem 1}%
  Let $R$ be a ring with identity and $I$ an ideal in $R$.
  \begin{enumerate}[(a)]
    \item We will prove that if $I$ contains a unit, then $I = R$.\\

      Specifically, by the definition of a unit $u$, for all $a\in R$, $ua = au = u$.\\

      If $u\in I$, then by the definition of ideal, $au\in I$ and $ua \in I$ for all $a\in R$. Therefore, $a\in I$ for all $a\in R$, meaning $I = R$.
    \item Let $F$ be a field. We will show that if $I$ is an ideal in $F$, then $I = \{0_F\}$ or $I = F$.\\

      Clearly, $I = \{0_F\}$ is an ideal --- $I$ is closed under subtraction, multiplication, and multiplication by elements of $F$ (as for $a\in F$, $a\cdot 0_F = 0_F \cdot a = 0_F$).\\

      Suppose that $I$ contains at least one element, $a$, where $a\neq 0_F$. Then, since $a \neq 0_F$, there is a multiplicative identity for $a$, $1/a$ such that $a\cdot 1/a = 1/a \cdot a = 1_F$. Since $I$ is an ideal, this means $I$ contains $a \cdot 1/a$ as $I$ is closed under multiplication by elements of the ring.\\

      Therefore, $I$ contains a unit of $F$ (namely, $1_F$), meaning $I = F$ by the result from (a).
  \end{enumerate}
  \section{Problem 2}%
  Let $I,J$ be ideals in ring $R$. Define $I+J = \{i+j\mid i\in I,j\in J\}$. This is referred to as the sum of the ideals.
  \begin{enumerate}[(a)]
    \item We will prove that $I + J$ is an ideal in $R$ that contains $I$ and $J$.\\

      To start, since $I$ and $J$ are ideals in $R$, $I$ and $J$ are each subrings of $R$, meaning both $I$ and $J$ contain $0_R$. Therefore, taking $j = 0_R$, we find that $\{i + 0_R\mid i\in I\}\subseteq I+J$, and similarly, taking $i = 0_R$, we find that $\{0_R + j\mid j\in J\}\subseteq I+J $. These sets are, respectively, $I$ and $J$, meaning $I$ and $J$ are both subsets of $I+J$.\\

      We will now show $I+J$ is an ideal in $R$. First, $I+J$ is non-empty since, as exhibited earlier, both $I$ and $J$ are subrings, meaning $0_R\in I$ and $0_R\in J$, so $0_R + 0_R = 0_R\in I+J$. Let $x,y\in I+J$. Then, $x = x_i + x_j$ and $y = y_i + y_j$ for some $x_i,y_i\in I$ and $x_j,y_j\in J$. Then,
      \begin{align*}
        x-y &= (x_i + x_j) - (y_i + y_j)\\
            &= (x_i - y_i) + (x_j - y_j),
      \end{align*}
      which is an element of $I + J$. Similarly, 
      \begin{align*}
        xy &= (x_i + x_j)(y_i + y_j)\\
           &= (x_iy_i) + (x_jy_j + x_iy_j + x_jy_i).
      \end{align*}
      Since $x_iy_i\in I$, as $I$ is a subring, and $x_jy_j\in J$, as $J$ is a subring, as well as $x_iy_j\in J$ and $x_jy_i\in J$ as $J$ is an ideal, we have that $x_jy_j + x_iy_j + x_jy_i\in J$, so $xy\in I+J$.\\

      Finally, we will show that $I+J$ is closed under multiplication by elements from $R$. Let $r\in R$, $a\in I+J$. Then, $a = a_i + a_j$ for $a_i\in I$ and $a_j\in J$. So,
      \begin{align*}
        ra &= r(a_i + a_j)\\
           &= ra_i + ra_j,
           \shortintertext{and}
        ar &= (a_i + a_j)r\\
           &= a_ir + a_jr,
      \end{align*}
      and since $I$ and $J$ are both ideals, $ra_i,a_ir\in I$ and $ra_j,a_jr\in J$, so $ar,ra\in I+J$.\\

      Therefore, $I+J$ is an ideal that contains $I$ and $J$.
    \item Let $a,b\in\mathbf{Z}$. We will show that $a\mathbf{Z} + b\mathbf{Z} = \gcd(a,b)\mathbf{Z}$.\\

      By Bezout's identity, it is the case that there are integers $x$ and $y$ such that $xa + yb = \gcd(a,b)$. Since $xa\in a\mathbf{Z}$, and $yb\in b\mathbf{Z}$, as $a\mathbf{Z}$ and $b\mathbf{Z}$ are ideals in $\mathbf{Z}$, it is the case that for any $n\in\mathbf{Z}$, $n(xa + yb) \in a\mathbf{Z} + b\mathbf{Z}$. Therefore, $\gcd(a,b)\mathbf{Z}\subseteq a\mathbf{Z} + b\mathbf{Z}$.\\

      For any $na + mb\in a\mathbf{Z} + b\mathbf{Z}$, there exist $k,\ell\in \mathbf{Z}$ such that $na = k\gcd(a,b)$ and $mb = \ell\gcd(a,b)$, by definition of greatest common divisor. Therefore, $na + mb = (k+\ell)\gcd(a,b)\in \gcd(a,b)\mathbf{Z}$, so $a\mathbf{Z}+b\mathbf{Z}\subseteq \gcd(a,b)\mathbf{Z}$.\\

      Since $\gcd(a,b)\mathbf{Z}\subseteq a\mathbf{Z} + b\mathbf{Z}$, and $a\mathbf{Z} + b\mathbf{Z}\subseteq \gcd(a,b)\mathbf{Z}$, it is the case that $a\mathbf{Z} + b\mathbf{Z} = \gcd(a,b)\mathbf{Z}$.
    \item We will prove that if $\gcd(a,b) = 1$, then $a\mathbf{Z}\cap b\mathbf{Z} = ab\mathbf{Z}$.\\

      To start, since $a$ divides all members of $ab\mathbf{Z}$, $ab\mathbf{Z}\subseteq a\mathbf{Z}$, and since $b$ divides all members of $ab\mathbf{Z}$, $ab\mathbf{Z}\subseteq b\mathbf{Z}$, meaning $ab\mathbf{Z}\subseteq a\mathbf{Z}\cap b\mathbf{Z}$.\\

      Let $k\in a\mathbf{Z} \cap b\mathbf{Z}$. Then, $k$ is a common multiple of $a$ and $b$. Therefore, $k$ is an integer multiple of $\text{lcm}(a,b)$, or $\frac{ab}{\gcd(a,b)}$. Since $\gcd(a,b) = 1$, $k$ is an integer multiple of $ab$. Therefore, $k\in ab\mathbf{Z}$, meaning $a\mathbf{Z}\cap b\mathbf{Z}\subseteq ab\mathbf{Z}$.\\

      Since $ab\mathbf{Z}\subseteq a\mathbf{Z}\cap b\mathbf{Z}$, and $a\mathbf{Z}\cap b\mathbf{Z} \subseteq ab\mathbf{Z}$, it is the case that $ab\mathbf{Z} = a\mathbf{Z} \cap b\mathbf{Z}$.
  \end{enumerate}
  \section{Problem 3}%
  Let $p$ be a prime number and let $T$ denote the set of rational numbers in reduced form whose denominators are not divisible by $p$.
  \begin{enumerate}[(a)]
    \item We will prove that $T$ is a ring by showing closure under addition, identity and inverse under addition, commutativity of addition, closure under multiplication, associativity under multiplication, and distribution of multiplication over addition.\\

      Let $\frac{a}{b},\frac{c}{d}\in T$ denote such rational numbers in lowest terms that satisfy the condition that $p$ does not divide $b$ and $d$, meaning that $p$ is not a prime factor of either $b$ or $d$. Then,
      \begin{align*}
        \frac{a}{b} + \frac{c}{d} &= \frac{ad + bc}{bd},
      \end{align*}
      and since the prime factors of $bd$ are precisely the prime factors multiplied by the prime factors of $d$, and $p$ is not a prime factor of $b$ or $d$, p is not a prime factor of $bd$, meaning $p$ does not divide $bd$. Therefore, $T$ is closed under addition.\\

      The additive identity in lowest terms in $T$ is inherited from the rational numbers --- namely, $0$. Since $p$ does not divide $0$, it is the case that $T$ contains the additive identity.\\

      The additive inverse to $\frac{a}{b}\in T$ is $\frac{-a}{b}\in T$; since $p$ does not divide $b$ by definition, it is the case that $\frac{-a}{b}$ satisfies the necessary condition for $T$.\\

      Since addition under $T$ is inherited from addition under the rational numbers, addition in $T$ is commutative, meaning $T$ is an abelian group under addition.\\

      Let $\frac{a}{b},\frac{c}{d}\in T$, meaning $p$ does not divide $c$ and $p$ does not divide $d$. Then,
      \begin{align*}
        \left(\frac{a}{b}\right)\left(\frac{c}{d}\right) &= \frac{ac}{bd},
      \end{align*}
      so by the same logic as with addition, $p$ does not divide $bd$, meaning $T$ is closed under multiplication.\\

      Since multiplication is associative and distributive under the rational numbers, and $T$ inherits these properties, it is the case that multiplication is associative and distributes over the rational numbers.\\

      Therefore, $T$ satisfies the necessary requirements for a ring.
    \item Let $I$ be the set of elements in $T$ such that the numerator is divisible by $p$. We will show that $I$ is an ideal by showing that $I$ is a subring and multiplication by any element of $T$ yields an element of $I$.\\

      Since $0\in I$, as the rational number $0$ is divisible by every number, it is the case that $I$ is non-empty. Let $\frac{a}{b},\frac{c}{d}\in I$. Then, $a = pk$ and $c = p\ell$ for some $k$ and $\ell$. Thus,
      \begin{align*}
        \frac{a}{b}-\frac{c}{d} &= \frac{pk}{b} - \frac{p\ell}{d}\\
                                &= \frac{pkd - p\ell b}{bd}\\
                                &= \frac{p(kd - \ell b)}{bd},
      \end{align*}
      meaning that $I$ is closed under subtraction. Similarly,
      \begin{align*}
        \left(\frac{a}{b}\right)\left(\frac{c}{d}\right) &= \frac{(pk)(p\ell)}{bd}\\
                                                         &= \frac{p(pk\ell)}{bd},
      \end{align*}
      meaning $I$ is closed under multiplication.
    \item We will show that $T/I$ has $p$ distinct cosets.\\

      To start, we will show that for some $p\not|b$, we will show that, modulo $p$, $\{0,b,2b,\dots,(p-1)b\}$ are distinct. Suppose toward contradiction that $kb \equiv \ell b$ modulo $p$ for some $k,\ell \in \{0,1,\dots,p-1\}$. Then, $kb - \ell b \equiv 0$ modulo $p$, implying that $p|b(k-\ell)$. However, since $p\not| b$, it is the case that $p|(k-\ell)$. However, since $k,\ell\in \{0,1,\dots,p-1\}$, $p|(k-\ell)$ if and only if $k = \ell$.\\

      By the definition of the equivalence relation of ideals, 
      \begin{align*}
        \frac{a}{b}&\sim \frac{k}{1}
        \shortintertext{if}
        \frac{a}{b}-\frac{k}{1} &\in I
      \end{align*}
      for some $k \in \{0,1,\dots,p-1\}$. Therefore, $\frac{a-kb}{b}\in I$, so $p|a-kb$, so $a-kb \equiv 0$ modulo $p$. Therefore, $a \equiv kb$ modulo $p$. Since $p$ is prime, and $p\not| b$, we can take inverses to find $\frac{a}{b} \equiv k$ modulo $p$.\\

      Therefore, the cosets of $T/I$ are the $\left[\frac{k}{1}\right]_{T/I}$ that each $\frac{a}{b}\in T$ is equal to within the quotient ring.
    \item Let $\varphi: T/I\rightarrow \mathbf{Z}/p\mathbf{Z}$ be defined as $\varphi \left(\frac{a}{b}\right) = \left[\frac{a}{b}\right]_{p}$. We will show that $\varphi$ is an isomorphism.\\

      Let $\left[\frac{a}{b}\right]_{T/I} = \left[\frac{c}{d}\right]_{T/I}$. Then, $ad-bc \equiv 0$ modulo $p$. Applying $\varphi$ to both sides, we get that $\left[\frac{a}{b}\right]_p = \left[\frac{c}{d}\right]_p$, meaning $ad - bc \equiv 0$ modulo $p$. Therefore, $\varphi$ is well-defined.\\

      We will now show that $\varphi$ is a ring homomorphism. Let $\frac{a}{b},\frac{c}{d}\in T/I$. Then,
      \begin{align*}
        \varphi\left(\left(\frac{a}{b}\right)\left(\frac{c}{d}\right)\right) &= \left[\frac{a}{b}\frac{c}{d}\right]_p,\\
        \shortintertext{and by the properties of $\mathbf{Z}/p\mathbf{Z}$,}
                                                   &= \left[\frac{a}{b}\right]_p \left[\frac{c}{d}\right]_p\\
                                                   &= \varphi\left(\frac{a}{b}\right)\varphi\left(\frac{c}{d}\right).
                                                   \shortintertext{Similarly,}
        \varphi\left(\frac{a}{b} + \frac{c}{d}\right) &= \left[\frac{a}{b} + \frac{c}{d}\right]_p,\\
        \shortintertext{and by the properties of $\mathbf{Z}/p\mathbf{Z}$,}
                                                      &= \left[\frac{a}{b}\right]_p + \left[\frac{c}{d}\right]_p\\
                                                      &= \varphi\left(\frac{a}{b}\right) + \varphi\left(\frac{c}{d}\right).
      \end{align*}
      Therefore, $\varphi$ is a ring homomorphism.\\

      We will now show that $\varphi$ is a bijection. Clearly, $\varphi$ is surjective, as we can select any $\frac{a}{b}\in T/I$ such that $\frac{a}{b}\in \mathbf{Z}/p\mathbf{Z}$. To show that $\varphi$ is injective, let $\varphi\left(\frac{a}{b}\right) = \varphi\left(\frac{c}{d}\right).$ Then,
      \begin{align*}
        \left[\frac{a}{b}\right]_p &= \left[\frac{c}{d}\right]_p,\\
        \shortintertext{so}
        \frac{a}{b} \equiv \frac{c}{d} \text{ modulo $p$.}\\
        \intertext{Therefore, by the definition of equivalence modulo $p$,}
        ad -bc \equiv 0 \text{ modulo $p$,}\\
        \intertext{so}
        \frac{a}{d} &\sim_{I}\frac{c}{d}.
      \end{align*}
      Since $\varphi$ is a bijective ring homomorphism, $\varphi$ is an isomorphism, meaning $T/I \cong \mathbf{Z}/p\mathbf{Z}$.
  \end{enumerate}
  \section{Problem 5}%
  Let $\varphi: R\rightarrow S$ be a ring homomorphism. We will prove that $\varphi$ is injective if and only if $\ker \varphi = \{0_F\}$.\\

  In the forwards direction, we let $\varphi$ be injective. Then, $\varphi(0_R) = 0_S$ by the definition of a ring homomorphism. Since, for any $a \in R, a\neq 0_R$, $\varphi(a)$ cannot equal $0_S$ (or else $\varphi$ would not be injective), this means $\ker\varphi = \{0_R\}$.\\

  In the reverse direction, we let $\ker\varphi = \{0_R\}$. Let $\varphi(a) = \varphi(b)$. Then, $\varphi(a) - \varphi(b) = \varphi(b) - \varphi(b)$, meaning $\varphi(a) - \varphi(b) = 0_S$. By the definition of a ring homomorphism, this is equivalent to $\varphi(a-b) = 0_S$. Since $\ker\varphi = \{0_R\}$, we have $a-b = 0_R$, or $a = b$. Thus, $\varphi$ is injective.
\end{document}
