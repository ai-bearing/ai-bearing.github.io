\documentclass[12pt]{extarticle}
\title{}
\author{}
\date{}
\usepackage[shortlabels]{enumitem}


%paper setup
\usepackage{geometry}
\geometry{letterpaper, portrait, margin=1in}
\usepackage{fancyhdr}
% sans serif font:
\usepackage{cmbright}
%symbols
\usepackage{amsmath}
\usepackage{bigints}
\usepackage{esint}
\usepackage{amssymb}
\usepackage{amsthm}
\usepackage{mathtools}
\usepackage{bbm}
\usepackage[hidelinks]{hyperref}
\usepackage{gensymb}
\usepackage{multirow,array}
\usepackage{multicol}

\newtheorem*{remark}{Remark}
\usepackage[T1]{fontenc}
\usepackage[utf8]{inputenc}

%chemistry stuff
%\usepackage[version=4]{mhchem}
%\usepackage{chemfig}

%plotting
\usepackage{pgfplots}
\usepackage{tikz}
\tikzset{middleweight/.style={pos = 0.5}}
%\tikzset{weight/.style={pos = 0.5, fill = white}}
%\tikzset{lateweight/.style={pos = 0.75, fill = white}}
%\tikzset{earlyweight/.style={pos = 0.25, fill=white}}

%\usepackage{natbib}

%graphics stuff
\usepackage{graphicx}
\graphicspath{ {./images/} }
\usepackage[style=numeric, backend=biber]{biblatex} % Use the numeric style for Vancouver
\addbibresource{the_bibliography.bib}
%code stuff
%when using minted, make sure to add the -shell-escape flag
%you can use lstlisting if you don't want to use minted
%\usepackage{minted}
%\usemintedstyle{pastie}
%\newminted[javacode]{java}{frame=lines,framesep=2mm,linenos=true,fontsize=\footnotesize,tabsize=3,autogobble,}
%\newminted[cppcode]{cpp}{frame=lines,framesep=2mm,linenos=true,fontsize=\footnotesize,tabsize=3,autogobble,}

%\usepackage{listings}
%\usepackage{color}
%\definecolor{dkgreen}{rgb}{0,0.6,0}
%\definecolor{gray}{rgb}{0.5,0.5,0.5}
%\definecolor{mauve}{rgb}{0.58,0,0.82}
%
%\lstset{frame=tb,
%	language=Java,
%	aboveskip=3mm,
%	belowskip=3mm,
%	showstringspaces=false,
%	columns=flexible,
%	basicstyle={\small\ttfamily},
%	numbers=none,
%	numberstyle=\tiny\color{gray},
%	keywordstyle=\color{blue},
%	commentstyle=\color{dkgreen},
%	stringstyle=\color{mauve},
%	breaklines=true,
%	breakatwhitespace=true,
%	tabsize=3
%}
% text + color boxes
\renewcommand{\mathbf}[1]{\mathbbm{#1}}
\usepackage[most]{tcolorbox}
\tcbuselibrary{breakable}
\tcbuselibrary{skins}
\newtcolorbox{problem}[1]{colback=white,enhanced,title={\small #1},
          attach boxed title to top center=
{yshift=-\tcboxedtitleheight/2},
boxed title style={size=small,colback=black!60!white}, sharp corners, breakable}
%including PDFs
%\usepackage{pdfpages}
\setlength{\parindent}{0pt}
\usepackage{cancel}
\pagestyle{fancy}
\fancyhf{}
\rhead{Avinash Iyer}
\lhead{Flux Integral Evaluation Methods}
\newcommand{\card}{\text{card}}
\newcommand{\ran}{\text{ran}}
\newcommand{\N}{\mathbbm{N}}
\newcommand{\Q}{\mathbbm{Q}}
\newcommand{\Z}{\mathbbm{Z}}
\newcommand{\R}{\mathbbm{R}}
\setcounter{secnumdepth}{0}
\begin{document}
\renewcommand{\arraystretch}{1.5}
  \section{Rectangular Coordinates}%
  \begin{itemize}
    \item Integrating over surface defined in rectangular coordinates.
    \item Primarily applies to non-closed surfaces defined in rectangular coordinates.
  \end{itemize}
  \begin{align*}
    z &= f(x,y)\\
    \vec{F} &= \vec{F}(x,y,f(x,y))\\
    \int_{S} \vec{F} \cdot d\vec{A} &= \int_{S} \vec{F} \cdot \begin{pmatrix}-\frac{\partial f}{\partial x}\\
  -\frac{\partial f}{\partial y}\\1\end{pmatrix}~dx~dy
  \end{align*}
  \section{Cylindrical Coordinates}%
  \begin{itemize}
    \item Integrating over side of non-closed cylinder with defined radius $R$.
  \end{itemize}
  \begin{align*}
    \vec{F} &= \vec{F}(R,\theta,z)\\
    x &= R\cos\theta\\
    y &= R\sin\theta\\
    z &= z\\
    \int_{S} \vec{F}\cdot d\vec{A} &= \int_{z_1}^{z_2}\int_{\theta_1}^{\theta_2}\vec{F} \cdot \begin{pmatrix}\cos\theta\\\sin\theta\\0\end{pmatrix}R~d\theta~dz
  \end{align*}
  \section{Spherical Coordinates}%
  \begin{itemize}
    \item Integrating over shell of non-closed sphere with defined radius $\rho$.
  \end{itemize}
  \begin{align*}
    \vec{F} &= \vec{F}(\rho,\theta,\varphi)\\
    x &= \rho\cos\theta\sin\varphi\\
    y &= \rho\sin\theta\sin\varphi\\
    z &= \rho\cos\varphi\\
    \int_{S} \vec{F} \cdot d\vec{A} &= \int_{\varphi_1}^{\varphi_2}\int_{\theta_1}^{\theta_2}\vec{F} \cdot \begin{pmatrix}\cos\theta\sin\varphi\\\sin\theta\sin\varphi\\\cos\varphi\end{pmatrix}\rho^2\sin\varphi~d\theta~d\varphi
  \end{align*}
  \section{Divergence Theorem}%
  \begin{itemize}
    \item Flux integral over closed surface $S$, defined by $W$ in any coordinate system.
  \end{itemize}
  \begin{align*}
    \oiint_{S}\vec{F}\cdot d\vec{A} &= \iiint_{W}\nabla\cdot \vec{F}~dV
  \end{align*}
  \section{Stokes Theorem}%
  \begin{itemize}
    \item Flux integral of curl of field over open surface $S$ (evaluate using above techniques), with border $C$ in any coordinate system.
  \end{itemize}
  \begin{align*}
    \int_{S}\nabla\times\vec{F} \cdot d\vec{A} &= \oint_{C}\vec{F}\cdot d\vec{r}
  \end{align*}
\end{document}
