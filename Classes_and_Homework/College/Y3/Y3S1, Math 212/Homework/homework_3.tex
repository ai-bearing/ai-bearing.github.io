\documentclass[10pt]{extarticle}
\title{}
\author{Avinash Iyer}
\date{}
\usepackage[shortlabels]{enumitem}


%paper setup
\usepackage{geometry}
\geometry{letterpaper, portrait, margin=1in}
\usepackage{fancyhdr}

%symbols
\usepackage{amsmath}
\usepackage{amssymb}
\usepackage{amsthm}
\usepackage{mathtools}
\usepackage{hyperref}
\usepackage{gensymb}
\usepackage{multirow,array}

\newtheorem*{remark}{Remark}
\usepackage[T1]{fontenc}
\usepackage[utf8]{inputenc}

%chemistry stuff
%\usepackage[version=4]{mhchem}
%\usepackage{chemfig}

%plotting
\usepackage{pgfplots}
\usepackage{tikz}
\tikzset{middleweight/.style={pos = 0.5, fill=white}}
\tikzset{weight/.style={pos = 0.5, fill = white}}
\tikzset{lateweight/.style={pos = 0.75, fill = white}}
\tikzset{earlyweight/.style={pos = 0.25, fill=white}}

%\usepackage{natbib}

%graphics stuff
\usepackage{graphicx}
\graphicspath{ {./images/} }
\usepackage[style=numeric, backend=biber]{biblatex} % Use the numeric style for Vancouver
\addbibresource{the_bibliography.bib}
%code stuff
%when using minted, make sure to add the -shell-escape flag
%you can use lstlisting if you don't want to use minted
%\usepackage{minted}
%\usemintedstyle{pastie}
%\newminted[javacode]{java}{frame=lines,framesep=2mm,linenos=true,fontsize=\footnotesize,tabsize=3,autogobble,}
%\newminted[cppcode]{cpp}{frame=lines,framesep=2mm,linenos=true,fontsize=\footnotesize,tabsize=3,autogobble,}

%\usepackage{listings}
%\usepackage{color}
%\definecolor{dkgreen}{rgb}{0,0.6,0}
%\definecolor{gray}{rgb}{0.5,0.5,0.5}
%\definecolor{mauve}{rgb}{0.58,0,0.82}
%
%\lstset{frame=tb,
%	language=Java,
%	aboveskip=3mm,
%	belowskip=3mm,
%	showstringspaces=false,
%	columns=flexible,
%	basicstyle={\small\ttfamily},
%	numbers=none,
%	numberstyle=\tiny\color{gray},
%	keywordstyle=\color{blue},
%	commentstyle=\color{dkgreen},
%	stringstyle=\color{mauve},
%	breaklines=true,
%	breakatwhitespace=true,
%	tabsize=3
%}
% text + color boxes
\usepackage[most]{tcolorbox}
\tcbuselibrary{breakable}
\newtcolorbox{problem}[1]{colback = white, title = {#1}, breakable}
\newtcolorbox{solution}{colback = white, colframe = black!75!white, title = Solution, breakable}
%including PDFs
%\usepackage{pdfpages}
\setlength{\parindent}{0pt}
\usepackage{cancel}
\pagestyle{fancy}
\fancyhf{}
\rhead{Avinash Iyer}
\lhead{Math 212: Homework 3}
\newcommand{\card}{\text{card}}
\newcommand{\ran}{\text{ran}}
\newcommand{\N}{\mathbb{N}}
\newcommand{\Q}{\mathbb{Q}}
\newcommand{\Z}{\mathbb{Z}}
\newcommand{\R}{\mathbb{R}}
\begin{document}
  \begin{problem}{13.2}
    \begin{description}[font=\normalfont]
      \item[10:]
        \begin{align*}
          \tan^{-1}\left(\frac{18}{15}\right) &= 50.2^{\circ}
        \end{align*}
      \item[16:]
        \begin{align*}
          F' &= -(F_1 + F_2)\\
             &= -11\hat{i} + 4\hat{j}
        \end{align*}
      \item[22:]
        \begin{align*}
          \shortintertext{Find vector sum of first two forces:}
          \vec{u} &= (70\cos(80\degree) + 100\cos(30\degree))\hat{i} + (-70\cos(80\degree) + 100\cos(30\degree))\hat{j} + 0\hat{k}\\
                  &= 98.8\hat{i} - 18.9\hat{j} + 0\hat{k}\\
          \shortintertext{Magnitude of $\vec{u}$:}
            \Vert\vec{u}\Vert &= \sqrt{(70\cos(80\degree) + 100\cos(30\degree))^2 + (-70\cos(80\degree) + 100\cos(30\degree))^2}\\
                              &= 100.55\\
                              \shortintertext{Magnitude of $\hat{k}$ component:}
            a &= \sqrt{(3000)^2 - \left((70\cos(80\degree) + 100\cos(30\degree))^2 + (-70\cos(80\degree) + 100\cos(30\degree))^2\right)}\\
              &= 2998.3\\
            \vec{v} &= -98.8\hat{i} + 18.9\hat{j} + 2998.3\hat{k}
        \end{align*}
      \item[24:]
        \begin{align*}
          \vec{f} &= \frac{2}{3}\vec{w} + \frac{1}{3}\vec{v}\\
                  &= (79,79.3,89,68.3,89.3)
        \end{align*}
    \end{description}
  \end{problem}
  \begin{problem}{13.3}
    \begin{description}[font=\normalfont]
      \item[4:] 
        \begin{align*}
          (2\hat{i} + 5\hat{k}) \cdot 10\hat{j} &= 0
        \end{align*}
      \item[6:] 
        \begin{align*}
          \vec{u} \cdot \vec{w} &= \Vert \vec{u} \Vert \Vert \vec{w} \Vert \cos(120\degree)\\
                                &= -100
        \end{align*}
      \item[12:]
        \begin{align*}
          \vec a \cdot (\vec{c} + \vec y) &= -2
        \end{align*}
      \item[14:]
        \begin{align*}
          (\vec a \cdot \vec y)(\vec c \cdot \vec z) &= 238
        \end{align*}
      \item[22:]
        \begin{align*}
          5x + 4y - z &= 3
        \end{align*}
      \item[24:]
        \begin{align*}
          5x + y - 2z &= 3
        \end{align*}
      \item[32:]
        \begin{align*}
          \theta &= \cos^{-1}\left(\frac{\vec u \cdot \vec v}{\Vert \vec u \Vert \Vert \vec v \Vert}\right)\\
                 &= 57.9\degree
        \end{align*}
      \item[44:]
        \begin{enumerate}[(a)]
          \item $\vec u = \hat{i} + 2\hat{j} - \hat{k}$
          \item
            \begin{align*}
              \theta &= \cos^{-1}\left(\frac{\vec u \cdot \vec v}{\Vert \vec u \Vert \Vert \vec v \Vert}\right)\\
                     &= 136.8\degree
            \end{align*}
          \item The angle between $\vec v$ and the plane is thus $46.8\degree$
        \end{enumerate}
      \item[54:] There are two vectors such that $\vec a \cdot \vec b = 4$, with $\cos(\theta) = 0.5 \Rightarrow \Theta = 60\degree$ and $\theta = 300\degree$
      \item[60:]
        \begin{align*}
          W &= \vec F \cdot \overrightarrow{PQ}\\
            &= -6\text{ J}\\
            &= 4.43\text{ ft-lb}
        \end{align*}
    \end{description}
  \end{problem}
  \begin{problem}{13.4}
    \begin{description}[font=\normalfont]
      \item[2:]
        \begin{align*}
          \vec v \times \vec w &= -\hat{i}
        \end{align*}
      \item[4:]
        \begin{align*}
          \vec v \cdot \vec w &= -2\hat{i} + 2\hat{j}
        \end{align*}
      \item[10:]
        \begin{align*}
          \left((\hat{i} + \hat{j})\times \hat{i}\right)\times \hat{j} &= (\hat{i}\times\hat{i} + \hat{i}\times\hat{j})\times\hat{j}\\
                                                                       &= \hat{k}\times\hat{j}\\
                                                                       &= -\hat{i}
        \end{align*}
      \item[12:]
        \begin{align*}
          \shortintertext{Finding $\vec a \times \vec b$}
          \vec a \times \vec b &= 
          \begin{vmatrix}
            \hat{i} & \hat{j} & \hat{k} \\
            3 & 1 & -1\\
            1 & -4 & 2
          \end{vmatrix}\\
                               &= -2\hat{i} - 7\hat{j} -13\hat{k}\\
          \shortintertext{Checking $\vec a\cdot(\vec a\times \vec b)$:}
            \vec a \cdot (\vec a \times \vec b) &= (-2)(3) + (-7)(1) + (-1)(-13)\\
                                                &= 0
          \shortintertext{Checking that $\vec b \cdot(\vec a \times \vec b)$:}
            \vec b \cdot(\vec a \times \vec b)&= (-2)(1) + (-4)(-7) + (2)(-13)\\
                                              &= 0
        \end{align*}
      \item[14:]
        \begin{align*}
          (-\hat{i} + \hat{j}) \times (-\hat{j} + \hat{k}) &= \hat{i} + \hat{j} + \hat{k}\\
          \shortintertext{Therefore, the plane is:}
          x+y+z = 1
        \end{align*}
      \item[34 (a):]
        \begin{align*}
          (4\hat{i} - \hat{j} + 4\hat{k})\times(-2\hat{i} - 3\hat{j} + \hat{k}) &= 11\hat{i} -12\hat{j}-14\hat{k}\\
          \shortintertext{Therefore, the plane is:}
          11x - 12y -14z = -45
        \end{align*}
      \item[36:] $\vec v \times \vec w$ is parallel to the $z$ axis, as both of its constituent vectors must lie on the $xy$-plane, so their cross product must be perpendicular to both.
      \item[38:] 
        \begin{align*}
          \Vert \vec v \times \vec w \Vert &= \sqrt{38}\\
          \tan\theta &= \frac{\Vert \vec v \times \vec w\Vert}{\vec v \cdot \vec w}\\
                     &= \frac{\sqrt{38}}{3}
        \end{align*}
      \item[40:] Since $\vec v \times (\hat i + \hat j + \hat k) = 0$, $\vec v$ is parallel to $(\hat i + \hat j + \hat k)$, so $\vec v = 2\hat i + 2\hat j + 2\hat k$.
    \end{description}
  \end{problem}
  \begin{problem}{14.1}
    \begin{description}[font=\normalfont]
      \item[2:]
        \begin{align*}
          f_x(3,2) &= \lim_{h \rightarrow 0}\frac{\frac{(3+h)^2}{3} - 3}{h}\\
                   &= \frac{(3+h)^2 - 9}{3h}\\
                   &= \frac{h^2 + 6h}{3h}
                   &= 2\\
          f_y(3,2) &= \lim_{k\rightarrow 0}\frac{\frac{9}{3 + h} - 3}{h}
                   &= \frac{3h}{(3+h)h}\\
                   &= 1
        \end{align*}
      \item[4:]
        \begin{enumerate}[(a)]
          \item Change in price as a function of age.
          \item Negative, due to depreciation of the car.
          \item Change in price as a function of original cost.
          \item Positive, because a more originally expensive car will have a higher price at any given age.
        \end{enumerate}
      \item[10:]
        \begin{enumerate}[(a)]
          \item $f(A) = 10$
          \item Positive
          \item Zero
        \end{enumerate}
      \item[14:]
        \begin{enumerate}[(a)]
          \item $f(A) = 40$
          \item Negative
          \item Positive
        \end{enumerate}
      \item[16:]
        \begin{itemize}
          \item $f_x > 0$
          \item $f_y > 0$
        \end{itemize}
      \item[20:]
        \begin{enumerate}[(a)]
          \item Positive
          \item Negative
          \item Positive
          \item Negative
        \end{enumerate}
      \item[22:]
        \begin{align*}
          f_x(3,5) \approx -\frac{2}{3}
        \end{align*}
      \item[36:]
        \begin{enumerate}[(a)]
          \item
            \begin{align*}
              \frac{\partial T}{\partial x}\bigr|_{15,20} &\approx -\frac{1}{2}~\text{\textdegree C per meter}\\
              \frac{\partial T}{\partial x}\bigr|_{15,20} &\approx \frac{1}{2}~\text{\textdegree C per minute}
            \end{align*}
            The wall heats up at $\frac{1}{2}$ a degree Celsius per minute, and cools by $\frac{1}{2}$ degree Celsius per meter away from the heat source.
          \item
            \begin{align*}
              \frac{\partial T}{\partial x}\bigr|_{5,12} &\approx 1~\text{\textdegree C per meter}\\
              \frac{\partial T}{\partial x}\bigr|_{5,12} &\approx \frac{1}{6}~\text{\textdegree C per minute}
            \end{align*}
            The wall heats up at $\frac{1}{6}$ of a degree Celsius per minute, and cools by $1$ degree Celsius per meter away from the heat source.
        \end{enumerate}
    \end{description}
  \end{problem}
\end{document}
