\documentclass[10pt]{extarticle}
\title{}
\author{Avinash Iyer}
\date{}
\usepackage[shortlabels]{enumitem}


%paper setup
\usepackage{geometry}
\geometry{letterpaper, portrait, margin=1in}
\usepackage{fancyhdr}

%symbols
\usepackage{amsmath}
\usepackage{amssymb}
\usepackage{amsthm}
\usepackage{mathtools}
\usepackage{hyperref}
\usepackage{gensymb}
\usepackage{multirow,array}

\newtheorem*{remark}{Remark}
\usepackage[T1]{fontenc}
\usepackage[utf8]{inputenc}

%chemistry stuff
%\usepackage[version=4]{mhchem}
%\usepackage{chemfig}

%plotting
\usepackage{pgfplots}
\usepackage{tikz}
\tikzset{middleweight/.style={pos = 0.5, fill=white}}
\tikzset{weight/.style={pos = 0.5, fill = white}}
\tikzset{lateweight/.style={pos = 0.75, fill = white}}
\tikzset{earlyweight/.style={pos = 0.25, fill=white}}

%\usepackage{natbib}

%graphics stuff
\usepackage{graphicx}
\graphicspath{ {./images/} }
\usepackage[style=numeric, backend=biber]{biblatex} % Use the numeric style for Vancouver
\addbibresource{the_bibliography.bib}
%code stuff
%when using minted, make sure to add the -shell-escape flag
%you can use lstlisting if you don't want to use minted
%\usepackage{minted}
%\usemintedstyle{pastie}
%\newminted[javacode]{java}{frame=lines,framesep=2mm,linenos=true,fontsize=\footnotesize,tabsize=3,autogobble,}
%\newminted[cppcode]{cpp}{frame=lines,framesep=2mm,linenos=true,fontsize=\footnotesize,tabsize=3,autogobble,}

%\usepackage{listings}
%\usepackage{color}
%\definecolor{dkgreen}{rgb}{0,0.6,0}
%\definecolor{gray}{rgb}{0.5,0.5,0.5}
%\definecolor{mauve}{rgb}{0.58,0,0.82}
%
%\lstset{frame=tb,
%	language=Java,
%	aboveskip=3mm,
%	belowskip=3mm,
%	showstringspaces=false,
%	columns=flexible,
%	basicstyle={\small\ttfamily},
%	numbers=none,
%	numberstyle=\tiny\color{gray},
%	keywordstyle=\color{blue},
%	commentstyle=\color{dkgreen},
%	stringstyle=\color{mauve},
%	breaklines=true,
%	breakatwhitespace=true,
%	tabsize=3
%}
% text + color boxes
\usepackage[most]{tcolorbox}
\tcbuselibrary{breakable}
\newtcolorbox{problem}[1]{colback = white, title = {#1}, breakable}
\newtcolorbox{solution}{colback = white, colframe = black!75!white, title = Solution, breakable}
%including PDFs
%\usepackage{pdfpages}
\setlength{\parindent}{0pt}
\usepackage{cancel}
\pagestyle{fancy}
\fancyhf{}
\rhead{Avinash Iyer}
\lhead{Math 212: Homework 6}
\newcommand{\card}{\text{card}}
\newcommand{\ran}{\text{ran}}
\newcommand{\N}{\mathbb{N}}
\newcommand{\Q}{\mathbb{Q}}
\newcommand{\Z}{\mathbb{Z}}
\newcommand{\R}{\mathbb{R}}
\begin{document}
\renewcommand{\arraystretch}{1.5}
  \begin{problem}{15.3}
    \begin{description}[font=\normalfont]
      \item[2:]
        \begin{align*}
          \nabla f &= \lambda \nabla g\\
          \begin{pmatrix}1\\3\end{pmatrix} &= \lambda \begin{pmatrix}2x\\2y\end{pmatrix}\\
          2\lambda x &= 1\\
          2\lambda y &= 3\\
          x &= \frac{1}{2\lambda}\\
          y &= \frac{3}{2\lambda}\\
          x^2 + y^2 &= 10\\
          \frac{10}{4\lambda^2} &= 10\\
          \lambda &= \pm\frac{1}{2}\\
          x &= \pm 1\\
          y &= \pm 3\\
          f &= 12,-8
        \end{align*}
        Therefore, $f$ is maximized subject to the constraint at $(1,3,12)$ and minimized at $(-1,-3,-8)$.
      \item[4:]
        \begin{align*}
          \nabla f &= \lambda \nabla g\\
          \begin{pmatrix}3x^2 \\ 1\end{pmatrix} &= \lambda \begin{pmatrix}6x\\2y\end{pmatrix}\\
          6\lambda x &= 3x^2\\
          3x\left(x - 6\lambda\right) &= 0\\
          x &= 0, 6\lambda\\
          2\lambda y &= 1\\
          y &= \frac{1}{2\lambda}\\
          3x^2 + y^2 &= 4\\
          \frac{1}{4\lambda^2} &= 4 \tag*{$x = 0$}\\
          \lambda &= \pm \frac{1}{4}\\
          (x,y) &= \left(0,\pm \frac{1}{4}\right)\\
          f(x,y) &= \pm\frac{1}{4}\\
          \frac{1}{4\lambda^2} + \frac{1}{4\lambda^2} &= 4\tag*{$x = 6\lambda$}\\
          \lambda &= \pm\frac{1}{2\sqrt{2}}\\
          x &= \pm\frac{3}{\sqrt{2}}\\
          y &= \pm \frac{1}{4\sqrt{2}}\\
          (x,y) &= \left(\pm \frac{3}{\sqrt{2}},\pm\frac{1}{4\sqrt{2}}\right)\\
          f(x,y) &= \pm\frac{55}{2\sqrt{2}}
        \end{align*}
        Therefore, $f$ is maximized subject to the constraint at $\left(\frac{3}{\sqrt{2}},\frac{1}{4\sqrt{2}},\frac{55}{2\sqrt{2}}\right)$ and minimized at $\left(-\frac{3}{\sqrt{2}},-\frac{1}{4\sqrt{2}},-\frac{55}{2\sqrt{2}}\right)$.
      \item[10:]
        \begin{align*}
          \nabla f &= \lambda \nabla g\\
          \begin{pmatrix}1\\3\\5\end{pmatrix} &= \lambda \begin{pmatrix}2x\\2y\\2z\end{pmatrix}\\
          2\lambda x &= 1\\
          2\lambda y &= 3\\
          2\lambda z &= 5\\
          x^2 + y^2 + z^2 &= 1\\
          \frac{35}{4\lambda^2} &= 1\\
          \lambda &= \frac{\pm\sqrt{35}}{2}\\
          (x,y,z) &= \left(\pm\frac{2}{\sqrt{35}},\pm\frac{6}{\sqrt{35}},\pm\frac{10}{\sqrt{35}}\right)\\
          f(x,y,z) &= \pm2\sqrt{35}
        \end{align*}
        Therefore, $f$ is maximized at $\left(\frac{2}{\sqrt{35}},\frac{6}{\sqrt{35}},\frac{10}{\sqrt{35}},2\sqrt{35}\right)$, and minimized at $\left(-\frac{2}{\sqrt{35}},-\frac{6}{\sqrt{35}},-\frac{10}{\sqrt{35}},-2\sqrt{35}\right)$
      \item[12:]
        \begin{align*}
          \nabla f &= \lambda \nabla g\\
          \begin{pmatrix}yz\\xz\\xy\end{pmatrix} &= \lambda \begin{pmatrix}2x\\2y\\8z\end{pmatrix}\\
          2\lambda x &= yz\\
          2\lambda y &= xz\\
          8\lambda z &= xy\\
          z &= \frac{xy}{8\lambda}\\
          2\lambda x &= \frac{y^2x}{8\lambda}\\
          16\lambda^2 &= y^2 \tag*{$x\neq 0$}\\
          16\lambda^2 &= x^2 \tag*{$y \neq 0$}\\
          z^2 &= 32\lambda^2\\
          x^2 + y^2 + 4z^2 &= 12\\
          32\lambda^2 + 128\lambda^2 &= 12\\
          \lambda^2 &= \frac{3}{40}\\
          \lambda &= \pm \sqrt{\frac{3}{40}}\\
          x &= \pm \sqrt{\frac{6}{5}}\\
          y &= \pm \sqrt{\frac{6}{5}}\\
          z &= \pm \sqrt{\frac{12}{5}}\\
          f(x,y,z) &= \pm\sqrt\frac{6\sqrt{12}}{5\sqrt{5}}
        \end{align*}
        Therefore, $f$ is maximized when $x,y,z$ are positive at $\frac{6\sqrt{12}}{5}$ and minimized when $x,y,z$ are negative at $-\frac{6\sqrt{12}}{5}$.
      \item[36:]
        \begin{align*}
          f &= 2\pi r^2 + 2\pi r h\\
          \pi r^2 h &= 100\\
          \nabla f &= \lambda \nabla g\\
          \begin{pmatrix}4\pi r + 2\pi h\\2\pi r\end{pmatrix} &= \lambda \begin{pmatrix}2\pi r h \\ \pi r^2\end{pmatrix}\\
          2\pi\lambda r h &= 4\pi r + 2\pi h\\
          \pi\lambda r^2 &= 2\pi r\\
          r &= \frac{2}{\lambda}\\
          4\pi h &= \frac{8\pi}{\lambda} + 2\pi h\\
          h &= \frac{4}{\lambda}\\
          \pi r^2 h &= 100\\
          \pi \frac{16}{\lambda^3} &= 100\\
          \lambda &= \sqrt[3]{\frac{16\pi}{100}}\\
          r &= \frac{2\sqrt[3]{100}}{\sqrt[3]{16\pi}}\\
          h &= \frac{4\sqrt[3]{100}}{\sqrt[3]{16\pi}}
        \end{align*}
    \end{description}
  \end{problem}
  \begin{problem}{16.1}
    \begin{description}[font=\normalfont]
      \item[2:]
        \begin{align*}
          \shortintertext{Lower Estimate:}
          \int_{R}f(x,y) dA &\approx (4)(0.1)(0.2) + (6)(0.1)(0.2) + (3)(0.1)(0.2) + (5)(0.1)(0.2)\\
                            &= 0.36\\
          \shortintertext{Upper Estimate:}
          \int_{R}f(x,y) dA &\approx (7)(0.1)(0.2) + (10)(0.1)(0.2) + (6)(0.1)(0.2) + (8)(0.1)(0.2)\\
                            &= 62
        \end{align*}
      \item[4:]
        \begin{align*}
          \shortintertext{Lower Estimate:}
          \int_{R}f(x,y) dA &\approx (50)\left(2+4+8+4+6+8+6+8+10\right)\\
                            &= 2800\\
          \shortintertext{Upper Estimate:}
          \int_{R}f(x,y) dA &\approx (50)\left(4+6+8+6+8+8+8+10+10\right)\\
                            &= 3400
        \end{align*}
      \item[6:] The integral represents total bacteria population.
      \item[8:] The integral is positive.
      \item[14:] The integral is negative.
      \item[20:] I don't know how to do this problem.
    \end{description}
  \end{problem}
\end{document}
