\documentclass[8pt]{extarticle}
\title{}
\author{}
\date{}
\usepackage[shortlabels]{enumitem}


%paper setup
\usepackage{geometry}
\geometry{letterpaper, portrait, margin=1in}
\usepackage{fancyhdr}
% sans serif font:
\usepackage{cmbright}
%symbols
\usepackage{amsmath}
\usepackage{bigints}
\usepackage{amssymb}
\usepackage{amsthm}
\usepackage{mathtools}
\usepackage{bbm}
\usepackage[hidelinks]{hyperref}
\usepackage{gensymb}
\usepackage{multirow,array}
\usepackage{multicol}

\newtheorem*{remark}{Remark}
\usepackage[T1]{fontenc}
\usepackage[utf8]{inputenc}

%chemistry stuff
%\usepackage[version=4]{mhchem}
%\usepackage{chemfig}

%plotting
\usepackage{pgfplots}
\usepackage{tikz}
\tikzset{middleweight/.style={pos = 0.5}}
%\tikzset{weight/.style={pos = 0.5, fill = white}}
%\tikzset{lateweight/.style={pos = 0.75, fill = white}}
%\tikzset{earlyweight/.style={pos = 0.25, fill=white}}

%\usepackage{natbib}

%graphics stuff
\usepackage{graphicx}
\graphicspath{ {./images/} }
\usepackage[style=numeric, backend=biber]{biblatex} % Use the numeric style for Vancouver
\addbibresource{the_bibliography.bib}
%code stuff
%when using minted, make sure to add the -shell-escape flag
%you can use lstlisting if you don't want to use minted
%\usepackage{minted}
%\usemintedstyle{pastie}
%\newminted[javacode]{java}{frame=lines,framesep=2mm,linenos=true,fontsize=\footnotesize,tabsize=3,autogobble,}
%\newminted[cppcode]{cpp}{frame=lines,framesep=2mm,linenos=true,fontsize=\footnotesize,tabsize=3,autogobble,}

%\usepackage{listings}
%\usepackage{color}
%\definecolor{dkgreen}{rgb}{0,0.6,0}
%\definecolor{gray}{rgb}{0.5,0.5,0.5}
%\definecolor{mauve}{rgb}{0.58,0,0.82}
%
%\lstset{frame=tb,
%	language=Java,
%	aboveskip=3mm,
%	belowskip=3mm,
%	showstringspaces=false,
%	columns=flexible,
%	basicstyle={\small\ttfamily},
%	numbers=none,
%	numberstyle=\tiny\color{gray},
%	keywordstyle=\color{blue},
%	commentstyle=\color{dkgreen},
%	stringstyle=\color{mauve},
%	breaklines=true,
%	breakatwhitespace=true,
%	tabsize=3
%}
% text + color boxes
\renewcommand{\mathbf}[1]{\mathbbm{#1}}
\usepackage[most]{tcolorbox}
\tcbuselibrary{breakable}
\tcbuselibrary{skins}
\newtcolorbox{problem}[1]{colback=white,enhanced,title={\small #1},
          attach boxed title to top center=
{yshift=-\tcboxedtitleheight/2},
boxed title style={size=small,colback=black!60!white}, sharp corners, breakable}
%including PDFs
%\usepackage{pdfpages}
\setlength{\parindent}{0pt}
\usepackage{cancel}
\pagestyle{fancy}
\fancyhf{}
\rhead{Avinash Iyer}
\lhead{Math 212: Homework 11}
\newcommand{\card}{\text{card}}
\newcommand{\ran}{\text{ran}}
\newcommand{\N}{\mathbbm{N}}
\newcommand{\Q}{\mathbbm{Q}}
\newcommand{\Z}{\mathbbm{Z}}
\newcommand{\R}{\mathbbm{R}}
\setcounter{secnumdepth}{0}
\begin{document}
  \section{19.4}%
  \begin{description}[font=\normalfont]
    \item[2:]\hfill
      \begin{itemize}
        \item Direct Calculation:
          \begin{align*}
            \int_{S} \vec{F} \cdot d\vec{A} &= \int_{0}^{2\pi}\int_{-1}^{1}\sin\theta~dz~d\theta\\
                                            &= 0
          \end{align*}
        \item Divergence Theorem:
          \begin{align*}
            \int_{S} \vec{F} \cdot d\vec{A} &= \int_{-1}^{1}\int_{0}^{2\pi}\int_{0}^{1}~dr~d\theta~dz\\
                                            &= 0
          \end{align*}
      \end{itemize}
    \item[4:]
    \item[6:]
      \begin{align*}
        \int_{S} \vec{F} \cdot d\vec{A} &= \int_{V} 10 dV\\
                                        &= 240.
      \end{align*}
    \item[8:]
      \begin{align*}
        \int_{S} \vec{H} \cdot d\vec{A} &= \int_{0}^{4}\int_{0}^{3}\int_{0}^{2}(y)~dx~dy~dz\\
                                        &= \int_{0}^{4}\int_{0}^{3}2y~dy~dz\\
                                        &= \int_{0}^{4}9~dz\\
                                        &= 36.
      \end{align*}
    \item[10:]
      \begin{align*}
        \int_{S}\vec{N}\cdot d\vec{A} &= \int_{V} \nabla \cdot \vec{N}~dV\\
                                      &= 0.
      \end{align*}
    \item[14:]
      \begin{align*}
        \int_{S}\vec{F} \cdot d\vec{A} &= \int_{V} \nabla \cdot \vec{F}~dV\\
                                       &= \int_{V} x+y+z~dV\\
                                       &= \int_{0}^{\pi}\int_{0}^{2\pi}\int_{0}^{1}\rho(\sin\phi\cos\theta + \sin\phi\sin\theta + \cos\phi)~\rho^2\sin\phi~d\rho~d\theta~d\phi\\
                                       &= 0
      \end{align*}
    \item[16:]
      \begin{align*}
        \int_{S}\vec{F} \cdot d\vec{A} &= \int_{V} \nabla \cdot \vec{F} d\vec{V}\\
                                       &= \int_{0}^{\pi/4}\int_{0}^{2\pi}\int_{2}^{3}3\rho^4\sin\phi~d\rho~d\theta~d\phi\\
                                       &= \frac{633(2-\sqrt{2})\pi}{5}
      \end{align*}
    \item[22:]
  \end{description}
  \section{20.1}%
  \begin{description}[font=\normalfont]
    \item[6:] 
      \begin{align*}
        \nabla \times \vec{F} &= \begin{pmatrix}6\\0\\2\end{pmatrix}
      \end{align*}
    \item[8:]
      \begin{align*}
        \nabla \times \vec{F} &= \begin{pmatrix}-1\\-1\\-1\end{pmatrix}
      \end{align*}
    \item[10:]
      \begin{align*}
        \nabla \times \vec{F} &= \begin{pmatrix}0\\0\\0\end{pmatrix}
      \end{align*}
    \item[12:]
      \begin{align*}
        \nabla \times \vec{F} &= \begin{pmatrix}2zx^3y+6y^5x^7-xy\\y-7x^6y^6\\zy-z\end{pmatrix}
      \end{align*}
    \item[22:]
      \begin{align*}
        \vec{F} &= \begin{pmatrix}a(x)\\b(y)\\c(z)\end{pmatrix}
      \end{align*}
    \item[24:]\hfill
      \begin{enumerate}[(a)]
        \item Counterclockwise.
        \item Clockwise.
        \item 
          \begin{align*}
            \nabla \times \vec{F} &= \begin{pmatrix}0\\0\\0\end{pmatrix}
          \end{align*}
      \end{enumerate}
  \end{description}
  \section{20.2}%
  \begin{description}[font=\normalfont]
    \item[2:] Zero.
    \item[4:]
      \begin{align*}
        \int_{C}\vec{F} \cdot d\vec{r} &= \bigint_{0}^{1} \begin{pmatrix}3(\cos(\pi t) + \sin(\pi t))\\3\cos(\pi t)\\0\end{pmatrix}\cdot\begin{pmatrix}-3\pi\sin(\pi t)\\0\\3\pi\cos(\pi t)\end{pmatrix} dt + \bigint_{0}^{1} \begin{pmatrix}6t-3\\6t-3\\0\end{pmatrix}\cdot \begin{pmatrix}6\\0\\0\end{pmatrix}dt\\
                                       &= \frac{9\pi}{2}\\
        \int_{S}\nabla\times\vec{F}\cdot d\vec{A} &= \frac{9\pi}{2}\int_{S} \begin{pmatrix}1\\1\\1\end{pmatrix} \cdot \begin{pmatrix}0\\1\\0\end{pmatrix}dA\\
                                                  &= \frac{9\pi}{2}
      \end{align*}
    \item[10:]
      \begin{align*}
        \int_{C} \vec{F} \cdot d\vec{r} &= \int_{S}\nabla\times\vec{F}\cdot d\vec{A}\\
                                        &= -5\pi
      \end{align*}
    \item[12:]
      \begin{align*}
        \int_{C}\vec{F} \cdot d\vec{r} &= \int_{S}\nabla\times\vec{F} \cdot d\vec{A}\\
                                       &= 0
      \end{align*}
    \item[28:]\hfill
      \begin{enumerate}[(a)]
        \item $\displaystyle\vec{C} = \begin{pmatrix}\sin t\\\cos t\\2\end{pmatrix}$
        \item
          \begin{align*}
            \int_{S}\nabla\times\vec{F}\cdot d\vec{A} &= \int_{C} \vec{F} \cdot d\vec{r}\\
                                                      &= \int_{C} \begin{pmatrix}-\cos t\\\sin t\\2\end{pmatrix}\cdot \begin{pmatrix}\cos t\\-\sin t\\0\end{pmatrix}~dt\\
                                                      &= -2\pi
          \end{align*}
      \end{enumerate}
    \item[34:] 
      \begin{align*}
        \int_{C_1}\vec{F} \cdot d\vec{r} &= \int_{S}\nabla\times\vec{F} \cdot d\vec{A}\\
                                        &= \int_{0}^{5}\int_{0}^{2\pi}12~d\theta~dz\\
                                        &= 120\pi\\
                                        &= \int_{C_2}\vec{F}\cdot d\vec{r}\\
        \int_{C_1}\vec{F}\cdot d\vec{r} + \int_{C_2}\vec{F}\cdot d\vec{r} &= 240\pi
      \end{align*}
  \end{description}
  \section{20.3}%
  \begin{description}[font=\normalfont]
    \item[4:]
      \begin{align*}
        \nabla\times\vec{F} &= \begin{pmatrix}0\\-4\\0\end{pmatrix}\\
                            &\neq \vec{0}
      \end{align*}
    \item[6:]
      \begin{align*}
        \nabla\times\vec{F} &= \vec{0}
      \end{align*}
    \item[8:]
      \begin{align*}
        \nabla\cdot \vec{F} &= 0
      \end{align*}
      Thus, the vector field is a curl field.
    \item[24:] There does exist a vector potential for the vector field since the divergence is zero. I don't know how to find it.
    \item[28:]
      \begin{align*}
        \int_{C} \begin{pmatrix}u\\v\end{pmatrix}\cdot d\vec{r} &= \int_{S} \nabla\times\begin{pmatrix}u\\v\\0\end{pmatrix}\cdot \begin{pmatrix}0\\0\\1\end{pmatrix}dA\\
                                                                &= \int_{R}\left(\frac{\partial v}{\partial x} - \frac{\partial u}{\partial y}\right)~dx~dy
      \end{align*}
  \end{description}
\end{document}
