\documentclass[8pt]{extarticle}
\title{}
\author{}
\date{}
\usepackage[shortlabels]{enumitem}


%paper setup
\usepackage{geometry}
\geometry{letterpaper, portrait, margin=1in}
\usepackage{fancyhdr}
% sans serif font:
\usepackage{cmbright}
%symbols
\usepackage{amsmath}
\usepackage{bigints}
\usepackage{amssymb}
\usepackage{amsthm}
\usepackage{mathtools}
\usepackage[hidelinks]{hyperref}
\usepackage{gensymb}
\usepackage{multirow,array}
\usepackage{multicol}

\newtheorem*{remark}{Remark}
\usepackage[T1]{fontenc}
\usepackage[utf8]{inputenc}

%chemistry stuff
%\usepackage[version=4]{mhchem}
%\usepackage{chemfig}

%plotting
\usepackage{pgfplots}
\usepackage{tikz}
\tikzset{middleweight/.style={pos = 0.5}}
%\tikzset{weight/.style={pos = 0.5, fill = white}}
%\tikzset{lateweight/.style={pos = 0.75, fill = white}}
%\tikzset{earlyweight/.style={pos = 0.25, fill=white}}

%\usepackage{natbib}

%graphics stuff
\usepackage{graphicx}
\graphicspath{ {./images/} }
\usepackage[style=numeric, backend=biber]{biblatex} % Use the numeric style for Vancouver
\addbibresource{the_bibliography.bib}
%code stuff
%when using minted, make sure to add the -shell-escape flag
%you can use lstlisting if you don't want to use minted
%\usepackage{minted}
%\usemintedstyle{pastie}
%\newminted[javacode]{java}{frame=lines,framesep=2mm,linenos=true,fontsize=\footnotesize,tabsize=3,autogobble,}
%\newminted[cppcode]{cpp}{frame=lines,framesep=2mm,linenos=true,fontsize=\footnotesize,tabsize=3,autogobble,}

%\usepackage{listings}
%\usepackage{color}
%\definecolor{dkgreen}{rgb}{0,0.6,0}
%\definecolor{gray}{rgb}{0.5,0.5,0.5}
%\definecolor{mauve}{rgb}{0.58,0,0.82}
%
%\lstset{frame=tb,
%	language=Java,
%	aboveskip=3mm,
%	belowskip=3mm,
%	showstringspaces=false,
%	columns=flexible,
%	basicstyle={\small\ttfamily},
%	numbers=none,
%	numberstyle=\tiny\color{gray},
%	keywordstyle=\color{blue},
%	commentstyle=\color{dkgreen},
%	stringstyle=\color{mauve},
%	breaklines=true,
%	breakatwhitespace=true,
%	tabsize=3
%}
% text + color boxes
\renewcommand{\mathbf}[1]{\mathbold{#1}}
\usepackage[most]{tcolorbox}
\tcbuselibrary{breakable}
\tcbuselibrary{skins}
\newtcolorbox{problem}[1]{colback=white,enhanced,title={\small #1},
          attach boxed title to top center=
{yshift=-\tcboxedtitleheight/2},
boxed title style={size=small,colback=black!60!white}, sharp corners, breakable}
%including PDFs
%\usepackage{pdfpages}
\setlength{\parindent}{0pt}
\usepackage{cancel}
\pagestyle{fancy}
\fancyhf{}
\rhead{Avinash Iyer}
\lhead{Math 212: Homework 9}
\newcommand{\card}{\text{card}}
\newcommand{\ran}{\text{ran}}
\newcommand{\N}{\mathbb{N}}
\newcommand{\Q}{\mathbb{Q}}
\newcommand{\Z}{\mathbb{Z}}
\newcommand{\R}{\mathbb{R}}
\begin{document}
\renewcommand{\arraystretch}{1.5}
  \begin{problem}{18.1}
    \begin{description}[font=\normalfont]
      \item[2:] Positive.
      \item[4:] Positive.
      \item[6:] Zero.
      \item[8:]
        \begin{align*}
          \int_{C}\vec{F}\cdot d\vec{r} &= \int_{0}^{5} 2~dx\\
                                        &= 10
        \end{align*}
      \item[20:]
        \begin{align*}
          \int_{C} \begin{pmatrix}2x\\3y\end{pmatrix}\cdot d\vec{r} &= 0
        \end{align*}
      \item[28:]\hfill
        \begin{itemize}
          \item $C_1$: Positive.
          \item $C_2$: Zero.
          \item $C_3$: Zero.
        \end{itemize}
      \item[30:]\hfill
        \begin{itemize}
          \item $C_1$: Zero.
          \item $C_2$: Zero.
          \item $C_3$: Zero.
        \end{itemize}
      \item[48:] $\int_{C_2}3\vec{G}\cdot d\vec{r} = 45$
      \item[50:] $\int_{C_1 + C_2}(\vec{G}-\vec{F})\cdot d\vec{r} = 15$
    \end{description}
  \end{problem}
  \begin{problem}{18.2}
    \begin{description}[font=\normalfont]
      \item[2:]
        \begin{align*}
          \int_{C}\vec{F}\cdot d\vec{r} &= \int_{\pi/2}^{-\pi/2} \cos^2(t)-\sin^2(t)~dt
        \end{align*}
      \item[10:]
        \begin{align*}
          \int_{C}\vec{F}\cdot d\vec{r} &= \int_{1}^{5}2t~dt\\
                                        &= 24
        \end{align*}
      \item[12:]
        \begin{align*}
          \int_{C}\vec{F}\cdot d\vec{r} &= -\int_{0}^{\pi/2} dt\\
                                        &= -\frac{\pi}{2}
        \end{align*}
      \item[14:]
        \begin{align*}
          \int_{C}\vec{F} \cdot d\vec{r} &= \int_{0}^{2} 2t\cos(t) - t^2\sin(t)~dt\\
                                         &= 4\cos(2)
        \end{align*}
      \item[18:]
        \begin{align*}
          \int_{C}\vec{F}\cdot d\vec{r} &= \int_{0}^{2}t+5~dt\\
                                        &= 12
        \end{align*}
      \item[22:]
        \begin{align*}
          \int_{C}\vec{F} \cdot d\vec{r} &= \int_{0}^{4\pi}6~dt\\
                                         &= 24\pi
        \end{align*}
      \item[24:]
        \begin{enumerate}[(a)]
          \item
            \begin{align*}
              \int_{C}3dx + xydy &= \int_{C}  \begin{pmatrix}3\\xy\end{pmatrix}\cdot d\vec{r}
            \end{align*}
          \item
            \begin{align*}
              \int_{C} \begin{pmatrix}100\cos x\\ e^y\sin x\end{pmatrix} &= \int_{C} 100\cos x dx + e^y\sin x dy
            \end{align*}
        \end{enumerate}
      \item[30:]
        \begin{align*}
          \int_{C} dx + ydy + zdz &= \int_{0}^{2\pi}9t + \sin t(\cos t - 1)~dt\\
                                  &= `18\pi^2
        \end{align*}
      \item[34:]
        \begin{align*}
          \int_{C} x dy &= \int_{0}^{\pi/2} 2\cos^2 t - 2\sin t~dt\\
                        &= \frac{1}{2}(\pi-4)
        \end{align*}
      \item[38:]
        \begin{align*}
          \int_{C} \begin{pmatrix}x\\y\end{pmatrix}\cdot d\vec{r} &= 0
        \end{align*}
    \end{description}
  \end{problem}
  \begin{problem}{18.3}
    \begin{description}[font=\normalfont]
      \item[14:]
        \begin{align*}
          \begin{pmatrix}\frac{\partial f}{\partial x}\\\frac{\partial f}{\partial y}\end{pmatrix} &= \begin{pmatrix}2xy\\x^2 + 8y^3\end{pmatrix}\\
          f(x,y) &= x^2y + 2y^4
        \end{align*}
      \item[16:]
        \begin{align*}
          \int_{C}\vec{F}\cdot d\vec{r} &= \int_{0}^{1} -4(4-4t)^2 + (15)(5t)^4~dt\\
                                       &= 1859
        \end{align*}
      \item[18:]
        \begin{align*}
          \int_{C}\vec{F}\cdot d\vec{r} &= f(3,1) - f(1,0)\\
                                        &= 33
        \end{align*}
      \item[20:]
        \begin{align*}
          \int_{C} \vec{F}\cdot d\vec{r} &= f(2,3,-1) - f(1,1,1)\\
                                         &= -10
        \end{align*}
      \item[24:]
        \begin{align*}
          \int_{C}\vec{F} \cdot d\vec{r} &= f(3,18) - f(1,2)\\
                                         &= \sin(54) - \sin(2)
        \end{align*}
      \item[30:]
      \item[32:] Since the partial derivative on $y$ is not symmetric or antisymmetric in the same way that the partial derivative on $x$ and $z$ are, it cannot be the case that $\vec{F}$ is a gradient vector field.
      \item[38:]
    \end{description}
  \end{problem}
\end{document}
