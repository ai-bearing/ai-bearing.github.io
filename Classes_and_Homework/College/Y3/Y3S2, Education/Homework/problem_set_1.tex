\documentclass[8pt]{extarticle}
\title{}
\author{}
\date{}
\usepackage[shortlabels]{enumitem}


%paper setup
\usepackage{geometry}
\geometry{letterpaper, portrait, margin=1in}
\usepackage{fancyhdr}
% sans serif font:
\usepackage{cmbright}
%symbols
\usepackage{amsmath}
\usepackage{bigints}
\usepackage{amssymb}
\usepackage{amsthm}
\usepackage{mathtools}
\usepackage{bbold}
\usepackage[hidelinks]{hyperref}
\usepackage{gensymb}
\usepackage{multirow,array}
\usepackage{multicol}

\newtheorem*{remark}{Remark}
\usepackage[T1]{fontenc}
\usepackage[utf8]{inputenc}

%chemistry stuff
%\usepackage[version=4]{mhchem}
%\usepackage{chemfig}

%plotting
\usepackage{pgfplots}
\usepackage{tikz}
\usetikzlibrary{cd}
\tikzset{middleweight/.style={pos = 0.5}}
%\tikzset{weight/.style={pos = 0.5, fill = white}}
%\tikzset{lateweight/.style={pos = 0.75, fill = white}}
%\tikzset{earlyweight/.style={pos = 0.25, fill=white}}

%\usepackage{natbib}

%graphics stuff
\usepackage{graphicx}
\graphicspath{ {./images/} }
\usepackage[style=numeric, backend=biber]{biblatex} % Use the numeric style for Vancouver
\addbibresource{the_bibliography.bib}
%code stuff
%when using minted, make sure to add the -shell-escape flag
%you can use lstlisting if you don't want to use minted
%\usepackage{minted}
%\usemintedstyle{pastie}
%\newminted[javacode]{java}{frame=lines,framesep=2mm,linenos=true,fontsize=\footnotesize,tabsize=3,autogobble,}
%\newminted[cppcode]{cpp}{frame=lines,framesep=2mm,linenos=true,fontsize=\footnotesize,tabsize=3,autogobble,}

%\usepackage{listings}
%\usepackage{color}
%\definecolor{dkgreen}{rgb}{0,0.6,0}
%\definecolor{gray}{rgb}{0.5,0.5,0.5}
%\definecolor{mauve}{rgb}{0.58,0,0.82}
%
%\lstset{frame=tb,
%	language=Java,
%	aboveskip=3mm,
%	belowskip=3mm,
%	showstringspaces=false,
%	columns=flexible,
%	basicstyle={\small\ttfamily},
%	numbers=none,
%	numberstyle=\tiny\color{gray},
%	keywordstyle=\color{blue},
%	commentstyle=\color{dkgreen},
%	stringstyle=\color{mauve},
%	breaklines=true,
%	breakatwhitespace=true,
%	tabsize=3
%}
% text + color boxes
\renewcommand{\mathbf}[1]{\mathbb{#1}}
\usepackage[most]{tcolorbox}
\tcbuselibrary{breakable}
\tcbuselibrary{skins}
\newtcolorbox{problem}[1]{colback=white,enhanced,title={\small #1},
          attach boxed title to top center=
{yshift=-\tcboxedtitleheight/2},
boxed title style={size=small,colback=black!60!white}, sharp corners, breakable}
%including PDFs
%\usepackage{pdfpages}
\setlength{\parindent}{0pt}
\usepackage{cancel}
\pagestyle{fancy}
\fancyhf{}
\rhead{Avinash Iyer}
\lhead{Economics of Education: Problem Set 1}
\newcommand{\card}{\text{card}}
\newcommand{\ran}{\text{ran}}
\newcommand{\N}{\mathbb{N}}
\newcommand{\Q}{\mathbb{Q}}
\newcommand{\Z}{\mathbb{Z}}
\newcommand{\R}{\mathbb{R}}
\newcommand{\C}{\mathbb{C}}
\newcommand{\iprod}[2]{\left\langle #1,#2\right\rangle}
\newcommand{\norm}[1]{\left\Vert #1\right\Vert}
\setcounter{secnumdepth}{0}
\begin{document}
  \section{Human Capital Model}%
  \begin{description}[font=\normalfont]
    \item[(1)]
      \begin{align*}
        \text{NPV}_{1} &= 100000 + \frac{110000}{1 + 0.2} + \frac{90000}{(1 + 0.2)^2}\\
                       &= 254166\\
        \text{NPV}_2 &= -50000 + \frac{180000}{1 + 0.2} + \frac{180000}{(1+0.2)^2}\\
                     &= 225000\\
        \text{NPV}_3 &= -50000 + \frac{400000}{(1+0.2)^2}\\
                     &= 227777.
      \end{align*}
      Therefore, path 1 maximizes the NPV of Peter's earnings.
    \item[(2)]\hfill
      \begin{enumerate}[(a)]
        \item The annual rate of return associated with the 16th year of schooling is $\frac{66000-60000}{60000}=0.1$, or 10\%. 
        \item If the criticism is true, then it would be the case that for someone who gets 16 years of schooling would have earned more had they had 15 years of schooling than is otherwise stated, implying that the true annual rate of return is lower than what our methodology shows above.
      \end{enumerate}
  \end{description}
  \section{Signaling Model}%
  \begin{description}[font=\normalfont]
    \item[(3)]\hfill
      \begin{enumerate}[(a)]
        \item Given the assumptions of the signaling model, the fact that Catrina has a lower cost of schooling than Jane suggests that she is a higher productivity worker than Jane.
        \item 
          \begin{enumerate}[(i)]
            \item If the firm offered everyone, regardless of schooling, earnings equal to \$500000 in NPV, then it is strictly worse for either Catrina or Jane to enter school, as their cost goes up while their benefit does not.
            \item This is a pooling equilibrium where both parties choose not to go to school.
          \end{enumerate}
        \item 
          \begin{enumerate}[(i)]
            \item The cost to Jane for obtaining 4 years of college is \$120000, while the cost to Catrina is \$100000 --- the benefit to both sides for taking 4 years of college (\$200000 above $S<4$) suggests that both sides will get an education.
            \item This is a pooling equilibrium where both Jane and Catrina get an education.
          \end{enumerate}
        \item 
          \begin{enumerate}[(i)]
            \item If Jane chooses 7 years of college, she will pay \$210000, above the benefit to her from obtaining that education, while if Catrina chooses 7 years of college, she will pay \$175000, which is less than the benefit from obtaining that education. Therefore, Jane will choose 0 years of education while Catrina will choose 7 years of education.
            \item This is a separating equilibrium.
          \end{enumerate}
      \end{enumerate}
  \end{description}
  \section{Reading Response}%
  \begin{enumerate}[(1)]
    \item I think one of the major ways that education has changed since the publishing of the article is greater delocalization, at least in California --- school funding across districts is largely equalized through LCFF and other methodologies, curriculum is more standardized, and generally we have more money and resources travelling through the state. I think these changes are just a part of the longer line of Goldin's argument, which is that the United States's early virtues of localized education, forgiveness, etc. have come to be seen more as vices. The rise of school choice movements in more conservative states also shows this trend --- Goldin mentioned that the American system had stifled the growth of voucherization, but that has started to change as education has become more delocalized. Overall, however, despite these changes, Goldin's broader thrust about the development of the human capital century in the United States has largely held up through the present day.
    \item Examining the chart of employment rates, we see that there is considerable divergence between employment rates between high school graduates and those with less than a high school education throughout the 2010s, with the gap increasing from approximately 10 percentage points to approximately 15 percentage points between 2010 and 2019. This suggests that, during the time period examined, the returns from the ``new economy'' skills that Goldin emphasizes as helping to spur the creation of mass secondary education are still present even in the modern era. In Figure 5, where Goldin charts the returns from education, we see that from the 1950s onward, the returns to high school and college education have increased substantially --- the employment results from the NCES suggest that this difference in labor market outcomes is still persistent, although it is likely that this disparity has reduced over time.
  \end{enumerate}
\end{document}
