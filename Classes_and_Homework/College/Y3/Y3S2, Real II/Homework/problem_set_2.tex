\documentclass[9pt]{extarticle}
\title{}
\author{}
\date{}
\usepackage[shortlabels]{enumitem}


%paper setup
\usepackage{geometry}
\geometry{letterpaper, portrait, margin=1in}
\usepackage{fancyhdr}
% sans serif font:
\usepackage{cmbright}
%symbols
\usepackage{amsmath}
\usepackage{bigints}
\usepackage{amssymb}
\usepackage{amsthm}
\usepackage{mathtools}
\usepackage{bbold}
\usepackage[hidelinks]{hyperref}
\usepackage{gensymb}
\usepackage{multirow,array}
\usepackage{multicol}

\newtheorem*{remark}{Remark}
\usepackage[T1]{fontenc}
\usepackage[utf8]{inputenc}

%chemistry stuff
%\usepackage[version=4]{mhchem}
%\usepackage{chemfig}

%plotting
\usepackage{pgfplots}
\usepackage{tikz}
\tikzset{middleweight/.style={pos = 0.5}}
%\tikzset{weight/.style={pos = 0.5, fill = white}}
%\tikzset{lateweight/.style={pos = 0.75, fill = white}}
%\tikzset{earlyweight/.style={pos = 0.25, fill=white}}

%\usepackage{natbib}

%graphics stuff
\usepackage{graphicx}
\graphicspath{ {./images/} }
\usepackage[style=numeric, backend=biber]{biblatex} % Use the numeric style for Vancouver
\addbibresource{the_bibliography.bib}
%code stuff
%when using minted, make sure to add the -shell-escape flag
%you can use lstlisting if you don't want to use minted
%\usepackage{minted}
%\usemintedstyle{pastie}
%\newminted[javacode]{java}{frame=lines,framesep=2mm,linenos=true,fontsize=\footnotesize,tabsize=3,autogobble,}
%\newminted[cppcode]{cpp}{frame=lines,framesep=2mm,linenos=true,fontsize=\footnotesize,tabsize=3,autogobble,}

%\usepackage{listings}
%\usepackage{color}
%\definecolor{dkgreen}{rgb}{0,0.6,0}
%\definecolor{gray}{rgb}{0.5,0.5,0.5}
%\definecolor{mauve}{rgb}{0.58,0,0.82}
%
%\lstset{frame=tb,
%	language=Java,
%	aboveskip=3mm,
%	belowskip=3mm,
%	showstringspaces=false,
%	columns=flexible,
%	basicstyle={\small\ttfamily},
%	numbers=none,
%	numberstyle=\tiny\color{gray},
%	keywordstyle=\color{blue},
%	commentstyle=\color{dkgreen},
%	stringstyle=\color{mauve},
%	breaklines=true,
%	breakatwhitespace=true,
%	tabsize=3
%}
% text + color boxes
\renewcommand{\mathbf}[1]{\mathbb{#1}}
\usepackage[most]{tcolorbox}
\tcbuselibrary{breakable}
\tcbuselibrary{skins}
\newtcolorbox{problem}[1]{colback=white,enhanced,title={\small #1},
          attach boxed title to top center=
{yshift=-\tcboxedtitleheight/2},
boxed title style={size=small,colback=black!60!white}, sharp corners, breakable}
%including PDFs
%\usepackage{pdfpages}
\setlength{\parindent}{0pt}
\usepackage{cancel}
\pagestyle{fancy}
\fancyhf{}
\rhead{Avinash Iyer}
\lhead{Real Analysis II: Problem Set 2}
\newcommand{\card}{\text{card}}
\newcommand{\ran}{\text{ran}}
\newcommand{\N}{\mathbb{N}}
\newcommand{\Q}{\mathbb{Q}}
\newcommand{\Z}{\mathbb{Z}}
\newcommand{\R}{\mathbb{R}}
\newcommand{\C}{\mathbb{C}}
\newcommand{\iprod}[2]{\left\langle #1,#2\right\rangle}
\newcommand{\norm}[1]{\left\Vert #1\right\Vert}
\setcounter{secnumdepth}{0}
\begin{document}
  \section{Problem 1}%
  Let $v_1,\dots,v_n$ be mutually orthogonal vectors in an inner product space $V$. Show that
  \begin{align*}
    \norm{\sum_{k=1}^{n}v_k}^2 &= \sum_{k=1}^{n}\norm{v_k}^2.
  \end{align*}
  \begin{description}
    \item[Proof:]
      \begin{align*}
        \norm{\sum_{k=1}^{n}v_k}^2 &=\iprod{\sum_{k=1}^{n}v_k}{\sum_{k=1}^{n}v_k} \\
                                   &= \sum_{i=1}^{n}\iprod{\sum_{k=1}^{n}v_k}{v_i}\\
                                   &= \sum_{i=1}^{n}\iprod{v_i}{v_i}\tag*{since for $i\neq j$, $\iprod{v_i}{v_j} = 0$}\\
                                   &= \sum_{i=1}^{n}\norm{v_i}^2
      \end{align*}
  \end{description}
  \section{Problem 2}%
  Let $V$ be an inner product space and fix $w\neq 0$ in $V$. We define the one-dimensional projection
  \begin{align*}
    P_w: V\rightarrow V;&P_w(v) := \frac{\iprod{v}{w}}{\iprod{w}{w}}w.
  \end{align*}
  \begin{enumerate}[(i)]
    \item Prove that $v-P_w(v)\perp P_w(v)$.
    \item Show that $P_w:V\rightarrow V$ is a linear operator with $\norm{P_w}_{\text{op}} = 1$.
    \item Show that $P_w\circ P_w = P_w$.
  \end{enumerate}
  \begin{description}
    \item[Proof of (i):]
      \begin{align*}
        \iprod{v-P_w(v)}{P_w(v)} &= \iprod{v}{P_w(v)} - \iprod{P_w(v)}{P_w(v)}\\
                                 &= \iprod{v}{P_w(v)} - \norm{P_w(v)}^2\\
                                 &= \iprod{v}{\frac{\iprod{v}{w}}{\iprod{w}{w}}w} - \norm{P_w(v)}^2\\
                                 &= \frac{\overline{\iprod{v}{w}}}{\iprod{w}{w}}\iprod{v}{w} - \norm{P_w(v)}^2\\
                                 &= \frac{|\iprod{v}{w}|^2}{\norm{w}^2} - \frac{|\iprod{v}{w}|}{\norm{w}^2}\\
                                 &= 0
      \end{align*}
    \item[Proof of (ii):]
      \begin{align*}
        \norm{P_w}_{\text{op}} &= \sup_{v\leq 1} \norm{\frac{\iprod{v}{w}}{\iprod{w}{w}}w}\\
                               &= \sup_{v\leq 1} \frac{|\iprod{v}{w}|}{\norm{w}}\\
                               &\leq \sup_{v\leq 1}\frac{\norm{v}{\norm{w}}}{\norm{w}}\\
                               &= 1
      \end{align*}
    \item[Proof of (iii):]
      \begin{align*}
        P_w(P_w(v)) &= P_w\left(\frac{\iprod{v}{w}}{\iprod{w}{w}} w\right)\\
                    &= \frac{\iprod{\frac{\iprod{v}{w}}{\iprod{w}{w}}w}{w}}{\iprod{w}{w}}w\\
                    &= \frac{\iprod{v}{w}}{\iprod{w}{w}}w\\
                    &= P_w(v).
      \end{align*}
  \end{description}
  \section{Problem 3}%
  Let $V$ be an inner product space. Prove the reverse Cauchy-Schwarz Inequality which states
  \begin{align*}
    v,w\in V,\text{ and } |\iprod{v}{w}| = \norm{v}\norm{w} \Rightarrow v = \alpha w.
  \end{align*}
  \begin{description}
    \item[Proof:] If $\norm{w} = 0$, then $w = 0$, so $\iprod{v}{w} = 0$ and $\alpha = 0$. Suppose $\norm{w} \neq 0$. Then,
      \begin{align*}
        |\iprod{v}{w}| &= \norm{v}\norm{w}\\
        \norm{w}\left|\frac{\iprod{v}{w}}{\iprod{w}{w}}\right| &= \norm{v},
      \end{align*}
      so $P_{w}(v) = v$, meaning $w = \alpha v$.
  \end{description}
  \section{Problem 4}%
    Let $V$ be an inner product space. Then, for any $v,w\in V$, show that
    \begin{align*}
      \norm{v+w}^2 + \norm{v-w}^2 &= 2\norm{v}^2 + 2\norm{w}^2
    \end{align*}
    \begin{description}
      \item[Proof:]
        \begin{align*}
          \iprod{v+w}{v+w} + \iprod{v-w}{v-w} &= \iprod{v}{v} + \iprod{v}{w} + \iprod{w}{v} + \iprod{w}{w} + \iprod{v}{v} - \iprod{w}{v} - \iprod{v}{w} + \iprod{-w}{-w} \\
                                              &= \iprod{v}{v} + \iprod{v}{v} + \iprod{w}{w} + \iprod{w}{w}\\
                                              &= 2\norm{v}^2 + 2\norm{w}^2
        \end{align*}
    \end{description}
  \section{Problem 5}%
  Let $\lambda = (\lambda_k)_k$ belong to $\ell_{\infty}$. Show that the map
  \begin{align*}
    D_{\lambda}: \ell_{2}\rightarrow \ell_{2}; D_{\lambda}((\xi_k)_k) &= (\lambda_k\xi_k)_k
  \end{align*}
  is well-defined, linear, and bounded with $\norm{D_{\lambda}}_{\text{op}} = \norm{\lambda}_{\infty}$
  \begin{description}
    \item[Proof:]
      \begin{align*}
        \intertext{Well-Defined: Let $(\zeta_k)_k = 0$ for all $k\in \N$. Then,}
        D_{\lambda}((\zeta_k)_k) &= (\lambda_k\zeta_k)_k\\
                                 &= ((\lambda_k)(0))_k\\
                                 &= 0
        \intertext{Linear:}
        D_{\lambda}((\alpha\xi_k)_k + (\beta\zeta_{k})_{k}) &= D_{\lambda}((\alpha\xi_k + \beta\zeta_k)_k)\\
                                                 &= (\lambda_{k}(\alpha\xi_k + \beta\zeta_k))_k\\
                                                 &= (\alpha\lambda_k\xi_k + \alpha\lambda_k\zeta_k)_k\\
                                                 &= (\alpha\lambda_k\xi_k)_k + (\beta\lambda_k\zeta_k)\\
                                                 &= \alpha(\lambda_k\xi_k)_k + \beta(\lambda_k\zeta_k)_k\\
                                                 &= \alpha D_{\lambda}((\xi_k)_k) + \beta D_{\lambda}((\zeta_k)_k)
                                                 \intertext{Bounded:}
        \norm{D_{\lambda}}_{\text{op}} &= \sup_{\norm{\xi_k}_k \leq 1}\norm{D_{\lambda}((\xi_k)_k)}\\
        \norm{D_{\lambda}((\xi_k)_k)} &= \left(\sum_{k=1}^{\infty}|\lambda_k\xi_k|^2\right)^{1/2}\\
                                      &\leq \left(\sum_{k=1}^{\infty}\left|\sup_{k\in \N}|\lambda_k| \xi_k\right|^2\right)^{1/2}\\
                                      &= \norm{\lambda}_{\infty}\left(\sum_{k=1}^{n}|\xi_k|^2\right)^{1/2}\\
                                      &= \norm{\lambda}_{\infty}\norm{\xi_k}\\
                                      \intertext{Therefore,}
        \norm{D_{\lambda}}_{\text{op}} &= \norm{\lambda}_{\infty}.
      \end{align*}
  \end{description}
  \section{Problem 6}%
  Consider the vector space $C([0,2\pi])$ equipped with
  \begin{align*}
    \iprod{f}{g} := \frac{1}{2\pi}\int_{0}^{2\pi}f(t)\overline{g(t)}dt.
  \end{align*}
  \begin{enumerate}[(i)]
    \item Show that this pairing defines an inner product on $C([0,2\pi])$.
      \begin{description}
        \item[Proof:] We will show that $\iprod{f}{g}$ satisfies the axioms of the inner product.
          \begin{align*}
            \intertext{Addition:}
            \iprod{f_1 + f_2}{g} &= \frac{1}{2\pi}\int_{0}^{2\pi}(f_1(t) + f_2(t)) \overline{g(t)}dt\\
                                 &= \frac{1}{2\pi}\int_{0}^{2\pi}\left(f_1(t) \overline{g(t)} + f_2(t) \overline{g(t)}\right)dt\\
                                 &= \frac{1}{2\pi} \int_{0}^{2\pi}f_1(t) \overline{g(t)}dt + \frac{1}{2\pi} \int_{0}^{2\pi}f_2(t) \overline{g(t)} dt\\
                                 &= \iprod{f_1}{g} + \iprod{f_2}{g}.
             \intertext{Scalar Multiplication:}
            \iprod{\alpha f}{g} &= \frac{1}{2\pi}\int_{0}^{2\pi} (\alpha f(t)) \overline{g(t)} dt\\
                                &= \frac{1}{2\pi}\int_{0}^{2\pi}\alpha\left(f(t)\overline{g(t)}\right)dt\\
                                &= \alpha \left(\frac{1}{2\pi}\int_{0}^{2\pi}f(t)\overline{g(t)}dt\right)\\
                                &= \alpha\iprod{f}{g}.
            \intertext{Conjugation:}
            \overline{\iprod{g}{f}} &= \frac{1}{2\pi}\int_{0}^{2\pi}\overline{g(t)\overline{f(t)}}dt\\
                                    &= \frac{1}{2\pi}\int_{0}^{2\pi}f(t)\overline{g(t)}dt\\
                                    &= \iprod{f}{g}.
            \intertext{Positive Definition:}
            \iprod{f}{f} &= \frac{1}{2\pi}\int_{0}^{2\pi}f(t)\overline{f(t)}dt\\
                         &= \frac{1}{2\pi}\int_{0}^{2\pi}|f(t)|^2dt\\
                         &\geq 0.
          \end{align*}
          For $\iprod{f}{f} = 0$, we have that the integral equals zero --- since $f$ is continuous, it means that if $|f(t)|^2 > 0$ for some $t_0\in [0,2\pi]$, then $|f(t)|^2 \neq 0$ on some interval $[t_0-\delta,t_0+\delta]$, meaning the integral can only equal zero if $f$ is $\mathbb{0}_f$ on $[0,2\pi]$.
      \end{description}
    \item For $n\in\Z$, set $e_n(t) = \cos(nt) + i\sin(nt)$. Show that the family $\{e_n\}_{n\in\Z}$ is orthonormal.
      \begin{description}
        \item[Proof:] We will show that $\{e_n\}_{n\in\Z}$ is orthonormal by showing that $\iprod{e_n}{e_n} = 1$ and $\iprod{e_n}{e_m} = 0$ for $m\neq n$.
          \begin{align*}
            \iprod{e_n}{e_n} &= \frac{1}{2\pi}\int_{0}^{2\pi}(\cos(nt) + i\sin(nt))(\cos(nt)-i\sin(nt))dt\\
                             &= \frac{1}{2\pi}\int_{0}^{2\pi}\left(\cos^{2}(nt) + \sin^{2}(nt)\right)dt\\
                             &= \frac{1}{2\pi}\int_{0}^{2\pi}dt\\
                             &= 1\\
            \iprod{e_n}{e_m} &= \frac{1}{2\pi}\int_{0}^{2\pi}(\cos(nt) + i\sin(nt))(\cos(mt) - i\sin(mt))dt\\
                             &= \frac{1}{2\pi}\int_{0}^{2\pi}\left(\cos(mt)\cos(nt) + i\sin(nt)\cos(mt) - i\sin(mt)\cos(nt) + \sin(nt)\sin(mt)\right)dt\\
                             &= \frac{1}{2\pi}\left(\int_{0}^{2\pi}(\cos(mt)\cos(nt) + \sin(nt)\sin(mt))dt + i\int_{0}^{2\pi}(\sin(nt)\cos(mt) - \sin(mt)\cos(nt))dt\right)\\
                             &= 0.
          \end{align*}
      \end{description}
  \end{enumerate}
  \section{Problem 7}%
  Let $V$ be any normed space, $p\in[1,\infty]$, and suppose $T:\ell_{p}^{n}\rightarrow V$ is linear. Show that $T$ is bounded.
  \begin{description}
    \item[Proof:] Let $T$ be a linear transformation from $\ell_{p}^{n}$ to $V$. Let $\xi = \sum_{k=1}^{n}\alpha_ke_k$ where $\norm{\xi}_{p} = 1$. Then,
      \begin{align*}
        \norm{T(\xi)} &= \norm{T\left(\sum_{k=1}^{n}\alpha_ke_k\right)}\\
                      &= \norm{\sum_{k=1}^{n}\alpha_kT(e_k)}\\
                      &\leq \sum_{k=1}^{n}|\alpha_k|\norm{T(e_k)}\\
                      &\leq \sum_{k=1}^{n}\sup|\alpha_k|\norm{T(e_k)}\\
                      &\leq \sum_{k=1}^{n}\norm{T(e_k)}\\
                      &\leq \sum_{k=1}^{n}\max_{k}\norm{T(e_k)}\\
                      &= n\norm{T(e_M)}\\
                      &< \infty.
      \end{align*}
  \end{description}
  \section{Problem 8}%
  Let $\mathbb{P}[0,1] = \left\{\sum_{0}^{n}a_kx^k\mid a_k\in\C\right\}\subseteq C([0,1])$ denote the linear subspace of all polynomial functions equipped with the uniform norm $\norm{\cdot}_{u}$ inherited from $C([0,1])$. We define the map
  \begin{align*}
    D:\mathbb{P}[0,1]\rightarrow \mathbb{P}[0,1]\\
    D(p(x)) = p'(x).
  \end{align*}
  Show that $D$ is unbounded.
  \begin{description}
    \item[Proof:] Let $p(x) = x^n$. Then, in $\mathbb{P}[0,1]$,
      \begin{align*}
        \norm{p}_u &= 1\\
        \norm{D(p)}_u &= n.
      \end{align*}
      For any $L\in \R$, we can find a $n\in \N$ sufficiently large such that $\norm{D(p)}_u = n > L$, by the Archimedean property. Therefore, $D$ is unbounded.
  \end{description}
  \section{Problem 9}%
  Let $V$ be an infinite-dimensional normed space. Show that there is a linear functional $\varphi: V\rightarrow \mathbb{F}$ that is unbounded.
  \begin{description}
    \item[Proof:] Let $B = \{x_n\}$ be the basis for $V$. We define $\varphi: V\rightarrow \mathbb{F}$ as $\varphi(x) = \sum_{n} n\alpha_n$ for the $\alpha_nx_n$ component in $x$. Then, $\varphi$ is linear and unbounded, as the values $n$ takes are not bounded, seeing as $V$ is infinite-dimensional.
  \end{description}
  \section{Problem 10}%
  Let $a,b\in \mathbb{M}_{n}$. Show the following properties of the operator norm.
  \begin{enumerate}[(i)]
    \item $\norm{a}_{\text{op}} = \sup\left\{|\iprod{a\xi}{\eta}|\mid \xi,\eta\in B_{\ell_{2}^{n}} \right\}$
    \item $\norm{a^{\ast}}_{\text{op}} = \norm{a}_{\text{op}}$
    \item $\norm{ab}_{\text{op}} \leq \norm{a}_{\text{op}}\norm{b}_{\text{op}}$
    \item $\norm{a^{\ast}a}_{\text{op}} = \norm{a}^{2}_{\text{op}}$
  \end{enumerate}
  \begin{description}
    \item[Proof:]\hfill
      \begin{enumerate}[(i)]
        \item 
          \begin{align*}
            \iprod{a\xi}{\eta} &\leq \norm{a\xi}\norm{\eta}\\
                               &= \norm{a\xi}\\
                               &\leq \sup_{\xi\in B_{\ell_2^{n}}}\norm{a\xi}\\
                               &= \norm{a}_{\text{op}}.\\
            \norm{a}{\text{op}} &= \sup_{\xi\in B_{\ell_2^{n}}}\norm{a\xi}\\
            \intertext{Set $\eta = \frac{a\xi}{\norm{a\xi}}$. Then,}
                                &= \sup_{\xi \in B_{\ell_2^{n}}} \frac{1}{\norm{a\xi}}\iprod{a\xi}{\eta}\\
                                &= \sup \left\{\iprod{a\xi}{\eta}\mid \xi,\eta \in B_{\ell_2^n}\right\}.
          \end{align*}
        \item
          \begin{align*}
            \norm{a^{\ast}}_{\text{op}} &= \sup_{\xi,\eta \in B_{\ell_2^n}}|\iprod{a^{\ast}\xi}{\eta}|\\
                                        &= \sup_{\xi,\eta\in B_{\ell_{2}^n}}|\iprod{\xi}{a^{\ast\ast}\eta}|\tag*{definition of conjugate transpose}\\
                                        &= \sup_{\xi,\eta\in B_{\ell_2^n}}|\iprod{a\xi}{\eta}| \tag*{by absolute value}\\
                                        &= \norm{a}_{\text{op}}.
          \end{align*}
        \item
          \begin{align*}
            \norm{ab}_{\text{op}} &= \sup_{\xi,\eta\in B_{\ell_2^n}} |\iprod{(ab)\xi}{\eta}|\\
                                  &= \sup_{\xi,\eta\in B_{\ell_2^n}}|\iprod{a(b\xi)}{\eta}|\\
                                  &= \sup_{\xi,\eta\in B_{\ell_2^n}}|\iprod{b\xi}{a^{\ast}\eta}|\\
                                  &\leq \sup_{\xi\in B_{\ell_2^n}}\norm{b\xi}\sup_{\eta\in B_{\ell_2^n}}\norm{a^{*}\eta}\\
                                  &= \norm{b}_{\text{op}}\norm{a^{\ast}}_{\text{op}}\\
                                  &= \norm{a}\norm{b}.
          \end{align*}
        \item 
          \begin{align*}
            \norm{a^{\ast}a}_{\text{op}} &= \sup_{\xi,\eta\in B_{\ell_2^n}}|\iprod{(a^{\ast}a)\xi}{\eta}|\\
                                         &= \sup_{\xi,\eta \in B_{\ell_2^n}}|\iprod{a\xi}{a^{\ast\ast}\eta}|\\
                                         &= \sup_{\xi\in B_{\ell_2^n}}\norm{a\xi}^{2}\\
                                         &= \norm{a}_{\text{op}}^{2}
          \end{align*}
      \end{enumerate}
  \end{description}
\end{document}
