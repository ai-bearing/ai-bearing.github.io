\documentclass[10pt]{extarticle}
\title{}
\author{}
\date{}
\usepackage[shortlabels]{enumitem}


%paper setup
\usepackage{geometry}
\geometry{letterpaper, portrait, margin=1in}
\usepackage{fancyhdr}
% sans serif font:
\usepackage{cmbright}
%symbols
\usepackage{amsmath}
\usepackage{bigints}
\usepackage{amssymb}
\usepackage{amsthm}
\usepackage{mathtools}
\usepackage{bbold}
\usepackage[hidelinks]{hyperref}
\usepackage{gensymb}
\usepackage{multirow,array}
\usepackage{multicol}

\newtheorem*{remark}{Remark}
\usepackage[T1]{fontenc}
\usepackage[utf8]{inputenc}

%chemistry stuff
%\usepackage[version=4]{mhchem}
%\usepackage{chemfig}

%plotting
\usepackage{pgfplots}
\usepackage{tikz}
\usetikzlibrary{cd}
\tikzset{middleweight/.style={pos = 0.5}}
%\tikzset{weight/.style={pos = 0.5, fill = white}}
%\tikzset{lateweight/.style={pos = 0.75, fill = white}}
%\tikzset{earlyweight/.style={pos = 0.25, fill=white}}

%\usepackage{natbib}

%graphics stuff
\usepackage{graphicx}
\graphicspath{ {./images/} }
%\usepackage[style=numeric, backend=biber]{biblatex} % Use the numeric style for Vancouver
%\addbibresource{the_bibliography.bib}
%code stuff
%when using minted, make sure to add the -shell-escape flag
%you can use lstlisting if you don't want to use minted
%\usepackage{minted}
%\usemintedstyle{pastie}
%\newminted[javacode]{java}{frame=lines,framesep=2mm,linenos=true,fontsize=\footnotesize,tabsize=3,autogobble,}
%\newminted[cppcode]{cpp}{frame=lines,framesep=2mm,linenos=true,fontsize=\footnotesize,tabsize=3,autogobble,}

%\usepackage{listings}
%\usepackage{color}
%\definecolor{dkgreen}{rgb}{0,0.6,0}
%\definecolor{gray}{rgb}{0.5,0.5,0.5}
%\definecolor{mauve}{rgb}{0.58,0,0.82}
%
%\lstset{frame=tb,
%	language=Java,
%	aboveskip=3mm,
%	belowskip=3mm,
%	showstringspaces=false,
%	columns=flexible,
%	basicstyle={\small\ttfamily},
%	numbers=none,
%	numberstyle=\tiny\color{gray},
%	keywordstyle=\color{blue},
%	commentstyle=\color{dkgreen},
%	stringstyle=\color{mauve},
%	breaklines=true,
%	breakatwhitespace=true,
%	tabsize=3
%}
% text + color boxes
%\renewcommand{\mathbf}[1]{\mathbb{#1}}
%\usepackage[most]{tcolorbox}
%\tcbuselibrary{breakable}
%\tcbuselibrary{skins}
%\newtcolorbox{problem}[1]{colback=white,enhanced,title={\small #1},
%          attach boxed title to top center=
%{yshift=-\tcboxedtitleheight/2},
%boxed title style={size=small,colback=black!60!white}, sharp corners, breakable}
%including PDFs
%\usepackage{pdfpages}
\setlength{\parindent}{0pt}
\usepackage{cancel}
\pagestyle{fancy}
\fancyhf{}
\rhead{Avinash Iyer}
\lhead{Real Analysis II: Problem Set 7}
\newcommand{\card}{\text{card}}
\newcommand{\ran}{\text{ran}}
\newcommand{\N}{\mathbb{N}}
\newcommand{\Q}{\mathbb{Q}}
\newcommand{\Z}{\mathbb{Z}}
\newcommand{\R}{\mathbb{R}}
\newcommand{\C}{\mathbb{C}}
\newcommand{\iprod}[2]{\left\langle #1,#2\right\rangle}
\newcommand{\norm}[1]{\left\Vert #1\right\Vert}
\setcounter{secnumdepth}{0}
\begin{document}
  \section{Problem 1}%
  Let $X$ be a metric space and consider a subset $Y\subseteq X$ viewed as a metric space. Show that $C\subseteq Y$ is connected in $Y$ if and only if it is connected as a subset of $X$.
  \begin{description}
    \item[Proof:] $C\subseteq Y$ is connected if and only if any splitting $C\subseteq (Y\cap U)\sqcup(Y\cap V)$ in $Y$ is trivial, for $U,V\subseteq X$ open. Thus, $C\subseteq Y\cap (U\sqcup V)$ is a trivial splitting, if and only if $C\subseteq U\sqcup V$ is trivial.
  \end{description}
  \section{Problem 2}%
  If $X$ is a metric space, and $Y\subseteq X$ is a connected subset of $X$, show that for every splitting $X = X_1\sqcup X_2$, $X_i\subseteq X$ open, we must have $Y\subseteq X_1$ or $Y\subseteq X_2$.
  \begin{description}
    \item[Proof:] Let $Y\subseteq X$ be connected. Then, for any splitting $Y\subseteq X_1\cup X_2$, with $X_1,X_2\subseteq X$ open, it is the case that $Y\cap X_1\cap X_2 = \emptyset$.

      Since the splitting is trivial, it is the case that either $Y\cap X_1 = \emptyset$ or $Y\cap X_2 = \emptyset$.

      We also have that $Y\cap (X_1 \cup X_2) = \left(Y\cap X_1\right) \cup \left(Y\cap X_2\right) = Y$. Therefore, it must be the case that $Y\cap X_1 = Y$ or $Y\cap X_2 = Y$, so $Y\subseteq X_1$ or $Y\subseteq X_2$.
  \end{description}
  \section{Problem 3}%
  For $n=0,1,2,3\dots$, let $X_n := [0,1]\times \{2^{-n}\}$, and consider the space
  \begin{align*}
    X &= \{(0,0),(1,0)\} \cup \left(\bigcup_{n=1}^{\infty} X_n\right).
  \end{align*}
  \begin{enumerate}[(i)]
    \item List all the connected components of $X$.
    \item If $X = U\sqcup V$ is a nontrivial splitting of $X$, show that there is a finite subset $F\subseteq \N$ with
      \begin{align*}
        U &= \bigcup_{n\in F}X_n,~~V = X\setminus U.
      \end{align*}
  \end{enumerate}
  \begin{description}
    \item[Proof:]\hfill
      \begin{enumerate}[(i)]
        \item Each of the $X_n$ are connected components, as each $X_n$ is connected, closed, and open in $X$ (by selecting an open subsets of $\R^2$ that splits two $X_n$ segments). Therefore, we must have $\{(0,0)\},\{(1,0)\},\{X_n\}_{n\geq 1}$ are the connected components of $X$.
        \item Let $U'\sqcup V'$ be open sets in $\R^2$ with $U = X\cap U'$, $V = X\cap V'$, with $U',V'$ non-empty and $U'\cap V' = \emptyset$. It must be the case that for any $n$, $X_n$ is wholly contained in either $U'$ or $V'$ --- otherwise, we would have a non-trivial splitting for a connected component, which is a contradiction.\\

          I don't know how to proceed from here.
      \end{enumerate}
  \end{description}
  \section{Problem 4}%
  Show that the $n$-sphere, $S^{n-1} = \{v\in \R^n\mid \norm{v}_2 = 1\}$ is path-connected.
  \begin{description}
    \item[Proof:] Let $x,y\in S^{n-1}$. Then, $\norm{x}_{2} = \norm{y}_{2} = 1$. Let $\gamma: [0,1]\rightarrow S^{n-1}$ be defined by $\gamma(0) = x$, $\gamma(1) = y$, and $\gamma(t) = \frac{(1-t)x + ty}{\norm{(1-t)x + ty}}$ (for $(1-t)x + ty \neq 0$). Since convex combinations and norms are continuous, $\gamma(t)$ is continuous and $\norm{\gamma(t)} = 1$ for all $t$, meaning every element of $\gamma(t)$ is an element of $S^{n-1}$, so $\gamma(t)$ is a path.\\

      If $x$ and $y$ are antipodes, then there is some $x^{\ast}$ in a $\varepsilon$-neighborhood of $x$, and a path from $x^{\ast}$ to $y$ found by the previous method, so by appending paths, we have a path from $x$ to $y$.
  \end{description}
  \section{Problem 5}%
  Let $X$ be a metric space. We define a relation on $X$, $x\sim y$ if and only if there exists a path $\gamma: [0,1] \rightarrow X$ with $\gamma(0) = x$ and $\gamma(1) = y$. Show that this defines an equivalence relation on $X$. Equivalence classes are called path-connected components.
  \begin{description}
    \item[Proof:] The relation is clearly reflexive.\\

      For symmetry, if $\gamma$ is a path from $x$ to $y$, we define $\gamma'$ as $\gamma(1-t)$, which is a path from $y$ to $x$.\\

      If $\gamma_1$ is a path from $x$ to $y$, and $\gamma_2$ is a path from $y$ to $z$, we define $\gamma: [0,1]\rightarrow X$ as
      \begin{align*}
        \gamma(t) &= \begin{cases}
          \gamma_1(2t) & 0 \leq t \leq 1/2\\
          \gamma_2(2t - 1) & 1/2 \leq t \leq 1
        \end{cases}.
      \end{align*}
      This is a path from $x$ to $z$, and thus the relation is transitive.
  \end{description}
  \section{Problem 6}%
  Show that $\R$ and $\R^2$ are not homeomorphic.
  \begin{description}
    \item[Proof:] Let $f: \R\rightarrow \R^2$ be a homeomorphism, meaning $f$ is continuous.\\

      Consider $f(\R\setminus \{0\})$. We have that $\R\setminus \{0\} = f^{-1}\left(f(\R\setminus \{0\})\right)$ is disconnected. However, $f(\R\setminus \{0\}) = f(\R)\setminus f(\{0\}) = \R^2\setminus f(0)$, but $\R^2\setminus f(0)$ is connected. $\bot$
  \end{description}
  \section{Problem 7}%
  Let $V$ be a normed space and suppose $Y\subseteq V$ is an open and connected subset. Fix a vector $y_0\in Y$, and set
  \begin{align*}
    W := \{w\in Y\mid \text{there is a path from $y_0$ to $w$}\}.
  \end{align*}
  \begin{enumerate}[(i)]
    \item Show that $W$ is open in $Y$.
    \item Show that $W$ is closed in $Y$.
    \item Conclude that $Y$ is path-connected.
  \end{enumerate}
  \begin{description}
    \item[Proof:]\hfill
      \begin{enumerate}[(i)]
        \item Let $y\in W$. Since $Y$ is open, $\exists \delta > 0$ with $U(y,\delta)\subseteq Y$. Letting $w\in U(y,\delta)$, we concatenate a path from $y_0$ to $y$, then a path from $y$ to $w$ to find that $w\in W$. Thus, $U(y,\delta)\subseteq W$.
        \item For any $w\in W$, the sequence $\displaystyle(w_n)_n = \frac{1}{n}y_0 + \left(1-\frac{1}{n}\right)w$ converges to $w$ --- thus, any sequence in $W$ converges to a point in $W$, as each of the $w_n$ are path connected to $y_0$ by restriction. Therefore, $W$ is closed in $Y$.
        \item Since $W$ is clopen in $Y$, and $Y$ is connected, it must be the case that $W = Y$, so the set of all path-connected elements of $Y$ is equal to the set of all elements of $Y$.
      \end{enumerate}
  \end{description}
  \section{Problem 8}%
  A group is a nonempty set $G$ with a binary operation $G\times G \rightarrow G$, $(s,t)\mapsto st$ satisfying
  \begin{itemize}
    \item $(st)r = s(tr)$;
    \item there is a unique neutral element $e\in G$ with $te = et$ for all $t\in G$;
    \item for every $t\in G$ there is a unique inverse $t^{-1}\in G$ with $t^{-1}t = tt^{-1} = e$.
  \end{itemize}
  A subgroup of $G$ is a nonempty subset $H\subseteq G$ such that $s,t\in H \Rightarrow st,t^{-1}\in H$. The subgroup $H$ is normal if $t\in G,s\in H$ implies $tst^{-1}\in H$.\\

  Consider a group $G$ equipped with a metric so that the operations $G\times G \rightarrow G$, $(s,t)\mapsto st$ and $G\rightarrow G$, $t\mapsto t^{-1}$ are both continuous. Show that the connected component containing the neutral element $e$, $G_0$, is a closed and normal subgroup of $G$.
  \begin{description}
    \item[Proof:] Let $s,t\in G_0$. We have some connected set $C$ such that $s,e\in C$, and some connected set $D$ such that $t,e\in D$. Notice that $CD$ is connected, since $C\times D$ is connected and $CD$ is $C\times D$ under a continuous map; since $st\in CD$ and $e\in CD$, $CD$ is a connected set containing $st$ and $e$, so $st\in G_0$.\\

      By a similar argument, we see that if $s,e\in D$, then $D\rightarrow D^{-1}$ is a connected set containing $s^{-1}$ and $e$ (as $e^{-1} = e$), meaning $s^{-1}$ is in a connected set with $e$. Thus, $G_0$ is a subgroup.\\

      Let $g\in G$. For $C$ a connected component with $e\in C$, we have that $gCg^{-1}$ is connected (by composing continuous mappings of connected sets), and $gCg^{-1}$ contains $e$, as $geg^{-1} = e$. Thus $gCg^{-1} = C$, so $G_0$ is normal.\\

      Let $C$ be a connected component with $e\in C$. Since $C$ is not contained in any other connected set (as otherwise, it would not be a connected component), it must be the case that $C = \overline{C}$, so $C$ is closed.
  \end{description}
  \section{Problem 9}%
  Show that the Cantor set is totally disconnected.
  \begin{description}
    \item[Proof:] Let $a,b\in \mathcal{C}$. We will show that the only components in $\mathcal{C}$ are singletons.\\

      Suppose $a\neq b$. Then, since $\R$ is Hausdorff, there exists $\varepsilon$ so small such that $(a-\varepsilon,a+\varepsilon) \cap (b-\varepsilon,b+\varepsilon) = \emptyset$. We find $n$ large such that $\frac{1}{3^n} < \varepsilon$, Then, it is the case that $[a-3^{-n},a+3^{-n}] \cap [b-3^{-n},b+3^{-n}] = \emptyset$.\\

      Since $a,b\in \mathcal{C}_n$ for all $n$, we have that for all $m\geq n$, $[a-3^{-m},a+3^{-m}] \cap [b-3^{-m},b+3^{-m}] = \emptyset$.\\

      Therefore, $(-\infty,a+3^{-m}) \cup (b-3^{-m},\infty)\cap \mathcal{C}$ is a non-trivial splitting.\\

      Since $\{a\}$ and $\{b\}$ are connected sets, it is the case that the only connected sets in $\mathcal{C}$ are singletons.
  \end{description}
  \section{Problem 10}%
  A metric space $X$ is called zero-dimensional if for any $x,y\in X$ with $x\neq y$, there are open subsets $U,V\subseteq X$ with $x\in U,y\in V$ and $X = U\sqcup V$.
  \begin{enumerate}[(i)]
    \item Show that every zero-dimensional metric space is totally disconnected.
    \item If $Y\subseteq \R$ is totally disconnected, show that $Y$ is zero-dimensional.
    \item Conclude that $\Q$ and the Cantor set are zero-dimensional.
  \end{enumerate}
  \begin{description}
    \item[Proof:]\hfill
      \begin{enumerate}[(i)]
        \item We know that for any $x\in X$, $\{x\}$ is connected, and for any $x,y\in X$ with $x\neq y$, there is a non-trivial splitting for any set containing $x$ and $y$, meaning the only connected components in $X$ are singletons.
        \item Without loss of generality, let $x,y\in Y$ with $x < y$. Then, $\exists z\in X$ with $z\notin Y$ and $x < z < y$ (as $\{x\}$ and $\{y\}$ are the connected components generated by $x$ and $y$). Thus, $U = (-\infty,z)\cap Y$ and $V = (z,\infty)\cap Y$ are disjoint, with $U\sqcup V = Y$, $x\in U$ and $y\in V$. Therefore, $Y$ is zero-dimensional.
        \item Since $\Q$ and the Cantor set are totally disconnected, both $\Q$ and the Cantor set are totally disconnected.
      \end{enumerate}
  \end{description}
\end{document}
