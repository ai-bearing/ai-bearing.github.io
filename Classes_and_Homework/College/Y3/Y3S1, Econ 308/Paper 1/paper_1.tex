\documentclass[12pt]{extarticle}
\title{}
\author{Avinash Iyer}
\date{}
\usepackage[shortlabels]{enumitem}


%paper setup
\usepackage{geometry}
\geometry{letterpaper, portrait, margin=1in}
\usepackage{fancyhdr}
\usepackage{setspace}
\usepackage{cmbright}
%symbols
\usepackage{amsmath}
\usepackage{amssymb}
\usepackage{amsthm}
\usepackage{mathtools}
\usepackage[hidelinks]{hyperref}
\usepackage{gensymb}
\usepackage{multirow,array}

\newtheorem*{remark}{Remark}
\usepackage[T1]{fontenc}
\usepackage[utf8]{inputenc}

%chemistry stuff
%\usepackage[version=4]{mhchem}
%\usepackage{chemfig}

%plotting
\usepackage{pgfplots}
\usepackage{tikz}
\tikzset{middleweight/.style={pos = 0.5, fill=white}}
\tikzset{weight/.style={pos = 0.5, fill = white}}
\tikzset{lateweight/.style={pos = 0.75, fill = white}}
\tikzset{earlyweight/.style={pos = 0.25, fill=white}}

%\usepackage{natbib}

%graphics stuff
\usepackage{graphicx}
\graphicspath{ {./images/} }
\usepackage[backend=biber,style=numeric,sorting=none]{biblatex}
\addbibresource{the_bibliography.bib}
%code stuff
%when using minted, make sure to add the -shell-escape flag
%you can use lstlisting if you don't want to use minted
%\usepackage{minted}
%\usemintedstyle{pastie}
%\newminted[javacode]{java}{frame=lines,framesep=2mm,linenos=true,fontsize=\footnotesize,tabsize=3,autogobble,}
%\newminted[cppcode]{cpp}{frame=lines,framesep=2mm,linenos=true,fontsize=\footnotesize,tabsize=3,autogobble,}

%\usepackage{listings}
%\usepackage{color}
%\definecolor{dkgreen}{rgb}{0,0.6,0}
%\definecolor{gray}{rgb}{0.5,0.5,0.5}
%\definecolor{mauve}{rgb}{0.58,0,0.82}
%
%\lstset{frame=tb,
%	language=Java,
%	aboveskip=3mm,
%	belowskip=3mm,
%	showstringspaces=false,
%	columns=flexible,
%	basicstyle={\small\ttfamily},
%	numbers=none,
%	numberstyle=\tiny\color{gray},
%	keywordstyle=\color{blue},
%	commentstyle=\color{dkgreen},
%	stringstyle=\color{mauve},
%	breaklines=true,
%	breakatwhitespace=true,
%	tabsize=3
%}
% text + color boxes
\usepackage[most]{tcolorbox}
\tcbuselibrary{breakable}
\newtcolorbox{problem}[1]{colback = white, title = {#1}, breakable}
\newtcolorbox{solution}{colback = white, colframe = black!75!white, title = Solution, breakable}
%including PDFs
%\usepackage{pdfpages}
\newlength\tindent
\setlength{\tindent}{\parindent}
\setlength{\parindent}{0pt}
\renewcommand{\indent}{\hspace*{\tindent}}
\usepackage{cancel}
\pagestyle{fancy}
\fancyhf{}
\rhead{Avinash Iyer}
\chead{SSWR}
\lhead{Econ 308: Paper 1}
\newcommand{\card}{\text{card}}
\newcommand{\ran}{\text{ran}}
\newcommand{\N}{\mathbb{N}}
\newcommand{\Q}{\mathbb{Q}}
\newcommand{\Z}{\mathbb{Z}}
\newcommand{\R}{\mathbb{R}}
\renewcommand{\footnotesize}{\scriptsize}
\begin{document}
  \doublespacing
  \section*{Abstract}
  \indent This paper seeks to analyze the effects of recent state housing laws in California on the incentives of local governments with regard to housing production, the economic welfare of individuals based on their tenure (tenant vs. owner-occupier), and the effects on migration between localities and states. I will start by introducing some of the state laws that limit the authority of local governments in the realm of housing approvals, then I will analyze the effects of these laws on localities' incentives to increase housing supply, the effects of local government housing restrictions on the welfare of tenants and owner-occupiers, and recommendations for further action on the part of the state.
  \section*{State Laws and Local Housing Restrictions}%
  \indent Every eight years, the state of California mandates that Metropolitan Planning Organizations, constituted by the local governments of incorporated municipalities and county governments, plan for the ``Regional Housing Needs Allocation,'' (RHNA), to accommodate future population growth by allowing for greater housing construction.\supercite{rhna} However, until recently, the effects of RHNA on facilitating housing supply growth were muted --- enforcement was lax, and cities often fell short of the mandated requirements. The RHNA system allows cities broad discretion to plan for housing, and until 2017, state laws provided little recourse for developers in cities that proved intransigent. According to the advocacy group California YIMBY, this intransigence exacerbates the statewide housing shortage.

  \indent In response, the state legislature passed Senate Bill 35 (2017), which allowed a ``release valve'' by creating a streamlined process for developers who want to build housing in cities that fell behind the RHNA target.\supercite{sb_35_423} Senate Bill 35 was one of a number of laws that sought to resolve housing undersupply in the state by restricting the ability of local governments to deny housing projects. Other laws include the Housing Accountability Act, which forbids localities from denying housing developments that substantially comply with objective standards, Senate Bill 1069 (2016), which mandates that localities allow accessory dwelling units,\supercite{sb_1069} and Assembly Bill 2097 (2022), which prohibits localities from mandating minimum quantities of parking near public transportation.\supercite{ab_2097}
  \section*{Effect on Local Government Incentives}%
  \indent Local governments, in deciding the stringency of restrictions on new construction, decide between two competing priorities. Localities benefit from new housing construction through increased property tax revenue and revenue from impact fees,\supercite{impact_fees} but also face costs in the form of increased risk of electoral backlash from a politically-active subset of owner-occupier voters who may see their asset value drop.\supercite{owner_nimbys}

  \indent Additionally, the benefits of allowing new housing construction in a given locality are not fully internalized, as the increased utility --- lower-cost housing inducing a substitution and income effect toward housing consumption --- from allowing new construction accrues to residents in other localities as well as to the residents of the locality in which new construction was permitted. Therefore, we can model the laxness of new housing construction as a public goods problem, where $G$ represents implementation of all other public goods, and $H$ represents the public good ``ease of new housing construction.''
  \[U_i(G,H) = \alpha \ln(G) + \beta \ln(H)\]
  \indent The effect of treating housing construction as a public goods problem is that we can see how local governments are incentivized to allow some level of housing construction, but below the efficient level. As with most public goods, we can see that cities will underproduce the public good (namely, ease of housing construction), thus leading to underproduction of housing.

  \indent State intervention is able to resolve the public goods problem via restricting the regulatory power of localities to restrict new housing. We can view this as $H_i$ is mandated to be at some minimum threshold --- localities for which $H_i$ is below said threshold will see their utility drop, while localities for which $H_i$ is above the threshold will see no change in their utility-maximizing quantity. The state resolves the public goods game by mandating the minimum ease of housing construction such that the quantity restrictions in place by localities are not binding.
  \section*{Effect on Welfare of Tenants and Homeowners}%
  \indent Glaeser and Gyourko (2002) found that local government housing restrictions are binding by showing that the intensive margin of land --- or the price that homeowners were willing to pay for an increase in parcel area --- was much lower than the extensive margin of land --- or the value of a plot of land with housing built on it --- whereas in a competitive market the prices would be identical.\supercite{glaeser_gyourko_2002} The effect of the public goods problem is to exacerbate these restrictions, implying that state intervention would help return the market to equilibrium.

  \indent Local government land use restrictions tend to decrease the price elasticity of housing. Evidence comparing supply growth and prices in different metro areas found that the price elasticity of supply in housing decreased most post-2008 in places with the most stringent land use restrictions.\supercite{albuquerque_2020}

  \indent In addition to local government restrictions decreasing the elasticity of supply, local government restrictions often serve as hard quantity restrictions. For example, the city of Boulder, Colorado had in place a hard limit of 1\% per annum growth,\supercite{growth_caps} regardless of whether or not the housing market desired more than 1\% growth.

  \indent The effect of quantity restrictions paired with inelastic supply serves to create, and widen, deadweight loss, primarily to the detriment of consumers of housing (tenants) as prices are increased above marginal costs. If we assume that a consumer has Cobb-Douglas preferences over square footage ($S$) and other consumption ($C$), the relative increase in the price of a square foot of housing induces a substitution and income effect away from housing consumption, reducing consumer utility in the process.
  \[U_{T} = a\ln(C) + b\ln(S)\]
  \indent The benefits of quantity restrictions accrue owner-occupiers able to take advantage of increased asset values. However, due to the fact that quantity restrictions cause deadweight loss, and that tenants, who are of lower income and wealth than homeowners,\supercite{tenant_income} are primarily affected by the commensurate increase in price, both utilitarian and Rawlsian social welfare are reduced under the regime of local government supply restrictions. However, the flipside is that 
  \section*{Recommendation}%
  \indent The strong negative effects of local government quantity restrictions on social welfare in both the utilitarian and Rawlsian understandings suggest that state intervention to reduce local governments' ability to restrict housing construction would be broadly welfare-improving. State laws such as Senate Bill 35, Senate Bill 1069, and Assembly Bill 2097 help contribute to a reduction in the restrictiveness of local government housing policy, thereby reducing the marginal cost of a square foot of housing, thus increasing the welfare of tenants at the marginal expense of owner-occupiers. These measures should be broadly favorable to improve social welfare, and thus should be implemented and expanded upon.
  \printbibliography
\end{document}
