\documentclass[10pt]{extarticle}
\title{}
\author{}
\date{}
\usepackage[shortlabels]{enumitem}


%paper setup
\usepackage{geometry}
\geometry{letterpaper, portrait, margin=1in}
\usepackage{fancyhdr}
% sans serif font:
\usepackage{cmbright}
%symbols
\usepackage{amsmath}
\usepackage{bigints}
\usepackage{amssymb}
\usepackage{amsthm}
\usepackage{mathtools}
\usepackage{bbm}
\usepackage[hidelinks]{hyperref}
\usepackage{gensymb}
\usepackage{multirow,array}
\usepackage{multicol}

\newtheorem*{remark}{Remark}
\usepackage[T1]{fontenc}
\usepackage[utf8]{inputenc}

%chemistry stuff
%\usepackage[version=4]{mhchem}
%\usepackage{chemfig}

%plotting
\usepackage{pgfplots}
\usepackage{tikz}
\tikzset{middleweight/.style={pos = 0.5}}
%\tikzset{weight/.style={pos = 0.5, fill = white}}
%\tikzset{lateweight/.style={pos = 0.75, fill = white}}
%\tikzset{earlyweight/.style={pos = 0.25, fill=white}}

%\usepackage{natbib}

%graphics stuff
\usepackage{graphicx}
\graphicspath{ {./images/} }
\usepackage[style=numeric, backend=biber]{biblatex} % Use the numeric style for Vancouver
\addbibresource{the_bibliography.bib}
%code stuff
%when using minted, make sure to add the -shell-escape flag
%you can use lstlisting if you don't want to use minted
%\usepackage{minted}
%\usemintedstyle{pastie}
%\newminted[javacode]{java}{frame=lines,framesep=2mm,linenos=true,fontsize=\footnotesize,tabsize=3,autogobble,}
%\newminted[cppcode]{cpp}{frame=lines,framesep=2mm,linenos=true,fontsize=\footnotesize,tabsize=3,autogobble,}

%\usepackage{listings}
%\usepackage{color}
%\definecolor{dkgreen}{rgb}{0,0.6,0}
%\definecolor{gray}{rgb}{0.5,0.5,0.5}
%\definecolor{mauve}{rgb}{0.58,0,0.82}
%
%\lstset{frame=tb,
%	language=Java,
%	aboveskip=3mm,
%	belowskip=3mm,
%	showstringspaces=false,
%	columns=flexible,
%	basicstyle={\small\ttfamily},
%	numbers=none,
%	numberstyle=\tiny\color{gray},
%	keywordstyle=\color{blue},
%	commentstyle=\color{dkgreen},
%	stringstyle=\color{mauve},
%	breaklines=true,
%	breakatwhitespace=true,
%	tabsize=3
%}
% text + color boxes
\renewcommand{\mathbf}[1]{\mathbbm{#1}}
\usepackage[most]{tcolorbox}
\tcbuselibrary{breakable}
\tcbuselibrary{skins}
\newtcolorbox{problem}[1]{colback=white,enhanced,title={\small #1},
          attach boxed title to top center=
{yshift=-\tcboxedtitleheight/2},
boxed title style={size=small,colback=black!60!white}, sharp corners, breakable}
%including PDFs
%\usepackage{pdfpages}
\setlength{\parindent}{0pt}
\usepackage{cancel}
\pagestyle{fancy}
\fancyhf{}
\rhead{Avinash Iyer}
\lhead{Econ 308: Problem Set 10}
\newcommand{\card}{\text{card}}
\newcommand{\ran}{\text{ran}}
\newcommand{\N}{\mathbbm{N}}
\newcommand{\Q}{\mathbbm{Q}}
\newcommand{\Z}{\mathbbm{Z}}
\newcommand{\R}{\mathbbm{R}}
\begin{document}
  \begin{problem}{Labor vs. Capital Taxation}
    Consider a model in which individuals live for two periods and have utility functions of the form $U = \ln(c_1) + \ln(c_2)$. They earn income of \$100 in the first period and save $s$ to finance consumption in the second period. The interest rate, $r$, is 10\%.
    \tcblower
    \begin{problem}{(a)}
      Set up the individual's lifetime utility maximization problem. Solve for the optimal $c_1,c_2,$ and $s$.
      \tcblower
      \begin{align*}
        c_1 + \frac{c_2}{1+r} = 100\\
        MRS_{c_1,c_2} &= \frac{c_2}{c_1}\\
        \frac{c_2}{c_1} &= 1+r\\
        c_2 &= c_1 (1+r)\\
        c_1 + \frac{c_1(1+r)}{1+r} &= 100\\
        c_1 &= 50\\
        s &= 50\\
        c_2 &= 55
      \end{align*}
    \end{problem}
    \begin{problem}{(b)}
      The government imposes a 20\% tax on labor income. Solve for the new optimal levels of $c_1,c_2,$ and $s$. 
      \tcblower
      \begin{align*}
        c_1 + \frac{c_2}{1+r} = 80\\
        MRS_{c_1,c_2} &= \frac{c_2}{c_1}\\
        \frac{c_2}{c_1} &= 1+r\\
        c_2 &= c_1 (1+r)\\
        c_1 + \frac{c_1(1+r)}{1+r} &= 80\\
        c_1 &= 40\\
        s &= 40\\
        c_2 &= 44
      \end{align*}
      The relative level of savings and consumption is exactly reduced by 20\%, meaning there are no substitution effects, all the relative shifts are due to income effects.
    \end{problem}
    \begin{problem}{(c)}
      Instead of the labor income tax, the government imposes a 20\% tax on interest income. Solve for the new optimal levels of $c_1,c_2$, and $s$.
      \tcblower
      \begin{align*}
        c_1 + \frac{c_2}{1+r} = 100\\
        MRS_{c_1,c_2} &= \frac{c_2}{c_1}\\
        \frac{c_2}{c_1} &= 1+r(1-\tau)\\
        c_2 &= c_1 (1+r(1-\tau))\\
        c_1 + \frac{c_1(1+r(1-\tau))}{1+r(1-\tau)} &= 100\\
        c_1 &= 50\\
        s &= 50\\
        c_2 &= 54
      \end{align*}
      The level of consumption is lowered in the second period --- because of the tax on capital income, there is an income effect towards saving more, while there is a substitution effect towards saving less. However, both of these effects cancel other out in this scenario.
    \end{problem}
  \end{problem}
  \begin{problem}{Reforming Taxes}
    Suppose that the world is populated by people who are identical in every dimension except for their savings behavior. People live for two periods, earning \$500 in period 1 and \$0 in period 2. The income tax on labor earnings and interest income is 40\%, and the interest rate earned on savings is 8\%. There are two types of people: ``Hand-to-Mouth'' consumers consume everything in period 1, and ``Smoothers'' split their consumption exactly equally between the two periods.
    \tcblower
    \begin{problem}{(a)}
      How much tax would Hand-to-Mouth consumers pay in each of the two periods? How much tax would the Smoothers pay in each of the two periods?
      \tcblower
      \begin{itemize}
        \item Hand-to-Mouth: \$200
        \item Smoothers: \$210
      \end{itemize}
    \end{problem}
    \begin{problem}{(b)}
      Suppose the income tax is replaced by an 80\% consumption tax. How much tax will each type of consumer pay in each period now?
      \tcblower
      \begin{itemize}
        \item Hand-to-Mouth: \$220
        \item Smoothers:
          \begin{itemize}
            \item Period 1: \$110
            \item Period 2: \$110
          \end{itemize}
      \end{itemize}
    \end{problem}
    \begin{problem}{(c)}
      Compare the present value of the taxes paid by the two types of consumers under the two types of tax system (using the interest rate to discount future taxes). Which tax system is more equitable?
    \end{problem}
  \end{problem}
  \begin{problem}{Estate Tax Calculation}
    When Mary died in 2023, she left her child \$2.2 million in cash (generated from labor earnings), \$5 million in stock that she had purchased (with labor earnings) for \$700,000 in 1995, and a \$6 million home purchased (with labor earnings) for \$800,000 in 1990.
    \tcblower
    \begin{problem}{(a)}
      What is the total value of Mary's estate?
      \tcblower
      \$13.2 million.
    \end{problem}
    \begin{problem}{(b)}
      How much of Mary's estate is subject to estate tax?
      \tcblower
      \$300,000
    \end{problem}
    \begin{problem}{(c)}
      How much of Mary's estate has already been subject to labor income taxes?
      \tcblower
      \$3.7 million
    \end{problem}
    \begin{problem}{(d)}
      Is the remainder of Mary's estate ever taxed? Explain.
      \tcblower
      It is not taxed, seeing as the estate subject to the tax is already taxed via the labor income tax, meaning none of it is taxed.
    \end{problem}
  \end{problem}
  \begin{problem}{Investment Effects}
    Suppose that the corporate tax rate is 25\%, there is an investment tax credit of 10\%, the depreciation rate is 5\%, and the dividend yield is 10\%. The official depreciation schedule is such that the PDV of depreciation allowances is 40\% of the purchase price.
    \tcblower
    \begin{problem}{(a)}
      Calculate the per-period marginal cost of each dollar the firm spends on the machine.
      \tcblower
      \begin{align*}
        MC &= (1-\tau z - \alpha)(0.15)\\
           &= 0.12
      \end{align*}
    \end{problem}
    \begin{problem}{(b)}
      If the marginal benefit per period is $MB = 40-0.6K$, what is the optimal amount of machinery purchased?
      \tcblower
      \begin{align*}
        \left(40-0.6K\right)(0.75) &= 0.12\\
        K &= 66
      \end{align*}
    \end{problem}
    \begin{problem}{(c)}
      How would your answer change if the ITC increased to 50\%
      \begin{align*}
        (1-\tau z - \alpha)(0.15) &= 0.056\\
        \left(40-0.6K\right)(0.75) &= 0.056\\
        K &= 67
      \end{align*}
    \end{problem}
  \end{problem}
\end{document}
