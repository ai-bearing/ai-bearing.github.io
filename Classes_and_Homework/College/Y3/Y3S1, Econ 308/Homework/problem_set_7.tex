\documentclass[8pt]{extarticle}
\title{}
\author{}
\date{}
\usepackage[shortlabels]{enumitem}


%paper setup
\usepackage{geometry}
\geometry{letterpaper, portrait, margin=1in}
\usepackage{fancyhdr}
% sans serif font:
\usepackage{cmbright}
%symbols
\usepackage{amsmath}
\usepackage{amssymb}
\usepackage{amsthm}
\usepackage{mathtools}
\usepackage[hidelinks]{hyperref}
\usepackage{gensymb}
\usepackage{multirow,array}
\usepackage{multicol}

\newtheorem*{remark}{Remark}
\usepackage[T1]{fontenc}
\usepackage[utf8]{inputenc}

%chemistry stuff
%\usepackage[version=4]{mhchem}
%\usepackage{chemfig}

%plotting
\usepackage{pgfplots}
\usepackage{tikz}
\tikzset{middleweight/.style={pos = 0.5}}
%\tikzset{weight/.style={pos = 0.5, fill = white}}
%\tikzset{lateweight/.style={pos = 0.75, fill = white}}
%\tikzset{earlyweight/.style={pos = 0.25, fill=white}}

%\usepackage{natbib}

%graphics stuff
\usepackage{graphicx}
\graphicspath{ {./images/} }
\usepackage[style=numeric, backend=biber]{biblatex} % Use the numeric style for Vancouver
\addbibresource{the_bibliography.bib}
%code stuff
%when using minted, make sure to add the -shell-escape flag
%you can use lstlisting if you don't want to use minted
%\usepackage{minted}
%\usemintedstyle{pastie}
%\newminted[javacode]{java}{frame=lines,framesep=2mm,linenos=true,fontsize=\footnotesize,tabsize=3,autogobble,}
%\newminted[cppcode]{cpp}{frame=lines,framesep=2mm,linenos=true,fontsize=\footnotesize,tabsize=3,autogobble,}

%\usepackage{listings}
%\usepackage{color}
%\definecolor{dkgreen}{rgb}{0,0.6,0}
%\definecolor{gray}{rgb}{0.5,0.5,0.5}
%\definecolor{mauve}{rgb}{0.58,0,0.82}
%
%\lstset{frame=tb,
%	language=Java,
%	aboveskip=3mm,
%	belowskip=3mm,
%	showstringspaces=false,
%	columns=flexible,
%	basicstyle={\small\ttfamily},
%	numbers=none,
%	numberstyle=\tiny\color{gray},
%	keywordstyle=\color{blue},
%	commentstyle=\color{dkgreen},
%	stringstyle=\color{mauve},
%	breaklines=true,
%	breakatwhitespace=true,
%	tabsize=3
%}
% text + color boxes
\renewcommand{\mathbf}[1]{\mathbold{#1}}
\usepackage[most]{tcolorbox}
\tcbuselibrary{breakable}
\tcbuselibrary{skins}
\newtcolorbox{problem}[1]{colback=white,enhanced,title={\small #1},
          attach boxed title to top center=
{yshift=-\tcboxedtitleheight/2},
boxed title style={size=small,colback=black!60!white}, sharp corners, breakable}
%including PDFs
%\usepackage{pdfpages}
\setlength{\parindent}{0pt}
\usepackage{cancel}
\pagestyle{fancy}
\fancyhf{}
\rhead{Avinash Iyer}
\lhead{Econ 308: Problem Set 7}
\newcommand{\card}{\text{card}}
\newcommand{\ran}{\text{ran}}
\newcommand{\N}{\mathbb{N}}
\newcommand{\Q}{\mathbb{Q}}
\newcommand{\Z}{\mathbb{Z}}
\newcommand{\R}{\mathbb{R}}
\begin{document}
  \begin{problem}{Elastic Incidence}
    Consider the following model for the soda market. Suppose the aggregate demand for soda is given by $Q^D = 800 - 100P$ where $P$ denotes the price and $Q$ denotes the quantity of bottles of soda demanded. The aggregate supply of soda is given by $Q^S = 300P$.
    \begin{enumerate}[(a)]
      \item Compute the soda market equilibrium. What are the equilibrium price and quantity?
      \item Calculate the price elasticity of supply and the price elasticity of demand at the equilibrium. Compare the values and explain which side you would expect to face a higher incidence if a tax is levied on soda.
      \item Use your price elasticities to compute the marginal effect of a tax $t$ on the producer price $p$.
      \item Now suppose a tax of $t = \$1$ is imposed on each soda that is purchased. Compute the soda market equilibrium with the tax. What are the equilibrium producer price, consumer price, and quantity? Check that the change in producer price aligns with your answer to part (c).
    \end{enumerate}
    \tcblower
    \begin{problem}{(a)}
      \begin{align*}
        300P &= 800 - 100P\\
        P^{\ast} &= 2\\
        Q^{\ast} &= 600
      \end{align*}
    \end{problem}
    \begin{problem}{(b)}
      \begin{align*}
        \varepsilon_S &= \frac{dQ}{dP}\frac{P}{Q}\\
                      &= \frac{(300)(2)}{600}\\
                      &= 1
        \varepsilon_D &= \frac{dQ}{dP}\frac{P}{Q}
                      &= (-100)\frac{2}{600}\\
                      &= -\frac{1}{3}
      \end{align*}
    \end{problem}
    Since $|\varepsilon_D| < |\varepsilon_S|$, we would expect the tax to be more incident on consumers than producers.
    \begin{problem}{(c)}
      \begin{align*}
        \frac{dp}{dt} &= \frac{\varepsilon_D}{\varepsilon_S - \varepsilon_D}\\
                      &= -\frac{1}{4}
      \end{align*}
    \end{problem}
    \begin{problem}{(d)}
      \begin{align*}
        300P &= 800 - 100(P + 1)\\
        P_{\text{cons}} &= \frac{7}{4}\\
        P_{\text{prod}} &= \frac{3}{4}\\
        Q &= \frac{900}{4}
      \end{align*}
      All these answers align with those expected in (c).
    \end{problem}
  \end{problem}
  \begin{problem}{Gas Tax Incidence}
    Taxing gasoline is one potential way to reduce consumption and CO$_2$ emissions. But are gas taxes actually passed onto consumers in the form of higher gas prices? Or do they simply lower the profits of oil companies and not change the gas consumption and consequently emissions? To answer this, Doyle and Sampatharank (2008) leveraged a natural experiment in which IN and IL temporarily suspended their 5\% gas taxes on July $1$, 2000 in response to spiking gas prices.
    \tcblower
    \begin{problem}{(a)}
      Suppose that you compare the difference in gas prices in Indiana and Illinois just before and after the tax repeal in Summer 2000. Explain why this simple difference is unlikely to be a causal estimate of the impact of the tax repeal on gas prices.
      \tcblower
      Since the tax was repealed when gas prices were increasing, we might see reverse causality (the high gas prices prompted a tax cut) if we just assumed the change in gas prices to be a causal effect.
    \end{problem}
    \begin{problem}{(b)}
      Suppose instead that you use a difference-in-differences estimator by comparing the gas price changes in Indiana and Illinois (the treatment group) to the gas price changes in neighboring states that did not change their gas tax (the control group). Explain why this can help address some of the concerns in part (a).
      \tcblower
      This can help address the concerns in part (a) by seeing the change in price on Illinois and Indiana in relation to gas price trends in similar locations --- any difference in price would not be due to trends that might have prompted the tax cut, but would be due to the tax cut itself.
    \end{problem}
    Doyle and Sampatharank (2008) use a DD estimator as in part (b), with Michigan, Ohio, Iowa, Missouri, and Wisconsin serving as a collective control group.
    \begin{problem}{(c)}
      Their results are visualized in Figure 2A. The diamonds indicate the (log) difference in gas prices between the treatment and control group, and the solid line ``smoothly connects'' these diamonds. What is the approximate DD estimate for the impact of the 5\% gas tax repeal in IL and IN on the percentage change in gas prices during the week of the repeal?
      \tcblower
      The approximate effect of the 5\% gas tax repeal was a 3\% drop in gas prices.
    \end{problem}
    \begin{problem}{(d)}
      Illinois reinstated the gas tax on December 31, 2000. Figure 2C presents the corresponding visualization for gas prices in Illinois relative to prices in its neighboring states. What is the approximate DD estimate for the impact of the 5\% gas tax reinstatement in Illinois on the percentage change in gas prices during the week of the reinstatement?
      \tcblower
      The approximate effect of the 5\% gas tax reinstatement was a 3\% increase in gas prices.
    \end{problem}
    \begin{problem}{(e)}
      Based on your results in part (c) and part (d), approximately what percent of a 5\% tax change do consumers bear? What does this result suggest about the effectiveness (or not) of gas taxes for reducing gas consumption and CO$_2$ emissions?
      \tcblower
      We can expect that approximately 60\% of the tax increase is held by consumers --- therefore, it should be modestly effective at reducing gas consumption.
    \end{problem}
  \end{problem}
  \begin{problem}{Optimal Commodity Taxation}
    The demand for snorkels in Berhama is given by $Q^{D}_s = 500 - 8P_s$, and the supply of snorkels in Berhama is given by $Q_{s}^S = 200 + 4P_s$. The demand for kayaks is given by $Q_{k}^D = 650 - 6P_k$, and the supply of kayaks is given by $Q_{k}^{S} = 50 + 1.5P_k$. Both goods are currently untaxed, but the government of Berhama needs to raise \$5,000 by taxing snorkels and kayaks.
    \begin{problem}{(a)}
      Compute the deadweight loss from a per-unit tax $T_s$ on snorkels. What is the marginal DWL?
      \tcblower
      \begin{align*}
        \shortintertext{Finding $Q^*$ and $Q^T$:}
        500 - 8P &= 200 + 4P\\
        P &= 25\\
        Q^{\ast} &= 300\\
        500 - 8(P + T_s) &= 200 + 4P\\
        P &= 25 - \frac{2}{3}T_s\\
        Q^T &= 300 - \frac{8}{3}T_s\\
        \shortintertext{Finding DWL:}
        \text{DWL} &= \frac{1}{2}(-\Delta Q)(T_s)\\
                   &= \frac{4}{3}T_s^2\\
        \frac{d\text{DWL}}{dT_s} &= \frac{8}{3}T_s
      \end{align*}
    \end{problem}
    \begin{problem}{(b)}
      Compute the tax revenue from a per-unit snorkel tax of $T_s$. What is the marginal tax revenue?
      \tcblower
      \begin{align*}
        R &= (Q^T)(T_s)\\
           &= \left(300 - \frac{8}{3}T_s\right)(T_s)\\
        \frac{dR}{dT_s} &= 300 - \frac{16}{3}T_s
      \end{align*}
    \end{problem}
    \begin{problem}{(c)}
      Compute the deadweight loss from a per-unit tax $T_k$ on kayaks. What is the marginal DWL?
      \tcblower
      \begin{align*}
        \shortintertext{Finding $Q^*$ and $Q^T$:}
        650 - 6P &= 50 + 1.5P\\
        P &= 80\\
        Q^{\ast} &= 170\\
        650 - 6(P + T_k) &= 50 + 1.5P\\
        P &= 80 - \frac{4}{5}T_k\\
        Q^T &= 170-\frac{6}{5}T_k\\
        \shortintertext{Finding DWL:}
        \text{DWL} &= \frac{1}{2}(-\Delta Q)(T_k)\\
                   &= \frac{3}{5}T_k^2\\
        \frac{d\text{DWL}}{dT_k} &= \frac{6}{5}T_k
      \end{align*}
    \end{problem}
    \begin{problem}{(d)}
      Compute the tax revenue from a per-unit kayak tax of $T_k$. What is the marginal tax revenue?
      \tcblower
      \begin{align*}
        R &= \left(Q^T\right)(T_k)\\
          &= \left(170-\frac{6}{5}T_k\right)T_k\\
        \frac{dR}{dT_k} &= 170 - \frac{12}{5}T_k
      \end{align*}
    \end{problem}
    \begin{problem}{(e)}
      Use the Ramsey rule and the revenue requirement of \$5000 to solve for the optimal commodity tax on each good.
      \tcblower
      \begin{align*}
        \shortintertext{Ramsey Rule:}
        \frac{M\text{DWL}_s}{MR_s} &= \frac{M\text{DWL}_k}{MR_k}\\
        \frac{\frac{8}{3}T_s}{300-\frac{16}{3}T_s} &= \frac{\frac{6}{5}T_k}{170-\frac{12}{5}T_k}\\
        \frac{8}{3}T_s\left(170-\frac{12}{5}T_k\right) &= \frac{6}{5}T_k\left(300-\frac{16}{3}T_s\right)\\
        \frac{1360}{3}T_s - \frac{96}{15}T_kT_s &= 360T_k - \frac{32}{5}T_kT_s\\
        T_s \left(\frac{1360}{3} - \frac{96}{15}T_k\right) &= T_k \left(360 - \frac{32}{5}T_s\right)\\
        \frac{1360}{3T_k} - \frac{96}{15} &= \frac{360}{T_s} - \frac{32}{5}\\
        \frac{3T_k}{1360} &= \frac{T_s}{360}\\
        T_s &= \frac{27}{34}T_k\\
        \shortintertext{Revenue Requirement:}
        R_{k} + R_s &= 5000\\
        5000 &= T_k\left(170 - \frac{12}{5}T_k\right) + T_s\left(300 - \frac{16}{3}T_s\right)\\
        0 &= -5000 + \frac{27}{34}T_s\left(170 - \frac{324}{170}T_s\right) + T_s \left(300 - \frac{16}{3}T_s\right)\\
        T_s &\approx 15.1 \tag*{found using Wolfram|Alpha}\\
        T_k &\approx 19.0
      \end{align*}
    \end{problem}
  \end{problem}
  \begin{problem}{Tax Inattention}
    Consider the following model for a local beer market. Market demand $D$ and market supply $S$ for six-packs of beer are $D(P) = 1200 - 50P$ and $S(P) = 100P$, where $P$ is the price of a six-pack.
    \tcblower
    \begin{problem}{(a)}
      Determine the price $P^{\ast}$ that equates demand and supply. What is the total market quantity $Q^{\ast}$ that is supplied at $P^{\ast}$?
      \tcblower
      \begin{align*}
        1200 - 50P &= 100P\\
        P^{\ast} &= 8\\
        Q^{\ast} &= 800
      \end{align*}
    \end{problem}
    \begin{problem}{(b)}
      Suppose that a tax of \$1 is imposed on each six-pack of beer that is purchased. If sellers receive price $P$, then the price paid by consumers is now $P+1$. Therefore, market demand is now
      \begin{align*}
        D(P+1) &= 1200 - 50(P+1).
      \end{align*}
      Determine the price that equates demand with supply. This is the new seller price $P^{S}$. Determine the new buyer price $P^{B} = P^{S} + 1$, and determine the new market quantity $Q^{\ast\ast} = S\left(P^{S}\right)$.
      \tcblower
      \begin{align*}
        1200 - 50(P+1) &= 100P\\
        P^{S}_{1} &= \frac{23}{3} \approx 7.67\\
        P^{B}_{1} &= \frac{26}{3} \approx 8.67\\
        Q^{\ast\ast}_{1} &= \frac{2300}{3} \approx 766.67
      \end{align*}
    \end{problem}
    \begin{problem}{(c)}
      Now assume that consumers are partially inattentive to the tax in part (b). That is, if the seller receives price $P$, consumers pay $P+1$, but only perceive their price to be $P + 0.2$. As a result, market demand is now
      \begin{align*}
        D(P+0.2) &= 1200-50(P + 0.2).
      \end{align*}
      Repeat the analysis in part (b) with this new demand.
      \tcblower
      \begin{align*}
        1200-50(P + 0.2) &= 100P\\
        P^{S}_{2} &= \frac{119}{15} \approx 7.933\\
        P^{B}_{2} &= \frac{134}{15} \approx 8.933\\
        Q^{\ast\ast}_{2} &= \frac{11900}{15} \approx 793.33
      \end{align*}
    \end{problem}
    \begin{problem}{(d)}
      Use your answers in parts (b) and (c) to discuss how inattention to taxes impacts the market effects of taxation.
      \tcblower
      Inattention to taxes brings market quantities and producer prices closer to equilibrium ($Q^{\ast} - Q^{\ast\ast}_{1} > Q^{\ast} - Q^{\ast\ast}_{2}$).
    \end{problem}
  \end{problem}
\end{document}
