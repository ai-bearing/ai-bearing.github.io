\documentclass[8pt]{extarticle}
\title{}
\author{Avinash Iyer}
\date{}
\usepackage[shortlabels]{enumitem}


%paper setup
\usepackage{geometry}
\geometry{letterpaper, portrait, margin=1in}
\usepackage{fancyhdr}

%symbols
\usepackage{amsmath}
\usepackage{amssymb}
\usepackage{amsthm}
\usepackage{mathtools}
\usepackage{hyperref}
\usepackage{gensymb}
\usepackage{multirow,array}

\newtheorem*{remark}{Remark}
\usepackage[T1]{fontenc}
\usepackage[utf8]{inputenc}

%chemistry stuff
%\usepackage[version=4]{mhchem}
%\usepackage{chemfig}

%plotting
\usepackage{pgfplots}
\usepackage{tikz}
\tikzset{middleweight/.style={pos = 0.5, fill=white}}
\tikzset{weight/.style={pos = 0.5, fill = white}}
\tikzset{lateweight/.style={pos = 0.75, fill = white}}
\tikzset{earlyweight/.style={pos = 0.25, fill=white}}

%\usepackage{natbib}

%graphics stuff
\usepackage{graphicx}
\graphicspath{ {./images/} }
\usepackage[style=numeric, backend=biber]{biblatex} % Use the numeric style for Vancouver
\addbibresource{the_bibliography.bib}
%code stuff
%when using minted, make sure to add the -shell-escape flag
%you can use lstlisting if you don't want to use minted
%\usepackage{minted}
%\usemintedstyle{pastie}
%\newminted[javacode]{java}{frame=lines,framesep=2mm,linenos=true,fontsize=\footnotesize,tabsize=3,autogobble,}
%\newminted[cppcode]{cpp}{frame=lines,framesep=2mm,linenos=true,fontsize=\footnotesize,tabsize=3,autogobble,}

%\usepackage{listings}
%\usepackage{color}
%\definecolor{dkgreen}{rgb}{0,0.6,0}
%\definecolor{gray}{rgb}{0.5,0.5,0.5}
%\definecolor{mauve}{rgb}{0.58,0,0.82}
%
%\lstset{frame=tb,
%	language=Java,
%	aboveskip=3mm,
%	belowskip=3mm,
%	showstringspaces=false,
%	columns=flexible,
%	basicstyle={\small\ttfamily},
%	numbers=none,
%	numberstyle=\tiny\color{gray},
%	keywordstyle=\color{blue},
%	commentstyle=\color{dkgreen},
%	stringstyle=\color{mauve},
%	breaklines=true,
%	breakatwhitespace=true,
%	tabsize=3
%}
% text + color boxes
\usepackage[most]{tcolorbox}
\tcbuselibrary{breakable}
\newtcolorbox{problem}[1]{colback = white, title = {#1}, breakable}
\newtcolorbox{solution}{colback = white, colframe = black!75!white, title = Solution, breakable}
%including PDFs
%\usepackage{pdfpages}
\setlength{\parindent}{0pt}
\usepackage{cancel}
\pagestyle{fancy}
\fancyhf{}
\rhead{Avinash Iyer}
\lhead{Econ 308: Problem Set 6}
\newcommand{\card}{\text{card}}
\newcommand{\ran}{\text{ran}}
\newcommand{\N}{\mathbb{N}}
\newcommand{\Q}{\mathbb{Q}}
\newcommand{\Z}{\mathbb{Z}}
\newcommand{\R}{\mathbb{R}}
\begin{document}
  \begin{problem}{Disability vs. Unemployment Hazards}
    Gruber (2000) found evidence that the elasticity of labor supply with respect to DI benefits is considerably smaller than the estimates of elasticity of unemployment durations with respect to UI benefits. Why might moral hazard be less of an issue in the DI program than in the UI program?
    \tcblower
    It is likely that the moral hazard in DI is lower due to the following:
    \begin{enumerate}[(1)]
      \item DI is screened for --- unlike UI, entering the process for DI is likely to be much more consuming, so people would only enter the process for DI if there were good reason for them to be eligible for DI.
      \item Unlike DI, it is easier to exit the adverse state with regard to UI than with DI (as it is much harder to become non-disabled than to become employed).
    \end{enumerate}
  \end{problem}
  \begin{problem}{Workers' Compensation Reform}
    In May 2004, the state of Vermont significantly reformed its workers' compensation system. One key profision in this reform was to reduce the window of time during which a claimant could file an initial workers' compensation claim.
    \begin{enumerate}[(a)]
      \item How might this reform have helped to reduce the degree of fraudulent use of the workers' compensation system?
      \item What is a potential drawback to this reform?
    \end{enumerate}
    \tcblower
    \begin{problem}{(a)}
      The reform may have helped reduce the degree of fraudulent claims by reducing the incidence of long-duration workers' compensation usage, which is one of the primary ways in which moral hazard in workers' compensation appears.
    \end{problem}
    \begin{problem}{(b)}
      A potential drawback to this reform is that it might lead to workers who were injured that forgot or otherwise failed to file for workers' compensation to pay a steep price.
    \end{problem}
  \end{problem}
  \begin{problem}{Optimal UI with Data}
    In class we showed that the optimal unemployment insurance benefit level, $b^*$, is characterized by the condition:
    \begin{align*}
      \frac{\varepsilon_{p,b}}{1-p} &\approx \gamma \times \frac{\Delta c}{c}(b^*)
    \end{align*}
    where $\varepsilon_{p,b}$ is the elasticity of unemployment with respect to benefits, $\gamma$ is the coefficient of relative risk aversion, and $\frac{\Delta c}{c}(b)$ is the magnitude of consumption drop during unemployment (relative to consumption while employed), as a function of benefits.\\

    Gruber (1997) takes this formula to the data by estimating the drop in consumption at unemployment. In particular, he estimates the following regression model:
    \begin{align*}
      \frac{\delta c}{c}(b) &= \alpha + \beta \frac{b}{w}
    \end{align*}
    where $w$ is pre-unemployment wages. The ratio $b/w$ is the wage ``replacement rate'' of UI.
    \tcblower
    \begin{problem}{(a)}
      Gruber finds the following estimates: $\hat{\alpha} = 0.24$ and $\hat{\beta} = -0.28$. To appreciate the economic meaning of these values, what is the estimated percentage drop in consumption during unemployment without UI, with a 30\% replacement rate, and with a 60\% replacement rate?
      \tcblower
      \begin{align*}
        \shortintertext{No UI:}
        \frac{\Delta c}{c}(b) &= \hat\alpha + \hat\beta(0)\\
                              &= 0.24\\
        \shortintertext{30\% replacement rate:}
        \frac{\Delta c}{c}(b) &= \hat \alpha + \hat\beta(0.3)\\
                              &= 0.16\\
        \shortintertext{60\% replacement rate:}
        \frac{\Delta c}{c}(b) &= \hat \alpha + \hat\beta(0.6)\\
                              &= 0.07
      \end{align*}
    \end{problem}
    \begin{problem}{(b)}
      Substitution regression equation (1) into $\frac{\Delta c}{c}(b)$ into the optimal UI formula (treating the approximation as an exact equality). Solve for the optimal replacement rate  in terms of $\varepsilon_{p,b},p,\gamma,\alpha,$ and $\beta$.
      \tcblower
      \begin{align*}
        \frac{\varepsilon_{p,b}}{1-p} &= \gamma \left(\hat\alpha + \hat\beta\frac{b}{w}\right)\\
        \frac{b^*}{w} &= \frac{\varepsilon_{p,b}}{\gamma\hat\beta(1-p)} - \frac{\hat\alpha}{\hat\beta}
      \end{align*}
    \end{problem}
    \begin{problem}{(c)}
      Plug the estimates in part (a) into your equation for the optimal replacement rate in part (b). In addition, plug in Gruber's estimates of $\varepsilon_{p,b} = 0.43$ and $p = 0.05$. You should now have a formula for the optimal replacement rate $b^*/w$ in terms of only $\gamma$.
      \tcblower
      \begin{align*}
        \frac{b^*}{w} &= -\frac{1.61}{\gamma} + 0.85
      \end{align*}
    \end{problem}
    \begin{problem}{(c)}
      Compute the optimal replacement rate for a range of $\gamma = 2,3,4,5,$ and $6$. How does greater risk aversion impact the optimal replacement rate? Provide intuition. And for what value of $\gamma$ would the typical UI replacement rate of 45\% be optimal?
      \tcblower
      \begin{align*}
        \frac{b^*_2}{w} &= 0.05\\
        \frac{b^*_3}{w} &= 0.31\\
        \frac{b^*_4}{w} &= 0.45\\
        \frac{b^*_5}{w} &= 0.53\\
        \frac{b^*_6}{w} &= 0.82
      \end{align*}
      Greater risk aversion implies that the benefits from consumption smoothing increase, while the moral hazard effect of UI remains constant, so the optimal benefit increases in turn. At $\gamma \approx 4$, the typical UI replacement rate of 45\% is optimal.
    \end{problem}
  \end{problem}
\end{document}
