\documentclass[8pt]{extarticle}
\title{}
\author{Avinash Iyer}
\date{}
\usepackage[shortlabels]{enumitem}


%paper setup
\usepackage{geometry}
\geometry{letterpaper, portrait, margin=1in}
\usepackage{fancyhdr}

%symbols
\usepackage{amsmath}
\usepackage{amssymb}
\usepackage{amsthm}
\usepackage{mathtools}
\usepackage{hyperref}
\usepackage{gensymb}
\usepackage{multirow,array}

\newtheorem*{remark}{Remark}
\usepackage[T1]{fontenc}
\usepackage[utf8]{inputenc}

%chemistry stuff
%\usepackage[version=4]{mhchem}
%\usepackage{chemfig}

%plotting
\usepackage{pgfplots}
\usepackage{tikz}
\tikzset{middleweight/.style={pos = 0.5, fill=white}}
\tikzset{weight/.style={pos = 0.5, fill = white}}
\tikzset{lateweight/.style={pos = 0.75, fill = white}}
\tikzset{earlyweight/.style={pos = 0.25, fill=white}}

%\usepackage{natbib}

%graphics stuff
\usepackage{graphicx}
\graphicspath{ {./images/} }
\usepackage[style=numeric, backend=biber]{biblatex} % Use the numeric style for Vancouver
\addbibresource{the_bibliography.bib}
%code stuff
%when using minted, make sure to add the -shell-escape flag
%you can use lstlisting if you don't want to use minted
%\usepackage{minted}
%\usemintedstyle{pastie}
%\newminted[javacode]{java}{frame=lines,framesep=2mm,linenos=true,fontsize=\footnotesize,tabsize=3,autogobble,}
%\newminted[cppcode]{cpp}{frame=lines,framesep=2mm,linenos=true,fontsize=\footnotesize,tabsize=3,autogobble,}

%\usepackage{listings}
%\usepackage{color}
%\definecolor{dkgreen}{rgb}{0,0.6,0}
%\definecolor{gray}{rgb}{0.5,0.5,0.5}
%\definecolor{mauve}{rgb}{0.58,0,0.82}
%
%\lstset{frame=tb,
%	language=Java,
%	aboveskip=3mm,
%	belowskip=3mm,
%	showstringspaces=false,
%	columns=flexible,
%	basicstyle={\small\ttfamily},
%	numbers=none,
%	numberstyle=\tiny\color{gray},
%	keywordstyle=\color{blue},
%	commentstyle=\color{dkgreen},
%	stringstyle=\color{mauve},
%	breaklines=true,
%	breakatwhitespace=true,
%	tabsize=3
%}
% text + color boxes
\usepackage[most]{tcolorbox}
\tcbuselibrary{breakable}
\newtcolorbox{problem}[1]{colback = white, title = {#1}, breakable}
\newtcolorbox{solution}{colback = white, colframe = black!75!white, title = Solution, breakable}
%including PDFs
%\usepackage{pdfpages}
\setlength{\parindent}{0pt}
\usepackage{cancel}
\pagestyle{fancy}
\fancyhf{}
\rhead{Avinash Iyer}
\lhead{Math 400: Homework 5}
\newcommand{\card}{\text{card}}
\newcommand{\ran}{\text{ran}}
\newcommand{\N}{\mathbb{N}}
\newcommand{\Q}{\mathbb{Q}}
\newcommand{\Z}{\mathbb{Z}}
\newcommand{\R}{\mathbb{R}}
\begin{document}
  \begin{problem}{2}
    Prove that a connected graph $G$ contains an Eulerian trail if and only if exactly two vertices of $G$ have odd degree. Furthermore, each Eulerian trail of $G$ begins at one of these odd vertices and ends at the other.
    \tcblower
    \begin{description}
      \item[$(\Rightarrow)$] Let $G$ be a graph with a non-closed Eulerian trail, $T = v_1..v_k$. Then, every internal vertex in $T$ must be even, as any edge on the trail ``entering'' an internal vertex $v_i$ must be paired with a different edge that ``exits'' the vertex.\\

        However, since $T$ is not a circuit, $d(v_1)\equiv 1$ mod 2, otherwise every edge entering $v_1$ would be paired with an edge exiting $v_1$, meaning $T$ would be a circuit, and $v_k$ must have one edge entering $v_k$ after all edges incident on it are paired up.

        Finally, $G$ cannot have only one vertex of odd degree, as $E(G) = \frac{\sum d(v)}{2}$, and if only $v_1$ were odd, $E(G)$ would not be an integer.
      \item[$(\Leftarrow)$] We will prove that if $G$ has exactly two vertices of odd degree, then $G$ contains a non-closed Eulerian trail.
        \begin{description}
          \item[Base Case:] The graph $K_2$ is the smallest graph with two odd vertices, and it contains a non-closed Eulerian trail $v_1v_2$.
          \item[Inductive Hypothesis:] Suppose that a graph of $n$ vertices with two vertices of odd degree has an Eulerian trail. Then, the graph with $n-1$ vertices will either be Eulerian or have an Eulerian trail.
          \item[Proof:] Let $T = v_1..v_k$ be an Eulerian trail where $v_1$ and $v_k$ are odd. In $G-v_1$. the following cases will occur:
            \begin{description}
              \item[Case 1:] One vertex will have its degree reduced --- because $\sum d(v)$ must be even, $v_k$ is the vertex that will have its degree reduced, in which case $G - v_1$ is Eulerian, so we can create an Eulerian trail by starting from $v_1$, entering $v_k$, and going along the Eulerian circuit of $G-v_k$ before ending back at $v_1$.
              \item[Case 2:] $2k+1$ vertices have their degree reduced for some $k>0$ --- In this case, $G-v_1$ must have more than one component. Each component must feature an Eulerian trail, as each component must have two vertices have their degree reduced by $1$. Therefore, we start at $v_1$, go along the Eulerian trail in one component, return to $v_1$, etc.
            \end{description}
        \end{description}
        In either case, we have that $G$ must yield a component with an Eulerian trail.
    \end{description}
  \end{problem}
  \begin{problem}{5}
    Show how Theorem 5.1 and Corollary 5.2 can be used to help answer the question on the first page of Chapter 5 concerning the drawing in Figure 5.2
    \tcblower
    We can draw vertices at each of the intersections of two lines, with each line leaving the vertex signifying an edge. If the graph is Eulerian (or has exactly two odd vertices), then the drawing can be made without lifting one's pencil, otherwise it requires lifting one's pencil.
  \end{problem}
  \begin{problem}{9}
    A connected graph $G$ of even size contains exactly four odd vertices. Therefore $G$ contains neither an Eulerian circuit nor an Eulerian trail.
    \begin{enumerate}[(a)]
      \item Show that $G$ contains two trails $T_1$ and $T_2$ such that every edge of $G$ belongs to exactly one of these trails.
      \item Show that $G$ contains two trails $T_1'$ and $T_2'$, each of even size, such that every edge of $G$ belongs to exactly one of these trails.
    \end{enumerate}
    \tcblower
    I don't know how to do this problem.
  \end{problem}
  \begin{problem}{11}
    What is the minimum number of bridges in Königsberg that must be traversed (counting multiplicities) to conduct a round trip that crosses each of the seven bridges at least once.
    \tcblower
    There must be at least two bridges crossed twice in order to create a round trip in Königsberg.
  \end{problem}
  \begin{problem}{X1}
    A connected graph $G$ has an Eulerian circuit if every vertex of $G$ has even degree.
    \tcblower
    Let $C$ be a maximal trail in $G$, and let $v\in C$. Suppose toward contradiction that $C$ is not closed. Then, $\exists v_i$ that serves as an endpoint to $C$: $C = v_i,\dots,v_k$. We know that $d(v_i) \geq 1$. $d(v_i) \neq 2$, as otherwise $C$ would be extendible by adding $v_iv_1$ for $v_1\leftrightarrow v_i$. Finally, any other vertices in the trail must maintain the parity of $v_i$, meaning $v_i$ is odd. $\bot$

    So, $C$ must be a circuit. In particular, every vertex in $C$ must have even degree (as every incident edge is paired with an outgoing edge).\newline

    Consider $G' = (V(G),E(G) - E(C))$. Then, since $G$ is an even graph, and every vertex in $C$ is even, the degree of every vertex in $G$ is reduced by an even number when creating $G'$. Consider the component of $G'$ that contains $v$. Since this component is connected, $\exists C'$ maximal circuit such that $v\in C'$. \newline

    We combine $C'$ and $C$ into a circuit by starting at $v$, going along $C'$, then returning to $v$, then going along $C$ back to $v$. Since $C$ and $C'$ are maximal, neither cannot be extended, so $C'$ concatenated with $C$ must be such that neither is extendible, so $C'$ concatenated with $C$ contains all edges of $G$.
  \end{problem}
  \begin{problem}{X2}
    \begin{problem}{(a)}
      Give an algorithm for finding an Eulerian circuit in a connected graph all of whose vertices have positive even degree.
      \tcblower
      \begin{itemize}
        \item Pick a vertex $v$.
        \item Trace a circuit $C$ in $G$ along $v$.
        \item Delete the edges of this circuit.
        \item Find components of $G-C$, trace a circuit in each component.
        \item Concatenate each circuit with $C$.
      \end{itemize}
    \end{problem}
    \begin{problem}{(b)}
      Give an algorithm for finding an Eulerian trail in a connected graph with vertices of odd degree.
      \tcblower
      \begin{itemize}
        \item Start at a vertex $v$ with odd degree.
        \item Trace a circuit returning to $v$.
        \item Delete this circuit.
        \item Trace a trail to the other vertex of odd degree.
        \item Concatenate.
      \end{itemize}
    \end{problem}
  \end{problem}
\end{document}
