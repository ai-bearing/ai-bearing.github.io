\documentclass[10pt]{extarticle}
\title{}
\author{Avinash Iyer}
\date{}
\usepackage[shortlabels]{enumitem}

%font setup
%
%\usepackage{newpxtext,eulerpx}

%paper setup
\usepackage{geometry}
\geometry{letterpaper, portrait, margin=1in}
\usepackage{fancyhdr}

%symbols
\usepackage{amsmath}
\usepackage{amssymb}
\usepackage{mathtools}
\usepackage{hyperref}
\usepackage{gensymb}

\usepackage{multirow,array}
\usepackage[T1]{fontenc}
\usepackage[utf8]{inputenc}

%chemistry stuff
\usepackage[version=4]{mhchem}
\usepackage{chemfig}
\usepackage{multicol}
%plotting
\usepackage{pgfplots}
\usepackage{tikz}
\tikzset{middleweight/.style={pos = 0.5, fill=white}}
\tikzset{weight/.style={pos = 0.5, fill = white}}
\tikzset{lateweight/.style={pos = 0.75, fill = white}}
\tikzset{earlyweight/.style={pos = 0.25, fill=white}}

%\usepackage{natbib}

%graphics stuff
\usepackage{graphicx}
\graphicspath{ {./images/} }

%code stuff
%when using minted, make sure to add the -shell-escape flag
%you can use lstlisting if you don't want to use minted
%\usepackage{minted}
%\usemintedstyle{pastie}
%\newminted[javacode]{java}{frame=lines,framesep=2mm,linenos=true,fontsize=\footnotesize,tabsize=3,autogobble,}
%\newminted[cppcode]{cpp}{frame=lines,framesep=2mm,linenos=true,fontsize=\footnotesize,tabsize=3,autogobble,}

\usepackage{listings}
\usepackage{color}
\definecolor{dkgreen}{rgb}{0,0.6,0}
\definecolor{gray}{rgb}{0.5,0.5,0.5}
\definecolor{mauve}{rgb}{0.58,0,0.82}

\lstset{frame=tb,
	language=Java,
	aboveskip=3mm,
	belowskip=3mm,
	showstringspaces=false,
	columns=flexible,
	basicstyle={\small\ttfamily},
	numbers=none,
	numberstyle=\tiny\color{gray},
	keywordstyle=\color{blue},
	commentstyle=\color{dkgreen},
	stringstyle=\color{mauve},
	breaklines=true,
	breakatwhitespace=true,
	tabsize=3
}
% text + color boxes
\usepackage{tcolorbox}
\tcbuselibrary{breakable}
\newtcolorbox{problem}[1]{colback = white, title = {#1}, breakable}
\newtcolorbox{solution}{colback = white, colframe = black!75!white, title = Solution, breakable}
%including PDFs
\usepackage{pdfpages}
\setlength{\parindent}{0pt}

\pagestyle{fancy}
\fancyhf{}
\rhead{Avinash Iyer}
\lhead{Econ 305: Problem Set 1}
\begin{document}
  \renewcommand{\arraystretch}{1.5}
  \begin{problem}{Tadelis 3.7: Public Good Contribution}
    Three players live in a town, and each can choose to contribute to fund a streetlamp. The value of having the streetlamp is $3$ for each player, and the value of not having it is 0. The mayor asks each player ot contribute either 1 or nothing. If at least two players contribute then the lamp will be erected. If one player or no players contribute then the lamp will not be erected, in which case any person who contributed will not get his money back. Write down the normal form of this game.
    \tcblower
    \begin{description}
      \item[Players:] $N = \{1,2,3\}$
      \item[Strategy Sets:] $S_i = \{F,A\}$ for $i\in \{1,2,3\}$
      \item[Payoffs:] Let $L = \{(F,F,F),(F,F,A),(F,A,F),(A,F,F)\}$ denote the set of possibilities that yields the streetlight being built, while $D = \{(A,A,A),(F,A,A),(A,F,A),(A,A,F)\}$ denotes the set of possibilities that yield the street light not being built. Then, the payoffs are as follows:
        \begin{align*}
          v_i &= \begin{cases}
            3 &\textrm{if } s_i = A,(s_1,s_2,s_3)\in L\\
            2 &\textrm{if } s_i = F,(s_1,s_2,s_3)\in L \\
            -1 &\textrm{if } s_i=F,(s_1,s_2,s_3)\in D\\
            0 &\textrm{if } s_i=A,(s_1,s_2,s_3)\in D\\
          \end{cases}
        \end{align*}
    \end{description}
  \end{problem}
  \begin{problem}{Hermaphroditic Fish}
    Members of some species of hermaphroditic fish choose, in each mating encounter, whether to play the role of a male or a female. Each fish has a preferred role, which uses up fewer resources and hence allows more future mating. A fish obtains a payoff $H$ if it mates in its preferred role and $L$ if it mates in the other role, where $H>L$. Consider an encounter between two fish whose preferred roles are the same. Each fish has two possible strategies: mate in either role or insist on its preferred role. If both fish offer to mate in either role, the roles are assigned randomly, and each fish's payoff is $\frac{1}{2}(H+L)$. If each fish insists on its preferred role, the fish do not mate; each goes off in search of another partner, and obtains the payoff $S$. The higher the chance of meeting another partner, the larger $S$ is. Formulate this situation as a normal-form game and determine the range of values of $S$, for any given values of $H$ and $L$, for which the game differs from the Prisoner's Dilemma only in the names of the actions.
    \tcblower
    \begin{description}
      \item[Players:] $i = \{1,2\}$
      \item[Strategies:] $S_i = \{P,N\}$, where $P$ stands for preferred role and $N$ stands for neutrality in role.
      \item[Payoffs:]\hfill
        \begin{align*}
          (v_1,v_2) &= \begin{cases}
            (S,S),&\text{if } S = (P,P)\\
            (H,L),&\text{if } S = (P,N)\\
            (L,H),&\text{if } S = (N,P)\\
            (\frac{H+L}{2},\frac{H+L}{2}),&\text{if } S = (N,N)
          \end{cases}
        \end{align*}
        If $L < S < \frac{H+L}{2} < H$, then the game is no different from the Prisoner's Dilemma.
    \end{description}
  \end{problem}
  \begin{problem}{Game Domination}
    In the following strategic games, first iteratively delete the strictly dominated strategies. Is the game pure strategy dominance solvable? If so, what is the iterated-elimination equilibrium? Finally, find all the pure strategy Nash equilibria.
    \begin{problem}{(a)}
      \begin{center}
        \begin{tabular}{c|c|c|c|}
          \multicolumn{1}{c}{}&\multicolumn{1}{c}{$L$}&\multicolumn{1}{c}{$C$}&\multicolumn{1}{c}{$R$}\\
          \cline{2-4}
          $T$ & $5,0$ & $1,1$ & $4,2$ \\
          \cline{2-4}
          $M$ & $3,4$ & $1,2$ & $2,3$ \\
          \cline{2-4}
          $B$ & $1,3$ & $0,2$ & $3,0$\\
          \cline{2-4}
        \end{tabular}
      \end{center}
      \tcblower
      This game is dominance solvable: by IESDS, we get that $(T,R)$ is the pure strategy Nash equilibrium.
    \end{problem}
    \begin{problem}{(b)}
      \begin{center}
        \begin{tabular}{c|c|c|c|c|}
          \multicolumn{1}{c}{} & \multicolumn{1}{c}{$L$} & \multicolumn{1}{c}{$CL$} & \multicolumn{1}{c}{$CR$} & \multicolumn{1}{c}{$R$}\\
          \cline{2-5}
          $T$ & $1,3$ & $1,1$ & $-2,2$ & $3,2$\\
          \cline{2-5}
          $M$ & $3,0$ & $2,0$ & $4,3$ & $4,2$ \\
          \cline{2-5}
          $B$ & $3,0$ & $3,2$ & $3,1$ & $5,0$\\
          \cline{2-5}
        \end{tabular}
      \end{center}
      \tcblower
      This game is dominance solvable: by IESDS, we get that $(M,CR)$ is the pure strategy Nash equilibrium.
    \end{problem}
    \begin{problem}{(c)}
      \begin{center}
        \begin{tabular}{c|c|c|c|c|}
          \multicolumn{1}{c}{} & \multicolumn{1}{c}{$L$} & \multicolumn{1}{c}{$CL$} & \multicolumn{1}{c}{$CR$} & \multicolumn{1}{c}{$R$}\\
          \cline{2-5}
          $T$ & $2,3$ & $-2,1$ & $-2,2$ & $3,-10$\\
          \cline{2-5}
          $MT$ & $2,0$ & $1,1$ & $0,0$ & $4,2$\\
          \cline{2-5}
          $MB$ & $10,0$ & $0,0$ & $1,1$ & $5,0$\\
          \cline{2-5}
          $B$ & $1,-2$ & $3,-2$ & $10,-1$ & $5,0$\\
          \cline{2-5}
        \end{tabular}
      \end{center}
      \tcblower
      This game is dominance solvable: by IESDS, we get that $(B,R)$ is the pure strategy Nash equilibrium.
    \end{problem}
  \end{problem}
  \begin{problem}{Prisoner's Altruism}
    Suppose that two players are asked to play a Prisoner's Dilemma game. In particular, each player has two possible strategies, Quiet ($Q$) and Fink ($F$), and each strategy profile results in the player receiving \textit{amounts of money} depicted below:
    \begin{center}
      \begin{tabular}{c|c|c|}
        \multicolumn{1}{c}{} & \multicolumn{1}{c}{$Q$} & \multicolumn{1}{c}{$F$}\\
        \cline{2-3}
        $Q$ & $2,2$ & $0,3$ \\
        \cline{2-3}
        $F$ & $3,0$ & $1,1$\\
        \cline{2-3}
      \end{tabular}
    \end{center}
    Instead of the standard game, however, the players are altruistic and player $i$ gets payoff $v_i(s) = m_i(s) + \alpha m_j(s)$ where $m_i(s)$ and $m_j(s)$ are the amounts of money received by players $i$ and $j$ respectively when the strategy profile is $s$. Assume that $\alpha \geq 0$, capturing the extent of altruism.
    \begin{problem}{(a)}
      Write down the modified game with altruism (payoffs will depend on $\alpha$).
      \tcblower
      \begin{center}
        \begin{tabular}{c|c|c|}
          \multicolumn{1}{c}{} & \multicolumn{1}{c}{$Q$} & \multicolumn{1}{c}{$F$}\\
          \cline{2-3}
          $Q$ & $2+2\alpha,2+2\alpha$ & $3\alpha,3$ \\
          \cline{2-3}
          $F$ & $3,3\alpha$ & $\alpha,\alpha$\\
          \cline{2-3}
        \end{tabular}
      \end{center}
    \end{problem}
    \begin{problem}{(b)}
      For what values of $\alpha$ is $(Q,Q)$ a Nash equilibrium? Explain.
      \tcblower
      Since this game is symmetric, all we need ensure is that player $1$ has no incentive to deviate from $Q,Q$, meaning that $3 \leq 2+2\alpha$, meaning $\alpha \geq 0.5$.
    \end{problem}
  \end{problem}
  \begin{problem}{Stag Hunt}
    Jean-Jacques Rousseau in his \textit{Discourse on the origin and foundations of inequality among men} discusses a group of hunters who wish to catch a stag. They will succeed if they all remain attentive, but each is tempted to desert her post and catch a hare.\\

    We can model this as a group of $n$ hunters, each of whom has two options: remain attentive to the stag ($S$) or catch a hare ($H$). There is only one stag, but more hares than hunters. Assume that at least $m$ hunters (with $2\leq m\leq n$) are needed to pursue the stag in order to catch it. A captured stag is only shared by the hunters who catch it. Find all the Nash equilibria of the strategic game under each of the following assumptions.
    \begin{problem}{(a)}
      All hunters are needed to catch the stag: $m=n$. For each hunter, the strategy profile in which all hunters choose $S$ is ranked highest, followed by any profile in which they choose $H$, followed by any profile in which they choose $S$ and one or more of the other players chooses $H$.
      \tcblower
      \[
        v_i(s_i,s_{-i}) = \begin{cases}
          H, & s_i = H\\
          0, & s_i = S, s_{-i}\neq (S,S,\dots,S)\\
          S, & s_i = S, s_{-i} = (S,S,\dots,S)
        \end{cases}
      \] 
      The cases are as follows:
      \begin{description}
        \item[Case 1:] $s = (S,S,\dots,S)$
          \begin{description}
            \item[Follow:] $v_i = S$
            \item[Deviate:] $v_i = H$
          \end{description}
        \item[Case 2:] $s = (H,H,\dots,H)$
          \begin{description}
            \item[Follow:] $v_i = H$
            \item[Deviate:] $v_i = 0$
          \end{description}
        \item[Case 3:] $s \neq (S,S,\dots,S), (H,H,\dots,H),~s_i = S$ 
          \begin{description}
            \item[Follow:] $v_i = 0$
            \item[Deviate:] $v_i = H$
          \end{description}
      \end{description}
      Therefore, our two pure strategy Nash equilibria are $s=(S,S,\dots,S)$ and $s = (H,H,\dots,H)$.
    \end{problem}
    \begin{problem}{(b)}
      Not all hunters are needed to catch the stag: $2\leq m < n$. Each hunter prefers $1/n$ of a stag over a hare to nothing.
      \tcblower
      \[
        v_i(s_i,s_{-i}) = \begin{cases}
          H, & s_i = H\\
          0, & s_i = S,s_{-i} = \underbrace{(S,S,\dots,S)}_{q(S) < m-1}\\
          S, & s_i = S,s_{-i} = \underbrace{(S,S,\dots,S)}_{q(S) \geq m-1}\\
        \end{cases}
      \] 
      The cases are as follows:
      \begin{multicols}{2}
        \begin{description}
          \item[Case 1:] $s = (S,S,\dots,S)$, $q(S) \geq m$, $s_i=S$
            \begin{description}
              \item[Follow:] $v_i = S$
              \item[Deviate:] $v_i = H$
            \end{description}
            $s_i = H$
            \begin{description}
              \item[Follow:] $v_i = H$
              \item[Deviate:] $v_i = S$
            \end{description}
          \item[Case 2:] $s = (S,S,\dots,S)$, $q(S) = m-1$, $s_i = S$
            \begin{description}
              \item[Follow:] $v_i = 0$
              \item[Deviate:] $v_i = H$
            \end{description}
            $s_i = H$
            \begin{description}
              \item[Follow:] $v_i = H$
              \item[Deviate:] $v_i = S$
            \end{description}
          \item[Case 3:] $s = (S,S,\dots,S)$, $q(S) < m-1$, $s_i = S$
            \begin{description}
              \item[Follow:] $v_i = 0$
              \item[Deviate:] $v_i = H$
            \end{description}
            $s_i = H$
            \begin{description}
              \item[Follow:] $v_i = H$
              \item[Deviate:] $v_i = 0$
            \end{description}
          \item[Case 4:] $s = (H,H,\dots,H)$, $s_i = H$
            \begin{description}
              \item[Follow:] $v_i = H$
              \item[Deviate:] $v_i = 0$
            \end{description} % Nash Equilibrium
        \end{description}
      \end{multicols}
      Therefore, the pure strategy Nash equilibrium is $s = (H,H,\dots,H)$.
    \end{problem}
    \begin{problem}{(c)}
      Not all the hunters are needed to catch the stag: $2\leq m < n$. Each hunter now prefers the fraction $1/k$ of the stag to a hare, but prefers a hare to any smaller fraction of the stag, where $m\leq k < n$.
      \tcblower
      Let $F$ denote a fraction of the stag, where $F < H$ if more than $k$ of the hunters go after the stag.
      \[
        v_i(s_i,s_{-i}) = \begin{cases}
          H, & s_i = H\\
          0, & s_i = S, s_{-i} = \underbrace{(S,S,\dots,S)}_{q(S) < m-1}\\
          S, & s_i = S, s_{-i} = \underbrace{(S,S,\dots,S)}_{m-1\leq q(S) \leq k}\\
          P, & s_i = S, s_{-i} = \underbrace{(S,S,\dots,S)}_{q(S) > k}
        \end{cases}
      \] 
      The cases are as follows:
      \begin{multicols}{2}
        \begin{description}
          \item[Case 1:] $s = (S,S,\dots,S)$, $q(S) > k$, $s_i = S$
            \begin{description}
              \item[Follow:] $v_i = P$
              \item[Deviate:] $v_i = H$
            \end{description}
          \item[Case 2:] $s = (S,S,\dots,S)$, $q(S) = k$, $s_i = S$
            \begin{description}
              \item[Follow:] $v_i = S$
              \item[Deviate:] $v_i = H$
            \end{description}
            $s_i = H$
            \begin{description}
              \item[Follow:] $v_i = H$
              \item[Deviate:] $v_i = P$
            \end{description} % Nash Equilibrium
          \item[Case 3:] $s = (S,S,\dots,S)$, $m\leq q(S) < k$, $s_i = S$
            \begin{description}
              \item[Follow:] $v_i = S$
              \item[Deviate:] $v_i = H$
            \end{description}
            $s_i = H$
            \begin{description}
              \item[Follow:] $v_i = H$
              \item[Deviate:] $v_i = S$
            \end{description} 
          \item[Case 4:] $s = (S,S,\dots,S)$, $q(S) = m-1$, $s_i = S$
            \begin{description}
              \item[Follow:] $v_i = 0$
              \item[Deviate:] $v_i = H$
            \end{description}
            $s_i = H$
            \begin{description}
              \item[Follow:] $v_i = H$
              \item[Deviate:] $v_i = S$
            \end{description}
          \item[Case 5:] $s = (S,S,\dots,S)$, $q(S) < m-1$, $s_i = S$
            \begin{description}
              \item[Follow:] $v_i = 0$
              \item[Deviate:] $v_i = H$
            \end{description}
            $s_i = H$
            \begin{description}
              \item[Follow:] $v_i = H$
              \item[Deviate:] $v_i = 0$
            \end{description}
          \item[Case 6:] $s = (H,H,\dots,H)$, $s_i = H$
            \begin{description}
              \item[Follow:] $v_i = H$
              \item[Deviate:] $v_i = 0$
            \end{description} % Nash Equilibrium
        \end{description}
      \end{multicols}
      Thus, the pure strategy Nash equilibria are $q(S) = k$ and $s = (H,H,\dots,H)$.
    \end{problem}
  \end{problem}
  \begin{problem}{Public Good Contribution}
    Three players live in a town, and each can choose to fund a streetlamp. The value of having the streetlamp is $3$ for each player, and the value of not having it is $0$. The mayor asks each player to contribute either $1$ or nothing. If at least two players contribute, then the lamp will be erected. If one player or no players contribute, then the lamp will not be erected, in which case any player who contributed will not get their money back.
    \begin{problem}{(a)}
      Write out each player's best-response correspondence.
      \tcblower
      Let $L$ indicate contribution and $D$ indicate non-contribution.
      \begin{align*}
        BR_i(s_{-i}) &= \begin{cases}
          L,& s_{-i} = \{L,D\}\\
          D,& s_{-i} = \{L,L\},\{D,D\}
        \end{cases}
      \end{align*}
    \end{problem}
    \begin{problem}{(b)}
      What outcomes can be supported as pure strategy Nash equilibria?
      \tcblower
      The outcomes of $\{L,L,D\}$ and $\{D,D,D\}$ are both pure strategy Nash equilibria, as they are indicative of the best response being remaining at one's position.
    \end{problem}
  \end{problem}
  \begin{problem}{The $n$-Player Tragedy of the Commons}
    Suppose there are $n$ players in the tragedy of the commons.
    \begin{problem}{(a)}
      Find the Nash equilibrium of this game. How does $n$ players affect the Nash outcome?
      \tcblower
      We know that the best response function for each player is:
      \[
        BR_i = \frac{K - \sum_{j \neq i}k_j}{2}
      \] 
      Since there are $n$ players, and this is a symmetric game, we have:
      \begin{align*}
        k_i^* &= \frac{K - (n-1)k_i^*}{2}\\
        (1+n)k_{i}^* &= K\\
        k_{i}^* &= \frac{K}{1+n}
      \end{align*}
    \end{problem}
    \begin{problem}{(b)}
      Find the socially optimal outcome with $n$ players. How does $n$ affect this outcome?
      \tcblower
      Since the game is symmetric, we will maximize the $n$ player payoff by maximizing the payoff of an arbitrary player $k_i$:
      \begin{align*}
        0 &= \frac{\partial w}{\partial k_i}\\
          &= \frac{1}{k_i} - \frac{2}{K-\sum_{j=1}^{n}k_j}\\
        \frac{2}{K-k_i(n)} &= \frac{1}{k_i} \tag*{since the game is symmetric}\\
        2 k_i &= K-k_i(n)\\
        k_i &= \frac{K}{2+n}
      \end{align*}
    \end{problem}
    \begin{problem}{(c)}
      How does the Nash equilibrium outcome compare to the socially efficient outcome as $n$ approaches infinity?
      \tcblower
      The Nash equilibrium and the socially efficient outcome converge toward each other as $n$ approaches infinity, as $\lim_{n\rightarrow\infty}\frac{n}{1+n} = 1 = \lim_{n\rightarrow\infty}\frac{n}{2+n}$
    \end{problem}
  \end{problem}
\end{document}
