\documentclass[8pt]{extarticle}
\title{}
\author{Avinash Iyer}
\date{}
\usepackage[shortlabels]{enumitem}


%paper setup
\usepackage{geometry}
\geometry{letterpaper, portrait, margin=1in}
\usepackage{fancyhdr}

%symbols
\usepackage{amsmath}
\usepackage{amssymb}
\usepackage{amsthm}
\usepackage{mathtools}
\usepackage{hyperref}
\usepackage{gensymb}
\usepackage{multirow,array}

\newtheorem*{remark}{Remark}
\usepackage[T1]{fontenc}
\usepackage[utf8]{inputenc}

%chemistry stuff
%\usepackage[version=4]{mhchem}
%\usepackage{chemfig}

%plotting
\usepackage{pgfplots}
\usepackage{tikz}
\tikzset{middleweight/.style={pos = 0.5, fill=white}}
\tikzset{weight/.style={pos = 0.5, fill = white}}
\tikzset{lateweight/.style={pos = 0.75, fill = white}}
\tikzset{earlyweight/.style={pos = 0.25, fill=white}}

%\usepackage{natbib}

%graphics stuff
\usepackage{graphicx}
\graphicspath{ {./images/} }
\usepackage[style=numeric, backend=biber]{biblatex} % Use the numeric style for Vancouver
\addbibresource{the_bibliography.bib}
%code stuff
%when using minted, make sure to add the -shell-escape flag
%you can use lstlisting if you don't want to use minted
%\usepackage{minted}
%\usemintedstyle{pastie}
%\newminted[javacode]{java}{frame=lines,framesep=2mm,linenos=true,fontsize=\footnotesize,tabsize=3,autogobble,}
%\newminted[cppcode]{cpp}{frame=lines,framesep=2mm,linenos=true,fontsize=\footnotesize,tabsize=3,autogobble,}

%\usepackage{listings}
%\usepackage{color}
%\definecolor{dkgreen}{rgb}{0,0.6,0}
%\definecolor{gray}{rgb}{0.5,0.5,0.5}
%\definecolor{mauve}{rgb}{0.58,0,0.82}
%
%\lstset{frame=tb,
%	language=Java,
%	aboveskip=3mm,
%	belowskip=3mm,
%	showstringspaces=false,
%	columns=flexible,
%	basicstyle={\small\ttfamily},
%	numbers=none,
%	numberstyle=\tiny\color{gray},
%	keywordstyle=\color{blue},
%	commentstyle=\color{dkgreen},
%	stringstyle=\color{mauve},
%	breaklines=true,
%	breakatwhitespace=true,
%	tabsize=3
%}
% text + color boxes
\usepackage[most]{tcolorbox}
\tcbuselibrary{breakable}
\newtcolorbox{problem}[1]{colback = white, title = {#1}, breakable}
\newtcolorbox{solution}{colback = white, colframe = black!75!white, title = Solution, breakable}
%including PDFs
%\usepackage{pdfpages}
\setlength{\parindent}{0pt}
\usepackage{cancel}
\pagestyle{fancy}
\fancyhf{}
\rhead{Avinash Iyer}
\lhead{Econ 305: Midterm 1 Review}
\newcommand{\card}{\text{card}}
\newcommand{\ran}{\text{ran}}
\newcommand{\N}{\mathbb{N}}
\newcommand{\Q}{\mathbb{Q}}
\newcommand{\Z}{\mathbb{Z}}
\newcommand{\R}{\mathbb{R}}
\begin{document}
  \begin{problem}{Exam Questions}
    \begin{enumerate}[(1)]
      \item Conceptual/Short Answer: understanding what a Nash equilibrium is or a strategic game is, for example.
      \item Find a Nash equilibrium in a 2 player finite strategy game. May include mixed strategies.
      \item (!) Arguing for/against strategy profiles being pure strategy Nash equilibria. Examples may include
        \begin{itemize}
          \item Stag Hunt
          \item Bertrand Competition
          \item Voter Participation
        \end{itemize}
        These are usually the hardest problems. Remember that a pure strategy profile $S$ is a Nash equilibrium if and only if
        \begin{align*}
          v_i(s_i,s_{-i}) \geq v_i(s'_i,s_{-i})
        \end{align*}
        \begin{enumerate}[(i)]
          \item Find the payoff in Nash equilibrium.
          \item Argue that any deviation on the part of any player does not improve one's payoff.
          \item In order to show that a strategy profile \textit{isn't} a Nash equilibrium, find a profitable deviation.
        \end{enumerate}
        If a strategy is \textit{dominated}, then there is a strategy profile of other strategies that has a payoff that is \textit{strictly greater than} the given strategy.
      \item Find a pure strategy Nash equilibrium in a game with a continuous strategy space. (find the intersection of the best response functions).
        \begin{itemize}
          \item Cournot Competition
          \item Tragedy of the Commons
        \end{itemize}
      \item Finding a mixed strategy Nash equilibrium in a game bigger than $2\times 2$.
        \begin{itemize}
          \item Voter participation with unequal participants
          \item To Catch a Thief
          \item Public Good Games
          \item Rock, Paper, Scissors
        \end{itemize}
        To solve, find the answer to this question: How does every other participant have to mix in order to be indifferent between the two strategies?
    \end{enumerate}
  \end{problem}
  \begin{problem}{$n$-Player Cournot with Quadratic Costs}
    \begin{align*}
      v_i &= q_i\left(20-q_i-\sum q_{-i}\right) - q_i^2
    \end{align*}
    \begin{description}
      \item[Finding Best Response:]\hfill
        \begin{align*}
          v_i &= 20q_i - 2q_i^2 - q_i\sum q_{-i}\\
          0 &= \frac{\partial v_i}{\partial q_i}\\
          0 &= 20 - 4q_i - \sum q_{-i}\\
          q_i &= \frac{20-\sum q_{-i}}{4}\\
          BR_i(q_{-i}) &= \frac{20-\sum q_{-i}}{4}
        \end{align*}
      \item[Symmetric Shortcut:]\hfill
        \begin{align*}
          q_i^* &= \frac{20-(n-1)q_{i}^*}{4}\\
          4q_i^* &= 20-(n-1)q_i^*\\
          q_i^* &= \frac{20}{n+3}
        \end{align*}
    \end{description}
  \end{problem}
\end{document}
