\documentclass[10pt]{extarticle}
\title{}
\author{}
\date{}
\usepackage[shortlabels]{enumitem}


%paper setup
\usepackage{geometry}
\geometry{letterpaper, portrait, margin=1in}
\usepackage{fancyhdr}
% sans serif font:
\usepackage{cmbright}
%symbols
\usepackage{amsmath}
\usepackage{amssymb}
\usepackage{amsthm}
\usepackage{mathtools}
\usepackage[hidelinks]{hyperref}
\usepackage{gensymb}
\usepackage{multirow,array}
\usepackage{multicol}

\newtheorem*{remark}{Remark}
\usepackage[T1]{fontenc}
\usepackage[utf8]{inputenc}

%chemistry stuff
%\usepackage[version=4]{mhchem}
%\usepackage{chemfig}

%plotting
\usepackage{pgfplots}
\usepackage{tikz}
\tikzset{middleweight/.style={pos = 0.5}}
%\tikzset{weight/.style={pos = 0.5, fill = white}}
%\tikzset{lateweight/.style={pos = 0.75, fill = white}}
%\tikzset{earlyweight/.style={pos = 0.25, fill=white}}

%\usepackage{natbib}

%graphics stuff
\usepackage{graphicx}
\graphicspath{ {./images/} }
\usepackage[style=numeric, backend=biber]{biblatex} % Use the numeric style for Vancouver
\addbibresource{the_bibliography.bib}
%code stuff
%when using minted, make sure to add the -shell-escape flag
%you can use lstlisting if you don't want to use minted
%\usepackage{minted}
%\usemintedstyle{pastie}
%\newminted[javacode]{java}{frame=lines,framesep=2mm,linenos=true,fontsize=\footnotesize,tabsize=3,autogobble,}
%\newminted[cppcode]{cpp}{frame=lines,framesep=2mm,linenos=true,fontsize=\footnotesize,tabsize=3,autogobble,}

%\usepackage{listings}
%\usepackage{color}
%\definecolor{dkgreen}{rgb}{0,0.6,0}
%\definecolor{gray}{rgb}{0.5,0.5,0.5}
%\definecolor{mauve}{rgb}{0.58,0,0.82}
%
%\lstset{frame=tb,
%	language=Java,
%	aboveskip=3mm,
%	belowskip=3mm,
%	showstringspaces=false,
%	columns=flexible,
%	basicstyle={\small\ttfamily},
%	numbers=none,
%	numberstyle=\tiny\color{gray},
%	keywordstyle=\color{blue},
%	commentstyle=\color{dkgreen},
%	stringstyle=\color{mauve},
%	breaklines=true,
%	breakatwhitespace=true,
%	tabsize=3
%}
% text + color boxes
\renewcommand{\mathbf}[1]{\mathbold{#1}}
\usepackage[most]{tcolorbox}
\tcbuselibrary{breakable}
\tcbuselibrary{skins}
\newtcolorbox{problem}[1]{colback=white,enhanced,title={\small #1},
          attach boxed title to top center=
{yshift=-\tcboxedtitleheight/2},
boxed title style={size=small,colback=black!60!white}, sharp corners, breakable}
%including PDFs
%\usepackage{pdfpages}
\setlength{\parindent}{0pt}
\usepackage{cancel}
\pagestyle{fancy}
\fancyhf{}
\rhead{Avinash Iyer}
\lhead{Math 310: Problem Set 7}
\newcommand{\card}{\text{card}}
\newcommand{\ran}{\text{ran}}
\newcommand{\N}{\mathbb{N}}
\newcommand{\Q}{\mathbb{Q}}
\newcommand{\Z}{\mathbb{Z}}
\newcommand{\R}{\mathbb{R}}
\begin{document}
  \begin{problem}{Problem 1}
    Let $D\subseteq \R$ and $c\in\R$. Show that the following are equivalent:
    \begin{enumerate}[(i)]
      \item $c$ is a limit point of $D$.
      \item There is a sequence $(x_n)_n$ in $D\setminus \{c\}$ with $(x_n)_n \rightarrow c$.
    \end{enumerate}
    \tcblower
    \begin{description}
      \item[$(\Rightarrow)$] Let $c$ be a limit point of $D$. Then, taking $\delta_n = 1/n$, let $x_n \in \dot{V}_{\delta_n}(c)$. Then, $(x_n)_n \rightarrow c$.
      \item[$(\Leftarrow)$] Let $(x_n)_n$ be a sequence in $D \setminus \{c\}$ with $(x_n)_n \rightarrow c$.\\

        Then, $\forall \varepsilon > 0$, $\exists N\in \N$ with, $ \forall n\geq N$, $|x_n - c| < \varepsilon$. Thus, $\forall \varepsilon > 0$, $\exists x_n$ such that $x_n \in \dot{V}_{\varepsilon}(c)$. Thus, $c$ is a limit point.
    \end{description}
  \end{problem}
  \begin{problem}{Problem 2}
    Show that $f$ can have at most one limit at $c$.
    \tcblower
    Suppose toward contradiction that $\lim_{x\rightarrow c}f(x) = L_1$ and $\lim_{x\rightarrow c}f(x) = L_2$, where $L_1 \neq L_2$. Then, $\exists \varepsilon_{0} > 0$ such that $V_{\varepsilon}(L_1) \cap V_{\varepsilon}(L_2) = \emptyset$.\\

    Let $\delta_1$ be such that $|x - c| < \delta_1 \Rightarrow |f(x) - L_1| < \varepsilon_{0}$, and $\delta_2$ be such that $|x-c| < \delta_2 \Rightarrow |f(x) - L_2| < \varepsilon_0$. Set $\delta = \min(\delta_1,\delta_2)$.\\

    Then, $|x-c| < \delta \Rightarrow |f(x) - L_1| < \varepsilon_0$ and $|x-c| < \delta \Rightarrow |f(x) - L_2| < \varepsilon_0$. So, $\exists k$ such that $f(k)\in V_{\varepsilon}(L_1)$ and $f(k)\in V_{\varepsilon}(L_2)$. $\bot$
  \end{problem}
  \begin{problem}{Problem 3}
    Show that the following are equivalent:
    \begin{enumerate}[(i)]
      \item $\lim_{x\rightarrow c} f(x) = L$
      \item For every sequence $(x_n)_n$ in $D\setminus \{c\}$ such that $(x_n)_n \rightarrow c$, we have $(f(x_n))_n \rightarrow L$. 
    \end{enumerate}
    \tcblower
    \begin{description}
      \item[$(\Rightarrow)$] Let $\lim_{x\rightarrow c} f(x) = L$. Then, $\forall \varepsilon > 0$, $\exists \delta > 0$ such that $|x-c| < \delta \Rightarrow |f(x) - L| < \varepsilon$.\\

        So, $\forall \varepsilon > 0$, $\exists f(x_k) \in V_{\varepsilon}(L)$, such that $x_k\in \dot{V}_{\delta}(c)$. So, we have a sequence $(x_n)_n \rightarrow c$ defined by $\delta(\varepsilon, c)$, where $\left(f(x_n)\right)_n \rightarrow L$.
      \item[$(\Leftarrow)$] Assume toward contradiction that $\lim_{x\rightarrow c} f(x) \neq L$. Then, $\exists \varepsilon_0$ such that $\forall \delta > 0$, $\exists x\in \dot{V}_{\delta}(c) \cap D$ such that $|f(x) - L| > \varepsilon_0$.\\

        Let $\delta_n = \frac{1}{n}$. Then, $\exists x_n\in \dot{V}_{1/n}(c) \cap D$ with $|f(x_n) - L| > \varepsilon_0$.\\

        Since $0 < |x-c| < 1/n$, $(x_n)_n \in D\setminus \{c\}$ and $(x_n)_n \rightarrow c$, meaning $(f(x_n))_n \rightarrow L$. However, $|f(x_n) - L| > \varepsilon_0$. $\bot$
    \end{description}
  \end{problem}
  \begin{problem}{Problem 4}
    If $\lim_{x\rightarrow c} f = L$ exists, show that there is a $\delta > 0$ such that
    \begin{align*}
      \sup_{x\in\dot{V}_{\delta}(c)}|f(x)| < \infty
    \end{align*}
    \tcblower
    Let $\varepsilon = 1$. Then, $\exists \delta > 0$ such that $\forall x\in \dot{V}_{\delta}(c)$, $|f(x) - L| < 1$. Therefore,
    \begin{align*}
      |f(x)| &= |f(x) - L + L|\\
             &\leq |f(x) - L| + |L|\tag*{Triangle Inequality}\\
             &< 1 + |L|\\
             \shortintertext{So,}
      \sup_{x\in \dot{V}_{\delta}(c)}|f(x)| &\leq 1 + |L|
    \end{align*}
  \end{problem}
  \begin{problem}{Problem 5}
    Establish the following limits:
    \begin{problem}{(a)}
      \begin{align*}
        \lim_{x\rightarrow 1} \frac{3x}{1+x} &= \frac{3}{2}
      \end{align*}
      \tcblower
      \begin{description}
        \item[Preliminary Work:] Let $\varepsilon > 0$.
          \begin{align*}
            \left|\frac{3x}{1+x} - \frac{3}{2}\right| &= \frac{3|x-1|}{2|x+1|} \\
            \shortintertext{If $x\in (0,2)$, or $|x-1| < 1$, then}
            \frac{3|x-1|}{2|x+1|} &< \frac{3}{2}|x-1|\\
                                  &< \varepsilon
          \end{align*}
        \item[Proof:] Given $\varepsilon > 0$, let $\delta = \frac{1}{2}\min\left(1,\frac{2}{3}\varepsilon\right)$. Then,
          \begin{align*}
            0 < |x-c| &< \delta\\
            \left|\frac{3x}{1+x}-\frac{3}{2}\right| &< \frac{3}{2}|x-1|\\
                                                    &< \frac{3}{2}\frac{2}{3}\varepsilon\\
                                                    &= \varepsilon
          \end{align*}
      \end{description}
    \end{problem}
    \begin{problem}{(b)}
      \begin{align*}
        \lim_{x\rightarrow 6} \frac{x^2 - 3x}{x+3} &= 2
      \end{align*}
      \tcblower
      \begin{description}
        \item[Preliminary Work:] Let $\varepsilon > 0$.
          \begin{align*}
            \left|\frac{x^2-3x}{x+3} - 2\right| &= \left|\frac{x^2 - 3x - 2(x+3)}{x+3}\right|\\
                                                &= \left|\frac{x^2 - 5x - 6}{x+3}\right|\\
                                                &= \frac{|x+1|}{|x-3|}|x-6|\\
                                                \shortintertext{for $|x-6| < 1$, we have}
                                                &< 3|x-6|\\
                                                &< \varepsilon
          \end{align*}
        \item[Proof:] Let $\varepsilon > 0$, and let $\delta = \frac{1}{2}\min\left(1,\frac{\varepsilon}{3}\right)$. Then,
          \begin{align*}
            0 < |x-6| &< \delta\\
            \left|\frac{x^2-3x}{x+3} - 2\right| &< 3|x-6|\\
                                                &< 3\frac{\varepsilon}{3}\\
                                                &= \varepsilon
          \end{align*}
      \end{description}
    \end{problem}
    \begin{problem}{(c)}
      \begin{align*}
        \lim_{x\rightarrow 0}\mathbf{1}_{\Q} &= 0
      \end{align*}
      \tcblower
      Let $(x_n)_n$ be a sequence defined by $\frac{1}{n}$, and let $(y_n)_n$ be a sequence defined by $\frac{1}{n\sqrt{2}}$. Then,
      \begin{align*}
        (x_n)_n &= (1,1,1,\dots)\\
        (y_n)_n &= (0,0,0,\dots)\\
        (z_n)_n &:= (x_1,y_1,x_2,y_2,\dots)\\
                &= (1,0,1,0,\dots)
      \end{align*}
      Then, $(z_n)_n$ contains two subsequences, namely $(x_n)_n$ and $(y_n)_n$ that converge to two different values ($1$ and $0$ respectively). Therefore $\lim_{x\rightarrow 0}\mathbf{1}_{\Q}$ does not exist.
    \end{problem}
    \begin{problem}{(d)}
      \begin{align*}
        \lim_{x\rightarrow 0}\frac{x^2}{|x|} &= 0
      \end{align*}
      \tcblower
      Let $(x_n)_n$ be a sequence such that $(x_n)_n \rightarrow 0$ and $x_n\neq 0$ $\forall n\in\N$. Then,
      \begin{align*}
        f(x_n) &= \frac{x_n^2}{|x_n|}\\
               &= \frac{|x_n|^2}{|x_n|}\\
               &= |x_n|\\
               &\rightarrow 0
      \end{align*}
    \end{problem}
  \end{problem}
  \begin{problem}{Problem 6}
    For which values of $k = 0,1,2,\dots$ does
    \begin{align*}
      \lim_{x\rightarrow 0}x^k\sin(1/x)
    \end{align*}
    exist?
    \tcblower
    \begin{description}[font=\normalfont]
      \item[$k = 0$:] Suppose $k = 0$. Let $(a_n)_n \in (0,1)$ be a sequence defined by $a_n = \frac{2}{(4n+1)\pi}$, and let $(b_n)_n\in (0,1)$ be a sequence defined by $\frac{1}{\pi n}$. Then,
        \begin{align*}
          (f(a_n))_n &= (1,1,1,\dots),
          \shortintertext{and}
          (f(b_n))_n &= (0,0,0,\dots),
        \end{align*}
        meaning that $(f(a_n))_n \rightarrow 1$ and $(f(b_n))_n \rightarrow 0$. Let $(c_n)_n = (a_1,b_1,a_2,b_2,\dots)$. Then, $\left(f(c_n)\right)_n$ has a subsequence $(f(a_n))_n \rightarrow 1$ and a subsequence $\left(f(b_n)\right)_n \rightarrow 0$. Therefore, $(f(c_n))_n$ is divergent, meaning the limit does not exist.
      \item[$k\neq 0$:] Suppose $k\neq 0$. Let $(x_n)_n$ be an arbitrary sequence in $D \setminus \{0\}$ such that $(x_n)_n \rightarrow 0$. Then,
        \begin{align*}
          |f(x_n)| &= \left|x_n \sin\left(\frac{1}{x_n}\right)\right|\\
                 &\leq \left|x_n\right|\\
                 &\rightarrow 0
        \end{align*}
        meaning $\left(f(x_n)\right)_n \rightarrow 0$.
    \end{description}
  \end{problem}
  \begin{problem}{Problem 7}
    Assume $f(x) \geq 0$ for all $x\in D$ and suppose $\lim_{x\rightarrow c} f :=: L$ exists. Show that $L\geq 0$ and
    \begin{align*}
      \lim_{x\rightarrow c} \sqrt{f} = \sqrt{L}
    \end{align*}
    \tcblower
    Let $(x_n)_n \in D\setminus \{c\}$ such that $(x_n)_n \rightarrow c$. Then, $(f(x_n))_n \rightarrow L$, by the sequential definition of limits. Since $f(x_n) \geq 0$ for all $x_n$, by the properties of sequences, it must be the case that $L \geq 0$.\\

    Similarly, it must be the case that $\left(\sqrt{f(x_n)}\right)_n \rightarrow \sqrt{L}$ by the properties of sequences --- meaning that $\lim_{x\rightarrow c}\sqrt{f} = \sqrt{L}$.
  \end{problem}
  \begin{problem}{Problem 8}
    Assume $f:\R\rightarrow \R$ such that $f(x+y) = f(x) + f(y)$ for all $x,y\in\R$. If $\lim_{x\rightarrow 0}f := L$ exists, show that $L = 0$ and show that $\lim_{x\rightarrow c}f$ exists for all $c\in\R$.
    \tcblower
    \begin{description}
      \item[Part 1:] Let $(x_n)_n \in \R-\{0\}$, $(x_n)_n \rightarrow 0$. Then, since $f(x+y) = f(x) + f(y)$ and $f$ is defined on $\R$, we have
        \begin{align*}
          f(x_n) &= f(0) + f(x_n)\\
          0 &= f(0),
        \end{align*}
        meaning $(f(x_n))_n \rightarrow f(0) = 0$.
      \item[Part 2:] Let $(x_n)_n \rightarrow c$. Then, $(x_n-c)_n \rightarrow 0$. So,
        \begin{align*}
          f(x_n) &= f(x_n - c + c)\\
                 &= f(x_n - c) + f(c)\\
                 &\rightarrow f(c)
        \end{align*}
    \end{description}
  \end{problem}
  \begin{problem}{Problem 9}
    Let $f:(0,1) \rightarrow \R$ be a bounded function such that $\lim_{x\rightarrow 0} f$ does not exist. Show that there are two sequences $(x_n)_n$ and $(y_n)_n$ with $(x_n)_n \rightarrow 0$, $(y_n)_n \rightarrow 0$, and $(f(x_n))_n$ and $(f(y_n))_n$ are both convergent, but with different limits.
    \tcblower
    Since $\lim_{x\rightarrow 0}f$ does not exist, $\exists \varepsilon_0 > 0$ such that $\forall \delta > 0$, $\exists x_0 \in (0,1)$ such that $|f(x_0) - L| \geq \varepsilon_0$.\\

    Let $\delta_{x,n} = \frac{1}{n}$, and let $\delta_{y,n} = \frac{1}{n^2}$. Select $x_n \in (0,\delta_{x,n})$, and $y_n \in (0,\delta_{y,n})$ for each $n$, where $x_n \neq y_n$. Set $L_x$ and $L_y$, where $L_x \neq L_y$ such that $|f(x_n) - L_x| \geq \varepsilon_0$, and $|f(y_n)-L_y|$ $\forall x_n,y_n$.\\

    Since $f$ is bounded, $a\leq f(x_n) \leq b$ and $c\leq f(y_n) \leq d$. Then, $\exists n_j,n_k$ such that $\left(f(x_{n_j})\right)\rightarrow a$ and $\left(f(y_{n_k})\right) \rightarrow d$. \\

    With proper selection of $x_n$ and $y_n$, we find that $(x_{n_j})_j \rightarrow 0$, $(y_{n_k})_k\rightarrow 0$, and the image of these sequences converges to different points.
  \end{problem}
  \begin{problem}{Problem 10}
    Suppose $f:(0,\infty) \rightarrow \R$. Show that the following are equivalent:
    \begin{enumerate}[(i)]
      \item $\displaystyle\lim_{x\rightarrow\infty} f = L$
      \item For every sequence $(x_n)_n$ in $(0,\infty)$ with $(x_n)_n \rightarrow \infty$, we have $(f(x_n))_n \rightarrow L$.
    \end{enumerate}
    \tcblower
    \begin{description}
      \item[$(\Rightarrow)$] Let $\lim_{x\rightarrow\infty} f = L$. Then, $\forall \varepsilon > 0,~\forall k > 0$, $\exists x \geq k$ such that $f(x)\in V_{\varepsilon}(L)$.\\

        Selecting $x_n$ such that $x_n > k$, we have $(x_n)_n \rightarrow +\infty$, and $f(x_n)\in V_{\varepsilon}(L)$.
      \item[$(\Leftarrow)$] Assume $\lim_{x\rightarrow\infty}f \neq L$. Then, $(\exists \varepsilon_0)(\forall k > 0)(\exists x > k)$ such that $|f(x) - L| > \varepsilon_0$. Let $k = n$. Then, $\exists x_n > n$ with $|f(x_n) - L|\geq \varepsilon_0$.\\

        Since $(x_n)_n \rightarrow +\infty$, it must be the case by (ii) that $(f(x_n))_n \rightarrow L$. However, $|f(x_n) - L| \geq \varepsilon_0$. $\bot$
    \end{description}
  \end{problem}
  \begin{problem}{Problem 11}
    If $f:(a,\infty) \rightarrow \R$ such that $\lim_{x\rightarrow\infty}xf(x) :=: L$ exists, show that
    \begin{align*}
      \lim_{x\rightarrow\infty}f(x) = 0.
    \end{align*}
    \tcblower
    Let $(x_n)_n \rightarrow +\infty$ where $(x_n)_n \in (a,\infty)$. Then, it must be the case that $\left(x_nf(x_n)\right)_n \rightarrow L$. So, for $\varepsilon > 0$,
    \begin{align*}
      |x_nf(x_n) - L| &< \varepsilon\\
      |f(x_n)| &= \frac{|x_nf(x_n)|}{|x_n|}\\
               &= \frac{|x_nf(x_n)-L+L|}{|x_n|}\\
               &\leq \frac{|x_nf(x_n) - L|}{|x_n|} + \frac{|L|}{x_n}\\
               &< \frac{\varepsilon}{N} + \frac{|L|}{N} \tag*{for $N$ large, by the Archimedean Property}\\
               &< \varepsilon,
    \end{align*}
    meaning $|f(x_n)| \rightarrow 0$.
  \end{problem}
  \begin{problem}{Problem 12}
    Suppose $f,g:(0,\infty)\rightarrow \R$ are such that $\lim_{x\rightarrow\infty} f:=: L > 0$, and $\lim_{x\rightarrow \infty}g = \infty$. Show that $\lim_{x\rightarrow \infty}fg = \infty$. Does this hold if $L = 0$?
    \tcblower
    Let $(x_n)_n \rightarrow \infty$. Then, $\forall M > 0$, $\exists N_1$ large such that $n\geq N \Rightarrow g(x_n) > M$, and $\exists N_2$ large such that $n\geq N_2 \Rightarrow |f(x_n) - L| < \varepsilon$. Let $N = \max(N_1,N_2)$.\\

    I don't know how to commence further on the problem.
  \end{problem}
\end{document}
