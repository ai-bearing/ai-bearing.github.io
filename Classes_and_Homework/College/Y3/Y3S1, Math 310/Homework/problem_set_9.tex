\documentclass[10pt]{extarticle}
\title{}
\author{}
\date{}
\usepackage[shortlabels]{enumitem}


%paper setup
\usepackage{geometry}
\geometry{letterpaper, portrait, margin=1in}
\usepackage{fancyhdr}
% sans serif font:
\usepackage{cmbright}
%symbols
\usepackage{amsmath}
\usepackage{bigints}
\usepackage{amssymb}
\usepackage{amsthm}
\usepackage{mathtools}
\usepackage{bbm}
\usepackage[hidelinks]{hyperref}
\usepackage{gensymb}
\usepackage{multirow,array}
\usepackage{multicol}

\newtheorem*{remark}{Remark}
\usepackage[T1]{fontenc}
\usepackage[utf8]{inputenc}

%chemistry stuff
%\usepackage[version=4]{mhchem}
%\usepackage{chemfig}

%plotting
\usepackage{pgfplots}
\usepackage{tikz}
\tikzset{middleweight/.style={pos = 0.5}}
%\tikzset{weight/.style={pos = 0.5, fill = white}}
%\tikzset{lateweight/.style={pos = 0.75, fill = white}}
%\tikzset{earlyweight/.style={pos = 0.25, fill=white}}

%\usepackage{natbib}

%graphics stuff
\usepackage{graphicx}
\graphicspath{ {./images/} }
\usepackage[style=numeric, backend=biber]{biblatex} % Use the numeric style for Vancouver
\addbibresource{the_bibliography.bib}
%code stuff
%when using minted, make sure to add the -shell-escape flag
%you can use lstlisting if you don't want to use minted
%\usepackage{minted}
%\usemintedstyle{pastie}
%\newminted[javacode]{java}{frame=lines,framesep=2mm,linenos=true,fontsize=\footnotesize,tabsize=3,autogobble,}
%\newminted[cppcode]{cpp}{frame=lines,framesep=2mm,linenos=true,fontsize=\footnotesize,tabsize=3,autogobble,}

%\usepackage{listings}
%\usepackage{color}
%\definecolor{dkgreen}{rgb}{0,0.6,0}
%\definecolor{gray}{rgb}{0.5,0.5,0.5}
%\definecolor{mauve}{rgb}{0.58,0,0.82}
%
%\lstset{frame=tb,
%	language=Java,
%	aboveskip=3mm,
%	belowskip=3mm,
%	showstringspaces=false,
%	columns=flexible,
%	basicstyle={\small\ttfamily},
%	numbers=none,
%	numberstyle=\tiny\color{gray},
%	keywordstyle=\color{blue},
%	commentstyle=\color{dkgreen},
%	stringstyle=\color{mauve},
%	breaklines=true,
%	breakatwhitespace=true,
%	tabsize=3
%}
% text + color boxes
\renewcommand{\mathbf}[1]{\mathbbm{#1}}
\usepackage[most]{tcolorbox}
\tcbuselibrary{breakable}
\tcbuselibrary{skins}
\newtcolorbox{problem}[1]{colback=white,enhanced,title={\small #1},
          attach boxed title to top center=
{yshift=-\tcboxedtitleheight/2},
boxed title style={size=small,colback=black!60!white}, sharp corners, breakable}
%including PDFs
%\usepackage{pdfpages}
\setlength{\parindent}{0pt}
\usepackage{cancel}
\pagestyle{fancy}
\fancyhf{}
\rhead{Avinash Iyer}
\lhead{Math 310: Problem Set 10}
\newcommand{\card}{\text{card}}
\newcommand{\ran}{\text{ran}}
\newcommand{\N}{\mathbbm{N}}
\newcommand{\Q}{\mathbbm{Q}}
\newcommand{\Z}{\mathbbm{Z}}
\newcommand{\R}{\mathbbm{R}}
\begin{document}
  \begin{problem}{Problem 1}
    Suppose $f:[0,1]\rightarrow \R$ is a continuous function with $f(0) = f(1)$. Show that there is a $c\in [0,1/2]$ with $f(c) = f(c + 1/2)$. Conclude that there are always antipodal points on the earth's equator with the same temperature.
    \tcblower
    Consider $g(x) = f(x) - f(x+1/2)$ on $[0,1/2]$. Then, $g(0) = f(0) - f(1/2)$, and $g(1/2) = f(1/2) - f(1)$. Since $f(0) = f(1)$, it must be the case that $g(0) = -g(1/2)$.\\

    Therefore, on $[0,1/2]$, if $g(0) = k$ for $k\in\R$, then $g(1/2) = -k$, meaning that by the Intermediate Value Theorem, $\exists c\in [0,1/2]$ with $g(c)=  0$. This is equivalent to $f(c) = f(c+1/2)$ by the definition of $g$.\\

    For any two antipodes on the earth's equator, let $t(x)$ be the temperature at point $x$. Then, moving from $x$ to $-x$, where $-x$ denotes the opposite point on the earth's equator, it must be the case that the values of $t$ at $x$ and $-x$ flip. Therefore, there is a point where $t(c) = t(-c)$.
  \end{problem}
  \begin{problem}{Problem 2}
    Suppose $f: [a,b] \rightarrow \R$ is injective and continuous. Show that $f$ is strictly monotone.
    \tcblower
    Let $f: [a,b] \rightarrow \R$ be injective and continuous. WLOG, let $p < q \in [a,b]$. Then, since $p\neq q$, $f(p) \neq f(q)$, meaning that $f(p) < f(q)$ and $f(p) > f(q)$.\\

    Since $f$ is continuous, $f$ by the Intermediate Value Theorem, $\forall x\in [f(p),f(q)]$ or  $[f(q),f(p)], \exists!x'\in [p,q]$ or $[q,p]$ such that $f(x') = x$. Therefore, $\forall p,q\in [a,b]$, $p < q \Rightarrow f(p) < f(q)$ or $f(p) > f(q)$, so $f$ is strictly monotone.
  \end{problem}
  \begin{problem}{Problem 3}
    Suppose $f: [0,1]\rightarrow \R$ is a map that takes on each of its values exactly twice. Show that $f$ cannot be continuous at every point.
    \tcblower
    I don't know how to do this problem.
  \end{problem}
  \begin{problem}{Problem 4}
    Show that the function $f(x) = \frac{1}{x^2}$ is uniformly continuous on $[1,\infty)$ but not on $(0,\infty)$.
    \tcblower
    Let $f(x) = \frac{1}{x^2}$ defined on $[1,\infty)$. Let $\varepsilon > 0$.
    \begin{align*}
      \left|f(x) - f(y)\right| &= \left|\frac{1}{x^2}-\frac{1}{y^2}\right|\\
                               &= \left|\frac{x^2 - y^2}{x^2y^2}\right|\\
                               &= \frac{x+y}{x^2y^2}\left|x - y\right|\\
                               & \leq 2\left|x-y\right|\\
                               &< \varepsilon
    \end{align*}
    Set $\delta = \frac{\varepsilon}{2}$.\\

    On $(0,\infty)$, let $u_n = \frac{1}{\sqrt{n+1}}$ and $v_n = \frac{1}{\sqrt{n}}$. Then,
    \begin{align*}
      \left|f(u_n) - f(v_n)\right| &= \left|n+1 - n\right|\\
                                   &= 1\\
                                   &= \varepsilon_0\\
      \left|u_n - v_n\right| &= \left|\frac{1}{\sqrt{n+1}} - \frac{1}{\sqrt{n}}\right|\\
                             &= \left|\frac{\sqrt{n+1} - \sqrt{n}}{\sqrt{n(n+1)}}\right|\\
                             &= \left|\frac{1}{\sqrt{n(n+1)}\left(\sqrt{n+1} + \sqrt{n}\right)}\right|\\
                             &\rightarrow 0.
    \end{align*}
    Therefore, $f$ is not uniformly continuous.
  \end{problem}
  \begin{problem}{Problem 5}
    Suppose $f: \R\rightarrow \R$ is periodic with period $p$; that is,
    \begin{align*}
      f(x+p) &= f(x)\tag*{$\forall x\in\R$}
    \end{align*}
    If $f$ is continuous, show that $f$ is bounded and uniformly continuous on $\R$.
    \tcblower
    Let $x\in\R$. Since $f$ is continuous on $\R$, $f$ is continuous on $[x,x+p]$, and $f$ takes every value on $[x,x+p]$ in all of $\R$, since if $q\in [x,x+p]$, then $f(q+np) = f(q)$.\\

    Since $f$ is continuous on $[x,x+p]$, $f$ is bounded on $[x,x+p]$, and so is bounded on $\R$. Additionally, $f$ is uniformly continuous on $[x,x+p]$, and so is uniformly continuous on $\R$.
  \end{problem}
  \begin{problem}{Problem 6}
    Show that $f(x) = x$ and $g(x) = \sin(x)$ are both uniformly continuous on $\R$, but the product
    \begin{align*}
      h(x) = x\sin(x)
    \end{align*}
    is not uniformly continuous on $\R$.
    \tcblower
    Let $f(x) = x$. Setting $\delta = \varepsilon$, we have that
    \begin{align*}
      |x-y| &< \delta\\
      |f(x) - f(y)| &< \delta\\
      |f(x) - f(y)| &< \varepsilon.
    \end{align*}
    Similarly, since $\sin(x)$ is periodic and continuous, it must be uniformly continuous.
  \end{problem}
  \begin{problem}{Problem 7}
    If $f: D\rightarrow \R$ is uniformly continuous and $|f(x)| \geq k > 0$ for some $k$, show that $\frac{1}{f}$ is uniformly continuous on $D$.
    \tcblower
    Since $f: D\rightarrow\R$ is uniformly continuous, $\forall u_n,v_n\in D$ with $\left(u_n - v_n\right)_n\rightarrow 0$, $\left(f(u_n)-f(v_n)\right)_n \rightarrow 0$.\\

    Since $|f|$ is bounded away from $0$, it must be the case that
    \begin{align*}
      \left(\frac{1}{f(u_n)} - \frac{1}{f(v_n)}\right)_n \rightarrow 0,
    \end{align*}
    so $\frac{1}{f}$ is uniformly continuous.
  \end{problem}
  \begin{problem}{Problem 8}
    If $D\subseteq \R$ is a bounded set and $f: D\rightarrow \R$ is uniformly continuous, show that $f$ is bounded.
    \tcblower
    Since $D$ is bounded, $\forall x\in D,~|x| < M$ for some $M$. Let $\varepsilon > 0$, and $\delta > 0$ be the corresponding value such that $|x-y| < \delta$. Then,
    \begin{align*}
      |f(x)| &= |f(x) - f(y) + f(y)|\\
             &\leq |f(x) - f(y)| + |f(y)|\\
             &< \varepsilon + |f(y)|
    \end{align*}
    So, for all $x$, $|f(x)|$ is bounded above, meaning that $f$ is bounded.
  \end{problem}
  \begin{problem}{Problem 9}
    Suppose $f_n: D\rightarrow \R$ is a sequence of uniformly continuous functions such that $(f_n)_n \rightarrow f$ uniformly on $D$. Show that $f$ is also continuous. Is this true with pointwise convergence?
    \tcblower
    Let $f_n: D\rightarrow \R$ be a sequence of uniformly continuous functions that uniformly converges to $f: D\rightarrow \R$.\\

    Let $c\in D$. Since $f_n$ is uniformly continuous, $\forall \varepsilon > 0$, $\exists \delta > 0$ such that $\forall y\in D$ where $|c-y| < \delta$, $|f_n(c)-f_n(y)| < \varepsilon$, for all $n\in\N$. Additionally, since $f_n\rightarrow f$ uniformly, if $|c-y| < \delta$, $|f(c) - f(y)| < \varepsilon$.\\

     Therefore, $f$ is continuous at $c$ for any arbitrary $c\in D$.\\

     This is not the case with pointwise convergence --- for example, $f_n = x^n$ on $[0,1]$ converges to the discontinuous function $\delta_1$.
  \end{problem}
  \begin{problem}{Problem 10}
    Prove that there does not exist a continuous function $f: \R\rightarrow \R$ with
    \begin{align*}
      f(\Q) &\subseteq \R\setminus\Q\\
      f(\R\setminus\Q) &\subseteq \Q.
    \end{align*}
    \tcblower
    Since $f$ does not map an interval to an interval, $f$ cannot be continuous.
  \end{problem}
\end{document}
