\documentclass[9pt]{extarticle}
\title{}
\author{Avinash Iyer}
\date{}
\usepackage[shortlabels]{enumitem}

%font setup
%
%\usepackage{newpxtext,eulerpx}

%paper setup
\usepackage{geometry}
\geometry{letterpaper, portrait, margin=1in}
\usepackage{fancyhdr}

%symbols
\usepackage{amsmath}
\usepackage{amsfonts}
\usepackage{mathtools}
\usepackage{hyperref}
\usepackage{gensymb}

\usepackage[OT1]{fontenc}
\usepackage[utf8]{inputenc}

%chemistry stuff
\usepackage[version=4]{mhchem}
\usepackage{chemfig}

%plotting
\usepackage{pgfplots}
\usepackage{tikz}
\tikzset{middleweight/.style={pos = 0.5, fill=white}}
\tikzset{weight/.style={pos = 0.5, fill = white}}
\tikzset{lateweight/.style={pos = 0.75, fill = white}}
\tikzset{earlyweight/.style={pos = 0.25, fill=white}}

%\usepackage{natbib}

%graphics stuff
\usepackage{graphicx}
\graphicspath{ {./images/} }

%code stuff
%when using minted, make sure to add the -shell-escape flag
%you can use lstlisting if you don't want to use minted
%\usepackage{minted}
%\usemintedstyle{pastie}
%\newminted[javacode]{java}{frame=lines,framesep=2mm,linenos=true,fontsize=\footnotesize,tabsize=3,autogobble,}
%\newminted[cppcode]{cpp}{frame=lines,framesep=2mm,linenos=true,fontsize=\footnotesize,tabsize=3,autogobble,}

\usepackage{listings}
\usepackage{color}
\definecolor{dkgreen}{rgb}{0,0.6,0}
\definecolor{gray}{rgb}{0.5,0.5,0.5}
\definecolor{mauve}{rgb}{0.58,0,0.82}

\lstset{frame=tb,
	language=Java,
	aboveskip=3mm,
	belowskip=3mm,
	showstringspaces=false,
	columns=flexible,
	basicstyle={\small\ttfamily},
	numbers=none,
	numberstyle=\tiny\color{gray},
	keywordstyle=\color{blue},
	commentstyle=\color{dkgreen},
	stringstyle=\color{mauve},
	breaklines=true,
	breakatwhitespace=true,
	tabsize=3
}
% text + color boxes
\usepackage{tcolorbox}
\tcbuselibrary{breakable}
\newtcolorbox{problem}[1]{colback = white, title = {#1}, breakable}
\newtcolorbox{solution}{colback = white, colframe = black!75!white, title = Solution, breakable}
%including PDFs
\usepackage{pdfpages}
\setlength{\parindent}{0pt}

\pagestyle{fancy}
\fancyhf{}
\rhead{Avinash Iyer}
\lhead{Problem Set 1}
\begin{document}
  \begin{problem}{Problem 1}
    If $F$ is a finite set and $k:F\rightarrow F$ is a self-map, prove that $k$ is injective if and only if $k$ is surjective.
    \tcblower
    \begin{description}[font=\normalfont]
      \item[($\Rightarrow$)] Suppose $k$ is injective. Then, $\text{card}(k(F)) = \text{card}(F)$, and since $k(F) \subseteq F$ and $F$ is finite, $k(F) = F$, so $k$ is surjective.
      \item[($\Leftarrow$)] Let $k$ be surjective. Since $k$ is a function, $\text{card}(k(F))\leq \text{card}(F)$.\\

    Suppose $\text{card}(k(F)) < F$. Then, $k(F)$ contains at most $n-1$ elements, for $\text{card}(F) = n$, which would violate surjectivity.\\

    Thus, $\text{card}(k(F)) = \text{card}(F)$, so $k$ is injective.
    \end{description}
  \end{problem}
  \begin{problem}{Problem 2}
    Prove that a set $A$ is infinite if and only if there is a non-surjective injection $f:A\xhookrightarrow{} A$.
    \tcblower
    \begin{description}[font=\normalfont]
      \item[$(\Rightarrow)$] Let $A$ be infinite. Then, $\exists i: \mathbb{N}\xhookrightarrow{} A$; $\forall n\in \mathbb{N}, a_n:=i(n)$. Let $f: A\rightarrow A$, $f(a_i) = a_{i+1}$. Then, for $a_{i_1} \neq a_{i_2}$, $f(a_{i_1}) = a_{i_1 + 1} \neq f(a_{i_2}) = a_{i_2 + 1}$. Therefore, $f$ is injective, but $a_1\notin\text{ran}(f)$, so $f$ is not surjective.
      \item[$(\Leftarrow)$] Suppose $A$ is finite. Then, by the result in Problem $1$, $\forall f: A\xhookrightarrow{} A$, $f$ must be surjective.
    \end{description}
  \end{problem}
  \begin{problem}{Problem 3}
    Let $A$, $B$, and $C$ be sets and suppose $\textrm{card}(A) < \textrm{card}(B) \leq \textrm{card}(C)$. Prove that $\textrm{card}(A) < \textrm{card}(C)$.
    \tcblower
    Since $\text{card}(A) < \text{card}(B)$, $\text{card}(A) \leq \text{card}(B)$, so $\text{card}(A) \leq \text{card}(C)$, by the transitive property.\\

    Since $\text{card}(A) \neq \text{card}(B)$, $\text{card}(A) \neq \text{card}(C)$, so $\text{card}(A) < \text{card}(C)$.
  \end{problem}
  \begin{problem}{Problem 4}
    If $A\subseteq B$ is an inclusion of sets with $A$ countable and $B$ uncountable, show that $B\setminus A$ is uncountable.
    \tcblower
    (Solution found with a friend)\\

    Suppose toward contradiction that $B\setminus A$ is countable.\\

    Then, $A\cup (B\setminus A)$ must be countable, by union of countable sets.\\

    However, $A\cup (B\setminus A) = B$, and $B$ is uncountable, meaning that $B\setminus A$ must be uncountable.
  \end{problem}
  \begin{problem}{Problem 5}
    Is the set $\{x\in\mathbb{R} \mid x>0~\textrm{and}~x^2\in\mathbb{Q}\}$ countable?
    \tcblower
    Since $x>0$, $t(x) = x^2$ is a bijection, as it has an inverse $t^{-1}(x) = \sqrt{x}$. Let $q: \mathbb{Q} \rightarrow \mathbb{N}$ denote the denumeration of the rationals (which is bijective).\\

    $q\circ t: \{x\in\mathbb{R} \mid x>0~\mid\text{and}~x^2\in\mathbb{Q}\} \rightarrow \mathbb{N}$ is the composition of bijections, so $q\circ t$ is a bijection, so $\{x\in\mathbb{R}\mid x>0~\text{and}~x^2\in\mathbb{Q}\}$ is countable.
  \end{problem}
  \begin{problem}{Problem 6}
    Consider the set $\mathcal{F}(\mathbb{N})$ of all finite subsets of $\mathbb{N}$. Is $\mathcal{F}(\mathbb{N})$ countable?
    \tcblower
    Let $f: \mathcal{F} \rightarrow \mathbb{N}$ be defined as follows, where $p_n$ denotes the $n$th prime number.
    \[
      f(\{a_1,a_2,\dots,a_n\}) = p_1^{a_1}\cdot p_2^{a_2}\cdots p_{n}^{a_n}
    \] 
    By the fundamental theorem of arithmetic, every natural number is equal to a unique product of powers of prime numbers, meaning that $f$ is injective, so $\mathcal{F}$ is countable.
  \end{problem}
  \begin{problem}{Problem 7}
    Let $k\in\mathbb{N}$.
    \begin{enumerate}[(i)]
      \item Prove that $\mathbb{N}^k = \underbrace{\mathbb{N}\times\mathbb{N}\times\cdots\mathbb{N}}_{k~\textrm{times}}$ is countable.
      \item Show that the set $\mathbb{N}^{\infty} := \{(n_k)_{k\geq 1}\mid n_k\in \mathbb{N}\}$ consisting of all sequences of natural numbers is uncountable.
      \item Prove that the set of \textbf{finitely-supported} natural sequences $c_c(\mathbb{N}) := \{(n_k)_{k\geq 1} \mid n_k\in\mathbb{N}, n_k=0~\text{for all but finitely many }k\}$ is countable.
    \end{enumerate}
    \tcblower
    \begin{problem}{(i)}
      Let $f: \mathbb{N}^k \rightarrow \mathbb{N}$ be defined as follows, where $p_n$ denotes the $n$th prime number in the sequence $\{2,3,5,\dots\}$
      \[
        f((a_1,a_2,\dots,a_k)) = p_1^{a_1}\cdot p_2^{a_2}\cdots p_k^{a_k}
      \] 
      By the fundamental theorem of arithmetic, $f$ is an injection, so $\mathbb{N}^k$ is countable.
    \end{problem}
    \begin{problem}{(ii)}
      Suppose toward contradiction that the set of all sequences of natural numbers is countable, so $\exists f:A_n \rightarrow \mathbb{N}$ is surjective.
      \begin{align*}
        A_1 &= a_{11},a_{12},a_{13},\dots\\
        A_2 &= a_{21},a_{22},a_{23},\dots\\
            &\vdots
      \end{align*}
      Create a new sequence $N$ defined as follows:
      \begin{align*}
        n_{k} &= a_{kk} + 1
      \end{align*}
      Since $f$ is surjective, $\exists A_m = a_{m1},a_{m2},\dots,a_{mm},\dots = n_{1},n_{2},\dots,n_{m},\dots$.However, $n_m \neq a_{mm}$, so $f$ must not be surjective. Thus, $\mathbb{N}^{\infty}$ is not countable.
    \end{problem}
    \begin{problem}{(iii)}
      Let $f: c_c(\mathbb{N}) \rightarrow \mathbb{N}$ be defined as follows, where $p_n$ denotes the $n$th prime number:
      \[
        f((n_i)) = p_1^{n_1}\cdot p_2^{n_2}\cdots p_{i}^{n_i}\cdots
      \] 
      Since there are a finite number of non-zero elements in $(n_i)$, by the fundamental theorem of arithmetic, $f$ must be injective, so $c_c(\mathbb{N})$ is countable.
    \end{problem}
    \begin{problem}{(iv)}
      Is the set of decreasing natural sequences $$D:= \{(n_k)_{k\geq 1}\mid n_k\in \mathbb{N}, n_{k+1}\leq n_k,~\forall k\geq 1\}$$ countable or uncountable?
      \tcblower
      (Solution found with a friend)\\

      Let $i: D\rightarrow \mathbb{N}$ be defined as follows, where $p_k$ denotes the $k$th prime number:
      \[
        i((n_i)) = p_1^{n_1}\cdot p_2^{n_2} \cdots p_k^{n_k}
      \] 
      where $n_k$ is the lower bound of the sequence. $k$ is the smallest index in the sequence with value $n_k$.\\

      By the fundamental theorem of arithmetic, $i$ must be injective, so $D$ is countable.
    \end{problem}
  \end{problem}
  \begin{problem}{Problem 8}
    Let $f:\mathbb{R} \rightarrow \mathbb{R}$ be a function that sends rational numbers to irrational numbers and irrational numbers to rational numbers. Prove that the range $\textrm{ran}(f)$ cannot contain any interval.
    \tcblower
    In $(a,b),~a<b$, there are countably many rational numbers (as $\mathbb{Q}$ is countable), but uncountably many irrational numbers.\\

    $f_{(a,b)}: (a,b) \rightarrow (a,b)$ implies that there are uncountably many irrational numbers not in $\textrm{ran}(f_{(a,b)})$. Therefore, no interval is in $\textrm{ran}(f)$, as there is no interval in $\textrm{ran}(f_{(a,b)})$.
  \end{problem}
  \begin{problem}{Problem 9}
    Prove that the set
    \[
      \mathcal{P} := \left\{\sum_{k=0}^{n}a_kx^k \mid n\subseteq \mathbb{N}_0,a_k\in\mathbb{Q}\right\}
    \] 
    consisting of all polynomials with rational coefficients, is countable.
    \tcblower
    Let $q: \mathbb{Q} \rightarrow \mathbb{N}$ be the denumeration of the rationals, and let $f: \mathcal{P} \rightarrow \mathbb{N}^k$ be defined as follows:
    \[
      f(a_0 + a_1x + a_2x^2 + \cdots + a_kx^k) = (q(a_0),q(a_1),\dots,q(a_k))
    \] 
    Since $\mathbb{Q}$ is countable, $\forall a\in \mathbb{Q},~q(a)\in \mathbb{N}$, so the output of $f$ is a bijection to $\mathbb{N}^k$, meaning $\mathcal{P}$ is countable.
  \end{problem}
  \begin{problem}{Problem 10}
    A real number $t$ is called \textbf{algebraic} if there is a nonzero polynomial $p$ with rational coefficients such that $p(t) = 0$. If $t\in \mathbb{R}$ is not algebraic, then it is called \textbf{transcendental}. For example, $\sqrt{2}$ is algebraic, but $\pi$ is transcendental. Show that the set of algebraic numbers is countable, and conclude that there are uncountably many transcendental numbers.
    \tcblower
    By the fundamental theorem of algebra, the set of real roots of a $k$ degree polynomial has cardinality at most $k$.\\

    $\forall p\in \mathcal{P},\exists A_p = \{a_1,\dots,a_k\}$ such that $a_i\in\mathbb{R},~\forall a_i\in \{a_1,\dots,a_k\},~p(a_i) = 0$. Therefore, $\mathbb{A} = \bigcup_{p\in\mathcal{P}}A_p$ is a countable union of countable sets, meaning $\mathbb{A}$ is countable.\\

    Since $\mathbb{T} = \mathbb{R} \setminus \mathbb{A}$, from Problem 4, $\mathbb{T}$ must be uncountable.
  \end{problem}
\end{document}
