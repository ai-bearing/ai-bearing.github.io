\documentclass[10pt]{extarticle}
\title{}
\author{Avinash Iyer}
\date{}
\usepackage[shortlabels]{enumitem}


%paper setup
\usepackage{geometry}
\geometry{letterpaper, portrait, margin=1in}
\usepackage{fancyhdr}

%symbols
\usepackage{amsmath}
\usepackage{amssymb}
\usepackage{amsthm}
\usepackage{mathtools}
\usepackage{hyperref}
\usepackage{gensymb}
\usepackage{multirow,array}

\newtheorem*{remark}{Remark}
\usepackage[T1]{fontenc}
\usepackage[utf8]{inputenc}

%chemistry stuff
%\usepackage[version=4]{mhchem}
%\usepackage{chemfig}

%plotting
\usepackage{pgfplots}
\usepackage{tikz}
\tikzset{middleweight/.style={pos = 0.5, fill=white}}
\tikzset{weight/.style={pos = 0.5, fill = white}}
\tikzset{lateweight/.style={pos = 0.75, fill = white}}
\tikzset{earlyweight/.style={pos = 0.25, fill=white}}

%\usepackage{natbib}

%graphics stuff
\usepackage{graphicx}
\graphicspath{ {./images/} }
\usepackage[style=numeric, backend=biber]{biblatex} % Use the numeric style for Vancouver
\addbibresource{the_bibliography.bib}
%code stuff
%when using minted, make sure to add the -shell-escape flag
%you can use lstlisting if you don't want to use minted
%\usepackage{minted}
%\usemintedstyle{pastie}
%\newminted[javacode]{java}{frame=lines,framesep=2mm,linenos=true,fontsize=\footnotesize,tabsize=3,autogobble,}
%\newminted[cppcode]{cpp}{frame=lines,framesep=2mm,linenos=true,fontsize=\footnotesize,tabsize=3,autogobble,}

%\usepackage{listings}
%\usepackage{color}
%\definecolor{dkgreen}{rgb}{0,0.6,0}
%\definecolor{gray}{rgb}{0.5,0.5,0.5}
%\definecolor{mauve}{rgb}{0.58,0,0.82}
%
%\lstset{frame=tb,
%	language=Java,
%	aboveskip=3mm,
%	belowskip=3mm,
%	showstringspaces=false,
%	columns=flexible,
%	basicstyle={\small\ttfamily},
%	numbers=none,
%	numberstyle=\tiny\color{gray},
%	keywordstyle=\color{blue},
%	commentstyle=\color{dkgreen},
%	stringstyle=\color{mauve},
%	breaklines=true,
%	breakatwhitespace=true,
%	tabsize=3
%}
% text + color boxes
\usepackage[most]{tcolorbox}
\tcbuselibrary{breakable}
\newtcolorbox{problem}[1]{colback = white, title = {#1}, breakable}
\newtcolorbox{solution}{colback = white, colframe = black!75!white, title = Solution, breakable}
%including PDFs
%\usepackage{pdfpages}
\setlength{\parindent}{0pt}
\usepackage{cancel}
\pagestyle{fancy}
\fancyhf{}
\rhead{Avinash Iyer}
\lhead{Math 310: Problem Set 4}
\newcommand{\card}{\text{card}}
\newcommand{\ran}{\text{ran}}
\newcommand{\N}{\mathbb{N}}
\newcommand{\Q}{\mathbb{Q}}
\newcommand{\Z}{\mathbb{Z}}
\newcommand{\R}{\mathbb{R}}
\begin{document}
  \begin{problem}{Problem 1}
    Prove the following limits:
    \begin{enumerate}[(i)]
      \item $\displaystyle \left(\frac{2n}{n+2}\right)_n \rightarrow 2$
      \item $\displaystyle \left(\frac{\sqrt{n}}{n+1}\right)_n \rightarrow 0$
      \item $\displaystyle \left(\frac{(-1)^n}{\sqrt{n+7}}\right)_n \rightarrow 0$
      \item $\displaystyle \left(n^kb^n\right)_n \rightarrow 0$ where $0\leq b < 1$ and $k\in\N$
      \item $\displaystyle \left(\frac{2^{n+1}+3^{n+1}}{2^n + 3^n}\right)_n \rightarrow 3$
    \end{enumerate}
    \tcblower
    \begin{problem}{(i)}
     We need to show that
      \[(\forall \varepsilon > 0)(\exists N\in\N) \ni n \geq N \Rightarrow \left|\frac{2n}{n+2} - 2\right| < \varepsilon\]
      \begin{description}
        \item[Preliminary Work] 
          \begin{align*}
            \frac{2n}{n+2} &> 2-\varepsilon\\
            2n &> (2n - \varepsilon n) - 2\varepsilon + 4\\
            n &> \frac{4-2\varepsilon}{\varepsilon}
          \end{align*}
        \item[Proof] Let $\varepsilon > 0,~\displaystyle N = \left\lceil\frac{4-2\varepsilon}{\varepsilon}\right\rceil$. Then,
          \begin{align*}
            n &> \frac{4-2\varepsilon}{\varepsilon}\\
            \varepsilon n &> 4-2\varepsilon\\
            0 &> 4-2\varepsilon-\varepsilon n\\
            2n &> 2n + 4 - \varepsilon (n+2)\\
            2n &> (2-\varepsilon)(n+2)\\
            \frac{2n}{n+2} - 2 &> -\varepsilon\\
            \left|\frac{2n}{n+2}-2\right| &< \varepsilon \tag*{$\displaystyle \frac{2n}{n+2}< 2~\forall n\in\N$} 
          \end{align*}
      \end{description}
    \end{problem}
    \begin{problem}{(ii)}
      We need to show that
      \begin{align*}
        (\forall \varepsilon > 0)(\exists N\in \N) \ni n > N \rightarrow \left|\left(\frac{\sqrt{n}}{n + 1}\right)\right| < \varepsilon
      \end{align*}
      \begin{description}
        \item[Preliminary Work] We will show that $\displaystyle\left(\frac{1}{\sqrt{n}}\right)_n\rightarrow 0$. Let $\varepsilon > 0$ and $\displaystyle N = 1 + \left\lceil \frac{1}{\varepsilon^2}\right\rceil$. Then,
          \begin{align*}
            n &\geq N\\
            n &> \frac{1}{\varepsilon^2}\\
            \frac{1}{\sqrt{n}} &< \varepsilon \\
            \left|\frac{1}{\sqrt{n}} - 0\right| &< \varepsilon
          \end{align*}
        \item[Proof] We know that $\forall n, \frac{\sqrt{n}}{n+1} > 0$ and $\frac{\sqrt{n}}{n+1} < \frac{1}{\sqrt{n}}$. Since we showed earlier that $\frac{1}{\sqrt{n}} \rightarrow 0$, it must be the case that $\frac{\sqrt{n}}{n+1} \rightarrow 0$.
      \end{description}
    \end{problem}
    \begin{problem}{(iii)}
      We need to show that
      \begin{align*}
        (\forall \varepsilon > 0)(\exists N \in \N) \ni n \geq N \Rightarrow \left|\frac{(-1)^n}{\sqrt{n+7}}\right| < \varepsilon
      \end{align*}
      \begin{description}
        \item[Preliminary Work]
          \begin{align*}
            \frac{1}{\sqrt{n+7}} &< \varepsilon\\
            \frac{1}{\varepsilon} & < \sqrt{n+7}\\
            n &> \frac{1}{\varepsilon^2} - 7 
          \end{align*}
        \item[Proof] Let $\varepsilon > 0,~\displaystyle N =  \left\lceil\frac{1}{\varepsilon^2}\right\rceil-7$. Then,
          \begin{align*}
            n &> \frac{1}{\varepsilon^2} - 7\\
            n + 7 &> \frac{1}{\varepsilon^2}\\
            \frac{1}{\sqrt{n+7}} & < \varepsilon\\
            -\varepsilon &< \frac{-1}{\sqrt{n+7}}\\
            \left|\frac{(-1)^{n}}{\sqrt{n+7}}\right| & < \varepsilon
          \end{align*}
      \end{description}
    \end{problem}
    \begin{problem}{(iv)}
      If $b = 0$, then $n^kb^n = 0 \rightarrow 0$.\newline

      Let $0 < b < 1$. To show that $(n^kb^n)_n \rightarrow 0$, we will find what the ratio of consecutive terms tends toward:
      \begin{align*}
        \frac{a_{n+1}}{a_n} &= \frac{(n+1)^kb^{n+1}}{n^kb^n}\\
                            &= b\left(\frac{n+1}{n}\right)^k
      \end{align*}
      We claim that $\left(\frac{n+1}{n}\right)^k \rightarrow 1$. For this, we need to show that
      \begin{align*}
        (\forall \varepsilon > 0)(\exists N\in\N)\ni n\geq N \Rightarrow \left|\left(\frac{n+1}{n}\right)^k - 1 \right| < \varepsilon
      \end{align*}
      \begin{description}
        \item[Preliminary Work]
          \begin{align*}
            \left|\left(1 + \frac{1}{n}\right)^k - 1 \right| &< \varepsilon\\
            \left(1+\frac{1}{n}\right)^k &< \varepsilon + 1\\
            1 + \frac{1}{n} &< \left(\varepsilon + 1\right)^{1/k}\\
            n &> \frac{1}{\left(\varepsilon + 1\right)^{1/k} -1}
          \end{align*}
        \item[Proof] Let $\varepsilon > 0$. Let $\displaystyle N = \left\lceil \frac{1}{\left(\varepsilon + 1\right)^{1/k}-1}\right\rceil + 1$. Then, for $n\geq N$, we have
          \begin{align*}
            n &> \frac{1}{\left(\varepsilon + 1\right)^{1/k}-1}\\
            (\varepsilon + 1)^{1/k} &> 1 + \frac{1}{n}\\
            \left(1 + \frac{1}{n}\right)^{k} - 1 &< \varepsilon
          \end{align*}
          whence $\displaystyle\left|\left(\frac{n+1}{n}\right)^{k} - 1\right| = \left(1 + \frac{1}{n}\right)^{k} - 1$. 
      \end{description}
      Therefore, since $\displaystyle \left(\frac{n+1}{n}\right)^k \rightarrow 1$, the ratio converges to $b < 1$, meaning $n^kb^n \rightarrow 0$.
    \end{problem}
    \begin{problem}{(v)}
      \begin{description}
        \item[Preliminary Work]
          \begin{align*}
            \left|\frac{2^{n+1} + 3^{n+1}}{2^n + 3^n} - 3 \right| &< \varepsilon\\
            3 - \frac{2^{n+1} + 3^{n+1}}{2^n + 3^n} &< \varepsilon\\
            \frac{3(2^n + 3^n) - 2^{n+1} - 3^{n+1}}{2^n + 3^n} &< \varepsilon\\
            \frac{2^n}{2^n + 3^n} &< \varepsilon\\
            2^n &< (2^n + 3^n)\varepsilon\\
            (1-\varepsilon)2^n &< \varepsilon \cdot 3^n\\
            \frac{1-\varepsilon}{\varepsilon} &< \left(\frac{3}{2}\right)^n\\
            n &> \frac{\ln(1-\varepsilon) - \ln\varepsilon}{\ln 3 - \ln 2}
          \end{align*}
        \item[Proof] Let $\varepsilon > 0$ and $\displaystyle N = \left\lceil \frac{\ln(1-\varepsilon) - \ln\varepsilon}{\ln 3 - \ln 2}\right\rceil + 1$. Then, for $n\geq N$, we have
          \begin{align*}
            n &> \frac{\ln(1-\varepsilon) - \ln\varepsilon}{\ln 3 - \ln 2}\\
            n\ln \left(\frac{3}{2}\right) &> \ln \left(\frac{1-\varepsilon}{\varepsilon}\right)\\
            \frac{3^n}{2^n} &> \frac{1-\varepsilon}{\varepsilon}\\
            \varepsilon (3^n + 2^n) &> 2^n\\
            \frac{2^n}{2^n + 3^n} &< \varepsilon
          \end{align*}
          whence $\displaystyle\left|\frac{2^{n+1} + 3^{n+1}}{2^n + 3^n} - 3\right| = \frac{2^n}{2^n + 3^n}$.
      \end{description}
    \end{problem}
  \end{problem}
  \begin{problem}{Problem 2}
    Show that the sequence $(\cos(n))_n$ does not converge.
    \tcblower
    We will show that $(\cos(n))_n$ does not converge to $L$ for any $L \in\R$
    \begin{description}
      \item[Case 1:] Suppose $L > 1$. Set $\varepsilon_0 = \frac{L-1}{2}$. Then, for any $N\in\N$, let $n = N$.
        \begin{align*}
          |\cos(n) - L| &= L - \cos(n)\\
                        &\geq L-1\\
                        &> \frac{L-1}{2}\\
                        &= \varepsilon_0
        \end{align*}
      \item[Case 2:] Suppose $L < -1$. Set $\varepsilon_0 = \frac{1-L}{2}$. Then, for any $N\in\N$, let $n = N$.
        \begin{align*}
          |\cos(n) - L| &= \cos(n) - L\\
                        &\geq 1 - L\\
                        &> \frac{1-L}{2}\\
                        &=\varepsilon_0
        \end{align*}
      \item[Case 3:] Suppose $L = 0$. Set $\varepsilon_0 = 1/2$. Given any $N\in\N$, find $n \geq N$ with $\cos(n) \geq 1/2$. Then, $|\cos(n) - 0| \geq \varepsilon_0$.
      \item[Case 4:] Suppose $0 < L < 1$. Set $\varepsilon_0 = L/2$. Given any $N\in\N$, we want to find $n\geq N$ such that $\cos(n) < 0$.\\

        Find $k$ large such that $N < \frac{(4k+1)\pi}{2}$, which is always possible by the Archimedean property. Then, $N < \frac{(4k+1)\pi}{2} < \frac{(4k+3)\pi}{2}$. So, we find $n = \left\lceil \frac{(4k+1)\pi}{2}\right\rceil$, meaning $\cos(n) < 0$, so $|L - \cos(n)| \geq \varepsilon_0$.
      \item[Case 5:] Suppose $-1 < L < 0$. Set $\varepsilon_0 = -L/2$. Given any $N\in\N$, we want to find $n\geq N$ such that $\cos(n) > 0$.\\

        Find $k$ large such that $N < \frac{(4k-1)\pi}{2}$. This is always possible by the Archimedean property. Then, $N < \frac{(4k-1)\pi}{2} < \frac{(4k+1)\pi}{2}$. So, we find $n = \left\lceil \frac{(4k-1)\pi}{2}\right\rceil$, meaning $\cos(n) > 0$, so $|L-\cos(n)| \geq \varepsilon_0$.
    \end{description}
  \end{problem}
  \begin{problem}{Problem 3}
    If $(x_n)_n$ is a real sequence converging to $x$, show that
    \begin{align*}
      (|x_n|)_n \rightarrow |x|
    \end{align*}
    Is the converse true?
    \tcblower
    If $(x_n)_n \rightarrow x$, then $|x_n - x| \rightarrow 0$. So
    \begin{align*}
      \left||x_n|-|x|\right| &\leq |x_n - x| \tag*{Reverse Triangle Inequality}\\
                  &\rightarrow 0
    \end{align*}
    So, $|x_n| \rightarrow |x|$.\newline

    The converse is not true. For example, the sequence $\left(|(-1)^n|\right)_n \rightarrow 1$, but $\left((-1)^n\right)_n$ does not converge.
  \end{problem}
  \begin{problem}{Problem 4}
    If $(x_n)_n$ is a real sequence converging to $x > 0$, show that there is an $N\in\N$ and $c > 0$ such that
    \begin{align*}
      x_n \geq c~\forall n\geq N
    \end{align*}
    \tcblower
    Since $(x_n)_n \rightarrow x$, we know that $(\forall \varepsilon > 0)(\exists N\in\N)$ such that $n\geq n \rightarrow x_n\in V_{\varepsilon}(x)$.\newline

    In particular, let $\varepsilon_0 = \frac{|0-x|}{3}$, $c = \frac{x}{3} < x$, and $\varepsilon_1$ small such that $V_{\varepsilon_1}(c) \cap V_{\varepsilon_0}(x) = \emptyset$.\newline

    Then, $\exists N\in\N$ such that $n\geq N \Rightarrow x_n\in V_{\varepsilon_0}(x) > c$.
  \end{problem}
  \begin{problem}{Problem 5}
    If $(x_n)_n$ is a real sequence of positive terms converging to $x$, show that $x \geq 0$ and 
    \begin{align*}
      \left(\sqrt{x_n}\right)_n &\rightarrow \sqrt{x}
    \end{align*}
    \tcblower
    \begin{problem}{$x \geq 0$}
      Suppose toward contradiction that $x < 0$. Let $\varepsilon = \frac{|0-x|}{2}$. Since $x_n \rightarrow x$, $\exists N\in \N$ large such that $x_n\in V_{\varepsilon}(x)$ for $n\geq N$. However, $\forall \ell\in V_{\varepsilon}(x)$, $\ell < 0$, meaning that $x_n < 0$ for large $n$. $\bot$
    \end{problem}
    \begin{problem}{$\left(\sqrt{x_n}\right)_n\rightarrow \sqrt{x}$}
      \begin{description}
        \item[Case 1:] Suppose $x = 0$. Let $\varepsilon > 0$. Then,
          \begin{align*}
            |x_n - 0| &< \varepsilon^2\\
            x_n &< \varepsilon^2\\
            \sqrt{x_n} &< \varepsilon\\
            |\sqrt{x_n} - 0| &< \varepsilon
          \end{align*}
          So, $\sqrt{x_n} \rightarrow 0$.
        \item[Case 2:] Suppose $x > 0$. Let $\varepsilon > 0$. Then,
          \begin{align*}
            \left|\sqrt{x_n} - \sqrt{x}\right| &= \left|\frac{x_n - x}{\sqrt{x_n} + \sqrt{x}}\right|\\
                                               &= \frac{1}{\sqrt{x_n}+\sqrt{x}}|x_n - x|\\
                                               &\leq \frac{1}{\sqrt{x}}|x_n - x|\\
                                               &\rightarrow 0
          \end{align*}
          Therefore, $|\sqrt{x_n}-\sqrt{x}| \rightarrow 0$, so $\sqrt{x_n}\rightarrow x$
      \end{description}
    \end{problem}
  \end{problem}
  \begin{problem}{Problem 6}
    If $(x_n)_n$ and $(y_n)_n$ are sequences with $(x_n)_n \rightarrow 0$ and $(y_n)_n$ bounded. Show that
    \begin{align*}
      (x_ny_n)_n \rightarrow 0
    \end{align*}
    \tcblower
    Let $y\in\R$ be an upper bound on $\left(y_n\right)_n$. Then,
    \begin{align*}
      |x_ny_n| &\leq |x_n||y|\\
               &\rightarrow 0
    \end{align*}
    Therefore, $x_ny_n \rightarrow 0$.
  \end{problem}
  \begin{problem}{Problem 7}
    If $(x_n)_n$ is a sequence of positive terms such that
    \begin{align*}
      \left(\frac{x_{n+1}}{x_n}\right)_n \xrightarrow{} L > 1,
    \end{align*}
    show that $(x_n)_n$ is not bounded, and thus not convergent. If $L = 1$, can we make any conclusions?
    \tcblower
    Since $L > 1$, $L = 1 + a$. Let $\varepsilon = \frac{a}{2}$. Then,
    \begin{align*}
      \left|\frac{x_{n+1}}{x_n} - (1+a)\right| &< \varepsilon\\
      1 + \frac{a}{2} < \frac{x_{n+1}}{x_n} &< 1 + \frac{3a}{2}\\
      \shortintertext{so, $\forall n\in\N$,}
      x_{n+1} &> x_n \left(1 + \frac{a}{2}\right)\\
      x_{n+2} &> x_{n+1}\left(1 + \frac{a}{2}\right)\\
              &> x_n \left(1 + \frac{a}{2}\right)^2\\
              &\geq x_n \left(1 + \frac{(2)a}{2}\right) \tag*{Bernoulli's Inequality}\\
              \shortintertext{Inductively, we have}
      x_{n+k} &> x_n \left(1 + \frac{(k)a}{2}\right)
    \end{align*}
    Since $\left(1 + \frac{(k)a}{2}\right)_k \rightarrow \infty$ and $x_n > 0~\forall n\in\N$, we have that $\left(x_{n}\right)_n$ goes to infinity, meaning it is not bounded.\\
    \vspace{4pt}
    \rule{\textwidth}{0.4pt}
    \vspace{4pt}
    If $L = 1$, we cannot make any conclusions as to the boundedness or convergence of the sequence.
  \end{problem}
  \begin{problem}{Problem 8}
    Let $a,b$ be positive numbers. Show that
    \begin{align*}
      \left((a^n + b^n)^{1/n}\right)_n \rightarrow \max\{a,b\}
    \end{align*}
    \tcblower
    Suppose $a = b$. Then,
    \begin{align*}
      \left(a^n + b^n\right)^{1/n} &= (2\cdot a^n)^{1/n}\\
                                   &= 2^{1/n}a\\
                                   &\rightarrow a \tag*{sequence of roots converges to $1$}\\
                                   &= \max\{a,b\}
    \end{align*}
    Otherwise, without loss of generality, let $a > b$. Then,
    \begin{align*}
      b^n < a^n &< a^n + b^n < 2\cdot a^n\\
      b < a &< \left(a^n + b^n\right)^{1/n} < 2^{1/n} a\\
            &\rightarrow\\
      b < a &< \left(a^n + b^n\right)^{1/n} < a \tag*{sequence of roots converges to $1$}
    \end{align*}
    So, by the squeeze theorem, $\left(a^n + b^n\right)^{1/n}\rightarrow a = \max\{a,b\}$.
  \end{problem}
  \begin{problem}{Problem 9}
    Let $(x_n)_n$ be a sequence of positive terms such that
    \begin{align*}
      \left(x_n^{1/n}\right)_n \rightarrow L < 1
    \end{align*}
    Prove that $(x_n)_n \rightarrow 0$. If $L = 1$, can we make any conclusion? What about $L > 1$?
    \tcblower
    Let $\rho = L + \frac{1-L}{2}$, and $\varepsilon = \rho - L = \frac{1-L}{2}$.\\

    Since $\left(x_n^{1/n}\right)_{n}$ converges, we know that
    \begin{align*}
      \left|(x_n)^{1/n} - L\right| &< \varepsilon \\
      \left(x_n\right)^{1/n} &< \rho\\
      x_n &< \rho^{n}
    \end{align*}
    Since $\rho < 1$, and as $n\rightarrow\infty$, $\rho^{n} \rightarrow 0$, therefore we know $\left(x_n\right)_n\rightarrow 0$.\\
    \vspace{4pt}
    \rule{\textwidth}{0.4pt}
    \vspace{4pt}
    We can't make any conclusions if $L = 1$, and if $L > 1$, we can assume that $(x_n)_n$ diverges, as we showed in the previous case with the ratio test.
  \end{problem}
\end{document}
