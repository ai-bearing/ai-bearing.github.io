\documentclass[12pt]{extarticle}
\title{}
\author{Avinash Iyer}
\date{}
\usepackage[shortlabels]{enumitem}

%font setup
%
%\usepackage{newpxtext,eulerpx}

%paper setup
\usepackage{geometry}
\geometry{letterpaper, portrait, margin=1in}
\usepackage{fancyhdr}

%symbols
\usepackage{amsmath}
\usepackage{amssymb}
\usepackage{mathtools}
\usepackage{hyperref}
\usepackage{gensymb}
\usepackage{multirow,array}

\usepackage[T1]{fontenc}
\usepackage[utf8]{inputenc}

%chemistry stuff
\usepackage[version=4]{mhchem}
\usepackage{chemfig}

%plotting
\usepackage{pgfplots}
\usepackage{tikz}
\tikzset{middleweight/.style={pos = 0.5, fill=white}}
\tikzset{weight/.style={pos = 0.5, fill = white}}
\tikzset{lateweight/.style={pos = 0.75, fill = white}}
\tikzset{earlyweight/.style={pos = 0.25, fill=white}}

%\usepackage{natbib}

%graphics stuff
\usepackage{graphicx}
\graphicspath{ {./images/} }

%code stuff
%when using minted, make sure to add the -shell-escape flag
%you can use lstlisting if you don't want to use minted
%\usepackage{minted}
%\usemintedstyle{pastie}
%\newminted[javacode]{java}{frame=lines,framesep=2mm,linenos=true,fontsize=\footnotesize,tabsize=3,autogobble,}
%\newminted[cppcode]{cpp}{frame=lines,framesep=2mm,linenos=true,fontsize=\footnotesize,tabsize=3,autogobble,}

%\usepackage{listings}
%\usepackage{color}
%\definecolor{dkgreen}{rgb}{0,0.6,0}
%\definecolor{gray}{rgb}{0.5,0.5,0.5}
%\definecolor{mauve}{rgb}{0.58,0,0.82}
%
%\lstset{frame=tb,
%	language=Java,
%	aboveskip=3mm,
%	belowskip=3mm,
%	showstringspaces=false,
%	columns=flexible,
%	basicstyle={\small\ttfamily},
%	numbers=none,
%	numberstyle=\tiny\color{gray},
%	keywordstyle=\color{blue},
%	commentstyle=\color{dkgreen},
%	stringstyle=\color{mauve},
%	breaklines=true,
%	breakatwhitespace=true,
%	tabsize=3
%}
% text + color boxes
\usepackage[most]{tcolorbox}
\tcbuselibrary{breakable}
\newtcolorbox{problem}[1]{colback = white, title = {#1}, breakable}
\newtcolorbox{solution}{colback = white, colframe = black!75!white, title = Solution, breakable}
%including PDFs
%\usepackage{pdfpages}
\setlength{\parindent}{0pt}

\pagestyle{fancy}
\fancyhf{}
\rhead{Avinash Iyer}
\lhead{Properties of the positive integers}
\newcommand{\card}{\text{card}}
\newcommand{\ran}{\text{ran}}
\newcommand{\N}{\mathbb{N}}
\newcommand{\Q}{\mathbb{Q}}
\newcommand{\Z}{\mathbb{Z}}
\newcommand{\R}{\mathbb{R}}
\begin{document}
  \begin{problem}{Properties of $\Z^+$}
    \begin{problem}{(i)}
      \[
        m,n\in\Z^+ \Rightarrow m+n\in\Z^+,~m\cdot n\in\Z^+
      \] 
      \tcblower
      Since $0\leq_a m$ and $0\leq_a n$, it must be the case that $0\leq_a m+n$, as $n$ cannot be less than zero. Similarly, $m\cdot n = \underbrace{m+m+\dots+m}_{\text{$n$ times}}$, meaning $0\leq_a m\cdot n$, so $m\cdot n$ and $m+n$ are both in $\Z^+$
    \end{problem}
    \begin{problem}{(ii)}
      \[
        m\in \Z \Rightarrow m\in \Z^+~\text{or}~-m\in\Z^+
      \] 
      \tcblower
      Let $m\in \Z$. Then, $0\leq_a m$ or $m\leq_a 0$. In the first case, $m\in \Z^+$, and in the second case, we know that $-m \prescript{}{a}\geq 0$, so $-m\in \Z^+$. 
    \end{problem}
    \begin{problem}{(iii)}
      \[
        m,-m\in Z^+ \Rightarrow m=0
      \] 
      \tcblower
      If $m\in \Z^+$, then $0\leq_a m$, and if $-m\in \Z^+$, $0\leq_a -m$, or $m\leq_a 0$. Therefore, $m=0$.
    \end{problem}
    \begin{problem}{(iv)}
      \[
        m\leq_a n \Leftrightarrow n-m\in \Z^+
      \] 
      \tcblower
      \begin{description}[font=\normalfont]
        \item[$(\Rightarrow)$] Let $m\leq_a n$. Then, $m-m \leq_a n-m$, so $0\leq_a n-m$, so $n-m\in \Z^+$.
        \item[$(\Leftarrow)$] Let $n-m\in \Z^+$. Then, $0\leq_a n-m$, so $m\leq_a n$.
      \end{description}
    \end{problem}
  \end{problem}
\end{document}
