\documentclass[8pt]{extarticle}
\title{}
\author{Avinash Iyer}
\date{}

%font setup
%
%\usepackage[math]{anttor}

%paper setup
\usepackage{geometry}
\geometry{letterpaper, portrait, margin=1in}
\usepackage{fancyhdr}

%symbols
\usepackage{amsmath}
\usepackage{amssymb}
\usepackage{hyperref}
\usepackage{gensymb}

\usepackage[T1]{fontenc}
\usepackage[utf8]{inputenc}

%chemistry stuff
\usepackage[version=4]{mhchem}
\usepackage{chemfig}

%plotting
\usepackage{pgfplots}
\usepackage{tikz}

%\usepackage{natbib}

%graphics stuff
\usepackage{graphicx}
\graphicspath{ {./images/} }

%a useful command
\newcommand{\plain}[1]{\textrm{#1}}

%code stuff
%when using minted, make sure to add the -shell-escape flag
%you can use lstlisting if you don't want to use minted
%\usepackage{minted}
%\usemintedstyle{pastie}
%\newminted[javacode]{java}{frame=lines,framesep=2mm,linenos=true,fontsize=\footnotesize,tabsize=3,autogobble,}
%\newminted[cppcode]{cpp}{frame=lines,framesep=2mm,linenos=true,fontsize=\footnotesize,tabsize=3,autogobble,}

\usepackage{listings}
\usepackage{color}
\definecolor{dkgreen}{rgb}{0,0.6,0}
\definecolor{gray}{rgb}{0.5,0.5,0.5}
\definecolor{mauve}{rgb}{0.58,0,0.82}

\lstset{frame=tb,
	language=Java,
	aboveskip=3mm,
	belowskip=3mm,
	showstringspaces=false,
	columns=flexible,
	basicstyle={\small\ttfamily},
	numbers=none,
	numberstyle=\tiny\color{gray},
	keywordstyle=\color{blue},
	commentstyle=\color{dkgreen},
	stringstyle=\color{mauve},
	breaklines=true,
	breakatwhitespace=true,
	tabsize=3
}
% text + color boxes
\usepackage{tcolorbox}
\newtcolorbox{mathbox}[1]{title = {#1}}

\pagestyle{fancy}
\fancyhf{}
\rhead{Avinash Iyer}
\lhead{Calculating GDP: Notes}
\begin{document}{
  \begin{center}
    \begin{tabular}{|c|c|c|c|c|}
      \hline
      Year & $P$ Apples & $Q$ Apples & $P$ Computers & $Q$ Computers\\
      \hline
      2026 & 2 & 500 & 1000 & 5 \\
      2027 & 3 & 550 & 1000 & 6\\
      \hline
    \end{tabular}
\end{center} 
\begin{itemize}
    \item Calculating Nominal GDP: Take prices of goods in that year and quantities produced in that year, and add up across the economy.
    \item For example, in 2026, we can find the following:
      \[\textrm{NGDP}_{2026} = (2)(500) + (1000)(5) = 6000\]
    \item Similarly for 2027, we can find the nominal GDP as follows:
      \[
        \textrm{NGDP}_{2027} = (3)(550) + (1000)(6) = 7100
      \]
    \item We can also calculate Real GDP by using different base years' prices:
      \[
      \textrm{RGDP}_{2026} = 6500
      \]
    \[
      \textrm{RGDP}_{2027} = 7650
    \]
    \item To calculate percentage change in RGDP, we set a base year's prices and find the change in output.
    \item Calculating the percentage change in RGDP using the initial year as the base price, we get the \textit{Laspeyres Index}
    \item Meanwhile, if we used the final year as the base price, we would get \textit{Paasche Index}
    \item The \textit{Fisher Index} is the arithmetic mean of the Laspeyres Index and the Paasche Index.
    \item We use chain weighting approach to find real GDP by taking the Fisher index, $f$, and solving for $x$ in the following equation:
      \[(1+f)(x) = \textrm{NGDP}_{\textrm{current}}\]
    \item We can also calculate inflation as follows:
        \begin{align*}
          \textrm{NGDP} &= P\times \textrm{RGDP} \\
          \%\Delta\textrm{NGDP} &\approx \%\Delta P \times \%\Delta\textrm{RGDP}
        \end{align*}
    \item The change in the price level, $\%\Delta P$, is the \textit{GDP Deflator}, a measure of inflation.
    \item $\%\Delta P$ can also be calculated as $\%\Delta \textrm{NGDP} - \%\Delta\textrm{RGDP}$, where RGDP is calculated using the various indices.
  \end{itemize}
}\end{document}
