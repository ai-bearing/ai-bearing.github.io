\documentclass[8pt]{extarticle}
\title{}
\author{Avinash Iyer}
\date{}

%font setup
%
%\usepackage[math]{anttor}

%paper setup
\usepackage{geometry}
\geometry{letterpaper, portrait, margin=1in}
\usepackage{fancyhdr}
\usepackage{multirow}
%symbols
\usepackage{amsmath}
\usepackage{amssymb}
\usepackage{hyperref}
\usepackage{gensymb}
\usepackage{tabularx}
\usepackage{pdflscape}
\usepackage{rotating}

\usepackage[T1]{fontenc}
\usepackage[utf8]{inputenc}

%chemistry stuff
\usepackage[version=4]{mhchem}
\usepackage{chemfig}

%plotting
\usepackage{pgfplots}
\usepackage{tikz}
\usepackage{array}

%\usepackage{natbib}

%graphics stuff
\usepackage{graphicx}
\graphicspath{ {./images/} }

%a useful command
\newcommand{\plain}[1]{\textrm{#1}}

%code stuff
%when using minted, make sure to add the -shell-escape flag
%you can use lstlisting if you don't want to use minted
%\usepackage{minted}
%\usemintedstyle{pastie}
%\newminted[javacode]{java}{frame=lines,framesep=2mm,linenos=true,fontsize=\footnotesize,tabsize=3,autogobble,}
%\newminted[cppcode]{cpp}{frame=lines,framesep=2mm,linenos=true,fontsize=\footnotesize,tabsize=3,autogobble,}

\usepackage{listings}
\usepackage{color}
\definecolor{dkgreen}{rgb}{0,0.6,0}
\definecolor{gray}{rgb}{0.5,0.5,0.5}
\definecolor{mauve}{rgb}{0.58,0,0.82}

\lstset{frame=tb,
	language=Java,
	aboveskip=3mm,
	belowskip=3mm,
	showstringspaces=false,
	columns=flexible,
	basicstyle={\small\ttfamily},
	numbers=none,
	numberstyle=\tiny\color{gray},
	keywordstyle=\color{blue},
	commentstyle=\color{dkgreen},
	stringstyle=\color{mauve},
	breaklines=true,
	breakatwhitespace=true,
	tabsize=3
}
% text + color boxes
\usepackage{tcolorbox}
\newtcolorbox{mathbox}[1]{colback = white, title = {#1}}

\pagestyle{fancy}
\fancyhf{}
\rhead{Avinash Iyer}
\lhead{Problem Set 2}
\begin{document}
\begin{mathbox}{Returns to Scale in Production}
    Do the following production functions exhibit increasing, constant, or decreasing returns to scale in inputs $K$ and $L$.
      \begin{itemize}
        \item $Y(K,L) = K^{1/2}L^{1/2}$
        \item $Y(K,L) = K^{1/3}L^{1/2}$
        \item $Y(K,L) = K + K^{1/3}L^{1/3}$
        \item $Y(K,L) = K^{1/3}L^{2/3} - \overline{A}$
      \end{itemize}
  \end{mathbox}
  \begin{itemize}
    \item $Y(2K, 2L) = (2K)^{1/2}(2L)^{1/2} = 2(K^{1/2}L^{1/2}) = 2Y(K,L)$ --- \textbf{constant returns to scale}
    \item $Y(2K, 2L) = (2K)^{1/3}(2L)^{1/2} < 2Y(K,L)$ --- \textbf{decreasing returns to scale}
    \item  $Y(2K, 2L) = 2K = (2K)^{1/3}(2L)^{1/3} < 2Y(K,L)$ --- \textbf{decreasing returns to scale}
    \item  $Y(2K,2L) = (2K)^{1/3}(2L)^{2/3} - \overline{A} < 2Y(K,L)$ --- \textbf{decreasing returns to scale}
  \end{itemize}
\begin{mathbox}{The Black Death}
      In the middle of the fourteenth century, an epidemic known as the Black Death killed about a third of Europe's population, about 34 million people. While this was an enormous tragedy, the macroeconomic consequences might surprise you: over the next century, wages are estimated to have been higher than before the Black Death.      
        \begin{itemize}
          \item Use the production model to explain why wages might have been higher.
          \item Can you attach a number to your explanation? In the model, how much would wages rise if a third of the population died from disease?
        \end{itemize}
    \end{mathbox}
    \noindent The production model would explain why wages went up because, after a large level of dying, the marginal product of labor, expressed as $MP_L = \frac{2}{3}\frac{Y}{L}$ went up as capital stayed constant, workers ``left,'' and output stayed constant.\\
    
    \noindent In this model, if a third of the population died of disease, then wages would have gone up by $50\%$ (which is equal to $1/(1-1/3)$), assuming output stayed constant.
\begin{mathbox}{A Variant of the Production Model}
    Suppose the production function at the core of our model is $Y = \overline{A}K^{3/4}L^{1/4}$.
      \begin{itemize}
        \item Create a new version of Table 4.1 from the textbook for the new model. What are the five equations and five unknowns?
        \item Now solve these equations to get the solution to the model. Put your solutions in the same form as Table 4.2 from the textbook.
        \item What is the solution for the equilibrium level of output per person?
      \end{itemize}
  \end{mathbox}
    \begin{center}
      \renewcommand{\arraystretch}{1.5}
      \begin{tabular}{c|c}
        \multicolumn{2}{c}{The Production Model, 5 Equations and 5 Unknowns} \\
        \hline
        Unknowns & $Y,K,L,r,w$ \\
        Production Function & $Y = \overline{A}K^{3/4}L^{1/4}$ \\
        Rule for hiring capital & $r = \frac{3}{4}\frac{Y}{K}$ \\
        Rule for hiring labor & $w = \frac{1}{4}\frac{Y}{L}$ \\
        Demand = Supply for Capital & $K = \overline{K}$ \\
        Demand = Supply for Labor & $L = \overline{L}$ \\
        Exogenous Variables & $\overline{A},\overline{K},\overline{L}$
      \end{tabular} \\

      \begin{tabular}{c|c}
        \multicolumn{2}{c}{The Solution to the Production Model}\\
        \hline
        Capital & $K^{*} = \overline{K}$\\
        Labor & $L^{*} = \overline{L}$ \\
        Rental rate & $r^{*} = \frac{3}{4}\frac{Y^{*}}{K^{*}} = \frac{3}{4}\overline{A}\left(\frac{\overline{L}}{\overline{K}}\right)^{1/4}$ \\
        Wage rate & $w^{*} = \frac{1}{4}\frac{Y^{*}}{L^{*}} = \frac{1}{4}\overline{A}\left(\frac{\overline{K}}{\overline{L}}\right)^{3/4}$ \\
        Output & $Y^{*} = \overline{A} \overline{K}^{3/4} \overline{L}^{1/4}$
      \end{tabular}
    \end{center}
    \[
      Y^{*}_{\tiny \textrm{per capita}} = \frac{\overline{A}\overline{K}^{3/4}\overline{L}^{1/4}}{\overline{L}} = \overline{A}\left(\frac{\overline{K}}{\overline{L}}\right)^{3/4}
    \]
\begin{mathbox}{Empirical Fit of the Production Model}
  The table below reports GDP per capita and capital per person in 2017 for ten countries. Fill out the missing data.      
\end{mathbox}
  \begin{center}
    \small
    \renewcommand{\arraystretch}{1.5}
    \begin{tabular}{c|cccccc}
      TFP Calculations & \multicolumn{2}{c}{In 2011 Dollars} & \multicolumn{4}{c}{Relative to the US} \\ \hline
      Country & Capital per person & GDP Per Capita & Capital per person & GDP Per Capita & Predicted $Y^{*}$ & Implied TFP \\ 
        \textbf{United States} & 175075 & 54087 & 1 & 1 & 1 & 1 \\ 
        \textbf{Canada} & 153390 & 42540 & 0.876 & 0.787 & 0.957 & 0.822 \\ 
        \textbf{France} & 136004 & 38841 & 0.887 & 0.718 & 0.961 & 0.747 \\ 
        \textbf{Hong Kong} & 154766 & 40603 & 1.138 & 0.751 & 1.044 & 0.719 \\ 
        \textbf{South Korea} & 142891 & 36521 & 0.923 & 0.675 & 0.974 & 0.693 \\ 
        \textbf{Indonesia} & 26620 & 10598 & 0.186 & 0.196 & 0.571 & 0.343 \\ 
        \textbf{Argentina} & 31589 & 16469 & 1.187 & 0.304 & 1.059 & 0.287 \\ 
        \textbf{Mexico} & 41866 & 17070 & 1.325 & 0.316 & 1.098 & 0.288 \\ 
        \textbf{Kenya} & 4179 & 3069 & 0.1 & 0.057 & 0.464 & 0.123 \\ 
        \textbf{Ethiopia} & 2938 & 1596 & 0.703 & 0.03 & 0.889 & 0.034 \\ \hline
    \end{tabular} \\
    \begin{tabular}{c|ccc}
        TFP Calculations & \multicolumn{3}{c}{Inverse Values (relative to the US)} \\ \hline
        Country & GDP per capita & Predicted Y* & Implied TFP \\ 
        \textbf{United States} & 1 & 1 & 1 \\ 
        \textbf{Canada} & 1.271 & 1.045 & 1.217 \\ 
        \textbf{France} & 1.393 & 1.041 & 1.339 \\ 
        \textbf{Hong Kong} & 1.332 & 0.958 & 1.391 \\ 
        \textbf{South Korea} & 1.481 & 1.027 & 1.443 \\ 
        \textbf{Indonesia} & 5.102 & 1.751 & 2.915 \\ 
        \textbf{Argentina} & 3.289 & 0.944 & 3.484 \\ 
        \textbf{Mexico} & 3.165 & 0.911 & 3.472 \\ 
        \textbf{Kenya} & 17.544 & 2.155 & 8.13 \\ 
        \textbf{Ethiopia} & 33.333 & 1.125 & 29.412 \\ \hline
    \end{tabular}
  \end{center}
We find that roughly $20\%$ of the difference between the United States and Kenya is due to differences in capital per person, and about $80\%$ of the difference is due to differences in TFP.
\begin{mathbox}{True/False}
    Decide whether the following statement is true, false, or uncertain, and why.
      \begin{quote}
        Differences in the quality of institutions are the sole cause of differences in per capita income levels across countries.
      \end{quote}
  \end{mathbox}
\noindent This statement is \textbf{false}, because there are other major factors determining differences in per capita income, such as technological development, human capital, natural resources, and geography.
\end{document}
