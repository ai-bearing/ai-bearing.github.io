\documentclass[10pt]{extarticle}
\title{}
\author{Avinash Iyer}
\date{}
\usepackage[shortlabels]{enumitem}

%font setup
%
\usepackage{newpxtext,eulerpx}

%paper setup
\usepackage{geometry}
\geometry{letterpaper, portrait, margin=1in}
\usepackage{fancyhdr}

%symbols
\usepackage{amsmath}
\usepackage{mathtools}
\usepackage{amssymb}
\usepackage{hyperref}
\usepackage{gensymb}

\usepackage[T1]{fontenc}
\usepackage[utf8]{inputenc}

%chemistry stuff
\usepackage[version=4]{mhchem}
\usepackage{chemfig}

%plotting
\usepackage{pgfplots}
\usepackage{tikz}
\tikzset{middleweight/.style={pos = 0.5, fill=white}}
\tikzset{weight/.style={pos = 0.5, fill = white}}
\tikzset{lateweight/.style={pos = 0.75, fill = white}}
\tikzset{earlyweight/.style={pos = 0.25, fill=white}}

%\usepackage{natbib}

%graphics stuff
\usepackage{graphicx}
\graphicspath{ {./images/} }

%code stuff
%when using minted, make sure to add the -shell-escape flag
%you can use lstlisting if you don't want to use minted
%\usepackage{minted}
%\usemintedstyle{pastie}
%\newminted[javacode]{java}{frame=lines,framesep=2mm,linenos=true,fontsize=\footnotesize,tabsize=3,autogobble,}
%\newminted[cppcode]{cpp}{frame=lines,framesep=2mm,linenos=true,fontsize=\footnotesize,tabsize=3,autogobble,}

\usepackage{listings}
\usepackage{color}
\definecolor{dkgreen}{rgb}{0,0.6,0}
\definecolor{gray}{rgb}{0.5,0.5,0.5}
\definecolor{mauve}{rgb}{0.58,0,0.82}

\lstset{frame=tb,
	language=Java,
	aboveskip=3mm,
	belowskip=3mm,
	showstringspaces=false,
	columns=flexible,
	basicstyle={\small\ttfamily},
	numbers=none,
	numberstyle=\tiny\color{gray},
	keywordstyle=\color{blue},
	commentstyle=\color{dkgreen},
	stringstyle=\color{mauve},
	breaklines=true,
	breakatwhitespace=true,
	tabsize=3
}
% text + color boxes
\usepackage{tcolorbox}
\tcbuselibrary{breakable}
\newtcolorbox{problem}[1]{colback = white, title = {#1}, breakable}
\newtcolorbox{solution}{colback = white, colframe = black!75!white, title = Solution, breakable}
%including PDFs
\usepackage{pdfpages}
\setlength{\parindent}{0pt}

\pagestyle{fancy}
\fancyhf{}
\rhead{Avinash Iyer}
\lhead{Problem Set 7}
\begin{document}{
  \begin{problem}{Raising the Nominal Interest Rate}
    Suppose the Fed announces today that it is raising the federal funds rate by 50 basis points. Using the IS-MP diagram, explain what happens to economic activity in the short run. What is the economics underlying this response in the economy?
    \tcblower
      \begin{center}
        \begin{tikzpicture}[scale = 0.75]
          \draw (0,10) -- (0,0) -- (10,0);
          \node[anchor = west] at (10,0){$\tilde{Y}$};
          \node[anchor = east] at (0,10) {$R$};
          \draw[thick] (1,10) -- (10,1);
          \node[anchor = west] at (10,1) {IS};
          \draw[thick] (0,5) -- (10,5);
          \node[anchor = west] at (10,5) {MP$_1$};
          \draw[blue!50!black, thick] (0,6) -- (10,6);
          \node[anchor = west] at (10,6) {MP$_2$};
          \draw[dashed, thick] (6,5) -- (6,0);
          \node[anchor = south west] at (6,0) {\tiny $\tilde{Y}_1 = 0$};
          \draw[dashed, thick] (5,6) -- (5,0);
          \node[anchor = south east] at (5,0) {\tiny $\tilde{Y}_2 < 0$};
        \end{tikzpicture}
      \end{center}
      In the short run, the output gap shifts in the negative direction. Since we are assuming that $\tilde{Y}_1$ is zero, we get that the output gap goes negative, because the higher interest rate means companies invest less and consumers buy less.
  \end{problem}
  \begin{problem}{Your Day as Chair of the Federal Reserve}
    With the goal of stabilizing output, explain how and why you would change the interest rate in response to these various shocks. Show the effects on the economy with an IS-MP diagram.
    \begin{enumerate}[(a)]
      \item Consumers become pessimistic about the state of the economy and future productivity.
      \item A booming economy in Europe this year leads to an unexpected increase in demand by European consumers of US goods.
      \item Americans develop an infatuation with all things made in New Zealand.
    \end{enumerate}
  \end{problem}

}\end{document}
