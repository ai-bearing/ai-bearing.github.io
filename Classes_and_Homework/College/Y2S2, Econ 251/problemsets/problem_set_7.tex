\documentclass[10pt]{extarticle}
\title{}
\author{Avinash Iyer}
\date{}
\usepackage[shortlabels]{enumitem}

%font setup
%
\usepackage{newpxtext,eulerpx}

%paper setup
\usepackage{geometry}
\geometry{letterpaper, portrait, margin=1in}
\usepackage{fancyhdr}

%symbols
\usepackage{amsmath}
\usepackage{mathtools}
\usepackage{amssymb}
\usepackage{hyperref}
\usepackage{gensymb}

\usepackage[T1]{fontenc}
\usepackage[utf8]{inputenc}

%chemistry stuff
\usepackage[version=4]{mhchem}
\usepackage{chemfig}

%plotting
\usepackage{pgfplots}
\usepackage{tikz}
\tikzset{middleweight/.style={pos = 0.5, fill=white}}
\tikzset{weight/.style={pos = 0.5, fill = white}}
\tikzset{lateweight/.style={pos = 0.75, fill = white}}
\tikzset{earlyweight/.style={pos = 0.25, fill=white}}

%\usepackage{natbib}

%graphics stuff
\usepackage{graphicx}
\graphicspath{ {./images/} }

%code stuff
%when using minted, make sure to add the -shell-escape flag
%you can use lstlisting if you don't want to use minted
%\usepackage{minted}
%\usemintedstyle{pastie}
%\newminted[javacode]{java}{frame=lines,framesep=2mm,linenos=true,fontsize=\footnotesize,tabsize=3,autogobble,}
%\newminted[cppcode]{cpp}{frame=lines,framesep=2mm,linenos=true,fontsize=\footnotesize,tabsize=3,autogobble,}

\usepackage{listings}
\usepackage{color}
\definecolor{dkgreen}{rgb}{0,0.6,0}
\definecolor{gray}{rgb}{0.5,0.5,0.5}
\definecolor{mauve}{rgb}{0.58,0,0.82}

\lstset{frame=tb,
	language=Java,
	aboveskip=3mm,
	belowskip=3mm,
	showstringspaces=false,
	columns=flexible,
	basicstyle={\small\ttfamily},
	numbers=none,
	numberstyle=\tiny\color{gray},
	keywordstyle=\color{blue},
	commentstyle=\color{dkgreen},
	stringstyle=\color{mauve},
	breaklines=true,
	breakatwhitespace=true,
	tabsize=3
}
% text + color boxes
\usepackage{tcolorbox}
\tcbuselibrary{breakable}
\newtcolorbox{problem}[1]{colback = white, title = {#1}, breakable}
\newtcolorbox{solution}{colback = white, colframe = black!75!white, title = Solution, breakable}
%including PDFs
\usepackage{pdfpages}
\setlength{\parindent}{0pt}

\pagestyle{fancy}
\fancyhf{}
\rhead{Avinash Iyer}
\lhead{Problem Set 7}
\begin{document}
  \begin{problem}{Raising the Nominal Interest Rate}
    Suppose the Fed announces today that it is raising the federal funds rate by 50 basis points. Using the IS-MP diagram, explain what happens to economic activity in the short run. What is the economics underlying this response in the economy?
    \tcblower
      \begin{center}
        \begin{tikzpicture}[scale = 0.75]
          \draw (0,10) -- (0,0) -- (10,0);
          \node[anchor = west] at (10,0){$\tilde{Y}$};
          \node[anchor = east] at (0,10) {$R$};
          \draw[thick] (1,10) -- (10,1);
          \node[anchor = west] at (10,1) {IS};
          \draw[thick] (0,5) -- (10,5);
          \node[anchor = west] at (10,5) {MP$_1$};
          \draw[blue!50!black, thick] (0,6) -- (10,6);
          \node[anchor = west] at (10,6) {MP$_2$};
          \draw[dashed, thick] (6,5) -- (6,0);
          \node[anchor = south west] at (6,0) {\small $\tilde{Y}_1 = 0$};
          \draw[dashed, thick] (5,6) -- (5,0);
          \node[anchor = south east] at (5,0) {\small $\tilde{Y}_2 < 0$};
          \node[anchor = east] at (0,5) {$\overline{r}$};
        \end{tikzpicture}
      \end{center}
      In the short run, the output gap shifts in the negative direction. Since we are assuming that $\tilde{Y}_1$ is zero, we get that the output gap goes negative, because the higher interest rate means companies invest less and consumers buy less.
  \end{problem}
  \begin{problem}{Your Day as Chair of the Federal Reserve}
    With the goal of stabilizing output, explain how and why you would change the interest rate in response to these various shocks. Show the effects on the economy with an IS-MP diagram.
    \begin{enumerate}[(a)]
      \item Consumers become pessimistic about the state of the economy and future productivity.
      \item A booming economy in Europe this year leads to an unexpected increase in demand by European consumers of US goods.
      \item Americans develop an infatuation with all things made in New Zealand.
    \end{enumerate}
    \tcblower
    \begin{tcolorbox}[colback = white, title = (a), breakable]
      Consumer pessimism about the state of the economy causes them to cut back on spending, which leads to the IS curve shifting inward. This is counteracted by interest rates shifting downward to ensure that the economy is at potential output.
      \begin{center}
        \begin{tikzpicture}[scale = 0.75]
          % \draw[thin, gray] (0,0) grid (10,10);
          \draw (0,10) -- (0,0) -- (10,0);
          \node[anchor = west] at (10,0){$\tilde{Y}$};
          \node[anchor = east] at (0,10) {$R$};
          \draw[thick] (1,10) -- (10,1);
          \draw[thick] (0,6) -- (10,6);
          \node[anchor = west] at (10,6) {MP$_1$};
          \node[anchor = east] at (0,6) {$R_1 = \overline{r}$};
          \draw[dashed, thick] (5,0) -- (5,6);
          \node[anchor = north] at (5,0) {\small $\tilde{Y}_0 = 0 = \tilde{Y}_2$};
          \node[anchor = west] at (10,1) {IS$_1$};
          \draw[blue, thick] (1,7) -- (7,1);
          \node[anchor = west] at (7,1) {IS$_2$};
          \draw[thick,red] (0,3) -- (10,3);
          \node[anchor = east] at (0,3) {$R_2$};
          \node[anchor = west] at (10,3) {MP$_2$};
          \draw[dashed, thick, blue] (2,0) -- (2,6);
          \node[anchor = north] at (2,0) {\small $\tilde{Y}_1 < 0$};
          \draw[-stealth] (9.5,6) -- (9.5,3);
        \end{tikzpicture}
      \end{center}
    \end{tcolorbox}
    \begin{tcolorbox}[colback = white, title = (b), breakable]
      An increase in exports leads the IS curve to shift outward, meaning that the central bank raises rates to ensure that the economy doesn't run too hot
      \begin{center}
        \begin{tikzpicture}[scale = 0.75]
          % \draw[thin, gray] (0,0) grid (10,10);
          \draw (0,10) -- (0,0) -- (10,0);
          \node[anchor = west] at (10,0){$\tilde{Y}$};
          \node[anchor = east] at (0,10) {$R$};
          \draw[thick] (1,10) -- (10,1);
          \node[anchor = west] at (10,1) {IS$_1$};
          \draw[blue, thick] (4,10) -- (10,4);
          \node[anchor = west] at (10,4) {IS$_2$};
          \draw[thick] (0,5) -- (10,5);
          \node[anchor = west] at (10,5) {MP$_1$};
          \draw[dashed, thick] (6,0) -- (6,8);
          \node[anchor = north] at (6,0) {\small $\tilde{Y}_0 = 0 = \tilde{Y}_2$};
          \draw[thick, red] (0,8) -- (10,8);
          \node[anchor = west] at (10,8) {MP$_{2}$};
          \draw[dashed, thick, blue] (9,0) -- (9,5);
          \node[anchor = north] at (9,0) {\small $\tilde{Y}_2 > 0$};
          \draw[-stealth] (9.5,8) -- (9.5,5);
          \node[anchor = east] at (0,5) {$R_1 = \overline{r}$};
          \node[anchor = east] at (0,8) {$R_2$};
        \end{tikzpicture}
      \end{center}
    \end{tcolorbox}
    \begin{tcolorbox}[colback = white, title = (c), breakable]
      The increase in imports from New Zealand reduces output within the country as imports crowd out domestic consumption, so the central bank lowers rates to stimulate the domestic economy.
      \begin{center}
        \begin{tikzpicture}[scale = 0.75]
          % \draw[thin, gray] (0,0) grid (10,10);
          \draw (0,10) -- (0,0) -- (10,0);
          \node[anchor = west] at (10,0){$\tilde{Y}$};
          \node[anchor = east] at (0,10) {$R$};
          \draw[thick] (1,10) -- (10,1);
          \draw[thick] (0,6) -- (10,6);
          \node[anchor = west] at (10,6) {MP$_1$};
          \node[anchor = east] at (0,6) {$R_1 = \overline{r}$};
          \draw[dashed, thick] (5,0) -- (5,6);
          \node[anchor = north] at (5,0) {\small $\tilde{Y}_0 = 0 = \tilde{Y}_2$};
          \node[anchor = west] at (10,1) {IS$_1$};
          \draw[blue, thick] (1,7) -- (7,1);
          \node[anchor = west] at (7,1) {IS$_2$};
          \draw[thick,red] (0,3) -- (10,3);
          \node[anchor = east] at (0,3) {$R_2$};
          \node[anchor = west] at (10,3) {MP$_2$};
          \draw[dashed, thick, blue] (2,0) -- (2,6);
          \node[anchor = north] at (2,0) {\small $\tilde{Y}_1 < 0$};
          \draw[-stealth] (9.5,6) -- (9.5,3);
        \end{tikzpicture}
      \end{center}
    \end{tcolorbox}
  \end{problem}
  \begin{problem}{The Summary Diagram}
    The end of section 12.4 in the textbook contains a summary diagram (recreated below) of the short-run model. Explain the economic reasoning that underlies each step in this summary. How does this summary diagram illustrate the essence of the Volcker disinflation?
    \begin{center}
      \underline{Summary: The Short-Run Model}\\
      \vspace{10pt}
      MP Curve: $\uparrow i_t \Rightarrow \uparrow R_t$\\
      \vspace{10pt}
      IS Curve: $\uparrow R_t \Rightarrow \downarrow \tilde{Y}_t$\\
      \vspace{10pt}
      Phillips Curve: $\downarrow \tilde{Y}_t \Rightarrow \downarrow \pi_t$
    \end{center}
    \tcblower
    \begin{description}[font = \normalfont\scshape]
      \item[MP Curve] In the MP curve, we raise the nominal interest rate, which raises the real interest rate.
      \item[IS Curve] When the real interest rate increases, investment and output slow as firms and households face higher borrowing costs.
      \item[Phillips Curve] As output slows, inflation falls because there is less pressure for higher wages and employment as a result of lower output.
    \end{description}
  \end{problem}
\end{document}
