\documentclass[10pt]{extarticle}
\title{}
\author{Avinash Iyer}
\date{}
\usepackage[shortlabels]{enumitem}

%font setup
%
\usepackage{newpxtext,eulerpx}

%paper setup
\usepackage{geometry}
\geometry{letterpaper, portrait, margin=1in}
\usepackage{fancyhdr}

%symbols
\usepackage{amsmath}
\usepackage{mathtools}
\usepackage{amssymb}
\usepackage{hyperref}
\usepackage{gensymb}

\usepackage[T1]{fontenc}
\usepackage[utf8]{inputenc}

%chemistry stuff
\usepackage[version=4]{mhchem}
\usepackage{chemfig}

%plotting
\usepackage{pgfplots}
\usepackage{tikz}
\tikzset{middleweight/.style={pos = 0.5, fill=white}}
\tikzset{weight/.style={pos = 0.5, fill = white}}
\tikzset{lateweight/.style={pos = 0.75, fill = white}}
\tikzset{earlyweight/.style={pos = 0.25, fill=white}}

%\usepackage{natbib}

%graphics stuff
\usepackage{graphicx}
\graphicspath{ {./images/} }

%code stuff
%when using minted, make sure to add the -shell-escape flag
%you can use lstlisting if you don't want to use minted
%\usepackage{minted}
%\usemintedstyle{pastie}
%\newminted[javacode]{java}{frame=lines,framesep=2mm,linenos=true,fontsize=\footnotesize,tabsize=3,autogobble,}
%\newminted[cppcode]{cpp}{frame=lines,framesep=2mm,linenos=true,fontsize=\footnotesize,tabsize=3,autogobble,}

\usepackage{listings}
\usepackage{color}
\definecolor{dkgreen}{rgb}{0,0.6,0}
\definecolor{gray}{rgb}{0.5,0.5,0.5}
\definecolor{mauve}{rgb}{0.58,0,0.82}

\lstset{frame=tb,
	language=Java,
	aboveskip=3mm,
	belowskip=3mm,
	showstringspaces=false,
	columns=flexible,
	basicstyle={\small\ttfamily},
	numbers=none,
	numberstyle=\tiny\color{gray},
	keywordstyle=\color{blue},
	commentstyle=\color{dkgreen},
	stringstyle=\color{mauve},
	breaklines=true,
	breakatwhitespace=true,
	tabsize=3
}
% text + color boxes
\usepackage{tcolorbox}
\tcbuselibrary{breakable}
\newtcolorbox{problem}[1]{colback = white, title = {#1}, breakable}
\newtcolorbox{solution}{colback = white, colframe = black!75!white, title = Solution, breakable}
%including PDFs
\usepackage{pdfpages}
\setlength{\parindent}{0pt}

\pagestyle{fancy}
\fancyhf{}
\rhead{Avinash Iyer}
\lhead{}
\begin{document}{
  \begin{problem}{Purely Statistical Evidence: The St. Louis Regression}
    Suppose we regress the change in log GDP to some measure of the stance of monetary policy such as the change in the federal funds rate.
    \[
      \Delta \ln y_t= \hat{\alpha} + \hat{\beta}\Delta X_t + \varepsilon_t,
    \]
    where $y_t$ represents GDP at time $t$, $X_t$ represents the level of the federal funds rate at $t$, and $\varepsilon_t$ is an error term.\\

    After performing the regression, we find that the coefficient on the federal funds rate term, $\hat\beta$, is statistically indistinguishable from zero. Does this finding mean that monetary policy does not matter?
    \tcblower
    The assumption that $\hat\beta \approx 0$ means that monetary policy does not matter is false. The causes of high or low interest rates in relation to output are twofold --- either the central bank has increased interest rates to counteract overheating in the economy, or the central bank's high interest rates lead to a slowdown in the economy, and similarly in the opposite direction for low interest rates. We would expect that the effect of interest rates on the economy with this simple regression would be statistically insignificant when we account for all these different effects.
  \end{problem}
  \begin{problem}{Questions about ``Does Monetary Policy Matter? A New Test in the Spirit of Friedman and Schwartz'' by Romer and Romer}
    \begin{enumerate}[(a)]
      \item \textit{The Narrative Approach}. Both Friedman and Schwartz (1963) and Romer and Romer (1989) utilize the narrative approach to deal with reverse causality concerns. How can the narrative approach overcome reverse causality concerns?
      \item \textit{The Identification of Monetary Shocks}. Friedman and Schwartz (1963) and Romer and Romer (1989) adopt different definitions of a monetary shock. What is the definition each study employs? From the perspective of identifying monetary shocks from the narrative record, which definition do you find more satisfying? Why? What are strengths and weaknesses with each definition?
      \item \textit{Hypothesis}. What is the main hypothesis of Friedman and Schwartz (1963) and Romer and Romer (1989)?
      \item \textit{Methodological Design}. What are some methodological concerns, if any, with the Friedman and Schwartz (1963) and Romer and Romer (1989) studies?
      \item \textit{Understanding Romer and Romer (1989) in an AS-AD framework}. In an AS-AD diagram, graphically depict the effects of a ``Romer and Romer'' monetary shock.
      \item \textit{Your Appraisal}. Based on the evidence presented in the paper, what is your view? Does monetary policy matter? Do you find the work of Romer and Romer (1989) convincing? Why or why not?
    \end{enumerate}
    \tcblower
    \begin{tcolorbox}[colback = white, title = (a), breakable]
      The narrative approach can overcome reverse causality concerns by showing that the historical record yields a monetary shock independent of those that are associated with recessions. Friedman and Schwartz used this identification strategy precisely because the history of monetary shocks does more justice towards the idea that monetary policy affects output, rather than simple regressions.
    \end{tcolorbox}
    \begin{tcolorbox}[colback = white, title = (b), breakable]
      Friedman and Schwartz identify monetary shocks as changes in the money supply \textit{unusual} compared to the economic conditions of the time. One of the central problems, though, is that this identification of monetary shocks is primarily up to the judgment of the duo, rather than a specific measure.\\

      Romer and Romer, on the other hand, focus entirely on whether the Federal Reserve sought to shift the aggregate demand inward, and focused their efforts on contractionary monetary policy. They use the record of FOMC policy actions to discern this aim.
    \end{tcolorbox}
    \begin{tcolorbox}[colback = white, title = (c), breakable]
      The main hypothesis of both Friedman and Schwartz and Romer and Romer is that monetary policy has an effect on output.
    \end{tcolorbox}
    \begin{tcolorbox}[colback = white, title = (d), breakable]
      The main methodological concern with Friedman and Schwartz's study is that the identification strategy relied primarily on the judgment of the duo, and was more of an open-ended strategy of identifying monetary shocks, rather than one that is very clear and well-defined.\\

      On the other hand, a potential methodological concern with Romer and Romer's identification strategy (and Friedman and Schwartz's) is that they focused entirely on negative monetary shocks, and did end up using some form of judgment to identify when the Federal Reserve was using contractionary monetary policy.
    \end{tcolorbox}
    \begin{tcolorbox}[colback = white, title = (e), breakable]
      \begin{center}
        \begin{tikzpicture}[scale=0.75]
          \draw (0,10) -- (0,0) -- (10,0);
          \node[anchor = east] at (0,10) {$\pi$};
          \node[anchor = north] at (10,0) {$\tilde{Y}$};
          \draw (1,10) -- (10,1);
          \node[anchor = west] at (10,1) {AD$_0$};
          \draw (1,1) -- (10,10);
          \node[anchor = west] at (10,10) {AS$_0$};
          \draw[dashed] (5.5,5.5) -- (0,5.5);
          \node[anchor = east] at (0,5.5) {$\overline{\pi}$};

          \draw[dashed] (5.5,5.5) -- (5.5,0);
          \node[anchor = north] at (5.5,0) {$\tilde{Y}_0 = 0$};
          \draw[blue!50!black] (1,7) -- (7,1);
          \node[anchor = west] at (7,1) {AD$_2$};

          \draw[dashed] (0,4) -- (4,4) -- (4,0);
          \node[anchor = south east] at (4,0) {$\tilde{Y}_1 < 0$};
          \node[anchor = east] at (0,4) {$\pi_1$};
        \end{tikzpicture}
      \end{center}
    \end{tcolorbox}
    \begin{tcolorbox}[colback = white, title = (f), breakable]
      Based on the evidence that Romer and Romer give, I think I'm fairly certain that monetary policy does have an effect on output in the short run.
    \end{tcolorbox}
  \end{problem}
  \begin{problem}{Questions about the ``Baby-Sitting Co-op'' by Paul Krugman}
    \begin{enumerate}[(a)]
      \item How does the tale of the baby-sitting co-op help us understand the causes of recessions and the appropriate policy response to recessions?
      \item In what sense does this story corroborate the conclusions Romer and Romer reach with their narrative, historical approach?
    \end{enumerate}
    \tcblower
    \begin{enumerate}[(a)]
      \item The tale of the baby-sitting co-op helps us understand the causes of recession by seeing the effects of a contraction of the ``money supply'' (or supply of baby-sitting scrips) on the market for babysitters. This model is a simple version of what happens during recessions when the money supply contracts and the ability for people to spend decreases, and reveals a simple solution, which is to increase the money supply when in recession so as to maintain aggregate demand.
      \item The story corroborates the historical approach of Romer and Romer since they showed that efforts to contract the money supply do show effects on real goods and services, just as effects on the supply of baby-sitting scrips affected real baby-sitting services.
    \end{enumerate}
  \end{problem}
}\end{document}
