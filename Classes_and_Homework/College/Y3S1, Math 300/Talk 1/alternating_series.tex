\documentclass{beamer}

% Theme
\usetheme{PaloAlto} % You can choose from various built-in themes

% Color theme (optional)
\usecolortheme{default} % You can choose from various built-in color themes

% Title page
\title{Alternating Series and Conditional Convergence}
\author{Avinash Iyer}
\institute{Occidental College}
\date{\today} % You can set a specific date here

% Additional packages (you can add more as needed)
\usepackage{graphicx} % For including images
\usepackage{caption}  % For customizing captions
\usepackage{amsmath}  % For mathematical symbols and equations
\usepackage{hyperref} % For hyperlinks

% Define custom colors (optional)
\definecolor{myblue}{RGB}{0,51,102}
\definecolor{myred}{RGB}{204,0,0}
\makeatletter
\renewrobustcmd{\beamer@@pause}[1][]{%
  \unless\ifmeasuring@%
  \ifblank{#1}%
    {\stepcounter{beamerpauses}}%
    {\setcounter{beamerpauses}{#1}}%
  \onslide<\value{beamerpauses}->\relax%
  \fi%
}
\makeatother
% Custom settings (optional)
\setbeamercolor{structure}{fg=myblue} % Change the color of the presentation elements
\setbeamertemplate{caption}[numbered] % Number captions for figures and tables
\AtBeginSection[]
{
  \begin{frame}
    \frametitle{Table of Contents}
    \tableofcontents[currentsection]
  \end{frame}
}
\begin{document}

\begin{frame}
    \titlepage
\end{frame}
\section{Alternating Harmonic Series: An Analysis}
\begin{frame}
  \frametitle{A Series}
  Consider the following series: \pause
  \begin{align*}
    \sum_{n=1}^{\infty} \frac{(-1)^{n + 1}}{n} &= 1 - \frac{1}{2} + \frac{1}{3} - \frac{1}{4} + \frac{1}{5} - \cdots
  \end{align*} \pause
  This series appears to be related to the harmonic series, but also very different:
  \begin{align*}
    \sum_{n=1}^{\infty}\frac{1}{n} &= 1 + \frac{1}{2} + \frac{1}{3} + \frac{1}{4} + \cdots \tag*{Harmonic Series}
  \end{align*}
\end{frame}
\begin{frame}
  \frametitle{Divergence of the Harmonic Series}
  We can show that the harmonic series is divergent as follows:
  \begin{align*}
    \sum_{n=1}^{\infty}\frac{1}{n} &= 1 + \frac{1}{2} + \frac{1}{3} + \frac{1}{4} + \frac{1}{5} + \cdots\\
                                   &\geq 1 + \frac{1}{2} + \frac{1}{4} + \frac{1}{4} + \frac{1}{8} + \cdots\\
                                   &= 1 + \frac{1}{2} + \frac{1}{2} + \frac{1}{2} + \cdots\\
                                   &= \infty
  \end{align*}
\end{frame}
\begin{frame}
  \frametitle{Differences}
  However, our alternating harmonic series is different. Taking partial sums, we get the following sequence:
  \begin{align*}
    s_{n} &= \sum_{k=1}^{n} \frac{(-1)^{n+1}}{n}\\
    s_{1} &= 1\\
    s_{2} &= \frac{1}{2}\\
    s_{3} &= \frac{5}{6}\\
    s_{4} &= \frac{7}{12}\\
          &\vdots
  \end{align*}
\end{frame}
\begin{frame}
  Clearly, this sequence does not grow without bound --- it is bounded above by $1$, and doesn't seem to dip below $\frac{1}{2}$
\end{frame}
\begin{frame}
  \frametitle{Convergence}
  The alternating harmonic does converge. Specifically,
  \begin{align*}
    \sum_{n=1}^{\infty}\frac{(-1)^{n+1}}{n} &= \ln 2
  \end{align*}\pause
  ...or does it? 
\end{frame}
\begin{frame}
  \frametitle{Rearranging the Alternating Harmonic Series}
  Rearrange the series as follows:
  \begin{align*}
    \sum_{n=1}^{\infty}\frac{(-1)^{n+1}}{n} &= \left(1-\frac{1}{2}\right) - \frac{1}{4} + \left(\frac{1}{3} - \frac{1}{6}\right) - \frac{1}{8} + \cdots\pause\\
                                            &= \frac{1}{2} - \frac{1}{4} + \frac{1}{6} - \frac{1}{8} + \cdots\pause\\
                                            &= \frac{1}{2}\sum_{n=1}^{\infty}\frac{(-1)^{n+1}}{n}\pause\\
                                            &= \frac{1}{2}\ln 2
  \end{align*}
\end{frame}
\end{document}
