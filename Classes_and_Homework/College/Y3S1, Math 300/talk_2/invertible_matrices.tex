\documentclass[12pt]{extarticle}
\title{}
\author{}
\date{}
\usepackage[shortlabels]{enumitem}


%paper setup
\usepackage{geometry}
\geometry{a4paper, portrait, margin=1in}
\usepackage{fancyhdr}
% sans serif font:
\usepackage{cmbright}
%symbols
\usepackage{amsmath}
\usepackage{bigints}
\usepackage{amssymb}
\usepackage{amsthm}
\usepackage{mathtools}
\usepackage[hidelinks]{hyperref}
\usepackage{gensymb}
\usepackage{multirow,array}
\usepackage{multicol}

\newtheorem*{remark}{Remark}
\usepackage[T1]{fontenc}
\usepackage[utf8]{inputenc}

%chemistry stuff
%\usepackage[version=4]{mhchem}
%\usepackage{chemfig}

%plotting
\usepackage{pgfplots}
\usepackage{tikz}
\tikzset{middleweight/.style={pos = 0.5}}
%\tikzset{weight/.style={pos = 0.5, fill = white}}
%\tikzset{lateweight/.style={pos = 0.75, fill = white}}
%\tikzset{earlyweight/.style={pos = 0.25, fill=white}}

%\usepackage{natbib}

%graphics stuff
\usepackage{graphicx}
\graphicspath{ {./images/} }
\usepackage[style=numeric, backend=biber]{biblatex} % Use the numeric style for Vancouver
\addbibresource{the_bibliography.bib}
%code stuff
%when using minted, make sure to add the -shell-escape flag
%you can use lstlisting if you don't want to use minted
%\usepackage{minted}
%\usemintedstyle{pastie}
%\newminted[javacode]{java}{frame=lines,framesep=2mm,linenos=true,fontsize=\footnotesize,tabsize=3,autogobble,}
%\newminted[cppcode]{cpp}{frame=lines,framesep=2mm,linenos=true,fontsize=\footnotesize,tabsize=3,autogobble,}

%\usepackage{listings}
%\usepackage{color}
%\definecolor{dkgreen}{rgb}{0,0.6,0}
%\definecolor{gray}{rgb}{0.5,0.5,0.5}
%\definecolor{mauve}{rgb}{0.58,0,0.82}
%
%\lstset{frame=tb,
%	language=Java,
%	aboveskip=3mm,
%	belowskip=3mm,
%	showstringspaces=false,
%	columns=flexible,
%	basicstyle={\small\ttfamily},
%	numbers=none,
%	numberstyle=\tiny\color{gray},
%	keywordstyle=\color{blue},
%	commentstyle=\color{dkgreen},
%	stringstyle=\color{mauve},
%	breaklines=true,
%	breakatwhitespace=true,
%	tabsize=3
%}
% text + color boxes
\renewcommand{\mathbf}[1]{\mathbold{#1}}
\usepackage[most]{tcolorbox}
\tcbuselibrary{breakable}
\tcbuselibrary{skins}
\newtcolorbox{problem}[1]{colback=white,enhanced,title={\small #1},
          attach boxed title to top center=
{yshift=-\tcboxedtitleheight/2},
boxed title style={size=small,colback=black!60!white}, sharp corners, breakable}
%including PDFs
%\usepackage{pdfpages}
\setlength{\parindent}{0pt}
\usepackage{cancel}
\pagestyle{fancy}
\fancyhf{}
\rhead{Avinash Iyer}
\lhead{Invertible Matrices}
\newcommand{\card}{\text{card}}
\newcommand{\ran}{\text{ran}}
\newcommand{\N}{\mathbb{N}}
\newcommand{\Q}{\mathbb{Q}}
\newcommand{\Z}{\mathbb{Z}}
\newcommand{\R}{\mathbb{R}}
\begin{document}
  Consider 4 vectors:
  \begin{align*}
    e_1 &= \begin{pmatrix}1\\0\\0\\0\end{pmatrix}\\
    e_2 &= \begin{pmatrix}0\\1\\0\\0\end{pmatrix}\\
    e_3 &= \begin{pmatrix}0\\0\\1\\0\end{pmatrix}\\
    e_4 &= \begin{pmatrix}0\\0\\0\\1\end{pmatrix}
  \end{align*}
  Is it possible to create 
  \begin{align*}
    v &= \begin{pmatrix}1\\2\\3\\4\end{pmatrix}
  \end{align*}
  from some combination of $e_1,e_2,e_3,$ and $e_4$?
  \newpage
  We say that $v$ is in the \textit{span} of $\{e_1,e_2,e_3,e_4\}$. The span of a set of vectors is the set of \textit{linear combinations} of these vectors.\\

  \vspace{15cm}

  Let's try writing this set of vectors differently --- instead of putting them into a set, let's just create a new structure to place them all together. Put each of $e_1$, $e_2$, $e_3$, and $e_4$ into one big set of parentheses --- and flip them horizontally too.
  \begin{align*}
    I &= \begin{pmatrix}1&0&0&0\\0&1&0&0\\0&0&1&0\\0&0&0&1\end{pmatrix}\\
    \\
    M &= \begin{pmatrix}1&0&0&0\\1&2&3&4\\0&0&1&0\\0&0&0&1\end{pmatrix}
  \end{align*}
  \newpage

  Both $I$ and $M$ are examples of a matrix. But, if you look more closely, you'll see that our vector $v$ is in the second row of $M$ in place of $e_2$. This has us wondering --- can we ``create'' $M$ from $I$, just as we created $v$ from $e_2$? 

  \newpage

  Consider the following matrix.
  \begin{align*}
    M' &= \begin{pmatrix}1&0&0&0\\2&0&0&0\\3&0&0&0 \\ 4&0&0&0\end{pmatrix}
  \end{align*}

  \vspace{10cm}

  For any vector $v$ that is a linear combination of the rows of $M$, this means we can find a vector $x$ such that linearly combining each row of $M$ "by" $x$ yields $v$.
  \newpage
  \textbf{Rank:} the minimum size of the set of vectors needed to create all the linear combinations of the rows of the matrix.\\

  \newpage
  A (square) matrix is \textit{invertible} if its rank is equal to the number of rows (or columns).\\

  \vspace{10cm}
  Intuitively, this means every vector with the same number of rows as $M$ can be created by linearly combining the "constituent vectors" of $M$, either rows or columns.
  \newpage
  A lot of the conditions in the Fundamental Theorem of Invertible Matrices, shown below, are essentially based on this fact of linear combinations.
  \begin{problem}{Fundamental Theorem of Invertible Matrices}
    Let $A $ be a $ n\times n$ matrix. The following statements are equivalent.
    \begin{enumerate}[(a)]
      \item $A$ is invertible.
      \item $Ax = b$ has a unique solution for every $b$ in $\R^n$.
      \item $Ax = 0$ has only the trivial solution.
      \item The reduced row echelon form of $A$ is $I_n$.
      \item $A$ is a product of elementary matrices.
      \item $\text{rank}(A) = n$.
      \item $\text{nullity}(A) = 0$.
      \item The column vectors of $A$ are linearly independent.
      \item The column vectors of $A$ span $\R^n$.
      \item The column vectors of $A$ form a basis for $\R^n$.
      \item The row vectors of $A$ are linearly independent.
      \item The row vectors of $A$ span $\R^n$.
      \item The row vectors of $A$ form a basis for $\R^n$.
    \end{enumerate}
    (Adapted from David Poole's \textit{Linear Algebra, A Modern Introduction}, pg. 206)
  \end{problem}
  \newpage
  \begin{center}
    \large Thank You. Any Questions?
  \end{center}
\end{document}
