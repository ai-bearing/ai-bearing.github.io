\documentclass[8pt]{extarticle}
\title{}
\author{Avinash Iyer}
\date{}
\usepackage[shortlabels]{enumitem}

%font setup
%
%\usepackage{newpxtext,eulerpx}

%paper setup
\usepackage{geometry}
\geometry{letterpaper, portrait, margin=1in}
\usepackage{fancyhdr}

%symbols
\usepackage{amsmath}
\usepackage{amssymb}
\usepackage{mathtools}
\usepackage{hyperref}
\usepackage{gensymb}
\usepackage{multirow,array}

\usepackage[T1]{fontenc}
\usepackage[utf8]{inputenc}

%chemistry stuff
\usepackage[version=4]{mhchem}
\usepackage{chemfig}

%plotting
\usepackage{pgfplots}
\usepackage{tikz}
\tikzset{middleweight/.style={pos = 0.5, fill=white}}
\tikzset{weight/.style={pos = 0.5, fill = white}}
\tikzset{lateweight/.style={pos = 0.75, fill = white}}
\tikzset{earlyweight/.style={pos = 0.25, fill=white}}

%\usepackage{natbib}

%graphics stuff
\usepackage{graphicx}
\graphicspath{ {./images/} }

%code stuff
%when using minted, make sure to add the -shell-escape flag
%you can use lstlisting if you don't want to use minted
%\usepackage{minted}
%\usemintedstyle{pastie}
%\newminted[javacode]{java}{frame=lines,framesep=2mm,linenos=true,fontsize=\footnotesize,tabsize=3,autogobble,}
%\newminted[cppcode]{cpp}{frame=lines,framesep=2mm,linenos=true,fontsize=\footnotesize,tabsize=3,autogobble,}

%\usepackage{listings}
%\usepackage{color}
%\definecolor{dkgreen}{rgb}{0,0.6,0}
%\definecolor{gray}{rgb}{0.5,0.5,0.5}
%\definecolor{mauve}{rgb}{0.58,0,0.82}
%
%\lstset{frame=tb,
%	language=Java,
%	aboveskip=3mm,
%	belowskip=3mm,
%	showstringspaces=false,
%	columns=flexible,
%	basicstyle={\small\ttfamily},
%	numbers=none,
%	numberstyle=\tiny\color{gray},
%	keywordstyle=\color{blue},
%	commentstyle=\color{dkgreen},
%	stringstyle=\color{mauve},
%	breaklines=true,
%	breakatwhitespace=true,
%	tabsize=3
%}
% text + color boxes
\usepackage[most]{tcolorbox}
\tcbuselibrary{breakable}
\newtcolorbox{problem}[1]{colback = white, title = {#1}, breakable}
\newtcolorbox{solution}{colback = white, colframe = black!75!white, title = Solution, breakable}
%including PDFs
%\usepackage{pdfpages}
\setlength{\parindent}{0pt}

\pagestyle{fancy}
\fancyhf{}
\rhead{Avinash Iyer}
\lhead{My Long Arc of Academic Interests}
\newcommand{\card}{\text{card}}
\newcommand{\ran}{\text{ran}}
\newcommand{\N}{\mathbb{N}}
\newcommand{\Q}{\mathbb{Q}}
\newcommand{\Z}{\mathbb{Z}}
\newcommand{\R}{\mathbb{R}}
\begin{document}
I like to think that choosing to do math was sort of the arc of my academic studies returning back to equilibrium from a long detour. I was pretty interested in math starting around 7th grade (when my cousin introduced me to Numberphile), and I obsessively watched the content of YouTube channels such as Numberphile, 3Blue1Brown, etc. Perhaps the most attention-grabbing of the videos I did watch was one showing that the sum of all natural numbers is $-1/12$ (it doesn't, but they made a very convincing case to young me).\\

Did I understand them during this early period of time? Not particularly. A video by 3Blue1Brown that discussed the brachistochrone (a special kind of curve for which something rolling down it will do so in least time) didn't really make much sense to me back then. But the graphics sure were attention-grabbing.\\

  However, when high school came around, math in school started taking precedent, and after a particularly difficult series of math tests and competitions, I stopped thinking I was able to \textit{do} math, in the same way others were able to do it. I was still able to get As in my math classes, but barely. I focused my efforts on other subjects like chemistry and physics, and while I wasn't much better in those other subjects, they felt unique enough that I didn't feel like I was being crowded out by every other kid who drilled math competitions like the AMC or AIME daily (and did much better in them than I did).\\

  However, then came college, and I decided to sign up for Linear Algebra, and all the math I had previously not cared about came rushing back, and I found the starry-eyed 7th grade me buried deep inside my psyche. The major's size was probably the pivotal factor --- I don't feel disconnected from other peers in the math major in the same year because all of us can fit comfortably in the math student lounge. Personalizing all the other people in a math class removes the tendency toward a cutthroat tenor that I think all of us are prone to.\\

  Before Linear Algebra, I thought that my preferred math class as a category was applied (e.g., calculus, which was my favorite class in high school). However, after taking Linear Algebra, then Algebra, then Topology, I found that pure math was much more fun. Abstracting away from potential real-world problems, and finding results through the fundamentals, was much more enjoyable (when we could find a solution, that was). For example, one of my favorite results that I learned was the proof for Fermat's Little Theorem, which is as follows:
  \begin{problem}{Proof of Fermat's Little Theorem}
    If $a\in\Z$ and $p$ is a prime number, then $a^p \equiv a\mod p$.
    \tcblower
    Consider $G = \{1,2,\dots,p-1\}$ under multiplication modulo $p$. $G$ must be a group:
    \begin{itemize}
      \item $\cdot$ is a binary operation.
      \item $\cdot$ is associative
      \item $1$ is the identity element.
      \item Since $\forall a\in G,~\gcd(a,p) = 1$, we know that $\exists n,k\in \Z$ such that $np + ka = 1$, or that $ka \equiv 1$ modulo $p$. By taking $k$ modulo $p$, we can have $k'\in G$ such that $k'a \equiv 1 \equiv ak'$, meaning $k' = a^{-1}$.
    \end{itemize}
    Consider the subgroup generated by $a$, where $a\in\{1,2,\dots,p-1\}$, $\langle a \rangle = \{1,a,a^2,a^{k-1}\}$ and $a^k \equiv 1\mod p$.\\

    Since this is a subgroup, it must divide $|G|$ (by Lagrange's Theorem), which is $p-1$, so $p-1 = kn$, where $a^k \equiv 1$ and $n\in\Z$. Therefore, $a^{p-1} \equiv 1$ modulo  $p$ (as we raise $a^k$ to $n$, meaning we raise $1$ to $n$), meaning $a^p \equiv a$ modulo $p$.\\

    If $a\notin \{1,2,\dots,p-1\}$, we can create $a'$ by taking the remainder of $a$ upon division by $p$. Then, $a^{p} \equiv a'^{p} \equiv a' \equiv a\mod p$.
  \end{problem}
  When I learned this proof in Algebra, seeing each step that made sense based on previous things we had learned at that time, and then proving, once and for all, a result that I had learned \textit{of} years before, but never really understood, I felt overjoyed. Finally, I was in the realm of the math I had thought in middle and high school would be forever unreachable, and I was actually keeping my head above water in it.\\

  I enjoy math a lot more than other subjects (including my other major of economics), simply because math is just rewarding. The idea that, based on some rules, one can prove, definitively, without a doubt, that something is true, is makes the subject even at its most abstract feel much more grounded in reality than something like economics, where much of our education is focused on models and abstractions so far removed from the everyday work of what economists do (which is mostly real analysis and regressions).\\

  After college, I'm not particularly sure what I'm going to do with my math degree. Based on my current course of study, I'm probably going to try to get a PhD in a pure field of mathematics such as analysis or algebra, though I might also try to go into a finance or consulting role. However, a lot can change in the course of a year, but I am fairly certain that throughout all this, I will still enjoy and cherish math for years to come.
\end{document}
