\documentclass[12pt]{extarticle}
\title{}
\author{}
\date{}
\usepackage[shortlabels]{enumitem}


%paper setup
\usepackage{geometry}
\geometry{letterpaper, portrait, margin=1in}
\usepackage{fancyhdr}
% sans serif font:
\usepackage{cmbright}
%symbols
\usepackage{amsmath}
\usepackage{bigints}
\usepackage{amssymb}
\usepackage{amsthm}
\usepackage{mathtools}
\usepackage[hidelinks]{hyperref}
\usepackage{gensymb}
\usepackage{multirow,array}
\usepackage{multicol}

\newtheorem*{remark}{Remark}
\usepackage[T1]{fontenc}
\usepackage[utf8]{inputenc}
\usepackage{titlesec}
%chemistry stuff
%\usepackage[version=4]{mhchem}
%\usepackage{chemfig}

%plotting
\usepackage{pgfplots}
\usepackage{tikz}
\tikzset{middleweight/.style={pos = 0.5}}
%\tikzset{weight/.style={pos = 0.5, fill = white}}
%\tikzset{lateweight/.style={pos = 0.75, fill = white}}
%\tikzset{earlyweight/.style={pos = 0.25, fill=white}}

%\usepackage{natbib}

%graphics stuff
\usepackage{graphicx}
\graphicspath{ {./images/} }
\usepackage[style=numeric, backend=biber]{biblatex} % Use the numeric style for Vancouver
\addbibresource{the_bibliography.bib}
%code stuff
%when using minted, make sure to add the -shell-escape flag
%you can use lstlisting if you don't want to use minted
%\usepackage{minted}
%\usemintedstyle{pastie}
%\newminted[javacode]{java}{frame=lines,framesep=2mm,linenos=true,fontsize=\footnotesize,tabsize=3,autogobble,}
%\newminted[cppcode]{cpp}{frame=lines,framesep=2mm,linenos=true,fontsize=\footnotesize,tabsize=3,autogobble,}

%\usepackage{listings}
%\usepackage{color}
%\definecolor{dkgreen}{rgb}{0,0.6,0}
%\definecolor{gray}{rgb}{0.5,0.5,0.5}
%\definecolor{mauve}{rgb}{0.58,0,0.82}
%
%\lstset{frame=tb,
%	language=Java,
%	aboveskip=3mm,
%	belowskip=3mm,
%	showstringspaces=false,
%	columns=flexible,
%	basicstyle={\small\ttfamily},
%	numbers=none,
%	numberstyle=\tiny\color{gray},
%	keywordstyle=\color{blue},
%	commentstyle=\color{dkgreen},
%	stringstyle=\color{mauve},
%	breaklines=true,
%	breakatwhitespace=true,
%	tabsize=3
%}
% text + color boxes
\renewcommand{\mathbf}[1]{\mathbold{#1}}
\usepackage[most]{tcolorbox}
\tcbuselibrary{breakable}
\tcbuselibrary{skins}
\newtcolorbox{problem}[1]{colback=white,enhanced,title={\small #1},
          attach boxed title to top center=
{yshift=-\tcboxedtitleheight/2},
boxed title style={size=small,colback=black!60!white}, sharp corners, breakable}
%including PDFs
%\usepackage{pdfpages}
\setlength{\parindent}{0pt}
\usepackage{cancel}
%\pagestyle{fancy}
%\fancyhf{}
%\rhead{Avinash Iyer}
%\lhead{}
\newcommand{\card}{\text{card}}
\newcommand{\ran}{\text{ran}}
\newcommand{\N}{\mathbb{N}}
\newcommand{\Q}{\mathbb{Q}}
\newcommand{\Z}{\mathbb{Z}}
\newcommand{\R}{\mathbb{R}}
\usepackage{setspace}
\begin{document}
\doublespacing
  \begin{center}
    {\large \scshape Existence of Equilibria in Non-Cooperative Games of Strategy}\\
    Avinash Iyer\\
    Occidental College
  \end{center}
  \section*{Abstract}%
  In this paper, I examine the background and the motivation behind the development of game theory, a discipline with broad applications to economics, but one which is still fundamentally based in mathematics. The most crucial set of papers that led to the development of game theory was John Nash's ``Non-Cooperative Games,'' and the paper I am examining today, ``Equilibrium Points in $N$-Person Games,'' published in the Proceedings of the National Academy of Sciences in 1950.

  \section*{Games of Strategy: Background and Motivation}%
 One of the earliest discussions of a game of strategy came in the 1838 book \textit{Researches Into the Mathematical Principles of the Theory of Wealth} by Augustin Cournot --- wherein he modeled economic competition between two profit-maximizing producers in the creation of spring water, in which each producer is a price-taker from market demand. In it, he found that each producer determined their quantity strategically in response to the production of the other producer. Additionally, Cournot found that, in this determination, there was a point where every producer was maximizing their profits; essentially, there was an ``equilibrium'' of sorts.\\

  Game theory as a discipline, however, was first examined in depth by John von Neumann and Oskar Morgenstern in the \textit{Theory of Games and Economic Behavior}, in which the duo developed a theory of examining \textit{zero-sum} games (where there is a fixed ``pool'' of payoffs that players compete over) and \textit{cooperative} games (where players will collaborate to maximize their payoffs). John Nash, on the other hand, was instrumental in developing theories and understandings of games where one player maximizing their payoff does not necessarily mean all players are equally worse off or maximally better off.\\

  The classic example of a non-cooperative game is the Prisoner's Dilemma. Two people are taken into custody for a crime they both committed --- however, the police do not have enough evidence to convict either for the maximum sentence. The options for them are to either stay quiet ($Q$), in which case both are locked up on a lesser charge for $1$ year, or ``fink'' ($F$), in which case the player who finks is set free and the player who stayed quiet is sentenced to $10$ years. If both players fink, they are both sentenced to a term of $5$ years. We can represent this non-cooperative game as a matrix --- player $1$'s payoff is represented as the first number and their strategy on the left side of the matrix, while player $2$'s payoff is represented as the second number and their strategy on the top of the matrix.
  \begin{center}
    \begin{tabular}{c|c|c|}
      \multicolumn{1}{c}{} & \multicolumn{1}{c}{$Q$} & \multicolumn{1}{c}{$F$}\\
      \cline{2-3}
      $Q$ & $-1,-1$ & $-10,0$\\
      \cline{2-3}
      $F$ & $-10,0$ & $-5,-5$\\
      \cline{2-3}
    \end{tabular}
  \end{center}
  We can see from this matrix that an equilibrium point emerges --- namely, both players choose to fink, as if player $1$ moves from $F$ to $Q$, their payoff is reduced by $5$, and if player $2$ moves from $F$ to $Q$, their payoff is reduced by $5$. Additionally, we can see that $Q$ is not an equilibrium point, as either player improves their payoff by finking.\\

  Nash further developed the theory of non-cooperative games by showing in his Ph.D. thesis (written in 1950), aptly titled ``Non-Cooperative Games,'' that every finite, $n$-player game has an equilibrium point consisting of some mix of every player's strategies. In it, he cited his 1949 paper, ``Equilibrium Points in $N$-Person Games,'' published in 1950 in the Proceedings of the National Academy of Sciences. This is the paper that I will summarize.
  \section*{Existence of Equilibria}%
  In an $n$-person game, there are, of course, the $n$ players who make up the game. Each of these players chooses from a certain \textit{strategy space}. A \textit{pure strategy} is one in which a player exclusively plays one of their allowed strategies, while a \textit{mixed strategy} is a probability distribution assigned over all the pure strategies; every player must play a strategy in any game. We will refer to this $n$-tuple of strategies as a ``strategy profile.''\\

  We can view the strategies that each of the $n$ players play in a game as a $n$-tuple. A \textit{payoff function} assigns each player a payoff based on their pure strategy; the payoff of a mixed strategy is the expected value of the pure strategies, weighted by the probability distribution placed on the pure strategies within the mixed strategy.\\

  Nash refers to a strategy profile ``counter[ing]'' another in the paper as a mixed strategy for a certain player that maximizes expected payoff holding the other $n-1$ strategies in the $n$-tuple; in most modern game theory textbooks, this is known as a best response, which is what I will refer to it as.\\

  Just as with the prisoner's dilemma, we view an equilibrium in a game as one where no player is better off from changing their strategy. Equivalently, this means every player is engaged in the best response to every other player's strategy --- in the language of functions, this means that a map from the set defined by all the strategy profiles to the space defined by the best responses to each strategy profile contains a fixed point (where a strategy is best-responding to itself). In the paper, Nash showed that the definition of an $n$-person game made it that such a fixed point must exist, by using Kakutani's Fixed Point Theorem.\\

  Kakutani's Fixed Point Theorem expands upon Brouwer's Fixed Point Theorem, which states that a continuous function from a closed unit ball (i.e., $\{x\mid d(x,p) \leq 1\}$) to itself must contain a fixed point. Specifically, Kakutani's fixed point theorem states that a continuous \textit{correspondence} that has a \textit{convex} image, and a convex, \textit{compact} domain must have a fixed point.
  \begin{description}
    \item[Correspondence:] A correspondence is a map from elements of a set $D$ to the power set, $\mathcal{P}(D)$.
    \item[Convex:] A convex set $S$ is a set where, if $p$ and $q$ are elements of $S$, then any linear combination $tp + (1-t)q$ where $0\leq t \leq 1$ is an element of $S$.
    \item[Compact:] For our purposes, this means a set that is closed and bounded (i.e., there is an ``edge'' to it).
  \end{description}
  In the set defined by the strategy profiles of each player, we can see that the set is compact --- the set of all mixed strategy profiles is bounded by the various pure strategy profiles, and every strategy profile in which every player plays a pure strategy is part of the set of strategy profiles. Additionally, by the definition of mixed strategies, linear combinations of pure strategies are part of the set of strategy profiles, meaning the set is convex. The payoff function is also defined over all mixed strategy profiles, meaning that the mapping from strategy profiles to the best responses is continuous.\\

  By the definition of best response, it must be the case that if $p$ is a strategy profile, and $q_1,q_2,\dots,q_n$ are best responses, the payoff to player $i$ from $q_1,\dots,q_n$ is identically maximal, meaning that a linear combination of these best responses is also a best response. Thus, the set of best responses is convex.\\

  So, by Kakutani's Fixed Point Theorem, it must be the case that there is a fixed point, or equilibrium point --- a strategy profile in which every player is playing their best response to every other player.
  \section*{Significance}%
  Nash's finding in the PNAS paper, further developed in his Ph.D. thesis, helped set the stage for further developments in the field of game theory. The idea of a ``Nash Equilibrium,'' which is the equilibrium point shown to exist above, is used heavily in the field of industrial organization to understand market entry and pricing decisions, as well as in modeling interactions in international relations.\\

  However, at the same time, Nash equilibria are not the end-all and be-all of modeling games. Further along in the development of game theory, mathematical economists such as Reinhard Selten and John Harsanyi refined the concept of Nash equilibrium to applications within dynamic games (i.e., ones that do not occur simultaneously) of complete and incomplete information. However, it was Nash that set the stage for the development of the discipline of game theory into what it is now through his groundbreaking mathematical work.
\end{document}
