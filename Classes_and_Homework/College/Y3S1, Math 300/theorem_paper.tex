\documentclass[12pt]{extarticle}
\title{Continuity in Metric Spaces and Topologies}
\author{Avinash Iyer, Occidental College}
\date{}
\usepackage[shortlabels]{enumitem}


%paper setup
\usepackage{geometry}
\geometry{letterpaper, portrait, margin=1in}
\usepackage{fancyhdr}

%symbols
\usepackage{amsmath}
\usepackage{amssymb}
\usepackage{amsthm}
\usepackage{mathtools}
\usepackage{hyperref}
\usepackage{gensymb}
\usepackage{multirow,array}

\newtheorem*{remark}{Remark}
\usepackage[T1]{fontenc}
\usepackage[utf8]{inputenc}

%chemistry stuff
%\usepackage[version=4]{mhchem}
%\usepackage{chemfig}

%plotting
\usepackage{pgfplots}
\usepackage{tikz}
\tikzset{middleweight/.style={pos = 0.5, fill=white}}
\tikzset{weight/.style={pos = 0.5, fill = white}}
\tikzset{lateweight/.style={pos = 0.75, fill = white}}
\tikzset{earlyweight/.style={pos = 0.25, fill=white}}

%\usepackage{natbib}

%graphics stuff
\usepackage{graphicx}
\graphicspath{ {./images/} }
\usepackage[style=numeric, backend=biber]{biblatex} % Use the numeric style for Vancouver
\addbibresource{the_bibliography.bib}
%code stuff
%when using minted, make sure to add the -shell-escape flag
%you can use lstlisting if you don't want to use minted
%\usepackage{minted}
%\usemintedstyle{pastie}
%\newminted[javacode]{java}{frame=lines,framesep=2mm,linenos=true,fontsize=\footnotesize,tabsize=3,autogobble,}
%\newminted[cppcode]{cpp}{frame=lines,framesep=2mm,linenos=true,fontsize=\footnotesize,tabsize=3,autogobble,}

%\usepackage{listings}
%\usepackage{color}
%\definecolor{dkgreen}{rgb}{0,0.6,0}
%\definecolor{gray}{rgb}{0.5,0.5,0.5}
%\definecolor{mauve}{rgb}{0.58,0,0.82}
%
%\lstset{frame=tb,
%	language=Java,
%	aboveskip=3mm,
%	belowskip=3mm,
%	showstringspaces=false,
%	columns=flexible,
%	basicstyle={\small\ttfamily},
%	numbers=none,
%	numberstyle=\tiny\color{gray},
%	keywordstyle=\color{blue},
%	commentstyle=\color{dkgreen},
%	stringstyle=\color{mauve},
%	breaklines=true,
%	breakatwhitespace=true,
%	tabsize=3
%}
% text + color boxes
\usepackage[most]{tcolorbox}
\tcbuselibrary{breakable}
\newtcolorbox{problem}[1]{colback = white, title = {#1}, breakable}
\newtcolorbox{solution}{colback = white, colframe = black!75!white, title = Solution, breakable}
%including PDFs
%\usepackage{pdfpages}
\setlength{\parindent}{0pt}
\usepackage{cancel}
\newcommand{\card}{\text{card}}
\newcommand{\ran}{\text{ran}}
\newcommand{\N}{\mathbb{N}}
\newcommand{\Q}{\mathbb{Q}}
\newcommand{\Z}{\mathbb{Z}}
\newcommand{\R}{\mathbb{R}}
\usepackage{setspace}
\begin{document}
\doublespacing
  \begin{center}
    \large \sc Homeomorphism of the Open-Ball Topology on $\R^2$ and the Rectangle Topology on $\R^2$
  \end{center}
  \begin{center}
    Avinash Iyer\\
    Occidental College
  \end{center}
  \section*{Abstract}%
  In this paper, I introduce the concept of metric spaces (including the definition of distance metrics, open sets, and continuity), topologies on sets, and homeomorphisms. With this knowledge, we can prove that two particular topologies on $\R^2$ are homeomorphic.
  \section*{Metric on $\R$ and $\R^2$}%
  Consider a set $X$, with $a,b\in X$. It's often useful to consider the idea of a distance between $a$ and $b$, $d(a,b)$. This distance function must map every pair of points to some positive number.
  \begin{align*}
    d: X\times X \rightarrow \R^+
  \end{align*}
  Additionally, it has to have the following properties:
  \begin{itemize}
    \item Zero distance property: $d(a,a) = 0$ and if $d(a,b) = 0$, then $a = b$.
    \item Commutativity: $d(a,b) = d(b,a)$.
    \item Triangle Inequality: $d(x,z) = d(x,y) + d(y,z)$.
  \end{itemize}
  This defines a \textbf{distance metric} on $X$. For instance, for $x,y\in\R^n$, we define the \textbf{Euclidean metric} as follows:
  \begin{align*}
    d(x,y) = \left(\sum_{i=1}^{n}(x_i-y_i)^2\right)^{1/2}
  \end{align*}
  In $\R^1 = \R$, this is equivalent to $|x-y|$.
  \section*{Open Balls and Open Sets}%
  In $\R^n$ with the Euclidean metric, we define the \textbf{open ball} of radius $\varepsilon > 0$ about the point $x$ as follows:
  \begin{align*}
    V_{\varepsilon} &= \left\{p \mid d(x,p) < \varepsilon\right\}
  \end{align*}
  In $\R$, this equivalent to the open interval $(x-\varepsilon,x+\varepsilon)$.\\

  In $A\subseteq X$, if $\forall x\in A,~\exists \varepsilon > 0$ such that $V_{\varepsilon}(x)\subseteq A$, then $A$ is an \textbf{open set}. If $A = \emptyset$ or $A = X$, then $A$ is open.\\

  Suppose $A$ and $B$ are open sets in a metric space $X$. We claim that $A\cap B$ and $A\cup B$ are open sets in $X$. Let $a\in A$ and $b\in B$.\\

  Then, $\exists \varepsilon_a>0$ such that $V_{\varepsilon_a}(a)\subseteq A$, meaning that $V_{\varepsilon_a}(a)\subseteq A\cup B$, and similarly for $b$ and $\varepsilon_b$.\\

  If $A = \emptyset$, then $A\cap B = \emptyset$, meaning $A\cap B$ is open. Otherwise, suppose $A\cap B \neq \emptyset$. Let $x\in A\cap B$, meaning $x\in A$ and $x\in B$. So, $\exists \varepsilon_1>0$ such that $V_{\varepsilon_1}(x)\in A$.\\

  Specifically, define $\varepsilon_1$ to be the maximum such value. Similarly, $\exists \varepsilon_2 > 0$ such that $V_{\varepsilon_2}(x)\in B$, where $\varepsilon_2$ is the maximum such value. Let $\varepsilon = \min\{\varepsilon_1,\varepsilon_2\}$ Then, $V_{\varepsilon} \subseteq V_{\varepsilon_1}$ and $V_{\varepsilon} \subseteq V_{\varepsilon_2}$, meaning $V_{\varepsilon} \subseteq A$ and $V_{\varepsilon}\subseteq B$, so $V_{\varepsilon} \subseteq A\cap B$.
  \section*{Continuity}%
  We can consider functions between metric spaces. For $f: X \rightarrow Y$ where $X$ and $Y$ are metric spaces.If $f$ is continuous, then $\forall \varepsilon > 0, \exists \delta > 0$ such that for $x_1,x_2\in X$,\[d(f(x_1),f(x_2)) < \varepsilon \Rightarrow d(x_1,x_2) < \delta\]. Alternatively, $\forall\varepsilon > 0$, if $f(x_2)\in V_{\varepsilon}(x_1)$, then $\exists \delta > 0$ such that $x_2\in V_{\delta}(x_1)$.\\

  We claim that $f:X\rightarrow Y$ is a continuous function if and only if the preimage of every open set in $Y$ is open in $X$.
  \begin{description}
    \item[$(\Rightarrow)$] Let $f: X\rightarrow Y$ be continuous, and let $B\subseteq Y$. We claim that $f^{-1}(B) = \{x\mid f(x) \in B\}$ is open. Let $f(b)\in B$. Then, since $B$ is open, $\exists \varepsilon > 0$ such that $V_{\varepsilon}(f(b)) \subseteq B$. Since $f$ is continuous, $\forall f(x)\in V_{\varepsilon}(f(b))$, $\exists \delta$ such that $x\in V_{\delta}(b)$. 
  \end{description}
\end{document}
