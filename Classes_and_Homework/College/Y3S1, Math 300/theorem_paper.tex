\documentclass[12pt]{extarticle}
\title{Continuity in Metric Spaces and Topologies}
\author{Avinash Iyer, Occidental College}
\date{}
\usepackage[shortlabels]{enumitem}


%paper setup
\usepackage{geometry}
\geometry{letterpaper, portrait, margin=1in}
\usepackage{fancyhdr}

%symbols
\usepackage{amsmath}
\usepackage{amssymb}
\usepackage{amsthm}
\usepackage{mathtools}
\usepackage{hyperref}
\usepackage{gensymb}
\usepackage{multirow,array}

\newtheorem*{remark}{Remark}
\usepackage[T1]{fontenc}
\usepackage[utf8]{inputenc}

%chemistry stuff
%\usepackage[version=4]{mhchem}
%\usepackage{chemfig}

%plotting
\usepackage{pgfplots}
\usepackage{tikz}
\tikzset{middleweight/.style={pos = 0.5, fill=white}}
\tikzset{weight/.style={pos = 0.5, fill = white}}
\tikzset{lateweight/.style={pos = 0.75, fill = white}}
\tikzset{earlyweight/.style={pos = 0.25, fill=white}}

%\usepackage{natbib}

%graphics stuff
\usepackage{graphicx}
\graphicspath{ {./images/} }
\usepackage[style=numeric, backend=biber]{biblatex} % Use the numeric style for Vancouver
\addbibresource{the_bibliography.bib}
%code stuff
%when using minted, make sure to add the -shell-escape flag
%you can use lstlisting if you don't want to use minted
%\usepackage{minted}
%\usemintedstyle{pastie}
%\newminted[javacode]{java}{frame=lines,framesep=2mm,linenos=true,fontsize=\footnotesize,tabsize=3,autogobble,}
%\newminted[cppcode]{cpp}{frame=lines,framesep=2mm,linenos=true,fontsize=\footnotesize,tabsize=3,autogobble,}

%\usepackage{listings}
%\usepackage{color}
%\definecolor{dkgreen}{rgb}{0,0.6,0}
%\definecolor{gray}{rgb}{0.5,0.5,0.5}
%\definecolor{mauve}{rgb}{0.58,0,0.82}
%
%\lstset{frame=tb,
%	language=Java,
%	aboveskip=3mm,
%	belowskip=3mm,
%	showstringspaces=false,
%	columns=flexible,
%	basicstyle={\small\ttfamily},
%	numbers=none,
%	numberstyle=\tiny\color{gray},
%	keywordstyle=\color{blue},
%	commentstyle=\color{dkgreen},
%	stringstyle=\color{mauve},
%	breaklines=true,
%	breakatwhitespace=true,
%	tabsize=3
%}
% text + color boxes
\usepackage[most]{tcolorbox}
\tcbuselibrary{breakable}
\newtcolorbox{problem}[1]{colback = white, title = {#1}, breakable}
\newtcolorbox{solution}{colback = white, colframe = black!75!white, title = Solution, breakable}
%including PDFs
%\usepackage{pdfpages}
\setlength{\parindent}{0pt}
\usepackage{cancel}
\newcommand{\card}{\text{card}}
\newcommand{\ran}{\text{ran}}
\newcommand{\N}{\mathbb{N}}
\newcommand{\Q}{\mathbb{Q}}
\newcommand{\Z}{\mathbb{Z}}
\newcommand{\R}{\mathbb{R}}
\usepackage{setspace}
\begin{document}
\doublespacing
  \begin{center}
    \large \sc Homeomorphism of the Open-Ball Topology on $\R^2$ and the Rectangle Topology on $\R^2$
  \end{center}
  \begin{center}
    Avinash Iyer\\
    Occidental College
  \end{center}
  \section*{Abstract}%
  In this paper, I introduce the concept of metric spaces (including the definition of distance metrics, open sets, and continuity), topologies on sets, and homeomorphisms. With this knowledge, we can prove that two particular topologies on $\R^2$ are homeomorphic.
  \section*{Sets and Metrics}%
  Consider a set $X$. If we let $a,b\in X$, we might ask what the ``distance'' between $a$ and $b$, as doing so might help tease out properties of $X$. We want the distance function to be defined for every pair of points, and to yield a positive number.
  \[
    d: X\times X \rightarrow \R^+
  \]
  In order to maintain a coherent idea of distance, we also need the following to hold:
  \begin{itemize}
    \item Commutativity: $d(a,b) = d(b,a)$.
    \item Zero distance condition: $d(x,x) = 0$, and if $d(a,b) = 0$, then $a = b$.
    \item Triangle inequality: $d(x,z) \leq d(x,y) + d(y,z)$. This property is important as it means that two points have arbitrarily small distance if they have arbitrarily small distance with any intermediate point.
  \end{itemize}
  We now have a well-defined distance metric. Every pair of points must have a unique distance, and no pair of points can map to two distance points. We call $X$ equipped with $d: X\times X \rightarrow \R$ a \textbf{metric space}.
\end{document}
