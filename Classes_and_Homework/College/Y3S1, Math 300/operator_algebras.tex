\documentclass[8pt]{extarticle}
\title{}
\author{Avinash Iyer}
\date{}
\usepackage[shortlabels]{enumitem}


%paper setup
\usepackage{geometry}
\geometry{letterpaper, portrait, margin=1in}
\usepackage{fancyhdr}

%symbols
\usepackage{amsmath}
\usepackage{amssymb}
\usepackage{amsthm}
\usepackage{mathtools}
\usepackage{hyperref}
\usepackage{gensymb}
\usepackage{multirow,array}

\newtheorem*{remark}{Remark}
\usepackage[T1]{fontenc}
\usepackage[utf8]{inputenc}

%chemistry stuff
%\usepackage[version=4]{mhchem}
%\usepackage{chemfig}

%plotting
\usepackage{pgfplots}
\usepackage{tikz}
\tikzset{middleweight/.style={pos = 0.5, fill=white}}
\tikzset{weight/.style={pos = 0.5, fill = white}}
\tikzset{lateweight/.style={pos = 0.75, fill = white}}
\tikzset{earlyweight/.style={pos = 0.25, fill=white}}

%\usepackage{natbib}

%graphics stuff
\usepackage{graphicx}
\graphicspath{ {./images/} }
\usepackage[style=numeric, backend=biber]{biblatex} % Use the numeric style for Vancouver
\addbibresource{the_bibliography.bib}
%code stuff
%when using minted, make sure to add the -shell-escape flag
%you can use lstlisting if you don't want to use minted
%\usepackage{minted}
%\usemintedstyle{pastie}
%\newminted[javacode]{java}{frame=lines,framesep=2mm,linenos=true,fontsize=\footnotesize,tabsize=3,autogobble,}
%\newminted[cppcode]{cpp}{frame=lines,framesep=2mm,linenos=true,fontsize=\footnotesize,tabsize=3,autogobble,}

%\usepackage{listings}
%\usepackage{color}
%\definecolor{dkgreen}{rgb}{0,0.6,0}
%\definecolor{gray}{rgb}{0.5,0.5,0.5}
%\definecolor{mauve}{rgb}{0.58,0,0.82}
%
%\lstset{frame=tb,
%	language=Java,
%	aboveskip=3mm,
%	belowskip=3mm,
%	showstringspaces=false,
%	columns=flexible,
%	basicstyle={\small\ttfamily},
%	numbers=none,
%	numberstyle=\tiny\color{gray},
%	keywordstyle=\color{blue},
%	commentstyle=\color{dkgreen},
%	stringstyle=\color{mauve},
%	breaklines=true,
%	breakatwhitespace=true,
%	tabsize=3
%}
% text + color boxes
\usepackage[most]{tcolorbox}
\tcbuselibrary{breakable}
\newtcolorbox{problem}[1]{colback = white, title = {#1}, breakable}
\newtcolorbox{solution}{colback = white, colframe = black!75!white, title = Solution, breakable}
%including PDFs
%\usepackage{pdfpages}
\setlength{\parindent}{0pt}
\usepackage{cancel}
\pagestyle{fancy}
\fancyhf{}
\rhead{Avinash Iyer}
\lhead{Operator Algebras Lecture}
\newcommand{\card}{\text{card}}
\newcommand{\ran}{\text{ran}}
\newcommand{\N}{\mathbb{N}}
\newcommand{\Q}{\mathbb{Q}}
\newcommand{\Z}{\mathbb{Z}}
\newcommand{\R}{\mathbb{R}}
\newcommand{\C}{\mathbb{C}}
\begin{document}
  We let $\C^n$ be defined as follows:
  \begin{align*}
    \C^n &= \left\{ \begin{pmatrix}
      z_1\\ \vdots \\z_n
    \end{pmatrix} \mid z_j\in\C \right\}
  \end{align*}
  with vector addition and scalar multiplication. $\C^n$ is a \textbf{vector space}, in which there is an inner product, defined as follows:
  \begin{align*}
    \left\langle \begin{pmatrix}z_1 \\ \vdots \\ z_n\end{pmatrix}, \begin{pmatrix}w_1 \\ \vdots \\ w_n\end{pmatrix}\right\rangle &= \sum z_i\overline{w}_i
  \end{align*}
  With this definition, we are able to have a \textbf{norm}, defined as follows:
  \begin{align*}
    \Vert v \Vert_2 &= \langle v,v\rangle^{1/2}
  \end{align*}
  with the given norm properties:
  \begin{itemize}
    \item $\Vert v+w\Vert \leq \Vert v\Vert + \Vert w \Vert$
    \item $\Vert \alpha v \Vert = |\alpha| \Vert v \Vert$
    \item $\Vert v \Vert = 0 \Rightarrow v = \vec{0}$
  \end{itemize}
  We also have the \textbf{Cauchy-Schwarz} inequality
  \begin{align*}
    |\langle v,w\rangle| \leq \Vert v \Vert \Vert w \Vert
  \end{align*}
  With these defined, we have
  \begin{align*}
    (\C^n,\langle \cdot , \cdot \rangle)
  \end{align*}
   denoted $\ell_2^n$.\\
  \rule{\textwidth}{0.4pt}
   With this settled, we will look at
   \begin{align*}
     \mathbb{M}_n(\C) &= \left\{(a_{ij}) \mid a_{ij}\in\C\right\}
   \end{align*}
   $\mathbb{M}_n(\C)$ also has
   \begin{itemize}
    \item Matrix Addition
    \item Matrix Multiplication
    \item Scalar Multiplication
   \end{itemize}
   Remember that $AB \neq BA$, matrices in $\C$ are still non-commutative. There is also the \textbf{adjoint} operation, which occurs when you take transpose and complex conjugate.
   \begin{align*}
     A &= (a_{ij})\\
     A^* &= (\overline{a}_{ji})_{ij}
   \end{align*}
   The adjoint operation has the following properties:
   \begin{itemize}
     \item $(A+B)^* = A^* + B^*$
     \item $(\alpha A)^* = \alpha A^*$
     \item $(AB)^* = B^*A^*$
     \item $A^{**} = A$
      \item $\langle Av,w\rangle = \langle v,A^*w\rangle$
  \end{itemize}
  With these properties, $\mathbb{M}_n(\C)$ is a \textbf{*-algebra}.\\
  \rule{\textwidth}{0.4pt}
  A matrix is a linear transformation:
  \begin{align*}
    T_A&: \ell_2^n \rightarrow \ell_2^n\\
    T_A(v) = Av
  \end{align*}
  for some matrix $A$.\\

  Given $A\in \mathbb{M}_n$, we have $\Vert A \Vert_{\text{op}}$, for the \textbf{operator norm}, defined as:
  \begin{align*}
    \Vert A \Vert_{\text{op}} &= \max\left\{\Vert Av\Vert_{2}\mid \Vert v\Vert_2 \leq 1\right\}
  \end{align*}
  \begin{description}
    \item[Exercise:] Show that $\Vert A\Vert_{\text{op}} \leq \left(\sum_{i,j=1}^{n}|a_{ij}|^2\right)^{1/2}$
  \end{description}
  Properties of $\Vert \cdot \Vert_{\text{op}}$:
  \begin{itemize}
    \item $\Vert A + B \Vert \leq \Vert A \Vert + \Vert B \Vert$
    \item $\Vert \alpha A \Vert = |\alpha| \Vert A \Vert$
    \item $\Vert A \Vert = 0 \Rightarrow A = \mathbf{0}$
    \item $\Vert AB\Vert \leq \Vert A \Vert \Vert B \Vert$
    \item $\Vert I_n\Vert = 1$
    \item $\Vert A^* \Vert = \Vert A \Vert$
    \item $\Vert A^* A\Vert = \Vert A \Vert^2$ (known as the $c^*$-property)
    \item $\Vert A \Vert = \max |\langle Av,v\rangle|, \Vert v \Vert \leq 1$
    \item $\Vert Av\Vert \leq \Vert A \Vert \cdot \Vert v \Vert$
  \end{itemize}
  \begin{itemize}
    \item $A$ is \textbf{normal} if $A A^* = A^* A$.
    \item $A$ is \textbf{self-adjoint} if $A = A^*$.
    \item $A$ is \textbf{positive} if $\langle Av,v\rangle \geq 0$
    \item $A$ is a \textbf{projection} if $A^2 = A^* = A$.
    \item $A$ is an \textbf{isometry} if $A^* A = I$.
    \item $A$ is a \textbf{unitary} if $A^* A = I$ and $AA^* = I$.
    \item $A$ is a \textbf{contraction} if $\Vert A \Vert_{\text{op}} \leq 1$.
  \end{itemize}
  Why use the word ``isometry?''
  \begin{align*}
    \Vert Av \Vert^2 &= \langle Av, Av\rangle\\
                     &= \langle v,A^*A v \rangle\\
                     &= \langle v,Iv\rangle\\
                     &= \langle v, v\rangle\\
                     &= \Vert v \Vert^2\\
                     \shortintertext{So,}
    \Vert Av\Vert &= \Vert v \Vert
  \end{align*}
  \rule{\textwidth}{0.4pt}
  \begin{description}
    \item[Spectral Theorem:] $A\in \mathbb{M}_n(\C)$ normal is always \textit{diagonalizable} via a unitary (i.e., $\exists$ unitary matrix $U$ with $U^* A U = \text{diag}(\lambda_1,\dots,\lambda_n)$), where $\{\lambda_1,\dots,\lambda_n\}$ are the eigenvalues (or the \textit{point spectrum} $\sigma_p(A)$)
  \end{description}
  Therefore, $A = UDU^*$.
  \begin{itemize}
    \item $A^2 = (UDU^*)(UDU^*) = UD^2U^* = U\text{diag}(\lambda_1^2,\dots,\lambda_n^2)U^*$
    \item $A^m = U\text{diag}(\lambda_1^m,\dots,\lambda_n^m)U^*$
    \item $p(A) = U\text{diag}(p(\lambda_1),\dots,p(\lambda_n))U^*$ for any polynomial $p$
  \end{itemize}
  We have $f: \sigma_p(A) \rightarrow \C$, $f(A) \mapsto U\text{diag}(f(\lambda_1),\dots,f(\lambda_n))U^*$.\\
  \rule{\textwidth}{0.4pt}
  \begin{description}
    \item[Von Neumann's Inequality] Given $A\in \mathbb{M}_n$, $\Vert A \Vert \leq 1$, then $\Vert p(A)\Vert_{\text{op}} \leq \max_{|z|\leq 1}|p(z)|$, where $p$ is any polynomial.
  \end{description}
  We will prove this where $A$ is unitary.
\end{document}
