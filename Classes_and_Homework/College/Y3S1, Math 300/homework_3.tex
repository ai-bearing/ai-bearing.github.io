\documentclass[8pt]{extarticle}
\title{}
\author{Avinash Iyer}
\date{}
\usepackage[shortlabels]{enumitem}


%paper setup
\usepackage{geometry}
\geometry{letterpaper, portrait, margin=1in}
\usepackage{fancyhdr}

%symbols
\usepackage{amsmath}
\usepackage{amssymb}
\usepackage{amsthm}
\usepackage{mathtools}
\usepackage{hyperref}
\usepackage{gensymb}
\usepackage{multirow,array}
\newtheorem{theorem}{Theorem}
\usepackage[T1]{fontenc}
\usepackage[utf8]{inputenc}

%chemistry stuff
%\usepackage[version=4]{mhchem}
%\usepackage{chemfig}

%plotting
\usepackage{pgfplots}
\usepackage{tikz}
\tikzset{middleweight/.style={pos = 0.5, fill=white}}
\tikzset{weight/.style={pos = 0.5, fill = white}}
\tikzset{lateweight/.style={pos = 0.75, fill = white}}
\tikzset{earlyweight/.style={pos = 0.25, fill=white}}

%\usepackage{natbib}

%graphics stuff
\usepackage{graphicx}
\graphicspath{ {./images/} }
\usepackage[style=numeric, backend=biber]{biblatex} % Use the numeric style for Vancouver
\addbibresource{the_bibliography.bib}
%code stuff
%when using minted, make sure to add the -shell-escape flag
%you can use lstlisting if you don't want to use minted
%\usepackage{minted}
%\usemintedstyle{pastie}
%\newminted[javacode]{java}{frame=lines,framesep=2mm,linenos=true,fontsize=\footnotesize,tabsize=3,autogobble,}
%\newminted[cppcode]{cpp}{frame=lines,framesep=2mm,linenos=true,fontsize=\footnotesize,tabsize=3,autogobble,}

%\usepackage{listings}
%\usepackage{color}
%\definecolor{dkgreen}{rgb}{0,0.6,0}
%\definecolor{gray}{rgb}{0.5,0.5,0.5}
%\definecolor{mauve}{rgb}{0.58,0,0.82}
%
%\lstset{frame=tb,
%	language=Java,
%	aboveskip=3mm,
%	belowskip=3mm,
%	showstringspaces=false,
%	columns=flexible,
%	basicstyle={\small\ttfamily},
%	numbers=none,
%	numberstyle=\tiny\color{gray},
%	keywordstyle=\color{blue},
%	commentstyle=\color{dkgreen},
%	stringstyle=\color{mauve},
%	breaklines=true,
%	breakatwhitespace=true,
%	tabsize=3
%}
% text + color boxes
\usepackage[most]{tcolorbox}
\tcbuselibrary{breakable}
\newtcolorbox{problem}[1]{colback = white, title = {#1}, breakable}
\newtcolorbox{solution}{colback = white, colframe = black!75!white, title = Solution, breakable}
%including PDFs
%\usepackage{pdfpages}
\setlength{\parindent}{0pt}
\usepackage{cancel}
\pagestyle{fancy}
\fancyhf{}
\rhead{Avinash Iyer}
\lhead{Editing Assignments}
\newcommand{\card}{\text{card}}
\newcommand{\ran}{\text{ran}}
\newcommand{\N}{\mathbb{N}}
\newcommand{\Q}{\mathbb{Q}}
\newcommand{\Z}{\mathbb{Z}}
\newcommand{\R}{\mathbb{R}}
\begin{document}
  \begin{description}
    \item[Theorem 1] \hfill
      \begin{itemize}
        \item The theorem statement is incorrect: for example, if $a=6,b=3,c=4$, then $a|(bc)$ but $a\not| b$ and $a\not|c$.
        \item The proof only looks at one case and generalizes to the entire integers.
      \end{itemize}
  \end{description}
  \begin{problem}{Corrected Theorem and Proof}
    \begin{theorem}
      Let $a,~b,~c\in \Z$ such that $a<b<c$. If $a|(bc)$, then $a|b$ or $a|c$
    \end{theorem}
    \begin{proof}
      Suppose toward contradiction that for  $a,b,c\in\Z$, $a \vert (bc)$, $a\not\vert b$, and $a\not\vert c$. Then $\forall x,y\in\Z$, $b\neq xa$ and $c \neq ya$. Then, $bc \neq (xy)a$. However, this means $a \not\vert bc$, as $xy\in\Z$. $\bot$
    \end{proof}
  \end{problem}
  %%%%%%%%%%%%%%%%%%%%%%%%%%%%%%%%%%%%%%%%%%%%%%
  %%%%%%%%%%%%%%%%%%%%%%%%%%%%%%%%%%%%%%%%%%%%%%
  \begin{theorem}
  If $a \in \mathbb{R}$ and $a>1$, then $0 < \displaystyle \frac{1}{a} <1$.
  \end{theorem}
  \begin{proof}
      Assume that $1 \le \displaystyle \frac{1}{1}$. Since $a>1$, we can divide both sides by $a$ (without reversing the inequality) to get $\displaystyle \frac{a}{a} > \displaystyle \frac{1}{a}$ so $1 > \displaystyle \frac{1}{a}$. This contradicts the assumption that 
      $1 \le \displaystyle \frac{1}{a}$. Thus it must be that $a> \displaystyle \frac{1}{a}$.
  \end{proof}

  %%%%%%%%%%%%%%%%%%%%%%%%%%%%%%%%%%%%%%%%%%%%%%
  %%%%%%%%%%%%%%%%%%%%%%%%%%%%%%%%%%%%%%%%%%%%%%
  \begin{theorem}
      If $abs{x} < \epsilon$ for every real number $\epsilon >0$, then $x=0$.
  \end{theorem}
  \begin{proof}
      Suppose that $|x|< \epsilon$ for every positive number $\epsilon$, but $x \neq 0$.
      Since $x \neq 0$, necessarily $\displaystyle \frac{|x|}{2}>0$, so in particular $|x| < \epsilon$ for the positive number $\epsilon=\displaystyle \frac{|x|}{2}>0$. This means
      $$ |x| < \displaystyle \frac{|x|}{2}.$$
      But, $|x| \neq 0$ by assumption, so we can divide both sides by $|x|$ to conclude that $1 < \frac{1}{2}$, which is a contradiction. Thus, if $|x| < \epsilon$ for every real number $\epsilon >0$, it must me the case that $x=0$.
  \end{proof}

  %%%%%%%%%%%%%%%%%%%%%%%%%%%%%%%%%%%%%%%%%%%%%%
  %%%%%%%%%%%%%%%%%%%%%%%%%%%%%%%%%%%%%%%%%%%%%%
  \begin{theorem}
      Let $a, b \in \mathbb{Z}$ where $a \equiv 1 \mod{3}$ and $b \equiv 2 \mod{3}$. Then 
      $(a+b) \equiv 0 \mod{3}.$
  \end{theorem}
  \begin{proof}
      Since $a \equiv 1 \mod{3}$ there is an integer $k$ in $\mathbb{Z}$ such that 
      $a=3k+1$. Since $b \equiv 2 \mod{3}$, we can write $b=3k+2$. Thus, $a+b=(3k+1)+(3k+2)=6k+3=3(2k+1)$,
      so $(a+b) \equiv 0 \mod{3}$.
  \end{proof}

  %%%%%%%%%%%%%%%%%%%%%%%%%%%%%%%%%%%%%%%%%%%%%%
  %%%%%%%%%%%%%%%%%%%%%%%%%%%%%%%%%%%%%%%%%%%%%%
  \begin{theorem}
      There are no integers $a, b$ for which $2a + 4b = 1$.
  \end{theorem}
  \begin{proof}
      Suppose the theorem is false, so that there are integers $a, b$ for which $2a + 4b = 1$. Dividing both sides of this equation by 2, we conclude that $a + 2b = \displaystyle \frac{1}{2}$ . Since $a and b$ are integers, $a + 2b$ is also an integer. But $\displaystyle \frac{1}{2}$ is not an integer, so this is impossible. Therefore, the theorem can not be false, so it must be true.
  \end{proof}

  %%%%%%%%%%%%%%%%%%%%%%%%%%%%%%%%%%%%%%%%%%%%%%
  %%%%%%%%%%%%%%%%%%%%%%%%%%%%%%%%%%%%%%%%%%%%%%
  \begin{theorem}
      Let $n$ be an integer. If $n^2 + 5$ is odd, then $n$ is even.
  \end{theorem}
  \begin{proof}
      Suppose, for the sake of contradiction, that $n^2+5$ is odd and $n$ is also odd. By definition,
  then, there exists  an integer $k$ so that $n^2+5=2k+1$ and $n=2k+1$. Hence we have
  $$ 2k+1=n^2+5=(2k+1)^2+1=4k^2+4k+1+5=2(2k^2+2k+3)$$.
  Therefore, $2k+1$ is even. This is clearly impossible, and hence we cannot have that $n^2 + 5$ is odd and $n$ is also odd.
  Therefore, if that $n^2 + 5$ is odd, we must have $n$ is even.
  \end{proof}
\end{document}
