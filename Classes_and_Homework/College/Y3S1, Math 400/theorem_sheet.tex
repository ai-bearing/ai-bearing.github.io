\documentclass[9pt]{extarticle}
\title{}
\author{Avinash Iyer}
\date{}
\usepackage[shortlabels]{enumitem}


%paper setup
\usepackage{geometry}
\geometry{letterpaper, portrait, margin=1in}
\usepackage{fancyhdr}

%symbols
\usepackage{amsmath}
\usepackage{amssymb}
\usepackage{amsthm}
\usepackage{mathtools}
\usepackage{hyperref}
\usepackage{gensymb}
\usepackage{multirow,array}

\newtheorem*{remark}{Remark}
\usepackage[T1]{fontenc}
\usepackage[utf8]{inputenc}

%chemistry stuff
%\usepackage[version=4]{mhchem}
%\usepackage{chemfig}

%plotting
\usepackage{pgfplots}
\usepackage{tikz}
\tikzset{middleweight/.style={pos = 0.5, fill=white}}
\tikzset{weight/.style={pos = 0.5, fill = white}}
\tikzset{lateweight/.style={pos = 0.75, fill = white}}
\tikzset{earlyweight/.style={pos = 0.25, fill=white}}

%\usepackage{natbib}

%graphics stuff
\usepackage{graphicx}
\graphicspath{ {./images/} }
\usepackage[style=numeric, backend=biber]{biblatex} % Use the numeric style for Vancouver
\addbibresource{the_bibliography.bib}
%code stuff
%when using minted, make sure to add the -shell-escape flag
%you can use lstlisting if you don't want to use minted
%\usepackage{minted}
%\usemintedstyle{pastie}
%\newminted[javacode]{java}{frame=lines,framesep=2mm,linenos=true,fontsize=\footnotesize,tabsize=3,autogobble,}
%\newminted[cppcode]{cpp}{frame=lines,framesep=2mm,linenos=true,fontsize=\footnotesize,tabsize=3,autogobble,}

%\usepackage{listings}
%\usepackage{color}
%\definecolor{dkgreen}{rgb}{0,0.6,0}
%\definecolor{gray}{rgb}{0.5,0.5,0.5}
%\definecolor{mauve}{rgb}{0.58,0,0.82}
%
%\lstset{frame=tb,
%	language=Java,
%	aboveskip=3mm,
%	belowskip=3mm,
%	showstringspaces=false,
%	columns=flexible,
%	basicstyle={\small\ttfamily},
%	numbers=none,
%	numberstyle=\tiny\color{gray},
%	keywordstyle=\color{blue},
%	commentstyle=\color{dkgreen},
%	stringstyle=\color{mauve},
%	breaklines=true,
%	breakatwhitespace=true,
%	tabsize=3
%}
% text + color boxes
\usepackage[most]{tcolorbox}
\tcbuselibrary{breakable}
\newtcolorbox{problem}[1]{colback = white, title = {#1}, breakable}
\newtcolorbox{solution}{colback = white, colframe = black!75!white, title = Solution, breakable}
%including PDFs
%\usepackage{pdfpages}
\setlength{\parindent}{0pt}
\usepackage{cancel}
\pagestyle{fancy}
\fancyhf{}
\chead{Math 400: Theorem Sheet}
\newcommand{\card}{\text{card}}
\newcommand{\ran}{\text{ran}}
\newcommand{\N}{\mathbb{N}}
\newcommand{\Q}{\mathbb{Q}}
\newcommand{\Z}{\mathbb{Z}}
\newcommand{\R}{\mathbb{R}}
\begin{document}
  \section*{Chapter 5 Theorems}%
  \begin{description}
    \item[Definitions:] A \textbf{walk} in a graph is a list of vertices such that any two consecutive vertices are adjacent to each other.

      A \textbf{trail} is a walk that does not repeat edges, but can repeat vertices.

      A \textbf{path} is a trail that does not repeat vertices.

      A \textbf{closed walk} is a walk that ends at the same vertex that it started at. A walk is open if it is not closed. A \textbf{circuit} is a closed trail, and a \textbf{cycle} is a closed path.

      An \textbf{Eulerian circuit} is a circuit that traverses all the edges of a graph. A graph is Eulerian if it contains an Eulerian circuit. An \textbf{Eulerian trail} traverses all the edges of a graph, and does not return to the same vertex it started from.
    \item[Theorem 5.1:] A connected graph $G$ is Eulerian if and only if every vertex of $G$ has even degree.
    \item[Corollary 5.2:] A connected graph $G$ contains an (open) Eulerian trail if and only if exactly two vertices of $G$ have odd degree. Furthermore, every Eulerian trail of $G$ begins at one of these odd vertices and ends at the other.
  \end{description}
  \section*{Chapter 6 Theorems}%
  \begin{description}
    \item[Definitions:] A \textbf{Hamiltonian cycle}  is a cycle that contains all the vertices of a graph. A graph is Hamiltonian if it contains a Hamiltonian cycle.
    \item[Theorem 6.2 (Dirac's Theorem):] If $G$ is a graph of order $n\geq 3$ such that $d(v) \geq n/2$ for all vertices of $G$, then $G$ is Hamiltonian.
    \item[Theorem 6.3 (Ore's Theorem):] If $G$ is a graph of order $n\geq 3$ such that $d(u) + d(v) \geq n$ for each pair $u,v$ of nonadjacent vertices of $G$, then $G$ is Hamiltonian.
    \item[Theorem 6.5:] For any graph $G$, if there is a positive integer $k$ such that deleting $k$ vertices results in a graph with more than $k$ components, then $G$ is not Hamiltonian.
  \end{description}
  \section*{Chapter 7 Theorems}%
  \begin{description}
    \item[Definitions:] A \textbf{matching} is a set of pairwise disjoint edges. A \textbf{perfect matching} is a matching that is incident on every vertex.

      The subgraph whose edges are a perfect matching is a $1$\textbf{-factor}.

      If $G$ contains $1$-factors $F_1,F_2,\dots,F_k$ such that $E(G)$ is partitioned by $E(F_1),E(F_2),\dots,E(F_k)$, then $\mathcal{F} = \{F_1,F_2,\dots,F_k\}$ is a $1$\textbf{-factorization} of $G$, and $G$ is $1$\textbf{-factorable}.

      A \textbf{bridge} is an edge upon whose deletion the number of components in a connected graph increases.

      If a graph can be decomposed into edge-disjoint Hamiltonian cycles, then the graph is \textbf{Hamiltonian-factorable}.
    \item[Theorem 7.1 (Hall's Theorem):] A sequence $(C_1,\dots,C_n)$ of $n$ nonempty finite sets has a system of distinct representatives $(s_1,\dots,s_i)$ where $s_i\in C_i$ if and only if for each subsequence $Y$, the union of the sets in $Y$ has at least as many elements as $Y$.\\

      Alternatively, if $G$ is a bipartite graph on vertices $C\sqcup S$, where $C = \{c_1,\dots,c_n\}$ and $S = \{s_1,\dots,s_m\}$, then $G$ has a $C$-perfect matching (a matching that contains every vertex in $C$) if and only if $\forall r$ where $1 \leq r \leq n$, any $r$ vertices in $C$ are adjacent to at least $r$ vertices in $S$.
    \item[Theorem 7.7 (Petersen's Theorem):] Every bridgeless $3$-regular graph contains a perfect matching.
    \item[Theorem 7.10:] For every even integer $n\geq 2$, $K_n$ is $1$-factorable.
    \item[Theorem 7.13:] For every odd integer $n\geq 3$, $K_n$ is Hamiltonian-factorable.
  \end{description}
\end{document}
