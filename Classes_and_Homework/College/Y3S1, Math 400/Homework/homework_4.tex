\documentclass[8pt]{extarticle}
\title{}
\author{Avinash Iyer}
\date{}
\usepackage[shortlabels]{enumitem}


%paper setup
\usepackage{geometry}
\geometry{letterpaper, portrait, margin=1in}
\usepackage{fancyhdr}

%symbols
\usepackage{amsmath}
\usepackage{amssymb}
\usepackage{amsthm}
\usepackage{mathtools}
\usepackage{hyperref}
\usepackage{gensymb}
\usepackage{multirow,array}

\newtheorem*{remark}{Remark}
\usepackage[T1]{fontenc}
\usepackage[utf8]{inputenc}

%chemistry stuff
%\usepackage[version=4]{mhchem}
%\usepackage{chemfig}

%plotting
\usepackage{pgfplots}
\usepackage{tikz}
\tikzset{middleweight/.style={pos = 0.5, fill=white}}
\tikzset{weight/.style={pos = 0.5, fill = white}}
\tikzset{lateweight/.style={pos = 0.75, fill = white}}
\tikzset{earlyweight/.style={pos = 0.25, fill=white}}

%\usepackage{natbib}

%graphics stuff
\usepackage{graphicx}
\graphicspath{ {./images/} }
\usepackage[style=numeric, backend=biber]{biblatex} % Use the numeric style for Vancouver
\addbibresource{the_bibliography.bib}
%code stuff
%when using minted, make sure to add the -shell-escape flag
%you can use lstlisting if you don't want to use minted
%\usepackage{minted}
%\usemintedstyle{pastie}
%\newminted[javacode]{java}{frame=lines,framesep=2mm,linenos=true,fontsize=\footnotesize,tabsize=3,autogobble,}
%\newminted[cppcode]{cpp}{frame=lines,framesep=2mm,linenos=true,fontsize=\footnotesize,tabsize=3,autogobble,}

%\usepackage{listings}
%\usepackage{color}
%\definecolor{dkgreen}{rgb}{0,0.6,0}
%\definecolor{gray}{rgb}{0.5,0.5,0.5}
%\definecolor{mauve}{rgb}{0.58,0,0.82}
%
%\lstset{frame=tb,
%	language=Java,
%	aboveskip=3mm,
%	belowskip=3mm,
%	showstringspaces=false,
%	columns=flexible,
%	basicstyle={\small\ttfamily},
%	numbers=none,
%	numberstyle=\tiny\color{gray},
%	keywordstyle=\color{blue},
%	commentstyle=\color{dkgreen},
%	stringstyle=\color{mauve},
%	breaklines=true,
%	breakatwhitespace=true,
%	tabsize=3
%}
% text + color boxes
\usepackage[most]{tcolorbox}
\tcbuselibrary{breakable}
\newtcolorbox{problem}[1]{colback = white, title = {#1}, breakable}
\newtcolorbox{solution}{colback = white, colframe = black!75!white, title = Solution, breakable}
%including PDFs
%\usepackage{pdfpages}
\setlength{\parindent}{0pt}
\usepackage{cancel}
\pagestyle{fancy}
\fancyhf{}
\rhead{Avinash Iyer}
\lhead{Math 400: Homework 4}
\newcommand{\card}{\text{card}}
\newcommand{\ran}{\text{ran}}
\newcommand{\N}{\mathbb{N}}
\newcommand{\Q}{\mathbb{Q}}
\newcommand{\Z}{\mathbb{Z}}
\newcommand{\R}{\mathbb{R}}
\begin{document}
  \begin{problem}{12}
    Suppose that $n$ teams $1,2,\dots,n$ are involved in a softball tournament in which every two teams play each other exactly once. For $n=9$ and $n=10$, set up a schedule of games that takes place during the smallest number of days so that no team plays more than one game per day.
    \tcblower
    \begin{problem}{$n=10$}
      \begin{description}[font=\normalfont\scshape]
        \item[Day 1:]\hfill
          \begin{itemize}
            \item $1$ vs $6$
            \item $2$ vs $7$
            \item $3$ vs $8$
            \item $4$ vs $9$
            \item $5$ vs $10$
          \end{itemize}
        \item[Day 2:]\hfill
          \begin{itemize}
            \item $1$ vs $7$
            \item $2$ vs $8$
            \item $3$ vs $9$
            \item $4$ vs $10$
            \item $5$ vs $6$
          \end{itemize}
        \item[Day 3:]\hfill
          \begin{itemize}
            \item $1$ vs $8$
            \item $2$ vs $9$
            \item $3$ vs $10$
            \item $4$ vs $6$
            \item $5$ vs $7$
          \end{itemize}
        \item[Day 4:]\hfill
          \begin{itemize}
            \item $1$ vs $9$
            \item $2$ vs $10$
            \item $3$ vs $6$
            \item $4$ vs $7$
            \item $5$ vs $8$
          \end{itemize}
        \item[Day 5:]\hfill
          \begin{itemize}
            \item $1$ vs $10$
            \item $2$ vs $6$
            \item $3$ vs $7$
            \item $4$ vs $8$
            \item $5$ vs $9$
          \end{itemize}
      \end{description}
    \end{problem}
  \end{problem}
\end{document}
