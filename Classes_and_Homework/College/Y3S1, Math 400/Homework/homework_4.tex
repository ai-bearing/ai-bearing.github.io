\documentclass[8pt]{extarticle}
\title{}
\author{Avinash Iyer}
\date{}
\usepackage[shortlabels]{enumitem}


%paper setup
\usepackage{geometry}
\geometry{letterpaper, portrait, margin=1in}
\usepackage{fancyhdr}

%symbols
\usepackage{amsmath}
\usepackage{amssymb}
\usepackage{amsthm}
\usepackage{mathtools}
\usepackage{hyperref}
\usepackage{gensymb}
\usepackage{multirow,array}

\newtheorem*{remark}{Remark}
\usepackage[T1]{fontenc}
\usepackage[utf8]{inputenc}

%chemistry stuff
%\usepackage[version=4]{mhchem}
%\usepackage{chemfig}

%plotting
\usepackage{pgfplots}
\usepackage{tikz}
\tikzset{middleweight/.style={pos = 0.5, fill=white}}
\tikzset{weight/.style={pos = 0.5, fill = white}}
\tikzset{lateweight/.style={pos = 0.75, fill = white}}
\tikzset{earlyweight/.style={pos = 0.25, fill=white}}

%\usepackage{natbib}

%graphics stuff
\usepackage{graphicx}
\graphicspath{ {./images/} }
\usepackage[style=numeric, backend=biber]{biblatex} % Use the numeric style for Vancouver
\addbibresource{the_bibliography.bib}
%code stuff
%when using minted, make sure to add the -shell-escape flag
%you can use lstlisting if you don't want to use minted
%\usepackage{minted}
%\usemintedstyle{pastie}
%\newminted[javacode]{java}{frame=lines,framesep=2mm,linenos=true,fontsize=\footnotesize,tabsize=3,autogobble,}
%\newminted[cppcode]{cpp}{frame=lines,framesep=2mm,linenos=true,fontsize=\footnotesize,tabsize=3,autogobble,}

%\usepackage{listings}
%\usepackage{color}
%\definecolor{dkgreen}{rgb}{0,0.6,0}
%\definecolor{gray}{rgb}{0.5,0.5,0.5}
%\definecolor{mauve}{rgb}{0.58,0,0.82}
%
%\lstset{frame=tb,
%	language=Java,
%	aboveskip=3mm,
%	belowskip=3mm,
%	showstringspaces=false,
%	columns=flexible,
%	basicstyle={\small\ttfamily},
%	numbers=none,
%	numberstyle=\tiny\color{gray},
%	keywordstyle=\color{blue},
%	commentstyle=\color{dkgreen},
%	stringstyle=\color{mauve},
%	breaklines=true,
%	breakatwhitespace=true,
%	tabsize=3
%}
% text + color boxes
\usepackage[most]{tcolorbox}
\tcbuselibrary{breakable}
\newtcolorbox{problem}[1]{colback = white, title = {#1}, breakable}
\newtcolorbox{solution}{colback = white, colframe = black!75!white, title = Solution, breakable}
%including PDFs
%\usepackage{pdfpages}
\setlength{\parindent}{0pt}
\usepackage{cancel}
\pagestyle{fancy}
\fancyhf{}
\rhead{Avinash Iyer}
\lhead{Math 400: Homework 4}
\newcommand{\card}{\text{card}}
\newcommand{\ran}{\text{ran}}
\newcommand{\N}{\mathbb{N}}
\newcommand{\Q}{\mathbb{Q}}
\newcommand{\Z}{\mathbb{Z}}
\newcommand{\R}{\mathbb{R}}
\begin{document}
  \begin{problem}{12}
    Suppose that $n$ teams $1,2,\dots,n$ are involved in a softball tournament in which every two teams play each other exactly once. For $n=9$ and $n=10$, set up a schedule of games that takes place during the smallest number of days so that no team plays more than one game per day.
    \tcblower
    In the $n=10$ case, it should take $9$ days for every team to play every other team (as there are 9 1-factorizations of $K_{10}$).\\

    In the $n=9$ case, we still have $9$ days, as we can reconceptualize the case as $n=10$ but team $10$ did not show up.
  \end{problem}
  \begin{problem}{13}
    Show that the graph in Figure 7.13 is not $1$-factorable.
    \tcblower
    In Figure 7.13, there are $10$ vertices, meaning there are $5$ edges in any given $1$-factor.\\

    Every $1$-factor must contain either $u_1v_1$ or $u_4v_4$, as otherwise the $1$-factors would be wholly contained within the pentagon components of $u$ or $v$ --- however, there can be at most $2$ $1$-factors in any given pentagon component.\\

    So, since there need to be $4$ $1$-factors for $G$ to be $1$-factorable, $G$ must not be $1$-factorable.
  \end{problem}
  \begin{problem}{15}
    Determine whether the $6$-regular graph $G$ below is Hamiltonian-factorable.
    \tcblower
    If $G$ is Hamiltonian-factorable, then there are $3$ Hamiltonian cycles. Each Hamiltonian cycle needs at least $2$ of $u_iv_i$, where $1\leq i\leq 4$.\\

    However, $3$ Hamiltonian cycles will require $6$ edges, meaning this is not possible.
  \end{problem}
\end{document}
