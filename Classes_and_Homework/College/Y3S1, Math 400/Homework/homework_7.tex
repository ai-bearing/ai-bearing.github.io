\documentclass[8pt]{extarticle}
\title{}
\author{}
\date{}
\usepackage[shortlabels]{enumitem}


%paper setup
\usepackage{geometry}
\geometry{letterpaper, portrait, margin=1in}
\usepackage{fancyhdr}
% sans serif font:
\usepackage{cmbright}
%symbols
\usepackage{amsmath}
\usepackage{amssymb}
\usepackage{amsthm}
\usepackage{mathtools}
\usepackage[hidelinks]{hyperref}
\usepackage{gensymb}
\usepackage{multirow,array}
\usepackage{multicol}

\newtheorem*{remark}{Remark}
\usepackage[T1]{fontenc}
\usepackage[utf8]{inputenc}

%chemistry stuff
%\usepackage[version=4]{mhchem}
%\usepackage{chemfig}

%plotting
\usepackage{pgfplots}
\usepackage{tikz}
\tikzset{middleweight/.style={pos = 0.5}}
%\tikzset{weight/.style={pos = 0.5, fill = white}}
%\tikzset{lateweight/.style={pos = 0.75, fill = white}}
%\tikzset{earlyweight/.style={pos = 0.25, fill=white}}

%\usepackage{natbib}

%graphics stuff
\usepackage{graphicx}
\graphicspath{ {./images/} }
\usepackage[style=numeric, backend=biber]{biblatex} % Use the numeric style for Vancouver
\addbibresource{the_bibliography.bib}
%code stuff
%when using minted, make sure to add the -shell-escape flag
%you can use lstlisting if you don't want to use minted
%\usepackage{minted}
%\usemintedstyle{pastie}
%\newminted[javacode]{java}{frame=lines,framesep=2mm,linenos=true,fontsize=\footnotesize,tabsize=3,autogobble,}
%\newminted[cppcode]{cpp}{frame=lines,framesep=2mm,linenos=true,fontsize=\footnotesize,tabsize=3,autogobble,}

%\usepackage{listings}
%\usepackage{color}
%\definecolor{dkgreen}{rgb}{0,0.6,0}
%\definecolor{gray}{rgb}{0.5,0.5,0.5}
%\definecolor{mauve}{rgb}{0.58,0,0.82}
%
%\lstset{frame=tb,
%	language=Java,
%	aboveskip=3mm,
%	belowskip=3mm,
%	showstringspaces=false,
%	columns=flexible,
%	basicstyle={\small\ttfamily},
%	numbers=none,
%	numberstyle=\tiny\color{gray},
%	keywordstyle=\color{blue},
%	commentstyle=\color{dkgreen},
%	stringstyle=\color{mauve},
%	breaklines=true,
%	breakatwhitespace=true,
%	tabsize=3
%}
% text + color boxes
\renewcommand{\mathbf}[1]{\mathbold{#1}}
\usepackage[most]{tcolorbox}
\tcbuselibrary{breakable}
\tcbuselibrary{skins}
\newtcolorbox{problem}[1]{colback=white,enhanced,title={\small #1},
          attach boxed title to top center=
{yshift=-\tcboxedtitleheight/2},
boxed title style={size=small,colback=black!60!white}, sharp corners, breakable}
%including PDFs
%\usepackage{pdfpages}
\setlength{\parindent}{0pt}
\usepackage{cancel}
\pagestyle{fancy}
\fancyhf{}
\rhead{Avinash Iyer}
\lhead{Math 400: Homework 7}
\newcommand{\card}{\text{card}}
\newcommand{\ran}{\text{ran}}
\newcommand{\N}{\mathbb{N}}
\newcommand{\Q}{\mathbb{Q}}
\newcommand{\Z}{\mathbb{Z}}
\newcommand{\R}{\mathbb{R}}
\begin{document}
  \begin{problem}{3}
    The graph $G$ in Figure 50 is connected and contains no bridges. Find a strong orientation of $G$.
  \end{problem}
  \begin{problem}{4}
    Suppose that $D$ is an orientation of a connected graph $G$ such that for each vertex $v$ of $G$, some edge is directed toward $v$ and some edge is directed away from $v$. Is $D$ a strong orientation on $G$.
    \tcblower
    Since $G$ is a connected graph, there must be a path $P$ between any two vertices $v_1$ and $v_2$. Call this path $P$, travelling along the orientation $D$. Then, the path ``exits'' $v_1$ and ``enters'' $v_2$. Suppose that there is no path from $v_2$ back to $v_1$.\\

    Then, within $G-P$ it must be the case that either $v_1$ or $v_2$ are of degree $0$, or there is a point in $G - P$ wherein the interior vertices have no edges directed ``out'' both of which would contradict the assumptions. Additionally, since there is at least one edge directed ``out'' from $v_2$ and directed ``in'' $v_1$.
  \end{problem}
  \begin{problem}{Extra Problem 1}
    Determine whether the following statements are true, and prove if so.
    \begin{enumerate}[(a)]
      \item A graph $G$ has a strong orientation if and only if $G$ is connected and has an orientation such that every pair of distinct vertices in $G$ is in a directed cycle.
      \item A graph $G$ has a strong orientation if and only if $G$ is connected and has an orientation such that every pair of distinct vertices in $G$ is in a directed circuit.
      \item A graph $G$ has a strong orientation if and only if $G$ is connected and has an orientation such that every pair of distinct vertices in $G$ is in a directed closed walk.
    \end{enumerate}
    \tcblower
    \begin{problem}{(a)}
      \begin{description}
        \item[$(\Rightarrow)$] If $G$ has a strong orientation, then for $u,v\in V(G)$ distinct, $\exists P = u,\dots,v$, and $P' = v,\dots,u\in G - P$ paths. Therefore, by appending $P$ and $P'$ together, we find that $u$ and $v$ are in a directed cycle.
        \item[$(\Leftarrow)$] Let $u,v\in V(G)$ distinct such that $u,v \in C$, where $C$ is a directed cycle. Thus, there must be no bridge between $u$ and $v$, as deletion of any edge must allow a path in $C - e$ --- so, by the condition of Robbin's Theorem, it must be the case that there is a strong orientation on $G$.
      \end{description}
    \end{problem}
    \begin{problem}{(b)}
      \begin{description}
        \item[$(\Rightarrow)$] Suppose $G$ has a strong orientation. Then, every pair of distinct vertices $u,v\in V(G)$ must have a path $P = u,\dots,v$ and a path $P' = v,\dots,u\in G-P$. By appending these paths together, we get a directed cycle, which is also a directed circuit.
        \item[$\Leftarrow$] Suppose $G$ is connected and has an orientation such that for every $u,v\in V(G)$ distinct, $u,v$ are in a directed circuit $C'$. This means there is a directed $u,v$ trail in $G$ --- and thus, a directed $u,v$ path $P$ in $G$. Similarly, in $G - P$, there must be a directed $v,u$ trail, and thus a directed $v,u$ path. So, by the conditions of Robbin's Theorem, it must be the case that $G$ has a strong orientation.
      \end{description}
    \end{problem}
    \begin{problem}{(c)}
      Since a closed walk is able to repeat edges, it is not necessarily the case that $G$ is a bridgeless graph, and thus has a strong orientation.
    \end{problem}
  \end{problem}
  \pagebreak
  \begin{problem}{Extra Problem 2}
    Let $K_n$ be a strong tournament with $n \geq 3$. 
    \begin{enumerate}[(a)]
      \item Prove that for every $j$ in $\{2,\dots,n-2\}$, $K_n$ has a directed cycle of length $1 + j$ or $1 + n - j$. 
      \item Prove that for every $j$ in $\{2,\dots,n-2\}$, $K_n$ has $n$ distinct directed cycles $C_1,\dots,C_n$ such that each $C_i$ has length $1 + j$ or $1 + n - j$.
    \end{enumerate}
    \tcblower
    \begin{problem}{(a)}
      Let $v_1,v_2,\dots,v_n,v_1$ be a Hamiltonian cycle in the strong tournament $K_n$, which we know exists by Theorem 9.5. Then, there is an edge connecting $v_1$ and $v_{j+1}$ for each $j\in \{2,\dots,n-2\}$.\\

      If $e = v_1\rightarrow v_{j+1}$, then we trace $v_1\rightarrow v_{j+1}\rightarrow \dots\rightarrow v_n\rightarrow v_1$ with length $n-j+1$. Otherwise, we have $v_1\rightarrow v_2\rightarrow \cdots \rightarrow v_j\rightarrow v_1$, with length $j+1$.
    \end{problem}
  \end{problem}
\end{document}
