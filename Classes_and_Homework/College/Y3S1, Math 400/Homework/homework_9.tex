\documentclass[8pt]{extarticle}
\title{}
\author{}
\date{}
\usepackage[shortlabels]{enumitem}


%paper setup
\usepackage{geometry}
\geometry{letterpaper, portrait, margin=1in}
\usepackage{fancyhdr}
% sans serif font:
\usepackage{cmbright}
%symbols
\usepackage{amsmath}
\usepackage{bigints}
\usepackage{amssymb}
\usepackage{amsthm}
\usepackage{mathtools}
\usepackage[hidelinks]{hyperref}
\usepackage{gensymb}
\usepackage{multirow,array}
\usepackage{multicol}

\newtheorem*{remark}{Remark}
\usepackage[T1]{fontenc}
\usepackage[utf8]{inputenc}

%chemistry stuff
%\usepackage[version=4]{mhchem}
%\usepackage{chemfig}

%plotting
\usepackage{pgfplots}
\usepackage{tikz}
\tikzset{middleweight/.style={pos = 0.5}}
%\tikzset{weight/.style={pos = 0.5, fill = white}}
%\tikzset{lateweight/.style={pos = 0.75, fill = white}}
%\tikzset{earlyweight/.style={pos = 0.25, fill=white}}

%\usepackage{natbib}

%graphics stuff
\usepackage{graphicx}
\graphicspath{ {./images/} }
\usepackage[style=numeric, backend=biber]{biblatex} % Use the numeric style for Vancouver
\addbibresource{the_bibliography.bib}
%code stuff
%when using minted, make sure to add the -shell-escape flag
%you can use lstlisting if you don't want to use minted
%\usepackage{minted}
%\usemintedstyle{pastie}
%\newminted[javacode]{java}{frame=lines,framesep=2mm,linenos=true,fontsize=\footnotesize,tabsize=3,autogobble,}
%\newminted[cppcode]{cpp}{frame=lines,framesep=2mm,linenos=true,fontsize=\footnotesize,tabsize=3,autogobble,}

%\usepackage{listings}
%\usepackage{color}
%\definecolor{dkgreen}{rgb}{0,0.6,0}
%\definecolor{gray}{rgb}{0.5,0.5,0.5}
%\definecolor{mauve}{rgb}{0.58,0,0.82}
%
%\lstset{frame=tb,
%	language=Java,
%	aboveskip=3mm,
%	belowskip=3mm,
%	showstringspaces=false,
%	columns=flexible,
%	basicstyle={\small\ttfamily},
%	numbers=none,
%	numberstyle=\tiny\color{gray},
%	keywordstyle=\color{blue},
%	commentstyle=\color{dkgreen},
%	stringstyle=\color{mauve},
%	breaklines=true,
%	breakatwhitespace=true,
%	tabsize=3
%}
% text + color boxes
\renewcommand{\mathbf}[1]{\mathbold{#1}}
\usepackage[most]{tcolorbox}
\tcbuselibrary{breakable}
\tcbuselibrary{skins}
\newtcolorbox{problem}[1]{colback=white,enhanced,title={\small #1},
          attach boxed title to top center=
{yshift=-\tcboxedtitleheight/2},
boxed title style={size=small,colback=black!60!white}, sharp corners, breakable}
%including PDFs
%\usepackage{pdfpages}
\setlength{\parindent}{0pt}
\usepackage{cancel}
\pagestyle{fancy}
\fancyhf{}
\rhead{Avinash Iyer}
\lhead{Math 400: Homework 9}
\newcommand{\card}{\text{card}}
\newcommand{\ran}{\text{ran}}
\newcommand{\N}{\mathbb{N}}
\newcommand{\Q}{\mathbb{Q}}
\newcommand{\Z}{\mathbb{Z}}
\newcommand{\R}{\mathbb{R}}
\begin{document}
  \begin{problem}{Problem 1}
    True or false: If $H$ is a minor of $G$, then $H$ is a contraction of a subgraph of $G$.
    \tcblower
    True.
  \end{problem}
  \begin{problem}{Problem 2}
    Prove each of the following.
    \begin{enumerate}[(a)]
      \item There exists an infinite family $F$ of graphs such that no graph in $F$ is a subgraph of another graph in $F$.
      \item There exists an infinite family $F$ of graphs such that no graph in $F$ is a contraction of another graph in $F$.
      \item There exists an infinite family $F$ of graphs such that no graph in $F$ is a subgraph or a contraction of another graph in $F$.
    \end{enumerate}
  \end{problem}
  \begin{problem}{Problem 3}
    Prove that the set of all planar graphs is minor-closed.
    \tcblower
    Let $G$ be any planar graph. Then, by Wagner's theorem, it must be the case that neither $K_{5}$ nor $K_{3,3}$ are minors of $G$. Therefore, any minor of $G$, $G'$, must also not have $K_{5}$ nor $K_{3,3}$ as a minor --- otherwise, we would take the steps to create $G'$, then the steps to create one of the forbidden minors, and $G$ would have the forbidden minors as a minor.\\

    Thus, since no minor of any planar graph can be non-planar, it must be the case that planarity is minor-closed.
  \end{problem}
  \begin{problem}{Problem 4}
    Let $P$ be an arbitrary set of graphs. Let $P'$ be the set of all graphs not in $P$. By the Graph Minor Theorem, $P$ has a finite subset $F$ of graphs that are minor-minimal in $P$. Similarly, $P'$ has a finite subset $F'$ of graphs that are minor minimal in $P'$. Prove that if $P$ is minor-closed, then a graph $G$ is in $P'$ if and only if $G$ has a minor in $F'$. So, if $P$ is minor-closed, then $P$ and $P'$ are both ``characterized'' by $F'$. In fact, if $P$ is minor-closed, then $F$ consists of only one graph, namely the graph with only one vertex. Why?
  \end{problem}
  \begin{problem}{Problem 5}
    A graph $G$ is apex if $G-v$ is planar for some vertex $v$ of $G$. Prove that the set of apex graphs is minor-closed.
  \end{problem}
\end{document}
