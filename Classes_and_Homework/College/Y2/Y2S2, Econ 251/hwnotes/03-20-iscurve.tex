\documentclass[10pt]{extarticle}
\title{}
\author{Avinash Iyer}
\date{}
\usepackage[shortlabels]{enumitem}

%font setup
%
\usepackage{newpxtext,eulerpx}

%paper setup
\usepackage{geometry}
\geometry{letterpaper, portrait, margin=1in}
\usepackage{fancyhdr}

%symbols
\usepackage{amsmath}
\usepackage{mathtools}
\usepackage{amssymb}
\usepackage{hyperref}
\usepackage{gensymb}

\usepackage[T1]{fontenc}
\usepackage[utf8]{inputenc}

%chemistry stuff
\usepackage[version=4]{mhchem}
\usepackage{chemfig}

%plotting
\usepackage{pgfplots}
\usepackage{tikz}

%\usepackage{natbib}

%graphics stuff
\usepackage{graphicx}
\graphicspath{ {./images/} }

%code stuff
%when using minted, make sure to add the -shell-escape flag
%you can use lstlisting if you don't want to use minted
%\usepackage{minted}
%\usemintedstyle{pastie}
%\newminted[javacode]{java}{frame=lines,framesep=2mm,linenos=true,fontsize=\footnotesize,tabsize=3,autogobble,}
%\newminted[cppcode]{cpp}{frame=lines,framesep=2mm,linenos=true,fontsize=\footnotesize,tabsize=3,autogobble,}

\usepackage{listings}
\usepackage{color}
\definecolor{dkgreen}{rgb}{0,0.6,0}
\definecolor{gray}{rgb}{0.5,0.5,0.5}
\definecolor{mauve}{rgb}{0.58,0,0.82}

\lstset{frame=tb,
	language=Java,
	aboveskip=3mm,
	belowskip=3mm,
	showstringspaces=false,
	columns=flexible,
	basicstyle={\small\ttfamily},
	numbers=none,
	numberstyle=\tiny\color{gray},
	keywordstyle=\color{blue},
	commentstyle=\color{dkgreen},
	stringstyle=\color{mauve},
	breaklines=true,
	breakatwhitespace=true,
	tabsize=3
}
% text + color boxes
\usepackage{tcolorbox}
\tcbuselibrary{breakable}
\newtcolorbox{problem}[1]{colback = white, title = {#1}, breakable}
\newtcolorbox{solution}{colback = white, colframe = black!75!white, title = Solution, breakable}
%including PDFs
\usepackage{pdfpages}
\setlength{\parindent}{0pt}

\pagestyle{fancy}
\fancyhf{}
\rhead{Avinash Iyer}
\lhead{IS Curve Derivation}
\begin{document}{
\begin{problem}{Derivation}
  \begin{align*}
    \shortintertext{Output:}
    Y_{t} &= C_t + I_t + G_t + EX_t - IM_t \\
    \frac{Y_t}{\overline{Y}_t} &= \frac{C_t + I_t + G_t + EX_t - IM_t}{\overline{Y}_t} \\
    \shortintertext{Assumptions:}
    \frac{C_t}{\overline{Y}_t} &= \overline{a}_{c} \tag*{$\left(\overline{a}_c = 0.67\right)$}\\
    \frac{G_t}{\overline{Y}_t} &= \overline{a}_g\\
    \frac{EX_t}{\overline{Y}_t} &= \overline{a}_{ex} \\
    \frac{IM_t}{\overline{Y}_t} &= \overline{a}_{im} \\
    \\
    \shortintertext{Investment:}
    \frac{I_t}{\overline{Y}_t} &= \overline{a}_i - \overline{b}(R_t - \overline{r}) \\
    \intertext{where $R_t$ represents the interest rate and $\overline{r}$ represents the marginal product of capital, and $\overline{b}$ represents the sensitivity of investment to interest rates} \\
    \frac{Y_t}{\overline{Y}_t} &= \overline{a}_c + \left(\overline{a}_i - \overline{b}(R_t - \overline{r})\right) + \overline{a}_g + \overline{a}_{ex} - \overline{a}_{im} \\
    \\
    \frac{Y_t}{\overline{Y}_t} - 1 &= \overline{a}_c + \left(\overline{a}_i - \overline{b}(R_t - \overline{r})\right) + \overline{a}_g + \overline{a}_{ex} - \overline{a}_{im} - 1 \\
    \tilde{Y}_t &= \frac{Y_t - \overline{Y}_t}{\overline{Y}_t} \\
                &= \frac{Y_t}{\overline{Y}_t} - 1 \\
                \\
    \tilde{Y}_t &= \left(\overline{a}_c + \overline{a}_i + \overline{a}_g + \overline{a}_{ex} - \overline{a}_{im} - 1\right) - \overline{b}(R_t - \overline{r}) \\
                &= \boxed{\overline{a} - \overline{b}(R_t - \overline{r})}\\
                \intertext{where $\overline{a} = \overline{a}_c + \overline{a}_i + \overline{a}_g + \overline{a}_{ex} - \overline{a}_{im} - 1$}
  \end{align*} 
  We can see from this derivation that the IS curve is downward sloping --- if $R_t$ increases, then $\tilde{Y}_t$ decreases. In long run equilibrium, we have $\tilde{Y}_t = 0$, meaning $\overline{a} = 0$ and $R_t = \overline{r}$.
\end{problem}
}\end{document}
