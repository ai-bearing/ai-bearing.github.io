\documentclass[10pt]{extarticle}
\title{}
\author{Avinash Iyer}
\date{}
\usepackage[shortlabels]{enumitem}

%font setup
%
%\usepackage[math]{anttor}

\usepackage{newpxtext,eulerpx}
%paper setup
\usepackage{geometry}
\geometry{letterpaper, portrait, margin=1in}
\usepackage{fancyhdr}

%symbols
\usepackage{amsmath}
\usepackage{mathtools}
\usepackage{amssymb}
\usepackage{hyperref}
\usepackage{gensymb}

\usepackage[T1]{fontenc}
\usepackage[utf8]{inputenc}

%chemistry stuff
\usepackage[version=4]{mhchem}
\usepackage{chemfig}

%plotting
\usepackage{pgfplots}
\usepackage{tikz}

%\usepackage{natbib}

%graphics stuff
\usepackage{graphicx}
\graphicspath{ {./images/} }

%code stuff
%when using minted, make sure to add the -shell-escape flag
%you can use lstlisting if you don't want to use minted
%\usepackage{minted}
%\usemintedstyle{pastie}
%\newminted[javacode]{java}{frame=lines,framesep=2mm,linenos=true,fontsize=\footnotesize,tabsize=3,autogobble,}
%\newminted[cppcode]{cpp}{frame=lines,framesep=2mm,linenos=true,fontsize=\footnotesize,tabsize=3,autogobble,}

\usepackage{listings}
\usepackage{color}
\definecolor{dkgreen}{rgb}{0,0.6,0}
\definecolor{gray}{rgb}{0.5,0.5,0.5}
\definecolor{mauve}{rgb}{0.58,0,0.82}

\lstset{frame=tb,
	language=Java,
	aboveskip=3mm,
	belowskip=3mm,
	showstringspaces=false,
	columns=flexible,
	basicstyle={\small\ttfamily},
	numbers=none,
	numberstyle=\tiny\color{gray},
	keywordstyle=\color{blue},
	commentstyle=\color{dkgreen},
	stringstyle=\color{mauve},
	breaklines=true,
	breakatwhitespace=true,
	tabsize=3
}
% text + color boxes
\usepackage{tcolorbox}
\tcbuselibrary{breakable}
\newtcolorbox{problem}[1]{colback = white, title = {#1}, breakable}
\newtcolorbox{solution}{colback = white, colframe = black!75!white, title = Solution, breakable}
%including PDFs
\usepackage{pdfpages}
\setlength{\parindent}{0pt}
\usepackage{qtree}
\pagestyle{fancy}
\fancyhf{}
\rhead{Avinash Iyer}
\lhead{Homework Section 1.3, Individual}
\begin{document}
  \begin{problem}{1.3.3}
    Let $u$ and $v$ be adjacent vertices in $G$. Prove that $uv$ belongs to at least $d(u) + d(v) - n(G)$ triangles in $G$.
    \tcblower
    Let $u\leftrightarrow v \in G$. In order for $uv$ to be in a triangle in $G$, $u$ and $v$ must share a common neighbor. By using the property of inclusion and exclusion, we can find the set as follows:
    \begin{align*}
      |N(u) \cup N(v)| &= |N(u)| + |N(v)| - |N(u) \cap N(v)| \\
      |N(u) \cap N(v)| &= |N(u)| + |N(v)| - |N(u) \cup N(v)| \\
                       &= d(u) + d(v) - n(G)
    \end{align*}

  \end{problem}
  \begin{problem}{1.3.4}
    Prove that the graph below is isomorphic to $Q_4$.
    \begin{center}
      \begin{tikzpicture}
        % \draw[thin, gray] (-3,-3) grid (3,3);
        % drawing points
        \filldraw (0.5,0.5) circle (2pt);
        \filldraw (-0.5,0.5) circle (2pt);
        \filldraw (0.5,-0.5) circle (2pt);
        \filldraw (-0.5,-0.5) circle (2pt);
        \filldraw (1,0.75) circle (2pt);
        \filldraw (-1,0.75) circle (2pt);
        \filldraw (-1,-0.75) circle (2pt);
        \filldraw (1,-0.75) circle (2pt);
        \filldraw (0.5,1.5) circle (2pt);
        \filldraw (-0.5,1.5) circle (2pt);
        \filldraw (-0.5,-1.5) circle (2pt);
        \filldraw (0.5,-1.5) circle (2pt);
        \filldraw (1.25,2) circle (2pt);
        \filldraw (-1.25,2) circle (2pt);
        \filldraw (1.25,-2) circle (2pt);
        \filldraw (-1.25,-2) circle (2pt);
        % drawing lines
        \draw (0.5,0.5) -- (-0.5,0.5) -- (-0.5,-0.5) -- (0.5,-0.5) -- (0.5,0.5);
        \draw (-1.25,-2) -- (1.25,-2) -- (1.25,2) -- (-1.25,2) -- (-1.25,-2);
        \draw (1,0.75) -- (-1,0.75) -- (-1,-0.75) -- (1,-0.75) -- (1,0.75);
        \draw (0.5,1.5) -- (0.5,0.5);
        \draw (-0.5,1.5) -- (-0.5,0.5);
        \draw (-0.5,-1.5) -- (-0.5,-0.5);
        \draw (0.5,-1.5) -- (0.5,-0.5);
        \draw (0.5,-1.5) -- (-0.5,-1.5);
        \draw (0.5,1.5) -- (-0.5,1.5);
        \draw (1.25,2) -- (0.5,1.5);
        \draw (-1.25,2) -- (-0.5,1.5);
        \draw (-1.25,-2) -- (-0.5,-1.5);
        \draw (1.25,-2) -- (0.5,-1.5);
        \draw (1,0.75) -- (1.25,2);
        \draw (-1,0.75) -- (-1.25,2);
        \draw (-1,-0.75) -- (-1.25,-2);
        \draw (1,-0.75) -- (1.25,-2);
        \draw (1,0.75) -- (0.5,0.5);
        \draw (1,-0.75) -- (0.5,-0.5);
        \draw (-1,-0.75) -- (-0.5,-0.5);
        \draw (-1,0.75) -- (-0.5,0.5);
        \draw (0.5,1.5) .. controls (3,5) and (3,-5) .. (0.5,-1.5);
        \draw (-0.5,1.5) .. controls (-3,5) and (-3,-5) .. (-0.5,-1.5);
      \end{tikzpicture} 
    \end{center}
    \tcblower
    We can assign tuples to the graph as follows:
    \begin{center}

      \begin{tikzpicture}
        % \draw[thin, gray] (-3,-3) grid (3,3);
        % drawing points
        \filldraw (0.5,0.5) circle (2pt);
        \node [anchor = west] at (0.5,0.5) {\tiny $0000$};
        \filldraw (-0.5,0.5) circle (2pt);
        \node [anchor = east] at (-0.5,0.5) {\tiny $0001$};
        \filldraw (0.5,-0.5) circle (2pt);
        \node [anchor = west] at (0.5,-0.5) {\tiny $0010$};
        \filldraw (-0.5,-0.5) circle (2pt);
        \node [anchor = east] at (-0.5,-0.5) {\tiny $0011$};
        \filldraw (1,0.75) circle (2pt);
        \node [anchor = west] at (1,0.75) {\tiny $0100$};
        \filldraw (-1,0.75) circle (2pt);
        \node [anchor = east] at (-1,0.75) {\tiny $0101$};
        \filldraw (-1,-0.75) circle (2pt);
        \node [anchor = east] at (-1,-0.75) {\tiny $0111$};
        \filldraw (1,-0.75) circle (2pt);
        \node [anchor = west] at (1,-0.75) {\tiny $0110$};
        \filldraw (0.5,1.5) circle (2pt);
        \node [anchor = west] at (0.5,1.5) {\tiny $1000$};
        \filldraw (-0.5,1.5) circle (2pt);
        \node [anchor = east] at (-0.5,1.5) {\tiny $1001$};
        \filldraw (-0.5,-1.5) circle (2pt);
        \node [anchor = east] at (-0.5,-1.5) {\tiny $1011$};
        \filldraw (0.5,-1.5) circle (2pt);
        \node [anchor = west] at (0.5,-1.5) {\tiny $1010$};
        \filldraw (1.25,2) circle (2pt);
        \node [anchor = west] at (1.25,2) {\tiny $1100$};
        \filldraw (-1.25,2) circle (2pt);
        \node [anchor = east] at (-1.25,2) {\tiny $1101$};
        \filldraw (1.25,-2) circle (2pt);
        \node [anchor = west] at (1.25,-2) {\tiny $1110$};
        \filldraw (-1.25,-2) circle (2pt);
        \node [anchor = east] at (-1.25,-2) {\tiny $1111$};
      \end{tikzpicture} \\ 
      \begin{tikzpicture}
        % \draw[thin, gray] (-3,-3) grid (3,3);
        % drawing points
        \filldraw (0.5,0.5) circle (2pt);
        \node [anchor = west] at (0.5,0.5) {\tiny $0000$};
        \filldraw (-0.5,0.5) circle (2pt);
        \node [anchor = east] at (-0.5,0.5) {\tiny $0001$};
        \filldraw (0.5,-0.5) circle (2pt);
        \node [anchor = west] at (0.5,-0.5) {\tiny $0010$};
        \filldraw (-0.5,-0.5) circle (2pt);
        \node [anchor = east] at (-0.5,-0.5) {\tiny $0011$};
        \filldraw (1,0.75) circle (2pt);
        \node [anchor = west] at (1,0.75) {\tiny $0100$};
        \filldraw (-1,0.75) circle (2pt);
        \node [anchor = east] at (-1,0.75) {\tiny $0101$};
        \filldraw (-1,-0.75) circle (2pt);
        \node [anchor = east] at (-1,-0.75) {\tiny $0111$};
        \filldraw (1,-0.75) circle (2pt);
        \node [anchor = west] at (1,-0.75) {\tiny $0110$};
        \filldraw (0.5,1.5) circle (2pt);
        \node [anchor = west] at (0.5,1.5) {\tiny $1000$};
        \filldraw (-0.5,1.5) circle (2pt);
        \node [anchor = east] at (-0.5,1.5) {\tiny $1001$};
        \filldraw (-0.5,-1.5) circle (2pt);
        \node [anchor = east] at (-0.5,-1.5) {\tiny $1011$};
        \filldraw (0.5,-1.5) circle (2pt);
        \node [anchor = west] at (0.5,-1.5) {\tiny $1010$};
        \filldraw (1.25,2) circle (2pt);
        \node [anchor = west] at (1.25,2) {\tiny $1100$};
        \filldraw (-1.25,2) circle (2pt);
        \node [anchor = east] at (-1.25,2) {\tiny $1101$};
        \filldraw (1.25,-2) circle (2pt);
        \node [anchor = west] at (1.25,-2) {\tiny $1110$};
        \filldraw (-1.25,-2) circle (2pt);
        \node [anchor = east] at (-1.25,-2) {\tiny $1111$};
        % drawing lines
        \draw (0.5,0.5) -- (-0.5,0.5) -- (-0.5,-0.5) -- (0.5,-0.5) -- (0.5,0.5);
        \draw (-1.25,-2) -- (1.25,-2) -- (1.25,2) -- (-1.25,2) -- (-1.25,-2);
        \draw (1,0.75) -- (-1,0.75) -- (-1,-0.75) -- (1,-0.75) -- (1,0.75);
        \draw (0.5,1.5) -- (0.5,0.5);
        \draw (-0.5,1.5) -- (-0.5,0.5);
        \draw (-0.5,-1.5) -- (-0.5,-0.5);
        \draw (0.5,-1.5) -- (0.5,-0.5);
        \draw (0.5,-1.5) -- (-0.5,-1.5);
        \draw (0.5,1.5) -- (-0.5,1.5);
        \draw (1.25,2) -- (0.5,1.5);
        \draw (-1.25,2) -- (-0.5,1.5);
        \draw (-1.25,-2) -- (-0.5,-1.5);
        \draw (1.25,-2) -- (0.5,-1.5);
        \draw (1,0.75) -- (1.25,2);
        \draw (-1,0.75) -- (-1.25,2);
        \draw (-1,-0.75) -- (-1.25,-2);
        \draw (1,-0.75) -- (1.25,-2);
        \draw (1,0.75) -- (0.5,0.5);
        \draw (1,-0.75) -- (0.5,-0.5);
        \draw (-1,-0.75) -- (-0.5,-0.5);
        \draw (-1,0.75) -- (-0.5,0.5);
        \draw (0.5,1.5) .. controls (3,5) and (3,-5) .. (0.5,-1.5);
        \draw (-0.5,1.5) .. controls (-3,5) and (-3,-5) .. (-0.5,-1.5);
      \end{tikzpicture} 
    \end{center}
  \end{problem}
  \begin{problem}{1.3.6}
    Given graphs $G$ and $H$, determine the number of components and maximum degree in $G+H$ in terms of the parameters for $G$ and $H$.
    \tcblower
    We can find the number of components in $G+H$ by summing the number of components in $G$ and the number of components in $H$.\\

    The maximum degree in $G+H$ is equal to $\textrm{max}\{\Delta(G) , \Delta(H)\}$.
  \end{problem}
  \begin{problem}{1.3.7}
    Determine the maximum number of edges in a bipartite subgraph of $P_n$, $C_n$, and $K_n$.
    \tcblower
    For the graph $P_n$, we will create a bipartition by starting at an endpoint of the path and alternating vertices in the sets $A$ and $B$. This is a bipartition since a path does not include any repeated vertices or edges, so $A$ and $B$ are independent sets. Therefore, the maximum number of edges in a bipartite subgraph of $P_n$ is the number of edges in $P_n$, which is $n-1$.\\

    For $C_n$, we have two values of the maximum number of edges in a bipartite subgraph of $C_n$:
    \begin{itemize}
      \item If $n$ is even, then $C_n$ is a bipartite graph already, meaning that the maximum number of edges in a bipartite subgraph of $C_n$ is the number of edges in $C_n$, which is $n$.
      \item If $n$ is odd, then $C_n$ is not a bipartite graph. After one edge deletion, we get that $C_n - e = P_n$, which is bipartite, so the maximum number of edges in a bipartite subgraph of $C_n$ is the number of edges in $P_n$, which is $n-1$.
    \end{itemize}
    For the graph $K_n$, there are two options for the maximum number of edges in a bipartite subgraph depending on the value of $n$:
    \begin{itemize}
      \item If $n$ is even, then the subgraph $K_{\frac{n}{2},\frac{n}{2}}$ is the maximal bipartite subgraph, meaning that the number of edges is equal to $n^2/4$. We know that $K_{\frac{n}{2},\frac{n}{2}}$ is a subgraph of $K_n$ because the vertex set is the same, and $K_n$ is complete, so any subset of edges is a subset of the edge set of $K_n$.
      \item If $n$ is odd, then the subgraph $K_{\lfloor \frac{n}{2}\rfloor, \lceil \frac{n}{2}\rceil}$ is the maximal bipartite subgraph, because $\lfloor \frac{n}{2}\rfloor + \lceil \frac{n}{2}\rceil = n$ and $K_n$ is complete. Therefore, the total number of edges is $\lfloor \frac{n^2}{4}\rfloor$.
    \end{itemize}
  \end{problem}
  \begin{problem}{1.3.26 (a)}
    Count the $6$-cycles in $Q_3$.
    \tcblower
    In order to find a $6$-cycle in $Q_3$, we select two vertices to delete and see if we can find a cycle from the graph $Q_3 - \{u,v\}$. Vertices are either adjacent, distance 2 (antipodal on the same face), or antipodal (distance 3) with each other.
    \begin{description}[font=\normalfont\scshape]
      \item[Adjacent:] If two vertices are adjacent, we can find one cycle from the remaining vertices after deletion. There are $(8)(3)/2$ sets of adjacent vertices, for a total of $12$ from this selection.
      \item[Antipodal on the same face:] If two vertices are antipodal, we cannot find a cycle from the remaining graph after deletion.
      \item[Antipodal:] If two vertices are antipodal, we can find one cycle from the remaining vertices after deletion. There are $4$ sets of antipodal vertices.
    \end{description}
    We find a total of $16$ $6$-cycles in $Q_3$.
  \end{problem}
\end{document}
