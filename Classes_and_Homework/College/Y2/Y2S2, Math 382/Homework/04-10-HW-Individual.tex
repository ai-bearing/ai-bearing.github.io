\documentclass[10pt]{extarticle}
\title{}
\author{Avinash Iyer}
\date{}
\usepackage[shortlabels]{enumitem}

%font setup
%
\usepackage{newpxtext,eulerpx}

%paper setup
\usepackage{geometry}
\geometry{letterpaper, portrait, margin=1in}
\usepackage{fancyhdr}

%symbols
\usepackage{amsmath}
\usepackage{mathtools}
\usepackage{amssymb}
\usepackage{hyperref}
\usepackage{gensymb}

\usepackage[T1]{fontenc}
\usepackage[utf8]{inputenc}

%chemistry stuff
\usepackage[version=4]{mhchem}
\usepackage{chemfig}

%plotting
\usepackage{pgfplots}
\usepackage{tikz}
\tikzset{middleweight/.style={pos = 0.5, fill=white}}
\tikzset{weight/.style={pos = 0.5, fill = white}}
\tikzset{lateweight/.style={pos = 0.75, fill = white}}
\tikzset{earlyweight/.style={pos = 0.25, fill=white}}

%\usepackage{natbib}

%graphics stuff
\usepackage{graphicx}
\graphicspath{ {./images/} }

%code stuff
%when using minted, make sure to add the -shell-escape flag
%you can use lstlisting if you don't want to use minted
%\usepackage{minted}
%\usemintedstyle{pastie}
%\newminted[javacode]{java}{frame=lines,framesep=2mm,linenos=true,fontsize=\footnotesize,tabsize=3,autogobble,}
%\newminted[cppcode]{cpp}{frame=lines,framesep=2mm,linenos=true,fontsize=\footnotesize,tabsize=3,autogobble,}

\usepackage{listings}
\usepackage{color}
\definecolor{dkgreen}{rgb}{0,0.6,0}
\definecolor{gray}{rgb}{0.5,0.5,0.5}
\definecolor{mauve}{rgb}{0.58,0,0.82}

\lstset{frame=tb,
	language=Java,
	aboveskip=3mm,
	belowskip=3mm,
	showstringspaces=false,
	columns=flexible,
	basicstyle={\small\ttfamily},
	numbers=none,
	numberstyle=\tiny\color{gray},
	keywordstyle=\color{blue},
	commentstyle=\color{dkgreen},
	stringstyle=\color{mauve},
	breaklines=true,
	breakatwhitespace=true,
	tabsize=3
}
% text + color boxes
\usepackage{tcolorbox}
\tcbuselibrary{breakable}
\newtcolorbox{problem}[1]{colback = white, title = {#1}, breakable}
\newtcolorbox{solution}{colback = white, colframe = black!75!white, title = Solution, breakable}
%including PDFs
\usepackage{pdfpages}
\setlength{\parindent}{0pt}

\pagestyle{fancy}
\fancyhf{}
\rhead{Avinash Iyer}
\lhead{Homework Section 3.3, Individual}
\begin{document}
  \begin{problem}{Our Hungarian Method}
    Use ``Our Hungarian Method'' to find a maximum matching in the bipartite graph below:
    \begin{center}
      \begin{tikzpicture}
        %drawing points
        \filldraw (1,1) circle (2pt)
                  (2,1) circle (2pt)
                  (3,1) circle (2pt)
                  (4,1) circle (2pt)
                  (5,1) circle (2pt)
                  (6,1) circle (2pt)
                  (7,1) circle (2pt);
        \filldraw (1,0) circle (2pt)
                  (2,0) circle (2pt)
                  (3,0) circle (2pt)
                  (4,0) circle (2pt)
                  (5,0) circle (2pt)
                  (6,0) circle (2pt)
                  (7,0) circle (2pt);

        % drawing labels
        \foreach \i in {1,2,...,7}[
          \node[anchor = south] at (\i,1.1){$\i$};
        ]
        \foreach \i\j in {1/a,2/b,3/c,4/d,5/e,6/f,7/g}[
          \node[anchor = north] at (\i,-0.1){$\j$};
        ]
        %drawing lines
        \draw (1,1) -- (1,0);
        \draw[very thick] (1,1) -- (2,0);
        \draw (1,1) -- (3,0);
        \draw (1,1) -- (4,0);

        \draw[very thick] (2,1) -- (1,0);
        \draw (2,1) -- (2,0);
        \draw (2,1) -- (4,0);
        \draw (2,1) -- (5,0);
        \draw (2,1) -- (7,0);

        \draw (3,1) -- (1,0);
        \draw (3,1) -- (3,0);
        \draw [very thick] (3,1) -- (5,0);
        \draw (3,1) -- (7,0);

        \draw (4,1) -- (3,0);

        \draw (5,1) -- (5,0);
        \draw [very thick] (5,1) -- (6,0);

        \draw (6,1) -- (5,0);
        \draw (6,1) -- (6,0);

        \draw[very thick](7,1) -- (3,0);
        \draw (7,1) -- (5,0);
        \draw (7,1) -- (6,0);

      \end{tikzpicture}
    \end{center}
    \tcblower
    \begin{description}[font=\scshape]
      \item[Run \#1]
      \item[Vertices Not Saturated]
    \end{description}
    \begin{align*}
      X_0 &= \{4,6\}\\
      Y_0 &= \{d,g\}
    \end{align*}
    \begin{description}[font=\scshape]
      \item[Hungarian Forest]
    \end{description}
    \begin{center}
      \begin{tikzpicture}[scale = 0.5]
        \node[anchor = south] at (1,0) {\tiny $4$};
        \filldraw (1,0) circle (2pt);
        \node[anchor = south] at (2,0) {\tiny $c$};
        \filldraw (2,0) circle (2pt);
        \draw (1,0) -- (2,0);

        \node[anchor = south] at (3,0) {\tiny $7$};
        \filldraw (3,0) circle (2pt);
        \draw[very thick] (2,0) -- (3,0);
      \end{tikzpicture}\\
      \vspace{10pt}
      \begin{tikzpicture}[scale = 0.5]
        \node[anchor = south] at (1,0) {\tiny $6$};
        \node[anchor = south] at (2,1) {\tiny $e$};
        \node[anchor = north] at (2,-1) {\tiny $f$};
        \filldraw (1,0) circle (2pt)
                  (2,1) circle (2pt)
                  (2,-1) circle (2pt);
        \draw (1,0) -- (2,1);
        \draw (1,0) -- (2,-1);
        
        \node[anchor = south] at (3,1) {\tiny $3$};
        \node[anchor = north] at (3,-1) {\tiny $5$};
        \filldraw (3,1) circle (2pt)
                  (3,-1) circle (2pt);
        \draw [very thick] (2,1) -- (3,1);
        \draw [very thick] (2,-1) -- (3,-1);

        \filldraw (4,2) circle (2pt)
                  (4,1) circle (2pt)
                  (4,0) circle (2pt);
        \node[anchor = south] at (4,2){\tiny $a$};
        \node[anchor = south] at (4,1) {\tiny $c$};
        \node[anchor = north] at (4,0) {\tiny $g$};
        \draw (3,1) -- (4,2);
        \draw (3,1) -- (4,1);
        \draw (3,1) -- (4,0);

        \filldraw (5,2) circle (2pt)
                  (5,1) circle (2pt);
        \node[anchor = south] at (5,2) {\tiny $2$};
        \node[anchor = south] at (5,1) {\tiny $7$};
        \draw[very thick] (4,2) -- (5,2);
        \draw[very thick] (4,1) -- (5,1);
      \end{tikzpicture}
    \end{center}
    \begin{description}[font=\scshape]
      \item[Flip Augmenting Path]
    \end{description}
    \begin{center}
      \begin{tikzpicture}
        %drawing points
        \filldraw (1,1) circle (2pt)
                  (2,1) circle (2pt)
                  (3,1) circle (2pt)
                  (4,1) circle (2pt)
                  (5,1) circle (2pt)
                  (6,1) circle (2pt)
                  (7,1) circle (2pt);
        \filldraw (1,0) circle (2pt)
                  (2,0) circle (2pt)
                  (3,0) circle (2pt)
                  (4,0) circle (2pt)
                  (5,0) circle (2pt)
                  (6,0) circle (2pt)
                  (7,0) circle (2pt);

        % drawing labels
        \foreach \i in {1,2,...,7}[
          \node[anchor = south] at (\i,1.1){$\i$};
        ]
        \foreach \i\j in {1/a,2/b,3/c,4/d,5/e,6/f,7/g}[
          \node[anchor = north] at (\i,-0.1){$\j$};
        ]
        %drawing lines
        \draw (1,1) -- (1,0);
        \draw[very thick] (1,1) -- (2,0);
        \draw (1,1) -- (3,0);
        \draw (1,1) -- (4,0);

        \draw[very thick] (2,1) -- (1,0);
        \draw (2,1) -- (2,0);
        \draw (2,1) -- (4,0);
        \draw (2,1) -- (5,0);
        \draw (2,1) -- (7,0);

        \draw (3,1) -- (1,0);
        \draw (3,1) -- (3,0);
        \draw (3,1) -- (5,0);
        \draw [very thick] (3,1) -- (7,0);

        \draw (4,1) -- (3,0);

        \draw (5,1) -- (5,0);
        \draw [very thick] (5,1) -- (6,0);

        \draw[very thick] (6,1) -- (5,0);
        \draw (6,1) -- (6,0);

        \draw[very thick](7,1) -- (3,0);
        \draw (7,1) -- (5,0);
        \draw (7,1) -- (6,0);
      \end{tikzpicture}
    \end{center}
    \begin{description}[font=\scshape]
      \item[Run \#2]
      \item[Vertices Not Saturated]
    \end{description}
    \begin{align*}
      X_0 &= \{4\}\\
      Y_0 &= \{d\}
    \end{align*}
    \begin{description}
      \item[Hungarian Forest]
    \end{description}
    \begin{center}
      \begin{tikzpicture}[scale = 0.5]
        \node[anchor = south] at (1,0) {\tiny $4$};
        \filldraw (1,0) circle (2pt);
        \node[anchor = south] at (2,0) {\tiny $c$};
        \filldraw (2,0) circle (2pt);
        \draw (1,0) -- (2,0);

        \node[anchor = south] at (3,0) {\tiny $7$};
        \filldraw (3,0) circle (2pt);
        \draw[very thick] (2,0) -- (3,0);

        \filldraw (4,1) circle (2pt)
                  (4,-1) circle (2pt);
        \node[anchor = south] at (4,1) {\tiny $e$};
        \node[anchor = north] at (4,-1) {\tiny $f$};
        \draw (3,0) -- (4,1);
        \draw (3,0) -- (4,-1);
        \filldraw (5,1) circle (2pt)
                  (5,-1) circle (2pt);
        \node[anchor = south] at (5,1) {\tiny $6$};
        \node[anchor = north] at (5,-1) {\tiny $5$};
        \draw[very thick] (4,1) -- (5,1);
        \draw[very thick] (4,-1) -- (5,-1);
      \end{tikzpicture}
    \end{center}
    \begin{description}[font=\scshape]
      \item[End Algorithm] Since our Hungarian Forest has no $M$-augmenting path, the following matching is a maximum matching in the graph.
    \end{description}
    \begin{center}
      \begin{tikzpicture}
        %drawing points
        \filldraw (1,1) circle (2pt)
                  (2,1) circle (2pt)
                  (3,1) circle (2pt)
                  (4,1) circle (2pt)
                  (5,1) circle (2pt)
                  (6,1) circle (2pt)
                  (7,1) circle (2pt);
        \filldraw (1,0) circle (2pt)
                  (2,0) circle (2pt)
                  (3,0) circle (2pt)
                  (4,0) circle (2pt)
                  (5,0) circle (2pt)
                  (6,0) circle (2pt)
                  (7,0) circle (2pt);

        % drawing labels
        \foreach \i in {1,2,...,7}[
          \node[anchor = south] at (\i,1.1){$\i$};
        ]
        \foreach \i\j in {1/a,2/b,3/c,4/d,5/e,6/f,7/g}[
          \node[anchor = north] at (\i,-0.1){$\j$};
        ]
        %drawing lines
        \draw (1,1) -- (1,0);
        \draw[very thick] (1,1) -- (2,0);
        \draw (1,1) -- (3,0);
        \draw (1,1) -- (4,0);

        \draw[very thick] (2,1) -- (1,0);
        \draw (2,1) -- (2,0);
        \draw (2,1) -- (4,0);
        \draw (2,1) -- (5,0);
        \draw (2,1) -- (7,0);

        \draw (3,1) -- (1,0);
        \draw (3,1) -- (3,0);
        \draw (3,1) -- (5,0);
        \draw [very thick] (3,1) -- (7,0);

        \draw (4,1) -- (3,0);

        \draw (5,1) -- (5,0);
        \draw [very thick] (5,1) -- (6,0);

        \draw[very thick] (6,1) -- (5,0);
        \draw (6,1) -- (6,0);

        \draw[very thick](7,1) -- (3,0);
        \draw (7,1) -- (5,0);
        \draw (7,1) -- (6,0);
      \end{tikzpicture}
    \end{center}
  \end{problem}
  \begin{problem}{3.3.1}
    Determine whether the following graph has a $1$-factor.
    \begin{center}
      \begin{tikzpicture}
        \filldraw (0,0) circle (2pt)
                  (0,2) circle (2pt)
                  (2,-2) circle (2pt)
                  (-2,-2) circle (2pt)
                  (0,1) circle (2pt)
                  (1,-1) circle (2pt)
                  (-1,-1) circle (2pt)
                  (1,0) circle (2pt)
                  (-1,0) circle (2pt)
                  (0,-2) circle (2pt);
        \draw (0,2) -- (1,0) -- (2,-2) -- (0,-2) -- (-2,-2) -- (-1,0) -- cycle;
        \draw (0,0) -- (0,2);
        \draw (0,0) -- (-2,-2);
        \draw (0,0) -- (2,-2);
      \end{tikzpicture}
    \end{center}
      \tcblower
      By letting $S$ be the following set of vertices, we find that $q(G-S) > |S|$, so the graph does not satisfy Tutte's condition, meaning there is no $1$-factor in the graph.
      \begin{center}
      \begin{tikzpicture}
        \filldraw (0,0) circle (2pt)
                  (0,2) circle (2pt)
                  (2,-2) circle (2pt)
                  (-2,-2) circle (2pt)
                  (0,1) circle (2pt)
                  (1,-1) circle (2pt)
                  (-1,-1) circle (2pt)
                  (1,0) circle (2pt)
                  (-1,0) circle (2pt)
                  (0,-2) circle (2pt);
          \draw (0,2) circle (5pt)
          (-2,-2) circle (5pt)
          (0,0) circle (5pt)
          (2,-2) circle (5pt);
        \draw (0,2) -- (1,0) -- (2,-2) -- (0,-2) -- (-2,-2) -- (-1,0) -- cycle;
        \draw (0,0) -- (0,2);
        \draw (0,0) -- (-2,-2);
        \draw (0,0) -- (2,-2);
      \end{tikzpicture}
    \end{center}
  \end{problem}
  \begin{problem}{3.3.2}
    Exhibit a maximum matching in the graph below, and use a result in this section to give a short proof that it has no larger matching.
    \begin{center}
      \begin{tikzpicture}
        \filldraw (0,0) circle (2pt)
                  (0,1) circle (2pt)
                  (0,2) circle (2pt)
                  (0,3) circle (2pt)
                  (0,-1) circle (2pt)
                  (0,-2) circle (2pt)
                  (0,-3) circle (2pt)
                  (1,0) circle (2pt)
                  (2,0) circle (2pt)
                  (-1,0) circle (2pt)
                  (-2,0) circle (2pt)
                  (-4,0) circle (2pt)
                  (4,0) circle (2pt)
                  (3,-1) circle (2pt)
                  (3,-2) circle (2pt)
                  (3,2) circle (2pt)
                  (-3,-2) circle (2pt)
                  (-3,2) circle (2pt);

        \draw (0,0) -- (-1,0);
        \draw (-3,2) -- (-1,0) -- (-3,-2) -- cycle;
        \draw (-3,2) -- (-2,0) -- (-3,-2) -- cycle;
        \draw (0,0) -- (1,0);
        \draw (3,2) -- (1,0) -- (3,-2) -- cycle;
        \draw (3,2) -- (2,0) -- (3,-2) -- cycle;

        \draw (4,0) -- (3,-1) -- (2,0);
        \draw (4,0) -- (3,-2);
        \draw (4,0) -- (2,0);
        \draw (4,0) -- (3,2);
        
        \draw (3,2) -- (0,2) -- (-3,2);
        \draw (3,2) -- (0,1) -- (-3,2);
        \draw (3,2) -- (0,3) -- (-3,2);

        \draw (3,-2) -- (0,-2) -- (-3,-2);
        \draw (3,-2) -- (0,-1) -- (-3,-2);
        \draw (3,-2) -- (0,-3) -- (-3,-2);
        
        \draw (-3,2) -- (-4,0) -- (-3,-2);
      \end{tikzpicture}
    \end{center}
    \tcblower
    \begin{center}
      \begin{tikzpicture}
        \filldraw (0,0) circle (2pt)
                  (0,1) circle (2pt)
                  (0,2) circle (2pt)
                  (0,3) circle (2pt)
                  (0,-1) circle (2pt)
                  (0,-2) circle (2pt)
                  (0,-3) circle (2pt)
                  (1,0) circle (2pt)
                  (2,0) circle (2pt)
                  (-1,0) circle (2pt)
                  (-2,0) circle (2pt)
                  (-4,0) circle (2pt)
                  (4,0) circle (2pt)
                  (3,-1) circle (2pt)
                  (3,-2) circle (2pt)
                  (3,2) circle (2pt)
                  (-3,-2) circle (2pt)
                  (-3,2) circle (2pt);

        \draw [very thick](0,0) -- (-1,0);
        \draw (-3,2) -- (-1,0);
        \draw(-1,0) -- (-3,-2);
        \draw (-3,-2) -- (-3,2);
        \draw (-3,2) -- (-2,0);
        \draw (-2,0) -- (-3,-2);
        \draw (0,0) -- (1,0);
        \draw [very thick] (3,2) -- (1,0);
        \draw (1,0) -- (3,-2);
        \draw (3,-2) -- (3,2);
        \draw (3,2) -- (2,0);
        \draw (2,0) -- (3,-2);
        \draw (4,0) -- (3,-1);
        \draw[very thick] (3,-1) -- (2,0);
        \draw[very thick] (4,0) -- (3,-2);
        \draw (4,0) -- (2,0);
        \draw (4,0) -- (3,2);
        
        \draw (3,2) -- (0,2);
        \draw (0,2) -- (-3,2);
        \draw (3,2) -- (0,1);
        \draw [very thick](0,1) -- (-3,2);
        \draw(3,2) -- (0,3);
        \draw (0,3) -- (-3,2);


        \draw (3,-2) -- (0,-2);
        \draw (0,-2) -- (-3,-2);

        \draw (3,-2) -- (0,-1);
        \draw[very thick] (0,-1) -- (-3,-2);

        \draw (3,-2) -- (0,-3);
        \draw (0,-3) -- (-3,-2);
        \draw (-3,2) -- (-4,0);
        \draw (-4,0) -- (-3,-2);
      \end{tikzpicture}
    \end{center}
    We can find the $S$ such that $n(G) - d(S)$ is minimized (satisfying the Berge-Tutte formula) as follows:
    \begin{center}
      \begin{tikzpicture}
        \filldraw (0,0) circle (2pt)
                  (0,1) circle (2pt)
                  (0,2) circle (2pt)
                  (0,3) circle (2pt)
                  (0,-1) circle (2pt)
                  (0,-2) circle (2pt)
                  (0,-3) circle (2pt)
                  (1,0) circle (2pt)
                  (2,0) circle (2pt)
                  (-1,0) circle (2pt)
                  (-2,0) circle (2pt)
                  (-4,0) circle (2pt)
                  (4,0) circle (2pt)
                  (3,-1) circle (2pt)
                  (3,-2) circle (2pt)
                  (3,2) circle (2pt)
                  (-3,-2) circle (2pt)
                  (-3,2) circle (2pt);
        \draw (3,2) circle (5pt)
              (3,-2) circle (5pt)
              (-3,2) circle (5pt)
              (-3,-2) circle (5pt);
        \draw (0,0) -- (-1,0);
        \draw (-3,2) -- (-1,0) -- (-3,-2) -- cycle;
        \draw (-3,2) -- (-2,0) -- (-3,-2) -- cycle;
        \draw (0,0) -- (1,0);
        \draw (3,2) -- (1,0) -- (3,-2) -- cycle;
        \draw (3,2) -- (2,0) -- (3,-2) -- cycle;

        \draw (4,0) -- (3,-1) -- (2,0);
        \draw (4,0) -- (3,-2);
        \draw (4,0) -- (2,0);
        \draw (4,0) -- (3,2);
        
        \draw (3,2) -- (0,2) -- (-3,2);
        \draw (3,2) -- (0,1) -- (-3,2);
        \draw (3,2) -- (0,3) -- (-3,2);

        \draw (3,-2) -- (0,-2) -- (-3,-2);
        \draw (3,-2) -- (0,-1) -- (-3,-2);
        \draw (3,-2) -- (0,-3) -- (-3,-2);
        
        \draw (-3,2) -- (-4,0) -- (-3,-2);
      \end{tikzpicture}
    \end{center}
    This deletion yields $10$ odd components, which means that we subtract $10 - 4 = 6$ off $n(G) = 18$ to get that there are $12$ vertices covered in the maximum matching, which we have here.
  \end{problem}
  \begin{problem}{3.3.5}
    Given graphs $G$ and $H$, determine the number of components and the maximum degree of $G\vee H$.
    \tcblower
    \begin{description}[font=\normalfont\scshape]
      \item[Components] There is $1$ component in $G\vee H$.
      \item[Maximum degree] The maximum degree of $G\vee H$ is $\max\{\Delta(G) + n(H),\Delta(H) + n(G)\}$.
    \end{description}
  \end{problem}
\end{document}
