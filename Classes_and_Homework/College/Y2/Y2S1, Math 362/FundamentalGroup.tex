
\documentclass[10pt]{extarticle}
\title{}
\author{Avinash Iyer}
\date{}

%font setup
%
%\usepackage[math]{anttor}

%paper setup
\usepackage{geometry}
\geometry{letterpaper, portrait, margin=1in}
\usepackage{fancyhdr}

%symbols
\usepackage{amsmath}
\usepackage{amssymb}
\usepackage{hyperref}
\usepackage{gensymb}

\usepackage[T1]{fontenc}
\usepackage[utf8]{inputenc}

%chemistry stuff
\usepackage[version=4]{mhchem}
\usepackage{chemfig}

%plotting
\usepackage{pgfplots}
\usepackage{tikz}

%\usepackage{natbib}

%graphics stuff
\usepackage{graphicx}
\graphicspath{ {./images/} }

%a useful command
\newcommand{\plain}[1]{\textrm{#1}}

%code stuff
%when using minted, make sure to add the -shell-escape flag
%you can use lstlisting if you don't want to use minted
%\usepackage{minted}
%\usemintedstyle{pastie}
%\newminted[javacode]{java}{frame=lines,framesep=2mm,linenos=true,fontsize=\footnotesize,tabsize=3,autogobble,}
%\newminted[cppcode]{cpp}{frame=lines,framesep=2mm,linenos=true,fontsize=\footnotesize,tabsize=3,autogobble,}

\usepackage{listings}
\usepackage{color}
\definecolor{dkgreen}{rgb}{0,0.6,0}
\definecolor{gray}{rgb}{0.5,0.5,0.5}
\definecolor{mauve}{rgb}{0.58,0,0.82}

\lstset{frame=tb,
	language=Java,
	aboveskip=3mm,
	belowskip=3mm,
	showstringspaces=false,
	columns=flexible,
	basicstyle={\small\ttfamily},
	numbers=none,
	numberstyle=\tiny\color{gray},
	keywordstyle=\color{blue},
	commentstyle=\color{dkgreen},
	stringstyle=\color{mauve},
	breaklines=true,
	breakatwhitespace=true,
	tabsize=3
}
\pagestyle{fancy}
\fancyhf{}
\rhead{Avinash Iyer}
\lhead{Notes on Fundamental Group}
\begin{document}{
\noindent Suppose we cut two disks out of a larger disk, let this equal to $X$. Pick $x_0\in A$, and let $f,g: I\rightarrow X$ be loops centered at $x_0$. Then, $f\cdot g$ is equivalent to moving along $f(I)$ for the first half of the interval $I$, then moving along $g(I)$ for the second half of the interval $I$. Note here that $f\cdot g$ and $g\cdot f$ (analogous to $f\cdot g$) need not self-intersect at $x_0$ because they are homotopic.\\
\\
Two paths are \textbf{path homotopic} if they start and end at the same point. In the example below, the points are not path homotopic.
  \begin{quote}
    Let $\alpha: I\rightarrow X$, $\beta: I\rightarrow X$, $\alpha(0)=\beta(0)=x_0$, $\alpha(1)=\beta(1)=x_1$. Then, $\alpha$ is path homotopic to $\beta$ if $\exists H:I\times I\rightarrow X$ where $H_0 = \alpha$, $H_1 = \beta$, $H_t(0)=x_0$ and $H_t(1)=x_1$.
  \end{quote}
  In this respect, it $f\cdot g$ and $g\cdot f$ are not path homotopic to each other in the above example. If $f_1$ and $f_2$ are path homotopic, then we say that they are in the same equivalence class.
  \[
    [f_1] = \{f: f\sim_{x_0}f_i\}
  \]
  Let $[f]\cdot[g] = [f\cdot g]$. We will have to show that this is well defined.\\
  \\
  Consider the following loops $f$ and $g$ on $T^2$. We know that $f$ is not path homotopic to $g$. However, we can show that $f\cdot g$ and $g\cdot f$ are path homotopic to each other with the following set of diagrams. Because $f\cdot g = g\cdot f$, we have that the group created by the equivalence class of functions is abelian.\\
  \\
  The fundamental group is the set of all equivalence classes of a loop under $\sim_p$. The product is defined as above. Because the fundamental group of $S^2\times S^1 = \mathbb{Z}$ while the fundamental group of $S^3 = \{1\}$, we know that $S^2\times S^1\not\simeq S^3$. 
}\end{document}
