
\documentclass[10pt]{extarticle}
\title{}
\author{Avinash Iyer}
\date{}

%font setup
%
%\usepackage[math]{anttor}

%paper setup
\usepackage{geometry}
\geometry{letterpaper, portrait, margin=1in}
\usepackage{fancyhdr}

%symbols
\usepackage{amsmath}
\usepackage{amssymb}
\usepackage{hyperref}
\usepackage{gensymb}

\usepackage[T1]{fontenc}
\usepackage[utf8]{inputenc}

%chemistry stuff
\usepackage[version=4]{mhchem}
\usepackage{chemfig}

%plotting
\usepackage{pgfplots}
\usepackage{tikz}

%\usepackage{natbib}

%graphics stuff
\usepackage{graphicx}
\graphicspath{ {./images/} }

%a useful command
\newcommand{\plain}[1]{\textrm{#1}}

%code stuff
%when using minted, make sure to add the -shell-escape flag
%you can use lstlisting if you don't want to use minted
%\usepackage{minted}
%\usemintedstyle{pastie}
%\newminted[javacode]{java}{frame=lines,framesep=2mm,linenos=true,fontsize=\footnotesize,tabsize=3,autogobble,}
%\newminted[cppcode]{cpp}{frame=lines,framesep=2mm,linenos=true,fontsize=\footnotesize,tabsize=3,autogobble,}

\usepackage{listings}
\usepackage{color}
\definecolor{dkgreen}{rgb}{0,0.6,0}
\definecolor{gray}{rgb}{0.5,0.5,0.5}
\definecolor{mauve}{rgb}{0.58,0,0.82}

\lstset{frame=tb,
	language=Java,
	aboveskip=3mm,
	belowskip=3mm,
	showstringspaces=false,
	columns=flexible,
	basicstyle={\small\ttfamily},
	numbers=none,
	numberstyle=\tiny\color{gray},
	keywordstyle=\color{blue},
	commentstyle=\color{dkgreen},
	stringstyle=\color{mauve},
	breaklines=true,
	breakatwhitespace=true,
	tabsize=3
}
\pagestyle{fancy}
\fancyhf{}
\rhead{Avinash Iyer}
\lhead{}
\begin{document}{
    \begin{quote}
      Show that if $X_1$ and $X_2$ are simply connected, then $X_1\times X_2$ is simply connected.
    \end{quote}
To start, we will show that $X_1\times X_2$ is path connected. Let $p = (p_1,p_2), q = (q_1,q_2)\subseteq X_1\times X_2$. By the definition of path connected, $\exists f_i:I\rightarrow X_i$ from $p_i$ to $q_i$ for $i = 1,2$. Let $f: I\rightarrow X_1\times X_2$ be given by $f(t) = (f_1(t),f_2(t))$. Then, $f$ is continuous since its ``component functions'' are continuous, and $f(0) = (p_1,p_2)$ and $f(1) = (q_1,q_2)$. Therefore, $\exists$ a continuous map from $I$ to $X_1\times X_2$, so $X_1\times X_2$ is continuous.\\
\\
Let $\ell = S^1 \rightarrow X_1\times X_2$. Then, $\pi_1\circ \ell = S^1\rightarrow X_1$, and similarly for $\pi_2$. Since $X_1$ is simply connected, $\pi_1\circ \ell$ is null homotopic. So $\exists H: S^1\times I \rightarrow X_1$ such that $H_0 = \pi_1\circ \ell$ and $H_1 = a$. Similarly, $\exists G: S^1\times I\rightarrow X_2$ such that $G_0 = \pi_2\circ \ell$ and $G_1 = b$ for constants $a,b$. Define $F: S^1\times I \rightarrow X_1 \times X_2$. Let $F(x,t) = (H(x,t),G(x,t))$. Then, $F_0 = (H(x,0),G(x,0)) = (\pi_1\circ l(x),\pi_2\circ l(x)) = l(x)$ and $F_1 = (H(x,1),G(x,1)) = (a,b)$, so $F$ is a homotopy between $\ell$ and the constant map.
}\end{document}
