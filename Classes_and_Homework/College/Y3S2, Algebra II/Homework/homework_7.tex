\documentclass[11pt]{extarticle}
\title{}
\author{}
\date{}
\usepackage[shortlabels]{enumitem}


%paper setup
\usepackage{geometry}
\geometry{letterpaper, portrait, margin=1in}
\usepackage{fancyhdr}
% sans serif font:
\usepackage{cmbright}
%symbols
\usepackage{amsmath}
\usepackage{bigints}
\usepackage{amssymb}
\usepackage{amsthm}
\usepackage{mathtools}
\usepackage{bbold}
\usepackage[hidelinks]{hyperref}
\usepackage{gensymb}
\usepackage{multirow,array}
\usepackage{multicol}

\newtheorem*{remark}{Remark}
\usepackage[T1]{fontenc}
\usepackage[utf8]{inputenc}

%chemistry stuff
%\usepackage[version=4]{mhchem}
%\usepackage{chemfig}

%plotting
\usepackage{pgfplots}
\usepackage{tikz}
\usetikzlibrary{cd}
\tikzset{middleweight/.style={pos = 0.5}}
%\tikzset{weight/.style={pos = 0.5, fill = white}}
%\tikzset{lateweight/.style={pos = 0.75, fill = white}}
%\tikzset{earlyweight/.style={pos = 0.25, fill=white}}

%\usepackage{natbib}

%graphics stuff
\usepackage{graphicx}
\graphicspath{ {./images/} }
%\usepackage[style=numeric, backend=biber]{biblatex} % Use the numeric style for Vancouver
%\addbibresource{the_bibliography.bib}
%code stuff
%when using minted, make sure to add the -shell-escape flag
%you can use lstlisting if you don't want to use minted
%\usepackage{minted}
%\usemintedstyle{pastie}
%\newminted[javacode]{java}{frame=lines,framesep=2mm,linenos=true,fontsize=\footnotesize,tabsize=3,autogobble,}
%\newminted[cppcode]{cpp}{frame=lines,framesep=2mm,linenos=true,fontsize=\footnotesize,tabsize=3,autogobble,}

%\usepackage{listings}
%\usepackage{color}
%\definecolor{dkgreen}{rgb}{0,0.6,0}
%\definecolor{gray}{rgb}{0.5,0.5,0.5}
%\definecolor{mauve}{rgb}{0.58,0,0.82}
%
%\lstset{frame=tb,
%	language=Java,
%	aboveskip=3mm,
%	belowskip=3mm,
%	showstringspaces=false,
%	columns=flexible,
%	basicstyle={\small\ttfamily},
%	numbers=none,
%	numberstyle=\tiny\color{gray},
%	keywordstyle=\color{blue},
%	commentstyle=\color{dkgreen},
%	stringstyle=\color{mauve},
%	breaklines=true,
%	breakatwhitespace=true,
%	tabsize=3
%}
% text + color boxes
%\renewcommand{\mathbf}[1]{\mathbb{#1}}
%\usepackage[most]{tcolorbox}
%\tcbuselibrary{breakable}
%\tcbuselibrary{skins}
%\newtcolorbox{problem}[1]{colback=white,enhanced,title={\small #1},
%          attach boxed title to top center=
%{yshift=-\tcboxedtitleheight/2},
%boxed title style={size=small,colback=black!60!white}, sharp corners, breakable}
%including PDFs
%\usepackage{pdfpages}
\setlength{\parindent}{0pt}
\usepackage{cancel}
%\pagestyle{fancy}
%\fancyhf{}
%\rhead{Avinash Iyer}
%\lhead{}
\newcommand{\card}{\text{card}}
\newcommand{\ran}{\text{ran}}
\newcommand{\N}{\mathbb{N}}
\newcommand{\Q}{\mathbb{Q}}
\newcommand{\Z}{\mathbb{Z}}
\newcommand{\R}{\mathbb{R}}
\newcommand{\C}{\mathbb{C}}
\newcommand{\iprod}[2]{\left\langle #1,#2\right\rangle}
\newcommand{\norm}[1]{\left\Vert #1\right\Vert}
\setcounter{secnumdepth}{0}
\begin{document}
  \begin{center}
    {\bf \Large Math 395 \\[0.1in]Homework 7 \\[0.1in]
    Due: 4/18/2024}\\[.25in]
    {\bf Name:} {Avinash Iyer}\\[0.15in]
    {\bf Collaborators:} {Antonio Cabello, Timothy Rainone, Nate Hall, Nora Manukyan, Jamie Perez-Schere} \\
  \end{center}
  \section{Problem 1}%
  We say a field $K/F$ is normal if $K$ is the splitting field of a collection of polynomials. Equivalently, every polynomial in $F[x]$ that has a root in $K$ splits into linear factors over $K$. Let $\alpha\in \R$ such that $\alpha^4 = 5$. We will show that $\Q(\alpha + i\alpha)$ is normal over $\Q(i\alpha^2)$, but $\Q(\alpha + i\alpha)$ is not normal over $\Q$.\\

  Note that $(\alpha + i\alpha)^2 = 2i\alpha^2$. Thus, $\Q(\alpha + i\alpha) = \text{Spl}_{\Q(i\alpha^2)}(x^2 - 2i\alpha^2)$, so $\Q(\alpha + i\alpha)$ is normal over $\Q(i\alpha^2)$.\\

  Suppose toward contradiction that $\Q(\alpha + i\alpha)$ is normal over $\Q$. Notice that $(\alpha + i\alpha)^4 = -20$, as is $(\alpha - i\alpha)^4$. Thus, $\alpha + i\alpha$ and $\alpha - i\alpha$ are roots of $x^4 + 20$. Since $\alpha,i,i\alpha\in \Q(\alpha + i\alpha)$, it is the case that $\Q(\alpha,i)\subseteq \Q(\alpha + i\alpha)$. However, we have
  \begin{align*}
    [\Q(\alpha,i):\Q] &= [\Q(\alpha,i):\Q(\alpha)][\Q(\alpha):\Q]\\
                      &= (2)(4)\\
                      &= 8,
  \end{align*}
  and $[\Q(\alpha + i\alpha):\Q] = 4$, as $m_{\alpha + i\alpha,\Q}(x) = x^4 + 20$. $\bot$
  \section{Problem 2}%
  The roots of $f(x) = (x^5 - 2)(x^2 - 2)$ are $\pm\sqrt{2}, \zeta_5^{k}\sqrt[5]{2}$ for $k=0,1,2,3,4$. We can see that $\Q(\zeta_5,\sqrt{2},\sqrt[5]{2})$ contains the roots of $(x^5-2)(x^2-2)$, so $\text{Spl}_{\Q}\left(f(x)\right) \subseteq \Q(\zeta_5,\sqrt{2},\sqrt[5]{2})$. Additionally, we see that $\sqrt[5]{2}\in \text{Spl}_{\Q}\left(f(x)\right)$, $\zeta_5 = \frac{\zeta_5\sqrt[5]{2}}{\sqrt[5]{2}}\in \text{Spl}_{\Q}\left(f(x)\right)$, and $\sqrt{2}\in \text{Spl}_{\Q}\left(f(x)\right)$. Thus, $\Q(\zeta_5,\sqrt[5]{2},\sqrt{2}) = \text{Spl}_{\Q}(f(x))$.\\

  For $x^6 + x^3 + 1$, we have that $x^6 + x^3 + 1 = \frac{x^9-1}{x^3 - 1}$. Therefore, the roots of $x^6 + x^3 + 1$ are $\zeta_9^{d}$, where $\gcd(d,9) = 1$ (since $9 = 3^2$, every $n\neq 0,3,6$ is a root of $x^6 + x^3 + 1$). Therefore, we can see that $x^6 + x^3 + 1 = \Phi_{9}(x)$, meaning $\text{Spl}_{\Q}(x^6 + x^3 + 1) = \Q(\zeta_9)$.
  %\section{Problem 3}%
  %For any prime $p$ and any nonzero $a\in \mathbb{F}_{p}$, we will prove that $f(x) = x^p - x + a$ is irreducible and separable over $\mathbb{F}_p$.\\

  %First, we have that $D_x(f(x)) = px^{p-1} - 1 = -1$, meaning that $\gcd(f(x),D_x(f(x))) = 1$, so $f$ is separable.\\

  %Let $\alpha$ be a root of $f$. Then, we have that $\alpha^p - \alpha + a = 0$. Notice that for $j\in \mathbb{F}_p$, $(\alpha + j)^{p} = \alpha^p + j^p = \alpha^p + j$, meaning that $(\alpha + j)^{p} - (\alpha + j) + a = 0$, so $\alpha + j$ is a root of $f$.\\

  %Suppose toward contradiction that $f$ is reducible over $\mathbb{F}_p$. Then, for some $\alpha \in \mathbb{F}_p$, we must have
  %\begin{align*}
  %  x^p - x + a &= \left(x-\alpha\right)\left(x-(\alpha + 1)\right)\left(x-(\alpha + 2)\right)\cdots\left(x-(\alpha + p - 1)\right),
  %\end{align*}
  %However, by definition, this means that there is some $k\in \mathbb{F}_{p}$ such that $\alpha + k = 0$, meaning $ a = \prod_{i=0}^{p-1}(\alpha + i) = 0$. $\bot$
  \section{Problem 6}%
  To find the subfields of $\Q(i,\sqrt[4]{3})$, we see that the basis of $\Q(i,\sqrt[4]{3})$ over $\Q$ is $\{1,\sqrt[4]{3},\sqrt{3},\sqrt[4]{27},i,i\sqrt[4]{3},i\sqrt{3},i\sqrt[4]{27}\}$, meaning $[\Q(i,\sqrt[4]{3}):\Q] = 8$. Finding subspaces of $\Q(i,\sqrt[4]{3})$, we arrive at the following diagram.
  \begin{center}
        % https://tikzcd.yichuanshen.de/#N4Igdg9gJgpgziAXAbVABwnAlgFyxMJZARgBoBmAXVJADcBDAGwFcYkQAdDgRRAF9S6TLnyEUABlIAmanSat2XbgAosASn6CQGbHgJEp02QxZtEnHsq5wAjgCccwcnw0Chu0UTLFj8sxZUsUmt7R2dXLR0RfRRyUh8aEwVzJSsOWwdkABZKJxdNd2ixElJxX1NFSyCQzJy811kYKABzeCJQADM7CABbJEkQHAgkMhAACxh6KHZIMDYaRnoAIxhGAAVhPTEQLDBsWBBEv3YpApAu3v6aIaRDccnp81n5uQrzU4Xl1Y2PGJ29rAHNznbp9RB3G6IOL3KYzAgvRYrdabTzmXb7F5JfynYEXMEQ4aILI0Cawp7ww4gRHfFF-dGAzHHd5nPEja6E6Gkx7gCmfJE-IrsekHI5vEA4rSsonspAAVj5NN+22FjLFHxh3OeLNBSGhkPlVK+yKVQoBIteyXFlK5cLm2suiFG+oVxsFaLNqstWWtD1tbFxOsQA0h0Kx7G9LoFW1NGJ9ZJ5doDDuDhOJIBWYEeAFpyANqa7o+7Y6KvXHNRSk-iZYgDTbyXbI7TlR7KWHzN6+JQ+EA
    \begin{tikzcd}
& {\Q(i,\sqrt[4]{3})}                                                                                                                                                  &                                                                                                                              &                                                         \\
   & {\Q(i,\sqrt{3})} \arrow[u, "2" description, no head]& & {\Q(\sqrt[4]{3})} \arrow[llu, "2" description, no head] \\
    \Q(i) \arrow[ru, "2" description, no head] \arrow[ruu, "4" description, no head] &                                                                                                                                                                      & \Q(\sqrt{3}) \arrow[lu, "2" description, no head] \arrow[ru, "2" description, no head] \arrow[luu, "4" description, no head] &                                                         \\
     & \Q \arrow[lu, "2" description, no head] \arrow[ru, "2" description, no head] \arrow[uu, "4" description, no head] \arrow[rruu, "4" description, no head, bend right] &                                                                                                                              &                                                        
    \end{tikzcd}
  \end{center}
  For any subfield $\Q\subseteq F \subseteq \Q(i,\sqrt[4]{3})$, it must be the case that $[F:\Q] = 2^k$ for some $k = 0,1,2,3$. Therefore, it must be the case that all subfields are of degree $1,2,4,8$.\\

  Suppose there is any subfield $\Q\subseteq E\subseteq \Q(i)$. Then, it must be the case that $[E:\Q] = 1$ or $[E:\Q] = 2$, meaning $E = \Q$ or $E = \Q(i)$. The same argument applies for all degree $2$ extensions in the above diagram. 
  \section{Problem 7}%
  Let $n= p^k m$ with $m$ relatively prime to prime $p$. We will show that there are $m$ distinct $n$th roots of unity over a field with characteristic $p$.\\

  Let $\zeta_n$ be an $n$th root of unity. Then, $\zeta_n^{n} - 1 = 0$, meaning
  \begin{align*}
    \zeta_n^{p^km}-1 &= 0\\
    \left(\zeta_n^{m}\right)^{p^k} - 1 &= 0\\
    \left(\zeta_n^m\right)^{p^k}- 1^{p^k} &= 0\\
    \left(\zeta_n^{m} - 1\right)^{p^k} &= 0.
  \end{align*}
  Since $m \neq p^{\ell}\alpha$, as $m$ and $p$ are relatively prime, it must be the case that, the $m$ roots of unity are distinct, and each $n$th root of unity is an $m$th root of unity, meaning there are $m$ distinct $n$th roots of unity.
\end{document}
