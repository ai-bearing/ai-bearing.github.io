\documentclass[8pt]{extarticle}
\title{}
\author{}
\date{}
\usepackage[shortlabels]{enumitem}


%paper setup
\usepackage{geometry}
\geometry{letterpaper, portrait, margin=1in}
\usepackage{fancyhdr}
% sans serif font:
\usepackage{cmbright}
%symbols
\usepackage{amsmath}
\usepackage{bigints}
\usepackage{amssymb}
\usepackage{amsthm}
\usepackage{mathtools}
\usepackage{bbold}
\usepackage[hidelinks]{hyperref}
\usepackage{gensymb}
\usepackage{multirow,array}
\usepackage{multicol}

\newtheorem*{remark}{Remark}
\usepackage[T1]{fontenc}
\usepackage[utf8]{inputenc}

%chemistry stuff
%\usepackage[version=4]{mhchem}
%\usepackage{chemfig}

%plotting
\usepackage{pgfplots}
\usepackage{tikz}
\usetikzlibrary{cd}
\tikzset{middleweight/.style={pos = 0.5}}
%\tikzset{weight/.style={pos = 0.5, fill = white}}
%\tikzset{lateweight/.style={pos = 0.75, fill = white}}
%\tikzset{earlyweight/.style={pos = 0.25, fill=white}}

%\usepackage{natbib}

%graphics stuff
\usepackage{graphicx}
\graphicspath{ {./images/} }
\usepackage[style=numeric, backend=biber]{biblatex} % Use the numeric style for Vancouver
\addbibresource{the_bibliography.bib}
%code stuff
%when using minted, make sure to add the -shell-escape flag
%you can use lstlisting if you don't want to use minted
%\usepackage{minted}
%\usemintedstyle{pastie}
%\newminted[javacode]{java}{frame=lines,framesep=2mm,linenos=true,fontsize=\footnotesize,tabsize=3,autogobble,}
%\newminted[cppcode]{cpp}{frame=lines,framesep=2mm,linenos=true,fontsize=\footnotesize,tabsize=3,autogobble,}

%\usepackage{listings}
%\usepackage{color}
%\definecolor{dkgreen}{rgb}{0,0.6,0}
%\definecolor{gray}{rgb}{0.5,0.5,0.5}
%\definecolor{mauve}{rgb}{0.58,0,0.82}
%
%\lstset{frame=tb,
%	language=Java,
%	aboveskip=3mm,
%	belowskip=3mm,
%	showstringspaces=false,
%	columns=flexible,
%	basicstyle={\small\ttfamily},
%	numbers=none,
%	numberstyle=\tiny\color{gray},
%	keywordstyle=\color{blue},
%	commentstyle=\color{dkgreen},
%	stringstyle=\color{mauve},
%	breaklines=true,
%	breakatwhitespace=true,
%	tabsize=3
%}
% text + color boxes
\renewcommand{\mathbf}[1]{\mathbb{#1}}
\usepackage[most]{tcolorbox}
\tcbuselibrary{breakable}
\tcbuselibrary{skins}
\newtcolorbox{problem}[1]{colback=white,enhanced,title={\small #1},
          attach boxed title to top center=
{yshift=-\tcboxedtitleheight/2},
boxed title style={size=small,colback=black!60!white}, sharp corners, breakable}
%including PDFs
%\usepackage{pdfpages}
\setlength{\parindent}{0pt}
\usepackage{cancel}
%\pagestyle{fancy}
%\fancyhf{}
%\rhead{Avinash Iyer}
%\lhead{}
\newcommand{\card}{\text{card}}
\newcommand{\ran}{\text{ran}}
\newcommand{\N}{\mathbb{N}}
\newcommand{\Q}{\mathbb{Q}}
\newcommand{\Z}{\mathbb{Z}}
\newcommand{\R}{\mathbb{R}}
\newcommand{\C}{\mathbb{C}}
\newcommand{\iprod}[2]{\left\langle #1,#2\right\rangle}
\newcommand{\norm}[1]{\left\Vert #1\right\Vert}
\setcounter{secnumdepth}{0}
\begin{document}
  \begin{center}
    {\bf \Large Math 395 \\[0.1in]Homework 5 \\[0.1in]
    Due: 3/5/2024}\\[.25in]
    {\bf Name:} {Avinash Iyer}\\[0.15in]
    {\bf Collaborators:} {Gianluca Crescenzo, Antonio Cabello, Nate Hall} \\
  \end{center}
  \section{Problem 1}%
  Let $R$ be a commutative ring with identity. Let $\Sigma$ be a multiplicative subset of $R$. Let $\mathcal{F} = \{(r,d)\mid r\in R, d\in \Sigma\}$. We will show by giving an explicit example that the relation $(r_1,d_1)\sim (r_2,d_2)$ if $r_1d_2 - r_2d_1 = 0$ is not necessarily an equivalence relation if $R$ is not an integral domain.\\

  Let $R = \Z/6\Z$, and consider the multiplicatively closed set $\{1,3\}$. Then,
  \begin{align*}
    \mathcal{F} &= \{(0,1),(0,3),(1,1),(1,3),(2,1),(2,3),(3,1),(3,3),(4,1),(4,3),(5,1),(5,3)\}.
  \end{align*}
  With the given equivalence relation, we can see that $(2,1)\sim (0,3)$, as $2\cdot 3 - 0\cdot 3 \equiv 0$ modulo 6, and $(0,3)\sim (2,3)$, as $0\cdot 3 - 2\cdot 3 \equiv 0$ modulo 3, but $(2,1)\nsim (2,3)$, as $2\cdot 3 - 1\cdot 2 \not\equiv 0$ modulo 3. Thus, the relation is not transitive, and is not an equivalence relation.
  \section{Problem 3}%
  Let $R_1$ and $R_2$ be rings with identity. We will show that if $I$ is an ideal in $R_1\times R_2$, then $I$ is of the form $I_1\times I_2$, where $I_j$ is an ideal in $R_j$.\\

  Let $I$ be an ideal in $R_1\times R_2$. Then, for any $(a,b),(c,d)\in I$ and any $(x,y)\in R$, then $(a,b)-(c,d) = (a-b,c-d)\in I, (a,b)(x,y) =(ax,by)\in I$, and $(x,y)(a,b) = (xa,yb)\in I$.\\

  Define $I_1 = \pi_1(I)$ and $I_2 = \pi_2(I)$. We will show that $I_1$ and $I_2$ are ideals in $R_1$ and $R_2$ respectively, with $I = I_1\times I_2$. Let $a,b\in I_1$ and $x\in R_1$. Then, $a = \pi_1((a,k))$ and $b = \pi_1((b,\ell))$for some $(a,k),(b,\ell)\in I$. Then, $a-b = \pi_1((a,k)) - \pi_1((b,\ell)) $, and since the projection map is a homomorphism, this is equivalent to $\pi_1((a-b),(k-\ell))$. Since $I$ is closed under subtraction, $(a-b,k-\ell)\in I$, so $a-b\in I_1$. Similarly, for $a,b\in I_2$, $a-b\in I_2$.\\

  Let $x\in I_1,r\in R_1$. Then, $x = \pi_1((x,t))$ for some $(x,t)\in I$, and $r = \pi_1((r,s))$ for some $(r,s)\in R_1\times R_2$. So,
  \begin{align*}
    xr &= \pi_1((x,t))\pi_1((r,s))\\
       &= \pi_1((x,t)(r,s))\\
       &= \pi_1((xr,ts)),\\
       \intertext{and since $(xr,ts)\in I$,}
       &\in I_1.\\
       \intertext{Similarly,}
    rx &= \pi_1((r,s))\pi_1(x,t)\\
       &= \pi_1((rx,st))\\
       \intertext{and since $(rx,st)\in I$,}
       &\in I_1.
  \end{align*}
  Similar results hold for $I_2$. Therefore, $I_1$ and $I_2$ are ideals.\\

  Clearly, $I\subseteq I_1\times I_2$.\\

  Let $(i_1,i_2)\in I_1\times I_2$. Then, $(i_1,b)\in I$ for some $b\in R_2$ and $(a,i_2)\in I$ for some $a\in R_1$. By the definition of ideal, $(1,0)(i_1,b)\in I$ and $(0,1)\times(a,i_2)\in I$, so $(i_1,0)\in I$ and $(0,i_2)\in I$. Since $I$ is an ideal, $(i_1,i_2)\in I$. Thus, $I_1\times I_2\subseteq I$.
  \section{Problem 8}%
  Let $V = \R^{n}$, and let $v = (a_1,\dots,a_n)\in V$ be fixed. We will prove that the collection $(x_1,\dots,x_n) \in V$ with $a_1x_1 + \cdots + a_nx_n = 0$ is a subspace of $V$.\\

  Let $T: \R^{n}\rightarrow \R$ be defined by $T(y) = \iprod{v}{y}$, where $\iprod{v}{y}$ denotes the traditional inner product on $\R^n$. Then, $\ker T = \{y\mid \iprod{v}{y} = 0\}$, which is precisely the collection of $(x_1,\dots,x_n)\in \R^n$ such that $a_1x_1 + \cdots + a_nx_n = 0$. For any $y_1,y_2\in \ker(T)$ and $\alpha\in \R$,
  \begin{align*}
    T(\alpha y_1 + y_2) &= \alpha T(y_1) + T(y_2)\\
                        &= 0,
  \end{align*}
  meaning $\ker T$ is a subspace. We know from a previous result that $\text{dim}_{\R}(\ker T) + \text{dim}_{\R}(\R) = \text{dim}_{\R}(\R^n)$, so $\text{dim}_{\R}(\ker T) = n-1$.\\

  To find a basis for $\ker T$, take $w_1$ a nonzero vector such that $\iprod{w_1}{v} = 0$. From there, select $w_2\notin \text{span}(w_1)\cup \text{span}(v)$, and iteratively for $w_3,\dots,w_{n-1}$. The collection $\{w_i\}_{i=1}^{n-1}$ is obviously spanning for $\ker T$. To show that it is linearly independent, let
  \begin{align*}
    c_1w_1 + \cdots + c_{n-1}w_{n-1} = 0.
  \end{align*}
  Since $0\in \text{span}(v)$, so too must $\sum_{i=1}^{n-1} c_iw_i$ (as, by definition), which is only the case if $c_1 = \cdots = c_{n-1} = 0$. Therefore, the set $\{w_i\}$ is linearly independent.
  \section{Problem 9}%
  Let $T: \R^4 \rightarrow \R$ be the linear transformation so that
  \begin{align*}
    T((1,0,0,0)) &= 1\\
    T((1,-1,0,0)) &= 0\\
    T((1,-1,1,0)) &= 1\\
    T((1,-1,1,-1)) &= 0.
  \end{align*}
  To determine $T((a,b,c,d))$ for any $(a,b,c,d)\in \R^4$, we will first convert the given basis vectors into the standard basis.
  \begin{align*}
    T((1,0,0,0)) &= 1\\
    T((0,1,0,0)) &= T((1,0,0,0) - (1,-1,0,0))\\
                 &= T((1,0,0,0)) - T((1,-1,0,0)) \\
                 &= 1\\
    T((0,0,1,0)) &= T((1,-1,1,0) - (1,-1,0,0))\\
                 &= T((1,-1,1,0)) - T((1,-1,0,0))\\
                 &= 1\\
    T((0,0,0,1)) &= T((1,-1,1,0) - T((1,-1,1,-1)))\\
                 &= T((1,-1,1,0)) - T((1,-1,1,-1))\\
                 &= 1.
  \end{align*}
  Therefore,
  \begin{align*}
    T((a,b,c,d)) &= aT((1,0,0,0)) + bT((0,1,0,0)) + cT((0,0,1,0)) + dT((0,0,0,1))\\
                 &= a + b + c + d.
  \end{align*}
\end{document}
