\documentclass[10pt]{extarticle}
\title{}
\author{}
\date{}
\usepackage[shortlabels]{enumitem}


%paper setup
\usepackage{geometry}
\geometry{letterpaper, portrait, margin=1in}
\usepackage{fancyhdr}
% sans serif font:
%\usepackage{cmbright}
%symbols
\usepackage{amsmath}
\usepackage{bigints}
\usepackage{amssymb}
\usepackage{amsthm}
\usepackage{mathtools}
\usepackage{bbold}
\usepackage[hidelinks]{hyperref}
\usepackage{gensymb}
\usepackage{multirow,array}
\usepackage{multicol}

\newtheorem*{remark}{Remark}
\usepackage[T1]{fontenc}
\usepackage[utf8]{inputenc}

%chemistry stuff
%\usepackage[version=4]{mhchem}
%\usepackage{chemfig}

%plotting
\usepackage{pgfplots}
\usepackage{tikz}
\tikzset{middleweight/.style={pos = 0.5}}
%\tikzset{weight/.style={pos = 0.5, fill = white}}
%\tikzset{lateweight/.style={pos = 0.75, fill = white}}
%\tikzset{earlyweight/.style={pos = 0.25, fill=white}}

%\usepackage{natbib}

%graphics stuff
\usepackage{graphicx}
\graphicspath{ {./images/} }
\usepackage[style=numeric, backend=biber]{biblatex} % Use the numeric style for Vancouver
\addbibresource{the_bibliography.bib}
%code stuff
%when using minted, make sure to add the -shell-escape flag
%you can use lstlisting if you don't want to use minted
%\usepackage{minted}
%\usemintedstyle{pastie}
%\newminted[javacode]{java}{frame=lines,framesep=2mm,linenos=true,fontsize=\footnotesize,tabsize=3,autogobble,}
%\newminted[cppcode]{cpp}{frame=lines,framesep=2mm,linenos=true,fontsize=\footnotesize,tabsize=3,autogobble,}

%\usepackage{listings}
%\usepackage{color}
%\definecolor{dkgreen}{rgb}{0,0.6,0}
%\definecolor{gray}{rgb}{0.5,0.5,0.5}
%\definecolor{mauve}{rgb}{0.58,0,0.82}
%
%\lstset{frame=tb,
%	language=Java,
%	aboveskip=3mm,
%	belowskip=3mm,
%	showstringspaces=false,
%	columns=flexible,
%	basicstyle={\small\ttfamily},
%	numbers=none,
%	numberstyle=\tiny\color{gray},
%	keywordstyle=\color{blue},
%	commentstyle=\color{dkgreen},
%	stringstyle=\color{mauve},
%	breaklines=true,
%	breakatwhitespace=true,
%	tabsize=3
%}
% text + color boxes
\usepackage[most]{tcolorbox}
\tcbuselibrary{breakable}
\tcbuselibrary{skins}
\newtcolorbox{problem}[1]{colback=white,enhanced,title={\small #1},
          attach boxed title to top center=
{yshift=-\tcboxedtitleheight/2},
boxed title style={size=small,colback=black!60!white}, sharp corners, breakable}
%including PDFs
%\usepackage{pdfpages}
\setlength{\parindent}{0pt}
\usepackage{cancel}
%\pagestyle{fancy}
%\fancyhf{}
%\rhead{Avinash Iyer}
%\lhead{}
\newcommand{\card}{\text{card}}
\newcommand{\ran}{\text{ran}}
\newcommand{\N}{\mathbb{N}}
\newcommand{\Q}{\mathbb{Q}}
\newcommand{\Z}{\mathbb{Z}}
\newcommand{\R}{\mathbb{R}}
\newcommand{\C}{\mathbb{C}}
\newcommand{\iprod}[2]{\left\langle #1,#2\right\rangle}
\newcommand{\norm}[1]{\left\Vert #1\right\Vert}
\setcounter{secnumdepth}{0}
\begin{document}
  \begin{center}
    {\bf \Large Math 395 \\[0.1in]Homework 2 \\[0.1in]
    Due: 2/8/2024}\\[.25in]
    {\bf Name:} {Avinash Iyer}\\[0.15in]
    {\bf Collaborators:} {} \\
  \end{center}
  \section{Problem 2}%
  Let $I,J$ be ideals in ring $R$. Define $I+J = \{i+j\mid i\in I,j\in J\}$. This is referred to as the sum of the ideals.
  \begin{enumerate}[(a)]
    \item We will prove that $I + J$ is an ideal in $R$ that contains $I$ and $J$.\\

      To start, since $I$ and $J$ are ideals in $R$, $I$ and $J$ are each subrings of $R$, meaning both $I$ and $J$ contain $0_R$. Therefore, taking $j = 0_R$, we find that $\{i + 0_R\mid i\in I\}\subseteq I+J$, and similarly, taking $i = 0_R$, we find that $\{0_R + j\mid j\in J\}\subseteq I+J $. These sets are, respectively, $I$ and $J$, meaning $I$ and $J$ are both subsets of $I+J$.\\

      We will show that $I+J$ is an ideal in $R$ by showing that $I+J$ is a subring that is closed under multiplication by all elements of $R$. Firstly, $I+J$ is non-empty since, as exhibited earlier, both $I$ and $J$ are subrings, meaning $0_R\in I$ and $0_R\in J$, so $0_R + 0_R = 0_R\in I+J$. Let $x,y\in I+J$. Then, $x = x_i + x_j$ and $y = y_i + y_j$ for some $x_i,y_i\in I$ and $x_j,y_j\in J$. Then,
      \begin{align*}
        x-y &= (x_i + x_j) - (y_i + y_j)\\
            &= (x_i - y_i) + (x_j - y_j),
      \end{align*}
      which is an element of $I + J$. Similarly, 
      \begin{align*}
        xy &= (x_i + x_j) + (y_i + y_j)\\
           &= (x_iy_i) + (x_jy_j + x_iy_j + x_jy_i).
      \end{align*}
      Since $x_iy_i\in I$, as $I$ is a subring, and $x_jy_j\in J$, as $J$ is a subring, as well as $x_iy_j\in J$ and $x_jy_i\in J$ as $J$ is an ideal, we have that $x_jy_j + x_iy_j + x_jy_i\in J$, so $xy\in I+J$.\\

      Finally, we will show that $I+J$ is closed under multiplication by elements from $R$. Let $r\in R$, $a\in I+J$. Then, $a = a_i + a_j$ for $a_i\in I$ and $a_j\in J$. So,
      \begin{align*}
        ra &= r(a_i + a_j)\\
           &= ra_i + ra_j,
           \shortintertext{and}
        ar &= (a_i + a_j)r\\
           &= a_ir + a_jr,
      \end{align*}
      and since $I$ and $J$ are both ideals, $ra_i,a_ir\in I$ and $ra_j,a_jr\in J$, so $ar,ra\in I+J$.\\

      Therefore, $I+J$ is an ideal that contains $I$ and $J$.
    \item Let $a,b\in\mathbf{Z}$. We will show that $a\mathbf{Z} + b\mathbf{Z} = \gcd(a,b)\mathbf{Z}$.\\

      By Bezout's identity, it is the case that there are integers $x$ and $y$ such that $xa + yb = \gcd(a,b)$. Since $xa\in a\mathbf{Z}$, and $yb\in b\mathbf{Z}$, as $a\mathbf{Z}$ and $b\mathbf{Z}$ are ideals in $\mathbf{Z}$, it is the case that $xa + yb \in a\mathbf{Z} + b\mathbf{Z}$. Therefore, $\gcd(a,b)\mathbf{Z}\subseteq a\mathbf{Z} + b\mathbf{Z}$.
  \end{enumerate}
\end{document}
