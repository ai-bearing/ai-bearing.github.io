\documentclass[11pt]{extarticle}
\title{}
\author{}
\date{}
\usepackage[shortlabels]{enumitem}


%paper setup
\usepackage{geometry}
\geometry{letterpaper, portrait, margin=1in}
\usepackage{fancyhdr}
% sans serif font:
\usepackage{cmbright}
%symbols
\usepackage{amsmath}
\usepackage{bigints}
\usepackage{amssymb}
\usepackage{amsthm}
\usepackage{mathtools}
\usepackage{bbold}
\usepackage[hidelinks]{hyperref}
\usepackage{gensymb}
\usepackage{multirow,array}
\usepackage{multicol}

\newtheorem*{remark}{Remark}
\usepackage[T1]{fontenc}
\usepackage[utf8]{inputenc}

%chemistry stuff
%\usepackage[version=4]{mhchem}
%\usepackage{chemfig}

%plotting
\usepackage{pgfplots}
\usepackage{tikz}
\usetikzlibrary{cd}
\tikzset{middleweight/.style={pos = 0.5}}
%\tikzset{weight/.style={pos = 0.5, fill = white}}
%\tikzset{lateweight/.style={pos = 0.75, fill = white}}
%\tikzset{earlyweight/.style={pos = 0.25, fill=white}}

%\usepackage{natbib}

%graphics stuff
\usepackage{graphicx}
\graphicspath{ {./images/} }
\usepackage[style=numeric, backend=biber]{biblatex} % Use the numeric style for Vancouver
\addbibresource{the_bibliography.bib}
%code stuff
%when using minted, make sure to add the -shell-escape flag
%you can use lstlisting if you don't want to use minted
%\usepackage{minted}
%\usemintedstyle{pastie}
%\newminted[javacode]{java}{frame=lines,framesep=2mm,linenos=true,fontsize=\footnotesize,tabsize=3,autogobble,}
%\newminted[cppcode]{cpp}{frame=lines,framesep=2mm,linenos=true,fontsize=\footnotesize,tabsize=3,autogobble,}

%\usepackage{listings}
%\usepackage{color}
%\definecolor{dkgreen}{rgb}{0,0.6,0}
%\definecolor{gray}{rgb}{0.5,0.5,0.5}
%\definecolor{mauve}{rgb}{0.58,0,0.82}
%
%\lstset{frame=tb,
%	language=Java,
%	aboveskip=3mm,
%	belowskip=3mm,
%	showstringspaces=false,
%	columns=flexible,
%	basicstyle={\small\ttfamily},
%	numbers=none,
%	numberstyle=\tiny\color{gray},
%	keywordstyle=\color{blue},
%	commentstyle=\color{dkgreen},
%	stringstyle=\color{mauve},
%	breaklines=true,
%	breakatwhitespace=true,
%	tabsize=3
%}
% text + color boxes
\renewcommand{\mathbf}[1]{\mathbb{#1}}
\usepackage[most]{tcolorbox}
\tcbuselibrary{breakable}
\tcbuselibrary{skins}
\newtcolorbox{problem}[1]{colback=white,enhanced,title={\small #1},
          attach boxed title to top center=
{yshift=-\tcboxedtitleheight/2},
boxed title style={size=small,colback=black!60!white}, sharp corners, breakable}
%including PDFs
%\usepackage{pdfpages}
\setlength{\parindent}{0pt}
\usepackage{cancel}
\pagestyle{fancy}
\fancyhf{}
\rhead{Avinash Iyer}
\lhead{}
\newcommand{\card}{\text{card}}
\newcommand{\ran}{\text{ran}}
\newcommand{\N}{\mathbb{N}}
\newcommand{\Q}{\mathbb{Q}}
\newcommand{\Z}{\mathbb{Z}}
\newcommand{\R}{\mathbb{R}}
\newcommand{\C}{\mathbb{C}}
\newcommand{\iprod}[2]{\left\langle #1,#2\right\rangle}
\newcommand{\norm}[1]{\left\Vert #1\right\Vert}
\setcounter{secnumdepth}{0}
\begin{document}
  \begin{center}
    {\bf \Large Math 395 \\[0.1in]Homework 6 \\[0.1in]
    Due: 3/28/2024}\\[.25in]
    {\bf Name:} {Avinash Iyer}\\[0.15in]
    {\bf Collaborators:} {Antonio Cabello, Nora Manukyan, Nate Hall} \\
  \end{center}
  \section{Problem 2}%
  We will show that $\{1,\sqrt{5},\sqrt{7},\sqrt{35}\}$ is linearly independent.\\

  Suppose $a + b\sqrt{5} + c\sqrt{7} + d\sqrt{35} = 0$. Then,
  \begin{align*}
    \left(a+d\sqrt{35}\right)^2 &= \left(b\sqrt{5} + c\sqrt{7}\right)^2\\
    a^2 + 35d^2 - 5b^2 - 7c^2 &= 2\sqrt{35}\left(bc-ad\right).
  \end{align*}
  Since $2\sqrt{35}\notin \Q$ and $a,b,c,d\in \Q$, this equation is only true if $bc-ad = 0$, so $bc = ad$.
  \begin{description}
    \item[Case 1:] Suppose $d=0$ and $a=0$. Then,
      \begin{align*}
        7c^2 + 5b^2 = 0,
      \end{align*}
      which is only true if $b=c=0$.
    \item[Case 2:] Suppose $d=0$ and $a$ is not necessarily equal to $0$. Then, it must be the case that either $b$ or $c$ is equal to $0$.\\

      If $b = c = 0$, then we have $a^2 = 0$, so $a = 0$.\\

      If $b = 0$ with $c$ not necessarily equal to $0$, we have
      \begin{align*}
        a^2 - 7c^2 &= 0\\
        (a-c\sqrt{7})(a + c\sqrt{7}) &= 0,
        \intertext{meaning $a = c\sqrt{7}$ or $a = -c\sqrt{7}$. Since $a\in \Q$ and $c\sqrt{7}\notin\Q$, this can only be the case if $a=c=0$.}
      \end{align*}
      If $c = 0$ with $b$ not necessarily equal to $0$, we have
      \begin{align*}
        a^2 - 5b^2 &= 0\\
        (a-b\sqrt{5})(a + b\sqrt{5}) &= 0\\
        \intertext{meaning $a = b\sqrt{5}$ or $a = -b\sqrt{5}$. Since $a\in\Q$ and $b\sqrt{5}\notin \Q$, this can only be the case if $a=b=0$.}
      \end{align*}
    \item[Case 3:] Suppose $a = 0$ and $d$ is not necessarily equal to $0$. Then, it must be the case that either $b$ or $c$ is equal to $0$.\\

      If $b=c=0$, we have $35d^2 = 0$, so $d =0$.\\

      If $b = 0$ with $c$ not necessarily equal to $0$, we have
      \begin{align*}
        35d^2 - 7c^2 &= 0\\
        7(5d^2 - c^2) &= 0\\
        7(d\sqrt{5} - c)(d\sqrt{5} + c) &= 0\\
        \intertext{meaning $d\sqrt{5} = c$ or $-d\sqrt{5} = c$. Since $c\in\Q$ and $d\sqrt{5}\notin \Q$, this can only be the case if $d=c=0$.}
      \end{align*}
      If $c = 0$ with $b$ not necessarily equal to $0$, we have
      \begin{align*}
        35d^2 - 5b^2 &= 0\\
        5(7d^2 - b^2) &= 0\\
        5(d\sqrt{7} - b)(d\sqrt{7} + b) &= 0\\
        \intertext{meaning $d\sqrt{7} = b$ or $-d\sqrt{7} = b$. Since $b\in\Q$ and $d\sqrt{7}\notin \Q$, this can only be the case if $d=b=0$.}
      \end{align*}
    \item[Case 4:] Suppose toward contradiction that $a\neq 0$ and $d\neq 0$. Then, $a = \frac{bc}{d}$. Substituting, we find
      \begin{align*}
        \left(\frac{bc}{d}\right)^2 + 35d^2 - 5b^2 - 7c^2 &= 0\\
        b^2c^2 + 35d^4 - 5b^2d^2 - 7c^2d^2 &= 0\\
        b^2(c^2 - 5d^2) - 7d^2(c^2 - 5d^2) &= 0\\
        (b-d\sqrt{7})(b+d\sqrt{7})(c-d\sqrt{5})(c+d\sqrt{5}) &= 0\\
        \intertext{meaning $b = \pm d\sqrt{7}$ or $c = \pm d\sqrt{5}$. Since $d\sqrt{7},d\sqrt{5}\notin \Q$, and $b,c\in\Q$, this is only the case if $b=d=0$ or $c=d=0$, which is a contradiction.}
      \end{align*}
  \end{description}
  \section{Problem 3}%
  We will show that $\Q(\sqrt{5} + \sqrt{7}) = \Q(\sqrt{5},\sqrt{7})$.\\

  Clearly, $\Q(\sqrt{5},\sqrt{7})\subseteq \Q(\sqrt{5} + \sqrt{7})$. We need to show that $\sqrt{7}$ and $\sqrt{5}$ can be written as elements of $\Q(\sqrt{5} + \sqrt{7})$. By difference of squares, we have
  \begin{align*}
    \sqrt{7} - \sqrt{5} &= \frac{2}{\left(\sqrt{7} + \sqrt{5}\right)},\\
    \intertext{meaning}
    \sqrt{7} &= \frac{\left(\sqrt{7} + \sqrt{5}\right) + \frac{2}{\left(\sqrt{7} + \sqrt{5}\right)}}{2}\\
    \sqrt{5} &= \frac{\left(\sqrt{7} + \sqrt{5}\right) - \frac{2}{\left(\sqrt{7} + \sqrt{5}\right)}}{2}\\
    \sqrt{35} &= \frac{1}{2}\left(\sqrt{5} + \sqrt{7}\right)^2 - 12
  \end{align*}
  Thus, $\Q(\sqrt{5} + \sqrt{7})\subseteq \Q(\sqrt{5},\sqrt{7})$.
  \section{Problem 4}%
  Let $F = \Q(\alpha_1,\dots,\alpha_n)$. Suppose $\alpha_i^2\in \Q$ for all $i$. We will show that $\sqrt[3]{2}\notin F$.\\

  If $\alpha_i^2\in \Q$, then $\alpha_i \in \Q$ or $\alpha_i \notin \Q$. If $\alpha_i \in \Q$, then $[\Q(\alpha_i):\Q] = 1$, and if $\alpha_i \notin \Q$, then $m_{\alpha_i,\Q}(x) = x^2 - \alpha_i^2$ is the unique monic irreducible polynomial over $\Q$, meaning $[\Q(\alpha_i):\Q] = 2$. Thus,
  \begin{align*}
    [\Q(\alpha_1,\dots,\alpha_n):\Q] &= [\Q(\alpha_1,\dots,\alpha_n):\Q(\alpha_1,\dots,\alpha_{n-1})][\Q(\alpha_1,\dots,\alpha_{n-1}):\Q],
  \end{align*}
  meaning that, inductively, we have that $[\Q(\alpha_1,\dots,\alpha_n):\Q] = 2^{k}$ for some $k\in \Z_{\geq 0}$.\\

  Suppose toward contradiction that $\sqrt[3]{2}\in \Q(\alpha_1,\dots,\alpha_n)$. Then, since $m_{\sqrt[3]{2},\Q}(x) = x^3 - 2$ (as it is irreducible by the Eisenstein criterion and monic, thus unique), we have that $[\Q(\sqrt[3]{2}):\Q] = 3$. This implies that $3|2^{k}$ for some $k\in \Z_{\geq 0}$, which is not possible. Thus, $\sqrt[3]{2}\notin \Q(\alpha_1,\dots,\alpha_n)$.
  \section{Problem 5}%
  We will show that $x^3 - 2x - 2$ is irreducible over $\Q$, then compute $(1+\theta)(1+\theta + \theta^2)$ and $\frac{1+\theta}{1+\theta+\theta^2}$ in $\Q(\theta)$ for $\theta$ a root.\\

  To start, we see that $x^3 - 2x - 2$ is a monic polynomial where $p = 2$, so by Eisenstein's criterion and Gauss's Lemma, $x^3 - 2x - 2$ is irreducible over $\Q$. Thus, we have that elements of $\Q[x]/\langle x^3 - 2x - 2\rangle = a\theta^2 + b\theta + c$ for $a,b,c\in \Q$.\\

  We have that $\theta^3 - 2\theta - 2 = 0$. So,
  \begin{align*}
    (1+\theta)(1+\theta + \theta^2) &= 1 + 2\theta + 2\theta^2 + \theta^3\\
                                    &= 3 + 4\theta + 2\theta^2 \in \Q(\theta).
  \end{align*}
  To find $\frac{1+\theta}{1+\theta+\theta^2}$, we find $\frac{1}{1+\theta+\theta^2}$ through the Euclidean algorithm and polynomial long division. Since $\gcd(1+x+x^2,x^3-2x-2) = 1$ (as both are irreducible in $\Q[x]$ and neither is a multiple of the other), we have
  \begin{align*}
    x^3 - 2x - 2 &= (1+x+x^2)(x-1) + (-2x-1)\\
    1+x+x^2 &= (-2x-1)\left(-\frac{1}{2}x - \frac{1}{4}\right) + \frac{3}{4}.
  \end{align*}
  Multiplying backwards, we have
  \begin{align*}
    1 &= \frac{4}{3}\left(1+x+x^2 -\left(-\frac{1}{2}x - \frac{1}{4}\right)(-2x-1)\right)\\
      &= \frac{4}{3} + \frac{4}{3}x + \frac{4}{3}x^2 - \frac{4}{3}\left(-\frac{1}{2}x - \frac{1}{4}\right)\left(x^3 - 2x - 2 - (x-1)(x^2 + x + 1)\right)\\
      &= \left(\frac{2}{3}x + \frac{1}{3}\right)\left(x^3-2x-2\right) + \left(-\frac{2}{3}x^2 + \frac{1}{3}x + \frac{5}{3}\right)\left(x^2 + x + 1\right).
  \end{align*}
  In particular, by taking $\theta$ as a root of $x^3 -2x -1$, we have
  \begin{align*}
    1 &= \left(\frac{2}{3}\theta + \frac{1}{3}\right)\left(\theta^3-2\theta-2\right) + \left(-\frac{2}{3}\theta^2 + \frac{1}{3}\theta + \frac{5}{3}\right)\left(\theta^2 + \theta + 1\right)\\
      &= \left(-\frac{2}{3}\theta^2 + \frac{1}{3}\theta + \frac{5}{3}\right)\left(\theta^2 + \theta + 1\right),
  \end{align*}
  so
  \begin{align*}
    \frac{1}{1+\theta + \theta^2} &= \left(-\frac{2}{3}\theta^2 + \frac{1}{3}\theta + \frac{5}{3}\right),
  \end{align*}
  so
  \begin{align*}
    \frac{1+\theta}{1+\theta+\theta^2} &= (1+\theta)\left(-\frac{2}{3}\theta^2 + \frac{1}{3}\theta + \frac{5}{3}\right)\\
                                       &= \frac{5}{3} + 2\theta - \frac{1}{3}\theta^2 - \frac{2}{3}\theta^3\\
                                       &= \frac{5}{3} + 2\theta - \frac{1}{3}\theta^2 - \frac{2}{3}\left(2\theta + 2\right)\\
                                       &= \frac{1}{3} + \frac{2}{3}\theta - \frac{1}{3}\theta^2.
  \end{align*}
\end{document}
