\documentclass[10pt]{extarticle}
\title{}
\author{}
\date{}
\usepackage[shortlabels]{enumitem}


%paper setup
\usepackage{geometry}
\geometry{letterpaper, portrait, margin=1in}
\usepackage{fancyhdr}
% sans serif font:
\usepackage{cmbright}
%symbols
\usepackage{amsmath}
\usepackage{bigints}
\usepackage{amssymb}
\usepackage{amsthm}
\usepackage{mathtools}
\usepackage{bbold}
\usepackage[hidelinks]{hyperref}
\usepackage{gensymb}
\usepackage{multirow,array}
\usepackage{multicol}

\newtheorem*{remark}{Remark}
\usepackage[T1]{fontenc}
\usepackage[utf8]{inputenc}

%chemistry stuff
%\usepackage[version=4]{mhchem}
%\usepackage{chemfig}

%plotting
\usepackage{pgfplots}
\usepackage{tikz}
\usetikzlibrary{cd}
\tikzset{middleweight/.style={pos = 0.5}}
%\tikzset{weight/.style={pos = 0.5, fill = white}}
%\tikzset{lateweight/.style={pos = 0.75, fill = white}}
%\tikzset{earlyweight/.style={pos = 0.25, fill=white}}

%\usepackage{natbib}

%graphics stuff
\usepackage{graphicx}
\graphicspath{ {./images/} }
%\usepackage[style=numeric, backend=biber]{biblatex} % Use the numeric style for Vancouver
%\addbibresource{the_bibliography.bib}
%code stuff
%when using minted, make sure to add the -shell-escape flag
%you can use lstlisting if you don't want to use minted
%\usepackage{minted}
%\usemintedstyle{pastie}
%\newminted[javacode]{java}{frame=lines,framesep=2mm,linenos=true,fontsize=\footnotesize,tabsize=3,autogobble,}
%\newminted[cppcode]{cpp}{frame=lines,framesep=2mm,linenos=true,fontsize=\footnotesize,tabsize=3,autogobble,}

%\usepackage{listings}
%\usepackage{color}
%\definecolor{dkgreen}{rgb}{0,0.6,0}
%\definecolor{gray}{rgb}{0.5,0.5,0.5}
%\definecolor{mauve}{rgb}{0.58,0,0.82}
%
%\lstset{frame=tb,
%	language=Java,
%	aboveskip=3mm,
%	belowskip=3mm,
%	showstringspaces=false,
%	columns=flexible,
%	basicstyle={\small\ttfamily},
%	numbers=none,
%	numberstyle=\tiny\color{gray},
%	keywordstyle=\color{blue},
%	commentstyle=\color{dkgreen},
%	stringstyle=\color{mauve},
%	breaklines=true,
%	breakatwhitespace=true,
%	tabsize=3
%}
% text + color boxes
%\renewcommand{\mathbf}[1]{\mathbb{#1}}
%\usepackage[most]{tcolorbox}
%\tcbuselibrary{breakable}
%\tcbuselibrary{skins}
%\newtcolorbox{problem}[1]{colback=white,enhanced,title={\small #1},
%          attach boxed title to top center=
%{yshift=-\tcboxedtitleheight/2},
%boxed title style={size=small,colback=black!60!white}, sharp corners, breakable}
%including PDFs
%\usepackage{pdfpages}
\setlength{\parindent}{0pt}
\usepackage{cancel}
%\pagestyle{fancy}
%\fancyhf{}
%\rhead{Avinash Iyer}
%\lhead{}
\newcommand{\card}{\text{card}}
\newcommand{\ran}{\text{ran}}
\newcommand{\N}{\mathbb{N}}
\newcommand{\Q}{\mathbb{Q}}
\newcommand{\Z}{\mathbb{Z}}
\newcommand{\R}{\mathbb{R}}
\newcommand{\C}{\mathbb{C}}
\newcommand{\iprod}[2]{\left\langle #1,#2\right\rangle}
\newcommand{\norm}[1]{\left\Vert #1\right\Vert}
\setcounter{secnumdepth}{0}
\begin{document}
  \begin{center}
    {\bf \Large Math 395 \\[0.1in]Homework 8 \\[0.1in]
    Due: 4/30/2024}\\[.25in]
    {\bf Name:} {Avinash Iyer}\\[0.15in]
    {\bf Collaborators:} {Nate Hall, Nora Manukyan} \\
  \end{center}
  \section{Problem 1}%
  Let $K/F$ be a Galois extension with $\text{Gal}(K/F)$ Abelian of order 10. We will compute the intermediate fields between $F$ and $K$, and their dimensions over $F$.\\

  Since $\text{Gal}(K/F)$ is Abelian and of order 10, $\text{Gal}(K/F)\cong \Z/10\Z$. {\tiny (OEIS A000001)}\\

  The subgroups of $\text{Gal}(K/F)$ are isomorphic to the subgroups of $\Z/10\Z$; since $10 = 2\cdot 5$, it must be the case that $\langle 2 \rangle$, with order $5$ and $\langle 5 \rangle$, with order $2$, are the two proper subgroups of $\Z/10\Z$ (by Lagrange's Theorem). We will let $H_1 \leq \text{Gal}(K/F)$ be isomorphic to $\langle 2 \rangle$, and $H_2 \leq \text{Gal}(K/F)$ be isomorphic to $\langle 5 \rangle$.\\

  Let $A = K^{H_1}$. Then, since $[\Z/10\Z : \langle 2 \rangle] = 2$, it is the case that $[A:F] = 2$. Similarly, for $B = K^{H_2}$, it is the case that $[\Z/10\Z : \langle 5 \rangle] = 5$, so $[B:F] = 5$.
  \section{Problem 3}% 
  We will find $\text{Gal}(x^4 - 5x^2 + 6)$ over $\Q$.\\

  To start, factoring $x^4 - 5x^2 + 6$, we find it is equal to $(x^2 - 3)(x^2 - 2) = (x-\sqrt{3})(x+\sqrt{3})(x-\sqrt{2})(x+\sqrt{2})$ in $\Q(\sqrt{2},\sqrt{3})$. Since $x^4 - 5x^2 + 6$ is separable in $\Q(\sqrt{2},\sqrt{3}) = \text{Spl}(x^4 - 5x^2 + 6)$, it must be the case that $\Q(\sqrt{2},\sqrt{3})/\Q$ is a Galois extension.\\

  We know that the basis for $\Q(\sqrt{2},\sqrt{3})$ is $\{1,\sqrt{2},\sqrt{3},\sqrt{6}\}$, meaning that for $\sigma\in \text{Gal}(K/F)$, we have $\sigma(a + b\sqrt{2} + c\sqrt{3} + d\sqrt{6}) + a + b\sigma(\sqrt{2}) + c\sigma(\sqrt{3}) + d\sigma(\sqrt{2})\sigma(\sqrt{6})$. Thus, the possible elements of $\text{Gal}(K/F)$ are
  \begin{align*}
    \sigma_0 &:= \text{id}\\
    \sigma_1 &:= \begin{cases}
      \sqrt{2} \mapsto -\sqrt{2}\\
      \sqrt{3} \mapsto \sqrt{3}
    \end{cases}\\
      \sigma_2 &:= \begin{cases}
        \sqrt{2} \mapsto \sqrt{2}\\
        \sqrt{3} \mapsto -\sqrt{3}
      \end{cases}\\
        \sigma_3 &:= \begin{cases}
          \sqrt{2} \mapsto -\sqrt{2}\\
          \sqrt{3} \mapsto -\sqrt{3}
        \end{cases}.
  \end{align*}
  Notice that $\sigma_1^{2} = \sigma_2^{2} = \sigma_3^{2} = \sigma_0$, meaning we have $\text{Gal}(K/F)\cong \Z/2\Z\times \Z/2\Z$.
  \section{Problem 4}%
  \begin{enumerate}[(a)]
    \item To find the splitting field of $f(x) = x^4 - 2$ over $\Q$, we find its roots, which are $\pm \sqrt[4]{2}$, $\pm i\sqrt[4]{2}$. Thus, $ K = \text{Spl}_{\Q}(f(x)) = \Q(i,\sqrt[4]{2})$.
    \item To find $[K:\Q]$, we see
      \begin{align*}
        [\Q(i,\sqrt[4]{2}) : \Q] &= [\Q(i,\sqrt[4]{2}):\Q(\sqrt[4]{2})][\Q(\sqrt[4]{2}):\Q]\\
                                 &= 8.
      \end{align*}
    \item To see that such a $\sigma$ exists, we will verify that it maps a basis for $\Q(i,\sqrt[4]{2})$ to a basis for $\Q(i,\sqrt[4]{2})$, and keeps $\Q$ fixed.
      \begin{align*}
        \sigma &: \begin{cases}
                    1 \mapsto 1\\
                    \sqrt[4]{2} \mapsto i\sqrt[4]{2}\\
                    \sqrt[4]{4} \mapsto -\sqrt[4]{4}\\
                    \sqrt[4]{8} \mapsto -i\sqrt[4]{8}\\
                    i \mapsto i\\
                    i\sqrt[4]{2} \mapsto -\sqrt[4]{2}\\
                    i\sqrt[4]{4} \mapsto -i\sqrt[4]{4}\\
                    i\sqrt[4]{8} \mapsto \sqrt[4]{8}
                  \end{cases}.
      \end{align*}
      Therefore, $\sigma \in \text{Gal}(K/\Q)$. We see that $\sigma^{2}(\sqrt[4]{2}) = -\sqrt[4]{2}$, $\sigma^3(\sqrt[4]{2}) = -i\sqrt[4]{2}$, meaning $\sigma^{4} = \text{id}$.
    \item Letting $\tau$ be the restriction of complex conjugation to $K$, we will show that $\tau\in \text{Gal}(K/\Q)$ and $\text{Gal}(K/\Q) = \{\text{id},\sigma,\sigma^2,\sigma^3,\tau,\sigma\tau,\sigma^2\tau,\sigma^3\tau\}$.\\

      To start, we will verify that $\tau$ maps a basis for $\Q(i,\sqrt[4]{2})$ to a basis for $\Q(i,\sqrt[4]{2})$, keeping $\Q$ fixed.
      \begin{align*}
        \tau &: \begin{cases}
                  1 \mapsto 1\\
                  \sqrt[4]{2} \mapsto \sqrt[4]{2}\\
                  \sqrt[4]{4} \mapsto \sqrt[4]{4}\\
                  \sqrt[4]{8} \mapsto \sqrt[4]{8}\\
                  i\mapsto -i\\
                  i\sqrt[4]{2} \mapsto -i\sqrt[4]{2}\\
                  i\sqrt[4]{4} \mapsto -i\sqrt[4]{4}\\
                  i\sqrt[4]{8} \mapsto -i\sqrt[4]{8}
                \end{cases}
      \end{align*}
      We see that $\tau^2 = \text{id}$, and $\tau \neq \sigma$. Defining $\sigma\tau\cdot x = \sigma(\tau(x))$, we see the elements of $\text{Gal}(K/\Q)$ are
      \begin{align*}
        e &= \text{id}\\
        \sigma &= \begin{cases}
                    \sqrt[4]{2} \mapsto i\sqrt[4]{2}\\
                    i \mapsto i
                  \end{cases}\\
          \sigma^2 &= \begin{cases}
                    \sqrt[4]{2} \mapsto -\sqrt[4]{2}\\
                    i \mapsto i
                  \end{cases}\\
            \sigma^3 &= \begin{cases}
                    \sqrt[4]{2} \mapsto -i\sqrt[4]{2}\\
                    i \mapsto i
                  \end{cases}\\
            \sigma^4 &= \begin{cases}
                    \sqrt[4]{2} \mapsto \sqrt[4]{2}\\
                    i \mapsto i
                  \end{cases}\\
                     &= \text{id}\\
              \tau &= \begin{cases}
                \sqrt[4]{2} \mapsto \sqrt[4]{2}\\
                i \mapsto -i
              \end{cases}\\
                \tau^2 &= \begin{cases}
                  \sqrt[4]{2} \mapsto \sqrt[4]{2}\\
                  i\mapsto i
                \end{cases}\\
                       &= \text{id}\\
              \sigma\tau &= \begin{cases}
                \sqrt[4]{2} \xmapsto{\tau} \sqrt[4]{2} \xmapsto{\sigma} i\sqrt[4]{2}\\
                i \xmapsto{\tau} -i \xmapsto{\sigma} -i
              \end{cases}\\
            \sigma^2\tau &= \begin{cases}
              \sqrt[4]{2} \xmapsto{\tau} \sqrt[4]{2} \xmapsto{\sigma^2} -\sqrt[4]{2}\\
              i \xmapsto{\tau} -i \xmapsto{\sigma^2} -i
            \end{cases}\\
            \sigma^3\tau &= \begin{cases}
              \sqrt[4]{2} \xmapsto{\tau} \sqrt[4]{2} \xmapsto{\sigma^3} -i\sqrt[4]{2}\\
              i \xmapsto{\tau} -i \xmapsto{\sigma^3} -i
            \end{cases}\\
              \tau\sigma &= \begin{cases}
                \sqrt[4]{2}\xmapsto{\sigma}i\sqrt[4]{2} \xmapsto{\tau} -i\sqrt[4]{2}\\
                i\xmapsto{\sigma}i\xmapsto{\tau}-i
              \end{cases}\\
                         &= \sigma^3\tau\\
              \tau\sigma^2 &= \begin{cases}
                \sqrt[4]{2}\xmapsto{\sigma^2}-\sqrt[4]{2} \xmapsto{\tau} -\sqrt[4]{2}\\
                i\xmapsto{\sigma^2}i\xmapsto{\tau}-i
              \end{cases}\\
                           &= \sigma^2\tau\\
                \tau\sigma^3 &= \begin{cases}
                  \sqrt[4]{2} \xmapsto{\sigma^3}-i\sqrt[4]{2}\xmapsto{\tau}i\sqrt[4]{2}\\
                  i \xmapsto{\sigma^3}i\xmapsto{\tau}-i
                \end{cases}\\
                             &= \sigma\tau.
      \end{align*}
      Since $|\text{Gal}(K/\Q)| = [K:\Q] = 8$, it must be the case that $\{e,\sigma,\sigma^2,\sigma^3,\tau,\sigma\tau,\sigma^2\tau,\sigma^3\tau\}$ are the elements of $\text{Gal}(K/\Q)$. This is isomorphic to the dihedral group of order 8, $D_4$.
    \item We can determine the fixed field of $\langle \sigma^2\tau\rangle$ as follows. We find that for $x = \sum_{j=1}^{8}a_je_j$, where $e_j$ denotes the $j$th basis vector of $\Q(i,\sqrt[4]{2})$, we have
      \begin{align*}
        \sigma^2\tau(x) &= a_1 - a_2\sqrt[4]{2} + a_3\sqrt[4]{4} - a_4\sqrt[4]{8} - a_5i + a_6i\sqrt[4]{2} - a_7i\sqrt[4]{4} + a_8i\sqrt[4]{8}\\
        \text{id}(x) &= a_1 + a_2\sqrt[4]{2} + a_3\sqrt[4]{4} + a_4\sqrt[4]{8} + a_5i + a_6i\sqrt[4]{2} + a_7i\sqrt[4]{4} + a_8i\sqrt[4]{8}.
      \end{align*}
      Therefore, $a_2 = -a_2$, $a_4 = -a_4$, $a_5 = -a_5$, and $a_7 = -a_7$, meaning the coefficients on the respective $a_i$ are identically $0$, or
      \begin{align*}
        x &= a_1 + a_3\sqrt[4]{4} + a_6i\sqrt[4]{2} + a_8i\sqrt[4]{8}\\
          &= a_1 + a_6i\sqrt[4]{2} - a_3\left(i\sqrt[4]{2}\right)^2 - a_8\left(i\sqrt[4]{2}\right)^3.
      \end{align*}
      Therefore, $\Q\left(i,\sqrt[4]{2}\right)^{\langle \sigma^2\tau \rangle} = \Q(i\sqrt[4]{2})$. {\tiny (Answer found with assistance from Adamson (1964), ``Introduction to Field Theory.'')}

    \item Letting $E = \Q(\sqrt{2},i)$, we have
      \begin{align*}
        [K:E] &= [\Q(\sqrt[4]{2},i):\Q(\sqrt{2},i)]\\
              &= 2.
      \end{align*}
      Additionally, since $\Q(\sqrt{2},i) = \text{Spl}_{\Q}(x^2 + 2)$, it is also Galois over $\Q$, meaning $\text{Gal}(K/E)\trianglelefteq \text{Gal}(K/\Q)$ with $|\text{Gal}(K/E)| = 2$. Thus, $\text{Gal}(K/E) = \langle \sigma^2\rangle$.
    \item To find the fixed fields for $\sigma\tau$ and $\sigma^3\tau$, we use the procedure that we used for $\sigma^2\tau$ to find $\Q(\sqrt[4]{2},i)^{\langle \sigma\tau\rangle} = \Q\left(\sqrt[4]{2}(1+i)\right)$ and $\Q(\sqrt[4]{2},i)^{\langle \sigma^3\tau\rangle} = \Q\left(\sqrt[4]{2}(1-i)\right)$. Thus, the lattice of subfields and subgroups is as follows.
      \begin{center}
% https://tikzcd.yichuanshen.de/#N4Igdg9gJgpgziAXAbVABwnAlgFyxMJZAJgBoAGAXVJADcBDAGwFcYkQAdDgRQAou4ARwBOOZABZKwYgF9SWAJQg56TLnyEU5UgEZqdJq3ZduXRjABmOfhyGiJU2bx0BqRV2FYA5gAscSlRAMbDwCIh1dfQYWNkROHjNLawERMUlpGWcAWncOT19-ZVJVEI0iMj0aaKM4kxs7HAz5AOKgtVDNZABmSKrDWPi+FPt02RaS9TCUcV6DGOMeXixhtMcZcbbSqeQI4ij+hb5l21SMjeDJzrI9vvnaxcUiiY6iHpu5msH607GnzcvyqQuvs7oNlPoYFAvPAiKALMIIABbJDaEA4CBIMggHwwehQdiQMBsGiMegAIxgjAACu0ynEsGBsLAQLdPsQ-vCkSiaOikBFsbj8XFCcSQKSKdTaVMQAymaLqgN2YFOcjEKjeYgegK8QSCKLxZSaVtNDLGVhmazFRyEar1RjEDNtULwHqWWLyYapSbZeb5Qc4krWiruWj7QBWGg4nXC10kj2S43sH0Wj5W5U2vk8+0ANkjgt1RLdBoTAPpZpTCvYgbhGcQWI1ufdEqNpdNcrdlYD1q5mqzSEbxZbLzL7bz0ZdhctVe7qq1GoA7HHm16k+W-aD2WPnSKZ0g5-aABxLz2Jke+jv+kCbp0FtjpnuOjVHpsn1vJ9dst1R7eu++qiOhkgz7freRbxkOdJtueU5dn+-Z9ogACcW6gceJbDlBFaXtWIDBogi6AUhKExpOL7oZB74Xhuu6IM+GrITeJH6uBK5nlh1EyJQMhAA
% https://tikzcd.yichuanshen.de/#N4Igdg9gJgpgziAXAbVABwnAlgFyxMJZAJgBoAGAXVJADcBDAGwFcYkQAdDgRQAou4ARwBOOZABZKwYgF9SWAJQg56TLnyEU5UgEZqdJq3ZduXRjABmOfhyGiJU2bx0BqRV2FYA5gAscSlRAMbDwCIh1dfQYWNkROHjNLawERMUlpGWcAWncOT19-ZVJVEI0iMj0aaKM4kxs7HAz5AOKgtVDNZABmSKrDWPi+FPt02RaS9TCUcV6DGOMeXixhtMcZcbbSqeQI4ij+hb5l21SMjeDJzrI9vvnaxcUiiY6iHpu5msH607GnzcvyqQuvs7oNlPoYFAvPAiKALMIIABbJDaEA4CBIMggHwwehQdiQMBsGiMegAIxgjAACu0ynEsGBsLAQLdPsQ-vCkSiaOikBFsbj8XFCcSQKSKdTaVMQAymaLqgN2YFOcjEKjeYgegK8QSCKLxZSaVtNDLGVhmazFRyEar1RjEDNtULwHqWWLyYapSbZeb5Qc4krWiruWj7QBWGg4nXC10kj2S43sH0Wj5W5U2vk8+0ANkjgt1RLdBoTAPpZpTCvYgbhGcQWI1ufdEqNpdNcrdlYDbqjzpF1q5mqzSEbxZbLzL7bz0ZdhctVf7qq1GoA7HHm16k+W-aD2VPe670wOl-aABxrz2Jie+jv+kC7p0FtiH1WOjVnpsX1vJ7ds7v5mOFs+SARqGSDvj2j5FvGY50m215zl2QGII2GoAJx7pB54luOcEVre1YgMGiCrqBiDoQ+AH6tBG5XnhO4LmBQ5kRhlFQeul64T+aaUDIQA
\begin{tikzcd}
                                                                       &                                                                       & {\Q(\sqrt[4]{2},i)} \arrow[ld, "2" description, no head] \arrow[lld, "2" description, no head] \arrow[d, "2" description, no head] \arrow[rd, "2" description, no head] \arrow[rrd, "2" description, no head] &                                                       &                                                         \\
{\Q\left(\sqrt[4]{2}(1+i)\right)} \arrow[rd, "2" description, no head] & {\Q\left(\sqrt[4]{2}(1-i)\right)} \arrow[d, "2" description, no head] & {\Q(\sqrt{2},i)} \arrow[ld, "2" description, no head] \arrow[d, "2" description, no head] \arrow[rd, "2" description, no head]                                                                                & {\Q(\sqrt[4]{2})} \arrow[d, "2" description, no head] & {\Q(i\sqrt[4]{2})} \arrow[ld, "2" description, no head] \\
                                                                       & \Q(i\sqrt{2}) \arrow[rd, "2" description, no head]                    & \Q(i) \arrow[d, "2" description, no head]                                                                                                                                                                     & \Q(\sqrt{2}) \arrow[ld, "2" description, no head]     &                                                         \\
                                                                       &                                                                       & \Q                                                                                                                                                                                                            &                                                       &                                                        
\end{tikzcd}\\
% https://tikzcd.yichuanshen.de/#N4Igdg9gJgpgziAXAbVABwnAlgFyxMJZAJgBoAGAXVJADcBDAGwFcYkQAdDx+sAc0YwuOGAA8cwLFAC+XAE68BbaaXSZc+QinKkAjNTpNW7Lj36Cu2PgFt6w+s3mLBIFWux4CRXXoMMWbIic3M5CHFa2AHoAzPaOHArmyqogGB6aRGT6NP7GQaahllg29JHETkmuKWkaXijRvjlGgcFmSnEVSlXutVrIACyNhgEmIUlFJWUdCaHdqeqefT7Efs2jbRbhxVFkE7bTiV1u8+l1JKQrTSP5Y+1bJZ0uxzWLRA2Xw3mthfc7pAezZ4LDIoMjRVbXb5JAAEe3o-w4OAcj2UBhgUD48CIoAAZnIINYkDoQDgIEgyCAABYwehQdiQMDJXH4wmIYmkpA+Kk0ulBBlMkB4glEmgcxANbm0+kEAVC1nssmIQaS3ngGVzOUikmKgCsNGpUr56uOmsQXLFADZ9TzpYyNSzyaLFVaVbbZQ7xU6kC6Dar+fbhZ7tUgAOzWw1qu0mj3KsUADnDfuNKVNeuDiATrqNUZTHol8cTboDrJdYoAnIXs+7A2H0xWs5Hq6zM+XK43XJRpEA
\begin{tikzcd}
                                             &                                                         & \langle\text{id}\rangle \arrow[ld, no head] \arrow[lld, no head] \arrow[d, no head] \arrow[rd, no head] \arrow[rrd, no head] &                                                   &                                                \\
\langle\sigma\tau\rangle \arrow[rd, no head] & \langle\sigma^3\tau\rangle \arrow[d, no head]           & \langle\sigma^2\rangle \arrow[ld, no head] \arrow[d, no head] \arrow[rd, no head]                                            & \langle\tau\rangle \arrow[d, no head]             & \langle\sigma^2\tau\rangle \arrow[ld, no head] \\
                                             & {\langle\sigma^2,\sigma\tau\rangle} \arrow[rd, no head] & \langle\sigma\rangle \arrow[d, no head]                                                                                      & {\langle\sigma^2,\tau\rangle} \arrow[ld, no head] &                                                \\
                                             &                                                         & {\langle \sigma,\tau\rangle}                                                                                                 &                                                   &                                               
\end{tikzcd}
      \end{center}
  \end{enumerate}
  \section{Problem 6}%
  We will prove that $\Q(\sqrt[3]{2})$ is not a subfield of $\Q(\zeta_n)$ for any $n\geq 1$.\\

  It is known that $\text{Gal}(\Q(\zeta_n)/\Q)\cong \left(\Z/n\Z\right)^{\times}$, which is an Abelian group. Therefore, any subgroup of $\text{Gal}\left(\Q(\zeta_n)/\Q\right)$ is normal, so any subfield $\Q \subseteq E\subseteq \Q(\zeta_n)$ is Galois over $\Q$. However, since $\Q(\sqrt[3]{2})/\Q$ is not a Galois extension, it cannot be the case that $\Q(\sqrt[3]{2})$ is a subfield of $\Q(\zeta_n)$. {\tiny (Answer found using hint from Stack Overflow.)}
\end{document}
