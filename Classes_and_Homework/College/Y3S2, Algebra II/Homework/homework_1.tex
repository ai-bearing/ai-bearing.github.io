\documentclass[10pt]{extarticle}
\title{}
\author{}
\date{}
\usepackage[shortlabels]{enumitem}


%paper setup
\usepackage{geometry}
\geometry{letterpaper, portrait, margin=1in}
\usepackage{fancyhdr}
% sans serif font:
% \usepackage{cmbright}
%symbols
\usepackage{amsmath}
\usepackage{bigints}
\usepackage{amssymb}
\usepackage{amsthm}
\usepackage{mathtools}
\usepackage{bbm}
\usepackage[hidelinks]{hyperref}
\usepackage{gensymb}
\usepackage{multirow,array}
\usepackage{multicol}

\newtheorem*{remark}{Remark}
\usepackage[T1]{fontenc}
\usepackage[utf8]{inputenc}

%chemistry stuff
%\usepackage[version=4]{mhchem}
%\usepackage{chemfig}

%plotting
\usepackage{pgfplots}
\usepackage{tikz}
\tikzset{middleweight/.style={pos = 0.5}}
%\tikzset{weight/.style={pos = 0.5, fill = white}}
%\tikzset{lateweight/.style={pos = 0.75, fill = white}}
%\tikzset{earlyweight/.style={pos = 0.25, fill=white}}

%\usepackage{natbib}

%graphics stuff
\usepackage{graphicx}
\graphicspath{ {./images/} }
\usepackage[style=numeric, backend=biber]{biblatex} % Use the numeric style for Vancouver
\addbibresource{the_bibliography.bib}
%code stuff
%when using minted, make sure to add the -shell-escape flag
%you can use lstlisting if you don't want to use minted
%\usepackage{minted}
%\usemintedstyle{pastie}
%\newminted[javacode]{java}{frame=lines,framesep=2mm,linenos=true,fontsize=\footnotesize,tabsize=3,autogobble,}
%\newminted[cppcode]{cpp}{frame=lines,framesep=2mm,linenos=true,fontsize=\footnotesize,tabsize=3,autogobble,}

%\usepackage{listings}
%\usepackage{color}
%\definecolor{dkgreen}{rgb}{0,0.6,0}
%\definecolor{gray}{rgb}{0.5,0.5,0.5}
%\definecolor{mauve}{rgb}{0.58,0,0.82}
%
%\lstset{frame=tb,
%	language=Java,
%	aboveskip=3mm,
%	belowskip=3mm,
%	showstringspaces=false,
%	columns=flexible,
%	basicstyle={\small\ttfamily},
%	numbers=none,
%	numberstyle=\tiny\color{gray},
%	keywordstyle=\color{blue},
%	commentstyle=\color{dkgreen},
%	stringstyle=\color{mauve},
%	breaklines=true,
%	breakatwhitespace=true,
%	tabsize=3
%}
% text + color boxes
\usepackage[most]{tcolorbox}
\tcbuselibrary{breakable}
\tcbuselibrary{skins}
\newtcolorbox{problem}[1]{colback=white,enhanced,title={\small #1},
          attach boxed title to top center=
{yshift=-\tcboxedtitleheight/2},
boxed title style={size=small,colback=black!60!white}, sharp corners, breakable}
%including PDFs
%\usepackage{pdfpages}
\setlength{\parindent}{0pt}
\usepackage{cancel}
%\pagestyle{fancy}
%\fancyhf{}
%\rhead{Avinash Iyer}
%\lhead{}
\newcommand{\card}{\text{card}}
\newcommand{\ran}{\text{ran}}
\newcommand{\N}{\mathbbm{N}}
\newcommand{\Q}{\mathbbm{Q}}
\newcommand{\Z}{\mathbbm{Z}}
\newcommand{\R}{\mathbbm{R}}
\setcounter{secnumdepth}{0}
\begin{document}
    \begin{center}
    {\bf \Large Math 395 \\[0.1in]Homework 1 \\[0.1in]
    Due: 2/1/2024}\\[.25in]
    {\bf Name:} {Avinash Iyer}\\[0.15in]
    {\bf Collaborators:} {\underline{\hspace*{4.5in}}} \\
    \end{center}
  \section{Problem 1}%
  Let $S$ be the subset of $\text{Mat}_{2}(\mathbf{R})$ be the set consisting of matrices of the form $ \begin{bmatrix}a&a\\b&b\end{bmatrix} $.
  \begin{enumerate}[(a)]
    \item To show that $S$ is a ring, we will show that $S$ is a subring of the ring $\text{Mat}_{2}(\mathbf{R})$, by showing that $S$ is not empty, $S$ is closed under subtraction, and $S$ is closed under multiplication.\\

      To show non-emptiness, we can see that the matrix $ \begin{bmatrix}0&0\\0&0\end{bmatrix} $ is an element of $S$ by its definition.\\

      To show $S$ is closed under subtraction, let $a,b,c,d\in \mathbf{R}$, and let $e = a-c$ and $f = b-d$. Then,
      \begin{align*}
        \begin{bmatrix}a&a\\b&b\end{bmatrix} - \begin{bmatrix}c&c\\d&d\end{bmatrix} &= \begin{bmatrix}a&a\\b&b\end{bmatrix} + \begin{bmatrix}-c&-c\\-d&-d\end{bmatrix}\\
                        &= \begin{bmatrix}a+(-c) & a+(-c) \\ b+(-d) & b+(-d)\end{bmatrix}\\
                        &= \begin{bmatrix}a-c&a-c\\b-d&b-d\end{bmatrix}\\
                        &= \begin{bmatrix}e&e\\f&f\end{bmatrix},
      \end{align*}
      which is an element of $S$. Thus, $S$ is closed under subtraction.\\

      Next, we need to show that $S$ is closed under multiplication. Letting $a,b,c,d\in \mathbf{R}$ as before, let $g = ac+ad$ and $h = bc+bd$. Then,
      \begin{align*}
        \begin{bmatrix}a&a\\b&b\end{bmatrix} \cdot \begin{bmatrix}c&c\\d&d\end{bmatrix} &= \begin{bmatrix}ac+ad & ac+ad\\bc+bd & bc+bd\end{bmatrix}\\
                        &= \begin{bmatrix}g&g\\h&h\end{bmatrix},
      \end{align*}
      which is an element of $S$. Thus, $S$ is closed under multiplication.\\

      Since $S$ is non-empty, closed under subtraction, and closed under multiplication, $S$ is a subring of $\text{Mat}_{2}(\mathbf{R})$, and so is a ring.
    \item To show that $J = \begin{bmatrix}1&1\\0&0\end{bmatrix}$ is a right identity, we multiply an arbitrary matrix in $S$ on the right by $J$.
      \begin{align*}
        AJ &=\begin{bmatrix}a&a\\b&b\end{bmatrix} \begin{bmatrix}1&1\\0&0\end{bmatrix} \\
        &= \begin{bmatrix}a&a\\b&b\end{bmatrix}\\
        &= A.
      \end{align*}
    \item Let $B = \begin{bmatrix}a&a\\b&b\end{bmatrix}$. Then, since
      \begin{align*}
        JB &= \begin{bmatrix}1&1\\0&0\end{bmatrix} \begin{bmatrix}a&a\\b&b\end{bmatrix}\\
           &= \begin{bmatrix}a+b&a+b\\0&0\end{bmatrix}\\
           &\neq B,
      \end{align*}
      $J$ is not a left identity for $S$.
  \end{enumerate}
  \section{Problem 3}%
  Let $a\oplus b = a+b-1$ and $a\odot b = ab - (a+b) + 2$ be defined as such on $\mathbf{Z}$. We will show that these operations under $\mathbf{Z}$ form an integral domain.\\

  First, we will show that $\mathbf{Z}$ under $\oplus$ is an Abelian group. Since $\mathbf{Z}$ is closed under ordinary addition and subtraction, $\mathbf{Z}$ is closed under $\oplus$. We can exhibit the associative property as follows:
  \begin{align*}
    a\oplus(b\oplus c) &= a + (b\oplus c) - 1\\
                       &= a+(b+c-1) - 1\\
                       &= (a+b-1) + c - 1\\
                       &= (a\oplus b) + c - 1\\
                       &= (a\oplus b)\oplus c.
  \end{align*}
  Additionally, $1$ is an additive identity for $\mathbf{Z}$ under $\oplus$, as $(a\oplus 1) = a+1-1 = a$. Therefore, $2-a$ is the additive inverse for $\mathbf{Z}$ under $\oplus$, exhibited as follows:
  \begin{align*}
    a\oplus(2-a) &= a+(2-a) - 1\\
                 &= 1.
  \end{align*}
  Finally, since $a\oplus b = a+b-1 = b+a-1 = b\oplus a$, the $\oplus$ operator is commutative.\\

  Next, we will show that $\mathbf{Z}$ under $\odot$ satisfies the necessary properties for a commutative ring with identity.\\

  Since $\odot$ consists of regular addition, subtraction, and multiplication under $\mathbf{Z}$, $\odot$ is closed under $\mathbf{Z}$. We will show associativity as follows. Let $a,b,c\in\mathbf{Z}$; then,
  \begin{align*}
    a\odot(b\odot c) &= a(b\odot c) - (a + (b\odot c)) + 2\\
                     &= a(bc - (b+c) + 2) - (a + (bc -(b+c) + 2)) + 2\\
                     &= abc - ab - ac + 2a - a - bc + b + c - 2 + 2\\
                     &= abc -ab - ac - bc + a + b + c,
                     \shortintertext{and}
    (a\odot b)\odot c &= (ab - (a+b) + 2)\odot c\\
                      &= (ab-(a+b) + 2)c - (ab -(a+b) + 2 + c) + 2\\
                      &= abc - ac-bc + 2c - ab + a + b - 2 - c + 2\\
                      &= abc - ab - ac - bc + a + b + c,\\
                      \shortintertext{so}
    (a\odot b)\odot c &= a\odot(b\odot c).
  \end{align*}
  We will show that $\odot$ is distributive over $\oplus$ as follows:
  \begin{align*}
    a\odot(b\oplus c) &= a\odot (b+c-1)\\
                      &= a(b+c-1) - (a + (b+c-1)) + 2\\
                      &= ab + ac - a - a - b - c + 1 + 2\\
                      &= \left(ab - (a+b) + 2\right) + \left(ac - (a+c) + 2\right) - 1\\
                      &= \left(a\odot b\right) \oplus \left(a\odot c\right)\\
    (a\oplus b)\odot c &= (a+b-1)\odot c\\
                       &= (a+b-1)c - (a+b-1 + c) + 2\\
                       &= ac + bc - c - a - b - c + 1 + 2\\
                       &= \left(ac - (a+c) + 2\right) + \left(bc - (b+c) + 2\right) - 1\\
                       &= (a\odot c)\oplus(b\odot c)
  \end{align*}
  To show commutativity, we can see that $a\odot b = ab - (a+b) + 2 = ba - (b+a) + 2 = b\odot a$. Additionally, we can show that $2$ is a multiplicative identity under $\odot$ as follows:
  \begin{align*}
    a\odot 2 &= (a)(2) - (a+2) + 2\\
             &= 2a - a - 2 + 2\\
             &= a,
  \end{align*}
  meaning $\odot$ is closed, associative, distributive, commutative, and has identity.\\

  In order to show that $(\mathbf{Z},\oplus,\odot)$ is an integral domain, we must show that this commutative ring with identity has no zero divisors (i.e., there is no number not equal to $1$ that yields $1$ when multiplied under $\odot$). Suppose toward contradiction that $a,b\neq 1$ and $a\odot b = 1$. Then,
  \begin{align*}
    1 &= a\odot b\\
    1&= ab - (a+b) + 2\\
           1 &= ab - a - b + 2\\
           0 &= ab - a - b + 1\\
           0 &= a(b-1) - (b-1)\\
           0 &= (b-1)(a-1),\\
           \shortintertext{meaning $a=1$ or $b=1$, and we have a contradiction.}
  \end{align*}
  Therefore, $(\mathbf{Z},\oplus,\odot)$ is a commutative ring with identity without zero divisors, meaning it is an integral domain.
  \section{Problem 4}%
  Let $R$ be a ring, and $Z(R) = \{a\mid ar = ra,\text{ for all }r\in R\}$. We will prove that $Z(R)$ is a subring of $R$.\\

  To show $Z(R)$ is a subring of $R$, we will show that $Z(R)$ is non-empty, closed under subtraction, and closed under multiplication. To start, we can see that $0_R\in Z(R)$, as $(0_R)(r) = (r)(0_R) = 0_R$. Therefore, $Z(R)$ is nonempty.\\

  Let $a,b\in Z(R)$. We will show that $a-b\in Z(R)$. Since $a\in Z(R)$, for any $r\in R$, it is the case that $ar = ra$. Subtracting $br$ from both sides, we have $ar-br = ra - br$. However, since $br \in Z(R)$, it is the case that $br = rb$, meaning we have $ar - br = ra - rb$. Using the distributive property of multiplication, we have $(a-b)r = r(a-b)$, meaning $a-b\in Z(R)$, and $Z(R)$ is closed under subtraction.\\

  Let $a,b\in Z(R)$. We will show that $ab\in Z(R)$. Since $b\in Z(R)$, for any $r\in R$, it is the case that $br = rb$. Multiplying $a$, where $a\in Z(R)$, on the left, we have $a(br) = a(rb)$. Using the associative property of multiplication, we have $(ab)r = (ar)b$. Finally, since $a\in Z(R)$, we have $ar = ra$, meaning $(ab)r = (ra)b$, and using the associative property once again, we have $(ab)r = r(ab)$. Thus, $ab\in Z(R)$, and $Z(R)$ is closed under multiplication.\\

  Since $Z(R)$ is non-empty, closed under subtraction, and closed under multiplication, $Z(R)$ is a subring of $R$.
  \section{Problem 6}%
  Let $S$ and $T$ be subrings of $R$.
  \begin{enumerate}[(a)]
    \item We will show that $S\cap T$ is a subring of $R$. We will show that $S\cap T$ is non-empty, closed under subtraction, and closed under multiplication. First, since the additive identity is an element of both $S$ and $T$, the additive identity is in $S\cap T$, meaning $S\cap T$ is non-empty.\\

      Let $a,b\in S\cap T$. Since $S$ is a subring, $S$ is closed under subtraction, meaning $a-b\in S$. Additionally, since $T$ is a subring, $a-b\in T$. Therefore, since $a-b \in S$ and $a-b\in T$, it is the case that $a-b\in S\cap T$, meaning $S\cap T$ is closed under subtraction.\\

      Let $a,b\in S\cap T$. Since $S$ is a subring, $S$ is closed under multiplication, meaning $ab\in S$. Additionally, since $T$ is a subring, $T$ is closed under multiplication, meaning $ab\in T$. Since $ab\in S$ and $ab\in T$, it is the case that $ab\in S\cap T$, meaning $S\cap T$ is closed under multiplication.
    \item Consider the subrings $S = 2\mathbf{Z}$ and $T = 5\mathbf{Z}$ of the ring $R = \mathbf{Z}$. The union, $S\cup T$, is not closed under addition, as for $a = 2\in S\cup T$ and $b = 5\in S\cup T $, $a + b = 2 + 5 = 7 \notin S\cup T$, meaning $S\cup T$ cannot be a subring of $\mathbf{Z}$.
  \end{enumerate}
\end{document}
