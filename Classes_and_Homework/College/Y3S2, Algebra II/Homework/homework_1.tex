\documentclass[10pt]{extarticle}
\title{}
\author{}
\date{}
\usepackage[shortlabels]{enumitem}


%paper setup
\usepackage{geometry}
\geometry{letterpaper, portrait, margin=1in}
\usepackage{fancyhdr}
% sans serif font:
% \usepackage{cmbright}
%symbols
\usepackage{amsmath}
\usepackage{bigints}
\usepackage{amssymb}
\usepackage{amsthm}
\usepackage{mathtools}
\usepackage{bbm}
\usepackage[hidelinks]{hyperref}
\usepackage{gensymb}
\usepackage{multirow,array}
\usepackage{multicol}

\newtheorem*{remark}{Remark}
\usepackage[T1]{fontenc}
\usepackage[utf8]{inputenc}

%chemistry stuff
%\usepackage[version=4]{mhchem}
%\usepackage{chemfig}

%plotting
\usepackage{pgfplots}
\usepackage{tikz}
\tikzset{middleweight/.style={pos = 0.5}}
%\tikzset{weight/.style={pos = 0.5, fill = white}}
%\tikzset{lateweight/.style={pos = 0.75, fill = white}}
%\tikzset{earlyweight/.style={pos = 0.25, fill=white}}

%\usepackage{natbib}

%graphics stuff
\usepackage{graphicx}
\graphicspath{ {./images/} }
\usepackage[style=numeric, backend=biber]{biblatex} % Use the numeric style for Vancouver
\addbibresource{the_bibliography.bib}
%code stuff
%when using minted, make sure to add the -shell-escape flag
%you can use lstlisting if you don't want to use minted
%\usepackage{minted}
%\usemintedstyle{pastie}
%\newminted[javacode]{java}{frame=lines,framesep=2mm,linenos=true,fontsize=\footnotesize,tabsize=3,autogobble,}
%\newminted[cppcode]{cpp}{frame=lines,framesep=2mm,linenos=true,fontsize=\footnotesize,tabsize=3,autogobble,}

%\usepackage{listings}
%\usepackage{color}
%\definecolor{dkgreen}{rgb}{0,0.6,0}
%\definecolor{gray}{rgb}{0.5,0.5,0.5}
%\definecolor{mauve}{rgb}{0.58,0,0.82}
%
%\lstset{frame=tb,
%	language=Java,
%	aboveskip=3mm,
%	belowskip=3mm,
%	showstringspaces=false,
%	columns=flexible,
%	basicstyle={\small\ttfamily},
%	numbers=none,
%	numberstyle=\tiny\color{gray},
%	keywordstyle=\color{blue},
%	commentstyle=\color{dkgreen},
%	stringstyle=\color{mauve},
%	breaklines=true,
%	breakatwhitespace=true,
%	tabsize=3
%}
% text + color boxes
\usepackage[most]{tcolorbox}
\tcbuselibrary{breakable}
\tcbuselibrary{skins}
\newtcolorbox{problem}[1]{colback=white,enhanced,title={\small #1},
          attach boxed title to top center=
{yshift=-\tcboxedtitleheight/2},
boxed title style={size=small,colback=black!60!white}, sharp corners, breakable}
%including PDFs
%\usepackage{pdfpages}
\setlength{\parindent}{0pt}
\usepackage{cancel}
%\pagestyle{fancy}
%\fancyhf{}
%\rhead{Avinash Iyer}
%\lhead{}
\newcommand{\card}{\text{card}}
\newcommand{\ran}{\text{ran}}
\newcommand{\N}{\mathbbm{N}}
\newcommand{\Q}{\mathbbm{Q}}
\newcommand{\Z}{\mathbbm{Z}}
\newcommand{\R}{\mathbbm{R}}
\setcounter{secnumdepth}{0}
\begin{document}
    \begin{center}
    {\bf \Large Math 395 \\[0.1in]Homework 1 \\[0.1in]
    Due: 2/1/2024}\\[.25in]
    {\bf Name:} {Avinash Iyer}\\[0.15in]
    {\bf Collaborators:} {\underline{\hspace*{4.5in}}} \\
    \end{center}
  \section{Problem 1}%
  Let $S$ be the subset of $\text{Mat}_{2}(\mathbf{R})$ be the set consisting of matrices of the form $ \begin{bmatrix}a&a\\b&b\end{bmatrix} $.
  \begin{enumerate}[(a)]
    \item To show that $S$ is a ring, we will show that $S$ is a subring of the ring $\text{Mat}_{2}(\mathbf{R})$, by showing that $S$ is not empty, $S$ is closed under subtraction, and $S$ is closed under multiplication.\\

      To show non-emptiness, we can see that the matrix $ \begin{bmatrix}0&0\\0&0\end{bmatrix} $ is an element of $S$ by its definition.\\

      To show $S$ is closed under subtraction, let $a,b,c,d\in \mathbf{R}$, and let $e = a-c$ and $f = b-d$. Then,
      \begin{align*}
        \begin{bmatrix}a&a\\b&b\end{bmatrix} - \begin{bmatrix}c&c\\d&d\end{bmatrix} &= \begin{bmatrix}a&a\\b&b\end{bmatrix} + \begin{bmatrix}-c&-c\\-d&-d\end{bmatrix}\\
                        &= \begin{bmatrix}a+(-c) & a+(-c) \\ b+(-d) & b+(-d)\end{bmatrix}\\
                        &= \begin{bmatrix}a-c&a-c\\b-d&b-d\end{bmatrix}\\
                        &= \begin{bmatrix}e&e\\f&f\end{bmatrix},
      \end{align*}
      which is an element of $S$. Thus, $S$ is closed under subtraction.\\

      Next, we need to show that $S$ is closed under multiplication. Letting $a,b,c,d\in \mathbf{R}$ as before, let $g = ac+ad$ and $h = bc+bd$. Then,
      \begin{align*}
        \begin{bmatrix}a&a\\b&b\end{bmatrix} \cdot \begin{bmatrix}c&c\\d&d\end{bmatrix} &= \begin{bmatrix}ac+ad & ac+ad\\bc+bd & bc+bd\end{bmatrix}\\
                        &= \begin{bmatrix}g&g\\h&h\end{bmatrix},
      \end{align*}
      which is an element of $S$. Thus, $S$ is closed under multiplication.\\

      Since $S$ is non-empty, closed under subtraction, and closed under multiplication, $S$ is a subring of $\text{Mat}_{2}(\mathbf{R})$, and so is a ring.
    \item
  \end{enumerate}
\end{document}
