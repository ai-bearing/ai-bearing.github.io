\documentclass[10pt]{extarticle}
\title{}
\author{}
\date{}
\usepackage[shortlabels]{enumitem}


%paper setup
\usepackage{geometry}
\geometry{letterpaper, portrait, margin=1in}
\usepackage{fancyhdr}
% sans serif font:
\usepackage{cmbright}
%symbols
\usepackage{amsmath}
\usepackage{bigints}
\usepackage{amssymb}
\usepackage{amsthm}
\usepackage{mathtools}
\usepackage{bbold}
\usepackage[hidelinks]{hyperref}
\usepackage{gensymb}
\usepackage{multirow,array}
\usepackage{multicol}

\newtheorem*{remark}{Remark}
\usepackage[T1]{fontenc}
\usepackage[utf8]{inputenc}

%chemistry stuff
%\usepackage[version=4]{mhchem}
%\usepackage{chemfig}

%plotting
\usepackage{pgfplots}
\usepackage{tikz}
\usetikzlibrary{cd}
\tikzset{middleweight/.style={pos = 0.5}}
%\tikzset{weight/.style={pos = 0.5, fill = white}}
%\tikzset{lateweight/.style={pos = 0.75, fill = white}}
%\tikzset{earlyweight/.style={pos = 0.25, fill=white}}

%\usepackage{natbib}

%graphics stuff
\usepackage{graphicx}
\graphicspath{ {./images/} }
\usepackage[style=numeric, backend=biber]{biblatex} % Use the numeric style for Vancouver
\addbibresource{the_bibliography.bib}
%code stuff
%when using minted, make sure to add the -shell-escape flag
%you can use lstlisting if you don't want to use minted
%\usepackage{minted}
%\usemintedstyle{pastie}
%\newminted[javacode]{java}{frame=lines,framesep=2mm,linenos=true,fontsize=\footnotesize,tabsize=3,autogobble,}
%\newminted[cppcode]{cpp}{frame=lines,framesep=2mm,linenos=true,fontsize=\footnotesize,tabsize=3,autogobble,}

%\usepackage{listings}
%\usepackage{color}
%\definecolor{dkgreen}{rgb}{0,0.6,0}
%\definecolor{gray}{rgb}{0.5,0.5,0.5}
%\definecolor{mauve}{rgb}{0.58,0,0.82}
%
%\lstset{frame=tb,
%	language=Java,
%	aboveskip=3mm,
%	belowskip=3mm,
%	showstringspaces=false,
%	columns=flexible,
%	basicstyle={\small\ttfamily},
%	numbers=none,
%	numberstyle=\tiny\color{gray},
%	keywordstyle=\color{blue},
%	commentstyle=\color{dkgreen},
%	stringstyle=\color{mauve},
%	breaklines=true,
%	breakatwhitespace=true,
%	tabsize=3
%}
% text + color boxes
\renewcommand{\mathbf}[1]{\mathbb{#1}}
\usepackage[most]{tcolorbox}
\tcbuselibrary{breakable}
\tcbuselibrary{skins}
\newtcolorbox{problem}[1]{colback=white,enhanced,title={\small #1},
          attach boxed title to top center=
{yshift=-\tcboxedtitleheight/2},
boxed title style={size=small,colback=black!60!white}, sharp corners, breakable}
%including PDFs
%\usepackage{pdfpages}
\setlength{\parindent}{0pt}
\usepackage{cancel}
%\pagestyle{fancy}
%\fancyhf{}
%\rhead{Avinash Iyer}
%\lhead{}
\newcommand{\card}{\text{card}}
\newcommand{\ran}{\text{ran}}
\newcommand{\N}{\mathbb{N}}
\newcommand{\Q}{\mathbb{Q}}
\newcommand{\Z}{\mathbb{Z}}
\newcommand{\R}{\mathbb{R}}
\newcommand{\C}{\mathbb{C}}
\newcommand{\iprod}[2]{\left\langle #1,#2\right\rangle}
\newcommand{\norm}[1]{\left\Vert #1\right\Vert}
\setcounter{secnumdepth}{0}
\begin{document}
  \begin{center}
    {\bf \Large Math 395 \\[0.1in]Homework 3 \\[0.1in]
    Due: 2/15/2024}\\[.25in]
    {\bf Name:} {Avinash Iyer}\\[0.15in]
    {\bf Collaborators:} {Nate Hall, Antonio Cabello, Gianluca Crescenzo} \\
  \end{center}
  \section{Problem 1}%
  Let $\varphi: R\rightarrow S$ be a ring homomorphism. Let $\mathfrak{p}\in \text{Spec}(S)$. We will prove that $\varphi^{-1}(\mathfrak{p})\subset R$ is an element of $\text{Spec}(R)$.\\

  Let $\mathfrak{p}\in \text{Spec}(S)$. Let $ab\in \varphi^{-1}(\mathfrak{p})$. Then, $\varphi(ab)\in \mathfrak{p}$. So, $\varphi(a)\varphi(b)\in \mathfrak{p}$, meaning either $\varphi(a)\in \mathfrak{p}$ or $\varphi(b)\in \mathfrak{p}$. Therefore, $a\in \varphi^{-1}(\mathfrak{p})$ or $b\in \varphi^{-1}(\mathfrak{p})$. Therefore, $\varphi^{-1}(\mathfrak{p})$. 
  \section{Problem 4}%
  Let $I,J$ be ideals of $R$ with $I\subseteq J$. We will show that $J/I$ is an ideal of $R/I$ and $(R/I)/(J/I)\cong R/J$.\\

  We know that $J/I$ is non-empty, as it contains $0_R$, so we will show that $J/I$ is closed under subtraction and multiplication by elements of $R/I$. Thus, by the rules of subrings, for $j_1,j_2,j\in J$, we have
  \begin{align*}
    (j_1 + I) - (j_2 + I) &= (j_1 - j_2) + I\\
                          &\in J/I,\\
                          \intertext{and, since $j\in R$, by the properties of the quotient ring,}
    (r+I)(j+I) &= (rj) + I,\\
    \intertext{and since $rj \in J$ as $J$ is an ideal,}
    (rj + I) &\in J/I.
    \intertext{Similarly,}
    (j+I)(r+I) &= (jr) + I\\
               &\in J/I.
  \end{align*}
  Therefore, since $J/I$ is non-empty, closed under subtraction, and closed under multiplication by elements of $R/I$, it is the case that $J/I$ is an ideal $R/I$.\\

  Let $\varphi: R/I \rightarrow R/J$, $r+I \mapsto r+J$. We will show that $\varphi$ is a well-defined homomorphism with $\ker(\varphi) = J/I$.\\

  Let $r_1 \sim_{R/I} r_2$. Then, $r_1 = r_2 + i$ for some $i\in I$. Then, 
  \begin{align*}
    \varphi(r_1) &= r_1 + J\\
                 &= (r_2 + i) + J,\\
                 \intertext{and since $I\subseteq J$,}
                 &= r_2 + J.
  \end{align*}
  Thus, $\varphi$ is well-defined. Additionally,
  \begin{align*}
    \varphi((r_1 + I) + (r_2 + I)) &= \varphi((r_1 + r_2) + I)\\
                                   &= (r_1 + r_2) + J\\
                                   &= (r_1 + J) + (r_2 + J)\\
                                   &= \varphi(r_1 + I) + \varphi(r_2 + I),\\
                                   \intertext{and}
    \varphi((r_1 + I)(r_2 + I)) &= \varphi(r_1r_2 + I)\\
                                &= r_1r_2 + J\\
                                &= (r_1 + J)(r_2 + J)\\
                                &= \varphi(r_1 + I)  \varphi(r_2 + I).
  \end{align*}
  Therefore, $\varphi$ is a homomorphism. The elements that $\varphi$ maps to $0 + J$ are precisely the elements of $j + I$ where $j\in J$, as
  \begin{align*}
    \varphi(j + I) &= j + J\\
                   &= 0 + J.
  \end{align*}
  Thus, $\ker(\varphi) = J/I$.\\

  By the first isomorphism theorem, $(R/I)/(J/I) \cong R/J$.
  \section{Problem 5}%
  Define $\varphi: \mathbb{F}_p \rightarrow \mathbb{F}_p$, where $\varphi(x) = x^p$ for $\mathbb{F}_p = \Z/p\Z$. We will show that $\varphi$ is an isomorphism.\\

  We will start by showing that $\varphi$ is a well-defined homomorphism. Let $[a]_{p} = [b]_p$. Then, $a = b + kp$ for some $k\in \Z$. By Fermat's Little Theorem, $\varphi(a) = a^{p} \equiv [a]_{p}$, and $\varphi(b + kp) = b^p + p(\ell)$ for some $\ell$, so $\varphi(b) \equiv [b]_{p}$ as well. Thus, $\varphi([a]_p) = \varphi([b]_p)$.\\

  Since $\varphi$ is well-defined, we find that, for $a,b\in \mathbb{F}_p$,
  \begin{align*}
    \varphi(a+b) &= \left([a+b]_p\right)^{p}\\
                 &\equiv [a+b]_p\\
                 &= [a]_p + [b]_p\\
                 &\equiv \left([a]_p\right)^p + \left([b]_p\right)^p \\
                 &= \varphi(a) + \varphi(b),
                 \intertext{and}
    \varphi(ab) &= \left([ab]_{p}\right)^{p}\\
                &\equiv [ab]_p\\
                &= [a]_p[b]_p\\
                &\equiv \left([a]_p\right)^p \left([b]_p\right)^p\\
                &= \varphi(a)\varphi(b),
  \end{align*}
  meaning $\varphi$ is a homomorphism.\\

  Since, for all $x\in \mathbb{F}_p$, $x \equiv x^p$, it is the case that $\varphi$ is surjective. Finally, since $\varphi(0) = 0$, and for $x\neq 0$, $\varphi(x) \neq 0$, it is the case that $\ker(\varphi) = \{0\}$, meaning $\varphi$ is injective.\\

  Since $\varphi$ is a bijective homomorphism, $\varphi$ is an isomorphism.
\end{document}
