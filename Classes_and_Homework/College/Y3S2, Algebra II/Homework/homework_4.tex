\documentclass[8pt]{extarticle}
\title{}
\author{}
\date{}
\usepackage[shortlabels]{enumitem}


%paper setup
\usepackage{geometry}
\geometry{letterpaper, portrait, margin=1in}
\usepackage{fancyhdr}
% sans serif font:
\usepackage{cmbright}
%symbols
\usepackage{amsmath}
\usepackage{bigints}
\usepackage{amssymb}
\usepackage{amsthm}
\usepackage{mathtools}
\usepackage{bbold}
\usepackage[hidelinks]{hyperref}
\usepackage{gensymb}
\usepackage{multirow,array}
\usepackage{multicol}

\newtheorem*{remark}{Remark}
\usepackage[T1]{fontenc}
\usepackage[utf8]{inputenc}

%chemistry stuff
%\usepackage[version=4]{mhchem}
%\usepackage{chemfig}

%plotting
\usepackage{pgfplots}
\usepackage{tikz}
\usetikzlibrary{cd}
\tikzset{middleweight/.style={pos = 0.5}}
%\tikzset{weight/.style={pos = 0.5, fill = white}}
%\tikzset{lateweight/.style={pos = 0.75, fill = white}}
%\tikzset{earlyweight/.style={pos = 0.25, fill=white}}

%\usepackage{natbib}

%graphics stuff
\usepackage{graphicx}
\graphicspath{ {./images/} }
\usepackage[style=numeric, backend=biber]{biblatex} % Use the numeric style for Vancouver
\addbibresource{the_bibliography.bib}
%code stuff
%when using minted, make sure to add the -shell-escape flag
%you can use lstlisting if you don't want to use minted
%\usepackage{minted}
%\usemintedstyle{pastie}
%\newminted[javacode]{java}{frame=lines,framesep=2mm,linenos=true,fontsize=\footnotesize,tabsize=3,autogobble,}
%\newminted[cppcode]{cpp}{frame=lines,framesep=2mm,linenos=true,fontsize=\footnotesize,tabsize=3,autogobble,}

%\usepackage{listings}
%\usepackage{color}
%\definecolor{dkgreen}{rgb}{0,0.6,0}
%\definecolor{gray}{rgb}{0.5,0.5,0.5}
%\definecolor{mauve}{rgb}{0.58,0,0.82}
%
%\lstset{frame=tb,
%	language=Java,
%	aboveskip=3mm,
%	belowskip=3mm,
%	showstringspaces=false,
%	columns=flexible,
%	basicstyle={\small\ttfamily},
%	numbers=none,
%	numberstyle=\tiny\color{gray},
%	keywordstyle=\color{blue},
%	commentstyle=\color{dkgreen},
%	stringstyle=\color{mauve},
%	breaklines=true,
%	breakatwhitespace=true,
%	tabsize=3
%}
% text + color boxes
\renewcommand{\mathbf}[1]{\mathbb{#1}}
\usepackage[most]{tcolorbox}
\tcbuselibrary{breakable}
\tcbuselibrary{skins}
\newtcolorbox{problem}[1]{colback=white,enhanced,title={\small #1},
          attach boxed title to top center=
{yshift=-\tcboxedtitleheight/2},
boxed title style={size=small,colback=black!60!white}, sharp corners, breakable}
%including PDFs
%\usepackage{pdfpages}
\setlength{\parindent}{0pt}
\usepackage{cancel}
%\pagestyle{fancy}
%\fancyhf{}
%\rhead{Avinash Iyer}
%\lhead{}
\newcommand{\card}{\text{card}}
\newcommand{\ran}{\text{ran}}
\newcommand{\N}{\mathbb{N}}
\newcommand{\Q}{\mathbb{Q}}
\newcommand{\Z}{\mathbb{Z}}
\newcommand{\R}{\mathbb{R}}
\newcommand{\C}{\mathbb{C}}
\newcommand{\iprod}[2]{\left\langle #1,#2\right\rangle}
\newcommand{\norm}[1]{\left\Vert #1\right\Vert}
\setcounter{secnumdepth}{0}
\begin{document}
  \begin{center}
    {\bf \Large Math 395 \\[0.1in]Homework 4 \\[0.1in]
    Due: 2/27/2024}\\[.25in]
    {\bf Name:} {Avinash Iyer}\\[0.15in]
    {\bf Collaborators:} {Ling Chen, Timothy Rainone} \\
  \end{center}
  \section{Problem 1}%
  Let $F$ be a field, with $F[x]$ denoting the ring of polynomials with coefficients in $F$. Let $f(x)\in F[x]$ be a monic polynomial. Let $g(x) \in F[x]$ be a nonzero polynomial. We will show that there exist unique $q(x)$ and $r(x)$ in $F[x]$ such that $f(x) = g(x)q(x) + r(x)$, where $r(x) = 0$ or $\text{deg}~r(x) < \text{deg}~g(x)$.\\

  If $\text{deg}~g(x) > \text{deg}~r(x)$, then we set $q(x) = 0$ and $r(x) = f(x)$.\\

  Consider $\{f(x) - g(x)q(x)\mid q(x)\in F[x]\}$, where $q(x)$ is such that $g(x)q(x)$ is monic. We select $r(x)$ to be the polynomial of smallest degree in this set.
  \section{Problem 4}%
  Let $p\in \Z$ be a prime. Let $\mathfrak{m} = \{(pa,b)\mid a,b\in \Z\}$. We will prove that $\mathfrak{m}$ is a maximal ideal in $\Z\times \Z$.\\

  We will do so by showing that $(\Z\times\Z)/\mathfrak{m}$ is isomorphic to the field $\Z/p\Z$. Let $\varphi: \Z\times\Z\rightarrow \Z/p\Z$ be defined by $\varphi((i,j)) = [i]_{p}$. We will show that $\varphi$ is a surjective homomorphism with kernel $\mathfrak{m}$.

  Let $(i,j),(k,\ell) \in \Z\times \Z$. Then,
  \begin{align*}
    \varphi((i,j) + (k,\ell)) &= \varphi((i+k,j+\ell))\\
                              &= [i+k]_{p}\\
                              &= [i]_p + [k]_{p}\\
                              &= \varphi((i,j)) + \varphi((k,\ell)),
                              \intertext{and}
    \varphi((i,j)(k,\ell)) &= \varphi((ik,j\ell))\\
                           &= [ik]_{p}\\
                           &= [i]_{p}[k]_{p}\\
                           &= \varphi((i,j))\varphi((k,\ell)).
  \end{align*}
  Finally, for any $[a]_{p}\in \Z/p\Z$, we set $(a,1)\in\Z\times\Z$ such that $\varphi((a,1)) = [a]_p$, meaning $\varphi$ is surjective.\\

  For $\varphi((x,y)) = [0]_{p}$, it must be the case that $[x]_{p} = [0]_{p}$, meaning $x = pa$ for some $a\in \Z$. Thus, $\ker\varphi = \{(pa,b)\mid a,b\in\Z\} = \mathfrak{m}$. By the first isomorphism theorem, it is the case that $(\Z\times\Z)/\mathfrak{m} = \Z/p\Z$. Since $\Z/p\Z$ is a field, $\mathfrak{m}$ must be maximal.
  \section{Problem 5}%
  Let $p$ be a prime, and let $J$ be the collection of polynomials in $\Z[x]$ whose constant term is divisible by $p$. We will show that $J$ is a maximal ideal in $\Z[x]$.\\
  
  Let $\varphi: \Z[x]\rightarrow \Z/p\Z$ be defined by
  \begin{align*}
    a_0 + a_1x + \cdots + a_nx^n \mapsto [a_0]_p.
  \end{align*}
  For any $[a]_p\in \Z/p\Z$, we select an element of $\Z[x]$ with constant term equal to $a$, meaning that $\varphi$ is a surjective map. We will show that $\varphi$ is a homomorphism. Let $a = a_0 + a_1x + \cdots + a_nx^n$ and $b = b_0 + b_1 x + \cdots + b_mx^{m}$ be elements of $\Z[x]$. Without loss of generality, $n \geq m$. Then,
  \begin{align*}
    \varphi(a+b) &= \varphi\left((a_0 + b_0) + (a_1 + b_1)x + \cdots + (a_m+b_m)x^m + \cdots + a_nx^n\right)\\
                 &= [a_0 + b_0]_{p}\\
                 &= [a_0]_p + [b_0]_p\\
                 &= \varphi(a_0 + a_1x + \cdots + a_nx^n) + \varphi(b_0 + b_1x + \cdots + b_mx^m),
                 \intertext{and}
    \varphi(ab) &= \varphi\left((a_0 + a_1x + \cdots + a_nx^n)(b_0 + b_1x + \cdots + b_mx^m)\right)\\
                &= \varphi\left((a_0b_0) + \cdots + (a_nb_m)x^{n+m}\right)\\
                &= [a_0b_0]_p\\
                &= [a_0]_p[b_0]_p\\
                &= \varphi(a_0 + a_1x + \cdots + a_nx^n)  \varphi(b_0 + b_1x + \cdots + b_mx^m)\\
                &= \varphi(a)\varphi(b).
  \end{align*}
  Therefore, $\varphi$ is a homomorphism with
  \begin{align*}
    \ker\varphi &= \left\{a_0 + a_1x + \cdots + a_nx^n\mid a_i\in \Z, [a_0]_p = [0]_p\right\},
  \end{align*}
  which is precisely the set of all polynomials in  $\Z[x]$ with with $a_0 | p$, or $J$. By the first isomorphism theorem, it is thus the case that $\Z[x]/J \cong \Z/p\Z$. Since $\Z/p\Z$ is a field, it must be the case that $J$ is a maximal ideal.
  \section{Problem 7}%
  Let $R$ be a commutative ring with identity. Let $I\subset R$ be an ideal. The radical of $I$ is defined as
  \begin{align*}
    \text{rad}~I &= \{r\in R\mid r^n\in I\text{ for some } n\in \Z_{>0}\}
  \end{align*}
  We say $I$ is a radical ideal if $\text{rad}~I = I$. We will show that every prime ideal of $R$ is a radical ideal.\\

  Let $I$ be a prime ideal. Let $r\in \text{rad}~I$. Then, $\exists n\in \Z_{>0}$ such that $r^{n} \in I$. We will show that $r\in I$ by induction.\\

  In the base case, we let $n = 1$. Then, since $r^{1} = (1)(r) \in I$. Since $I$ is prime, it must be the case that either $1$ or $r$ is an element of $I$; however, since $I \neq R$, it must be the case that $1\notin I$ (as $1$ is a unit in $R$), so $r\in I$.\\

  Suppose that for $2,\dots,n-1$, it is the case that if $r^{n-1}\in I$, then $r\in I$. Then, if $r^{n}\in I$, we have $r^{n} = (r^{n-1})(r)\in I$. Since $I$ is prime, either $r\in I$ or $r^{n-1}\in I$. If the first is the case, then we are done; otherwise, if $r^{n-1}\in I$, the inductive hypothesis holds that $r\in I$. Thus, $\text{rad}~I \subseteq I$.\\

  Let $a\in I$. Then, since $a \in R$, we have that $a^{1}\in I$, meaning $n=1$, so $a\in \text{rad}~I$. Thus, $I\subseteq \text{rad}~I$. Therefore, for $I$ a prime ideal, $\text{rad}~I = I$.
\end{document}
