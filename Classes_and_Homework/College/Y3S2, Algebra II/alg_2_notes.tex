\documentclass[10pt]{extarticle}
\title{}
\author{}
\date{}
\usepackage[shortlabels]{enumitem}


%paper setup
\usepackage{geometry}
\geometry{letterpaper, portrait, margin=1in}
\usepackage{fancyhdr}
% sans serif font:
\usepackage{cmbright}
%symbols
\usepackage{amsmath}
\usepackage{bigints}
\usepackage{amssymb}
\usepackage{amsthm}
\usepackage{mathtools}
\usepackage{bbm}
\usepackage{bbold}
\usepackage[hidelinks]{hyperref}
\usepackage{gensymb}
\usepackage{multirow,array}
\usepackage{multicol}

\newtheorem*{remark}{Remark}
\usepackage[T1]{fontenc}
\usepackage[utf8]{inputenc}

%chemistry stuff
%\usepackage[version=4]{mhchem}
%\usepackage{chemfig}

%plotting
\usepackage{pgfplots}
\usepackage{tikz}
\usetikzlibrary{cd}
\tikzset{middleweight/.style={pos = 0.5}}
%\tikzset{weight/.style={pos = 0.5, fill = white}}
%\tikzset{lateweight/.style={pos = 0.75, fill = white}}
%\tikzset{earlyweight/.style={pos = 0.25, fill=white}}

%\usepackage{natbib}

%graphics stuff
\usepackage{graphicx}
\graphicspath{ {./images/} }
\usepackage[style=numeric, backend=biber]{biblatex} % Use the numeric style for Vancouver
\addbibresource{the_bibliography.bib}
%code stuff
%when using minted, make sure to add the -shell-escape flag
%you can use lstlisting if you don't want to use minted
%\usepackage{minted}
%\usemintedstyle{pastie}
%\newminted[javacode]{java}{frame=lines,framesep=2mm,linenos=true,fontsize=\footnotesize,tabsize=3,autogobble,}
%\newminted[cppcode]{cpp}{frame=lines,framesep=2mm,linenos=true,fontsize=\footnotesize,tabsize=3,autogobble,}

%\usepackage{listings}
%\usepackage{color}
%\definecolor{dkgreen}{rgb}{0,0.6,0}
%\definecolor{gray}{rgb}{0.5,0.5,0.5}
%\definecolor{mauve}{rgb}{0.58,0,0.82}
%
%\lstset{frame=tb,
%	language=Java,
%	aboveskip=3mm,
%	belowskip=3mm,
%	showstringspaces=false,
%	columns=flexible,
%	basicstyle={\small\ttfamily},
%	numbers=none,
%	numberstyle=\tiny\color{gray},
%	keywordstyle=\color{blue},
%	commentstyle=\color{dkgreen},
%	stringstyle=\color{mauve},
%	breaklines=true,
%	breakatwhitespace=true,
%	tabsize=3
%}
% text + color boxes
\usepackage[most]{tcolorbox}
\tcbuselibrary{breakable}
\tcbuselibrary{skins}
\newtcolorbox{problem}[1]{colback=white,enhanced,title={\small #1},
          attach boxed title to top center=
{yshift=-\tcboxedtitleheight/2},
boxed title style={size=small,colback=black!60!white}, sharp corners, breakable}
%including PDFs
%\usepackage{pdfpages}
\setlength{\parindent}{0pt}
\usepackage{cancel}
\pagestyle{fancy}
\fancyhf{}
\rhead{Avinash Iyer}
\lhead{Algebra II: Class Notes}
\newcommand{\card}{\text{card}}
\newcommand{\ran}{\text{ran}}
\newcommand{\N}{\mathbb{N}}
\newcommand{\Q}{\mathbb{Q}}
\newcommand{\Z}{\mathbb{Z}}
\newcommand{\R}{\mathbb{R}}
\newcommand{\C}{\mathbb{C}}
\setcounter{secnumdepth}{0}
\begin{document}
  \section{Motivation and Introduction}%
  Main purpose of this course is to study Galois theory --- a field that arose in trying to study roots of polynomials.\\

  Consider $f(x) = ax^2 + bx + c$. If we want to find a general, closed-form expression for the roots of the function, we complete the square.
  \begin{align*}
    \text{roots} &= \frac{-b \pm \sqrt{b^2-4ac}}{2a}.
  \end{align*}
  We found these roots by by the coefficients, $\Q$, addition, subtraction, multiplication, division, and square root (raising to the $1/2$ power: see Math 310 notes, Page 104). Naturally, this leads us to ask whether we can do this for cubic polynomials with the same operations. Obviously, we have to change from $1/2$ power to the $1/3$ power, but Cardano showed that it was possible to solve a cubic and quartic equation using these traditional operations and radicals.\\

  Évariste Galois invented his theory to prove there is no such closed formula by radicals for any polynomial of degree $5$ or above.\\

  For example, $x^5 - x + 1$ does not have roots given by radicals.
  \subsection{Example: A Solvable Polynomial}%
  Consider the polynomial $f(x) = x^2 - 2$. We know that the roots of this polynomial are $\pm \sqrt{2}$. From this, we want to create a set $K(f)$ that satisfies the following rules:
  \begin{itemize}
    \item $\Q \subseteq K(f)$.
    \item $K(f)$ must contain the roots of $f$.
    \item $K(f)$ must be closed under the traditional operations: $+,-,\times,/$
    \item $K(f)$ must be the smallest field that satisfies the above three requirements.
  \end{itemize}
  \textbf{Claim:} $K(f) = \Q(\sqrt{2}) = \{a + b\sqrt{2}\mid a,b\in \Q\}$.
  \begin{itemize}
    \item $\Q\subseteq K(f)$, because we can set $b=0$.
    \item $\sqrt{2} = 0 + (1)(\sqrt{2})$, $-\sqrt{2} = 0 + (-1)(\sqrt{2})$
    \item Let $a+b\sqrt{2}$ and $c+d\sqrt{2}$ be elements of $K(f)$. Then,
      \begin{itemize}
        \item $(a+b\sqrt{2})\pm (c+d\sqrt{2}) = (a\pm c) + (b\pm d)\sqrt{2}$
        \item $(a+b\sqrt{2})(c+d\sqrt{2}) = (ac + 2bd) + (ad + bc)\sqrt{2}$
        \item Set $c+d\sqrt{2} \neq 0$
          \begin{align*}
            \frac{a+b\sqrt{2}}{c+d\sqrt{2}} &= \frac{(a+b\sqrt{2})(c-d\sqrt{2})}{c^2-2d^2}\\
                                            &= \frac{1}{c^2-2d^2}\left((ac-2bd) + (bc-ad)\sqrt{2}\right)\\
                                            &= \frac{ac-2bd}{c^2-2d^2} + \frac{bc-ad}{c^2-2d^2}\sqrt{2}
          \end{align*}
      \end{itemize}
    \item $K(f)$ is indeed the smallest set.
      \begin{itemize}
        \item Note that $K(f)$ is a $\Q$-vector space, with basis $\{1,\sqrt{2}\}$. Therefore, $\text{dim}_{\Q} K(f) = 2$. $K(f)$ is known as the ``splitting field'' of $f$.
      \end{itemize}
  \end{itemize}

  We want to consider a bijective function $\varphi: K(f)\rightarrow K(f)$ with the following properties:
  \begin{itemize}
    \item $\varphi(r) = r$ for every $r\in\Q$
    \item $\varphi(x+y) = \varphi(x) + \varphi(y)$
    \item $\varphi(xy) = \varphi(x)\varphi(y)$
  \end{itemize}
  We denote the collection of all such $\varphi$ as $\text{Aut}(K(f)/\Q)$. This is a group under the operation $\circ$ (composition). Specifically, we have
  \begin{align*}
    \varphi(a+b\sqrt{2}) &= \varphi(a) + \varphi(b)\varphi(\sqrt{2})\\
                         &= a + b\varphi(\sqrt{2}).
  \end{align*}
  Notice
  \begin{align*}
    \left(\varphi(\sqrt{2})\right)^2 - 2 &= \varphi \left(\left(\sqrt{2}\right)^2 - 2\right)\\
                            &= \varphi(0)\\
                            &= 0.
  \end{align*}
  Therefore, $\varphi(\sqrt{2}) = \pm \sqrt{2}$. Therefore, we have that the elements of $\text{Aut}(K(f)/\Q)$ as the following:
  \begin{align*}
    \varphi_0: a+b\sqrt{2} \mapsto a+b\sqrt{2}\\
    \varphi_1: a+b\sqrt{2} \mapsto a-b\sqrt{2}\\
    \varphi_1\circ\varphi_1 = \varphi_0\\
    \shortintertext{Thus,}
    \text{Aut}(K(f)/\Q) &= \{\varphi_0,\varphi_1\}\\
                        &\cong \Z/2\Z
  \end{align*}
  \subsection{Example: A Harder Polynomial}%
  Let $f(x) = (x^2-2)(x^2-3)$. Our roots are $\{\pm\sqrt{2},\pm\sqrt{3}\}$. We want to form $K(f)$ with the same properties. Let
  \begin{align*}
    K(f) &= \Q(\sqrt{2},\sqrt{3})\\
    &= \{a+b\sqrt{2}+c\sqrt{3}+d\sqrt{6}\mid a,b,c,d\in\Q\}.
  \end{align*}
  Just as with our previous example, $K(f)$ is a vector space over $\Q$, with basis $\{1,\sqrt{2},\sqrt{3},\sqrt{6}\}$, so $\text{dim}_{\Q}K(f) = 4$.\\

  Now, we want $\text{Aut}(K(f)/\Q)$. If $\varphi\in \text{Aut}(K(f)/\Q)$, then
  \begin{align*}
    \varphi(a+b\sqrt{2}+c\sqrt{3}+d\sqrt{6}) &= a+b\varphi(\sqrt{2}) + c\varphi(\sqrt{3}) + d\varphi(\sqrt{6})\\
                                             &= a+b\varphi(\sqrt{2}) + c\varphi(\sqrt{3}) + d\varphi(\sqrt{2})\varphi(\sqrt{3}).
  \end{align*}
  Thus, we need to know $\varphi(\sqrt{2})$ and $\varphi(\sqrt{3})$. So,
  \begin{align*}
    f(\varphi(\sqrt{2})) &= \left(\left(\varphi(\sqrt{2})\right)^2 - 2\right)\left(\left(\varphi(\sqrt{2})\right)^2-3\right)\\
                         &= 0\\
                         \shortintertext{and the same is the case with $\varphi(\sqrt{3})$. So,}
    \varphi(\sqrt{2}) &\in \{\pm\sqrt{2}, \pm\sqrt{3}\}\\
    \varphi(\sqrt{3}) &\in \{\pm\sqrt{2},\pm\sqrt{3}\}.\\
    \shortintertext{Suppose $\varphi(\sqrt{2}) = \sqrt{3}$. Then,}
    \left(\left(\varphi(\sqrt{2})\right)^2\right) &= (\sqrt{3}^2-1)\\
                                                  &= 0\\
                                                  &=\left(\varphi(2)-3\right)\\
                                                  &= -1.~\bot\\
                                                  \shortintertext{Thus,}
    \varphi(\sqrt{2}) &\in \{\pm\sqrt{2}\}\\
    \varphi(\sqrt{3}) &\in \{\pm\sqrt{3}\},\\
    \shortintertext{and we have the maps as:}
    \varphi_0&: \sqrt{2}\mapsto\sqrt{2},\sqrt{3}\mapsto\sqrt{3}\\
    \varphi_1&: \sqrt{2}\mapsto-\sqrt{2},\sqrt{3}\mapsto\sqrt{3}\\
    \varphi_2&: \sqrt{2}\mapsto\sqrt{2},\sqrt{3}\mapsto-\sqrt{3}\\
    \varphi_3&: \sqrt{2}\mapsto-\sqrt{2},\sqrt{3}\mapsto-\sqrt{3}\\
  \end{align*}
  %\begin{center}
  %  \includegraphics[width=\textwidth]{images/automorphism_lattice.png}
  %\end{center}
  % automorphism group lattice
  \subsection{Example: A Cubic Polynomial}%
  Consider the function $f(x) = x^3 - 2$. The function has one real root, $r_1=\sqrt[3]{2}$, and two complex roots. Let's examine $\Q(\sqrt[3]{2}) = \{a + b\sqrt[3]{2} + c\sqrt[3]{4}\mid a,b,c\in\Q\}$; $r_2$ and $r_3$ are not in $Q(\sqrt[3]{2})$. We could instead consider $\Q(\sqrt[3]{2},r_1,r_2)$.
  \begin{align*}
    x^3-2 &= (x-r_1)(x^2 + r_1x + r_1^2)\\
    r_2 &= \frac{-r_1 + \sqrt{r_1^2-4r_1^2}}{2}\\
        &= r_1\frac{-1+\sqrt{-3}}{2}\\
        &= r_1\zeta_3\\
    r_3 &= r_1\frac{-1 - \sqrt{-3}}{2}\\
        &= r_1\zeta_3^2
  \end{align*}
  However, including $r_2$ and $r_3$ is excessive --- all we need is $\Q(\sqrt[3]{2},\zeta_3)$. Therefore, the basis of this vector space is $\{1,r_1,r_1^2,\zeta_3,\zeta_3 r_1,\zeta_3 r_1^2\}$ (note that $\zeta_3^2 = -1-\zeta_3$). Therefore, $\text{dim}_{\Q}\Q(\sqrt[3]{2},\zeta_3)=6$, and $\Q(\sqrt[3]{2},\zeta_3) = K(f)$.Additionally, we have $\text{Aut}(\Q(\sqrt[3]{2})/\Q) = \{\varphi_0\}$, but $\dim_{\Q}\Q(\sqrt[3]{2}) = 3$. For the full field extension, we need to find $\varphi(\sqrt[3]{2})$ and $\varphi(\zeta_3)$.
  \begin{align*}
    \varphi(\sqrt[3]{2}) &\in \{r_1,\zeta_3 r_1,\zeta_3^2 r_1\}\\
    \varphi(\zeta) &\in \{\zeta_3,\zeta_3^2\}\\
    \varphi_0 &: r_1\mapsto r_1,\zeta_3\mapsto \zeta_3\\
    \varphi_1 &: r_1\mapsto \zeta_3r_1,\zeta_3\mapsto \zeta_3\\
    \varphi_2 &: r_1\mapsto r_1,\zeta_3\mapsto \zeta_3^2\\
    \varphi_3 &: r_1\mapsto \zeta_3^2r_1,\zeta_3\mapsto \zeta_3\\
    \varphi_4 &: r_1\mapsto \zeta_3r_1,\zeta_3\mapsto \zeta_3^2\\
    \varphi_5 &: r_1\mapsto \zeta_3^2r_1,\zeta_3\mapsto \zeta_3^2\\
    \shortintertext{Therefore,}
    \text{Aut}(\Q(\sqrt[3]{2},\zeta_3)/\Q) &= 6\\
                                           &= \dim_{\Q}\Q(\sqrt[3],\sqrt[3]{2})
  \end{align*}
  \section{Rings}%
  Consider the integers under the normal operations, $(\Z,+,\cdot)$; this will serve as the motivation for rings in the future.
  \subsection{Definition of a Ring}%
   Let $R$ be a nonempty set with operations $(+,\cdot)$, with the following properties:
    \begin{enumerate}[(1)]
      \item $(R,+)$ is an abelian group:
        \begin{itemize}
          \item Closed: $r_1 + r_2\in R,~ \forall r_1,r_2\in R$
          \item Identity: $\exists 0_R,~r + 0_R = 0_R+r = r$
          \item Associativity: $r_1 + (r_2 + r_3) = (r_1 + r_2) + r_3$
          \item Inverse: $\forall r\in R,~\exists -r\in R, r + (-r) = 0_R$
          \item Commutativity: $r_1 + r_2 = r_2 + r_1$
        \end{itemize}
      \item Closure under Multiplication: $r_1\cdot r_2\in R,~\forall r_1,r_2\in R$
      \item Associativity under Multiplication: $r_1\cdot (r_2 \cdot r_3) = (r_1\cdot r_2)\cdot r_3$
      \item Distributivity: $r_1\cdot (r_2 + r_3) = r_1\cdot r_2 + r_2\cdot r_3, (r_1 + r_2)\cdot r_3 = r_1\cdot r_3 + r_2\cdot r_3$
    \end{enumerate}
  We say $(R,+,\cdot)$ is a ring if it satisfies all these properties.\\

  If $\exists 1_R\in R$ such that $r\cdot 1_R = 1_R \cdot r = r$, then we say $R$ is a ring with identity, and $1_R$ is the multiplicative identity. If multiplication is commutative, then $R$ is known as a commutative ring.
  \subsubsection{Examples}%
  \begin{enumerate}[(1)]
    \item $(\Z,+,\cdot)$, $(\Q,+,\cdot)$, $(\R,+,\cdot)$, $(\mathbb{C},+,\cdot)$ are commutative rings with identity value of $1$.
    \item $(\Z/n\Z,+,\cdot)$ is a commutative ring with identity $1_{R} = [1]_n$.
    \item $(\R[x],+,\cdot)$, where $\displaystyle\R[x] = \left\{\sum_{i=0}^{n}a_ix^i\mid a_i\in\R\right\}$, is a commutative ring with identity.
    \item $(2\Z,+,\cdot)$ is a commutative ring \textit{without} identity.
    \item $(\text{Mat}_{n}(\R),+,\cdot)$, where $\text{Mat}_n(\R)$ refers to $n\times n$ matrices with real entries, is a \textit{non}commutative ring with identity.
  \end{enumerate}
  \subsection{Division Rings and Fields}%
  Let $R$ be a ring with identity. We say $R$ is a \textit{division ring} if $\forall r\in R\setminus \{0_R\}$, $\exists r^{-1}\in R$ with $r\cdot r^{-1} = 1_R = r^{-1}\cdot r$. If $R$ is also commutative, then $R$ is a \textit{field}.
  \subsubsection{Examples}%
  \begin{enumerate}[(1)]
    \item $(\Q,+,\cdot)$, $(\R,+,\cdot)$, and $(\mathbb{C},+,\cdot)$ are all fields.
    \item Let $p$ be prime, and set $F = \Z/p\Z$. Then, $F$ is a field; we denote this $\mathbb{F}_p$.
    \item Define 
      \begin{align*}
        \mathbb{H} = \{a+bi+cj+dk\mid a,b,c,d\in\R,i^2=j^2=k^2=-1,ij=k=-ji,jk=i=-kj,ki=j=-ik\}.
      \end{align*}
      Then, $\mathbb{H}$ is a division ring, known as the Hamiltonian quaternions. Note that $\mathbb{C}\subset \mathbb{H}$.
  \end{enumerate}
  \subsection{Properties of Rings}%
  \begin{description}
    \item[Proposition 4.1:] Let $R$ be a ring.
      \begin{enumerate}[(1)]
        \item $0_R a = a0_r = 0$ $\forall a\in R$
        \item $(-a)b = a(-b) = -(ab)$ $\forall a,b\in R$
        \item $(-a)(-b) = ab$ $\forall a,b\in R$
        \item If $\exists 1_R\in R$, then $1_R$ is unique, and $-a = (-1_R)a$.
      \end{enumerate}
    \item[Proof of (1):] Let $a\in R$. Then,
      \begin{align*}
        0_Ra &= (0_R + 0_R)a \tag*{Additive Inverse}\\
        0_Ra &= 0_Ra + 0_Ra \tag*{Distributivity}\\
        0_Ra + (-0_Ra) &= 0_Ra + 0_Ra (-0_Ra)\\
        0_R &= 0_Ra. \tag*{Additive Inverse}
      \end{align*}
    \item[Proof of (2):] Let $a,b\in R$. Note that $-(ab)$ is the unique inverse such that $ab + (-(ab)) = 0_R$ via group theory. We have
      \begin{align*}
        ab + (-a)b &= (a+(-a))b \tag*{Distributivity}\\
                   &= (0_R)b\tag*{Additive Inverse}\\
                   &= 0_R. \tag*{By Property (1)}
      \end{align*}
      Thus, $(-a)b = -(ab)$.
  \end{description}
  \subsection{Zero Divisor and Units in Rings}%
    Let $a\in R$, $a\neq 0_R$. If $\exists b\in R$ with $b\neq 0_R$ such that $ab = 0_R = ba$, then we say $a$ is a zero divisor.\\

    If $1_R \in R$, we say $u\in R$ is a unit if $\exists v\in R$ (can be equal to $u$) with $uv = 1_R = vu$. The collection of units in $R$ is denoted $R^{\times}$.
    \begin{description}
      \tiny
      \item[Exercise:] Show that $R^{\times}$ is a group under multiplication.
    \end{description}
    \subsubsection{Examples}%
  \begin{enumerate}[(1)]
    \item Let $R = \Z/6\Z$. Note that $[2]_6 [3]_{6} = [6]_{6} = [0]_{6}$, so both $[2]_6$ and $[3]_{6}$ are both zero divisors. Additionally, $[4]_6[3]_6 = [6]_{6} = [0]_{6}$. Meanwhile, since $(\Z/6\Z)^{\times}=\{[1]_{6},[5]_{6}\}$, those are the two units of $\Z/6\Z$.
    \item $\Z$ has no zero divisors. $\Z^{\times} = \{\pm 1\}$.
    \item $\Q$ has no zero divisors. $\Q^{\times} = \Q\setminus \{0\}$.
    \item $\Z[i] = \{a+bi\mid a,b\in\Z,i^2=-1\}$ has no zero divisors (as $\mathbb{C}$ is a field). $\Z[i]^{\times} = \{\pm 1,\pm i\}$.
  \end{enumerate}
  \subsection{Subrings}%
  Let $(R,+,\times)$. If $S\subseteq R$ is a nonempty subset, and $(S,+,\cdot)$ is a ring, then $S$ is a subring of $R$. To see $S$ is a subring, it is enough to show:
  \begin{itemize}
    \item $S\neq \emptyset$.
    \item $S$ is closed under subtraction.
    \item $S$ is closed under multiplication of elements in $S$.
  \end{itemize}
  \subsubsection{Examples}%
  \begin{enumerate}[(1)]
    \item 
    \begin{align*}
      \underbrace{\Z\subseteq\Q\subseteq\R\subseteq \C}_{\text{subrings}}
    \end{align*}
  \item $\R\subseteq \R[x]$ is a subring.
  \item $S = \{[0]_4,[2]_4\}\subseteq \Z/4\Z$ is a subring.
  \end{enumerate}
  \subsection{Integral Domains}%
  Let $R$ be a commutative ring with identity. We say $R$ is an integral domain if $R$ has no zero divisors.
  \subsubsection{Examples}%
  \begin{enumerate}[(1)]
    \item $\Z$, the integers, is an integral domain, that is not a field.
    \item All fields are integral domains.
    \item $\Z/6\Z$ is \textit{not} an integral domain, as it has zero divisors.
    \item $\Z/n\Z$ is not an integral domain if $n$ is composite.
  \end{enumerate}
  Integral domains are nice due to allowance of cancellations. For example, if $2m = 2n$ in $\Z$, then we find $2(m-n) = 0$, and since $\Z$ has no zero divisors, it must be the case that $m=n$.\\

  However, in a ring that is not an integral domain, such as $\Z/6\Z$, we cannot use the same technique to find the solution to a similar equation. For example, $3\cdot 2 = 0 = 3\cdot 4$, but $2\neq 4$.
  \subsubsection{Proposition: Equations in Integral Domains}%
  Let $R$ be an integral domain. If $a,b,c\in R$ with $a\neq 0_R$, and $ab = ac$, then $b=c$.\\

  \textbf{Proof:}
  \begin{align*}
    ab &= ac\\
    a(b-c) &= 0_R\\
    \shortintertext{Since $a\neq 0$,}
    b-c &= 0_R\\
    b &= c.
  \end{align*}
  \subsubsection{Theorem: Finite Integral Domains and Fields}%
  If $R$ is an integral domain, and $\card(R) < \infty$ ,then $R$ is a field.\\

  \textbf{Proof:} Let $a\in R$, $a\neq 0_{R}$. Note $ab \neq 0_R$ for all $b\in R,~b\neq 0_R$.\\

  Define $\varphi_a: R\setminus\{0_R\} \rightarrow R\setminus\{0_R\}$, $b\mapsto ab$. If $\varphi_a(b) = \varphi_a(c)$, then $ab = ac$, and by our previous result, $b=c$ --- therefore, $\varphi_a$ is injective.\\

  Since $R\setminus \{0_R\}$ is finite, and $\varphi_a$ is injective, then $\varphi_a$ is surjective. In particular, this means $\exists b\in R\setminus\{0_R\}$ with $\varphi_a(b) = 1_R$; therefore, $ab = 1_R$. Since $R$ is commutative, $ba = 1_R$, so $b = a^{-1}$.
  \subsubsection{Examples of Abstract Rings}%
  
  \subsubsection{Ring of Integers in a Field}%
  Let $d\in \Z$, $d$ is square-free (there is no square that divides $d$). Set $\Q(\sqrt{d}) = \{a + b\sqrt{d}\mid a,b\in\Q\} \subseteq \C$. This is a field (can be verified as a subfield of $\C$).\\

  We can define
  \begin{align*}
    \mathcal{O}_{\Q\left(\sqrt{d}\right)} &= \begin{cases}
      \Z[\sqrt{d}] = \{a + b\sqrt{d}\mid a,b\in\Z\} & d \equiv 2,3\mod 4\\
      \Z\left[\frac{1 + \sqrt{d}}{2}\right] = \{a + b\left(\frac{1+\sqrt{d}}{2}\right)\mid a,b\in\Z\} & d\equiv 1\mod 4
    \end{cases}.
  \end{align*}
  Then, $\mathcal{O}_{\Q(\sqrt{d})}$ is a subring of $\Q(\sqrt{d})$. This is known as the ring of integers of $\Q(\sqrt{d})$. This set behaves in $\Q(\sqrt{d})$ the same say that $\Z$ does inside $\Q$. The set $\mathcal{O}_{\Q(\sqrt{d})}$ is the collection of all roots in $\Q(\sqrt{d})$ of monic (coefficient of highest degree is 1) polynomials with coefficients in $\Z$.\\

  For example, if $d = -1$, defining $\Q(i)$, then we can verify that $\Z[i]$ is a root of a monic polynomial with coefficients in $\Z$.
  \subsubsection{Ring of Matrices}%
  Let $R$ be a ring. Then,
  \begin{align*}
    \text{Mat}_{n}(R) &= \{\text{$n\times n$ matrices with entries in $R$}\}
  \end{align*}
  is a ring under matrix addition and multiplication.
  \subsubsection{Ring of Functions}%
  Let $L^{1}(\R)$ be all functions $f: \R\rightarrow\R$ such that
  \begin{align*}
    \int_{\R}|f(x)|dx
  \end{align*}
  exists. The set $L^{1}(\R)$ is a ring under pointwise addition and convolution, where convolution is defined as
  \begin{align*}
    \left(f\ast g\right)(x) &= \int_{\R}f(x-y)g(y)dy.
  \end{align*}
  This is a commutative ring without identity.
  \subsubsection{Group Ring}%
  Let $K$ be a field and $G$ a group. Set $K[G]$ to be all formal linear combinations of the form
  \begin{align*}
    \alpha = \sum_{x\in G} a_x x,
  \end{align*}
  with $a_x\in K$, $x\in G$, with $a_x = 0$ for all but finitely many $x$.\\

  Given
  \begin{align*}
    \alpha &= \sum_{x\in G}a_x x\\
    \alpha &= \sum_{y\in G}b_y y,
  \end{align*}
  define
  \begin{align*}
    \alpha + \beta &= \sum_{x\in G}(a_x + b_x) x\\
    \alpha\beta &= \sum_{x\in G}\sum_{y\in G}a_xb_yxy\\
                &= \sum_{z\in G}\left(\sum_{xy=z}a_xb_y\right)z.
  \end{align*}
  This is a ring under these operations, known as the group ring. It is commutative if and only if $G$ is abelian.
  \subsubsection{Polynomials under a Ring}%
  Let $R$ be a ring. Set
  \begin{align*}
    R[x] = \left\{\sum_{i=1}^{n}a_ix^i\mid a_i\in R,n\in \Z_{\geq 0}\right\}
  \end{align*}
  to be the all polynomials with coefficients in $R$. This is a ring under polynomial addition and multiplication. If $R$ is commutative, then $R[x]$ is commutative.\\

  \subsubsection{Proposition: Polynomial Properties}%
  Let $R$ be an integral domain, with $p(x),q(x)\in R[x]\setminus\{0\}$. Then:
  \begin{enumerate}[(1)]
    \item $\text{deg}(p(x)q(x)) = \text{deg}(p(x)) + \text{deg}(q(x))$
    \item $R[x]^{\times} = R^{\times}$
    \item $R[x]$ is an integral domain.
  \end{enumerate}
  \begin{description}
    \item[Proof of (1):] Let
      \begin{align*}
        p(x) &= a_mx^m + \cdots + a_1x + a_0\\
        q(x) &= b_nx^n + \cdots + b_1x + b_0
      \end{align*}
      with $a_m,b_n\neq 0$ --- $\text{deg}(p) = m$ and $\text{deg}(q) = n$. Then,
      \begin{align*}
        p(x)q(x) = a_mb_nx^{m+n} + \text{lower degree terms},
      \end{align*}
      and since $a_mb_n\neq 0$ as $R$ is an integral domain with $a_m,b_n\neq 0$, $\text{deg}(pq) = m+n$.
  \end{description}
  \subsection{Ring Homomorphism}%
  Let $R$ and $S$ be rings. A ring homomorphism between $R$ and $S$ is a map $\varphi: R\rightarrow S$ that satisfies the following properties for all $r_1,r_2\in R$:
  \begin{enumerate}[(1)]
    \item $\displaystyle\varphi\left(r_1 +_{\tiny R} r_2\right) = \varphi(r_1) +_{\tiny S} \varphi(r_2)$
    \item $\displaystyle\varphi \left(r_1 \cdot_{\tiny R} r_2\right) = \varphi(r_1) \cdot_{\tiny S} \varphi(r_2)$
  \end{enumerate}
  The kernel of a ring homomorphism $\varphi$ is given by
  \begin{align*}
    \ker(\varphi): \{r\in R\mid \varphi(r) = 0_S\}
  \end{align*}
  A bijective ring homomorphism is called an isomorphism. If there exists such a bijection between $R$ and $S$, we say $R$ and $S$ are isomorphic.\\

  If $\varphi$ is an isomorphism, we write
  \begin{align*}
    \varphi: R\xrightarrow{\simeq}S
  \end{align*}
  \subsection{Examples: Ring Homomorphisms}%
  \subsubsection{Not a Ring Homomorphism}%
  Let $R = \Z$ and $S = 2\Z$. Define
  \begin{align*}
    \varphi: \Z&\rightarrow 2\Z\\
    n&\mapsto 2n.
  \end{align*}
  Let $m,n\in\Z$. We have
  \begin{align*}
    \varphi(m+n) &= 2(m+n)\\
                 &= 2m + 2n\\
                 &= \varphi(m) + \varphi(n).
  \end{align*}
  However,
  \begin{align*}
    \varphi(mn) &= 2(mn)\\
    \varphi(m)\varphi(n) &= 4(mn).
  \end{align*}
  \subsubsection{Homomorphism between Integers and Integers Modulo $n$}%
  Consider $R = \Z$ and $S = \Z/n\Z$. Define
  \begin{align*}
    \varphi:\Z&\rightarrow \Z/n\Z\\
    a&\mapsto [a]_{n}.
  \end{align*}
  Let $a,b\in\Z$. We have
  \begin{align*}
    \varphi(a+b) &= [a+b]_{n}\\
                 &= [a]_{n} + [b]_n\\
                 &= \varphi(a) + \varphi(b).
  \end{align*}
  Additionally, we have
  \begin{align*}
    \varphi(ab) &= [ab]_n\\
                &= [a]_n[b]_n\\
                &= \varphi(a)\varphi(b).
  \end{align*}
  So, $\varphi$ is a ring homomorphism. Note that
  \begin{align*}
    \ker(\varphi) &= \{a\in \Z\mid \varphi(a) = [0]_n\}\\
                  &= \{a\in \Z\mid [a]_n = [0]_n\}\\
                  &= \{a\in \Z\mid n | a\}\\
                  &= n\Z.
  \end{align*}
  \subsubsection{Homomorphism Between the Polynomials and Reals}%
  Let $S = \R[x]$ and $T = \R$. Define
  \begin{align*}
    \varphi_{a}: \R[x]&\rightarrow \R\\
    f&\mapsto f(a)\\
  \end{align*}
  Let $f(x),g(x) = \R[x]$. Then,
  \begin{align*}
    \varphi_a(f(x) + \varphi(g)(x)) &= \varphi_a((a_0 + b_0) + \cdots + (a_m + b_m)x^m + b_{m+1}x^{m+1} + \cdots b_nx^n)\\
                                    &= (a_0 + b_0) + \cdots + (a_m + b_m)a^m + b_{m+1}a^{m+1} + \cdots + b_na^n\\
                                    &= \varphi_a(f(x)) + \varphi_a(g(x)).
  \end{align*}
  Similarly, we can verify that $\varphi_a(f(x)g(x)) = \varphi_a(f(x))\varphi_a(g(x))$. So, $\varphi_a$ is a ring homomorphism. Note that
  \begin{align*}
    \ker(\varphi_a) &= \{f(x)\in \R[x]\mid f(a) = 0\}\\
                    &= \{f(x)\in \R[x]\mid (x-a)|f(x)\}\\
                    &= (x-a)\R[x]
  \end{align*}
  \subsubsection{Homomorphism between Matrices}%
  Define
  \begin{align*}
    R &= \left\{ \begin{bmatrix}a&b\\0&d\end{bmatrix}\in \text{Mat}_{2}(\R)\right\}\\
    S &= \R,
  \end{align*}
  and
  \begin{align*}
    \varphi:R&\rightarrow S\\
    \begin{bmatrix}a&b\\0&d\end{bmatrix}&\mapsto a.
  \end{align*}
  Then,
  \begin{align*}
    \varphi \left(\begin{bmatrix}a_1&b_1\\0&d_1\end{bmatrix}+\begin{bmatrix}a_2&b_2\\0&d_2\end{bmatrix}\right) &= \varphi \left(\begin{bmatrix}a_1+a_2&b_1+b_2\\0&d_1+d_2\end{bmatrix}\right)\\
                                    &= a_1 + a_2\\
                                    &= \varphi \left(\begin{bmatrix}a_1&b_1\\0&d_1\end{bmatrix}\right) + \varphi \left(\begin{bmatrix}a_2&b_2\\0&d_2\end{bmatrix}\right),
  \end{align*}
  and
  \begin{align*}
    \varphi \left(\begin{bmatrix}a_1&b_1\\0&d_1\end{bmatrix}\begin{bmatrix}a_2&b_2\\0&d_2\end{bmatrix}\right) &= \varphi \left(\begin{bmatrix}a_1a_2&a_1b_2 + b_1d_2\\0&d_1d_2\end{bmatrix}\right)\\
                                    &= a_1 a_2\\
                                    &= \varphi \left(\begin{bmatrix}a_1&b_1\\0&d_1\end{bmatrix}\right)  \varphi \left(\begin{bmatrix}a_2&b_2\\0&d_2\end{bmatrix}\right).
  \end{align*}
  So $\varphi$ is a ring homomorphism that is surjective but not injective. Note
  \begin{align*}
    \ker(\varphi) &= \left\{ \begin{bmatrix}0&b\\0&d\end{bmatrix}\mid b,d\in\R\right\}.
  \end{align*}
  \subsubsection{Proposition: Fundamental Theorem of Ring Homomorphisms}%
  Let $\varphi: R\rightarrow S$ be a ring homomorphism.
  \begin{enumerate}[(1)]
    \item The image of $\varphi$, $\varphi(R) = \{s\in S\mid s = \varphi(r)\text{ for some }r\in R\}$, is a subring of $S$.
    \item The kernel, $\ker(\varphi)$, is a subring of $R$.\\

      Additionally, for any $r\in R$, and $a\in \ker(\varphi)$, $ar\in \ker(\varphi)$ and $ra\in \ker(\varphi)$.
  \end{enumerate}
  \begin{description}
    \item[Proof of (2):] To show $\ker(\varphi)$ is a subring, we must show that $\ker(\varphi)$ is non-empty, closed under subtraction, and closed under multiplication.\\

      First, since $\varphi(0_R) = 0_S$ (verify this), $\ker(\varphi)$ is non-empty.\\

      Let $a,b\in\ker(\varphi)$. We have
      \begin{align*}
        \varphi(a-b) &= \varphi(a + (-b))\\
                     &= \varphi(a) + \varphi(-b)\\
                     &= \varphi(a)-\varphi(b) \tag*{check $\varphi(-b) = -\varphi(b)$}\\
                     &= 0_S - 0_S\\
                     &= 0_S.
      \end{align*}
      Thus, $a-b\in\ker(\varphi)$, and $\ker(\varphi)$ is closed under subtraction.\\

      To show $\ker(\varphi)$ is closed under multiplication, we will prove the general case. Let $a\in\ker(\varphi)$ and $r\in R$. We have
      \begin{align*}
        \varphi(ra) &= \varphi(r)\varphi(a)\\
                    &= \varphi(r)0_S\\
                    &= 0_S.
      \end{align*}
      Similarly, $\varphi(ar) = 0_S$. So, $ar,ra\in\ker(\varphi)$.
  \end{description}
  The stronger condition that we found for $\ker(\varphi)$ (closed under multiplication of all elements of the ring, not merely those from the subring) forms what we call an ideal.
  \subsection{Quotient Rings}%
  \subsubsection{Defining an Equivalence Relation on a Ring}%
  Set $K = \ker(\varphi)$. We will define a relation on $R$, $\sim$, where $r_1 \sim r_2$ if $r_1-r_2\in K$. We want to see if $\sim$ is an equivalence relation:
  \begin{itemize}
    \item Reflexive: $r\sim r$ since $r-r=0_R \in K$.
    \item Symmetric: $r_1\sim r_2$ implies $r_1 - r_2 = k$ for some $k\in K$. Since $k$ is a subring, $-k\in K$, so $r_2 - r_1\in K$.
    \item Transitive: suppose $r_1 \sim r_2$ and $r_2\sim r_3$. This means there are elements $k_1,k_2\in K$ with $r_1-r_2 = k_1$ and $r_2-r_3 = k_2$. Since $K$ is a subring, $(r_1 - r_2) + (r_2 - r_3) = r_1 - r_3 = k_1 + k_2\in K$. Thus, $r_1 \sim r_3$.
  \end{itemize}
  Since $\sim$ is reflexive, symmetric, and transitive, $\sim$ is an equivalence relation on $R$.\\
  
  Since $\sim$ is an equivalence relation on $R$, we will want to examine equivalence classes of $R$ under $\sim$. Specifically, for $r\in R$, we have
  \begin{align*}
    [r]_K &= \{\tilde{r}\in R \mid r-\tilde{r}\in K\}\\
          &= \{\tilde{r}\in R \mid r - \tilde{r} = k\text{ for some }k\in K\}\\
          &= \{r + k\mid k\in K\}\\
          &= r+K.
  \end{align*}
  We will define the set
  \begin{align*}
    R/K = \{r + K\mid r\in R\}
  \end{align*}
  to be the set of all equivalence classes.
  \begin{description}
    \item[Example:] Let $\varphi: \Z\rightarrow \Z/n\Z$, $a\mapsto [a]_n$. Then, $\ker(\varphi) = n\Z$. Then, $R/K = \Z/n\Z$.
  \end{description}
  Let $r_1 + K,r_2+K\in R/K$. The new question is whether or not we can define addition and multiplication on $R/K$. Suppose that the following are the definition of multiplication and addition on $R/K$.
  \begin{align*}
    (r_1 + K) + (r_2 + K) &= (r_1 + r_2) + K\\
    (r_1 + K)(r_2 + K) &= (r_1r_2) + K.
  \end{align*}
  Suppose $r_1 + K = \tilde{r_1} + K$ and $r_2 + K = \tilde{r_2} + K$. This means there are $k_1,k_2\in K$ with $r_1 - \tilde{r_1} = k_1$, $r_2 - \tilde{r_2} = k_2$, or that $r_1 = \tilde{r_1} + k_1$, $r_2 = \tilde{r_2} + k_2$.\\

  To see if the map is well-defined, we have
  \begin{align*}
    (r_1 + K) + (r_2 + K) &= (r_1 + r_2) + K\\
                          &= (\tilde{r_1} + k_1 + \tilde{r_2} + k_2) + K\\
                          &= (\tilde{r_1} + k_1) + K + (\tilde{r_2} + k_2) + K\\
                          &= (\tilde{r_1} + K) + (\tilde{r_2} + K)\\
    \shortintertext{since $\tilde{r_1} + k_1 - \tilde{r_1} = k\in K$.}
  \end{align*}
  Thus, our addition is well-defined.\\

  Examining multiplication, we see that
  \begin{align*}
    (r_1 + K)  (r_2 + K) &= r_1r_2 + K\\
                         &= (\tilde{r_1} + k_1)(\tilde{r_2} + k_2) + K\\
                         &= \tilde{r_1}\tilde{r_2} + \underbrace{k_1\tilde{r_2} + \tilde{r_1}k_2 + k_1k_2}_{\in K \text{ since $K = \ker(\varphi)$}} + K\\
                         &= \tilde{r_1}\tilde{r_2} + K.
  \end{align*}
  Therefore, our multiplication is well-defined.\\

  We can show that $R/K$ is a ring (verify for yourself).
  \begin{description}
    \small
    \item[Note:] This construction would not have worked if $K$ was merely a subring, as multiplication would not be well-defined.
  \end{description}
  \subsubsection{Ideals}%
  Let $I\subseteq R$ be a subring.
  \begin{enumerate}[(1)]
    \item If $ra\in I$ for every $r \in R$, we say $I$ is a left-ideal of $R$.
    \item If $ar\in I$ for every $r\in R$, then we say $I$ is a right-ideal of $R$.
    \item If $I$ is a left-ideal and a right-ideal of $R$, then we say $I$ is an ideal of $R$.
  \end{enumerate}
  If $I\subseteq R$ is an ideal, we define $r_1\sim_{I}r_2$ if $r_1-r_2 \in I$, and $R/I = \{r+I\mid r\in I\}$. Addition and multiplication in $R/I$ are defined as
  \begin{align*}
    (r_1 + I) + (r_2 + I) &= (r_1 + r_2) + I\\
    (r_1 + I)(r_2 + I) &= r_1r_2 + I.
  \end{align*}
  \subsubsection{Examples of Ideals}%
  \begin{enumerate}[(1)]
    \item $n\Z\subseteq \Z$ is an ideal; if $nk\in n\Z$, and $m\in\Z$, then $m(nk) = n(mk) \in n\Z$.
    \item Let $R = \Z[x]$. Set $\langle x^2\rangle = \{f(x)x^2\mid f(x)\in \Z[x]\}$. This is an ideal.
    \item Let $R$ be a ring. If $r\in R$, we define $\langle r \rangle = \{ar \mid a\in R\}$.
    \item Set $I = \{(2n,0)\mid n\in\Z\}$ in $\Z\times\Z$. Let $(a,b)\in \Z\times\Z$. Then, $(a,b)(2n,0) = (2an,0)\in I$, meaning $I$ is an ideal.
    \item Define $R = \left\{ \begin{bmatrix}a&b\\0&d\end{bmatrix}\in \text{Mat}_{2}(\R)\right\}$. Consider $I = \left\{ \begin{bmatrix}a&0\\0&d\end{bmatrix}\mid a,b\in\R\right\}$. Then,
      \begin{align*}
        \begin{bmatrix}a&b\\0&d\end{bmatrix} \begin{bmatrix}s&0\\0&t\end{bmatrix} &= \begin{bmatrix}as & bt\\0&dt\end{bmatrix}\\
        \begin{bmatrix}s&0\\0&t\end{bmatrix}\begin{bmatrix}a&b\\0&d\end{bmatrix} &= \begin{bmatrix}sa & sb\\0&td\end{bmatrix}.
      \end{align*}
      Therefore, $I$ is a subring but not an ideal.
    \item Let $R = \Z[x]$. Consider $I = \langle 2,x\rangle = \{2f(x) + g(x)\mid f(x),g(x)\in \Z[x]\}$. Then,
      \begin{align*}
        (2f_1(x) + xg(x))(2f_2(x) + xg_2(x)) &= 2\left(f_1(x)(2f_2(x) + xg_2(x))\right) + x(g_1(x)(2f_2(x) + xg_2(x)))\\
        h(x)\left(2f(x) + xg(x)\right) &= 2(f(x)h(x)) + x(g(x)h(x)),
      \end{align*}
      meaning $I$ is an ideal.
  \end{enumerate}
  \subsubsection{Examples of Quotient Rings}%
  \begin{enumerate}[(1)]
    \item Let $R = \Z$, $I = n\Z$. Then, $R/I = \Z/n\Z$.
    \item Let $R = \R[x]$, $I = \langle x^2\rangle$ as defined earlier. Then,
      \begin{align*}
        R/I &= \R[x]/\langle x^2\rangle\\
            &= f(x) + \langle x^2 \rangle.\\
            \shortintertext{Other examples include}
        f(x) &= a_nx^n + \cdots + a_1x + a_0 \in \R[x]\\
        f(x) + \langle x^2 \rangle &= a_1 x + a_0 + \langle x^2 \rangle \in \R[x]/\langle x^2\rangle\\
        \R[x]/\langle x^2\rangle &= \{a+bx + \langle x^2 \rangle\mid a,b\in\R\}.\\
        (a+bx + \langle x^2 \rangle)(c + dx \langle x^2 \rangle) &= ac + ad x + bc x + bdx^2 + \langle x^2 \rangle\\
                                                                 &= (ac) + (ad + bc) x + \langle x^2 \rangle\\
        (x+\langle x^2\rangle)^2 &= x^2 + \langle x^2 \rangle\\
                                 &= \langle x^2 \rangle.
      \end{align*}
    \item Let $R = \Z\times\Z$, $I = \{(2n,0)\mid n\in\Z\}$. Then,
      \begin{align*}
        R/I &= \{(a,b) + I\mid a,b\in\Z\}.\\
        (a,b) + I &= ([a]_2,b) + I \tag*{where $[a]_2$ is $a$ modulo 2.}
      \end{align*}
      We would expect that $\varphi: \Z/2\Z\times \Z \rightarrow R/I$, $([a]_2,b)\rightarrow (a,b) + I$ is an isomorphism (verify for yourself).
  \end{enumerate}
  \subsubsection{Isomorphisms to Quotient Rings}%
  Let $R = \Z[x]$, $I = \langle 2,x\rangle$, $J = \langle 2 \rangle = \{2f(x)\mid f(x)\in\Z[x]\}$.
  \begin{align*}
    R/J &= \{f(x) + \langle 2 \rangle \mid f(x)\in \Z[x]\}\\
    f(x) + \langle 2 \rangle &= g(x) + \langle 2 \rangle\\
    \shortintertext{if $2|(f(x)-g(x))$, meaning all coefficients of $f(x)-g(x)$ are divisible by $2$. Therefore,}
    f(x) + \langle 2 \rangle &= 5 + 4x + 7x^2 - 5x^3 \langle 2 \rangle\\
                             &= (1 + (2)(2)) + 2(2x) + x^2 + 2(3x^2) -x^3 - 2(2x^3) + \langle 2 \rangle\\
                             &= 1 + x^2 - x^3 + \langle 2 \rangle\\
                             &= 1 + x^2 - 2(x^3) + x^3 + \langle 2 \rangle\\
                             &= 1 + x^2 + x^3 + \langle 2 \rangle.\\
    (1 + x + x^2 + \langle 2 \rangle) + (x + \langle 2 \rangle) &= 1 + 2x + x^2 + \langle 2 \rangle\\
                                                                &= 1 + x^2 + \langle 2 \rangle.\\
                                                                \shortintertext{Therefore, we can consider}
    \Z[x]/\langle 2 \rangle &= \Z[x]/2\Z[x]\\
                            &\cong \Z/2\Z.
  \end{align*}
  \begin{align*}
    R/I &= \Z[x]/\langle 2,x\rangle\\
    f(x) + \langle 2,x\rangle &= a_nx^n + \cdots + a_1 x + a_0 + \langle 2,x\rangle\\
                              &= a_0 + \langle 2,x \rangle\\
                              &= \begin{cases}
                                0 & 2|a_0\\
                                1 & 2\not| a_0
                              \end{cases},
                              \shortintertext{So, we can consider}
    \Z[x]/\langle 2,x\rangle &\cong \Z/2\Z.
  \end{align*}
  \subsubsection{Isomorphism Example: Complex Numbers to Matrices}%
  Consider the set
  \begin{align*}
    R = \left\{ \begin{bmatrix}a&b\\-b & a\end{bmatrix}\in \text{Mat}_{2}(\R)\right\}.
  \end{align*}
  We can verify that $R$ is a ring.\\

  Define
  \begin{align*}
    \varphi&: \C\rightarrow R\\
    a + bi &\mapsto \begin{bmatrix}a&b\\-b&a\end{bmatrix}.
  \end{align*}
  We can verify that $\varphi$ is a bijective map.\\

  Let $a+bi,c+di\in \C$. Then,
  \begin{align*}
    \varphi((a+bi) + (c+di)) &= \varphi((a+c) + (b+d)i)\\
                             &= \begin{bmatrix}a+c & b+ d\\-(b+d) & a+c\end{bmatrix}\\
                             &= \begin{bmatrix}a&b\\-b&a\end{bmatrix} + \begin{bmatrix}c & d\\-d&c\end{bmatrix}\\
                             &= \varphi(a+bi) + \varphi(c+di),\\
                             \shortintertext{and}
    \varphi((a+bi)(c+di)) &= \varphi((ac-bd) + (ad + bc)i)\\
                          &= \begin{bmatrix}ac-bd & ad + bc\\ -(ad + bc) & ac - bd\end{bmatrix}\\
    \varphi(a+bi)\varphi(c+di) &= \begin{bmatrix}a&b\\-b&a\end{bmatrix} \begin{bmatrix}c&d\\-d&c\end{bmatrix}\\
                               &= \begin{bmatrix}ac-bd & ad + bc\\-(ad+bc) & ac-bd\end{bmatrix}.
  \end{align*}
  Therefore, $\C\cong R$.
  \section{First Isomorphism Theorem}%
  Let $\varphi: R\rightarrow S$ be a homomorphism. We have $R/\ker\varphi \cong \varphi(R)$.
  \subsection{Proof of the First Isomorphism Theorem}%
  We want to show that $R/\ker(\varphi)\cong \varphi(R)$. Without loss of generality, assume $\varphi$ is surjective. Let $K = \ker(\varphi)$.\\

  We define $\Phi: R/K \rightarrow S$, $r+K\mapsto \varphi(r)$. We must show that $\Phi$ is a well-defined map. Let $r_1 + K = r_2 + K$ (meaning $r_1 - r_2 \in K$). This means $r_1 = r_2 + k$ for some $k\in K$. Applying $\Phi$, we have
  \begin{align*}
    \Phi(r_1 + K) &= \varphi(r_1)\\
                  &= \varphi(r_2 + k)\\
                  &= \varphi(r_2) + \varphi(k)\\
                  &= \varphi(r_2)\\
                  &= \Phi(r_2 + K).
  \end{align*}
  Let $r_1 + K$, $r_2 + K \in R/K$. Observe
  \begin{align*}
    \Phi((r_1 + K) + (r_2 + K)) &= \Phi((r_1 + r_2) + K)\\
                                &= \varphi(r_1 + r_2)\\
                                &= \varphi(r_1) + \varphi(r_2)\\
                                &= \Phi(r_1 + K) + \Phi(r_2 + K),
                                \intertext{and}
    \Phi((r_1 + K)(r_2 + K)) &= \Phi(r_1r_2 + K)\\
                             &= \varphi(r_1r_2)\\
                             &= \varphi(r_1)\varphi(r_2)\\
                             &= \Phi(r_1 + K)\Phi(r_2 + K),
  \end{align*}
  meaning $\Phi$ is a homomorphism.\\

  Let $s\in S$. Since $\varphi$ is surjective, there exists $r\in R$ with $\varphi(r) = s$. So, $\Phi(r+K) = \varphi(r) = s$. Thus, $\Phi$ is surjective.\\

  Let $r+K\in \ker(\Phi)$. Then,
  \begin{align*}
    \Phi(r+k) &= 0_S\\
              &= \varphi(r),
  \end{align*}
  meaning $r\in \ker(\varphi) = K$. So, $r+K = 0_{R} + K = 0_{R/K}$. Thus, $\Phi$ is injective.
  \subsection{Using the First Isomorphism Theorem: Example 1}%
  Let $\varphi: \Z[x]\rightarrow \Z/2\Z$, $a_0 + a_1x + \cdots + a_nx^n \mapsto [a_0]_2$.\\

  To apply the first isomorphism theorem, we must check that this is a ring homomorphism. Let
  \begin{align*}
    f &= a_0 + a_1 x + \cdots + a_mx^m\\
    g &= b_0 + b_1 x + \cdots + b_mx^m
  \end{align*}
  be elements in $\Z[x]$. Note that 
  \begin{align*}
    \varphi(f+g) &= \varphi((a_0 + b_0) + \cdots)\\
                 &= [a_0 + b_0]_2\\
                 &= [a_0]_2 + [b_0]_2\\
                 &= \varphi(f) + \varphi(g)
                 \intertext{and}
    \varphi(fg) &= \varphi((a_0b_0) + \cdots)\\
                &=[a_0b_0]_2\\
                &= [a_0]_2 + [b_0]_2\\
                &= \varphi(f)\varphi(g).
  \end{align*}
  So $\varphi$ is a homomorphism. Note that $\varphi(0) = [0]_2$ and $\varphi(1) = [1]_2$. The first isomorphism theorem gives that $\Z[x]/\ker\varphi \cong \Z/2\Z$.\\

  We claim that $\ker\varphi = \langle 2,x\rangle$.\\

  If $2f(x) + xg(x)\in \langle 2,x\rangle$, and we write $f(x) = a_0 + a_1 x + \cdots + a_nx^n$, then
  \begin{align*}
    \varphi(2f(x) + g(x)) &= \varphi(2)\varphi(f(x)) + \varphi(x)\varphi(g(x))\\
                          &= [0]_{2}[a_0]_2 + [0]_2\varphi(g(x))\\
                          &= [0]_2,
  \end{align*}
  so $\langle 2,x \rangle \subseteq \ker\varphi$.\\

  Let $f(x) = a_0 + a_1 x + \cdots + a_nx^n\in \ker(\varphi)$, meaning
  \begin{align*}
  [0]_2 &= \varphi(f(x))\\
        &= [a_0]_2.
  \end{align*}
  Therefore, $a_0 = 2k$. So,
  \begin{align*}
    f(x) &= 2k x(a_1 + a_2x + \cdots + a_nx^{n-1})\\
         &\in \langle 2,x\rangle.
  \end{align*}
  Thus, $\ker(\varphi)\subseteq \langle 2,x\rangle$, meaning $\ker(\varphi) = \langle 2,x\rangle$.\\

  By the first isomorphism theorem, $\Z[x]/\langle 2,x\rangle \cong \Z/2\Z$.
  \subsection{Using the First Isomorphism Theorem: Example 2}%
  We want to find the ring that is isomorphic to $(\Z\times\Z)/(2\Z\times 5\Z)$. We define
  \begin{align*}
    \varphi: \Z\times\Z \rightarrow \Z/2\Z \times \Z/5\Z\\
    (m,n)\mapsto ([m]_2,[n]_5).
  \end{align*}
  We will start by showing homomorphism as follows:
  \begin{align*}
    \varphi((m_1,n_1) + (m_2,n_2)) &= \varphi((m_1 + m_2,n_1+n_2))\\
                                   &= ([m_1 + m_2]_2,[n_1 + n_2]_5)\\
                                   &= ([m_1]_2 + [m_2]_2,[n_1]_5 + [n_2]_5)\\
                                   &= ([m_1]_2,[n_1]_5) + ([m_2]_2,[n_2]_5)\\
                                   &= \varphi((m_1,n_1)) + \varphi((m_2,n_2)),
     \intertext{and similarly for multiplication}
      \varphi((m_1,n_1)(m_2,n_2)) &= \varphi((m_1m_2,n_1n_2))\\
                                &= ([m_1m_2]_2,[n_1n_2]_5)\\
                                &\vdots\\
                                &= \varphi((m_1,n_1))\varphi((m_2,n_2))
  \end{align*}
  Let $([a]_2,[b]_5) \in \Z/2\Z\times\Z/5\Z$. Then, $\varphi((a,b)) = ([a]_2,[b]_5)$. Thus, $\varphi$ is surjective.\\

  Finally, we have $(m,n)\in \ker(\varphi)$ if and only if $[m]_2 = [0]_2$ and $[n]_5 = [0]_5$, meaning $m\in 2\Z$ and $n\in 5\Z$. Therefore, $\ker(\varphi) = 2\Z\times 5\Z$.
  \subsection{Using the First Isomorphism Theorem: Example 3}%
  Consider the map $\Z\rightarrow \Z/2\Z\times \Z/5\Z$, $n\mapsto ([n]_2,[n]_5)$. Note
  \begin{align*}
    \varphi(m+n) &= ([m+n]_2,[m+n]_5)\\
                 &= ([m]_2 + [n]_2,[m]_5 + [n]_5)\\
                 &= ([m]_2,[m]_5) + ([n]_2,[n]_5)\\
                 &= \varphi(m) + \varphi(n),
                 \intertext{and}
    \varphi(mn) &= \varphi(m)\varphi(n).
  \end{align*}
  We want to find if this map is surjective. Let $([a]_2,[b]_5)\in \Z/2\Z\times \Z/5\Z$. We are trying to find $n\in\Z$ such that $[n]_2 = [a]_2$ and $[n]_5 = [b]_5$, or $n\equiv a$ modulo 2 and $n\equiv b$ modulo 5.
  \begin{align*}
    n-a &\equiv 2k \text{ for some $k\in \Z$}\\
    n &\equiv a + 2k\\
    a+2k &\equiv b \text{ modulo 5}\\
    2k &= b-a \text{ modulo 5}\\
    k &= 3(b-a) \text{ modulo 5}\\
    n &= a + 2(3(b-a))\\
      &= a + 6(b-a).
  \end{align*}
  So $\varphi(a + 6(b-a)) = ([a]_2,[b]_5)$. Thus, $\varphi$ is surjective.\\

  Finally, we desire $\ker(\varphi)$. Observe that 
  \begin{align*}
    \ker(\varphi) &= \{n\in\Z \mid [n]_2=[0]_2,[n]_5=[0]_5\}\\
                  &= \{n\in\Z\mid 2|n,5|n\}\\
                  &= \{n\in\Z\mid 10|n\}\\
                  &= 10\Z.
  \end{align*}
  Thus, the first isomorphism theorem gives $\Z/10\Z \equiv \Z/2\Z \times \Z/5\Z$.
  \subsection{Proposition: Ring Homomorphisms and Ideals}%
  Let $R$ be a ring and $I\subseteq R$ be an ideal. The map
  \begin{align*}
    \varphi: R\rightarrow R/I\\
    r\mapsto r+I
  \end{align*}
  is a surjective ring homomorphism with $\ker(\varphi) = I$. The proof is left as an exercise to the reader.
  \subsection{Using the First Isomorphism Theorem: Example 3}%
  Let $A$ be a ring and $X$ be any non-empty set. Let $R$ be the set of functions from $X$ to $A$.\\

  We have $R$ is a ring.
  \begin{align*}
    (f+g)(x) &= f(x) +_{A} g(x)\\
    (fg)(x) &= f(x)\cdot_{A}g(x).
  \end{align*}
  Fix $x_0 \in X$. We define $E_{x_0}: R\rightarrow A$ by
  \begin{align*}
    E_{x_0}(f) &= f(x_0).
  \end{align*}
  We have
  \begin{align*}
    E_{x_0}(f+g) &= (f+g)(x_0)\\
                 &= f(x_0) + g(x_0)\\
                 &= E_{x_0}(f) + E_{x_0}(g)\\
                 \intertext{and}
    E(x_0)(fg) &= (fg)(x_0)\\
               &= f(x_0)g(x_0)\\
               &= E_{x_0}(f)E_{x_0}(g).
  \end{align*}
  Therefore, $E_{x_0}$ is a homomorphism. Additionally, $E_{x_0}$ is surjective, since we can find $f_a: X\rightarrow A$, $x\mapsto a$, meaning $E_{x_0}(f_a) = f_a(x_0) = a$.\\

  If $f\in \ker(E_{x_0})$, then $E_{x_0}(f) = 0_A$. However, $E_{x_0}(f) = f(x_0)$. Then,
  \begin{align*}
    \ker(\varphi) &= \{f: X\rightarrow A \mid f(x_0) = 0_A\}\\
                  &= \mathcal{M}_{x_0}.
  \end{align*}
  By the first isomorphism theorem, we can see that $R/\mathcal{M}_{x_0} \cong A$.
  \subsection{Other Isomorphism Theorems}%
  Let $R$ be a ring.
  \begin{description}
    \item[Diamond Isomorphism Theorem:] Let $A$ be a subring of $R$ and $I$ an ideal of $R$. Define $A + I = \{a+i\mid a\in A,i\in I\}$. This is an ideal of $R$. We also have that $A\cap I$ is an ideal in $A$, and $(A+I)/I \equiv A/A\cap I$.
  \end{description}
  % https://tikzcd.yichuanshen.de/#N4Igdg9gJgpgziAXAbVABwnAlgFyxMJZARgBoAmAXVJADcBDAGwFcYkQBBAHS4GN60AAgCSIAL6l0mXPkIoADKWLU6TVu1ESp2PASLklKhizaJO4ySAw7ZRMvKNrTnANSaVMKAHN4RUADMAJwgAWyRFEBwIJGItECDQ8JoopANVE3YeXgIvCwDgsMQySOjEAGYaY3UzLJy8+ILU5NKysUoxIA
  \begin{center}
    \begin{tikzcd}
                          & A+I                                    &              \\
    I \arrow[ru, "\simeq"] &                                        & A \arrow[lu] \\
                          & A\cap I \arrow[lu] \arrow[ru, "\simeq"] &             
    \end{tikzcd}
  \end{center}
  \begin{description}
    \item[Third Isomorphism Theorem:] Let $I,J$ be ideals of $R$ with $I\subseteq J$. Then, $J/I$ is an ideal of $R/I$ with $(R/I)/(J/I)\cong R/J$.
    \item[Lattice Isomorphism Theorem:] Let $I\subseteq R$ be an ideal. The correspondence $A\leftrightarrow A/I$ is an inclusion-preserving bijection between the subrings $A$ of $R$ that contain $I$ and the subrings of $R/I$. Moreover, $A$ is an ideal if and only if $A/I$ is an ideal.
  \end{description}
  \subsection{Using the Third Isomorphism Theorem}%
  Let $R = \Z$, $I = 12\Z$, and $J = 4\Z$. By the third isomorphism theorem, $J/I = 4\Z/12\Z$ is an ideal of $R/I = \Z/12\Z$, and
  \begin{align*}
    (R/I)/(J/I) &= (\Z/12\Z)/(4\Z/12\Z)\\
                &\cong \Z/4\Z.
  \end{align*}
  \subsection{Applying the Isomorphism Theorems}%
  Consider the rings $3\Z$ and $12\Z$. We have that $12\Z \subseteq 3\Z$ as an ideal. Therefore, we can form the quotient ring $3\Z/12\Z$. We might ask how it's related to other $\Z/n\Z$, or to $\Z/12\Z$.\\

  Note that $3\Z/12\Z$ starts with elements in $3\Z$ and examines elements in $12\Z$. We might ask whether or not $3\Z/12\Z \cong \Z/4\Z$. However,
  \begin{align*}
    3\Z/12\Z &= \{a + 12\Z\mid a\in 3\Z\}\\
             &= \{3b + 12\Z \mid b\in\Z\}.
  \end{align*}
  We can define 
  \begin{align*}
    \varphi: 3\Z\rightarrow \Z/4\Z\\
    0 + 12\Z \mapsto [0]_4,\\
    3 + 12\Z \mapsto [3]_4,\\
    6 + 12\Z \mapsto [2]_4,\\
    9 + 12\Z \mapsto [1]_4.
  \end{align*}
  which we look at by aiming for $12\Z$ to be the kernel of $\varphi$. Then, by the first isomorphism theorem, $3\Z/12\Z \cong \Z/4\Z$.\\

  If we want to examine $3\Z/12\Z$ in relation to $\Z/12\Z$, we see that $3\Z/12\Z \cong \langle[3]_{12}\rangle \subseteq \Z/12\Z$.
  \section{Further Examination of Ideals}%
  Let $I,J\subseteq R$ be ideals. We define
  \begin{enumerate}[(1)]
    \item the sum, $I+J = \{i+j\mid i\in I,j\in J\}$,
    \item the product, $IJ$, the collection of finite sums of elements of the form $xy$, where $x\in I$ and $y\in J$, and
    \item The $n$th power of $I$, denoted $I^n$, which is the collection of finite sums of elements of the form $x_1,\dots,x_n\in I$.
  \end{enumerate}
  \begin{description}
    \item[Exercises:]\hfill
      \begin{enumerate}[(1)]
        \item $I+J$ is the smallest ideal containing $I$ and $J$.
        \item $IJ\subseteq I\cap J$.
      \end{enumerate}
  \end{description}
  Let $R$ be a ring with $1_R\neq 0_R$. Let $A\subseteq R$.
  \begin{enumerate}[(1)]
    \item Let $\langle A\rangle$ be the smallest ideal that contains $A$. It is called the ideal \textit{generated} by $A$.
    \item We set $RA = \{r_1a_1 + \cdots + r_na_n\mid r_i\in R, a_i\in A\}$ for any $n\in \Z_{\geq 0}$. Additionally, $AR$ is analogous to $RA$. We set $RAR = \{r_1a_1\tilde{r_1} + \cdots + r_na_n\tilde{r_n}\mid r_i,\tilde{r}_i\in R, a_i\in A\}$.
    \item If $A$ is a single element $a$, we write $\langle a \rangle$ to denote the ideal generated by $A$ and refer to this as a principal ideal. If $A$ is finite, then we say $\langle A \rangle$ is a finitely generated ideal.
  \end{enumerate}
  For example, if $R = \Z[x_1,x_2,\dots]$, then $I = \langle x_1,x_2,\dots\rangle$ is not finitely generated.
  \begin{description}
    \item[Note:] If $R$ is commutative, then $\langle a \rangle = Ra$ and if $R$ is not commutative, $\langle a \rangle = RaR$. For $R$ commutative, we say that for $b\in \langle a \rangle$, $b = ra$ for some $r\in R$. We say $a$ divides $b$ --- if $a$ divides $b$, then $\langle b \rangle\subseteq \langle a \rangle$.
  \end{description}
  \subsection{Principal Ideal: Example 1}%
  Every ideal in $\Z$ is a principal ideal.\\

  Let $I\subseteq \Z$ be a nonzero ideal (the zero ideal is generated by $0$). Let $m\in I, m\neq 0$. Since $I$ is an ideal, if $m\in I$, so too is $-m\in I$. Therefore, we know there is a positive integer in $I$.\\

  By the well-ordering principle, let $n\in I$ be the smallest positive integer in $I$. Let $a\in I$, $a\neq 0$. Write $a = nq + r$ for $q,r\in \Z$, and $0\leq r < n$. Then, we have $r = a-nq$. Since $a\in I$ and $n\in I$, $r\in I$. Therefore, $r = 0$, and $n|a$. Thus, $I = n\Z$.
  \subsection{Principal Ideal: Example 2}%
  Let $R = \Z[x]$. Consider $I = \langle 2,x\rangle$. We claim that $I$ is not a principal ideal.\\

  Suppose toward contradiction that $\langle 2,x\rangle = \langle f(x)\rangle$ for some $f(x)\in \Z[x]$. Therefore, $2 = f(x)g(x)$ for some $g(x)\in \Z[x]$. Since degrees add, $\text{deg}(2) = \text{deg}(f) + \text{deg}(g)$, or $0= f(x)g(x)$. Therefore, $f(x),g(x)\in \Z$. Therefore, we must have that $f(x) \in \{\pm 1,\pm 2\}.$\\

  So, we have elements of $\langle 2,x\rangle$ of the form $2s(x) + xt(x)$. So we have constant term divisible by $2$, meaning $f(x) \neq \pm 1$, so $f(x) = \pm 2$.\\

  Then, $x = 2h(x)$ for some $h(x) \in \Z[x]$. However, we have that $h(x)$ has integer coefficients. Therefore, $\langle 2,x\rangle \neq \langle f(x)\rangle$ for any $f(x)\in \Z[x]$.
  \subsection{Proposition: Ideals in Unital Rings}%
  Let $I$ be an ideal of $R$.
  \begin{enumerate}[(1)]
    \item $I = R$ if and only if $I$ contains a unit.
    \item If $R$ is commutative, then $R$ is a field if and only if the only ideals in $R$ are $\langle 0_R\rangle$ and $R$.
  \end{enumerate}
  \begin{description}[font=\normalfont]
    \item[Proof of (1):] Suppose $I = R$. Then, $1_R \in I$, and $1_R$ is a unit.\\

      Suppose $I$ contains a unit, $u$. Then, we have $u^{-1}\in R$. Since $I$ is an ideal, we have $uu^{-1} \in I$, and $uu^{-1} = 1_R$. Letting $r\in R$, using the fact that $I$ is an ideal, $(r)(1_R) = r\in I$. Thus, $I = R$.
    \item[Proof of (2):] Suppose $R$ is a field. Let $I$ be any nonzero ideal. Every nonzero element in $I$ is a unit, meaning $I = R$.\\

      Suppose $\langle 0_R\rangle$ and $R$ are the only ideals in $R$. Let $r \in R$, $r\neq 0_R$. Since $r\neq 0$, $\langle r \rangle = R$. Thus, $1_R \in \langle r \rangle$. Thus, $1_R = sr$ for some $s\in R$, implying every nonzero element of $R$ has an inverse.
  \end{description}
  \subsection{Corollary: Field Homomorphisms}%
  Let $F$ be a field, and $\varphi: F\rightarrow R$ be a homomorphism. Then, $\varphi$ is either the zero map ($\varphi(f) = 0_R$) or $\varphi$ is injective.
  \begin{description}[font=\normalfont]
    \item[Proof:] Since $\ker(\varphi)$ is an ideal in $F$ by the first isomorphism theorem, then $\ker(\varphi) = \langle 0_F\rangle$ or $\ker(\varphi) = R$. If $\ker(\varphi) = \langle 0_F\rangle$, then $\varphi$ is injective, and if $\ker(\varphi) = F$, then $\varphi$ is the zero map.
  \end{description}
  \subsection{Maximal Ideals}%
  \begin{enumerate}[(1)]
    \item An ideal $\mathcal{M}\subseteq R$ is a maximal ideal if $\mathcal{M}\neq R$ and the only ideals containing $\mathcal{M}$ are $\mathcal{M}$ and $R$. The collection of maximal ideals is denoted $\text{m-spec}(R)$ or $\text{maxspec}(R)$.
    \item An ideal $\mathfrak{p}\subseteq R$ with $\mathfrak{p} \neq R$ is a prime ideal if whenever $ab \in \mathfrak{p}$, then $a\in \mathfrak{p}$ or $b\in \mathfrak{p}$. We denote the collection of prime ideals $\text{Spec}(R)$.
  \end{enumerate}
  For example, $\text{Spec}(\Z) = \{0\Z,p\Z\}$ for $p$ prime, and $\text{maxspec}(\Z) = \{p\Z\}$.
  \begin{description}
    \item[Aside:] Let $R$ be commutative. The set $\text{Spec}(R)$ is a topological space. Let $A\subseteq R$ be any subset. Closed sets look like
      \begin{align*}
        V(A) &= \{\mathcal{P}\in \text{Spec}(R) \mid A \subset \mathcal{P}\}\\
             &= V(I)\\
             &= \langle A \rangle
      \end{align*}
      For example, if $R = \R[x,y]$, if $f(x,y) = y-x^2$, then $V(f) = \{(a,b)\in \R^2\mid f(a,b) = 0\}$. The topology on $\text{Spec}(R)$ is called the Zariski topology.
  \end{description}
  Let $\varphi: R\rightarrow S$ be a ring homomorphism. If $\mathcal{P}\in \text{Spec}(S)$, then $\varphi^{-1}(\mathcal{P})$ is a prime ideal in $R$. We get a map $\varphi^{\ast}(\text{Spec}(S)) \rightarrow \text{Spec}(R)$ given by $\mathcal{P}\rightarrow \varphi^{-1}(\mathcal{P})$.\\

  We get a contravariant functor that takes $R\mapsto \text{Spec}(R)$, mapping from the category of rings to the category of topological spaces. 
  \subsection{Proposition: Existence of Maximal Ideals}%
  Let $R$ be a ring. Every proper ideal is contained in a maximal ideal.\\

  Let $I$ be a proper ideal. Let $\mathcal{S}$ be the collection of all proper ideals that contain $I$. We know that $\mathcal{S}$ is non-empty as $I\in \mathcal{S}$. Then, $\mathcal{S}$ has a partial ordering under inclusion.\\

  Let $\mathcal{C}$ be a chain of ideals (that is, totally ordered subset) in $\mathcal{S}$, and 
  \begin{align*}
    J &= \bigcup_{A\in\mathcal{C}}A.
  \end{align*}
  Since $\mathcal{C} \neq \emptyset$, there is at least one $A$ in the union with $0_R \in A$. So, $J \neq \emptyset$. Let $a,b\in J$. There exists $A$ with $a\in A$ and $b$ with $b\in B$. Since $\mathcal{C}$ is a chain, either $A\subseteq B$ or $B\subseteq A$. So, $a$ and $b$ are both in either $A$ or $B$. Thus, $a-b$ and $ab$ are in either $A$ or $B$. Thus, $a-b$ and $ab$ are elements in $J$, meaning $J$ is an ideal.\\

  If $J = R$, then $1_R \in J$, meaning $1_R$ is an element of some $A\in \mathcal{C}$. Since $A\in \mathcal{S}$ is a proper ideal, this would be a contradiction.\\

  Therefore, $J$ is an upper bound for $\mathcal{C}$. Since every chain in $\mathcal{S}$ has an upper bound in $\mathcal{S}$, then, by Zorn's Lemma, there is a maximal element in $\mathcal{S}$.
  
  \subsection{Proposition: Maximal Ideals, Quotient Rings, and Fields}%
  An ideal $\mathcal{M}\subseteq R$ of a commutative ring with identity is maximal if and only if $R/\mathcal{M}$ is a field.\\

  Suppose $\mathcal{M}$ is maximal. Let $x + \mathcal{M} \neq 0 + \mathcal{M}$. We want to show that $x + \mathcal{M}$ has an inverse.\\

  Consider $\langle x,\mathcal{M}\rangle$, the ideal generated by $x$ and $\mathcal{M}$. We have $\mathcal{M} \subset \langle x,\mathcal{M}\rangle$, as $x\notin \mathcal{M}$. Therefore, $\langle x,\mathcal{M}\rangle = R$ by the definition of a maximal ideal. Therefore, $1_R \in \langle x,\mathcal{M}\rangle$, meaning $1_R = xu + mv$ for some $u,v\in R$, $m\in \mathcal{M}$. Note
  \begin{align*}
    (x + \mathcal{M})(u + \mathcal{M}) &= xu + \mathcal{M}\\
                                       &= (1_R - mv) + \mathcal{M}\\
                                       &= 1_R + \mathcal{M},
  \end{align*}
  meaning $x + \mathcal{M}$ has an inverse, meaning $R/\mathcal{M}$ is a field.\\

  Suppose $R/\mathcal{M}$ is a field. Assume we have $\mathcal{M} \subset I \subset R$ for some ideal $I$. From the third isomorphism theorem, we have $I/\mathcal{M}$ is an ideal of $R/\mathcal{M}$. Specifically, by our construction, $I/\mathcal{M}$ is a proper nonzero ideal of $R/\mathcal{M}$, but since $R/\mathcal{M}$ is a field, no such proper nonzero ideal exists, meaning no such $I$ exists.
  \subsection{Examples: Maximal Ideals}%
  \begin{enumerate}[(1)]
    \item Let $R = \Z$. Given $m\in \Z$, we know $m\Z$ is a maximal ideal if and only if $m$ is prime. If $p|m$ and $p\neq m$, then $m\Z \subseteq p\Z$. Additionally, if $p$ is prime, then $\Z/p\Z$ is a field. Additionally, $\Z/m\Z$ is not an integral domain if $m$ is composite.
    \item Let $R = F[x]$ for $F$ a field. Let $\alpha \in F$ and consider $\mathcal{M}_{\alpha} = \langle x-\alpha\rangle$. We claim that $F[x]/\mathcal{M}_{\alpha} \cong \mathcal{F}$, meaning $\mathcal{M}$ is a maximal ideal.\\

      Let $\varphi: F[x] \rightarrow F$, $x \mapsto \alpha, f(x) \mapsto f(\alpha)$. Let $f(x),g(x)\in F[x]$. Then,
      \begin{align*}
              \varphi(f + g) &= (f+g)(\alpha)\\
                             &= f(\alpha) + g(\alpha)\\
                             &= \varphi(f) + \varphi(g)\\
                             \intertext{and}
              \varphi(fg) &= (fg)(\alpha)\\
                          &= f(\alpha)g(\alpha)\\
                          &= \varphi(f)\varphi(g).
      \end{align*}
      Let $\beta \in F$. Then,
      \begin{align*}
        \varphi(\beta + (x-\alpha)) &= \beta + (\alpha - \alpha)\\
                                    &= \beta.
      \end{align*}
      Thus, $\varphi$ is surjective. Finally, we have $f(x)\in \ker(\varphi)$ if and only if $f(\alpha) = 0$. However, $f(\alpha) = 0$ if and only if $(x-\alpha)|f(x)$. Therefore, $\ker(\varphi) = \langle x-\alpha \rangle$.
    \item Let $R = \Z[x]$. Let $\mathcal{M} = \langle 2,x\rangle$. We saw that $\Z[x]/\langle 2,x\rangle \cong \mathbb{F}_2 = \Z/2\Z$. Therefore, we know that $\mathcal{M}$ is a maximal ideal by the above categorization.
    \item Let $R = \mathbb{F}_2[x]$. Consider the ideal $\mathcal{M} = \langle x^2 + x + 1\rangle$.
      \begin{align*}
        R/\mathcal{M} &= \left\{f(x) + \langle x^2 + x + 1\rangle\mid f(x)\in \mathbb{F}_2[x]\right\}\\
        f(x) &= \left\{(x^2 + x + 1)q(x) + r(x) \mid q(x),r(x)\in \mathbb{F}_2[x],~r(x)=0 \text{ or } \text{deg}r(x) < 2\right\}.\\
        \intertext{So,}
        f(x) + \mathcal{M} &= r(x) + \mathcal{M},
        \intertext{meaning}
        R\mathcal{M} &= \{0 + \mathcal{M}, 1 + \mathcal{M}, x + \mathcal{M}, 1 + x + \mathcal{M}\}.
      \end{align*}
      This is a field.
      \begin{center}
        \renewcommand{\arraystretch}{1.5}
        \begin{tabular}{c|cccc}
          $+$ & $0 + \mathcal{M}$ & $1 + \mathcal{M}$ & $x + \mathcal{M}$ & $x+1 + \mathcal{M}$\\
          \hline
          $0 + \mathcal{M}$ & $0$ & $1$ & $x$ & $x+1$\\
          $1 + \mathcal{M}$ & $1$ & $0$ & $1+x$ & $x$\\
          $x + \mathcal{M}$ & $x$ & $1+x$ & $0$ & $1$\\
          $x+1 + \mathcal{M}$ & $1+x$ & $x$ & $1$ & $0$
        \end{tabular}\\
        \begin{tabular}{c|cccc}
          $\times$ & $0 + \mathcal{M}$ & $1 + \mathcal{M}$ & $x + \mathcal{M}$ & $x+1 + \mathcal{M}$\\
          \hline
          $0 + \mathcal{M}$ & $0$ & $0$ & $0$ & $0$\\
          $1 + \mathcal{M}$ & $0$ & $1$ & $x$ & $x+1$\\
          $x + \mathcal{M}$ & $0$ & $x$ & $1+x$ & $1$\\
          $x+1 + \mathcal{M}$ & $0$ & $1+x$ & $x$ & $1$
        \end{tabular}\\
      \end{center}
      Specifically, this is a field of order $4$. Note that $\mathbb{F}_2 \hookrightarrow R/\mathcal{M}$. We say $R/\mathcal{M} \cong \mathbb{F}_4$. 
      \begin{description}
        \item[Note:] For every $p$ prime and every $n\in \Z$ positive, there is exactly one field of order $p^n$ up to isomorphism.
      \end{description}
    \item Let $R = \Z[i]$. Set $\mathcal{M} = \langle 3\rangle$. This is a maximal ideal, and $|\Z[i]/\langle 3 \rangle| = 9$.
  \end{enumerate}
  \subsection{Proposition: Prime Ideals, Quotient Rings, and Integral Domains}%
  Let $R$ be a commutative ring with identity. An ideal $\mathfrak{p}\subseteq R$ is a prime ideal if and only if $R/\mathfrak{p}$ is an integral domain.\\

  Let $\mathfrak{p}\subseteq R$ be a prime ideal. Let $x,y\in R$ with $(x+\mathfrak{p})(y + \mathfrak{p}) = 0 + \mathfrak{p}$. We have
  \begin{align*}
    xy + \mathfrak{p} &= 0 + \mathfrak{p}\\
    \intertext{meaning}
    xy &\in \mathfrak{p},\\
    \intertext{so, since $\mathfrak{p}$ is prime,}
    x&\in \mathfrak{p}\\
    \intertext{or}
    y&\in \mathfrak{p}\\
    \intertext{so $x + \mathfrak{p} = 0 + \mathfrak{p}$ or $y + \mathfrak{p} = 0\mathfrak{p}$.}
  \end{align*}
  In the reverse direction, assume $R/\mathfrak{p}$ is an integral domain. Let $xy\in \mathfrak{p}$. Then,
  \begin{align*}
    (x+\mathfrak{p})  (y + \mathfrak{p}) &= xy + \mathfrak{p}\\
                                         &= 0 + \mathfrak{p},
  \end{align*}
  implying that $x + \mathfrak{p}$ or $y + \mathfrak{p}$ is equal to $0 + \mathfrak{p}$, or $x \in \mathfrak{p}$ or $y\in \mathfrak{p}$.
  \subsection{Examples: Prime Ideals}%
  \begin{enumerate}[(1)]
    \item If $R = \Z[x]$, then $\mathfrak{p} = \langle x \rangle$ is a prime ideal that is not a maximal ideal, as $\Z[x]/\langle x \rangle\cong \Z$.
  \end{enumerate}
  \subsection{Corollary: Maximal Ideals and Prime Ideals}%
  Let $R$ be a commutative ring with identity. Then, $\text{maxspec}(R)\subseteq \text{Spec}(R)$.
  \subsection{Direct Products}%
  Let $R$ and $S$ be rings. The set
  \begin{align*}
    R\times S &= \{(r,s)\mid r\in R, s\in S\}
  \end{align*}
   is a ring under component-wise multiplication and addition.
   \begin{description}
     \item[Exercise:] Let $R_1,\dots,R_n$ be rings. Let 
       \begin{align*}
         \varphi: R\rightarrow R_1\times \cdots \times R_n
       \end{align*}
      be a map. Define
      \begin{align*}
        \pi_j: R_1\times\cdots\times R_n \rightarrow R_j\\
        (r_1,\dots,r_n)\mapsto r_j.
      \end{align*}
      Show $\varphi$ is a homomorphism if and only if $\pi_j\circ \varphi$ is a homomorphism for each $j$.
   \end{description}
   \subsection{Comaximal Ideals}%
   Recall that $a\Z + b\Z = \gcd(a,b)\Z$. If $\gcd(a,b) = 1$, then $a\Z + b\Z = \Z$. Conversely, if $a\Z + b\Z = \Z$, then $am + bn = 1$ for some $m,n\in\Z$. Thus, $\gcd(a,b) = 1$.\\

   Let $I,J$ be ideals in a commutative ring $R$. We say $I$ and $J$ are comaximal if $I+J = R$.
   \subsection{Chinese Remainder Theorem}%
   Let $I_1,\dots,I_n$ be ideals in a commutative ring $R$. The map
   \begin{align*}
     \varphi: R\rightarrow R/I_1 \times R/I_2\times \cdots \times R/I_n\\
     r \mapsto (r + I_1,r+I_2,\dots,r+I_n)
   \end{align*}
   is a ring homomorphism with kernel $I_1\cap \cdots \cap I_n$. If $I_i,I_j$ are comaximal for all $1\leq i,j\leq n$ with $i\neq j$, then $\varphi$ is surjective, and $I_1\cap \cdots \cap I_n = (I_1)(I_2)\cdots(I_n)$, so
   \begin{align*}
     R/\left((I_1)(I_2)\cdots(I_n)\right) \cong R/(I_1\cap \cdots \cap I_n) \cong R/I_1\times\cdots\times R/I_n.
   \end{align*}
   \subsubsection{Corollary to the Chinese Remainder Theorem (1)}%
   Let $n = p_1^{e_1}\cdots p_r^{e_r}\in \Z$. Then,
   \begin{align*}
     \Z/n\Z \cong \Z/p_1^{e_1}\Z \times \cdots \times \Z/p_{r}^{e_r}\Z.
   \end{align*}
   Moreover,
   \begin{align*}
     \left(\Z/n\Z\right)^{\times} \cong \left(\Z/p_1^{e_1}\Z\right)^{\times} \times \cdots \times \left(\Z/p_{r}^{e_r}\Z\right)^{\times}.
   \end{align*}
  \subsubsection{Corollary to the Chinese Remainder Theorem (2)}%
  Let $n_1,\dots,n_k$ be positive integers that are pairwise relatively prime. Then, for any $a_1,\dots,a_k\in\Z$, there is a $x\in\Z$ satisfying
  \begin{align*}
   x &\equiv a_1\mod n_1\\
     &\vdots\\
   x &\equiv a_k\mod n_k
  \end{align*}
  This solution is unique modulo $n_1,\dots,n_k$. If we set
  \begin{align*}
   m_i = n_1\cdots \hat{n_i}\cdots n_k,
  \end{align*}
  and $y_i$ as the inverse of $m_i$ mod $n_i$. The solution $x$ is given by
  \begin{align*}
   x &= a_1y_1m_1 + \cdots + a_ky_km_k.
  \end{align*}
  We will prove the Chinese Remainder Theorem by induction, with the base case of $n=2$:
  \begin{align*}
   \varphi: R\rightarrow R/I_1 \times R/I_2\\
   r \mapsto (r+I_1,r+I_2).
  \end{align*}
  We can verify that this is a homomorphism, with $\ker(\varphi) = I_1\cap I_2$. Assume $I_1$ and $I_2$ are comaximal: $I_1 + I_2 = R$. In particular, there exist $x\in I_1$ and $y\in I_2$ such that $x + y = 1_R$. Note that
  \begin{align*}
   \varphi(x) &= (x+I_1,x+I_2) \\
              &= (0+I_1,1_R - y + I_2)\\
              &= (0+I_1,1_R + I_2)\\
              \intertext{and}
   \varphi(y) &= (1_R + I_1,0+I_2).
  \end{align*}
  Let $(r_1 + I_1,r_2 + I_2)\in R/I_1 \times R/I_2$. Set $z = r_2x + r_1y$. Then,
  \begin{align*}
   \varphi(z) &= (r_2x + r_1y + I_1,r_2x + r_1y + I_2)\\
              &= (r_1 + I_1,r_2 + I_2).
  \end{align*}
  So, $\varphi$ is surjective, and we get $R/I_1\cap I_2 \cong R/I_1\times R/I_2$.\\

  We also have that $(I_1)(I_2)\subseteq I_1\cap I_2$. Let $z\in I_1\cap I_2$. We have
  \begin{align*}
   z &= z(1_R)\\
   &= z(x+y)\\
   &= zx + zy\\
   &\in (I_1)(I_2).
  \end{align*}
  Therefore, $R/(I_1)(I_2)\cong R/I_1\cap I_2$.\\

  Suppose the result holds for all values up to $2\leq n \leq k-1$. Write $J_1 = I_1$ and $J_2 = (I_2)(I_3)\cdots(I_k)$. We only need to show that $J_1$ and $J_2$ are comaximal, then apply $n=2$ to $J_1,J_2$ and $n=k-1$ to split up $J_2$.\\

  For each $i\in \{2,\dots,k\}$, there are elements $x_i\in I_1$ and $y_i\in I_{i}$ such that $x_i + y_i = 1_R$. We have $x_i + y_i \equiv y_i (\text{mod } I_1)$, so
  \begin{align*}
   1_R &= (x_2 + y_2)(x_3 + y_3)\times(x_k + y_k)
  \end{align*}
  is an element of $J_1 + J_2$.
  \section{Localization}%
  Where does $\Q$ come from?\\

  Consider the sets $\Z$ and $\Sigma = \Z\setminus \{0\}$. Set
  \begin{align*}
   \Sigma^{-1}\Z &= \left\{(a,b)\mid a\in\Z,b\in\Sigma\right\}.
  \end{align*}
  Define $\sim$ on $\Sigma^{-1}\Z$ by
  \begin{align*}
   (a,b)\sim (c,d) \text{ if } ad = bc.
  \end{align*}
  This is an equivalence relation:
  \begin{description}
   \item[Reflexivity:] 
     \begin{align*}
       (a,b)\sim (a,b)\\
       ab = ab.
     \end{align*}
   \item[Symmetry:] 
     \begin{align*}
       (a,b)\sim (c,d)\\
       ad = bc\\
       bc = ad\\
       (c,d) = (a,b)
     \end{align*}
   \item[Transitivity:] Suppose $(a,b)\sim (c,d)$ and $(c,d)\sim (e,f)$, meaning $ad = bc$ and $cf = de$. We need to show $af = be$.
     \begin{align*}
       ad - bc &= 0\\
       cf - de &= 0\\
       adf - bcf &= 0\\
       bcf - bde &= 0\\
       (adf - bcf) + (bcf - bde) &= 0\\
       (af-be)(d) &= 0\\
       \intertext{and since $d\neq 0$ and we are in $\Z$,}
       af = be,
     \end{align*}
     meaning $(a,b)\sim (e,f)$.
  \end{description}
  Let $\frac{a}{b}$ denote the equivalence class containing $(a,b)$. We define
  \begin{align*}
   \frac{a}{b} + \frac{c}{d} &= \frac{ad + bc}{bd}\\
   \frac{a}{b}\cdot \frac{c}{d} &= \frac{ac}{bd}.
  \end{align*}
  \begin{description}
   \item[Exercise:] Show that addition and multiplication are well-defined, and make the collection of equivalence classes into a field.
  \end{description}
  The field of equivalence classes $\Sigma^{-1}\Z$ under the defined addition and multiplication forms the field $\Q$.\\

  Let $R$ be a ring. We say $\Sigma \subseteq R$ is multiplicatively closed if, given $a,b\in \Sigma$, $ab \in \Sigma$.
  \begin{enumerate}[(1)]
    \item $\Sigma = \Z\setminus\{0\}$ is multiplicatively closed.
    \item Let $r\in R$. Then, $\Sigma = \{r^n\mid n\in\Z\}$.
    \item Let $\mathfrak{p}\in R$. Then, $R\setminus \mathfrak{p}$ is multiplicatively closed (verify this).
  \end{enumerate}
  \subsection{Universal Property}%
  Let $R$ be a commutative ring with identity and $\Sigma\subseteq R$ a multiplicatively closed subset with $1_R\in \Sigma$. There is a unique commutative ring $\Sigma^{-1}R$ and ring homomorphism
  \begin{align*}
   \pi: R\rightarrow \Sigma^{-1}R
  \end{align*}
  satisfying for any homomorphism $\psi: R\rightarrow S$ that sends $1_R$ to $1_S$ and $\psi(\Sigma) \subseteq S^{\times}$, there is a unique homomorphism
  \begin{align*}
   \Psi: \Sigma^{-1}R \rightarrow S
  \end{align*}
  such that $\Psi \circ \pi = \psi$.
  \begin{center}
    \begin{tikzcd}
    R \arrow[rd, "\psi"'] \arrow[r, "\pi"] & \Sigma^{-1} R \arrow[d, "\Psi"] \\
                                       & S                              
    \end{tikzcd}
  \end{center}
  For example, if $R = \Z$ and $\Sigma = \Z\setminus\{0\}$, then $\Sigma^{-1}\Z = \Q$, then for $\pi: \Z\hookrightarrow \Q$, and a homomorphism from $\Z$ into a set $S$, there must exist a map from $\Q$ to $S$.\\

  Consider $\Z$ with $\Sigma = \Z\setminus p\Z$. Then, $\Sigma^{-1}\Z = \{(a,b)\mid a\in\Z, p\not|b\} = \Z_{\langle p \rangle}$. We saw on an earlier homework assignment that $\Z_{\langle p \rangle}/p\Z_{\langle p \rangle} \cong \mathbb{F}_p$, meaning it is a maximal ideal (as if $a\not| p$, then $a/b$ is a unit in $\Z_{\langle p \rangle}$). The only other ideals are $p^m \Z_{\langle p \rangle}$, so we have a chain
  \begin{align*}
    p\Z_{\langle p \rangle} \supseteq p^2\Z_{\langle p \rangle} \supseteq \cdots.
  \end{align*}
\end{document}
