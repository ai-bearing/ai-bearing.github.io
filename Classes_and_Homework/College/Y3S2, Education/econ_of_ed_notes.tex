\documentclass[10pt]{extarticle}
\title{}
\author{}
\date{}
\usepackage[shortlabels]{enumitem}


%paper setup
\usepackage{geometry}
\geometry{letterpaper, portrait, margin=1in}
\usepackage{fancyhdr}
% sans serif font:
\usepackage{cmbright}
%symbols
\usepackage{amsmath}
\usepackage{bigints}
\usepackage{amssymb}
\usepackage{amsthm}
\usepackage{mathtools}
\usepackage{bbm}
\usepackage[colorlinks=true,urlcolor=blue]{hyperref}
\usepackage{gensymb}
\usepackage{multirow,array}
\usepackage{multicol}

\newtheorem*{remark}{Remark}
\usepackage[T1]{fontenc}
\usepackage[utf8]{inputenc}

%chemistry stuff
%\usepackage[version=4]{mhchem}
%\usepackage{chemfig}

%plotting
\usepackage{pgfplots}
\usepackage{tikz}
\tikzset{middleweight/.style={pos = 0.5}}
%\tikzset{weight/.style={pos = 0.5, fill = white}}
%\tikzset{lateweight/.style={pos = 0.75, fill = white}}
%\tikzset{earlyweight/.style={pos = 0.25, fill=white}}

%\usepackage{natbib}

%graphics stuff
\usepackage{graphicx}
\graphicspath{ {./images/} }
\usepackage[style=numeric, backend=biber]{biblatex} % Use the numeric style for Vancouver
\addbibresource{the_bibliography.bib}
%code stuff
%when using minted, make sure to add the -shell-escape flag
%you can use lstlisting if you don't want to use minted
%\usepackage{minted}
%\usemintedstyle{pastie}
%\newminted[javacode]{java}{frame=lines,framesep=2mm,linenos=true,fontsize=\footnotesize,tabsize=3,autogobble,}
%\newminted[cppcode]{cpp}{frame=lines,framesep=2mm,linenos=true,fontsize=\footnotesize,tabsize=3,autogobble,}

%\usepackage{listings}
%\usepackage{color}
%\definecolor{dkgreen}{rgb}{0,0.6,0}
%\definecolor{gray}{rgb}{0.5,0.5,0.5}
%\definecolor{mauve}{rgb}{0.58,0,0.82}
%
%\lstset{frame=tb,
%	language=Java,
%	aboveskip=3mm,
%	belowskip=3mm,
%	showstringspaces=false,
%	columns=flexible,
%	basicstyle={\small\ttfamily},
%	numbers=none,
%	numberstyle=\tiny\color{gray},
%	keywordstyle=\color{blue},
%	commentstyle=\color{dkgreen},
%	stringstyle=\color{mauve},
%	breaklines=true,
%	breakatwhitespace=true,
%	tabsize=3
%}
% text + color boxes
\renewcommand{\mathbf}[1]{\mathbbm{#1}}
\usepackage[most]{tcolorbox}
\tcbuselibrary{breakable}
\tcbuselibrary{skins}
\newtcolorbox{problem}[1]{colback=white,enhanced,title={\small #1},
          attach boxed title to top center=
{yshift=-\tcboxedtitleheight/2},
boxed title style={size=small,colback=black!60!white}, sharp corners, breakable}
%including PDFs
%\usepackage{pdfpages}
\setlength{\parindent}{0pt}
\usepackage{cancel}
\pagestyle{fancy}
\fancyhf{}
\rhead{Avinash Iyer}
\lhead{Economics of Education: Class Notes}
\newcommand{\card}{\text{card}}
\newcommand{\ran}{\text{ran}}
\newcommand{\N}{\mathbbm{N}}
\newcommand{\Q}{\mathbbm{Q}}
\newcommand{\Z}{\mathbbm{Z}}
\newcommand{\R}{\mathbbm{R}}
\setcounter{secnumdepth}{0}
\begin{document}
\section{Introduction}%
Education is one of the largest sectors in the economy, and thus can be studied from a large amount of angles.
\begin{itemize}
  \item Early Childhood Education (beyond just ``being watched'')
  \item Elementary/Secondary School
  \item Postsecondary Education
\end{itemize}
Education can be studied from a lot of angles:
\begin{description}
  \item[Micro:] Applying theories of labor economics and consumer theory to education.
  \item[Econometrics:] Use data to analyze educational policies.
  \item[Macro:] Investigate global demand for education-as-a-commodity.
\end{description}
\subsection{Education System Basics}%
\begin{description}
  \item[Returns to Education:] There is a large return to education; those with a high school education tend to make far less than those with a bachelor's degree and up. Perceived value of being more education in private or public market.
  \item[Labor Market Outcomes:] The more educated you are, the more likely to have a job; unemployment rates for high school graduates are higher than unemployment rates for college graduates.
  \item[Public Spending:] Approximately 5--6\% of GDP is spent on education in most OECD countries.
  \item[Funding Structure:] Public schools are primarily funded through state and local governments --- property taxes the largest source of funding for education, but federal government has started to fund more schools in recent years.
  \item[Growth of Education over Time:] Claudia Goldin's 1993 paper ``The Human-Capital Century and American Leadership'' shows that the 20th century was really the century of greater and greater access and attainment in education.
\end{description}
\section{Why Do We Get Educated?}%
\subsection{Human Capital}%
  \begin{description}
    \item[What is human capital?]\hfill
      \begin{itemize}
        \item Labor. 
        \item Complexity or efficiency of work.
      \end{itemize}
    \item[How does human capital differ from capital?]\hfill
      \begin{itemize}
        \item Less static.
        \item Differential depreciation --- potential for appreciation (people can skill up).
        \item Higher variance.
        \item Unionization/collective bargaining.
        \item Idea generation.
        \item Potentially greater mobility.
        \item Returns to human capital come in the form of wages --- human capital is owned by the human that holds it.
        \item Cannot be collateralized.
        \item Divisibility (or lack thereof).
      \end{itemize}
  \end{description}
\subsection{Education: how much?}%
  \begin{description}
    \item[Discrete Model:] To college or not?
      \begin{itemize}
        \item Direct costs: tuition, room and board.
        \item Indirect costs: foregone earnings.
        \item Returns: expected future earnings (requires college degree or not). 
      \end{itemize}
      We will assume that ``college'' is period 1, and college grads earn more post-college, and there is a discount rate $r$. 
      \begin{center}
        \includegraphics[width=10cm]{images/discrete_education_model.png}
      \end{center}
      The discount rate of \$100 in $t>0$ periods is worth $\frac{100}{(1+r)^t}$ in period $0$ (aka today).\\

      We generally think about $r$ in terms of the interest rate --- money today is worth more than money in the future due to the ability to invest.\\

      The \textit{present value} of a stream of money is found as follows:
      \begin{align*}
        \text{PV} &= \frac{100}{(1+r)} + \frac{100}{(1+r)^2} + \cdots + \frac{100}{(1+r)^n}\\
                  &= \sum_{t=1}^{n} \frac{100}{(1+r)^t} \tag*{$(1)$}\\
        (1+r)\text{PV} &= 100 + \frac{100}{(1+r)} + \cdots + \frac{100}{(1+r)^{n-1}}\\
                       &= 100 + \sum_{t=1}^{n-1} \frac{100}{(1+r)^t} \tag*{$(2)$}\\
        (1+r)\text{PV} - \text{PV} &= 100 + \sum_{t=1}^{n-1}\frac{100}{(1+r)^t} - \sum_{t=1}^{n-1}\frac{100}{(1+r)^t} - \frac{100}{(1+r)^n}\tag*{$(2)-(1)$}\\
        r\text{PV} &= 100 - \frac{100}{(1+r)^n}\\
        \text{PV} &= \frac{100}{r}\left(1-\frac{100}{(1+r)^n}\right)
      \end{align*}
      As $n$ becomes larger, then the PV of the asset is larger. For example, if $n = 40$, $Y = 60,000$, and $r = 0.05$, then the PV of this revenue stream is approximately \$1 million.\\

      Bringing this to the model, where $F$ denotes direct tuition cost, $Y_0$ denotes earnings with no schooling, and $Y_S$ denotes earnings with schooling (where school occurs in period $1$).
      \begin{align*}
        \text{PV}_{0} &= \frac{Y_0}{(1+r)} + \frac{Y_0}{(1+r)^2} + \cdots + \frac{Y_0}{(1+r)^n}\\
        \text{PV}_{S} &= -F + \frac{Y_S}{(1+r)^2} + \cdots + \frac{Y_S}{(1+r)^n}\\
        \text{NPV}_{S} &= \text{PV}_{S} - \text{PV}_{0}\\
                       &= \underbrace{-F - \frac{Y_0}{(1+r)}}_{\small\text{Cost}} + \underbrace{\sum_{t=2}^{n}\frac{Y_S - Y_0}{(1+r)^t}}_{\small\text{Benefit}}\\
                       &= -F-\frac{Y_0}{1+r} + \frac{Y_S-Y_0}{r}\left(1-\frac{1}{(1+r)}\right)\frac{1}{1+r}
      \end{align*}
      To find if education is worth it, we calculate if $\text{NPV}_{S} > 0$.
    \item[Continuous Model (or Mincer Model):] To take an extra year of education or not?
      \begin{itemize}
        \item $S$ is a discrete, integer choice (denoting a year of education).
        \item $Y_S$ is salary after schooling for $S$ years.
        \item There are zero direct costs of school.
        \item Years in labor force, $K$, are equivalent regardless of $S$.
      \end{itemize}
      We choose $S$ where marginal benefit is equal to marginal cost.
      \begin{align*}
        \text{PV}_{S} &= \text{PV}_{S+1}\\
        \sum_{t=1}^{K}\frac{Y_S}{(1+r)^t} &= \sum_{t=2}^{K+1}\frac{Y_{S+1}}{(1+r)^t}\\
        \frac{Y_S}{r}\left(1-\frac{1}{(1+r)^K}\right) &= \frac{Y_{S+1}}{r}\left(1-\frac{1}{(1+r)^K}\right)\frac{1}{1+r}\\
        Y_S &= Y_{S+1}\frac{1}{1+r}\\
        1+r &= \frac{Y_{S+1}}{Y_S}
      \end{align*}
      We choose school until the marginal rate of return is equal to the discount rate.
  \end{description}
  \textbf{Housekeeping, January 30:} Schedule for discussion and presentation is located \href{https://docs.google.com/document/d/1uy96HLNuZVGlrT8oiC49i3M6p2E57fopxGwizHv4_Hs/edit}{at this link}, and the guidelines for classroom activities are located \href{https://docs.google.com/document/d/1tnPmI21LJLdKblCUbuFMuDGobuAz2IOaineH_I4BTFg/edit}{at this link}.
  \subsection{Educational Landscape}%
  The human capital system consists of a number of components.
  \begin{itemize}
    \item Trade, technical, and vocational education (generally falls under post-secondary education)
    \item Early childhood education --- Ages 6 weeks--5, includes day care and pre-K
    \item Primary education --- Ages 5--12, Grades K--5/6
    \item Secondary education --- Ages 12--18, Grades 6--12
    \item Post-secondary education --- two year/community college, four year college
    \item Graduate education --- profession-oriented (MBA, JD), research-oriented (master's, PhD), certification (CPA, CFA, actuarial credentialing)
    \item Adult education (GED, college)
  \end{itemize}
  In primary and secondary education, primary choice facing consumers of education is between public and private education.
  \subsection{Human Capital Model: Choice of Schooling Quantity}%
  The human capital model indicates that consumers of education choose their amount of schooling, $S$, based on the following factors:
  \begin{itemize}
    \item Discrete: $Y_S$ (income from having been schooled) vs $Y_0$ (income without schooling)
    \item Continuous: $\frac{Y_{S+1}}{Y_S}$ (marginal rate of return from schooling)
    \item $F$ (the cost of schooling)
    \item $r$ (discount rate)
  \end{itemize}
  However, this leads us to ask an important question --- why might $S$ differ?
  \begin{itemize}
    \item Differing (marginal) rates of return --- job-specific factors, overqualification, ability, quality of education
    \item Different cost of education --- borrowing, aid, credit constraints
      \begin{description}
        \small
        \item[Comment:] Credit constraints increase exponentially as quantity of schooling increases.
      \end{description}
  \end{itemize}
  A model of credit constraints' effects on choices of education can be seen as follows:
  \begin{center}
    \includegraphics[width=10cm]{images/credit_constraints.png}
  \end{center}
  Broadly speaking, if $S$ differs because of marginal rate of return, then subsidies may be inefficient --- subsidies will cause inefficient excess schooling.\\

  However, if $S$ differs because of cost, then subsidies improve overall output and efficiency.
  \subsection{Signaling}%
  The basic idea behind the human capital model is that by getting more educated, you become smarter and have a higher rate of return --- regardless of whether or not you get a degree. Now, we will discuss a model where schooling does not indicate one's level of smartness.
  \begin{description}
    \item[Assumptions:]\hfill
      \begin{enumerate}[(1)]
        \item No human capital accrued at school.
        \item Two types of workers: low ability ($L$) of proportion $p$ with productivity $1$ and high ability ($H$) of $1-p$ with productivity $2$.
        \item Cost of education is lower for type $H$. For type $L$, the cost of education is $c$, and for type $H$ the cost of education is $c/2$.
        \item Generic employer who, if they distinguish $H$ and $L$, pay marginal benefit --- wage to $L$ is $1$, wage to $H$ is $2$.
        \item If the employer cannot distinguish between $H$ and $L$, then they pay the expected marginal benefit, $(1-p)(2) + (p)(1) = 2-p$.
      \end{enumerate}
    \item[Game Play:] \hfill
      \begin{itemize}
        \item Employer forms belief $w(S)$ about the worker productivity
        \item Employer sets $w(S)$
        \item Workers observe $w(S)$ and decide on $S$
        \item Workers are hired and firms observe their productivity
      \end{itemize}
    \item[Types of Equilibria:]\hfill
      \begin{itemize}
        \item Separating equilibrium: a situation where $H$ chooses education and $L$ does not choose education. In this case, education serves as a pure signal of high productivity --- there is no separating equilibrium where $H$ chooses no education and $L$ chooses education.
        \item Pooling equilibrium: all workers choose education, and the employer cannot differentiate, meaning the employer pays $2-p$ to all workers.
      \end{itemize}
    \item[Solving an Equilibrium:] We assume that there is a separating equilibrium --- $H$ chooses $S=1$ and $L$ chooses $S=0$. Then, the employer forms beliefs to set a wage structure as follows:
      \begin{align*}
        w(S) &= \begin{cases}
          2 & S=1\\
          1 & S=0
        \end{cases}.
      \end{align*}
      In order to be an equilibrium, both $H$ and $L$ types need to have an incentive not to deviate.
      \begin{itemize}
        \item $H$ Type Equilibrium Condition: Return to education is higher than return to non-education.
          \begin{align*}
            2 - \frac{c}{2} &> 1\\
            c &< 2
          \end{align*}
        \item $L$ Type Equilibrium Condition:
          \begin{align*}
            1 &> 2-c\\
            c &> 1
          \end{align*}
      \end{itemize}
      Therefore, if $c\in (1,2)$, we can find a separating equilibrium.
  \end{description}
\end{document}
