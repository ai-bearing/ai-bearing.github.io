\documentclass[8pt]{extarticle}
\title{}
\author{}
\date{}
\usepackage[shortlabels]{enumitem}

%paper setup
\usepackage{geometry}
\geometry{letterpaper, portrait, margin=1in}
\usepackage{fancyhdr}
% sans serif font:
\usepackage{cmbright}
%symbols
\usepackage{amsmath}
\usepackage{bigints}
\usepackage{amssymb}
\usepackage{amsthm}
\usepackage{mathtools}
\usepackage{bbold}
\usepackage[hidelinks]{hyperref}
\usepackage{gensymb}
\usepackage{multirow,array}
\usepackage{multicol}

\newtheorem*{remark}{Remark}
\usepackage[T1]{fontenc}
\usepackage[utf8]{inputenc}

%chemistry stuff
%\usepackage[version=4]{mhchem}
%\usepackage{chemfig}

%plotting
\usepackage{pgfplots}
\usepackage{tikz}
\usetikzlibrary{cd}
\tikzset{middleweight/.style={pos = 0.5}}
%\tikzset{weight/.style={pos = 0.5, fill = white}}
%\tikzset{lateweight/.style={pos = 0.75, fill = white}}
%\tikzset{earlyweight/.style={pos = 0.25, fill=white}}

%\usepackage{natbib}

%graphics stuff
\usepackage{graphicx}
\graphicspath{ {./images/} }
\usepackage[style=numeric, backend=biber]{biblatex} % Use the numeric style for Vancouver
\addbibresource{ps4.bib}
%code stuff
%when using minted, make sure to add the -shell-escape flag
%you can use lstlisting if you don't want to use minted
%\usepackage{minted}
%\usemintedstyle{pastie}
%\newminted[javacode]{java}{frame=lines,framesep=2mm,linenos=true,fontsize=\footnotesize,tabsize=3,autogobble,}
%\newminted[cppcode]{cpp}{frame=lines,framesep=2mm,linenos=true,fontsize=\footnotesize,tabsize=3,autogobble,}

%\usepackage{listings}
%\usepackage{color}
%\definecolor{dkgreen}{rgb}{0,0.6,0}
%\definecolor{gray}{rgb}{0.5,0.5,0.5}
%\definecolor{mauve}{rgb}{0.58,0,0.82}
%
%\lstset{frame=tb,
%	language=Java,
%	aboveskip=3mm,
%	belowskip=3mm,
%	showstringspaces=false,
%	columns=flexible,
%	basicstyle={\small\ttfamily},
%	numbers=none,
%	numberstyle=\tiny\color{gray},
%	keywordstyle=\color{blue},
%	commentstyle=\color{dkgreen},
%	stringstyle=\color{mauve},
%	breaklines=true,
%	breakatwhitespace=true,
%	tabsize=3
%}
% text + color boxes
%\renewcommand{\mathbf}[1]{\mathbb{#1}}
%\usepackage[most]{tcolorbox}
%\tcbuselibrary{breakable}
%\tcbuselibrary{skins}
%\newtcolorbox{problem}[1]{colback=white,enhanced,title={\small #1},
%          attach boxed title to top center=
%{yshift=-\tcboxedtitleheight/2},
%boxed title style={size=small,colback=black!60!white}, sharp corners, breakable}
%including PDFs
%\usepackage{pdfpages}
\setlength{\parindent}{0pt}
\usepackage{cancel}
\pagestyle{fancy}
\fancyhf{}
\rhead{Avinash Iyer}
\lhead{Economics of Education: Problem Set 4}
\newcommand{\card}{\text{card}}
\newcommand{\ran}{\text{ran}}
\newcommand{\N}{\mathbb{N}}
\newcommand{\Q}{\mathbb{Q}}
\newcommand{\Z}{\mathbb{Z}}
\newcommand{\R}{\mathbb{R}}
\newcommand{\C}{\mathbb{C}}
\newcommand{\iprod}[2]{\left\langle #1,#2\right\rangle}
\newcommand{\norm}[1]{\left\Vert #1\right\Vert}
\setcounter{secnumdepth}{0}
\begin{document}
  \section{Summary}%
  In Morton et al.'s paper, ``A multi-state, student-level analysis of the effects of the four-day school week on student achievement and growth,''\supercite{morton_et_al} the authors examine the adoption of four-day school weeks --- which have become increasingly popular, especially in the post-COVID era, where schools find staffing increasingly difficult --- and its effects on students' academic achievement, in order to provide proper evaluation of the policy. The NWEA Measure of Academic Progress Growth exams are administered every fall, winter, and spring, which Morton et al. cite as a superior way to measure the effect of the four-day week on gains, as opposed to annual spring exams (which are confounded by the effects of summer vacation). As a result, the authors use a two-way fixed effects difference-in-difference estimate to examine the effects of adoption of a four-day week, using samples from six states (CO, IA, KS, MT, ND, and WY), as well as examine whether the effects of a four-day school week on gains are even across rural vs. non-rural districts, as well as on the basis of grades and particular subgroups. They find that school enrollment did not change significantly as a result of adoption (removing a potential source of bias), and after accounting for grade-term, school, and student fixed effects, find a statistically significant drop in fall-to-spring gains in both reading and math as a result of adoption of the four-day week. The drop in fall-to-spring gains was higher in non-rural school districts (at a stronger statistical significance) than in rural districts. At the same time, the authors do qualify some of their results --- it is possible that the four-day school week could allow for greater teacher retention (in which case, the positive effects of teacher retention could outweigh the negative effects on student achievement from lower instructional time), but such research has not been taken up yet.
  \section{Critique}%
  My impression is that the paper is very comprehensive and takes a much more thorough look at the effects of the four-day work week than the previous papers it cites. Using a standardized exam that is administered throughout the year (as opposed to once in the spring or fall) allows for a better measure of the true effects that the length of school instruction has on student achievement. Additionally, assessing gains (as opposed to changes or levels) is a much more appropriate measure of the effects on student achievement, as the particular measure of gains assesses the same student at the same school in the same year --- this purely isolates the effect of the instructional period length, rather than potentially catching factors that bias the perceived gains downward (or upward). The combination of these factors --- measuring within-year gains on a standardized exam --- suggests that the researchers have a plausible causal estimate of the effects of a four-day school week, which stands in contrast to previous attempts that did not successfully fully isolate the effects of adoption. At the same time, even if the paper documents a negative effect of the four-day week on student achievement, the researchers were also careful to note that the potential alternative (of lower teacher retention from maintaining the five-day week) could yield worse overall results for students than the four-day week; this consideration to potential alternatives shows that the researchers still maintain their epistemic humility, which I find quite admirable.\\

  Overall, I think this paper is a very useful resource for policymakers when considering the tradeoffs of adopting a four-day school week. The authors thoroughly documented the effects in both rural and non-rural settings, for instance, which may inform how different districts view the effects of a four-day policy (although the authors do note that the deleterious effects of four-day policies increase over time for rural areas). While the relationship between four-day policies and teacher retention (and academic achievement from smaller class sizes) is not fully answered by the authors, they do still give it a nod, which does allow policymakers some leeway to experiment with ideas to maintain academic achievement while keeping the four-day week.\\

  On a more personal note, this paper does effectively back up the thesis I have had surrounding primary and secondary education --- that more school in K--12 is good, and less school in K--12 is bad --- by documenting that the four-day week meaningfully negatively affected instruction time, and thus academic achievement. This can also serve as a template for understanding the potential deleterious effects of extended summer breaks on academic achievement (for instance), and how policymakers could work to mitigate those effects.
  \section{Discussion Questions}%
  \begin{itemize}
    \item What are some potential explanations for the heterogeneous effects of adoption of the four-day week between rural and non-rural school districts? Must there necessarily be convergence of effect size between rural and non-rural districts?
    \item What are some possible sources of selection bias in the effects of adoption on test scores, and how do the authors test for and mitigate such potential selection bias?
    \item Keeping the effect of the adoption of the four-day week on test scores in mind, what is a potential advantage of adoption relative to the most likely alternative.
  \end{itemize}
  \section{Table Breakdown}%
  While the primary results of Morton et al. are contained in Table 4 of the paper, I will focus on Table 6 as it focuses on the heterogeneity of their findings (a result that is much more interesting and relevant than just the results themselves). Table 6 accounts for all the relevant grade-term, school, and student fixed effects, which ensures the results are as accurate as possible.\\

  Column (1) of the table looks at overall effects of adoption of the four-day week on math scores, with Panel A looking at year-over-year spring achievement --- in it, they find statistically insignificant negative effects of adoption on achievement --- the drop in scores is $0.89$ standard deviations below $0$. When looking at Panel B, which measures fall-to-spring gains (a measure that specifically isolates the effects of the four-day week on achievement without the confounder of summer break), they find a statistically significant drop at the $p = 0.10$ (or, in other words, the drop is at least 1.64 standard deviations below $0$).\\

  Column (2) and (3) split the effects on math scores between rural areas and non-rural areas. In rural areas, the effect of the four-day week on year-over-year spring achievement is positive and statistically insignificant (with the increase in scores being $0.52$ standard deviations above $0$). In non-rural areas, the drop in math scores is $0.081$ exam standard deviations, which is statistically significant (specifically more than $2$ standard deviations below $0$ in effect). Examining fall-to-spring gains, rural areas see a statistically insignificant drop in gains, while non-rural areas see a statistically significant drop in gains of $0.083$ exam standard deviations (which is also more than $2$ standard deviations below $0$ in effect).\\

  Columns (4)--(6) focus on the effects of the four-day week on reading. Looking at column (4), which focuses on overall effect for both rural and non-rural districts, we see a statistically significant drop in year-over-year spring achievement of $0.072$ exam standard deviations (which is more than $2$ standard deviations below $0$ in effect), in addition to a statistically significant drop in fall-to-spring gains of $0.062$ exam standard deviations (which is more than $3$ standard deviations below $0$ in effect).\\

  Column (5) focuses on rural districts that adopted the four-day week, and finds a statistically insignificant drop in year-over-year spring achievement (specifically, the drop is within 1 standard deviation of $0$ in effect), it does find a statistically significant drop in fall-to-spring reading gains of 0.038 exam standard deviations (which is $2$ standard deviations below $0$ in effect). Column (6), which focuses on non-rural districts, finds stronger and more statistically significant drops in scores. For year-over-year spring achievement, they find a drop of $0.106$ exam standard deviations (which is nearly $3$ standard deviations below $0$ in effect), and for fall-to-spring gains, they find a drop of $0.086$ exam standard deviations (which is more than $3$ standard deviations below $0$ in effect).\\

  Overall, the results in Table 6 provide the strongest support to the authors' thesis --- that adoption of the four-day week results in deleterious outcomes for academic achievement, and this effect is larger for non-rural school districts than for rural school districts.
  \section{Outside Sources}%
  The adoption of four-day weeks, especially in rural areas, seems to be a very old trend. The New York Times published the article ``Four-Day Week: A Trend in Rural Schools'' in 1982 (written by Edward B. Fiske)\supercite{fiske_four-day_1982}, where they found that ``35 of Colorado's 181 districts'' considered adopting a four-day week. The article discusses how districts tended to adopt the week for a few reasons, primarily budget-related.\\

  For instance, the Edison school district in Colorado noted a drop of 26\% in utility costs and 20\% in transportation costs by the adoption of the four-day week, which was especially important in the early 1980s due to the energy crisis and corresponding increases in cost of living --- not unlike the past few years. Fiske notes that the move toward four-day weeks is broadly popular among parents, students, and teachers; however, in a marked contrast to the results found in Morton et al., Fiske says that school officials believed the four-day week yielded positive academic results. Fiske cites a finding that absenteeism among students and faculty dropped upon adoption of the four-day week, which provides some evidence for the idea that Morton et al. bring up in the discussion of their paper (specifically, that adoption may be a better alternative due to the deleterious effects of absenteeism on school achievement).\\

  Morton et al. cite various state laws mandating minimum instructional hours as a reason why some schools switched to four-day weeks --- Fiske discusses this challenge, such as instituting longer school days to satisfy minimum instructional hours (as well as changing instruction time mandates from days to hours). Fiske cites the longer school days as a source of stress, especially for elementary school-aged children, but the extended instructional hours may be what informs the academic achievement measures.\\

  Overall, Fiske's article presents a nuanced understanding of the four-day school week, primarily focused on a rural setting, and provides some evidence to back up Morton et al.'s theses, but does not provide extensive evidence to support moving towards a four-day week.
  \printbibliography
\end{document}
