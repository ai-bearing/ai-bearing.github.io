\documentclass[8pt]{extarticle}
\title{}
\author{}
\date{}
\usepackage[shortlabels]{enumitem}


%paper setup
\usepackage{geometry}
\geometry{letterpaper, portrait, margin=1in}
\usepackage{fancyhdr}
% sans serif font:
\usepackage{cmbright}
%symbols
\usepackage{amsmath}
\usepackage{bigints}
\usepackage{amssymb}
\usepackage{amsthm}
\usepackage{mathtools}
\usepackage{bbm}
\usepackage[hidelinks]{hyperref}
\usepackage{gensymb}
\usepackage{multirow,array}
\usepackage{multicol}

\newtheorem*{remark}{Remark}
\usepackage[T1]{fontenc}
\usepackage[utf8]{inputenc}

%chemistry stuff
%\usepackage[version=4]{mhchem}
%\usepackage{chemfig}

%plotting
\usepackage{pgfplots}
\usepackage{tikz}
\tikzset{middleweight/.style={pos = 0.5}}
%\tikzset{weight/.style={pos = 0.5, fill = white}}
%\tikzset{lateweight/.style={pos = 0.75, fill = white}}
%\tikzset{earlyweight/.style={pos = 0.25, fill=white}}

%\usepackage{natbib}

%graphics stuff
\usepackage{graphicx}
\graphicspath{ {./images/} }
\usepackage[style=numeric, backend=biber]{biblatex} % Use the numeric style for Vancouver
\addbibresource{the_bibliography.bib}
%code stuff
%when using minted, make sure to add the -shell-escape flag
%you can use lstlisting if you don't want to use minted
%\usepackage{minted}
%\usemintedstyle{pastie}
%\newminted[javacode]{java}{frame=lines,framesep=2mm,linenos=true,fontsize=\footnotesize,tabsize=3,autogobble,}
%\newminted[cppcode]{cpp}{frame=lines,framesep=2mm,linenos=true,fontsize=\footnotesize,tabsize=3,autogobble,}

%\usepackage{listings}
%\usepackage{color}
%\definecolor{dkgreen}{rgb}{0,0.6,0}
%\definecolor{gray}{rgb}{0.5,0.5,0.5}
%\definecolor{mauve}{rgb}{0.58,0,0.82}
%
%\lstset{frame=tb,
%	language=Java,
%	aboveskip=3mm,
%	belowskip=3mm,
%	showstringspaces=false,
%	columns=flexible,
%	basicstyle={\small\ttfamily},
%	numbers=none,
%	numberstyle=\tiny\color{gray},
%	keywordstyle=\color{blue},
%	commentstyle=\color{dkgreen},
%	stringstyle=\color{mauve},
%	breaklines=true,
%	breakatwhitespace=true,
%	tabsize=3
%}
% text + color boxes
\renewcommand{\mathbf}[1]{\mathbbm{#1}}
\usepackage[most]{tcolorbox}
\tcbuselibrary{breakable}
\tcbuselibrary{skins}
\newtcolorbox{problem}[1]{colback=white,enhanced,title={\small #1},
          attach boxed title to top center=
{yshift=-\tcboxedtitleheight/2},
boxed title style={size=small,colback=black!60!white}, sharp corners, breakable}
%including PDFs
%\usepackage{pdfpages}
\setlength{\parindent}{0pt}
\usepackage{cancel}
\pagestyle{fancy}
\fancyhf{}
\rhead{Avinash Iyer}
\lhead{Mathematical Statistics: Class Notes}
\newcommand{\card}{\text{card}}
\newcommand{\ran}{\text{ran}}
\newcommand{\N}{\mathbbm{N}}
\newcommand{\Q}{\mathbbm{Q}}
\newcommand{\Z}{\mathbbm{Z}}
\newcommand{\R}{\mathbbm{R}}
\setcounter{secnumdepth}{0}
\begin{document}
\section{$T$ and $F$ Distributions}%
  The purpose of both of these distributions is to allow for inferences about $\mu$ and $\sigma$ in an unknown distribution. Both are quotients of known distributions.\\
\subsubsection{Preliminaries}%
  \begin{description}
    \item[Sample Mean:] Let $Y_1,\dots,Y_n$ be a random, independent sample from a distribution with mean $\mu$ and variance $\sigma^2$. Then,
      \begin{align*}
        \overline{Y} &= \frac{1}{n}\sum_{i=1}^{n}Y_i \tag*{Sample Mean}
      \end{align*}
      is a distribution with mean $\overline{\mu} = \mu$ and variance $\overline{\sigma}^2 = \frac{\sigma^2}{n}$. If the underlying distribution is a normal distribution, then $\frac{\overline{Y}-\mu}{\sigma/\sqrt{n}}$ is a \textit{standard} normal distribution.
    \item[Sample Variance:] The \textit{sample variance} is defined as
      \begin{align*}
        S^2 &= \frac{1}{n-1}\sum_{i=1}^{n}(Y_i-\overline{Y})^2.\tag*{Sample Variance}
      \end{align*}
      It is important to note that the sample variance is found for samples drawn from a distribution; for population standard deviation/variance, we use $n$ instead of $n-1$ in the denominator.\\

      When $Y_i$ is a normal distribution, then $\frac{(n-1)S^2}{\sigma^2}$ is a $\chi^2$ distribution with $n-1$ df --- $S^2$ and $\overline{Y}$ are independent.
  \end{description}
\subsubsection{Definition of $T$ Distribution}%
  Let $Z$ be a standard normal distribution, $W$ be $\chi^2$ with $\nu$ df, and $Z$ and $W$ be independent. Then,
  \begin{align*}
    T &= \frac{Z}{\sqrt{W/\nu}}
  \end{align*}
  has a $T$ distribution with $\nu$ df.
  \begin{description}
    \item[Creating a $T$ Distribution:] Let $Y_i$ be sampled from a normal distribution with mean $\mu$ and standard deviation $\sigma$.\\
      
      Then, $Z = \frac{\overline{Y}-\mu}{\sigma/\sqrt{n}}$ is a standard normal distribution, and $W = \frac{(n-1)S^2}{\sigma^2}$ is $\chi^2$ with $n-1$ df.\\

      So,
      \begin{align*}
        T &= \frac{Z}{\sqrt{W/(n-1)}}\\
          &= \frac{(\overline{Y}-\mu)\sqrt{n}}{\sigma}\sqrt{\frac{(n-1)\sigma^2}{S^2}}\\
          &= \frac{(\overline{Y}-\mu)\sqrt{n}}{S}
      \end{align*}
      has a $T$ distribution with $n-1$ df.
    \item[$T$ Distribution:] Let $Y_1,\dots,Y_6$ be samples from a normal distribution with unknown $\mu$, $\sigma$. Estimate $P(|\overline{Y}-\mu|<(2S/\sqrt{n}))$.\\

      Thus, we have
      \begin{align*}
        P\left(|\overline{Y}-\mu| \leq \frac{2S}{\sqrt{n}}\right) &= P\left(-2\leq \frac{\sqrt{n}(\overline{Y}-\mu)}{S}\leq 2\right)\\
                                                                  &= P(-2 \leq T \leq 2)
      \end{align*}
      Thus, for $n=6$, we have that our random variable $T$ has 5 df. By looking at a $T$ distribution table, we can find that $P \approx 0.9$. We can also use R.
  \end{description}
\end{document}
