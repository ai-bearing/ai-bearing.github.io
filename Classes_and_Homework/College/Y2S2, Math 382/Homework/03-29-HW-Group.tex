\documentclass[10pt]{extarticle}
\title{}
\author{Avinash Iyer}
\date{}
\usepackage[shortlabels]{enumitem}

%font setup
%
\usepackage{newpxtext,eulerpx}

%paper setup
\usepackage{geometry}
\geometry{letterpaper, portrait, margin=1in}
\usepackage{fancyhdr}

%symbols
\usepackage{amsmath}
\usepackage{mathtools}
\usepackage{amssymb}
\usepackage{hyperref}
\usepackage{gensymb}

\usepackage[T1]{fontenc}
\usepackage[utf8]{inputenc}

%chemistry stuff
\usepackage[version=4]{mhchem}
\usepackage{chemfig}

%plotting
\usepackage{pgfplots}
\usepackage{tikz}
\tikzset{middleweight/.style={pos = 0.5, fill=white}}
\tikzset{weight/.style={pos = 0.5, fill = white}}
\tikzset{lateweight/.style={pos = 0.75, fill = white}}
\tikzset{earlyweight/.style={pos = 0.25, fill=white}}

%\usepackage{natbib}

%graphics stuff
\usepackage{graphicx}
\graphicspath{ {./images/} }

%code stuff
%when using minted, make sure to add the -shell-escape flag
%you can use lstlisting if you don't want to use minted
%\usepackage{minted}
%\usemintedstyle{pastie}
%\newminted[javacode]{java}{frame=lines,framesep=2mm,linenos=true,fontsize=\footnotesize,tabsize=3,autogobble,}
%\newminted[cppcode]{cpp}{frame=lines,framesep=2mm,linenos=true,fontsize=\footnotesize,tabsize=3,autogobble,}

\usepackage{listings}
\usepackage{color}
\definecolor{dkgreen}{rgb}{0,0.6,0}
\definecolor{gray}{rgb}{0.5,0.5,0.5}
\definecolor{mauve}{rgb}{0.58,0,0.82}

\lstset{frame=tb,
	language=Java,
	aboveskip=3mm,
	belowskip=3mm,
	showstringspaces=false,
	columns=flexible,
	basicstyle={\small\ttfamily},
	numbers=none,
	numberstyle=\tiny\color{gray},
	keywordstyle=\color{blue},
	commentstyle=\color{dkgreen},
	stringstyle=\color{mauve},
	breaklines=true,
	breakatwhitespace=true,
	tabsize=3
}
% text + color boxes
\usepackage{tcolorbox}
\tcbuselibrary{breakable}
\newtcolorbox{problem}[1]{colback = white, title = {#1}, breakable}
\newtcolorbox{solution}{colback = white, colframe = black!75!white, title = Solution, breakable}
%including PDFs
\usepackage{pdfpages}
\setlength{\parindent}{0pt}

\pagestyle{fancy}
\fancyhf{}
\rhead{Ling Chen, Avinash Iyer, Nora Manukyan, Vincent Ng}
\lhead{Homework Section 3.1, Group}
\begin{document}{
  \begin{problem}{Problem 1}
    Let $G = (X,E,Y)$ be a bipartite graph, and let $d\in \mathbb{N}$ be a fixed constant. Show that if $|N(S)| \geq |S|-d$ for every $S\subseteq X$, then $G$ contains a matching of cardinality $|X|-d$.
    \tcblower
    We add $d$ vertices to the $Y$ partition such that $|N(S)| + d \geq |S|$ for all $S\subseteq X$. Then, we will create an edge between every vertex $x\subseteq X$ and every auxiliary vertex. Let $G' = (X,E',Y')$ denote this new graph.\\

    Let $S'\subseteq X$ be a set that contains all vertices of $X$ --- then, $N(S') \subseteq Y'$ must be of cardinality at least $|S'|$. So, for all $S'\subseteq X$, it follows that $|N(S')| \geq |S'|$, so $G'$ has an $X$-perfect matching by Hall's Theorem.\\

    Since there is a matching in $G'$ that saturates every vertex in $X$, this matching maps every $x\in X$ to every $y'\in Y'$. We remove $d$ edges from the matching, corresponding to the $d$ auxiliary vertices in $Y'$. Thus, $G$ has a matching of $|X|-d$ edges.
  \end{problem}
  \begin{problem}{3.1.19}
    Let $\mathbf{A} = (A_1,\dots,A_m)$ be a collection of subsets of $Y$. A \textbf{system of distinct representatives}, or SDR, for $\mathbf{A}$ is a set of \textit{distinct} elements $a_1,\dots,a_m\in Y$ where $a_i\in A_i$. Prove that $\mathbf{A}$ has a SDR if and only if $\left|\bigcup_{i\in S} A_i\right| \geq |S|$ for every $S\subseteq \{1,\dots,m\}$.
    \tcblower
    Let $G = (\{1,\dots,m\},E,\mathbf{A})$ where edges are defined from every element in $\{1,\dots,m\}$ to every element in $A_i$ for $i\in \{1,\dots,m\}$. The definition of $N(S)$ that follows from this definition of $G$ is $\bigcup_{i\in S} A_i$.
    \begin{description}[font=\normalfont\scshape]
      \item[$(\Rightarrow)$] If $\mathbf{A}$ has a SDR, this is equivalent to a perfect matching in $G$ from every $i\in \{1,\dots,m\}$ to every $a_i\in A_i$ --- so, by Hall's theorem, we know that $|N(S)| \geq |S|$ for every $S\subseteq \{1,\dots,m\}$. So, $\mathbf{A} \Rightarrow \left|\bigcup_{i\in S} A_i\right| \geq |S|$.\\
      \item[$(\Leftarrow)$] Let $\left|\bigcup_{i\in S}A_i\right| \geq |S|$ for all $S\subseteq \{1,\dots,m\}$. Then, by the definition of $N(S)$ in $G$, we know that $|N(S)| \geq |S|$ for every $S\subseteq \{1,\dots,m\}$, meaning $G$ has a perfect matching. This means that every $i\subseteq \{1,\dots,m\}$ maps to a unique element $a_i$ in $\mathbf{A}$, as $G$ has a perfect matching. This is, thus, equivalent to $\mathbf{A}$ having a SDR.
    \end{description}

  \end{problem}
  \begin{problem}{3.1.25}
    A doubly stochastic matrix $Q$ is a nonnegative real matrix in which every row and column sums to one. Prove that a doubly stochastic matrix $Q$ can be expressed as a convex combination of permutation matrices --- i.e., there exist permutation matrices $P_1,\dots,P_m$ and \textit{nonnegative} real coefficients $c_1,\dots,c_m$ such that $Q = c_1P_1 + c_2P_2 + \dots + c_mP_m$, where $\sum_{i = 1}^{m} c_i = 1$.
    \tcblower
    We will prove via induction as follows:
    \begin{description}[font = \normalfont\scshape]
      \item[Base Case] If $Q$ is a permutation matrix itself, then it is a doubly stochastic matrix and satisfies the base case.
      \item[Inductive Step] Suppose that $Q$ is a $m\times m$ matrix with at least $m+1$ nonzero entries. Let $G$ represent a bipartite graph, where $I$ represents the set of $m$ rows, and $J$ represents the set of $m$ columns. Each nonzero entry in $(i,j)$ represents an edge between the $i$th vertex in $I$ and the $j$th vertex in $J$.\\

        Let $S\subseteq I$ and $|S| = d$ for some $d\in \mathbb{N}$. Then, $N(S) \subseteq J$ represents the columns of at least one nonzero entry for each of the rows in $S$. The sum of each of these rows is $1$, meaning the sum of the rows in $S$ is $d$.\\

        Each column sums to maximum $1$, meaning $|S| \leq |N(S)|$, satisfying the condition for Hall's Theorem. Since $|I| = |J|$, the graph has a perfect matching, meaning we can find a permutation matrix $P_1$ and a positive number $c_1$. By the inductive hypothesis, $Q - c_1P_1 = c_2P_2+\dots+c_mP_m$, so $Q = c_1P_1 + c_2P_2 + \dots + c_mP_m$.
    \end{description}
  \end{problem}
  \begin{problem}{3.1.29}
    Use the König-Egerváry theorem to prove that every bipartite graph $G$ has a matching of size at least $e(G)/\Delta(G)$. Use this to conclude that every subgraph of $K_{n,n}$ with more than $(k-1)n$ has a matching of size at least $k$.
    \tcblower
    Let $G$ be bipartite. Then, from the König-Egerváry theorem, we know that $\alpha'(G) = \beta(G)$.\\

    Let $Q$ represent the minimum vertex cover, meaning $|Q| = \beta(G)$. Every edge is incident on some vertex $v\in Q$, and the upper bound on $d(v)$ is $\Delta(G)$. This means that $|Q|\Delta(G) \geq e(G)$ (assuming that there would be double counting in $|Q|\Delta(G)$). So, $\beta(G)\Delta(G) \geq n(G)$. Therefore, $\beta(G) \geq \frac{n(G)}{\Delta(G)}$. So, $\alpha'(G) \geq \frac{n(G)}{\Delta(G)}$, as $\alpha'(G) = \beta(G)$.
  \end{problem}
}\end{document}
