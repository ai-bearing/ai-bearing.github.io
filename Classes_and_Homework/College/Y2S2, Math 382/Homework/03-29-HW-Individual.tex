\documentclass[10pt]{extarticle}
\title{}
\author{Avinash Iyer}
\date{}
\usepackage[shortlabels]{enumitem}

%font setup
%
\usepackage{newpxtext,eulerpx}

%paper setup
\usepackage{geometry}
\geometry{letterpaper, portrait, margin=1in}
\usepackage{fancyhdr}

%symbols
\usepackage{amsmath}
\usepackage{mathtools}
\usepackage{amssymb}
\usepackage{hyperref}
\usepackage{gensymb}

\usepackage[T1]{fontenc}
\usepackage[utf8]{inputenc}

%chemistry stuff
\usepackage[version=4]{mhchem}
\usepackage{chemfig}

%plotting
\usepackage{pgfplots}
\usepackage{tikz}
\tikzset{middleweight/.style={pos = 0.5, fill=white}}
\tikzset{weight/.style={pos = 0.5, fill = white}}
\tikzset{lateweight/.style={pos = 0.75, fill = white}}
\tikzset{earlyweight/.style={pos = 0.25, fill=white}}

%\usepackage{natbib}

%graphics stuff
\usepackage{graphicx}
\graphicspath{ {./images/} }

%code stuff
%when using minted, make sure to add the -shell-escape flag
%you can use lstlisting if you don't want to use minted
%\usepackage{minted}
%\usemintedstyle{pastie}
%\newminted[javacode]{java}{frame=lines,framesep=2mm,linenos=true,fontsize=\footnotesize,tabsize=3,autogobble,}
%\newminted[cppcode]{cpp}{frame=lines,framesep=2mm,linenos=true,fontsize=\footnotesize,tabsize=3,autogobble,}

\usepackage{listings}
\usepackage{color}
\definecolor{dkgreen}{rgb}{0,0.6,0}
\definecolor{gray}{rgb}{0.5,0.5,0.5}
\definecolor{mauve}{rgb}{0.58,0,0.82}

\lstset{frame=tb,
	language=Java,
	aboveskip=3mm,
	belowskip=3mm,
	showstringspaces=false,
	columns=flexible,
	basicstyle={\small\ttfamily},
	numbers=none,
	numberstyle=\tiny\color{gray},
	keywordstyle=\color{blue},
	commentstyle=\color{dkgreen},
	stringstyle=\color{mauve},
	breaklines=true,
	breakatwhitespace=true,
	tabsize=3
}
% text + color boxes
\usepackage{tcolorbox}
\tcbuselibrary{breakable}
\newtcolorbox{problem}[1]{colback = white, title = {#1}, breakable}
\newtcolorbox{solution}{colback = white, colframe = black!75!white, title = Solution, breakable}
%including PDFs
\usepackage{pdfpages}
\setlength{\parindent}{0pt}

\pagestyle{fancy}
\fancyhf{}
\rhead{Avinash Iyer}
\lhead{Homework Section 3.1, Individual}
\begin{document}{
  \begin{problem}{3.1.1}
    Find a maximum matching in each graph below. Prove that it is a maximum matching by exhibiting an optimal solution to the dual problem (minimum vertex cover). Explain why this proves that the matching is optimal.
    \begin{center}
      Graph 1:\\
      \vspace{10pt}
      \begin{tikzpicture}
        \filldraw (0,0) circle (2pt)
              (1,0) circle (2pt)
              (2,0) circle (2pt)
              (3,0) circle (2pt)
              (4,0) circle (2pt)
              (0,1) circle (2pt)
              (1,1) circle (2pt)
              (2,1) circle (2pt)
              (3,1) circle (2pt);
        \draw (0,0) -- (2,1);
        \draw (1,0) -- (2,1);
        \draw (2,1) -- (2,0);
        \draw (1,0) -- (3,1);
        \draw (3,1) -- (2,0);
        \draw (3,0) -- (3,1);
        \draw (3,1) -- (4,0);
        \draw (0,1) -- (2,0);
        \draw (1,1) -- (2,0);
        \draw (2,1) -- (4,0);
      \end{tikzpicture}\\
      \vspace{20pt}
      Graph 2:\\
      \vspace{10pt}
      \begin{tikzpicture}
        \filldraw (0,0) circle (2pt)
                  (1,0) circle (2pt)
                  (2,0) circle (2pt)
                  (3,0) circle (2pt)
                  (4,0) circle (2pt);
        \filldraw (0,1) circle (2pt)
                  (1,1) circle (2pt)
                  (2,1) circle (2pt)
                  (3,1) circle (2pt)
                  (4,1) circle (2pt);
        \draw (0,1) -- (1,0);
        \draw (0,1) -- (2,0);
        \draw (1,1) -- (2,0);
        \draw (2,1) -- (1,0);
        \draw (3,1) -- (0,0);
        \draw (3,1) -- (2,0);
        \draw (3,1) -- (3,0);
        \draw (3,1) -- (4,0);
        \draw (4,1) -- (1,0);
        \draw (4,1) -- (2,0);
      \end{tikzpicture}\\
      \vspace{20pt}
      Graph 3:\\
      \vspace{10pt}
      \begin{tikzpicture}
        \filldraw (0,0) circle (2pt)
                  (1,0) circle (2pt)
                  (2,0) circle (2pt)
                  (3,0) circle (2pt)
                  (4,0) circle (2pt);
        \filldraw (0,1) circle (2pt)
                  (1,1) circle (2pt)
                  (2,1) circle (2pt)
                  (3,1) circle (2pt)
                  (4,1) circle (2pt);

        \draw (0,1) -- (1,0);
        \draw (0,1) -- (3,0);

        \draw (1,1) -- (0,0);
        \draw (1,1) -- (1,0);
        \draw (1,1) -- (2,0);
        \draw (1,1) -- (3,0);

        \draw (2,1) -- (1,0);
        \draw (2,1) -- (3,0);

        \draw (3,1) -- (0,0);
        \draw (3,1) -- (1,0);
        \draw (3,1) -- (3,0);
        \draw (3,1) -- (4,0);

        \draw (4,1) -- (1,0);
        \draw (4,1) -- (3,0);
      \end{tikzpicture}
    \end{center}
  \end{problem}
}\end{document}
