\documentclass[9pt]{extarticle}
\title{}
\author{Avinash Iyer}
\date{}
\usepackage[shortlabels]{enumitem}

%font setup
%
%\usepackage[math]{anttor}

%paper setup
\usepackage{geometry}
\geometry{letterpaper, portrait, margin=1in}
\usepackage{fancyhdr}

%symbols
\usepackage{amsmath}
\usepackage{mathtools}
\usepackage{amssymb}
\usepackage{hyperref}
\usepackage{gensymb}

\usepackage[T1]{fontenc}
\usepackage[utf8]{inputenc}

%chemistry stuff
\usepackage[version=4]{mhchem}
\usepackage{chemfig}

%plotting
\usepackage{pgfplots}
\usepackage{tikz}

%\usepackage{natbib}

%graphics stuff
\usepackage{graphicx}
\graphicspath{ {./images/} }

%code stuff
%when using minted, make sure to add the -shell-escape flag
%you can use lstlisting if you don't want to use minted
%\usepackage{minted}
%\usemintedstyle{pastie}
%\newminted[javacode]{java}{frame=lines,framesep=2mm,linenos=true,fontsize=\footnotesize,tabsize=3,autogobble,}
%\newminted[cppcode]{cpp}{frame=lines,framesep=2mm,linenos=true,fontsize=\footnotesize,tabsize=3,autogobble,}

\usepackage{listings}
\usepackage{color}
\definecolor{dkgreen}{rgb}{0,0.6,0}
\definecolor{gray}{rgb}{0.5,0.5,0.5}
\definecolor{mauve}{rgb}{0.58,0,0.82}

\lstset{frame=tb,
	language=Java,
	aboveskip=3mm,
	belowskip=3mm,
	showstringspaces=false,
	columns=flexible,
	basicstyle={\small\ttfamily},
	numbers=none,
	numberstyle=\tiny\color{gray},
	keywordstyle=\color{blue},
	commentstyle=\color{dkgreen},
	stringstyle=\color{mauve},
	breaklines=true,
	breakatwhitespace=true,
	tabsize=3
}
% text + color boxes
\usepackage{tcolorbox}
\tcbuselibrary{breakable}
\newtcolorbox{problem}[1]{colback = white, title = {#1}, breakable}
\newtcolorbox{solution}{colback = white, colframe = black!75!white, title = Solution, breakable}
%including PDFs
\usepackage{pdfpages}
\setlength{\parindent}{0pt}

\pagestyle{fancy}
\fancyhf{}
\rhead{Ling Chen, Avinash Iyer, Nora Manukyan, Vincent Ng}
\lhead{Homework Section 1.3, Group}
\begin{document}
  \begin{problem}{1.3.17}
    Let $G$ be a graph with at least two vertices. Prove or disprove:
    \begin{enumerate}[(a)]
      \item Deleting a vertex of degree $\Delta(G)$ cannot increase the average degree.
      \item Deleting a vertex of degree $\delta(G)$ cannot decrease the average degree.
    \end{enumerate}
  \end{problem}
  \begin{solution}
    \begin{tcolorbox}[colback = white, title = (a), breakable]
      Assume toward contradiction that deleting a vertex of degree $\Delta(G)$ increases the average degree.
      \begin{align*}
        d_{\textrm{avg}}' &> d_{\textrm{avg}} \\
        \frac{2e(G) - 2\Delta(G)}{n(G) - 1} &> \frac{2e(G)}{n(G)}\\
        \frac{2e(G) - 2\Delta(G)}{2e(G)} &> \frac{n(G) - 1}{n(G)}\\
        1 - \frac{\Delta(G)}{e(G)} &> 1 - \frac{1}{n(G)}\\
        \frac{1}{n(G)} - \frac{\Delta(G)}{e(G)} &> 0 \\
        \frac{1}{n(G)} - \frac{2\Delta(G)}{n(G)d_{\textrm{avg}}} &> 0 \\
        \frac{d_{\textrm{avg}} - 2\Delta(G)}{n(G)} &> 0\\
        d_{\textrm{avg}} - 2\Delta(G) &> 0\\
        d_{\textrm{avg}} &> 2\Delta(G)
      \end{align*}
      However, we have reached a contradiction --- by definition, $\Delta(G) \geq d_{\textrm{avg}}$, meaning that $d_{\textrm{avg}}\not> \Delta(G)$, let alone $2\Delta(G)$.
    \end{tcolorbox}
    \begin{tcolorbox}[colback = white, title = (b), breakable]
      Deleting a vertex of the graph $K_{1,1}$ yields a graph with one vertex of degree zero, which is lower than the average degree of $1$ in $K_{1,1}$.
    \end{tcolorbox}
  \end{solution}
  \begin{problem}{1.3.25}
    Prove that every cycle of length $2r$ in a hypercube is contained within a subcube of dimension at most $r$. Can a cycle of length $2r$ be contained in a subcube of dimension less than $r$.
  \end{problem}
  \begin{solution}
    Let $C$ be a cycle of length $2r$ in $Q_n$. Then, $C$ contains $2r$ $n$-tuples. For every tuple in $C$, there exists a ``switched'' tuple where every coordinate is equal to its other, corresponding coordinate, except for one. Since $C$ is a cycle, every coordinate that is switched must be returned to its original state at the end of the cycle --- since there are $2r$ switches (corresponding to the $2r$ edges in $C$), this means there are at most $r$ coordinates that are switched, then switched back sometime along the cycle's path. This means the other $n-r$ coordinates are fixed, implying that $C\subseteq Q_r$, the $r$-dimensional subcube of $Q_k$.\\

    There is a cycle of length $8$ in $Q_3$, defined as follows: $000 \rightarrow 001 \rightarrow 011 \rightarrow 010 \rightarrow 110 \rightarrow 111 \rightarrow 101 \rightarrow 100 \rightarrow 000$.
  \end{solution}
  \begin{problem}{1.3.31}
    Using complete graphs and counting arguments, prove the following:
    \begin{enumerate}[(a)]
      \item $\begin{pmatrix}n\\2\end{pmatrix} = \begin{pmatrix} k\\2 \end{pmatrix} + k(n-k) + \begin{pmatrix} n-k \\ 2 \end{pmatrix}$ for $0\leq k \leq n$.
      \item If $\sum n_i = n$, then $\sum \begin{pmatrix}n_i \\ 2\end{pmatrix} \leq \begin{pmatrix}n \\ 2\end{pmatrix}$.
    \end{enumerate}
  \end{problem}
  \begin{solution}
    \begin{tcolorbox}[colback = white, title = (a), breakable]
      We can consider a decomposition of the edges of $K_n$ into the edge set of $K_k$ and $K_{n-k}$, and some connector edges.\\

      The edge set of $K_n$ has cardinality $\begin{pmatrix}n\\2\end{pmatrix}$, the edge set of $K_k$ has cardinality $\begin{pmatrix}k\\2\end{pmatrix}$, and the edge set of $K_{n-k}$ has cardinality $\begin{pmatrix}n-k \\ 2\end{pmatrix}$. In order for this set of edges to be a full decomposition, we need a graph that connects all the vertices in $K_k$ with all the vertices in $K_{n-k}$, which takes $k(n-k)$ edges. Therefore, we have shown the following result:
      \[
        \begin{pmatrix}
          n\\
          2
        \end{pmatrix} = \begin{pmatrix}
          k\\
          2
        \end{pmatrix} + \begin{pmatrix}
          n-k\\
          2
        \end{pmatrix} + k(n-k)
      \] 
    \end{tcolorbox}
    \begin{tcolorbox}[colback = white, title = (b), breakable]
      Consider the graph $G$, where $|V(G)| = n$ with maximal clique components $H_{1},\dots,H_{k}$. Each of these components has $e(H_i) = \begin{pmatrix}|V(H_i)|\\2\end{pmatrix}$, with total $\sum_{i=1}^{k}\begin{pmatrix}|V(H_i)| \\ 2\end{pmatrix}$. In comparison, if we consider $e(K_{G})$, where $K_G$ is defined as the complete graph on the vertices of $G$, then that value is $\begin{pmatrix}n \\ 2\end{pmatrix}$, and $n = \sum_{i = 1}^{k} |V(H_i)|$. Therefore, the size of the edge set of $G$ is less than or equal to the sum of the sizes of the edge sets of maximal clique components $H_i$ (because the maximal clique components of $G$ could just be $G$ itself).
    \end{tcolorbox}
  \end{solution}
  \begin{problem}{1.3.41}
    Prove or disprove: if $G$ is an $n$-vertex simple graph with maximum degree $\lceil n/2 \rceil$ and minimum degree $\lfloor n/2 \rfloor - 1$, then $G$ is connected.
  \end{problem}
  \begin{solution}
    Let $u,v\in V(G)$ and let $d(u) = \lceil \frac{n}{2}\rceil$. Then, $u$ is adjacent to $\lceil \frac{n}{2}\rceil$ vertices and nonadjacent to $\lfloor \frac{n}{2}\rfloor$ vertices. Let $u\not\leftrightarrow v$.\\

    We want to show that there exists some other vertex such that there exists a $u,v$ path through that vertex. We know that $|N(u)| = d(u) = \lceil \frac{n}{2}\rceil$ and $|N(v)| = d(v) \geq \delta(G) = \lfloor \frac{n}{2}\rfloor - 1$.\\

    Since $u\not\leftrightarrow v$, $N(u),N(v) \subseteq V(G) - \{u,v\}$. So, $|N(u) \cap N(v)| = |N(u)| + |N(v)| - |N(u) \cup N(v)| \geq \left(\lceil \frac{n}{2}\rceil\right) + \left(\lfloor \frac{n}{2}\rfloor - 1\right) - (n-2) = 1$.\\

    Therefore, there must be at least one element in $N(u)\cap N(v)$, meaning $G$ is connected.
  \end{solution}
\end{document}
