
\documentclass[8pt]{extarticle}
\title{}
\author{Avinash Iyer}
\date{}

%font setup
%
%\usepackage[math]{anttor}

%paper setup
\usepackage{geometry}
\geometry{letterpaper, portrait, margin=1in}
\usepackage{fancyhdr}

%symbols
\usepackage{amsmath}
\usepackage{amssymb}
\usepackage{hyperref}
\usepackage{gensymb}

\usepackage[T1]{fontenc}
\usepackage[utf8]{inputenc}

%chemistry stuff
\usepackage[version=4]{mhchem}
\usepackage{chemfig}

%plotting
\usepackage{pgfplots}
\usepackage{tikz}

%\usepackage{natbib}

%graphics stuff
\usepackage{graphicx}
\graphicspath{ {./images/} }

%a useful command
\newcommand{\plain}[1]{\textrm{#1}}

%code stuff
%when using minted, make sure to add the -shell-escape flag
%you can use lstlisting if you don't want to use minted
%\usepackage{minted}
%\usemintedstyle{pastie}
%\newminted[javacode]{java}{frame=lines,framesep=2mm,linenos=true,fontsize=\footnotesize,tabsize=3,autogobble,}
%\newminted[cppcode]{cpp}{frame=lines,framesep=2mm,linenos=true,fontsize=\footnotesize,tabsize=3,autogobble,}

\usepackage{listings}
\usepackage{color}
\definecolor{dkgreen}{rgb}{0,0.6,0}
\definecolor{gray}{rgb}{0.5,0.5,0.5}
\definecolor{mauve}{rgb}{0.58,0,0.82}

\lstset{frame=tb,
	language=Java,
	aboveskip=3mm,
	belowskip=3mm,
	showstringspaces=false,
	columns=flexible,
	basicstyle={\small\ttfamily},
	numbers=none,
	numberstyle=\tiny\color{gray},
	keywordstyle=\color{blue},
	commentstyle=\color{dkgreen},
	stringstyle=\color{mauve},
	breaklines=true,
	breakatwhitespace=true,
	tabsize=3
}
\pagestyle{fancy}
\fancyhf{}
\rhead{Avinash Iyer}
\lhead{Homework 28 Proof}
\begin{document}{
  \begin{quote}
    Show that every continuous function $f: X\rightarrow Y$ maps limit points to limit points. 
  \end{quote}
  We will prove in the forward direction first (i.e., if a function maps limit points to limit points, then it is continuous). Suppose $S\subseteq Y$ is open. Then, $\overline{S}$ is closed. We claim that $f^{-1}(S)$ cannot contain a limit point of $A = f^{-1}(\overline(S))$. Suppose toward contradiction that $\overline{A}$ contains a limit point $p$ of $A$. Then, since $p$ is a limit point of $A$, $f(p)$ is a limit point of $f(A)$ by our assumption; so, $f(p)\in f(f^{-1}(\overline{S}))\subseteq \overline{S}$ as $\overline{S}$ is closed. But, since $p$ is in $\overline{A}$, so $f(p)\in f(\overline{A} = \overline{\overline{S}}) = S$. So $f(p)\in S$ and $f(p)\in \overline{S}$, which is a contradiction.\\
  \\
  By claim, no limit point of $A$ is in $\overline{A}$. So every limit point of $A$ is in $A$. So $A$ is closed, so $f^{-1}(\overline{S})$ is closed, so $f$ is continuous.

}\end{document}
