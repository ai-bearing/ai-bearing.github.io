\documentclass[8pt]{extarticle}
\title{}
\author{}
\date{}
\usepackage[shortlabels]{enumitem}


%paper setup
\usepackage{geometry}
\geometry{letterpaper, portrait, margin=1in}
\usepackage{fancyhdr}
% sans serif font:
\usepackage{cmbright}
%symbols
\usepackage{amsmath}
\usepackage{bigints}
\usepackage{amssymb}
\usepackage{amsthm}
\usepackage{mathtools}
\usepackage{bbm}
\usepackage[hidelinks]{hyperref}
\usepackage{gensymb}
\usepackage{multirow,array}
\usepackage{multicol}

\newtheorem*{remark}{Remark}
\usepackage[T1]{fontenc}
\usepackage[utf8]{inputenc}

%chemistry stuff
%\usepackage[version=4]{mhchem}
%\usepackage{chemfig}

%plotting
\usepackage{pgfplots}
\usepackage{tikz}
\tikzset{middleweight/.style={pos = 0.5}}
%\tikzset{weight/.style={pos = 0.5, fill = white}}
%\tikzset{lateweight/.style={pos = 0.75, fill = white}}
%\tikzset{earlyweight/.style={pos = 0.25, fill=white}}

%\usepackage{natbib}

%graphics stuff
\usepackage{graphicx}
\graphicspath{ {./images/} }
\usepackage[style=numeric, backend=biber]{biblatex} % Use the numeric style for Vancouver
\addbibresource{the_bibliography.bib}
%code stuff
%when using minted, make sure to add the -shell-escape flag
%you can use lstlisting if you don't want to use minted
%\usepackage{minted}
%\usemintedstyle{pastie}
%\newminted[javacode]{java}{frame=lines,framesep=2mm,linenos=true,fontsize=\footnotesize,tabsize=3,autogobble,}
%\newminted[cppcode]{cpp}{frame=lines,framesep=2mm,linenos=true,fontsize=\footnotesize,tabsize=3,autogobble,}

%\usepackage{listings}
%\usepackage{color}
%\definecolor{dkgreen}{rgb}{0,0.6,0}
%\definecolor{gray}{rgb}{0.5,0.5,0.5}
%\definecolor{mauve}{rgb}{0.58,0,0.82}
%
%\lstset{frame=tb,
%	language=Java,
%	aboveskip=3mm,
%	belowskip=3mm,
%	showstringspaces=false,
%	columns=flexible,
%	basicstyle={\small\ttfamily},
%	numbers=none,
%	numberstyle=\tiny\color{gray},
%	keywordstyle=\color{blue},
%	commentstyle=\color{dkgreen},
%	stringstyle=\color{mauve},
%	breaklines=true,
%	breakatwhitespace=true,
%	tabsize=3
%}
% text + color boxes
\renewcommand{\mathbf}[1]{\mathbbm{#1}}
\usepackage[most]{tcolorbox}
\tcbuselibrary{breakable}
\tcbuselibrary{skins}
\newtcolorbox{problem}[1]{colback=white,enhanced,title={\small #1},
          attach boxed title to top center=
{yshift=-\tcboxedtitleheight/2},
boxed title style={size=small,colback=black!60!white}, sharp corners, breakable}
%including PDFs
%\usepackage{pdfpages}
\setlength{\parindent}{0pt}
\usepackage{cancel}
\pagestyle{fancy}
\fancyhf{}
\rhead{Avinash Iyer}
\lhead{Econ 305: Problem Set 6}
\newcommand{\card}{\text{card}}
\newcommand{\ran}{\text{ran}}
\newcommand{\N}{\mathbbm{N}}
\newcommand{\Q}{\mathbbm{Q}}
\newcommand{\Z}{\mathbbm{Z}}
\newcommand{\R}{\mathbbm{R}}
\setcounter{secnumdepth}{0}
\begin{document}
  \begin{problem}{More Limit Pricing}
    An incumbent firm (player 1) is either a low-cost type $\theta_1 = \theta_L$ or a high cost type $\theta_1 = \theta_H$, each with equal probability. In period $t=1$, the incumbent is a monopolist and sets one of two prices, $p_L$ or $p_H$, and its profits in this period depend on its type and the price it chooses, given by the following table:
    \begin{center}
      \begin{tabular}{ccc}
        Type & Profit from $p_L$ & Profit from $p_H$\\
        \hline
        $\theta_L$ & $6$ & $8$\\
        $\theta_H$ & $1$ & $5$
      \end{tabular}
    \end{center}
    After observing the period $t=1$ price, a potential entrant (player 2), which does not know the incumbent's type but knows the distribution of types, can choose to either enter the market ($E$), or stay out ($O$) in period $t=2$. The payoffs of both players in period $2$ depend on the entrant's choice and on the incumbent's type and are given by the following table:
    \begin{center}
      \begin{tabular}{cccc}
        Incumbent's type & Entrant's choice & Incumbent's payoff & Entrant's payoff\\
        \hline
        $\theta_L$ & $E$ & $0$ & $-2$\\
        $\theta_L$ & $O$ & $8$ & $0$\\
        $\theta_H$ & $E$ & $0$ & $1$\\
        $\theta_H$ & $O$ & $5$ & $0$
      \end{tabular}
    \end{center}
    At the beginning of the game the incumbent discounts profits for period $t=2$ using a discount factor $\delta \leq 1$.
    \begin{problem}{(a)}
      Draw the extensive-form game tree.
      \tcblower
    \begin{center}
      \small
      \begin{tikzpicture}[scale=0.75]
        \draw (-5,5) -- (5,5);
        \draw (-5,-5) -- (5,-5);
        \draw (6,6) --node[pos=0.5,anchor=south]{$E$} (5,5);
        \draw (6,4) -- node[pos=0.5,anchor=north]{$O$}(5,5);
        \draw (-6,6) -- node[pos=0.5,anchor=south]{$E$}(-5,5);
        \draw (-6,4) -- node[pos=0.5,anchor=north]{$O$}(-5,5);
        \draw (-6,-4) -- node[anchor=south,pos=0.5]{$E$}(-5,-5);
        \draw (-6,-6) -- node[anchor=north,pos=0.5]{$O$}(-5,-5);
        \draw (6,-4) -- node[anchor=south,pos=0.5]{$E$}(5,-5);
        \draw (6,-6) -- node[anchor=north,pos=0.5]{$O$}(5,-5);
        \filldraw (-5,5) circle (2pt)
          (-5,-5) circle (2pt)
          (5,5) circle (2pt)
          (5,-5) circle (2pt)
          (0,5) circle (2pt)
          (0,-5) circle (2pt)
          (0,0) circle (2pt);
        \draw[thick] (0,5) -- (0,-5);
        \node[anchor = east] at (0,0) {$N$};
        \node[anchor = east] at (0,2.5) {$0.5$};
        \node[anchor = east] at (0,-2.5) {$0.5$};
        \node[anchor = north] at (0,-5) {$\theta_H$};
        \node[anchor = south] at (0,5) {$\theta_L$};
        \draw[dashed] (-5,-5) -- (-5,5);
        \draw[dashed] (5,-5) -- (5,5);
        \node[anchor = west] at (5,0) {$2$};
        \node[anchor = east] at (-5,0) {$2$};
        \node[anchor = south] at (2.5,5) {$p_H$};
        \node[anchor = south] at (-2.5,5) {$p_L$};
        \node[anchor = north] at (2.5,-5) {$p_H$};
        \node[anchor = north] at (-2.5,-5) {$p_L$};
        \node[anchor = west] at (6,6){$8,-2$};
        \node[anchor = west] at (6,4){$8+8\delta,0$};
        \node[anchor = west] at (6,-4){$5,1$};
        \node[anchor = west] at (6,-6){$5+5\delta,0$};
        \node[anchor = east] at (-6,-6){$1+5\delta,0$};
        \node[anchor = east] at (-6,-4){$1,1$};
        \node[anchor = east] at (-6,4){$6+8\delta,0$};
        \node[anchor = east] at (-6,6){$6,-2$};
      \end{tikzpicture}
    \end{center}
    \end{problem}
    \begin{problem}{(b)}
      For $\delta = 1$ find a pooling perfect Bayesian equilibrium of the game in which both types of player $1$ choose $p_L$ in period $t=1$.
      %prevent a deviation after finding best responses
    \end{problem}
  \end{problem}
\end{document}
