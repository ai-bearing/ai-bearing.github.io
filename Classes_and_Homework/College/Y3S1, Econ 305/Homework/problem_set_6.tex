\documentclass[10pt]{extarticle}
\title{}
\author{}
\date{}
\usepackage[shortlabels]{enumitem}


%paper setup
\usepackage{geometry}
\geometry{letterpaper, portrait, margin=1in}
\usepackage{fancyhdr}
% sans serif font:
\usepackage{cmbright}
%symbols
\usepackage{amsmath}
\usepackage{bigints}
\usepackage{amssymb}
\usepackage{amsthm}
\usepackage{mathtools}
\usepackage{bbm}
\usepackage[hidelinks]{hyperref}
\usepackage{gensymb}
\usepackage{multirow,array}
\usepackage{multicol}

\newtheorem*{remark}{Remark}
\usepackage[T1]{fontenc}
\usepackage[utf8]{inputenc}

%chemistry stuff
%\usepackage[version=4]{mhchem}
%\usepackage{chemfig}

%plotting
\usepackage{pgfplots}
\usepackage{tikz}
\tikzset{middleweight/.style={pos = 0.5}}
%\tikzset{weight/.style={pos = 0.5, fill = white}}
%\tikzset{lateweight/.style={pos = 0.75, fill = white}}
%\tikzset{earlyweight/.style={pos = 0.25, fill=white}}

%\usepackage{natbib}

%graphics stuff
\usepackage{graphicx}
\graphicspath{ {./images/} }
\usepackage[style=numeric, backend=biber]{biblatex} % Use the numeric style for Vancouver
\addbibresource{the_bibliography.bib}
%code stuff
%when using minted, make sure to add the -shell-escape flag
%you can use lstlisting if you don't want to use minted
%\usepackage{minted}
%\usemintedstyle{pastie}
%\newminted[javacode]{java}{frame=lines,framesep=2mm,linenos=true,fontsize=\footnotesize,tabsize=3,autogobble,}
%\newminted[cppcode]{cpp}{frame=lines,framesep=2mm,linenos=true,fontsize=\footnotesize,tabsize=3,autogobble,}

%\usepackage{listings}
%\usepackage{color}
%\definecolor{dkgreen}{rgb}{0,0.6,0}
%\definecolor{gray}{rgb}{0.5,0.5,0.5}
%\definecolor{mauve}{rgb}{0.58,0,0.82}
%
%\lstset{frame=tb,
%	language=Java,
%	aboveskip=3mm,
%	belowskip=3mm,
%	showstringspaces=false,
%	columns=flexible,
%	basicstyle={\small\ttfamily},
%	numbers=none,
%	numberstyle=\tiny\color{gray},
%	keywordstyle=\color{blue},
%	commentstyle=\color{dkgreen},
%	stringstyle=\color{mauve},
%	breaklines=true,
%	breakatwhitespace=true,
%	tabsize=3
%}
% text + color boxes
\renewcommand{\mathbf}[1]{\mathbbm{#1}}
\usepackage[most]{tcolorbox}
\tcbuselibrary{breakable}
\tcbuselibrary{skins}
\newtcolorbox{problem}[1]{colback=white,enhanced,title={\small #1},
          attach boxed title to top center=
{yshift=-\tcboxedtitleheight/2},
boxed title style={size=small,colback=black!60!white}, sharp corners, breakable}
%including PDFs
%\usepackage{pdfpages}
\setlength{\parindent}{0pt}
\usepackage{cancel}
\pagestyle{fancy}
\fancyhf{}
\rhead{Avinash Iyer}
\lhead{Econ 305: Problem Set 6}
\newcommand{\card}{\text{card}}
\newcommand{\ran}{\text{ran}}
\newcommand{\N}{\mathbbm{N}}
\newcommand{\Q}{\mathbbm{Q}}
\newcommand{\Z}{\mathbbm{Z}}
\newcommand{\R}{\mathbbm{R}}
\setcounter{secnumdepth}{0}
\begin{document}
\section{More Limit Pricing}%
  An incumbent firm (player 1) is either a low-cost type $\theta_1 = \theta_L$ or a high cost type $\theta_1 = \theta_H$, each with equal probability. In period $t=1$, the incumbent is a monopolist and sets one of two prices, $p_L$ or $p_H$, and its profits in this period depend on its type and the price it chooses, given by the following table:
  \begin{center}
    \begin{tabular}{ccc}
      Type & Profit from $p_L$ & Profit from $p_H$\\
      \hline
      $\theta_L$ & $6$ & $8$\\
      $\theta_H$ & $1$ & $5$
    \end{tabular}
  \end{center}
  After observing the period $t=1$ price, a potential entrant (player 2), which does not know the incumbent's type but knows the distribution of types, can choose to either enter the market ($E$), or stay out ($O$) in period $t=2$. The payoffs of both players in period $2$ depend on the entrant's choice and on the incumbent's type and are given by the following table:
  \begin{center}
    \begin{tabular}{cccc}
      Incumbent's type & Entrant's choice & Incumbent's payoff & Entrant's payoff\\
      \hline
      $\theta_L$ & $E$ & $0$ & $-2$\\
      $\theta_L$ & $O$ & $8$ & $0$\\
      $\theta_H$ & $E$ & $0$ & $1$\\
      $\theta_H$ & $O$ & $5$ & $0$
    \end{tabular}
  \end{center}
  At the beginning of the game the incumbent discounts profits for period $t=2$ using a discount factor $\delta \leq 1$.
  \subsection{(a)}%
  \begin{center}
    \small
    \begin{tikzpicture}[scale=0.75]
      \draw (-5,5) -- (5,5);
      \draw (-5,-5) -- (5,-5);
      \draw (6,6) --node[pos=0.5,anchor=south]{$E$} (5,5);
      \draw (6,4) -- node[pos=0.5,anchor=north]{$O$}(5,5);
      \draw (-6,6) -- node[pos=0.5,anchor=south]{$E$}(-5,5);
      \draw (-6,4) -- node[pos=0.5,anchor=north]{$O$}(-5,5);
      \draw (-6,-4) -- node[anchor=south,pos=0.5]{$E$}(-5,-5);
      \draw (-6,-6) -- node[anchor=north,pos=0.5]{$O$}(-5,-5);
      \draw (6,-4) -- node[anchor=south,pos=0.5]{$E$}(5,-5);
      \draw (6,-6) -- node[anchor=north,pos=0.5]{$O$}(5,-5);
      \filldraw (-5,5) circle (2pt)
        (-5,-5) circle (2pt)
        (5,5) circle (2pt)
        (5,-5) circle (2pt)
        (0,5) circle (2pt)
        (0,-5) circle (2pt)
        (0,0) circle (2pt);
      \draw[thick] (0,5) -- (0,-5);
      \node[anchor = east] at (0,0) {$N$};
      \node[anchor = east] at (0,2.5) {$0.5$};
      \node[anchor = east] at (0,-2.5) {$0.5$};
      \node[anchor = north] at (0,-5) {$\theta_H$};
      \node[anchor = south] at (0,5) {$\theta_L$};
      \draw[dashed] (-5,-5) -- (-5,5);
      \draw[dashed] (5,-5) -- (5,5);
      \node[anchor = west] at (5,0) {$2$};
      \node[anchor = east] at (-5,0) {$2$};
      \node[anchor = south] at (2.5,5) {$p_H$};
      \node[anchor = south] at (-2.5,5) {$p_L$};
      \node[anchor = north] at (2.5,-5) {$p_H$};
      \node[anchor = north] at (-2.5,-5) {$p_L$};
      \node[anchor = west] at (6,6){$8,-2$};
      \node[anchor = west] at (6,4){$8+8\delta,0$};
      \node[anchor = west] at (6,-4){$5,1$};
      \node[anchor = west] at (6,-6){$5+5\delta,0$};
      \node[anchor = east] at (-6,-6){$1+5\delta,0$};
      \node[anchor = east] at (-6,-4){$1,1$};
      \node[anchor = east] at (-6,4){$6+8\delta,0$};
      \node[anchor = east] at (-6,6){$6,-2$};
    \end{tikzpicture}
  \end{center}
  \subsection{(b)}%
  \textbf{Problem:} For $\delta = 1$ find a pooling perfect Bayesian equilibrium of the game in which both types of player $1$ choose $p_L$ in period $t=1$.
  \begin{center}
  \rule{\textwidth}{0.4pt}
  \end{center}
    \begin{description}
      \item[Strategy for Player 1:]
        \begin{align*}
          s_1^{\ast}(\theta)&= p_L \tag*{$\forall \theta$}
        \end{align*}
      \item[Belief System for Player 2:]
        \begin{align*}
          \mu_{2}(\theta_L | p_L) &= \frac{1}{2}\\
          \mu_2(\theta_H|p_L) &= \frac{1}{2}\\
          \mu_2(\theta_L|p_H) &= \lambda\\
          \mu_2(\theta_H|p_H) &= 1-\lambda
        \end{align*}
      \item[Best Responses for Player 2:]
        \begin{align*}
          \shortintertext{After seeing $p_L$:}
          Ev_2(E,p_L;\theta) &= \mu_{2}(\theta_L | p_L)v_2(E,p_L;\theta_L) + \mu_{2}(\theta_H | p_L)v_2(E,p_L;\theta_H)\\
                             &=-\frac{1}{2}\\
           Ev_2(O,p_L;\theta) &= 0\\
           \shortintertext{After seeing $p_H$:}
           Ev_2(E,p_H;\theta) &= \mu_{2}(\theta_L | p_H)v_2(E,p_H;\theta_L) + \mu_{2}(\theta_H | p_H)v_2(E,p_H;\theta_H)\\
                              &= 1-3\lambda\\
           Ev_2(O,p_H;\theta) &= 0\\
           \shortintertext{Condition to play entry upon $p_H$:}
           1-3\lambda &> 0\\
           \lambda &< \frac{1}{3}.\\
           \shortintertext{So,}
           s_2^{\ast}(a_1) &= \begin{cases}
             O,&a_1 = p_L,\lambda\in[0,1]\\
             E,&a_1 = p_H,\lambda < 1/3\\
             O,&a_1 = p_H,\lambda > 1/3\\
             qE + (1-q)O,&a_1 = p_H,\lambda = 1/3
           \end{cases}
        \end{align*}
      \item[Best Responses for Player 1:]\hfill
        \begin{description}
          \item[Follow:]
            \begin{align*}
              v_1(p_L,s_2^{\ast}(a_1);\theta_L) &= 14\\
              v_1(p_L,s_2^{\ast}(a_1);\theta_H) &= 6
            \end{align*}
          \item[Deviate:] 
            \begin{align*}
              v_1(p_H,s_2^{\ast}(a_1);\theta_L) &= 8\\
              v_1(p_H,s_2^{\ast}(a_1);\theta_H) &= 5
            \end{align*}
        \end{description}
    \end{description}
    \subsection{(c)}%
    \textbf{Problem:} Find the range of discount factors for which a separating perfect Bayesian equilibrium exists in which type $\theta_L$ chooses $p_L$ and type $\theta_H$ chooses $p_H$ in period $t=1$.
    \begin{center}
    \rule{\textwidth}{0.4pt}
    \end{center}
    \begin{description}
      \item[Strategy for Player 1:]
        \begin{align*}
          s_1^{\ast}(\theta) &= \begin{cases}
            p_L,&\theta = \theta_L\\
            p_H,&\theta=\theta_H
          \end{cases}
        \end{align*}
      \item[Best Response for Player 2:]
        \begin{align*}
          s_2^{\ast}(a_1) &= \begin{cases}
            E,&a_1 = p_H\\
            O,&a_1=p_L
          \end{cases}
        \end{align*}
      \item[Player 1 Equilibrium Conditions:]\hfill
        \begin{itemize}
          \item[Low Cost:]
            \begin{align*}
              v_1(p_L,s_2^{\ast}(a_1);\theta_L) &= 6 + 8\delta\\
              v_1(p_H,s_2^{\ast}(a_1);\theta_L) &= 8\\
              6 + 8\delta &\geq 8\\
              \delta &\geq \frac{3}{4}
            \end{align*}
          \item[High Cost:]
            \begin{align*}
              v_1(p_H,s_2^{\ast}(a_1);\theta_H) &= 5\\
              v_1(p_L,s_2^{\ast}(a_1);\theta_L) &= 1 + 5\delta\\
              1 + 5\delta &\leq 5\\
              \delta &\leq \frac{4}{5}
            \end{align*}
        \end{itemize}
    \end{description}
    Therefore, $\delta \in [3/4,4/5]$.
    \pagebreak
    \section{Third Type's a Charm}%
    The following three-type signaling game begins with a move by Nature (not shown in the tree) that yields one of three types with equal probability.
    \begin{center}
      \small
      \begin{tikzpicture}[scale=0.75]
        \draw (-5,5) -- (5,5);
        \draw (-5,0) -- (5,0);
        \draw (-5,-5) -- (5,-5);
        \filldraw (0,0) circle (2pt)
          (0,5) circle (2pt)
          (0,-5) circle (2pt)
          (-5,5) circle (2pt)
          (5,5) circle (2pt)
          (5,0) circle (2pt)
          (-5,0) circle (2pt)
          (-5,-5) circle (2pt)
          (5,-5) circle (2pt);
        \draw[dashed] (-5,5) -- (-5,-5);
        \draw[dashed] (5,5) -- (5,-5);
        \draw (-5,5) -- node[pos=0.5,anchor=south]{$u$}(-6,6);
        \draw (-5,5) -- node[pos=0.5,anchor=north]{$d$}(-6,4);
        \draw (-5,0) -- node[pos = 0.5,anchor=south]{$u$}(-6,1);
        \draw (-5,0) -- node[pos = 0.5,anchor=north]{$d$}(-6,-1);
        \draw (-5,-5) -- node[pos=0.5,anchor=north]{$d$}(-6,-6);
        \draw (-5,-5) -- node[pos=0.5,anchor=south]{$u$}(-6,-4);
        \draw (5,5) -- node[pos=0.5,anchor=south]{$u$}(6,6);
        \draw (5,5) -- node[pos=0.5,anchor=north]{$d$}(6,4);
        \draw (5,0) -- node[pos = 0.5,anchor=south]{$u$}(6,1);
        \draw (5,0) -- node[pos = 0.5,anchor=north]{$d$}(6,-1);
        \draw (5,-5) -- node[pos=0.5,anchor=north]{$d$}(6,-6);
        \draw (5,-5) -- node[pos=0.5,anchor=south]{$u$}(6,-4);
        \node[anchor = south] at (0,5) {$t_1$};
        \node[anchor = south] at (0,0) {$t_2$};
        \node[anchor = south] at (0,-5) {$t_3$};
        \node[anchor = north] at (0,5) {$1/3$};
        \node[anchor = north] at (0,0) {$1/3$};
        \node[anchor = north] at (0,-5) {$1/3$};
        \node[anchor = south] at (-2.5,5) {$L$};
        \node[anchor = south] at (-2.5,0) {$L$};
        \node[anchor = south] at (-2.5,-5) {$L$};
        \node[anchor = south] at (2.5,5) {$R$};
        \node[anchor = south] at (2.5,0) {$R$};
        \node[anchor = south] at (2.5,-5) {$R$};
        \node[anchor = east] at (-6,6) {$1,1$};
        \node[anchor = east] at (-6,4) {$1,0$};
        \node[anchor = east] at (-6,1) {$2,1$};
        \node[anchor = east] at (-6,-1) {$0,1$};
        \node[anchor = east] at (-6,-4) {$1,1$};
        \node[anchor = east] at (-6,-6) {$0,0$};
        \node[anchor = west] at (6,6) {$0,1$};
        \node[anchor = west] at (6,4) {$0,0$};
        \node[anchor = west] at (6,1) {$1,1$};
        \node[anchor = west] at (6,-1) {$1,0$};
        \node[anchor = west] at (6,-4) {$0,0$};
        \node[anchor = west] at (6,-6) {$2,1$};
      \end{tikzpicture}
    \end{center}
    \subsection{(a)}%
    \textbf{Problem:} Find the pure strategy pooling perfect Bayesian equilibrium.
    \begin{center}
    \rule{\textwidth}{0.4pt}
    \end{center}
    \begin{description}
      \item[Strategy for Player 1:]
        \begin{align*}
          s_1^{\ast}(\theta) &= L \tag*{$\forall \theta$}
        \end{align*}
      \item[Belief System for Player 2:]
        \begin{align*}
          \mu_2(t_1|L) &= \frac{1}{3}\\
          \mu_2(t_2|L) &= \frac{1}{3}\\
          \mu_2(t_3|L) &= \frac{1}{3}\\
          \mu_2(t_1|R) &= \lambda_1\\
          \mu_2(t_2|R) &= \lambda_2\\
          \mu_2(t_3|R) &= \lambda_3
        \end{align*}
      \item[Best Response for Player 2:]
        \begin{align*}
          Ev_2(u,L;\theta) &= \sum_{i} \mu_2(t_i|L)v_2(u,L;t_i)\\
                           &= 1\\
          Ev_2(d,L;\theta) &= 0\\
          Ev_2(u,R;\theta) &= \sum_{i}\mu_2(t_i|R)v_2(u,R;t_i)\\
                           &= \frac{\lambda_1 + \lambda_2}{3}\\
                           &= 1-\frac{\lambda_3}{3}\\
          Ev_2(d,R;\theta) &= \sum_{i}\mu_2(t_i|R)v_2(d,R;t_i)\\
                           &= \frac{\lambda_3}{3}\\
                           \shortintertext{Therefore,}
          s_2^{\ast}(a_1) &= u \tag*{$\forall \lambda_1,\lambda_2,\lambda_3$}
        \end{align*}
      \item[Best Responses for Player 1:]\hfill
        \begin{description}
          \item[Follow:]
            \begin{align*}
              v_1(L,s_2^{\ast}(a_1);t_1) &= 1\\
              v_1(L,s_2^{\ast}(a_1);t_2) &= 2\\
              v_1(L,s_2^{\ast}(a_1);t_3) &= 1\\
            \end{align*}
          \item[Deviate:]
            \begin{align*}
              v_1(R,s_2^{\ast}(a_1);t_1) &= 0\\
              v_1(R,s_2^{\ast}(a_1);t_2) &= 1\\
              v_1(R,s_2^{\ast}(a_1);t_3) &= 0\\
            \end{align*}
        \end{description}
    \end{description}
    Therefore, $(s_1^{\ast}(\theta),s_2^{\ast}(a_1), \mu_2(\theta|a_1))$ is a pooling perfect Bayesian equilibrium.
    \subsection{(b)}%
    \textbf{Problem:} Find the pure strategy separating perfect Bayesian equilibrium.
    \begin{center}
    \rule{\textwidth}{0.4pt}
    \end{center}
    \begin{description}
      \item[Strategy for Player 1:]
        \begin{align*}
          s_1^{\ast}(\theta) &= \begin{cases}
            L, & \theta = t_1,t_2\\
            R, & \theta = t_3
          \end{cases}
        \end{align*}
      \item[Belief System for Player 2:]
        \begin{align*}
          \mu_2(t_1|L) &= \frac{1}{2}\\
          \mu_2(t_2|L) &= \frac{1}{2}\\
          \mu_2(t_3|L) &= 0\\
          \mu_2(t_1|R) &= 0\\
          \mu_2(t_2|R) &= 0\\
          \mu_2(t_3|R) &= 1
        \end{align*}
      \item[Best Response for Player 2:]
        \begin{align*}
          Ev_2(u,L;\theta) &= \sum_{i}\mu_2(t_i|L)v_2(u,L;t_i)\\
                           &= 1\\
          Ev_2(d,L;\theta) &= \sum_{i}\mu_2(t_i|L)v_2(d,L;t_i)\\
                           &= 0\\
          Ev_2(u,R;\theta) &= \sum_{i} \mu_2(t_i|R)v_2(u,R;t_i)\\
                           &= 2\\
          Ev_2(d,R;\theta) &= \sum_{i}\mu_2(t_i|R) v_2(d,R;t_i)\\
                           &= 1\\
                           \shortintertext{Therefore,}
          s_2^{\ast}(a_1) &= \begin{cases}
            u,&a_1 = L\\
            d,&a_1 = R
          \end{cases}
        \end{align*}
      \item[Best Responses for Player 1:]\hfill
        \begin{description}
          \item[Follow:]
            \begin{align*}
              v_1(L,s_2^{\ast}(a_1);t_1) &= 1\\
              v_1(L,s_2^{\ast}(a_1);t_2) &= 2\\
              v_1(R,s_2^{\ast}(a_1);t_3) &= 2
            \end{align*}
          \item[Deviate:]
            \begin{align*}
              v_1(R,s_2^{\ast}(a_1);t_1) &= 0\\
              v_1(R,s_2^{\ast}(a_1);t_2) &= 1\\
              v_1(L,s_2^{\ast}(a_1);t_3) &= 1
            \end{align*}
        \end{description}
    \end{description}
    Therefore, $(s_1^{\ast}(\theta),s_2^{\ast}(a_1),\mu_2(\theta|a_1))$ is a separating perfect Bayesian equilibrium.
    \pagebreak
    \section{Semi-Separating Signals}%
    \textbf{Problem:} Find a semi-separating PBE in the below signaling game.
    \begin{center}
      \small
      \begin{tikzpicture}[scale=0.75]
        \draw (-5,5) -- (5,5);
        \draw (-5,-5) -- (5,-5);
        \draw (6,6) --node[pos=0.5,anchor=south]{$u$} (5,5);
        \draw (6,4) -- node[pos=0.5,anchor=north]{$d$}(5,5);
        \draw (-6,6) -- node[pos=0.5,anchor=south]{$u$}(-5,5);
        \draw (-6,4) -- node[pos=0.5,anchor=north]{$d$}(-5,5);
        \draw (-6,-4) -- node[anchor=south,pos=0.5]{$u$}(-5,-5);
        \draw (-6,-6) -- node[anchor=north,pos=0.5]{$d$}(-5,-5);
        \draw (6,-4) -- node[anchor=south,pos=0.5]{$u$}(5,-5);
        \draw (6,-6) -- node[anchor=north,pos=0.5]{$d$}(5,-5);
        \filldraw (-5,5) circle (2pt)
          (-5,-5) circle (2pt)
          (5,5) circle (2pt)
          (5,-5) circle (2pt)
          (0,5) circle (2pt)
          (0,-5) circle (2pt)
          (0,0) circle (2pt);
        \draw[thick] (0,5) -- (0,-5);
        \node[anchor = east] at (0,0) {$N$};
        \node[anchor = east] at (0,2.5) {$0.5$};
        \node[anchor = east] at (0,-2.5) {$0.5$};
        \node[anchor = north] at (0,-5) {$t_2$};
        \node[anchor = south] at (0,5) {$t_1$};
        \draw[dashed] (-5,-5) -- (-5,5);
        \draw[dashed] (5,-5) -- (5,5);
        \node[anchor = west] at (5,0) {$2$};
        \node[anchor = east] at (-5,0) {$2$};
        \node[anchor = south] at (2.5,5) {$R$};
        \node[anchor = south] at (-2.5,5) {$L$};
        \node[anchor = north] at (2.5,-5) {$R$};
        \node[anchor = north] at (-2.5,-5) {$L$};
        \node[anchor = west] at (6,6){$2,2$};
        \node[anchor = west] at (6,4){$0,0$};
        \node[anchor = west] at (6,-4){$1,0$};
        \node[anchor = west] at (6,-6){$1,1$};
        \node[anchor = east] at (-6,-6){$0,1$};
        \node[anchor = east] at (-6,-4){$0,0$};
        \node[anchor = east] at (-6,4){$2,0$};
        \node[anchor = east] at (-6,6){$1,1$};
      \end{tikzpicture}
    \end{center}
    \begin{center}
    \rule{\textwidth}{0.4pt}
    \end{center}
    \begin{description}
      \item[Strategy for Player 1:]
        \begin{align*}
          s_1^{\ast}(\theta) &= \begin{cases}
            qR + (1-q)L,& \theta = t_1\\
            R,&\theta = t_2
          \end{cases}
        \end{align*}
      \item[Belief System for Player 2:]
        \begin{align*}
          \mu_2(t_1|L) &= 1\\
          \mu_2(t_2|L) &= 0\\
          \mu_2(t_1|R) &= \frac{q}{1+q}\\
          \mu_2(t_2|R) &= \frac{1}{1+q}
        \end{align*}
      \item[Best Response for Player 2:]
        \begin{align*}
          Ev_2(u,L;\theta) &= \sum_{i}\mu_2(t_i|L)v_2(u,L;t_i)\\
                           &= 1\\
          Ev_2(d,L;\theta) &= \sum_{i}\mu_2(t_i|L)v_2(d,L;t_i)\\
                           &= 0\\
          Ev_2(u,R;\theta) &= \sum_{i}\mu_2(t_i|R)v_2(u,R;t_i) \\
                           &= \frac{2q}{1+q}\\
          Ev_2(d,R;\theta) &= \sum_{i}\mu_2(t_i|R)v_2(d,R;t_i)\\
                           &= \frac{1}{1+q}\\
          \shortintertext{Indifference Condition:}
          \frac{2q}{1+q} &= \frac{1}{1+q}\\
          q &= \frac{1}{2}\\
          \shortintertext{therefore,}
          s_2^{\ast}(a_1) &= \begin{cases}
            u,&a_1 = L\\
            u,&a_1=R,q > \frac{1}{2}\\
            d,&a_1 = R,q < \frac{1}{2}\\
            pu + (1-p)d,&a_1 = R,q = \frac{1}{2}
          \end{cases}
        \end{align*}
      \item[Best Response for Player 1:]
        \begin{align*}
          v_1(L,s_2^{\ast}(a_1);t_1) &= 1\\
          v_1(R,s_2^{\ast}(a_1);t_1) &= 2\\
          \shortintertext{Indifference Condition:}
          q(v_1(R,s_2^{\ast}(a_1);t_1)) &= (1-q)(v_1(L,s_2^{\ast}(a_1);t_1))\\
          2q &= 1-q\\
          q &= \frac{1}{3}
        \end{align*}
        Type $t_2$ will always play $R$ as $R$ dominates $L$ for $t_2$.
    \end{description}
    Therefore, $(s_1^{\ast}(\theta),s_2^{\ast}(a_1),\mu_2(\theta|a_1))$ is a PBE.
  \end{document}
