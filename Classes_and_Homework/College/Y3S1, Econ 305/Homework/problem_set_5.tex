\documentclass[9pt]{extarticle}
\title{}
\author{}
\date{}
\usepackage[shortlabels]{enumitem}


%paper setup
\usepackage{geometry}
\geometry{letterpaper, portrait, margin=1in}
\usepackage{fancyhdr}
% sans serif font:
\usepackage{cmbright}
%symbols
\usepackage{amsmath}
\usepackage{bigints}
\usepackage{amssymb}
\usepackage{amsthm}
\usepackage{mathtools}
\usepackage{bbm}
\usepackage[hidelinks]{hyperref}
\usepackage{gensymb}
\usepackage{multirow,array}
\usepackage{multicol}

\newtheorem*{remark}{Remark}
\usepackage[T1]{fontenc}
\usepackage[utf8]{inputenc}

%chemistry stuff
%\usepackage[version=4]{mhchem}
%\usepackage{chemfig}

%plotting
\usepackage{pgfplots}
\usepackage{tikz}
\tikzset{middleweight/.style={pos = 0.5}}
%\tikzset{weight/.style={pos = 0.5, fill = white}}
%\tikzset{lateweight/.style={pos = 0.75, fill = white}}
%\tikzset{earlyweight/.style={pos = 0.25, fill=white}}

%\usepackage{natbib}

%graphics stuff
\usepackage{graphicx}
\graphicspath{ {./images/} }
\usepackage[style=numeric, backend=biber]{biblatex} % Use the numeric style for Vancouver
\addbibresource{the_bibliography.bib}
%code stuff
%when using minted, make sure to add the -shell-escape flag
%you can use lstlisting if you don't want to use minted
%\usepackage{minted}
%\usemintedstyle{pastie}
%\newminted[javacode]{java}{frame=lines,framesep=2mm,linenos=true,fontsize=\footnotesize,tabsize=3,autogobble,}
%\newminted[cppcode]{cpp}{frame=lines,framesep=2mm,linenos=true,fontsize=\footnotesize,tabsize=3,autogobble,}

%\usepackage{listings}
%\usepackage{color}
%\definecolor{dkgreen}{rgb}{0,0.6,0}
%\definecolor{gray}{rgb}{0.5,0.5,0.5}
%\definecolor{mauve}{rgb}{0.58,0,0.82}
%
%\lstset{frame=tb,
%	language=Java,
%	aboveskip=3mm,
%	belowskip=3mm,
%	showstringspaces=false,
%	columns=flexible,
%	basicstyle={\small\ttfamily},
%	numbers=none,
%	numberstyle=\tiny\color{gray},
%	keywordstyle=\color{blue},
%	commentstyle=\color{dkgreen},
%	stringstyle=\color{mauve},
%	breaklines=true,
%	breakatwhitespace=true,
%	tabsize=3
%}
% text + color boxes
\renewcommand{\mathbf}[1]{\mathbbm{#1}}
\usepackage[most]{tcolorbox}
\tcbuselibrary{breakable}
\tcbuselibrary{skins}
\newtcolorbox{problem}[1]{colback=white,enhanced,title={\small #1},
          attach boxed title to top center=
{yshift=-\tcboxedtitleheight/2},
boxed title style={size=small,colback=black!60!white}, sharp corners, breakable}
%including PDFs
%\usepackage{pdfpages}
\setlength{\parindent}{0pt}
\usepackage{cancel}
\pagestyle{fancy}
\fancyhf{}
\rhead{Avinash Iyer}
\lhead{Econ 305: Problem Set 5}
\newcommand{\card}{\text{card}}
\newcommand{\ran}{\text{ran}}
\newcommand{\N}{\mathbbm{N}}
\newcommand{\Q}{\mathbbm{Q}}
\newcommand{\Z}{\mathbbm{Z}}
\newcommand{\R}{\mathbbm{R}}
\begin{document}
  \begin{problem}{Conquering an Island}
    Two opposed armies are ready to conquer an island. Each army's general can choose to either attack $(A)$ not attack $(N)$. In addition, each army is either strong or weak, and there is common knowledge that these two events are equally likely and independent. The island is worth $M$ if captured and it is captured either by attacking when its opponent is not or by attacking when its rival does if its own army is strong and the rival is weak. If two armies of equal strength both attack, neither captures the island. An army has a cost of fighting, $s$ if strong and $w > s$ if weak ($M > 2w$). There is no cost of attacking if the rival does not. Find all of the pure strategy Bayesian Nash equilibria.
    \tcblower
    Without loss of generality, we will analyze Player $1$'s payoffs.
    \begin{description}
      \item[Strong Army Strategies:]\hfill
        \begin{description}
          \item[Not Attack:]
            \begin{align*}
              E\left(v_1(N,s_2^{\ast}(\theta_2);S,\theta_2)\right) &= 0
            \end{align*}
          \item[Attack, Opponent Attacks:]
            \begin{align*}
              E\left(v_1(A,A;S,\theta_2)\right) &= P(\theta_2 = S)v_1(A,A,;S,\theta_2) + P(\theta_2 = W)v_1(A,A;S,\theta_2)\\
                                                &= \frac{1}{2}(-s) + \frac{1}{2}(M-s)\\
                                                &= \frac{1}{2}M - s\\
                                                &> 0
            \end{align*}
          \item[Attack, Opponent Does Not Attack:]
            \begin{align*}
              E\left(v_1(A,N;S,\theta_2)\right) &= M\\
                                                &> 0
            \end{align*}
        \end{description}
        Therefore, a strong army's dominant strategy is to Attack.
      \item[Weak Army Strategies:]
        \begin{description}
          \item[Not Attack:]
            \begin{align*}
              E\left(v_1(N,s_2^{\ast}(\theta_2);W,\theta_2)\right) &= 0
            \end{align*}
          \item[Attack, Opponent Attacks:]
            \begin{align*}
              E\left(v_1(A,A;W,\theta_2)\right) &= -w\\
                                                &< 0\\
                                                \shortintertext{meaning}
              BR_i(A;W,\theta_2) &= N
            \end{align*}
          \item[Attack, Opponent Does Not Attack:]
            If opponent does not attack, this means opponent must be of type $W$; a player of type $S$ always attacks.
            \begin{align*}
              E\left(v_1(A,N;W,W)\right) &= M\\
                                         &> 0\\
                                         \shortintertext{meaning}
              BR_i(N; W,W) &= A
            \end{align*}
        \end{description}
    \end{description}
    Since the best response functions are symmetric with respect to each other, this means the pure strategy Bayesian Nash equilibria are:
    \begin{itemize}
      \item $((A,A),(A,N))$
      \item $((A,N),(A,A))$
    \end{itemize}
    % Iterated dominance: find dominant strategy for particular type, then find expected payoffs for other type given dominant strategy.
  \end{problem}
  \begin{problem}{Matching Pennies, Revisited}
    Consider the modified version of Matching Pennies shown below. The players derive extra utility from playing Heads regardless of whether they win or lose. Player $1$ gets extra utility $a$ from playing Heads and player $2$ gets extra utility $b$ from playing heads. Assume that $a$ and $b$ are private information independently drawn from the uniform distribution on $[0,1]$.
    \begin{center}
      \renewcommand{\arraystretch}{1.5}
      \begin{tabular}{c|c|c|}
        \multicolumn{1}{c}{} & \multicolumn{1}{c}{$H$} & \multicolumn{1}{c}{$T$}\\
        \cline{2-3}
        $H$ & $1 + a, -1 + b$ & $-1 + a,1$\\
        \cline{2-3}
        $T$ & $-1,1+b$ & $1,-1$\\
        \cline{2-3}
      \end{tabular}
    \end{center}
    \tcblower
    \begin{problem}{(a)}
      Find a Bayesian Nash equilibrium of this game.
      \tcblower
      \begin{description}
        \item[Cutoff Strategies:]
          \begin{align*}
            s_1^{\ast}(a) &= \begin{cases}
              H & a\geq a^{\ast}\\
              T & a < a^{\ast}
            \end{cases}\\
              s_2^{\ast}(b) &= \begin{cases}
                H & b\geq b^{\ast}\\
                T & b < b^{\ast}
              \end{cases}
          \end{align*}
        \item[Indifference Condition for Player 1:]
          \begin{align*}
            Ev_{1}(H,s_2^{\ast}(\theta_2);\theta_1^{\ast},\theta_2) &= Ev_{1}(T,s_2^{\ast}(\theta_2);\theta_1^{\ast},\theta_2)\\
            (1+a^{\ast})(1-b^{\ast}) + (-1+a^{\ast})(b^{\ast}) &= (-1)(1-b^{\ast}) + (1)(b^{\ast})
          \end{align*}
        \item[Indifference Condition for Player 2:]
          \begin{align*}
            Ev_{2}(s_1^{\ast}(\theta_1),H;\theta_1^{\ast},\theta_2) &= Ev_{1}(s_1^{\ast}(\theta_1),T;\theta_1^{\ast},\theta_2)\\
            (-1+b^{\ast})(1-a^{\ast}) + (1+b^{\ast})(a^{\ast}) &= (1)(1-a^{\ast}) + (-1)(a^{\ast})
          \end{align*}
        \item[Solution to System:]
          \begin{align*}
            1 + a^{\ast} - b^{\ast} - a^{\ast}b^{\ast} + a^{\ast}b^{\ast} - b^{\ast} &= 2b^{\ast} - 1\tag*{Player 1 Indifference Condition}\\
            4b^{\ast} - 2 &= a^{\ast}\\
            a^{\ast} + a^{\ast}b^{\ast} - 1 + a^{\ast} - a^{\ast}b^{\ast} + b^{\ast} &= 1-2a^{\ast}\tag*{Player 2 Indifference Condition}\\
            b^{\ast} &= 1-4a^{\ast}\\
            1-4(4b^{\ast} - 2) &= b^{\ast}\\
            b^{\ast} &= \frac{10}{17}\\
            a^{\ast} &= \frac{6}{17}
          \end{align*}
      \end{description}
      Therefore, the Bayesian Nash equilibrium is:
      \begin{align*}
        s_1^{\ast}(a) &= \begin{cases}
          H & a\geq \frac{6}{17}\\
          T & a < \frac{6}{17}
        \end{cases}\\
          s_2^{\ast}(b) &= \begin{cases}
            H & b\geq \frac{10}{17}\\
            T & b < \frac{10}{17}
          \end{cases}
      \end{align*}
    \end{problem}
    \begin{problem}{(b)}
      The probability that player $1$ plays $T$ in the BNE is $\frac{6}{17}$ and the probability that player $2$ plays $T$ in the BNE is $\frac{10}{17}$.
    \end{problem}
    \begin{problem}{(c)}
      We would expect the probabilities to approach the mixed strategy Nash equilibrium of the regular matching pennies game, which are $\frac{1}{2}$ and $\frac{1}{2}$ respectively.
    \end{problem}
    %Cutoff strategies: guess strategy of player 1 as function of type is action 1 if type is at least one cutoff, other action if type is less than cutoff; type = cutoff is the indifference condition — indifference condition is expected payoff to player 1 of action 1 at  = expected payoff
  \end{problem}
  \begin{problem}{Final Times}
    Consider the following model of Prof. Brandon Lehr and Prof. Andy Jalil choosing dates for the 305 and 251 final exams, respectively. Each exam will either be on Tuesday or on Friday. The exam dates are chosen by Brandon and Andy simultaneously; they send messages of either Tuesday or Friday to the final exam office, which schedules each exam on the requested date.\\

    Brandon and Andy, however, like to have the exam on different days because they share some of the same students. Specifically, they get $1$ util if the exams are on different dates. However, both prefer to have their exam on Tuesday. Brandon gets extra utility $\theta_B$ from holding his exam on Tuesday and Andy gets extra utility $\theta_A$ from holding his exam on Tuesday. Suppose that $\theta_B$ and $\theta_A$ are private information, uniformly distributed on $[0, \overline{\theta}]$.
    \begin{center}
      \renewcommand{\arraystretch}{1.5}
      \begin{tabular}{c|c|c|}
        \multicolumn{1}{c}{} & \multicolumn{1}{c}{Tuesday} & \multicolumn{1}{c}{Friday}\\
        \cline{2-3}
        Tuesday & $\theta_B,\theta_A$ & $1 + \theta_B,1$\\
        \cline{2-3}
        Friday & $1,1+\theta_A$ & $0,0$\\
        \cline{2-3}
      \end{tabular}
    \end{center}
    \tcblower
    \begin{problem}{(a)}
      Find a symmetric pure strategy Bayesian Nash equilibrium of this game.
      \tcblower
      \begin{description}
        \item[Cutoff Strategy:]
          \begin{align*}
            s_{B}^{\ast}(\theta_B) &= \begin{cases}
              T & \theta_{B} \geq \theta_B^{\ast}\\
              F & \theta_{B} < \theta_{B}^{\ast}
            \end{cases}\\
              s_{A}^{\ast}(\theta_A) &= \begin{cases}
                T & \theta_{A} \geq \theta_{A}^{\ast}\\
                F & \theta_{A} < \theta_{A}^{\ast}
              \end{cases}
          \end{align*}
        \item[Indifference Condition for Player B:]
          \begin{align*}
            Ev_A(T,s_{A}^{\ast}(\theta_A);\theta_B^{\ast},\theta_A) &= Ev_A(F,s_{A}^{\ast}(\theta_A);\theta_B^{\ast},\theta_A)\\
            \theta_B^{\ast}\frac{1}{ \overline{\theta} }( \overline{\theta}-\theta_A^{\ast}) + (1+\theta_B^{\ast})\frac{1}{ \overline{\theta} }(\theta_A^{\ast}) &= (1) \frac{1}{ \overline{\theta} }( \overline{\theta}-\theta_A^{\ast})
          \end{align*}
        \item[Indifference Condition for Player A:]
          \begin{align*}
            Ev_A(s_{B}^{\ast}(\theta_{B}),T;\theta_B,\theta_A^{\ast}) &= Ev_A(s_{B}^{\ast}(\theta_{B}),F;\theta_B,\theta_A^{\ast})\\
            \theta_{A}^{\ast}\frac{1}{ \overline{\theta} }( \overline{\theta}-\theta_{B}^{\ast}) + (1+\theta_A^{\ast})\frac{1}{ \overline{\theta}}(\theta_B^{\ast}) &= (1)\frac{1}{\overline{\theta}}(\overline{\theta}-\theta_B^{\ast})
          \end{align*}
        \item[Solution to System:]
          \begin{align*}
            \theta_B^{\ast}\frac{1}{ \overline{\theta} }( \overline{\theta}-\theta_A^{\ast}) + (1+\theta_B^{\ast})\frac{1}{ \overline{\theta} }(\theta_A^{\ast}) &= (1) \frac{1}{ \overline{\theta} }( \overline{\theta}-\theta_A^{\ast}) \tag*{Player A Indifference Condition}\\
            \theta_{A}^{\ast}\frac{1}{ \overline{\theta} }( \overline{\theta}-\theta_{B}^{\ast}) + (1+\theta_A^{\ast})\frac{1}{ \overline{\theta}}(\theta_B^{\ast}) &= (1)\frac{1}{\overline{\theta}}(\overline{\theta}-\theta_B^{\ast})\tag*{Player B Indifference Condition}\\
            \overline{\theta}\theta_B^{\ast} - \theta_{B}^{\ast}\theta_{A}^{\ast} + \theta_{A}^{\ast} + \theta_{A}^{\ast}\theta_{B}^{\ast} &= \overline{\theta} - \theta_{A}^{\ast}\\
            \theta_{B}^{\ast} &= 1-\frac{2\theta_{A}^{\ast}}{ \overline{\theta} }\\
            \theta_{A}^{\ast}\overline{\theta} - \theta_{B}^{\ast}\theta_{A}^{\ast} + \theta_{B}^{\ast} + \theta_{A}^{\ast}\theta_{B}^{\ast} &= \overline{\theta} - \theta_{B}^{\ast}\\
            \theta_{A}^{\ast} &= 1-\frac{2\theta_{B}^{\ast}}{\overline{\theta}}\\
            \theta_{B}^{\ast} &= 1 - \frac{2\left(1 - \frac{2\theta_{B}^{\ast}}{\overline{\theta}}\right)}{\overline{\theta}}\\
            \overline{\theta}\theta_{B}^{\ast} &= \overline{\theta} - 2 + \frac{2\theta_{B}^{\ast}}{\overline{\theta}}\\
            \frac{ \theta_{B}^{\ast}\left(\overline{\theta}^2 - 2\right) }{ \overline{\theta} }&= \overline{\theta} - 2\\
            \theta_{B}^{\ast} &= \frac{\overline{\theta}\left(\overline{\theta} - 2\right)}{\overline{\theta}^2 - 2}\\
            \theta_{A}^{\ast} &= \frac{\overline{\theta}^{2}-2 \overline{\theta} + 4}{\overline{\theta}^{2} - 2}
          \end{align*}
      \end{description}
    \end{problem}
    \begin{problem}{(b)}
      What does Harsanyi's theorem say about the equilibria of this game when $\overline{\theta}$ is small?
      \tcblower
      As $\overline{\theta}$ becomes small, the Bayesian Nash equilibrium of the game approaches the mixed strategy Nash equilibrium.
    \end{problem}
  \end{problem}
  \begin{problem}{Trading Places}
    Two players, $1$ and $2$, each own a house. Each player $i$ values their own house at $v_i$. The value of player $i$'s house to the other player is $\frac{3}{2}v_i$. Each player knows $v_i$ to themselves, but not the value of the other player's house. The values $v_i$ are drawn independently from $[0,1]$ with uniform distribution.
    \tcblower
    \begin{problem}{(a)}
      Suppose players announce simultaneously whether they want to exchange their houses. If both players agree to an exchange, the exchange takes place. Otherwise no exchange takes place. Find a Bayesian Nash equilibrium of this game in pure strategies in which each player $i$ accepts an exchange if and only if the value $v_i$ does not exceed some threshold $\theta_{i}$.
      \tcblower
      The simultaneous exchange game has the structure below, where $E$ denotes ``exchange'' and $D$ denotes ``don't exchange.'
      \begin{center}
        \renewcommand{\arraystretch}{1.65}
        \begin{tabular}{c|c|c|}
          \multicolumn{1}{c}{} & \multicolumn{1}{c}{$E$} & \multicolumn{1}{c}{$D$}\\
          \cline{2-3}
          $E$ & $\frac{3}{2}\theta_2 - \theta_1,\frac{3}{2}\theta_1 - \theta_2$ & $\theta_1,\theta_2$\\
          \cline{2-3}
          $D$ & $\theta_1,\theta_2$ & $\theta_1,\theta_2$\\
          \cline{2-3}
        \end{tabular}
      \end{center}
      \begin{description}
        \item[Cutoff Strategy:]
          \begin{align*}
            s_{i}^{\ast}(v_i) &= \begin{cases}
              D & \theta_i \geq \theta_i^*\\
              E & \theta_i < \theta_i^*
            \end{cases}
          \end{align*}
        \item[Best Responses:] Without loss of generality, we will consider the conditions for the best response of player $1$ to player $2$ playing Exchange.
          \begin{align*}
            v_1(E,E;\theta_1,\theta_2) &\geq v_1(D,E;\theta_1,\theta_2)\\
            \frac{3}{2}\theta_2 - \theta_1 &\geq \theta_1\\
            \theta_1 &\leq \frac{3}{4}\theta_2\\
            \theta_2 &\leq \frac{3}{4}\theta_1
          \end{align*}
        \item[Indifference Conditions:]
          \begin{align*}
            \frac{3}{4}\theta_2\left(\frac{3}{2}\theta_2-\theta_1\right) &= \theta_1 \left(1-\frac{3}{4}\theta_2\right)\\
            \theta_1 &= \frac{9}{8}\theta_2^2\\
            \theta_2 &= \frac{9}{8}\theta_1^2\\
            \theta_2 &= \frac{81}{64}\theta_2^4\\
            0 &= \left(\frac{81}{64}\theta_2^3-1\right)\theta_2\\
            \theta_2^{\ast} &= \sqrt[3]{\frac{64}{81}}\\
            \theta_1^{\ast} &= \sqrt[3]{\frac{64}{81}}
          \end{align*}
      \end{description}
    \end{problem}
    \begin{problem}{(b)}
      How would your answer to (a) change if player $j$'s valuation of player $i$'s house were $\frac{5}{2}v_i$.
      \tcblower
      Inserting a $\frac{5}{2}$ for a $\frac{3}{2}$ in every location, we have
      \begin{align*}
        \frac{5}{2}\theta_2 - \theta_1 &\geq \theta_1\tag*{Best Responses}\\
        \theta_1 &\leq \frac{5}{4}\theta_2\\
        \theta_2 &\leq \frac{5}{4}\theta_1\\
        \frac{5}{4}\theta_2\left(\frac{5}{2}\theta_2 - \theta_1\right) &= \theta_1 \left(1-\frac{5}{4}\theta_2\right)\tag*{Indifference Condition}\\
        \theta_1 &= \frac{25}{8}\theta_2^2\\
        \theta_2 &= \frac{25}{8}\theta_1^2\\
        \theta_2 &= \frac{625}{64}\theta^4\\
        0 &= \left(\frac{625}{64}\theta_2^3 - \theta_2\right)\theta_2\\
        \theta_2 &= \sqrt[3]{\frac{64}{625}}\\
        \theta_1 &= \sqrt[3]{\frac{64}{625}}
      \end{align*}
    \end{problem}
    \begin{problem}{(c)}
      Try to explain why any Bayesian Nash equilibrium of the game described in (a) must involve threshold strategies of the type postulated in (a).
      \tcblower
      There is a minimum threshold such that the best response for being offering an exchange is to accept the exchange --- namely, accepting the exchange needs to yield a net increase in payoff. Therefore, it follows that the best responses would yield a Bayesian Nash equilibrium with cutoff strategies.
    \end{problem}
  \end{problem}
  \begin{problem}{Risk-Averse Bidders}
    %Symmetric Bayesian Nash Equilibrium: $s_i^{\ast}(\theta_i) = k\theta_i + c$.
    %\begin{enumerate}[(1)]
    %  \item Everyone except player $1$ is playing $s_i^{\ast}(\theta_i) = k\theta_i + c$
    %  \item Find $Ev_{1}(b_1,s_{-1}(\theta_{-1});\theta_1,\theta_{-1})$
    %  \item Find $b_1^{\ast}$ such that $\frac{\partial Ev_1}{\partial b_1} = 0$
    %  \item In BNE, $b_1^{\ast} = k\theta_1 + c$
    %\end{enumerate}
    Imagine an auction for a single item with $n$ bidders whose valuations $\theta_i$ are drawn uniformly from $[0,1]$. Assume that the players are risk averse and that the utility to player $i$ of type $\theta_i$ from winning the item at price $p$ is given by $\sqrt{\theta_i-p}$ for $p \leq \theta_i$.
    \tcblower
    \begin{problem}{(a)}
      Write down a bidder's expected payoff function for the first and second price sealed-bid auctions.
      \tcblower
      \begin{description}
        \item[First Price Auction:] Assume that $s_j(\theta_j) = k\theta_j$ for some $k > 0$. Then,
          \begin{align*}
            E\left(v_i(b_i,b_{-i};\theta_i,\theta_{-i})\right) &= \sqrt{\theta_i - b_i}P(b_i > k\theta_{-i})\\
                                                               &= \sqrt{\theta_i - b_i}\left(\frac{b_i}{k}\right)^{n-1}
          \end{align*}
      \end{description}
    \end{problem}
    \begin{problem}{(b)}
      Show that in the first price sealed-bid auction each bidder of type $\theta_i$ choosing the bidding strategy $s(\theta_i) = \theta_i\frac{m(n-1)}{m(n-1)+1}$ forms a symmetric Bayesian Nash equilibrium.
      \tcblower
      \begin{align*}
        b_i^* &= \arg\max_{b_i}\left(E\left(v_i(b_i,b_{-i};\theta_i,\theta_{-i})\right)\right)\\
        0 &= \frac{\partial E(v_i(b_i,b_{-i};\theta_i,\theta_{-i}))}{\partial b_i}\\
          &= \frac{-1}{2\sqrt{\theta_i - b_i}}\left(\frac{b_i}{k}\right)^{n-1} + (n-1)\sqrt{\theta_i - b_i}\left(\frac{b_i}{k}\right)^{n-2}\\
          b_i &= \frac{2k(n-1)}{2k(n-1) + 1}
      \end{align*}
      Letting $m=2k$, we have the given result.
    \end{problem}
    \begin{problem}{(c)}
      Show that in a second price sealed-bid auction, it is a weakly dominant strategy for each bidder to bid their valuation.
      \tcblower
      Letting $b^*$ denote the second-highest or highest-bid, we have the following situations:
      \begin{description}[font=\normalfont]
        \item[$b_i < \theta_i < b^*$:]\hfill
          \begin{description}[font=\normalfont]
            \item[Follow:] $v_i = 0$
            \item[Deviate to $\theta_i$:] $v_i = 0$
          \end{description}
        \item[$b_i <  b^*<\theta_i $:]\hfill
          \begin{description}[font=\normalfont]
            \item[Follow:] $v_i = 0$
            \item[Deviate to $\theta_i$:] $v_i = \sqrt{\theta_i - b^*} > 0$
          \end{description}
        \item[$b^*<\theta_i < b_i $:]\hfill
          \begin{description}[font=\normalfont]
            \item[Follow:] $v_i = \sqrt{\theta_i - b_i^*} > 0$
            \item[Deviate to $\theta_i$:] $v_i = \sqrt{\theta_i - b^*} > 0$
          \end{description}
      \item[$\theta_i < b_i < b^*$:]\hfill
        \begin{description}[font=\normalfont]
          \item[Follow:] $v_i = 0$
          \item[Deviate to $\theta_i$:] $v_i = 0$
        \end{description}
      \item[$\theta_i < b^*< b_i$:]\hfill
        \begin{description}[font=\normalfont]
          \item[Follow:] $v_i = \sqrt{\theta_i - b^*} < 0$ (technically complex, but still)
          \item[Deviate to $\theta_i$:] $v_i = 0$
        \end{description}
      \item[$ b^*<\theta_i < b_i$:]\hfill
        \begin{description}[font=\normalfont]
          \item[Follow:] $v_i = \theta_i - b^* > 0$
          \item[Deviate to $\theta_i$:] $v_i = \theta_i - b_i^* > 0$
        \end{description}
      \end{description}
    \end{problem}
    \begin{problem}{(d)}
      Compare the seller's expected revenue from the first and second price sealed-bid auctions. In what way does this example differ from the case of risk-neutral bidders?
      \tcblower
      We know that the expected revenue from the second price auction with risk aversion is equal to the expected revenue from the second price auction with risk neutrality, or
      \begin{align*}
        E(\theta_2^{[2]}) &= \frac{n-1}{n+1}.
      \end{align*}
      Similarly, for any $m > 0$ and $n > 1$, it is the case that
      \begin{align*}
        \frac{m(n-1)}{m(n-1) + 1} &< \frac{n-1}{n}
      \end{align*}
      meaning that the expected revenue from the first price auction is less. This is the primary effect of risk aversion --- with risk neutrality, we know from the Revenue Equivalence Theorem that the expected revenue from the first price and second price auctions are the same.
    \end{problem}
  \end{problem}
  \begin{problem}{Private Value Linear Bidding}
    Two players participate in a first price sealed-bid auction with independent private values. The value of the good for each player is given by $0.5 + \theta_i$, where $\theta_i$ is only observed by player $i$ and is uniformly distributed over the interval $[0,1]$.
    \tcblower
    \begin{problem}{(a)}
      Show there is a symmetric Bayesian Nash equilibrium in which each bidder $i$'s bidding strategy is of the form $s_i^*(\theta_i) = a + b\theta_i$.
      \tcblower
      \begin{align*}
        Ev_1(b_1,b_2;\theta_1,\theta_2) &= \left(0.5 + \theta_1 - b_1\right)P\left(b_1 > c + k\theta_{2}\right)\\
                                        &= \left(0.5 + \theta_1 - b_1\right)\left(\frac{b_1 - c}{k}\right)\\
        b_1^* &= \arg\max_{b_i}\left(\left(0.5 + \theta_1 - b_1\right)\left(\frac{b_1 - c}{k}\right)\right)\\
        0 &= \frac{1}{k}\left(0.5 + \theta_1 - b_1\right) - \frac{b_1-c}{k}\\
          &= 0.5 + \theta_1 - b_1 - b_1 + c\\
        b_1 &= \frac{0.5 + c}{2} + \frac{\theta_1}{2},\\
        \shortintertext{therefore,}
        b_1 &= \frac{1}{2} + \frac{\theta_1}{2}\\
        b_2 &= \frac{1}{2} + \frac{\theta_2}{2}
      \end{align*}
    \end{problem}
    \begin{problem}{(b)}
      What is the expected payoff of a player of type $\theta_i$?
      \tcblower
      \begin{align*}
        Ev_{i}(b_i,b_{-i};\theta_i,\theta_{-i}) &= \left(\frac{\theta_i}{2}\right)\left(\frac{\theta_i}{4}\right)\\
                                                &= \frac{\theta_i^2}{8}
      \end{align*}
    \end{problem}
  \end{problem}
\end{document}
