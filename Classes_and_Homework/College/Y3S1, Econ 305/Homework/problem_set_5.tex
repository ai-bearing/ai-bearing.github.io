\documentclass[8pt]{extarticle}
\title{}
\author{}
\date{}
\usepackage[shortlabels]{enumitem}


%paper setup
\usepackage{geometry}
\geometry{letterpaper, portrait, margin=1in}
\usepackage{fancyhdr}
% sans serif font:
\usepackage{cmbright}
%symbols
\usepackage{amsmath}
\usepackage{bigints}
\usepackage{amssymb}
\usepackage{amsthm}
\usepackage{mathtools}
\usepackage{bbm}
\usepackage[hidelinks]{hyperref}
\usepackage{gensymb}
\usepackage{multirow,array}
\usepackage{multicol}

\newtheorem*{remark}{Remark}
\usepackage[T1]{fontenc}
\usepackage[utf8]{inputenc}

%chemistry stuff
%\usepackage[version=4]{mhchem}
%\usepackage{chemfig}

%plotting
\usepackage{pgfplots}
\usepackage{tikz}
\tikzset{middleweight/.style={pos = 0.5}}
%\tikzset{weight/.style={pos = 0.5, fill = white}}
%\tikzset{lateweight/.style={pos = 0.75, fill = white}}
%\tikzset{earlyweight/.style={pos = 0.25, fill=white}}

%\usepackage{natbib}

%graphics stuff
\usepackage{graphicx}
\graphicspath{ {./images/} }
\usepackage[style=numeric, backend=biber]{biblatex} % Use the numeric style for Vancouver
\addbibresource{the_bibliography.bib}
%code stuff
%when using minted, make sure to add the -shell-escape flag
%you can use lstlisting if you don't want to use minted
%\usepackage{minted}
%\usemintedstyle{pastie}
%\newminted[javacode]{java}{frame=lines,framesep=2mm,linenos=true,fontsize=\footnotesize,tabsize=3,autogobble,}
%\newminted[cppcode]{cpp}{frame=lines,framesep=2mm,linenos=true,fontsize=\footnotesize,tabsize=3,autogobble,}

%\usepackage{listings}
%\usepackage{color}
%\definecolor{dkgreen}{rgb}{0,0.6,0}
%\definecolor{gray}{rgb}{0.5,0.5,0.5}
%\definecolor{mauve}{rgb}{0.58,0,0.82}
%
%\lstset{frame=tb,
%	language=Java,
%	aboveskip=3mm,
%	belowskip=3mm,
%	showstringspaces=false,
%	columns=flexible,
%	basicstyle={\small\ttfamily},
%	numbers=none,
%	numberstyle=\tiny\color{gray},
%	keywordstyle=\color{blue},
%	commentstyle=\color{dkgreen},
%	stringstyle=\color{mauve},
%	breaklines=true,
%	breakatwhitespace=true,
%	tabsize=3
%}
% text + color boxes
\renewcommand{\mathbf}[1]{\mathbbm{#1}}
\usepackage[most]{tcolorbox}
\tcbuselibrary{breakable}
\tcbuselibrary{skins}
\newtcolorbox{problem}[1]{colback=white,enhanced,title={\small #1},
          attach boxed title to top center=
{yshift=-\tcboxedtitleheight/2},
boxed title style={size=small,colback=black!60!white}, sharp corners, breakable}
%including PDFs
%\usepackage{pdfpages}
\setlength{\parindent}{0pt}
\usepackage{cancel}
\pagestyle{fancy}
\fancyhf{}
\rhead{Avinash Iyer}
\lhead{Econ 305: Problem Set 5}
\newcommand{\card}{\text{card}}
\newcommand{\ran}{\text{ran}}
\newcommand{\N}{\mathbbm{N}}
\newcommand{\Q}{\mathbbm{Q}}
\newcommand{\Z}{\mathbbm{Z}}
\newcommand{\R}{\mathbbm{R}}
\begin{document}
  \begin{problem}{Conquering an Island}
    Two opposed armies are ready to conquer an island. Each army's general can choose to either attack $(A)$ not attack $(N)$. In addition, each army is either strong or weak, and there is common knowledge that these two events are equally likely and independent. The island is worth $M$ if captured and it is captured either by attacking when its opponent is not or by attacking when its rival does if its own army is strong and the rival is weak. 
    \tcblower
    % Iterated dominance: find dominant strategy for particular type, then find expected payoffs for other type given dominant strategy.
  \end{problem}
  \begin{problem}{Matching Pennies, Revisited}
    %Cutoff strategies: guess strategy of player 1 as function of type is action 1 if type is at least one cutoff, other action if type is less than cutoff; type = cutoff is the indifference condition — indifference condition is expected payoff to player 1 of action 1 at  = expected payoff
    \begin{description}
      \item[Indifference Condition for Player 1:]
        \begin{align*}
          Ev_{1}(H,s_2^{\ast}(\theta_2);\theta_1^{\ast},\theta_2) &= Ev_{1}(T,s_2^{\ast}(\theta_2);\theta_1^{\ast},\theta_2)
        \end{align*}
    \end{description}
  \end{problem}
  \begin{problem}{Risk-Averse Bidders}
    Symmetric Bayesian Nash Equilibrium: $s_i^{\ast}(\theta_i) = k\theta_i + c$.
    \begin{enumerate}[(1)]
      \item Everyone except player $1$ is playing $s_i^{\ast}(\theta_i) = k\theta_i + c$
      \item Find $Ev_{1}(b_1,s_{-1}(\theta_{-1});\theta_1,\theta_{-1})$
      \item Find $b_1^{\ast}$ such that $\frac{\partial Ev_1}{\partial b_1} = 0$
      \item In BNE, $b_1^{\ast} = k\theta_1 + c$
    \end{enumerate}
  \end{problem}
\end{document}
