\documentclass[8pt]{extarticle}
\title{}
\author{Avinash Iyer}
\date{}
\usepackage[shortlabels]{enumitem}


%paper setup
\usepackage{geometry}
\geometry{letterpaper, portrait, margin=1in}
\usepackage{fancyhdr}
\usepackage{cmbright}

%symbols
\usepackage{amsmath}
\usepackage{amssymb}
\usepackage{amsthm}
\usepackage{mathtools}
\usepackage{hyperref}
\usepackage{gensymb}
\usepackage{multirow,array}
\usepackage{multicol}

\newtheorem*{remark}{Remark}
\usepackage[T1]{fontenc}
\usepackage[utf8]{inputenc}

%chemistry stuff
%\usepackage[version=4]{mhchem}
%\usepackage{chemfig}

%plotting
\usepackage{pgfplots}
\usepackage{tikz}
\tikzset{middleweight/.style={pos = 0.5, fill=white}}
\tikzset{weight/.style={pos = 0.5, fill = white}}
\tikzset{lateweight/.style={pos = 0.75, fill = white}}
\tikzset{earlyweight/.style={pos = 0.25, fill=white}}

%\usepackage{natbib}

%graphics stuff
\usepackage{graphicx}
\graphicspath{ {./images/} }
\usepackage[style=numeric, backend=biber]{biblatex} % Use the numeric style for Vancouver
\addbibresource{the_bibliography.bib}
%code stuff
%when using minted, make sure to add the -shell-escape flag
%you can use lstlisting if you don't want to use minted
%\usepackage{minted}
%\usemintedstyle{pastie}
%\newminted[javacode]{java}{frame=lines,framesep=2mm,linenos=true,fontsize=\footnotesize,tabsize=3,autogobble,}
%\newminted[cppcode]{cpp}{frame=lines,framesep=2mm,linenos=true,fontsize=\footnotesize,tabsize=3,autogobble,}

%\usepackage{listings}
%\usepackage{color}
%\definecolor{dkgreen}{rgb}{0,0.6,0}
%\definecolor{gray}{rgb}{0.5,0.5,0.5}
%\definecolor{mauve}{rgb}{0.58,0,0.82}
%
%\lstset{frame=tb,
%	language=Java,
%	aboveskip=3mm,
%	belowskip=3mm,
%	showstringspaces=false,
%	columns=flexible,
%	basicstyle={\small\ttfamily},
%	numbers=none,
%	numberstyle=\tiny\color{gray},
%	keywordstyle=\color{blue},
%	commentstyle=\color{dkgreen},
%	stringstyle=\color{mauve},
%	breaklines=true,
%	breakatwhitespace=true,
%	tabsize=3
%}
% text + color boxes
\usepackage[most]{tcolorbox}
\tcbuselibrary{breakable}
\tcbuselibrary{skins}
\newtcolorbox{problem}[1]{colback=white,enhanced,title={\small #1},
          attach boxed title to top center=
{yshift=-\tcboxedtitleheight/2},
boxed title style={size=small,colback=black!60!white}, breakable}
%including PDFs
%\usepackage{pdfpages}
\setlength{\parindent}{0pt}
\usepackage{cancel}
\pagestyle{fancy}
\fancyhf{}
\rhead{Avinash Iyer}
\lhead{Econ 305: Problem Set 4}
\newcommand{\card}{\text{card}}
\newcommand{\ran}{\text{ran}}
\newcommand{\N}{\mathbb{N}}
\newcommand{\Q}{\mathbb{Q}}
\newcommand{\Z}{\mathbb{Z}}
\newcommand{\R}{\mathbb{R}}
\begin{document}
  \renewcommand{\arraystretch}{1.75}
  \begin{problem}{Déjà Vu}
    Consider the game in which the following stage game is repeated twice.
    \begin{center}
      \begin{tabular}{c|c|c|c|}
        \multicolumn{1}{c}{} & \multicolumn{1}{c}{$L$} & \multicolumn{1}{c}{$C$} & \multicolumn{1}{c}{$R$}\\
        \cline{2-4}
        $T$ & $8,8$ & $0,0$ & $16,0$\\
        \cline{2-4}
        $M$ & $0,0$ & $4,4$ & $16,-1$\\
        \cline{2-4}
        $B$ & $0,16$ & $-1,16$ & $12,12$\\
        \cline{2-4}
      \end{tabular}
    \end{center}
    For each of the action profiles below, construct a SPE in which the action profile is played in the first stage of the game, or show that no such SPE exists.
    \begin{enumerate}[(a)]
      \item $(B,L)$
      \item $(B,C)$
    \end{enumerate}
    \tcblower
    In order to deduce the potential action profiles, we start by finding the Nash equilibria of the game.
    \begin{itemize}
      \item $R$ is strictly dominated by $\frac{1}{2}L + \frac{1}{2}C$.
    \end{itemize}
    \begin{center}
      \begin{tabular}{c|c|c|c|}
        \multicolumn{1}{c}{} & \multicolumn{1}{c}{$L$} & \multicolumn{1}{c}{$C$} & \multicolumn{1}{c}{$\xcancel{R}$}\\
        \cline{2-4}
        $T$ & $8,8$ & $0,0$ & $\xcancel{16,0}$\\
        \cline{2-4}
        $M$ & $0,0$ & $4,4$ & $\xcancel{16,-1}$\\
        \cline{2-4}
        $B$ & $0,16$ & $-1,16$ & $\xcancel{12,12}$\\
        \cline{2-4}
      \end{tabular}
    \end{center}
    \begin{itemize}
      \item Now, $B$ is strictly dominated by $\frac{1}{2}T + \frac{1}{2}M$ .
    \end{itemize}
    \begin{center}
      \begin{tabular}{c|c|c|c|}
        \multicolumn{1}{c}{} & \multicolumn{1}{c}{$L$} & \multicolumn{1}{c}{$C$} & \multicolumn{1}{c}{$\xcancel{R}$}\\
        \cline{2-4}
        $T$ & $8,8$ & $0,0$ & $\xcancel{16,0}$\\
        \cline{2-4}
        $M$ & $0,0$ & $4,4$ & $\xcancel{16,-1}$\\
        \cline{2-4}
        $\xcancel{B}$ & $\xcancel{0,16}$ & $\xcancel{-1,16}$ & $\xcancel{12,12}$\\
        \cline{2-4}
      \end{tabular}
    \end{center}
    We can see that there are three Nash equilibria:
    \begin{itemize}
      \item $(T,L)$, with payoffs $(8,8)$
      \item $(M,C)$, with payoffs $(4,4)$
      \item $\displaystyle \left(\frac{1}{3}T + \frac{2}{3}M,\frac{1}{3}L + \frac{2}{3}C\right)$, with payoffs $\displaystyle\left(\frac{16}{3},\frac{16}{3}\right)$
    \end{itemize}
    Thus, we would expect that conceivable SPE play as follows:
    \begin{itemize}
      \item In stage 1, if both players cooperate, the second stage plays the $(T,L)$ Nash equilibrium.
      \item In stage 1, if a player defects, the second stage plays the $(M,C)$ Nash equilibrium.
    \end{itemize}
    \begin{enumerate}[(a)]
      \item If we were to construct a SPE with $(B,L)$ in the first stage, we have the following:
        \begin{itemize}
          \item The incentive for Player $1$ to deviate to playing $T$ yields a net payoff increase of $8$
          \item The maximum possible punishment is a net loss of $4$ --- therefore, Player 1 has an incentive to deviate.
        \end{itemize}
      \item If we were to construct a SPE with $(B,C)$ in the first stage, we have the following:
        \begin{itemize}
          \item The incentive for Player 1 to deviate to playing $M$ yields a net payoff increase of $5$.
          \item The maximum possible punishment is a net loss of $4$ --- therefore, Player 1 has an incentive to deviate.
        \end{itemize}
    \end{enumerate}
    Therefore, both of the proposed strategies cannot be constructed into SPE.
  \end{problem}
  \begin{problem}{Two Seagulls}
    Seagulls love shellfish. In order to break the shell, they need to fly high up and drop the shellfish. However, the other seagulls will steal the shellfish from the seagull that dropped it. Consider the case of two seagulls, Nina (player 1) and Irina (player 2). The seagulls have two options: Up or Down. There is a cost of $10$ to going up and the value of eating the shellfish is $20$. Nina and Irina repeat this game infinitely many times, with a common discount factor of $\delta$. In particular, the stage game is given by the following payoff matrix:
    \begin{center}
      \begin{tabular}{c|c|c|}
        \multicolumn{1}{c}{} & \multicolumn{1}{c}{Up} & \multicolumn{1}{c}{Down}\\
        \cline{2-3}
        Up & $-10,-10$ & $-10,20$\\
        \cline{2-3}
        Down & $20,-10$ & $0,0$\\
        \cline{2-3}
      \end{tabular}
    \end{center}
    For each strategy profile below, find the range of discount factors (if any) such that the strategy profile is a subgame perfect equilibrium.
    \begin{enumerate}[(a)]
      \item (No Punishment) In the first stage Nina plays Up and Irina plays Down. They alternate Up and Down each day thereafter, irrespective of the history. A deviation by some player in some period does not change the prescription of play.
      \item (Grim Trigger) In the first stage Nina plays Up and Irina plays Down. They alternate Up and Down each day thereafter. If some player ever fails to follow this scheme, then both birds switch to playing Down forever.
    \end{enumerate}
    \tcblower
    \begin{problem}{(a)}
      \begin{description}
        \item[Current period, Player $i$ plays down:]\hfill
          \begin{align*}
            \shortintertext{Follow:}
            v_i &= (1-\delta)\left(-10 - 10\delta^2 - 10\delta^4 - \cdots + 20\delta + 20\delta^3 + \cdots\right)\\
                &= (1-\delta)\left(\frac{20\delta-10}{1-\delta^2}\right)\\
                &= \frac{20\delta - 10}{1 + \delta}\\
            \shortintertext{Deviate in current period:}
            v_i &= (1-\delta)\left(0 - 10\delta^2 - 10\delta^4 - \cdots + 20 \delta + 20\delta^3 + \cdots\right)\\
                &= (1-\delta)\left(\frac{20\delta - 10\delta^2}{1-\delta^2}\right)\\
                &= \frac{20\delta - 10\delta^2}{1 + \delta}\\
                \shortintertext{SPE Condition:}
            \frac{20\delta - 10}{1 + \delta} &\geq \frac{20\delta - 10\delta^2}{1 + \delta}\\
            20\delta - 10 &\geq 20\delta - 10\delta^2\\
            \delta^2 &\geq 1\\
            \delta &\geq 1
          \end{align*}
          However, since $\delta < 1$ necessarily by the One Stage Deviation Property, the given strategy is not a SPE.
      \end{description}
    \end{problem}
    \begin{problem}{(b)}
      \begin{description}
        \item[Current period, Player $i$ plays down:]\hfill
          \begin{align*}
            \shortintertext{Follow:}
            v_i &= (1-\delta)\left(-10 - 10\delta^2 - 10\delta^4 - \cdots + 20\delta + 20\delta^3 + \cdots\right)\\
                &= (1-\delta)\left(\frac{20\delta-10}{1-\delta^2}\right)\\
                &= \frac{20\delta - 10}{1 + \delta}\\
                \shortintertext{Deviate in current period:}
            v_i &= (1-\delta)(0)\\
                &= 0
              \shortintertext{SPE Condition:}
            \frac{20\delta-10}{1+\delta} &\geq 0\\
            \delta &\geq \frac{1}{2}
          \end{align*}
      \end{description}
    \end{problem}
  \end{problem}
  \begin{problem}{Cournot Competition Collusion}
    Suppose that two firms are engaged in an infinitely repeated Cournot competition game with market demand in each period given by $P(Q) = 1-Q$, where $Q = q_1 + q_2$. Assume that each firm has zero costs. Show that by using a grim trigger strategy of permanent reversion to the static Cournot equilibrium $(q_1 = q_2 = 1/3)$, firms can sustain full collusion $(q_1 = q_2 = 1/4)$ as a subgame perfect equilibrium for $\delta$ sufficiently high. Find the range of $\delta$ such that this is true.
    \tcblower
    \begin{align*}
      v_i &= q_i(1-q_i-q_{-i})\\
      \shortintertext{Best response of firm $i$:}
      \frac{\partial v_i}{\partial q_i} &= 0\\
      1 - 2q_i - q_{-i} &= 0\\
      q_i &= \frac{1-q_{-i}}{2}\\
      \shortintertext{Best response to attempted collusion:}
      q_i &= \frac{1-\frac{1}{4}}{2}\\
          &= \frac{3}{8}\\
      \shortintertext{Payoff from best response to attempted collusion:}
      v_i &= \frac{3}{8}\left(1-\frac{3}{8}-\frac{1}{4}\right)\\
          &= \frac{9}{64}\\
      \shortintertext{Payoff in Nash equilibrium of stage game:}
      v_i &= \frac{1}{3}\left(1-\frac{1}{3}-\frac{1}{3}\right)\\
          &= \frac{1}{9}
      \shortintertext{Discounted average payoff from full collusion with grim trigger:}
      v_i^{\ast} &= (1-\delta)\left(\frac{1}{8} + \frac{1}{8}\delta + \frac{1}{8}\delta^2 + \cdots\right)\\
                 &= (1-\delta)\frac{1}{8(1-\delta)}\\
                 &= \frac{1}{8}\\
      \shortintertext{Discounted average payoff from deviating in Period 1 with grim trigger:}
      v_i' &= (1-\delta)\left(\frac{9}{64} + \underbrace{\frac{1}{9}\delta + \frac{1}{9}\delta^2 + \cdots}_{\substack{\text{discounted future payoff}\\\text{from grim trigger}}}\right)\\
           &=\frac{9}{64}-\frac{9}{64}\delta + (1-\delta)\frac{\delta}{9(1-\delta)}\\
           &= \frac{9}{64}-\frac{9}{64}\delta + \frac{\delta}{9}\\
      \shortintertext{One Stage Deviation Condition:}
      \frac{1}{8} &\geq \frac{9}{64}-\frac{9}{64}\delta + \frac{\delta}{9}\\
      \frac{17}{576}\delta &\geq \frac{1}{64}\\
      \delta &\geq \frac{9}{17}
    \end{align*}
  \end{problem}
\end{document}
