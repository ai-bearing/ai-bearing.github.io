\documentclass[8pt]{extarticle}
\title{}
\author{Avinash Iyer}
\date{}
\usepackage[shortlabels]{enumitem}

%font setup
%
%\usepackage{newpxtext,eulerpx}

%paper setup
\usepackage{geometry}
\geometry{letterpaper, portrait, margin=1in}
\usepackage{fancyhdr}

%symbols
\usepackage{amsmath}
\usepackage{amssymb}
\usepackage{mathtools}
\usepackage{hyperref}
\usepackage{gensymb}
\usepackage{multirow,array}

\usepackage[T1]{fontenc}
\usepackage[utf8]{inputenc}

%chemistry stuff
\usepackage[version=4]{mhchem}
\usepackage{chemfig}

%plotting
\usepackage{pgfplots}
\usepackage{tikz}
\tikzset{middleweight/.style={pos = 0.5, fill=white}}
\tikzset{weight/.style={pos = 0.5, fill = white}}
\tikzset{lateweight/.style={pos = 0.75, fill = white}}
\tikzset{earlyweight/.style={pos = 0.25, fill=white}}

%\usepackage{natbib}

%graphics stuff
\usepackage{graphicx}
\graphicspath{ {./images/} }

%code stuff
%when using minted, make sure to add the -shell-escape flag
%you can use lstlisting if you don't want to use minted
%\usepackage{minted}
%\usemintedstyle{pastie}
%\newminted[javacode]{java}{frame=lines,framesep=2mm,linenos=true,fontsize=\footnotesize,tabsize=3,autogobble,}
%\newminted[cppcode]{cpp}{frame=lines,framesep=2mm,linenos=true,fontsize=\footnotesize,tabsize=3,autogobble,}

%\usepackage{listings}
%\usepackage{color}
%\definecolor{dkgreen}{rgb}{0,0.6,0}
%\definecolor{gray}{rgb}{0.5,0.5,0.5}
%\definecolor{mauve}{rgb}{0.58,0,0.82}
%
%\lstset{frame=tb,
%	language=Java,
%	aboveskip=3mm,
%	belowskip=3mm,
%	showstringspaces=false,
%	columns=flexible,
%	basicstyle={\small\ttfamily},
%	numbers=none,
%	numberstyle=\tiny\color{gray},
%	keywordstyle=\color{blue},
%	commentstyle=\color{dkgreen},
%	stringstyle=\color{mauve},
%	breaklines=true,
%	breakatwhitespace=true,
%	tabsize=3
%}
% text + color boxes
\usepackage[most]{tcolorbox}
\tcbuselibrary{breakable}
\newtcolorbox{problem}[1]{colback = white, title = {#1}, breakable}
\newtcolorbox{solution}{colback = white, colframe = black!75!white, title = Solution, breakable}
%including PDFs
%\usepackage{pdfpages}
\setlength{\parindent}{0pt}
\usepackage{cancel}
\pagestyle{fancy}
\fancyhf{}
\rhead{Avinash Iyer}
\lhead{Econ 305: Problem Set 2}
\newcommand{\card}{\text{card}}
\newcommand{\ran}{\text{ran}}
\newcommand{\N}{\mathbb{N}}
\newcommand{\Q}{\mathbb{Q}}
\newcommand{\Z}{\mathbb{Z}}
\newcommand{\R}{\mathbb{R}}
\begin{document}
  \begin{problem}{Cops and Robbers}
    Player 1 is a police officer who must decide whether to patrol the streets or to hang out at the coffee shop. His payoff from hanging out at the coffee shop is $10$, while his payoff from patrolling the streets depends on whether he catches a robber, who is player $2$. If the robber prowls the streets then the police officer will catch him and obtain a payoff of $20$. If the robber stays in his hideaway then the officer's payoff is $0$. The robber must choose between staying hidden or prowling the streets. If he stays hidden then his payoff is $0$, while if he prowls the streets his payoff is $-10$ if the officer is patrolling the streets and $10$ if the officer is at the coffee shop.
    \begin{enumerate}[(a)]
      \item Write down the matrix form of this game.
      \item Draw the best-response function of each player.
      \item Find the Nash equilibrium of this game.
    \end{enumerate}
    \tcblower
    \begin{problem}{(a)}
      \begin{center}
        \renewcommand{\arraystretch}{1.5}
        \begin{tabular}{cc|c|c|}
          &\multicolumn{1}{c}{} & \multicolumn{2}{c}{Robber}\\
          &\multicolumn{1}{c}{} & \multicolumn{1}{c}{Stay} & \multicolumn{1}{c}{Prowl} \\
          \cline{3-4}
          \multirow{2}{1em}{Cop} & Coffee & $10,0$ & $10,10$\\
          \cline{3-4}
                                 & Patrol & $0,0$ & $20,-10$\\
                                 \cline{3-4}
        \end{tabular}
      \end{center}
    \end{problem}
    \begin{problem}{(b)}
      In order to find the best response correspondences for each player, we assume that both are mixing strategies.
      \begin{align*}
        \shortintertext{Cop's Indifference:}
        v_1(C,\sigma_2) &= 10\\
        v_1(P,\sigma_2) &= 0(q) + 20 (1-q)\\
        q &= \frac{1}{2}\\
        \shortintertext{Robber's Indifference:}
        v_2(\sigma_1,S) &= 10p + 0(1-p)\\
        v_2(\sigma_1,P) &= 10p -10(1-p)\\
        10p &= 10p - 10(1-p)\\
        p &= 1
      \end{align*}
      The best response correspondences are plotted below:
      \begin{center}
        \begin{tikzpicture}[scale=0.75]
          \draw (0,6) -- (0,0) -- (6,0);
          \node[anchor = north] at (0,0) {$0$};
          \node[anchor= north] at (3,-1) {$p$};
          \node[anchor = north] at (6,0) {$1$};

          \node[anchor = east] at (0,0) {$0$};
          \node[anchor= east] at (-1,3) {$q$};
          \node[anchor = east] at (0,6) {$1$};

          \draw[thick] (0,6) -- (0,3) -- (6,3) -- (6,0);
          \node[anchor = west] at (6,3) {$BR_1(q)$};
          \node[anchor = east] at (0,3) {$\frac{1}{2}$};

          \draw[gray, thick] (0,6) -- (2,6) -- (2,0) -- (6,0);
          \node[anchor = north] at (2,0) {$\frac{1}{3}$};
          \node[anchor = west] at (2,6) {$BR_2(p)$};
          \filldraw (2,3) circle (2pt);
          \node[anchor = south west] at (2,3) {MSNE};
        \end{tikzpicture}
      \end{center}
    \end{problem}
    \begin{problem}{(c)}
      The Nash equilibria are at $p^* = \frac{1}{3}$ and $q^* = \frac{1}{2}$.
    \end{problem}
  \end{problem}
  \begin{problem}{Discrete All-Pay Auction}
    Each bidder submits a bid. The highest bidder gets the good, but \textit{all players pay their bids}. Consider an auction in which player $1$ values the item at $3$ and player $2$ values the item at $5$. Each player can bid either $0$ ($Z$), $1$ ($O$), or $2$ ($T$). If both players bid the same amount, a coin is flipped, but both players pay their bids nonetheless.\\

    The bids correspond to the payoff matrix below.
    \begin{center}
      \renewcommand{\arraystretch}{1.5}
      \begin{tabular}{cc|c|c|c|}
        & \multicolumn{1}{c}{} & \multicolumn{3}{c}{Player 2} \\
        & \multicolumn{1}{c}{} & \multicolumn{1}{c}{$Z$} & \multicolumn{1}{c}{$O$} & \multicolumn{1}{c}{$T$}\\
        \cline{3-5}
        \multirow{3}{2em}{Player 1} & $Z$ & $1.5,2.5$ & $0,4$ & $0,3$ \\
        \cline{3-5}
                                    & $O$ & $2,0$ & $0.5,1.5$ & $-1,3$\\
                                    \cline{3-5}
                                    & $T$ & $1,0$ & $1,-1$ & $-0.5,0.5$\\
                                    \cline{3-5}
      \end{tabular}
    \end{center}
    \tcblower
    \begin{problem}{(a)}
      Write down the reduced game that results from iteratively eliminating strictly dominated strategies.
      \tcblower
      \begin{itemize}
        \item For Player 2, $T$ strictly dominates $Z$.
      \end{itemize}
      \begin{center}
        \renewcommand{\arraystretch}{1.5}
        \begin{tabular}{cc|c|c|c|}
          & \multicolumn{1}{c}{} & \multicolumn{3}{c}{Player 2} \\
          & \multicolumn{1}{c}{} & \multicolumn{1}{c}{$Z$} & \multicolumn{1}{c}{$O$} & \multicolumn{1}{c}{$T$}\\
          \cline{3-5}
          \multirow{3}{2em}{Player 1} & $Z$ & $\xcancel{1.5,2.5}$ & $0,4$ & $0,3$ \\
          \cline{3-5}
                                      & $O$ & $\xcancel{2,0}$ & $0.5,1.5$ & $-1,3$\\
                                      \cline{3-5}
                                      & $T$ & $\xcancel{1,0}$ & $1,-1$ & $-0.5,0.5$\\
                                      \cline{3-5}
        \end{tabular}
      \end{center}
      \begin{itemize}
        \item For Player 1, $T$ strictly dominates $O$
      \end{itemize}
      \begin{center}
        \renewcommand{\arraystretch}{1.5}
        \begin{tabular}{cc|c|c|c|}
          & \multicolumn{1}{c}{} & \multicolumn{3}{c}{Player 2} \\
          & \multicolumn{1}{c}{} & \multicolumn{1}{c}{$Z$} & \multicolumn{1}{c}{$O$} & \multicolumn{1}{c}{$T$}\\
          \cline{3-5}
          \multirow{3}{2em}{Player 1} & $Z$ & $\xcancel{1.5,2.5}$ & $0,4$ & $0,3$ \\
          \cline{3-5}
                                      & $O$ & $\xcancel{2,0}$ & $\xcancel{0.5,1.5}$ & $\xcancel{-1,3}$\\
                                      \cline{3-5}
                                      & $T$ & $\xcancel{1,0}$ & $1,-1$ & $-0.5,0.5$\\
                                      \cline{3-5}
        \end{tabular}
      \end{center}
    \end{problem}
    \begin{problem}{(b)}
      Find the Nash equilibria for this game.
      \tcblower
      We will let $p$ denote Player 1's chance of playing $Z$ and let $q$ denote Player 2's chance of playing $O$.
      \begin{align*}
        \shortintertext{Player 1's Indifference:}
        v_1(Z,\sigma_2) &= 0\\
        v_1(T,\sigma_2) &= q - 0.5(1-q)\\
        0 &= q - 0.5(1-q) \\
        q &= \frac{2}{3}\\
        \shortintertext{Player 2's Indifference:}
        v_2(\sigma_1,O) &= 4p - (1-p)\\
        v_2(\sigma_1,T) &= 3p + 0.5(1-p)\\
        5p-1 &= 2.5p + 0.5\\
        p &= \frac{3}{5}
      \end{align*}
      Therefore, the Nash equilibrium is $p^* = \frac{3}{5},q^* = \frac{2}{3}$.
    \end{problem}
  \end{problem}
  \begin{problem}{Mixed Up}
    In the following normal-form games, find all the Nash equilibria.
    \tcblower
    \begin{problem}{(a)}
      \begin{center}
        \renewcommand{\arraystretch}{1.5}
        \begin{tabular}{c|c|c|}
          \multicolumn{1}{c}{} & \multicolumn{1}{c}{$L$} & \multicolumn{1}{c}{$R$}\\
          \cline{2-3}
          $T$ & $0,0$ & $10,12$\\
          \cline{2-3}
          $B$ & $4,4$ & $6,0$\\
          \cline{2-3}
        \end{tabular}
      \end{center}
      \tcblower
      \begin{align*}
        \shortintertext{Player 1's Indifference:}
        v_1(T,\sigma_2) &= 10(1-q)\\
        v_1(B,\sigma_2) &= 4q + 6(1-q)\\
        10-10q &= 6-2q\\
        q &= \frac{1}{2}\\
        \shortintertext{Player 2's Indifference:}
        v_2(\sigma_1,L) &= 4(1-p)\\
        v_2(\sigma_1,R) &= 12p\\
        12p &= 4-4p\\
        p &= \frac{1}{2}
      \end{align*}
      Therefore, the Nash equilibrium is $p^* = \frac{1}{2},q^* = \frac{1}{2}$
    \end{problem}
    \begin{problem}{(b)}
      \begin{center}
        \renewcommand{\arraystretch}{1.5}
        \begin{tabular}{c|c|c|}
          \multicolumn{1}{c}{} & \multicolumn{1}{c}{$L$} & \multicolumn{1}{c}{$R$}\\
          \cline{2-3}
          $T$ & $2,4$ & $2,6$\\
          \cline{2-3}
          $B$ & $6,6$ & $2,4$\\
          \cline{2-3}
        \end{tabular}
      \end{center}
      \tcblower
      \begin{align*}
        \shortintertext{Player 1's Indifference:}
        v_1(T,\sigma_2) &= 2\\
        v_1(B,\sigma_2) &= 6q + 2(1-q)\\
        2 &= 6q + 2(1-q)\\
        q &= 0\\
        \shortintertext{Player 2's Indifference:}
        v_2(\sigma_1,L) &= 4p + 6(1-p)\\
        v_2(\sigma_1,R) &= 6p + 4(1-p)\\
        6-2p &= 4 + 2p\\
        p &= \frac{1}{2}
      \end{align*}
      So the Nash equilibria are $(T,R)$ and $(B,R)$, each with equal frequency.
    \end{problem}
    \begin{problem}{(c)}
      \begin{center}
        \small
        \renewcommand{\arraystretch}{1.5}
        \begin{tabular}{c|c|c|c|c|}
          \multicolumn{1}{c}{} & \multicolumn{1}{c}{$W$} & \multicolumn{1}{c}{$X$} & \multicolumn{1}{c}{$Y$} & \multicolumn{1}{c}{$Z$}\\
          \cline{2-5}
          $A$ & $3,1$ & $1,6$ & $3,1$ & $6,5$ \\
          \cline{2-5}
          $B$ & $6,6$ & $2,3$ & $1,1$ & $1,1$ \\
          \cline{2-5}
          $C$ & $1,1$ & $2,3$ & $6,6$ & $5,2$ \\
          \cline{2-5}
          $D$ & $2,3$ & $7,2$ & $2,3$ & $2,1$\\
          \cline{2-5}
        \end{tabular}
      \end{center}
      \tcblower
      \begin{itemize}
        \item $Z$ is strictly dominated by $X$
      \end{itemize}
      \begin{center}
        \small
        \renewcommand{\arraystretch}{1.5}
        \begin{tabular}{c|c|c|c|c|}
          \multicolumn{1}{c}{} & \multicolumn{1}{c}{$W$} & \multicolumn{1}{c}{$X$} & \multicolumn{1}{c}{$Y$} & \multicolumn{1}{c}{$Z$}\\
          \cline{2-5}
          $A$ & $3,1$ & $1,6$ & $3,1$ & $\xcancel{6,5}$ \\
          \cline{2-5}
          $B$ & $6,6$ & $2,3$ & $1,1$ & $\xcancel{1,1}$ \\
          \cline{2-5}
          $C$ & $1,1$ & $2,3$ & $6,6$ & $\xcancel{5,2}$ \\
          \cline{2-5}
          $D$ & $2,3$ & $7,2$ & $2,3$ & $\xcancel{2,1}$\\
          \cline{2-5}
        \end{tabular}
      \end{center}
        \begin{itemize}
          \item $A$ is strictly dominated by $\frac{1}{2}B + \frac{1}{2}C$
        \end{itemize}
        \begin{center}
          \small
          \renewcommand{\arraystretch}{1.5}
          \begin{tabular}{c|c|c|c|c|}
            \multicolumn{1}{c}{} & \multicolumn{1}{c}{$W$} & \multicolumn{1}{c}{$X$} & \multicolumn{1}{c}{$Y$} & \multicolumn{1}{c}{$Z$}\\
            \cline{2-5}
            $A$ & $\xcancel{3,1}$ & $\xcancel{1,6}$ & $\xcancel{3,1}$ & $\xcancel{6,5}$ \\
            \cline{2-5}
            $B$ & $6,6$ & $2,3$ & $1,1$ & $\xcancel{1,1}$ \\
            \cline{2-5}
            $C$ & $1,1$ & $2,3$ & $6,6$ & $\xcancel{5,2}$ \\
            \cline{2-5}
            $D$ & $2,3$ & $7,2$ & $2,3$ & $\xcancel{2,1}$\\
            \cline{2-5}
          \end{tabular}
        \end{center}
        \begin{itemize}
          \item $X$ is strictly dominated by $\frac{1}{2}W + \frac{1}{2}Y$
        \end{itemize}
        \begin{center}
          \small
          \renewcommand{\arraystretch}{1.5}
          \begin{tabular}{c|c|c|c|c|}
            \multicolumn{1}{c}{} & \multicolumn{1}{c}{$W$} & \multicolumn{1}{c}{$X$} & \multicolumn{1}{c}{$Y$} & \multicolumn{1}{c}{$Z$}\\
            \cline{2-5}
            $A$ & $\xcancel{3,1}$ & $\xcancel{1,6}$ & $\xcancel{3,1}$ & $\xcancel{6,5}$ \\
            \cline{2-5}
            $B$ & $6,6$ & $\xcancel{2,3}$ & $1,1$ & $\xcancel{1,1}$ \\
            \cline{2-5}
            $C$ & $1,1$ & $\xcancel{2,3}$ & $6,6$ & $\xcancel{5,2}$ \\
            \cline{2-5}
            $D$ & $2,3$ & $\xcancel{7,2}$ & $2,3$ & $\xcancel{2,1}$\\
            \cline{2-5}
          \end{tabular}
        \end{center}
        \begin{itemize}
          \item $D$ is strictly dominated by $\frac{1}{2}B + \frac{1}{2}C$
        \end{itemize}
        \begin{center}
          \small
          \renewcommand{\arraystretch}{1.5}
          \begin{tabular}{c|c|c|c|c|}
            \multicolumn{1}{c}{} & \multicolumn{1}{c}{$W$} & \multicolumn{1}{c}{$X$} & \multicolumn{1}{c}{$Y$} & \multicolumn{1}{c}{$Z$}\\
            \cline{2-5}
            $A$ & $\xcancel{3,1}$ & $\xcancel{1,6}$ & $\xcancel{3,1}$ & $\xcancel{6,5}$ \\
            \cline{2-5}
            $B$ & $6,6$ & $\xcancel{2,3}$ & $1,1$ & $\xcancel{1,1}$ \\
            \cline{2-5}
            $C$ & $1,1$ & $\xcancel{2,3}$ & $6,6$ & $\xcancel{5,2}$ \\
            \cline{2-5}
            $D$ & $\xcancel{2,3}$ & $\xcancel{7,2}$ & $\xcancel{2,3}$ & $\xcancel{2,1}$\\
            \cline{2-5}
          \end{tabular}
        \end{center}
      \begin{align*}
        \shortintertext{Player 1's Indifference:}
        v_1(B,\sigma_2) &= 6q + (1-q)\\
        v_1(C,\sigma_2) &= q + 6(1-q)\\
        6q + 1 - q &= q + 6 - 6q\\
        q &= \frac{1}{2}\\
        \shortintertext{Player 2's Indifference:}
        v_2(\sigma_1,W) &= 6p + (1-p)\\
        v_2(\sigma_1,Y) &= p + 6(1-p)\\
        6p + 1 - p &= p + 6 - 6p\\
        p &= \frac{1}{2}
      \end{align*}
      Therefore, the Mixed Strategy Nash Equilibria are $(B,W)$ and $(C,Y)$ with equal probability.
    \end{problem}
  \end{problem}
  \begin{problem}{Voter Participation}
    Two candidates, Joe and Donald, compete in a national Presidential election. Of the registered voters in the U.S., $h$ individuals support Joe and $d\leq h$ individuals support Donald. Each individual decides whether to vote, at a cost, for the candidate she supports, or to abstain. A citizen who abstains receives a payoff of $2$ if the candidate they support wins, $1$ if the candidate ties, and $0$ if the candidate loses. A citizen who votes receives the payoffs $2-c$, $1-c$, and $-c$ in these three cases, where $0 < c < 1$.
    \begin{enumerate}[(a)]
      \item There is a mixed strategy Nash equilibrium of this game when $h > d$ in which every supporter of Donald votes with probability $p\in (0,1)$, $d$ supporters of Joe vote with certainty, and the remaining $h-d$ supporters of Joe abstain with certainty. Determine $p$.
      \item How does the probability $p$ that a Donald supporter votes depend on $c$ and $d$.
    \end{enumerate}
    \tcblower
    \begin{problem}{(a)}
      Each Donald voter must have a probability assigned such that $1-c$ is the payoff for any Joe supporter regardless of whether they would vote or abstain (as that must be when they tie). There must be $d$ Joe voters at the Nash equilibrium, so we have
      \[
        p = (1-c)^d
      \] 
    \end{problem}
    \begin{problem}{(b)}
      If the cost of voting goes up, the probability must go down, and similarly if the number of voters goes up, the likelihood of one's vote being the deciding factor in whether one ties or loses goes down.
    \end{problem}
  \end{problem}
  \begin{problem}{To Catch a Thief}
    A crime is observed by a group of $n$ people. Each person can choose to call ($C$) or don't call ($D$) the police and would like the police to be informed but prefers that someone else makes the phone call. Suppose that each person attaches the benefit $b$ to the police being informed and bears the cost $c$ if they make the phone call, where $b > c > 0$. In particular:
    \[
      v_i(s) = \begin{cases}
        0,& s = (D,D,\dots,D)\\
        b-c,& s_i = C\\
        b,&\text{otherwise}
      \end{cases}
    \] 
    \tcblower
    \begin{problem}{(a)}
      Argue that any strategy profile in which exactly one person calls is a PSNE.
      \tcblower
      Consider the given strategy profile.\\

      Any player playing $D$ receives payoff of $b$, and deviating would yield payoff of $b-c$.\\

      The player playing $C$ receives payoff of $b-c$, and deviating would yield payoff of $0$.
    \end{problem}
    \begin{problem}{(b)}
      Show that this game has no PSNE in which everyone chooses the same strategy.
      \tcblower
      If everyone chooses $D$, there is an incentive for any individual to deviate to $C$, as their payoff increases from $0$ to $b-c$.\\

      If everyone chooses $C$, there is an incentive for any individual to deviate to $D$, as their payoff increases from $b-c$ to $b$.
    \end{problem}
    \begin{problem}{(c)}
      There exists a symmetric mixed strategy Nash equilibrium of this game. In particular, denote the strategy played by each player as $\sigma_i = pC + (1-p)D$, where $p$ is the probability that each person calls. Determine $p$.
      \tcblower
      \begin{align*}
        \shortintertext{Player $i$'s Indifference:}
        v_1(C,\sigma_{-i}) &= b-c\\
        v_1(D,\sigma_{-i}) &= b\left(1-(1-p)^{n-1}\right)\\
        b-b(1-p)^{n-1} &= b-c \\
        c &= b(1-p)^{n-1}\\
        1-p &= \left(\frac{c}{b}\right)^{\frac{1}{n-1}}\\
        p &= 1-\left(\frac{c}{b}\right)^{\frac{1}{n-1}}
      \end{align*}
    \end{problem}
    \begin{problem}{(d)}
      As $n$ increases, $p$ decreases, which means that any \textit{individual}'s chance of calling the police decreases. However, the \textit{aggregate} chance that police are called increases, and tends to $1-\frac{c}{b}$.
    \end{problem}
  \end{problem}
\end{document}
