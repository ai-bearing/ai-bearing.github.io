\documentclass[10pt]{extarticle}
\title{}
\author{Avinash Iyer}
\date{}
\usepackage[shortlabels]{enumitem}

%font setup
%
\usepackage{newpxtext,eulerpx}

%paper setup
\usepackage{geometry}
\geometry{letterpaper, portrait, margin=1in}
\usepackage{fancyhdr}

%symbols
\usepackage{amsmath}
\usepackage{mathtools}
%\usepackage{amssymb}
\usepackage{hyperref}
\usepackage{gensymb}

\usepackage[T1]{fontenc}
\usepackage[utf8]{inputenc}

%chemistry stuff
%\usepackage[version=4]{mhchem}
%\usepackage{chemfig}

%plotting
\usepackage{pgfplots}
\usepackage{tikz}
\tikzset{middleweight/.style={pos = 0.5, fill=white}}
\tikzset{weight/.style={pos = 0.5, fill = white}}
\tikzset{lateweight/.style={pos = 0.75, fill = white}}
\tikzset{earlyweight/.style={pos = 0.25, fill=white}}

%\usepackage{natbib}

%graphics stuff
\usepackage{graphicx}
\graphicspath{ {./images/} }

%code stuff
%when using minted, make sure to add the -shell-escape flag
%you can use lstlisting if you don't want to use minted
%\usepackage{minted}
%\usemintedstyle{pastie}
%\newminted[javacode]{java}{frame=lines,framesep=2mm,linenos=true,fontsize=\footnotesize,tabsize=3,autogobble,}
%\newminted[cppcode]{cpp}{frame=lines,framesep=2mm,linenos=true,fontsize=\footnotesize,tabsize=3,autogobble,}

%\usepackage{listings}
%\usepackage{color}
%\definecolor{dkgreen}{rgb}{0,0.6,0}
%\definecolor{gray}{rgb}{0.5,0.5,0.5}
%\definecolor{mauve}{rgb}{0.58,0,0.82}

%\lstset{frame=tb,
%	language=Java,
%	aboveskip=3mm,
%	belowskip=3mm,
%	showstringspaces=false,
%	columns=flexible,
%	basicstyle={\small\ttfamily},
%	numbers=none,
%	numberstyle=\tiny\color{gray},
%	keywordstyle=\color{blue},
%	commentstyle=\color{dkgreen},
%	stringstyle=\color{mauve},
%	breaklines=true,
%	breakatwhitespace=true,
%	tabsize=3
%}
% text + color boxes
\usepackage{tcolorbox}
\tcbuselibrary{breakable}
\newtcolorbox{problem}[1]{colback = white, title = {#1}, breakable}
\newtcolorbox{solution}{colback = white, colframe = black!75!white, title = Solution, breakable}
%including PDFs
\usepackage{pdfpages}
\setlength{\parindent}{0pt}

\pagestyle{fancy}
\fancyhf{}
\rhead{Avinash Iyer}
\lhead{Econ 305: Class Notes}
\begin{document}
  \begin{problem}{Introduction to Game Theory}
    Game Theory analyzes the \textit{interaction} among a \textit{group} of \textit{rational} agents who \textit{behave strategically}.
    \begin{itemize}
      \item A group consists of at least two individuals who are free to make decisions.
      \item An interaction means that the decisions of at least one member of the group must affect at least one other member of the group.
      \item In strategic behavior, members of the group account for the interaction in their decision making process.
      \item Rational agents act in their best decisions based on their knowledge.
    \end{itemize}
    Keynes's Beauty Contest: Choose the face that is the most chosen in a newspaper contest.\\

    In many games, we are not asked to pick \textit{our} favorite, we are asked to pick \textit{everyone else's} favorite.
    \begin{problem}{Applications of Game Theory}
      \begin{itemize}
        \item Labor Economics (compensation interactions, promotions)
        \item Industrial Organization (pricing, entry, exit, etc.)
        \item Public Finance (public goods games)
        \item Political Economy (strategic voting)
        \item Trade (tariff wars)
        \item Biology (hunting and mating)
        \item Linguistics
      \end{itemize}
    \end{problem}
    It's important to remember that game theory is a subfield of \textit{mathematics}, not economics.
  \end{problem}
\end{document}
