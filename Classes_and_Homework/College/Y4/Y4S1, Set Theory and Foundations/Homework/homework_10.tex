\documentclass[10pt]{mypackage}

% sans serif font:
%\usepackage{cmbright}
%\usepackage{sfmath}
%\usepackage{bbold} %better blackboard bold

%serif font + different blackboard bold for serif font
\usepackage{newpxtext,eulerpx}
\renewcommand*{\mathbb}[1]{\varmathbb{#1}}
\renewcommand*{\hbar}{\hslash}

\pagestyle{fancy} %better headers
\fancyhf{}
\rhead{Avinash Iyer}
\lhead{Set Theory and Foundations of Mathematics: Homework 10}

\setcounter{secnumdepth}{0}

\begin{document}
\RaggedRight
\section{4.7}%
\begin{problem}
  Show that the function $P\left(x,y\right) = x^y$ is primitive recursive.
\end{problem}
\begin{solution}
  We have
  \begin{align*}
    P\left(x,y+1\right) &= M\left(x,P(x,y)\right),\\
    P\left(x,0\right) &= 1\\
    &= S\left(C_0\left(x\right)\right)
  \end{align*}
  We have
  \begin{align*}
    P\left(x,y+1\right) &= g\left(x,y,P\left(x,y\right)\right)
    \intertext{where}
    g\left(x,y,z\right) &= M\left(P_{1}^{(3)}\left(x,y,z\right),P_{3}^{(3)}\left(x,y,z\right)\right)
    \intertext{and}
    P\left(x,0\right) &= f(x)\\
                      &= S\left(C_0\left(x\right)\right).
  \end{align*}
  It is the case that $P$ is computed by primitive recursion on $f$ and $g$, since $M\left(x,y\right)$ is primitive recursive by a previous result.
\end{solution}

\section{4.8}%
\begin{problem}
  Given a primitive recursive function $p\left(x_1,\dots,x_n,y\right)$, show that the function
  \begin{align*}
    h\left(x_1,\dots,x_n,z\right) &= \prod_{k=0}^{z}p\left(x_1,\dots,x_n,k\right)
  \end{align*}
  is primitive recursive.
\end{problem}
\section{Extra Problem 1}%
\begin{problem}
  Show that every primitive recursive function is total.
\end{problem}
\begin{solution}[Proof 1]
  Let $f$ be primitive recursive. Then, $f$ is obtained via $S,C_0,P_{i}^{(k)}$ by composition and primitive recursion. We want to show that composition and primitive recursion preserve totality.\newline

  Turning our attention to composition, we have
  \begin{align*}
    h\left(x_1,\dots,x_n\right) &= f\left(g_1\left(x_1,\dots,x_n\right),\dots,g_m\left(x_1,\dots,x_n\right)\right).
  \end{align*}
  We will use the established result that if $f$ and $g_1,\dots,g_m$ are total functions, then $h$ is total.\newline

  In primitive recursion, we have
  \begin{align*}
    h\left(x_1,\dots,x_n,y+1\right) &= g\left(x_1,\dots,x_n,y,h\left(x_1,\dots,x_n,y\right)\right)\\
    h\left(x_1,\dots,x_n,0\right) &= f\left(x_1,\dots,x_n\right).
  \end{align*}
  We will use the established result that if $f,g$ are total, then $h$ is total.\newline

  Additionally, we will use the established result that $S,C_0,P_{i}^{(k)}$ are total.\newline

  We say $h$ has depth $n$ if $h$ is obtained by one application of composition or primitive recursion with one or more functions of depth less than or equal to $n-1$, one of which has depth exactly equal to $n-1$. We say $h$ has depth $0$ if $h = S,C_0,P_{i}^{(k)}$.\newline

  We then use induction on the depth of $h$ to prove the totality of $h$.
  \begin{description}[font=\normalfont]
    \item[$d = 0$:] $h = S,C_0,P_{i}^{(k)}$, so by the previous result, $h$ is total.
    \item[$d = n+1$:] If every function with depth $n$ is total, then for $h$ with depth $n+1$, $h$ is obtained by composition of functions with depth $n$, or 
  \end{description}
\end{solution}
\begin{solution}[Proof 2]
Instead of using depth, we will let
\begin{align*}
  PR_0 &= \set{S,C_0,P_i^{(k)}},
\end{align*}
where $i\leq k\in \N$. We let
\begin{align*}
  PR_{n+1} &= \set{h|h\text{ is obtained by one composition or one primitive recursion from functions in $PR_n$}}\cup PR_n
\end{align*}
Then,
\begin{align*}
  PR &= \bigcup_{i=1}^{\infty} PR_i
\end{align*}
consists of all primitive recursive functions.\newline

We want to show that if $h\in PR_{n}$ is total, then $h\in PR_{n+1}$ is total.
\end{solution}
\section{Extra Problem 2}%
\begin{problem}
\end{problem}
\begin{solution}
  Let $g\left(n,x\right) = f\left(n\right) = A\left(n,n\right)$. Suppose toward contradiction that $f$ is primitive recursive. Then, for all $n,x > M$
  \begin{align*}
    A\left(n,x\right) &> g\left(n,x\right)\\
                      &= f\left(n\right)\\
                      &= A\left(n,n\right).
  \end{align*}
  However, since $x > M$ and $n > M$, we can take $n = x$, meaning $A\left(n,n\right) > A\left(n,n\right)$, which is a contradiction.
\end{solution}

\end{document}
