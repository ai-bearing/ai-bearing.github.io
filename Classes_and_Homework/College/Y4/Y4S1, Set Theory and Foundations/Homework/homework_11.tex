\documentclass[10pt]{mypackage}

% sans serif font:
%\usepackage{cmbright}
%\usepackage{sfmath}
%\usepackage{bbold} %better blackboard bold

%serif font + different blackboard bold for serif font
\usepackage{newpxtext,eulerpx}
\renewcommand*{\mathbb}[1]{\varmathbb{#1}}
\renewcommand*{\hbar}{\hslash}

\pagestyle{fancy} %better headers
\fancyhf{}
\rhead{Avinash Iyer}
\lhead{Set Theory and Foundations of Mathematics: Homework 11}

\setcounter{secnumdepth}{0}

\begin{document}
\RaggedRight
\section{4.11}%
\begin{problem}
  Show that if $A$ is a decidable set, then so is its complement. Then, show that if $A$ and $B$ are decidable sets, then so are $A\cup B$ and $A\cap B$.
\end{problem}
\begin{solution}
  Let $f_A$ be the function that computes $\chi_A$, and let $f_B$ be the function that computes $\chi_B$.\newline

  We define $g_A$, which computes $\N\setminus A$, by composing $f_A$ with the partial function that computes $1$ if the input is $0$ and computes $0$ if the input is $1$.\newline

  To define $f_{A\cup B}$ and $f_{A\cap B}$, we take
  \begin{align*}
    f_{A\cup B} &= f_A + f_B - \left(f_A\right)\left(f_B\right)\\
    f_{A\cap B} &= \left(f_A\right)\left(f_B\right),
  \end{align*}
  in which we use the multiplication and addition operations composed with $f_A$ and $f_B$.
\end{solution}
\section{Extra Problem 1}%
\begin{problem}
  Give an example of a relation that is not computable.
\end{problem}
\begin{solution}
  Let $\set{T_m}_{m\in\N}$ be a denumeration of the set of all Turing machines with one input. We define the relation $R\subseteq \N\times \N$ with the membership $\left(m,n\right)\in R$ if and only if $T_{m}\left(n\right)$ halts.\newline

  Since it is not possible to compute the halting problem, we know that the relation $R$ is not computable.
\end{solution}
\section{Extra Problem 2}%
\begin{problem}
  Suppose $R,S,T$ are relations with $\left(a,b\right)\in T$ if and only if $\left(a,b\right)\in R$ or $\left(a,b\right)\in S$.
  \begin{enumerate}[(a)]
    \item Prove or disprove: if $R$ and $S$ are computable, then $T$ is computable.
    \item Prove or disprove: if $T$ and $S$ are computable, then $R$ is computable.
  \end{enumerate}
\end{problem}
\begin{solution}\hfill
  \begin{enumerate}[(a)]
    \item We can define a computation of $T$ by saying
      \begin{align*}
        T\left(a,b\right) &= R\left(a,b\right) + S\left(a,b\right) - M\left(S(a,b),R(a,b)\right);
      \end{align*}
      which is computable as addition and multiplication are computable.
    \item Consider $R$ as the halting evaluation --- that is, $\left(a,b\right)\in R$ if and only if $T_{a}(b)$ halts. If we let $S = \N\times \N$, then
      \begin{align*}
        T &= R\cup S\\
          &= R\cup \N\times \N\\
          &= \N\times \N,
      \end{align*}
      meaning $T$ is computable, $S$ is computable, but $R$ is not computable.
  \end{enumerate}
\end{solution}

\section{Extra Problem 3}%
\begin{problem}
  Prove that if $R\subseteq \N\times \N$ is computable, then so too is $\N\times \N\setminus R$.
\end{problem}
\section{Extra Problem 4}%
\begin{problem}
  Show that the relation $a|b$ is primitive recursive.
\end{problem}
\begin{solution}
  The relation $a|b$ means there exists $k$ such that $ak = b$.\newline

  So, we know that $a|b$ if and only if
  \begin{align*}
    \left(\left(0\right)\left(a\right) = b\right) \vee \left(\left(1\right)\left(a\right) = b\right) \vee \cdots \vee \left(ba = b\right).
  \end{align*}
  We then evaluate the product
  \begin{align*}
    d\left(a,b\right) &= \prod_{i=0}^{b}\left(1\dot{-}E(b\dot{-}M(i,a),0)\right)
  \end{align*}
  to find the truth value of the statement.
\end{solution}
\section{Extra Problem 5}%
\begin{problem}
  Define a function $\pi(n)$ by $\pi(n) = 1$ if $n$ is prime and $0$ otherwise. Use minimalization to show that $\pi$ is computable.
\end{problem}
\begin{solution}
  We define
  \begin{align*}
    \pi(n) &= \min_{z}\set{}
  \end{align*}
  
\end{solution}

\section{Extra Problem 7}%
\begin{problem}
  Let
  \begin{align*}
    \pi(n) &= \begin{cases}
      1 & \text{$n$ prime}\\
      0 & \text{else}
    \end{cases}.
  \end{align*}
  Show that $\pi$ is primitive recursive.
\end{problem}
\begin{solution}
  We know that $n$ is prime if and only if the only $k$ with $k > 1$ and $k | n$ is $k=n$.\newline

  Take
  \begin{align*}
    \pi\left(n\right) &= \left(1\dot{-}D\left(2,n\right)\right)\left(1\dot{-}D\left(3,n\right)\right)\cdots \left(1\dot{-}D\left(n-1,n\right)\right)\\
                      &= \prod_{i=2}^{n-1}\left(1\dot{-}D\left(i,n\right)\right).
  \end{align*}
  
\end{solution}

\end{document}
