\documentclass[10pt]{mypackage}

% sans serif font:
%\usepackage{cmbright}
%\usepackage{sfmath}
%\usepackage{bbold} %better blackboard bold

%serif font + different blackboard bold for serif font
\usepackage{newpxtext,eulerpx}
\renewcommand*{\mathbb}[1]{\varmathbb{#1}}
\renewcommand*{\hbar}{\hslash}

\pagestyle{fancy} %better headers
\fancyhf{}
\rhead{Avinash Iyer}
\lhead{Set Theory and Foundations of Mathematics: Homework 5}

\setcounter{secnumdepth}{0}

\begin{document}
\RaggedRight
\section{2.6}%
\begin{problem}
  Suppose $X = \set{\alpha\mid \alpha\text{ is an ordinal}}$ were a set. Show that it would follow that $X$ is transitive and well-ordered by $\in$.
\end{problem}
\begin{solution}
  Let $s\in t$ and $t\in X$. Then, $t = \alpha$ for some ordinal $\alpha$, meaning $s\in \alpha$. By the definition of an ordinal, this implies $s$ is an ordinal, implying $s\in X$.\newline

  It is necessarily the case that $X$ is (strictly) totally ordered, since for any two distinct ordinals $\alpha$ and $\beta$, either $\alpha \in \beta$ or $\beta \in \alpha$. Let $A\subseteq X$ be nonempty. Suppose toward contradiction that $A$ did not have a least element. Since $A$ is a subset of a totally ordered set, $A$ is totally ordered, meaning that for any $\alpha \in A$, there is a $\beta$ such that $\beta \in \alpha$, and so on. In particular, if it were not the case that $A$ had a least element, there would be an infinite descending membership chain, which is a violation of the axiom of regularity.
\end{solution}
\section{2.7}%
\begin{problem}
  Suppose $\alpha$ is an ordinal. Show that $\alpha \cup \set{\alpha}$ is an ordinal.
\end{problem}
\begin{solution}
  We will start by showing that $\alpha \cup \set{\alpha}$ is transitive with respect to $\in$. Let $s\in t$ and $t\in \alpha \cup \alpha$. Since $\alpha\cap \set{\alpha} = \emptyset$, it is the case that $t \in \alpha$ or $t\in \set{\alpha}$.\newline

  If $t\in \alpha$, then $t$ is also an ordinal, and since ordinals are transitive with respect to $\in$, $s\in \alpha$, so $s\in \alpha \cup \set{\alpha}$. If $t\in \set{\alpha}$, then $t = \alpha$, meaning $s \in \alpha$, so $s\in \alpha \cup \set{\alpha}$.\newline

  Let $A\subseteq \alpha \cup \set{\alpha}$ be nonempty. If $A\cap \alpha = \emptyset$, then $A \subseteq \set{\alpha}$, meaning $A = \alpha$, and $A$ has the least element of $\alpha$. Else, we define the least element of $A$ by taking the least element of $A\cap \alpha$; since $A\cap \alpha$ is a nonempty subset of $\alpha$, $A\cap \alpha$ has a least element since ordinals are well-ordered.
\end{solution}
\section{2.8}%
\begin{problem}
  Let $\alpha$ and $\beta$ be ordinals, and let $S = \set{(0,x)\mid x\in \alpha}$ and $T = \set{(1,x)\mid x\in \beta}$. Define an ordering on $S\cup T$ by taking $(m,n) < (n,y)$ if $m < n$ or if $m = n$ and $x < y$. Show that this is a well-ordering of $S\cup T$.
\end{problem}
\begin{solution}
  Let $A\subseteq S\cup T$ be nonempty. If $A\cap S = \emptyset$, then $A \subseteq T$ is nonempty, meaning any element of $A$ is of the form $(1,t)$, where $t\in \beta$. By the definition of an ordinal, it is the case that the $A_T = \set{t\in \beta\mid (1,t)\in T}$ contains a least element, which means $A$ has a least element.\newline

  If $A\cap S \neq \emptyset$, then the ordering gives the least element of $A$ to be the least element in $A\cap S$, which exists since $A_S = \set{t\in \alpha\mid (0,t)\in A}$ is a nonempty set of ordinals.\newline

  Thus, this is a well-ordering.
\end{solution}
\section{2.9}%
\begin{problem}
  Let $\alpha$ and $\beta$ be ordinals. We define an ordering on $\alpha \times \beta = \set{(x,y)\mid x\in \alpha,y\in\beta}$ by taking $(x,y) < (t,u)$ if $y < u$ or if $y = u$ and $x < t$. Show this is a well-ordering on $\alpha \times \beta$.
\end{problem}
\begin{solution}
  I don't know how to do this problem.
\end{solution}
\section{Extra Problem 3}%
\begin{problem}\hfill
  \begin{enumerate}[(a)]
    \item If $T$ is $\in$-transitive, then $\bigcup T \subseteq T$.
    \item If $\bigcup T \subseteq T$, then $T$ is $\in$-transitive.
  \end{enumerate}
\end{problem}
\begin{solution}\hfill
  \begin{enumerate}[(a)]
    \item I don't know how to do this problem.
    \item Let $\bigcup T\subseteq T$. Let $s\in T$, $x\in s$. Then, $s\subseteq \bigcup T$, so $x\in \bigcup T$, so $x\in T$.
  \end{enumerate}
\end{solution}
\section{Extra Problem 4}%
\begin{problem}
  Prove that ordinal addition is associative.
\end{problem}
\begin{solution}
  Let $\alpha,\beta,\gamma$ be ordinals.
  \begin{align*}
    \alpha + \beta &\cong \set{0}\times \alpha \cup \set{1}\times \beta
  \end{align*}
  under the lexicographical order.\newline

  Additionally,
  \begin{align*}
    \left(\alpha + \beta\right) + \gamma &\cong \underbrace{\set{0}\times \left(\alpha + \beta\right) \cup \set{1}\times \gamma}_{S},
  \end{align*}
  ordered lexicographically.\newline

  Finally,
  \begin{align*}
    \alpha + \left(\beta + \gamma\right) &\cong \underbrace{\set{0}\times \alpha \cup \set{1}\times \left(\beta + \gamma\right)}_{T}
  \end{align*}
  ordered lexicographically.\newline

  It is enough to show that $S$ is order isomorphic to $T$, since ordinals are unique up to order isomorphism.\newline

  Let $f: S\rightarrow T$. Then, for $x\in S$, we have $x\in \set{0}\times \left(\alpha + \beta\right)$ or $x\in \set{1}\times \gamma$.
  \begin{align*}
    f(x) &= \begin{cases}
      (0,a) & x = (0,a);~\text{for some }a\in \alpha\\
      (1,a) & x = (0,a);~\text{for some }a\in \left(\alpha + \beta\right)\setminus \alpha\\
      \left(1,\beta + c\right) & x = (1,c);~c\in \gamma
    \end{cases}
  \end{align*}
  We need to show that $f$ is well-defined and order-preserving.
\end{solution}
\end{document}
