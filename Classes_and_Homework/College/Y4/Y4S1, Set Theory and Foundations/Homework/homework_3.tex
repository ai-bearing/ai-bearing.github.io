\documentclass[10pt]{mypackage}

% sans serif font:
%\usepackage{cmbright}
%\usepackage{sfmath}
%\usepackage{bbold} %better blackboard bold

%serif font + different blackboard bold for serif font
\usepackage{newpxtext,eulerpx}
\renewcommand*{\mathbb}[1]{\varmathbb{#1}}
\renewcommand*{\hbar}{\hslash}

\pagestyle{fancy} %better headers
\fancyhf{}
\rhead{Avinash Iyer}
\lhead{Set Theory and Foundations of Mathematics: Homework 3}

\setcounter{secnumdepth}{0}

\begin{document}
\RaggedRight
\section{1.15}%
\begin{problem}
  Define $f: \N\times\N\rightarrow\N$ by $f(a,b) = 2^a3^b$. Show that $f$ is injective. Use the Cantor--Schröder--Bernstein theorem to deduce that $\N\times\N$ is countably infinite.
\end{problem}
\begin{solution}
  Suppose $2^{a_1}3^{b_1} = 2^{a_2}3^{b_2}$. By the fundamental theorem of arithmetic, it must be the case that $a_1 = a_2$ and $b_1 = b_2$, meaning $f$ is injective.\newline

  Since we have an injection $g: \N\rightarrow \N\times\N$ with $g(n) = (n,0)$, it is the case that, by the Cantor--Schröder--Bernstein theorem, there exists some bijection between $\N\times\N$ and $\N$, meaning they have the same cardinality.
\end{solution}
\section{1.16}%
\begin{problem}
  Let $A$ be the set of all finite subsets of $\N$. Find injective functions from $\N$ to $A$ and vice versa. Use the Cantor--Schröder--Bernstein theorem to deduce that $A$ is countably infinite. Then, prove that the number of infinite subsets of $\N$ is uncountable.
\end{problem}
\begin{solution}
  There is a simple injection from $\N$ to $A = \mathcal{F}\left(\N\right)$ by taking $f(n) = \set{n}$.\newline

  In the reverse direction, for some $X\in A$, define $X = \set{x_1,\dots,x_n}$ with $x_1 < x_2 < \cdots < x_n $. Let $p_i$ denote the $i$th prime number, and
  \begin{align*}
    f(X) &= \prod_{i=1}^{n}p_i^{x_i}.
  \end{align*}
  Suppose $f(X) = f(Y)$ for some $X,Y\in A$. Then, $X = \set{x_1,\dots,x_m}$ and $y = \set{y_1,\dots,y_n}$. Since $f(X) = f(Y)$, we have
  \begin{align*}
    \prod_{i=1}^{m}p_i^{x_i} &= \prod_{i=1}^{n} p_i^{y_i}.
  \end{align*}
  Suppose toward contradiction that $m\neq n$. Without loss of generality, we have $m > n$, implying that $p_m^{x_m}|f(X) = f(Y)$, meaning $p_m^{x_m}|p_1^{y_1}\cdots p_n^{y_n}$, but $p_m > p_1,\dots,p_n$, which is not possible.\newline

  Thus, we have
  \begin{align*}
    p_1^{x_1}p_2^{x_2}\cdots p_m^{x_m} &= p_1^{y_1}p_2^{y_2}\cdots p_m^{y_m},
  \end{align*}
  which by the fundamental theorem of arithmetic, means $x_i = y_i$ for all $i$.\newline

  Since the set of all subsets of $\N$, $P\left(\N\right)$, is uncountable, and $A = \mathcal{F}\left(\N\right)$ is countable, it is the case that the set of infinite subsets of $\N$, $P\left(\N\right) \setminus \mathcal{F}\left(\N\right)$, is uncountable. To show this, suppose toward contradiction that $P\left(\N\right)\setminus \mathcal{F}\left(\N\right)$ is countable. Then, we would have $\mathcal{F}\left(\N\right)\cup \left(P\left(\N\right)\setminus \mathcal{F}\left(\N\right)\right)$ is a countable union of countable sets, implying $P\left(\N\right)$ is countable, which is a contradiction.
\end{solution}
\section{1.17}%
\begin{problem}
  Let $\R^{\times}$ denote the set of nonzero real numbers. Use the Cantor--Schröder--Bernstein theorem to deduce that $\left\vert \R^{\times} \right\vert = \left\vert \R \right\vert$. Now, try to explicitly define a bijection between the sets.
\end{problem}
\begin{solution}
  The inclusion map $\iota: \R^{\times}\rightarrow \R$ is an injection, implying that $\left\vert \R^{\times} \right\vert\leq \left\vert \R \right\vert$. Additionally, the map $f: \R\rightarrow \R^{\times}$ defined by $f(x) = \arctan\left(x\right) + \pi/2$ is an injection from $\R$ into $\R^{\times}$, meaning $\left\vert \R \right\vert\leq \left\vert \R^{\times} \right\vert$. Thus, by Cantor--Schröder--Bernstein, there is a bijection from $\R$ to $\R^{\times}$.\newline

  The function
  \begin{align*}
    f: \R \rightarrow \R^{\times}
  \end{align*}
  defined by
  \begin{align*}
    f(x) &= \begin{cases}
      x + 1 & x\in\N\\
      x & x\notin \N
    \end{cases}
  \end{align*}
  is a bijection from $\R$ to $\R^{\times}$.
\end{solution}
%{\color{blue}
%  
%\begin{solution}[Alternative Solution]
%\end{solution}
%}
\section{1.18}%
\begin{problem}
  Let $A = \set{x\in\R\mid 0 < x < 1}$ and $B = \set{x\in\R\mid 0\leq x \leq 1}$. Find injective functions $f: A\rightarrow B$ and $g: B\rightarrow A$, and deduce that $|A| = |B|$. Try to define an explicit bijection between $A$ and $B$.
\end{problem}
\begin{solution}
  The inclusion map $\iota: A\hookrightarrow B$ is an injection between $(0,1)$ and $[0,1]$. Additionally, $g: [0.1]\rightarrow (0,1)$ defined by $g(x) = \frac{1}{3}x + \frac{1}{3}$ is also an injection between $[0,1]$ and $(0,1)$. Thus, by Cantor--Schröder--Bernstein, there is a bijection between $A$ and $B$.\newline

 % For an attempt at a bijection between $[0,1]$ and $(0,1)$, let
 % \begin{align*}
 %   f_n: [0,1]\rightarrow \left[\frac{1}{n},1-\frac{1}{n}\right]
 % \end{align*}
 % be a linear function. Notice that $\lim_{n\rightarrow\infty}\left[\frac{1}{n},1-\frac{1}{n}\right] = (0,1)$. We define $f(x) = \lim_{n\rightarrow\infty}f_n(x)$. I don't know exactly how to prove that this is a bijection, but it should be one.\newline
 We take
 \begin{align*}
   \set{\frac{1}{n}\mid n\geq 2},
 \end{align*}
 and map $\frac{1}{2}$ to $0$, $\frac{1}{3}\mapsto 1$, and $\frac{1}{n+2} \mapsto \frac{1}{n}$ for $n\geq 2$. For $x\notin \set{\frac{1}{n}\mid n\geq 2}$, we map $x\mapsto x$. This yields a bijection from $\left(0,1\right)$ to $\left[0,1\right]$.
\end{solution}
{
  \color{blue}
  \begin{solution}[Alternative using Chasing]
    We let $[0,1]$ be the set of dogs and $(0,1)$ be the set of cats, with $f(x) = x$ mapping $(0,1)$ into $[0,1]$, and $g(x) = \frac{1}{3} + \frac{1}{3}x$ mapping $[0,1]$ into $(0,1)$.\newline

    The first dog-sequence maps
    \begin{align*}
      g(0) &= \frac{1}{3}\\
      g\left(\frac{1}{3}\right) &= \frac{1}{3} + \frac{1}{3}\left(\frac{1}{3}\right)\\
      g\left(\frac{1}{3} + \frac{1}{3^2}\right) &= \underbrace{\frac{1}{3} + \frac{1}{3^2} + \frac{1}{3^3}}_{\sum_{i=1}^{3}\frac{1}{3^i}}.
    \end{align*}
    Inductively, we have $\underbrace{g\circ\cdots\circ g}_{n\text{ times}}(0)$ is
    \begin{align*}
      g^{n}\left(0\right) &= \sum_{i=0}^{n}\frac{1}{3^i}.
    \end{align*}
    For some cat $c\in (0,1)$, we have
    \begin{align*}
      h(c) &= \begin{cases}
        f(c) & \text{otherwise}\\
        g^{-1}\left(c\right) & \text{$x$ is in a dog sequence}
      \end{cases}.
    \end{align*}
    In particular, our dog-sequence elements are the ones that are of the form
    \begin{align*}
      \sum_{i=0}^{n}\frac{1}{3^i} &= \frac{1}{2}\left(1-\frac{1}{3^i}\right)
    \end{align*}
    for $n\in\N$, and the corresponding sequence that starts with $g(1)$.
  \end{solution}
}
\section{1.19}%
\begin{problem}
  Let $S = \set{s_1,\dots,s_n}$ be a nonempty set of finitely many symbols. Show that the number of finite strings consisting of elements of $S$ is countably infinite. What happens if $S$ is countably infinite?
\end{problem}
\begin{solution}
  We let $S_i$ be the set of strings of length $i$; there are $n^i$ elements of $S_i$, which is finite. The set of all finite strings in $S$ is
  \begin{align*}
    \bigcup_{i=1}^{\infty}S_i.
  \end{align*}
  Since the set $S_i$ are disjoint, it is the case that the set of all finite strings in $S$ is a countably infinite union of finite disjoint sets, which is countably infinite.\newline

  If $S$ is countably infinite, then by ordering the finite subsets of $S$ by length and lexicographical order, we find that the set of finite subsets of $S$ is countably infinite.
\end{solution}
\section{1.20}%
\begin{problem}
  The two questions below refer to Hilbert's Hotel, discussed at the end of the chapter.
  \begin{enumerate}[(a)]
    \item A fleet of countably infinite busses arives with countably infinite passengers. Describe a way to assign rooms to everyone, including those currently in the hotel, such that no rooms are left empty.
    \item There are now a countably infinite number of fleets of countably infinite buses with a countably infinite number of people. Find a way for the desk attendant to accommodate all guests.
  \end{enumerate}
\end{problem}
\begin{solution}\hfill
  \begin{enumerate}[(a)]
    \item Move every current resident of the hotel to $2$ multiplied by their current room number. Use the Cantor pairing function to map $\N\times \N$ to map each of the countably infinite busses' countably infinite members to $\N$. Then, for each new resident, multiply their room number by $2$ and add $1$.
    \item Proceeding in a similar manner, we can compose the Cantor pairing function with itself to create a bijection from $\N\times\N\times \N$ to $\N$, then multiply by $2$ and add $1$ to map every new resident to an odd room, while mapping every current resident to an even room.
  \end{enumerate}
\end{solution}
\end{document}
