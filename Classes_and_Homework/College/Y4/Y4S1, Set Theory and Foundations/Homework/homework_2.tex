\documentclass[10pt]{mypackage}

% sans serif font:
%\usepackage{cmbright}
%\usepackage{sfmath}
%\usepackage{bbold} %better blackboard bold

%serif font + different blackboard bold for serif font
\usepackage{newpxtext,eulerpx}
\renewcommand*{\mathbb}[1]{\varmathbb{#1}}

\pagestyle{fancy} %better headers
\fancyhf{}
\rhead{Avinash Iyer}
\lhead{Set Theory and Foundations of Mathematics: Homework 2}

\setcounter{secnumdepth}{0}

\begin{document}
\RaggedRight
\section{1.8}%
\begin{problem}
  Fix a natural number $b\geq 2$. Show that every positive real number in $x$ in $[0,1]$ has a $b$-adic expansion of the form
  \begin{align*}
    x &= \sum_{n=1}^{\infty}\frac{x_n}{b^n},
  \end{align*}
  with each $0\leq x_n \leq b-1$.
\end{problem}
\begin{solution}
  I don't know how to do this problem.
\end{solution}
\section{1.9}%
\begin{problem}
  Suppose
  \begin{align*}
    \sum_{n=1}^{\infty}\frac{x_n}{b^n} &= \sum_{n=1}^{\infty}\frac{y_n}{b^n},
  \end{align*}
  with $0 \leq x_n \leq b-1$ and $0 \leq y_n \leq b-1$ integers. Show that either $x_n = y_n$ for all n, or there is an $m$ such that one of the following two cases occurs:
  \begin{itemize}
    \item $x_m = y_m + 1$ and for $n\geq m+1$, $y_n = b-1$ and $ x_n = 0$;
    \item $y_m = x_m + 1$ and for $n\geq m+1$, $x_n = b-1$ and $y_n = 0$.
  \end{itemize}
\end{problem}
\begin{solution}
  I don't know how to do this problem.
\end{solution}
\section{1.10}%
\begin{problem}
  Show that a number $x\in [0,1]$ is rational if and only if its decimal expansion is eventually periodic. Deduce that irrational numbers have unique decimal expansions.
\end{problem}
\begin{solution}
  Let $x$ be rational. Then, $x = \frac{p}{q}$, with $p\in \Z_{\geq0}$, $q\in \Z_{> 0}$, with $\frac{p}{q}$ in lowest terms, with $q > p$.\newline

  We write $10x = x_1 + y_1$, with $x_1 = \left\lfloor 10x\right\rfloor$ and $y_1 = 10x - \left\lfloor 10x\right\rfloor$. Thus, we have
  \begin{align*}
    y_1 &= \frac{10p}{q} - \frac{qx_1}{q}\\
        &= \frac{10p - qx_1}{q}\\
        &= \frac{m_1}{q}.
  \end{align*}
  We want to show that $0 \leq m_1 < q$.\newline

  Now, we take $10y_1 = x_2 + y_2$, with
  \begin{align*}
    y_2 &= \frac{10 m_1}{q} - \frac{qx_2}{q}\\
        &= \frac{m_2}{q}.
  \end{align*}
  Repeatedly, we get $y_n = \frac{m_n}{q}$.\newline

  We have $0 \leq x_i < 10$, and $0 \leq m_i < q$. Thus, looking at the set of pairs $\left(x_1,m_1\right),\left(x_2,m_2\right),\dots$. Since $x_i$ and $m_i$ are limited, there cannot be infinitely many distinct pairs; thus, there will necessarily be a value of $n$ such that $\left(x_k,m_k\right) = \left(x_{k+n},m_{k+n}\right)$.
\end{solution}
\section{1.11}%
\begin{problem}
  Show that the collection of polynomials with rational coefficients is a countably infinite set.
\end{problem}
\begin{solution}
  Let $\mathcal{P}_n\left(\Q\right)$ denote the set of polynomials with degree $n$ with coefficients in $\Q$. We construct a bijection
  \begin{align*}
    \mathcal{P}_n\left(\Q\right) &\rightarrow \prod_{k=0}^{n}\Q,
  \end{align*}
  where $\prod$ denotes the Cartesian product, by taking
  \begin{align*}
    a_0 + a_1x + \cdots + a_nx^n \mapsto \left(a_0,a_1,\dots,a_n\right).
  \end{align*}
  Since $\prod_{k=0}^{n}\Q$ is a countable Cartesian product of countable sets, this means $\mathcal{P}_n\left(\Q\right)$ is countable.\newline

  Finally, we have $\Q[x]$, the set of all polynomials with rational coefficients, is
  \begin{align*}
    \Q[x] &= \bigcup_{k=0}^{\infty}\mathcal{P}_k\left(\Q\right),
  \end{align*}
  meaning $\Q[x]$ is countable.\newline

  Since $\Q[x]$ is countable, and for any $p(x)\in \Q[x]$, $p(x)$ has at most $\deg\left(p(x)\right)$ roots, it must be the case that the algebraic numbers are countable.
\end{solution}
\section{1.12}%
\begin{problem}
  Show that the collection of infinite sequences made up of the elements $0$ and $1$ is uncountable.
\end{problem}
\begin{solution}
  Let $S$ denote the set of all infinite sequences consisting of the elements $0$ and $1$. Suppose toward contradiction $S$ is countable. In particular, $S$ is infinite (as the subset of sequences consisting of $0$ everywhere except for $1$ at position $n$ is infinite), meaning we are supposing that $S$ is denumerable.\newline

  Let $f: \N\rightarrow S$ be a bijection from $S$ to $\N$, defining $f\left(i\right) = s_i$, where $s_i$ is a sequence. We let $s_{i,j}$ denote the $j$th position of sequence $i$.\newline

  Define a new sequence $a$ by taking
  \begin{align*}
    a_{j} &= \begin{cases}
      0 & s_{j,j} = 1\\
      1 & s_{j,j} = 0
    \end{cases}.
  \end{align*}
  It is then the case that $a\in S$, but $a$ is not in $\img\left(f\right)$. Thus, $f$ cannot be a bijection, meaning $S$ is not countable.
\end{solution}
\section{1.13}%
\begin{problem}
  Show that the number of functions mapping from $\N$ to $\N$ is uncountable.
\end{problem}
\begin{solution}
  Since the set of functions $f: \N\rightarrow \set{0,1}$ is a subset of the set of functions $f:\N\rightarrow \N$, and we have shown that the set of functions $f: \N\rightarrow \set{0,1}$ is uncountable (as a sequence is a function from $\N$ to some codomain), so too is the set of functions $f: \N\rightarrow \N$.
\end{solution}
\section{Extra Problem 1}%
\begin{problem}
  Prove that every infinite subset of a denumerable set is denumerable.
\end{problem}
\begin{solution}
  Let $A$ be a denumerable set, and let $S\subseteq A$ be infinite. We will create a denumeration of $S$.\newline

  Let $f: \N\rightarrow A$ be a bijection, which exists as $A$ is denumerable. We define $a_i = f(i)$ for each $i\in \N$.\newline

  It is then the case that $S = \set{a_{i_j}}$ for some $\set{i_j}_{j}\subseteq \N$, with $\set{i_j}$ infinite. Define $s_0$ to be $a_{i_0}$, where $i_0$ denotes the least element in $\set{i_j}_{j}$. It is the case that $i_0$ exists by the well-ordering principle. We then define $s_1 = a_{i_1}$, where $i_1$ is the least element in $\set{i_j}_{j}\setminus \set{i_0}$. Repeatedly, we define $s_n = a_{i_n}$, where $i_n$ is the least element in $\set{i_j}_{j}\setminus \set{i_0,\dots,i_{n-1}}$.\newline

  Finally, we have the bijection $g: S\rightarrow \N$ defined by $g\left(s_i\right) = i$, meaning $S$ is denumerable.
\end{solution}
\section{Extra Problem 2}%
\begin{problem}
  If $|A| \leq |B|$, then $|P(A)| \leq |P(B)|$.
\end{problem}
\begin{solution}
  Let $f: A\hookrightarrow B$ be an injection. Given $S\subseteq A$, we have $f(S) \subseteq B$, meaning $S\in P(A)$ implies $f(S)\in P(B)$. We let $g: P(A)\rightarrow P(B)$ be induced by $f$, with 
  \begin{align*}
    g(S) &= f(S)\\
         &= \set{f(x)\mid x\in S}.
  \end{align*}
\end{solution}
\section{Extra Problem 3}%
\begin{problem}
  If $|A| = |B|$, then $|P(A)| = |P(B)|$.
\end{problem}
\begin{solution}
  Let $f: A\rightarrow B$ be a bijection. Given $S\subseteq A$, we know that $f(S)\subseteq B$, meaning $S\in P\left(A\right)$ and $f(S)\in P\left(B\right)$. We define $g: P(A) \rightarrow P(B)$ to be induced by $f$ as follows:
  \begin{align*}
    g(S) &= \set{f(x)\mid x\in S}.
  \end{align*}
  Then, $g$ is a bijection, as $f$ is a bijection.
\end{solution}
\end{document}
