\documentclass[11pt]{mypackage}

% sans serif font:
%\usepackage{cmbright}
%\usepackage{sfmath}
%\usepackage{bbold} %better blackboard bold

%serif font + different blackboard bold for serif font
\usepackage{newpxtext,eulerpx}
\renewcommand*{\mathbb}[1]{\varmathbb{#1}}

\pagestyle{fancy}
\fancyhf{}
\rhead{Avinash Iyer}
\lhead{Set Theory and Foundations of Mathematics: Homework 1}

\setcounter{secnumdepth}{0}

\begin{document}
\RaggedRight
\section{Preliminary Statements}%
\begin{theorem}[Definition of Countability]
  A set $S$ is countable if and only if there exists an injection $f: S\hookrightarrow \N$.
\end{theorem}
\begin{proof}
  Let $S$ be countable.
  \begin{description}
    \item[Case 1:] We have $S$ is finite if and only if there is a map $f: S\rightarrow \set{1,2,\dots,n}$, where $f$ is a bijection. Letting $\iota: \set{1,2,\dots,n} \rightarrow \N$ be defined by $\iota(n) = n$, it is clear that $\iota$ is an injection.\newline

      Considering the map $\iota\circ f: S\rightarrow \N$, since $\iota$ is injective and $f$ is injective, so too is $\iota\circ f$, meaning our desired injection is $\iota\circ f$.
    \item[Case 2:] By definition, a set $S$ is countably infinite if and only if there exists a bijection $g: S\rightarrow \N$, which is our desired injection.
  \end{description}
\end{proof}
\begin{theorem}[Injection into a Finite Set]
  Let $S$ be a nonempty set. If there exists an injection $S\hookrightarrow \set{1,2,\dots,n}$ for some $n\in \N$, then $S$ is finite.

  %Then, $S$ is finite if and only if there exists an injection $S\hookrightarrow \set{1,2,\dots,n}$ for some $n\in \N$.
\end{theorem}
\begin{proof}
  We begin by showing the reverse direction.\newline

  Let $\sigma: S\hookrightarrow \set{1,2,\dots,n}$ be an injection for some $n\in \N$. Define $s_i$ by $\sigma\left(s_i\right) = i$ for $i\in \text{im}\left(\sigma\right)$.\newline

  Notice that $\sigma': S\rightarrow \sigma(S)$ is a bijection, since $\sigma$ is injective and any map of the form $f: A\rightarrow f(A)$ is surjective by definition.\newline

  We define $r: \sigma(S) \hookrightarrow \N$ selecting $i_1$ to be the least element in $\sigma(S)$ (which exists by the well-ordering principle since $\set{1,2,\dots,n}\subseteq \N$ is nonempty), and mapping $r\left(i_1\right)= 1$. Similarly, we inductively select $i_k$ to be the least element in $\sigma\left(S\right) \setminus \set{i_1,i_2,\dots,i_{k-1}}$, and map $r\left(i_k\right) = k$. From this construction, it is clear that $r$ is injective.\newline

  Then, defining $r': \sigma\left(S\right)\rightarrow r\left(\sigma(S)\right)$, we can see that $r'$ is a bijection, with $r\left(\sigma(S)\right) = \set{1,2,\dots,j}$ for some $j\leq n$ (since, by definition, $\sigma$ is an injection, meaning $\sigma\left(s_i\right) \leq n$ for all $n$).\newline

  Taking $r'\circ \sigma': S\rightarrow \set{1,2,\dots,j}$, we see that this is a composition of bijections, meaning it is a bijection. Thus, $S$ is finite.\newline

  In the forward direction, we can see that if $S$ is finite, then the bijection $h: S\rightarrow \set{1,2,\dots,n}$ is an injection, and we are done.
\end{proof}
%\begin{theorem}[Definition of Infinite]
%  Let $A$ be a set. Then, $A$ is not finite if and only if there exists an injection $f: \N\hookrightarrow A$.
%\end{theorem}
%\begin{proof}
%
%\end{proof}
\section{1.1}%
\begin{problem}
  Show that the function $f: \N\rightarrow \Z$ given by
  \begin{align*}
    f(n) &= \left(-1\right)^{n+1}\left\lfloor \frac{n+1}{2}\right\rfloor
  \end{align*}
  is a bijection.
\end{problem}
\begin{solution}
  We begin by showing that $f$ is injective. Let $f\left(n_1\right) = f\left(n_2\right)$. Then, we have two cases: one if $f\left(n_1\right)$ and $f\left(n_2\right)$ are positive, and one if $f\left(n_1\right)$ and $f\left(n_2\right)$ are negative. In the either case, we have
  \begin{align*}
    f\left(n_1\right) &= \left(-1\right)^{n_1 + 1}\left\lfloor \frac{n_1 + 1}{2}\right\rfloor,\\
    f\left(n_2\right) &= \left(-1\right)^{n_2 + 1}\left\lfloor \frac{n_2 + 1}{2}\right\rfloor,
  \end{align*}
  meaning 
  \begin{align*}
    \left\lfloor \frac{n_1 + 1}{2}\right\rfloor = \left\lfloor \frac{n_2 + 1}{2}\right\rfloor.
  \end{align*}
  If $f\left(n_1\right)$ and $f\left(n_2\right)$ are positive, this implies that $n_1$ and $n_2$ are odd (so that $n_1 + 1$, $ n_2+ 1$ are even). Since $n_1 + 1$ and $n_2 + 1$ are even, this implies
  \begin{align*}
    \left\lfloor\frac{n_1 + 1}{2}\right\rfloor &= \frac{n_1 + 1}{2}\\
    \left\lfloor \frac{n_2 + 1}{2}\right\rfloor &= \frac{n_2 + 1}{2},
  \end{align*}
  meaning $n_1 = n_2$.\newline

  If $f\left(n_1\right)$ and $f\left(n_2\right)$ are odd, this implies that $n_1$ and $n_2$ are even, so
  \begin{align*}
    \left\lfloor\frac{n_1 + 1}{2}\right\rfloor &= \frac{n_1 }{2}\\
    \left\lfloor \frac{n_2 + 1}{2}\right\rfloor &= \frac{n_2 }{2},
  \end{align*}
  once again implying that $n_1 = n_2$.\newline

  To show surjectivity, let $z\in \Z$. Suppose $z < 0$. Then, we find $n\in \N$ by taking $n = -2z$. If $z > 0$, we take $n = 2z - 1$, and if $z = 0$, we take $n = 0$.
\end{solution}
\section{1.2}%
\begin{problem}
  Given bijections $f: \N\rightarrow \Z$ and $P: \N\times\N\rightarrow \N$, show that the function $h: \Z\times \Z \rightarrow \N$ defined by $h(x,y) = P\left(f^{-1}\left(x\right),f^{-1}\left(y\right)\right)$ is bijective.
\end{problem}
\begin{solution}
  We begin by showing injectivity. Since $f$ is bijective, so too is $f^{-1}$, meaning that for
  \begin{align*}
    h\left(x,y\right) &= h\left(x',y'\right),
  \end{align*}
  we have
  \begin{align*}
    P\left(f^{-1}(x),f^{-1}(y)\right) &= P\left(f^{-1}\left(x'\right),f^{-1}\left(y'\right)\right)\\
    f^{-1}(x) &= f^{-1}\left(x'\right)\\
    f^{-1}\left(y\right) &= f^{-1}\left(y'\right) \tag*{since $P$ is bijective}\\
    \intertext{meaning}
    x &= x'\\
    y &= y'\tag*{since $f^{-1}$ is bijective.}
  \end{align*}
  Thus, $h$ is injective.\newline

  Let $n\in \N$. Since $P$ is surjective, there exist $a,b$ such that $P\left(a,b\right) = n$. Since $f^{-1}$ is surjective, there exists $x,y\in \Z$ such that $f^{-1}\left(x\right) = a$ and $f^{-1}\left(y\right) = b$. Thus, there exist $x,y\in \Z$ such that $h\left(x,y\right) = n$.
\end{solution}
\section{1.3}%
\begin{problem}
  If $A$ and $B$ are countably infinite, show that $A\times B$ is countably infinite.%\footnote{Assuming the axiom of choice.}
\end{problem}
\begin{solution}
  By the definition of countably infinite sets, there exist bijections $\alpha: A\rightarrow \N$ and $\beta: B\rightarrow \N$. Additionally, we know that there exists a bijection $P: \N\times \N \rightarrow \N$.\newline

  Define $h: A\times B \rightarrow \N$ by $h(a,b) = P\left(\alpha\left(a\right),\beta\left(b\right)\right)$. Then, since $h$ is a composition of bijections, $h$ is a bijection between $A\times B$ and $\N$.
\end{solution}
\section{1.5}%
\begin{problem}
  If $A_1,A_2,\dots$ is an infinite sequence of (pairwise) disjoint finite sets, show that the union $\bigcup_{n=1}^{\infty}A_n$ is countably infinite.
\end{problem}
\begin{solution}
  For all $i\in \N$, there exists $f_i: A_i\rightarrow \set{1,2,\dots,n_i}$ such that $f_i$ is a bijection, by the definition of finitude.\newline

  Let $x\in \bigcup_{i=1}^{\infty}A_i$. Then, $x\in A_i$ for exactly one value of $i$, since the sets $A_i$ are pairwise disjoint.\newline

  Define
  \begin{align*}
    p(x) &= f_i(x) - 1 + \sum_{j=1}^{i-1}n_j.
  \end{align*}
  Then, $p$ is a bijection, meaning $\bigcup_{i=1}^{\infty}A_i$ is denumerable.
%  Let $\chi_n: A_n \rightarrow \set{1,2,\dots,\alpha_n}$ be the bijection that defines the cardinality of each $A_n$. We define $a_{i,n}\in A_n$ to be the unique element of $a_n$ such that $\chi_{n}\left(a_{i,n}\right) = i$. Let $p_n$ denote the $n$th prime number.\newline
%
%  The function $h: \bigcup_{n=1}^{\infty}A_n \rightarrow \N$ defined by $h\left(a_{i,k}\right) = p_k^{\chi_{k}\left(a_{i,k}\right)}$, where $a_{i,k}\in A_k$, is a well-defined injection, as the $A_k$ are disjoint and primes do not divide each other. Thus, we know that $\bigcup_{n=1}^{\infty}A_n$ is countable (and, since sequence of $A_i$ are infinite, denumerable by Extra Problem 4).
\end{solution}
%{\color{blue}
%\begin{solution}[Proposed]
%  For all $i\in \N$, there exists $f_i: A_i \rightarrow \set{1,2,\dots,n_i}$ such that $f_i$ is a bijection (by the definition of finitude).\newline
%
%  Let $x\in \bigcup_{i=1}^{\infty}A_i$.\newline
%
%  Then, $x\in A_i$ for some $i$, and only one $i$, because the sets $A_i$ are pairwise disjoint.\newline
%
%  Define
%  \begin{align*}
%    p(x) &= f_i(x) - 1 + \sum_{j=1}^{i-1}n_j.
%  \end{align*}
%  Thus, $\bigcup_{i=1}^{\infty}A_i$ is denumerable.
%\end{solution}
%}
\section{1.6}%
\begin{problem}
  If $A_1,A_2,\dots$ is an infinite sequence of disjoint countably infinite sets, show that the union $\bigcup_{n=1}^{\infty} A_n$ is countably infinite.
\end{problem}
\begin{solution}
  We define $\chi_n: A_n\rightarrow \N$ to be bijections that define the cardinality of $A_n$, and let $a_{i,n}\in A_n$ be defined by $\chi_n\left(a_{i,n}\right) = i$. We let $p_n$ denote the $n$th prime number.\newline

  The function $h: \bigcup_{n=1}^{\infty}A_n\rightarrow \N$ defined by $h\left(a_{i,k}\right) = p_k^{\chi_k\left(a_{i,k}\right)}$ is an injection, as each $A_k$ is disjoint and prime numbers do not divide each other. Thus, we know that $\bigcup_{n=1}^{\infty}A_n$ is countable.
\end{solution}
%{
%  \color{blue}
%  \begin{solution}
%    Since $A_i$ is denumerable for all $i\in \N$ , there exists a bijection $f_i: A_i \rightarrow \N$.\newline
%
%    Define $f: \bigcup_{i=1}^{\infty}A_i\rightarrow \N\times \N$ by 
%    \begin{align*}
%      f(x) &= \left(i-1, f_{i}\left(x\right)\right)
%    \end{align*}
%    for $x\in A_i$.\newline
%
%    We know $f$ is well-defined, since $A_i$ are pairwise disjoint, so $x\in A_i$ for exactly one value of $i$.\newline
%
%    To define $g: \bigcup_{i=1}^{\infty}A_i \rightarrow \N$, we take $g = P\circ f$ (where $P$ denotes the Cantor pairing function). We can see that $g$ is a bijection since $P$ and $f$ are bijections.
%  \end{solution}
%}
\section{1.7}%
\begin{problem}
  Construct an explicit polynomial bijection between $\N\times\N\times\N$ and $\N$.
\end{problem}
\begin{solution}
  Let $Q: \N\times\N\times\N \rightarrow \N$ be defined by $Q(x,y,z) = P\left(P(x,y),z\right)$, where $P(x,y) = \frac{(x+y)(x+y+1)}{2} + x$ is a bijection between $\N\times\N$ and $\N$.\newline

  We know that $Q$ is a bijection since it is a composition of bijections. I do not want to expand this expression.
\end{solution}
\section{Extra Problem 1}%
\begin{problem}
  Prove that if $A$ and $B$ are finite sets, then $A\cup B$ is finite.
\end{problem}
%\begin{solution}
%  We have $A\cup B = A\setminus B \cup B\setminus A \cup A\cap B$. Since $A\setminus B\subseteq A$, $B\setminus A\subseteq B$, and $A\cap B\subseteq A$, with all three disjoint, this is a finite disjoint union of finite sets, meaning it is finite.\footnote{In the order of my completing homework, I proved the injection to finite sets, then the subset of a finite set, then this problem.}
%\end{solution}

  \begin{solution}
    We know $A\setminus B \subseteq A$; since $A$ is finite, so too is $A\setminus B$ (by Extra Problem 3).\newline

    Since $A\cup B = \left(A\setminus B\right)\cup B$ is a disjoint union of finite sets, $A\cup B$ is finite.
  \end{solution}
\begin{remark}[Disjoint Union of Finite Sets is Finite]
  Let $A,B$ be disjoint finite sets. Then, $A\cup B$ is finite.\newline

  To prove this, by the definition of finitude, there exist $\alpha: A\rightarrow \set{1,2,\dots,m}$ and $\beta: B\rightarrow \set{1,2,\dots,n}$ bijections for some $m,n\in \N$.\newline

  We can create a new function $f: A\cup B \rightarrow \set{1,2,\dots,m+n}$ by 
  \begin{align*}
    h(x) &= \begin{cases}
      f(x) & x\in A\\
      g(x) + m & x\in B
    \end{cases}.
  \end{align*}
  We can see that $h$ is a well-defined bijection since $A\cap B = \emptyset$.
\end{remark}

%\begin{quote}
%  Prove that if $A$ and $B$ are finite sets, then $A\cup B$ is finite.
%\end{quote}
\section{Extra Problem 2}%
\begin{problem}
  Prove that for every $n\in \N$, every subset of $\set{0,1,\dots,n}$ is finite.
\end{problem}
\begin{solution}
  For any subset $P\subseteq \set{0,1,\dots,n}$, the inclusion map is an injection into $\set{0,1,\dots,n}$; composing the inclusion map with the bijection $a: \set{0,1,\dots,n} \rightarrow \set{1,2,\dots,n+1}$ defined by $a(m) = m+1$, we see that there is an injection $a\circ \text{id}: P\hookrightarrow \set{1,2,\dots,n+1}$, meaning $P$ is finite by the theorem above.
\end{solution}
%\begin{quote}
%  Prove that for every $n\in \N$, every subset of $\set{0,1,\dots,n}$ is finite.
%\end{quote}

\section{Extra Problem 3}%
\begin{problem}
  Prove that every subset of a finite set is finite.
\end{problem}
%\begin{quote}
%  Prove that every subset of a finite set is finite.
%\end{quote}
\begin{solution}
  Since every empty set is finite, so too is every subset of the empty set. Similarly, any empty subset of a given finite set is also finite.\newline

  Let $A$ be a nonempty finite set. Then, there exists a bijection $\alpha: A\rightarrow \set{1,2,\dots,n}$ for some $n\in\N$.\newline

  Let $B\subseteq A$ be nonempty. The inclusion map $\iota: B\hookrightarrow A$ defined by $\iota(x) = x$ is an injection.\newline

  Thus, $\alpha \circ \text{id}: B\hookrightarrow \set{1,2,\dots,n}$ is an injection, as it is a composition of injections. By the established theorem above, this means $B$ is finite.
\end{solution}
\section{Extra Problem 4}%
\begin{problem}
  Prove that every infinite subset of $\N$ is denumerable.
\end{problem}
\begin{solution}
  Let $A\subseteq \N$ be infinite.\newline

  Since $A$ is nonempty, by the well-ordering principle, there must exist a least element of $A$, which we label as $a_0$.\newline

  Consider $A\setminus \set{a_0}$. Since $A$ is infinite, $A\setminus \set{a_0}$ must also be infinite, meaning there is a least element of $A\setminus \set{a_0}$ by the well-ordering principle. We label this element as $\set{a_1}$.\newline

  Now, we consider $A\setminus \set{a_0,a_1}$, and use the well-ordering principle to extract $a_2$, and inductively extract $a_i$ by using the well ordering principle on $A\setminus \set{a_0,a_1,\dots,a_{i-1}}$.\newline

  The function $f: A\rightarrow \N$ defined by $f\left(a_i\right) = i$ is a bijection, since $f\left(a_i\right) = f\left(a_j\right)$ if and only if $i = j$.\newline

   Thus, $f$ is a denumeration of $A$.
\end{solution}
\end{document}
