
\documentclass[10pt]{mypackage}

% sans serif font:
%\usepackage{cmbright}
%\usepackage{sfmath}
%\usepackage{bbold} %better blackboard bold

%serif font + different blackboard bold for serif font
\usepackage{newpxtext,eulerpx}
\renewcommand*{\mathbb}[1]{\varmathbb{#1}}

\fancyhf{}
\rhead{Avinash Iyer}
\lhead{Set Theory and Foundations of Mathematics: Homework 1}

\setcounter{secnumdepth}{0}

\begin{document}
\section{Preliminary Statements}%
\begin{theorem}[Definition of Countability]
  A set $S$ is countable if and only if there exists an injection $f: S\hookrightarrow \N$.
\end{theorem}
\begin{proof}
  Let $S$ be countable.
  \begin{description}
    \item[Case 1:] We have $S$ is finite if and only if there is a map $f: S\rightarrow \set{1,2,\dots,n}$, where $f$ is a bijection. Letting $\text{id}: \set{1,2,\dots,n} \rightarrow \N$ be defined by $\text{id}(n) = n$, it is clear that $\text{id}$ is an injection.\newline

      Considering the map $\text{id}\circ f: S\rightarrow \N$, since $\text{id}$ is injective and $f$ is injective, so too is $\text{id}\circ f$, meaning our desired injection is $\text{id}\circ f$.
    \item[Case 2:] By definition, a set $S$ is countably infinite if and only if there exists a bijection $g: S\rightarrow \N$, which is our desired injection.
  \end{description}
\end{proof}
%\begin{theorem}[Definition of Infinite]
%  Let $A$ be a set. Then, $A$ is not finite if and only if there exists an injection $f: \N\hookrightarrow A$.
%\end{theorem}
%\begin{proof}
%
%\end{proof}
\section{1.1}%
\section{1.2}%
\begin{problem}
  Given bijections $f: \N\rightarrow \Z$ and $P: \N\times\N\rightarrow \N$, show that the function $h: \Z\times \Z \rightarrow \N$ defined by $h(x,y) = P\left(f^{-1}\left(x\right),f^{-1}\left(y\right)\right)$ is bijective.
\end{problem}
\begin{solution}
  We begin by showing injectivity. Since $f$ is bijective, so too is $f^{-1}$, meaning that for
  \begin{align*}
    h\left(x,y\right) &= h\left(x',y'\right),
  \end{align*}
  we have
  \begin{align*}
    P\left(f^{-1}(x),f^{-1}(y)\right) &= P\left(f^{-1}\left(x'\right),f^{-1}\left(y'\right)\right)\\
    f^{-1}(x) &= f^{-1}\left(x'\right)\\
    f^{-1}\left(y\right) &= f^{-1}\left(y'\right) \tag*{since $P$ is bijective}\\
    \intertext{meaning}
    x &= x'\\
    y &= y'\tag*{since $f^{-1}$ is bijective.}
  \end{align*}
  Thus, $h$ is injective.\newline

  Let $n\in \N$. Since $P$ is surjective, there exist $a,b$ such that $P\left(a,b\right) = n$. Since $f^{-1}$ is surjective, there exists $x,y\in \Z$ such that $f^{-1}\left(x\right) = a$ and $f^{-1}\left(y\right) = b$. Thus, there exist $x,y\in \Z$ such that $h\left(x,y\right) = n$.
\end{solution}
\section{1.3}%
\begin{problem}
  If $A$ and $B$ are countably infinite, show that $A\times B$ is countably infinite.\footnote{Assuming the axiom of choice.}
\end{problem}
\begin{solution}
  By the definition of countably infinite sets, there exist bijections $\alpha: A\rightarrow \N$ and $\beta: B\rightarrow \N$. Additionally, we know that there exists a bijection $P: \N\times \N \rightarrow \N$.\newline

  Define $h: A\times B \rightarrow \N$ by $h(a,b) = P\left(\alpha\left(a\right),\beta\left(b\right)\right)$. Then, since $h$ is a composition of bijections, $h$ is a bijection between $A\times B$ and $\N$.
\end{solution}
\section{1.5}%
\begin{problem}
  If $A_1,A_2,\dots$ is an infinite sequence of disjoint finite sets, show that the union $\bigcup_{n=1}^{\infty}A_n$ is countably infinite.
\end{problem}
\begin{solution}
  Let $a_n$ be defined by the bijection $\alpha_n: A_n \rightarrow \set{1,2,\dots,a_n}$.
\end{solution}
\section{1.6}%
\section{1.7}%
\begin{problem}
  Construct an explicit polynomial bijection between $\N\times\N\times\N$ and $\N$.
\end{problem}
\begin{solution}
  Let $Q: \N\times\N\times\N \rightarrow \N$ be defined by $Q(x,y,z) = P\left(P(x,y),z\right)$, where $P(x,y) = \frac{(x+y)(x+y+1)}{2} + x$ is a bijection between $\N\times\N$ and $\N$.\newline

  We know that $Q$ is a bijection since it is a composition of bijections. I do not want to expand this expression.
\end{solution}
\section{Extra Problem 1}%
\begin{quote}
  Prove that if $A$ and $B$ are finite sets, then $A\cup B$ is finite.
\end{quote}
\section{Extra Problem 2}%
\begin{quote}
  Prove that for every $n\in \N$, every subset of $\set{0,1,\dots,n}$ is finite.
\end{quote}

\section{Extra Problem 3}%
\begin{quote}
  Prove that every subset of a finite set is finite.
\end{quote}

\section{Extra Problem 4}%
\begin{problem}
  Prove that every infinite subset of $\N$ is denumerable.
\end{problem}
\begin{solution}
  Let $A\subseteq \N$ be infinite.\newline

  Since $A$ is nonempty, by the well-ordering principle, there must exist a least element of $A$, which we label as $a_0$.\newline

  Consider $A\setminus \set{a_0}$. Since $A$ is infinite, $A\setminus \set{a_0}$ must also be infinite, meaning there is a least element of $A\setminus \set{a_0}$ by the well-ordering principle. We label this element as $\set{a_1}$.\newline

  Now, we consider $A\setminus \set{a_0,a_1}$, and use the well-ordering principle to extract $a_2$, and inductively extract $a_i$ by using the well ordering principle on $A\setminus \set{a_0,a_1,\dots,a_{i-1}}$.\newline

  The function $f: A\rightarrow \N$ defined by $f\left(a_i\right) = i$ is a bijection, since $f\left(a_i\right) = f\left(a_j\right)$ if and only if $i = j$.\newline

   Thus, $f$ is a denumeration of $A$.
\end{solution}
\end{document}
