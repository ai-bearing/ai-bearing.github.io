\documentclass[10pt]{mypackage}

% sans serif font:
%\usepackage{cmbright}
%\usepackage{sfmath}
%\usepackage{bbold} %better blackboard bold

%serif font + different blackboard bold for serif font
\usepackage{newpxtext,eulerpx}
\renewcommand*{\mathbb}[1]{\varmathbb{#1}}
\renewcommand*{\hbar}{\hslash}

\pagestyle{fancy} %better headers
\fancyhf{}
\rhead{Avinash Iyer}
\lhead{Set Theory and Foundations of Mathematics: Homework 12}

\setcounter{secnumdepth}{0}

\begin{document}
\RaggedRight
\section{Problem 1}%
\begin{problem}
  Determine whether each of the following statements is true or false. Prove your answers.
  \begin{enumerate}[(1)]
    \item If $A$ is a limit ordinal, then $A+B$ is a limit ordinal.
    \item If $B$ is a limit ordinal, then $A+B$ is a limit ordinal.
    \item If $A+B$ is a limit ordinal, then $A$ is a limit ordinal.
    \item If $A+B$ is a limit ordinal, then $B$ is a limit ordinal.
  \end{enumerate}
\end{problem}
\begin{solution}\hfill
  \begin{enumerate}[(1)]
    \item False --- the ordinal $\omega + 1$ is a successor ordinal to $\omega$, but $\omega$ is a limit ordinal.
    \item True --- by the lexicographical ordering on $A+B$, we must have that any element of $ \set{1}\times B$ is greater than any element of $\set{0}\times A$. Since $B$ is a limit ordinal, it does not have a maximal element (or else it would be a successor ordinal), so $\set{1}\times B$ has no maximal element, so $A+B$ has no maximal element. Thus, $A+B$ is a limit ordinal.
    \item False --- the limit ordinal $\omega$ is equal to $2 + \omega$, but $2$ is not a limit ordinal.
    \item True --- by similar reasoning to (2), we see that there is no maximal element in $A + B$, and by the lexicographical ordering, this means there is no maximal element in $\set{1}\times B\times$, so there is no maximal element in $B$. Thus, $B$ is a limit ordinal.
  \end{enumerate}
\end{solution}
\section{Problem 2}%
\begin{problem}
  Let $A$, $B$, and $C$ be nonzero ordinals. Determine whether each of the following is true or false. Prove your answers.
  \begin{enumerate}[(1)]
    \item $A < A+B$;
    \item $B < A+B$;
    \item if $A < B$, then $A+C < B+C$;
    \item if $A < B$, then $C+A < C+B$.
  \end{enumerate}
\end{problem}
\begin{solution}\hfill
  \begin{enumerate}[(1)]
    \item Since $A\cong \set{0} \times A$ are order isomorphic, and $\set{0}\times A \subsetneq \set{0}\times A \cup \set{1}\times B\cong A+B$, we have $A < A+B$.
    \item Since $B \cong \set{1}\times B$ are order isomorphic, and $\set{1}\times B \subsetneq \set{0}\times A \cup \set{1}\times B\cong A+B$, we have $B < A+B$.
    \item If $A < B$, then $A\subsetneq B$, so $\set{0}\times A\subsetneq \set{0}\times B$, so $\set{0}\times A\cup \set{1}\times C \subsetneq \set{0}\times B \cup \set{1}\times C$, so $A+C < B+C$.
    \item By a similar reasoning, we have $\set{1}\times A\subsetneq \set{1}\times B$, so $\set{0}\times C\cup \set{1}\times A \subsetneq \set{0}\times C \cup \set{1}\times A$, so $C+A < C+B$.
  \end{enumerate}
\end{solution}

\end{document}
