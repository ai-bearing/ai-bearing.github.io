\documentclass[10pt]{mypackage}

% sans serif font:
%\usepackage{cmbright}
%\usepackage{sfmath}
%\usepackage{bbold} %better blackboard bold

%serif font + different blackboard bold for serif font
\usepackage{newpxtext,eulerpx}
\renewcommand*{\mathbb}[1]{\varmathbb{#1}}
\renewcommand*{\hbar}{\hslash}

\pagestyle{fancy} %better headers
\fancyhf{}
\rhead{Avinash Iyer}
\lhead{Set Theory and Foundations of Mathematics: Homework 12}

\setcounter{secnumdepth}{0}

\begin{document}
\RaggedRight
\section{Problem 1}%
\begin{problem}
  Determine whether each of the following statements is true or false. Prove your answers.
  \begin{enumerate}[(a)]
    \item If $A$ is a limit ordinal, then $A+B$ is a limit ordinal.
    \item If $B$ is a limit ordinal, then $A+B$ is a limit ordinal.
    \item If $A+B$ is a limit ordinal, then $A$ is a limit ordinal.
    \item If $A+B$ is a limit ordinal, then $B$ is a limit ordinal.
  \end{enumerate}
\end{problem}
\begin{solution}\hfill
  \begin{enumerate}[(a)]
    \item False --- the ordinal $\omega + 1$ is a successor ordinal to $\omega$, but $\omega$ is a limit ordinal.
    \item True --- we consider $A+B \cong \set{0}\times A \cup \set{1}\times B = S$ with the lexicographical ordering. By a previous result, we know that $B$ is a limit ordinal if and only if $B$ has no maximal element. By the lexicographical ordering, we know that for all $x\in \set{0}\times A$ and $y\in \set{1}\times B$, $x < y$.\newline

      Thus, since $\set{1}\times B\cong B$, we know that $\set{1}\times B$ has no maximal element (\textasteriskcentered).\newline

      Let $t\in S$. If $t\in \set{0}\times A$, then we know that $0\in B$, so $t < (1,0)$. If $t\in \set{1}\times B$, then by (\textasteriskcentered), there is $t'\in \set{1}\times B$ with $t < t'$, so $t'\in S$ and $t < t'$. Thus, $t$ is not a maximal element.\newline

      Thus, $A+B$ has no maximal element, so $A+B$ is a limit ordinal.
    \item False --- the limit ordinal $\omega$ is equal to $2 + \omega$, but $2$ is not a limit ordinal.
    \item True --- by similar reasoning to (a), we see that there is no maximal element in $A + B$, and by the lexicographical ordering, this means there is no maximal element in $\set{1}\times B$, so there is no maximal element in $B$. Thus, $B$ is a limit ordinal.
  \end{enumerate}
\end{solution}
\section{Problem 2}%
\begin{problem}
  Let $A$, $B$, and $C$ be nonzero ordinals. Determine whether each of the following is true or false. Prove your answers.
  \begin{enumerate}[(a)]
    \item $A < A+B$;
    \item $B < A+B$;
    \item if $A < B$, then $A+C < B+C$;
    \item if $A < B$, then $C+A < C+B$.
  \end{enumerate}
\end{problem}
\begin{solution}\hfill
  \begin{enumerate}[(a)]
    \item We know $A\cong \set{0} \times A$ are order isomorphic, and $\set{0}\times A \subsetneq \set{0}\times A \cup \set{1}\times B\cong A+B$. We wish to show that $\set{0}\times A $ is an ``initial segment'' of $\set{0}\times A \cup \set{1}\times B$.
      \begin{definition}
        Let $S$ be a totally ordered set, $x\in S$. We define the initial segment $S_x$ to be
        \begin{align*}
          S_x &= \set{y\in S|y \leq x}.
        \end{align*}
        We say $S_x$ is the initial segment of $S$ less than or equal to $x$.
      \end{definition}
      
    \item Since $\omega \not< 2 + \omega$, this is false.
    \item If $A = 1$ and $B = 2$, then $1 < 2$, but $1 + \omega = \omega \not< 2 + \omega = \omega$.
    \item We know that $\set{0}\times C \cup \set{1}\times A \subsetneq \set{0}\times C \cup \set{1}\times B$. We want to show there exists a sequence
      \begin{align*}
        C+A\xrightarrow[\cong]{f} \set{0}\times C \cup \set{1}\times A = \text{initial segment of }\set{0}\times C \cup \set{1}\times B \xrightarrow[\cong]{g} C+B.
      \end{align*}

  \end{enumerate}
\end{solution}
\section{Problem 3}%
\begin{problem}
  Prove that for all ordinals $A,B,C$, if $C+A = C+B$, then $A=B$.
\end{problem}
\begin{solution}
  Let $A,B,C$ be ordinals, and let $C+A = C+B$. Then, the identity map $\id\colon C+A \rightarrow C+B$ is a bijection, so we have
  \begin{align*}
    \id\bigr\vert_{\set{1}\times A}\colon \set{1}\times A \rightarrow \set{1}\times \set{B}
  \end{align*}
  is a bijection as well, so there is a bijection between $A$ and $B$. Since $A$ and $B$ are ordinals and there is a bijection between $A$ and $B$, this means $A = B$.
\end{solution}
\begin{question}
  True or false? If $\alpha$ and $\beta$ are ordinals, and if $f: \alpha \hookrightarrow \beta$ is injective and preserves order, then $\alpha \leq \beta$.
\end{question}
\section{Problem 4}%
\begin{problem}
  Prove that for every infinite ordinal $A$, there exists a limit ordinal $B$ and a natural number $n$ such that $A = B + n$.
\end{problem}
\begin{solution}
  For infinite ordinals, the principle of induction says that $P(\alpha)$ holds if $P(\omega)$ holds and, we show that if $P(k)$ holds for all $k < \alpha$, then $P(\alpha)$ holds. We will use strong induction to prove this.\newline

  The induction hypothesis states that if $B < A$ and $B$ is infinite, then $B = C+n$ for some limit ordinal $C$ and natural number $n$.\newline

  If $A = \omega$, then $A = A+ 0$.\newline

  If $A$ is a limit ordinal, then $A = A + 0$.\newline

  If $A$ is a successor ordinal, then there exists $\alpha$ such that $A = \alpha \cup \set{\alpha}$ for some ordinal $\alpha$, meaning $A = \alpha + 1$. Since $\alpha$ is an infinite ordinal and $\alpha < A$, $\alpha = C + n$ for some limit ordinal $C$ and natural number $n$. Thus, $A = \alpha + 1 = \left(C+n\right)+1=C + \left(n+1\right)$.
\end{solution}

\end{document}
