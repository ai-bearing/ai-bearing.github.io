\documentclass[10pt]{mypackage}

% sans serif font:
%\usepackage{cmbright}
%\usepackage{sfmath}
%\usepackage{bbold} %better blackboard bold

%serif font + different blackboard bold for serif font
\usepackage{newpxtext,eulerpx}
\renewcommand*{\mathbb}[1]{\varmathbb{#1}}
\renewcommand*{\hbar}{\hslash}

\pagestyle{fancy} %better headers
\fancyhf{}
\rhead{Avinash Iyer}
\lhead{Set Theory and Foundations of Mathematics: Homework 4}

\setcounter{secnumdepth}{0}

\begin{document}
\RaggedRight
\section{2.1}%
\begin{problem}
  \normalsize
  Recall that an ordered pair $(a,b)$ can be defined as the set $\set{\set{a},\set{a,b}}$. Show that $(a,b) = (c,d)$ if and only if $a=c$ and $b=d$
\end{problem}
\begin{solution}
  Let $L = \set{\set{a},\set{a,b}}$ and $R = \set{c,\set{c,d}}$. Suppose $L = R$. Since $\set{a}\in L$, we have $\set{a}\in R$. Thus, $\set{a} = \set{c}$ or $\set{a} = \set{c,d}$.
  \begin{description}[leftmargin=0pt]
    \item[Case 1:] If $\set{a} = \set{c}$, then $a\in \set{c}$, meaning $a = c$.
    \item[Case 2:] If $\set{a} = \set{c,d}$, then $c\in \set{a}$, meaning $c = a$.
  \end{description}
  Since $\set{a,b}\in L$, we have $\set{a,b} \in R$, meaning $\set{a,b} = \set{c}$ or $\set{a,b} = \set{c,d}$.
  \begin{description}
    \item[Case 3:] If $\set{a,b} = \set{c}$, then it must be the case that $\set{a} = \set{c,d}$, meaning $a = b = c = d$, so $b = d$.
    \item[Case 4:] If $\set{a,b} = \set{c,d}$, then it must be the case that $\set{a} = \set{c}$, meaning $a = c$, and thus $b = d$.
  \end{description}
  %If $a = c$, then $b = d$, since $\set{a,b} = \set{c,b} = \set{c,d}$. 
\end{solution}
\section{2.3}%
\begin{problem}
  Show that the replacement schema implies the comprehension schema.
\end{problem}
\begin{solution}
  Let $\psi(u,v) = \phi(v) \wedge u = v$. Then, the replacement schema becomes
  \begin{align*}
    \forall a\: \exists b\:\forall v\left(v\in b \Leftrightarrow \exists u\left(u\in a \wedge \psi(u,v)\right)\right)\\
    \forall a\: \exists b \:\forall v \left(v\in b \Leftrightarrow \exists u \left(u\in a \wedge \forall u\left(\phi(v) \wedge u = v\right)\right)\right)\\
    \forall a\:\exists b \: \forall v \left(v\in b \Leftrightarrow v\in a \wedge \phi(v)\right)
  \end{align*}
\end{solution}
\section{2.4}%
\begin{problem}
  In this question, we show how the pairing axiom follows from the replacement schema. Let sets $a$ and $b$ be given.
  \begin{enumerate}[(a)]
    \item We originally used the pairing axiom to construct the set $\set{\emptyset,\set{\emptyset}}$. Instead, us the power set axiom.
    \item Let $\psi(u,v)$ be the formula
      \begin{align*}
        \left(u=\emptyset \wedge v = a\right)\vee \left(u \neq \emptyset \wedge v = b\right).
      \end{align*}
      Show that this is a function-like formula.
    \item Use the replacement schema on the set $\set{\emptyset,\set{\emptyset}}$ and the function-like formula $\psi\left(u,v\right)$ to show the existence of the set with elements $a$ and $b$.
  \end{enumerate}
\end{problem}
\begin{solution}\hfill
  \begin{enumerate}[(a)]
    \item Consider $\set{\emptyset}$. By the power set axiom, there exists a set $c$ such that $c$ consists of all subsets of $\set{\emptyset}$. Thus, $c = \set{\emptyset,\set{\emptyset}}$.
    \item Let $\psi(u,v) = \left(u = \emptyset \wedge v = a\right)\vee \left(u\neq\emptyset \wedge v = b\right)$. Then, if $\psi\left(u,v\right) = \psi\left(u,w\right) = \text{true}$,
      \begin{align*}
        \left(u = \emptyset \wedge v = a\right) \vee \left(u\neq \emptyset \wedge v = b\right)
      \end{align*}
      and
      \begin{align*}
        \left(u = \emptyset \wedge w = a\right) \vee \left(u\neq \emptyset \wedge w = b\right)
      \end{align*}
      If $v = b$, then $u \neq \emptyset$, implying $w = b$, and similarly, if $v = a$, then $w = a$. Thus, $u = w$.
    \item Using the replacement schema on $\set{\emptyset,\set{\emptyset}}$, we see there is a set $b$ such that for $\emptyset\in \set{\emptyset,\set{\emptyset}}$, $\psi(u,v)$ maps $\emptyset$ to $a$, and for $\set{\emptyset} \in \set{\emptyset,\set{\emptyset}}$, $\psi(u,v)$ maps $\set{\emptyset}$ to $b$.
  \end{enumerate}
\end{solution}
\section{Extra Problem 1}%
\begin{problem}\hfill
  \begin{enumerate}[(a)]
    \item Explain what would go wrong if we defined $(a,b) = \set{a,\set{b}}$.
    \item Can you figure out why the book defines $(a,b) = \set{\set{a},\set{a,b}}$ instead of $\set{a,\set{a,b}}$.
  \end{enumerate}
\end{problem}
\begin{solution}\hfill
  \begin{enumerate}[(a)]
    \item 
    \item If we consider $(a,b) = (a,b)$, we must then have $\set{a,\set{a,b}} = \set{a,\set{a,b}}$, meaning our cases would yield $a\in \set{a,\set{a,b}}$, or $a = \set{a,b}$, implying $a\in a$ or $a\in b$. In particular, for $a\in a$, we get a descending membership chain, which ends up requiring the regularity axiom.
  \end{enumerate}
\end{solution}
\section{Extra Problem 2}%
\begin{problem}
  Let $s$ be a set. Use mathematical symbols exclusively to express $t$, the set of all singleton subsets of $s$.
\end{problem}
\begin{solution}
  \begin{align*}
    \forall s\:\exists t\:\forall x\left(x\in t \Leftrightarrow x\in s \wedge \forall a\:\forall b\left(a\in x \wedge b\in x \Rightarrow a = b\right)\right)
  \end{align*}
\end{solution}
\section{Extra Problem 4}%
\begin{problem}
  Show that if $A$ and $B$ are nonempty sets, then $\bigcap\left(A\cup B\right) = \bigcap A \cup \bigcap B$.
\end{problem}
\begin{solution}
  \begin{align*}
  \bigcap\left(A\cup B\right) &= \forall A \forall B \exists C \: \forall x\left(x\in C \wedge \left(x\in A \vee x\in B\right)\right)\\
                              &= \forall A\forall B\exists C\: \forall x\left(\left(x\in C \wedge x\in A\right) \vee \left(x\in C \wedge x\in B\right)\right)\\
                              &= \bigcap A \cup \bigcap B.
  \end{align*}
\end{solution}
\section{Extra Problem 5}%
\begin{problem}
  Show there exists a set $s$ such that $x\in s$ if and only if $x$ is a natural number.
\end{problem}
\begin{solution}
  \begin{align*}
    \exists s \: \forall x \left(\underbrace{\left(x\in s \wedge x\cup \set{x}\in s\right)}_{\text{Axiom of Infinity}}\wedge \forall y\left(y\in s \Rightarrow \exists z\left(y = z \cup \set{z}\right)\right)\right).
  \end{align*}
\end{solution}
\end{document}
