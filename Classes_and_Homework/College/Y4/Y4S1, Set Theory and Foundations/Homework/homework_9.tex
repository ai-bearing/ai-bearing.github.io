\documentclass[10pt]{mypackage}

% sans serif font:
%\usepackage{cmbright}
%\usepackage{sfmath}
%\usepackage{bbold} %better blackboard bold

%serif font + different blackboard bold for serif font
\usepackage{newpxtext,eulerpx}
\renewcommand*{\mathbb}[1]{\varmathbb{#1}}
\renewcommand*{\hbar}{\hslash}

\pagestyle{fancy} %better headers
\fancyhf{}
\rhead{Avinash Iyer}
\lhead{Set Theory and Foundations: Homework 9}

\setcounter{secnumdepth}{0}

\begin{document}
\RaggedRight
\section{4.1}%
\begin{problem}
  Write a Turing machine that computes the constant function $C_0(n) = 0$.
\end{problem}
\begin{solution}
  The Turing machine with the instructions
  \begin{align*}
    q_1 1 B q_2\\
    q_2 B R q_3\\
    q_3 1 B q_2
  \end{align*}
  computes $C_0(n)$.
  \begin{itemize}
    \item The Turing machine deletes the first $1$, then moves into state $q_2$.
    \item The Turing machine reads the blank, moves right, and enters state $q_3$.
    \item The Turing machine reads the next input in state $q_3$ --- if $1$, then blank, and go back into $q_2$, and if blank, it stops.
  \end{itemize}
\end{solution}
\section{4.2}%
\begin{problem}
  Write a Turing machine that computes the identity function $f(m) = m$ on $\N$. Then write a machine for the identity function on $\N^{n}$.
\end{problem}
\begin{solution}
  The Turing machine with instructions
  \begin{align*}
    q_1 1 B q_2
  \end{align*}
  computes $f(m) = m$.\newline

  The Turing machine with instructions
  \begin{align*}
    q_1 1 B q_2\\
    q_2 B R q_3\\
    q_3 1 R q_3\\
    q_3 B R q_4\\
    q_4 B B q_5\\
    q_4 1 B q_2
  \end{align*}
  computes $\id_{\N^{n}}$.
\end{solution}
\section{4.3}%
\begin{problem}
  Write a Turing machine that computes the function
  \begin{align*}
    Z(n) &= \begin{cases}
      1 & n = 0\\
      0 & n \geq 1
    \end{cases}.
  \end{align*}
\end{problem}
\begin{solution}
  The Turing machine with the instructions
  \begin{align*}
    q_1 1 R q_2\\
    q_2 1 B q_3\\
    q_3 B L q_4\\
    q_4 1 B q_5\\
    q_5 B R q_5\\
    q_5 1 1 q_6\\
    q_6 1 B q_7\\
    q_7 B R q_6
  \end{align*}
  computes $Z(n)$.
\end{solution}
\section{4.4}%
\begin{problem}
  Write a Turing machine that computes $f\left(m,n\right) = \left\vert m-n \right\vert$.
\end{problem}
\begin{solution}
  The Turing machine with the instructions
  \begin{align*}
    q_1 1 B q_2\\
    q_2 B R q_3\\
    q_3 1 R q_3\\
    q_3 B R q_4\\
    q_4 1 R q_5\\
    q_5 1 R q_5\\
    q_5 B L q_6\\
    q_6 1 B q_7\\
    q_7 B L q_8\\
    q_8 1 L q_8\\
    q_8 B L q_9\\
    q_9 1 L q_9\\
    q_9 B R q_1
  \end{align*}
  computes $f\left(m,n\right) = \left\vert m-n \right\vert$. In short, for an input tape
  \begin{align*}
    B,\underbrace{1,1,\dots,1}_{m+1},B,\underbrace{1,1,\dots,1}_{n+1},B
  \end{align*}
  the machine successively deletes the left-most $1$ on the tape denoting $m$ and deletes the right-most $1$ on the tape denoting $n$, until it reaches the end of one of the domain elements (since both $q_1$ and $q_4$ are not defined for blank inputs).
\end{solution}
\section{4.6}%
\begin{problem}
  Show that there is no Turing machine that can determine if a given Turing machine acting no a given input $m$ will yield an output that contains the symbol $s_k$ for a fixed $k\geq 1$.
\end{problem}
\section{Extra Problem 2}%
\begin{problem}
  Write a Turing machine that computes the function $f(n) = 0$ if $n$ is even and $f(n) = 1$ if $n$ is odd.
\end{problem}
\begin{solution}

\end{solution}

\end{document}
