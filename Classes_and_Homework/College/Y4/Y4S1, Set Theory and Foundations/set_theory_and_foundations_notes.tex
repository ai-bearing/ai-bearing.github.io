\documentclass[10pt]{mypackage}

% sans serif font:
%\usepackage{cmbright}
%\usepackage{sfmath}
%\usepackage{bbold} %better blackboard bold

%serif font + different blackboard bold for serif font
\usepackage{newpxtext,eulerpx}
\renewcommand*{\mathbb}[1]{\varmathbb{#1}}

\pagestyle{fancy}
\fancyhf{}
\rhead{Avinash Iyer}
\lhead{Set Theory and Foundations of Mathematics: Class Notes}

\setcounter{secnumdepth}{0}

\begin{document}
\section{Cardinality and Countability}%
\subsection{Section 1.1: Countable Sets}%
\begin{definition}[Denumerable Set]
  A set S is denumerable if there exists a function $f: S\rightarrow \N$ with $f$ a bijection. We also say $S$ is countably infinite.
\end{definition}
\begin{definition}[Countable Set]
  We say $S$ is countable if $S$ is either finite or denumerable.
\end{definition}
\begin{theorem}[Countability of Unions]
  If $A$ and $B$ are countable sets, then $A\cup B$ is countable.
\end{theorem}
\begin{theorem}[Countability of Subsets]
  If $A\subseteq B$, then if $B$ is countable, then $A$ is countable.
\end{theorem}
\begin{theorem}[Union of Finite Sets]
  If $A$ and $B$ are finite, then $A\cup B$ is finite.
\end{theorem}
\begin{proof}
  If $A$ is finite and $B$ has one element, then we show that $A\cup B$ is finite (with two cases).\newline
  
  Afterward, for $|B| > 1$, we use induction on $|B|$.
\end{proof}
\begin{definition}[Finite Set]
  A set $A$ is finite if there exists a bijection $f: S\rightarrow \set{1,2,\dots,n}$ for some $n \in \N = \set{0,1,\dots}$.\newline

  We write $|A| = n$.
\end{definition}
\begin{theorem}[Disjoint Union of Countable Sets]
  If $A$ is denumerable, $B$ is finite, and $A\cap B = \emptyset$, then $A\cup B$ is denumerable.
\end{theorem}
\begin{proof}
  There exists a bijection $f: A\rightarrow \N$ (since $A$ is denumerable), and a bijection $g: B\rightarrow \set{1,2,\dots,n}$ for some $n \in \N$ (since $B$ is finite).\newline

  We create a new bijection $h: A\cup B \rightarrow \N$ by:
  \begin{align*}
    h(x) &= \begin{cases}
      g(x) - 1 & x\in B\\
      f(x) + n & x\in A
    \end{cases}.
  \end{align*}
  Since $A\cap B = \emptyset$, we know that $h$ is well-defined.\newline

  Now, we must show that $h$ is a bijection.\newline

  Suppose $h(x) = h(y)$.
  \begin{description}
    \item[Case 1:] If $x,y\in B$, then $h(x) = g(x) - 1$, and $h(y) = g(y) - 1$, meaning $g(x) - 1 = g(y) - 1$, meaning $g(x) = g(y)$. Since $g$ is a bijection, $x = y$.
    \item[Case 2:] If $x,y\in A$, a similar argument yields that $x = y$
    \item[Case 3:] Without loss of generality, let $x\in A$ and $y\in B$. If $x\in A$, then $h(x) = f(x) + n$ and $h(y) = g(y) - 1$. Thus, $f(x) + n = g(y) - 1$. However, since $f(x) + n \geq n$ and $0 \leq g(y) - 1 \leq n - 1$. Thus, we get that $0 \leq n \leq n-1$, which is a contradiction.
  \end{description}
  Thus, we have shown that $h$ is injective.
\end{proof}
\begin{theorem}[Cartesian Product of Natural Numbers]
  $\N\times \N$ is denumerable.
\end{theorem}
\begin{proof}
  We consider $\N\times \N$ as
  \begin{align*}
    \N \times \N &= \N \times \set{0} \cup \N\times \set{1} \cup \cdots,
  \end{align*}
  \begin{center}
    \begin{tabular}{cccccc}
      $\N\times \set{0}:$ & $(0,0)$ & $(1,0)$ & $(2,0)$ & $(3,0)$ & $\cdots$\\
      $\N\times \set{1}:$ & $(0,1)$ & $(1,1)$ & $(2,1)$ & $(3,1)$ & $\cdots$\\
      $\N\times \set{2}:$ & $(0,2)$ & $(1,2)$ & $(2,2)$ & $(3,2)$ & $\cdots$\\
      $\N\times \set{3}:$ & $(0,3)$ & $(1,3)$ & $(2,3)$ & $(3,3)$ & $\cdots$\\
      $\vdots$ & $\vdots$ & $\vdots$ & $\vdots$ & $\vdots$ & $\ddots$
    \end{tabular}
  \end{center}
  Then, we can find an (informal) bijection as follows:
  \begin{center}
    \begin{tabular}{cccccc}
      $\N\times \set{0}:$ & $\cancelto{0}{(0,0)}$ & $\cancelto{2}{(1,0)}$ & $\cancelto{5}{(2,0)}$ & $\cancelto{9}{(3,0)}$ & $\cdots$\\
      $\N\times \set{1}:$ & $\cancelto{1}{(0,1)}$ & $\cancelto{4}{(1,1)}$ & $\cancelto{8}{(2,1)}$ & $(3,1)$ & $\cdots$\\
      $\N\times \set{2}:$ & $\cancelto{3}{(0,2)}$ & $\cancelto{7}{(1,2)}$ & $(2,2)$ & $(3,2)$ & $\cdots$\\
      $\N\times \set{3}:$ & $\cancelto{6}{(0,3)}$ & $(1,3)$ & $(2,3)$ & $(3,3)$ & $\cdots$\\
      $\vdots$ & $\vdots$ & $\vdots$ & $\vdots$ & $\vdots$ & $\ddots$
    \end{tabular}
  \end{center}
  We can also find a bijection $P: \N\times \N \rightarrow \N$, with
  \begin{align*}
    P\left(x,y\right) &= \frac{(x+y)(x+y+1)}{2} + x
  \end{align*}
  A fun challenge is to prove that $P$ is a bijection.
\end{proof}
\begin{theorem}[Countability of the Rationals]
  $\Q$ is denumerable.
\end{theorem}
\begin{theorem}[Countability of the Integers]
  The set $\Z$ is denumerable.
\end{theorem}
\begin{proof}
  Let $f: \Z\rightarrow \N$ be defined by
  \begin{align*}
    f(x) &= \begin{cases}
      2x & x\geq 0\\
      -2x - 1 & x < 0
    \end{cases}
  \end{align*}
\end{proof}
\begin{definition}[Cardinality]
  We say two sets, $A$ and $B$, have the same cardinality if there exists a bijection $f: A\rightarrow B$.
\end{definition}
\begin{theorem}[Finite Subset Cardinality]
  If $m,n\in \N$ and $m\neq n$, then $\set{1,2,\dots,m}$ and $\set{1,2,\dots,n}$ do not have the same cardinality.
\end{theorem}
\begin{theorem}[Infinitude of the Natural Numbers]
  $\N$ is not finite.
\end{theorem}
\begin{example}
  If $A\subsetneq B$ and $|A| = |B|$, then both $A$ and $B$ are infinite.\newline

  In order to prove this, we need to show that every injection from a finite set to itself is a bijection.
\end{example}
\subsection{Section 1.2}%
\begin{definition}[Uncountable Set]
A set is uncountable if it is not countable.
\end{definition}
\begin{theorem}[Uncountability of $\R$]
  $\R$ is uncountable.
\end{theorem}
\begin{proof}
  For all $x\in \R$, and for all $j\in\N$, we define $\left[x\right]_j$ to denote the $j+1$-th digit after the decimal point in the decimal expansion of $x$.\newline

  For example, $\left[\pi\right]_0 = 1$, $\left[\pi\right]_1 = 4$, etc.\newline

  Let $f: \N\rightarrow \R$. We will show that $f$ is not surjective.\newline

  Let $y\in [0,1)\subseteq \R$ defined by $\forall j\in \N$,
  \begin{align*}
    \left[y\right]_j &= \begin{cases}
      0 & \left[f(j)\right]_j = 1\\
      1 & \left[f(j)\right]_j \neq 1
    \end{cases}.
  \end{align*}
  We claim that $y\notin f\left(\N\right)$. We will show that $\forall j\in \N$, $f(j) \neq y$.\newline

  We can see that if $\left[f(j)\right]_j = 1$, then $\left[y\right]_j = 0$. Similarly, if $\left[f(j)\right]_j\neq 1$, then $\left[y\right]_j = 1$. Either way, $\left[f(j)\right]_j \neq \left[y\right]_j$ for all $j\in\N$.
\end{proof}
\begin{remark}
The above proof is an example of a diagonalization proof. It can be imagined as
\begin{center}
  \renewcommand{\arraystretch}{1.5}
  \begin{tabular}{c|c}
    f(0) & $\ast.\cancelto{\neq}{a_1}\:a_2a_3\dots$\\
    f(1) & $\ast.b_1\cancelto{\neq}{b_2}\:b_3\dots$\\
    f(2) & $\ast.c_1c_2\cancelto{\neq}{c_3}\dots$\\
    $\vdots$ & $\vdots$
  \end{tabular}
\end{center}
\end{remark}
\begin{note}
  A substantial problem that we might need to deal with is that a real number does not necessarily have a unique decimal representation. For instance, $3.999\dots = 4.000\dots$.\newline

  In order to resolve this issue, we can default to the option with trailing $0$ over trailing $9$.
\end{note}
\begin{definition}[Power Set]
The power set of a set $S$ is 
\begin{align*}
  P(S) &= \set{A\mid A\subseteq S}.
\end{align*}
\end{definition}
\begin{theorem}[Power Set Surjection]
  Let $f: S\rightarrow P\left(S\right)$. Then, $f$ is not surjective.
\end{theorem}
\begin{proof}
  Let $T = \set{x\in S\mid x\notin f(x)}$. Then, $T\notin f(S)$.\newline

  Let $y\in S$. We want to show that $f(y)\neq T$. Suppose toward contradiction that $f(y) = T$. Then, if $y\in T$, then $y\in f(y)$, which implies that $y\notin T$.\newline

  If $y\notin T$, then $y\notin f(y)$, which implies that $y\in T$.\newline

  Thus, it cannot be the case that $f(y) = T$.
\end{proof}
\begin{definition}[Cardinality Comparison]
  Let $A$ and $B$ be sets. Then, we write $\Card(A) \leq \Card(B)$ if there exists an injective map $f: A\hookrightarrow B$.\newline

  We write $\Card(A) < \Card(B)$ if there exists an injection $f: A\hookrightarrow B$ but no bijection.
\end{definition}
\begin{example}[Cardinality of the Power Set]
  For every set,
  \begin{align*}
    \Card(S) < \Card\left(P(S)\right).
  \end{align*}
  \begin{enumerate}[(1)]
    \item We know that $\Card(S) \leq \Card\left(P\left(S\right)\right)$, defining $f: S\hookrightarrow P\left(S\right)$, $f(a) = \set{a}$, since if $f(x) = f(y)$, then $\set{x} = \set{y}$, meaning $x \in \set{y}$, so $x = y$.\newline

      In the case of $f: \emptyset \rightarrow \set{\emptyset}$, we define $\emptyset = f\subseteq \emptyset\times\set{\emptyset}$.
    \item Since there exists no bijection $f: S\rightarrow P\left(S\right)$, it is the case that $\Card(S)\neq \Card\left(P\left(S\right)\right)$.
  \end{enumerate}
\end{example}
\begin{example}[Decimal Expansion]
  We know that for some decimal expansion
  \begin{align*}
    3.14159\dots &= 3 + \frac{1}{10} + \frac{4}{100} + \cdots\\
                 &= \sum_{i=0}^{\infty}\frac{n_i}{10^i},
  \end{align*}
  with $0 \leq n_i \leq 9$ for $i \geq 1$.\newline

  However, we can also write any real number as
  \begin{align*}
    \sum_{i=0}^{\infty}\frac{n_i}{3^i}
  \end{align*}
  with $0 \leq n_i \leq 2$ for all $i \geq 1$.
\end{example}
\begin{example}[Finite Strings]
  Let $S$ be the set of all finite strings of $0$ and $1$. $S$ is countable.
  \begin{description}
    \item[Proof 1:] We define $f: S\rightarrow \N$ by, for a string $x\in S$, $x$ starts with $n_1$ zeroes, then has $n_2$ ones, then $n_3$ zeroes, etc. We define $f(x) := 2^{n_1}\times 3^{n_2}\times 5^{n_3}\times 7^{n_4}\times 11^{n_5}\cdots$, or
      \begin{align*}
        f(x) &= \prod_{i}^{\infty}p_{i}^{n_i},
      \end{align*}
      where $p_i$ denotes the $i$th prime number. We can see that $f$ is an injection.

      Since $S$ is infinite (proof omitted), we can see that $f(S)$ is also infinite.\footnote{If $f(S)$ is finite, then there exists a bijection $g: f(S)\rightarrow\set{1,\dots,n}$. Composing $g$ and $f$, we find $S$ is finite as $g\circ f|_{S}$ is a bijection.} Since $f(S)$ is an infinite subset of $\N$, $f(S)$ is denumerable, meaning there exists a bijection $q: f(S) \rightarrow \N$. Therefore, we have $q\circ f: S\rightarrow \N$ is a bijection, meaning $S$ is denumerable.
    \item[Proof 2:] List the elements of $S$ by length and lexicographic order: short strings come before long strings, and $0$s come before $1$s.
      \begin{center}
        \begin{tabular}{c|c}
          Rank & String\\
          \hline
          0 & 0\\
          1 & 1\\
          2 & 00\\
          3 & 01\\
          4 & 10\\
          5 & 11\\
          \vdots & \vdots
        \end{tabular}
      \end{center}
      This pattern yields a systematic way to map $S$ to the natural numbers.
    \item[Proof 3:] We can see that
      \begin{align*}
        S &= \bigcup_{i=1}^{\infty} S_i,
      \end{align*}
      where $S_i$ is the set of all strings of length $i$, each of which contains $2^i$ elements.\newline

      Since each $S_i$ is finite, and $S_i \cap S_j = \emptyset$ (by definition). Thus, $S$ is a countable union of pairwise disjoint countable sets, so $S$ is countable.
  \end{description}
\end{example}
\begin{example}[All Possible Writings]
  Let $W$ be the set of all possible writings in English. We let $W_n$ denote the writing with $n$ characters. Then,
  \begin{align*}
    W &= \bigcup_{n=1}^{\infty}W_n,
  \end{align*}
  which is a countable union of disjoint finite sets, which is countable.\newline

  Similarly, we can list all the writings by length and lexicographic order.\newline

  This result implies that ``almost all'' real numbers, in a sense, are unable to be described.
\end{example}
\subsection{Section 1.3: Cantor--Schröder--Bernstein Theorem}%
\begin{example}
  If we have $|A| \leq |B|$ and $|B| \leq |A|$, it does not necessarily imply $|A| = |B|$.\newline

  This is because the $\leq$ in the cardinality comparison implies there exist injections $f: A\hookrightarrow B$ and $g: B\hookrightarrow A$, not that the cardinalities are necessarily ``less than or equal to'' each other.\newline

  However, at the same time, this fact is true --- this is what is known as the Cantor--Schröder--Bernstein Theorem.
\end{example}
\begin{theorem}[Cantor--Schröder--Bernstein]
  Let $f: C\hookrightarrow D$ and $g: D\hookrightarrow C$ be injective maps. Then, $|C| = |D|$.
\end{theorem}
\begin{proof}[An Informal Proof Sketch]
  Consider $C$ to be a set of cats and $D$ to be a set of dogs. Every cat chases a dog, and every dog chases a cat, with different cats chasing different dogs and vice versa.\newline

  There are four potential arrangements:
  \begin{enumerate}[(1)]
    \item A set of cats and dogs are chasing each other in a circle.
    \item A chain of dogs chasing cats that starts with a dog.
    \item A chain of cats chasing dogs that starts with a cat.
    \item An endless chain of cats chasing dogs with no discernible start or end point.
  \end{enumerate}
  These four cases create a bijection from $C$ to $D$:
  \begin{enumerate}[(1)]
    \item Pair each cat with the dog that it is chasing.
    \item Pair each cat with the dog that it is chasing.
    \item Pair each cat with the dog that \textit{is chasing it}.
    \item Pair each cat with the dog that it is chasing.
  \end{enumerate}
\end{proof}
\begin{proof}[A More Formal Proof Sketch]
  For $C = \set{c_i}_{i\in I}$ and $D = \set{d_i}_{i}$, we have four types of sequences.
  \begin{enumerate}[(i)]
    \item Circular sequence: for some $m\in\N$, there exist $c_1,\dots,c_m$ and $d_1,\dots,d_m$ such that $f\left(c_i\right) = d_i$ and $g\left(d_i\right) = c_{i+1}$, where $c_{m+1} = c_1$.
    \item Cat sequence: there is $c_1,c_2,\dots$ and $d_1,d_2,\dots$ such that $f(c_i) = d_i$ and $g(d_i) = c_{i+1}$.
    \item Dog sequence: there is $c_1,c_2,\dots$ and $d_1,d_2,\dots$ such that $f\left(c_i\right) = d_{i+1}$ and $g\left(d_i\right) = c_{i}$.
    \item Bi-infinite sequence: $\set{c_i}_{i\in \Z}$ and $\set{d_i}_{i\in\Z}$ such that $f\left(c_i\right) = d_i$ and $g\left(d_i\right) = c_{i+1}$.
  \end{enumerate}
  \begin{description}
    \item[Claim 1:] For every $c\in C$, $c$ is in exactly one sequence that is either a circular sequence, a cat sequence, a dog sequence, or a bi-infinite sequence.
  \end{description}
  We define our bijection $h: C\rightarrow D$ by
  \begin{align*}
    h(c) &= \begin{cases}
      g^{-1}\left(c\right) & \text{$c$ in a dog sequence}\\
      f(c) & \text{else}
    \end{cases}.
  \end{align*}
  \begin{description}
    \item[Claim 2:] $h$ is well-defined.
    \item[Claim 3:] $h$ is a bijection.
  \end{description}
\end{proof}
\begin{theorem}
  For every set $A,B$, either $|A| \leq |B|$ or $|B| \leq |A|$.\newline

  In order to prove this, we need the axiom of choice.
\end{theorem}
\begin{example}[Cardinality of the Reals]
  Recall that $|\N| < |P(\N)|$ and $|\N| < |\R|$. According to the previous theorem, it is the case that either $\left\vert P(\N) \right\vert \leq |\R|$ or $\left\vert \R \right\vert\leq \left\vert P\left(\N\right) \right\vert$.\newline

  In particular, $\left\vert P(\N) \right\vert = \left\vert \R \right\vert$.
\end{example}
\begin{proof}[An Informal Proof]
  Let $S$ be the set of all functions $f: \N\rightarrow \set{0,1}$. We will show that $|S| = \left\vert P\left(\N\right) \right\vert$ and $|S| = |\R|$. This will show that $\left\vert P(\N) \right\vert = \left\vert \R \right\vert$ (by composing bijections).\newline

  To show that $|S| = |P\left(\N\right)|$, define a subset of $\N$ by the support\footnote{The elements that $f$ does not map to $0$ for some $f\in S$.} of some element of $S$. This is a bijection between $P\left(\N\right)$ and $S$.\newline

  To show $|S| = |\R|$, we place a decimal point in front of the string, and consider it as a real number in base 2, which yields a bijection between $S$ and $[0,1]$.\newline

  Next, we show that $|[0,1]| = |(0,1)|$.\newline

  Finally, we show that $|(0,1)| = \R$. Take $f: (0,1)\rightarrow \R$ to be $\cot\left(\pi x\right)$ --- or $\tan(\pi x - \pi/2)$. These are bijections from $(0,1)$ to $\R$.
\end{proof}
\begin{definition}[Continuum Hypothesis]
  We are aware that
  \begin{align*}
    |\N| < |\R| = |P\left(\N\right)|.
  \end{align*}
  The continuum hypothesis states that there exists no set $S$ such that
  \begin{align*}
    |\N| < |S| < |\R|.
  \end{align*}
  The continuum hypothesis is independent of the ZFC axioms.\footnote{Zermelo--Frankel Axioms with the Axiom of Choice.}
\end{definition}
\end{document}
