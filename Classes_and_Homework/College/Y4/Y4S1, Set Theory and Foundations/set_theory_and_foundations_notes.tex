\documentclass[10pt]{mypackage}

% sans serif font:
%\usepackage{cmbright}
%\usepackage{sfmath}
%\usepackage{bbold} %better blackboard bold

%serif font + different blackboard bold for serif font
\usepackage{newpxtext,eulerpx}
\renewcommand*{\mathbb}[1]{\varmathbb{#1}}

\pagestyle{fancy}
\fancyhf{}
\rhead{Avinash Iyer}
\lhead{Set Theory and Foundations of Mathematics: Class Notes}

\setcounter{secnumdepth}{0}

\begin{document}
\section{Cardinality and Countability}%
\subsection{Section 1.1: Countable Sets}%
\begin{definition}[Denumerable Set]
  A set S is denumerable if there exists a function $f: S\rightarrow \N$ with $f$ a bijection. We also say $S$ is countably infinite.
\end{definition}
\begin{definition}[Countable Set]
  We say $S$ is countable if $S$ is either finite or denumerable.
\end{definition}
\begin{theorem}[Countability of Unions]
  If $A$ and $B$ are countable sets, then $A\cup B$ is countable.
\end{theorem}
\begin{theorem}[Countability of Subsets]
  If $A\subseteq B$, then if $B$ is countable, then $A$ is countable.
\end{theorem}
\begin{theorem}[Union of Finite Sets]
  If $A$ and $B$ are finite, then $A\cup B$ is finite.
\end{theorem}
\begin{proof}
  If $A$ is finite and $|B|$ has one element, then we show that $A\cup B$ is finite (with two cases).\newline
  
  Afterward, for $|B| > 1$, we use induction on $|B|$.
\end{proof}
\begin{definition}[Finite Set]
  A set $A$ is finite if there exists a bijection $f: S\rightarrow \set{1,2,\dots,n}$ for some $n \in \N = \set{0,1,\dots}$.\newline

  We write $|A| = n$.
\end{definition}
\begin{theorem}[Disjoint Union of Countable Sets]
  If $A$ is denumerable, $B$ is finite, and $A\cap B = \emptyset$, then $A\cup B$ is denumerable.
\end{theorem}
\begin{proof}
  There exists a bijection $f: A\rightarrow \N$ (since $A$ is denumerable), and a bijection $g: B\rightarrow \set{1,2,\dots,n}$ for some $n \in \N$ (since $B$ is finite).\newline

  We create a new bijection $h: A\cup B \rightarrow \N$ by:
  \begin{align*}
    h(x) &= \begin{cases}
      g(x) - 1 & x\in B\\
      f(x) + n & x\in A
    \end{cases}.
  \end{align*}
  Since $A\cap B = \emptyset$, we know that $h$ is well-defined.\newline

  Now, we must show that $h$ is a bijection.\newline

  Suppose $h(x) = h(y)$.
  \begin{description}
    \item[Case 1:] If $x,y\in B$, then $h(x) = g(x) - 1$, and $h(y) = g(y) - 1$, meaning $g(x) - 1 = g(y) - 1$, meaning $g(x) = g(y)$. Since $g$ is a bijection, $x = y$.
    \item[Case 2:] If $x,y\in A$, a similar argument yields that $x = y$
    \item[Case 3:] Without loss of generality, let $x\in A$ and $y\in B$. If $x\in A$, then $h(x) = f(x) + n$ and $h(y) = g(y) - 1$. Thus, $f(x) + n = g(y) - 1$. However, since $f(x) + n \geq n$ and $0 \leq g(y) - 1 \leq n - 1$. Thus, we get that $0 \leq n \leq n-1$, which is a contradiction.
  \end{description}
  Thus, we have shown that $h$ is injective.
\end{proof}
\begin{theorem}[Cartesian Product of Natural Numbers]
  $\N\times \N$ is denumerable.
\end{theorem}
\begin{proof}
  We consider $\N\times \N$ as
  \begin{align*}
    \N \times \N &= \N \times \set{0} \cup \N\times \set{1} \cup \cdots,
  \end{align*}
  \begin{center}
    \begin{tabular}{cccccc}
      $\N\times \set{0}:$ & $(0,0)$ & $(1,0)$ & $(2,0)$ & $(3,0)$ & $\cdots$\\
      $\N\times \set{1}:$ & $(0,1)$ & $(1,1)$ & $(2,1)$ & $(3,1)$ & $\cdots$\\
      $\N\times \set{2}:$ & $(0,2)$ & $(1,2)$ & $(2,2)$ & $(3,2)$ & $\cdots$\\
      $\N\times \set{3}:$ & $(0,3)$ & $(1,3)$ & $(2,3)$ & $(3,3)$ & $\cdots$\\
      $\vdots$ & $\vdots$ & $\vdots$ & $\vdots$ & $\vdots$ & $\ddots$
    \end{tabular}
  \end{center}
  Then, we can find an (informal) bijection as follows:
  \begin{center}
    \begin{tabular}{cccccc}
      $\N\times \set{0}:$ & $\cancelto{0}{(0,0)}$ & $\cancelto{2}{(1,0)}$ & $\cancelto{5}{(2,0)}$ & $\cancelto{9}{(3,0)}$ & $\cdots$\\
      $\N\times \set{1}:$ & $\cancelto{1}{(0,1)}$ & $\cancelto{4}{(1,1)}$ & $\cancelto{8}{(2,1)}$ & $(3,1)$ & $\cdots$\\
      $\N\times \set{2}:$ & $\cancelto{3}{(0,2)}$ & $\cancelto{7}{(1,2)}$ & $(2,2)$ & $(3,2)$ & $\cdots$\\
      $\N\times \set{3}:$ & $\cancelto{6}{(0,3)}$ & $(1,3)$ & $(2,3)$ & $(3,3)$ & $\cdots$\\
      $\vdots$ & $\vdots$ & $\vdots$ & $\vdots$ & $\vdots$ & $\ddots$
    \end{tabular}
  \end{center}
  We can also find a bijection $P: \N\times \N \rightarrow \N$, with
  \begin{align*}
    P\left(x,y\right) &= \frac{(x+y)(x+y+1)}{2} + x
  \end{align*}
  A fun challenge is to prove that $P$ is a bijection.
\end{proof}
\begin{theorem}[Countability of the Rationals]
  $\Q$ is denumerable.
\end{theorem}
\begin{theorem}[Countability of the Integers]
  The set $\Z$ is denumerable.
\end{theorem}
\begin{proof}
  Let $f: \Z\rightarrow \N$ be defined by
  \begin{align*}
    f(x) &= \begin{cases}
      2x & x\geq 0\\
      -2x - 1 & x < 0
    \end{cases}
  \end{align*}
\end{proof}
\begin{definition}[Cardinality]
  We say two sets, $A$ and $B$, have the same cardinality if there exists a bijection $f: A\rightarrow B$.
\end{definition}
\begin{theorem}[Finite Subset Cardinality]
  If $m,n\in \N$ and $m\neq n$, then $\set{1,2,\dots,m}$ and $\set{1,2,\dots,n}$ do not have the same cardinality.
\end{theorem}
\begin{theorem}[Infinitude of the Natural Numbers]
  $\N$ is not finite.
\end{theorem}
\begin{example}
  If $A\subsetneq B$ and $|A| = |B|$, then both $A$ and $B$ are infinite.\newline

  In order to prove this, we need to show that every injection from a finite set to itself is a bijection.
\end{example}
\subsection{Section 1.2}%
\begin{definition}[Uncountable Set]
A set is uncountable if it is not countable.
\end{definition}
\begin{theorem}[Uncountability of $\R$]
  $\R$ is uncountable.
\end{theorem}
\begin{proof}
  For all $x\in \R$, and for all $j\in\N$, we define $\left[x\right]_j$ to denote the $j+1$-th digit after the decimal point in the decimal expansion of $x$.\newline

  For example, $\left[\pi\right]_0 = 1$, $\left[\pi\right]_1 = 4$, etc.\newline

  Let $f: \N\rightarrow \R$. We will show that $f$ is not surjective.\newline

  Let $y\in [0,1)\subseteq \R$ defined by $\forall j\in \N$,
  \begin{align*}
    \left[y\right]_j &= \begin{cases}
      0 & \left[f(j)\right]_j = 1\\
      1 & \left[f(j)\right]_j \neq 1
    \end{cases}.
  \end{align*}
  We claim that $y\notin f\left(\N\right)$. We will show that $\forall j\in \N$, $f(j) \neq y$.\newline

  We can see that if $\left[f(j)\right]_j = 1$, then $\left[y\right]_j = 0$. Similarly, if $\left[f(j)\right]_j\neq 1$, then $\left[y\right]_j = 1$. Either way, $\left[f(j)\right]_j \neq \left[y\right]_j$ for all $j\in\N$.
\end{proof}
\begin{remark}
The above proof is an example of a diagonalization proof. It can be imagined as
\begin{center}
  \renewcommand{\arraystretch}{1.5}
  \begin{tabular}{c|c}
    f(0) & $\ast.\cancelto{\neq}{a_1}\:a_2a_3\dots$\\
    f(1) & $\ast.b_1\cancelto{\neq}{b_2}\:b_3\dots$\\
    f(2) & $\ast.c_1c_2\cancelto{\neq}{c_3}\dots$\\
    $\vdots$ & $\vdots$
  \end{tabular}
\end{center}
\end{remark}
\begin{note}
  A substantial problem that we might need to deal with is that a real number does not necessarily have a unique decimal representation. For instance, $3.999\dots = 4.000\dots$.\newline

  In order to resolve this issue, we can default to the option with trailing $0$ over trailing $9$.
\end{note}
\begin{definition}[Power Set]
The power set of a set $S$ is 
\begin{align*}
  \mathcal{P}(S) &= \set{A\mid A\subseteq S}.
\end{align*}
\end{definition}
\begin{theorem}[Power Set Surjection]
  Let $f: S\rightarrow \mathcal{P}\left(S\right)$. Then, $f$ is not surjective.
\end{theorem}
\begin{proof}
  Let $T = \set{x\in S\mid x\notin f(x)}$. Then, $T\notin f(S)$.\newline

  Let $y\in S$. We want to show that $f(y)\neq T$. Suppose toward contradiction that $f(y) = T$. Then, if $y\in T$, then $y\in f(y)$, which implies that $y\notin T$.\newline

  If $y\notin T$, then $y\notin f(y)$, which implies that $y\in T$.\newline

  Thus, it cannot be the case that $f(y) = T$.
\end{proof}
\begin{definition}[Cardinality Comparison]
  Let $A$ and $B$ be sets. Then, we write $\Card(A) \leq \Card(B)$ if there exists an injective map $f: A\hookrightarrow B$.\newline

  We write $\Card(A) < \Card(B)$ if there exists an injection $f: A\hookrightarrow B$ but no bijection.
\end{definition}
\begin{example}[Cardinality of the Power Set]
  For every set,
  \begin{align*}
    \Card(S) < \Card\left(\mathcal{P}(S)\right).
  \end{align*}
  \begin{enumerate}[(1)]
    \item We know that $\Card(S) \leq \Card\left(\mathcal{P}\left(S\right)\right)$, defining $f: S\hookrightarrow \mathcal{P}\left(S\right)$, $f(a) = \set{a}$, since if $f(x) = f(y)$, then $\set{x} = \set{y}$, meaning $x \in \set{y}$, so $x = y$.\newline

      In the case of $f: \emptyset \rightarrow \set{\emptyset}$, we define $\emptyset = f\subseteq \emptyset\times\set{\emptyset}$.
    \item Since there exists no bijection $f: S\rightarrow \mathcal{P}\left(S\right)$, it is the case that $\Card(S)\neq \Card\left(\mathcal{P}\left(S\right)\right)$.
  \end{enumerate}
\end{example}
\begin{example}[Decimal Expansion]
  We know that for some decimal expansion
  \begin{align*}
    3.14159\dots &= 3 + \frac{1}{10} + \frac{4}{100} + \cdots\\
                 &= \sum_{i=0}^{\infty}\frac{n_i}{10^i},
  \end{align*}
  with $0 \leq n_i \leq 9$ for $i \geq 1$.\newline

  However, we can also write any real number as
  \begin{align*}
    \sum_{i=0}^{\infty}\frac{n_i}{3^i}
  \end{align*}
  with $0 \leq n_i \leq 2$ for all $i \geq 1$.
\end{example}
\end{document}
