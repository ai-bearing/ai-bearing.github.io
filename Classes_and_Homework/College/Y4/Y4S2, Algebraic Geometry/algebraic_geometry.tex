\documentclass[10pt]{mypackage}

% sans serif font:
%\usepackage{cmbright,sfmath,bbold}
%\renewcommand{\mathcal}{\mathtt}

%Euler:
\usepackage{newpxtext,eulerpx,eucal,eufrak}
\renewcommand*{\mathbb}[1]{\varmathbb{#1}}
\renewcommand*{\hbar}{\hslash}
\newcommand{\A}{\mathbb{A}}
\usepackage{homework}

%kp fonts:
%\usepackage{kpfonts}
%\renewcommand{\mathbb}{\mathds}
%\usepackage{homework}

\pagestyle{fancy} %better headers
\fancyhf{}
\rhead{Avinash Iyer}
\lhead{Algebraic Geometry}

\setcounter{secnumdepth}{0}

\begin{document}
\RaggedRight
\section{Introduction}%
Oh hey, it's another one of these independent studies. Me and a friend are going to be going through William Fulton's \textit{Algebraic Curves}. It will be hard, it will be long, and it might not work out for me, but who cares.
\tableofcontents
\section{Affine Algebraic Sets}%
\subsection{Algebraic Preliminaries}%
We will assume all rings are commutative with unity, where $\Z$ is the integers, $\Q$ is the rationals, $\R$ is the reals, and $\C$ is the complex numbers.\newline

Any integral domain $R$ has a quotient field $K$, which contains $R$ as a subring, and any element in $K$ may be written as a not necessarily unique ratio of two elements of $R$. Any one-to-one ring homomorphism from $R$ to a field $L$ extends uniquely to a ring homomorphism from $K$ to $L$.\newline

If $R$ is a ring, then $R[x]$ is the ring of polynomials with coefficients in $R$. The degree of a nonzero polynomial $\sum a_ix^i$ is the largest integer $d$ such that $a_d\neq 0$. The polynomial is monic if $a_d = 1$.\newline

The ring of polynomials in $n$ variables over $R$ is $R\left[x_1,\dots,x_n\right]$. We write $R\left[x,y\right]$ and $R\left[x,y,z\right]$ if $n=2$ and $3$ respectively. Monomials in $R\left[x_1,\dots,x_n\right]$ are of the form $x^{(i)} \coloneq x_1^{i_1}x_2^{i_2}\cdots x_n^{i_n}$, where $i_j$ are nonnegative integers, and the degree of the monomial is $i_1 + \cdots i_n$. Every $F\in R\left[x_1,\dots,x_n\right]$ has a unique expression $F = \sum a_{(i)}x^{(i)}$, where $x^{(i)}$ are monomials, and $a_{(i)}\in R$. We say $F$ is homogeneous of degree $d$ if all $a_{(i)}$ are zero except for monomials of degree $d$. The polynomial $F$ is written as $F = F_0 + F_1 + \cdots F_d$, where $F_i$ is a form of degree $i$, and $d = \deg(F)$ for $F_d\neq 0$.\newline

The ring $R$ is a subring of $R\left[x_1,\dots,x_n\right]$, and the ring $R\left[x_1,\dots,x_n\right]$ is characterized by the following: if $\varphi\colon R\rightarrow S$ is a ring homomorphism, and $s_1,\dots,x_n$ are elements in $S$, then there is a unique extension of $\varphi$ to a ring homomorphism $\overline{\varphi}\colon R\left[x_1,\dots,x_n\right]\rightarrow S$ such that $\overline{\varphi}\left(x_i\right) = s_i$. The image of $F$ under $\overline{\varphi}$ is written $F\left(s_1,\dots,s_n\right)$. The ring $R\left[x_1,\dots,x_n\right]$ is canonically isomorphic to $R\left[x_1,\dots,x_{n-1}\right]\left[x_n\right]$.\newline

An element $a\in R$ is called irreducible if it is not a unit or zero, and any factorization $a=bc$ with $b,c\in R$ is such that either $b $ or $c$ is a unit. A domain $R$ is a unique factorization domain (UFD) if every nonzero element in $R$ can be factored uniquely up to units and ordering.\newline

If $R$ is a UFD with quotient field $K$, then any irreducible element $F\in R[x]$ remains irreducible when considered in $K[x]$.
\begin{theorem}[Gauss's Lemma for $\Z$]
  If $F\in \Z[x]$ is a monic polynomial that is irreducible, then $F$ is irreducible in $\Q[x]$.
\end{theorem}
If $F$ and $G$ are polynomials in $R[x]$ with no common factors in $R[x]$, then they have no common factors in $K[x]$.\newline

If $R$ is a UFD, then $R[x]$ is also a UFD, and consequently $k\left[x_1,\dots,x_n\right]$ is a UFD for any field $k$. The quotient field of $k\left[x_1,\dots,x_n\right]$ is written $k\left(x_1,\dots,x_n\right)$ is called the field of rational functions in $n$ variables over $k$.\newline

If $\varphi\colon R\rightarrow S$ is a ring homomorphism, $\ker\left(\varphi\right)\coloneq \varphi^{-1}(0)$. The kernel is an ideal in $R$. An ideal in $R$ is proper if $I\neq R$, and a proper ideal is known as maximal if it is not contained in any larger proper ideal.\footnote{Alternatively, an ideal $I$ is maximal if the quotient ring $R/M$ is a field.} An ideal $\mathfrak{p}$ is prime if, whenever $ab\in \mathfrak{p}$, then $a\in \mathfrak{p}$ or $b\in \mathfrak{p}$.\footnote{Alternatively, an ideal $\mathfrak{p}$ is prime if $R/\mathfrak{p}$ is an integral domain.}\newline

Let $k$ be a field and $I$ a proper ideal in $k\left[x_1,\dots,x_n\right]$. The canonical homomorphism $\pi$ from $k\left[x_1,\dots,x_n\right]$ to $k\left[x_1,\dots,x_n\right]/I$ restricts to a ring homomorphism from $k$ to $k\left[x_1,\dots,x_n\right]/I$. We regard $k$ as a subring of $k\left[x_1,\dots,x_n\right]/I$, which is a vector space over $k$.\newline

If $R$ is an integral domain, then $\operatorname{char}\left(R\right)$, the characteristic of $R$, is the smallest integer $p$ such that 
\begin{align*}
  \underbrace{1+1\cdots +1}_{p\text{ times}} = 0. 
\end{align*}
If $p$ exists, we say $\operatorname{char}\left(R\right) = p$, else $0$.\newline

Note that if $\varphi\colon \Z\rightarrow R$ is the unique ring homomorphism from $\Z$ to $R$,\footnote{This is because $\Z$ is initial in the category of rings. See Aluffi.} then $\ker\left(\varphi\right) = \left\langle p \right\rangle$, so $\operatorname{char}\left(R\right)$ is prime or $0$.\newline

If $R$ is a ring, and $F\in R\left[x\right]$, and $a$ is a root of $F$, then $F = \left(x-a\right)G$ for some unique polynomial $G\in R[x]$. A field $k$ is algebraically closed if any nonconstant $F\in k\left[x\right]$ has a root.
\begin{exercise}[Exercise 1.1]
Let $R$ be an integral domain.
\begin{enumerate}[(a)]
  \item If $F$ and $G$ are forms of degree $r$ and $s$ respectively in $R\left[x_1,\dots,x_n\right]$, show that $FG$ is a form of degree $r+s$.
  \item Show that any factor of a form in $R\left[x_1,\dots,x_n\right]$ is also a form.
\end{enumerate}
\end{exercise}
\begin{solution}\hfill
  \begin{enumerate}[(a)]
    \item Let $H = FG$, where $F$ is a form of degree $r$ and $G$ is a form of degree $s$. Note that since $F$ and $G$ are forms, we know that $F = F_{r}$, where $F_r$ is the form with degree $r$, and $G = G_s$, where $G_s$ is the form with degree $s$.
  \end{enumerate}
  
\end{solution}

\begin{exercise}[Exercise 1.2]
  Let $R$ be a UFD and $K$ the quotient field of $R$. Show that every element $z\in K$ may be written as $z = a/b$, where $a,b\in R$ have no common factors. This representative is unique up to units of $R$.
\end{exercise}
\begin{solution}
  Since $K = \operatorname{Frac}\left(R\right)$, we know that every $z\in K$ is of the form $z = \frac{a}{b}$. Since $R$ a unique factorization domain, $\gcd\left(a,b\right)$ is unique and well-defined. Set $c\cdot \gcd\left(a,b\right) = a$ and $d\cdot \gcd\left(a,b\right) = b$. Then,
  \begin{align*}
    z &= \frac{a}{b}\\
      &= \frac{c\cdot \gcd\left(a,b\right)}{d\cdot \gcd\left(a,b\right)}\\
      &= \frac{c}{d}.
  \end{align*}
  We show that this is unique up to units. Suppose
  \begin{align*}
    z &= \frac{c}{d}\\
      &= \frac{c'}{d'}.
  \end{align*}
  Then, by the properties of the field of fractions, we know that
  \begin{align*}
    c'd &= cd',
  \end{align*}
  and since $R$ is a UFD, we know that $\gcd\left(c,d\right) = \gcd\left(c',d'\right) = 1$, so $c = u_1c'$ and $d = u_2d'$.
\end{solution}
\begin{exercise}[Exercise 1.3]
  Let $R$ be a principal ideal domain, and let $P$ be a nonzero proper prime ideal in $R$.
  \begin{enumerate}[(a)]
    \item Show that $P$ is generated by an irreducible element.
    \item Show that $P$ is maximal.
  \end{enumerate}
\end{exercise}
\begin{solution}\hfill
  \begin{enumerate}[(a)]
    \item Since $P$ is principal, we know that $ P = \left\langle a \right\rangle $ for some $a\in R$. We know that $a$ cannot be a unit, as otherwise $P = R$, contradicting the assumption that $P$ is proper, and that $a\neq 0$ as $P$ is not zero.\newline

      Suppose toward contradiction that $\left\langle a \right\rangle\subsetneq \left\langle b \right\rangle$ for some $b\in R$. Then, $a = bc$ for some $c\in R$. If $c\notin \left\langle a \right\rangle$, then since $\left\langle a \right\rangle$ is prime, we must have $b\in \left\langle a \right\rangle$, contradicting strict inclusion. Thus, $c\in \left\langle a \right\rangle$, so $c = at$ for some $t\in R$. Therefore, we have $a = abt$, so $bt = 1_R$, and $\left\langle b \right\rangle = R$.
    \item Since $R$ is a PID, and $P$ is prime, we know that $P = \left\langle a \right\rangle$ is generated by an irreducible element. Thus, if $\left\langle a \right\rangle\subseteq \left\langle b \right\rangle$, then $a = bc$ for some $c\in R$. Since we have unique factorization (as all PIDs are UFDs), and $a$ is irreducible, this means either $b$ or $c$ is a unit. If $b$ is a unit, then $\left\langle b  \right\rangle = R$, and if $c$ is a unit, then $\left\langle b \right\rangle = \left\langle a \right\rangle$. Thus, $\left\langle a \right\rangle$ is maximal.
  \end{enumerate}
\end{solution}
\begin{exercise}[Exercise 1.4]
Let $k$ be an infinite field, $f\in k\left[x_1,\dots,x_n\right]$. Suppose $F\left(a_1,\dots,a_n\right) = 0$ for all $a_1,\dots,a_n\in k$. Show that $f = 0$.
\end{exercise}
\begin{exercise}[Exercise 1.5]
Let $k$ be any field. Show that there are an infinite number of irreducible monic polynomials in $k\left[x\right]$.
\end{exercise}
\begin{solution}
  Suppose $F_1,\dots,F_n$ were all the irreducible monic polynomials in $k\left[x\right]$. Consider the polynomial $P = F_1F_2\cdots F_n + 1$. We note that $P$ is monic. We will show that $P$ is irreducible.\newline

  Suppose toward contradiction that $P$ were reducible. We know that $k\left[x\right]$ is a principal ideal domain, so $P\in \left\langle F_i \right\rangle$ for some irreducible monic $F_i$. However, we know that, for any $F_i$, $1\leq i\leq n$, $P \nmid F_i$, as, applying the division algorithm to $P$, we get
  \begin{align*}
    P &= \left(F_i\right)\prod_{j\neq i}F_j + 1,
  \end{align*}
  where $r \neq 0$. Thus, $P$ is not reducible and monic, so there are infinitely many irreducible monic polynomials in $k\left[x\right]$.
\end{solution}

\begin{exercise}[Exercise 1.6]
Show that any algebraically closed field is infinite.
\end{exercise}
\begin{solution}
  Note that if $k$ is any field, then there are infinitely many irreducible monic polynoimals in $k\left[x\right]$. If $k$ is algebraically closed, then $\left(x-a\right)$, for $a\in k$, is the only irreducible monic polynomial. Since there are infinitely many irreducible monic polynomials in $k\left[x\right]$, there are infinitely many $a\in k$ such that $\left(x-a\right)$ is irreducible in $k\left[x\right]$. Thus, $k$ is infinite.
\end{solution}

\begin{exercise}[Exercise 1.7]
Let $k$ be any field, and $F\in k\left[x_1,\dots,x_n\right]$, with $a_1,\dots,a_n\in k$.
\begin{enumerate}[(a)]
  \item Show that
    \begin{align*}
      F &= \sum\lambda_{(i)}\left(x_1-a_1\right)^{i_1}\cdots \left(x_n-a_n\right)^{i_n},
    \end{align*}
    where $\lambda_{(i)}\in k$.
  \item If $F\left(a_1,\dots,a_n\right) = 0$, show that $F = \sum_{i=1}^{n}\left(x_i - a_i\right)G_i$ for some not necessarily unique $G_i\in k\left[x_1,\dots,x_n\right]$.
\end{enumerate}
\end{exercise}
\begin{solution}\hfill
  \begin{enumerate}[(a)]
    \item We let
      \begin{align*}
        G &= F\left(x_1+a_1,x_2 + a_2,\dots,x_n + a_n\right).
      \end{align*}
      Then, since $G\in k\left[x_1,\dots,x_n\right]$, we have
      \begin{align*}
        G &= \sum \lambda_{(i)}x_1^{i_1}\cdots x_n^{i_n}.
      \end{align*}
      Then, we have
      \begin{align*}
        F &= \sum \lambda_{(i)}\left(x_1 - a_1\right)^{i_1} \cdots \left(x_n-a_n\right)^{i_n}.
      \end{align*}
    \item Note that if $F\left(a_1,\dots,a_n\right) = 0$, then $\left(x_i - a_i\right) \mid F\left(a_1,\dots,a_{i-1},x_i,a_{i+1},\dots,a_n\right)$. Thus, we have
      \begin{align*}
        F\left(a_1,\dots,a_{i-1},x_i,a_{i+1},\dots,a_n\right) &= \left(x_i - a_i\right)\underbrace{g\left(a_1,\dots,a_{i-1},x_i,a_{i+1},\dots,a_n\right)}_{G_i}.
      \end{align*}
      This yields
      \begin{align*}
        F\left(x_1,\dots,x_n\right) &= \sum_{i=1}^{n}\left(x_i - a_i\right)G_i.
      \end{align*}
  \end{enumerate}
\end{solution}
\subsection{Affine Space and Algebraic Sets}%
\begin{definition}
  If $k$ is a field, then when we write $\mathbb{A}^n\left(k\right)$, or $\mathbb{A}^n$, to be the cartesian product of $k$ with itself $n$ times.\newline

  We call $\mathbb{A}^n\left(k\right)$ the affine $n$-space over $k$. Its elements are called points. We call $\mathbb{A}^1(k)$ the affine line and $\A^2(k)$ the affine plane.
\end{definition}
\begin{definition}
  If $F\in k\left[x_1,\dots,x_n\right]$, then $P = \left(a_1,\dots,a_n\right)\in \A^n\left(k\right)$ is called a zero of $F$ if $F(P) = \left(a_1,\dots,a_n\right) = 0$.\newline

  If $F$ is not constant, then the zeros of $F$ are called the hypersurface defined by $F$, defined by $V(F)$. A hypersurface in $\A^2\left(k\right)$ is called an affine plane curve.\newline

  If $F$ is a polynomial of degree $1$, then $V(F)$ is called a hyperplane in $\A^n\left(k\right)$; if $n = 2$, then an affine hyperplane is a line.
\end{definition}
\begin{definition}
  If $S$ is any set of polynomials in $k\left[x_1,\dots,x_n\right]$, then $V(S) = \set{P\in \A^n | F(P) = 0\text{ for all }F\in S}$. In other words, $V(S) = \bigcap_{F\in S}V(F)$. If $S = \set{F_1,\dots,F_r}$, we write $V\left(F_1,\dots,F_r\right)$.\newline

  A subset $X\subseteq \A^n\left(k\right)$ is an affine algebraic set (or algebraic set) if $X = V(S)$ for some $S$.
\end{definition}
\begin{proposition}\hfill
  \begin{enumerate}[(1)]
    \item If $I$ is the ideal in $k\left[x_1,\dots,x_n\right]$ generated by $S$, then $V(S) = V(I)$; thus, every algebraic set is equal to $V(I)$ for some ideal $I$.
    \item If $\set{I_{\alpha}}$ is a collection of ideals, then $V\left(\bigcup_{\alpha}I_{\alpha}\right) = \bigcap_{\alpha}V\left(I_{\alpha}\right)$.
    \item If $I\subseteq J$, then $V(I)\supseteq V(J)$.
    \item For any polynomials$F,G$, $V\left(FG\right) = V\left(F\right)\cup V\left(G\right)$. Furthermore, $V(I)\cup V(J) = V\left(\set{FG | F\in I, G\in J}\right)$.
    \item We have that $V(0) = \A^n(k)$, $V(1) = \emptyset$, $V\left(x_1-a_1,\dots,x_n-a_n\right) = \set{(a_1,\dots,a_n)}$ for $a_i\in k$. Thus, any finite subset of $\A^n(k)$ is an algebraic set.
  \end{enumerate}
\end{proposition}
\begin{exercise}[Exercise 1.8]
  Show that the algebraic subsets of $\A^1(k)$ are just the finite subsets together with $\A^1(k)$ itself.
\end{exercise}
\begin{solution}
  Since $k\left[x\right]$ is a principal ideal domain, we know that the zero set $V(S)$ for any $S\subseteq k\left[x\right]$ is of the form $V\left(\left\langle f \right\rangle\right) = V\left(f\right)$, where $f\in k\left[x\right]$. Since $f$ is a polynomial, $f$ has finitely many roots, so there are finitely many elements in the algebraic subset.\newline

  Additionally, since $0\in k\left[x\right]$, we know that $k$ is also an algebraic subset.
\end{solution}
\begin{exercise}[Exercise 1.14]
Let $F$ be a nonconstant polynomial in $k\left[x_1,\dots,x_n\right]$, where $k$ is algebraically closed. Show that $\A^n(k)\setminus V(F)$ is infinite if $n\geq 1$ and that $V(F)$ is infinite if $n\geq 2$. Conclude that the complement of any proper algebraic set is infinite.
\end{exercise}
\begin{solution}
  We know that $k$ is infinite as $k$ is algebraically closed.\newline

  Let $F\in k\left[x_1,\dots,x_n\right] \cong k\left[x_1,\dots,x_{n-1}\right]\left[x_n\right]$.\newline

  In the base case with $n=1$, we know that there are finitely many roots in $\A^1\left(k\right)$, so we have the base case. If $n\geq 2$, then we write $F = \sum G_ix_n^{i}$. We know that since $F$ is nonzero, then there is at least one nonzero $G_i$. We showed in Exercise 1.4 that there is some $a_1,\dots,a_{n-1}\in k$ such that $G_i\left(a_1,\dots,a_{n-1}\right)\neq 0$. Thus, $F\left(a_{1},\dots,a_{n-1},x_n\right)$ is not the zero polynomial, meaning there are finitely many roots, and thus infinitely many non-roots.\newline

  Thus, there are infinitely many $a_1,\dots,a_n\in k $ with $a_1,\dots,a_n\neq 0$.\newline

  We write $F = \sum G_ix_n^{i}$. We know that if all the $G_i$ are constant, then we have a single-variable polynomial in $x_n$, and any choice of $a_1,\dots,a_{n-1}\in k$ provide other elements of $V(F)$. We assume that there is some $G_i$ that is a nonconstant polynomial in $x_1,\dots,x_{n-1}$.\newline

  Since $G_i$ is nonzero, we may use the previous paragraph to state that $G_i$ has infinitely many non-roots, and for each choice of those $a_1,\dots,a_{n-1}$, we have a polynomial in $x_n$. This polynomial has a root, meaning there are infinitely many roots.
\end{solution}
\begin{exercise}[Exercise 1.15]
Let $V\subseteq \A^n\left(k\right)$ and $W\subseteq \A^m\left(k\right)$ be algebraic sets. Show that
\begin{align*}
  V\times W &= \set{\left(a_1,\dots,a_n,b_1,\dots,b_m\right) | \left(a_1,\dots,a_n\right)\in V,\left(b_1,\dots,b_m\right)\in W}
\end{align*}
is an algebraic set in $\A^{n+m}\left(k\right)$. It is called the product of $V$ and $W$.
\end{exercise}
\begin{solution}
  Consider the set of polynomials in $k\left[x_1,\dots,x_n,x_{n+1},\dots,x_{n+m}\right]$ given by $P = F\left(x_1,\dots,x_n\right) + G\left(x_{n+1},\dots,x_m\right)$, where $F$ is a polynomial in the ideal whose algebraic set is $V$ and $G$ is an ideal in the algebraic set whose ideal is $W$. Then, the collection of zeros are those of the form $\left(a_1,\dots,a_n,b_1,\dots,b_m\right)$, where $\left(a_1,\dots,a_n\right)\in V$ and $\left(b_1,\dots,b_m\right)\in W$.
\end{solution}
\begin{solution}[A Real Solution]
  We have that $V$ and $W$ are defined by $\set{F_1,\dots,F_r}$ and $\set{G_1,\dots,G_s}$ for some polynomials. We define $V\times W$ to be the algebraic set defined by the polynomials in $\set{F_1,\dots,F_r,G_1,\dots,G_s}$ that are constant with respect to the other variables.
\end{solution}

\subsection{The Ideal of a Set of Points}%
\begin{definition}
  If $X\subseteq \A^n\left(k\right)$, then the polynomials that vanish on $X$ form an ideal in $k\left[x_1,\dots,x_n\right]$, called the ideal of $X$, or $I(X)$.
  \begin{align*}
    I(X) &\coloneq \set{F\in k\left[x_1,\dots,x_n\right] | F\left(a_1,\dots,a_n\right) = 0\text{ for all }\left(a_1,\dots,a_n\right)\in X}.
  \end{align*}
\end{definition}
The following hold.
\begin{itemize}
  \item If $X\subseteq Y$, then $I(X)\supseteq I(Y)$.
  \item We have $I\left(\emptyset\right) = k\left[x_1,\dots,x_n\right]$, $I\left(\A^n\left(k\right)\right) = \left\langle 0 \right\rangle$ if $k$ is infinite, and $I\left(\set{\left(a_1,\dots,a_n\right)}\right) = \left\langle x_1-a_1,\dots,x_n-a_n \right\rangle$ for $a_1,\dots,a_n\in k$.
  \item We have $I\left(V(S)\right) \supseteq S$ for any set $S$ of polynomials, and $V\left(I(X)\right)\supseteq X$ for any set $X$ of points.
  \item We have $V(I(V(S))) = V(S)$ for any set of polynomials $S$, and $I(V(I(X))) = I(X)$ for any set $X$ of points. If $V$ is an algebraic set, $V = V(I(V))$ and if $I$ is the ideal of an algebraic set, then $I = I(V(I))$.
\end{itemize}
\begin{definition}
  If $I$ is any ideal in a ring $R$, we define the radical of $I$, written $\operatorname{rad}\left(I\right) = \set{a^n | a\in I\text{ for some }n > 0}$. We have that $\operatorname{rad}\left(I\right)$ is an ideal containing $I$. An ideal $I$ is called a radical ideal if $I = \operatorname{rad}\left(I\right)$.
\end{definition}
\begin{itemize}
  \item We have $I(X)$ is a radical ideal for any $X\subseteq \A^n\left(k\right)$.
\end{itemize}
\begin{exercise}[Exercise 1.16]
Let $V$ and $W$ be algebraic sets in $\A^n\left(k\right)$. Show that $V = W$ if and only if $I(V) = I(W)$.
\end{exercise}
\begin{solution}
  Let $V = W$. Then, if $F\in I(V)$, then $F = 0$ on $W$, so $F\in I(W)$, and vice versa.\newline

  Suppose $I(V) = I(W)$. We know that $V(I(V)) = V$ and $V(I(W)) = W$. Thus, if $\left(a_1,\dots,a_n\right)\in V$, we know that for all $F\in I(W)$, that $F\left(a_1,\dots,a_n\right) = 0$ as $F\in I(V)$, meaning $\left(a_1,\dots,a_n\right)\in V(I(W)) = W$. By symmetry, we have $V = W$.
\end{solution}

\begin{exercise}[Exercise 1.17]\hfill
\begin{enumerate}[(a)]
  \item Let $V$ be an algebraic set in $\A^n\left(k\right)$ and $P\in \A^n\left(k\right)$ not a point in $V$. Show that there is a polynomial $F\in k\left[x_1,\dots,x_n\right]$ such that $F(Q) = 0$ for all $Q\in V$ but $F(P) = 1$.
  \item Let $P_1,\dots,P_r$ e distinct points in $\A^n\left(k\right)$ not in an algebraic set $V$. Show that there are polynomials $F_1,\dots,F_r\in I(V)$ such that $F_i\left(P_j\right) = \delta_{ij}$.
  \item With $P_1,\dots,P_r$ and $V$ as in (b), and $a_{ij}\in k$ for $1\leq i,j\leq r$, show that there are $G_i\in I(V)$ such that $G_i\left(P_j\right) = a_{ij}$ for all $i$ and $j$.
\end{enumerate}
\end{exercise}
\begin{solution}\hfill
  \begin{enumerate}[(a)]
    \item We know that there is some $F\in I(V)$ such that $F(P)\neq 0$. Letting $a = F(P)$, we have that $\frac{1}{a}F(P) = 1$.
    \item We find $F_i\in I\left(V \cup \set{P_{-i}}\right)$, where $\set{P_{-i}} = \set{P_1,\dots,P_r}\setminus\set{P_i}$. Applying (a) to $F_i$, we get that $F_i\left(P_i\right) = 1$ and $F_i\left(P_{j}\right) = 0$ for $j\neq i$. By symmetry, this holds for $F_1,\dots,F_r$.
    \item With $P_1,\dots,P_r$ and $V$ as in (b), find $F_1,\dots,F_r$ as in (b). Then, $G_i = \sum_{j}a_{ij}F_j$ yields our desired outcome.
  \end{enumerate}
\end{solution}
\begin{exercise}[Exercise 1.18]
  Let $I$ be an ideal in a ring $R$. If $a^n\in I$ and $b^m\in I$, show that $\left(a+b\right)^{n+m}\in I$. Show that $\operatorname{rad}\left(I\right)$ is a (radical) ideal. Show that any prime ideal is radical.
\end{exercise}
\begin{solution}\hfill
  \begin{itemize}
    \item Applying binomial theorem, we have
      \begin{align*}
        \left(a+b\right)^{n+m} &= \sum_{k=0}^{n+m}{n+m\choose k}a^{n+m-k}b^{k}\\
                               &\in I,
      \end{align*}
      where $a^0 = b^{0}\coloneq 1$.
    \item We have $I\subseteq \operatorname{rad}\left(I\right)$, since we can take $n = 1$. If $a,b\in \operatorname{rad}\left(I\right)$, we know that there is some $n$ such that $a^n,b^m\in I$, so by the same logic as above, $\left(a-b\right)^{n+m}\in I$, meaning $a-b\in \operatorname{rad}\left(I\right)$. Now, if $a\in \operatorname{rad}\left(I\right)$ and $x\in R$, then we have that $a^n\in I$ for some $n$, meaning $x^na^n\in I$ as $I$ is an ideal, so $\left(xa\right)^n\in I$, so $xa\in \operatorname{rad}\left(I\right)$, so $\operatorname{rad}\left(I\right)$ is an ideal.
    \item Let $I$ be prime, and let $a\in \operatorname{rad}\left(I\right)$. Then, $a^{n}\in I$ for some $n > 0$, meaning $(a)\left(a^{n-1}\right)\in I$. Then, either $a\in I$, or $a^{n-1}\in I$, so by the implicit inductive hypothesis,  we have $a\in I$, so $\operatorname{rad}\left(I\right)\subseteq I$, so $\operatorname{rad}\left(I\right) = I$.
  \end{itemize}
\end{solution}
\begin{exercise}[Exercise 1.20]
  Show that for any ideal $I$ in $k\left[x_1,\dots,x_n\right]$, $V(I) = V\left(\operatorname{rad}\left(I\right)\right)$, and $\operatorname{rad}\left(I\right)\subseteq I\left(V\left(I\right)\right)$.
\end{exercise}
\begin{solution}\hfill
  \begin{itemize}
    \item Clearly, $V\left(\operatorname{rad}\left(I\right)\right)\subseteq V(I)$ because $I\subseteq \operatorname{rad}\left(I\right)$. We know that if $P\in V(I)$, then there is some polynomial $F\in I$ such that $F(P) = 0$.
  \end{itemize}
\end{solution}

\begin{exercise}[Exercise 1.21]
Show that any $I = \left\langle x_1 - a_1,\dots,x_n-a_n \right\rangle\subseteq k\left[x_1,\dots,x_n\right]$ is a maximal ideal, and that the natural homomorphism from $k$ to $k\left[x_1,\dots,x_n\right]/I$ is an isomorphism.
\end{exercise}
\begin{solution}
  Note that $\left\langle x_1-a_1,\dots,x_n-a_n \right\rangle\subseteq k\left[x_1,\dots,x_n\right]$ is isomorphic to $\left\langle x_1,\dots,x_n \right\rangle\subseteq k\left[x_1 + a_1,\dots,x_n + a_n\right]$, $k\left[x_1,\dots,x_n\right]/I \cong k$.
\end{solution}
\subsection{The Hilbert Basis Theorem}%
Earlier, we allowed any algebraic set $V(S)$ to be defined by an arbitrary set $\set{F_i}_{i\in I}\subseteq k\left[ x_1,\dots,x_n \right]$. However, the Hilbert Basis Theorem will show that a finite number will do.
\begin{theorem}
  Every algebraic set is the intersection of a finite number of hypersurfaces.
\end{theorem}
\begin{proof}
  We know that $V(I)$ is the algebraic set for some $I\subseteq k\left[ x_1,\dots,x_n \right]$. It is enough to show that $I$ is finitely generated, as if $I = \left\langle F_1,\dots,F_n \right\rangle$, then $V(I) = V\left( F_1 \right)\cap\cdots\cap V\left( F_n \right)$.
\end{proof}
Now, to prove this, we need to show that any arbitrary ideal $I\subseteq k\left[ x_1,\dots,x_n \right]$ is finitely generated. This is where the Hilbert Basis Theorem comes into play.
\begin{definition}
  If $R$ is a commutative ring, with identity, we say $R$ is Noetherian if every ideal of $R$ is finitely generated.
\end{definition}
Note that all PIDs are Noetherian.\newline

Now, we may state and prove the Hilbert Basis Theorem.
\begin{theorem}[Hilbert Basis Theorem]
  If $R$ is a Noetherian ring, then $R\left[ x_1,\dots,x_n \right]$ is a Noetherian ring.
\end{theorem}
\begin{proof}
  Since $R\left[ x_1,\dots,x_n \right]$ is canonically isomorphic to $R\left[ x_1,\dots,x_{n-1} \right]\left[ x_n \right]$. The theorem will follow by induction if we can prove that $R\left[ x \right]$ is Noetherian whenever $R$ is Noetherian.\newline

  Let $I\subseteq R\left[ x \right]$ be an ideal. We wish to find a finite set of generators for $I$.\newline

  Let $F = a_dx^d + \cdots a_1x + a_0\in R\left[ x \right]$ with $a_d\neq 0$. We call $a_d$ the leading coefficient of $F$. Let $J$ be the set of leading coefficients of polynomials in $I$. Then, $J\subseteq R$ is an ideal, so there are polynomials $F_1,\dots,F_r\in I$ whose leading coefficients generate $J$.\newline

  Select $N$ larger than the degree of each $F_i$. For each $m\leq N$, let $J_m$ be the ideal in $R$ consisting of all leading coefficients of polynomials $F\in I$ with $\deg\left( F \right) \leq m$. Let $\set{F_{m_j}}$ be the finite set of polynomials in $I$ with degree $\leq m$ such that their leading coefficients generate $J_m$. Let $I'$ be the ideal generated by $F_i$ and $F_{m_j}$ for each $i,m_j$. It is enough to show that $I = I'$.\newline

  Suppose $I'\subsetneq I$. Let $G$ be an element of $I$ of minimal degree such that $G\notin I'$. If $\deg(G) > N$, then we may find $Q_i$ such that $\sum Q_iF_i$ and $G$ have the same leading term. However, this means $\deg\left( G - \sum Q_iF_i \right) < \deg(G)$, so $G - \sum Q_iF_i\in I'$, meaning $G\in I'$. Similarly, if $\deg\left( G \right) = m \leq N$, then we may lower the degree by subtracting $\sum Q_jF_{m_j}$ for some $Q_j$.
\end{proof}
\begin{exercise}[Exercise 1.22]
Let $I$ be an ideal in a ring $R$, $\pi\colon R\rightarrow R/I$ the canonical projection.
\begin{enumerate}[(a)]
  \item Show that for every ideal $J'\subseteq R/I$, that $\pi^{-1}\left( J' \right) = J$ is an ideal of $R$ containing $I$. Furthermore, show that for every ideal $J\subseteq R$, that $\pi\left( J \right) = J'$ is an ideal of $R/I$. This establishes a natural correspondence between ideals of $R/I$ and ideals of $R$ that contain $I$.
  \item Show that $J'$ is a radical ideal if and only if $J$ is radical. Similarly, show this for $J$ prime and maximal.
  \item Show that $J'$ is finitely generated if $J$ is. Conclude that $R/I$ is Noetherian if $R$ is Noetherian. Thus, we get that $k\left[ x_1,\dots,x_n \right]/I$ is Noetherian for any ideal $I\subseteq k\left[ x_1,\dots,x_n \right]$ by the Hilbert Basis Theorem.
\end{enumerate}
\end{exercise}
\begin{solution}\hfill
  \begin{enumerate}[(a)]
    \item We know that $I\subseteq \pi^{-1}\left( J' \right)$, as $ I = \pi^{-1}\left(0 + I\right) \subseteq \pi^{-1}\left( J' \right)$. Notice that, if $a,b\in \pi^{-1}\left(J'\right)$ and $r\in R$, then $a + I,b + I\in J'$ and $r + I\in R/I$. Then, $a-b +I\in J'$, so $a-b\in \pi^{-1}\left( J' \right)$, and $ra + I \in J'$, so $ra\in \pi^{-1}\left( J' \right)$, so $\pi^{-1}\left( J' \right)$ is an ideal of $R$.\newline

      Now, let $a+I,b+I\in \pi(J)$. Then, we know that there exist $c_1,c_2\in J$ such that $a-c_1,b-c_2\in I$. Thus, $\left( a-b \right) + \left( c_2 - c_1 \right)\in I$. Since we have $c_2 - c_1\in J$ as $J$ is an ideal, so $\pi\left( a-b \right) = \pi\left( c_2 - c_1 \right)$, and $\left( a-b \right) + I\in \pi(J)$. Now, let $a + I\in \pi(J)$, and let $r + I\in R/I$. Then, there exist $c_1\in R$, $c_2\in J$ such that $r-c_1\in I$ and $a - c_2\in I$, meaning that $\pi\left( c_1c_2 \right) = \pi\left( ra \right) = ra + I \in \pi(J)$.
    \item Let $J$ be maximal. Then, $R/J \cong \left( R/I \right)/\left( \pi(J) \right)$, is a field, meaning $\pi(J)\subseteq R/I$ is also maximal. This gives both directions.\newline

    Similarly, if $J$ is prime, then $R/J \cong \left( R/I \right)/\left( \pi(J) \right)$ is an integral domain, so $\pi(J)\subseteq R/I$ is also an integral domain. This gives both directions.\newline

    Let $J$ be a radical ideal. Then, $J = \bigcap\set{\mathfrak{p} | J\subseteq \mathfrak{p},\mathfrak{p}\text{ is prime}}$. We know that for all $\mathfrak{p}$, $\pi\left( \mathfrak{p} \right)\subseteq R/I$ is prime. We know that $\pi(J)\subseteq \pi\left( \mathfrak{p} \right)$ if and only if $J\subseteq \mathfrak{p}$, so $\pi(J) = \bigcap\set{\pi(\mathfrak{p}) | J\subseteq \mathfrak{p},\mathfrak{p}\text{ is prime}}$. In the reverse direction, we se that if $a\in \pi^{-1}(J)$, then $a + I\in J$, so $a^n + I\in J$ for some $n\in \N$, so $a^n\in \pi^{-1}(J)$, so $\pi^{-1}(J)$ is a radical ideal.
  \item Letting $\left\langle a_1,\dots,a_n \right\rangle = J$, then we know that $ \left\langle \pi\left(a_1\right),\dots,\pi\left(a_n\right) \right\rangle= \pi(J)$. Thus, $\pi(J)$ is finitely generated.\newline

    Since $R$ is an ideal, if $R$ is Noetherian, then $R/I$ is Noetherian, so by the Hilbert Basis Theorem, any ring of the form $k\left[ x_1,\dots,x_n \right]/I$ is Noetherian.
  \end{enumerate}
\end{solution}
\subsection{Irreducible Components of an Algebraic Set}%
An algebraic set can be the union of several smaller algebraic sets. If $V\subseteq \A^n$ is such that $V = V_1 \cup V_2$, where $V_1,V_2$ are algebraic sets and $V_i\neq V$ for each $i$, then we say $V$ is reducible. Else, we say $V$ is irreducible.
\begin{proposition}
  An algebraic set $V$ is irreducible if and only if $I(V)$ is prime.
\end{proposition}
\begin{proof}
  If $I(V)$ is not prime, then we have $F_1F_2\in I(V)$ with $F_i\notin I(V)$. Then, $V = \left( V\cap V\left(F_1\right) \right)\cup\left( V\cap V\left(F_2\right) \right)$, with $V\cap V\left( F_i \right)\subsetneq V$, meaning $V$ is irreducible.\newline

  If $V = V_1\cup V_2$ with $V_i\subsetneq V$, then $I\left( V_i \right)\supseteq I(V)$. Let $F_i\in I\left( V_i \right)$ with $F_i\notin I(V)$. Then, $F_1F_2\in I(V)$, so $I(V)$ is not prime.
\end{proof}
Now, we want to show that an algebraic set is a finite union of irreducible algebraic sets. To see this, we need to show an equivalent definition of a Noetherian ring.
\begin{lemma}
  Let $\mathcal{I}$ be a nonempty collection of ideals in a Noetherian ring $R$. Then, $\mathcal{I}$ has a maximal member.
\end{lemma}
\begin{proof}
  We will choose an ideal from each subset of $\mathcal{I}$. Letting $I_0$ be the chosen ideal for $\mathcal{I}$ itself, we let $\mathcal{I}_1 = \set{I\in \mathcal{I} | I\supsetneq I_0}$, with $I_1$ as the chosen ideal of $\mathcal{I}_1$. Continuing, we define
  \begin{align*}
    \mathcal{I}_j &= \set{I\in \mathcal{I} | I\supsetneq I_{j-1}},
  \end{align*}
  and select $I_j\in \mathcal{I}_j$. It suffices to show that some $\mathcal{I}_n$ is empty.\newline

  Define $I = \bigcup_{n=0}^{\infty}I_n$ to be an ideal of $R$, and let $F_1,\dots,F_r$ be generators of $I$. We must have $F_i\in I_n$ for all $i$ if $n$ is sufficient large. Then, $I_n = I$, meaning $I_{n+1} = I_{n}$, which is a contradiction.
\end{proof}
Effectively, we have shown that every Noetherian ring satisfies the ascending chain condition on its ideals.\newline

It follows that any collection of algebraic sets $\set{V_{\alpha}}$ in $\A^n\left( k \right)$ has a minimal element, by selecting the maximal member of $\set{I\left( V_{\alpha} \right)}$.
\begin{theorem}
  Let $V$ be an algebraic set in $\A^n\left( k \right)$. Then, there rae unique irreducible algebraic sets $V_1,\dots,V_m$ such that $V=  V_1\cup \cdots \cup V_m$, and $V_i\nsubseteq V_j$ for all $i\neq j$.
\end{theorem}
\begin{proof}
  Let $\mathcal{I}$ be the set of algebraic sets in $\A^n\left( k \right)$ such that $V$ is not the union of a finite number of irreducible algebraic sets. We wish to show that $\mathcal{I}$ is empty.\newline

  If not, let $V$ be a minimal member of $\mathcal{I}$. Since $V\in \mathcal{I}$, $V$ is not irreducible, so $V = V_1\cup V_2$ with $V_i\subsetneq V$, meaning $V_i\notin \mathcal{I}$, so $V_i = V_{i,1}\cup\cdots V_{i,m_i}$, with $V_{i,j}$ irreducible. However, $V = \bigcup_{i,j} V_{i,j}$, which is a finite union.\newline

  Thus, any algebraic set $V$ may be written as $V = V_1\cup\cdots\cup V_m$ with $V_i$ irreducible. To obtain the second condition, we may discard any $V_i$ with $V_i\subseteq V_j$ with $i\neq j$.\newline

  To show uniqueness, let $V = W_1\cup\cdots\cup W_m$ be another decomposition. Then, $V_i= \bigcup_{j}\left( W_j\cap V_i \right)$, so $V_i\subseteq W_{j(i)}$ for some $j(i)$. Similarly, $W_{j(i)}\subseteq V_k$ for some $k$. However, this means $V_i\subseteq V_k$, so $i = k$, so $V_i = W_{j(i)}$. Likewise, $W_j = V_{i(j)}$ for some $i(j)$.
\end{proof}
We call $V_i$ the irreducible components of $V$, and $V = V_1\cup\cdots\cup V_m$ is the decomposition of $V$ into irreducible components.
\begin{exercise}[Exercise 1.25]\hfill
  \begin{enumerate}[(a)]
    \item Show that $V\left(y-x^2\right)\subseteq \A^2\left(\C\right)$ is irreducible; in fact, $I\left(V\left(y-x^2\right)\right) = \left\langle y-x^2 \right\rangle$.
    \item Decompose $V\left( y^4-x^2,y^4-x^2y^2+xy^2-x^3 \right)\subseteq \A^2\left( \C \right)$ into irreducible components.
  \end{enumerate}
\end{exercise}
\begin{solution}\hfill
  \begin{enumerate}[(a)]
    \item Suppose there exists $g\in \C\left[ x,y \right]$ such that $g | y-x^2$, meaning there exists $f\in \C\left[ x,y \right]$ such that $fg = y-x^2$. Since $y-x^2$ has degree in $y$ equal to $1$, one of either $f$ or $g$ has degree in $y$ equal to zero.\newline

      Therefore, without loss of generality, $f\in \C\left[ x \right]$. Then, $g = yh_1 + h_2$, where $h_1,h_2\in \C\left[ x \right]$. Note that $h_1\neq 0$, then $fg = fyh_1 + fh_2 = yfh_1 + fh_2$; since $fh_1\neq 0$, we must have $fh_1 = 1$, so $f$ is constant, so $g$ is some constant multiple of $y-x^2$, so $y-x^2$ is irreducible. Thus, $\left\langle y-x^2 \right\rangle$ is maximal, hence prime, so $I\left( V\left( y-x^2 \right) \right) = \left\langle y-x^2 \right\rangle$.
    \item Factoring, we see that both polynomials vanish whenever $y^2 + x = 0$. Finding all pairs, we get
      \begin{align*}
        V &= V\left( y^2-x,y^2 + x \right) \cup V\left( y^2- x,y-x \right) \cup \cdots\\
          &= V\left( y^2 + x \right) \cup V\left( x-1,y-1 \right) \cup V\left( x-1,y+1 \right).
      \end{align*}
  \end{enumerate}
\end{solution}
\begin{solution}\hfill
  \color{red}
  \begin{enumerate}[(a)]
    \item Let $g\in I(V)$. Then,
      \begin{align*}
        g(x,y) &= f_0\left( x \right) + \left( y-x^2 \right)f_1\left( x,y \right),
      \end{align*}
      wherein we order $y > x$ and do polynomial long division over $y$. This yields $f_0(x) = 0$ for all $x$, so that $I\left( V \right)$ is prime.
  \end{enumerate}
\end{solution}

\begin{exercise}[Exercise 1.29]
  Show that $\A^n\left( k \right)$ is irreducible if $k$ is infinite.
\end{exercise}
\begin{solution}
  We know that any polynomial that vanishes on $\A^n\left( k \right)$ is the zero polynomial, and $k\left[ x_1,\dots,x_n \right]$ is an integral domain, so $\left\langle 0 \right\rangle\subseteq k\left[ x_1,\dots,x_n \right]$ prime ideal.
\end{solution}

\subsection{Algebraic Subsets of the Plane}%
\begin{exercise}[Exercise 1.30]
Let $k = \R$. 
\begin{enumerate}[(a)]
  \item Show that $I\left(V\left(x^2 + y^2 + 1\right)\right) = \left\langle 1 \right\rangle$.
  \item Show that every algebraic subset of $\A^2\left(\R\right)$ is equal to $V(F)$ for some $F\in \R\left[x,y\right]$.
\end{enumerate}
\end{exercise}
\begin{exercise}[Exercise 1.31]\hfill
  \begin{enumerate}[(a)]
    \item Find the irreducible components of $V\left( y^2 - xy - x^2y + x^3 \right)$ in $\A^2\left( \R \right)$, and in $\A^2\left( \C \right)$.
    \item Do the same for $V\left( y^2 - x\left( x^2 - 1 \right) \right)$, and for $V\left( x^3 + x - x^2 y - y \right)$.
  \end{enumerate}
\end{exercise}
\subsection{Hilbert's Nullstellensatz}%
\begin{exercise}[Exercise 1.33]\hfill
  \begin{enumerate}[(a)]
    \item Decompose $V\left( x^2 + y^2 - 1,x^2 - z^2 - 1 \right)\subseteq \A^3\left( \C \right)$ into irreducible components.
    \item Let $V = \set{\left( t,t^2,t^3 \right)\in \A^3\left( \C \right) | t\in \C}$. Find $I(V)$ and show that $V$ is irreducible.
  \end{enumerate}
\end{exercise}
\begin{exercise}[Exercise 1.36]
  Let $I = \left\langle y^2-x^2,y^2 + x^2 \right\rangle\subseteq \C\left[ x,y \right]$. Find $V(I)$ and $\Dim_{\C}\left( \C\left[ x,y \right]/I \right)$.
\end{exercise}
\begin{exercise}[Exercise 1.37]
  Let $K$ be any field, $F\in K\left[ x \right]$ a polynomial of degree $n > 0$. Show that the residues $\overline{1},\overline{x},\dots,\overline{x}^{n-1}$ form a basis for $K\left[ x \right]/\left\langle F \right\rangle$ over $K$.
\end{exercise}
\begin{exercise}[Exercise 1.38]
  Let $R = k\left[ x_1,\dots,x_n \right]$ with $k$ algebraically closed. Let $V = V(I)$. Show that there is a natural one-to-one correspondence between algebraic subsets of $V$ and radical ideals in $k\left[ x_1,\dots,x_n \right]/I$, and that irreducible algebraic sets (points) correspond to prime ideals (maximal ideals).
\end{exercise}
\subsection{Modules and Finiteness}%

\end{document}
