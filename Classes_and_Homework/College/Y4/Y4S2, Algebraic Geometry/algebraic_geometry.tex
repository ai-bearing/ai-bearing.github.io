\documentclass[10pt]{mypackage}

% sans serif font:
%\usepackage{cmbright,sfmath,bbold}
%\renewcommand{\mathcal}{\mathtt}

%Euler:
\usepackage{newpxtext,eulerpx,eucal,eufrak}
\renewcommand*{\mathbb}[1]{\varmathbb{#1}}
\renewcommand*{\hbar}{\hslash}
\newcommand{\A}{\mathbb{A}}

%kp fonts:
%\usepackage{kpfonts}
%\renewcommand{\mathbb}{\mathds}
\usepackage{homework}
\usepackage{microtype}

\pagestyle{fancy} %better headers
\fancyhf{}
\rhead{Avinash Iyer}
\lhead{Algebraic Geometry}

\setcounter{secnumdepth}{0}

\begin{document}
\RaggedRight
\section{Introduction}%
Oh hey, it's another one of these independent studies. Me and a friend are going to be going through William Fulton's \textit{Algebraic Curves}. It will be hard, it will be long, and it might not work out for me, but who cares.
\tableofcontents
\section{Affine Algebraic Sets}%
\subsection{Algebraic Preliminaries}%
We will assume all rings are commutative with unity, where $\Z$ is the integers, $\Q$ is the rationals, $\R$ is the reals, and $\C$ is the complex numbers.\newline

Any integral domain $R$ has a quotient field $K$, which contains $R$ as a subring, and any element in $K$ may be written as a not necessarily unique ratio of two elements of $R$. Any one-to-one ring homomorphism from $R$ to a field $L$ extends uniquely to a ring homomorphism from $K$ to $L$.\newline

If $R$ is a ring, then $R[x]$ is the ring of polynomials with coefficients in $R$. The degree of a nonzero polynomial $\sum a_ix^i$ is the largest integer $d$ such that $a_d\neq 0$. The polynomial is monic if $a_d = 1$.\newline

The ring of polynomials in $n$ variables over $R$ is $R\left[x_1,\dots,x_n\right]$. We write $R\left[x,y\right]$ and $R\left[x,y,z\right]$ if $n=2$ and $3$ respectively. Monomials in $R\left[x_1,\dots,x_n\right]$ are of the form $x^{(i)} \coloneq x_1^{i_1}x_2^{i_2}\cdots x_n^{i_n}$, where $i_j$ are nonnegative integers, and the degree of the monomial is $i_1 + \cdots i_n$. Every $F\in R\left[x_1,\dots,x_n\right]$ has a unique expression $F = \sum a_{(i)}x^{(i)}$, where $x^{(i)}$ are monomials, and $a_{(i)}\in R$. We say $F$ is homogeneous of degree $d$ if all $a_{(i)}$ are zero except for monomials of degree $d$. The polynomial $F$ is written as $F = F_0 + F_1 + \cdots F_d$, where $F_i$ is a form of degree $i$, and $d = \deg(F)$ for $F_d\neq 0$.\newline

The ring $R$ is a subring of $R\left[x_1,\dots,x_n\right]$, and the ring $R\left[x_1,\dots,x_n\right]$ is characterized by the following: if $\varphi\colon R\rightarrow S$ is a ring homomorphism, and $s_1,\dots,x_n$ are elements in $S$, then there is a unique extension of $\varphi$ to a ring homomorphism $\overline{\varphi}\colon R\left[x_1,\dots,x_n\right]\rightarrow S$ such that $\overline{\varphi}\left(x_i\right) = s_i$. The image of $F$ under $\overline{\varphi}$ is written $F\left(s_1,\dots,s_n\right)$. The ring $R\left[x_1,\dots,x_n\right]$ is canonically isomorphic to $R\left[x_1,\dots,x_{n-1}\right]\left[x_n\right]$.\newline

An element $a\in R$ is called irreducible if it is not a unit or zero, and any factorization $a=bc$ with $b,c\in R$ is such that either $b $ or $c$ is a unit. A domain $R$ is a unique factorization domain (UFD) if every nonzero element in $R$ can be factored uniquely up to units and ordering.\newline

If $R$ is a UFD with quotient field $K$, then any irreducible element $F\in R[x]$ remains irreducible when considered in $K[x]$.
\begin{theorem}[Gauss's Lemma for $\Z$]
  If $F\in \Z[x]$ is a monic polynomial that is irreducible, then $F$ is irreducible in $\Q[x]$.
\end{theorem}
If $F$ and $G$ are polynomials in $R[x]$ with no common factors in $R[x]$, then they have no common factors in $K[x]$.\newline

If $R$ is a UFD, then $R[x]$ is also a UFD, and consequently $k\left[x_1,\dots,x_n\right]$ is a UFD for any field $k$. The quotient field of $k\left[x_1,\dots,x_n\right]$ is written $k\left(x_1,\dots,x_n\right)$ is called the field of rational functions in $n$ variables over $k$.\newline

If $\varphi\colon R\rightarrow S$ is a ring homomorphism, $\ker\left(\varphi\right)\coloneq \varphi^{-1}(0)$. The kernel is an ideal in $R$. An ideal in $R$ is proper if $I\neq R$, and a proper ideal is known as maximal if it is not contained in any larger proper ideal.\footnote{Alternatively, an ideal $I$ is maximal if the quotient ring $R/M$ is a field.} An ideal $\mathfrak{p}$ is prime if, whenever $ab\in \mathfrak{p}$, then $a\in \mathfrak{p}$ or $b\in \mathfrak{p}$.\footnote{Alternatively, an ideal $\mathfrak{p}$ is prime if $R/\mathfrak{p}$ is an integral domain.}\newline

Let $k$ be a field and $I$ a proper ideal in $k\left[x_1,\dots,x_n\right]$. The canonical homomorphism $\pi$ from $k\left[x_1,\dots,x_n\right]$ to $k\left[x_1,\dots,x_n\right]/I$ restricts to a ring homomorphism from $k$ to $k\left[x_1,\dots,x_n\right]/I$. We regard $k$ as a subring of $k\left[x_1,\dots,x_n\right]/I$, which is a vector space over $k$.\newline

If $R$ is an integral domain, then $\operatorname{char}\left(R\right)$, the characteristic of $R$, is the smallest integer $p$ such that 
\begin{align*}
  \underbrace{1+1\cdots +1}_{p\text{ times}} = 0. 
\end{align*}
If $p$ exists, we say $\operatorname{char}\left(R\right) = p$, else $0$.\newline

Note that if $\varphi\colon \Z\rightarrow R$ is the unique ring homomorphism from $\Z$ to $R$,\footnote{This is because $\Z$ is initial in the category of rings. See Aluffi.} then $\ker\left(\varphi\right) = \left\langle p \right\rangle$, so $\operatorname{char}\left(R\right)$ is prime or $0$.\newline

If $R$ is a ring, and $F\in R\left[x\right]$, and $a$ is a root of $F$, then $F = \left(x-a\right)G$ for some unique polynomial $G\in R[x]$. A field $k$ is algebraically closed if any nonconstant $F\in k\left[x\right]$ has a root.
\begin{exercise}[Exercise 1.1]
Let $R$ be an integral domain.
\begin{enumerate}[(a)]
  \item If $F$ and $G$ are forms of degree $r$ and $s$ respectively in $R\left[x_1,\dots,x_n\right]$, show that $FG$ is a form of degree $r+s$.
  \item Show that any factor of a form in $R\left[x_1,\dots,x_n\right]$ is also a form.
\end{enumerate}
\end{exercise}
\begin{solution}\hfill
  \begin{enumerate}[(a)]
    \item Let $H = FG$, where $F$ is a form of degree $r$ and $G$ is a form of degree $s$. Note that since $F$ and $G$ are forms, we know that $F = F_{r}$, where $F_r$ is the form with degree $r$, and $G = G_s$, where $G_s$ is the form with degree $s$.
  \end{enumerate}
  
\end{solution}

\begin{exercise}[Exercise 1.2]
  Let $R$ be a UFD and $K$ the quotient field of $R$. Show that every element $z\in K$ may be written as $z = a/b$, where $a,b\in R$ have no common factors. This representative is unique up to units of $R$.
\end{exercise}
\begin{solution}
  Since $K = \operatorname{Frac}\left(R\right)$, we know that every $z\in K$ is of the form $z = \frac{a}{b}$. Since $R$ a unique factorization domain, $\gcd\left(a,b\right)$ is unique and well-defined. Set $c\cdot \gcd\left(a,b\right) = a$ and $d\cdot \gcd\left(a,b\right) = b$. Then,
  \begin{align*}
    z &= \frac{a}{b}\\
      &= \frac{c\cdot \gcd\left(a,b\right)}{d\cdot \gcd\left(a,b\right)}\\
      &= \frac{c}{d}.
  \end{align*}
  We show that this is unique up to units. Suppose
  \begin{align*}
    z &= \frac{c}{d}\\
      &= \frac{c'}{d'}.
  \end{align*}
  Then, by the properties of the field of fractions, we know that
  \begin{align*}
    c'd &= cd',
  \end{align*}
  and since $R$ is a UFD, we know that $\gcd\left(c,d\right) = \gcd\left(c',d'\right) = 1$, so $c = u_1c'$ and $d = u_2d'$.
\end{solution}
\begin{exercise}[Exercise 1.3]
  Let $R$ be a principal ideal domain, and let $P$ be a nonzero proper prime ideal in $R$.
  \begin{enumerate}[(a)]
    \item Show that $P$ is generated by an irreducible element.
    \item Show that $P$ is maximal.
  \end{enumerate}
\end{exercise}
\begin{solution}\hfill
  \begin{enumerate}[(a)]
    \item Since $P$ is principal, we know that $ P = \left\langle a \right\rangle $ for some $a\in R$. We know that $a$ cannot be a unit, as otherwise $P = R$, contradicting the assumption that $P$ is proper, and that $a\neq 0$ as $P$ is not zero.\newline

      Suppose toward contradiction that $\left\langle a \right\rangle\subsetneq \left\langle b \right\rangle$ for some $b\in R$. Then, $a = bc$ for some $c\in R$. If $c\notin \left\langle a \right\rangle$, then since $\left\langle a \right\rangle$ is prime, we must have $b\in \left\langle a \right\rangle$, contradicting strict inclusion. Thus, $c\in \left\langle a \right\rangle$, so $c = at$ for some $t\in R$. Therefore, we have $a = abt$, so $bt = 1_R$, and $\left\langle b \right\rangle = R$.
    \item Since $R$ is a PID, and $P$ is prime, we know that $P = \left\langle a \right\rangle$ is generated by an irreducible element. Thus, if $\left\langle a \right\rangle\subseteq \left\langle b \right\rangle$, then $a = bc$ for some $c\in R$. Since we have unique factorization (as all PIDs are UFDs), and $a$ is irreducible, this means either $b$ or $c$ is a unit. If $b$ is a unit, then $\left\langle b  \right\rangle = R$, and if $c$ is a unit, then $\left\langle b \right\rangle = \left\langle a \right\rangle$. Thus, $\left\langle a \right\rangle$ is maximal.
  \end{enumerate}
\end{solution}
\begin{exercise}[Exercise 1.4]
Let $k$ be an infinite field, $f\in k\left[x_1,\dots,x_n\right]$. Suppose $F\left(a_1,\dots,a_n\right) = 0$ for all $a_1,\dots,a_n\in k$. Show that $f = 0$.
\end{exercise}
\begin{exercise}[Exercise 1.5]
Let $k$ be any field. Show that there are an infinite number of irreducible monic polynomials in $k\left[x\right]$.
\end{exercise}
\begin{solution}
  Suppose $F_1,\dots,F_n$ were all the irreducible monic polynomials in $k\left[x\right]$. Consider the polynomial $P = F_1F_2\cdots F_n + 1$. We note that $P$ is monic. We will show that $P$ is irreducible.\newline

  Suppose toward contradiction that $P$ were reducible. We know that $k\left[x\right]$ is a principal ideal domain, so $P\in \left\langle F_i \right\rangle$ for some irreducible monic $F_i$. However, we know that, for any $F_i$, $1\leq i\leq n$, $P \nmid F_i$, as, applying the division algorithm to $P$, we get
  \begin{align*}
    P &= \left(F_i\right)\prod_{j\neq i}F_j + 1,
  \end{align*}
  where $r \neq 0$. Thus, $P$ is not reducible and monic, so there are infinitely many irreducible monic polynomials in $k\left[x\right]$.
\end{solution}

\begin{exercise}[Exercise 1.6]
Show that any algebraically closed field is infinite.
\end{exercise}
\begin{solution}
  Note that if $k$ is any field, then there are infinitely many irreducible monic polynoimals in $k\left[x\right]$. If $k$ is algebraically closed, then $\left(x-a\right)$, for $a\in k$, is the only irreducible monic polynomial. Since there are infinitely many irreducible monic polynomials in $k\left[x\right]$, there are infinitely many $a\in k$ such that $\left(x-a\right)$ is irreducible in $k\left[x\right]$. Thus, $k$ is infinite.
\end{solution}

\begin{exercise}[Exercise 1.7]
Let $k$ be any field, and $F\in k\left[x_1,\dots,x_n\right]$, with $a_1,\dots,a_n\in k$.
\begin{enumerate}[(a)]
  \item Show that
    \begin{align*}
      F &= \sum\lambda_{(i)}\left(x_1-a_1\right)^{i_1}\cdots \left(x_n-a_n\right)^{i_n},
    \end{align*}
    where $\lambda_{(i)}\in k$.
  \item If $F\left(a_1,\dots,a_n\right) = 0$, show that $F = \sum_{i=1}^{n}\left(x_i - a_i\right)G_i$ for some not necessarily unique $G_i\in k\left[x_1,\dots,x_n\right]$.
\end{enumerate}
\end{exercise}
\begin{solution}\hfill
  \begin{enumerate}[(a)]
    \item We let
      \begin{align*}
        G &= F\left(x_1+a_1,x_2 + a_2,\dots,x_n + a_n\right).
      \end{align*}
      Then, since $G\in k\left[x_1,\dots,x_n\right]$, we have
      \begin{align*}
        G &= \sum \lambda_{(i)}x_1^{i_1}\cdots x_n^{i_n}.
      \end{align*}
      Then, we have
      \begin{align*}
        F &= \sum \lambda_{(i)}\left(x_1 - a_1\right)^{i_1} \cdots \left(x_n-a_n\right)^{i_n}.
      \end{align*}
    \item Note that if $F\left(a_1,\dots,a_n\right) = 0$, then $\left(x_i - a_i\right) \mid F\left(a_1,\dots,a_{i-1},x_i,a_{i+1},\dots,a_n\right)$. Thus, we have
      \begin{align*}
        F\left(a_1,\dots,a_{i-1},x_i,a_{i+1},\dots,a_n\right) &= \left(x_i - a_i\right)\underbrace{g\left(a_1,\dots,a_{i-1},x_i,a_{i+1},\dots,a_n\right)}_{G_i}.
      \end{align*}
      This yields
      \begin{align*}
        F\left(x_1,\dots,x_n\right) &= \sum_{i=1}^{n}\left(x_i - a_i\right)G_i.
      \end{align*}
  \end{enumerate}
\end{solution}
\subsection{Affine Space and Algebraic Sets}%
\begin{definition}
  If $k$ is a field, then when we write $\mathbb{A}^n\left(k\right)$, or $\mathbb{A}^n$, to be the cartesian product of $k$ with itself $n$ times.\newline

  We call $\mathbb{A}^n\left(k\right)$ the affine $n$-space over $k$. Its elements are called points. We call $\mathbb{A}^1(k)$ the affine line and $\A^2(k)$ the affine plane.
\end{definition}
\begin{definition}
  If $F\in k\left[x_1,\dots,x_n\right]$, then $P = \left(a_1,\dots,a_n\right)\in \A^n\left(k\right)$ is called a zero of $F$ if $F(P) = \left(a_1,\dots,a_n\right) = 0$.\newline

  If $F$ is not constant, then the zeros of $F$ are called the hypersurface defined by $F$, defined by $V(F)$. A hypersurface in $\A^2\left(k\right)$ is called an affine plane curve.\newline

  If $F$ is a polynomial of degree $1$, then $V(F)$ is called a hyperplane in $\A^n\left(k\right)$; if $n = 2$, then an affine hyperplane is a line.
\end{definition}
\begin{definition}
  If $S$ is any set of polynomials in $k\left[x_1,\dots,x_n\right]$, then $V(S) = \set{P\in \A^n | F(P) = 0\text{ for all }F\in S}$. In other words, $V(S) = \bigcap_{F\in S}V(F)$. If $S = \set{F_1,\dots,F_r}$, we write $V\left(F_1,\dots,F_r\right)$.\newline

  A subset $X\subseteq \A^n\left(k\right)$ is an affine algebraic set (or algebraic set) if $X = V(S)$ for some $S$.
\end{definition}
\begin{proposition}\hfill
  \begin{enumerate}[(1)]
    \item If $I$ is the ideal in $k\left[x_1,\dots,x_n\right]$ generated by $S$, then $V(S) = V(I)$; thus, every algebraic set is equal to $V(I)$ for some ideal $I$.
    \item If $\set{I_{\alpha}}$ is a collection of ideals, then $V\left(\bigcup_{\alpha}I_{\alpha}\right) = \bigcap_{\alpha}V\left(I_{\alpha}\right)$.
    \item If $I\subseteq J$, then $V(I)\supseteq V(J)$.
    \item For any polynomials$F,G$, $V\left(FG\right) = V\left(F\right)\cup V\left(G\right)$. Furthermore, $V(I)\cup V(J) = V\left(\set{FG | F\in I, G\in J}\right)$.
    \item We have that $V(0) = \A^n(k)$, $V(1) = \emptyset$, $V\left(x_1-a_1,\dots,x_n-a_n\right) = \set{(a_1,\dots,a_n)}$ for $a_i\in k$. Thus, any finite subset of $\A^n(k)$ is an algebraic set.
  \end{enumerate}
\end{proposition}
\begin{exercise}[Exercise 1.8]
  Show that the algebraic subsets of $\A^1(k)$ are just the finite subsets together with $\A^1(k)$ itself.
\end{exercise}
\begin{solution}
  Since $k\left[x\right]$ is a principal ideal domain, we know that the zero set $V(S)$ for any $S\subseteq k\left[x\right]$ is of the form $V\left(\left\langle f \right\rangle\right) = V\left(f\right)$, where $f\in k\left[x\right]$. Since $f$ is a polynomial, $f$ has finitely many roots, so there are finitely many elements in the algebraic subset.\newline

  Additionally, since $0\in k\left[x\right]$, we know that $k$ is also an algebraic subset.
\end{solution}
\begin{exercise}[Exercise 1.14]
Let $F$ be a nonconstant polynomial in $k\left[x_1,\dots,x_n\right]$, where $k$ is algebraically closed. Show that $\A^n(k)\setminus V(F)$ is infinite if $n\geq 1$ and that $V(F)$ is infinite if $n\geq 2$. Conclude that the complement of any proper algebraic set is infinite.
\end{exercise}
\begin{solution}
  We know that $k$ is infinite as $k$ is algebraically closed.\newline

  Let $F\in k\left[x_1,\dots,x_n\right] \cong k\left[x_1,\dots,x_{n-1}\right]\left[x_n\right]$.\newline

  In the base case with $n=1$, we know that there are finitely many roots in $\A^1\left(k\right)$, so we have the base case. If $n\geq 2$, then we write $F = \sum G_ix_n^{i}$. We know that since $F$ is nonzero, then there is at least one nonzero $G_i$. We showed in Exercise 1.4 that there is some $a_1,\dots,a_{n-1}\in k$ such that $G_i\left(a_1,\dots,a_{n-1}\right)\neq 0$. Thus, $F\left(a_{1},\dots,a_{n-1},x_n\right)$ is not the zero polynomial, meaning there are finitely many roots, and thus infinitely many non-roots.\newline

  Thus, there are infinitely many $a_1,\dots,a_n\in k $ with $a_1,\dots,a_n\neq 0$.\newline

  We write $F = \sum G_ix_n^{i}$. We know that if all the $G_i$ are constant, then we have a single-variable polynomial in $x_n$, and any choice of $a_1,\dots,a_{n-1}\in k$ provide other elements of $V(F)$. We assume that there is some $G_i$ that is a nonconstant polynomial in $x_1,\dots,x_{n-1}$.\newline

  Since $G_i$ is nonzero, we may use the previous paragraph to state that $G_i$ has infinitely many non-roots, and for each choice of those $a_1,\dots,a_{n-1}$, we have a polynomial in $x_n$. This polynomial has a root, meaning there are infinitely many roots.
\end{solution}
\begin{exercise}[Exercise 1.15]
Let $V\subseteq \A^n\left(k\right)$ and $W\subseteq \A^m\left(k\right)$ be algebraic sets. Show that
\begin{align*}
  V\times W &= \set{\left(a_1,\dots,a_n,b_1,\dots,b_m\right) | \left(a_1,\dots,a_n\right)\in V,\left(b_1,\dots,b_m\right)\in W}
\end{align*}
is an algebraic set in $\A^{n+m}\left(k\right)$. It is called the product of $V$ and $W$.
\end{exercise}
\begin{solution}
  Consider the set of polynomials in $k\left[x_1,\dots,x_n,x_{n+1},\dots,x_{n+m}\right]$ given by $P = F\left(x_1,\dots,x_n\right) + G\left(x_{n+1},\dots,x_m\right)$, where $F$ is a polynomial in the ideal whose algebraic set is $V$ and $G$ is an ideal in the algebraic set whose ideal is $W$. Then, the collection of zeros are those of the form $\left(a_1,\dots,a_n,b_1,\dots,b_m\right)$, where $\left(a_1,\dots,a_n\right)\in V$ and $\left(b_1,\dots,b_m\right)\in W$.
\end{solution}
\begin{solution}[A Real Solution]
  We have that $V$ and $W$ are defined by $\set{F_1,\dots,F_r}$ and $\set{G_1,\dots,G_s}$ for some polynomials. We define $V\times W$ to be the algebraic set defined by the polynomials in $\set{F_1,\dots,F_r,G_1,\dots,G_s}$ that are constant with respect to the other variables.
\end{solution}

\subsection{The Ideal of a Set of Points}%
\begin{definition}
  If $X\subseteq \A^n\left(k\right)$, then the polynomials that vanish on $X$ form an ideal in $k\left[x_1,\dots,x_n\right]$, called the ideal of $X$, or $I(X)$.
  \begin{align*}
    I(X) &\coloneq \set{F\in k\left[x_1,\dots,x_n\right] | F\left(a_1,\dots,a_n\right) = 0\text{ for all }\left(a_1,\dots,a_n\right)\in X}.
  \end{align*}
\end{definition}
The following hold.
\begin{itemize}
  \item If $X\subseteq Y$, then $I(X)\supseteq I(Y)$.
  \item We have $I\left(\emptyset\right) = k\left[x_1,\dots,x_n\right]$, $I\left(\A^n\left(k\right)\right) = \left\langle 0 \right\rangle$ if $k$ is infinite, and $I\left(\set{\left(a_1,\dots,a_n\right)}\right) = \left\langle x_1-a_1,\dots,x_n-a_n \right\rangle$ for $a_1,\dots,a_n\in k$.
  \item We have $I\left(V(S)\right) \supseteq S$ for any set $S$ of polynomials, and $V\left(I(X)\right)\supseteq X$ for any set $X$ of points.
  \item We have $V(I(V(S))) = V(S)$ for any set of polynomials $S$, and $I(V(I(X))) = I(X)$ for any set $X$ of points. If $V$ is an algebraic set, $V = V(I(V))$ and if $I$ is the ideal of an algebraic set, then $I = I(V(I))$.
\end{itemize}
\begin{definition}
  If $I$ is any ideal in a ring $R$, we define the radical of $I$, written $\operatorname{rad}\left(I\right) = \set{a^n | a\in I\text{ for some }n > 0}$. We have that $\operatorname{rad}\left(I\right)$ is an ideal containing $I$. An ideal $I$ is called a radical ideal if $I = \operatorname{rad}\left(I\right)$.
\end{definition}
\begin{itemize}
  \item We have $I(X)$ is a radical ideal for any $X\subseteq \A^n\left(k\right)$.
\end{itemize}
\begin{exercise}[Exercise 1.16]
Let $V$ and $W$ be algebraic sets in $\A^n\left(k\right)$. Show that $V = W$ if and only if $I(V) = I(W)$.
\end{exercise}
\begin{solution}
  Let $V = W$. Then, if $F\in I(V)$, then $F = 0$ on $W$, so $F\in I(W)$, and vice versa.\newline

  Suppose $I(V) = I(W)$. We know that $V(I(V)) = V$ and $V(I(W)) = W$. Thus, if $\left(a_1,\dots,a_n\right)\in V$, we know that for all $F\in I(W)$, that $F\left(a_1,\dots,a_n\right) = 0$ as $F\in I(V)$, meaning $\left(a_1,\dots,a_n\right)\in V(I(W)) = W$. By symmetry, we have $V = W$.
\end{solution}

\begin{exercise}[Exercise 1.17]\hfill
\begin{enumerate}[(a)]
  \item Let $V$ be an algebraic set in $\A^n\left(k\right)$ and $P\in \A^n\left(k\right)$ not a point in $V$. Show that there is a polynomial $F\in k\left[x_1,\dots,x_n\right]$ such that $F(Q) = 0$ for all $Q\in V$ but $F(P) = 1$.
  \item Let $P_1,\dots,P_r$ e distinct points in $\A^n\left(k\right)$ not in an algebraic set $V$. Show that there are polynomials $F_1,\dots,F_r\in I(V)$ such that $F_i\left(P_j\right) = \delta_{ij}$.
  \item With $P_1,\dots,P_r$ and $V$ as in (b), and $a_{ij}\in k$ for $1\leq i,j\leq r$, show that there are $G_i\in I(V)$ such that $G_i\left(P_j\right) = a_{ij}$ for all $i$ and $j$.
\end{enumerate}
\end{exercise}
\begin{solution}\hfill
  \begin{enumerate}[(a)]
    \item We know that there is some $F\in I(V)$ such that $F(P)\neq 0$. Letting $a = F(P)$, we have that $\frac{1}{a}F(P) = 1$.
    \item We find $F_i\in I\left(V \cup \set{P_{-i}}\right)$, where $\set{P_{-i}} = \set{P_1,\dots,P_r}\setminus\set{P_i}$. Applying (a) to $F_i$, we get that $F_i\left(P_i\right) = 1$ and $F_i\left(P_{j}\right) = 0$ for $j\neq i$. By symmetry, this holds for $F_1,\dots,F_r$.
    \item With $P_1,\dots,P_r$ and $V$ as in (b), find $F_1,\dots,F_r$ as in (b). Then, $G_i = \sum_{j}a_{ij}F_j$ yields our desired outcome.
  \end{enumerate}
\end{solution}
\begin{exercise}[Exercise 1.18]
  Let $I$ be an ideal in a ring $R$. If $a^n\in I$ and $b^m\in I$, show that $\left(a+b\right)^{n+m}\in I$. Show that $\operatorname{rad}\left(I\right)$ is a (radical) ideal. Show that any prime ideal is radical.
\end{exercise}
\begin{solution}\hfill
  \begin{itemize}
    \item Applying binomial theorem, we have
      \begin{align*}
        \left(a+b\right)^{n+m} &= \sum_{k=0}^{n+m}{n+m\choose k}a^{n+m-k}b^{k}\\
                               &\in I,
      \end{align*}
      where $a^0 = b^{0}\coloneq 1$.
    \item We have $I\subseteq \operatorname{rad}\left(I\right)$, since we can take $n = 1$. If $a,b\in \operatorname{rad}\left(I\right)$, we know that there is some $n$ such that $a^n,b^m\in I$, so by the same logic as above, $\left(a-b\right)^{n+m}\in I$, meaning $a-b\in \operatorname{rad}\left(I\right)$. Now, if $a\in \operatorname{rad}\left(I\right)$ and $x\in R$, then we have that $a^n\in I$ for some $n$, meaning $x^na^n\in I$ as $I$ is an ideal, so $\left(xa\right)^n\in I$, so $xa\in \operatorname{rad}\left(I\right)$, so $\operatorname{rad}\left(I\right)$ is an ideal.
    \item Let $I$ be prime, and let $a\in \operatorname{rad}\left(I\right)$. Then, $a^{n}\in I$ for some $n > 0$, meaning $(a)\left(a^{n-1}\right)\in I$. Then, either $a\in I$, or $a^{n-1}\in I$, so by the implicit inductive hypothesis,  we have $a\in I$, so $\operatorname{rad}\left(I\right)\subseteq I$, so $\operatorname{rad}\left(I\right) = I$.
  \end{itemize}
\end{solution}
\begin{exercise}[Exercise 1.20]
  Show that for any ideal $I$ in $k\left[x_1,\dots,x_n\right]$, $V(I) = V\left(\operatorname{rad}\left(I\right)\right)$, and $\operatorname{rad}\left(I\right)\subseteq I\left(V\left(I\right)\right)$.
\end{exercise}
\begin{solution}\hfill
  \begin{itemize}
    \item Clearly, $V\left(\operatorname{rad}\left(I\right)\right)\subseteq V(I)$ because $I\subseteq \operatorname{rad}\left(I\right)$. We know that if $P\in V(I)$, then there is some polynomial $F\in I$ such that $F(P) = 0$.
  \end{itemize}
\end{solution}

\begin{exercise}[Exercise 1.21]
Show that any $I = \left\langle x_1 - a_1,\dots,x_n-a_n \right\rangle\subseteq k\left[x_1,\dots,x_n\right]$ is a maximal ideal, and that the natural homomorphism from $k$ to $k\left[x_1,\dots,x_n\right]/I$ is an isomorphism.
\end{exercise}
\begin{solution}
  Note that $\left\langle x_1-a_1,\dots,x_n-a_n \right\rangle\subseteq k\left[x_1,\dots,x_n\right]$ is isomorphic to $\left\langle x_1,\dots,x_n \right\rangle\subseteq k\left[x_1 + a_1,\dots,x_n + a_n\right]$, $k\left[x_1,\dots,x_n\right]/I \cong k$.
\end{solution}
\subsection{The Hilbert Basis Theorem}%
Earlier, we allowed any algebraic set $V(S)$ to be defined by an arbitrary set $\set{F_i}_{i\in I}\subseteq k\left[ x_1,\dots,x_n \right]$. However, the Hilbert Basis Theorem will show that a finite number will do.
\begin{theorem}
  Every algebraic set is the intersection of a finite number of hypersurfaces.
\end{theorem}
\begin{proof}
  We know that $V(I)$ is the algebraic set for some $I\subseteq k\left[ x_1,\dots,x_n \right]$. It is enough to show that $I$ is finitely generated, as if $I = \left\langle F_1,\dots,F_n \right\rangle$, then $V(I) = V\left( F_1 \right)\cap\cdots\cap V\left( F_n \right)$.
\end{proof}
Now, to prove this, we need to show that any arbitrary ideal $I\subseteq k\left[ x_1,\dots,x_n \right]$ is finitely generated. This is where the Hilbert Basis Theorem comes into play.
\begin{definition}
  If $R$ is a commutative ring, with identity, we say $R$ is Noetherian if every ideal of $R$ is finitely generated.
\end{definition}
Note that all PIDs are Noetherian.\newline

Now, we may state and prove the Hilbert Basis Theorem.
\begin{theorem}[Hilbert Basis Theorem]
  If $R$ is a Noetherian ring, then $R\left[ x_1,\dots,x_n \right]$ is a Noetherian ring.
\end{theorem}
\begin{proof}
  Since $R\left[ x_1,\dots,x_n \right]$ is canonically isomorphic to $R\left[ x_1,\dots,x_{n-1} \right]\left[ x_n \right]$. The theorem will follow by induction if we can prove that $R\left[ x \right]$ is Noetherian whenever $R$ is Noetherian.\newline

  Let $I\subseteq R\left[ x \right]$ be an ideal. We wish to find a finite set of generators for $I$.\newline

  Let $F = a_dx^d + \cdots a_1x + a_0\in R\left[ x \right]$ with $a_d\neq 0$. We call $a_d$ the leading coefficient of $F$. Let $J$ be the set of leading coefficients of polynomials in $I$. Then, $J\subseteq R$ is an ideal, so there are polynomials $F_1,\dots,F_r\in I$ whose leading coefficients generate $J$.\newline

  Select $N$ larger than the degree of each $F_i$. For each $m\leq N$, let $J_m$ be the ideal in $R$ consisting of all leading coefficients of polynomials $F\in I$ with $\deg\left( F \right) \leq m$. Let $\set{F_{m_j}}$ be the finite set of polynomials in $I$ with degree $\leq m$ such that their leading coefficients generate $J_m$. Let $I'$ be the ideal generated by $F_i$ and $F_{m_j}$ for each $i,m_j$. It is enough to show that $I = I'$.\newline

  Suppose $I'\subsetneq I$. Let $G$ be an element of $I$ of minimal degree such that $G\notin I'$. If $\deg(G) > N$, then we may find $Q_i$ such that $\sum Q_iF_i$ and $G$ have the same leading term. However, this means $\deg\left( G - \sum Q_iF_i \right) < \deg(G)$, so $G - \sum Q_iF_i\in I'$, meaning $G\in I'$. Similarly, if $\deg\left( G \right) = m \leq N$, then we may lower the degree by subtracting $\sum Q_jF_{m_j}$ for some $Q_j$.
\end{proof}
\begin{exercise}[Exercise 1.22]
Let $I$ be an ideal in a ring $R$, $\pi\colon R\rightarrow R/I$ the canonical projection.
\begin{enumerate}[(a)]
  \item Show that for every ideal $J'\subseteq R/I$, that $\pi^{-1}\left( J' \right) = J$ is an ideal of $R$ containing $I$. Furthermore, show that for every ideal $J\subseteq R$, that $\pi\left( J \right) = J'$ is an ideal of $R/I$. This establishes a natural correspondence between ideals of $R/I$ and ideals of $R$ that contain $I$.
  \item Show that $J'$ is a radical ideal if and only if $J$ is radical. Similarly, show this for $J$ prime and maximal.
  \item Show that $J'$ is finitely generated if $J$ is. Conclude that $R/I$ is Noetherian if $R$ is Noetherian. Thus, we get that $k\left[ x_1,\dots,x_n \right]/I$ is Noetherian for any ideal $I\subseteq k\left[ x_1,\dots,x_n \right]$ by the Hilbert Basis Theorem.
\end{enumerate}
\end{exercise}
\begin{solution}\hfill
  \begin{enumerate}[(a)]
    \item We know that $I\subseteq \pi^{-1}\left( J' \right)$, as $ I = \pi^{-1}\left(0 + I\right) \subseteq \pi^{-1}\left( J' \right)$. Notice that, if $a,b\in \pi^{-1}\left(J'\right)$ and $r\in R$, then $a + I,b + I\in J'$ and $r + I\in R/I$. Then, $a-b +I\in J'$, so $a-b\in \pi^{-1}\left( J' \right)$, and $ra + I \in J'$, so $ra\in \pi^{-1}\left( J' \right)$, so $\pi^{-1}\left( J' \right)$ is an ideal of $R$.\newline

      Now, let $a+I,b+I\in \pi(J)$. Then, we know that there exist $c_1,c_2\in J$ such that $a-c_1,b-c_2\in I$. Thus, $\left( a-b \right) + \left( c_2 - c_1 \right)\in I$. Since we have $c_2 - c_1\in J$ as $J$ is an ideal, so $\pi\left( a-b \right) = \pi\left( c_2 - c_1 \right)$, and $\left( a-b \right) + I\in \pi(J)$. Now, let $a + I\in \pi(J)$, and let $r + I\in R/I$. Then, there exist $c_1\in R$, $c_2\in J$ such that $r-c_1\in I$ and $a - c_2\in I$, meaning that $\pi\left( c_1c_2 \right) = \pi\left( ra \right) = ra + I \in \pi(J)$.
    \item Let $J$ be maximal. Then, $R/J \cong \left( R/I \right)/\left( \pi(J) \right)$, is a field, meaning $\pi(J)\subseteq R/I$ is also maximal. This gives both directions.\newline

    Similarly, if $J$ is prime, then $R/J \cong \left( R/I \right)/\left( \pi(J) \right)$ is an integral domain, so $\pi(J)\subseteq R/I$ is also an integral domain. This gives both directions.\newline

    Let $J$ be a radical ideal. Then, $J = \bigcap\set{\mathfrak{p} | J\subseteq \mathfrak{p},\mathfrak{p}\text{ is prime}}$. We know that for all $\mathfrak{p}$, $\pi\left( \mathfrak{p} \right)\subseteq R/I$ is prime. We know that $\pi(J)\subseteq \pi\left( \mathfrak{p} \right)$ if and only if $J\subseteq \mathfrak{p}$, so $\pi(J) = \bigcap\set{\pi(\mathfrak{p}) | J\subseteq \mathfrak{p},\mathfrak{p}\text{ is prime}}$. In the reverse direction, we se that if $a\in \pi^{-1}(J)$, then $a + I\in J$, so $a^n + I\in J$ for some $n\in \N$, so $a^n\in \pi^{-1}(J)$, so $\pi^{-1}(J)$ is a radical ideal.
  \item Letting $\left\langle a_1,\dots,a_n \right\rangle = J$, then we know that $ \left\langle \pi\left(a_1\right),\dots,\pi\left(a_n\right) \right\rangle= \pi(J)$. Thus, $\pi(J)$ is finitely generated.\newline

    Since $R$ is an ideal, if $R$ is Noetherian, then $R/I$ is Noetherian, so by the Hilbert Basis Theorem, any ring of the form $k\left[ x_1,\dots,x_n \right]/I$ is Noetherian.
  \end{enumerate}
\end{solution}
\subsection{Irreducible Components of an Algebraic Set}%
An algebraic set can be the union of several smaller algebraic sets. If $V\subseteq \A^n$ is such that $V = V_1 \cup V_2$, where $V_1,V_2$ are algebraic sets and $V_i\neq V$ for each $i$, then we say $V$ is reducible. Else, we say $V$ is irreducible.
\begin{proposition}
  An algebraic set $V$ is irreducible if and only if $I(V)$ is prime.
\end{proposition}
\begin{proof}
  If $I(V)$ is not prime, then we have $F_1F_2\in I(V)$ with $F_i\notin I(V)$. Then, $V = \left( V\cap V\left(F_1\right) \right)\cup\left( V\cap V\left(F_2\right) \right)$, with $V\cap V\left( F_i \right)\subsetneq V$, meaning $V$ is irreducible.\newline

  If $V = V_1\cup V_2$ with $V_i\subsetneq V$, then $I\left( V_i \right)\supseteq I(V)$. Let $F_i\in I\left( V_i \right)$ with $F_i\notin I(V)$. Then, $F_1F_2\in I(V)$, so $I(V)$ is not prime.
\end{proof}
Now, we want to show that an algebraic set is a finite union of irreducible algebraic sets. To see this, we need to show an equivalent definition of a Noetherian ring.
\begin{lemma}
  Let $\mathcal{I}$ be a nonempty collection of ideals in a Noetherian ring $R$. Then, $\mathcal{I}$ has a maximal member.
\end{lemma}
\begin{proof}
  We will choose an ideal from each subset of $\mathcal{I}$. Letting $I_0$ be the chosen ideal for $\mathcal{I}$ itself, we let $\mathcal{I}_1 = \set{I\in \mathcal{I} | I\supsetneq I_0}$, with $I_1$ as the chosen ideal of $\mathcal{I}_1$. Continuing, we define
  \begin{align*}
    \mathcal{I}_j &= \set{I\in \mathcal{I} | I\supsetneq I_{j-1}},
  \end{align*}
  and select $I_j\in \mathcal{I}_j$. It suffices to show that some $\mathcal{I}_n$ is empty.\newline

  Define $I = \bigcup_{n=0}^{\infty}I_n$ to be an ideal of $R$, and let $F_1,\dots,F_r$ be generators of $I$. We must have $F_i\in I_n$ for all $i$ if $n$ is sufficient large. Then, $I_n = I$, meaning $I_{n+1} = I_{n}$, which is a contradiction.
\end{proof}
Effectively, we have shown that every Noetherian ring satisfies the ascending chain condition on its ideals.\newline

It follows that any collection of algebraic sets $\set{V_{\alpha}}$ in $\A^n\left( k \right)$ has a minimal element, by selecting the maximal member of $\set{I\left( V_{\alpha} \right)}$.
\begin{theorem}
  Let $V$ be an algebraic set in $\A^n\left( k \right)$. Then, there rae unique irreducible algebraic sets $V_1,\dots,V_m$ such that $V=  V_1\cup \cdots \cup V_m$, and $V_i\nsubseteq V_j$ for all $i\neq j$.
\end{theorem}
\begin{proof}
  Let $\mathcal{I}$ be the set of algebraic sets in $\A^n\left( k \right)$ such that $V$ is not the union of a finite number of irreducible algebraic sets. We wish to show that $\mathcal{I}$ is empty.\newline

  If not, let $V$ be a minimal member of $\mathcal{I}$. Since $V\in \mathcal{I}$, $V$ is not irreducible, so $V = V_1\cup V_2$ with $V_i\subsetneq V$, meaning $V_i\notin \mathcal{I}$, so $V_i = V_{i,1}\cup\cdots V_{i,m_i}$, with $V_{i,j}$ irreducible. However, $V = \bigcup_{i,j} V_{i,j}$, which is a finite union.\newline

  Thus, any algebraic set $V$ may be written as $V = V_1\cup\cdots\cup V_m$ with $V_i$ irreducible. To obtain the second condition, we may discard any $V_i$ with $V_i\subseteq V_j$ with $i\neq j$.\newline

  To show uniqueness, let $V = W_1\cup\cdots\cup W_m$ be another decomposition. Then, $V_i= \bigcup_{j}\left( W_j\cap V_i \right)$, so $V_i\subseteq W_{j(i)}$ for some $j(i)$. Similarly, $W_{j(i)}\subseteq V_k$ for some $k$. However, this means $V_i\subseteq V_k$, so $i = k$, so $V_i = W_{j(i)}$. Likewise, $W_j = V_{i(j)}$ for some $i(j)$.
\end{proof}
We call $V_i$ the irreducible components of $V$, and $V = V_1\cup\cdots\cup V_m$ is the decomposition of $V$ into irreducible components.
\begin{exercise}[Exercise 1.25]\hfill
  \begin{enumerate}[(a)]
    \item Show that $V\left(y-x^2\right)\subseteq \A^2\left(\C\right)$ is irreducible; in fact, $I\left(V\left(y-x^2\right)\right) = \left\langle y-x^2 \right\rangle$.
    \item Decompose $V\left( y^4-x^2,y^4-x^2y^2+xy^2-x^3 \right)\subseteq \A^2\left( \C \right)$ into irreducible components.
  \end{enumerate}
\end{exercise}
\begin{solution}\hfill
  \begin{enumerate}[(a)]
    \item Suppose there exists $g\in \C\left[ x,y \right]$ such that $g | y-x^2$, meaning there exists $f\in \C\left[ x,y \right]$ such that $fg = y-x^2$. Since $y-x^2$ has degree in $y$ equal to $1$, one of either $f$ or $g$ has degree in $y$ equal to zero.\newline

      Therefore, without loss of generality, $f\in \C\left[ x \right]$. Then, $g = yh_1 + h_2$, where $h_1,h_2\in \C\left[ x \right]$. Note that $h_1\neq 0$, then $fg = fyh_1 + fh_2 = yfh_1 + fh_2$; since $fh_1\neq 0$, we must have $fh_1 = 1$, so $f$ is constant, so $g$ is some constant multiple of $y-x^2$, so $y-x^2$ is irreducible. Thus, $\left\langle y-x^2 \right\rangle$ is maximal, hence prime, so $I\left( V\left( y-x^2 \right) \right) = \left\langle y-x^2 \right\rangle$.
    \item Factoring, we see that both polynomials vanish whenever $y^2 + x = 0$. Finding all pairs, we get
      \begin{align*}
        V &= V\left( y^2-x,y^2 + x \right) \cup V\left( y^2- x,y-x \right) \cup \cdots\\
          &= V\left( y^2 + x \right) \cup V\left( x-1,y-1 \right) \cup V\left( x-1,y+1 \right).
      \end{align*}
  \end{enumerate}
\end{solution}
\begin{solution}\hfill
  \color{red}
  \begin{enumerate}[(a)]
    \item Let $g\in I(V)$. Then,
      \begin{align*}
        g(x,y) &= f_0\left( x \right) + \left( y-x^2 \right)f_1\left( x,y \right),
      \end{align*}
      wherein we order $y > x$ and do polynomial long division over $y$. This yields $f_0(x) = 0$ for all $x$, so that $I\left( V \right)$ is prime.
  \end{enumerate}
\end{solution}

\begin{exercise}[Exercise 1.29]
  Show that $\A^n\left( k \right)$ is irreducible if $k$ is infinite.
\end{exercise}
\begin{solution}
  We know that any polynomial that vanishes on $\A^n\left( k \right)$ is the zero polynomial, and $k\left[ x_1,\dots,x_n \right]$ is an integral domain, so $\left\langle 0 \right\rangle\subseteq k\left[ x_1,\dots,x_n \right]$ is a prime ideal.
\end{solution}

\subsection{Algebraic Subsets of the Plane}%
We focus on the affine plane, $\A^2\left( k \right)$, and find its algebraic subsets.\newline

It is enough to look at the irreducible algebraic subsets.
\begin{exercise}[Exercise 1.30]
Let $k = \R$. 
\begin{enumerate}[(a)]
  \item Show that $I\left(V\left(x^2 + y^2 + 1\right)\right) = \left\langle 1 \right\rangle$.
  \item Show that every algebraic subset of $\A^2\left(\R\right)$ is equal to $V(F)$ for some $F\in \R\left[x,y\right]$.
\end{enumerate}
\end{exercise}
\begin{solution}\hfill
  \begin{enumerate}[(a)]
    \item Since $x^2 + y^2 + 1 = 0$ if and only if $x^2 + y^2 = -1$, which means $V\left(x^2 + y^2 + 1\right) = \emptyset$. Thus, $I\left( V\left( x^2 + y^2 + 1 \right) \right) = \R\left[ x,y \right] = \left\langle 1 \right\rangle$.
    \item 
  \end{enumerate}
\end{solution}

\begin{exercise}[Exercise 1.31]\hfill
  \begin{enumerate}[(a)]
    \item Find the irreducible components of $V\left( y^2 - xy - x^2y + x^3 \right)$ in $\A^2\left( \R \right)$, and in $\A^2\left( \C \right)$.
    \item Do the same for $V\left( y^2 - x\left( x^2 - 1 \right) \right)$, and for $V\left( x^3 + x - x^2 y - y \right)$.
  \end{enumerate}
\end{exercise}
\subsection{Hilbert's Nullstellensatz}%
Given an algebraic set $V$, we have a criterion for determining whether or not $V$ is irreducible. However, we do not have a way to describe $V$ in terms of the set that defines $V$. This is what the Nullstellensatz, or zero locus theorem, will tell us.\newline

We assume throughout this section that $k$ is algebraically closed.
\begin{theorem}[Weak Nullstellensatz]
  If $I$ is a proper ideal in $k\left[ x_1,\dots,x_n \right]$, then $V(I)\neq \emptyset$.
\end{theorem}
\begin{proof}
  We may assume that $I$ is a maximal ideal, as $J\supseteq I$ is maximal and $V(J)\subseteq V(I)$.\newline

  Thus, $L = k\left[ x_1,\dots,x_n \right]/I$ is a field, and $k$ is a subfield of $L$.\newline

  Suppose we knew that $k = L$. For each $i$, there is $a_i\in k$ such that $x_i - a_i\in I$. However, $\left\langle x_1-a_1,\dots,x_n-a_n \right\rangle$ is a maximal ideal. Thus, $I = \left\langle x_1-a_1,\dots,x_n-a_n \right\rangle$, and $V(I) = \set{\left( a_1,\dots,a_n \right)}\neq \emptyset$.
\end{proof}
Now, we have reduced the problem to showing that if an algebraically closed field $k$ is a subfield of a field $L$, and there is a ring homomorphism of $k\left[ x_1,\dots,x_n \right]$ onto $L$ that is the identity on $k$, then $k = L$.
\begin{theorem}[Hilbert's Nullstellensatz]
  Let $I$ be an ideal in $k\left[ x_1,\dots,x_n \right]$ with $k$ algebraically closed. Then, $I(V(I)) = \operatorname{rad}\left( I \right)$.
\end{theorem}

\begin{remark}
  In concrete terms, if $F_1,\dots,F_r,G$ are in $k\left[ x_1,\dots,x_n \right]$, and $G$ vanishes wherever $F_1,\dots,F_r$ vanish, then there is some equation $G^N = A_1F_1 + \cdots A_rF_r$ for some $N > 0$ and $A_i\in k\left[ x_1,\dots,x_n \right]$.
\end{remark}

\begin{proof}
  We can see that $\operatorname{rad}\left( I \right)\subseteq I\left(V(I)\right)$. Now, let $G$ be in the ideal $I\left( V\left( F_1,\dots,F_r \right) \right)$, where $F_i\in k\left[ x_1,\dots,x_n \right]$. Let $J = \left\langle F_1,\dots,F_r,x_{n+1}G - 1 \right\rangle\subseteq k\left[ x_1,\dots,x_n,x_{n+1} \right]$.\newline

  Then, $V(J)\subseteq \A^{n+1}\left( k \right)$ is empty, since $G$ vanishes wherever all the $G_i$ are zero. Applying the weak Nullstellensatz to $J$, we have $1\in J$, so there is an equation $1= \sum A_i\left( x_1,\dots,x_{n+1} \right)F_i + B\left( x_1,\dots,x_{n+1} \right)\left( x_{n+1}G - 1 \right)$. Now, let $y = 1/x_{n+1}$, and multiply the equation by a high power of $y$ such that $y^N = \sum C_i\left( x_1,\dots,x_n,y \right)F_i + D\left( x_1,\dots,x_n,y \right)\left( g-y \right)$ in $k\left[ x_1,\dots,x_n,y \right]$. Now, substituting $G$ for $y$, we obtain our desired result.
\end{proof}
\begin{corollary}
  If $I$ i a radical ideal in $k\left[ x_1,\dots,x_n \right]$, then $I(V(I)) = I$. Thus, there is a one-to-one correspondence between radical ideals and algebraic sets.
\end{corollary}
\begin{corollary}
  If $I$ is a prime ideal, then $V(I)$ is irreducible. Thus, there is a one-to-one correspondence between prime ideals and irreducible algebraic sets. The maximal ideals correspond to points.
\end{corollary}
\begin{corollary}
  Let $F$ be a nonconstant polynomial in $k\left[ x_1,\dots,x_n \right]$, and $F = F_1^{n_1}\cdots F_r^{n_r}$ is a decomposition into irreducible factors. Then, $V(F) = V\left(F_1\right)\cup\cdots\cup V\left( F_r \right)$ is the decomposition of $V(F)$ into irreducible components, and $I(V(F)) = \left\langle F_1,\dots,F_r \right\rangle$. There is a one-to-one correspondence between irreducible polynomials $F\in k\left[ x_1,\dots,x_n \right]$ and irreducible hypersurfaces in $\A^n\left( k \right)$.
\end{corollary}
\begin{corollary}
  Let $I$ be an ideal in $k\left[ x_1,\dots,x_n \right]$. Then, $V(I)$ is a finite set if and only if $k\left[ x_1,\dots,x_n \right]/I$ is a finite-dimensional vector space over $k$. If so, the number of points in $V(I)$ is at most $\Dim_{k}\left( k\left[ x_1,\dots,x_n \right]/I \right)$.
\end{corollary}
\begin{proof}
  Let $P_1,\dots,P_r\in V(I)$. Let $F_1,\dots,F_r\in k\left[ x_1,\dots,x_n \right]$ such that $F_i(P_j) = \delta_{ij}$. Let $\overline{F_i}$ be the residue of $F_i$ in $k\left[ x_1,\dots,x_n \right]/I$.\newline

  If $\sum \lambda_i \overline{F_i} = 0$, where $\lambda_i\in k$, then $\sum \lambda_iF_i\in I$, so that $\lambda_j = \left( \sum \lambda_iF_i \right)\left( P_j \right) = 0$, meaning the $\overline{F_i}$ are linearly independent over $k$, and $\Dim_k\left( k\left[ x_1,\dots,x_n \right]/I \right)$.\newline

  Now, conversely, if $V(I) = \set{P_1,\dots,P_r}$ is finite, let $P_i = \left( a_{i1},\dots,a_{in} \right)$, and define $F_j$ by $F_j = \prod_{i=1}^{r}\left( x_i-a_{ij} \right)$ for $j=1,\dots,n$.\newline

  Then, $F_j\in I(V(I))$, so $F_j^{N}\in I$ for some $N > 0$, and we may take $N$ large enough such that $N$ works for all $F_j$. Taking residues in $I$, we have $\overline{F_j}^N = 0$, so that $\overline{x_j}^{rN}$ is a $k$-linear combination of $\overline{1},\overline{x_j},\dots,\overline{x_j}^{rN-1}$. Thus, by induction, $\overline{x_j}^{s}$ is a $k$-linear combination of $1,\overline{x_j},\dots,\overline{x_j}^{rN - 1}$ for all $s$, so the set $\set{\overline{x_1}^{m_1}\dots\overline{x_n}^{m_n} | m_i < rN}$ generates $k\left[ x_1,\dots,x_n \right]/I$ as a $k$-vector space.
\end{proof}

\begin{exercise}[Exercise 1.33]\hfill
  \begin{enumerate}[(a)]
    \item Decompose $V\left( x^2 + y^2 - 1,x^2 - z^2 - 1 \right)\subseteq \A^3\left( \C \right)$ into irreducible components.
    \item Let $V = \set{\left( t,t^2,t^3 \right)\in \A^3\left( \C \right) | t\in \C}$. Find $I(V)$ and show that $V$ is irreducible.
  \end{enumerate}
\end{exercise}
\begin{solution}\hfill
  \begin{enumerate}[(a)]
    \item We have that $x^2 = 1-y^2$, so that $1-y^2 - z^2 - 1 = 0$, and $y = \pm iz$. Thus, $V\left( x^2 + y^2 - 1,x^2 - z^2 - 1 \right) = V\left( x^2 + y^2 - 1,y+iz \right)\cup V\left( x^2 + y^2 - 1,y-iz \right)$. We want to show that these are irreducible sets. Let $I_2 = \left\langle x^2 + y^2 - 1,y+iz \right\rangle$, $I_3 = \left\langle x^2+y^2-1,y-iz \right\rangle$, and $I_1 = \left\langle x^2 + y^2 - 1,x^2 - z^2 - 1 \right\rangle$.\newline

      By the Third Isomorphism Theorem,
      \begin{align*}
        \C\left[ x,y,z \right]/I_{2,3} &\cong \left( \C\left[ x,y,z \right]/\left\langle y\pm iz \right\rangle \right)/\left( \left\langle x^2 + y^2 - 1,y\pm iz \right\rangle/\left\langle y \pm iz \right\rangle \right)\\
                                       &\cong \C[x,y]/\left\langle x^2 + y^2 - 1 \right\rangle.
      \end{align*}
      To show that $I_2$ is prime, we show that $\C\left[ x,y \right]/\left\langle x^2 + y^2 - 1 \right\rangle$ is an integral domain.\newline

      Note that $\C\left[ x,y \right] = \C\left[ x + iy,x-iy \right]\coloneq \C\left[ a,b \right]$. Then,
      \begin{align*}
        \C\left[ x,y \right]/\left\langle x^2 + y^2 - 1 \right\rangle &\cong \C\left[ a,b \right]/\left\langle ab-1 \right\rangle\\
                                                                      &\cong \left( \C\left[ a \right] \right)\left[ b \right]/\left\langle ab-1 \right\rangle.
      \end{align*}
      Since $ab-1$ is a degree $1$ polynomial in $\left( \C\left[ a \right] \right)\left[ b \right]$, we have $ab - 1$ is irreducible, so that $\left\langle ab-1 \right\rangle$ is prime, as $\left( \C\left[ a \right] \right)\left[ b \right]$ is a unique factorization domain.
    \item We have $I(V) = \left\langle x^2 - y,x^3 - z \right\rangle$. To show that this is irreducible, consider the surjective homomorphism $\varphi\colon \C\left[ x,y,z \right]\rightarrow \C\left[ t \right]$, given by $f\left( x,y,z \right)\mapsto f\left( t,t^2,t^3 \right)$. This has kernel $I(V)$, so that $\C\left[ x,y,z \right]/I(V)\cong \C\left[ t \right]$, and $I(V)$ is prime, so $V$ is irreducible.
  \end{enumerate}
\end{solution}
\begin{exercise}[Exercise 1.36]
  Let $I = \left\langle y^2-x^2,y^2 + x^2 \right\rangle\subseteq \C\left[ x,y \right]$. Find $V(I)$ and $\Dim_{\C}\left( \C\left[ x,y \right]/I \right)$.
\end{exercise}
\begin{solution}
  We see that $I$ is generated by $\left\langle \left( y-x \right)\left( y+x \right),\left( y-ix \right)\left( y+ix \right) \right\rangle$. This gives $\set{(0,0)}$ as $V(I)$.\newline

  Note that we have $y^2 + x^2 + I \cong 0$ and $y^2 - x^2 + I \cong 0$, so $x^2\cong 0$ and $y^2\cong 0$, meaning the basis for $\Dim_{\C}\left( \C\left[ x,y \right]/I \right) $ is $ \set{1,x,y,xy}$.
\end{solution}
\begin{exercise}[Exercise 1.37]
  Let $K$ be any field, $F\in K\left[ x \right]$ a polynomial of degree $n > 0$.\newline

  Show that the residues $\overline{1},\overline{x},\dots,\overline{x}^{n-1}$ form a basis for $K\left[ x \right]/\left\langle F \right\rangle$ over $K$.
\end{exercise}
\begin{solution}
  Without loss of generality, we may assume $F$ is monic, meaning that $x^{n} = -\left( a_{n-1}x^{n-1} + \cdots + a_1x + a_0 \right)$, meaning that $\overline{x}^n \in \Span\set{\overline{1},\overline{x},\dots,\overline{x}^{n-1}}$. Thus, we know that the set $\set{\overline{1},\overline{x},\dots,\overline{x}^{n-1}}$ is spanning for $K\left[ x \right]/\left\langle F \right\rangle$.\newline

  To show that this set is linearly independent in $K\left[ x \right]/\left\langle F \right\rangle$, we suppose $gF = s_0\overline{1} + s_1\overline{x} + \cdots + s_{n-1}\overline{x}^{n-1}$. Then $g = 0$ by polynomial long division.
\end{solution}
\begin{exercise}[Exercise 1.38]
  Let $R = k\left[ x_1,\dots,x_n \right]$ with $k$ algebraically closed. Let $V = V(I)$. Show that there is a natural one-to-one correspondence between algebraic subsets of $V$ and radical ideals in $k\left[ x_1,\dots,x_n \right]/I$, and that irreducible algebraic sets (points) correspond to prime ideals (maximal ideals).
\end{exercise}
\begin{solution}
  This follows from the correspondence in Exercise 1.22.
\end{solution}

\subsection{Modules and Finiteness}%
\begin{definition}
  Let $R$ be a ring. An $R$-module is a commutative group $M$ with a scalar multiplication $R\times M \rightarrow M$ satisfying
  \begin{enumerate}[(i)]
    \item $\left( a+b \right)m = am + bm$ for $a,b\in R$, $m\in M$;
    \item $a\left( m + n \right)= am + an$ for $a\in R$, $m,n\in M$;
    \item $\left( ab \right)m = a\left( bm \right)$ for $a,b\in R$, $m\in M$;
    \item $1_R m = m$ for $m\in M$, where $1_R$ is the multiplicative unit for $R$.
  \end{enumerate}
\end{definition}
\begin{example}\hfill
  \begin{enumerate}[(1)]
    \item A $\Z$-module is an abelian group.
    \item If $R$ is a field, an $R$-module is an $R$-vector space.
    \item The multiplication in $R$ makes any ideal of $R$ into an $R$-module.
    \item If $\varphi\colon R\rightarrow S$ is a ring homomorphism, we define $r\cdot s$ by the equation $r\cdot s \coloneq \varphi(r)s$, which makes $S$ into an $R$-module. If $R$ is a subring of $S$, then $S$ is an $R$-module.
  \end{enumerate}
\end{example}
\begin{definition}
  A subgroup $N$ of an $R$-module $M$ is called a submodule if $am\in N$ for all $a\in R$ and $m\in N$.\newline

  If $S$ is a set of elements of an $R$-module $M$, the submodule generated by $S$ is defined to be
  \begin{align*}
    \set{\sum r_is_i | r_i\in R,s_i\in S};
  \end{align*}
  it is the smallest submodule of $M$ that contains $S$. If $S = \set{s_1,\dots,s_n}$ is finite, the submodule generated by $S$ is denoted $\sum R s_i$.\newline

  The module $M$ is said to be finitely generated if $M = \sum Rs_i$ for some $s_1,\dots,s_n\in M$.
\end{definition}
\begin{definition}
  Let $R$ be a subring of $S$.
  \begin{enumerate}[(a)]
    \item We say $S$ is module-finite over $R$ if $S$ is finitely generated as an $R$-module. If $S$ and $R$ are  fields, then we denote the dimension of $S$ over $R$ by $\left[ R:S \right]$.
    \item Let $v_1,\dots,v_n\in S$, and $\varphi\colon R\left[ x_1,\dots,x_n \right]\rightarrow S$ be the ring homomorphism taking $x_i$ to $v_i$. The image of $\varphi$ is written $R\left[ v_1,\dots,v_n \right]$, which is a subring of $S$ containing $R$ and $v_1,\dots,v_n$.\newline

      Explicitly, we write
      \begin{align*}
        R\left[ v_1,\dots,v_n \right] &= \set{\sum a_{(i)}v_1^{i_1}\cdots v_n^{i_n} | a_{(i)} \in R}.
      \end{align*}
      The ring $S$ is ring-finite over $R$ if $S = R\left[ v_1,\dots,v_n \right]$ for some $v_1,\dots,v_n\in S$.
    \item Suppose $R = K$ and $S = L$ are fields. If $v_1,\dots,v_n\in L$ and $K\left( v_1,\dots,v_n \right)$ is the quotient field of $K\left[ v_1,\dots,v_n \right]$. Consider $K\left( v_1,\dots,v_n \right)\subseteq L$ as a subfield, which is the smallest subfield of $L$ containing $K$ and $v_1,\dots,v_n$.\newline

      We say $L$ is a finitely generated extension of $K$ if $L = K\left( v_1,\dots,v_n \right)$ for some $v_1,\dots,v_n\in L$.
  \end{enumerate}
\end{definition}
\begin{exercise}[Exercise 1.41]
  If $S$ is module-finite over $R$, then $S$ is ring-finite over $R$.
\end{exercise}
\begin{solution}
  Let $S$ be module-finite. Then, $v\in S$ can be expressed as $v = r_1s_1 + \cdots + r_ns_n$, so that $v\in R\left[ s_1,\dots,s_n \right]$. Thus, $S\subseteq R\left[ s_1,\dots,s_n \right]$. Since $r\in R$ and $s_1,\dots,s_n\in S$, we have that $R\left[ s_1,\dots,s_n \right]\subseteq S$, and $S$ is ring-finite over $R$.
\end{solution}
\begin{exercise}[Exercise 1.43]
  If $L$ is ring-finite over $K$, where $L$ and $K$ are fields, then $L$ is a finitely generated field extension of $K$.
\end{exercise}
\begin{solution}
  Let $L$ be ring-finite over $K$, where $L$ and $K$ are fields. Then, $L = K\left[ v_1,\dots,v_n \right]$. For each $v_i\in K\left[ v_1,\dots,v_n \right]$, we have that $v_i^{-1}\in K\left[ v_1,\dots,v_n \right]$, so $L = K\left( v_1,\dots,v_n \right)$.
\end{solution}
\begin{exercise}[Exercise 1.44]
  Show that $L = K(x)$ is a finitely generated field extension of $K$, but $L$ is not ring-finite over $K$.
\end{exercise}
\begin{solution}
  Suppose toward contradiction that $K\left( x \right) = L = K\left[ \frac{f_1}{g_1},\dots,\frac{f_n}{g_n} \right]$.\newline

  Then, for all $h\in L$, we have that 
  \begin{align*}
    \frac{1}{h} &= \sum_{i}b_{(i)}\frac{f_1^{j_1}\cdots f_n^{j_n}}{g_1^{i_1}\cdots g_n^{i_n}},
  \end{align*}
  meaning that
  \begin{align*}
    \frac{g_1^{i_1}\cdots g_n^{i_n}}{h}\in L\left[ x \right].
  \end{align*}
  However, since there are infinitely many irreducible monic polynomials in $L\left[ x \right]$, choose $h$ to not be equal to any of these.
\end{solution}

\begin{exercise}[Exercise 1.45]
  Let $R$ be a subring of $S$, $S$ a subring of $T$.
  \begin{enumerate}[(a)]
    \item If $S = \sum Rv_i$ and $T = \sum Sw_j$, then $T = \sum Rv_iw_j$.
    \item If $S = R\left[ v_1,\dots,v_n \right]$ and $T = S\left[ w_1,\dots,w_m \right]$, show that $T = R\left[ v_1,\dots,v_n,w_1,\dots,w_m \right]$.
    \item If $R,S,T$ are fields, and $S = R\left( v_1,\dots,v_n \right)$, $T = S\left( w_1,\dots,w_m \right)$, show that $T = R\left( v_1,\dots,v_n,w_1,\dots,w_m \right)$.
  \end{enumerate}
  Thus, each of the three finiteness conditions is a transitive relation.
\end{exercise}
\subsection{Integral Elements}%
\begin{definition}
  Let $R$ be a subring of a ring $S$. An element $v\in S$ is said to be integral over $R$ if there is a monic polynomial $f = x^n + a_{n-1}x^{n-1} + \cdots + a_{1}x + a_0\in R\left[ x \right]$ such that $f(v) = 0$.\newline

  If $R$ and $S$ are fields, then we say $v$ is algebraic over $R$ if $v$ is integral over $R$.
\end{definition}
\begin{proposition}
  Let $R$ be a subring of an integral domain $S$, with $v\in S$. The following are equivalent:
  \begin{enumerate}[(i)]
    \item $v$ is integral over $R$;
    \item $R\left[ v \right]$ is module-finite over $R$;
    \item there is a subring $R'$ of $S$ containing $R\left[ v \right]$ that is module-finite over $R$.
  \end{enumerate}
\end{proposition}
\begin{proof}
  If $0 = v^n + a_{n-1}v^{n-1} + \cdots + a_{1}v + a_0 = 0$, then $v^n\in \sum_{i=0}^{n-1}Rv^i$, so $v^{m}\in \sum_{i=0}^{n-1}Rv^i$ for all $m$, so $R\left[ v \right] = \sum_{i=0}^{n-1}Rv^i$.\newline

  Now, to show (ii) implies (iii), all we need to is take $R' = R\left[ v \right]$.\newline

  To show (iii) implies (i), we let $R' = \sum_{i=1}^{n}Rw_i$, so that $vw_i = \sum_{j=1}^{n}a_{ij}w_j$ for some $a_{ij}\in R$. Then,
  \begin{align*}
    \sum_{j=1}^{n} \left( \delta_{ij}v - a_{ij} \right)w_j = 0
  \end{align*}
  for all $i$, where $\delta_{ij}$ is the Kronecker delta function.\newline

  If we consider these equations in the quotient field of $S$, then $\left( w_1,\dots,w_n \right)$ is a nontrivial solution, so
  \begin{align*}
    \det\left( \delta_{ij}v - a_{ij} \right) = 0.
  \end{align*}
   Since $v$ only appears on the diagonal of this matrix, we have the form $0 = v^n + a_{n-1}v^{n-1} + \cdots + a_{1}v + a_0$, where $a_i\in R$. Thus, $v$ is integral over $R$.
\end{proof}
\begin{corollary}
  The set of elements of $S$ that are integral over $R$ is a subring of $S$ containing $R$.
\end{corollary}
\begin{proof}
  If $a,b$ are integral over $R$, then $b$ is integral over $R\left[ a \right]\supseteq R$, so $R\left[ a,b \right]$ is module-finite over $R$, and $a\pm b,ab\in R\left[ a,b \right]$, so they are integral over $R$.
\end{proof}
\begin{exercise}[Exercise 1.46]
  Let $R$ be a subring of $S$, $S$ a subring of an integral domain $T$. If $S$ is integral over $R$, and $T$ is integral over $S$, show that $T$ is integral over $R$.
\end{exercise}
\begin{solution}
  Let $z\in T$. Then, $z^n + a_{n-1}z^{n-1} + \cdots + a_1 z + a_0 = 0$, where each $a_i\in S$. Note that we have $\set{1,z,\dots,z^{n-1}}$ as a basis for $R\left[ a_0,\dots,a_{n-1} \right]\left[ z \right]$, so that $R\left[ a_0,\dots,a_{n-1} \right]\left[ z \right]\subseteq T$ is module-finite over $R$. This ring contains the subring $R\left[ z \right]$, so $T$ is integral over $R$ by part (3) of the proposition.
\end{solution}
\begin{exercise}[Exercise 1.47]
  Suppose $S$ is an integral domain that is ring-finite over $R$. Show that $S$ is module-finite over $R$ if and only if $S$ is integral over $R$.
\end{exercise}
\begin{solution}
  Let $S$ be ring-finite over $R$, so $S = R\left[ a_1,\dots,a_n \right]$.\newline

  If $S$ is integral over $R$, then for any $z\in S$, there is some polynomial $z^n + r_{n-1}z^{n-1} + \cdots + r_1 z + r_0 = 0$. Therefore, $\set{1,z,\dots,z^{n-1}}$ serves as a basis for $R\left[ z \right]\subseteq S$ for any $z\in S$. However, this applies for each $a_1,\dots,a_n$, so $S$ is finitely generated as a module over $R$.\newline

  If $S$ is module-finite over $R$, then for any $v\in S$, $R\left[ v \right]\subseteq R\left[ a_1,\dots,a_n \right]\left[ v \right] = R\left[ a_1,\dots,a_n,v \right] = S$, so $R\left[ v \right]$ is module-finite over $S$, so $S$ is integral over $R$.
\end{solution}
\begin{exercise}[Exercise 1.48]
  Let $L$ be a field, $k$ an algebraically closed subfield of $L$.
  \begin{enumerate}[(a)]
    \item Show that any element of $L$ that is algebraic over $k$ is in $k$.
    \item An algebraically closed field has no module-finite field extensions except itself.
  \end{enumerate}
\end{exercise}
\begin{solution}\hfill
  \begin{enumerate}[(a)]
    \item If $z\in L$ is algebraic over $k$, then $z^{n} + a_{n-1}z^{n-1} + \cdots + a_1 z + a_0 = 0$, where $a_{n-1},\dots,a_0\in k$. However, since $k$ is algebraically closed, this means $z\in k$, as $z$ is a root of the polynomial $x^{n} + a_{n-1}x^{n-1} + \cdots + a_1 x + a_0$.
    \item We know that $z$ is integral over $k$ if and only if $k\left[ z \right]$ is module-finite over $k$. However, since every integral/algebraic element over an algebraically closed field is in the field, there cannot be any module-finite extensions over $k$.
  \end{enumerate}
\end{solution}

\begin{exercise}[Exercise 1.49]
  Let $K$ be any field, $L=  K(x)$.
  \begin{enumerate}[(a)]
    \item Show that any element of $L$ that is integral over $K\left[ x \right]$ is in $K\left[ x \right]$.
    \item Show that there is no nonzero element $F\in K\left[ x \right]$ such that for every $z\in L$, $F^{n}z$ is integral over $K\left[ x \right]$ for some $n > 0$.
  \end{enumerate}
\end{exercise}
\begin{exercise}[Exercise 1.50]
  Let $K$ be a subfield of $L$.
  \begin{enumerate}[(a)]
    \item Show that the set of elements of $L$ that are algebraic over $K$ is a subfield of $L$ containing $K$.
    \item Suppose $L$ is module-finite over $K$ and $R$ is a ring such that $K\subseteq R\subseteq L$. Show that $R$ is a field.
  \end{enumerate}
\end{exercise}
\begin{solution}\hfill
  \begin{enumerate}[(a)]
    \item Let $a,b$ be algebraic over $K$. Then, $K\left( a,b \right)$ is module-finite over $K$, so $K\left( a,b \right)$ is an algebraic extension of $K$. Therefore, since $a+b,ab,a^{-1}\in K\left( a,b \right)$, all such elements algebraic over $K$, and $K$ is trivially algebraic over $K$. Thus, the set of elements in $L$ that are algebraic over $K$ forms a subfield of $L$.
    \item Let $K\subseteq R \subseteq L$. Now, since $L$ is module-finite over $K$, $L$ is ring-finite over $K$, so $R$ is ring-finite over $K$. Now, since $R\subseteq L$, $R$ is module-finite over $L$, so for any $v\in R$, there is a polynomial such that
      \begin{align*}
        v^n + b_{n-1}v^{n-1} + \cdots + b_1v + b_0 &= 0.
      \end{align*}
      Now, if $b_0 \neq 0$, we have
      \begin{align*}
        v\left( v^{n-1} + b_{n-1}v^{n-2} + \cdots + b_1 \right) &= -b_0,
      \end{align*}
      meaning that
      \begin{align*}
        v\left( \frac{-1}{b_0}\left( v^{n-1} + b_{n-1}v^{n-2} + \cdots + b_1 \right) \right) &= 1,
      \end{align*}
      and $v$ has an inverse in $R$.
  \end{enumerate}
\end{solution}
\subsection{Field Extensions}%
Let $K$ be a subfield of $L$, and suppose $L = K(v)$ for some $v\in L$. Let $\varphi\colon K\left[ x \right]\rightarrow L$ be the homomorphism mapping $x\mapsto v$. Let $\ker\left( \varphi \right) = \left\langle f \right\rangle$ for some $f\in k\left[ x \right]$. Then, $k\left[ x \right]/\left\langle f \right\rangle\cong K\left[ v \right]$, so $\left\langle f \right\rangle$ is prime.\newline

We may consider two cases.\newline

In the first case, if $f = 0$, then $K\left[ v \right]\cong K\left[ x \right]$, so $K\left( v \right) = L$ is isomorphic to $k\left( X \right)$, and thus $L$ is not ring-finite or module-finite over $K$.\newline

In the second case, if $f\neq 0$, then we may assume $f$ is monic, meaning $\left\langle f \right\rangle$ is monic, and $f$ is irreducible, so $\left\langle f \right\rangle$ is maximal, and $K\left[ v \right]$ is a field. Thus, $K\left[ v \right] = K\left( v \right)$, and $f\left( v \right) = 0$. Therefore, $v$ is algebraic over $K$, and $L=K\left[ v \right]$ is module-finite over $K$.\newline

To finish the proof of the Nullstellensatz, we must prove that if a field $L$ is a ring-finite extension of an algebraically closed field $k$, then $L = k$.\newline

Thus, it is enough to show that $L$ is module-finite over $k$ --- we already know that any ring-finite extensions are already module-finite. Now, we will show that this is always true, proving the Nullstellensatz.
\begin{proposition}
  If $L$ is ring-finite over a subfield $K$, then $L$ is module-finite over $K$.
\end{proposition}
\begin{proof}
  Let $L = K\left[ v_1,\dots,v_n \right]$. The case for $n = 1$ is taken care of by above, so we assume the result holds for all extensions generated by $n-1$ elements. Let $K_1 = K\left( v_1 \right)$; by induction, $L = K_1\left[ v_2,\dots,v_n \right]$ is module-finite over $K_1$. Assume towards contradiction that $v_1$ is not algebraic over $K$.\newline

  Each $v_i$ satisfies an equation $v_i^{n_i} + a_{i,n_i-1}v_i^{n_i-1} + \cdots = 0$, where $a_{ij}\in K_1$. Letting $a\in K\left[ v_1 \right]$ --- a multiple of the denominators of $a_{ij}$ --- we have equations $\left( av_i \right)^{n_i} + aa_{i,n_i-1}\left( av_i \right)^{n_i - 1} + \cdots = 0$.\newline

  Therefore, for any $z\in L$, there is some $N$ such that $a^N z$ is integral over $K\left[ v_1 \right]$. This must hold for all $z\in K\left( v_1 \right)$; however, since $K\left( v_1 \right)$ is isomorphic to the field of rational functions in one variable over $K$, this is impossible.
\end{proof}
\begin{exercise}[Exercise 1.51]
  Let $K$ be a field, $F\in K\left[ x \right]$ an irreducible monic polynomial of degree $n > 0$.
  \begin{enumerate}[(a)]
    \item Show that $L = K\left[ x \right]/\left\langle F \right\rangle$ is a field, and if $\overline{x}$ is the residue of $x$ in $L$, then $F\left( \overline{x} \right) = 0$.
    \item Suppose $L'$ is a field extension of $K$, $y\in L'$ such that $F(y) = 0$. Show that the homomorphism from $K\left[ x \right]$ to $L'$ that takes $x$ to $y$ induces an isomorphism of $L$ with $K\left( y \right)$.
    \item With $L'$ and $y$ as in (b), suppose $G\in K\left[ x \right]$ with $G(y) = 0$. Show that $F$ divides $G$.
    \item Show that $F = \left( x-\overline{x} \right)f_1$, where $f_1\in L\left[ x \right]$.
  \end{enumerate}
\end{exercise}
\begin{solution}\hfill
  \begin{enumerate}[(a)]
    \item Let $L = K\left[ X \right]/\left\langle F \right\rangle$, $x = X + \left\langle F \right\rangle$. Then, $F(x) = F\left( X + \left\langle X \right\rangle \right) = \left( X + \left\langle F \right\rangle \right)^{n} + \cdots + a_1\left( X + \left\langle F \right\rangle \right) + a_0 = F(X) + \left\langle F \right\rangle = 0 + \left\langle F \right\rangle$.
    \item Let $\varphi\colon K\left[ X \right]\rightarrow L'$ map $X\mapsto Y$. By the first isomorphism theorem, since $F(y) = 0$ and $F$ is irreducible, $\ker\varphi = \left\langle F \right\rangle$, so $K\left[ X \right]/\left\langle F \right\rangle = K\left( y \right)$.
    \item Since $G\in \ker\left( \varphi \right)$, and $F$ is irreducible, we have $G = FQ$ for some polynomial $Q$.
    \item This problem statement is too confusing.
  \end{enumerate}
\end{solution}
\begin{exercise}[Exercise 1.52]
  Let $K$ be a field, $F\in K\left[ x \right]$.\newline

  Show that there is a field $L$ containing $K$ such that $F = \prod_{i=1}^{n}\left( x-x_i \right)\in L\left[ x \right]$.
\end{exercise}
\begin{solution}
  Suppose this is the case for a polynomial of degree $\leq n$. Now, if $F$ is a polynomial of degree $n + 1$ in $K\left[ X \right]$. We may find $\left( X-x_i \right)$ such that $F = \left( X-x_i \right)F_1$ with $F_1\in K\left[ X \right]$. Splitting $F_1$, we obtain $F=  \prod_{i=1}^{n+1}\left( X-x_i \right)$.
\end{solution}
\begin{exercise}[Exercise 1.53]
  Suppose $K$ is a field of characteristic zero, $F$ an irreducible monic polynomial in $K\left[ x \right]$ of degree $n > 0$, and let $L$ be the splitting field of $F$. Show that the $x_i$ are distinct.
\end{exercise}
\begin{solution}
  See \href{https://ai.avinash-iyer.com/Classes_and_Homework/College/Y3/Y3S2,%20Real%20II/real_2_notes.pdf}{Algebra II Notes} regarding splitting fields over characteristic $0$ fields.
\end{solution}
\begin{exercise}[Exercise 1.54]
  Let $R$ be an integral domain with quotient field $K$, $L$ a finite algebraic extension of $K$.
  \begin{enumerate}[(a)]
    \item For any $v\in L$, show that there is a nonzero $a\in R$ such that $av$ is integral over $R$.
    \item Show that there is a basis $v_1,\dots,v_n$ for $L$ over $K$ such that each $v_i$ is integral over $R$.
  \end{enumerate}
\end{exercise}
\section{Affine Varieties}%
From now on, $k$ is a fixed algebraically closed field, with affine algebraic sets in $\A^n = \A^n\left( k \right)$. Irreducible affine algebraic sets are called \textit{affine varieties}.\newline

All rings and fields contain $k$ as a subring, with all homomorphisms of rings $\varphi\colon R\rightarrow S$ fixing $k$. We call affine varieties ``varieties'' this section since we are not dealing with other types of varieties yet.
\subsection{Coordinate Rings}%
Let $V\subseteq \A^n$ be a nonempty variety. Then, $I(V)$ is prime in $k\left[ x_1,\dots,x_n \right]$, meaning $k\left[ x_1,\dots,x_n \right]/I(V)$ is an integral domain.
\begin{definition}
  Let $\Gamma(V)\coloneq k\left[ x_1,\dots,x_n \right]/I(V)$. Then, we call $\Gamma\left( V \right)$ the \textit{coordinate ring} of $V$.
\end{definition}
If $V$ is any nonempty set, $\mathcal{F}\left( V,k \right)$ consists of all functions from $V$ to $k$ with pointwise operations. We identify $k$ with the subring of $\mathcal{F}\left( V,k \right)$ consisting of constants.
\begin{definition}
  If $V\subseteq \A^n$ is a variety, a function $f\in \mathcal{F}\left( V,k \right)$ is called a \textit{polynomial function} if there exists a polynomial $F\in k\left[ x_1,\dots,x_n \right]$ such that $f\left( a_1,\dots,a_n \right) = F\left( a_1,\dots,a_n \right)$ for all $\left( a_1,\dots,a_n \right)\in V$.\newline

  The polynomial functions form a subring of $\mathcal{F}\left( V,k \right)$ containing $k$. Two polynomials determine the same function if $\left( F-G \right)\left( a_1,\dots,a_n \right) = 0$ for all $\left( a_1,\dots,a_n \right)\in V$.\newline

  We may identify $\Gamma\left( V \right)$ with the subring of $\mathcal{F}\left( V,k \right)$ consisting of all the polynomial functions on $\mathcal{F}\left( V,k \right)$.
\end{definition}
\begin{exercise}[Exercise 2.1]
  Show that the map that associates to each $F\in k\left[ x_1,\dots,x_n \right]$ a polynomial function in $\mathcal{F}\left( V,k \right)$ is a ring homomorphism whose kernel is $I(V)$.
\end{exercise}
\begin{solution}
  The map $\varphi\colon k\left[ x_1,\dots,x_n \right]\rightarrow \mathcal{F}\left( V,k \right)$ sends to zero functions all the polynomials that are identically zero on $V$, which is equal to $I(V)$.
\end{solution}
\begin{exercise}[Exercise 2.2]
  Let $V\subseteq \A^n$ be a variety. A subvariety of $V$ is a variety $W\subseteq \A^n$ that is contained in $V$. Show that there is a natural one-to-one correspondence between algebraic subsets (resp. subvarieties, points) and radical ideals (resp. prime ideals, maximal ideals) in $\Gamma\left( V \right)$.
\end{exercise}
\begin{solution}
  We know that: algebraic subsets of $V$ correspond to radical ideals in $I(V)$; subvarieties of $V$ correspond to prime ideals in $I(V)$; points in $V$ correspond to maximal ideals in $I(V)$. Since radical ideals, prime ideals, and maximal ideals are preserved under quotients, we see that they correspond to the same objects in $\Gamma(V)$.
\end{solution}
\begin{exercise}[Exercise 2.3]
  Let $W$ be a subvariety of $V$, and let $I_V(W)$ be the ideal of $\Gamma(V)$ corresponding to $W$.
  \begin{enumerate}[(a)]
    \item Show that every polynomial function on $V$ restricts to a polynomial function on $W$.
    \item Show that the map $\varphi\colon \Gamma(V)\rightarrow \Gamma(W)$ defined in part (a) is a surjective homomorphism with kernel $I_V(W)$, so $\Gamma(W)$ is isomorphic to $\Gamma(V)/I_V(W)$.
  \end{enumerate}
\end{exercise}
\begin{solution}\hfill
  \begin{enumerate}[(a)]
    \item If $f\colon V\rightarrow k$ is a polynomial map, then by defining $f|_{W}\colon W\rightarrow k$.
    \item Let $\varphi\colon \Gamma(V)\rightarrow \Gamma(W)$ be the map defined by $\varphi\left( \left[ f \right] \right) = \left[ f|_{W} \right]$; the kernel of this map consists of all polynomials $F\in k\left[ x_1,\dots,x_n \right]$ such that $F|_{W} = 0$, which is precisely $I_V(W)$.
  \end{enumerate}
\end{solution}

\begin{exercise}[Exercise 2.4]
  Let $V\subseteq \A^n$ be a nonempty variety. Show that the following are equivalent:
  \begin{enumerate}[(i)]
    \item $V$ is a point;
    \item $\Gamma(V) = k$;
    \item $\Dim_{k}\left( \Gamma(V) \right) < \infty$.
  \end{enumerate}
\end{exercise}
\begin{solution}
  If $V$ is a point, then $V = \left( a_1,\dots,a_n \right)$ is the zero of $P = s_1\left( x_1-a_1 \right) + \cdots + s_n\left( x_n-a_n \right)$, so $I(V) = \left\langle P \right\rangle$. Since $k\left[ x_1,\dots,x_n \right] \cong k\left[ x_1-a_1,\dots,x_n-a_n \right]$ (by a translation), we have
  \begin{align*}
    \Gamma(V) &= k\left[ x_1,\dots,x_n \right]/\left\langle x_1-a_1,\dots,x_n-a_n \right\rangle\\
              &= k\left[ x_1-a_1,\dots,x_n-a_n \right]/\left\langle x_1-a_1,\dots,x_n-a_n \right\rangle\\
              &= k.
  \end{align*}
  Since $k$ is a dimension $1$ $k$-vector space, this implies (iii).\newline

  If $\Dim_{k}\left( \Gamma(V) \right) < \infty$, then $\Gamma(V)$ is a finite-dimensional $k$-algebra, meaning it is an \href{https://en.wikipedia.org/wiki/Artinian_ring}{Artinian ring}, hence has Krull dimension zero. Thus, $\left\langle \overline{0} \right\rangle\subseteq \Gamma(V)$ is prime and is not contained in any other prime ideals, meaning $I(V)$ is maximal, hence $V$ is a point.
  % take field of fractions, then is module-finite, hence algebraic, so L = k
\end{solution}
\subsection{Polynomial Maps}%
\begin{definition}
  Let $V\subseteq \A^n$, $W\subseteq \A^m$ be varieties. A map $\varphi\colon V\rightarrow W$ is called a polynomial map if there are polynomials $T_1,\dots,T_m\in k\left[ x_1,\dots,x_m \right]$ such that $\varphi\left( a_1,\dots,a_n \right) = \left( T\left( a_1,\dots,a_n \right),\dots,T_m\left( a_1,\dots,a_n \right) \right)$ for all $\left( a_1,\dots,a_n \right)\in V$.
\end{definition}
Any map $\varphi\colon V\rightarrow W$ induces a homomorphism $\widetilde{\varphi}\colon \mathcal{F}\left( W,k \right)\rightarrow \mathcal{F}\left( V,k \right)$ by $\widetilde{\varphi}\left( f \right) = f\circ\varphi$.\newline

If $\varphi$ is a polynomial map, then $\widetilde{\varphi}\left( \Gamma\left( W \right) \right)\subseteq \Gamma\left( V \right)$, so $\widetilde{\varphi}$ restricts to a homomorphism, also written $\widetilde{\varphi}$, from $\Gamma(W)$ to $\Gamma(V)$. If $f\in \Gamma(W)$ is the $I(W)$ residue of $F$, then $\widetilde{\varphi}\left( f \right) = f\circ\varphi$ is the $I(V)$ residue of the polynomial $F\left( T_1,\dots,T_m \right)$.\newline

If $V = \A^n$, $W = \A^m$, and $T_1,\dots,T_m\in k\left[ x_1,\dots,x_n \right]$ determine a polynomial map $T\colon \A^n\rightarrow \A^m$, then the $T_i$ are uniquely determined by $T$, so we usually write $T = \left( T_1,\dots,T_m \right)$.
\begin{proposition}
  Let $V\subseteq \A^n$ and $W\subseteq \A^m$ be affine varieties. There is a natural one to one correspondence between polynomial maps $\varphi\colon V\rightarrow W$ and homomorphisms $\widetilde{\varphi}\colon \Gamma(W)\rightarrow \Gamma(V)$. Any such $\varphi$ is the restriction of a polynomial map from $\A^n$ to $\A^m$.
\end{proposition}
\begin{proof}
  Let $\alpha\colon \Gamma\left( W \right)\rightarrow \Gamma\left( V \right)$ be a homomorphism. Set $T_i\in k\left[ x_1,\dots,x_n \right]$ such that $\alpha\left( \overline{x_i} \right) = \overline{T_i}$, where the residue of $x_i$ is taken in $I(W)$ and the residue of $T_i$ is taken in $I(V)$. Then, $T = \left( T_1,\dots,T_m \right)$ is a polynomial map from $\A^n$ to $\A^m$ that induces $\widetilde{T}\colon k\left[ x_1,\dots,x_m \right]\rightarrow k\left[ x_1,\dots,x_n \right]$. Note that $\widetilde{T}\left( I(W) \right)\subseteq I(V)$ by construction, so $T(V)\subseteq W$, and $T$ restricts to a polynomial map $\varphi\colon V\rightarrow W$. Now, on $\Gamma(W)$, we have
  \begin{align*}
    \widetilde{\varphi}\left( f \right)\left(\overline{x_1},\dots,\overline{x_n}\right) &= f\circ \varphi\left( x_1,\dots,x_n \right)\\
                                                                                        &= \left( T_1,\dots,T_m \right)\left( x_1,\dots,x_n \right),
  \end{align*}
  so $\widetilde{\varphi} = \alpha$.
\end{proof}
\begin{definition}
  A polynomial map $\varphi\colon V\rightarrow W$ is an isomorphism if there is a polynomial map $\psi\colon W\rightarrow V$ such that $\psi = \varphi^{-1}$.
\end{definition}
Two affine varieties are isomorphic if and only if their coordinate rings are isomorphic.
\begin{exercise}[Exercise 2.6]
Let $\varphi\colon V\rightarrow W$ and $\psi\colon W\rightarrow Z$ be polynomial maps. Show that $\widetilde{\psi\circ\varphi} = \widetilde{\varphi}\circ\widetilde{\psi}$. Show that the composition of polynomial maps is a polynomial map.
\end{exercise}
\begin{solution}
  Let $f\in \mathcal{F}\left( V,k \right)$ be a polynomial function. Then,
  \begin{align*}
    \widetilde{\psi\circ\varphi}\left( f \right) &= f\circ \left( \psi\circ\varphi \right)\\
                                                 &= \left( f\circ\psi \right)\circ\varphi\\
                                                 &= \widetilde{\varphi}\circ \widetilde{\psi}\left( f \right).
  \end{align*}
  A polynomial map $\varphi\colon V\rightarrow W$ is defined by polynomials $T_1,\dots,T_m$; similarly, a polynomial map $\psi\colon W\rightarrow Z$ is defined by polynomials $S_1,\dots,S_r$; since the composition of two polynomials is another polynomial, the composition of their respective maps is also a polynomial map.
\end{solution}
\begin{exercise}[Exercise 2.7]
  Let $\varphi\colon V\rightarrow W$ be a polynomial map, and $X$ an algebraic subset of $W$. Then, $\varphi^{-1}\left( X \right)$ is an algebraic subset of $V$. If $\varphi^{-1}\left( X \right)$ is irreducible and $X$ is contained in the image of $\varphi$, show that $X$ is irreducible.
\end{exercise}
\begin{solution}
  Let $\varphi\colon V\rightarrow W$ be a polynomial map, and let $X$ be an algebraic subset of $W$, with corresponding radical ideal $I$ in $\Gamma(W)$. There is a homomorphism of coordinate rings, $\widetilde{\varphi}\colon \Gamma(W)\rightarrow \Gamma(V)$, and since the homomorphic image of a radical ideal is a radical ideal, the corresponding radical ideal $\widetilde{\varphi}\left( I \right)\subseteq \Gamma(V)$ corresponds to $\varphi^{-1}\left( X \right)$.\newline

  Now, if $\varphi^{-1}\left( X \right)$ is irreducible, then there is a corresponding prime ideal $\mathfrak{p}\subseteq \Gamma(V)$. Taking inverse images, $\widetilde{\varphi}^{-1}\circ \widetilde{\varphi}\left( \mathfrak{p} \right)$ corresponds to $\varphi\circ\varphi^{-1}\left( X \right)$. If $X\subseteq \varphi\circ\varphi^{-1}\left( X \right)\subseteq X$, then $\mathfrak{p}\subseteq \widetilde{\varphi}^{-1}\circ\widetilde{\varphi}\left( \mathfrak{p} \right)\subseteq \mathfrak{p}$, meaning that $X$ has corresponding prime ideal $\widetilde{\varphi}^{-1}\left( \mathfrak{p} \right)$, and $X$ is irreducible.
\end{solution}
\begin{exercise}[Exercise 2.8]\hfill
  \begin{enumerate}[(a)]
    \item Show that $\set{\left( t,t^2,t^3 \right)\in \A^3\left( k \right) | t\in k}$ is an affine variety.
    \item Show that $V\left( xz-y^2,yz-x^3,x^2-x^2y \right)\subseteq \A^2\left( \C \right)$ is a variety.
  \end{enumerate}
\end{exercise}
\begin{solution}\hfill
  \begin{enumerate}[(a)]
    \item The set $S = \set{\left( t,t^2,t^3 \right)\in \A^3\left( k \right) | t\in k}$ has $I(S) = \left\langle x^2 - y, x^3 - z \right\rangle\subseteq k\left[ x,y,z \right]$. From Exercise 1.33 (b), we have that
      \begin{align*}
        k\left[ x,y,z \right]/I(S) &\cong k\left[ t \right],
      \end{align*}
      given by the surjective ring homomorphism $f\left( x,y,z \right)\mapsto f\left( t,t^2,t^3 \right)$. Since $k\left[ t \right]$ is an integral domain, this means $I(S)$ is prime, so $S$ is a variety.
    \item Using the hint, we know that $V = V\left( \left\langle y^3-x^4,z^3-x^5,z^4-y^5 \right\rangle \right)$, with algebraic set of $\set{\left( t^3,t^4,t^5 \right) | t\in k}$. This means we have a map $\varphi\colon \A^1\left( \C \right)\rightarrow V$ by taking $t\mapsto \left( t^3,t^4,t^5 \right)$. This map is bijective, so the induced homomorphism $\varphi\colon \Gamma\left( V \right)\rightarrow \Gamma\left( \A^1\left( \C \right) \right)$ is an isomorphism. Since $\Gamma\left( \A^1\left( \C \right) \right) = \C\left[ x \right]$ is an integral domain, so too is $\Gamma\left( V \right)$, so $I(V)$ is prime, and $V$ is a variety.
  \end{enumerate}
\end{solution}
\begin{exercise}[Exercise 2.9]
  Let $\varphi\colon V\rightarrow W$ be a polynomial map of affine varieties, with $V'\subseteq V$ and $W'\subseteq W$ subvarieties. Suppose $\varphi\left( V' \right)\subseteq W'$.
  \begin{enumerate}[(a)]
    \item Show that $\widetilde{\varphi}\left( I_W\left( W' \right) \right) \subseteq I_V\left( V' \right)$.
    \item Show that the restriction of $\varphi$ gives a polynomial map from $V'$ to $W'$.
  \end{enumerate}
\end{exercise}
\begin{solution}\hfill
  \begin{enumerate}[(a)]
    \item Let $ \overline{x_i} $ be the image of $x_i$ in $\Gamma(V)$, and let $ \overline{y_i} $ be the image of $y_i$ in $\Gamma(W)$, where
      \begin{align*}
        \Gamma(V) &= k\left[ x_1,\dots,x_m \right]/I(V)\\
        \Gamma(W) &= k\left[ y_1,\dots,y_n \right]/I(W).
      \end{align*}
      Let $f\left( \overline{y}_1,\dots, \overline{y}_n \right)\in I_W\left( W' \right)$, meaning $f\left( a_1,\dots,a_n \right)= 0$ for all $\left( a_1,\dots,a_n \right)\in W'$. Let $\left( b_1,\dots,b_m \right)\in V'$. Then,
      \begin{align*}
        \widetilde{\varphi}\left( f \right)\left( b_1,\dots,b_m \right) &= f\left( \varphi\left( b_1,\dots,b_m \right) \right)\\
                                                                        &= 0,
      \end{align*}
      where we use the fact that $\varphi\left( V' \right)\subseteq W'$. Thus, $\varphi\left( b_1,\dots,b_n \right)\in W'$, and $\widetilde{\varphi}\left( I_W\left( W' \right) \right)\subseteq I_V\left( V' \right)$.
    \item Using Exercise 2.3 and the duality relation, we notice that $\widetilde{\varphi}\colon \Gamma\left( W' \right)\rightarrow \Gamma\left( V' \right)$ is a homomorphism, so we use the proposition to determine that $\varphi|_{V'}$ is a polynomial map.
  \end{enumerate}
\end{solution}

\begin{exercise}[Exercise 2.10]
  Show that the projection map $P\colon \A^n\rightarrow \A^r$, where $n\geq r$, defined by $P\left( a_1,\dots,a_n \right) = \left( a_1,\dots,a_r \right)$ is a polynomial map.
\end{exercise}
\begin{solution}
  Define $T_1,\dots,T_r$ to be identity.
\end{solution}
\begin{exercise}[Exercise 2.12]\hfill
  \begin{enumerate}[(a)]
    \item Let $\varphi\colon \A^1\rightarrow V = V\left( y^2-x^3 \right)\subseteq \A^2$ be defined by $\varphi\left( t \right) = \left( t^2,t^3 \right)$. Show that, although $\varphi$ is an injective polynomial map, $\varphi$ is not an isomorphism.
    \item Let $\varphi\colon \A^1\rightarrow V = V\left( \left\langle y^2-x^2\left( x+1 \right) \right\rangle \right)$ be defined by $\varphi\left( t^2-1,t\left( t^2-1 \right) \right)$. Show that $\varphi$ is one-to-one and onto except that $\varphi\left( \pm 1 \right) = (0,0)$.
  \end{enumerate}
\end{exercise}
\begin{solution}\hfill
  \begin{enumerate}[(a)]
    \item 
  \end{enumerate}
\end{solution}

\subsection{Coordinate Changes}%
If $T = \left( T_1,\dots,T_m \right)$ is a polynomial map from $\A^n$ to $\A^m$, and $F$ is a polynomial in $k\left[ x_1,\dots,x_m \right]$, we let $F^T = \widetilde{T}\left( F \right) = F\left( T_1,\dots,T_m \right)$.\newline

For ideals $I$ and algebraic sets $V$ in $\A^m$, $I^T$ is the ideal in $k\left[ x_1,\dots,x_m \right]$ generated by $\set{F^T | F\in I}$, and $V^T$ denotes $T^{-1}\left( V \right) = V\left( I^T \right)$, where $I = I(V)$. If $V$ is the hypersurface of $F$, then $V^T$ is the hypersurface of $F^T$ if $F^T$ is not constant.\newline

A \textit{change of coordinates} on $\A^n$ is a polynomial map $T\colon \A^n\rightarrow \A^n$ such that each $T_i$ is a polynomial of degree $1$ and $T$ is bijective. If $T_i = \sum a_{ij}x_j + a_{i0}$, then $T = T'' \circ T'$, where $T'$ is a linear map and $T''$ is a translation. Since translations are invertible, it follows that $T$ is bijective if and only if $T'$ is invertible.\newline

If $T$ and $U$ are affine changes of coordinates on $\A^n$, then so are $T\circ U$ and $T^{-1}$; in other words, $T$ is an automorphism of the variety $\A^n$.
\begin{exercise}[Exercise 2.14]
A set $V\subseteq \A^n\left( k \right)$ is called a linear subvariety of $\A^n\left( k \right)$ if $V = V\left( \left\langle F_1,\dots,F_r \right\rangle \right)$, where the $F_i$ are polynomials of degree $1$.
\begin{enumerate}[(a)]
  \item Show that if $T$ is an affine change of coordinates on $\A^n$, then $V^T$ is also a linear subvariety of $\A^n\left( k \right)$.
  \item If $V\neq\emptyset$ is a linear subvariety, show that there is an affine change of coordinates $T$ of $\A^n$ such that $V^{T} = V\left( x_{m+1},\dots,x_n \right)$
  \item Show that the $m$ that appears in part (b) is independent of the choice of $T$. It is called the dimension of $V$.
\end{enumerate}
\end{exercise}
\begin{solution}\hfill
  \begin{enumerate}[(a)]
    \item If $T$ is an affine change of coordinates, then each $T_i$ is of the form $T_i = \sum a_{ij}x_j + a_{i0}$. Considering $F_i^T = F_i\left( T_1,\dots,T_i \right)$, we must have each $F_i$ as a function of exactly one $T_i$. Since each $T_i$ is also a polynomial of degree $1$, $V^T = T^{-1}\left( V \right)$ is a variety generated by a family of polynomials of degree $1$, so $V^T$ is a linear subvariety.
    \item Let $V = V\left(F_1\right)$ for some degree $1$ polynomial $F = \sum a_i x_i + a_0$. Define $T = \left( T_1,\dots,T_m \right)$. We may take $T_m$ by defining
      \begin{align*}
        T_m\left( x_n \right) &= - \frac{a_0}{a_n} -\frac{a_1}{a_n}x_1 - \frac{a_2}{a_n} \cdots + \frac{1}{a_n}x_m\\
        T_m\left( x_i \right) &= x_i.\tag*{$i\leq n-1$}
      \end{align*}
      Then, $F_1\circ T = x_m$, so $V^T = V\left( x_m \right)$.\newline

      For the inductive step, we take $V = V\left( F_1,\dots,F_r,F_{r+1} \right)$, and suppose $T$ is defined for $V\left( F_1,\dots, F_r \right)$. Then, we may define
      \begin{align*}
        V^T &= T^{-1}\left( V\left( F_1,\dots,F_r \right) \right)\cap T^{-1}\left( F_{r+1} \right)\\
            &= V\left( x_{m+1},\dots,x_n \right)\cap T^{-1}\left( F_{r+1} \right),
      \end{align*}
      and we may set $T$ to be such that $T^{-1}\left( V\left( F_{r+1} \right) \right) = V\left( x_m \right)$, satisfying the inductive step.
    \item Suppose there were a change of coordinates $T = \left( T_1,\dots,T_n \right)$ such that $V\left( x_{m+1},\dots,n \right)^{T} = V\left( x_{s+1},\dots,x_n \right)$, where $s < m$. Then, by definition,
      \begin{align*}
        T^{-1}\left( V\left( x_{m+1},\dots,x_{n} \right) \right)  &= V\left( x_{s+1},\dots,x_n \right),
      \end{align*}
      meaning that, since affine transformations are bijective,
      \begin{align*}
        T\left( V\left( x_{s+1},\dots,x_n \right) \right) &= V\left( x_{m+1},\dots,x_n \right).
      \end{align*}
      This means that any polynomial in $x_{s+1},\dots,x_n$ yields a polynomial exclusively in $x_{m+1},\dots,x_n$; this means that at least one of the affine transformations in $T_1,\dots,T_n$ yields $0$ by the pigeonhole principle, so the transformations in $T_1,\dots,T_n$ are not independent.
  \end{enumerate}
\end{solution}
\begin{exercise}[Exercise 2.15]
  Let $P = \left( a_1,\dots,a_n \right)$ and $Q = \left( b_1,\dots,b_n \right)$ be distinct points in $\A^n$. The line through $P,Q$ is defined by $\set{a_1 + t\left( b_1-a_1 \right),\dots,a_n + t\left( b_n-a_n \right) | t\in k}$.
  \begin{enumerate}[(a)]
    \item Show that if $L$ is defined through $P$ and $Q$, and $T$ is an affine change of coordinates, then $T(L)$ is the line through $T(P)$ and $T(Q)$.
    \item Show that a line is a linear subvariety of dimension $1$, and that any linear subvariety of dimension $1$ is the line through any two of its points.
    \item Show that, in $\A^2$, a line is the same thing as a hyperplane.
    \item Let $P,P'\in \A^2$, $L_1,L_2$ be two distinct lines through $P$, and $L_1',L_2'$ distinct lines through $P'$. Show that there is an affine change of coordinates of $\A^2$ such that $T(P)= P'$ and $T\left( L_i \right) = L_i'$.
  \end{enumerate}
\end{exercise}

\subsection{Local Rings}%
Let $V$ be a nonempty variety in $\A^n$, and let $\Gamma\left( V \right)$ be its coordinate ring. We may define the quotient field on $\Gamma(V)$, giving the \textit{field of rational functions} on $V$, written $k(V)$.\newline

If $f$ is a rational function on $V$, and $P\in V$, we say $f$ is defined at $P$ if for some $a,b\in \Gamma\left( V \right)$, $f = \frac{a}{b}$, and $b(P)\neq 0$. If $\Gamma(V)$ is a unique factorization domain, there is an essentially unique representation $f = a/b$ with $a,b$ having no common factors.
\begin{example}
  If $V = V\left( xw-yz \right)\subseteq \A^4\left( k \right)$, then $\Gamma(V) = k\left[ x,y,z,w \right]/\left\langle xw-yz \right\rangle$. Letting $ \overline{x}, \overline{y}, \overline{z}, \overline{w}$ represent the residues, we have $\frac{ \overline{x} }{ \overline{y} }= \frac{ \overline{z} }{ \overline{w} } = f\in k(V)$ is defined at $p \left( x,y,z,w \right)$ whenever $y$ or $w$ are not equal to $0$.
\end{example}
Letting $P\in V$, we define $\mathcal{O}_P(V)$ to be the set of rational functions on $V$ that are defined at $P$. It turns out that $\mathcal{O}_{P}(V)$ defines a subring of $k(V)$ containing $\Gamma(V)$, which we call the \textit{local ring} of $V$ at $P$.\newline

The set of points $P\in V$ where a rational function is not defined is called the pole set of $f$.
\begin{proposition}\hfill
  \begin{enumerate}[(1)]
    \item The pole set of a rational function is an algebraic subset of $V$.
    \item 
      \begin{align*}
        \Gamma\left( V \right) &= \bigcap_{P\in V}\mathcal{O}_{P}\left( V \right).
      \end{align*}
  \end{enumerate}
\end{proposition}
\begin{proof}
  Suppose $V\subseteq \A^n$. Let $ \overline{G} $ be the residue of $G\in k\left[ x_1,\dots,x_n \right]$ in $\Gamma(V)$. Let $f\in k(V)$, and let
  \begin{align*}
     J_f= \set{G | \overline{G}f\in \Gamma(V)}.
  \end{align*}
  Note that $J_f$ is an ideal containing $I(V)$, and points of $V\left(J_f\right)$ are those points where $f$ is not defined.\newline

  Now, if $f\in \bigcap_{P\in V}\mathcal{O}_P\left( V \right)$, $V\left(J_f\right) = \emptyset$, so $1\in J_f$ by the Nullstellensatz, meaning $f\in \Gamma(V)$.
\end{proof}
Let $f\in \mathcal{O}_P\left( V \right)$. We can define the value of $f$ at $P$, written $f(P)$, to be $a(P)/b(P)$. The ideal 
\begin{align*}
  \mathfrak{m}_P(V) = \set{f\in \mathcal{O}_P(V) | f(P) = 0}
\end{align*}
is called the \textit{maximal ideal} of $V$ at $P$. It is the kernel of the evaluation homomorphism $f\mapsto f(P)$ onto $k$, so $\mathcal{O}_P(V)/\mathfrak{m}_P(V)$ is isomorphic to $k$.\newline

In particular, note that all elements of $\mathcal{O}_P(V)$ that are not in $\mathfrak{m}_P(V)$ are units.
\begin{lemma}
  The following conditions on a ring $R$ are equivalent.
  \begin{enumerate}[(1)]
    \item The set of non-units in $R$ forms an ideal.
    \item $R$ has a unique maximal ideal that contains every proper ideal of $R$.
  \end{enumerate}
\end{lemma}
\begin{proof}
  Let $\mathfrak{m} = \set{\text{non-units of }R}$. Every proper ideal of $R$ is contained in $\mathfrak{m}$.
\end{proof}
A ring that satisfies these conditions is known as a local ring. The units are those elements not belonging to the maximal ideal. 
\begin{proposition}
  $\mathcal{O}_P(V)$ is a Noetherian local integral domain.
\end{proposition}
\begin{proof}
  We only need to show that every ideal $I$ of $\mathcal{O}_P(V)$ is finitely generated. Since $\Gamma(V)$ is Noetherian, we may choose generators $f_1,\dots,f_r$ for the ideal $I\cap \Gamma(V)$ of $\Gamma(V)$. We claim that $f_1,\dots,f_r$ generate $I$ in $\mathcal{O}_P(V)$. If $f\in I\subseteq \mathcal{O}_P(V)$, there is a $b\in \Gamma(V)$ with $b(P)\neq 0$ and $bf\in \Gamma(V)$. Then, $bf\in \Gamma(V)\cap I$, so $bf = \sum a_if_i$ for some $a_i\in \Gamma(V)$, meaning $f = \sum (a_i/b)f_i$ as desired.
\end{proof}

\begin{exercise}[Exercise 2.17]
  Let $V = V\left( y^2 - x^2\left( x+1 \right) \right)$, and $ \overline{x}, \overline{y} $ residues in $\Gamma(V)$. Let $z = \frac{ \overline{y} }{ \overline{x} }$. Find the pole sets of $z$ and $z^2$.
\end{exercise}
\begin{solution}
  We start by verifying the pole sets for $z^2$. Taking $z^2$, we have
  \begin{align*}
    z^2 &= \frac{ \overline{y}^2 }{ \overline{x}^2}\\
        &= \frac{ \overline{x}^2\left( \overline{x} + 1 \right) }{ \overline{x}^2 }\\
        &= \overline{x} + 1,
  \end{align*}
  meaning $z^2$ has no poles.\newline

  Now, since $z = \frac{ \overline{y} }{ \overline{x} }$, the only possible poles are points $\left( a,b \right)$ where $a = 0$. However, if $P\in V$ and $a = 0$, we must have $b^2 = 0$, so $b = 0$. Therefore, the only possible pole is where $P = (0,0)$. However, we must verify that this is indeed a pole.\newline

  Suppose $z$ is defined at $(0,0)$, so we may write $z = \frac{f\left( \overline{x}, \overline{y} \right)}{g\left( \overline{x}, \overline{y} \right)}$, for some $f,g\in \Gamma(V)$ with $g(0,0)\neq 0$. Since $ \overline{y}^2 = \overline{x}^2 \left( \overline{x} + 1 \right) $, we may write $g\left( \overline{x}, \overline{y} \right) = g_0\left(  \overline{x} \right) + \overline{y}g_1\left( \overline{x} \right)$ (any other factors of $ \overline{y} $ can be rewritten in terms of $ \overline{x} $), and similarly writing $f\left( \overline{x}, \overline{y} \right) = f_0\left( \overline{x} \right) + \overline{y}f_1\left( \overline{x} \right)$. Therefore,
  \begin{align*}
    \frac{ \overline{y} }{ \overline{x} } &= \frac{f_0\left( \overline{x} \right) + \overline{y}f_1\left(  \overline{x} \right)}{ g_0\left( \overline{x} \right) + \overline{y} g_1\left( \overline{x} \right) },
  \end{align*}
  so
  \begin{align*}
    \overline{y}\left( g_0\left( \overline{x} \right) + \overline{y}g_1\left( \overline{x} \right) \right) &= \overline{x}\left( f_0\left( \overline{x} \right) + \overline{y}f_1\left( \overline{x} \right) \right).
  \end{align*}
  Writing $ \overline{y}^2 = \overline{x}^2\left( \overline{x} + 1 \right) $, we get
  \begin{align*}
    g_0\left( \overline{x} \right) \overline{y} g_1\left( \overline{x} \right)\left( \overline{x}^2\left( \overline{x} + 1 \right) \right) &= f_0\left( \overline{x} \right) \overline{x} + \overline{x} \overline{y} f_1\left( \overline{x} \right),
  \end{align*}
  so that $g_0\left( \overline{x} \right) = \overline{x}f_1\left( \overline{x} \right)$, and $g_0 = 0$. Therefore, $g\left( 0,0 \right) = g_0(0) + 0\cdot g_1(0) = 0$, which is a contradiction.
\end{solution}
\begin{exercise}[Exercise 2.18]
  Let $\mathcal{O}_P(V)$ be the local ring of a variety $V$ at point $P$. Show that there is a natural one-to-one correspondence between the prime ideals in $\mathcal{O}_P(V)$ and the subvarieties of $V$ that pass through $P$. 
\end{exercise}
\begin{solution}
  Let $I$ be prime in $\mathcal{O}_P(V)$. Then, $I\cap \Gamma(V)\subseteq \Gamma(V)$  is prime, so $I\cap \Gamma(V)$ corresponds to a unique subvariety of $V$. Specifically, since $I\subseteq \mathcal{O}_P(V)$ is an ideal, it is contained in $\mathfrak{m}_P$, so $f$ is zero at $P$, meaning the subvariety corresponding to $I\cap \Gamma(V)$ passes through $P$.
\end{solution}
\begin{exercise}[Exercise 2.21]
Let $\varphi\colon V\rightarrow W$ be a polynomial map of affine varieties, $\widetilde{\phi}\colon \Gamma(W)\rightarrow \Gamma(V)$ the induced map of coordinate rings.\newline

Suppose $P\in V$, $\varphi\left( P \right) = Q$. Show that $\widetilde{\varphi}$ extends uniquely to a ring homomorphism $ \overline{\varphi}\colon\mathcal{O}_Q(W)\to \mathcal{O}_P(V)$. Show that $ \overline{\varphi}\left( \mathfrak{m}_Q(W) \right) \subseteq \mathfrak{m}_P(V) $.
\end{exercise}
\begin{solution}
  Let $f = a/b\in \mathcal{O}_Q(W)$ be in reduced form. Define
  \begin{align*}
    \overline{\varphi}\left( f \right) &= (a\circ\varphi)/(b\circ\varphi)\\
                                       &= \widetilde{\varphi}(a)/\widetilde{\varphi}(b).
  \end{align*}
  Since $\widetilde{\varphi}$ is unique, and $f$ is written in its unique reduced form, this gives a unique map $ \overline{\varphi}\colon \mathcal{O}_Q(W)\rightarrow \mathcal{O}_P(V) $. 
\end{solution}
\begin{exercise}[Exercise 2.22]
Let $T\colon \A^n\rightarrow \A^n$ be an affine change of coordinates, with $T(P) = Q$. Show that $\widetilde{T}\colon \mathcal{O}_Q\left( \A^n \right)\rightarrow \mathcal{O}_P\left( \A^n \right)$ is an isomorphism. Show that $\widetilde{T}$ induces an isomorphism from $\mathcal{O}_Q\left( V \right)$ to $\mathcal{O}_P\left( V^T \right)$ if $P\in V^T$ for any subvariety $V\subseteq \A^n$.
\end{exercise}

\subsection{Discrete Valuation Rings}%
\begin{proposition}
  Let $R$ be an integral domain that is not a field. The following are equivalent:
  \begin{enumerate}[(1)]
    \item $R$ is a local, Noetherian, and the maximal ideal is principal;
    \item there is an irreducible element $t\in R$ such that every nonzero $z\in R$ may be written uniquely in the form $z = ut^{n}$ for some unit $u\in R$ and $n$ a nonnegative integer.
  \end{enumerate}
\end{proposition}
\begin{proof}
  Assume (1). Let $\mathfrak{m}$ be the maximal ideal, and $t$ a generator for $\mathfrak{m}$. Suppose $ut^{n} = vt^{m}$ with $u,v$ units and $n\geq m$. Then, $ut^{n-m} = v$ is a unit, so $n = m$ and $u = v$. Thus, any expression of $z$ is unique.\newline

  To show that $z$ has an expression, we may assume $z = z_1 t$ for some $z_1\in R$. If $z_1$ is a unit, we are done. Then, we assume $z_1 = z_2 t$, so that we have a sequence $\left( z_k \right)_k$, where $z_{k} = z_{k+1}t$. Since $R$ is Noetherian, the chain of ideals $\left\langle z_1 \right\rangle\subseteq \left\langle z_2 \right\rangle\subseteq \cdots$ has a maximal member, so $\left\langle z_n \right\rangle = \left\langle z_{n+1} \right\rangle$ for some $n$. Thus, $z_{n+1} = vz_n$ for some $v\in R$, and $z_n = vtz_n$, and $vt = 1_R$, but $t$ is not a unit.\newline

  Assume (2). We note that $\mathfrak{m}= \left\langle t \right\rangle$ is the set of non-units, and that the only ideals in $R$ are the principal ideals, $\left\langle t^n \right\rangle$ for some nonnegative integer, meaning $R$ is a principal ideal domain.
\end{proof}
Any ring that satisfies these conditions is called a \textit{discrete valuation ring}, which we call a DVR. The element $t$ is known as a uniformizing parameter for $R$, and any other uniformizing parameter is of the form $ut$ for some unit $u\in R$.\newline

If $K$ is the field of fractions for $R$, then for fixed $t$, a nonzero element $z\in K$ has an expression $z = ut^{n}$ for a unit $u$ and $n\in\Z$. The exponent $n$ is called the \textit{order} of $z$, which we write $\operatorname{ord}\left( z \right)$. We define $\operatorname{ord}\left( 0 \right) = \infty$.
\subsection{Forms}%
Let $R$ be an integral domain. If $F\in R\left[ x_1,\dots,x_{n+1} \right]$ is a form, then we define $F_{\ast}\in F\left[ x_1,\dots,x_n \right]$ by taking $F_{\ast} = F\left( x_1,\dots,x_n,1 \right)$.\newline

Conversely, for any polynomial $f\in R\left[ x_1,\dots,x_n \right]$ of degree $d$, we write $f = f_0 + f_1 + \cdots + f_d$, and define $f^{\ast}\in R\left[ x_1,\dots,x_{n+1} \right]$ to be
\begin{align*}
  f^{\ast} &= x^{d}_{n+1}f\left( x_1/x_{n+1},\dots,x_n/x_{n+1} \right).
\end{align*}
Then, $f^{\ast}$ is a form of degree $d$.
\subsection{Direct Products}%
If $R_1,\dots,R_n$ are rings, the cartesian product $R_1\times\cdots\times R_n$ is made into a ring by taking pointwise addition and pointwise multiplication.\newline

This ring is known as the direct product of $R_1,\dots,R_n$, written $\prod_{i=1}^{n}R_i$. The natural projection maps $\pi_i\colon \prod_{j=1}^{n}R_j\rightarrow R_i$, given by $\left( a_1,\dots,a_n \right)\mapsto a_i$ are ring homomorphism.\newline

The direct product is characterized by the following universal property: given any ring $R$ and family of ring homomorphisms $\varphi_i\colon R\rightarrow R_i$, there is a unique ring homomorphism $\varphi\colon R\rightarrow \prod_{i=1}^{n}R_i$ such that $\pi_i\circ\varphi = \varphi_i$.\newline

In particular, if a field $k$ is a subring of each $R_i$, we may regard $k$ as a subring of the product.
\subsection{Operations with Ideals}%
\subsection{Ideals with a Finite Number of Zeros}%

\end{document}
