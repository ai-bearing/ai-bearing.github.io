\documentclass[10pt]{mypackage}

% sans serif font:
%\usepackage{cmbright,sfmath,bbold}
%\renewcommand{\mathcal}{\mathtt}

%Euler:
\usepackage{newpxtext,eulerpx,eucal,eufrak}
\renewcommand*{\mathbb}[1]{\varmathbb{#1}}
\renewcommand*{\hbar}{\hslash}

%kp fonts:
%\usepackage{kpfonts}
%\renewcommand{\mathbb}{\mathds}
%\usepackage{homework}

\pagestyle{fancy} %better headers
\fancyhf{}
\rhead{Avinash Iyer}
\lhead{Algebraic Geometry}

\setcounter{secnumdepth}{0}

\begin{document}
\RaggedRight
\section{Introduction}%
Oh hey, it's another one of these independent studies. Me and a friend are going to be going through William Fulton's \textit{Algebraic Curves}. It will be hard, it will be long, and it might not work out for me, but who cares.
\tableofcontents
\section{Affine Algebraic Sets}%
\subsection{Algebraic Preliminaries}%
We will assume all rings are commutative with unity, where $\Z$ is the integers, $\Q$ is the rationals, $\R$ is the reals, and $\C$ is the complex numbers.\newline

Any integral domain $R$ has a quotient field $K$, which contains $R$ as a subring, and any element in $K$ may be written as a not necessarily unique ratio of two elements of $R$. Any one-to-one ring homomorphism from $R$ to a field $L$ extends uniquely to a ring homomorphism from $K$ to $L$.\newline

If $R$ is a ring, then $R[x]$ is the ring of polynomials with coefficients in $R$. The degree of a nonzero polynomial $\sum a_ix^i$ is the largest integer $d$ such that $a_d\neq 0$. The polynomial is monic if $a_d = 1$.\newline

The ring of polynomials in $n$ variables over $R$ is $R\left[x_1,\dots,x_n\right]$. We write $R\left[x,y\right]$ and $R\left[x,y,z\right]$ if $n=2$ and $3$ respectively. Monomials in $R\left[x_1,\dots,x_n\right]$ are of the form $x^{(i)} \coloneq x_1^{i_1}x_2^{i_2}\cdots x_n^{i_n}$, where $i_j$ are nonnegative integers, and the degree of the monomial is $i_1 + \cdots i_n$. Every $F\in R\left[x_1,\dots,x_n\right]$ has a unique expression $F = \sum a_{(i)}x^{(i)}$, where $x^{(i)}$ are monomials, and $a_{(i)}\in R$. We say $F$ is homogeneous of degree $d$ if all $a_{(i)}$ are zero except for monomials of degree $d$. The polynomial $F$ is written as $F = F_0 + F_1 + \cdots F_d$, where $F_i$ is a form of degree $i$, and $d = \deg(F)$ for $F_d\neq 0$.\newline

The ring $R$ is a subring of $R\left[x_1,\dots,x_n\right]$, and the ring $R\left[x_1,\dots,x_n\right]$ is characterized by the following: if $\varphi\colon R\rightarrow S$ is a ring homomorphism, and $s_1,\dots,x_n$ are elements in $S$, then there is a unique extension of $\varphi$ to a ring homomorphism $\overline{\varphi}\colon R\left[x_1,\dots,x_n\right]\rightarrow S$ such that $\overline{\varphi}\left(x_i\right) = s_i$. The image of $F$ under $\overline{\varphi}$ is written $F\left(s_1,\dots,s_n\right)$. The ring $R\left[x_1,\dots,x_n\right]$ is canonically isomorphic to $R\left[x_1,\dots,x_{n-1}\right]\left[x_n\right]$.\newline

An element $a\in R$ is called irreducible if it is not a unit or zero, and any factorization $a=bc$ with $b,c\in R$ is such that either $b $ or $c$ is a unit. A domain $R$ is a unique factorization domain (UFD) if every nonzero element in $R$ can be factored uniquely up to units and ordering.\newline

If $R$ is a UFD with quotient field $K$, then any irreducible element $F\in R[x]$ remains irreducible when considered in $K[x]$.
\begin{theorem}[Gauss's Lemma for $\Z$]
  If $F\in \Z[x]$ is a monic polynomial that is irreducible, then $F$ is irreducible in $\Q[x]$.
\end{theorem}
If $F$ and $G$ are polynomials in $R[x]$ with no common factors in $R[x]$, then they have no common factors in $K[x]$.\newline

If $R$ is a UFD, then $R[x]$ is also a UFD, and consequently $k\left[x_1,\dots,x_n\right]$ is a UFD for any field $k$. The quotient field of $k\left[x_1,\dots,x_n\right]$ is written $k\left(x_1,\dots,x_n\right)$ is called the field of rational functions in $n$ variables over $k$.\newline

If $\varphi\colon R\rightarrow S$ is a ring homomorphism, $\ker\left(\varphi\right)\coloneq \varphi^{-1}(0)$. The kernel is an ideal in $R$. An ideal in $R$ is proper if $I\neq R$, and a proper ideal is known as maximal if it is not contained in any larger proper ideal.\footnote{Alternatively, an ideal $I$ is maximal if the quotient ring $R/M$ is a field.} An ideal $\mathfrak{p}$ is prime if, whenever $ab\in \mathfrak{p}$, then $a\in \mathfrak{p}$ or $b\in \mathfrak{p}$.\footnote{Alternatively, an ideal $\mathfrak{p}$ is prime if $R/\mathfrak{p}$ is an integral domain.}\newline

Let $k$ be a field and $I$ a proper ideal in $k\left[x_1,\dots,x_n\right]$. The canonical homomorphism $\pi$ from $k\left[x_1,\dots,x_n\right]$ to $k\left[x_1,\dots,x_n\right]/I$ restricts to a ring homomorphism from $k$ to $k\left[x_1,\dots,x_n\right]/I$. We regard $k$ as a subring of $k\left[x_1,\dots,x_n\right]/I$, which is a vector space over $k$.\newline

If $R$ is an integral domain, then $\operatorname{char}\left(R\right)$, the characteristic of $R$, is the smallest integer $p$ such that $\underbrace{1+1\cdots +1}_{p\text{ times}} = 0$. If $p$ exists, we say $\operatorname{char}\left(R\right) = p$, else $0$.\newline

Note that if $\varphi\colon \Z\rightarrow R$ is the unique ring homomorphism from $\Z$ to $R$,\footnote{This is because $\Z$ is initial in the category of rings. See Aluffi.} then $\ker\left(\varphi\right) = \left\langle p \right\rangle$, so $\operatorname{char}\left(R\right)$ is prime or $0$.\newline

If $R$ is a ring, and $F\in R\left[x\right]$, and $a$ is a root of $F$, then $F = \left(x-a\right)G$ for some unique polynomial $G\in R[x]$. A field $k$ is algebraically closed if any nonconstant $F\in k\left[x\right]$ has a root.
\begin{exercise}[Exercise 1.1]
Let $R$ be an integral domain.
\begin{enumerate}[(a)]
  \item If $F$ and $G$ are forms of degree $r$ and $s$ respectively in $R\left[x_1,\dots,x_n\right]$, show that $FG$ is a form of degree $r+s$.
  \item Show that any factor of a form in $R\left[x_1,\dots,x_n\right]$ is also a form.
\end{enumerate}
\end{exercise}
\begin{exercise}
  Let $R$ be a UFD and $K$ the quotient field of $R$. Show that every element $z\in K$ may be written as $z = a/b$, where $a,b\in R$ have no common factors. This representative is unique up to units of $R$.
\end{exercise}
\begin{exercise}
  Let $R$ be a principal ideal domain, and let $P$ be a nonzero proper prime ideal in $R$.
  \begin{enumerate}[(a)]
    \item Show that $P$ is generated by an irreducible element.
    \item Show that $P$ is maximal.
  \end{enumerate}
\end{exercise}

\end{document}
