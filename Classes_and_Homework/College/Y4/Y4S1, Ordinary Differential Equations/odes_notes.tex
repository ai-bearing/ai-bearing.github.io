\documentclass[10pt]{mypackage}

% sans serif font:
%\usepackage{cmbright}
%\usepackage{sfmath}
%\usepackage{bbold} %better blackboard bold

%serif font + different blackboard bold for serif font
\usepackage{newpxtext,eulerpx}
\renewcommand*{\mathbb}[1]{\varmathbb{#1}}

\fancyhf{}
\rhead{Avinash Iyer}
\lhead{Ordinary Differential Equations: Class Notes}

\setcounter{secnumdepth}{0}

\begin{document}
\section{First-Order Differential Equations}%
\subsection{Introduction to first-order ODEs}%
  Recall that for $y = f(x)$, $x$ is the independent variable and $y$ is the dependent variable.
  \begin{definition}[Differential Equation]
  A differential equation is an equation which contains derivatives of a dependent variable with respect to one or more independent variables.
  \end{definition}
  \begin{example}[A Basic Differential Equation]
    \begin{align*}
      \frac{dy}{dx} - 5y - 1 &= 
    \end{align*}
  \end{example}
  We can classify differential equations by
  \begin{itemize}
    \item type;
    \item order;
    \item linearity.
  \end{itemize}
  \begin{definition}[Classification by Type]
  There are two types of differential equations:
  \begin{itemize}
    \item ordinary differential equations (ODEs);
    \item partial differential equations (PDEs).
  \end{itemize}
  ODEs are characterized by derivatives of the dependent variable with respect to one independent variable. PDEs are characterized by derivatives of the dependent variable with respect to multiple independent variables.
  \end{definition}
  \begin{example}[ODEs and PDEs]\hfill
    \begin{enumerate}[(1)]
      \item An ODE:
        \begin{align*}
          \frac{d^2y}{dx^2} - 2\frac{dy}{dx} + 6y &= 0
        \end{align*}
      \item A PDE:
        \begin{align*}
          \pd{^2u}{x^2} + \pd{^2u}{y^2} &= 0
        \end{align*}
    \end{enumerate}
  \end{example}
  \begin{definition}[Classificiation by Order]
    The order of the highest derivative in a differential equation is the order of the differential equation.
  \end{definition}
  \begin{example}[Differential Equations of Varying Orders]\hfill
    \begin{enumerate}[(1)]
      \item 
        \begin{align*}
          \diff{^2y}{x^2} + 5\left(\frac{dy}{dx}\right) - 4y &= x \tag*{order 2}
        \end{align*}
      \item 
        \begin{align*}
          2\frac{dy}{dx} + y &= 0\tag*{order 1}
        \end{align*}
      \item 
        \begin{align*}
          \sin(x)y''' - \left(\cos x\right)y' &= 2\tag*{order 3}
        \end{align*}
    \end{enumerate}
  \end{example}
  In general, we write a differential equation of order $n$ in the form
  \begin{align*}
    \underbrace{F\left(x,y,\frac{dy}{dx},\frac{d^2y}{dx^2},\dots,\frac{d^ny}{dx^n}\right)}_{\text{$n+2$ variables}} = 0.
  \end{align*}
  \begin{example}
    Suppose we have the differential equation
    \begin{align*}
      \frac{d^3y}{dx^3} + y^2 &= x.
    \end{align*}
    Then, we rewrite as
    \begin{align*}
      \underbrace{\frac{d^3y}{dx^3} + y^2 - x}_{F\left(x,y,\frac{dx}{dy},\frac{d^2x}{dy^2},\frac{d^3x}{dy^3}\right)} &= 0.
    \end{align*}
    Alternatively, we can write as
    \begin{align*}
      \frac{d^3y}{dx^3} &= \underbrace{x - y^2}_{f(x,y)}.
    \end{align*}
  \end{example}
  \begin{definition}[Classification by Linearity]
    We much prefer to analyze linear differential equations over nonlinear differential equations.\newline

    A differential equation is linear if it has the following form:
    \begin{align*}
      a_n(x)\frac{d^ny}{dx^n} + a_{n-1}(x)\frac{d^{n-1}y}{dx^{n-1}} + \cdots + a_{1}(x)\frac{dy}{dx} + a_0(x)y &= g(x).
    \end{align*}
    \begin{enumerate}[(1)]
      \item The power of each term involving $y$ and all its derivatives is one.
      \item All coefficients are exclusively functions of $x$.
    \end{enumerate}
    A differential equation that is not linear is called nonlinear.
  \end{definition}
  \begin{example}[Linear Differential Equations (or lack thereof)]\hfill
    \begin{enumerate}[(1)]
      \item 
        \begin{align*}
          x^3\frac{d^3y}{dx^3} - x^2\frac{d^2y}{dx^2} + 3x\frac{dy}{dx} + 5y &= e^x \tag*{Linear}
        \end{align*}
      \item 
        \begin{align*}
          yy'' - 2y' &= x\tag*{Nonlinear}
        \end{align*}
      \item 
        \begin{align*}
          \frac{d^3y}{dx^3} + y^2 &= 0 \tag*{Nonlinear}
        \end{align*}
    \end{enumerate}
  \end{example}
  \begin{definition}[Autonomous Differential Equations]
    An autonomous (first-order) differential equation is a differential equation in the following form:
    \begin{align*}
      \frac{dy}{dx} &= f(y).
    \end{align*}
  \end{definition}
  \begin{definition}[Solution to an ODE]
    Consider the general ODE
    \begin{align*}
      F\left(x,y,\frac{dy}{dx},\frac{d^2y}{dx^2},\dots,\frac{d^ny}{dx^n}\right) &= 0.\tag*{(\textasteriskcentered)}
    \end{align*}
    A solution of (\textasteriskcentered) is a function $y = f(x)$ that satisfies the ODE; that is,
    \begin{align*}
      F\left(x,f(x),f'(x),f''(x),\dots,f^{(n)}(x)\right) &= 0
    \end{align*}
    for every $x$ in the domain of $f(x)$.\newline

    Notice that $f$ is an element of a family of functions that satisfy the differential equation.
  \end{definition}
  \begin{example}[Verifying a Solution]
    We wish to show that $y = xe^{x}$ is a solution to
    \begin{align*}
      y'' - 2y' + y &= 0
    \end{align*}
    on $(-\infty,\infty)$.\newline

    In order to do this, we plug the proposed solution into the ODE:
    \begin{align*}
      y'' - 2y' + y &= \frac{d^2}{dx^2}\left(xe^{x}\right) - 2\frac{d}{dx}\left(xe^x\right) + xe^x\\
                    &= \left(xe^x + 2e^x\right) - 2\left(xe^x + e^x\right) + xe^x\\
                    &= 0.
    \end{align*}
  \end{example}
  \subsection{Modeling with Differential Equations}%
  \begin{definition}[Initial Value Problem]
  An initial value problem is a problem with a given ODE and an initial condition.
  \end{definition}
  \begin{example}[Initial Value Problems]\hfill
    \begin{enumerate}[(1)]
      \item We want to find
    \begin{align*}
      \frac{dy}{dx} &= f\left(x,y\right)
    \end{align*}
    such that $y\left(x_0\right) = y_0$.
    \item
      \begin{align*}
        \frac{d^2y}{dx^2} &= f\left(x,y,\frac{dy}{dx}\right)
      \end{align*}
      must satisfy $y\left(x_0\right) = y_0$, $y'\left(x_0\right) = y_1$.
    \end{enumerate}
  \end{example}
  Modeling primarily occurs via the following feedback loop:
  \begin{itemize}
    \item real-world problem;
    \item mathematical model;
    \item solution;
    \item result/prediction.
  \end{itemize}
  As predictions from the model begin to stray from real-world observations, we update the model to reflect these new observations.
  \begin{example}[Vertical Motion]
    Consider someone who throws a rock off a building.\newline

    We let $y(t)$ denote the height of the ball at time $t$, with $y_0$ denoting initial height. The acceleration due to gravity is equal to $a(t) = v'(t) = y''(t) = g$.\newline

    Our ODE is
    \begin{align*}
      y''(t) &= -g \tag*{for $0 \leq t \leq T$.}
    \end{align*}
    We require some initial conditions:
    \begin{itemize}
      \item $y(0) = y_0$ (initial position);
      \item $y'(0) = v_0$ (initial velocity).
    \end{itemize}
    Thus, we have created our second-order initial value problem.\newline

    To solve this second-order initial value problem analytically, we start with
    \begin{align*}
      y'' &= -g.
    \end{align*}
    Taking our first integral with respect to $t$, we have
    \begin{align*}
      y' &= -gt + c_1.
    \end{align*}
    Now, taking our second integral, 
    \begin{align*}
      y(t) &= -\frac{1}{2}gt^2 + c_1 t + c_2.
    \end{align*}
    This version of $y(t)$ is the general solution.\newline

    Applying our initial condition on $y'(t)$, we have $y'(0) = c_1$, meaning $c_1 = v_0$, and applying the initial condition to $y(t)$, we have $y(0) = c_2$, meaning $c_2 = y_0$.\newline

    Thus, the solution to this initial value problem is
    \begin{align*}
      y(t) &= -\frac{1}{2}gt^2 + v_0 t + y_0.
    \end{align*}
  \end{example}
  \begin{example}[Population Growth]
    Let $P(t)$ be the population of living fish in a lake at time $t$.\newline

    We know that the rate of growth in population is proportional to the population. In other words,
    \begin{align*}
      \frac{dP}{dt} &= kP(t)
    \end{align*}
    for some constant $k > 0$.\newline

    We can also include an initial condition, $P(0) = P_0$.\newline

    We can see (relatively easily) that
    \begin{align*}
      P(t) &= P_0e^{kt}.
    \end{align*}
  \end{example}
\end{document}
