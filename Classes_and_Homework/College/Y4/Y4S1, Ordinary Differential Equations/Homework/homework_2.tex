\documentclass[10pt]{mypackage}

% sans serif font:
%\usepackage{cmbright}
%\usepackage{sfmath}
%\usepackage{bbold} %better blackboard bold

%serif font + different blackboard bold for serif font
\usepackage{newpxtext,eulerpx}
\renewcommand*{\mathbb}[1]{\varmathbb{#1}}
\renewcommand*{\hbar}{\hslash}

\pagestyle{fancy} %better headers
\fancyhf{}
\rhead{Avinash Iyer}
\lhead{Ordinary Differential Equations: Homework 2}

\setcounter{secnumdepth}{0}

\begin{document}
\RaggedRight
\section{Part 1}%
\subsection{1.4, Problem 2}%
\begin{align*}
  \frac{dy}{dt} = 2y + 1\\
  y(0) &= 3\\
  0 \leq t \leq 2\\
  \Delta t = 0.5
\end{align*}
\begin{center}
  \begin{tabular}{c|c|c|c}
    $k$ & $t$ & $y$ & $f$\\
    \hline
    $0$ & $0$ & $3$ & $7$\\
    1 & $0.5$ & $6.5$ & $14$\\
    2 & $1$ & $13.5$ & $28$\\
    3 & $1.5$ & $27.5$ & $56$\\
    4 & $2$ & $55.5$ & ---
  \end{tabular}
\end{center}
\begin{center}
  \begin{tikzpicture}
    \begin{axis}[
        axis x line*=center,
        axis y line*=center,
        xmin = 0, xmax = 2,
        ymin = 0, ymax = 60,
        xtick={0,0.5,1,1.5,2},
        ytick = {0,10,20,30,40,50,60}
    ]
    \addplot[color=black,mark=circle*] coordinates {(0,3)(0.5,6.5)(1,13.5)(1.5,27.5)(2,55.5)};
    \end{axis}
  \end{tikzpicture}
\end{center}
Excel was used for calculation and TikZ/PGF was used to graph the coordinate outcomes from Euler's Method.
\subsection{1.4, Problem 6}%
\begin{align*}
  \frac{dw}{dt} &= (3-w)(w+1)\\
  w(0) &= 0\\
  0 \leq t \leq 2\\
  \Delta t = 0.5
\end{align*}
\begin{center}
  \begin{tabular}{c|c|c|c}
    $k$ & $t$ & $w$ & $f$\\
    \hline
    $1$ & $0$ & $3$ & $3$\\
    $2$ & $0.5$ & $1.5$ & $3.75$\\
    $3$ & $1$ & $3.375$ & $-1.641$\\
    $4$ & $1.5$ & $2.555$ & $1.583$\\
    $5$ & $2.0$ & $3.346$ & ---
  \end{tabular}
\end{center}
\begin{center}
  \begin{tikzpicture}
    \begin{axis}[
      axis x line*=center,
      axis y line*=center,
      xmin = 0, xmax = 2.2,
      ymin = 0, ymax = 3.7,
      xtick = {0,0.5,1,1.5,2},
      ytick = {0,0.5,1,1.5,2,2.5,3,3.5},
      ]
      \addplot[color=black,mark=circle*] coordinates{(0,0)(0.5,1.5)(1,3.375)(1.5,2.555)(2,3.346)};
    \end{axis}
  \end{tikzpicture}
\end{center}
\subsection{1.4, Problem 11}%
The equilibrium solutions to $\frac{dw}{dt} = (3-w)(w+1)$ occur for $w(t) = 3$; however, we had our initial condition at $w(0) = 0$ and yet the solution seemed to oscillate around the equilibrium point (rather than approaching it from below, as we would expect for a solution that started below the equilibrium value).
\subsection{1.4, Problem 15}%
%\begin{center}
%    \begin{tabular}{c|c|c|c}
%    \hline
%        $k$ & $x$ & $y$ & $f$  \\
%        \hline
%        1 & $(0,1)$  \\
%        2 & $(0.25,1.25)$  \\
%        3 & $(0.5,1.530)$  \\
%        4 & (0.75,1.839)$  \\
%        5 & (1,2.178)$  \\
%        6 & (1.25,2.547)$  \\
%        7 & (1.5,2.946),1.716)$  \\
%        8 & (1.75,3.375),1.837)$  \\
%        9 & (2,3.834),1.958)$  \\ 
%        10 & (2.25,4.323),2.079)$  \\
%    \end{tabular}
%\end{center}
\begin{itemize}
  \item $\Delta t = 1$:
    \begin{center}
  \begin{tikzpicture}
    \begin{axis}[
      axis x line*=center,
      axis y line*=center,
      xmin = 0, xmax=5.2,
      ymin=0,ymax=10.2,
      xtick={0,0.5,1,1.5,2,2.5,3,3.5,4,4.5,5},
      ytick={0,1,2,3,4,5,6,7,8,9,10},
      ]
  %    \addplot[color=black,mark=circle*] coordinates{(4,8.601)(3.75,7.899)(3.5,7.227)(2.75,5.393)(3,5.974)(2.5,4.843)(2.25,4.323)(2,3.834)(1.75,3.375)(1.5,2.946)(1.25,2.547)(1,2.178)(0.75,1.839)(0.5,1.530)(0.25,1.25)(0,1)};
  %    \addplot[color=blue,mark=circle*] coordinates{(0,1)(0.5,1.5)(1,2.112)(1.5,2.839)(2,3.682)(2.5,4.641)(3,5.718)(3.5,6.913)(4,8.228)};
      \addplot[color=blue,mark=circle*] coordinates{(0,1)(1,2)(2,3.414)(3,5.262)(4,7.556)};
    \end{axis}
  \end{tikzpicture}
    \end{center}
  \item $\Delta t = 0.5$:
    \begin{center}
  \begin{tikzpicture}
    \begin{axis}[
      axis x line*=center,
      axis y line*=center,
      xmin = 0, xmax=5.2,
      ymin=0,ymax=10.2,
      xtick={0,0.5,1,1.5,2,2.5,3,3.5,4,4.5,5},
      ytick={0,1,2,3,4,5,6,7,8,9,10},
      ]
  %    \addplot[color=black,mark=circle*] coordinates{(4,8.601)(3.75,7.899)(3.5,7.227)(2.75,5.393)(3,5.974)(2.5,4.843)(2.25,4.323)(2,3.834)(1.75,3.375)(1.5,2.946)(1.25,2.547)(1,2.178)(0.75,1.839)(0.5,1.530)(0.25,1.25)(0,1)};
      \addplot[color=blue,mark=circle*] coordinates{(0,1)(0.5,1.5)(1,2.112)(1.5,2.839)(2,3.682)(2.5,4.641)(3,5.718)(3.5,6.913)(4,8.228)};
  %    \addplot[color=blue,mark=circle*] coordinates{(0,1)(1,2)(2,3.414)(3,5.262)(4,7.556)};
    \end{axis}
  \end{tikzpicture}
    \end{center}
  \item $\Delta t = 0.25$:
    \begin{center}
  \begin{tikzpicture}
    \begin{axis}[
      axis x line*=center,
      axis y line*=center,
      xmin = 0, xmax=5.2,
      ymin=0,ymax=10.2,
      xtick={0,0.5,1,1.5,2,2.5,3,3.5,4,4.5,5},
      ytick={0,1,2,3,4,5,6,7,8,9,10},
      ]
      \addplot[color=black,mark=circle*] coordinates{(0,1)(0.25,1.25)(0.5,1.530)(0.75,1.839)(1,2.178)(1.25,2.547)(1.5,2.946)(1.75,3.375)(2,3.834)(2.25,4.323)(2.5,4.843)(2.75,5.393)(3,5.974)(3.25,6.585)(3.5,7.227)(3.75,7.899)(4,8.601)};
  %    \addplot[color=blue,mark=circle*] coordinates{(0,1)(0.5,1.5)(1,2.112)(1.5,2.839)(2,3.682)(2.5,4.641)(3,5.718)(3.5,6.913)(4,8.228)};
  %    \addplot[color=blue,mark=circle*] coordinates{(0,1)(1,2)(2,3.414)(3,5.262)(4,7.556)};
    \end{axis}
  \end{tikzpicture}
    \end{center}
\end{itemize}
The actual solution to the initial value problem should be some quadratic function.
%\begin{center}
%  \begin{tikzpicture}
%    \begin{axis}[
%      axis x line*=center,
%      axis y line*=center,
%      xmin = 0, xmax=5.2,
%      ymin=0,ymax=10.2,
%      xtick={0,0.5,1,1.5,2,2.5,3,3.5,4,4.5,5},
%      ytick={0,1,2,3,4,5,6,7,8,9,10},
%      ]
%      \addplot[color=black,mark=circle*] coordinates{(4,8.601)(3.75,7.899)(3.5,7.227)(2.75,5.393)(3,5.974)(2.5,4.843)(2.25,4.323)(2,3.834)(1.75,3.375)(1.5,2.946)(1.25,2.547)(1,2.178)(0.75,1.839)(0.5,1.530)(0.25,1.25)(0,1)};
%      \addplot[color=blue,mark=circle*] coordinates{(0,1)(0.5,1.5)(1,2.112)(1.5,2.839)(2,3.682)(2.5,4.641)(3,5.718)(3.5,6.913)(4,8.228)};
%      \addplot[color=blue,mark=circle*] coordinates{(0,1)(1,2)(2,3.414)(3,5.262)(4,7.556)};
%    \end{axis}
%  \end{tikzpicture}
%\end{center}
\section{Part 2}%
\subsection{1.5, Problem 2}%
Since $0 < y(0) < 2$ and $y_2(t) = 2$, $y_3(t) = 0$ are equilibrium solutions for $\frac{dy}{dt} = f(y)$, it is the case that $y(t)$ that solves the initial value problem with $y(0) = 1$ will be restricted between $0$ and $2$ for all $t$.
\subsection{1.5, Problem 3}%
For the initial condition $y(0) = 1$, we can see that it is, in a sense, trapped between $y_1(t)$ and $y_2(t)$, meaning that $y(t)$ that solves the initial value problem will be restricted between $y_1(t) = t + 2$ and $y_2(t) = -t^2$.
\subsection{1.5, Problem 12}%
\begin{enumerate}[(a)]
  \item 
    \begin{align*}
      \frac{dy_1}{dt} &= -\frac{1}{\left(t-1\right)^2}\\
                      &= -y_1^2\\
      \frac{dy_2}{dt} &= -\frac{1}{\left(t-2\right)^2}\\
                      &= -y_2^2
    \end{align*}
  \item If $-1 < y(0) < -1/2$, we know that $y$ will be of the form $\frac{1}{t+y(0)}$, as this satisfies the initial value problem, and $f(y)$, $\frac{\partial f}{\partial y}$ are continuous in a region about $\left(0,y(0)\right)$ (satisfying the uniqueness condition).
\end{enumerate}
\subsection{1.5, Problem 14}%
\begin{align*}
  \frac{dy}{dt} &= \frac{1}{\left(y+1\right)\left(t-2\right)}\\
  \int_{}^{} \left(y + 1\right)\:dy &= \int_{}^{} \frac{1}{t-2}\:dt\\
  \frac{1}{2}\left(y+1\right)^2 &= \ln \left\vert t-2 \right\vert + C\\
  y &= \sqrt{ 2\ln \left\vert t-2 \right\vert + K} - 1.
\end{align*}
Thus, we have
\begin{align*}
  0 &= \sqrt{2\ln \left\vert (0)-2 \right\vert + K} - 1\\
  2ln\left\vert -2 \right\vert + K &= 1\\
  K &= \frac{1}{2}-\ln 2.
\end{align*}
The solution is defined for all $t \neq 2$ and $y \neq -1$, implying that the solution's domain is $-2 < t < 2$.. The solution's slope blows up (to negative infinity) as it approaches the edge of its domain.
\subsection{1.5, Problem 15}%
\begin{align*}
  \frac{dy}{dt} &= \frac{1}{\left(y+2\right)^2}\\
  \int_{}^{} \left(y+2\right)^2\:dy &= \int_{}^{} \:dt\\
  \frac{1}{3}\left(y+2\right)^3 &= t + C\\
  y &= \sqrt[3]{3t + K} - 2.
\end{align*}
Evaluating the initial condition, we have
\begin{align*}
  1 &= \sqrt[3]{3(0) + K} - 2\\
  3 &= \sqrt[3]{K}\\
  K &= 27.
\end{align*}
In particular, since $y(0) > -2$ , the allowed values for $y$ are $-2 < y < 2$, meaning $-9 < t < 9$.
\end{document}

