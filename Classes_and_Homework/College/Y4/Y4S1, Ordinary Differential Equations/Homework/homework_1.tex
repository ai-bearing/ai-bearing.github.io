\documentclass[10pt]{mypackage}

% sans serif font:
%\usepackage{cmbright}
%\usepackage{sfmath}
%\usepackage{bbold} %better blackboard bold

%serif font + different blackboard bold for serif font
\usepackage{newpxtext,eulerpx}
\renewcommand*{\mathbb}[1]{\varmathbb{#1}}

\fancyhf{}
\rhead{Avinash Iyer}
\lhead{Ordinary Differential Equations: Homework 1}

\setcounter{secnumdepth}{0}

\begin{document}
\section{Part 1}%
\subsection{1.1, Problem 2}%
The equilibrium solution occurs when $\frac{dy}{dt} = 0$, meaning
\begin{align*}
  0 &= \frac{\left(t^2 - 1\right)\left(y^2 - 2\right)}{y^2 - 4}\\
  y^2 - 2 &= 0\\
  y(t) = \sqrt{2}\\
  y(t) &= -\sqrt{2}.
\end{align*}
\subsection{1.1, Problem 3}%
\begin{enumerate}[(a)]
  \item When $P = 0$ or $P = 230$, the population is in equilibrium.
  \item If $P$ is between $0$ and $230$, the population is increasing.
  \item If $P$ is greater than $230$ or less than $0$, the population is decreasing.
\end{enumerate}
\subsection{1.1, Problem 13}%
Learning occurs most rapidly when $L = 0$.
\subsection{1.1, Problem 14}%
\begin{enumerate}[(a)]
  \item The student who knows one half of the list at $t=0$ learns at a slower rate than the student who knows none of the list.
  \item The student who starts out knowing none of the list will never catch up to the student who starts out knowing one half of the list because solutions to initial value problems cannot intersect.
\end{enumerate}
\section{Part 2}%
\subsection{1.2, Problem 1}%
\begin{enumerate}[(a)]
  \item Paul and Bob are correct; taking $y(t) = t^2 - 2$, we see that $\frac{dy}{dt} = 2t = 2\frac{t^2 - 1}{t+1}$, and similarly, taking $y(t) = t$, we see $\frac{dy}{dt} = 1 = \frac{t+1}{t+1}$.
  \item The solution of $y(t) = t$ should be immediately obvious from separation of variables.
\end{enumerate}
\subsection{1.2, Problem 2}%
Substituting $y=e^{2t}$, we find that $t = \frac{1}{2}\ln y$, meaning we have $y(t) = e^{2t}$ is a solution to $\frac{dy}{dt} = 2y - t + \frac{1}{2}\ln y$.
\subsection{1.2, Problem 3}%
The derivative $\frac{dy}{dt}$ for $y(t) = e^{t^3}$ is $3t^2 e^{t^3}$. Substituting $y = e^{t^3}$, we find that $y(t) = e^{t^3}$ is a solution to the equation $\frac{dy}{dt} = 3t^2y$.
\subsection{1.2, Problem 27}%
Since $y(t) = 0$ is an equilibrium solution for $\frac{dy}{dt} = -y^2$, and $y(0) = 0$, we have that $y(t) = 0$ solves the initial value problem.
\subsection{1.2, Problem 32}%
\begin{align*}
  \frac{dy}{dt} &= ty^2 + 2y^2\\
  \frac{dy}{dt} &= y^2\left(t+2\right)\\
  \frac{1}{y^2}\:dy &= \frac{1}{t+2}\:dt\\
  \int_{}^{} \frac{1}{y^2}\:dy &= \int_{}^{} \frac{1}{t+2}\:dt\\
  -\frac{1}{y} &= \ln\left\vert t+2 \right\vert + C_1\\
  \frac{1}{y} &= C  - \ln \left\vert t+2 \right\vert\\
  y &= \frac{1}{C - \ln \left\vert t+2 \right\vert}.
\end{align*}
Including our initial value $y(0) = 1$, we find that
\begin{align*}
  C = 1 + \ln 2.
\end{align*}
Thus, the initial value problem has a solution of
\begin{align*}
  y(t) &= \frac{1}{\left(1 + \ln 2\right) - \ln \left\vert t+2 \right\vert}.
\end{align*}
\end{document}
