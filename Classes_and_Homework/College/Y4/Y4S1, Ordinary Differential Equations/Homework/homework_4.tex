\documentclass[10pt]{mypackage}

% sans serif font:
%\usepackage{cmbright}
%\usepackage{sfmath}
%\usepackage{bbold} %better blackboard bold

%serif font + different blackboard bold for serif font
\usepackage{newpxtext,eulerpx}
\renewcommand*{\mathbb}[1]{\varmathbb{#1}}
\renewcommand*{\hbar}{\hslash}

\pagestyle{fancy} %better headers
\fancyhf{}
\rhead{Avinash Iyer}
\lhead{Ordinary Differential Equations: Homework 4}

\setcounter{secnumdepth}{0}

\begin{document}
\RaggedRight
\section{Part 1}%
\subsection{1.7, Problem 10}%
For
\begin{align*}
  \diff{y}{t} &= e^{-y^2} + \alpha,
\end{align*}
there are zero equilibrium solutions for $\alpha \geq 0$ and $\alpha < -1$, while there are two equilibrium solutions for $\alpha \in \left(-1,0\right)$ and one equilibrium solution for $\alpha = -1$.
\subsection{1.7, Problem 13}%
\begin{enumerate}[(a)]
  \item This is a graph of (iii), as (iii) decomposes into $y(A-y^2)$, meaning $0$ is always an equilibrium solution, as well as the map $A = y^2$.
  \item This is a graph of (v), as $y^2 - A = 0$ when $A = y^2$, so it yields a source when $y > 0$ and a sink when $y < 0$.
  \item This is a graph of (iv) as it is the opposite sign of (v).
  \item This is a graph of (iv), as (iv) decomposes into $\left(A-y\right)y$, meaning $0$ is always an equilibrium solution, as well as some linear factor.
\end{enumerate}
\subsection{Chapter 1 Review, Problem 3}%
There are no equilibrium solutions for $\diff{y}{t} = t^2(t^2 + 1)$
\subsection{Chapter 1 Review, Problem 4}%
One of the solutions to $\diff{y}{t} = -\left\vert \sin^5 y \right\vert$ is the equilibrium solution $y = 0$.
\subsection{Chapter 1 Review, Problem 10}%
The bifurcation occurs at $a = -4$, where there is one equilibrium solution, with zero equilibrium solutions on either side of $a = -4$.
\subsection{Chapter 1 Review, Problem 11}%
This is true. We can see that $\diff{y}{t} = e^{-t} = \left|-e^{-t}\right|$.
\subsection{Chapter 1 Review, Problem 12}%
This is false. For example, the differential equation
\begin{align*}
  \diff{y}{t} &= \left(y+5\right)\left(t^2 + 2\right)
\end{align*}
is separable, but it is not autonomous.
\subsection{Chapter 1 Review, Problem 13}%
This is true. We can see that
\begin{align*}
  \diff{y}{t} &= a(y)
  \intertext{implies}
  \int_{}^{} \frac{1}{a(y)}\:dy &= \int_{}^{} \:dt,
\end{align*}
so $\diff{y}{t} = a(y)$ is separable.
\subsection{Chapter 1 Review, Problem 14}%
This is false. The differential equation
\begin{align*}
  \diff{y}{t} &= 3yt^2 + 2t
\end{align*}
is linear but is not separable.
\subsection{Chapter 1 Review, Problem 49}%

\subsection{Chapter 1 Review, Problem 52}%

\end{document}
