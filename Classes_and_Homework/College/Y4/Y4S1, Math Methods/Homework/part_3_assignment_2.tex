\documentclass[10pt]{mypackage}

% sans serif font:
%\usepackage{cmbright}
%\usepackage{sfmath}
%\usepackage{bbold} %better blackboard bold

%serif font + different blackboard bold for serif font
\usepackage{newpxtext,eulerpx}
\renewcommand*{\mathbb}[1]{\varmathbb{#1}}
\renewcommand*{\hbar}{\hslash}

\pagestyle{fancy} %better headers
\fancyhf{}
\rhead{Avinash Iyer}
\lhead{Physics 310: Assignment 8}

\setcounter{secnumdepth}{0}

\begin{document}
\RaggedRight
I am using $\overline{z}$ to denote the conjugate of a complex number and $T^{\ast}$ to denote the adjoint of an operator.
\section{Chapter 25 Problems}%
\subsection{Problem 1}%
\begin{enumerate}[(a)]
  \item 
    \begin{align*}
      \norm{\ket{1}}^2 &= \braket{1}{1}\\
                       &= \left(1\right)\overline{\left(1\right)} + \left(i\right)\overline{\left(i\right)}\\
                  &= 2\\
      \norm{\ket{2}}^2 &= \braket{2}{2}\\
                       &= \left(-i\right)\overline{\left(-i\right)} + \left(2i\right)\overline{\left(2i\right)}\\
                  &= 5\\
      \norm{\ket{3}}^2 &= \braket{3}{3}\\
                       &= \left(e^{i\phi}\right)\overline{\left(e^{i\phi}\right)} + \left(-1\right)\overline{\left(-1\right)}\\
                       &= 2\\
      \norm{\ket{4}}^2 &= \braket{4}{4}\\
                       &= \left(1\right)\overline{\left(1\right)} + \left(-2i\right)\overline{\left(-2i\right)} + \left(1\right)\overline{\left(1\right)}\\
                       &= 6\\
      \norm{\ket{5}}^2 &= \braket{5}{5}\\
                       &= \left(i\right)\overline{\left(i\right)} + \left(1\right)\overline{\left(1\right)} + \left(i\right)\overline{\left(i\right)}\\
                       &= 3.
    \end{align*}
  \item
    \begin{align*}
      \braket{2}{1} &= \left(1\right)\overline{\left(-i\right)} + \left(i\right)\overline{\left(2i\right)}\\
                    &= \overline{-i\overline{\left(1\right)} + 2i\overline{\left(i\right)}}\\
                    &= 2 + i\\
                    &= \overline{\braket{1}{2}}\\
      \braket{3}{1} &= \left(1\right)\overline{\left(e^{i\phi}\right)} + \left(i\right)\overline{\left(-1\right)}\\
                    &= \overline{e^{i\phi}\overline{\left(1\right)} + \left(-1\right)\overline{\left(i\right)}}\\
                    &= e^{-i\phi} - i\\
                    &= \overline{\braket{1}{3}}\\
      \braket{3}{2} &= \left(-i\right)\overline{\left(e^{i\phi}\right)} + \left(2i\right)\overline{\left(-1\right)}\\
                    &= \overline{\left(e^{i\phi}\right)\overline{\left(-i\right)} + \left(-1\right)\overline{\left(2i\right)}}\\
                    &= -ie^{-i\phi} -2i\\
                    &= \overline{\braket{2}{3}}.\\
      \braket{5}{4} &= \left(1\right)\overline{\left(i\right)} + \left(-2i\right)\overline{\left(1\right)} + \left(1\right)\overline{\left(i\right)}\\
                    &= \overline{i\overline{\left(1\right)} + \left(1\right)\overline{\left(-2i\right)} + \left(i\right)\overline{\left(1\right)}}\\
                    &= -4i\\
                    &= \overline{\braket{4}{5}}
    \end{align*}
\end{enumerate}
\subsection{Problem 4}%
\begin{enumerate}[(a)]
  \item 
    \begin{align*}
      \ket{u}^{\ast} &= \left({M}\ket{v}\right)^{\ast}\\
                     &= \left( \begin{pmatrix}1 & i \\ 2 & 1\end{pmatrix} \begin{pmatrix}1\\-i\end{pmatrix}\right)^{\ast}\\
                     &= \left( \begin{pmatrix}2\\2-i\end{pmatrix}\right)^{\ast}\\
                     &= \begin{pmatrix}2 & 2+i\end{pmatrix}\\
                     &= \begin{pmatrix}1 & i\end{pmatrix} \begin{pmatrix}1 & 2 \\ -i & 1\end{pmatrix}\\
                     &= \bra{u}.
    \end{align*}
  \item 
    \begin{align*}
      \braket{w}{v} &= \braket{w}{{M}v}\\
                    &= \braket{w}{u}\\
                    &= \begin{pmatrix}-1 & 1\end{pmatrix} \begin{pmatrix}2\\2-i\end{pmatrix}\\
                    &= -i\\
                    &= \overline{\braket{u}{w}}\\
                    &= \overline{ \begin{pmatrix}2 & 2+i\end{pmatrix} \begin{pmatrix}-1\\1\end{pmatrix} }\\
                    &= \overline{\left(i\right)}\\
                    &= -i.
    \end{align*}
\end{enumerate}
\subsection{Problem 5}%
\begin{align*}
  \braket{v}{{L}w} &= \bra{v}{L}\ket{w}\\
                           &= \braket{{L}^{\ast}v}{w}\\
                           &= \overline{\braket{w}{{L}^{\ast}v}}\\
                           &= \overline{\bra{w}{L}^{\ast}\ket{v}}.
\end{align*}
\subsection{Problem 6}%
\begin{enumerate}[(a)]
  \item 
      \begin{align*}
        \overline{\overline{\bra{v}T\ket{w}}} &= \overline{\bra{w}T^{\ast}\ket{v}}\\
                                              &= \bra{v}T^{\ast\ast}\ket{w}.
      \end{align*}
  \item 
    \begin{align*}
      \bra{v}\left(ST\right)^{\ast}\ket{w} &= \overline{\bra{w}S\left(T\ket{v}\right)}\\
                                           &= \overline{\bra{w}S\ket{u}}\tag*{$\ket{u} = T\ket{v}$}\\
                                           &= \bra{u}S^{\ast}\ket{w}\\
                                           &= \bra{Tv}S^{\ast}\ket{w}\\
                                           &= \bra{v}T^{\ast}S^{\ast}\ket{w}.\\
                                           \intertext{Alternatively,}
      \bra{v}\left(ST\right)^{\ast}\ket{w} &= \braket{\left(ST\right)v}{w}\\
                                           &= \bra{Tv}S^{\ast}\ket{w}\\
                                           &= \bra{v}T^{\ast}S^{\ast}\ket{w}.
    \end{align*}
\end{enumerate}
\subsection{Problem 8}%
\begin{enumerate}[(a)]
  \item It is clear that
    \begin{align*}
      \norm{\ket{1}} &= 1\\
      \norm{\ket{2}} &= 1,
    \end{align*}
    and
    \begin{align*}
      \braket{1}{2} &= \overline{ \begin{pmatrix}i & 0\end{pmatrix} } \begin{pmatrix}1\\0\end{pmatrix}\\
                    &= 0.
    \end{align*}
  \item It is clear that
    \begin{align*}
      \norm{\ket{+}} &= 1\\
      \norm{\ket{-}} &= 1,
    \end{align*}
    and
    \begin{align*}
      \braket{+}{-} &= \frac{1}{2} \overline{ \begin{pmatrix}1 & i\end{pmatrix} } \begin{pmatrix}1\\-i\end{pmatrix}\\
                    &=0.
    \end{align*}
  \item Similarly, it is clear that
    \begin{align*}
      \norm{\ket{\uparrow}} &= 1\\
      \norm{\ket{\downarrow}} &= 1.
    \end{align*}
    Taking inner products, we have
    \begin{align*}
      \braket{\uparrow}{\downarrow} &= \frac{1}{2} \overline{ \begin{pmatrix}1 & 1\end{pmatrix} } \begin{pmatrix}1\\-1\end{pmatrix}\\
                                    &= 0.
    \end{align*}
  \item 
    \begin{align*}
      \norm{\ket{I}}^2 &= \frac{1}{9}\left(\left\vert i-\sqrt{3} \right\vert^2 + \left\vert 1 + 2i \right\vert^2\right)\\
                       &= \frac{1}{9}\left(9\right)\\
                       &= 1\\
                       \\
      \norm{\ket{II}}^2 &= \frac{1}{9}\left(\left\vert 1 - 2i \right\vert^2 + \left\vert i + \sqrt{3} \right\vert^2\right)\\
                        &= \frac{1}{9}\left(9\right)\\
                        &= 1\\
                        \\
      \braket{I}{II} &= \frac{1}{9} \overline{ \begin{pmatrix}i - \sqrt{3} & 1 + 2i\end{pmatrix} } \begin{pmatrix}1-2i\\i + \sqrt{3}\end{pmatrix}\\
                     &= 0.
    \end{align*}
\end{enumerate}
\subsection{Problem 9}%
\begin{enumerate}[(a)]
  \item 
    \begin{align*}
      a_{1} &= \braket{1}{A}\\
            &= 1-2i\\
      a_{2} &= \braket{2}{A}\\
            &= 2 + 2i\\
      \norm{\ket{A}}^2 &= \left\vert a_1 \right\vert^2 + \left\vert a_2 \right\vert^2\\
                     &= 10\\
                     \\
      b_{1} &= \braket{1}{B}\\
            &= 1+i\\
      b_{2} &= \braket{2}{B}\\
            &= 2i\\
      \norm{\ket{B}}^2 &= \left\vert b_1 \right\vert^2 + \left\vert b_2 \right\vert^2\\
                       &= 6
    \end{align*}
  \item 
    \begin{align*}
      a_{+} &= \braket{+}{A}\\
            &= \frac{1}{\sqrt{2}} \overline{ \begin{pmatrix}1 & i\end{pmatrix} } \begin{pmatrix}1-i \\ 2 + 2i\end{pmatrix}\\
            &= \frac{1}{\sqrt{2}} \left(3 -3i\right)\\
      a_{-} &= \braket{-}{A}\\
            &= \frac{1}{\sqrt{2}} \overline{ \begin{pmatrix}1 & -i\end{pmatrix} } \begin{pmatrix}1-i\\2 + 2i\end{pmatrix}\\
            &= \frac{1}{\sqrt{2}} \left(-1 + i\right)\\
      \norm{\ket{A}}^2 &= \left\vert a_{+} \right\vert^2 + \left\vert a_{-} \right\vert^2\\
                       &= 10\\
                       \\
      b_{+} &= \braket{+}{B}\\
            &= \frac{1}{\sqrt{2}} \overline{ \begin{pmatrix}1 & i\end{pmatrix} } \begin{pmatrix}1+i\\2i\end{pmatrix}\\
            &= \frac{1}{\sqrt{2}} \left(3+i\right)\\
      b_{-} &= \braket{-}{B}\\
            &= \frac{1}{\sqrt{2}} \overline{ \begin{pmatrix}1 & -i\end{pmatrix} } \begin{pmatrix}1 + i \\ 2i\end{pmatrix}\\
            &= \frac{1}{\sqrt{2}} \left(2i\right)\\
      \norm{\ket{B}}^2 &= \left\vert b_{+} \right\vert^2 + \left\vert b_{-} \right\vert^2\\
                       &= 6
    \end{align*}
\end{enumerate}
\subsection{Problem 13}%
\begin{align*}
  \ket{\hat{A}} &= \frac{\ket{A}}{\sqrt{\braket{A}{A}}}\\
                &= \frac{1}{\sqrt{3}}\left(\ket{\hat{e}_1} + \ket{\hat{e}_2} + \ket{\hat{e}_{3}}\right).
\end{align*}
\subsection{Problem 17}%
\begin{align*}
  \ket{e_3} &= \ket{M_{3}} - \braket{\hat{e}_1}{M_{3}}\ket{\hat{e}_1} - \braket{\hat{e}_{2}}{M_3}\ket{\hat{e}_2}\\
            &= \begin{pmatrix}1 & 0 \\ 0 & i\end{pmatrix} - \frac{1}{2}\tr\left(\hat{e}_{1}^{\ast}M_{3}\right) \frac{1}{\sqrt{2}} \begin{pmatrix}1 & 1 \\ 1 & 1\end{pmatrix} - \frac{1}{2}\tr\left(\hat{e}_{2}^{\ast}M_{3}\right) \left(\frac{i}{\sqrt{2}}\right)\begin{pmatrix}-1 & 1 \\ 1 & -1\end{pmatrix}\\
            &= \begin{pmatrix}1 & 0 \\ 0 & i\end{pmatrix} - \frac{1+i}{4} \begin{pmatrix}1 & 1\\1 & 1\end{pmatrix} + \frac{1+i}{4} \begin{pmatrix}-1 & 1 \\ 1 & -1\end{pmatrix}\\
            &= \begin{pmatrix}\frac{1-i}{2} & 0 \\ 0 & -\frac{1-i}{2}\end{pmatrix}\\
  \ket{\hat{e}_3} &= \frac{\ket{e_3}}{\norm{\ket{e_3}}}\\
                  &= \frac{1-i}{\sqrt{2}} \begin{pmatrix}1 & 0 \\ 0 & -1\end{pmatrix}\\
\end{align*}
To find $\ket{e_4}$, we can see that $\norm{\ket{M_4}} = 1$ and $\braket{\hat{e}_1}{M_4} = \braket{\hat{e}_2}{M_4} = \braket{\hat{e}_3}{M_4} = 0$. Thus, $\ket{\hat{e}_4} = \ket{M_4}$.
\subsection{Problem 18}%
\begin{enumerate}[(a)]
  \item Note that
    \begin{align*}
      \braket{\phi_m}{\phi_m} &= k_m
    \end{align*}
    so
    \begin{align*}
      1 &= \frac{1}{k_m}\braket{\phi_m}{\phi_m}\\
          &= \braket{\frac{1}{\sqrt{k_m}}\phi_m}{\frac{1}{\sqrt{k_m}}\phi_m}\\
                  &= \braket{\hat{\phi}_m}{\hat{\phi}_{m}}.
    \end{align*}
    Thus,
    \begin{align*}
      \ket{v} &= \sum_{m}\braket{\hat{\phi}_m}{v}\ket{\hat{\phi}_m}\\
              &= \sum_{m}\braket{\frac{1}{\sqrt{k_m}}\phi_m}{v}\ket{\frac{1}{\sqrt{k_m}}\phi_m}\\
              &= \sum_{m}\frac{1}{k_m}\braket{\phi_m}{v}\ket{\phi_m}\\
              &= \sum_{m}c_m\ket{\phi_m}.
    \end{align*}
    Thus, $c_m = \frac{1}{k_m}\braket{\phi_m}{v}$.
  \item 
    \begin{align*}
      \id_{V} &= \sum_{m}\ket{\hat{\phi}_m}\bra{\hat{\phi}_m}\\
              &= \sum_{m}\ket{\frac{1}{\sqrt{k_m}}\phi_m}\bra{\frac{1}{\sqrt{k_m}}\phi_m}\\
              &= \sum_{m}\frac{1}{k_m}\ket{\phi_m}\bra{\phi_m}.
    \end{align*}
\end{enumerate}
\subsection{Problem 19}%
\begin{align*}
  M_{ij}^{\ast} &= \left(\bra{\hat{e}_i}\mathcal{M}\ket{\hat{e}_j}\right)^{\ast}\\
                &= \bra{\hat{e}_j}\overline{\mathcal{M}}\ket{\hat{e}_i}\\
                &= \overline{M_{ji}}.
\end{align*}
\subsection{Problem 26}%
We can see that $\ket{\phi_1}$ and $\ket{\phi_2}$ are linearly independent, and similarly are $\ket{\phi_3}$ and $\ket{\phi_4}$.  Since $\ket{\phi_2}$ and $\ket{\phi_3}$ are necessarily linearly independent, the collection of $\ket{\phi_i}$ are linearly independent.\newline

Therefore,
\begin{align*}
  \ket{\hat{e}_1} &= \frac{1}{2} \begin{pmatrix}1\\1\\1\\1\end{pmatrix}\\
  \\
  \ket{e_2} &= \begin{pmatrix}1\\-1\\1\\-1\end{pmatrix} - \frac{1}{4} \left[\overline{ \begin{pmatrix}1 & 1 & 1 & 1\end{pmatrix} } \begin{pmatrix}1\\-1\\1\\-1\end{pmatrix}\right] \begin{pmatrix}\end{pmatrix}\\
            &= \begin{pmatrix}1 \\ -1 \\ 1 \\ -1\end{pmatrix}\\
  \ket{\hat{e}_2} &= \frac{1}{2} \begin{pmatrix}1\\-1\\1\\-1\end{pmatrix}\\
  \\
  \ket{e_3} &= \begin{pmatrix}1\\-i\\1\\i\end{pmatrix} - \left[\frac{1}{4} \overline{ \begin{pmatrix}1 & 1 & 1 & 1\end{pmatrix} } \begin{pmatrix}1\\-i\\1\\i\end{pmatrix}\right] \begin{pmatrix}1\\1\\1\\1\end{pmatrix} - \left[\frac{1}{4} \overline{ \begin{pmatrix}1 & -1 & 1 & -1\end{pmatrix} } \begin{pmatrix}1\\-i\\1\\i\end{pmatrix}\right] \begin{pmatrix}1\\-1\\1\\-1\end{pmatrix}
\end{align*}

\subsection{Problem 29}%
\subsection{Problem 30}%

\end{document}
