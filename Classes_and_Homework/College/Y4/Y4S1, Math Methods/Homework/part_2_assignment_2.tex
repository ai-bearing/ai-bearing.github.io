\documentclass[10pt]{mypackage}

% sans serif font:
%\usepackage{cmbright}
%\usepackage{sfmath}
%\usepackage{bbold} %better blackboard bold

%serif font + different blackboard bold for serif font
\usepackage{newpxtext,eulerpx}
\renewcommand*{\mathbb}[1]{\varmathbb{#1}}
\renewcommand*{\hbar}{\hslash}

\pagestyle{fancy} %better headers
\fancyhf{}
\rhead{Avinash Iyer}
\lhead{Physics 310: Assignment 5}

\setcounter{secnumdepth}{0}

\begin{document}
\RaggedRight
\section{Problem 1}%
\begin{enumerate}[(a)]
  \item 
    \begin{align*}
      \int_{C_1}^{} \left(x^2 + y^2\right)\:d\ell &= \int_{0}^{1} x^2\:dx + \int_{0}^{1} y^2 + 1\:dy\\
                                               &= \frac{5}{3}.
    \end{align*}
  \item 
    \begin{align*}
      \int_{C_2}^{} \left(x^2 + y^2\right)\:d\ell &= \int_{0}^{1} 2x^2\:dx\\
                                                  &= \frac{2}{3}.
    \end{align*}
  \item 
    \begin{align*}
      \int_{C_3}^{} \left(x^2 + y^2\right)\:d\ell &= \int_{0}^{1} x^2 + x^4\:dx\\
                                                  &= \frac{8}{15}.
    \end{align*}
\end{enumerate}
\section{Problem 2}%
\begin{enumerate}[(a)]
  \item Since $\oint_{C}\:d\ell$ ``adds up'' the infinitesimal lengths along $C$, this integral gives the length of $C$.
  \item Since $\oint_{C}d\vec{\ell}$ is a vector-valued integral along $C$, and since $C$ is closed, this integral gives $0$.
\end{enumerate}
\section{Problem 3}%
\begin{enumerate}[(a)]
  \item We have $d\ell = dx\sqrt{1 + \left(\frac{dy}{dx}\right)^2}$, and $\diff{y}{x} = \frac{-x}{\sqrt{a^2 - x^2}}$, so
    \begin{align*}
      \int_{C}^{} \:d\ell &= \int_{}^{} \sqrt{1 + \frac{x^2}{a^2 - x^2}}\:dx\\
                          &= \int_{}^{} \frac{1}{\sqrt{a^2 - x^2}}\:dx\\
                          &= a\arcsin\left(\frac{x}{a}\right).
    \end{align*}
    Evaluated from $x = -a$ to $x = a$, we get that $\int_{C}^{} \:d\ell = \pi a$.
  \item We have $d\ell = \sqrt{dr^2 + r^2d\theta^2}$, so
    \begin{align*}
      \int_{}^{} \:d\ell &= \int_{0}^{\pi} a\:d\theta\\
                         &= \pi a
    \end{align*}
\end{enumerate}
\section{Problem 7}%
\begin{enumerate}[(a)]
  \item $\oint_{S}dA$ gives the area of the sphere, as we do not have to integrate with respect to a direction.
  \item $\oint_{S}d\mathbf{A}$ yields zero, as $\hat{n}$ is symmetrical with respect to $S$.
\end{enumerate}
\section{Problem 11}%
\begin{align*}
  \int_{S}^{} \mathbf{r}\cdot d\mathbf{A} &= \int_{}^{} \left(R\hat{r}\right)\cdot \hat{r}\left( R^2d\Omega\right)\\
                                          &= R^3\int_{0}^{2\pi}\int_{0}^{\pi/2}\sin^2\theta \:d\phi d\theta\\
                                          &= 2\pi R^3\frac{\pi}{4}\\
                                          &= \frac{\pi^2}{2}R^3.
\end{align*}
\section{Problem 18}%
\section{Problem 19}%
\section{Problem 20}%
\section{Problem 21}%
\section{Problem 22}%
\section{Problem 26}%
\section{Problem 28}%

\end{document}
