\documentclass[10pt]{mypackage}

% sans serif font:
%\usepackage{cmbright}
%\usepackage{sfmath}
%\usepackage{bbold} %better blackboard bold

%serif font + different blackboard bold for serif font
\usepackage{newpxtext,eulerpx}
\renewcommand*{\mathbb}[1]{\varmathbb{#1}}

\pagestyle{fancy}
\fancyhf{}
\rhead{Avinash Iyer}
\lhead{Physics 310, Assignment 2}

\setcounter{secnumdepth}{0}

\begin{document}
\RaggedRight

\section{Chapter 4 Problems}%
\subsection{4.7}%
\subsubsection{Cylindrical Coordinates}%
In cylindrical coordinates, we have
\begin{align*}
  d\mathbf{r} &= \rho \cos \phi \hat{i} + \rho \sin\phi \hat{j} + z\hat{k}.
\end{align*}
We let $\hat{e}_1 = \hat{\rho}$, $\hat{e}_2 = \hat{\phi}$, and $\hat{e}_3 = \hat{z}$, with $u_1 = \rho$, $u_2 = \phi$, and $u_3 = z$. Thus, we get
\begin{itemize}
  \item Line element:
    \begin{align*}
      \left(ds\right)^2 &= \sum_{i,j} \frac{\partial \mathbf{r}}{\partial u_i} \frac{\partial \mathbf{r}}{\partial u_j}\left(\hat{e}_i \cdot \hat{e}_j\right) du_idu_j\\
                        &= \sum_{i=1}\left(\frac{\partial \mathbf{r}}{\partial u_i}\right) \left(du_i\right)^2\tag*{The $\hat{\rho},\hat{\phi},\hat z$ basis is orthogonal}\\
                        &= \left(d\rho\right)^2 + \rho^2 \left(d\phi\right)^2 + \left(dz\right)^2.
    \end{align*}
  \item Area element:
    \begin{align*}
      d\mathbf{a} &= \left(\sum_{k}\epsilon_{ijk}\hat{e}_k\right)\frac{\partial \mathbf{r}}{\partial u_i}\cdot\frac{\partial \mathbf{r}}{\partial u_j} du_i du_j
    \end{align*}
\end{itemize}
\subsection{4.9}%
Without loss of generality, we have
\begin{align*}
  \sum_{\ell}\epsilon_{mn\ell}\epsilon_{ij\ell} &= \epsilon_{mn1}\epsilon_{ij1},
\end{align*}
where $m,n,i,j = 2,3$. If we have $m=i,n=j$, then $\epsilon_{mn1}\epsilon_{ij1} = 1$; if $m=j,n=i$, then $\epsilon_{mn1}\epsilon_{ij1} = -1$; else, $\epsilon_{mn1}\epsilon_{ij1} = 0$.
\subsection{4.11}%
\begin{enumerate}[(a)]
  \item 
    \begin{align*}
      \mathbf{A}\times \mathbf{B} &= \sum_{i,j,k}\epsilon_{ijk}A_iB_j\hat{e}_k\\
                &= -\sum_{i,j,k}\epsilon_{jik}B_jA_i\hat{e}_k\\
                &= -\left(\mathbf{B}\times \mathbf{A}\right)
    \end{align*}
  \item
    \begin{align*}
      \mathbf{A}\cdot \left(\mathbf{A}\times \mathbf{B}\right) &= \sum_{i,j,k}\left(\epsilon_{ijk}A_iB_j\hat{e}_k\right)\cdot A_i\hat{e}_i\\
                                                               &= \sum_{i,j,k}\delta_{ik}\left(\epsilon_{ijk}A_i^2B_j\right)\\
                                                               &= 0.
    \end{align*}
  \item 
    \begin{align*}
      \mathbf{A}\cdot \left(\mathbf{B}\times \mathbf{C}\right) &= \sum_{i,j,\ell}\epsilon_{ij\ell}A_{\ell}B_iC_j\\
                                                               &= \sum_{i,j,\ell}\left(\epsilon_{\ell i j}A_{\ell}B_i\right)C_j\\
                                                               &= \mathbf{C}\cdot \left(\mathbf{A}\times \mathbf{B}\right)
    \end{align*}
    and
    \begin{align*}
      \mathbf{A}\cdot \left(\mathbf{B}\times \mathbf{C}\right) &= \sum_{i,j,\ell}\epsilon_{ij\ell}A_{\ell}B_iC_j\\
                                                               &= \sum_{i,j,\ell}\left(\epsilon_{j \ell i}C_{j}A_{i}\right)B_i\\
                                                               &= \mathbf{B}\cdot \left(\mathbf{C}\times \mathbf{A}\right).
    \end{align*}
  \item 
    \begin{align*}
      \mathbf{A}\times \left(\mathbf{B}\times \mathbf{C}\right) &= \sum_{i,j}\epsilon_{ijk}A_i\left(\sum_{\alpha,\beta}\epsilon_{\alpha \beta j}B_{\alpha}C_{\beta}\right)\\
                                                                &= \sum_{i,j,\alpha,\beta}\epsilon_{ijk}\epsilon_{\alpha\beta j}A_{i} B_{\alpha}C_{\beta}\\
                                                                &= -\left(\sum_{i,j,\alpha,\beta}\epsilon_{ikj}\epsilon_{\alpha \beta j}A_{i}B_{\alpha}C_{\beta}\right)\\
                                                                &= -\left(\sum_{i,j,\alpha,\beta}\left(\delta_{i\alpha}\delta_{k\beta} - \delta_{i\beta}\delta_{k\alpha}\right)A_{i}B_{\alpha}C_{\beta}\right)\\
                                                                &= \sum_{i,j,\alpha,\beta}\left(\delta_{k\alpha}\delta_{i\beta} - \delta_{i\alpha}\delta_{k\beta}\right)A_iB_{\alpha}C_{\beta}\\
                                                                &= \sum_{i,j,\alpha,\beta}\left(B_{\alpha}\delta_{k\alpha}\right)\left(A_{i}C_{\beta}\delta_{i\beta}\right) - \left(C_{\beta}\delta_{k\beta}\right)\left(A_{i}B_{\alpha}\delta_{i\alpha}\right)\\
                                                                &= \mathbf{B}\left(\mathbf{A}\cdot \mathbf{C}\right) - \mathbf{C}\left(\mathbf{A}\cdot \mathbf{B}\right).
    \end{align*}
  \item 
    \begin{align*}
      \left(\mathbf{A}\times \mathbf{B}\right)\cdot \left(\mathbf{C}\cdot \mathbf{D}\right) &= \sum_{\alpha,\beta}\left(\mathbf{A}\times \mathbf{B}\right)_{\alpha}\left(\mathbf{C}\times \mathbf{D}\right)_{\beta}\delta_{\alpha\beta}\\
                  &= \sum_{\substack{\alpha,\beta,\\i,j,\\m,n}}\epsilon_{ij\alpha}\epsilon_{mn\beta}A_iB_jC_mD_n\delta_{\alpha\beta}\\
                  &= \sum_{\substack{\alpha,\\i,j,\\m,n}}\epsilon_{ij\alpha}\epsilon_{mn\alpha}A_iB_jC_mD_n\\
                  &= \sum_{\substack{i,j,\\m,n}}A_iB_jC_mD_n\left(\delta_{mi}\delta_{nj} - \delta_{mj}\delta_{ni}\right)\\
                  &= \sum_{\substack{i,j,\\m,n}}\left(\left(A_iC_m\delta_{mi}\right)\left(B_jD_n\delta_{nj}\right)\right) - \left(\left(B_jC_m\delta_{mj}\right) - \left(A_iD_n\delta_{ni}\right)\right)\\
                  &= \left(\mathbf{A}\cdot \mathbf{C}\right)\left(\mathbf{B}\cdot \mathbf{D}\right) - \left(\mathbf{B}\cdot \mathbf{C}\right)\left(\mathbf{A}\cdot \mathbf{D}\right).
    \end{align*}
\end{enumerate}
\section{Chapter 5 Problems}%
\subsection{5.1}%
Let $f(x) = x^n$. We use linearity for the general case.
\begin{align*}
  \frac{df}{dx} &= \lim_{h\rightarrow 0}\frac{\left(x + h\right)^n - x^n}{h}\\
                &= \lim_{h\rightarrow 0}\frac{x^n + n\left(hx^{n-1}\right) + \cdots + nh^{n-1}x + h^{n} - x^n }{h}\\
                &= \lim_{h\rightarrow 0}\left(nx^{n-1} + \cdots + nh^{n-2}x + h^{n-1}\right)\\
                &= nx^{n-1}.
\end{align*}
\subsection{5.6}%
\begin{align*}
  \cos\left(N\phi\right) + i\sin\left(N\phi\right) &= \left(\cos \phi + i\sin\phi\right)^N\\
                                                   &= \sum_{k=0}^{N}{N\choose k}\left(\cos\phi\right)^{k}\left(\sin\phi^{N-k}\right)\left(e^{i\frac{\pi}{2}}\right)^{N-k}\\
                                                   &= \sum_{k=0}^{N}{N\choose k}\left(\cos\phi\right)^k\left(\sin\phi\right)^{N-k}\left(\cos\left(\left(N-k\right)\frac{\pi}{2}\right) + i\sin\left(\left(N-k\right)\frac{\pi}{2}\right)\right)\\
                                                   &= \sum_{k=0}^{N}{N\choose k}\cos\left(\left(N-k\right)\frac{\pi}{2}\right)\left(\cos\phi\right)^k\left(\sin\phi\right)^{N-k} \\
                                                   & +i\left(\sum_{k=0}^{N}{N\choose k}\sin\left(\left(N-k\right)\frac{\pi}{2}\right)\left(\cos\phi\right)^k\left(\sin\phi\right)^{N-k}\right).
\end{align*}
We get the final answer by equating real and imaginary parts.
\section{Chapter 6 Problems}%
\subsection{6.3}%
\begin{enumerate}[(a)]
  \item Looking at the ratio test first, we find
    \begin{itemize}
      \item Ratio test:
    \begin{align*}
      \lim_{n\rightarrow\infty}\left\vert \frac{a_{n+1}}{a_n} \right\vert &= \lim_{n\rightarrow\infty}\left\vert \sqrt{\frac{n}{n+1}} \right\vert\\
                                                                          &= 1,
    \end{align*}
    which is an inconclusive result.
      \item Comparison test:
        \begin{align*}
          \frac{1}{\sqrt{n}} &> \frac{1}{n} \tag*{$\forall n\geq 1$.}
        \end{align*}
        Since $\sum_{n=1}^{\infty}\frac{1}{n}$ diverges, so too does $\sum_{n=1}^{\infty}\frac{1}{\sqrt{n}}$.
    \end{itemize}
  \item 
    \begin{itemize}
      \item Ratio test:
        \begin{align*}
          \lim_{n\rightarrow\infty}\left\vert \frac{a_{n+1}}{a_n} \right\vert &= \lim_{n\rightarrow\infty}\left\vert \left(\frac{n}{n+1}\right)\left(\frac{1}{2}\right) \right\vert\\
                                                                              &= \frac{1}{2}\\
                                                                              &< 1,
        \end{align*}
        meaning the series converges by the ratio test.
      \item 
        \begin{align*}
          \frac{1}{n2^n} < \frac{1}{2^n} \tag*{for all $n\geq 1$,}
        \end{align*}
        and since $\sum_{n=1}^{\infty}\frac{1}{2^n}$ converges, it must be the case that $\sum_{n=1}^{\infty}\frac{1}{n2^n}$ converges.
    \end{itemize}
\end{enumerate}
\subsection{6.9}%
\begin{align*}
  \sum_{n=-N}^{N}e^{inx} &= 1 + \sum_{n=1}^{N}e^{-inx} + \sum_{n=1}^{N}e^{inx}\\
                         &= 1 + e^{-ix}\sum_{n=1}^{N}e^{i\left(n-1\right)x} + e^{ix}\sum_{n=1}^{N}e^{i\left(n-1\right)x}\\
                         &= 1 + e^{-ix}\sum_{n=0}^{N-1}e^{-inx} + e^{ix}\sum_{n=0}^{N-1}e^{inx}\\
                         &= 1 + e^{-ix}\frac{1 - e^{-iNx}}{1-e^{-ix}} + e^{ix}\frac{1-e^{iNx}}{1-e^{ix}}\\
                         &= 1 + \frac{e^{-ix} - e^{-i\left(N+1\right)x}}{1-e^{-ix}} + \frac{1-e^{iNx}}{e^{-ix} - 1}\\
                         &= 1 + \frac{\left(e^{-ix} - 1\right) + e^{iNx} - e^{-i\left(N+1\right)x}}{1-e^{-ix}}\\
                         &= \frac{e^{iNx} - e^{-i\left(N+1\right)x}}{1-e^{-ix}}\\
                         &= \frac{e^{iNx} - e^{-i\left(N+1\right)x}}{e^{-i\left(\frac{x}{2}\right)}\left(e^{i\left(\frac{x}{2}\right)} - e^{-i\left(\frac{x}{2}\right)}\right)}\\
                         &= \frac{e^{i\left(N + \frac{1}{2}\right)x} - e^{-i\left(N + \frac{1}{2}\right)x}}{e^{-i\left(\frac{x}{2}\right)} - e^{-i\left(\frac{x}{2}\right)}}\\
                         &= \frac{\sin\left(\left(N + \frac{1}{2}\right)x\right)}{\sin \left(\frac{x}{2}\right)}.
\end{align*}
\subsection{6.13}%
\begin{enumerate}[(a)]
  \item 
    \begin{align*}
      \frac{d}{dz}\left(\arctan(z)\right) &= 1 - z^2 + z^4 - z^6 + \cdots\\
                                          &= \sum_{i=0}^{\infty}(-1)^{n}z^{2n}\\
                                          &= \frac{1}{1+z^2}.
    \end{align*}
  \item 
    \begin{align*}
      \rho &= \limsup_{k\rightarrow\infty}\sqrt[k]{\left|\left(-1\right)^k\right|}\\
           &= 1\\
      r &= \frac{1}{\rho}\\
        &= 1.
    \end{align*}
  \item 
    \begin{align*}
      \rho &= \limsup_{k\rightarrow\infty}\left(\left|\frac{(-1)^k}{2k+1}\right|\right)^{1/k}\\
           &= \limsup_{k\rightarrow\infty}\frac{1}{\left(2k+1\right)^{1/k}}\\
           &= 1\\
      r &= \frac{1}{\rho}\\
        &= 1.
    \end{align*}
\end{enumerate}
\subsection{6.25}%
\begin{align*}
  e^{i\theta} &= \sum_{k=0}^{\infty}\frac{\left(i\theta\right)^{k}}{k!}\\
              &= 1 + \left(i\theta\right) + \frac{\left(i\theta\right)^2}{2!} + \frac{\left(i\theta\right)^3}{3!} + \cdots\\
              &= \sum_{k=0}^{\infty}\frac{(-1)^{k}\theta^{2k}}{\left(2k\right)!} + i\sum_{k=0}^{\infty}\frac{\left(-1\right)^{k}\theta^{2k+1}}{\left(2k+1\right)!}\\
              &= \cos\theta + i\sin\theta.
\end{align*}
\subsection{6.37}%
\subsection{6.42}%
\end{document}
