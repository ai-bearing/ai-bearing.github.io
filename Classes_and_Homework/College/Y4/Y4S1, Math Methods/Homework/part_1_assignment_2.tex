\documentclass[10pt]{mypackage}

% sans serif font:
%\usepackage{cmbright}
%\usepackage{sfmath}
%\usepackage{bbold} %better blackboard bold

%serif font + different blackboard bold for serif font
\usepackage{newpxtext,eulerpx}
\renewcommand*{\mathbb}[1]{\varmathbb{#1}}

\fancyhf{}
\rhead{Avinash Iyer}
\lhead{Physics 310, Assignment 2}

\setcounter{secnumdepth}{0}

\begin{document}
\RaggedRight

\section{Chapter 4 Problems}%
\subsection{4.7}%
\subsubsection{Cylindrical Coordinates}%
In cylindrical coordinates, we have
\begin{align*}
  d\mathbf{r} &= \rho \cos \phi \hat{i} + \rho \sin\phi \hat{j} + z\hat{k}.
\end{align*}
We let $\hat{e}_1 = \hat{\rho}$, $\hat{e}_2 = \hat{\phi}$, and $\hat{e}_3 = \hat{z}$, with $u_1 = \rho$, $u_2 = \phi$, and $u_3 = z$. Thus, we get
\begin{itemize}
  \item Line element:
    \begin{align*}
      \left(ds\right)^2 &= \sum_{i,j} \frac{\partial \mathbf{r}}{\partial u_i} \frac{\partial \mathbf{r}}{\partial u_j}\left(\hat{e}_i \cdot \hat{e}_j\right) du_idu_j\\
                        &= \sum_{i=1}\left(\frac{\partial \mathbf{r}}{\partial u_i}\right) \left(du_i\right)^2\tag*{The $\hat{\rho},\hat{\phi},\hat z$ basis is orthogonal}\\
                        &= \left(d\rho\right)^2 + \rho^2 \left(d\phi\right)^2 + \left(dz\right)^2.
    \end{align*}
  \item Area element:
    \begin{align*}
      d\mathbf{a} &= \left(\sum_{k}\epsilon_{ijk}\hat{e}_k\right)\frac{\partial \mathbf{r}}{\partial u_i}\cdot\frac{\partial \mathbf{r}}{\partial u_j} du_i du_j
    \end{align*}
\end{itemize}
\section{Chapter 6 Problems}%
\subsection{6.3}%
\begin{enumerate}[(a)]
  \item Looking at the ratio test first, we find
    \begin{itemize}
      \item Ratio test:
    \begin{align*}
      \lim_{n\rightarrow\infty}\left\vert \frac{a_{n+1}}{a_n} \right\vert &= \lim_{n\rightarrow\infty}\left\vert \sqrt{\frac{n}{n+1}} \right\vert\\
                                                                          &= 1,
    \end{align*}
    which is an inconclusive result.
      \item Comparison test:
        \begin{align*}
          \frac{1}{\sqrt{n}} &> \frac{1}{n} \tag*{$\forall n\geq 1$.}
        \end{align*}
        Since $\sum_{n=1}^{\infty}\frac{1}{n}$ diverges, so too does $\sum_{n=1}^{\infty}\frac{1}{\sqrt{n}}$.
    \end{itemize}
  \item 
    \begin{itemize}
      \item Ratio test:
        \begin{align*}
          \lim_{n\rightarrow\infty}\left\vert \frac{a_{n+1}}{a_n} \right\vert &= \lim_{n\rightarrow\infty}\left\vert \left(\frac{n}{n+1}\right)\left(\frac{1}{2}\right) \right\vert\\
                                                                              &= \frac{1}{2}\\
                                                                              &< 1,
        \end{align*}
        meaning the series converges by the ratio test.
      \item 
        \begin{align*}
          \frac{1}{n2^n} < \frac{1}{2^n} \tag*{for all $n\geq 1$,}
        \end{align*}
        and since $\sum_{n=1}^{\infty}\frac{1}{2^n}$ converges, it must be the case that $\sum_{n=1}^{\infty}\frac{1}{n2^n}$ converges.
    \end{itemize}
\end{enumerate}
\end{document}
