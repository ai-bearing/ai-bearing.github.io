\documentclass[10pt]{mypackage}

% sans serif font:
%\usepackage{cmbright}
%\usepackage{sfmath}
%\usepackage{bbold} %better blackboard bold

%serif font + different blackboard bold for serif font
\usepackage{newpxtext,eulerpx}
\renewcommand*{\mathbb}[1]{\varmathbb{#1}}
\renewcommand*{\hbar}{\hslash}

\pagestyle{fancy} %better headers
\fancyhf{}
\rhead{Avinash Iyer}
\lhead{Physics 310: Assignment 6}

\setcounter{secnumdepth}{0}

\begin{document}
\RaggedRight
\section{Chapter 15 Problems}%
\subsection{Problem 1}%
Let $S$ denote the surface of the hemisphere with $z\geq 0$, $S_1$ denote the full hemisphere including the disk in the plane at $z=0$, and $S_2$ is the disk in the plane at $z=0$. Then,
\begin{align*}
  \int_{S}^{} \mathbf{r}\cdot d\mathbf{a} &= \oint_{S_1}\mathbf{r}\cdot d\mathbf{a} - \int_{S_2}\mathbf{r}\cdot d\mathbf{a}\\
                                          &= \oint_{S_1}r\hat{r}\cdot d\mathbf{a}\\
                                          &= \int_{V_1}d\tau\\
                                          &= \frac{2}{3}\pi R^3.
\end{align*}
\subsection{Problem 2 (a)}%
We let $S$ be the square in the $xy$-plane. Then,
\begin{align*}
  \oint_{C}\left(x\hat{i} - y\hat{j}\right)\cdot d\vec{\ell} &= \int_{S}\left(\nabla\times \left(x\hat{i} - y\hat{j}\right)\right)\cdot d\mathbf{a}\\
                                                             &= 0.
\end{align*}
\subsection{Problem 3 (a)}%
Let $V$ denote the cube of side length $a$. Then,
\begin{align*}
  \int_{S} \mathbf{F}\cdot d\mathbf{a} &= \oint_{V}\left(3x^2 + 3y^2 + 3z^2\right)d\tau\\
                                       &= 6a^5.
\end{align*}
\subsection{Problem 4}%
\begin{enumerate}[(a)]
  \item 
    \begin{align*}
      \oint_{S}\mathbf{F}\cdot d\mathbf{a} &= \int_{0}^{\pi}\int_{0}^{2\pi}\left(R\sin\theta\hat{r}\right)\cdot \hat{r} \left(R^2\sin\theta\right)d\phi d\theta\\
                                           &= 2\pi R^3\int_{0}^{\pi}\sin^2\theta\:d\theta\\
                                           &= 2\pi^2R^3.\\
      \oint_{S}\mathbf{F}\cdot d\mathbf{a} &= \int_{V}\sin\theta d\tau\\
                                           &= \int_{0}^{\pi}\int_{0}^{2\pi}\int_{0}^{R} r^2\sin^2\theta\:dr\:d\phi\:d\theta\\
                                           &= 2\pi^2R^3.
    \end{align*}
  \item 
    \begin{align*}
      \oint_{S}\mathbf{F}\cdot d\mathbf{a} &= \int_{0}^{\pi}\int_{0}^{2\pi}\left(R\sin\theta \hat{\theta}\right)\cdot \hat{r}\left(R^2\sin\theta\right)d\phi d\theta\\
                                           &= 0\\
      \oint_{S}\mathbf{F}\cdot d\mathbf{a} &= \int_{V}0\:d\tau\\
                                           &= 0.
    \end{align*}
  \item 
    \begin{align*}
      \oint_{S}\mathbf{F}\cdot d\mathbf{a} &= \int_{0}^{\pi}\int_{0}^{2\pi}\left(R\sin\theta \hat{\phi}\right)\cdot \hat{r}\left(R^2\sin\theta\right)d\phi d\theta\\
                                           &= 0\\
      \oint_{S}\mathbf{F}\cdot d\mathbf{a} &= \int_{V}0\:d\tau\\
                                           &= 0.
    \end{align*}
\end{enumerate}
\subsection{Problem 9}%
Let $C = \partial S_1 = \partial S_2$. Let $\mathbf{B} = \nabla \times \mathbf{A}$ (which must exist as $\mathbf{B}$ is solenoidal).
\begin{enumerate}[(a)]
  \item 
    \begin{align*}
      \int_{S_1}\mathbf{B}\cdot d\mathbf{a} &= \int_{S_1}\left(\nabla \times \mathbf{A}\right)\cdot d\mathbf{a}\\
                                            &= \oint_{C}\mathbf{A}\cdot d\vec{\ell}\\
                                            &= \int_{S_2}\left(\nabla \times \mathbf{A}\right)\cdot d\mathbf{a}\\
                                            &= \int_{S_2}\mathbf{B}\cdot d\mathbf{a}.
    \end{align*}
  \item For all closed surfaces $S$, it is the case that $\partial S_1 = \partial S_2 = 0$. Thus,
    \begin{align*}
      \oint_{S_1}\mathbf{B}\cdot d\mathbf{a} &= \int_{V_1}\nabla \cdot \mathbf{B}\:d\tau\\
                                             &= 0\\
                                             &= \int_{V_2}\nabla \cdot \mathbf{B}\:d\tau\\
                                             &= \oint_{S_2}\mathbf{B}\cdot d\mathbf{a}.
    \end{align*}
\end{enumerate}
\subsection{Problem 16}%
We have $\mathbf{E} = x^3\hat{i} + y^3\hat{j} + z^3\hat{k} = r^3\hat{r}$. Thus,
\begin{align*}
  \oint_{S}\mathbf{E}\cdot d\mathbf{a} &= \int_{V}\nabla \cdot \mathbf{E}\:d\tau\\
                                       &= \int_{0}^{R}\int_{0}^{2\pi}\int_{0}^{\pi}\left(3r^2\right)\left(r^2\sin\theta\right)\:d\theta d\phi dr\\
                                       &= \frac{3}{5}R^5\left(4\pi\right)\\
                                       &= \frac{12}{5}\pi R^5
\end{align*}
\subsection{Problem 17}%
We have $\mathbf{E} = \hat{i} + \hat{j} + z\left(x^2 + y^2\right)\hat{k}$, so $\nabla \cdot \mathbf{E} = x^2 + y^2 = r^2$. Thus,
\begin{align*}
  \oint_{S}\mathbf{E}\cdot d\mathbf{a} &= \int_{V}\nabla \cdot \mathbf{E}\:d\tau\\
                                       &= \int_{0}^{1}\int_{0}^{1}\int_{0}^{2\pi}\left(r^2\right)r\:d\phi dz dr\\
                                       &= \frac{\pi}{2}.
\end{align*}
\subsection{Problem 19 (a)}%
\begin{enumerate}[(a)]
  \item 
    \begin{align*}
      \oint_{S}f\nabla g\cdot d\mathbf{a} &= \int_{V}\nabla \cdot \left(f\nabla g\right)d\tau\\
                                          &= \int_{V}\left(\nabla f\cdot \nabla g + f\nabla^2 g\right)d\tau.
    \end{align*}
\end{enumerate}
\subsection{Problem 22}%
\begin{align*}
  \oint_{C}\mathbf{A}\cdot d\vec{\ell} &= \int_{S}\nabla \times \mathbf{A}\cdot d\mathbf{a}\\
                                       &= \int_{S} \left(\nabla \times \mathbf{A} + \nabla \times \nabla \lambda\right)\cdot d\mathbf{a}\\
                                       &= \int_{S}\nabla \times \left(\mathbf{A}+ \nabla \lambda\right)\cdot d\mathbf{a}\\
                                       &= \oint_{C}\left(\mathbf{A} + \nabla \lambda\right)\cdot d\vec{\ell}.
\end{align*}
\subsection{Problem 26}%
\begin{enumerate}[(a)]
  \item Note that
    \begin{align*}
      \nabla \left(\frac{1}{r}\right) &= -\frac{\hat{r}}{r^2}.
    \end{align*}
    We evaluate the surface integral of $-\frac{\hat{r}}{r^2}$ over a sphere of radius $R$ centered at the origin.
    \begin{align*}
      \oint_{S}-\frac{\hat{r}}{r^2}\cdot d\mathbf{a} &= -\oint_{S} \frac{1}{r^2}\left(r^2\sin\theta\right)d\Omega\\
                                                     &= -4\pi.
    \end{align*}
    We also evaluate the surface integral of $-\frac{\hat{r}}{r^2}$ over a surface $T$ that does not contain the origin.
    \begin{align*}
      \oint_{T}-\frac{\hat{r}}{r^2}\cdot d\mathbf{a} &= 0.
    \end{align*}
    Thus, we must have $\nabla \cdot \left(\nabla\left(\frac{1}{r}\right)\right) = -4\pi \df\left(\mathbf{r}\right)$.
  \item For $V$ a sphere that is centered at the origin, we have
    \begin{align*}
      \Phi &= \int_{V}\nabla\cdot \left(-\frac{\hat{r}}{r^2}\right)d\tau\\
           &= -\int_{0}^{\pi}\int_{0}^{2\pi} \sin\theta\:d\phi d\theta\\
           &= -4\pi,
    \end{align*}
    while if $V$ does not contain the origin, the divergence is zero. Thus, we get $\nabla^2\left(\frac{1}{r}\right) = -4\pi\delta\left(\mathbf{r}\right)$.
\end{enumerate}
\subsection{Problem 32}%
\begin{enumerate}[(a)]
  \item 
    \begin{align*}
      \nabla \times \mathbf{B}_0 &= \frac{1}{r}\left(\pd{}{r}\left(\frac{1}{3}r^3\right)\right)\hat{k}\\
                                 &= r\hat{k}\\
      \left(\nabla \times \mathbf{B}_0\right)\cdot \mathbf{B}_0 &= 0.
    \end{align*}
  \item 
    \begin{align*}
      \nabla \times \left(\mathbf{B}_0 + \hat{k}\right) &= \nabla \times \mathbf{B}_0 + \nabla \times \hat{k}\\
                                                        &= \nabla \times \mathbf{B}_0\\
                                                        &= r\hat{k}\\
      \left(\nabla \times \mathbf{B}_1\right)\cdot \mathbf{B}_1 &= r.\\
      \\
      \nabla \times \left(\mathbf{B}_0 + z\hat{r} + r\hat{k}\right) &= \nabla \times \mathbf{B}_0 + \nabla \times \left(z\hat{r} + r\hat{k}\right)\\
                                                                    &= r\hat{k}\\
      \left(\nabla \times \mathbf{B}_2\right)\cdot \mathbf{B}_2 &= r^2\hat{k}.
    \end{align*}
    It seems like the addition of a divergence-free component affects the orthogonality of the curl to the original vector field.
  \item With $\Lambda = -\frac{1}{3}rz$, we have
    \begin{align*}
      \left(\nabla \times \mathbf{B}_3\right)\cdot \mathbf{B}_3 &= \left(r\hat{k} + \hat{\phi}\right) \cdot \left(\frac{1}{3}r^2\hat{\phi} + z\hat{r} + \nabla\Lambda\right)\\
                                                                &= \frac{1}{3}r^2 + \hat{\phi}\cdot \left(\nabla\Lambda\right) + r\hat{k}\left(-\frac{1}{3}r\hat{k}\right)\\
                                                                &= 0.
    \end{align*}
\end{enumerate}
\subsection{Problem 37 (a)}%
\begin{align*}
  \nabla \cdot \mathbf{E} &= \nabla \cdot \left(\frac{1}{4\pi\epsilon_0}\int_{V}\rho\left(\mathbf{x}\right) \frac{\mathbf{r} - \mathbf{x}}{\norm{\mathbf{r} - \mathbf{x}}^3}d^3x\right)\\
                          &= \frac{1}{4\pi\epsilon_0}\int_{V}\nabla \cdot \left(\rho\left(\mathbf{x}\right) \frac{\mathbf{r} - \mathbf{x}}{\norm{\mathbf{r} - \mathbf{x}}^3}\right)d^3 x\\
                          &= \frac{1}{4\pi\epsilon_0}\oint_{S} -\frac{\rho\left(\mathbf{x}\right)\mathbf{x}}{\norm{\mathbf{r} - \mathbf{x}}^3}x^2d\Omega\\
                          &= \frac{\rho}{\epsilon_0}.
\end{align*}
\end{document}
