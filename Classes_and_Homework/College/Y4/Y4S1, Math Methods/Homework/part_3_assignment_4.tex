\documentclass[10pt]{mypackage}

% sans serif font:
%\usepackage{cmbright}
%\usepackage{sfmath}
%\usepackage{bbold} %better blackboard bold

%serif font + different blackboard bold for serif font
\usepackage{newpxtext,eulerpx}
\renewcommand*{\mathbb}[1]{\varmathbb{#1}}
\renewcommand*{\hbar}{\hslash}

\pagestyle{fancy} %better headers
\fancyhf{}
\rhead{Avinash Iyer}
\lhead{Mathematical Methods of Physics: Assignment 10}

\setcounter{secnumdepth}{0}

\begin{document}
\RaggedRight
\section{Chapter 27 Problems}%
\subsection{Problem 11}%
\begin{enumerate}[(a)]
  \item 
    \begin{align*}
      \lambda \braket{v}{v} &= \bra{v}\lambda \ket{v} \tag{Moving $\lambda$ into braket.}\\
                            &= \bra{v}H\ket{v} \tag{Definition of $\lambda$.}\\
                            &= \overline{\bra{v}H^{\ast}\ket{v}} \tag{Definition of adjoint operator.}\\
                            &= \overline{\bra{v}H\ket{v}} \tag{Definition of Hermitian operator.}\\
                            &= \overline{\bra{v}\lambda\ket{v}} \tag{Definition of $\lambda$.}\\
                            &= \overline{\lambda}\braket{v}{v}\tag{Moving $\lambda$ out of braket.}
    \end{align*}
  \item It is the case that $\braket{Hv_1}{v_2} = \overline{\lambda_1}\braket{v_1}{v_2}$ for any operator --- since our operator is Hermitian, it must be the case that $\lambda_1 = \overline{\lambda_1}$, else it would be possible for there to be $\lambda_2 - \overline{\lambda_1} = 0$ with $\lambda_1,\lambda_2$ distinct in (27.52b).
\end{enumerate}
\subsection{Problem 22}%
\begin{align*}
  M &= \sum_{i}\lambda_i \ket{\hat{v}_i}\bra{\hat{v}_i}\\
    &= \left(2\right)\left(\frac{1}{6}\right) \begin{pmatrix}1\\2\\1\end{pmatrix} \begin{pmatrix}1 & 2 & 1\end{pmatrix}\\
    &+ \left(-1\right)\left(\frac{1}{2}\right) \begin{pmatrix}1\\0\\-1\end{pmatrix} \begin{pmatrix}1 & 0 & -1\end{pmatrix}\\
    &+ \left(1\right)\left(\frac{1}{3}\right) \begin{pmatrix}1 \\ -1 \\ 1\end{pmatrix} \begin{pmatrix}1 & -1 & 1\end{pmatrix}\\
    \\
    &= \begin{pmatrix}-1 & 1 & -1 \\ 1 & 1 & 1 \\ -1 & 1 & -1\end{pmatrix}.
\end{align*}
\subsection{Problem 26}%
\begin{enumerate}[(a)]
  \item Let $M$ be a normal matrix. Then, there exists a unitary operator $U$ such that
    \begin{align*}
      U\Lambda U^{\ast} &= M,
    \end{align*}
    where $\Lambda$ is the diagonal matrix of eigenvalues. Since $\Lambda$ and $M$ are in the same similarity class, they have the same trace, so
    \begin{align*}
      \tr\left(M\right) &= \tr\left(\Lambda\right)\\
                        &= \sum_{i}\lambda_i.
    \end{align*}
  \item Let $M$ be a normal matrix. Then, there exists a unitary operator $U$ such that
    \begin{align*}
      U\Lambda U^{\ast} &= M,
    \end{align*}
    where $\Lambda$ is the diagonal matrix of eigenvalues. Since $\Lambda$ and $M$ are in the same similarity class, they have the same determinant, so
    \begin{align*}
      \det\left(M\right) &= \det\left(\Lambda\right)\\
                         &= \prod_{i}\lambda_i.
    \end{align*}
\end{enumerate}
\subsection{Problem 27}%
I don't know what you can say about their eigenvalues.
\section{Chapter 28 Problems}%
\subsection{Problem 1}%
\begin{align*}
  M\ket{\ddot{Q}} &= -K\ket{Q}\\
  m\ddot{q}_1 &= -2kq_1 + kq_2\\
  m\ddot{q}_2 &= -2kq_2 + kq_1\\
  \\
  m\ddot{q}_1 &= k\left(-2q_1 + q_2\right)\\
  m\ddot{q}_2 &= k\left(-2q_2 + q_1\right)
\end{align*}
We have
\begin{align*}
  m\left(\ddot{q}_1 + \ddot{q}_2\right) &= -k\left(q_1 + q_2\right)\\
  m\left(\ddot{q}_1 - \ddot{q}_2\right) &= -3k\left(q_1 - q_2\right).
\end{align*}
Thus, we have
\begin{align*}
  \frac{d^2}{dt^2} \left(q_1 - q_2\right) &= -\frac{3k}{m}\left(q_1 - q_2\right)\\
  \frac{d^2}{dt^2} \left(q_1 + q_2\right) &= -\frac{k}{m}\left(q_1 + q_2\right),
\end{align*}
so
\begin{align*}
  q_1 + q_2 &= A_1\cos\left(\omega_1t + \delta_1\right)\\
  q_1 - q_2 &= A_2\cos\left(\omega_2 t + \delta_2\right)\\
  q_1 &= a_1\cos\left(\omega_1 t + \delta_1\right) + a_2\cos\left(\omega_2 t + \delta_2\right)\\
  q_2 &= a_1\cos\left(\omega_1 t + \delta_1\right) - a_2\cos\left(\omega_2 t + \delta_2\right).
\end{align*}

\subsection{Problem 2}%
We have a matrix of eigenvectors
\begin{align*}
  A &= \begin{pmatrix}1 & 1 \\ 1 & -1\end{pmatrix}\\
  A^{-1} &= \begin{pmatrix}1/2 & 1/2 \\ 1/2 & -1/2\end{pmatrix}.
\end{align*}
We find
\begin{align*}
  A^{-1}MA &= \begin{pmatrix}1/2 & 1/2 \\ 1/2 & -1/2\end{pmatrix} \begin{pmatrix}2k & k \\ k & 2k\end{pmatrix} \begin{pmatrix}1 & 1 \\ 1 & -1\end{pmatrix}\\
           &= \begin{pmatrix}1/2 & 1/2 \\ 1/2 & -1/2\end{pmatrix} \begin{pmatrix}3k & k \\ 3k & -k\end{pmatrix}\\
           &= \begin{pmatrix}3k  & 0\\ 0 & k\end{pmatrix}.
\end{align*}

\subsection{Problem 3}%
We let $\delta_1 = \delta_2 = 0$, and take
\begin{align*}
  q_1(0) &= a\\
         &= A_1 + A_2\\
  q_2(0) &= b\\
         &= A_1 - A_2.
\end{align*}
Therefore, we have $A_1 = \frac{a+b}{2}$ and $A_2 = \frac{a-b}{2}$. Inserting into their respective formula, we get
\begin{align*}
  q_1(t) &= \frac{a+b}{2}\cos\left(\omega_1t\right) + \frac{a-b}{2}\cos\left(\omega_2 t\right)\\
         &= \frac{a}{2}\left(\cos\left(\omega_1 t\right) + \cos\left(\omega_2 t\right)\right) + \frac{b}{2}\left(\cos\left(\omega_1 t\right) - \cos\left(\omega_2\right)t\right)\\
  q_2(t) &= \frac{a+b}{2}\cos\left(\omega_1t\right) - \frac{a-b}{2}\cos\left(\omega_2 t\right)\\
         &= \frac{a}{2}\left(\cos\left(\omega_1 t\right) - \cos\left(\omega_2 t\right)\right) + \frac{b}{2}\left(\cos\left(\omega_1 t\right) + \cos\left(\omega_2 t\right)\right).
\end{align*}

\subsection{Problem 6}%
We have the matrix
\begin{align*}
  A &= \begin{pmatrix}1 & 1 \\ -\sqrt{2} & \sqrt{2}\end{pmatrix}\\
  A^{-1}  &= \begin{pmatrix}1/2 & 1/2 \\ -1/\sqrt{2} & 1/\sqrt{2}\end{pmatrix}.
\end{align*}
Thus, we find
\begin{align*}
  \Lambda &= \begin{pmatrix}1/2 & 1/2 \\ -1/\sqrt{2} & 1/\sqrt{2}\end{pmatrix}\begin{pmatrix}2 & -1 \\ -2 & -2\end{pmatrix} \begin{pmatrix}1 & 1 \\ -\sqrt{2} & \sqrt{2}\end{pmatrix}\\
          &= \begin{pmatrix}2+\sqrt{2} & 0 \\ 0 & 2-\sqrt{2}\end{pmatrix},
\end{align*}
implying that our eigenmodes do indeed solve the generalized eigenvalue problem for this system.
\subsection{Problem 7}%
Using Newton's second law, we get
\begin{align*}
  m_1\ddot{q_1} &= -k_1q_1 + k_2\left(q_2 - q_1\right)\\
  m_2\ddot{q_2} &= -k_2q_2 - k_2\left(q_2 - q_1\right).
\end{align*}
Using our initial conditions, we get the equations
\begin{align*}
  M\ket{\ddot{Q}} &= -\begin{pmatrix}5k & -2k \\ k & 2k\end{pmatrix} \ket{Q},
\end{align*}
where
\begin{align*}
  \ket{Q} &= \begin{pmatrix}q_1\\q_2\end{pmatrix}.
\end{align*}
We then seek to solve the generalized eigenvalue equation
\begin{align*}
  K\ket{\Phi} &= \omega^2M\ket{\Phi}.
\end{align*}
We find eigenvalues of
\begin{align*}
  \omega_1^2 &= \frac{3k}{m}\\
  \omega_2^2 &= \frac{4k}{m},
\end{align*}
with respective eigenvectors of
\begin{align*}
  \ket{\Phi_1} &= \begin{pmatrix}1\\1\end{pmatrix}\\
  \ket{\Phi_2} &= \begin{pmatrix}2\\1\end{pmatrix}.
\end{align*}
\subsection{Problem 10}%
Calculating
\begin{align*}
  k &= m\omega^2,
\end{align*}
we get
\begin{align*}
  k &\approx 187\text{ N/m.}
\end{align*}
\subsection{Problem 15}%
The two normal modes are the mode where both masses are swinging in the same direction, with frequency $\frac{1}{2\pi}\sqrt{\frac{g}{l}}$, and where both masses are swinging in the opposite direction, with frequency $\frac{\sqrt{3}}{2\pi}\sqrt{\frac{g}{l}}$.
\end{document}
