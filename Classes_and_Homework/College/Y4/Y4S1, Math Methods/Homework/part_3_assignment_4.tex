\documentclass[10pt]{mypackage}

% sans serif font:
%\usepackage{cmbright}
%\usepackage{sfmath}
%\usepackage{bbold} %better blackboard bold

%serif font + different blackboard bold for serif font
\usepackage{newpxtext,eulerpx}
\renewcommand*{\mathbb}[1]{\varmathbb{#1}}
\renewcommand*{\hbar}{\hslash}

\pagestyle{fancy} %better headers
\fancyhf{}
\rhead{Avinash Iyer}
\lhead{Mathematical Methods of Physics: Assignment 10}

\setcounter{secnumdepth}{0}

\begin{document}
\RaggedRight
\section{Chapter 27 Problems}%
\subsection{Problem 11}%
\begin{enumerate}[(a)]
  \item 
    \begin{align*}
      \lambda \braket{v}{v} &= \bra{v}\lambda \ket{v} \tag{Moving $\lambda$ into braket.}\\
                            &= \bra{v}H\ket{v} \tag{Definition of $\lambda$.}\\
                            &= \overline{\bra{v}H^{\ast}\ket{v}} \tag{Definition of adjoint operator.}\\
                            &= \overline{\bra{v}H\ket{v}} \tag{Definition of Hermitian operator.}\\
                            &= \overline{\bra{v}\lambda\ket{v}} \tag{Definition of $\lambda$.}\\
                            &= \overline{\lambda}\braket{v}{v}\tag{Moving $\lambda$ out of braket.}
    \end{align*}
  \item It is the case that $\braket{Hv_1}{v_2} = \overline{\lambda_1}\braket{v_1}{v_2}$ for any operator --- since our operator is Hermitian, it must be the case that $\lambda_1 = \overline{\lambda_1}$, else it would be possible for there to be $\lambda_2 - \overline{\lambda_1} = 0$ with $\lambda_1,\lambda_2$ distinct in (27.52b).
\end{enumerate}
\subsection{Problem 22}%
\begin{align*}
  M &= \sum_{i}\lambda_i \ket{\hat{v}_i}\bra{\hat{v}_i}\\
    &= \left(2\right)\left(\frac{1}{6}\right) \begin{pmatrix}1\\2\\1\end{pmatrix} \begin{pmatrix}1 & 2 & 1\end{pmatrix}\\
    &+ \left(-1\right)\left(\frac{1}{2}\right) \begin{pmatrix}1\\0\\-1\end{pmatrix} \begin{pmatrix}1 & 0 & -1\end{pmatrix}\\
    &+ \left(1\right)\left(\frac{1}{3}\right) \begin{pmatrix}1 \\ -1 \\ 1\end{pmatrix} \begin{pmatrix}1 & -1 & 1\end{pmatrix}\\
    \\
    &= \begin{pmatrix}-1 & 1 & -1 \\ 1 & 1 & 1 \\ -1 & 1 & -1\end{pmatrix}.
\end{align*}
\subsection{Problem 26}%
\begin{enumerate}[(a)]
  \item Let $M$ be a normal matrix. Then, there exists a unitary operator $U$ such that
    \begin{align*}
      U\Lambda U^{\ast} &= M,
    \end{align*}
    where $\Lambda$ is the diagonal matrix of eigenvalues. Since $\Lambda$ and $M$ are in the same similarity class, they have the same trace, so
    \begin{align*}
      \tr\left(M\right) &= \tr\left(\Lambda\right)\\
                        &= \sum_{i}\lambda_i.
    \end{align*}
  \item Let $M$ be a normal matrix. Then, there exists a unitary operator $U$ such that
    \begin{align*}
      U\Lambda U^{\ast} &= M,
    \end{align*}
    where $\Lambda$ is the diagonal matrix of eigenvalues. Since $\Lambda$ and $M$ are in the same similarity class, they have the same determinant, so
    \begin{align*}
      \det\left(M\right) &= \det\left(\Lambda\right)\\
                         &= \prod_{i}\lambda_i.
    \end{align*}
\end{enumerate}
\subsection{Problem 27}%
I don't know what you can say about their eigenvalues.
\section{Chapter 28 Problems}%
\subsection{Problem 1}%
\begin{align*}
  M\ket{\ddot{Q}} &= -K\ket{Q}\\
  m\ddot{q}_1 &= -2kq_1 + kq_2\\
  m\ddot{q}_2 &= -2kq_2 + kq_1\\
  \\
  m\ddot{q}_1 &= k\left(-2q_1 + q_2\right)\\
  m\ddot{q}_2 &= k\left(-2q_2 + q_1\right)
\end{align*}
We have
\begin{align*}
  m\left(\ddot{q}_1 + \ddot{q}_2\right) &= -k\left(q_1 + q_2\right)\\
  m\left(\ddot{q}_1 - \ddot{q}_2\right) &= -3k\left(q_1 - q_2\right).
\end{align*}
Thus, we have
\begin{align*}
  \frac{d^2}{dt^2} \left(q_1 - q_2\right) &= -\frac{3k}{m}\left(q_1 - q_2\right)\\
  \frac{d^2}{dt^2} \left(q_1 + q_2\right) &= -\frac{k}{m}\left(q_1 + q_2\right),
\end{align*}
so
\begin{align*}
  q_1 + q_2 &= A_1\cos\left(\omega_1t + \delta_1\right)\\
  q_1 - q_2 &= A_2\cos\left(\omega_2 t + \delta_2\right)\\
  q_1 &= a_1\cos\left(\omega_1 t + \delta_1\right) + a_2\cos\left(\omega_2 t + \delta_2\right)\\
  q_2 &= a_1\cos\left(\omega_1 t + \delta_1\right) - a_2\cos\left(\omega_2 t + \delta_2\right).
\end{align*}

\subsection{Problem 2}%

\subsection{Problem 3}%
\subsection{Problem 6}%
\subsection{Problem 7}%
\subsection{Problem 10}%
\subsection{Problem 15}%
\end{document}
