\documentclass[12pt]{mypackage}

% sans serif font:
%\usepackage{cmbright}
%\usepackage{sfmath}
%\usepackage{bbold} %better blackboard bold

%serif font + different blackboard bold for serif font
\usepackage{newpxtext,eulerpx}
\renewcommand*{\mathbb}[1]{\varmathbb{#1}}
\renewcommand*{\hbar}{\hslash}

\pagestyle{fancy} %better headers
\fancyhf{}
\rhead{Avinash Iyer}
\lhead{Physics 310: Assignment 7}

\setcounter{secnumdepth}{0}

\begin{document}
\RaggedRight
\section{Chapter 23 Problems}%
\subsection{23.1}%
\begin{enumerate}
  \item Since
    \begin{align*}
      \hat{M} &= \begin{pmatrix}a & b \\ c & d\end{pmatrix},
    \end{align*}
    and all matrices are representations of linear transformations over a given basis, it is the case that $\hat{M}$ is a linear operator.
  \item $\hat{T}$ is not a linear operator.
  \item $\hat{S}$ is not a linear operator.
  \item $\hat{N}$ is not a linear operator.
  \item Since
    \begin{align*}
      \hat{C}\left(\alpha \mathbf{A} + \mathbf{B}\right) &= \alpha \nabla \times \mathbf{A} + \nabla \times \mathbf{B}\\
                                                         &= \alpha \hat{C}\left(\mathbf{A}\right) + \hat{C}\left(\mathbf{B}\right),
    \end{align*}
    $\hat{C}$ is a linear operator.
  \item Since
    \begin{align*}
      \hat{D}\left(\alpha f + g\right) &= \diff{\left(\alpha f + g\right)}{x}\\
                                       &= \alpha \diff{f}{x} + \diff{g}{x}\\
                                       &= \alpha\hat{D}\left(f\right) + \hat{D}\left(g\right),
    \end{align*}
    $\hat{D}$ is a linear operator.
  \item Since the derivative is linear, multiple applications of the derivative are still linear.
  \item Since the derivative and multiplication by $x^2$ are linear operators, their sum is linear.
  \item $\hat{D}^2 + \sin^2$ is not a linear operator.
  \item Since
    \begin{align*}
      \hat{K}\left(\alpha f + g\right) &= \int_{P_1}^{P_2} K\left(x,t\right)\left(\alpha f(t) + g(t)\right)\:dt\\
                                       &= \alpha \int_{P_1}^{P_2} K(x,t)f(t)\:dt + \int_{P_1}^{P_2} K\left(x,t\right)g(t)\:dt,
    \end{align*}
    it is the case that $\hat{K}$ is a linear operator.
\end{enumerate}
\subsection{23.2}%
Linear combinations of solutions to the Schrödinger equation are also solutions to the Schrödinger equation since the equation consists entirely of linear operators.
\subsection{23.3}%
Let $L$ be a linear operator, $v\in V$. Then,
\begin{align*}
  L\left(0\right) &= L\left(v-v\right)\\
                  &= L(v) - L(v)\\
                  &= 0.
\end{align*}
\section{Chapter 24 Problems}%
\subsection{24.1}%
\begin{enumerate}[(a)]
  \item The set of scalars is a vector space.
  \item The set of two-dimensional column vectors whose first element is smaller than the second element is a vector space.
  \item The set of $n$-dimensional column vectors consisting of integer-valued elements is a $\Z$-module, but since $\Z$ is not a field, it is not a vector space.
  \item The set of $n$-dimensional column vectors whose elements sum to zero is a vector space.
  \item The set of $n\times n$ antisymmetric matrices combining under addition is a vector space.
  \item The set of $n\times n$ matrices combining under multiplication does not form a vector space.
  \item The set of polynomials with real coefficients is an $\R$-vector space but is not a $\C$-vector space.
  \item The set of periodic functions with $f(0) = 1$ is not a vector space.
  \item The set of periodic functions with $f(0) = 0$ is a vector space.
  \item The set of functions with $cf(a) + df'(a) = 0$ is a vector space.
\end{enumerate}
\subsection{24.2}%
The product space $U = V\times W$ under the operation
\begin{align*}
  a\left(v_1,w_1\right) + b\left(v_2,w_2\right) = \left(av_1 + bv_2,aw_1 + bw_2\right)
\end{align*}
is a vector space.
\subsection{24.3}%
\begin{enumerate}[(a)]
  \item This is a linearly independent set.
  \item This is not a linearly independent set.
  \item This is a linearly independent set.
  \item This is a linearly independent set.
  \item This is a linearly independent set.
\end{enumerate}
\subsection{24.4}%
Consider $z = bi$. Then, $\overline{z} = -bi$, and $z + \overline{z} = 0$, so $z$ is not necessarily linearly independent of $\overline{z}$.
\subsection{24.5}%
Calculating the determinant, we have
\begin{align*}
  \begin{vmatrix}1 & 1 & 1 \\ 1 & -2 & 0 \\ 1 & 1 & -1\end{vmatrix} &= 6,
\end{align*}
so the vectors form a basis.
\subsection{24.6}%
Let $\set{v_i}_{i\in I}$ be a basis for a given $\C$-vector space $V$, and let $w\in V$ be expressed by
\begin{align*}
  w &= \sum_{i\in I}a_iv_i\\
    &= \sum_{i\in I}b_iv_i,
\end{align*}
where the sums are finite linear combinations. Then,
\begin{align*}
  0 &= \left(w-w\right)\\
    &= \sum_{i\in I}\left(a_i-b_i\right)v_i,
\end{align*}
and since the $\set{v_i}_{i\in I}$ form a basis, this can only be the case if $a_i = b_i$ for each $i$.
\subsection{24.7}%
\begin{enumerate}[(a)]
  \item 
    \begin{align*}
      v &= \begin{pmatrix}a_1\\a_2\\a_3\end{pmatrix}\\
      w &= \begin{pmatrix}b_1\\b_2\\b_3\end{pmatrix}.
    \end{align*}
  \item 
    \begin{align*}
      \begin{pmatrix}a_1\\a_2\\a_3\end{pmatrix} + \begin{pmatrix}b_1\\b_2\\b_3\end{pmatrix} &= \begin{pmatrix}a_1 + b_1\\a_2 + b_2\\a_3 + b_3\end{pmatrix}.
    \end{align*}
  \item 
    \begin{align*}
      \begin{pmatrix}b_1\\b_2\\b_3\end{pmatrix} &= k \begin{pmatrix}a_1\\a_2\\a_3\end{pmatrix}\\
                                                &= \begin{pmatrix}ka_1 \\ ka_2\\ka_3\end{pmatrix}.
    \end{align*}
  \item 
    \begin{align*}
      \begin{pmatrix}a_1 - a_1\\a_2 - a_2\\a_3 - a_3\end{pmatrix} &= \begin{pmatrix}a_1\\a_2\\a_3\end{pmatrix}  - \begin{pmatrix}a_1\\a_2\\a_3\end{pmatrix}\\
                                                                  &= \begin{pmatrix}0\\0\\0\end{pmatrix}.
    \end{align*}
\end{enumerate}
\subsection{24.8}%
We can let
\begin{align*}
  \begin{pmatrix}1 &0 & 0 \\ 0 & 1 & 0 \\ 0 & 0 & 1\end{pmatrix}
\end{align*}
be the matrix representation for $\set{ \ket{R}, \ket{G}, \ket{B}  }$. We have
\begin{align*}
  \ket{\text{white}} &= 255 \ket{R} + 255 \ket{G} + 255 \ket{B}\\
  \ket{\text{black}} &= 0 \ket{R} + 0 \ket{G} + 0 \ket{B}\\
  \ket{\text{orange}} &= 255 \ket{R} + 165 \ket{G} + 0 \ket{B}.
\end{align*}

\end{document}
