\documentclass[10pt]{mypackage}

% sans serif font:
%\usepackage{cmbright}
%\usepackage{sfmath}
%\usepackage{bbold} %better blackboard bold

%serif font + different blackboard bold for serif font
\usepackage{newpxtext,eulerpx}
\renewcommand*{\mathbb}[1]{\varmathbb{#1}}
\renewcommand*{\hbar}{\hslash}

\pagestyle{fancy} %better headers
\fancyhf{}
\rhead{Avinash Iyer}
\lhead{Mathematical Methods of Physics: Assignment 11}

\setcounter{secnumdepth}{0}

\begin{document}
\RaggedRight
\section{Chapter 31 Problems}%
\subsection{Problem 1}%
Since the inner product is positive definite, and
\begin{align*}
  k_{m} &= \braket{P_m}{P_m},
\end{align*}
we must have $k_m \geq 0$. Since $P_m\neq 0$, we must have $k_m \neq 0$.
\subsection{Problem 2}%
I don't know how to do this problem.
\subsection{Problem 3}%
\begin{align*}
  \braket{f+g}{f+g} &= \braket{f}{f} + \braket{g}{g} + 2\re\left(\braket{f}{g}\right)\\
                    &\leq \braket{f}{f} + \braket{g}{g} + 2\left\vert \braket{f}{g} \right\vert\\
                    &\leq \braket{f}{f} + \braket{g}{g} + 2\sqrt{\braket{f}{f}\braket{g}{g}}\\
                    &< \infty.
\end{align*}
\section{Chapter 32 Problems}%
\subsection{Problem 2}%
\begin{enumerate}[(a)]
  \item We have
    \begin{align*}
      \ket{P_0} &= 1\\
      \ket{P_1} &= x\\
      \ket{P_2} &= \frac{1}{2} \left(3x^2 - 1\right)\\
      \ket{P_3} &= \ket{\chi_3} - \braket{\hat{P}_0}{\chi_3}\ket{\hat{P}_0} - \braket{\hat{P}_1}{\chi_3}\ket{\hat{P}_0}- \braket{\hat{P}_2}{\chi_3}\ket{\hat{P}_2}\\
                &= \ket{\chi_3} - \braket{\hat{P}_1}{\chi_3}\ket{\hat{P}_1}\\
                &= x^3 - \left(\sqrt{\frac{3}{2}} \int_{-1}^{1} t^4\:dt\right)\left(\sqrt{\frac{3}{2}}x\right)\\
                &= \frac{1}{2}\left(5x^3 - 3x\right).
    \end{align*}
  \item We have
    \begin{align*}
      \ket{L_0} &= 1\\
      \ket{L_1} &= \ket{\chi_1} - \braket{L_0}{\chi_1}\ket{L_0}\\
                &= x - \left(\int_{0}^{\infty} xe^{-x}\:dx\right)\left(1\right)\\
                &= 1-x\\
      \ket{L_2} &= \ket{\chi_2} - \braket{L_1}{\chi_2}\ket{L_1} - \braket{L_0}{\chi_2}\ket{L_0}\\
                &= \frac{1}{2}\left(x^2 - 4x + 2\right)\\
      \ket{L_3} &= \ket{\chi_3} - \braket{L_2}{\chi_3}\ket{L_2} - \braket{L_1}{\chi_3}\ket{L_1} - \braket{L_0}{\chi_3}\ket{L_0}\\
                &= \frac{1}{6}\left(-x^3 + 9x^2 -18x + 6\right).
    \end{align*}
\end{enumerate}
\subsection{Problem 7}%
\begin{align*}
  \braket{P_0}{f} &= \frac{1}{2}\int_{-1}^{1} P_0(x)e^{ikx}\:dx\\
                  &= \frac{2\sin(k)}{k}\\
  \braket{P_1}{f} &= \frac{3}{2}\int_{-1}^{1} P_1(x)e^{ikx}\:dx\\
                  &= 2i\frac{-k\cos\left(k\right) + \sin\left(k\right)}{k^2}\\
  \braket{P_2}{f} &= \frac{5}{2}\int_{-1}^{1} \frac{1}{2}\left(3x^2 - 1\right)e^{-ikx}\:dx\\
                  &= 6\frac{\cos\left(k\right)}{k^2} + 2\frac{\left(-3 + k^2\right)\sin\left(k\right)}{k^3}.
\end{align*}
\subsection{Problem 8}%
\begin{align*}
  \braket{L_0}{f} &= \int_{0}^{\infty} e^{-x/2}\:dx\\
                  &= 2\\
  \braket{L_1}{f} &= \int_{0}^{\infty}\left(1-x\right)e^{-x/2}dx\\
                  &= -2\\
  \braket{L_2}{f} &= \int_{0}^{\infty} \frac{1}{2}\left(x^2 - 4x + 2\right)e^{-x/2}\:dx\\
                  &= 2.
\end{align*}
\subsection{Problem 16}%
\begin{align*}
  P_0\left(x\right) &= 1\\
  P_1\left(x\right) &= -\frac{1}{2}\diff{}{x}\left(1-x^2\right)\\
                    &= x\\
  P_2\left(x\right) &= \frac{1}{8}\diff{^2}{x^2}\left(x^4 - 2x^2 + 1\right)\\
                    &= \frac{1}{2}\left(3x^2-1\right).
\end{align*}

\end{document}
