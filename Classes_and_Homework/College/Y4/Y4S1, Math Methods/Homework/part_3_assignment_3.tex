\documentclass[10pt]{mypackage}

% sans serif font:
%\usepackage{cmbright}
%\usepackage{sfmath}
%\usepackage{bbold} %better blackboard bold

%serif font + different blackboard bold for serif font
\usepackage{newpxtext,eulerpx}
\renewcommand*{\mathbb}[1]{\varmathbb{#1}}
\renewcommand*{\hbar}{\hslash}

\pagestyle{fancy} %better headers
\fancyhf{}
\rhead{Avinash Iyer}
\lhead{Math 310: Assignment 9}

\setcounter{secnumdepth}{0}

\begin{document}
\RaggedRight
I will be using $A^{\ast}$ to denote the adjoint of an operator $A$ throughout this assignment.
\section{Chapter 26 Problems}%
\subsection{Problem 1}%
\begin{align*}
  v_i &= \braket{\hat{e}_i}{v}\\
      &= \braket{v}{\hat{e}_i}\\
      &= \bra{v}R^{T}\ket{\hat{e}_i'}\\
      &= \sum_{k}\braket{v}{\hat{e}_k'}\bra{\hat{e}_k'}R^{T}\ket{\hat{e}_i'}\\
      &= \sum_{k}v_k'\ket{\hat{e}_k'}.
\end{align*}
\subsection{Problem 2}%
\begin{align*}
  \ket{v'} &= v_1'\ket{\hat{e}_1} + v_2'\ket{\hat{e}_2}\\
           &= \left(v_1\cos\phi -v_2\sin\phi\right)\ket{\hat{e}_1} + \left(v_1\sin\phi + v_2\cos\phi\right)\ket{\hat{e}_2}\\
           &= v_1\left(\cos\phi\ket{\hat{e}_1} + \sin\phi\ket{\hat{e}_2}\right) + v_2\left(-\sin\phi\ket{\hat{e}_1} + \cos\phi\ket{\hat{e}_2}\right)\\
           &= v_1\ket{\hat{e}_1'} + v_2\ket{\hat{e}_2'}.
\end{align*}
This is a clockwise rotation of the unprimed basis.
\subsection{Problem 4}%
\begin{align*}
  \left(R_n\left(\varphi\right)\right)^m &= \left(e^{-i\phi\hat{n}\cdot \mathbf{L}}\right)^m\\
                                         &= e^{-im\phi \hat{n}\cdot \mathbf{L}}\\
                                         &= R_{n}\left(m\varphi\right).
\end{align*}
We then have
\begin{align*}
  R_n\left(3\varphi\right) &= \left(R_n\left(\varphi\right)\right)^{3}\\
                           &= \left(\cos\left(\varphi \hat{n}\cdot \mathbf{L}\right) + i\sin\left(\varphi\hat{n}\cdot \mathbf{L}\right)\right)^3\\
                           &= \left(\cos^3\left(\varphi \hat{n}\cdot \mathbf{L}\right) - 3\sin^2\left(\varphi \hat{n}\cdot \mathbf{L}\right)\cos\left(\varphi\hat{n}\cdot \mathbf{L}\right)\right) + i\left(\cos^2\left(\varphi\hat{n}\cdot \mathbf{L}\right)\sin\left(\varphi\hat{n}\cdot \mathbf{L}\right) - \sin^3\left(\varphi\hat{n}\cdot \mathbf{L}\right)\right)\\
                           &= \cos\left(3\varphi\hat{n}\cdot \mathbf{L}\right) + i\sin\left(3\varphi\hat{n}\cdot \mathbf{L}\right).
\end{align*}
\subsection{Problem 5}%
\begin{enumerate}[(a)]
  \item Since $A^{T} = A^{-1}$, this matrix is orthogonal (and unitary). However, it is not Hermitian.
  \item Since $\det(B) =0$, this matrix is not unitary, but it is Hermitian.
  \item Since neither $C^{\ast} = C$, $C^{\ast} = C^{-1}$, nor $C^{T} = C^{-1}$, it is the case that $C$ is neither orthogonal, unitary, nor Hermitian.
  \item Since
    \begin{align*}
      \tr\left(i \begin{pmatrix}0 & -i \\ i & 0\end{pmatrix}\right) = 0,
    \end{align*}
    $D$ is a unitary operator.
  \item Since $E^{\ast} = E$, $E$ is Hermitian. Additionally, $E^{\ast} = E^{-1}$, so $E$ is also unitary.
\end{enumerate}
\subsection{Problem 8}%
\begin{align*}
  UU^{\ast} &= I\\
  \det\left(UU^{\ast}\right) &= \det(I)\\
  \det\left(U\right)\det\left(U^{\ast}\right) &= 1\\
  \left(\det\left(U\right)\right)^2 &= 1\\
                                    &= \det\left(U\right)\det\left(U\right)^{-1}.
\end{align*}
meaning $\det(U)\in \mathds{T}$, so $\det(U)$ is pure phase.
\subsection{Problem 11}%
\begin{align*}
\norm{\mathbf{v}}^2 &= \sum_{i}\braket{v_i}{v_i}\\
                    &= \sum_{i}\braket{\sum_{k}U_{ik}v_k}{\sum_{\ell}U_{i\ell}v_{\ell}}\\
                    &= \sum_{i,k,\ell}\overline{U_{ik}}U_{i\ell}\braket{v_{k}}v_{\ell}\\
                    &= \sum_{k,\ell}\left(\sum_{i}U_{ki}^{\ast}U_{i\ell}\right)\braket{v_k}{v_{\ell}},
\end{align*}
meaning $\sum_{i}U_{ki}^{\ast}U_{i\ell} = \delta_{k\ell}$, so $U^{\ast}U = I$.
\subsection{Problem 16}%
\begin{align*}
  \braket{\delta \mathbf{r}}{\mathbf{r}} &= \braket{\delta\vec{\varphi}\times \mathbf{r}}{\mathbf{r}}\\
                                         &= \sum_{k}\braket{\sum_{i,j}\delta\epsilon_{ijk}\varphi_{i}r_j}{r_{k}}\\
                                         &= \sum_{i,j,k}\delta\epsilon_{ijk}\varphi_i\braket{r_j}{r_k}.
\end{align*}
I don't know where to go from here.
\subsection{Problem 18}%
\begin{enumerate}[(a)]
  \item Solving the eigenvector equation $A\hat{n} = \hat{n}$, we get
    \begin{align*}
      \hat{n} &= \frac{1}{\sqrt{2}} \begin{pmatrix}-1\\0\\1\end{pmatrix}.
    \end{align*}
    Thus, we have
    \begin{align*}
      \phi &= \arccos\left(\frac{1}{2}\left(0\right)\right)\\
           &= \pi/2.
    \end{align*}
  \item This matrix is a reflection about the line $y=x$.
  \item This matrix is a flip and $\pi/6$ rotation about the $y$ axis.
  \item This matrix is a flip and a $\pi/4$ rotation about the $x$ axis.
\end{enumerate}
\subsection{Problem 20}%
We have
\begin{align*}
  \phi &= \arccos\left(\frac{1}{2}\left(\frac{2}{3}\right)\right)\\
       &\approx 70.53^{\circ},
\end{align*}
and, solving
\begin{align*}
  R_q\hat{q} &= \hat{q}
\end{align*}
with Mathematica, we also get
\begin{align*}
  \hat{q} &= \frac{1}{\sqrt{2}} \begin{pmatrix}-1\\1\\0\end{pmatrix}.
\end{align*}
\subsection{Problem 21}%
We get
\begin{align*}
  R_{n}\left(\alpha\right) &= e^{-i\alpha \hat{n}\cdot \mathbf{L}}\\
                           &= e^{-i\alpha \left(\frac{1}{\sqrt{2}}L_2 + \frac{1}{\sqrt{2}}L_3\right)}\\
                           &= e^{-i\frac{\alpha}{\sqrt{2}}L_2}e^{-i\frac{\alpha}{\sqrt{2}}L_3}\\
                           &= R_{y}\left(\frac{\alpha}{\sqrt{2}}\right)R_{z}\left(\frac{\alpha}{\sqrt{2}}\right).
\end{align*}
\section{Chapter 27 Problems}%
\subsection{Problem 1}%
\begin{enumerate}[(a)]
  \item 
    \begin{align*}
      \det\left(A - \lambda I\right) &= \left(\lambda - 1\right)^2 - 4\\
      \left(\lambda - 1\right)^2 - 4 &= 0\\
      \lambda_{1,2} &= -1,3.
    \end{align*}
    We have
    \begin{align*}
      \begin{pmatrix}1 & 2 \\ 2 & 1\end{pmatrix} \begin{pmatrix}x\\y\end{pmatrix} &= -1 \begin{pmatrix}x\\y\end{pmatrix}\\
      x + 2y &= -x\\
      x &= -y\\
      \ket{v_1} &= \frac{1}{\sqrt{2}}\begin{pmatrix}-1\\1\end{pmatrix}\\
      \ket{v_2} &= \frac{1}{\sqrt{2}} \begin{pmatrix}1\\1\end{pmatrix}.
    \end{align*}
  \item 
    \begin{align*}
      \det\left(A - \lambda I\right) &= \left(\lambda - 1\right)^2 + 4\\
      \lambda_{1,2} &= 1\pm 2i.
    \end{align*}
    We have
    \begin{align*}
      \begin{pmatrix}1 & 2 \\ -2 & 1\end{pmatrix} \begin{pmatrix}x\\y\end{pmatrix} &= 1 \pm 2i \begin{pmatrix}x\\y\end{pmatrix}\\
      x + 2y &= \left(1+2i\right)x\\
      x &= -iy\\
      \ket{v_1} &= \frac{1}{\sqrt{2}} \begin{pmatrix}1\\i\end{pmatrix}\\
      \ket{v_2} &= \frac{1}{\sqrt{2}} \begin{pmatrix}1\\-i\end{pmatrix}.
    \end{align*}
  \item 
    \begin{align*}
      \det\left(A - \lambda I\right) &= \left(\lambda - 1\right)^2-4\\
      \lambda_{1,2} &= -1,3,
    \end{align*}
    meaning
    \begin{align*}
      \ket{v_1} &= \frac{1}{\sqrt{2}}\begin{pmatrix}-i\\1\end{pmatrix}\\
      \ket{v_2} &= \frac{1}{\sqrt{2}} \begin{pmatrix}i\\1\end{pmatrix}.
    \end{align*}
  \item 
    \begin{align*}
      \det\left(A - \lambda I\right) &= \left(\lambda - 1\right)^2 + 4\\
      \lambda_{1,2} &= 1\pm 2i,
    \end{align*}
    meaning
    \begin{align*}
      \ket{v_1} &= \frac{1}{\sqrt{2}} \begin{pmatrix}1\\1\end{pmatrix}\\
      \ket{v_2} &= \frac{1}{\sqrt{2}} \begin{pmatrix}-1\\1\end{pmatrix}.
    \end{align*}
\end{enumerate}
\subsection{Problem 2}%
The trace is equal to $\lambda_1 + \lambda_2$, while the determinant is equal to $\lambda_1\lambda_2$, meaning we have two equations and two unknowns.\newline

For
\begin{align*}
  A &= \begin{pmatrix}1 & 2\\ 2 & 1\end{pmatrix},
\end{align*}
it is the case that $\tr\left(A\right) = 2$  and $\det\left(A\right) = -3$, so $\lambda_{1,2} = -1,3$.
\subsection{Problem 4}%
Computing
\begin{align*}
  \det\left(S - \lambda I\right) &= -\lambda^3 + 3\lambda - 2.
\end{align*}
We find
\begin{align*}
  \lambda_{1} &= -2\\
  \lambda_{2,3} &= 1.
\end{align*}
The eigenvector for $\lambda_1$ is
\begin{align*}
  \ket{v_1} &= \frac{1}{\sqrt{6}}\begin{pmatrix}1\\-2\\1\end{pmatrix}.
\end{align*}
To solve for $\ket{v_2}$ and $\ket{v_3}$, we find
\begin{align*}
  \frac{1}{2} \begin{pmatrix}1 & 2 & -1 \\ 2 & -2 & 2 \\ -1 & 2 & 1\end{pmatrix} \begin{pmatrix}x\\y\\z\end{pmatrix} &= \begin{pmatrix}x\\y\\z\end{pmatrix}\\
  \frac{1}{2}x + y -\frac{1}{2}z &= x\\
  2y &= x + z.
\end{align*}
Two orthogonal eigenvectors corresponding to $\lambda_{2,3} = 1$ are
\begin{align*}
  \ket{v_2} &= \frac{1}{\sqrt{2}}\begin{pmatrix}-1\\0\\1\end{pmatrix}\\
  \ket{v_3} &= \frac{1}{\sqrt{3}} \begin{pmatrix}1\\1\\1\end{pmatrix}.
\end{align*}
\subsection{Problem 6}%
I don't know how to do this problem.
\end{document}
