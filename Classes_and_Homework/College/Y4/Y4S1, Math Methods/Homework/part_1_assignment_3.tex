\documentclass[10pt]{mypackage}

% sans serif font:
%\usepackage{cmbright}
%\usepackage{sfmath}
%\usepackage{bbold} %better blackboard bold

%serif font + different blackboard bold for serif font
\usepackage{newpxtext,eulerpx}
\renewcommand*{\mathbb}[1]{\varmathbb{#1}}

\pagestyle{fancy} %better headers
\fancyhf{}
\rhead{Avinash Iyer}
\lhead{Physics 310, Assignment 3}

\setcounter{secnumdepth}{0}

\begin{document}
\RaggedRight
\section{8.1}%
\begin{enumerate}[(a)]
  \item 
    \begin{align*}
      \int_{0}^{1} 2^x\:dx &= \int_{0}^{1} e^{x\left(\ln 2\right)}\:dx\\
                           &= \frac{1}{\ln 2}\left(e^{x\left(\ln 2\right)}\bigr\vert_{0}^{1}\right)\tag*{$u = x\left(\ln 2\right)$}\\
                           &= \frac{1}{\ln 2}\left(2^{x}\bigr\vert_{0}^{1}\right)\\
                           &= \frac{1}{\ln 2}\left(2-1\right)\\
                           &= \frac{1}{\ln 2}.
    \end{align*}
  \item
    \begin{align*}
      \int_{-\infty}^{\infty} e^{-\frac{x^2}{2}}e^{x}\:dx &= \int_{-\infty}^{\infty} e^{\left(-\frac{x^2}{2} + x - \frac{1}{2} \right)+ \frac{1}{2}}\:dx\tag*{Completing the square.}\\
                                                          &= e^{\frac{1}{2}}\int_{-\infty}^{\infty}e^{-\frac{1}{2}\left(x-1\right)^2} \:dx\\
                                                          &= \sqrt{2\pi e}\tag*{Gaussian Integral}
    \end{align*}
  \item 
  \item 
    \begin{align*}
      \int_{-a}^{a} \sin x e^{-\alpha x^2}\:dx &= 0 \tag*{Even/odd.}
    \end{align*}
  \item 
    \begin{align*}
      \int_{0}^{1} e^{\sqrt{x}}\:dx &= xe^{\sqrt{x}}\bigr\vert_{0}^{1} - \frac{1}{2}\int_{0}^{1} xe^{\sqrt{x}}\:dx\tag*{Integration by Parts}\\
                                    &= e - \int_{0}^{1} u^3e^{u}\:du\tag*{$u = \sqrt{x}$}\\
                                    &= e - \left(u^3e^{u}\bigr\vert_{0}^{1} - 3u^2e^{u}\bigr\vert_{0}^{1} + 6u\bigr\vert_{0}^{1} - 6e^{u}\bigr\vert_{0}^{1}\right)\tag*{Repeated integration by parts.}\\
                                    &= 3e-6.
    \end{align*}
    To evaluate $\int_{0}^{1} u^{3}e^{u}\:du$, we used tabular integration as follows:
    \begin{center}
      \begin{tabular}{c|c|c}
        Sign & Differentiate & Integrate\\
        \hline
        + & $u^3$ & $e^u$\\
        - & $3u^2$ & $e^u$\\
        + & $6u$ & $e^u$\\
        - & $6$ & $e^u$\\
        + & 0 & $e^u$
      \end{tabular}
    \end{center}
      Taking the boundary integrals, we obtain
      \begin{align*}
          u^3e^u\bigr\vert_{0}^{1} - 3u^2e^u\bigr\vert_{0}^{1} + 6ue^u\bigr\vert_{0}^{1} - 6e^u\bigr\vert_{0}^{1} &= 6-2e
      \end{align*}
  \item 
    \begin{align*}
    \int_{}^{} \frac{1}{\sqrt{1+x^2}}\:dx &= \int\frac{1}{\cosh(u)}\cosh(u)\:du \tag*{$x = \sinh(u)$}\\
                                          &= u + C\\
                                          &= \sinh^{-1}\left(x\right) + C.
    \end{align*}
  \item 
    \begin{align*}
      \int_{}^{} \tanh x\:dx &= \int_{}^{} \frac{\sinh x}{\cosh x}\:dx\\
                             &= \int_{}^{} \frac{1}{u}\:du\tag*{$u = \cosh x$}\\
                             &= \ln\left\vert u \right\vert + C\\
                             &= \ln\left\vert \cosh x \right\vert + C.
    \end{align*}
  \item 
    \begin{align*}
      \int_{}^{} \tan^{-1}x\:dx &= x\tan^{-1}x - \int_{}^{} \frac{x}{1+x^2}\:dx\tag*{integration by parts}\\
                                &= x\tan^{-1}x - \frac{1}{2}\ln|1 + x^2| + C.\tag*{$u$-substitution implicit}
    \end{align*}
  \item 
    \begin{align*}
      \int_{S}^{} z^2\:d\mathbf{a} &= \int_{0}^{\pi/2}\int_{0}^{\pi/2} \cos^2\theta \sin\theta\:d\phi d\theta\\
                                   &= \frac{\pi}{2}\int_{0}^{\pi/2} \cos^2\theta\sin\theta\:d\theta\\
                                   &= -\frac{\pi}{2}\int_{0}^{-1} t^2\:dt \tag*{$t = \cos\theta$}\\
                                   &= \frac{\pi}{2}\left(\frac{t^3}{3}\bigr\vert_{-1}^{0}\right)\\
                                   &= \frac{\pi}{6}
    \end{align*}
\end{enumerate}
\subsection{8.8}%
\begin{enumerate}[(a)]
  \item 
    \begin{align*}
      \int_{0}^{\infty} \frac{x}{e^{x}-1}\:dx &= \int_{0}^{\infty} \frac{xe^{-x}}{1-e^{-x}}\:dx\\
                                              &= \int_{0}^{\infty} xe^{-x}\left(\sum_{k=0}^{\infty}e^{-kx}\right)\:dx\\
                                              &= \sum_{k=0}^{\infty}\int_{0}^{\infty} xe^{-\left(k+1\right)x}\:dx\\
                                              &= \sum_{k=0}^{\infty}\frac{1}{\left(k+1\right)^2}\int_{0}^{\infty} ue^{-u}\:du \tag*{$u = \left(k+1\right)x$}\\
                                              &= \frac{\pi^2}{6}\tag*{Basel Problem}.
    \end{align*}
  \item 
    \begin{align*}
      \int_{0}^{\infty} \frac{x}{e^x + 1}\:dx &= \int_{0}^{\infty} \frac{xe^{-x}}{1 + e^{-x}}\:dx\\
                                              &= \int_{0}^{\infty} xe^{-x}\sum_{k=0}^{\infty}\left(-1\right)^ke^{-kx}\:dx\\
                                              &= \sum_{k=0}^{\infty}\left(-1\right)^k\int_{0}^{\infty} xe^{-(k+1)x}\:dx\\
                                              &= \sum_{k=0}^{\infty}\frac{\left(-1\right)^k}{\left(k+1\right)^2}\int_{0}^{\infty} ue^{-u}\:dx\tag*{$u = \left(k+1\right)x$}\\
                                              &= \sum_{k=0}^{\infty}\frac{\left(-1\right)^k}{\left(k+1\right)^2}.
                                              \intertext{To resolve}
      \sum_{k=0}^{\infty}\frac{\left(-1\right)^k}{\left(k+1\right)^2} &= 1 - \frac{1}{4} + \frac{1}{9} - \frac{1}{16} + \cdots
      \intertext{we take}
                                                                      &= \left(1 + \frac{1}{9} + \frac{1}{25} + \cdots\right) - \frac{1}{4}\underbrace{\left(1 + \frac{1}{4} + \frac{1}{9} + \frac{1}{16} + \cdots\right)}_{\frac{\pi^2}{6}},
                                                                      \intertext{meaning}
                                                                      \int_{0}^{\infty} \frac{x}{e^x + 1}\:dx&= \frac{\pi^2}{12}.
    \end{align*}
\end{enumerate}
\subsection{8.14}%
\begin{align*}
  I_0\left(a\right) &= \int_{0}^{\infty} x^{0}e^{-ax^2}\:dx\\
                    &= \frac{1}{2}\sqrt{\frac{\pi}{a}}\\
                    \\
  I_1\left(a\right) &= \int_{0}^{\infty} xe^{-ax^2}\:dx\\
                    &= \frac{1}{2}\left(\frac{1}{-a}e^{-ax^2}\bigr\vert_{0}^{\infty}\right)\\
                    &= \frac{1}{2a}\\
                    \\
  I_2\left(a\right) &= \int_{0}^{\infty} x^2e^{-ax^2}\:dx\\
                    &= -\frac{1}{2a}xe^{-ax^2}\bigr\vert_{0}^{\infty} + \frac{1}{2a}\int_{0}^{\infty} xe^{-ax^2}\:dx\\
                    &= \frac{1}{2a}I_1\\
                    &= \frac{1}{4a^2}\\
                    \\
  I_3\left(a\right) &= \int_{0}^{\infty} x^3e^{-ax^2}\:dx\\
                    &= -\frac{1}{2a}x^2e^{-ax^2}\bigr\vert_{0}^{\infty} + \frac{1}{a}\int_{}^{} xe^{-ax^2}\:dx\\
                    &= \frac{1}{a}I_1\\
                    &= \frac{1}{2a^2}\\
                    \\
  I_4\left(a\right) &= \int_{0}^{\infty} x^4e^{-ax^2}\:dx\\
                    &= -\frac{1}{2a}x^3e^{-ax^2}\bigr\vert_{0}^{\infty} + \frac{3}{2a}\int_{0}^{\infty} x^2e^{-ax^2}\:dx\\
                    &= \frac{3}{2a}I_2\\
                    &= \frac{3}{8a^3}.
\end{align*}
\end{document}
