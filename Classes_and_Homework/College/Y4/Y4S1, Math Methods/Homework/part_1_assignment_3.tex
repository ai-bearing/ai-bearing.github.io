\documentclass[10pt]{mypackage}

% sans serif font:
%\usepackage{cmbright}
%\usepackage{sfmath}
%\usepackage{bbold} %better blackboard bold

%serif font + different blackboard bold for serif font
\usepackage{newpxtext,eulerpx}
\renewcommand*{\mathbb}[1]{\varmathbb{#1}}
\newcommand{\df}{\delta\!}

\pagestyle{fancy} %better headers
\fancyhf{}
\rhead{Avinash Iyer}
\lhead{Physics 310, Assignment 3}

\setcounter{secnumdepth}{0}

\begin{document}
\RaggedRight
\section{Chapter 8 Problems}%

\subsection{8.1}%
\begin{enumerate}[(a)]
  \item 
    \begin{align*}
      \int_{0}^{1} 2^x\:dx &= \int_{0}^{1} e^{x\left(\ln 2\right)}\:dx\\
                           &= \frac{1}{\ln 2}\left(e^{x\left(\ln 2\right)}\bigr\vert_{0}^{1}\right)\tag*{$u = x\left(\ln 2\right)$}\\
                           &= \frac{1}{\ln 2}\left(2^{x}\bigr\vert_{0}^{1}\right)\\
                           &= \frac{1}{\ln 2}\left(2-1\right)\\
                           &= \frac{1}{\ln 2}.
    \end{align*}
  \item
    \begin{align*}
      \int_{-\infty}^{\infty} e^{-\frac{x^2}{2}}e^{x}\:dx &= \int_{-\infty}^{\infty} e^{\left(-\frac{x^2}{2} + x - \frac{1}{2} \right)+ \frac{1}{2}}\:dx\tag*{Completing the square.}\\
                                                          &= e^{\frac{1}{2}}\int_{-\infty}^{\infty}e^{-\frac{1}{2}\left(x-1\right)^2} \:dx\\
                                                          &= \sqrt{2\pi e}\tag*{Gaussian Integral}
    \end{align*}
  \item 
  \item 
    \begin{align*}
      \int_{-a}^{a} \sin x e^{-\alpha x^2}\:dx &= 0 \tag*{Even/odd.}
    \end{align*}
  \item 
    \begin{align*}
      \int_{0}^{1} e^{\sqrt{x}}\:dx &= xe^{\sqrt{x}}\bigr\vert_{0}^{1} - \frac{1}{2}\int_{0}^{1} xe^{\sqrt{x}}\:dx\tag*{Integration by Parts}\\
                                    &= e - \int_{0}^{1} u^3e^{u}\:du\tag*{$u = \sqrt{x}$}\\
                                    &= e - \left(u^3e^{u}\bigr\vert_{0}^{1} - 3u^2e^{u}\bigr\vert_{0}^{1} + 6u\bigr\vert_{0}^{1} - 6e^{u}\bigr\vert_{0}^{1}\right)\tag*{Repeated integration by parts.}\\
                                    &= 3e-6.
    \end{align*}
    To evaluate $\int_{0}^{1} u^{3}e^{u}\:du$, we used tabular integration as follows:
    \begin{center}
      \begin{tabular}{c|c|c}
        Sign & Differentiate & Integrate\\
        \hline
        + & $u^3$ & $e^u$\\
        - & $3u^2$ & $e^u$\\
        + & $6u$ & $e^u$\\
        - & $6$ & $e^u$\\
        + & 0 & $e^u$
      \end{tabular}
    \end{center}
      Taking the boundary integrals, we obtain
      \begin{align*}
          u^3e^u\bigr\vert_{0}^{1} - 3u^2e^u\bigr\vert_{0}^{1} + 6ue^u\bigr\vert_{0}^{1} - 6e^u\bigr\vert_{0}^{1} &= 6-2e
      \end{align*}
  \item 
    \begin{align*}
    \int_{}^{} \frac{1}{\sqrt{1+x^2}}\:dx &= \int\frac{1}{\cosh(u)}\cosh(u)\:du \tag*{$x = \sinh(u)$}\\
                                          &= u + C\\
                                          &= \sinh^{-1}\left(x\right) + C.
    \end{align*}
  \item 
    \begin{align*}
      \int_{}^{} \tanh x\:dx &= \int_{}^{} \frac{\sinh x}{\cosh x}\:dx\\
                             &= \int_{}^{} \frac{1}{u}\:du\tag*{$u = \cosh x$}\\
                             &= \ln\left\vert u \right\vert + C\\
                             &= \ln\left\vert \cosh x \right\vert + C.
    \end{align*}
  \item 
    \begin{align*}
      \int_{}^{} \tan^{-1}x\:dx &= x\tan^{-1}x - \int_{}^{} \frac{x}{1+x^2}\:dx\tag*{integration by parts}\\
                                &= x\tan^{-1}x - \frac{1}{2}\ln|1 + x^2| + C.\tag*{$u$-substitution implicit}
    \end{align*}
  \item 
    \begin{align*}
      \int_{S}^{} z^2\:d\mathbf{a} &= \int_{0}^{\pi/2}\int_{0}^{\pi/2} \cos^2\theta \sin\theta\:d\phi d\theta\\
                                   &= \frac{\pi}{2}\int_{0}^{\pi/2} \cos^2\theta\sin\theta\:d\theta\\
                                   &= -\frac{\pi}{2}\int_{0}^{-1} t^2\:dt \tag*{$t = \cos\theta$}\\
                                   &= \frac{\pi}{2}\left(\frac{t^3}{3}\bigr\vert_{-1}^{0}\right)\\
                                   &= \frac{\pi}{6}
    \end{align*}
\end{enumerate}
\subsection{8.8}%
\begin{enumerate}[(a)]
  \item 
    \begin{align*}
      \int_{0}^{\infty} \frac{x}{e^{x}-1}\:dx &= \int_{0}^{\infty} \frac{xe^{-x}}{1-e^{-x}}\:dx\\
                                              &= \int_{0}^{\infty} xe^{-x}\left(\sum_{k=0}^{\infty}e^{-kx}\right)\:dx\\
                                              &= \sum_{k=0}^{\infty}\int_{0}^{\infty} xe^{-\left(k+1\right)x}\:dx\\
                                              &= \sum_{k=0}^{\infty}\frac{1}{\left(k+1\right)^2}\int_{0}^{\infty} ue^{-u}\:du \tag*{$u = \left(k+1\right)x$}\\
                                              &= \frac{\pi^2}{6}\tag*{Basel Problem}.
    \end{align*}
  \item 
    \begin{align*}
      \int_{0}^{\infty} \frac{x}{e^x + 1}\:dx &= \int_{0}^{\infty} \frac{xe^{-x}}{1 + e^{-x}}\:dx\\
                                              &= \int_{0}^{\infty} xe^{-x}\sum_{k=0}^{\infty}\left(-1\right)^ke^{-kx}\:dx\\
                                              &= \sum_{k=0}^{\infty}\left(-1\right)^k\int_{0}^{\infty} xe^{-(k+1)x}\:dx\\
                                              &= \sum_{k=0}^{\infty}\frac{\left(-1\right)^k}{\left(k+1\right)^2}\int_{0}^{\infty} ue^{-u}\:dx\tag*{$u = \left(k+1\right)x$}\\
                                              &= \sum_{k=0}^{\infty}\frac{\left(-1\right)^k}{\left(k+1\right)^2}.
                                              \intertext{To resolve}
      \sum_{k=0}^{\infty}\frac{\left(-1\right)^k}{\left(k+1\right)^2} &= 1 - \frac{1}{4} + \frac{1}{9} - \frac{1}{16} + \cdots
      \intertext{we take}
                                                                      &= \left(1 + \frac{1}{9} + \frac{1}{25} + \cdots\right) - \frac{1}{4}\underbrace{\left(1 + \frac{1}{4} + \frac{1}{9} + \frac{1}{16} + \cdots\right)}_{\frac{\pi^2}{6}},
                                                                      \intertext{meaning}
                                                                      \int_{0}^{\infty} \frac{x}{e^x + 1}\:dx&= \frac{\pi^2}{12}.
    \end{align*}
\end{enumerate}
\subsection{8.14}%
\begin{align*}
  I_0\left(a\right) &= \int_{0}^{\infty} x^{0}e^{-ax^2}\:dx\\
                    &= \frac{1}{2}\sqrt{\frac{\pi}{a}}\\
                    \\
  I_1\left(a\right) &= \int_{0}^{\infty} xe^{-ax^2}\:dx\\
                    &= \frac{1}{2}\left(\frac{1}{-a}e^{-ax^2}\bigr\vert_{0}^{\infty}\right)\\
                    &= \frac{1}{2a}\\
                    \\
  I_2\left(a\right) &= \int_{0}^{\infty} x^2e^{-ax^2}\:dx\\
                    &= -\frac{1}{2a}\left(xe^{-ax^2}\bigr\vert_{0}^{\infty}\right) + \frac{1}{2a}\int_{0}^{\infty} e^{-ax^2}\:dx\\
                    &= \frac{1}{4a}\sqrt{\frac{\pi}{a}}.\\
                    \\
  I_3\left(a\right) &= \int_{0}^{\infty} x^3e^{-ax^2}\:dx\\
                    &= -\frac{1}{2a}\left(x^2e^{-ax^2}\bigr\vert_{0}^{\infty}\right) + \frac{1}{a}\int_{0}^{\infty} xe^{-ax^2}\:dx\\
                    &= \frac{1}{2a^2}\\
                    \\
  I_4\left(a\right) &= \int_{0}^{\infty} x^4e^{-ax^2}\:dx\\
                    &= -\frac{1}{2a}x^3e^{-ax^2}\bigr\vert_{0}^{\infty} + \frac{3}{2a}\int_{0}^{\infty} x^2e^{-ax^2}\:dx\\
                    &= \frac{3}{2a}I_2\\
                    &= \frac{3}{8a^3}\sqrt{\frac{\pi}{a}}.
\end{align*}
\subsection{8.24}%
\begin{align*}
  J(a) &= \lim_{n\rightarrow\infty}\int_{0}^{1}\int_{0}^{1}\cdots\int_{0}^{1} e^{-a\frac{\sum_{i=1}^{n}x_i}{n}}\frac{n}{\sum_{i=1}^{n}x_i}\:dx_1dx_2\cdots dx_{n}\\
  J'(a) &= -\lim_{n\rightarrow\infty}\int_{0}^{1}\int_{0}^{1}\cdots\int_{0}^{1}e^{-a}\:dx_1dx_2\cdots dx_n\\
        &= -e^{-a}\\
        \intertext{meaning}
  J(a) &= e^{-a}\\
  J(0) &= 1.
\end{align*}
\subsection{8.26}%
\begin{enumerate}[(a)]
  \item 
    \begin{align*}
      \int_{0}^{\infty} e^{-ax}\sin kx\:dx &= -\frac{1}{k}e^{-ax}\cos x\bigr\vert_{0}^{\infty} - \frac{a}{k}e^{-ax}\sin x\bigr\vert_{0}^{\infty} - \frac{a^2}{k^2}\int_{0}^{\infty} e^{-ax}\sin x\:dx\\
      \left(1 + a^2\right)\int_{0}^{\infty}e^{-ax} \sin x\:dx &= -e^{-ax}\cos x\bigr\vert_{0}^{\infty} - ae^{-ax}\sin x\bigr\vert_{0}^{\infty}\\
      \left(1 + a^2\right)\int_{0}^{\infty}e^{-ax} \sin x\:dx &= -\frac{1}{k}\\
      \int_{0}^{\infty} e^{-ax}\sin x\:dx &= -\frac{1}{k\left(1+\frac{a^2}{k^2}\right)}\\
                                          &= -\frac{k}{k^2 + a^2}
    \end{align*}
  \item 
    \begin{align*}
      \int_{0}^{\infty} e^{-ax}\sin kx\:dx &= \frac{1}{2i}\int_{0}^{\infty} e^{-ax}\left(e^{ikx} - e^{-ikx}\right)\:dx\\
                                           &= \frac{1}{2i}\left(\frac{1}{-a+ik}e^{-x\left(a-ik\right)} - \frac{1}{-a-ik}e^{-x\left(a + ik\right)}\right)\bigr\vert_{0}^{\infty}\\
                                           &= \frac{1}{2i}\left(\frac{1}{-a + ik}\left(e^{a-ik}\right)^{-x} - \frac{1}{-a-ik}\left(e^{a + ik}\right)^{-x}\right)\bigr\vert_{0}^{\infty}\\
                                           &= \frac{1}{2i}\left(\frac{1}{-a + ik} - \frac{1}{-a-ik}\right)\\
                                           &= \frac{1}{2}\left(\frac{1}{-k-ia} - \frac{1}{k - ia}\right)\\
                                           &= \frac{1}{2}\left(\frac{(k-ia)-\left(-k-ia\right)}{\left(-k-ia\right)\left(k-ia\right)}\right)\\
                                           &= \frac{1}{2}\left(\frac{2k}{-k^2 - a^2}\right)\\
                                           &= -\frac{k}{k^2 + a^2}.
    \end{align*}
  \item
    \begin{align*}
      \int_{0}^{\infty} e^{-ax}\sin kx\:dx &= \int_{0}^{\infty} e^{-ax}\im\left(e^{ikx}\right)\:dx\\
                                           &= \im \left(\int_{0}^{\infty} e^{-ax}e^{ikx}\:dx\right)\\
                                           &= \im \left(\frac{1}{-a+ik}e^{\left(-a + ik\right)x}\bigr\vert_{0}^{\infty}\right)\\
                                           &= \im \left(\frac{1}{-a + ik}\left(e^{a - ik}\right)^{-x}\bigr\vert_{0}^{\infty}\right)\\
                                           &= \im \left(\frac{1}{-a + ik}\right)\\
                                           &= \im \left(\frac{-a-ik}{a^2 + k^2}\right)\\
                                           &= \im \left(\frac{-a}{a^2 + k^2} - \frac{k}{k^2 + a^2}i\right)\\
                                           &= -\frac{k}{k^2 + a^2}.
    \end{align*}
\end{enumerate}
\section{Chapter 9 Problems}%
\subsection{9.1}%
\begin{enumerate}[(a)]
  \item 
    \begin{align*}
      \int_{-2}^{3} \left(x^3 -\left(2x + 5\right)^2\right)\df\left(x-1\right)\:dx &= -48.
    \end{align*}
  \item
    \begin{align*}
      \int_{0}^{3} \left(5x^2 - 3x + 4\right)\df\left(x+2\right)\:dx &= 0.
    \end{align*}
  \item 
    \begin{align*}
      \int_{0}^{1} \cos x \df\left(x-\pi/6\right)\:dx &= \frac{\sqrt{3}}{2}.
    \end{align*}
  \item 
    \begin{align*}
      \int_{-\pi}^{\pi} \ln \left(\sin x + 2\right)\df\left(x+\pi/2\right)\:dx &= \ln(3).
    \end{align*}
  \item 
    \begin{align*}
      \int_{-1}^{1} \left(x^3 - 3x^2 + 2\right)\df\left(x/7\right)\:dx &= 14.
    \end{align*}
  \item
    \begin{align*}
      \int_{-1}^{1} (x-1)e^{x^2}\df\left(-3x\right)\:dx &= -\frac{1}{3}.
    \end{align*}
  \item
    \begin{align*}
      \int_{-\pi}^{\pi} 4x^2\arccos(x)\df\left(2x-1\right)\:dx &= \frac{\pi}{6}.
    \end{align*}
  \item
    \begin{align*}
      \int_{p}^{\infty} \df\left(x+q\right)\:dx &= \begin{cases}
        1 & p < q\\
        0 & p > q
      \end{cases}.
    \end{align*}
  \item 
    \begin{align*}
      \int_{0}^{2b} x\df\left(\left(x^2 - b^2\right)\left(x-\frac{b}{2}\right)\right)\:dx &= \int_{0}^{2b} x\left(b^2\delta\left(x-b\right) + 3b^2\delta\left(x+b\right) - \frac{3}{4}b^2\delta\left(x-b/2\right)\right)\:dx\\
                                                                                               &= b^3 - \frac{3}{8}b^3\\
                                                                                               &= \frac{5}{8}b^3.
    \end{align*}
  \item
    \begin{align*}
      \int_{-\pi/2}^{\pi/2} e^{x}\df\left(\tan(x)\right)\:dx &= 1.
    \end{align*}
\end{enumerate}
\subsection{9.2}%
We can see that $\frac{d\Theta}{dx} = 0$ for $x < 0$ and $x > 0$, and $\frac{d\Theta}{dx}$ is undefined for $x = 0$.\newline

Additionally, for any $f$, we have
\begin{align*}
  \int_{-\infty}^{\infty} f(x)\frac{d\Theta}{dx}\:dx &= \int_{-\infty}^{\infty} f(x)\:d\Theta\\
                                                     &= f(0).
\end{align*}
Thus, since $\frac{d\Theta}{dx}$ has the two necessary conditions to be a delta distribution. $\frac{d\Theta}{dx } = \delta(x)$.

\subsection{9.3}%
First, we can see that
\begin{align*}
  \int_{-\infty}^{\infty} \phi_n(x)\:dx &= \frac{n}{\sqrt{\pi}}\int_{-\infty}^{\infty} e^{-n^2x^2}\:dx\\
                                        &= \frac{n}{\sqrt{\pi}}\left(\frac{\sqrt{\pi}}{n}\right)\\
                                        &= 1,
\end{align*}
meaning that we satisfy normalization. Additionally, for $x_0\neq 0$,
\begin{align*}
  \lim_{n\rightarrow\infty} \phi_n\left(x_0\right) &= \lim_{n\rightarrow\infty}\frac{n}{\sqrt{\pi}}e^{-n^2x_0^2}\\
                                                   &= \frac{1}{\sqrt{\pi}}\lim_{n\rightarrow\infty}n\left(e^{x_0^2}\right)^{-n^2}\\
                                                   &= 0,
\end{align*}
and for $x = 0$, $\lim_{n\rightarrow}\phi_n(x)$ diverges. Finally, for $f(x)$, we have
\begin{align*}
  \int_{-\infty}^{\infty} f(x)\frac{n}{\pi}e^{-n^2x^2}\:dx &= f(c)\int_{-\infty}^{\infty}\frac{n}{\sqrt{\pi}}e^{-n^2x^2}\:dx\\
                                                           &= f(c)
\end{align*}
for some $c\in (-\infty,\infty)$. In particular, since $\frac{n}{\sqrt{\pi}}f(x)e^{-n^2x^2}$ tends to zero for any $x\neq 0$, it must be the case that $f(c) = f(0)$.
\subsection{9.10}%
\begin{enumerate}[(a)]
  \item Knowing that $\delta$ instantiates an integral at a particular value, we know that
    \begin{align*}
      M &= \int_{V}^{} \rho\:d\tau\\
        &= \int_{-\infty}^{\infty}\int_{0}^{2\pi}\int_{0}^{\infty}\rho r\:drd\phi dz\\
        &= \int_{-\infty}^{\infty}\int_{0}^{2\pi}\int_{0}^{\infty}M\df\left(r-R\right)\df\left(z\right)r\:drd\phi dz,
    \end{align*}
    meaning $\rho = M\df\left(r-R\right)\df\left(z\right)$.
  \item Similarly, we have
    \begin{align*}
      M &= \int_{V}^{} \rho\:d\tau\\
        &= \int_{0}^{\pi}\int_{0}^{2\pi}\int_{0}^{\infty}\rho r^2\sin\theta\:drd\phi d\theta\\
        &= \int_{0}^{\pi}\int_{0}^{2\pi}\int_{0}^{\infty}M\df\left(r-R\right)\df\left(\theta - \frac{\pi}{2}\right) r^2\sin\theta\:dr d\phi d\theta,
    \end{align*}
    meaning $\rho = M\df\left(r-R\right)\df\left(\theta - \frac{\pi}{2}\right)$.
\end{enumerate}
\subsection{9.11}%
Notice that
\begin{align*}
  \frac{d}{dx}\left(i\pi \sgn(x)\right) &= \frac{d}{dx}\left(\int_{-\infty}^{\infty} \frac{e^{ikx}}{k}\:dk\right)\\
                                        &= i\int_{-\infty}^{\infty} e^{ikx}\:dk.
\end{align*}
Thus,
\begin{align*}
  \frac{1}{2\pi}\int_{-\infty}^{\infty} e^{ikx}\:dk &= \frac{1}{2}\left(\frac{d}{dx}\sgn(x)\right),
\end{align*}
meaning
\begin{align*}
  \int_{a}^{b} f(x)\left(\frac{1}{2\pi}\int_{-\infty}^{\infty} e^{ikx}\:dk\right)\:dx &= \int_{a}^{b} f(x)\left(\frac{d}{dx}\sgn(x)\right)\:dx\\
                                                                                      &= \frac{1}{2}f(x)\sgn(x)\bigr\vert_{a}^{b} - \frac{1}{2}\int_{a}^{b} f'(x)\sgn(x)\:dx\\
                                                                                      &= \frac{1}{2}\left(f(b)\sgn(b) - f(a)\sgn(a)\right) - \frac{1}{2}\left(\int_{a}^{0} f'(x)\sgn(x)\:dx + \int_{0}^{b} f'(x)\sgn(x)\:dx\right)\\
                                                                                      &= \frac{1}{2}\left(f(b)\sgn(b) - f(a)\sgn(a)\right) - \frac{1}{2}\left(-\sgn(a)\left(f(a) - f(0)\right) + \sgn(b)\left(f(b) - f(0)\right)\right).
\end{align*}
Without loss of generality, we say $a < b$. If $\sgn(a) = \sgn(b)$, then this expression resolves to $0$, and if $\sgn(a) \neq \sgn(b)$, this expression resolves to $f(0)$.
\end{document}
