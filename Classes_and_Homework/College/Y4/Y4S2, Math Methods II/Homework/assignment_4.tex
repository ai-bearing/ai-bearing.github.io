\documentclass[10pt]{mypackage}

% sans serif font:
%\usepackage{cmbright,sfmath,bbold}
%\renewcommand{\mathcal}{\mathtt}

%Euler:
\usepackage{newpxtext,eulerpx,eucal,eufrak}
\renewcommand*{\mathbb}[1]{\varmathbb{#1}}
\renewcommand*{\hbar}{\hslash}

\DeclareMathOperator{\Ln}{Ln}
\DeclareMathOperator{\Arg}{Arg}
\DeclareMathOperator{\Arctan}{Arctan}
\DeclareMathOperator{\Res}{Res}
\DeclareMathOperator{\res}{Res}
\usepackage{homework}

\pagestyle{fancy} %better headers
\fancyhf{}
\rhead{Avinash Iyer}
\lhead{}

\setcounter{secnumdepth}{0}

\begin{document}
\RaggedRight
\begin{solution}[21.1]\hfill
  \begin{enumerate}[(a)]
    \item Doing a partial fraction decomposition, we find
      \begin{align*}
        \frac{1}{\left( z-1 \right)\left( z+2 \right)} &= \frac{1}{3}\frac{1}{z-1} - \frac{1}{3}\frac{1}{z+2},
      \end{align*}
      giving giving a residue of $\frac{1}{3}$ at $z = 1$ and a residue of $-\frac{1}{3}$ at $z = -2$.
    \item Evaluating the residue at $z = 1$, we may use the cover-up method to find
      \begin{align*}
        \res\left[ f(z),1 \right] &= \frac{e^{2i}}{27}.
      \end{align*}
      To evaluate the residue at $z = -2$, we use the formula to calculate residues, giving
      \begin{align*}
        \res\left[ f(z),-2 \right] &= \frac{1}{2}\diff{^2}{z^2}\left( \frac{e^{2iz}}{z-1} \right)\bigr\vert_{z = -2}\\
                                   &= \frac{38}{27}e^{-4i}
      \end{align*}
    \item Note that $\sin(z)$ is a simple zero at $z = n\pi$. Therefore, we evaluate
      \begin{align*}
        \res\left[ f(z),n\pi \right] &= \left( -1 \right)^n e^{n\pi}.
      \end{align*}
    \item Using the Laurent series for $e^{1/z}$, we find that
      \begin{align*}
        e^{1/z} &= 1 + \frac{1}{z} + \frac{1}{2z^2} + \cdots,
      \end{align*}
      so that
      \begin{align*}
        \res\left[ f(z),0 \right] &= 1.
      \end{align*}
    \item Note that $e^{2z} + 1 = 0$ whenever $z = i(2n + 1)\pi/2$. These are all simple zeros, so we may evaluate
      \begin{align*}
        \res\left[ f(z),i\left( 2n + 1 \right)\pi/2 \right] &= \frac{-\left( 2n + 1 \right)^2\pi^2\left( -1 \right)}{4\left( -2 \right)}\\
                                                            &= -\frac{\left( 2n+1 \right)^2\pi^2}{8}.
      \end{align*}
  \end{enumerate}
\end{solution}
\begin{solution}[21.2]

\end{solution}
\begin{solution}[21.6]

\end{solution}
\begin{solution}[21.8]

\end{solution}
\begin{solution}[21.10]

\end{solution}
\begin{solution}[21.12]

\end{solution}
\begin{solution}[21.16]

\end{solution}
\begin{solution}[21.17]

\end{solution}
\begin{solution}[21.22]

\end{solution}
\end{document}
