\documentclass[10pt]{mypackage}

% sans serif font:
%\usepackage{cmbright,sfmath,bbold}
%\renewcommand{\mathcal}{\mathtt}

%Euler:
\usepackage{newpxtext,eulerpx,eucal,eufrak}
\renewcommand*{\mathbb}[1]{\varmathbb{#1}}
\renewcommand*{\hbar}{\hslash}

\DeclareMathOperator{\Ln}{Ln}
\DeclareMathOperator{\Arg}{Arg}
\DeclareMathOperator{\Arctan}{Arctan}
\DeclareMathOperator{\Res}{Res}
\DeclareMathOperator{\res}{Res}
\usepackage{homework}

\pagestyle{fancy} %better headers
\fancyhf{}
\rhead{Avinash Iyer}
\lhead{}

\setcounter{secnumdepth}{0}

\begin{document}
\RaggedRight
\begin{solution}[21.1]\hfill
  \begin{enumerate}[(a)]
    \item Doing a partial fraction decomposition, we find
      \begin{align*}
        \frac{1}{\left( z-1 \right)\left( z+2 \right)} &= \frac{1}{3}\frac{1}{z-1} - \frac{1}{3}\frac{1}{z+2},
      \end{align*}
      giving giving a residue of $\frac{1}{3}$ at $z = 1$ and a residue of $-\frac{1}{3}$ at $z = -2$.
    \item Evaluating the residue at $z = 1$, we may use the cover-up method to find
      \begin{align*}
        \res\left[ f(z),1 \right] &= \frac{e^{2i}}{27}.
      \end{align*}
      To evaluate the residue at $z = -2$, we use the formula to calculate residues, giving
      \begin{align*}
        \res\left[ f(z),-2 \right] &= \frac{1}{2}\diff{^2}{z^2}\left( \frac{e^{2iz}}{z-1} \right)\bigr\vert_{z = -2}\\
                                   &= \frac{38}{27}e^{-4i}
      \end{align*}
    \item Note that $\sin(z)$ is a simple zero at $z = n\pi$. Therefore, we evaluate
      \begin{align*}
        \res\left[ f(z),n\pi \right] &= \left( -1 \right)^n e^{n\pi}.
      \end{align*}
    \item Using the Laurent series for $e^{1/z}$, we find that
      \begin{align*}
        e^{1/z} &= 1 + \frac{1}{z} + \frac{1}{2z^2} + \cdots,
      \end{align*}
      so that
      \begin{align*}
        \res\left[ f(z),0 \right] &= 1.
      \end{align*}
    \item Note that $e^{2z} + 1 = 0$ whenever $z = i(2n + 1)\pi/2$. These are all simple zeros, so we may evaluate
      \begin{align*}
        \res\left[ f(z),i\left( 2n + 1 \right)\pi/2 \right] &= \frac{-\left( 2n + 1 \right)^2\pi^2\left( -1 \right)}{4\left( -2 \right)}\\
                                                            &= -\frac{\left( 2n+1 \right)^2\pi^2}{8}.
      \end{align*}
  \end{enumerate}
\end{solution}
\begin{solution}[21.2]
  Since $\lim_{|z|\rightarrow\infty}f(z) = 0$, we may evaluate the residue at the pole at infinity by evaluating (and accounting for the sign flip)
  \begin{align*}
    \lim_{|z|\rightarrow\infty} zf(z) &= -1.
  \end{align*}
  Pairing with the (negative of the) residue at $z = 3$, we have
  \begin{align*}
    \oint_{|z|=2}f(z)\:dz &= 2\pi i \left( -1 + \frac{1}{6} \right)\\
                          &= -\frac{5\pi}{3}i.
  \end{align*}
\end{solution}
\begin{solution}[21.6]
  We start by factoring and using the cover-up method to obtain
  \begin{align*}
    f(z) &= \frac{4-2z}{\left( z-i \right)\left( z+i \right)\left( z-1 \right)^2}\\
         &= \frac{1}{\left( z-1 \right)^2} + \left( 1-\frac{1}{2}i \right)\frac{1}{z-i} + \left( 1 + \frac{1}{2}i \right)\frac{1}{z+i} + \frac{B}{z-1},
  \end{align*}
  where
  \begin{align*}
    B &= \res\left[ f(z),1 \right]\\
      &= \diff{}{z}\left( \frac{4-2z}{z^2+1} \right)\bigr\vert_{z=1}\\
      &= -\frac{3}{2}.
  \end{align*}
  Thus, we obtain the partial fraction decomposition of
  \begin{align*}
    f(z) &= \frac{1}{\left( z-1 \right)^2} -\frac{3}{2}\left( \frac{1}{z-1} \right) + \left( 1-\frac{1}{2}i \right)\frac{1}{z-i} + \left( 1 + \frac{1}{2}i \right) \frac{1}{z+i}.
  \end{align*}
\end{solution}
\begin{solution}[21.8]
  Closing the contour in the upper half plane, we evaluate the residues of the roots at $e^{i\pi/6}$ and $e^{i5\pi/6}$, giving
  \begin{align*}
    \res\left[ f(z),e^{i\pi/6} \right] &= \frac{1}{3\left( e^{i\pi/6} \right)^2}\\
                                       &= \frac{1}{3e^{i\pi/3}}\\
    \res\left[ f(z),e^{i5\pi/6} \right] &= \frac{1}{3\left( e^{i5\pi/6} \right)}\\
                                        &= -\frac{1}{3e^{i2\pi/3}}.
  \end{align*}
  Thus, calculating the integral, we have
  \begin{align*}
    \oint_{C}\frac{1}{z^3 - i}\:dz &= \frac{2i\pi}{3}.
  \end{align*}
\end{solution}
\begin{solution}[21.10]
  We start by doing a partial fraction decomposition of $f$, giving
  \begin{align*}
    f(z) &= \frac{1}{8}\left( \frac{1}{z-1} \right)  + \frac{1}{8}\left( \frac{1}{z-i} \right) - \frac{1}{8}\left( \frac{1}{z+i} \right)- \frac{1}{8}\left( \frac{1}{z+1} \right) - \frac{i}{8}\left( \frac{1}{\left( z+i \right)^2} \right) + \frac{i}{8}\left( \frac{1}{\left( z-i \right)^2} \right).
  \end{align*}
  \begin{enumerate}[(a)]
    \item If we evaluate the integral inside the circle $\left\vert z \right\vert < 1/2$, then there are no poles inside the contour, so the integral evaluates to $0$.
    \item If we evaluate the integral inside $\left\vert z \right\vert < 2$, then all the poles are inside the contour, so the integral once again evaluates to $0$.
    \item If we evaluate the integral inside the contour $\left\vert z-i \right\vert < 1$, only the pole at $z = i$ is inside the contour, meaning we have the integral of $\frac{\pi i}{4}$.
    \item If we evaluate the integral inside the elliptical contour, only the poles at $z = \pm 1$ are inside the contour, yet again meaning the integral evaluates to $0$.
  \end{enumerate}
\end{solution}
\begin{solution}[21.12]\hfill
  \begin{enumerate}[(a)]
    \item Since the integrand goes to zero at infinity, we may close the contour in the upper half-plane, giving two poles inside the contour at $e^{i\pi/4}$ and $e^{i3\pi/4}$. Evaluating the residues, we have
      \begin{align*}
        \res\left[ f(z),e^{i\pi/4} \right] &= \frac{i}{4e^{i3\pi/4}}\\
        \res\left[ f(z),e^{i3\pi/4} \right] &= \frac{-i}{4e^{i\pi/4}}
      \end{align*}
      Thus, we have
      \begin{align*}
        \int_{-\infty}^{\infty}\frac{x^2}{x^4 + 1}\:dx &= \oint_{C}\frac{z^2}{z^4 + 1}\:dz\\
                                                       &= 2\pi i \left( \frac{-\sqrt{2}i}{4} \right)\\
                                                       &= \frac{\sqrt{2}}{2}\pi.
      \end{align*}
    \item The integrand goes to zero at infinity, so we may close the contour in the upper half-plane, giving the residues of $i$ and $2i$. Evaluating these residues, we have
      \begin{align*}
        \res\left[ f(z),i \right] &= \diff{}{z}\left( \frac{1}{z^2 + 4} \right)\bigr\vert_{z=i}\\
                                  &= -\frac{2z}{\left( z^2 + 4 \right)^2}\bigr\vert_{z=i}\\
                                  &= -\frac{2i}{9}\\
        \res\left[ f(z),2i \right] &= \frac{1}{\left( z^2 + 1 \right)^2\left( z+2i \right)}\bigr\vert_{z=2i}\\
                                   &= -\frac{i}{36}.
      \end{align*}
      Thus, we have the integral of
      \begin{align*}
        \int_{-\infty}^{\infty} \frac{1}{\left( x^2 + 1 \right)^2\left( x^2 + 4 \right)}\:dx &= \oint_{C}\frac{1}{\left( z^2 + 1 \right)^2\left( z^2 + 4 \right)}\:dz\\
                                                                                             &= 2\pi i \left( -\frac{2i}{9} - \frac{i}{36} \right)\\
                                                                                             &= \frac{\pi}{2}.
      \end{align*}
    \item Using the cubic factorization, we have
      \begin{align*}
        \int_{-\infty}^{\infty} \frac{x^2}{x^4 + x^2 + 1}\:dx &= \int_{-\infty}^{\infty} \frac{x^2\left( x^2 - 1 \right)}{x^6 - 1}\:dx,
      \end{align*}
      with removable discontinuity at $x = \pm 1$. Furthermore, since the integrand goes to zero at infinity, we may close the contour in the upper half-plane, giving residues at $z = e^{i\pi/3}$ and $z = e^{i2\pi/3}$. Evaluating at these residues using the cover-up method, we find
      \begin{align*}
        \res\left[ f(z),e^{i\pi/3} \right] &= \frac{e^{i2\pi/3}}{\left( e^{i\pi/3}-e^{i2\pi/3} \right)\left( e^{i\pi/3} - e^{i4\pi/3} \right) \left( e^{i\pi/3} -  e^{i5\pi/3}\right)}\\
                                           &= \frac{1}{2\sqrt{3}}e^{-i\pi/6}\\
        \res\left[ f(z),e^{i2\pi/3} \right] &= \frac{e^{i4\pi/3}}{\left( e^{i2\pi/3} - e^{i\pi/3} \right)\left( e^{i2\pi/3} - e^{i4\pi/3} \right)\left( e^{i2\pi/3} - e^{i5\pi/3} \right)}\\
                                            &= -\frac{1}{2\sqrt{3}}e^{i\pi/6}.
      \end{align*}
      Therefore, our integral gives
      \begin{align*}
        \int_{-\infty}^{\infty} \frac{x^2}{x^4 + x^2 + 1}\:dx &= \oint_{C}\frac{z^2}{z^4 + z^2 + 1}\:dz\\
                                                              &= 2\pi i \left( \frac{1}{2\sqrt{3}}\left( -2i\sin\left( \pi/6 \right) \right) \right)\\
                                                              &= \frac{\pi}{\sqrt{3}}.
      \end{align*}
    \item We have that
      \begin{align*}
        \int_{-\infty}^{\infty} \frac{\cos\left( \pi x \right)}{x^2 + 1}\:dx &= \frac{1}{2}\re\left( \oint_{C}\frac{e^{i\pi z}}{\left( z-i \right)\left( z+i \right)}\:dz \right).
      \end{align*}
      Since $\pi > 0$, we close our contour in the upper half-plane, giving
      \begin{align*}
        \oint_{C}\frac{e^{i\pi z}}{\left( z-i \right)\left( z+i \right)}\:dz &= 2\pi i\res\left[ f(z),i \right]\\
                                                                             &= 2\pi i \left( \frac{e^{-\pi}}{2i} \right)\\
                                                                             &= \pi e^{-\pi}.
      \end{align*}
  \end{enumerate}
\end{solution}
\begin{solution}[21.16]

\end{solution}
\begin{solution}[21.17]
  We evaluate
  \begin{align*}
    I &= \int_{0}^{\infty} \frac{x}{1 + x^4}\:dx
  \end{align*}
  by considering 
  \begin{align*}
    I' &= \oint_{C}\frac{z}{1 + z^4}\:dz,
  \end{align*}
  where $C$ is the contour that goes to a large radius $r$ and returns along the imaginary axis. The integral along this component is equal to $-e^{i\pi/2} I$, giving $I' = \left( 1 - e^{i\pi/2} \right)I$.
  \begin{align*}
    \int_{0}^{\infty} \frac{x}{1 + x^4}\:dx &= \frac{1}{1 - i}\oint_{C}\frac{z}{1 + z^{4}}\:dz\\
                                            &= \left( \frac{1}{1 - i} \right)\left( 2\pi i \res\left[ f(z),e^{i\pi/4} \right] \right)\\
                                            &= 2\pi i \left( \frac{e^{i\pi/4}}{\left( e^{i\pi/4} - e^{i3\pi/4} \right)\left( e^{i\pi/4} - e^{i5\pi/4} \right)\left( e^{i\pi/4} - e^{i7\pi/4} \right)\left( 1 - i \right)} \right)\\
                                            &= \frac{\pi}{4}.
  \end{align*}
\end{solution}
\begin{solution}[21.22]

\end{solution}
\end{document}
