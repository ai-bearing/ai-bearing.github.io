\documentclass[10pt]{mypackage}

% sans serif font:
%\usepackage{cmbright,sfmath,bbold}
%\renewcommand{\mathcal}{\mathtt}

%Euler:
\usepackage{newpxtext,eulerpx,eucal,eufrak}
\renewcommand*{\mathbb}[1]{\varmathbb{#1}}
\renewcommand*{\hbar}{\hslash}

%\renewcommand{\mathbb}{\mathds}
%\usepackage{homework}

\pagestyle{fancy} %better headers
\fancyhf{}
\rhead{Avinash Iyer}
\lhead{Assignment 6}

\setcounter{secnumdepth}{0}

\begin{document}
\RaggedRight
\begin{solution}[29.5]\hfill
  \begin{enumerate}[(a)]
    \item We have
      \begin{align*}
        \left( \vec{w}\cdot \doublevec{T} \right)_{k} &= \sum_{i,j,k}w_{i}T_{jk}\delta_{ij}\\
                                   &= \sum_{i,k}w_iT_{ik},
      \end{align*}
      which is a first-rank tensor.
    \item Since $\vec{w}\cdot \doublevec{T}$ is a first-rank tensor, and we are taking the dot product of two first rank tensors the expression $\vec{w}\cdot \doublevec{T}\cdot \vec{v}$ is a scalar (or rank zero tensor).
    \item We have
      \begin{align*}
        \doublevec{T}\cdot \doublevec{U} &= \left( \sum_{i,j}T_{ij}e_i\otimes e_j \right)\cdot \left( \sum_{k,\ell}U_{k\ell} e_{k}\otimes e_{\ell} \right)\\
                                         &= \sum_{i,j,k,\ell}T_{ij}U_{k \ell} \left( e_k\cdot e_i \right)\left( e_j\cdot e_{\ell} \right),
      \end{align*}
      which is a scalar.
    \item The expression $\doublevec{T}\vec{v}$ expresses the operation of
      \begin{align*}
        \doublevec{T} &= \sum_{i,j}T_{ij}e_i\otimes e_j
      \end{align*}
      on
      \begin{align*}
        \vec{v} &= \sum_{i}v_ie_i,
      \end{align*}
      meaning that $\doublevec{T}\vec{v}$ is a vector.
    \item The expression $\doublevec{T}\doublevec{U}$ is a composition of two linear maps on $V\otimes V$, so it is a rank $2$ tensor (or another linear map on $V\otimes V$).
  \end{enumerate}
\end{solution}
\begin{solution}[29.7]
  We have $2^{4}$ or $16$ components in $A_{ijkl}$.
\end{solution}
\begin{solution}[29.10]

\end{solution}
\begin{solution}[29.11]\hfill
\end{solution}
\begin{enumerate}[(a)]

\end{enumerate}
\begin{solution}[29.12]

\end{solution}
\begin{solution}[29.14]

\end{solution}
\begin{solution}[29.23]

\end{solution}
\begin{solution}[29.24]

\end{solution}
\begin{solution}[29.25]

\end{solution}

\end{document}
