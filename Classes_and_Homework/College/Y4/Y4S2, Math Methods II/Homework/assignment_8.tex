\documentclass[10pt]{mypackage}

% sans serif font:
%\usepackage{cmbright,sfmath,bbold}
%\renewcommand{\mathcal}{\mathtt}

%Euler:
\usepackage{newpxtext,eulerpx,eucal,eufrak}
\renewcommand*{\mathbb}[1]{\varmathbb{#1}}
\renewcommand*{\hbar}{\hslash}

%\renewcommand{\mathbb}{\mathds}
\usepackage{homework}
%\usepackage{exposition}

\pagestyle{fancy} %better headers
\fancyhf{}
\rhead{Avinash Iyer}
\lhead{Assignment 8}

\setcounter{secnumdepth}{0}

\begin{document}
\RaggedRight
\begin{solution}[32.20]
  We start by taking the recurrence relation
  \begin{align*}
    \left( 1-x^2 \right)P_n' &= -nxP_n + nP_{n-1}.\label{eq:derivative_recursion}\tag{$\ast$}
  \end{align*}
  Differentiating, this gives
  \begin{align*}
    \left( 1-x^2 \right)P_n'' - 2xP_n' &= n\left( -P_n - xP_n' + P_{n-1}' \right). 
  \end{align*}
  We seek to show that
  \begin{align*}
    -xP_n' + P_{n-1}' &= -nP_n.
  \end{align*}
  At this point, I ran out of board space to deal with the generating functions and their ensuing mess of partial derivatives.
\end{solution}
\begin{solution}[32.21]
  Using $dv=P_m'(x)$, we integrate by parts to get
  \begin{align*}
    \int_{-1}^{1} \left( 1-x^2 \right)P_n'(x)P_m'(x)\:dx &= P_m(x)P_n'(x)\left( 1-x^2 \right)\biggr\vert_{-1}^{1} - \int_{-1}^{1} \diff{}{x}\left( \left( 1-x^2 \right)P_n'(x) \right)P_m(x)\:dx\\
                                                         &= - \int_{-1}^{1} \left( \left( 1-x^2 \right)P_n''(x) - 2xP_n'(x) \right)P_{m}(x)\:dx\\
                                                         &= n\left( n+1 \right) \int_{-1}^{1} P_n(x)P_m(x)\:dx\\
                                                         &= \frac{2n\left( n+1 \right)}{2n+1} \delta_{mn}.
  \end{align*}
\end{solution}
\begin{solution}[32.23]
  Upon taking $m$ derivatives of Legendre's equation, and using the Leibniz rule for differentiation, we get
  \begin{align*}
    \left( 1-x^2 \right)\diff{^{m+2}P_{\ell}}{x^{m+2}} - 2x\left( m+1 \right)\diff{^{m+1}P_{\ell}}{x^{m+2}} + \left( \left( \ell \right)\left( \ell + 1 \right) - \left( m\left( m-1 \right) + 2m \right) \right)\diff{^{m}P_{\ell}}{x^{m}} &= 0.
  \end{align*}
  Rewriting $u(x) = \diff{^mP_{\ell}}{x^{m}}$, we obtain
  \begin{align*}
    0 &= \left( 1-x^2 \right)\diff{^2u}{x^2} -2x\left( m+1 \right)\diff{u}{x} + \left( \ell\left( \ell + 1 \right)-m^2-m \right)u(x).
  \end{align*}
  Setting $u(x) = \left( 1-x^2 \right)^{-m/2} v(x)$, we find
  \begin{align*}
    \diff{u}{x} &= \left( 1-x^2 \right)^{-m/2}\diff{v}{x} + mxv(x)\left( 1-x^2 \right)^{-m/2 - 1}\\
                &= \left( 1-x^2 \right)^{-m/2}\left( \diff{v}{x} + \frac{mxv(x)}{1-x^2} \right)\\
    \diff{^2u}{x^2} &= -mx\left( 1-x^2 \right)^{-m/2-1}\left( \diff{v}{x} + \frac{mxv(x)}{1-x^2} \right) + \left( 1-x^2 \right)^{-m/2}\left( \diff{^2v}{x^2} + \frac{mv(x)}{1-x^2} + \frac{mx}{1-x^2}\diff{v}{x} + \frac{2mx^2v(x)}{\left( 1-x^2 \right)^2} \right)\\
                    &= \left( 1-x^2 \right)^{-m/2}\left( \diff{^2v}{x^2} + \frac{2mx}{1-x^2}\diff{v}{x} + \frac{mv(x)}{1-x^2} + \frac{2mx^2v(x)}{\left( 1-x^2 \right)^2} \right).
  \end{align*}
  Substituting, we have the equation
  \begin{align*}
    0 &= \left( 1-x^2 \right)\left( 1-x^2 \right)^{-m/2}\left( \diff{^2v}{x^2} + \frac{2mx}{1-x^2}\diff{v}{x} + \frac{mv(x)}{1-x^2} + \frac{2mx^2v(x)}{\left( 1-x^2 \right)^2} \right)\\
      &- 2x(m+1) \left( \left( 1-x^2 \right)^{-m/2}\left( \diff{v}{x} + \frac{mxv(x)}{1-x^2} \right) \right)\\
      &+\left( \ell\left( \ell + 1 \right)-m^2-m \right)\left( 1-x^2 \right)^{-m/2}v(x),
  \end{align*}
  which after much more tedious algebra, yields
  \begin{align*}
    0 &= \diff{^2v}{x^2} - 2x \diff{v}{x} + \left( \left( \ell \right)\left( \ell + 1 \right) - \frac{m^2}{1-x^2} \right)v(x),
  \end{align*}
  so $v$ satisfies the differential equation. Thus, we have
  \begin{align*}
    v(x) &= \left( 1-x^2 \right)^{m/2}\diff{^{m}P_{\ell}}{x^{m}}.
  \end{align*}
  
\end{solution}
\begin{solution}[35.4]
  Using the expression
  \begin{align*}
    J_n\left( x \right) &= \frac{1}{2\pi}\int_{-\pi}^{\pi} e^{ix\sin\left( \gamma \right) - in\gamma}\:d\gamma\\
                        &= \frac{1}{2\pi} \int_{-\pi}^{\pi} e^{ix\sin\left( \gamma \right)}e^{-in\gamma}\:d\gamma,
  \end{align*}
  we expand the first term in a Taylor series, giving
  \begin{align*}
    J_n\left( x \right) &= \frac{1}{2\pi}\sum_{k=0}^{\infty} \frac{i^{k}x^{k}}{k!} \int_{-\pi}^{\pi} \sin^{k}\left( \gamma \right)e^{-in\gamma}\:d\gamma.
  \end{align*}
  Now, $k$ has to be even (else we have an odd integrand over a symmetric interval).
\end{solution}
\begin{solution}[35.5]
  Differentiating,
  \begin{align*}
    \diff{J_0}{x} &= \frac{1}{2\pi} \int_{-\pi}^{\pi} \pd{}{x}\left( e^{ix\sin\left( \gamma \right)} \right)\:d\gamma\\
                  &= \frac{1}{2\pi} \int_{-\pi}^{\pi} \left( i\sin\left( \gamma \right) \right)e^{ix\sin\left( \gamma \right)}\:d\gamma\\
                  &= \frac{1}{2\pi} \int_{-\pi}^{\pi} i\left( \frac{1}{2i}\left( e^{i\gamma} - e^{-i\gamma} \right) \right)\:d\gamma\\
                  &= \frac{1}{2\pi} \int_{-\pi}^{\pi} \frac{1}{2}e^{ix\sin\left( \gamma \right) + i\gamma} - \frac{1}{2}e^{ix\sin\left( \gamma \right) - i\gamma}\:d\gamma\\
                  &= \frac{1}{2\pi} \int_{-\pi}^{\pi} \frac{1}{2}\left( \cos\left( x\sin\left( \gamma \right) + i\gamma \right) + i\sin\left( x\sin\left( \gamma \right) + i\gamma \right) - \left( \cos\left( x\sin\left( \gamma \right) - i\gamma \right) + i\sin\left( x\sin\left( \gamma \right) - i\gamma \right) \right) \right)\:d\gamma
                  \intertext{and with more tedious algebra, we obtain}
                  &= -\frac{1}{\pi} \int_{0}^{\pi} \cos\left( x\sin\left( \gamma \right) - \gamma \right) \:d\gamma\\
                  &= -J_1(x).
  \end{align*}
  Evaluating
  \begin{align*}
    \diff{}{x}\left( xJ_1 \right) &= J_1 + x\diff{J_1}{x},
  \end{align*}
  we take
  \begin{align*}
    \diff{}{x}\left( xJ_1 \right) &= \frac{1}{\pi}\int_{0}^{\pi} \cos\left( x\sin\left( \gamma \right) - \gamma \right) - x\sin\left( \gamma \right)\sin\left( x\sin\left( \gamma \right) - \gamma \right)\:d\gamma\\
                                  &= \frac{1}{\pi} \int_{0}^{\pi} \cos\left( x\sin\left( \gamma \right) \right)\cos\left( \gamma \right) + \sin\left( x\sin\left( \gamma \right) \right)\sin\left( \gamma \right) - x\sin\left( \gamma \right)\sin\left( x\sin\left( \gamma \right)-\gamma \right)\:d\gamma\\
                                  &= \frac{1}{\pi} \int_{0}^{\pi} \cos\left( \gamma \right)\cos\left( x\sin\left( \gamma \right) \right) + \sin\left( \gamma \right)\sin\left( x\sin\left( \gamma \right) \right) - x\sin\left( \gamma \right)\left( \sin\left( x\sin\left( \gamma \right) \right)\cos\left( \gamma \right) - \sin\left( \gamma \right)\cos\left( x\sin\left( \gamma \right) \right) \right)\:d\gamma\\
                                  &= \frac{1}{\pi} \int_{0}^{\pi} x\cos\left( x\sin\left( \gamma \right) \right)\:d\gamma\\
                                  &= xJ_0.
  \end{align*}
\end{solution}
\begin{solution}[35.7]
  Solving
  \begin{align*}
    x^2\diff{^2u}{x^2} + x\diff{u}{x} + \left( x^2 - n^2 \right)u(x) &= 0,
  \end{align*}
  we plug in the expression for $J_n(x)$ to get
  \begin{align*}
    x^2\diff{^2u}{x^2} + x\diff{u}{x} + \left( x^2 - n^2 \right)u(x) &=x^2\left( \sum_{m=0}^{\infty} \frac{1}{2^{2m + n}}\left( 2m+n-1 \right)\left( 2m+n \right)\frac{\left( -1 \right)^{m}}{m!\left( m+n! \right)} x^{2m + n - 2}  \right)\\
                                                                     &+ x \left( \sum_{m=0}^{\infty}\frac{1}{2^{2m + n}}\left( 2m + n \right) \frac{\left( -1 \right)^{m}}{m!\left( m+n \right)!}x^{2m + n - 1} \right)\\
                                                                     &+ \sum_{m=0}^{\infty} \frac{1}{2^{2m + n}} \frac{\left( -1 \right)^{m}}{m!\left( m+n \right)!}x^{2m + n + 2}\\
                                                                     & - \sum_{m=0}^{\infty}\frac{n^2}{2^{2m + n}}\frac{\left( -1 \right)^{m}}{m!\left( m+n \right)!}x^{2m + n}\\
                                                                     &= \sum_{m=0}^{\infty}\frac{1}{2^{2m + n}}\frac{\left( -1 \right)^{m}}{m!\left( m+n! \right)}\left( x^{2m + n} \right)\left( \left( 2m + n - 1 \right)\left( 2m + n \right) + 2m + n + x^2 - n^2 \right)\\
                                                                     &= \sum_{m=0}^{\infty}\frac{\left( -1 \right)^{m}}{2^{2m + n}m!\left( m+n! \right)}x^{2m + n}\left( x^2 + 4m^2 + 4mn \right)
  \end{align*}
  From here, I'm not sure how to manipulate this series to get $0$ as the final answer.
\end{solution}
\begin{solution}[35.8]\hfill
  \begin{enumerate}[(a)]
    \item We have
      \begin{align*}
        e^{ix\sin\left( \phi \right)} &= \sum_{n=-\infty}^{\infty}c_ne^{in\phi},
      \end{align*}
      where
      \begin{align*}
        c_n &= \frac{1}{2\pi}\int_{-\pi}^{\pi} e^{ix\sin\left( \phi \right)}e^{-in\phi}\:d\phi\\
            &= J_n\left( x \right).
      \end{align*}
    \item Splitting into real and imaginary parts, we have
      \begin{align*}
        e^{ix\sin\left( \phi \right)} &= \cos\left( x\sin\left( \phi \right) \right) + i\sin\left( x\sin\left( \phi \right) \right),
      \end{align*}
      so that
      \begin{align*}
        e^{ix\sin\left( \phi \right)} &= \sum_{n=-\infty}^{\infty}c_ne^{in\phi}\\
                                      &= \sum_{n=-\infty}^{\infty} J_n\left( x \right)\left( \cos\left( n\phi \right) + i\sin\left( n\phi \right) \right)\\
                                      &= \sum_{n=-\infty}^{\infty}J_n\left( x \right)\cos\left( n\phi \right) + i\sum_{n=-\infty}^{\infty}J_n\left( x \right)\sin\left( n\phi \right).
      \end{align*}
      Equating real and imaginary parts gives the desired result.
    \item We use the angle summation identity to get
      \begin{align*}
        A\cos\left( \omega_c t \right)\cos\left( \beta\sin\left( \omega_m t \right) \right) - A\sin\left( \omega_c t \right)\sin\left( \beta\sin\left( \omega_m t \right) \right) &= A\cos\left( \omega_c t \right)\sum_{n=-\infty}^{\infty}J_n\left( \beta \right)\cos\left( n\omega_m t \right)\\
                                                                                                                                                                                                &- A\sin\left( \omega_c t \right)\sum_{n=-\infty}^{\infty}J_n\left( \beta \right)\sin\left( n\omega_m t \right)\\
                                                                                                                                                                                                &= \sum_{n=-\infty}^{\infty} J_n\left( \beta \right)\cos\left( \omega_c t + n\omega_m t \right).
      \end{align*}
  \end{enumerate}
\end{solution}
\begin{solution}[35.10]
  \begin{align*}
    P_{3}\left( x \right) &= \frac{1}{2}\left( 5x^3 - 3x \right)\\
    P_{3,1}\left( x \right) &= \frac{1}{2}\left( 1-x^2 \right)^{1/2}\left( 15x^2 - 3 \right)\\
    P_{3,-1}\left( x \right) &= -\frac{1}{6}\left( 1-x^2 \right)^{1/2}\left( 15x^2 - 3 \right)\\
    P_{3,2} \left( x \right)  &= 15x\left( 1-x^2 \right)\\
    P_{3,-2}\left( x \right) &= \frac{1}{8}x\left( 1-x^2 \right).\\
    P_{3,3}\left( x \right) &= 15\left( 1-x^2 \right)^{3/2}\\
    P_{3,-3}\left( x \right) &= -\frac{1}{48}\left( 1-x^2 \right)^{3/2}.
  \end{align*}
\end{solution}
\begin{solution}[35.11]
  \begin{align*}
    Y_{\ell,m}\left( \pi-\theta,\phi + \pi \right)  &= \left( -1 \right)^{m}\sqrt{\frac{2\ell + 1}{4\pi}}\sqrt{\frac{\left( \ell-m \right)!}{\left( \ell + m \right)!}}P_{\ell,m}\left( \cos\left( \pi-\theta \right) \right)e^{im\left( \phi + \pi \right)}\\
                                                    &= \left( -1 \right)^{m}\sqrt{\frac{2\ell + 1}{4\pi}}\sqrt{\frac{\left( \ell - m \right)!}{\left( \ell + m \right)!}}P_{\ell,m}\left( -\cos\left( \theta \right) \right)e^{im\phi}\left( -1 \right)^{m}\\
                                                    &= \left( -1 \right)^{m}\sqrt{\frac{2\ell + 1}{4\pi}}\sqrt{\frac{\left( \ell - m \right)!}{\left( \ell + m \right)!}}\left( -1 \right)^{\ell -m}P_{\ell,m}\left( \cos\left( \theta \right) \right)\left( -1 \right)^{m}e^{im\phi}\\
                                                    &= \left( -1 \right)^{\ell}\left( \left( -1 \right)^{m}\sqrt{\frac{2\ell + 1}{4\pi}}\sqrt{\frac{\left( \ell - m \right)!}{\left( \ell + m \right)!}}P_{\ell,m}\left( \cos\left( \theta \right) \right)e^{im\phi} \right)\\
                                                    &= \left( -1 \right)^{\ell}Y_{\ell,m}\left( \theta,\phi \right).
  \end{align*}
  
\end{solution}
\begin{solution}[35.12]
  We have
  \begin{align*}
    Y_{\ell,0}\left( \hat{n} \right) &= \left( -1 \right)^{0}\sqrt{\frac{2\ell + 1}{4\pi}}\sqrt{\frac{\ell!}{\ell!}}P_{\ell,0}\left( \cos\left( \theta \right) \right)e^{i\left( 0 \right)\phi}\\
                                     &= \sqrt{\frac{2\ell + 1}{4\pi}}P_{\ell}\left( \cos\left( \theta \right) \right).
  \end{align*}
  Furthermore, since $P_{\ell,m}\left( 1 \right) = \delta_{0,m}$, we have
  \begin{align*}
    Y_{\ell,m}\left( 0,\phi \right) &= \left( -1 \right)^{m}\sqrt{\frac{2\ell + 1}{4\pi}}\sqrt{\frac{\left( \ell-m \right)!}{\left( \ell + m \right)!}}P_{\ell,m}\left( 1 \right)e^{im\phi}\\
                                    &= \sqrt{\frac{2\ell + 1}{4\pi}}\delta_{0,m}.
  \end{align*}
  
\end{solution}
\begin{solution}[35.16]
  Using the addition theorem, where $\hat{a} = \hat{b} = \hat{n}$, we get
  \begin{align*}
    \sum_{m=-\ell}^{\ell}\left\vert Y_{\ell,m}\left( \hat{n} \right) \right\vert^2 &= \frac{2\ell + 1}{4\pi}P_{\ell}\left( 1 \right)\\
                                                                                   &= \frac{2\ell + 1}{4\pi}.
  \end{align*}
\end{solution}
\begin{solution}[35.17 (c)]

\end{solution}
\begin{solution}[35.21]

\end{solution}
\begin{solution}[35.25]

\end{solution}
\end{document}
