\documentclass[10pt]{mypackage}

% sans serif font:
%\usepackage{cmbright,sfmath,bbold}
%\renewcommand{\mathcal}{\mathtt}

%Euler:
\usepackage{newpxtext,eulerpx,eucal,eufrak}
\renewcommand*{\mathbb}[1]{\varmathbb{#1}}
\renewcommand*{\hbar}{\hslash}

\usepackage{homework}

\pagestyle{fancy} %better headers
\fancyhf{}
\rhead{Avinash Iyer}
\lhead{Math Methods II: Assignment 5}

\setcounter{secnumdepth}{0}

\begin{document}
\RaggedRight
\begin{solution}[21.20a]
  There are poles at $2,2e^{2i\pi/3},2e^{4i\pi/3}$, and since the integrand falls off with $1/r^2$, we may close the contour in the upper half-plane, with one of the poles on the contour. Thus, we get the solution
  \begin{align*}
    \oint_{C} \frac{z}{z^3 - 8}\:dz &= 2\pi i \res\left[ f(z),2e^{2i\pi/3} \right] + \pi i \res\left[ f(z),2 \right]\\
                                    &= \frac{\pi}{2\sqrt{3}}.
  \end{align*}
\end{solution}
\begin{solution}[21.25a]
  Evaluating
  \begin{align*}
    \sum_{n=-\infty}^{\infty}\frac{1}{1 + n^2} &= -\sum_{j} \res\left[ \frac{\pi\cot\left( \pi z \right)}{1 + z^2}, z_j \right]\\
                                               &= 2\pi\tanh\left(\pi\right).
  \end{align*}
\end{solution}
\begin{solution}[21.26]

\end{solution}
\begin{solution}[21.28]

\end{solution}
\begin{solution}[21.31]

\end{solution}
\begin{solution}[21.32]

\end{solution}
\begin{solution}[21.33]

\end{solution}
\begin{solution}[22.7]

\end{solution}

\end{document}
