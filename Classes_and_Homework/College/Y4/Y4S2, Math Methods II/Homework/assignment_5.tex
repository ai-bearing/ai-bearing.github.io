\documentclass[10pt]{mypackage}

% sans serif font:
%\usepackage{cmbright,sfmath,bbold}
%\renewcommand{\mathcal}{\mathtt}

%Euler:
\usepackage{newpxtext,eulerpx,eucal,eufrak}
\renewcommand*{\mathbb}[1]{\varmathbb{#1}}
\renewcommand*{\hbar}{\hslash}
\DeclareMathOperator{\res}{Res}

\usepackage{homework}

\pagestyle{fancy} %better headers
\fancyhf{}
\rhead{Avinash Iyer}
\lhead{Math Methods II: Assignment 5}

\setcounter{secnumdepth}{0}

\begin{document}
\RaggedRight
\begin{solution}[21.20a]
  There are poles at $2,2e^{2i\pi/3},2e^{4i\pi/3}$, and since the integrand falls off with $1/r^2$, we may close the contour in the upper half-plane, with one of the poles on the contour. Thus, we get the solution
  \begin{align*}
    \oint_{C} \frac{z}{z^3 - 8}\:dz &= 2\pi i \res\left[ f(z),2e^{2i\pi/3} \right] + \pi i \res\left[ f(z),2 \right]\\
                                    &= \frac{\pi}{2\sqrt{3}}.
  \end{align*}
\end{solution}
\begin{solution}[21.25a]
  Evaluating
  \begin{align*}
    \sum_{n=-\infty}^{\infty}\frac{1}{1 + n^2} &= -\sum_{j} \res\left[ \frac{\pi\cot\left( \pi z \right)}{1 + z^2}, z_j \right]\\
                                               &= 2\pi\tanh\left(\pi\right).
  \end{align*}
\end{solution}
\begin{solution}[21.26]
  I don't know how to do this problem.
\end{solution}
\begin{solution}[21.28]
  If $t > 0$ is fixed, then we close the integral in the lower half-plane, which has the pole at $\omega = -i\ve$ with residue $-2\pi i e^{-\ve t}$. Taking $\ve \rightarrow 0$, we get $1$ for $\frac{i}{2\pi}\int_{-\infty}^{\infty} \frac{e^{-i\omega t}}{\omega + i\ve}\:d\omega$.\newline

  If $t < 0$ is fixed, then we close the integral in the upper half-plane, which has no poles, so the integral $\frac{i}{2\pi}\int_{-\infty}^{\infty} \frac{e^{-i\omega t}}{\omega + i\ve}\:d\omega$ is equal to zero. A similar case holds for $t = 0$.
\end{solution}
\begin{solution}[21.31]\hfill
  \begin{enumerate}[(a)]
    \item If $z = a$ is an $n$th order zero, then $w(z) = \left( z-a \right)^n g(z)$ for some $g(z)\neq 0$ on $z = a$. This gives
      \begin{align*}
        \diff{w}{z} &= n\left( z-a \right)^{n-1}g(z) + g'(z)\left( z-a \right)^n
      \end{align*}
      Note that $g'(z)\neq 0$ on $z = a$.
      \begin{align*}
        \frac{w'(z)}{w(z)} &= \frac{n}{\left( z-a \right)} + \frac{g'(z)}{g(z)}.
      \end{align*}
      Thus, we have a residue of $n$ at $z = a$.
    \item If $z= a$ is a $p$th order pole, then $w(z) = \left( z-a \right)^{-p}g(z)$ for some $g(z)\neq 0$ at $z = p$. This gives
      \begin{align*}
        \diff{w}{z} &= -p\left( z-a \right)^{-p-1}g(z) + \left( z-a \right)^{-p}g'(z)\\
        \frac{w'(z)}{w(z)} &= \frac{-p}{\left( z-a \right)} + \frac{g'(z)}{g(z)},
      \end{align*}
      so we have a residue of $-p$ at $z = a$.
    \item We have
      \begin{align*}
        \oint \frac{w'(z)}{w(z)}\:dz &= 2\pi i \sum_{i}\res\left[ \frac{w'(z)}{w(z)},z_i \right]\\
                                     &= 2\pi i \left( \sum_{i}n_i - \sum_{i}p_i \right).
      \end{align*}
  \end{enumerate}
\end{solution}
\begin{solution}[21.32]\hfill
  \begin{enumerate}[(a)]
    \item Inside $\left\vert z \right\vert = \frac{1}{2}$, there is an order $3$ zero at $z = 0$, so the integral evaluates to $6\pi i$.
    \item Inside $\left\vert z \right\vert = 2$, there is a pole of order $3$ at $z = 1$, a zero of order $1$ at $z = -1$, and a zero of order $3$ at $z = 0$. Thus, the integral evaluates to $2\pi i$.
    \item Inside $\left\vert z \right\vert = 9/2$, there is an order $2$ pole at $z  =-3$, an order $3$ pole at $z = 1$, an order $3$ zero at $z = 0$, an order $1$ zero at $z = -1$, and an order $1$ zero at $z = 4i$. Thus, the integral evaluates to $0$.
  \end{enumerate}
\end{solution}
\begin{solution}[21.33]\hfill
  \begin{enumerate}[(a)]
    \item The phase change in $\arg(w)$ is equal to the amount of times that the contour crosses the branch cut along $(-\infty,0]$. This gives
    \begin{align*}
      \oint_{C}\frac{w'(z)}{w(z)}\:dz &= \oint_{C}\diff{}{z}\left( \ln\left( w(z) \right) \right)\:dz\\
                                      &= i\left( \text{\# of times $C$ crosses branch cut} \right)\\
                                      &= i\Delta_{C}\arg\left( w \right).
    \end{align*}
    In the case of $w(z) = \frac{1}{z}$, this yields the winding number of $C$.
    \item If $w$ is nonvanishing on $C$, then since $C$ crosses the branch cut three times, we have that
      \begin{align*}
        6\pi i &= 2\pi i \sum_{i}n_i,
      \end{align*}
      so there are three orders worth of zeros of $w$.
  \end{enumerate}
\end{solution}
\begin{solution}[22.7]
  Due to poor time management, I am unable to complete this problem with sufficient attention investment.
\end{solution}

\end{document}
