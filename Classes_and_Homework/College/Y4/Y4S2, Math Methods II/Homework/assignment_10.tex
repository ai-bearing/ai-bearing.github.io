\documentclass[10pt]{mypackage}

% sans serif font:
%\usepackage{cmbright,sfmath,bbold}
%\renewcommand{\mathcal}{\mathtt}

%Euler:
\usepackage{newpxtext,eulerpx,eucal,eufrak}
\renewcommand*{\mathbb}[1]{\varmathbb{#1}}
\renewcommand*{\hbar}{\hslash}

%\renewcommand{\mathbb}{\mathds}
\usepackage{homework}
%\usepackage{exposition}

\pagestyle{fancy} %better headers
\fancyhf{}
\rhead{Avinash Iyer}
\lhead{Assignment 10}

\setcounter{secnumdepth}{0}

\begin{document}
\RaggedRight
\begin{solution}[40.7]
  We have
  \begin{align*}
    \braket{\psi}{\mathcal{L}\phi} &= \int_{a}^{b} \overline{\psi(x)}\left( \alpha(x)\diff{^2\phi}{x^2} + \beta(x)\diff{\phi}{x} + \gamma(x)\phi(x) \right)\:dx\\
                                   &= \int_{a}^{b} \overline{\psi(x)}\alpha(x)\diff{^2\phi}{x^2}\:dx + \int_{a}^{b} \overline{\psi(x)}\beta(x)\diff{\phi}{x}\:dx + \int_{a}^{b} \overline{\psi(x)}\gamma(x)\phi(x)\:dx.
  \end{align*}
  We evaluate these integrals separately. Assuming that $\alpha,\beta,\gamma$ are real-valued, we have
  \begin{align*}
    \int_{a}^{b} \overline{\psi(x)}\alpha(x)\diff{^2\phi}{x^2}\:dx &= \diff{\phi}{x} \overline{\psi(x)}\alpha(x)\biggr\vert_{a}^{b} - \int_{a}^{b} \left( \diff{\alpha}{x} \overline{\psi(x)} + \overline{\diff{\psi}{x}} \alpha(x) \right)\diff{\phi}{x}\:dx\\
                                                                   &= \underbrace{\left( \diff{\phi}{x}\alpha(x) \overline{\psi(x)} - \phi(x)\left( \diff{\alpha}{x} \overline{\psi(x)} + \overline{\diff{\psi}{x}} \alpha(x) \right) \right) \biggr\vert_{a}^{b}}_{S_1}\\
                                                                   &+ \int_{a}^{b} \overline{\left( \alpha(x)\diff{^2}{x^2} + 2\diff{\alpha}{x}\diff{}{x} + \diff{^2\alpha}{x^2}\right)\psi(x)}\phi(x)\:dx.\\
    \int_{a}^{b} \overline{\psi(x)}\beta(x) \diff{\phi}{x}\:dx &= \underbrace{\left( \phi(x) \beta(x) \overline{\psi(x)} \right)\biggr\vert_{a}^{b}}_{S_2} - \int_{a}^{b} \phi(x)\left( \diff{\beta}{x} \overline{\psi(x)} + \overline{\diff{\psi}{x}} \beta(x) \right)\:dx.
  \end{align*}
  Thus, we have
  \begin{align*}
    \int_{a}^{b} \overline{\psi(x)} \left( \alpha(x)\diff{^2\phi}{x^2} \right)\:dx &= S_1 + S_2 + \int_{a}^{b} \overline{\left( \alpha(x)\diff{^2}{x^2} + \left( 2\diff{\alpha}{x}-\beta(x) \right)\diff{}{x} + \left( \diff{^2\alpha}{x^2} - \diff{\beta}{x} + \gamma(x) \right) \right)\psi(x)} \phi(x)\:dx.
  \end{align*}
\end{solution}
\begin{solution}[40.23]\hfill
  \begin{enumerate}[(a)]
    \item We have $p(x) = 1$, and
      \begin{align*}
        \int_{0}^{a} \overline{\sin\left( n\pi x/a \right)}\sin\left( m\pi x/a \right)\:dx &= \frac{a}{m\pi - n\pi} \left( n\pi\cos\left( n\pi x/ \right) \overline{\sin\left( m\pi x/a \right)} - m\pi\cos\left( m\pi x/a \right) \overline{\sin\left( n\pi x/a \right)}  \right)\biggr\vert_{0}^{a}\\
                                                                                           &= 0.
      \end{align*}
    \item With the eigenfunctions $J_0\left( \alpha_i r/a \right)$, we have
      \begin{align*}
        \int_{0}^{a} r \overline{J_0\left( \frac{\alpha_m}{a}r \right)}J_0\left( \frac{\alpha_n}{a}r \right)\:dx &= \frac{r\left( \frac{\alpha_n}{a}J_0'\left( \frac{\alpha_n}{a}r \right) \right)\Bigr\vert_{0}^{a}}{\frac{\alpha_m}{a} - \frac{\alpha_n}{a}}.
      \end{align*}
      We use the identity that
      \begin{align*}
        J_0' &= -J_1
      \end{align*}
      to use $J_1(0) = 0$ and $J_0\left( \frac{\alpha_i}{a}(a) \right) = 0$, so we recover the orthogonality relation.
    \item We have
      \begin{align*}
        \int_{0}^{\infty} \operatorname{Ai}\left( \kappa x + \alpha_n \right) \operatorname{Ai}\left( \kappa x + \alpha_m \right)\:dx &= \frac{\kappa x\left( \operatorname{Ai}'\left( \kappa x + \alpha_n \right)\operatorname{Ai}\left( \kappa x + \alpha_m \right) - \operatorname{Ai}'\left( \kappa x + \alpha_m \right)\operatorname{Ai}'\left( \kappa x + \alpha_n \right) \right)\bigr\vert_{0}^{\infty}}{\kappa^2\left( \alpha_n - \alpha_m \right)}\\
                                                                                                                                      &= 0.
      \end{align*}
  \end{enumerate}
\end{solution}
\begin{solution}[40.27]

\end{solution}
\begin{solution}[41.8]

\end{solution}
\begin{solution}[41.13]

\end{solution}
\begin{solution}[41.14]

\end{solution}
\begin{solution}[41.16]

\end{solution}
\begin{solution}[41.25]

\end{solution}
\begin{solution}[41.28]

\end{solution}
\begin{solution}[42.1]

\end{solution}
\begin{solution}[42.2]

\end{solution}
\begin{solution}[42.11]

\end{solution}

\end{document}
