\documentclass[10pt]{mypackage}

% sans serif font:
%\usepackage{cmbright,sfmath,bbold}
%\renewcommand{\mathcal}{\mathtt}

%Euler:
\usepackage{newpxtext,eulerpx,eucal,eufrak}
\renewcommand*{\mathbb}[1]{\varmathbb{#1}}
\renewcommand*{\hbar}{\hslash}

%\renewcommand{\mathbb}{\mathds}
\usepackage{homework}
%\usepackage{exposition}

\pagestyle{fancy} %better headers
\fancyhf{}
\rhead{Avinash Iyer}
\lhead{Assignment 10}

\setcounter{secnumdepth}{0}

\begin{document}
\RaggedRight
\begin{solution}[40.7]
  We have
  \begin{align*}
    \braket{\psi}{\mathcal{L}\phi} &= \int_{a}^{b} \overline{\psi(x)}\left( \alpha(x)\diff{^2\phi}{x^2} + \beta(x)\diff{\phi}{x} + \gamma(x)\phi(x) \right)\:dx\\
                                   &= \int_{a}^{b} \overline{\psi(x)}\alpha(x)\diff{^2\phi}{x^2}\:dx + \int_{a}^{b} \overline{\psi(x)}\beta(x)\diff{\phi}{x}\:dx + \int_{a}^{b} \overline{\psi(x)}\gamma(x)\phi(x)\:dx.
  \end{align*}
  We evaluate these integrals separately. Assuming that $\alpha,\beta,\gamma$ are real-valued, we have
  \begin{align*}
    \int_{a}^{b} \overline{\psi(x)}\alpha(x)\diff{^2\phi}{x^2}\:dx &= \diff{\phi}{x} \overline{\psi(x)}\alpha(x)\biggr\vert_{a}^{b} - \int_{a}^{b} \left( \diff{\alpha}{x} \overline{\psi(x)} + \overline{\diff{\psi}{x}} \alpha(x) \right)\diff{\phi}{x}\:dx\\
                                                                   &= \underbrace{\left( \diff{\phi}{x}\alpha(x) \overline{\psi(x)} - \phi(x)\left( \diff{\alpha}{x} \overline{\psi(x)} + \overline{\diff{\psi}{x}} \alpha(x) \right) \right) \biggr\vert_{a}^{b}}_{S_1}\\
                                                                   &+ \int_{a}^{b} \overline{\left( \alpha(x)\diff{^2}{x^2} + 2\diff{\alpha}{x}\diff{}{x} + \diff{^2\alpha}{x^2}\right)\psi(x)}\phi(x)\:dx.\\
    \int_{a}^{b} \overline{\psi(x)}\beta(x) \diff{\phi}{x}\:dx &= \underbrace{\left( \phi(x) \beta(x) \overline{\psi(x)} \right)\biggr\vert_{a}^{b}}_{S_2} - \int_{a}^{b} \phi(x)\left( \diff{\beta}{x} \overline{\psi(x)} + \overline{\diff{\psi}{x}} \beta(x) \right)\:dx.
  \end{align*}
  Thus, we have
  \begin{align*}
    \int_{a}^{b} \overline{\psi(x)} \left( \mathcal{L}\phi \right)(x)\:dx &= S_1 + S_2 + \int_{a}^{b} \overline{\left( \alpha(x)\diff{^2}{x^2} + \left( 2\diff{\alpha}{x}-\beta(x) \right)\diff{}{x} + \left( \diff{^2\alpha}{x^2} - \diff{\beta}{x} + \gamma(x) \right) \right)\psi(x)} \phi(x)\:dx.
  \end{align*}
\end{solution}
\begin{solution}[40.23]\hfill
  \begin{enumerate}[(a)]
    \item We have $p(x) = 1$, and
      \begin{align*}
        \int_{0}^{a} \overline{\sin\left( n\pi x/a \right)}\sin\left( m\pi x/a \right)\:dx &= \frac{a}{m\pi - n\pi} \left( n\pi\cos\left( n\pi x/ \right) \overline{\sin\left( m\pi x/a \right)} - m\pi\cos\left( m\pi x/a \right) \overline{\sin\left( n\pi x/a \right)}  \right)\biggr\vert_{0}^{a}\\
                                                                                           &= 0.
      \end{align*}
    \item With the eigenfunctions $J_0\left( \alpha_i r/a \right)$, we have
      \begin{align*}
        \int_{0}^{a} r \overline{J_0\left( \frac{\alpha_m}{a}r \right)}J_0\left( \frac{\alpha_n}{a}r \right)\:dx &= \frac{r\left( \frac{\alpha_n}{a}J_0'\left( \frac{\alpha_n}{a}r \right) \right)\Bigr\vert_{0}^{a}}{\frac{\alpha_m}{a} - \frac{\alpha_n}{a}}.
      \end{align*}
      We use the identity that
      \begin{align*}
        J_0' &= -J_1
      \end{align*}
      to use $J_1(0) = 0$ and $J_0\left( \frac{\alpha_i}{a}(a) \right) = 0$, so we recover the orthogonality relation.
    \item We have
      \begin{align*}
        \int_{0}^{\infty} \operatorname{Ai}\left( \kappa x + \alpha_n \right) \operatorname{Ai}\left( \kappa x + \alpha_m \right)\:dx &= \frac{\kappa x\left( \operatorname{Ai}'\left( \kappa x + \alpha_n \right)\operatorname{Ai}\left( \kappa x + \alpha_m \right) - \operatorname{Ai}'\left( \kappa x + \alpha_m \right)\operatorname{Ai}'\left( \kappa x + \alpha_n \right) \right)\bigr\vert_{0}^{\infty}}{\kappa^2\left( \alpha_n - \alpha_m \right)}\\
                                                                                                                                      &= 0.
      \end{align*}
  \end{enumerate}
\end{solution}
\begin{solution}[40.27]\hfill
  \begin{enumerate}[(a)]
    \item We may express the Rayleigh quotient as
      \begin{align*}
        \rho\left( v \right) &= \frac{\braket{v}{Av}}{\braket{v}{v}}.
      \end{align*}
    \item We note that if $\mathcal{L}\phi = -\lambda w(x)\phi$, then by multiplying by $ \overline{\phi} $, integrating, and dividing we get
      \begin{align*}
        \lambda &= \frac{ \int_{a}^{b} \overline{\phi(x)} \left( \diff{}{x}\left( p(x)\diff{}{x} \right) + q(x) \right)\phi(x)\:dx }{ \int_{a}^{b} \left\vert \phi(x) \right\vert^2w(x)\:dx }\\
                &= \frac{1}{k_n} \int_{a}^{b} \overline{\phi(x)}\left( p(x)\diff{^2\phi}{x^2} + \diff{p}{x}\diff{\phi}{x} + q(x)\phi(x) \right)\:dx
      \end{align*}
    \item Splitting things up, we get
      \begin{align*}
        \lambda &= \frac{1}{k_n} \left( \int_{a}^{b} \overline{\phi(x)}p(x)\diff{^2\phi}{x^2}\:dx + \int_{a}^{b} \diff{p}{x}\diff{\phi}{x} \overline{\phi(x)}\:dx + \int_{a}^{b} q(x)\left\vert \phi(x) \right\vert^2\:dx \right).
      \end{align*}
      In the ``best case'' scenario, we may assume that $\diff{p}{x}$ vanishes everywhere, so we are left with
      \begin{align*}
        \lambda &\geq \frac{1}{k_n} \left( \int_{a}^{b} \overline{\phi(x)}p(x)\diff{^2\phi}{x^2}\:dx + \int_{a}^{b} q(x)\left\vert \phi(x) \right\vert^2\:dx \right).
      \end{align*}
      Integrating the first term by parts, we may implement the condition that
      \begin{align*}
        p(x)\left( \left( \diff{\phi}{x} \right)\phi(x) - \overline{\phi(x)}\diff{\phi}{x} \right)\biggr\vert_{a}^{b} &= 0
      \end{align*}
      to simplify down to
      \begin{align*}
        \lambda &\geq \frac{1}{k_n}\left( -p(x) \overline{\diff{\phi}{x}} \phi(x)\biggr\vert_{a}^{b} + \int_{a}^{b} q(x) \left\vert \phi(x) \right\vert^2\:dx \right).
      \end{align*}
  \end{enumerate}
\end{solution}
\begin{solution}[41.8]
  Using the Laplacian in spherical coordinates, we have
  \begin{align*}
    \nabla^2 &= \frac{1}{r^2} \pd{}{r}\left( r^2 \pd{}{r} \right) + \frac{1}{r^2\sin^2\left( \theta \right)} \pd{^2}{\phi^2} + \frac{1}{r^2\sin\left( \theta \right)}\pd{}{\theta}\left( \sin\left( \theta \right)\pd{}{\theta} \right),
  \end{align*}
  which separates
  \begin{align*}
    \psi\left( \mathbf{r} \right) &= R(r) \Theta(\theta)\Phi(\phi)
  \end{align*}
  into
  \begin{align*}
    \frac{1}{R}\diff{}{r}\left( r^2\diff{R}{r} \right) + \frac{1}{\Phi}\frac{1}{\sin^2\left( \theta \right)}\diff{^2\Phi}{\phi^2} + \frac{1}{\Theta} \frac{1}{\sin\left( \theta \right)} \diff{}{\theta}\left( \sin\left( \theta \right)\diff{\Theta}{\theta} \right) &= -k^2r^2.
  \end{align*}
  The latter two terms are functions of $\theta,\phi$ exclusively, so we have
  \begin{align*}
    \frac{1}{\Theta}\frac{1}{\sin\left( \theta \right)}\diff{}{\theta}\left( \sin\left( \theta \right) \diff{\Theta}{\theta} \right) + \frac{1}{\sin^2\left( \theta \right)} \diff{^2\Phi}{\phi^2} &= -\lambda,
    \intertext{and multiplying out by $\sin^2\left( \theta \right)$, we have}
    \frac{1}{\Theta}\sin\left( \theta \right)\diff{}{\theta}\left( \sin\left( \theta \right)\diff{\Theta}{\theta} \right) + \frac{1}{\Phi}\diff{^2\Phi}{\phi^2} &= -\lambda\sin^2\left( \theta \right).
  \end{align*}
  Therefore, we recover
  \begin{align*}
    \frac{1}{\Phi}\diff{^2\Phi}{\phi^2} &= -m^2\\
    \frac{1}{\Theta}\sin\left( \theta \right)\diff{}{\theta}\left( \sin\left( \theta \right)\diff{\Theta}{\theta} \right) &= -\lambda\sin^2\left( \theta \right) + m^2\\
    \diff{^2\Phi}{\phi^2} &= -m^2\Phi(\phi)\\
    \frac{1}{\sin\left( \theta \right)}\diff{}{\theta}\left( \sin\left( \theta \right)\diff{\Theta}{\theta} \right) + \left( \lambda - \frac{m^2}{\sin^2\left( \theta \right)} \right)\Theta(\theta) &= 0.
  \end{align*}
  Examining the term in $r$, we get
  \begin{align*}
    \frac{1}{R}\diff{}{r}\left( r^2\diff{R}{r} \right) &= -k^2r^2 + \lambda\\
    \diff{}{r}\left( r^2\diff{R}{r} \right) + \left( k^2r^2 - \lambda \right)R(r) &= 0.
  \end{align*}
  Using $\lambda = \ell\left( \ell + 1 \right)$, we get
  \begin{align*}
    \diff{}{r}\left( r^2\diff{R}{r} \right) + \left( k^2r^2 - \ell\left( \ell + 1 \right) \right)R(r) &= 0\\
    \frac{1}{\sin\left( \theta \right)}\diff{}{\theta}\left( \sin\left( \theta \right)\diff{\Theta}{\theta} \right) + \left( \ell\left( \ell + 1 \right) - \frac{m^2}{\sin^2\left( \theta \right)} \right)\Theta(\theta) &= 0\\
    \diff{^2\Phi}{\phi^2} &= -m^2\Phi.
  \end{align*}
  Using $x = \cos\left( \theta \right)$ and $X(x) = \Theta(\theta)$, we have
  \begin{align*}
    \diff{X}{x} &= \diff{\Theta}{\left( \cos\left( \theta \right) \right)}\\
                &= -\frac{1}{\sin\left( \theta \right)} \diff{\Theta}{\theta},
  \end{align*}
  and
  \begin{align*}
    \diff{}{x}\left( \left( 1-x^2 \right)\diff{X}{x} \right) &= \frac{1}{\sin\left( \theta \right)}\diff{}{\theta}\left( \sin\left( \theta \right)\diff{\Theta}{\theta} \right).
  \end{align*}
  Therefore, we have
  \begin{align*}
    R(r) &= a_1j_{\ell}\left( kr \right) + a_2n_{\ell}\left( kr \right)\\
    \Theta(\theta) &= b_1P_{\ell,m}\left( \cos\left( \theta \right) \right) + b_2Q_{\ell,m}\left( \cos\left( \theta \right) \right)\\
    \Phi(\phi) &= c_1e^{im\phi} + c_2e^{-im\phi}.
  \end{align*}
\end{solution}
\begin{solution}[41.13]

\end{solution}
\begin{solution}[41.14]

\end{solution}
\begin{solution}[41.16]

\end{solution}
\begin{solution}[41.25]

\end{solution}
\begin{solution}[41.28]

\end{solution}
\begin{solution}[42.1]

\end{solution}
\begin{solution}[42.2]

\end{solution}
\begin{solution}[42.11]

\end{solution}

\end{document}
