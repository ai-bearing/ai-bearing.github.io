\documentclass[10pt]{mypackage}

% sans serif font:
%\usepackage{cmbright,sfmath,bbold}
%\renewcommand{\mathcal}{\mathtt}

%Euler:
\usepackage{newpxtext,eulerpx,eucal,eufrak}
\renewcommand*{\mathbb}[1]{\varmathbb{#1}}
\renewcommand*{\hbar}{\hslash}

%\renewcommand{\mathbb}{\mathds}
\usepackage{homework}
%\usepackage{exposition}

\pagestyle{fancy} %better headers
\fancyhf{}
\rhead{Avinash Iyer}
\lhead{Assignment 10}

\setcounter{secnumdepth}{0}

\begin{document}
\RaggedRight
\begin{solution}[40.7]
  We have
  \begin{align*}
    \braket{\psi}{\mathcal{L}\phi} &= \int_{a}^{b} \overline{\psi(x)}\left( \alpha(x)\diff{^2\phi}{x^2} + \beta(x)\diff{\phi}{x} + \gamma(x)\phi(x) \right)\:dx\\
                                   &= \int_{a}^{b} \overline{\psi(x)}\alpha(x)\diff{^2\phi}{x^2}\:dx + \int_{a}^{b} \overline{\psi(x)}\beta(x)\diff{\phi}{x}\:dx + \int_{a}^{b} \overline{\psi(x)}\gamma(x)\phi(x)\:dx.
  \end{align*}
  We evaluate these integrals separately. Assuming that $\alpha,\beta,\gamma$ are real-valued, we have
  \begin{align*}
    \int_{a}^{b} \overline{\psi(x)}\alpha(x)\diff{^2\phi}{x^2}\:dx &= \diff{\phi}{x} \overline{\psi(x)}\alpha(x)\biggr\vert_{a}^{b} - \int_{a}^{b} \left( \diff{\alpha}{x} \overline{\psi(x)} + \overline{\diff{\psi}{x}} \alpha(x) \right)\diff{\phi}{x}\:dx\\
                                                                   &= \underbrace{\left( \diff{\phi}{x}\alpha(x) \overline{\psi(x)} - \phi(x)\left( \diff{\alpha}{x} \overline{\psi(x)} + \overline{\diff{\psi}{x}} \alpha(x) \right) \right) \biggr\vert_{a}^{b}}_{S_1}\\
                                                                   &+ \int_{a}^{b} \overline{\left( \alpha(x)\diff{^2}{x^2} + 2\diff{\alpha}{x}\diff{}{x} + \diff{^2\alpha}{x^2}\right)\psi(x)}\phi(x)\:dx.\\
    \int_{a}^{b} \overline{\psi(x)}\beta(x) \diff{\phi}{x}\:dx &= \underbrace{\left( \phi(x) \beta(x) \overline{\psi(x)} \right)\biggr\vert_{a}^{b}}_{S_2} - \int_{a}^{b} \phi(x)\left( \diff{\beta}{x} \overline{\psi(x)} + \overline{\diff{\psi}{x}} \beta(x) \right)\:dx.
  \end{align*}
  Thus, we have
  \begin{align*}
    \int_{a}^{b} \overline{\psi(x)} \left( \mathcal{L}\phi \right)(x)\:dx &= S_1 + S_2 + \int_{a}^{b} \overline{\left( \alpha(x)\diff{^2}{x^2} + \left( 2\diff{\alpha}{x}-\beta(x) \right)\diff{}{x} + \left( \diff{^2\alpha}{x^2} - \diff{\beta}{x} + \gamma(x) \right) \right)\psi(x)} \phi(x)\:dx.
  \end{align*}
\end{solution}
\begin{solution}[40.23]\hfill
  \begin{enumerate}[(a)]
    \item We have $p(x) = 1$, and
      \begin{align*}
        \int_{0}^{a} \overline{\sin\left( n\pi x/a \right)}\sin\left( m\pi x/a \right)\:dx &= \frac{a}{m\pi - n\pi} \left( n\pi\cos\left( n\pi x/ \right) \overline{\sin\left( m\pi x/a \right)} - m\pi\cos\left( m\pi x/a \right) \overline{\sin\left( n\pi x/a \right)}  \right)\biggr\vert_{0}^{a}\\
                                                                                           &= 0.
      \end{align*}
    \item With the eigenfunctions $J_0\left( \alpha_i r/a \right)$, we have
      \begin{align*}
        \int_{0}^{a} r \overline{J_0\left( \frac{\alpha_m}{a}r \right)}J_0\left( \frac{\alpha_n}{a}r \right)\:dx &= \frac{r\left( \frac{\alpha_n}{a}J_0'\left( \frac{\alpha_n}{a}r \right) \right)\Bigr\vert_{0}^{a}}{\frac{\alpha_m}{a} - \frac{\alpha_n}{a}}.
      \end{align*}
      We use the identity that
      \begin{align*}
        J_0' &= -J_1
      \end{align*}
      to use $J_1(0) = 0$ and $J_0\left( \frac{\alpha_i}{a}(a) \right) = 0$, so we recover the orthogonality relation.
    \item We have
      \begin{align*}
        \int_{0}^{\infty} \operatorname{Ai}\left( \kappa x + \alpha_n \right) \operatorname{Ai}\left( \kappa x + \alpha_m \right)\:dx &= \frac{\kappa x\left( \operatorname{Ai}'\left( \kappa x + \alpha_n \right)\operatorname{Ai}\left( \kappa x + \alpha_m \right) - \operatorname{Ai}'\left( \kappa x + \alpha_m \right)\operatorname{Ai}'\left( \kappa x + \alpha_n \right) \right)\bigr\vert_{0}^{\infty}}{\kappa^2\left( \alpha_n - \alpha_m \right)}\\
                                                                                                                                      &= 0.
      \end{align*}
  \end{enumerate}
\end{solution}
\begin{solution}[40.27]\hfill
  \begin{enumerate}[(a)]
    \item We may express the Rayleigh quotient as
      \begin{align*}
        \rho\left( v \right) &= \frac{\braket{v}{Av}}{\braket{v}{v}}.
      \end{align*}
    \item We note that if $\mathcal{L}\phi = -\lambda w(x)\phi$, then by multiplying by $ \overline{\phi} $, integrating, and dividing we get
      \begin{align*}
        \lambda &= \frac{ \int_{a}^{b} \overline{\phi(x)} \left( \diff{}{x}\left( p(x)\diff{}{x} \right) + q(x) \right)\phi(x)\:dx }{ \int_{a}^{b} \left\vert \phi(x) \right\vert^2w(x)\:dx }\\
                &= \frac{1}{k_n} \int_{a}^{b} \overline{\phi(x)}\left( p(x)\diff{^2\phi}{x^2} + \diff{p}{x}\diff{\phi}{x} + q(x)\phi(x) \right)\:dx
      \end{align*}
    \item Splitting things up, we get
      \begin{align*}
        \lambda &= \frac{1}{k_n} \left( \int_{a}^{b} \overline{\phi(x)}p(x)\diff{^2\phi}{x^2}\:dx + \int_{a}^{b} \diff{p}{x}\diff{\phi}{x} \overline{\phi(x)}\:dx + \int_{a}^{b} q(x)\left\vert \phi(x) \right\vert^2\:dx \right).
      \end{align*}
      In the ``best case'' scenario, we may assume that $\diff{p}{x}$ vanishes everywhere, so we are left with
      \begin{align*}
        \lambda &\geq \frac{1}{k_n} \left( \int_{a}^{b} \overline{\phi(x)}p(x)\diff{^2\phi}{x^2}\:dx + \int_{a}^{b} q(x)\left\vert \phi(x) \right\vert^2\:dx \right).
      \end{align*}
      Integrating the first term by parts, we may implement the condition that
      \begin{align*}
        p(x)\left( \left( \diff{\phi}{x} \right)\phi(x) - \overline{\phi(x)}\diff{\phi}{x} \right)\biggr\vert_{a}^{b} &= 0
      \end{align*}
      to simplify down to
      \begin{align*}
        \lambda &\geq \frac{1}{k_n}\left( -p(x) \overline{\diff{\phi}{x}} \phi(x)\biggr\vert_{a}^{b} + \int_{a}^{b} q(x) \left\vert \phi(x) \right\vert^2\:dx \right).
      \end{align*}
  \end{enumerate}
\end{solution}
\begin{solution}[41.8]
  Using the Laplacian in spherical coordinates, we have
  \begin{align*}
    \nabla^2 &= \frac{1}{r^2} \pd{}{r}\left( r^2 \pd{}{r} \right) + \frac{1}{r^2\sin^2\left( \theta \right)} \pd{^2}{\phi^2} + \frac{1}{r^2\sin\left( \theta \right)}\pd{}{\theta}\left( \sin\left( \theta \right)\pd{}{\theta} \right),
  \end{align*}
  which separates
  \begin{align*}
    \psi\left( \mathbf{r} \right) &= R(r) \Theta(\theta)\Phi(\phi)
  \end{align*}
  into
  \begin{align*}
    \frac{1}{R}\diff{}{r}\left( r^2\diff{R}{r} \right) + \frac{1}{\Phi}\frac{1}{\sin^2\left( \theta \right)}\diff{^2\Phi}{\phi^2} + \frac{1}{\Theta} \frac{1}{\sin\left( \theta \right)} \diff{}{\theta}\left( \sin\left( \theta \right)\diff{\Theta}{\theta} \right) &= -k^2r^2.
  \end{align*}
  The latter two terms are functions of $\theta,\phi$ exclusively, so we have
  \begin{align*}
    \frac{1}{\Theta}\frac{1}{\sin\left( \theta \right)}\diff{}{\theta}\left( \sin\left( \theta \right) \diff{\Theta}{\theta} \right) + \frac{1}{\sin^2\left( \theta \right)} \diff{^2\Phi}{\phi^2} &= -\lambda,
    \intertext{and multiplying out by $\sin^2\left( \theta \right)$, we have}
    \frac{1}{\Theta}\sin\left( \theta \right)\diff{}{\theta}\left( \sin\left( \theta \right)\diff{\Theta}{\theta} \right) + \frac{1}{\Phi}\diff{^2\Phi}{\phi^2} &= -\lambda\sin^2\left( \theta \right).
  \end{align*}
  Therefore, we recover
  \begin{align*}
    \frac{1}{\Phi}\diff{^2\Phi}{\phi^2} &= -m^2\\
    \frac{1}{\Theta}\sin\left( \theta \right)\diff{}{\theta}\left( \sin\left( \theta \right)\diff{\Theta}{\theta} \right) &= -\lambda\sin^2\left( \theta \right) + m^2\\
    \diff{^2\Phi}{\phi^2} &= -m^2\Phi(\phi)\\
    \frac{1}{\sin\left( \theta \right)}\diff{}{\theta}\left( \sin\left( \theta \right)\diff{\Theta}{\theta} \right) + \left( \lambda - \frac{m^2}{\sin^2\left( \theta \right)} \right)\Theta(\theta) &= 0.
  \end{align*}
  Examining the term in $r$, we get
  \begin{align*}
    \frac{1}{R}\diff{}{r}\left( r^2\diff{R}{r} \right) &= -k^2r^2 + \lambda\\
    \diff{}{r}\left( r^2\diff{R}{r} \right) + \left( k^2r^2 - \lambda \right)R(r) &= 0.
  \end{align*}
  Using $\lambda = \ell\left( \ell + 1 \right)$, we get
  \begin{align*}
    \diff{}{r}\left( r^2\diff{R}{r} \right) + \left( k^2r^2 - \ell\left( \ell + 1 \right) \right)R(r) &= 0\\
    \frac{1}{\sin\left( \theta \right)}\diff{}{\theta}\left( \sin\left( \theta \right)\diff{\Theta}{\theta} \right) + \left( \ell\left( \ell + 1 \right) - \frac{m^2}{\sin^2\left( \theta \right)} \right)\Theta(\theta) &= 0\\
    \diff{^2\Phi}{\phi^2} &= -m^2\Phi.
  \end{align*}
  Using $x = \cos\left( \theta \right)$ and $X(x) = \Theta(\theta)$, we have
  \begin{align*}
    \diff{X}{x} &= \diff{\Theta}{\left( \cos\left( \theta \right) \right)}\\
                &= -\frac{1}{\sin\left( \theta \right)} \diff{\Theta}{\theta},
  \end{align*}
  and
  \begin{align*}
    \diff{}{x}\left( \left( 1-x^2 \right)\diff{X}{x} \right) &= \frac{1}{\sin\left( \theta \right)}\diff{}{\theta}\left( \sin\left( \theta \right)\diff{\Theta}{\theta} \right).
  \end{align*}
  Therefore, we have
  \begin{align*}
    R(r) &= a_1j_{\ell}\left( kr \right) + a_2n_{\ell}\left( kr \right)\\
    \Theta(\theta) &= b_1P_{\ell,m}\left( \cos\left( \theta \right) \right) + b_2Q_{\ell,m}\left( \cos\left( \theta \right) \right)\\
    \Phi(\phi) &= c_1e^{im\phi} + c_2e^{-im\phi}.
  \end{align*}
\end{solution}
\begin{solution}[41.13]\hfill
  \begin{enumerate}[(a)]
    \item Separating variables, we have
      \begin{align*}
        0 &= \frac{1}{X}\diff{^2X}{x^2} + \frac{1}{Y}\diff{^2Y}{y^2} + \frac{1}{Z}\diff{^2Z}{z^2}.
      \end{align*}
      We assume that
      \begin{align*}
        \diff{^2 X}{x^2} &= -\alpha^2 X\\
        \diff{^2Y}{y^2} &= -\beta^2 Y\\
        \diff{^2Z}{z^2} &= -\gamma^2 Z,
      \end{align*}
      subject to the condition that
      \begin{align*}
        \alpha^2 + \beta^2 + \gamma^2 &= 0.
      \end{align*}
      We have the boundary conditions of
      \begin{align*}
        V_0 &= V\left( 0,y,z \right)\\
        0 &= V\left( a,y,z \right)\\
          &= V\left( x,0,z \right)\\
          &= V\left( x,a,z \right)\\
          &= V\left( x,y,0 \right)\\
          &= V\left( x,y,a \right)
      \end{align*}
      Due to the Neumann boundary conditions in fixed $x$, we know that our eigenfunctions in $y$ and $z$ are of the form $\sin\left( \frac{n\pi}{a}y \right)$ and $\sin\left( \frac{m\pi}{a}z \right)$. This gives
      \begin{align*}
        Y(y)Z(z) &= \sum_{m=1}^{\infty}\sum_{n=1}^{\infty} a_{m,n}\sin\left( \frac{n\pi}{a}y \right)\sin\left( \frac{m\pi}{a}z \right)\\
        a_{m,n} &= \frac{4V_0}{a^2} \int_{0}^{a} \int_{0}^{a} \sin\left( \frac{n\pi}{a}y \right)\sin\left( \frac{m\pi}{a}z \right)\:dz\:dy\\
                &= \int_{0}^{a} \frac{2\sqrt{V_0}}{a}\sin\left( \frac{m\pi}{a} \right)\:dz \int_{0}^{a} \frac{2\sqrt{V_0}}{a}\sin\left( \frac{n\pi}{a} \right)\:dy\\
                &= \frac{4V_0}{\pi^2 mn},
      \end{align*}
      and
      \begin{align*}
        V &= \sum_{m=1}^{\infty}\sum_{n=1}^{\infty}\frac{4V_0}{\pi^2 mn}e^{-\frac{\pi}{a}x\sqrt{m^2 + n^2}}\sin\left( \frac{n\pi}{a}y \right)\sin\left( \frac{m\pi}{a}z \right).
      \end{align*}
      \begin{remark}
      I do not know where I lost the $V\left( a,y,z \right) = 0$ condition, but I did.
      \end{remark}
    \item Via linearity, we may consider the cube as being a sum of cubes with faces at $x = 0$ and $z = a$ held at $V_0$, then add together.\newline

      Solving for this case by using the Dirichlet conditions in $x$ and $y$, we get
      \begin{align*}
        X_mY_n(x,y) &= \sum_{m=1}^{\infty}\sum_{n=1}^{\infty}a_{n,m}\sin\left( \frac{n\pi}{a}x \right)\sin\left( \frac{m\pi}{a}y \right)\\
        a_{n,m} &= \frac{4V_0}{a^2} \int_{0}^{a} \int_{0}^{a} \sin\left( \frac{m\pi}{a}y \right)\sin\left( \frac{n\pi}{a}x \right)\:dy\:dx\\
                &= \frac{4V_0}{\pi^2mn},
      \end{align*}
      and
      \begin{align*}
        Z_{m,n}(z) &= a_1\cosh\left( \frac{\pi}{a}\sqrt{m^2 + n^2} z \right) + a_2\sinh\left( \frac{\pi}{a}\sqrt{m^2 + n^2} z \right).
      \end{align*}
      Evaluating the condition that $Z_{m,n}(a) =1 $, we get
      \begin{align*}
        Z_{m,n}(z) &= a_1\cosh\left( \pi\sqrt{m^2 + n^2} \right) + a_2\sinh\left( \pi\sqrt{m^2 + n^2} \right).
      \end{align*}
      Using the power of safe assumptions, we will assume $a_2 = 0$ for all such $a_2$, giving
      \begin{align*}
        Z_{m,n}(z) &= \tanh\left( \pi\sqrt{m^2 + n^2} \right)\cosh\left( \frac{\pi}{a}\sqrt{m^2 + n^2} z \right).
      \end{align*}
      Thus, we get the solution in the case of \textit{only} $z=a$ at $V_0$ of
      \begin{align*}
        V &= \sum_{m=1}^{\infty}\sum_{n=1}^{\infty} \frac{4V_0}{\pi^2mn}\tanh\left( \pi\sqrt{m^2 + n^2} \right)\cosh\left( \frac{\pi}{a}\sqrt{m^2 + n^2} z \right)\sin\left( \frac{m\pi}{a}y \right)\sin\left( \frac{n\pi}{a}x \right).
      \end{align*}
      and the solution to the full cube of
      \begin{align*}
        V &= \sum_{m=1}^{\infty}\sum_{n=1}^{\infty} \frac{4V_0}{\pi^2mn}\tanh\left( \pi\sqrt{m^2 + n^2} \right)\cosh\left( \frac{\pi}{a}\sqrt{m^2 + n^2} z \right)\sin\left( \frac{m\pi}{a}y \right)\sin\left( \frac{n\pi}{a}x \right)\\
          &+ \frac{4V_0}{\pi^2 mn}e^{-\frac{\pi}{a}x\sqrt{m^2 + n^2}}\sin\left( \frac{n\pi}{a}y \right)\sin\left( \frac{m\pi}{a}z \right)
      \end{align*}
  \end{enumerate}
\end{solution}
\begin{solution}[41.14]
  We know that solutions of Laplace's equation in cylindrical coordinates are of the form
  \begin{align*}
    R(r) &= a_1J_m\left( \beta r \right) + a_2N_m\left( \beta r \right)\\
    \Phi(\phi) &= b_1\cos\left( m\phi \right) + b_2\sin\left( m\phi \right)\\
    Z(z) &= c_1e^{\beta z} + c_2e^{-\beta z}.
  \end{align*}
  Instead of $R$, we will use $S$ to denote the radius of the cylinder. Given the boundary conditions of
  \begin{align*}
    V\left(r,\phi,0\right) &= 0\\
    V\left( S,\phi,z \right) &= 0\\
    V\left( r,\phi,L \right) &= V_0,
  \end{align*}
  we know that $a_2$ must be zero, as the $N_m$ blow up at the origin. We let $\alpha_{m,n}$ denote the $n$th zero of $J_m$, and since $Z(0) = 0$, we must have $Z = d\sinh\left( \frac{\alpha_{m,n}}{S}z \right)$. We are left with the expression
  \begin{align*}
    V &= \sum_{m=0}^{\infty}\sum_{n=1}^{\infty}\sinh\left( \frac{\alpha_{m,n}}{S}z \right)J_{m}\left( \frac{\alpha_{m,n}}{S}r \right) \left( A_{m,n}\cos\left( m\phi \right) + B_{m,n}\sin\left( m\phi \right) \right).
  \end{align*}
  Since we have polar symmetry, we may disregard $m$ for $m\neq 0$. Renaming $\alpha_{0,n}\eqcolon \alpha_n$ and $A_{0,n}\eqcolon A_n$, we have
  \begin{align*}
    V(r,\phi,z) &= \sum_{n=1}^{\infty}A_n\sinh\left( \frac{\alpha_{n}}{S} z \right)J_0\left( \frac{\alpha_n}{S} r \right).
  \end{align*}
  Evaluating
  \begin{align*}
    V\left( r,\phi,L \right) &= V_0\\
                             &= \sum_{n=1}^{\infty}A_n\sinh\left( \frac{\alpha_n}{S}L \right)J_0\left( \frac{\alpha_n}{S}r \right),
  \end{align*}
  so
  \begin{align*}
    A_n &= \frac{2V_0}{\sinh\left( \frac{\alpha_n L}{S} \right)S^2J_1\left( \alpha_n \right)^2} \int_{0}^{S} J_0\left( \frac{\alpha_n}{S}r \right) r\:dr.
  \end{align*}
\end{solution}
\begin{solution}[41.16]
  Since we have oscillation in $z$, our separated solutions to Laplace's equations are of the form
  \begin{align*}
    R(r) &= a_1I_{m}\left( kr \right) + a_2K_{m}\left( kr \right)\\
    \Phi(\phi) &= b_1\cos\left( m\phi \right) + b_2\sin\left( m\phi \right)\\
    Z(z) &= c_1\cos\left( kz \right) + c_2\sin\left( kz \right).
  \end{align*}
  We may disregard the term in $K_m$ as the function blows up towards the origin. We may also disregard the term in $\sin(kz)$ as we have periodic Neumann conditions rather than Dirichlet conditions. Thus, we get
  \begin{align*}
    V(r,\phi,z) &= \sum_{k=1}^{\infty}\sum_{m=0}^{\infty} \frac{4V_0\sin\left( \frac{k\pi}{2} \right)}{k\pi}\cos\left( \frac{k\pi}{L}z \right) I_{m}\left( kr \right)\left( A_{k,m}\cos\left( m\phi \right) + B_{k,m}\sin\left( m\phi \right) \right).
  \end{align*}
  As we have polar symmetry, we may disregard all but the $m=0$ term, giving
  \begin{align*}
    V\left( r,\phi,z \right) &= \sum_{k=1}^{\infty}A_{k}\frac{4V_0\sin\left( \frac{k\pi}{2} \right)}{k\pi}\cos\left( \frac{k\pi}{L}z \right)I_0\left( kr \right).
  \end{align*}
  Finally, we must have
  \begin{align*}
    A_k &= \frac{1}{I_0\left( kR \right)},
  \end{align*}
  so we have a full solution of
  \begin{align*}
    V\left( r,\phi,z \right) &= \sum_{k=1}^{\infty}\frac{4V_0\sin\left( \frac{k\pi}{2} \right)}{k\pi I_0\left( kS \right)} \cos\left( \frac{k\pi}{L}z \right)I_0\left( kr \right).
  \end{align*}
\end{solution}
\begin{solution}[41.25]\hfill
  \begin{enumerate}[(a)]
    \item The only $J_m$ such that there is displacement at the center is $J_0$, so only the modes $\alpha_{0,n}$ are excited. Evaluating the ratios, they are not particularly harmonic. The only ratios that appear are $\frac{\alpha_{0,5}}{\alpha_{0,4}}$ approximating a major third and $\frac{\alpha_{0,4}}{\alpha_{0,3}}$ a perfect fourth.
    \item Exciting the preferred modes of $J_{1}$ suppresses the $J_0$ modes because $J_{m}$ for all $m \geq 1$ have no displacement at the center.
    \item The respective frequencies are
      \begin{align*}
        \frac{\alpha_{1,2}}{\alpha_{1,1}} &\approx 1.83\\
        \frac{\alpha_{1,3}}{\alpha_{1,2}} &\approx 1.45\\
        \frac{\alpha_{1,4}}{\alpha_{1,3}} &\approx 1.31\\
        \frac{\alpha_{1,5}}{\alpha_{1,4}} &\approx 1.24\\
        \frac{\alpha_{1,6}}{\alpha_{1,5}} &\approx 1.19
      \end{align*}
      These are quite close to their respective harmonies of equal temperament.
  \end{enumerate}
\end{solution}
\begin{solution}[41.28]\hfill
  \begin{enumerate}[(a)]
    \item We have
      \begin{align*}
        0 &= a_1J_{n}\left( ka \right) + a_2N_n\left( ka \right)\\
        0 &= a_1J_n\left( kb \right) + a_2N_n\left( kb \right).
      \end{align*}
      Therefore,
      \begin{align*}
        a_1 &= -a_2\frac{N_n\left( kb \right)}{J_n\left( kb \right)}.
      \end{align*}
      Here, $k$ is the parameter of eigenmodes, with units of inverse radius.
  \end{enumerate}
  I don't know how to do the other two parts of the problems.
\end{solution}
\begin{solution}[42.1]\hfill
  \begin{enumerate}[(a)]
    \item We have the Green's Function of
      \begin{align*}
        G\left( x,t \right) &= \frac{1}{L} \begin{cases}
          x\left( t-L \right) & x < t\\
          t\left( x-L \right) & x > t
        \end{cases}
      \end{align*}
    \item With $\psi\left( 0 \right) = 0$ and $\psi'\left( L \right) = 0$, we have
      \begin{align*}
        G\left( x,t \right) &= \frac{1}{L} \begin{cases}
          ax & x < t\\
          k & x > t
        \end{cases}.
      \end{align*}
      We must have $at = k$ and $-a = 1$. Thus,
      \begin{align*}
        G\left( x,t \right) &= \begin{cases}
          -x & x < t\\
          -t & x > t
        \end{cases}
      \end{align*}
    \item With $\psi'(0) = 0$ and $\psi(L) = 0$, we have
      \begin{align*}
        G\left( x,t \right) &= \begin{cases}
          k & x < t\\
          b\left( x-L \right) & x > t
        \end{cases}.
      \end{align*}
      We must have $b\left( t-L \right) = k$ and $b = 1$. Therefore,
      \begin{align*}
        G\left( x,t \right) &= \begin{cases}
          t-L & x < t\\
          x-L & x > t
        \end{cases}
      \end{align*}
  \end{enumerate}
\end{solution}
\begin{solution}[42.2]
  We have
  \begin{align*}
    \psi_{p,1}(x) &= \int_{0}^{L} G\left( x,t \right)t^2\:dt\\
               &= \frac{1}{L}\int_{0}^{x} t\left( x-L \right)t^2\:dt + \frac{1}{L} \int_{x}^{L} x\left( t-L \right)t^2 \:dt\\
               &= \frac{1}{12}x\left( x^3-L^3 \right).\\
    \psi_{p,2} &= \int_{0}^{L} G\left( x,t \right)t^2\:dt\\
            &= \int_{0}^{x} -t^3\:dt + \int_{x}^{L} -xt^2\:dx\\
            &= \frac{1}{12}x\left( x^3-4L^3 \right)\\
    \psi_{p,3} &= \int_{0}^{L} G\left( x,t \right)t^2\:dt\\
            &= \int_{0}^{x} t^2\left( x-L \right)\:dt + \int_{x}^{L} t^2\left( t-L \right)\:dt\\
            &= \frac{1}{12}\left( x^4 - L^4 \right).
  \end{align*}
  \begin{itemize}
    \item We see that $\psi_{p,1}(0) = 0$ and $\psi_{p,1}(L) = 0$.
    \item We see that $\psi_{p,2} = 0$ and $\diff{\psi_{p,2}}{x}\bigr\vert_{x=L} = 4L^3-4L^3 = 0$.
    \item We see that $\diff{\psi_{p,3}}{x}\bigr\vert_{x=0} = 0$ and $\psi_{p,3}\left( L \right)=0$.
  \end{itemize}
\end{solution}
\begin{solution}[42.11]\hfill
  \begin{enumerate}[(a)]
    \item Since $\delta\left( x-t \right) = 0$ whenever $x\neq t$, we may implement the Neumann conditions to take
      \begin{align*}
        G\left( x,t \right) &= \begin{cases}
          a\cos\left( 3x \right) & x < t\\
          b \cos\left( 3\left( x-L \right) \right) & x > t,
        \end{cases}
      \end{align*}
      subject to the conditions that
      \begin{align*}
        a\cos\left( 3t \right) &= b\cos\left( 3\left( t-L \right) \right)\\
        -3b\sin\left( 3\left( t-L \right) \right) + 3a\sin\left( 3t \right) &= 1\\
        3a\sin\left( 3t \right) &= 1 + 3b\sin\left( 3\left( t-L \right) \right).
      \end{align*}
      Therefore,
      \begin{align*}
        b &= a\frac{\cos\left( 3t \right)}{\cos\left( 3\left( t-L \right) \right)}\\
        3a\sin\left( 3t \right) &= 1 + 3a\sin\left( 3\left( t-L \right) \right)\frac{\cos\left( 3t \right)}{\cos\left( 3\left( t-L \right) \right)}\\
        3a\sin\left( 3t \right) &= 1 + 3a\tan\left( 3\left( t-L \right) \right)\cos\left( 3t \right)\\
        a &= \frac{1}{3\left( \sin\left( 3t \right) - \tan\left( 3\left( t-L \right) \right) \right)\cos\left( 3t \right)}\\
        b &= \frac{\cos\left( 3t \right)}{3\left( \cos\left( 3(t-L) \right)\sin\left( 3t \right) -\sin\left( 3\left( t-L \right) \right)\cos\left( 3t \right) \right)}.
      \end{align*}
    \item The eigenfunctions of the Sturm--Liouville operator $\diff{^2}{x^2} + 9$ subject to the Neumann boundary conditions are $3\cos\left( \frac{n\pi}{L}t \right)$. Therefore,
      \begin{align*}
        G(x,t) &= \frac{9L}{\pi}\sum_{n=1}^{\infty}\frac{\cos\left( \frac{n\pi}{L}x \right)\cos\left( \frac{n\pi}{L}t \right)}{n}.
      \end{align*}
    \item We will use the eigenfunction expansion for this purpose. This gives
      \begin{align*}
        y_p(x) &= \int_{0}^{L} \frac{9L}{\pi}\sum_{n=1}^{\infty}\frac{\cos\left( \frac{n\pi}{L}x \right)\cos\left( \frac{n\pi}{L}t \right)}{n} t^2\:dt\\
               &= \frac{9L}{\pi} \sum_{n=1}^{\infty} \frac{1}{n}\cos\left( \frac{n\pi}{L}x \right) \int_{0}^{L} t^2\cos\left( \frac{n\pi}{L}t \right)\:dt\\
               &= \frac{9L}{\pi} \sum_{n=1}^{\infty}\frac{2L^3}{n^3\pi^2}\cos\left( \frac{n\pi}{L}x \right).
      \end{align*}
      
  \end{enumerate}
\end{solution}

\end{document}
