\documentclass[10pt]{mypackage}

% sans serif font:
%\usepackage{cmbright,sfmath,bbold}
%\renewcommand{\mathcal}{\mathtt}

%Euler:
\usepackage{newpxtext,eulerpx,eucal,eufrak}
\renewcommand*{\mathbb}[1]{\varmathbb{#1}}
\renewcommand*{\hbar}{\hslash}

%\renewcommand{\mathbb}{\mathds}
\usepackage{homework}
%\usepackage{exposition}

\pagestyle{fancy} %better headers
\fancyhf{}
\rhead{Avinash Iyer}
\lhead{Assignment 9}

\setcounter{secnumdepth}{0}

\begin{document}
\RaggedRight
\begin{solution}[38.5]
  Copying the template equation, we have
  \begin{align*}
    \diff{v}{t} &= -\frac{c}{m}v^2 + g,
  \end{align*}
  where $c$ is some constant. We see that the terminal velocity is
  \begin{align*}
    v_t &= \sqrt{\frac{mg}{c}}.
  \end{align*}
  Separating variables, we have
  \begin{align*}
    \frac{dv}{-\frac{c}{m}v^2 + g} &= dt\\
    \frac{1}{g}\left( \frac{dv}{1 - \frac{c}{mg}v^2} \right) &= dt\\
    \frac{1}{g}\left( \frac{dv}{1 - \left(v/v_t\right)^2} \right) &= dt.
  \end{align*}
  Using the substitution $u \coloneq v/v_t$, we have $du = \frac{1}{v_t} dv$, meaning that
  \begin{align*}
    v_t \int_{}^{} \frac{1}{1-u^2}\:du &= \int_{}^{} g\:dt.
  \end{align*}
  The integral of $\frac{1}{1-u^2}$ is $\frac{1}{2}\ln\left( \frac{1+u}{1-u} \right) = \operatorname{arctanh}\left( u \right)$. Therefore, we have
  \begin{align*}
    \frac{v}{v_t} &= \tanh\left( \frac{g}{v_t}t \right) + K\\
    v &= v_t \tanh\left( \frac{g}{v_t}t \right) + v_0\\
      &= \sqrt{\frac{mg}{c}}\tanh\left( \sqrt{\frac{c}{mg}}t \right) + v_0.
  \end{align*}
\end{solution}
\begin{solution}[38.6]\hfill
  \begin{enumerate}[(a)]
    \item Using the chain rule and letting $\diff{m}{t} = km^{2/3}$, we have
      \begin{align*}
        \diff{v}{t} &= km^{2/3}\diff{v}{m}\\
        \diff{v}{m} + \frac{v}{m} &= -\frac{b}{km}v + \frac{g}{km^{2/3}}.
      \end{align*}
      With integrating factor $m^{1 + \frac{b}{k}}$, we have
      \begin{align*}
        m^{1 + \frac{b}{k}}v &= \frac{g}{k}\frac{m^{\frac{4}{3} + \frac{b}{k}}}{\frac{4}{3} + \frac{b}{k}} + C\\
        v &= \frac{g}{k\left( \frac{4}{3} + \frac{b}{k} \right)}m^{\frac{1}{3} + \frac{b}{k}} + Cm^{-1-\frac{b}{k}}.
      \end{align*}
      We let $v\left( m_0 \right)= 0$, so that
      \begin{align*}
        C &= -\frac{g}{k\left( \frac{4}{3} + \frac{b}{k} \right)}m_0^{\frac{4}{3} + \frac{b}{k}},
      \end{align*}
      so
      \begin{align*}
        v &= \frac{g}{\frac{4}{3}k + b}m^{\frac{1}{3}}\left( 1-\left( \frac{m_0}{m} \right)^{\frac{4}{3} + \frac{b}{k}} \right).
      \end{align*}
      Thus,
      \begin{align*}
        \diff{v}{t} &= g - \frac{1}{m}\diff{m}{t} v\\
                    &= g- \frac{1}{m}\left( km^{2/3} \right)\left(\frac{g}{\frac{4}{3}k + b}m^{\frac{1}{3}}\left( 1-\left( \frac{m_0}{m} \right)^{\frac{4}{3} + \frac{b}{k}} \right)  \right).
      \end{align*}
    \item Using $\diff{m}{t} = km^{2/3}v$, and $\diff{v}{t} = km^{2/3}v\diff{v}{m}$, we obtain
      \begin{align*}
        m\diff{v}{t} + v\diff{m}{t} &= -bm^{2/3}v^{2} + mg\\
        v\:dv + \left( \frac{v^2}{m}\left( 1 + \frac{b}{k} \right) - \frac{g}{km^[2/3]} \right)\:dm &= 0.
      \end{align*}
      This gives $\alpha = v$ and $\beta = \frac{v^2}{m}\left( 1 + \frac{b}{k} \right)- \frac{g}{km^{2/3}}$. Solving for $p(m)$, we get
      \begin{align*}
        p(m) &= \frac{1}{v}\left( \frac{2v}{m}\left( 1 + \frac{b}{k} \right) \right)\\
             &= \frac{2}{m}\left( 1 + \frac{b}{k} \right).
      \end{align*}
      Therefore, our integrating factor is
      \begin{align*}
        w(x) &= m^{2 + \frac{2b}{k}}.
      \end{align*}
      This gives
      \begin{align*}
        \pd{\Phi}{v} &= \alpha\\
                     \Phi &= \frac{1}{2}m^{2 + \frac{2b}{k}}v^2 + c_1(m)\\
        \pd{\Phi}{m} &= \beta\\
        \Phi &= \frac{1}{2}m^{2 + \frac{2b}{k}}v^2 - \frac{g}{k\left( \frac{7}{3} + \frac{2b}{k} \right)}m^{\frac{7}{3} + \frac{2b}{k}} + c_2(v).
      \end{align*}
      Thus, $c_2(v) = 0$, and
      \begin{align*}
        \frac{1}{2}m^{2 + \frac{2b}{k}}v^2 - \frac{g}{k\left( \frac{7}{3} + \frac{2b}{k} \right)}m^{\frac{7}{3} + \frac{2b}{k}} &= C.
      \end{align*}
      Using $v\left( m_0 \right) = 0$, we obtain the solution of
      \begin{align*}
        \frac{1}{2}m^{2 + \frac{2b}{k}}v^2 &= \frac{g}{k\left( \frac{7}{3} + \frac{2b}{k} \right)}m^{\frac{7}{3} + \frac{2b}{k}} \left( 1-\left( \frac{m_0}{m} \right)^{\frac{7}{3} + \frac{2b}{k}} \right).
      \end{align*}
      Simplifying, this gives
      \begin{align*}
        v^2 &= \frac{2g}{k\left( \frac{7}{3} + \frac{2b}{k} \right)}m^{\frac{1}{3}}\left( 1-\left( \frac{m_0}{m} \right)^{\frac{7}{3} + \frac{2b}{k}} \right).
      \end{align*}
      Therefore,
      \begin{align*}
        2v \diff{v}{m} &= \frac{2g}{3k\left( \frac{7}{3} + \frac{2b}{k} \right)}m^{-2/3}\left( 1-\left( \frac{m_0}{m} \right)^{\frac{7}{3} + \frac{2b}{k}} \right) + \frac{2g}{km}\left( \frac{m_0}{m} \right)^{\frac{7}{3} + \frac{2b}{k}},
      \end{align*}
      and
      \begin{align*}
        \diff{v}{t} &= \frac{k}{2}m^{2/3}\left( 2v\diff{v}{m} \right)\\
                    &= \frac{g}{3\left( \frac{7}{3} + \frac{2b}{k} \right)}\left( 1 - \left( \frac{m_0}{m} \right)^{\frac{7}{3} + \frac{2b}{k}} \right) + \frac{g}{m^{\frac{1}{3}}}\left( \frac{m_0}{m} \right)^{\frac{7}{3} + \frac{2b}{k}}.
      \end{align*}
  \end{enumerate}
\end{solution}
\begin{solution}[38.7]

\end{solution}
\begin{solution}[39.5]

\end{solution}
\begin{solution}[39.7]

\end{solution}
\begin{solution}[39.8]

\end{solution}
\begin{solution}[39.13]

\end{solution}
\begin{solution}[39.17]

\end{solution}
\begin{solution}[39.18]

\end{solution}
\begin{solution}[39.21]

\end{solution}
\begin{solution}[39.22 (b)]

\end{solution}
\begin{solution}[39.28]

\end{solution}

\end{document}
