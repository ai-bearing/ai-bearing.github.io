\documentclass[10pt]{mypackage}

% sans serif font:
%\usepackage{cmbright,sfmath,bbold}
%\renewcommand{\mathcal}{\mathtt}

%Euler:
\usepackage{newpxtext,eulerpx,eucal,eufrak}
\renewcommand*{\mathbb}[1]{\varmathbb{#1}}
\renewcommand*{\hbar}{\hslash}

\usepackage{homework}

\pagestyle{fancy} %better headers
\fancyhf{}
\rhead{Avinash Iyer}
\lhead{Assignment 2}

\setcounter{secnumdepth}{0}

\begin{document}
\RaggedRight
\begin{solution}[19.1]\hfill
  \begin{enumerate}[(a)]
    \item There is a simple pole at $z = 0$. The residue at this pole is $0$.
    \item There is a pole of order $4$ at $z = 0$. The residue at this pole is $0$.
    \item There is a pole of order $4$ at $z = 0$. The residue at this pole is $\frac{1}{120}$.
    \item There is an essential singularity at $z = 0$.
    \item There is a removable singularity at $z = 0$.
  \end{enumerate}
\end{solution}
\begin{solution}[19.2]
  The poles of $\frac{e^z}{\sin z}$ occur when $\sin z = 0$, which happens when $z = n\pi$.
\end{solution}
\begin{solution}[19.4]
  There are no residues within $\left\vert z \right\vert < 1$.\newline

  For $1 < \left\vert z \right\vert < 2$, evaluating the $a_{-1}$ term, we have the residue of $\frac{1}{3}$.\newline

  For $\left\vert z \right\vert > 2$, evaluating the $a_{-1}$ term, we have a residue of $\frac{1}{3}$.
\end{solution}
\begin{solution}[19.5]\hfill
  \begin{enumerate}[(a)]
    \item There is a pole of order $2$ at $z = 1$ and a pole of order $1$ at $z = 0$.
    \item Around $z = 0$, we have the expansion
      \begin{align*}
        \frac{1}{z\left( z-1 \right)^2} &= \frac{1}{z\left( 1-z \right)^2}\\
                                        &= \frac{1}{z}\left( \sum_{k=1}^{\infty}kz^{k-1} \right)\\
                                        &= \sum_{k=1}^{\infty}kz^{k-2},
      \end{align*}
      which converges for all $0 \left\vert z \right\vert < 1$. Around $z = 1$, we have the expansion
      \begin{align*}
        \frac{1}{\left( z-1 \right)^2 z} &= \frac{1}{\left( z-1 \right)^2 \left( 1 + z-1 \right)}\\
                                         &= \frac{1}{\left( z- 1\right)^2} \left( \sum_{k=0}^{\infty}\left( -1 \right)^k\left( z-1 \right)^k \right)\\
                                         &= \sum_{k=0}^{\infty} \left( -1 \right)^k\left( z-1 \right)^{k-2}.
      \end{align*}
      This series converges for all $0 < \left\vert z-1 \right\vert < 1$.
    \item The residue at $z = 0$ is $1$, and the residue at $z = 1$ is $-1$.
  \end{enumerate}
\end{solution}
\begin{solution}[19.9]
  If $a$ is not a singularity of $w(z)$, the Laurent expansion collapses into the Taylor expansion.
\end{solution}
\begin{solution}[19.11]

\end{solution}
\begin{solution}[19.13]

\end{solution}
\begin{solution}[19.18]

\end{solution}
\begin{solution}[19.24]
  We must move $e^{2\pi i}$ back into the principal branch to evaluate the square root.
\end{solution}
\begin{solution}[19.28]\hfill
  \begin{enumerate}[(a)]
    \item We have
      \begin{align*}
        e^{iz} &= \cos(z) + i\sin(z)\\
               &= \left( 1-\sin^2(z) \right)^{1/2} + i\sin(z).
      \end{align*}
      Thus, defining $w = \sin(z)$, we have
      \begin{align*}
        iz &= \ln\left( iw + \left( 1- w^2\right)^{1/2} \right)\\
        z &= -i\ln\left( iw + \left( 1-w^2 \right)^{1/2} \right).
      \end{align*}
      Similarly, defining $w = \cos(z)$, we have
      \begin{align*}
        e^{iz} &= \cos(z) + i\left( 1-\cos^2(z) \right)^{1/2}\\
        iz &= \ln\left( w + i\left( 1-w^2 \right)^{1/2} \right)\\
           &= \ln\left( w + i\left( \left( -1 \right)\left( w^2 - 1 \right) \right)^{1/2} \right)\\
           &= \ln\left( w + i\left( -i \right)\left( w^2 - 1 \right)^{1/2} \right)\\
           &= \ln\left( w + \left( w^2 - 1 \right)^{1/2} \right).
      \end{align*}
  \end{enumerate}
\end{solution}

\end{document}
