\documentclass[10pt]{mypackage}

% sans serif font:
%\usepackage{cmbright,sfmath,bbold}
%\renewcommand{\mathcal}{\mathtt}

%Euler:
\usepackage{newpxtext,eulerpx,eucal,eufrak}
\renewcommand*{\mathbb}[1]{\varmathbb{#1}}
\renewcommand*{\hbar}{\hslash}

%\renewcommand{\mathbb}{\mathds}
\usepackage{homework}

\pagestyle{fancy} %better headers
\fancyhf{}
\rhead{Avinash Iyer}
\lhead{Assignment 7}

\setcounter{secnumdepth}{0}

\begin{document}
\RaggedRight
\begin{solution}[30.1]
  \begin{enumerate}[(a)]
    \item This is a legal expression.
    \item This is a legal expression.
    \item This is not a legal expression; we should obtain a dual vector upon acting with $B_{ij}$ on the vector $C^{i}$.
    \item This is not a legal expression assuming summation convention; we cannot have a repeated index on the same tensor.
    \item This is not a legal expression assuming summation convention; we cannot have a repeated index on the same tensor.
    \item This is a legal expression.
    \item This is not a legal expression; we should have a $(0,2)$ tensor in the dual space, $A_{ij}$, rather than a $(1,1)$ tensor of the form $A_{i}^{j}$.
    \item This is a legal expression.
  \end{enumerate}
\end{solution}
\begin{solution}[30.3]
  Using the chain rule, we obtain
  \begin{align*}
    A^{j'}B_{j'} &= \pd{u^{j'}}{u^{j}}A^{j}\pd{u^{j}}{u^{j'}}B_{j}\\
                 &= \delta_{j}^{j}A^jB_j\\
                 &= A^jB_j.
  \end{align*}
  Meanwhile,
  \begin{align*}
    A^{j'}B^{j'} &= \pd{u^{j'}}{u^{j}}A^{j}\pd{u^{j'}}{u^{j}}B^{j}\\
                 &= \pd{u^{j'}}{u^{j}}\pd{u^{j'}}{u^{j}}A^{j}B^{j},
  \end{align*}
  which means $A^{j'}B^{j'}$ is a rank $(2,0)$ tensor.
\end{solution}
\begin{solution}[30.5]

\end{solution}
\begin{solution}[30.6]

\end{solution}
\begin{solution}[30.16]

\end{solution}
\begin{solution}[30.20]

\end{solution}
\begin{solution}[30.21]

\end{solution}
\begin{solution}[30.22]

\end{solution}
\begin{solution}[30.28]

\end{solution}
\end{document}
