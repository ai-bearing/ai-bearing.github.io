\documentclass[10pt]{mypackage}

% sans serif font:
%\usepackage{cmbright,sfmath,bbold}
%\renewcommand{\mathcal}{\mathtt}

%Euler:
\usepackage{newpxtext,eulerpx,eucal,eufrak}
\renewcommand*{\mathbb}[1]{\varmathbb{#1}}
\renewcommand*{\hbar}{\hslash}

%\renewcommand{\mathbb}{\mathds}
\usepackage{homework}

\pagestyle{fancy} %better headers
\fancyhf{}
\rhead{Avinash Iyer}
\lhead{Assignment 7}

\setcounter{secnumdepth}{0}

\begin{document}
\RaggedRight
\begin{solution}[30.1]
  \begin{enumerate}[(a)]
    \item This is a legal expression.
    \item This is a legal expression.
    \item This is not a legal expression; we should obtain a dual vector upon acting with $B_{ij}$ on the vector $C^{i}$.
    \item This is not a legal expression assuming summation convention; we cannot have a repeated index on the same tensor.
    \item This is not a legal expression assuming summation convention; we cannot have a repeated index on the same tensor.
    \item This is a legal expression.
    \item This is not a legal expression; we should have a $(0,2)$ tensor in the dual space, $A_{ij}$, rather than a $(1,1)$ tensor of the form $A_{i}^{j}$.
    \item This is a legal expression.
  \end{enumerate}
\end{solution}
\begin{solution}[30.3]
  Using the chain rule, we obtain
  \begin{align*}
    A^{j'}B_{j'} &= \pd{u^{j'}}{u^{j}}A^{j}\pd{u^{j}}{u^{j'}}B_{j}\\
                 &= \delta_{j}^{j}A^jB_j\\
                 &= A^jB_j.
  \end{align*}
  Meanwhile,
  \begin{align*}
    A^{j'}B^{j'} &= \pd{u^{j'}}{u^{j}}A^{j}\pd{u^{j'}}{u^{j}}B^{j}\\
                 &= \pd{u^{j'}}{u^{j}}\pd{u^{j'}}{u^{j}}A^{j}B^{j},
  \end{align*}
  which means $A^{j'}B^{j'}$ is a rank $(2,0)$ tensor.
\end{solution}
\begin{solution}[30.5]
  The matrix
  \begin{align*}
    g_{ab} &= \begin{pmatrix}1 & \cos\left( \phi \right)\\\cos\left( \phi \right)1\end{pmatrix}
  \end{align*}
  has inverse
  \begin{align*}
    g^{ab} &= \begin{pmatrix}\csc^2\left( \phi \right) & -\cot\left( \phi \right)\csc\left( \phi \right) \\ -\cot\left( \phi \right)\csc\left( \phi \right) & \csc^2\left( \phi \right)\end{pmatrix},
  \end{align*}
  where $\cos\left( \phi \right) = \sin\left( \alpha + \beta \right)$. Therefore, we may calculate
  \begin{align*}
    \vec{e}^{a} &= g^{ab}\vec{e}_b\\
                &= \frac{1}{\cos\left( \alpha + \beta \right)} \begin{pmatrix}\cos\left( \beta \right)\\-\sin\left( \beta \right)\end{pmatrix}\\
    \vec{e}^{b} &= g^{ab}\vec{e}_a\\
                &= \frac{1}{\cos\left( \alpha + \beta \right)}\begin{pmatrix} -\sin\left( \alpha \right)\\\cos\left( \alpha \right) \end{pmatrix}.
  \end{align*}
\end{solution}
\begin{solution}[30.6]\hfill
  \begin{enumerate}[(a)]
    \item We have the downstairs basis of $\set{\hat{r},r\hat{\phi},\hat{z}}$ for cylindrical coordinates.
    \item Using the metric of
      \begin{align*}
        g^{ab} &= \begin{pmatrix}1 & & \\ & r^2 & \\ & & 1\end{pmatrix},
      \end{align*}
      we calculate
      \begin{align*}
        \vec{e}^{r} &= g^{ab}\vec{e}_r\\
                    &= \hat{r}\\
        \vec{e}^{\phi} &= g^{ab}\vec{e}_{\phi}\\
                       &= \frac{1}{r}\hat{\phi}\\
        \vec{e}^{z} &= g^{ab}\vec{e}_{z}\\
                    &= \hat{z}.
      \end{align*}
    \item We calculate
      \begin{align*}
        A_{r}\vec{e}^{r} &= A_{r}\hat{r}\\
                         &= A^{r}\hat{r}\\
        A_{\phi}\vec{e}^{\phi} &= \frac{1}{r}A_{\phi}\hat{\phi}\\
                               &= A^{\phi}\hat{\phi}\\
        A_{z}\vec{e}^{z}  &= A_{z}\hat{z}\\
                          &= A^{z}\hat{z}.
      \end{align*}
      Thus,
      \begin{align*}
        A_r &= A^r\\
        A_{\phi} &= \frac{1}{r}A^{\phi}\\
        A_{z} &= A^{z}.
      \end{align*}
    \item We have
      \begin{align*}
        A^r\vec{e}_r &= A^r\hat{r}\\
                     &= A_r\hat{r}\\
                     &= A_r\vec{e}^r\\
        A^{\phi}\vec{e}_{\phi} &= A^{\phi}\hat{\phi}\\
                               &= rA_{\phi}\hat{\phi}\\
                               &= A_{\phi}\vec{e}^{\phi}.\\
        A^z\vec{e}_z &= A^z\hat{z}\\
                     &= A_z\hat{z}\\
                     &= A_z\vec{e}_z.
      \end{align*}
  \end{enumerate}
\end{solution}
\begin{solution}[30.16]\hfill
  \begin{enumerate}[(a)]
    \item Let $g_{ab} = \operatorname{diag}\left( 1,r^2,r^2\sin^2\left( \theta \right) \right)$ be the metric for spherical coordinates. Then,
      \begin{align*}
        g_{rr} &= \left( \pd{\rho}{r} \right)^2 + \left( \pd{\phi}{r} \right)^2 + \left( \pd{\theta}{r} \right)^2\\
               &= \frac{z^2}{r^2} + 1 + \frac{1}{z^2}\\
        g_{\phi\phi} &= 1\\
        g_{zz} &= \frac{r^2}{z^2} + 1 + \frac{1}{r^2}.
      \end{align*}
    \item We convert from spherical to cylindrical by converting from spherical to Cartesian by taking $g^{ab}$ on spherical coordinates, summing over $\delta_{ij}$, then multiplying by the cylindrical metric, with some relabeling, giving
      \begin{align*}
        g_{rr} &= \frac{z^2}{r^2} + 1 + \frac{1}{z^2}\\
        g_{\phi\phi} &= 1\\
        g_{zz} &= \frac{r^2}{z^2} + 1 + \frac{1}{r^2}.
      \end{align*}
      This approach is simpler because we get to take advantage of the properties of the Cartesian metric (i.e., that it is independent of location).
  \end{enumerate}
\end{solution}
\begin{solution}[30.20]
  Calculating
  \begin{align*}
    \sqrt{\det\left( g_{ab} \right)} &= \sqrt{1-\cos^2\left( \varphi \right)},
  \end{align*}
  we may calculate
  \begin{align*}
    \int_{A}^{} \:d\tau' &= \int_{0}^{b}\int_{0}^{a} \sqrt{1-\cos^2\left( \varphi \right)}\:du\:dv\\
                         &= ab\sqrt{1-\cos^2\left( \varphi \right)}.
  \end{align*}
\end{solution}
\begin{solution}[30.21]\hfill
  \begin{enumerate}[(a)]
    \item We take
      \begin{align*}
        g_{ab} &= \begin{pmatrix}R^2 & \\ & R^2\sin^2\left( \theta \right)\end{pmatrix}.
      \end{align*}
    \item Calculating
      \begin{align*}
        \int_{C}^{} \:ds &= \int_{C}^{}\sqrt{\left\vert g_{ab}du^{a}du^{b} \right\vert}\\
                         &= R\int_{0}^{\theta_0} \:d\theta\\
                         &= R\theta_{0}.
      \end{align*}
    \item The circumference $C$ of the circle $\theta = \theta_0$ is equal to $2\pi R \sin\left( \theta_0 \right)$, which is not equal to $2\pi s$ because observers on the sphere do not perceive its curvature.
    \item Without loss of generality we may assume we are on the north pole. Then, very close to the north pole, we have very small $\theta_0$, or that $\theta_0\approx \sin\left( \theta_0 \right)$.\newline

      Thus, a small neighborhood of the sphere with radius $r \ll R$ will appear as a disc of radius $r$ rather than a curved section whose boundary has radius $r\sin\left( \theta_0 \right)$.
  \end{enumerate}
\end{solution}
\begin{solution}[30.22]\hfill
  \begin{enumerate}[(a)]
    \item Assuming $r_2 > r_1 > r_s$, we calculate
      \begin{align*}
        \int_{r_1}^{r_2} \:ds &= \int_{r_1}^{r_2}\sqrt{\left\vert g_{ab}du^{a}du^{b} \right\vert}\\
                              &= \int_{r_1}^{r_2} \left( 1-\frac{r_s}{r} \right)^{-1}\:dr\\
                              &= \int_{r_1}^{r_2} \frac{r}{r-r_s}\:dr\\
                              &= \int_{r_1}^{r_2} 1 + \frac{r_s}{r-r_s}\:dr\\
                              &= \left( r_2-r_1 \right) + r_s\ln\left( \frac{r_2-r_s}{r_1-r_s} \right)\\
                              &\geq r_2-r_1.
      \end{align*}
    \item Fixing $R$, we have
      \begin{align*}
        \int_{}^{} \:dA &= \int_{0}^{2\pi}\int_{0}^{\pi}R^2\sin\left( \theta \right)\:d\theta d\phi\\
                        &= 4\pi R^2.
      \end{align*}
      In this sense, $R$ is a measure of (warped) constant radius.
    \item I don't know how to do this.
    \item As Bob approaches $r = r_s$, Alice perceives time to slow down for Bob.
  \end{enumerate}
\end{solution}
\begin{solution}[30.28]\hfill
  \begin{enumerate}[(a)]
    \item Using $u^{c} = \hat{\theta}$ and $u^a,u^b = r\sin\theta \hat{\phi}$, with the Christoffel symbol $\Gamma_{\phi\phi}^{\theta} = -\sin\left( \theta \right)\cos\left( \theta \right)$, we substitute to obtain
      \begin{align*}
        \diff{^2\theta}{t^2} - \sin\theta\cos\theta \left( \diff{\phi}{t} \right)^2 &= 0.
      \end{align*}
      Similarly, expanding the other equation, we have
      \begin{align*}
        \diff{}{t}\left( \sin^2\left( \theta \right)\diff{\phi}{t} \right) &= 2\sin\left( \theta \right)\cos\left( \theta \right)\diff{\theta}{t}\diff{\phi}{t} - \sin^2\left( \theta \right)\diff{^2\phi}{t^2}\\
                                                                           &= 0.
      \end{align*}
      When we take $u^{c} = \phi$ and $\Gamma_{\phi\theta}^{\phi} = \cot\left( \theta \right)$, we get the geodesic equation yet again.
    \item With initial velocity along $\hat{\phi}$ at the equator, the geodesic equation evaluates to yield constant $\theta = \pi/2$, constant $r = R$, and $\dot{\phi} = k$ for some constant $k$. In other words, this yields a great circle.
    \item Since the geodesic equation is a covariant expression, we may use a series of transformations to give any other starting position to be equal to the case in part (b), meaning that all geodesics are great circles.
  \end{enumerate}
\end{solution}
\end{document}
