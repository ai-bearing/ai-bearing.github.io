\documentclass[10pt]{mypackage}

% sans serif font:
%\usepackage{cmbright,sfmath,bbold}
%\renewcommand{\mathcal}{\mathtt}

%Euler:
\usepackage{newpxtext,eulerpx,eucal,eufrak}
\renewcommand*{\mathbb}[1]{\varmathbb{#1}}
\renewcommand*{\hbar}{\hslash}

\usepackage{homework}

\pagestyle{fancy} %better headers
\fancyhf{}
\rhead{Avinash Iyer}
\lhead{Assignment 3}

\setcounter{secnumdepth}{0}

\begin{document}
\RaggedRight
\begin{solution}[20.1]
  We know that $\sin(z)$ is conformal when $\diff{}{z}\left( \sin(z) \right) \neq 0$, meaning that we verify when $\cos(z) \neq 0$. This occurs at $z = n\pi$, where $n\in \Z$.\newline

  We know that $\sin(z) = 0$ when $z = \pi$, with $\cos(z) = -1 = e^{i\pi}$. Therefore, the image of $z = \pi$ is not stretched, and is rotated by an angle of $\pi$.\newline

  We know that the image of $z = i\pi$ is stretched by a factor of $\left\vert \cos\left( i\pi \right) \right\vert = \left\vert \cosh\left( \pi \right) \right\vert$. Since $\cosh\left( \pi \right) = \left\vert \cosh\left( \pi \right) \right\vert$, the image is rotated by an angle of $0$.\newline

  Evaluating $\cos\left( \pi/2 + i\pi\right)$, we get that it is equal to $-\sin\left( \pi/2 \right)\sin\left( i \right)$, or $-i\sinh(1) = \sinh(1)e^{-i\pi/2}$. Therefore, the image of $z = \pi/2 + i$ is stretched by a factor of $\sinh(1)$ and rotated by an angle of $-\pi/2$.
\end{solution}
\begin{solution}[20.9]
  Mapping $\left\vert z-1 \right\vert < 1$ to the plane $\re(w) > 0$, with $w(0) = \infty$, we have
  \begin{align*}
    w(z) &= \frac{az + b}{cz}.
  \end{align*}
  Now, we want $z = 2$ to map to zero, giving
  \begin{align*}
    w(z) &= \frac{a\left( z-2 \right)}{cz}.
  \end{align*}
  Finally, an entirely arbitrary decision made by the problem solver has it such that $z = 1$ maps to $z = 1$. Thus, we have
  \begin{align*}
    \frac{-a}{c} &= 1.
  \end{align*}
  Therefore, we get the Möbius transformation of
  \begin{align*}
    w(z) &= \frac{2-z}{z}.
  \end{align*}
\end{solution}
{\tiny \textbf{Problem Solver's Note:} It is not possible for $\left\vert z-1 \right\vert < 0$, as norms are always at least equal to zero. The problem solver has decided to interpret the question such that it becomes nontrivial.}
\begin{solution}[20.10]
  The first map of $e^z$ has it such that $\re(w)$ ranges from $e^{\re\left(z_1\right)}$ to $e^{\re\left( z_2 \right)}$, while $\arg(w)$ ranges from $0$ to $\pi$, which agrees with the map showing an annular strip in the $w$-plane.\newline

  The second map of $e^{z}$ maps $z_1$, $z_2$, and $z_3$ to $e^{\re\left( z_1 \right)}$, 1, and $e^{\re\left( z_3 \right)}$, eventually converging to $0$ as $z_3$ becomes more and more negative. Similarly, $e^{z}$ maps $z_4$, $z_5$, and $z_6$ to $e^{i\pi\re\left( z_4 \right)}$, $-1$, and $e^{i\pi\re\left( z_6 \right)}$, similarly converging to $0$ as $z_4$ becomes more and more negative.
\end{solution}
\begin{solution}[20.11]

\end{solution}
\begin{solution}[20.12]

\end{solution}
\begin{solution}[20.14]

\end{solution}
\begin{solution}[20.15]

\end{solution}
\begin{solution}[20.16]

\end{solution}
\begin{solution}[20.17]

\end{solution}
\end{document}
