\documentclass[10pt]{mypackage}

% sans serif font:
%\usepackage{cmbright,sfmath,bbold}
%\renewcommand{\mathcal}{\mathtt}

%Euler:
%\usepackage{newpxtext,eulerpx,eucal,eufrak}
%\renewcommand*{\mathbb}[1]{\varmathbb{#1}}
%\renewcommand*{\hbar}{\hslash}

%kp fonts:
\usepackage{kpfonts}
\renewcommand{\mathbb}{\mathds}
\usepackage{homework}

\pagestyle{fancy} %better headers
\fancyhf{}
\rhead{Avinash Iyer}
\lhead{Math 400: Homework 1}

\setcounter{secnumdepth}{0}

\begin{document}
\RaggedRight
\begin{problem}[Problem 1]
  Every square $S$ in an infinite chessboard contains a positive integer equal to the average of the four non-diagonal neighbors of $S$. Are the integers necessarily equal?
\end{problem}
\begin{problem}[Problem 2]
  Alice has $n+1$ binary strings of length $n\geq 2$. Bob knows this but knows nothing about the bits in the strings. Bob wants to specify a binary string of length n that is not in Alice’s set. Bob can ask Alice questions of the form: ``What is the $j$th bit of the $k$th string?'' What is the minimal number of questions that Bob needs to ask to guarantee success?
\end{problem}
\begin{problem}[Problem 3]
  Show that among $n+1$ integers not greater than $2n$ that there are two such that one divides the other.
\end{problem}
\begin{problem}[Problem 4]
Show that for every graph there is an orientation of the edges such that for every vertex the out-degree and in-degree differ by at most $1$.
\end{problem}
\begin{solution}
  Without loss of generality, we assume $G$ is connected. If $G$ is disconnected, we apply the following procedure to each of the connected components.\newline

  We define the ``net degree'' of a vertex in a directed graph to be equal to the difference between its out-degree and its in-degree. Additionally, we say the directed graph $G$ is strongly oriented if, for any $v$ and $w$ in $G$, there is a directed path from $v$ to $w$, and a directed path from $w$ to $v$. Note that if $G$ is strongly oriented, then every vertex in $G$ has net degree zero, as any pair of vertices is contained in a directed circuit.\newline

  By Robbins's theorem, if $G$ does not contain a cut-edge (or bridge), then $G$ always admits a strong orientation. If $G$ contains a cut-edge (or multiple cut-edges), we define $G_1,\dots,G_n$ to be the components that result after all cut-edges are removed. Each of these components can be strongly oriented by construction. If any of the components are isolated vertices, we apply the orientation on the edges incident on these components such that the net degree of the vertex is either $0$, $-1$ or $1$, which is always possible as the number of edges incident on a vertex is either even or odd. Otherwise, all cut-edges incident on vertices in a more-than-one-vertex connected component still maintain a satisfactory orientation.
\end{solution}

\begin{problem}[Problem 5]
  A convex board is surrounded by some nails hammered into a table; the nails make it impossible to slide the board in any direction, but if any of them is missing then this is no longer true. What is the maximal number of nails?
\end{problem}

\end{document}
