\documentclass[10pt]{mypackage}

% sans serif font:
%\usepackage{cmbright,sfmath,bbold}
%\renewcommand{\mathcal}{\mathtt}

%Euler:
\usepackage{newpxtext,eulerpx,eucal,eufrak}
\renewcommand*{\mathbb}[1]{\varmathbb{#1}}
\renewcommand*{\hbar}{\hslash}

%\renewcommand{\mathbb}{\mathds}
\usepackage{homework}
%\usepackage{exposition}

\pagestyle{fancy} %better headers
\fancyhf{}
\rhead{Avinash Iyer}
\lhead{Partial Differential Equations: Homework 6}

\setcounter{secnumdepth}{0}

\begin{document}
\RaggedRight
\begin{solution}[11.2, Problem 2]
  We evaluate
  \begin{align*}
    a_n &= \frac{1}{\pi}\int_{-\pi}^{\pi} \cos\left( nx \right)f(x)\:dx\\
        &= \begin{cases}
          1 & n = 0\\
          0 & \text{else}
        \end{cases}\\
    b_n &= \frac{1}{\pi}\int_{-\pi}^{\pi} \sin\left( nx \right)f(x)\:dx\\
        &= \frac{1}{\pi}\int_{0}^{\pi} \sin\left( nx \right)\:dx\\
        &= \frac{1}{\pi}\left( \frac{1}{n}\cos\left( nx \right)\biggr\vert_{0}^{\pi} \right)\\
        &= \frac{1}{n\pi}\left( \left( -1 \right)^{n}-1 \right).
  \end{align*}
  Therefore, our Fourier series is
  \begin{align*}
    f(x) &= 1 + \sum_{n=1}^{\infty}\frac{\left( -1 \right)^{n}-1}{n\pi} \sin\left( nx \right).
  \end{align*}
\end{solution}
\begin{solution}[11.2, Problem 8]
  We evaluate
  \begin{align*}
    a_n &= \frac{1}{\pi}\int_{-\pi}^{\pi} \cos\left( nx \right)\left( 3-2x \right)\:dx\\
        &= \frac{1}{\pi} \int_{-\pi}^{\pi} 3\cos\left( nx \right) - 2x\cos\left( nx \right)\:dx\\
        &= \begin{cases}
          3 & n = 0\\
          0 & \text{else}.
        \end{cases}\\
    b_n &= \frac{1}{\pi} \int_{-\pi}^{\pi} \sin\left( nx \right)\left( 3-2x \right)\:dx\\
        &= \frac{1}{\pi} \int_{-\pi}^{\pi} 3\sin\left( nx \right) - 2x\sin\left( nx \right)\:dx\\
        &= \frac{1}{\pi}\left( \frac{3}{n}\cos\left( nx \right)\biggr\vert_{-\pi}^{\pi} - 2\left( \frac{x\cos\left( nx \right)}{n} + \frac{\sin\left( nx \right)}{n^2} \right)\biggr\vert_{-\pi}^{\pi}\right)\\
        &= \frac{4\left( -1 \right)^{n}}{n}.
  \end{align*}
  Thus, our Fourier series is
  \begin{align*}
    f(x) &= 3 + 4\sum_{n=1}^{\infty}\frac{\left( -1 \right)^{n}}{n}\sin\left( nx \right).
  \end{align*}
\end{solution}
\begin{solution}[11.2, Problem 10]
  Using integration by parts, we evaluate
  \begin{align*}
    a_n &= \frac{2}{\pi} \int_{0}^{\pi/2} \cos\left( \frac{n}{2}x \right)\cos\left( x \right)\:dx\\
        &= \begin{cases}
          \frac{8}{\pi\left( 4-n^2 \right)} \cos\left( \frac{n\pi}{4} \right) & n\neq 2\\
          \frac{1}{2} & n = 2
        \end{cases}.\\
    b_n &= \frac{2}{\pi} \int_{0}^{\pi/2} \sin\left( \frac{n}{2}x \right)\cos\left( x \right)\:dx\\
        &= 
        \begin{cases}
          \frac{8}{\pi\left( 4-n^2 \right)}\left( \sin\left( \frac{n\pi}{4} \right) - \frac{n}{2}\cos\left( \frac{n\pi}{4} \right) \right). & n\neq 2\\
          \frac{1}{\pi} & n = 2
        \end{cases}.
  \end{align*}
  Thus, with $a_0 = \frac{2}{\pi}$, we have the Fourier series 
  \begin{align*}
    f(x) &= \frac{1}{\pi} + \sum_{n=1}^{\infty}a_n\cos\left( \frac{n}{2}x \right) + b_n\sin\left( \frac{n}{2}x \right).
  \end{align*}
\end{solution}
\begin{solution}[11.2, Problem 17]
  We first start by finding the series expansion. Evaluating, we have
  \begin{align*}
    a_n &= \frac{1}{\pi} \int_{0}^{\pi} x^2\cos\left( nx \right)\:dx\\
        &= 
        \begin{cases}
          \frac{2\left( -1 \right)^{n}}{n^2} & n > 0\\
          \frac{\pi^2}{3} & n = 0
        \end{cases}\\
    b_n &= \frac{1}{\pi} \int_{0}^{\pi} x^2\sin\left( nx \right)\:dx\\
        &= \frac{\pi}{n}\left( -1 \right)^{n+1} + \frac{2}{\pi n^3}\left( \left( -1 \right)^{n} - 1 \right).
  \end{align*}
  Thus, we have the Fourier series
  \begin{align*}
    x^2 &= \frac{\pi^2}{6} + \sum_{n=1}^{\infty}\frac{2\left( -1 \right)^{n}}{n^2}\cos\left( nx \right) + \left( \frac{\pi}{n}\left( -1 \right)^{n+1} + \frac{2}{\pi n^3}\left( \left( -1 \right)^{n} - 1 \right) \right)\sin\left( nx \right).
  \end{align*}
  Using the input $x = 0$, we get
  \begin{align*}
    0 &= \frac{\pi^2}{6} + \sum_{n=1}^{\infty}\frac{2\left( -1 \right)^{n}}{n^2}\cos\left( nx \right)\\
    \sum_{n=1}^{\infty}\frac{\left( -1 \right)^{n+1}}{n^2} &= \frac{\pi^2}{12}.
  \end{align*}
  Meanwhile, using the input of $-\pi$, we have $f(-\pi) = 0$, and
  \begin{align*}
    0 &= \frac{\pi^2}{3} + \sum_{n=1}^{\infty} \frac{2}{n^2}\\
    \frac{\pi^2}{6} &= \sum_{n=1}^{\infty}\frac{1}{n^2}.
  \end{align*}
\end{solution}
\begin{solution}[11.2, Problem 18]
  Adding, we get
  \begin{align*}
    \frac{\pi^2}{8} &= \sum_{n\text{ odd}} \frac{1}{n^2}.
  \end{align*}
\end{solution}
\begin{solution}[11.3, Problem 6]
  The function
  \begin{align*}
    f(x) &= e^{x} - e^{-x}
  \end{align*}
  is odd.
\end{solution}
\begin{solution}[11.3, Problem 10]
  The function
  \begin{align*}
    f(x) &= \left\vert x^{5} \right\vert
  \end{align*}
  is even.
\end{solution}
\begin{solution}[11.3, Problem 12]
  This function is even, so we expand in the cosine series. This gives
  \begin{align*}
    a_n &= \int_{0}^{2} f(x)\cos\left( \frac{n\pi }{2} x\right)\:dx\\
        &= \int_{1}^{2} \cos\left( \frac{n\pi }{2}x \right)\:dx\\
        &= \begin{cases}
          \frac{2}{n\pi} & n\text{ even}\\
          0 & n\text{ odd}
        \end{cases}.
  \end{align*}
\end{solution}
\begin{solution}[11.3, Problem 18]
  This function is odd, so we expand in the sine series. This gives
  \begin{align*}
    b_n &= \frac{2}{\pi}\int_{0}^{\pi} x^3\sin\left( nx \right)\:dx\\
        &= \frac{\left( -1 \right)^{n+1}\pi^3}{n} + \frac{6\pi\left( -1 \right)^{n}}{n^3}.
  \end{align*}
\end{solution}
\begin{solution}[11.3, Problem 20]
  This function is odd, so we expand in a sine series.
  \begin{align*}
    b_n &= \frac{2}{\pi} \int_{0}^{\pi} \left( x + 1 \right)\sin\left( nx \right)\:dx\\
        &= \frac{2}{\pi} \int_{0}^{\pi} x\sin\left( nx \right) + \sin\left( nx \right)\:dx\\
        &= \frac{2\left( 1 + \left( -1 \right)^{n+1} + \left( -1 \right)^{n+1}\pi \right)}{n\pi}.
  \end{align*}
\end{solution}
\begin{solution}[11.3, Problem 34]
  Since $f(0) = 0$, we expand in a sine series. This gives
  \begin{align*}
    b_n &= \int_{0}^{2} x\left( 2-x \right)\sin\left( \frac{n\pi}{2}x \right)\:dx
        &= \frac{16}{n^3\pi^3}\left( 1 + \left( -1 \right)^{n+1} \right).
  \end{align*}
\end{solution}
\begin{solution}[11.4, Problem 2]
  We have the cases of $\lambda = \alpha^2,0,-\alpha^2$, with corresponding solution forms of
  \begin{align*}
    y &= c_1\cos\left( \alpha x \right) + c_2\sin\left( \alpha x \right)\\
    y &= c_1 + c_2 x\\
    y &= c_1\cosh\left( \alpha x \right) + c_2\sinh\left( \alpha x \right).
  \end{align*}
  Substituting our boundary conditions for each of these cases, where the first listed equation is the case $y(1) = 0$ and the second listed equation is the case $y(0) + y'(0) = 0$, we have
  \begin{align*}
    c_1\cos\left( \alpha \right) + c_2\sin\left( \alpha \right) &= 0\\
    \alpha\left( c_2-c_1 \right) &= 0\\
    \\
    c_1 + c_2 &= 0\\
    c_1 + c_2 &= 0\\
    \\
    c_1\cosh\left( \alpha \right) + c_2\sinh\left( \alpha \right) &= 0\\
    \alpha\left( c_1 + c_2 \right) &= 0.
  \end{align*}
  In the case with $\lambda = \alpha^2$, we have $c_1 = c_2$ (as $\alpha\neq 0$), giving $y(1) = c_1\cos\left( \alpha \right) + c_1\sin\left( \alpha \right)$, which simplifies to $\sqrt{2}c_1\sin\left( \alpha + \frac{\pi}{4} \right) = 0$. This has solutions of $\alpha = n\pi - \frac{\pi}{4}$, where $n\in\Z$.\newline

  In the case with $\lambda = 0$, we have the case $y = c_1 - c_1 x$.\newline

  The case with $\lambda = -\alpha^2$ simplifies to $c_1 = -c_2$, and $c_1\cosh\left( \alpha \right) + c_1\sinh\left( \alpha \right) = 0$, which is only true if $c_1 = c_2 = 0$. This is a trivial solution.\newline

  Therefore, we have eigenvalues $\lambda = 0,\left( \pi n - \frac{\pi}{4} \right)^2$. 
  \begin{align*}
    \lambda_1 &= 0\\
    \lambda_2 &= \frac{\pi^2}{16}\\
    \lambda_3 &= \frac{9\pi^2}{16}\\
    \lambda_4 &= \frac{25\pi^2}{16}.
  \end{align*}
\end{solution}
\begin{solution}[11.4, Problem 4]
  We have the periodic Sturm--Liouville boundary condition of
  \begin{align*}
    y\left(-L\right) - y\left(L\right) &= 0\\
    y'\left(-L\right) - y'\left(L\right) &= 0.
  \end{align*}
  Letting $\lambda = \alpha^2$, we have solutions of the form
  \begin{align*}
    y(x) &= c_1\cos\left( \alpha x \right) + c_2\sin\left( \alpha x \right).
  \end{align*}
  Using the first boundary condition, we have
  \begin{align*}
    2c_2\sin\left( \alpha L \right) &= 0,
  \end{align*}
  so $\alpha = \frac{n\pi}{L}$, where $n\in\Z$. Varying our coefficients and our values of $n$, we recover the family
  \begin{align*}
    y &= \set{1,\sin\left( \frac{n\pi}{L} \right),\cos\left( \frac{n\pi}{L} \right) | n\in\Z}.
  \end{align*}
\end{solution}
\begin{solution}[11.4, Problem 8]\hfill
  \begin{enumerate}[(a)]
    \item 
  \end{enumerate}
\end{solution}
\begin{solution}[11.4, Problem 10]

\end{solution}
\begin{solution}[12.3, Problem 2]

\end{solution}
\begin{solution}[12.3, Problem 4]

\end{solution}

\end{document}
