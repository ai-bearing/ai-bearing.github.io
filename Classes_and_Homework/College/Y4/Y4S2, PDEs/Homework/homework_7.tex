\documentclass[10pt]{mypackage}

% sans serif font:
%\usepackage{cmbright,sfmath,bbold}
%\renewcommand{\mathcal}{\mathtt}

%Euler:
\usepackage{newpxtext,eulerpx,eucal,eufrak}
\renewcommand*{\mathbb}[1]{\varmathbb{#1}}
\renewcommand*{\hbar}{\hslash}

%\renewcommand{\mathbb}{\mathds}
\usepackage{homework}
%\usepackage{exposition}

\pagestyle{fancy} %better headers
\fancyhf{}
\rhead{Avinash Iyer}
\lhead{Partial Differential Equations: Homework 7}

\setcounter{secnumdepth}{0}

\begin{document}
\RaggedRight
\begin{solution}[12.4, Problem 6]
  Upon separation of variables, we get
  \begin{align*}
    \frac{1}{a^2 T}\diff{^2T}{t^2} &= \frac{1}{X}\diff{^2X}{x^2}\\
                               &\ \begin{cases}
                                 k^2\\
                                 0\\
                                 -k^2
                               \end{cases}.
  \end{align*}
  Using some black magic, we get the cases of
  \begin{align*}
    T(x) &= \begin{cases}
      Ae^{akt} & k^2\\
      At + B & 0\\
      A\cos\left( akt \right) + B\sin\left( akt \right) & -k^2
    \end{cases}\\
    X(x) &= \begin{cases}
      Ce^{kx} & k^2\\
      Cx + D & 0\\
      C\cos\left( kx \right) + D\sin\left( kx \right) & -k^2
    \end{cases}.
  \end{align*}
  By plugging in the boundary conditions of $u(0,t) = u\left( 1,t \right) = 0$, we quickly remove the former two cases, we are of the form
  \begin{align*}
    T(t) &= A\cos\left( akt \right) + B\sin\left( akt \right)\\
    X(x) &= C\cos\left( kx \right) + D\sin\left( kx \right).
  \end{align*}
  Since $X(0) = 0$, we must have $C = 0$, and since $X(1) = 0$, we have $k= n\pi$, $n\in\Z$. Thus, we have functions of the form
  \begin{align*}
    u_n\left( x,t \right) &= \left( A_n\cos\left( n\pi at \right) + B_n\sin\left( n\pi at \right) \right) \sin\left( n\pi x \right),
  \end{align*}
  and the general solution of
  \begin{align*}
    u\left( x,t \right) &= \sum_{n=1}^{\infty} \left( A_n\cos\left( n\pi a t \right) + B_n\sin\left( n\pi a t \right) \right)\sin\left( n\pi x \right).
  \end{align*}
  Plugging in the initial condition, we have
  \begin{align*}
    u\left( x,0 \right) &= \sum_{n=1}^{\infty} A_n\sin\left( n\pi x \right)\\
                        &= \frac{1}{100}\sin\left( 3\pi x \right),
  \end{align*}
  so that $A_n = \frac{1}{100}$ at $x = 3$ and $0$ elsewhere. Writing our amended solution, we have
  \begin{align*}
    u\left( x,0 \right) &= \left( \frac{1}{100}\cos\left( 3\pi a t \right) + B_{3}\sin\left( 3\pi a t \right) \right)\sin\left( 3\pi a x \right).
  \end{align*}
  Taking derivatives, we have
  \begin{align*}
    \pd{u}{t}\biggr\vert_{(x,0)} &= B_3\sin\left( 3\pi a x \right)\\
                                 &= 0,
  \end{align*}
  so $B_3 = 0$, and we arrive at the solution
  \begin{align*}
    u\left( x,t \right) &= \frac{1}{100}\cos\left( 3\pi a t \right)\sin\left( 3\pi x \right).
  \end{align*}
\end{solution}
\begin{solution}[12.4, Problem 8]
  Upon separation of variables, we get
  \begin{align*}
    \frac{1}{a^2 T}\diff{^2T}{t^2} &= \frac{1}{X}\diff{^2X}{x^2}\\
                               &\ \begin{cases}
                                 k^2\\
                                 0\\
                                 -k^2
                               \end{cases}.
  \end{align*}
  Using some black magic, we get the cases of
  \begin{align*}
    T(x) &= \begin{cases}
      Ae^{akt} & k^2\\
      At + B & 0\\
      A\cos\left( akt \right) + B\sin\left( akt \right) & -k^2
    \end{cases}\\
    X(x) &= \begin{cases}
      Ce^{kx} & k^2\\
      Cx + D & 0\\
      C\cos\left( kx \right) + D\sin\left( kx \right) & -k^2
    \end{cases}.
  \end{align*}
  We plug in the boundary conditions of $\pd{u}{x}\biggr\vert_{x=0} = \pd{u}{x}\biggr\vert_{x=L} = 0$ to obtain
  \begin{align*}
    X_n\left( x \right) &= \begin{cases}
      C_n\cos\left( \frac{n\pi}{L}x \right) & -k^2\\
      Cx + D & 0
    \end{cases}\\
      T_n(t) &= \begin{cases}
        B_n\cos\left( \frac{n\pi a}{L}t \right) & -k^2\\
        At + B & 0
      \end{cases}
  \end{align*}
  We may evaluate the solution
  \begin{align*}
    u\left( x,t \right) &= X_0(x)T_0(t) + \sum_{n=1}^{\infty}D_n\cos\left( \frac{n\pi}{L} x \right) \cos\left( \frac{n\pi a}{L}t \right).
  \end{align*}
  To do this, we start with the initial condition, giving $T_0(t) = 1$ and $X_0(x) = x$. Taking the partial derivative with respect to $t$, we get
  \begin{align*}
    \pd{u}{t} &= X_0(x)\diff{T_0}{t} - \sum_{n=1}^{\infty}D_n\cos\left( \frac{n\pi}{L}x \right)\left( \frac{n\pi a}{L} \right)\sin\left( \frac{n\pi a}{L}t \right).
  \end{align*}
  Therefore,
  \begin{align*}
    u\left( x,t \right) &= x
  \end{align*}
\end{solution}
\begin{solution}[12.5, Problem 2]
  Separating variables, we have
  \begin{align*}
    \frac{1}{X}\diff{^2X}{x^2} &= -\frac{1}{Y}\diff{^2Y}{y^2}\\
                               &= \begin{cases}
                                 -\lambda^2\\
                                 0\\
                                 \lambda^2
                               \end{cases}.
  \end{align*}
  Thus, we have
  \begin{align*}
    X_n &= A_n\cos\left( \lambda x \right) + B_n\sin\left( \lambda x \right).
  \end{align*}
  Using the boundary conditions of $X_n(a) = X_n(0) = 0$, we simplify to
  \begin{align*}
    X_n &= B_n\sin\left( \frac{n\pi}{a}x \right).
  \end{align*}
  Similarly, we have
  \begin{align*}
    Y_n(y) &= C_n\cosh\left( \frac{n\pi}{a}y \right) + D_n\sinh\left( \frac{n\pi}{a}y \right).
  \end{align*}
  Applying the boundary condition of $\pd{u}{y}\biggr\vert_{(x,0)} = 0$, we have $D_n = 0$, and
  \begin{align*}
    u(x,y) &= \sum_{n=1}^{\infty}K_n\sinh\left( \frac{n\pi}{a}y \right) \sin\left( \frac{n\pi}{a}x \right).
  \end{align*}
  We have
  \begin{align*}
    f(x) &= u\left( x,b \right)\\
         &= \sum_{n=1}^{\infty}K_n\sinh\left( \frac{n\pi b}{a} \right)\sin\left( \frac{n\pi}{a}x \right).
  \end{align*}
  Using the expansion of Fourier coefficients, we have
  \begin{align*}
    K_n &= \frac{2}{a\sinh\left( \frac{n\pi b}{a} \right)} \int_{0}^{a} f(x)\sin\left( \frac{n\pi}{a}x \right)\:dx.
  \end{align*}
\end{solution}
\begin{solution}[12.5, Problem 4]
  Separating variables, we get
  \begin{align*}
    \frac{1}{X}\diff{^2X}{x^2} &= -\frac{1}{Y}\diff{^2Y}{y^2}\\
                               &= \begin{cases}
                                 -\lambda^2\\
                                 0\\
                                 \lambda^2
                               \end{cases}.
  \end{align*}
  This evaluates to
  \begin{align*}
    X &= A\cos\left( \lambda x \right) + B\sin\left( \lambda x \right)\\
    Y &= C\cosh\left( \lambda y \right) + B\sinh\left( \lambda y \right).
  \end{align*}
  Using the Neumann boundary condition, we get
  \begin{align*}
    X_n &= A_n\cos\left( \frac{n\pi}{a}x \right)\\
    Y_n &= C_n\cosh\left( \frac{n\pi}{a}y \right) + D_n\sinh\left( \frac{n\pi}{a}y \right).
  \end{align*}
  Therefore, $u\left( x,y \right)$ is of the form
  \begin{align*}
    u\left( x,y \right) &= \sum_{n=0}^{\infty}B_n\cos\left( \frac{n\pi}{a}x \right)\cosh\left( \frac{n\pi}{a}y \right) + \sum_{n=1}^{\infty}D_n\cos\left( \frac{n\pi}{a}x \right)\sinh\left( \frac{n\pi}{a}y \right).
  \end{align*}
  We may plug this into an expression for $u\left( x,0 \right)$ to get
  \begin{align*}
    x &= \sum_{n=0}^{\infty}B_n\cos\left( \frac{n\pi}{a}x \right),
  \end{align*}
  meaning
  \begin{align*}
    B_0 &= a\\
    B_n &= \frac{2\left( \left( -1 \right)^{n} -1 \right)a}{n^2\pi^2},
  \end{align*}
  and plugging in the condition that $u\left( x,b \right) = 0$, we have
  \begin{align*}
    D_n &= -\coth\left( \frac{n\pi b}{a} \right)B_n.
  \end{align*}
\end{solution}
\begin{solution}[12.5, Problem 6]
  Separating variables, we have 
  \begin{align*}
    -\frac{1}{X}\diff{^2X}{x^2} &= \frac{1}{Y}\diff{^2Y}{y^2}\\
                               &= \begin{cases}
                                 -\lambda^2\\
                                 0\\
                                 \lambda^2
                               \end{cases}.
  \end{align*}
  We thus have
  \begin{align*}
    Y &= A\cos\left( \lambda y \right) + B\sin\left( \lambda y \right).
  \end{align*}
  Using the Neumann boundary condition in $y$, we may simplify this to
  \begin{align*}
    Y_n &= A_n\cos\left( ny \right).
  \end{align*}
  This gives
  \begin{align*}
    X_n &= B_n\cosh\left( nx \right) + C_n\sinh\left( nx \right).
  \end{align*}
  We thus have
  \begin{align*}
    u\left( x,y \right) &= \sum_{n=0}^{\infty}A_n\cosh\left( nx \right)\cos\left( ny \right) + \sum_{n=1}^{\infty}C_n\sinh\left( nx \right)\cos\left( ny \right).
  \end{align*}
  Evaluating the boundary condition at $u\left( 0,y \right)$, we have
  \begin{align*}
    g(y) &= \sum_{n=0}^{\infty}A_n\cos\left( ny \right),
  \end{align*}
  so that
  \begin{align*}
    A_n &= \frac{2}{\pi} \int_{0}^{\pi} g(y)\cos\left( ny \right)\:dy.
  \end{align*}
  Evaluating the derivative at $x = 1$, we have
  \begin{align*}
    \pd{u}{x}\biggr\vert_{(1,y)} &= \sum_{n=1}^{\infty}nA_n\sinh\left( n \right)\cos\left( ny \right) + \sum_{n=1}^{\infty}nC_n\cosh\left( n \right)\cos\left( ny \right)\\
                                 &= 0.
  \end{align*}
  Therefore, $C_n = -A_n\tanh\left( n \right)$.
\end{solution}
\begin{solution}[12.5, Problem 8]
  Separating variables, we have
  \begin{align*}
    \frac{1}{X}\diff{^2X}{x^2} &= -\frac{1}{Y}\diff{^2Y}{y^2}\\
                               &= \begin{cases}
                                 -\lambda^2\\
                                 0\\
                                 \lambda^2
                               \end{cases}
  \end{align*}
  Using the Dirichlet boundary condition on $X$, we get
  \begin{align*}
    X_n &= A_n\sin\left( n\pi x \right)\\
    Y_n &= B_n\cosh\left( n\pi y \right) + C_n\sinh\left( n\pi y \right).
  \end{align*}
  Thus, we have $u\left( x,y \right)$ of the form
  \begin{align*}
    u\left( x,y \right) &= \sum_{n=1}^{\infty}A_n\sin\left( n\pi x \right)\cosh\left( n\pi y \right) + \sum_{n=1}^{\infty}B_n\sin\left( n\pi x \right)\sinh\left( n\pi y \right).
  \end{align*}
  Setting
  \begin{align*}
    \pd{u}{y}\biggr\vert_{x=0} &= \sum_{n=1}^{\infty} n\pi B_n\sin\left( n\pi x \right)\\
                               &= 0,
  \end{align*}
  we have $B_n = 0$, so
  \begin{align*}
    u\left( x,y \right) &= \sum_{n=1}^{\infty}A_n\sin\left( n\pi x \right)\cosh\left( n\pi y \right).
  \end{align*}
  Solving for the Fourier coefficients, we get
  \begin{align*}
    A_n &= \frac{2}{\cosh\left( n\pi \right)} \int_{0}^{1} f(x)\sin\left( n\pi x \right)\:dx.
  \end{align*}
\end{solution}
\begin{solution}[12.6, Problem 2]
  We may homogenize the boundary condition by letting $u\left( x,t \right) = v\left( x,t \right) + \psi(x)$, and solving
  \begin{align*}
    k\diff{^2\psi}{x^2} &= 0,
  \end{align*}
  where $\psi(0) = u_0$ and $\psi(1) = 0$, so $\psi = -\frac{1}{u_0}x + u_0$. This gives the boundary value problem
  \begin{align*}
    \pd{v}{t} - k\pd{^2v}{x^2} &= 0
  \end{align*}
  with Dirichlet boundary of $v(0) = v(1) = 0$. Separating variables as $v\left( x,t \right) = X(x)T(t)$, we then get
  \begin{align*}
    \frac{1}{T}\diff{T}{t} &= \frac{k}{X}\diff{^2X}{x^2}\\
                           &= \begin{cases}
                             -\alpha^2\\
                             0\\
                             \alpha^2
                           \end{cases},
  \end{align*}
  so that
  \begin{align*}
    X_n &= A_n\sin\left( n\pi\sqrt{k}x \right)\\
    T_n &= e^{-kn^2\pi^2 t},
  \end{align*}
  and
  \begin{align*}
    v\left( x,t \right) &= \sum_{n=1}^{\infty}A_n\sin\left( n\pi\sqrt{k}x \right)e^{-kn^2\pi^2 t}.
  \end{align*}
  We have the initial condition $v\left( x,0 \right) = f(x)-\psi(x)$, so
  \begin{align*}
    A_n &= 2 \int_{0}^{1} \left( f(x)+\frac{1}{u_0}x-u_0 \right)\sin\left( n\pi x \right)\:dx,
  \end{align*}
  and
  \begin{align*}
    u\left( x,t \right) &= v\left( x,t \right)-\frac{1}{u_0}x + u_0.
  \end{align*}
\end{solution}
\begin{solution}[12.6, Problem 4]
  Homogenizing the boundary conditions, we have
  \begin{align*}
    u\left( x,t \right) &= v\left( x,t \right) + \psi(x),
  \end{align*}
  where
  \begin{align*}
    k\diff{^2\psi}{x^2} &= -r.
  \end{align*}
  This gives
  \begin{align*}
    \psi\left( x \right) &= -\frac{r}{2k}x^2 + Bx + C.
  \end{align*}
  Plugging in the boundary conditions, we get
  \begin{align*}
    \psi(x) &= -\frac{r}{2k}x^2 + \left( u_1 - u_0 + \frac{r}{2k} \right)x + u_0.
  \end{align*}
  The homogeneous heat equation is given by
  \begin{align*}
    \pd{v}{t} - k\pd{^2v}{x^2} &= 0,
  \end{align*}
  where $v\left( x,0 \right) = f(x) - \psi(x)$, $v\left( 0,t  \right) = v\left( 1,t \right) = 0$. By separating variables, we get
  \begin{align*}
    \frac{1}{T}\diff{^2T}{t^2} &= \frac{k}{X}\diff{^2X}{x^2}\\
                               &= \begin{cases}
                                 -\alpha^2\\
                                 0\\
                                 \alpha^2
                               \end{cases}
  \end{align*}
  Using the Dirichlet boundary conditions, we get
  \begin{align*}
    X_n &= A_n\sin\left( n\pi\sqrt{k} x \right)\\
    T_n &= e^{-kn^2\pi^2 t}.
  \end{align*}
  This gives
  \begin{align*}
    v\left( x,t \right) &= \sum_{n=1}^{\infty}A_n\sin\left( n\pi\sqrt{k}x \right)e^{-kn^2\pi^2 t}.
  \end{align*}
  The Fourier coefficients are
  \begin{align*}
    A_n &= 2 \int_{0}^{1} \left( f(x)-\psi(x) \right)\sin\left( n\pi x \right)\:dx,
  \end{align*}
  and we get the solution of $u\left( x,t \right) = v\left( x,t \right) + \psi(x)$.
\end{solution}
\begin{solution}[12.6, Problem 10]

\end{solution}
\begin{solution}[Extra Problems]

\end{solution}

\end{document}
