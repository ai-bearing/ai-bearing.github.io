\documentclass[10pt]{mypackage}

% sans serif font:
%\usepackage{cmbright,sfmath,bbold}
%\renewcommand{\mathcal}{\mathtt}

%Euler:
\usepackage{newpxtext,eulerpx,eucal}
\renewcommand*{\mathbb}[1]{\varmathbb{#1}}
\renewcommand*{\hbar}{\hslash}

%kp fonts:
%\usepackage{kpfonts}
%\renewcommand{\mathbb}{\mathds}
\usepackage{homework}

\pagestyle{fancy} %better headers
\fancyhf{}
\rhead{Avinash Iyer}
\lhead{Partial Differential Equations: Homework 1}

\setcounter{secnumdepth}{0}

\begin{document}
\RaggedRight
\section{Section 4.1}%
\renewcommand{\arraystretch}{1.5}
\begin{solution}[Problem 4]
  Evaluating with the initial conditions, we get
  \begin{align*}
    c_1- c_2 &= 0\\
    - c_3 &= 2\\
    c_2 &= -1.
  \end{align*}
  We see that $c_1 = -1$, $c_2 = -1$, and $c_3 = -2$. This yields the particular solution of
  \begin{align*}
    y &= -1 - \cos x - 2 \sin x.
  \end{align*}
\end{solution}

\begin{solution}[Problem 10]
  The interval $(-\pi,\pi)$ contains a unique solution to the initial value problem.
\end{solution}
\begin{solution}[Problem 14]\hfill
  \begin{enumerate}[(a)]
    \item We have
      \begin{align*}
        c_1 + c_2 + 3 &= 0\\
        c_1 + c_2 + 3 &= 4,
      \end{align*}
      which is not possible.
    \item We have
      \begin{align*}
        3 &= 0\\
        c_1 + c_2 + 3 &= 2,
      \end{align*}
      which is yet again not possible.
    \item We have
      \begin{align*}
        3 &= 3\\
        c_1 + c_2 + 3 &= 0,
      \end{align*}
      meaning that the solution set is all pairs $\left(c_1,c_2\right)$ such that $c_1 + c_2 = -3$.
    \item We have
      \begin{align*}
        c_1 + c_2 + 3 &= 3\\
        4c_1 + 16 c_2 + 3 &= 15,\\
        \intertext{or}
        c_1 + c_2 &= 0\\
        4c_1 + 16 c_2 &= 12\\
        \intertext{meaning}
        c_1 &= -1\\
        c_2 &= 1.
      \end{align*}
  \end{enumerate}
\end{solution}
\begin{solution}[Problem 22]
  Since
  \begin{align*}
    \sinh(x) &= \frac{1}{2}\left(e^{x} + e^{-x}\right),
  \end{align*}
  the functions are not linearly independent anywhere on $\left(-\infty,\infty\right)$.
\end{solution}
\begin{solution}[Problem 28]
  First, we verify that both solutions work.
  {\footnotesize
    \begin{align*}
    x^2 \diff{^2}{x^2}\left(\cos\left(\ln(x)\right)\right) + x \diff{}{x}\left(\cos\left(\ln(x)\right)\right) + \cos\left(\ln\left(x\right)\right) &= x^2\left(-\frac{\cos\left(\ln\left(x\right)\right)}{x^2} + \frac{\sin\left(\ln\left(x\right)\right)}{x^2}\right) + x\left(-\frac{\sin\left(\ln\left(x\right)\right)}{x}\right) + \cos\left(\ln\left(x\right)\right)\\
                                                                                                                                                   &= 0\\
    x^2 \diff{^2}{x^2}\left(\sin\left(\ln\left(x\right)\right)\right) + x \diff{}{x}\left(\sin\left(\ln\left(x\right)\right)\right) + \sin\left(\ln\left(x\right)\right) &= x^2 \left(-\frac{\cos\left(\ln\left(x\right)\right)}{x^2} - \frac{\sin\left(\ln\left(x\right)\right)}{x^2}\right) + x\left(\frac{\cos\left(\ln\left(x\right)\right)}{x}\right) + \sin\left(\ln\left(x\right)\right)\\
                                                                                                                                                                         &= 0.
  \end{align*}
}
Additionally, we find that
\begin{align*}
  \det \begin{pmatrix}\cos\left(\ln\left(x\right)\right) & \sin\left(\ln\left(x\right)\right) \\ \frac{-\sin\left(\ln\left(x\right)\right)}{x} & \frac{\cos\left(\ln\left(x\right)\right)}{x}\end{pmatrix} &= \frac{1}{x}\\
                                                         &\neq 0,
\end{align*}
so the solutions are linearly independent. Since the differential equation $x^2y'' + xy' + y' = 0$ is a second order equation, there are no other linearly independent solutions. Thus, we have the general solution of
\begin{align*}
  y &= \alpha \cos\left(\ln\left(x\right)\right) + \beta \sin\left(\ln\left(x\right)\right).
\end{align*}
\end{solution}
\begin{solution}[Problem 30]
  I'm not checking the Wronskian on this one, they're clearly linearly independent. However, I will be doing the derivatives.
  \begin{align*}
    \diff{^4}{x^4}\left(1\right) + \diff{^2}{x^2}\left(1\right) &= 0\\
    \diff{^4}{x^4}\left(x\right) + \diff{^2}{x^2}\left(x\right) &= 0\\
    \diff{^4}{x^4}\left(\cos\left(x\right)\right) + \diff{^2}{x^2}\left(\cos(x)\right) &= \cos\left(x\right) - \cos\left(x\right)\\
                                                                                       &= 0\\
    \diff{^4}{x^4}\left(\sin(x)\right) + \diff{^2}{x^2}\left(\sin(x)\right) &= \sin(x) - \sin(x)\\
                                                                            &= 0.
  \end{align*}
  Thus, since the solutions are linearly independent and have dimension $4$, they form a basis for the general solution of $y^{(4)} + y'' = 0$. The general solution is
  \begin{align*}
    y(x) &= c_1 + c_2x + c_3\cos(x) + c_4\sin(x).
  \end{align*}
\end{solution}
\begin{solution}[Problem 36]\hfill
  \begin{enumerate}[(a)]
    \item We have $y=5$ is a particular solution to $y'' + 2y = 10$.
    \item We have $y=-2x$ is a particular solution to $y'' + 2y = 10$.
    \item Using linearity, we get that $y = -2x + 5$ is a particular solution to $y'' + 2y = -4x + 10$.
    \item Using a similar process, we have a particular solution of $y = 4x + \frac{5}{2}$.
    \item Neither of these linear combinations are general solutions of the differential equation, as the linearity principle only applies to solutions of the corresponding homogeneous equation.
  \end{enumerate}
\end{solution}
\section{Section 4.2}%

\begin{solution}[Problem 2]
  Using the power of inspection, we find that our other solution is $y_2 = e^{-2x}$.
\end{solution}
\begin{solution}[Problem 8]
  \begin{align*}
    y_2(x) &= e^{x/3}\int_{}^{} \frac{e^{-x/6}}{e^{2x/3}}\:dx\\
           &= -\frac{6}{5}e^{-x/2}
  \end{align*}
\end{solution}
\begin{solution}[Problem 16]
  \begin{align*}
    y_2(x) &= \int_{}^{} e^{-\int_{}^{} \frac{2x}{1-x^2}\:dx}\:dx\\
           &= \frac{1}{3}x^2 - x.
  \end{align*}
\end{solution}
\begin{solution}[Problem 20]
  Using the power of inspection, the other homogeneous solution is $y_2(x) = e^{3x}$. Letting $y_p(x) = v(x)e^{x}$, we get
  \begin{align*}
    \left(v'' - 2v'\right)e^{x} &= x\\
    v''-2v' &= xe^{-x}\\
    \diff{}{x}\left(v'(x)\right) - 2v'(x) &= xe^{-x}.
  \end{align*}
  Using the integrating factor $e^{-2x}$, we have
  \begin{align*}
    \diff{}{x}\left(e^{-2x}v'(x)\right) &= xe^{-x}\\
    e^{-2x}v'(x) &= \int_{}^{} xe^{-x}\:dx\\
                 &= -xe^{-x} - e^{-x}\\
    v(x) &= -xe^{x}.
  \end{align*}
  Thus, we have the general solution of
  \begin{align*}
    y(x) &= c_1e^{x} + c_2e^{3x} + c_3xe^{2x}.
  \end{align*}
\end{solution}
\begin{solution}[Problem 22]
  We know that
  \begin{align*}
    x\diff{^2}{x^2}\left(x\right) - x \diff{}{x}\left(x\right) + x &= 0.
  \end{align*}
  Let $y_2(x) = v(x) x$. Then,
\end{solution}
\section{Section 4.3}%
\begin{solution}[Problem 4]
  Using the power of inspection, we have
  \begin{align*}
    y &= c_1e^{x} + c_2e^{2x}.
  \end{align*}
\end{solution}
\begin{solution}[Problem 6]
  Using the power of inspection, we have
  \begin{align*}
    y &= c_1e^{5x} + c_2xe^{5x}.
  \end{align*}
\end{solution}
\begin{solution}[Problem 12]
  Using the power of inspection$+ \ve$, we have
  \begin{align*}
    y(x) &= e^{-\frac{1}{2}x} \left(c_1\cos\left(\frac{1}{2}x\right) + c_2\sin\left(\frac{1}{2}x\right)\right).
  \end{align*}
\end{solution}
\begin{solution}[Problem 16]
  Using the power of inspection$+\ve$, we have
  \begin{align*}
    y(x) &= c_1e^{x} + c_2e^{-\frac{1}{2}x}\cos\left(\frac{\sqrt{3}}{2}x\right) + c_3e^{-\frac{1}{2}x}\sin\left(\frac{\sqrt{3}}{2}x\right).
  \end{align*}
\end{solution}
\begin{solution}[Problem 22]
  Using the power of inspection$+2\ve$, we have
  \begin{align*}
    y(x) &= c_1e^{2x} + c_2xe^{2x} + c_3x^2e^{2x}.
  \end{align*}
\end{solution}
\begin{solution}[Problem 36]
  Using the power of inspection$+\ve$, we have
  \begin{align*}
    y(x) &= c_1e^{-x} + c_2e^{-3x} + c_3e^{2x}.
  \end{align*}
  Evaluating the initial conditions, we have
  \begin{align*}
    1 &= c_1 + c_2 + c_3\\
    1 &= -c_1 - 3c_2 + 2c_3\\
    1 &= c_1 + 9c_2 + 4c_3,
  \end{align*}
  and using the power of inspection$+\ve$, we get
  \begin{align*}
    c_1 &= \frac{2}{3}\\
    c_2 &= -\frac{1}{5}\\
    c_3 &= \frac{8}{15}.
  \end{align*}
  
\end{solution}
\begin{solution}[Problem 38]

\end{solution}
\begin{solution}[Problem 50]

\end{solution}
\end{document}
