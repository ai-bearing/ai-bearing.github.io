\documentclass[10pt]{mypackage}

% sans serif font:
%\usepackage{cmbright,sfmath,bbold}
%\renewcommand{\mathcal}{\mathtt}

%Euler:
\usepackage{newpxtext,eulerpx,eucal}
\renewcommand*{\mathbb}[1]{\varmathbb{#1}}
\renewcommand*{\hbar}{\hslash}

%kp fonts:
%\usepackage{kpfonts}
%\renewcommand{\mathbb}{\mathds}
\usepackage{homework}

\pagestyle{fancy} %better headers
\fancyhf{}
\rhead{Avinash Iyer}
\lhead{Partial Differential Equations: Homework 1}

\setcounter{secnumdepth}{0}

\begin{document}
\RaggedRight
\section{Section 4.1}%
\begin{solution}[Problem 4]
  Evaluating with the initial conditions, we get
  \begin{align*}
    c_1- c_2 &= 0\\
    - c_3 &= 2\\
    c_2 &= -1.
  \end{align*}
  We see that $c_1 = -1$, $c_2 = -1$, and $c_3 = -2$. This yields the particular solution of
  \begin{align*}
    y &= -1 - \cos x - 2 \sin x.
  \end{align*}
\end{solution}

\begin{solution}[Problem 10]
  The interval $(-\pi,\pi)$ contains a unique solution to the initial value problem.
\end{solution}
\begin{solution}[Problem 14]\hfill
  \begin{enumerate}[(a)]
    \item We have
      \begin{align*}
        c_1 + c_2 + 3 &= 0\\
        c_1 + c_2 + 3 &= 4,
      \end{align*}
      which is not possible.
    \item We have
      \begin{align*}
        3 &= 0\\
        c_1 + c_2 + 3 &= 2,
      \end{align*}
      which is yet again not possible.
    \item We have
      \begin{align*}
        3 &= 3\\
        c_1 + c_2 + 3 &= 0,
      \end{align*}
      meaning that the solution set is all pairs $\left(c_1,c_2\right)$ such that $c_1 + c_2 = -3$.
    \item We have
      \begin{align*}
        c_1 + c_2 + 3 &= 3\\
        4c_1 + 16 c_2 + 3 &= 15,\\
        \intertext{or}
        c_1 + c_2 &= 0\\
        4c_1 + 16 c_2 &= 12\\
        \intertext{meaning}
        c_1 &= -1\\
        c_2 &= 1.
      \end{align*}
  \end{enumerate}
\end{solution}
\begin{solution}[Problem 22]
  Since
  \begin{align*}
    \sinh(x) &= \frac{1}{2}\left(e^{x} + e^{-x}\right),
  \end{align*}
  the functions are not linearly independent anywhere on $\left(-\infty,\infty\right)$.
\end{solution}
\begin{solution}[Problem 28]
  First, we verify that both solutions work.
  {\footnotesize
    \begin{align*}
    x^2 \diff{^2}{x^2}\left(\cos\left(\ln(x)\right)\right) + x \diff{}{x}\left(\cos\left(\ln(x)\right)\right) + \cos\left(\ln\left(x\right)\right) &= x^2\left(-\frac{\cos\left(\ln\left(x\right)\right)}{x^2} + \frac{\sin\left(\ln\left(x\right)\right)}{x^2}\right) + x\left(-\frac{\sin\left(\ln\left(x\right)\right)}{x}\right) + \cos\left(\ln\left(x\right)\right)\\
                                                                                                                                                   &= 0\\
    x^2 \diff{^2}{x^2}\left(\sin\left(\ln\left(x\right)\right)\right) + x \diff{}{x}\left(\sin\left(\ln\left(x\right)\right)\right) + \sin\left(\ln\left(x\right)\right) &= x^2 \left(-\frac{\cos\left(\ln\left(x\right)\right)}{x^2} - \frac{\sin\left(\ln\left(x\right)\right)}{x^2}\right) + x\left(\frac{\cos\left(\ln\left(x\right)\right)}{x}\right) + \sin\left(\ln\left(x\right)\right)\\
                                                                                                                                                                         &= 0.
  \end{align*}
}
Additionally, we find that
\begin{align*}
  \det \begin{pmatrix}\cos\left(\ln\left(x\right)\right) & \sin\left(\ln\left(x\right)\right) \\ \frac{-\sin\left(\ln\left(x\right)\right)}{x} & \frac{\cos\left(\ln\left(x\right)\right)}{x}\end{pmatrix} &= \frac{1}{x}\\
                                                         &\neq 0,
\end{align*}
so the solutions are linearly independent. Since the differential equation $x^2y'' + xy' + y' = 0$ is a second order equation, there are no other linearly independent solutions. Thus, we have the general solution of
\begin{align*}
  y &= \alpha \cos\left(\ln\left(x\right)\right) + \beta \sin\left(\ln\left(x\right)\right).
\end{align*}
\end{solution}
\begin{solution}[Problem 30]

\end{solution}
\begin{solution}[Problem 36]

\end{solution}
\section{Section 4.2}%

\begin{solution}[Problem 2]

\end{solution}

\begin{solution}[Problem 8]

\end{solution}
\begin{solution}[Problem 16]

\end{solution}
\begin{solution}[Problem 20]

\end{solution}
\begin{solution}[Problem 22]

\end{solution}
\section{Section 4.3}%
\begin{solution}[Problem 4]

\end{solution}
\begin{solution}[Problem 6]

\end{solution}
\begin{solution}[Problem 12]

\end{solution}
\begin{solution}[Problem 16]

\end{solution}
\begin{solution}[Problem 22]

\end{solution}
\begin{solution}[Problem 36]

\end{solution}
\begin{solution}[Problem 38]

\end{solution}
\begin{solution}[Problem 50]

\end{solution}
\end{document}
