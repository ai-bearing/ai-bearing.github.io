\documentclass[10pt]{mypackage}

% sans serif font:
%\usepackage{cmbright,sfmath,bbold}
%\renewcommand{\mathcal}{\mathtt}

%Euler:
\usepackage{newpxtext,eulerpx,eucal,eufrak}
\renewcommand*{\mathbb}[1]{\varmathbb{#1}}
\renewcommand*{\hbar}{\hslash}

\usepackage{homework}

\pagestyle{fancy} %better headers
\fancyhf{}
\rhead{Avinash Iyer}
\lhead{Partial Differential Equations: Homework 4}

\setcounter{secnumdepth}{0}

\begin{document}
\RaggedRight
\renewcommand{\arraystretch}{1.5}
\begin{solution}[8.3, Problem 12]
  Expressing in matrix form, we have
  \begin{align*}
    \diff{\mathbf{x}}{t} &= \begin{pmatrix}2 & -1\\3 & -2\end{pmatrix}\mathbf{x} + \begin{pmatrix}0\\4t\end{pmatrix}.
  \end{align*}
  Solving for the fundamental matrix, we have
  \begin{align*}
    \det \begin{pmatrix}2-\lambda & -1 \\ 3 & -2-\lambda\end{pmatrix} &= \lambda^2 - 4 + 3\\
                                  &= \lambda^2 - 1,
  \end{align*}
  so we have eigenvalues of $\pm 1$, with eigenvectors
  \begin{align*}
    \mathbf{v}_1 &= \begin{pmatrix}1\\3\end{pmatrix}\\
    \mathbf{v}_2 &= \begin{pmatrix}1\\1\end{pmatrix}.
  \end{align*}
  Thus, our fundamental matrix is
  \begin{align*}
    \Phi(t) &= \begin{pmatrix}e^{t} & e^{-t} \\ 3e^{t} & e^{-t}\end{pmatrix} 
  \end{align*}
  Using the power of Mathematica, we may find the inverse of $\Phi$ to be
  \begin{align*}
    \Phi^{-1}\left( t \right) &= \begin{pmatrix}-\frac{1}{2}e^{-t} & \frac{1}{2}e^{-t} \\ \frac{3}{2}e^{t} & -\frac{1}{2}e^{t}\end{pmatrix}.
  \end{align*}
  Using yet more of the power of Mathematica, we evaluate
  \begin{align*}
    \Phi^{-1}(t) \begin{pmatrix}0\\4t\end{pmatrix} &= \begin{pmatrix}2te^{-t} \\ -2te^{t}\end{pmatrix},
  \end{align*}
  with integrals of
  \begin{align*}
    \int_{}^{} \Phi^{-1}(t)\mathbf{F}(t)\:dt &= \begin{pmatrix}-2\left( te^{-t} + e^{-t} \right) \\ -2\left( te^{t} - e^{t} \right)\end{pmatrix}\\
    \begin{pmatrix}e^{t} & e^{-t} \\ 3e^{t} & e^{-t}\end{pmatrix} \begin{pmatrix}-2\left( te^{-t} + e^{-t} \right) \\ -2\left( te^{t} - e^{t} \right)\end{pmatrix} &= \begin{pmatrix}-4t\\-8t-4\end{pmatrix}
  \end{align*}
  Thus, the general solution to the system is
  \begin{align*}
    \mathbf{x}(t) &= \begin{pmatrix}e^{t} & e^{-t} \\ 3e^{t} & e^{-t}\end{pmatrix} \begin{pmatrix}c_1\\c_2\end{pmatrix} + \begin{pmatrix}-4t\\-8t-4\end{pmatrix}.
  \end{align*}
\end{solution}
\begin{solution}[8.3, Problem 16]
  We have
  \begin{align*}
    \det \begin{pmatrix}-\lambda & 2 \\ -1 & 3-\lambda\end{pmatrix} &= \lambda^2 - 3\lambda + 2,
  \end{align*}
  so $\lambda = 1,2$. These have respective eigenvectors of
  \begin{align*}
    \begin{pmatrix}-1 & 2 \\ -1 & 2\end{pmatrix} \begin{pmatrix}a\\b\end{pmatrix} &= \begin{pmatrix}0\\0\end{pmatrix}\\
    \mathbf{v}_1 &= \begin{pmatrix}2\\1\end{pmatrix}\\
    \begin{pmatrix}-2 & 2 \\ -1 & 1\end{pmatrix} \begin{pmatrix}a\\b\end{pmatrix} &= \begin{pmatrix}0\\0\end{pmatrix}\\
    \mathbf{v}_2 &= \begin{pmatrix}1\\1\end{pmatrix}.
  \end{align*}
  Thus, we have the solution matrix of
  \begin{align*}
    \Phi(t) &= \begin{pmatrix}2e^{t} & e^{2t} \\ e^{t} & e^{2t}\end{pmatrix}.
  \end{align*}
  Using the power of Mathematica, we find the inverse and calculate
  \begin{align*}
    \Phi^{-1}(t) &= \begin{pmatrix}e^{-t} & -e^{-t} \\ -e^{-2t} & 2e^{-2t}\end{pmatrix}\\
    \Phi^{-1}\left( t \right) \begin{pmatrix}2\\e^{-3t}\end{pmatrix} &= \begin{pmatrix}-e^{-4t} + 2e^{-t} \\ 2e^{-5t} - 2e^{-2t}\end{pmatrix}\\
    \begin{pmatrix}2e^{t} & e^{2t} \\ e^{t} & e^{2t}\end{pmatrix}\begin{pmatrix}-e^{-4t} + 2e^{-t} \\ 2e^{-5t} - 2e^{-2t}\end{pmatrix} &= \begin{pmatrix}2\\e^{-3t}\end{pmatrix}.
  \end{align*}
  Thus, we have the solution of
  \begin{align*}
    \mathbf{x}(t) &= \begin{pmatrix}2e^{t} & e^{2t} \\ e^{t} & e^{2t}\end{pmatrix} \begin{pmatrix}c_1\\c_2\end{pmatrix} + \begin{pmatrix}2\\e^{-3t}\end{pmatrix}.
  \end{align*}
\end{solution}
\begin{solution}[8.3, Problem 20]
  We have
  \begin{align*}
    \det \begin{pmatrix}3-\lambda & 2 \\ -2 & -1-\lambda\end{pmatrix} &= \lambda^2 - 2\lambda + 1,
  \end{align*}
  giving eigenvalues of $1,1$. We need to use generalized eigenvectors here, meaning that we have
  \begin{align*}
    \mathbf{v}_1 &= \begin{pmatrix}1\\-1\end{pmatrix}\\
    \mathbf{w} &= \frac{1}{4} \begin{pmatrix}1\\1\end{pmatrix}.
  \end{align*}
  Thus, we have the solution matrix of
  \begin{align*}
    \Phi(t) &= \begin{pmatrix}e^{t} & \frac{1}{4}te^{t} + e^{t} \\ -e^{t} & \frac{1}{4}te^{t} - e^{t}\end{pmatrix}.
  \end{align*}
  This solution matrix has the inverse of
  \begin{align*}
    \Phi^{-1}\left( t \right) &= \begin{pmatrix}\frac{1}{2t}\left( -4e^{-t} + t \right) & -\frac{1}{2t}\left( 4e^{-t} + t \right) \\ \frac{2e^{-t}}{t} & \frac{2e^{-t}}{t}\end{pmatrix}.
  \end{align*}
  Multiplying by $\mathbf{F}$, we have
  \begin{align*}
    \Phi^{-1}(t) \mathbf{F}\left( t \right) &= \begin{pmatrix}-\frac{4e^{-t}}{t} \\ \frac{4e^{-t}}{t}\end{pmatrix}.
  \end{align*}
  This integral cannot be solved analytically. Thus, we have the solution of
  \begin{align*}
    \mathbf{x} &= \begin{pmatrix}e^{t} & \frac{1}{4}te^{t} + e^{t} \\ -e^{t} & \frac{1}{4}te^{t} - e^{t}\end{pmatrix} \begin{pmatrix}c_1\\c_2\end{pmatrix} + \begin{pmatrix}e^{t} & \frac{1}{4}te^{t} + e^{t} \\ -e^{t} & \frac{1}{4}te^{t} - e^{t}\end{pmatrix}\int_{}^{} \begin{pmatrix}-\frac{4e^{-t}}{t} \\ \frac{4e^{-t}}{t}\end{pmatrix}\:dt.
  \end{align*}
\end{solution}
\begin{solution}[8.3, Problem 28]
  We have
  \begin{align*}
    \det \begin{pmatrix}1-\lambda & -2 \\ 1 & -1-\lambda\end{pmatrix} &= \lambda^2 + 1,
  \end{align*}
  so there are eigenvalues of $\lambda = \pm i$, which admit eigenvectors of
  \begin{align*}
    \mathbf{v} &= \begin{pmatrix}1\\1\end{pmatrix} \pm i\begin{pmatrix}1\\0\end{pmatrix}.
  \end{align*}
  Thus, obtain solution vectors and fundamental matrix of
  \begin{align*}
    \mathbf{x}_1 &= \begin{pmatrix}\cos (t) - \sin (t)\\ \cos (t)\end{pmatrix}\\
    \mathbf{x}_2 &= \begin{pmatrix}\cos (t) + \sin (t) \\ \sin (t)\end{pmatrix}\\
    \Phi(t) &= \begin{pmatrix}\cos(t) - \sin\left( t \right) & \cos\left( t \right) + \sin\left( t \right) \\ \cos\left( t \right) & \sin\left( t \right)\end{pmatrix}.
  \end{align*}
  Using the power of Mathematica, we find the inverse
  \begin{align*}
    \Phi^{-1}\left( t \right) &= \begin{pmatrix}-\sin\left( t \right) & \cos\left( t \right) + \sin\left( t \right) \\ \cos\left( t \right) & -\cos\left( t \right) + \sin\left( t \right)\end{pmatrix}\\
    \Phi^{-1}\left( t \right) \begin{pmatrix}\tan(t) \\ 1\end{pmatrix} &= \begin{pmatrix}\cos(t) + \sin(t) - \sin(t)\tan(t) \\ -\cos(t) + 2\sin(t)\end{pmatrix}\\
    \Phi(t) \int_{}^{} \Phi^{-1}(t)\mathbf{F}(t)\:dt &= \begin{pmatrix}-3\cos^2(t)+ \cos\left( t \right)\left( \ln\left( \sec\left( t \right) + \tan\left( t \right) \right) - \sin\left( t \right) \right) -\sin\left( t \right)\left( \ln\left( \sec\left( t \right) + \tan\left( t \right) \right) + 2\sin\left( t \right) \right) \\ -1 + \cos\left( t \right)\left( \ln\left( \sec\left( t \right) + \tan\left( t \right) \right) - \sin\left( t \right) \right)\end{pmatrix}.
  \end{align*}
  Thus, our general solution is
  \begin{align*}
    \mathbf{x}(t) &= \begin{pmatrix}\cos(t) - \sin\left( t \right) & \cos\left( t \right) + \sin\left( t \right) \\ \cos\left( t \right) & \sin\left( t \right)\end{pmatrix} \begin{pmatrix}c_1\\c_2\end{pmatrix} \\
                  &+ \begin{pmatrix}-3\cos^2(t)+ \cos\left( t \right)\left( \ln\left( \sec\left( t \right) + \tan\left( t \right) \right) - \sin\left( t \right) \right) -\sin\left( t \right)\left( \ln\left( \sec\left( t \right) + \tan\left( t \right) \right) + 2\sin\left( t \right) \right) \\ -1 + \cos\left( t \right)\left( \ln\left( \sec\left( t \right) + \tan\left( t \right) \right) - \sin\left( t \right) \right)\end{pmatrix}.
  \end{align*}
\end{solution}
\begin{solution}[8.3, Problem 30]
  Finding the eigenvectors and eigenvalues, we have $\lambda = 2,2,1$, and
  \begin{align*}
    \mathbf{v}_1 &= \begin{pmatrix}1\\0\\1\end{pmatrix}\\
    \mathbf{v}_2 &= \begin{pmatrix}1\\1\\0\end{pmatrix}\\
    \mathbf{v}_3 &= \begin{pmatrix}1\\1\\1\end{pmatrix},
  \end{align*}
  which gives the fundamental solution matrix of
  \begin{align*}
    \Phi(t) &= \begin{pmatrix}e^{2t} & e^{2t} & e^{t} \\ 0 & e^{2t} & e^{t} \\ e^{2t} & 0 & e^{t}\end{pmatrix}.
  \end{align*}
  We calculate the inhomogeneous term using Mathematica to find
  \begin{align*}
    \Phi\left( t \right)\int_{}^{} \Phi^{-1}\left( t \right)\mathbf{F}(t)\:dt &= \begin{pmatrix}\frac{1}{4}\left( -3-2t + 8e^{t}\left( 1+t \right) \right) \\ \left( 1+t \right)\left( -1 + 2e^{t} \right) \\ \frac{1}{4}\left( -3 + \left( -2 + 8e^{t} \right)t \right)\end{pmatrix},
  \end{align*}
  giving the general solution of
  \begin{align*}
    \mathbf{x}(t) &= \begin{pmatrix}e^{2t} & e^{2t} & e^{t} \\ 0 & e^{2t} & e^{t} \\ e^{2t} & 0 & e^{t}\end{pmatrix} \begin{pmatrix}c_1\\c_2\\c_3\end{pmatrix} + \begin{pmatrix}\frac{1}{4}\left( -3-2t + 8e^{t}\left( 1+t \right) \right) \\ \left( 1+t \right)\left( -1 + 2e^{t} \right) \\ \frac{1}{4}\left( -3 + \left( -2 + 8e^{t} \right)t \right)\end{pmatrix}
  \end{align*}
\end{solution}
\begin{solution}[8.3, Problem 32]
  From inspection, the matrix has an eigenvalue of $1$ with eigenvector
  \begin{align*}
    \mathbf{v}_1 &= \begin{pmatrix}1\\0\end{pmatrix},
  \end{align*}
  and since the matrix has linearly dependent rows, $0$ is an eigenvalue for the homogeneous system, with eigenvector
  \begin{align*}
    \mathbf{v}_2 &= \begin{pmatrix}1\\1\end{pmatrix}.
  \end{align*}
  Therefore, we have a fundamental solution matrix of
  \begin{align*}
    \Phi(t) &= \begin{pmatrix}e^{t} & 1 \\ 0 & 1\end{pmatrix}.
  \end{align*}
  Finding $\Psi_A$, we have
  \begin{align*}
    \Phi(t)\Phi^{-1}(0) &= \begin{pmatrix}e^{t} & 1 - e^{t} \\ 0 & 1\end{pmatrix}.
  \end{align*}
  We see that
  \begin{align*}
    \Psi_A(t)\int_{}^{} \Psi_A(-t)\mathbf{F}(t)\:dt &= \begin{pmatrix}\ln(t)\\\ln(t)\end{pmatrix}.
  \end{align*}
  Now, finding the initial conditions, we have
  \begin{align*}
    \begin{pmatrix}2\\-1\end{pmatrix} &= \begin{pmatrix}e & 1-e \\ 0 & 1\end{pmatrix} \begin{pmatrix}c_1\\c_2\end{pmatrix}\\
    \begin{pmatrix}c_1\\c_2\end{pmatrix} &= \begin{pmatrix}\frac{3}{e}-1\\-1\end{pmatrix}.
  \end{align*}
  Thus, our solution is
  \begin{align*}
    \mathbf{x}(t) &= \begin{pmatrix}e^{t} & 1-e^{t} \\ 0 & 1\end{pmatrix} \begin{pmatrix}\frac{3}{e}-1\\-1\end{pmatrix} + \begin{pmatrix}\ln(t)\\\ln(t)\end{pmatrix}.
  \end{align*}
\end{solution}
\begin{solution}[8.3, Problem 35]\hfill
  \begin{enumerate}[(a)]
    \item We find the eigenvalues $\lambda = 4,3,1,0$, with respective eigenvectors of
      \begin{align*}
        \mathbf{v}_1 &= \begin{pmatrix}-1\\1\\0\\0\end{pmatrix}\\
        \mathbf{v}_2 &= \begin{pmatrix}3\\1\\2\\1\end{pmatrix}\\
        \mathbf{v}_3 &= \begin{pmatrix}2\\1\\0\\0\end{pmatrix}\\
        \mathbf{v}_4 &= \begin{pmatrix}-6\\-4\\1\\2\end{pmatrix}.
      \end{align*}
    \item These give the fundamental solution matrix of
      \begin{align*}
        \Phi(t) &= \begin{pmatrix}-e^{4t} & 3e^{3t} & 2e^{t} & -6\\ e^{4t} & e^{3t} & e^{t} & -4\\ 0 & 2e^{3t} & 0 & 1\\0 & e^{3t} & 0 & 2 \end{pmatrix}\\
        \Phi^{-1}(t) &= \begin{pmatrix}-\frac{1}{3}e^{-4t} & \frac{2}{3}e^{-4t} & 0 & \frac{1}{3}e^{-4t} \\ 0 & 0 & \frac{2}{3}e^{-3t} & -\frac{1}{3}e^{-3t} \\ \frac{1}{3}e^{-t} & \frac{1}{3}e^{-t} & -2e^{-t} & \frac{8}{3}e^{-2t} \\ 0 & 0 & -\frac{1}{3} & \frac{2}{3}\end{pmatrix}.
      \end{align*}
    \item 
      \begin{align*}
        \Phi^{-1}\mathbf{F} &= 
\begin{pmatrix}
 \frac{1}{3} e^{-5 t} \left(-e^{2 t} t+e^t+2\right) \\
 \frac{1}{3} e^{-3 t} \left(2 e^{2 t}-1\right) \\
 \frac{1}{3} \left(t+e^{-2 t}+8 e^{-t}-6 e^t\right) \\
 \frac{1}{3} \left(2-e^{2 t}\right) \\
\end{pmatrix}\\
        \int_{}^{} \Phi^{-1}\mathbf{F} \:dt &= 
\begin{pmatrix}
 \frac{1}{540} e^{-5 t} \left(20 e^{2 t} (3 t+1)-45 e^t-72\right) \\
 \frac{1}{9} e^{-3 t} \left(1-6 e^{2 t}\right) \\
 \frac{1}{6} \left(t^2-e^{-2 t}-16 e^{-t}-12 e^t\right) \\
 \frac{1}{6} \left(4 t-e^{2 t}\right) \\
\end{pmatrix}\\
        \Phi \int_{}^{} \Phi^{-1}\mathbf{F}\:dt &= 
\begin{pmatrix}
 \frac{1}{27} e^t \left(9 t^2-3 t-1\right)-4 t-\frac{e^{-t}}{5}-5 e^{2 t}-\frac{59}{12} \\
 \frac{1}{54} e^t \left(9 t^2+6 t+2\right)-\frac{8 t}{3}-\frac{3 e^{-t}}{10}-2 e^{2 t}-\frac{95}{36} \\
 \frac{2 t}{3}-\frac{3 e^{2 t}}{2}+\frac{2}{9} \\
 \frac{4 t}{3}-e^{2 t}+\frac{1}{9} \\
\end{pmatrix}
      \end{align*}
      Thus, we have
      \begin{align*}
        \mathbf{x} &= \begin{pmatrix}-e^{4t} & 3e^{3t} & 2e^{t} & -6\\ e^{4t} & e^{3t} & e^{t} & -4\\ 0 & 2e^{3t} & 0 & 1\\0 & e^{3t} & 0 & 2 \end{pmatrix} \begin{pmatrix}c_1\\c_2\\c_3\\c_4\end{pmatrix} + \begin{pmatrix}
 \frac{1}{27} e^t \left(9 t^2-3 t-1\right)-4 t-\frac{e^{-t}}{5}-5 e^{2 t}-\frac{59}{12} \\
 \frac{1}{54} e^t \left(9 t^2+6 t+2\right)-\frac{8 t}{3}-\frac{3 e^{-t}}{10}-2 e^{2 t}-\frac{95}{36} \\
 \frac{2 t}{3}-\frac{3 e^{2 t}}{2}+\frac{2}{9} \\
 \frac{4 t}{3}-e^{2 t}+\frac{1}{9} \\
\end{pmatrix}
      \end{align*}
    \item 
      \begin{align*}
        \mathbf{x}(t) &= c_1e^{4t} \begin{pmatrix}-1\\1\\0\\0\end{pmatrix} + c_2 e^{3t} \begin{pmatrix}3\\1\\2\\1\end{pmatrix} + c_3e^{t} \begin{pmatrix}2\\1\\0\\0\end{pmatrix} + c_4 \begin{pmatrix}-6\\-4\\1\\2\end{pmatrix} + \begin{pmatrix}
 \frac{1}{27} e^t \left(9 t^2-3 t-1\right)-4 t-\frac{e^{-t}}{5}-5 e^{2 t}-\frac{59}{12} \\
 \frac{1}{54} e^t \left(9 t^2+6 t+2\right)-\frac{8 t}{3}-\frac{3 e^{-t}}{10}-2 e^{2 t}-\frac{95}{36} \\
 \frac{2 t}{3}-\frac{3 e^{2 t}}{2}+\frac{2}{9} \\
 \frac{4 t}{3}-e^{2 t}+\frac{1}{9} \\
\end{pmatrix}
      \end{align*}
  \end{enumerate}
\end{solution}
\begin{solution}[8.4, Problem 2]
  We see that $A^2 = I$, so that
  \begin{align*}
    e^{At} &= \left( I + \frac{1}{2!}I t^2 + \frac{1}{4!}I t^{4} + \cdots \right) + \left( At + \frac{1}{3!}At^3 + \frac{1}{5!}At^{5} + \cdots \right)\\
           &= \begin{pmatrix}\cosh(t) & \sinh(t) \\ \sinh(t) & \cosh(t)\end{pmatrix},
  \end{align*}
  with
  \begin{align*}
    e^{-At} &= \begin{pmatrix}\cosh(t) & -\sinh(t) \\ -\sinh(t) & \cosh(t)\end{pmatrix}.
  \end{align*}
\end{solution}
\begin{solution}[8.4, Problem 4]
  Calculating, we have
  \begin{align*}
    A^2 &= \begin{pmatrix}0 & 0 & 0 \\ 0 & 0 & 0 \\ 3 & 0 & 0\end{pmatrix}\\
    A^3 &= \mathbf{0}.
  \end{align*}
  Thus, we have the matrix of
  \begin{align*}
    e^{At} &= \begin{pmatrix}1 & 0 & 0\\ 3t & 0 & 0 \\ 5t + \frac{3}{2}t^2 & t & 0\end{pmatrix}.
  \end{align*}
\end{solution}
\begin{solution}[8.4, Problem 6]
  Now that we know
  \begin{align*}
    e^{At} &= \begin{pmatrix}\cosh(t) & \sinh(t) \\ \sinh(t) & \cosh(t)\end{pmatrix},
  \end{align*}
  we have the solution of
  \begin{align*}
    \mathbf{x} &= \begin{pmatrix}c_1\cosh(t) + c_2\sinh(t) \\ c_1\sinh(t) + c_2\cosh(t)\end{pmatrix}.
  \end{align*}
\end{solution}
\begin{solution}[8.4, Problem 8]
  We have the general solution of
  \begin{align*}
    \mathbf{x}(t) &= \begin{pmatrix}1 & 0 & 0\\ 3t & 0 & 0 \\ 5t + \frac{3}{2}t^2 & t & 0\end{pmatrix} \begin{pmatrix}c_1\\c_2\\c_3\end{pmatrix}.
  \end{align*}
\end{solution}
\begin{solution}[8.4, Problem 26]
  We calculate
  \begin{align*}
    A^2 &= \begin{pmatrix}-1 & 0 & 1 \\ 0 & 0 & 0 \\ -1 & 0 & 1\end{pmatrix}\\
    A^3 &= \mathbf{0},
  \end{align*}
  so $A$ is nilpotent. Thus, we may calculate
  \begin{align*}
    e^{At} &= \begin{pmatrix}1 - t - \frac{t^2}{2} & t & t + \frac{t^2}{2}\\ -t & 1 & t \\ -t-\frac{t^2}{2} & t & 1 + t + \frac{t^2}{2}\end{pmatrix},
  \end{align*}
  and find the solution
  \begin{align*}
    \mathbf{x} &= \begin{pmatrix}1 - t - \frac{t^2}{2} & t & t + \frac{t^2}{2}\\ -t & 1 & t \\ -t-\frac{t^2}{2} & t & 1 + t + \frac{t^2}{2}\end{pmatrix} \begin{pmatrix}c_1\\c_2\\c_3\end{pmatrix}.
  \end{align*}
\end{solution}
\begin{solution}[8.4, Problem 27]\hfill
  \begin{enumerate}[(a)]
    \item Using Mathematica, we find the matrix exponential of
      \begin{align*}
        e^{At} &= \begin{pmatrix}\frac{2}{5}e^{t} + \frac{3}{5}e^{6t} & -\frac{2}{5}e^{t} + \frac{2}{5}e^{6t} \\ -\frac{3}{5}e^{t} + \frac{3}{5}e^{6t} & \frac{3}{5}e^{t} + \frac{2}{5}e^{6t}\end{pmatrix}.
      \end{align*}
      Thus, we find the solution of
      \begin{align*}
        \mathbf{x} &= \begin{pmatrix}\frac{2}{5}e^{t} + \frac{3}{5}e^{6t} & -\frac{2}{5}e^{t} + \frac{2}{5}e^{6t} \\ -\frac{3}{5}e^{t} + \frac{3}{5}e^{6t} & \frac{3}{5}e^{t} + \frac{2}{5}e^{6t}\end{pmatrix} \begin{pmatrix}c_1\\c_2\end{pmatrix}.
      \end{align*}
      Using some tedious algebra, we are able to convert this solution to the form
      \begin{align*}
        \mathbf{x} &= k_1e^{t} \begin{pmatrix}2\\-3\end{pmatrix} + k_2 e^{6t} \begin{pmatrix}1\\1\end{pmatrix},
      \end{align*}
      by setting $k_1 \coloneq \frac{1}{5}\left( c_1 - c_2 \right)$ and $k_2 \coloneq \frac{1}{5}\left( c_1 + c_2 \right)$.
    \item We find the matrix exponential of
      \begin{align*}
        e^{At} &= 
\begin{pmatrix}
 e^{-2 t} \left(\cos (t)-\sin (t)\right) & -e^{-2 t} \sin (t) \\
 2 e^{-2 t} \sin (t) & e^{-2 t} \left(\sin (t)+\cos (t)\right) \\
\end{pmatrix}
      \end{align*}
      and general solution of
      \begin{align*}
        \mathbf{x} &= 
\begin{pmatrix}
 e^{-2 t} \left(\cos (t)-\sin (t)\right) & -e^{-2 t} \sin (t) \\
 2 e^{-2 t} \sin (t) & e^{-2 t} \left(\sin (t)+\cos (t)\right) \\
\end{pmatrix} \begin{pmatrix}c_1\\c_2\end{pmatrix}.
      \end{align*}
      
  \end{enumerate}
\end{solution}
\end{document}
