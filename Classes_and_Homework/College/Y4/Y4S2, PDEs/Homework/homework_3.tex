\documentclass[10pt]{mypackage}

% sans serif font:
%\usepackage{cmbright,sfmath,bbold}
%\renewcommand{\mathcal}{\mathtt}

%Euler:
\usepackage{newpxtext,eulerpx,eucal,eufrak}
\renewcommand*{\mathbb}[1]{\varmathbb{#1}}
\renewcommand*{\hbar}{\hslash}

%\usepackage{homework}

\pagestyle{fancy} %better headers
\fancyhf{}
\rhead{Avinash Iyer}
\lhead{Partial Differential Equations: Homework 3}

\setcounter{secnumdepth}{0}

\begin{document}
\RaggedRight
\section{Textbook Problems}%
\begin{solution}[8.1, Problem 6]
  We have
  \begin{align*}
    \diff{\mathbf{x}}{t} &= \begin{pmatrix}-3 & 4 & 0\\5 & 0 & 9 \\ 0 & 1 & 6\end{pmatrix} \begin{pmatrix}x(t)\\y(t)\\z(t)\end{pmatrix} + e^{-t}\begin{pmatrix}\sin(2t) \\ 4\cos(2t) \\-1 \end{pmatrix}.
  \end{align*}
\end{solution}
\begin{solution}[8.1, Problem 12]
  We have
  \begin{align*}
    \diff{x}{t} &= 5e^{t}\cos(t) - 5e^{t}\sin(t)\\
                &= -10e^{t}\cos(t) + \left( 15e^{t}\cos(t) - 5e^{t}\sin(t) \right)\\
                &= -2x + 5y\\
                \\
    \diff{y}{t} &= 3e^{t}\cos(t) - e^{t}\sin(t) - 3e^{t}\sin(t) - e^{t}\cos(t)\\
                &= 2e^{t}\cos(t) - 4e^{t}\sin(t)\\
                &= \left( -10e^{t}\cos(t) \right) + 12e^{t}\cos(t) - 4e^{t}\sin(t)\\
                &= -2x + 4y.
  \end{align*}
\end{solution}
\begin{solution}[8.1, Problem 20]
  Writing our Wronskian, and using the power of Mathematica, we have
  \begin{align*}
    W(t) &= \det \begin{pmatrix}1 & e^{-4t} & 2e^{3t} \\ 6 & -2e^{-4t} & 3e^{3t} \\ -13 & -e^{-4t} & -2e^{3t}\end{pmatrix}\\
         &= -4e^{-t},
  \end{align*}
  which is zero nowhere along $(-\infty,\infty)$, so the vectors $\mathbf{x}_1,\mathbf{x}_2,\mathbf{x}_3$ form a fundamental set of solutions.
\end{solution}
\begin{solution}[8.1, Problem 26]
  We start with the homogeneous equation:
  \begin{align*}
    \diff{\mathbf{x}}{t} &= \begin{pmatrix}-1 & -1 \\ -1 & 1\end{pmatrix} \mathbf{x}.
  \end{align*}
  Finding the eigenvalues and eigenvectors, we have
  \begin{align*}
    \det \begin{pmatrix}-1-\lambda & -1 \\ -1 & 1-\lambda\end{pmatrix} &= \lambda^2 - 2\\
    \lambda_1 &= \sqrt{2}\\
    \lambda_2 &= -\sqrt{2},
  \end{align*}
  with associated eigenvectors of
  \begin{align*}
    \begin{pmatrix}-1 & -1 \\ -1 & 1\end{pmatrix} \begin{pmatrix}a\\b\end{pmatrix} &= \pm\sqrt{2} \begin{pmatrix}a\\b\end{pmatrix}\\
    \mathbf{v}_1 &= \begin{pmatrix}1\\-1-\sqrt{2}\end{pmatrix} \\
    \mathbf{v}_2 &= \begin{pmatrix}1\\-1+\sqrt{2}\end{pmatrix}.
  \end{align*}
  Therefore, we have the homogeneous solution of
  \begin{align*}
    \mathbf{x}_h(t) &= c_1\begin{pmatrix}-1\\-1-\sqrt{2}\end{pmatrix}e^{\sqrt{2} t} + c_2 \begin{pmatrix}-1\\-1+\sqrt{2}\end{pmatrix} e^{-\sqrt{2}t}.
  \end{align*}
  Note that the Wronskian gives
  \begin{align*}
    W(t) &= \det \begin{pmatrix}e^{\sqrt{2}t} & e^{-\sqrt{2}}t \\ \left( -1-\sqrt{2} \right)e^{\sqrt{2}t} & \left( -1+\sqrt{2} \right)\left( e^{-\sqrt{2}t} \right)\end{pmatrix} \\
         &= 2\sqrt{2},
  \end{align*}
  meaning that these solutions are linearly independent. Next, we examine the particular solution.
  \begin{align*}
    \mathbf{x}_p(t) &= \begin{pmatrix}t^2 - 2t + 1\\4t\end{pmatrix}\\
    \diff{\mathbf{x}_p}{t} &= \begin{pmatrix}2t-2\\4\end{pmatrix}.
  \end{align*}
  Meanwhile, we have
  \begin{align*}
    \begin{pmatrix}-1 & -1 \\ -1 & 1\end{pmatrix} \begin{pmatrix}t^2 - 2t + 1 \\ 4t\end{pmatrix} + \begin{pmatrix}1\\1\end{pmatrix}t^2 + \begin{pmatrix}4\\-6\end{pmatrix} + \begin{pmatrix}-1\\5\end{pmatrix} &= \begin{pmatrix}-t^2-2t-1 + t^2 + 4t - 1 \\ -t^2+2t-1 + 4t - 6t + 5\end{pmatrix}\\
                      &= \diff{\mathbf{x}_{p}}{t}.
  \end{align*}
  Thus, this is the general solution of our equation on $\left( -\infty,\infty \right)$.
\end{solution}
\begin{solution}[8.2, Problem 12]
  Computing the characteristic polynomial, we get
  \begin{align*}
    p(x) &= \left( 6-\lambda \right)\left( 3-\lambda \right)\left( -5-\lambda \right),
  \end{align*}
  giving
  \begin{align*}
    \lambda_1 &= 3\\
    \lambda_2 &= -5\\
    \lambda_3 &= 6.
  \end{align*}
  Plugging these into
  \begin{align*}
    \begin{pmatrix}-1 & 4 & 2 \\ 4 & -1 & -2 \\ 0 & 0 & 6\end{pmatrix} \begin{pmatrix}a\\b\\c\end{pmatrix} &= \lambda_i \begin{pmatrix}a\\b\\c\end{pmatrix},
  \end{align*}
  we get
  \begin{align*}
    \mathbf{v}_1 &= \begin{pmatrix}1\\1\\0\end{pmatrix}\\
    \mathbf{v}_2 &= \begin{pmatrix}-1\\1\\0\end{pmatrix}\\
    \mathbf{v}_3 &= \begin{pmatrix}7\\4\\1\end{pmatrix}.
  \end{align*}
  Thus, the general solution is
  \begin{align*}
    \mathbf{x}(t) &= c_1\begin{pmatrix}1\\1\\0\end{pmatrix} e^{3t} + c_2 \begin{pmatrix}-1\\1\\0\end{pmatrix}e^{-5t} + c_3 \begin{pmatrix}7\\4\\1\end{pmatrix} e^{6t}.
  \end{align*}
\end{solution}
\begin{solution}[8.2, Problem 16]

\end{solution}
\begin{solution}[8.2, Problem 24]

\end{solution}
\begin{solution}[8.2, Problem 28]

\end{solution}
\begin{solution}[8.2, Problem 40]

\end{solution}
\begin{solution}[8.2, Problem 42]

\end{solution}
\begin{solution}[8.2, Problem 46]

\end{solution}
\section{Extra Problems}%

\end{document}
