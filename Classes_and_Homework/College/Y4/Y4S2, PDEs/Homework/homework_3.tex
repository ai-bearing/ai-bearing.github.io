\documentclass[10pt]{mypackage}

% sans serif font:
%\usepackage{cmbright,sfmath,bbold}
%\renewcommand{\mathcal}{\mathtt}

%Euler:
\usepackage{newpxtext,eulerpx,eucal,eufrak}
\renewcommand*{\mathbb}[1]{\varmathbb{#1}}
\renewcommand*{\hbar}{\hslash}

\usepackage{homework}

\pagestyle{fancy} %better headers
\fancyhf{}
\rhead{Avinash Iyer}
\lhead{Partial Differential Equations: Homework 3}

\setcounter{secnumdepth}{0}

\begin{document}
\RaggedRight
\section{Textbook Problems}%
\begin{solution}[8.1, Problem 6]
  We have
  \begin{align*}
    \diff{\mathbf{x}}{t} &= \begin{pmatrix}-3 & 4 & 0\\5 & 0 & 9 \\ 0 & 1 & 6\end{pmatrix} \begin{pmatrix}x(t)\\y(t)\\z(t)\end{pmatrix} + e^{-t}\begin{pmatrix}\sin(2t) \\ 4\cos(2t) \\-1 \end{pmatrix}.
  \end{align*}
\end{solution}
\begin{solution}[8.1, Problem 12]
  We have
  \begin{align*}
    \diff{x}{t} &= 5e^{t}\cos(t) - 5e^{t}\sin(t)\\
                &= -10e^{t}\cos(t) + \left( 15e^{t}\cos(t) - 5e^{t}\sin(t) \right)\\
                &= -2x + 5y\\
                \\
    \diff{y}{t} &= 3e^{t}\cos(t) - e^{t}\sin(t) - 3e^{t}\sin(t) - e^{t}\cos(t)\\
                &= 2e^{t}\cos(t) - 4e^{t}\sin(t)\\
                &= \left( -10e^{t}\cos(t) \right) + 12e^{t}\cos(t) - 4e^{t}\sin(t)\\
                &= -2x + 4y.
  \end{align*}
\end{solution}
\begin{solution}[8.1, Problem 20]
  Writing our Wronskian, and using the power of Mathematica, we have
  \begin{align*}
    W(t) &= \det \begin{pmatrix}1 & e^{-4t} & 2e^{3t} \\ 6 & -2e^{-4t} & 3e^{3t} \\ -13 & -e^{-4t} & -2e^{3t}\end{pmatrix}\\
         &= -4e^{-t},
  \end{align*}
  which is zero nowhere along $(-\infty,\infty)$, so the vectors $\mathbf{x}_1,\mathbf{x}_2,\mathbf{x}_3$ form a fundamental set of solutions.
\end{solution}
\begin{solution}[8.1, Problem 26]
  We start with the homogeneous equation:
  \begin{align*}
    \diff{\mathbf{x}}{t} &= \begin{pmatrix}-1 & -1 \\ -1 & 1\end{pmatrix} \mathbf{x}.
  \end{align*}
  Finding the eigenvalues and eigenvectors, we have
  \begin{align*}
    \det \begin{pmatrix}-1-\lambda & -1 \\ -1 & 1-\lambda\end{pmatrix} &= \lambda^2 - 2\\
    \lambda_1 &= \sqrt{2}\\
    \lambda_2 &= -\sqrt{2},
  \end{align*}
  with associated eigenvectors of
  \begin{align*}
    \begin{pmatrix}-1 & -1 \\ -1 & 1\end{pmatrix} \begin{pmatrix}a\\b\end{pmatrix} &= \pm\sqrt{2} \begin{pmatrix}a\\b\end{pmatrix}\\
    \mathbf{v}_1 &= \begin{pmatrix}1\\-1-\sqrt{2}\end{pmatrix} \\
    \mathbf{v}_2 &= \begin{pmatrix}1\\-1+\sqrt{2}\end{pmatrix}.
  \end{align*}
  Therefore, we have the homogeneous solution of
  \begin{align*}
    \mathbf{x}_h(t) &= c_1\begin{pmatrix}-1\\-1-\sqrt{2}\end{pmatrix}e^{\sqrt{2} t} + c_2 \begin{pmatrix}-1\\-1+\sqrt{2}\end{pmatrix} e^{-\sqrt{2}t}.
  \end{align*}
  Note that the Wronskian gives
  \begin{align*}
    W(t) &= \det \begin{pmatrix}e^{\sqrt{2}t} & e^{-\sqrt{2}}t \\ \left( -1-\sqrt{2} \right)e^{\sqrt{2}t} & \left( -1+\sqrt{2} \right)\left( e^{-\sqrt{2}t} \right)\end{pmatrix} \\
         &= 2\sqrt{2},
  \end{align*}
  meaning that these solutions are linearly independent. Next, we examine the particular solution.
  \begin{align*}
    \mathbf{x}_p(t) &= \begin{pmatrix}t^2 - 2t + 1\\4t\end{pmatrix}\\
    \diff{\mathbf{x}_p}{t} &= \begin{pmatrix}2t-2\\4\end{pmatrix}.
  \end{align*}
  Meanwhile, we have
  \begin{align*}
    \begin{pmatrix}-1 & -1 \\ -1 & 1\end{pmatrix} \begin{pmatrix}t^2 - 2t + 1 \\ 4t\end{pmatrix} + \begin{pmatrix}1\\1\end{pmatrix}t^2 + \begin{pmatrix}4\\-6\end{pmatrix} + \begin{pmatrix}-1\\5\end{pmatrix} &= \begin{pmatrix}-t^2-2t-1 + t^2 + 4t - 1 \\ -t^2+2t-1 + 4t - 6t + 5\end{pmatrix}\\
                      &= \diff{\mathbf{x}_{p}}{t}.
  \end{align*}
  Thus, this is the general solution of our equation on $\left( -\infty,\infty \right)$.
\end{solution}
\begin{solution}[8.2, Problem 12]
  Computing the characteristic polynomial, we get
  \begin{align*}
    p(x) &= \left( 6-\lambda \right)\left( 3-\lambda \right)\left( -5-\lambda \right),
  \end{align*}
  giving
  \begin{align*}
    \lambda_1 &= 3\\
    \lambda_2 &= -5\\
    \lambda_3 &= 6.
  \end{align*}
  Plugging these into
  \begin{align*}
    \begin{pmatrix}-1 & 4 & 2 \\ 4 & -1 & -2 \\ 0 & 0 & 6\end{pmatrix} \begin{pmatrix}a\\b\\c\end{pmatrix} &= \lambda_i \begin{pmatrix}a\\b\\c\end{pmatrix},
  \end{align*}
  we get
  \begin{align*}
    \mathbf{v}_1 &= \begin{pmatrix}1\\1\\0\end{pmatrix}\\
    \mathbf{v}_2 &= \begin{pmatrix}-1\\1\\0\end{pmatrix}\\
    \mathbf{v}_3 &= \begin{pmatrix}7\\4\\1\end{pmatrix}.
  \end{align*}
  Thus, the general solution is
  \begin{align*}
    \mathbf{x}(t) &= c_1\begin{pmatrix}1\\1\\0\end{pmatrix} e^{3t} + c_2 \begin{pmatrix}-1\\1\\0\end{pmatrix}e^{-5t} + c_3 \begin{pmatrix}7\\4\\1\end{pmatrix} e^{6t}.
  \end{align*}
\end{solution}
\begin{solution}[8.2, Problem 16]
  Using the power of Mathematica, we find the eigenvalues of $\set{5.05452, 4.09561, -2.92362, 2.02882, -0.155338}$, with corresponding eigenvectors yielding the general solution of
  \begin{align*}
    \mathbf{x}(t) &= c_1e^{5.05452 t} \begin{pmatrix}-0.0312209\\-0.949058\\-0.239535\\-0.195825\\-0.0508861\end{pmatrix}\\
                  &+ c_2e^{4.09561 t} \begin{pmatrix}-0.280232\\-0.836611\\-0.275304\\0.176045\\0.338775\end{pmatrix}\\
                  &+ c_3e^{-2.92362 t} \begin{pmatrix}0.262219\\=0.162664\\-0.826218\\-0.346439\\0.31957\end{pmatrix}\\
                  &+ c_4e^{2.02882 t} \begin{pmatrix}-0.313235\\-0.64181\\-0.31754\\-0.173787\\0.599018\end{pmatrix} \\
                  &+ c_5e^{-0.155338 t} \begin{pmatrix}-0.301294\\0.466599\\0.222136\\0.0534311\\-0.799567\end{pmatrix}.
  \end{align*}
\end{solution}
\begin{solution}[8.2, Problem 24]
  We start by finding the eigenvalues of the matrix of the equation
  \begin{align*}
    \diff{\mathbf{x}}{t} &= \begin{pmatrix}3 & 2 & 4 \\ 2 & 0 & 2 \\ 4 & 2 & 3\end{pmatrix} \mathbf{x}.
  \end{align*}
  Using computational assistance, we find eigenvalues of $\lambda_1 = 8$, $\lambda_2 = -1$, and $\lambda_3 = -1$. By solving the eigenvector equation for $\lambda_1 = 8$, we obtain the eigenvector of
  \begin{align*}
    \mathbf{v}_1 &= \begin{pmatrix}1\\0\\2\end{pmatrix}.
  \end{align*}
  Meanwhile, the eigenvector equation for $\lambda_2,\lambda_3$ gives eigenvectors of
  \begin{align*}
    \mathbf{v}_2 &= \begin{pmatrix}-1\\-1\\2\end{pmatrix}\\
    \mathbf{v}_3 &= \begin{pmatrix}0\\2\\1\end{pmatrix}.
  \end{align*}
  Therefore, we get the general solution of
  \begin{align*}
    \mathbf{x}(t) &= c_1e^{-t} \begin{pmatrix}-1\\-1\\2\end{pmatrix} + c_2e^{-t} \begin{pmatrix}0\\2\\1\end{pmatrix} + c_3e^{8t} \begin{pmatrix}1\\0\\2\end{pmatrix}.
  \end{align*}
\end{solution}
\begin{solution}[8.2, Problem 28]
  The matrix in the equation
  \begin{align*}
    \diff{\mathbf{x}}{t} &= \begin{pmatrix}4 & 1 & 0 \\ 0 & 4 & 1 \\ 0 & 0 & 4\end{pmatrix} \mathbf{x}
  \end{align*}
  is in Jordan canonical form, meaning that we know that its eigenvalues are $4,4,4$, and the generalized eigenvectors are
  \begin{align*}
    \mathbf{v}_1 &= \begin{pmatrix}1\\0\\0\end{pmatrix}\\
    \mathbf{w}_2 &= \begin{pmatrix}0\\1\\0\end{pmatrix}\\
    \mathbf{w}_3 &= \begin{pmatrix}0\\0\\1\end{pmatrix},
  \end{align*}
  subject to the chain $\mathbf{v}_1\rightarrow \mathbf{w}_2 \rightarrow \mathbf{w}_3$. Thus, our general solution is
  \begin{align*}
    \mathbf{x}(t) &= c_1e^{4t}  \begin{pmatrix}1\\0\\0\end{pmatrix} + c_2e^{4t} \left( t\begin{pmatrix}1\\0\\0\end{pmatrix} + \begin{pmatrix}0\\1\\0\end{pmatrix} \right) + c_3e^{4t} \left( \frac{t^2}{2} \begin{pmatrix}1\\0\\0\end{pmatrix} + t \begin{pmatrix}0\\1\\0\end{pmatrix} + \begin{pmatrix}0\\0\\1\end{pmatrix} \right).
  \end{align*}
\end{solution}
\begin{solution}[8.2, Problem 40]
  We start by taking the equation
  \begin{align*}
    \diff{\mathbf{x}}{t} &= \begin{pmatrix}2 & 1 & 2 \\ 3 & 0 & 6 \\ -4 & 0 & -3\end{pmatrix} \mathbf{x},
  \end{align*}
  and finding the eigenvalues. With computational assistance, we find the characteristic polynomial of
  \begin{align*}
    0 &= \lambda^3 + \lambda^2 - \lambda + 15.
  \end{align*}
  This has roots of
  \begin{align*}
    \lambda &= 3,1\pm 2i.
  \end{align*}
  The corresponding eigenvectors are
  \begin{align*}
    \mathbf{v}_1 &= \begin{pmatrix}0\\-2\\1\end{pmatrix}\\
    \mathbf{v}_{2,3} &= \begin{pmatrix}1 \\-9/2\\1\end{pmatrix}\pm i \begin{pmatrix}1/2\\0\\0\end{pmatrix}.
  \end{align*}
  Now, we have
  \begin{align*}
    \mathbf{x}(t) &= c_1e^{3t} \begin{pmatrix}0\\-2\\1\end{pmatrix} + e^{t} \left( c_2\cos\left( \frac{1}{2}t \right) + c_3\sin\left( \frac{1}{2}t \right) \right) \begin{pmatrix}1\\-9/2\\1\end{pmatrix} + e^{t}\left( c_3\cos\left( \frac{1}{2}t \right) - c_2\sin\left( \frac{1}{2}t \right) \right) \begin{pmatrix}1/2\\0\\0\end{pmatrix}.
  \end{align*}
\end{solution}
\begin{solution}[8.2, Problem 42]
  We start by taking the equation
  \begin{align*}
    \diff{\mathbf{x}}{t} &= \begin{pmatrix}4 & 0 & 1 \\ 0 & 6 & 0 \\ -4 & 0 & 4\end{pmatrix} \mathbf{x}
  \end{align*}
\end{solution}
\begin{solution}[8.2, Problem 46]
  Solving the eigenvectors and eigenvalues for $A$, we have
  \begin{align*}
    \det \left( A - \lambda I \right) &= \lambda^2 - 10 \lambda + 29,
  \end{align*}
  giving eigenvalues of $5\pm 2i$ and corresponding eigenvectors of
  \begin{align*}
    \mathbf{v}_{1,2} &= \begin{pmatrix}1\\0\end{pmatrix} \pm i \begin{pmatrix}0\\2\end{pmatrix}.
  \end{align*}
  Therefore, the general solution is
  \begin{align*}
    \mathbf{x} &= e^{5t} \begin{pmatrix}a\cos(2t) + b\sin(2t) \\ -2a\sin(2t) + 2b\sin(2t)\end{pmatrix}.
  \end{align*}
  Solving the initial condition, we have $a = -2$ and $b = 4$. Therefore we have the full solution of
  \begin{align*}
    \mathbf{x} &= e^{5t} \begin{pmatrix}-2\cos(2t) + 4\sin(2t) \\ -4\sin(2t) + 8\cos(2t)\end{pmatrix}.
  \end{align*}
\end{solution}
\section{Extra Problems}%
\begin{solution}[Extra Problem 1]
To find the linearly independent solutions for $\diff{\mathbf{x}}{t} = A \mathbf{x}$, where
\begin{align*}
  A &= \begin{pmatrix}1 & -4 & & \\ 2 & 5 & & \\ & & 5 & 2 \\ & & -4 & 1\end{pmatrix},
\end{align*}
we find the eigenvectors and eigenvalues of the two block matrices. Using the power of computational linear algebra, we get
\begin{align*}
  \lambda_{1,2} &= 3\pm 2i\\
  \mathbf{v}_{1,2} &= \begin{pmatrix}-1\\1\\0\\0\end{pmatrix} \pm i \begin{pmatrix}1\\0\\0\\0\end{pmatrix}\\
  \lambda_{3,4} &= 3\pm 2i\\
  \mathbf{v}_{3,4} &= \begin{pmatrix}0\\0\\-1\\2\end{pmatrix} \pm i \begin{pmatrix}0\\0\\1\\0\end{pmatrix}.
\end{align*}
Thus, we get the general solution of
\begin{align*}
  \mathbf{x} &= e^{3t} \begin{pmatrix}-\cos\left( 2t \right)-\sin\left(2t\right) \\ \cos\left(2t\right) \\ -\cos\left(2t\right)-\sin\left(2t\right) \\ 2\cos\left(2t\right)\end{pmatrix}.
\end{align*}
\end{solution}
\begin{solution}[Extra Problem 2]\hfill
  \begin{enumerate}[(i)]
    \item We have the generalized eigenvector chain $\mathbf{v}_1 \rightarrow \mathbf{w}_1$, where
      \begin{align*}
        \mathbf{v}_1 &= \begin{pmatrix}1\\0\\0\\0\end{pmatrix}\\
        \mathbf{w}_1 &= \begin{pmatrix}0\\1\\0\\0\end{pmatrix}\\
        \mathbf{v}_3 &= \begin{pmatrix}0\\0\\1\\0\end{pmatrix}\\
        \mathbf{v}_4 &= \begin{pmatrix}0\\0\\0\\1\end{pmatrix}.
      \end{align*}
    \item We have the generalized eigenvector chain $\mathbf{v}_3 \rightarrow \mathbf{w}_1$, where
      \begin{align*}
        \mathbf{v}_1 &= \begin{pmatrix}1\\0\\0\\0\end{pmatrix}\\
        \mathbf{v}_2 &= \begin{pmatrix}0\\1\\0\\0\end{pmatrix}\\
        \mathbf{v}_3 &= \begin{pmatrix}0\\0\\1\\0\end{pmatrix}\\
        \mathbf{w}_1 &= \begin{pmatrix}0\\0\\0\\1\end{pmatrix}.
      \end{align*}
    \item We have the generalized eigenvector chain $\mathbf{v}_1 \rightarrow \mathbf{w}_1\rightarrow \mathbf{w}_2$, where
      \begin{align*}
        \mathbf{v}_1 &= \begin{pmatrix}1\\0\\0\\0\end{pmatrix}\\
        \mathbf{w}_1 &= \begin{pmatrix}0\\1\\0\\0\end{pmatrix}\\
        \mathbf{w}_2 &= \begin{pmatrix}0\\0\\1\\0\end{pmatrix}\\
        \mathbf{v}_4 &= \begin{pmatrix}0\\0\\0\\1\end{pmatrix}.
      \end{align*}
    \item We have the generalized eigenvector chain $\mathbf{v}_1 \rightarrow \mathbf{w}_1 \rightarrow \mathbf{w}_2 \rightarrow \mathbf{w}_3$, where
      \begin{align*}
        \mathbf{v}_1 &= \begin{pmatrix}1\\0\\0\\0\end{pmatrix}\\
        \mathbf{w}_1 &= \begin{pmatrix}0\\1\\0\\0\end{pmatrix}\\
        \mathbf{w}_2 &= \begin{pmatrix}0\\0\\1\\0\end{pmatrix}\\
        \mathbf{w}_3 &= \begin{pmatrix}0\\0\\0\\1\end{pmatrix}.
      \end{align*}
  \end{enumerate}
  An addition to the chain of generalized eigenvectors adds a $1$ to the Jordan block.
\end{solution}

\end{document}
