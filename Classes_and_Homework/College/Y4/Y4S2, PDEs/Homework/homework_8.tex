\documentclass[10pt]{mypackage}

% sans serif font:
%\usepackage{cmbright,sfmath,bbold}
%\renewcommand{\mathcal}{\mathtt}

%Euler:
\usepackage{newpxtext,eulerpx,eucal,eufrak}
\renewcommand*{\mathbb}[1]{\varmathbb{#1}}
\renewcommand*{\hbar}{\hslash}

%\renewcommand{\mathbb}{\mathds}
\usepackage{homework}
%\usepackage{exposition}

\pagestyle{fancy} %better headers
\fancyhf{}
\rhead{Avinash Iyer}
\lhead{Partial Differential Equations: Homework 8}

\setcounter{secnumdepth}{0}

\begin{document}
\RaggedRight
\renewcommand{\arraystretch}{1.75}
\begin{solution}[12.4, Problem 14]\hfill
  \begin{enumerate}[(a)]
    \item We may write
      \begin{align*}
        \xi &=x + at\\
        \eta &= x- at,
      \end{align*}
      so
      \begin{align*}
        \pd{}{\xi} &= \pd{}{x} + a\pd{}{t}\\
        \pd{}{\eta} &= \pd{}{x} - a\pd{}{t}.
      \end{align*}
      Thus,
      \begin{align*}
        \left( \pd{}{x}-a\pd{}{t} \right)\left( \pd{}{x} + a\pd{}{t} \right) u &= \pd{^2u}{\xi\partial\eta}.
      \end{align*}
    \item Integrating, we have
      \begin{align*}
        \int_{}^{} \pd{^2u}{\eta\partial \xi}\:d\eta &= c_1(\xi)\\
        \int_{}^{} \pd{u}{\xi}\:d\xi &= c_2(\eta),
      \end{align*}
      so
      \begin{align*}
        \int_{}^{} \int_{}^{} \pd{^2u}{\eta\partial \xi}\:d\xi\:d\eta &= c_1(x+at) + c_2(x-at).
      \end{align*}
      Using this solution, we know that if
      \begin{align*}
        u\left( x,t \right) &= c_1(x+at) + c_2(x-at),
      \end{align*}
      then
      \begin{align*}
        c_1(x) + c_2(x) &= f(x)\\
        a\diff{}{x}\left( c_1(x) - c_2(x) \right) &= g(x).
      \end{align*}
      Using the fundamental theorem of calculus, we get that
      \begin{align*}
        c_1(x) + c_2(x) &= f(x)\\
        c_1(x) - c_2(x) &= \frac{1}{a} \int_{x_0}^{x} g(s)\:ds + k
      \end{align*}
      which gives the solutions
      \begin{align*}
        c_1(x) &= \frac{1}{2}f(x) + \frac{1}{2a} \int_{x_0}^{x} g(s)\:ds + k/2\\
        c_2(x) &= \frac{1}{2}f(x) - \frac{1}{2a} \int_{x_0}^{x} g(s)\:ds - k/2.
      \end{align*}
    \item Summing and substituting, we get
      \begin{align*}
        u\left( x,t \right) &= c_1(x+at) + c_2(x-at)\\
                            &= \frac{1}{2}\left( f(x+at) + f(x-at) \right) + \frac{1}{2a} \int_{x_0}^{x+at} g(s)\:ds - \frac{1}{2a} \int_{x_0}^{x-at} g(s)\:ds\\
                            &= \frac{1}{2} \left( f(x+at) + f(x-at) \right) + \frac{1}{2a} \left( \int_{x_0}^{x+at} g(s)\:ds + \int_{x-at}^{x_0} g(s)\:ds \right)\\
                            &= \frac{1}{2} \left( f(x+at) + f(x-at) \right) + \frac{1}{2a} \int_{x-at}^{x+at} g(s)\:ds.
      \end{align*}
  \end{enumerate}
\end{solution}
\begin{solution}[12.4, Problem 16]
      \begin{align*}
        u\left( x,t \right) &= \frac{1}{2}\left( \sin\left( x+at \right) + \sin\left( x-at \right) \right) + \frac{1}{2a} \int_{x-at}^{x+at} \cos(s)\:ds\\
                            &= \frac{1}{2}\left( \sin\left( x+at \right) + \sin\left( x-at \right) \right) + \frac{1}{2a}\left( \sin\left( x+at \right) - \sin\left( x-at \right) \right) 
      \end{align*}
\end{solution}
\begin{solution}[12.4, Problem 18]
  \begin{align*}
    u\left( x,t \right) &= \frac{1}{2}\left( e^{-\left( x+at \right)^2} + e^{-\left( x-at \right)^2} \right).
  \end{align*}
\end{solution}
\begin{solution}[Method of Characteristics Problems]\hfill
  \begin{enumerate}[(i)]
    \item We have
      \begin{align*}
        \pd{u}{t} + u^2\pd{u}{x} &= 1,
      \end{align*}
      giving the vector identity
      \begin{align*}
        \begin{pmatrix}\pd{u}{t}\\\pd{u}{x}\\-1\end{pmatrix} \cdot \begin{pmatrix}1 \\ u^2 \\ -1\end{pmatrix} &= 0.
      \end{align*}
      Writing the Lagrange--Charpit equations, we get
      \begin{align*}
        \diff{t}{s} &= 1\\
        \diff{x}{s} &= u^2\\
        \diff{u}{s} &= 1.
      \end{align*}
      We have the parametrization of $x(s) = su^2 + x_0$, so $x_0 = x - tu^2$, substituting $t = s$. Finally, we have $u(s) = s + u_0$, so $u_0 = u - t$. Therefore, we have the implicitly defined function
      \begin{align*}
        u\left( x,t \right) &= u_0\left( x_0 \right)\\
                            &= e^{x-tu^2} - t.
      \end{align*}
    \item Putting into standard form, we have the equation
      \begin{align*}
        \pd{u}{t} + te^{t} \pd{u}{x} &= e^{t}u.
      \end{align*}
      There are two equations here to solve. We start with the equation in $x$, which gives
      \begin{align*}
        \diff{x}{t} &= te^{t},
      \end{align*}
      so
      \begin{align*}
        x &= te^{t}-e^{t} + x_0,
      \end{align*}
      and
      \begin{align*}
        x_0 &= x+e^{t}-te^{t}.
      \end{align*}
      Solving the equation in $u$, we have
      \begin{align*}
        \diff{u}{t} &= e^{t} u\\
        u &= u_0e^{e^{t}}.
      \end{align*}
      Therefore,
      \begin{align*}
        u\left( x,t \right) &= u_0\left( x_0 \right)e^{e^{t}}\\
                            &= \cos\left( \left( x + e^{t} - te^{t} \right)^2 \right)e^{e^{t}}.
      \end{align*}
    \item In standard form, the equation becomes
      \begin{align*}
        \pd{u}{t} + t\pd{u}{x} &= 0,
      \end{align*}
      so we have the solution
      \begin{align*}
        u\left( x,t \right) &= u_0\left( x-\frac{1}{2}t^2 \right)\\
                            &= \frac{1}{\left( x-\frac{1}{2}t^2 \right)^2 + 2}.
      \end{align*}
    \item We have
      \begin{align*}
        \diff{x}{t} &= 4x\\
        x &= x_0e^{4t}\\
        x_0 &= xe^{-4t}.
      \end{align*}
      Furthermore, since
      \begin{align*}
        \diff{u}{t} &= t,
      \end{align*}
      we get
      \begin{align*}
        u\left( x,t \right) &= \frac{1}{2}t^2 + u_0,
      \end{align*}
      so
      \begin{align*}
        u\left( x,t \right) &= \frac{1}{2}t^2 + \left( xe^{-4t} \right)^3.
      \end{align*}
    \item We have
      \begin{align*}
        \diff{y}{t} &= y\\
        y_0 &= ye^{-t}\\
        \diff{x}{t} &= 1\\
        x_0 &= x-t,
      \end{align*}
      so our solution is
      \begin{align*}
        u\left( x,y,t \right) &= u_0\left( x_0,y_0 \right)\\
                              &= x-t + ye^{-t}.
      \end{align*}
    \item We have
      \begin{align*}
        \diff{y}{t} &= y\\
        y_0 &= ye^{-t}\\
        \diff{x}{t} &= 1\\
        x_0 &= x-t\\
        \diff{u}{t} &= u\\
        u &= u_0e^{t}\\
          &= \left( x-t + ye^{-t} \right)e^{t}.
      \end{align*}
  \end{enumerate}
\end{solution}
\begin{solution}[D'Alembert's Method Problems]\hfill
  \begin{enumerate}[(i)]
    \item We have
      \begin{align*}
        \left( \pd{}{t} + 2\pd{}{x} \right)\left( \pd{}{t} - \pd{}{x} \right) u &= 0,
      \end{align*}
      giving characteristic curves of $x+2t$ and $x-t$. Therefore, if
      \begin{align*}
        u\left( x,t \right) &= h\left( x+2t \right) + k\left( x-t \right),
      \end{align*}
      we have
      \begin{align*}
        h(x) + k(x) &= u\left( x,0 \right)\\
        2h(x) - k(x) &= \int_{x_0}^{x} g(s)\:ds,
      \end{align*}
      giving
      \begin{align*}
        h(x) &= \frac{1}{3}u\left( x,0 \right) + \frac{1}{3} \int_{x_0}^{x} g(s)\:ds\\
        k(x) &= \frac{2}{3} u\left( x,0 \right) - \frac{1}{3} \int_{x_0}^{x} g(s)\:ds,
      \end{align*}
      so that
      \begin{align*}
        u\left( x,t \right) &= \frac{1}{3}\sin\left( x+2t \right) + \frac{2}{3}\sin\left( x-t \right) + \frac{1}{3} \int_{x-t}^{x+2t} e^{s}\:ds\\
                            &= \frac{1}{3}\sin\left( x+2t \right) + \frac{2}{3}\sin\left( x-t \right) + \frac{1}{3}\left( e^{x+2t} - e^{x-t} \right).
      \end{align*}
    \item We have
      \begin{align*}
        \pd{}{t}\left( \pd{}{t} + 9\pd{}{x} \right)u &= 0,
      \end{align*}
      so we have characteristic curves of $x$ and $x + 9t$. Therefore, if
      \begin{align*}
        u\left( x,t \right) &= h\left( x + 9t \right) + k\left( x \right),
      \end{align*}
      we have
      \begin{align*}
        h\left( x \right) + k\left( x \right) &= u\left( x,0 \right)\\
        9h(x) &= \int_{x_0}^{x} g(s)\:ds,
      \end{align*}
      so
      \begin{align*}
        u\left( x,t \right) &= x^2 + 1 + \frac{1}{9} \int_{x}^{x + 9t} s\:ds\\
                            &= x^2 + 1 + \frac{1}{9}\left( \frac{1}{9}\left( x+9t \right)^2 - \frac{1}{2}x^2 \right).
      \end{align*}
    \item Factoring, we have
      \begin{align*}
        \left( \pd{}{t}-3\pd{}{x} \right)\left( \pd{}{t}-2\pd{}{x} \right)\left( \pd{}{t}-\pd{}{x} \right)u &= 0,
      \end{align*}
      with the three characteristic curves of $x-3t$, $x-2t$, and $x-t$. We thus have the equations
      \begin{align*}
        c_1 + c_2 + c_3 &= u\left( x,0 \right)\\
        -3c_1-2c_2-c_3 &= \int_{x_0}^{x} g(s)\:ds\\
        9c_1 + 4c_2 + c_3 &= \int_{x_0}^{x} h(s)\:ds,
      \end{align*}
      which yields solutions of
      \begin{align*}
        c_1(x) &= u\left( x,0 \right) + \frac{3}{2} \int_{x_0}^{x} g(s)\:ds + \frac{1}{2} \int_{x_0}^{x} h(s)\:ds\\
        c_2(x) &= -3u\left( x,0 \right) - 4 \int_{x_0}^{x} g(s)\:ds - \int_{x_0}^{x} h(s)\:ds\\
        c_3(x) &= 3u\left( x,0 \right) + \frac{5}{2} \int_{x_0}^{x} g(s)\:ds + \frac{1}{2} \int_{x_0}^{x} h(s)\:ds.
      \end{align*}
      This gives the solution of
      \begin{align*}
        u\left( x,t \right) &= \left( x-3t \right)^2 + 1 - 3\left( x-2t \right)^2-3 + 3\left( x-t \right)^2 + 3\\
                            &+ \frac{1}{2}\left( \frac{1}{2}\left( x-3t \right)^2 - \frac{1}{2}\left( x-2t \right)^2 \right) + \frac{5}{2}\left( \frac{1}{2}\left( x-t \right)^2 - \frac{1}{2}\left( x-2t \right)^2 \right)\\
                            &+ \frac{1}{2}\left( \frac{1}{2}\left( x-3t \right)^2 - \frac{1}{2}\left( x-2t \right)^2 \right) + \frac{1}{2}\left( \frac{1}{2}\left( x-t \right)^2-\frac{1}{2}\left( x-2t \right)^2 \right)\\
                            \\
                            &= \left( x-3t \right)^2 - 3\left( x-2t \right)^2 + 3\left( x-t \right)^2 + \frac{1}{2}\left( x-3t \right)^2 - 2\left( x-2t \right)^2 + \frac{3}{2}\left( x-t \right)^2 + 1\\
                            &= \frac{3}{2}\left( x-3t \right)^2 - 5\left( x-2t \right)^2 + \frac{9}{2}\left( x-t \right)^2 + 1.
      \end{align*}
  \end{enumerate}
\end{solution}
\begin{solution}[Hyperbolic System Problem]
  We have the hyperbolic system
  \begin{align*}
    \mathbf{w}(x,t) &= \begin{pmatrix}u(x,t)\\v(x,t)\end{pmatrix}\\
    \pd{\mathbf{w}}{t} &= \begin{pmatrix}-4 & -2 \\ -3 & 1\end{pmatrix} \pd{\mathbf{w}}{x}.
  \end{align*}
  with
  \begin{align*}
    \mathbf{w}_0 &= \begin{pmatrix}x+2\\x^2 + 1\end{pmatrix}.
  \end{align*}
  The diagonalization of the matrix $A$ is
  \begin{align*}
    P &= \begin{pmatrix}1 & 2 \\ -3 & 1\end{pmatrix}\\
    D &= \begin{pmatrix}2 & \\ & -5\end{pmatrix}\\
    P^{-1} &= \begin{pmatrix}-1/7 & 2/7\\-3/7 & -1/7\end{pmatrix}.
  \end{align*}
  We take the decoupled system
  \begin{align*}
    \mathbf{z}_0 &= \begin{pmatrix}-1/7 & 2/7\\-3/7 & -1/7\end{pmatrix} \begin{pmatrix}x+ 2\\x^2 + 1\end{pmatrix}\\
                 &= \frac{1}{7} \begin{pmatrix}-(x+2) + 2\left( x^2 + 1 \right)\\ -3(x+2) - 1(x^2 + 1)\end{pmatrix}\\
                 &= \frac{1}{7} \begin{pmatrix}2x^2-x\\ -x^2-3x-7\end{pmatrix}.
  \end{align*}
  The decoupled system is
  \begin{align*}
    \pd{z_1}{t} &= 2 \pd{z_1}{x}\\
    \pd{z_2}{t} &= -5\pd{z_1}{x},
  \end{align*}
  with
  \begin{align*}
    z_1(x,0) &= \frac{1}{7}\left( 2x^2 - x \right)\\
    z_2(x,0) &= \frac{1}{7}\left( -x^2-3x-7 \right).
  \end{align*}
  Therefore, we have
  \begin{align*}
    z_1\left( x,t \right) &= \frac{1}{7}\left( 2\left( x+2t \right)^2-(x+2t) \right)\\
    z_2\left( x,t \right) &= \frac{1}{7}\left( -\left( x-5t \right)^2-3\left( x-5t \right)-7 \right),
  \end{align*}
  and
  \begin{align*}
    \mathbf{w}\left( x,t \right) &= \frac{1}{7} \begin{pmatrix}1 & 2 \\ -3 & 1\end{pmatrix} \begin{pmatrix}2\left( x+2t \right)^2-(x+2t)\\ -\left( x-5t \right)^2-3\left( x-5t \right)-7\end{pmatrix}\\
                                 &= \frac{1}{7} \begin{pmatrix}\left( x+2t \right)^2-(x+2t) - 2\left( x-5t \right)^2-6\left( x-5t \right)-14\\ -6\left( x+2t \right)^2+3\left( x+2t \right) - \left( x-5t \right)^2-3\left( x-5t \right)-7\end{pmatrix}.
  \end{align*}
  
\end{solution}

\end{document}
