\documentclass[10pt]{mypackage}

% sans serif font:
%\usepackage{cmbright,sfmath,bbold}
%\renewcommand{\mathcal}{\mathtt}

%Euler:
\usepackage{newpxtext,eulerpx,eucal,eufrak}
\renewcommand*{\mathbb}[1]{\varmathbb{#1}}
\renewcommand*{\hbar}{\hslash}

\usepackage{homework}

\pagestyle{fancy} %better headers
\fancyhf{}
\rhead{Avinash Iyer}
\lhead{Partial Differential Equations: Homework 5}

\setcounter{secnumdepth}{0}

\begin{document}
\RaggedRight
I have not shown most of the extraneous work because it is tedious to show.
\begin{solution}[12.1, Problem 2]
  Separating with $u=X(x)Y(y)$, we have
  \begin{align*}
    Y\diff{X}{x} + 3X\diff{Y}{y} &= 0,
  \end{align*}
  so that
  \begin{align*}
    \diff{X}{x} &= CX\\
    \diff{Y}{y} &= -\frac{C}{3}Y,
  \end{align*}
  meaning
  \begin{align*}
    u\left( x,y \right) &= Ke^{Cx - \frac{C}{3}y}.
  \end{align*}
\end{solution}
\begin{solution}[12.1, Problem 4]
  Separating by taking $u\left( x,y \right) = X(x)Y(y)$, we have
  \begin{align*}
    \frac{1}{X}\left( \diff{X}{x} \right) &= \frac{1}{Y}\left( \diff{Y}{y} \right) + 1.
  \end{align*}
  Therefore, this equation splits into
  \begin{align*}
    \diff{X}{x} &= CX\\
    \diff{Y}{y} &= \left( C-1 \right) Y,
  \end{align*}
  yielding the solution of
  \begin{align*}
    u\left( x,y \right) &= Ke^{Cx + \left( C-1 \right)y}.
  \end{align*}
\end{solution}
\begin{solution}[12.1, Problem 10]
  Separating with $u\left( x,t \right) = X(x)T(t)$, we have
  \begin{align*}
    kT(t)\diff{^2X}{x^2} = X(t)\diff{T}{t},
  \end{align*}
  so that
  \begin{align*}
    \frac{k}{X} \left( \diff{^2X}{x^2} \right) &= \frac{1}{T}\left( \diff{T}{t} \right).
  \end{align*}
  Setting these quantities equal to $C$, we have
  \begin{align*}
    u\left( x,t \right)  &= \begin{cases}
      e^{Ct}\left( A\cos\left( \sqrt{\frac{-C}{k}}x \right) + B\cos\left( \sqrt{\frac{-C}{k}}x \right) \right) & C < 0\\
      e^{Ct}\left( Ae^{\sqrt{\frac{C}{k}}x} + Be^{-\sqrt{\frac{C}{k}}x} \right) & C > 0\\
      Ax + B & C = 0.
    \end{cases}
  \end{align*}
\end{solution}
\begin{solution}[12.1, Problem 12]
  Separating with $u\left( x,t \right) = X(x)T(t)$, we get
  \begin{align*}
    \frac{a^2}{X}\left( \diff{^2X}{x^2} \right) &= \frac{1}{T}\left( \diff{^2T}{t^2} + 2k\diff{T}{t} \right).
  \end{align*}
  Setting equal to $C$ and going through tedious algebra, we have the solution
  \begin{align*}
    u\left( x,t \right) &= \begin{cases}
      \left( a_1e^{\left( -k+\sqrt{k^2 + C} \right)t} + a_2e^{\left( -k+ \sqrt{k^2 + C} \right)t} \right)\left( b_1e^{\frac{\sqrt{C}}{a}x} + b_2e^{-\frac{\sqrt{C}}{a}x} \right) & c > 0\\
      \left( a_1e^{\left( -k+\sqrt{k^2 + C} \right)t} + a_2e^{\left( -k+ \sqrt{k^2 + C} \right)t} \right)\left( Ax + B \right) & C = 0\\
      \left( a_1e^{\left( -k+\sqrt{k^2 + C} \right)t} + a_2e^{\left( -k+ \sqrt{k^2 + C} \right)t} \right)\left( b_1\cos\left( \sqrt{\frac{-C}{a}}x  \right)+ b_2\sin\left( \sqrt{\frac{-C}{a}}x \right) \right) & -k^2 < C < 0\\
      \left( a_1e^{-kt} + a_2te^{-kt} \right)\left( b_1\cos\left( \sqrt{\frac{-C}{a}}x  \right)+ b_2\sin\left( \sqrt{\frac{-C}{a}}x \right) \right) & C = -k^2\\
      e^{-kt}\left( a_1\cos\left( \sqrt{\left\vert k^2 + c \right\vert}x \right) + a_2\sin\left( \sqrt{\left\vert k^2 + c \right\vert}x \right) \right)\left( b_1\cos\left( \sqrt{\frac{-C}{a}}x  \right)+ b_2\sin\left( \sqrt{\frac{-C}{a}}x \right) \right) & C < -k^2
    \end{cases}
  \end{align*}
\end{solution}
\begin{solution}[12.1, Problem 18]
  Since $B = 5$, $A = 3$, and $C = 1$, this is a hyperbolic PDE.
\end{solution}
\begin{solution}[12.2, Problem 2]
  The boundary value problem is
  \begin{align*}
    u\left( x,0 \right) &= 0\\
    u\left( 0,t \right) &= u_0\\
    u\left( L,t \right) &= u_1.
  \end{align*}
\end{solution}
\begin{solution}[12.2, Problem 4]
  The boundary value problem is
  \begin{align*}
    \pd{u}{x}\biggr\vert_{(0,t)} &= 0\\
    \pd{u}{x}\biggr\vert_{(0,t)} &= 0\\
    u\left( x,0 \right) &= 100\\
    \pd{u}{t}\biggr\vert_{(x,t)} &= -50.
  \end{align*}
  
\end{solution}
\begin{solution}[12.2, Problem 6]

\end{solution}
\begin{solution}[11.1, Problem 2]

\end{solution}
\begin{solution}[11.1, Problem 4]

\end{solution}
\begin{solution}[11.1, Problem 10]

\end{solution}
\begin{solution}[11.1, Problem 12]

\end{solution}
\begin{solution}[Extra Problem]\hfill

\end{solution}

\end{document}
