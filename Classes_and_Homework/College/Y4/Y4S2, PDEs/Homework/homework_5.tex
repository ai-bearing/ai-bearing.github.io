\documentclass[10pt]{mypackage}

% sans serif font:
%\usepackage{cmbright,sfmath,bbold}
%\renewcommand{\mathcal}{\mathtt}

%Euler:
\usepackage{newpxtext,eulerpx,eucal,eufrak}
\renewcommand*{\mathbb}[1]{\varmathbb{#1}}
\renewcommand*{\hbar}{\hslash}

\usepackage{homework}

\pagestyle{fancy} %better headers
\fancyhf{}
\rhead{Avinash Iyer}
\lhead{Partial Differential Equations: Homework 5}

\setcounter{secnumdepth}{0}

\begin{document}
\RaggedRight
I have not shown most of the extraneous work because it is tedious to show.
\begin{solution}[12.1, Problem 2]
  Separating with $u=X(x)Y(y)$, we have
  \begin{align*}
    Y\diff{X}{x} + 3X\diff{Y}{y} &= 0,
  \end{align*}
  so that
  \begin{align*}
    \diff{X}{x} &= CX\\
    \diff{Y}{y} &= -\frac{C}{3}Y,
  \end{align*}
  meaning
  \begin{align*}
    u\left( x,y \right) &= Ke^{Cx - \frac{C}{3}y}.
  \end{align*}
\end{solution}
\begin{solution}[12.1, Problem 4]
  Separating by taking $u\left( x,y \right) = X(x)Y(y)$, we have
  \begin{align*}
    \frac{1}{X}\left( \diff{X}{x} \right) &= \frac{1}{Y}\left( \diff{Y}{y} \right) + 1.
  \end{align*}
  Therefore, this equation splits into
  \begin{align*}
    \diff{X}{x} &= CX\\
    \diff{Y}{y} &= \left( C-1 \right) Y,
  \end{align*}
  yielding the solution of
  \begin{align*}
    u\left( x,y \right) &= Ke^{Cx + \left( C-1 \right)y}.
  \end{align*}
\end{solution}
\begin{solution}[12.1, Problem 10]
  Separating with $u\left( x,t \right) = X(x)T(t)$, we have
  \begin{align*}
    kT(t)\diff{^2X}{x^2} = X(t)\diff{T}{t},
  \end{align*}
  so that
  \begin{align*}
    \frac{k}{X} \left( \diff{^2X}{x^2} \right) &= \frac{1}{T}\left( \diff{T}{t} \right).
  \end{align*}
  Setting these quantities equal to $C$, we have
  \begin{align*}
    u\left( x,t \right)  &= \begin{cases}
      e^{Ct}\left( A\cos\left( \sqrt{\frac{-C}{k}}x \right) + B\cos\left( \sqrt{\frac{-C}{k}}x \right) \right) & C < 0\\
      e^{Ct}\left( Ae^{\sqrt{\frac{C}{k}}x} + Be^{-\sqrt{\frac{C}{k}}x} \right) & C > 0\\
      Ax + B & C = 0.
    \end{cases}
  \end{align*}
\end{solution}
\begin{solution}[12.1, Problem 12]
  Separating with $u\left( x,t \right) = X(x)T(t)$, we get
  \begin{align*}
    \frac{a^2}{X}\left( \diff{^2X}{x^2} \right) &= \frac{1}{T}\left( \diff{^2T}{t^2} + 2k\diff{T}{t} \right).
  \end{align*}
  Setting equal to $C$ and going through tedious algebra, we have the solution
  \begin{align*}
    u\left( x,t \right) &= \begin{cases}
      \left( a_1e^{\left( -k+\sqrt{k^2 + C} \right)t} + a_2e^{\left( -k+ \sqrt{k^2 + C} \right)t} \right)\left( b_1e^{\frac{\sqrt{C}}{a}x} + b_2e^{-\frac{\sqrt{C}}{a}x} \right) & c > 0\\
      \left( a_1e^{\left( -k+\sqrt{k^2 + C} \right)t} + a_2e^{\left( -k+ \sqrt{k^2 + C} \right)t} \right)\left( Ax + B \right) & C = 0\\
      \left( a_1e^{\left( -k+\sqrt{k^2 + C} \right)t} + a_2e^{\left( -k+ \sqrt{k^2 + C} \right)t} \right)\left( b_1\cos\left( \sqrt{\frac{-C}{a}}x  \right)+ b_2\sin\left( \sqrt{\frac{-C}{a}}x \right) \right) & -k^2 < C < 0\\
      \left( a_1e^{-kt} + a_2te^{-kt} \right)\left( b_1\cos\left( \sqrt{\frac{-C}{a}}x  \right)+ b_2\sin\left( \sqrt{\frac{-C}{a}}x \right) \right) & C = -k^2\\
      e^{-kt}\left( a_1\cos\left( \sqrt{\left\vert k^2 + c \right\vert}x \right) + a_2\sin\left( \sqrt{\left\vert k^2 + c \right\vert}x \right) \right)\left( b_1\cos\left( \sqrt{\frac{-C}{a}}x  \right)+ b_2\sin\left( \sqrt{\frac{-C}{a}}x \right) \right) & C < -k^2
    \end{cases}
  \end{align*}
\end{solution}
\begin{solution}[12.1, Problem 18]
  Since $B = 5$, $A = 3$, and $C = 1$, this is a hyperbolic PDE.
\end{solution}
\begin{solution}[12.2, Problem 2]
  The boundary value problem is
  \begin{align*}
    u\left( x,0 \right) &= 0\\
    u\left( 0,t \right) &= u_0\\
    u\left( L,t \right) &= u_1.
  \end{align*}
\end{solution}
\begin{solution}[12.2, Problem 4]
  The boundary value problem is
  \begin{align*}
    \pd{u}{x}\biggr\vert_{(0,t)} &= 0\\
    \pd{u}{x}\biggr\vert_{(L,t)} &= 0\\
    u\left( x,0 \right) &= 100\\
    \pd{u}{t}\biggr\vert_{(x,t)} &= -50.
  \end{align*}
\end{solution}
\begin{solution}[12.2, Problem 6]
  The boundary value problem is
  \begin{align*}
    \pd{u}{t} &= \sin\left( \pi x/L \right)\\
    u\left( 0,t \right) &= 0\\
    u\left( L,t \right) &= 0\\
    u\left( x,0 \right) &= 0.
  \end{align*}
\end{solution}
\begin{solution}[11.1, Problem 2]
  We evaluate
  \begin{align*}
    \iprod{f_1}{f_2} &= \int_{-1}^{1} \left( x^3 \right)\left( x^2 + 1 \right)\:dx\\
                     &= \int_{-1}^{1} x^5 + x^3\:dx\\
                     &= 0,
  \end{align*}
  by even/odd rules.
\end{solution}
\begin{solution}[11.1, Problem 4]
  We evaluate
  \begin{align*}
    \iprod{f_1}{f_2} &= \int_{0}^{\pi} \cos\left( x \right)\sin^2\left( x \right)\:dx\\
                     &= - \int_{0}^{0} u^2\:dx\tag*{$u=\sin\left( x \right)$}\\
                     &= 0.
  \end{align*}
\end{solution}
\begin{solution}[11.1, Problem 10]
  We evaluate
  \begin{align*}
    \iprod{\sin\left( \frac{n\pi x}{p} \right)}{ \sin\left( \frac{m \pi x}{p} \right) } &= \int_{0}^{p} \sin\left( \frac{n\pi x}{p} \right) \sin\left( \frac{m\pi x}{p} \right)\:dx\\
                                                                                        &= \int_{0}^{\pi} \sin\left( nt \right)\sin\left( mt \right)\:dt\\
                                                                                        &= \begin{cases}
                                                                                          0 & m\neq n\\
                                                                                          \frac{\pi}{2} & m=n
                                                                                        \end{cases}.
  \end{align*}
\end{solution}
\begin{solution}[11.1, Problem 12]
  For two separate ``classes'' of functions, we have
  \begin{align*}
    \int_{-p}^{p} \sin\left( \frac{m\pi x}{p} \right)\left( 1 \right)\:dx &= 0\\
    \int_{-p}^{p} \cos\left( \frac{m\pi x}{p} \right)\left( 1 \right)\:dx &= 0\\
    \int_{-p}^{p} \cos\left( \frac{m\pi x}{p} \right)\sin\left( \frac{n\pi x}{p} \right)\:dx &= 0\\
    \int_{-p}^{p} \cos\left( \frac{n\pi x}{p} \right)\sin\left( \frac{n\pi x}{p} \right)\:dx &= 0.
  \end{align*}
  Furthermore, for two members of the same ``class'' of functions with different $m,n$, we know that
  \begin{align*}
    \int_{-p}^{p} \cos\left( \frac{n\pi x}{p} \right) \cos\left( \frac{m\pi x}{p} \right)\:dx &= \int_{-\pi}^{\pi} \cos\left( nx \right)\cos\left( mx \right)\:dx\\
                                                                                              &= 0\\
    \int_{-p}^{p} \sin\left( \frac{n\pi x}{p} \right) \sin\left( \frac{m\pi x}{p} \right)\:dx &= \int_{-\pi}^{\pi} \sin\left( nx \right)\sin\left( mx \right)\:dx\\
                                                                                              &= 0.
  \end{align*}
  Evaluating norms, we get
  \begin{align*}
    \int_{-p}^{p} \sin^2\left( \frac{n\pi x}{p} \right)\:dx &= p\\
    \int_{-p}^{p} \cos^2\left( \frac{n\pi x}{p} \right)\:dx &= p\\
    \int_{-p}^{p} \:dx &= 2p\\
  \end{align*}
  
\end{solution}
\begin{solution}[Extra Problem]\hfill
  \begin{enumerate}[(i)]
    \item We recognize this as the transport equation with $a = -3$, so the solution is
      \begin{align*}
        u\left( x,t \right) &= \ln\left( x + 3t - 1 \right),
      \end{align*}
      with
      \begin{align*}
        u\left( 3,40 \right) &= \ln\left( 122 \right)\\
        u\left( 40,3 \right) &= \ln\left( 48 \right).
      \end{align*}
    \item We use separation of variables to solve the heat equation, taking $u\left( x,t \right) = X(x)T(t)$. After some tedious algebra, we get
      \begin{align*}
        \frac{1}{T}\left( \diff{T}{t} \right) &= \frac{2}{X}\left( \diff{^2X}{x^2} \right)\\
                                              &= \begin{cases}
                                                \lambda^2\\
                                                0\\
                                                -\lambda^2
                                              \end{cases}.
      \end{align*}
      In the case with $\lambda^2$, we get $u = e^{\lambda^2 t}\left( Ae^{\lambda/\sqrt{2} x} + Be^{-\lambda/\sqrt{2} x} \right)$, which does not satisfy the boundary conditions.\newline

      Similarly, in the case with $0$, we get $u = Ax + B$, which only satisfies the boundary conditions when $u = 0$, and does not satisfy the initial conditions.\newline

      Therefore, taking the case of $-\lambda^2$, we have
      \begin{align*}
        X &= A\sin\left( \frac{\lambda}{\sqrt{2}} x \right) + B\cos\left( \frac{\lambda}{\sqrt{2}} x \right)\\
        T &= Ce^{-\lambda^2 t}.
      \end{align*}
      Plugging in our boundary conditions, we get that $\lambda\in \frac{1}{\sqrt{2}}\Z^{+}$ and $B = 0$, yielding
      \begin{align*}
        u\left( x,t \right) &= \sum_{n=1}^{\infty}C_ne^{-n^2/2 t}\sin\left( \frac{n}{2}x \right).
      \end{align*}
      Finally, plugging in our initial condition, we get
      \begin{align*}
        \sin\left( 2x \right) &= \sum_{n=1}^{\infty}C_ne^{-n^2/2 t}\sin\left( \frac{n}{2}x \right),
      \end{align*}
      or that 
      \begin{align*}
        u\left( x,t \right) &= e^{-8t}\sin\left( 2x \right).
      \end{align*}
    \item Using separation of variables to solve the heat equation, we take $u\left( x,t \right) = X(x)T(t)$. After some algebra, we get
      \begin{align*}
        \frac{1}{T}\left( \diff{T}{t} \right) &= \frac{1}{X} \left( \diff{^2X}{x^2} \right)\\
                                              &= \begin{cases}
                                                \lambda^2 \\
                                                0\\
                                                -\lambda^2
                                              \end{cases}.
      \end{align*}
      Using a similar method as with (ii) to narrow down our possibilities, we get that $\lambda\in \pi\Z^+$, and
      \begin{align*}
        X_n &= A_n\cos\left( \pi n x \right) + B_n\sin\left( \pi n x \right)\\
        T_n &= Ce^{-\pi^2 n^2 t}.
      \end{align*}
      Using the Neumann boundary condition, we get that $B_n = 0$ for all $n$, meaning
      \begin{align*}
        u\left( x,t \right) &= \sum_{n=0}^{\infty}C_ne^{-\pi^2n^2 t}\cos\left( \pi n x \right).
      \end{align*}
      Plugging in our initial condition, we get that $C_0 = 8$, $C_3 = -4$, and everything else is $0$, so
      \begin{align*}
        u\left( x,t \right) &= 8 - 4e^{-9\pi^2 t}\cos\left( 3\pi x \right).
      \end{align*}
    \item Using separation of variables on the wave equation, we write $u\left( x,t \right) = X(x)T(t)$, and get
      \begin{align*}
        \frac{1}{T} \left( \diff{^2T}{t^2} \right) &= \frac{1}{X}\left( \diff{^2X}{x^2} \right)\\
                                                   &= \begin{cases}
                                                     \lambda^2\\
                                                     0\\
                                                     -\lambda^2
                                                   \end{cases}.
      \end{align*}
      As in (ii) and (iii), both $\lambda^2$ and $0$ yield trivial solutions when we plug in the boundary conditions $u\left( 0,t \right) = u\left( 2,t \right) = 0$, so we are left with the form
      \begin{align*}
        X(x) &= A\cos\left( \lambda x \right) + B\sin\left( \lambda x \right)\\
        T(t) &= C\cos\left( \lambda t \right) + D\sin\left( \lambda t \right).
      \end{align*}
      By plugging in the boundary condition $u\left( 0,t \right) = 0$, we get that $A = 0$, and by plugging in $u\left( 2,t \right) = 0$, we get that $B\sin\left( 2\lambda \right) = 0$, so $\lambda = \frac{\pi}{2}n$. Our solution is of the form
      \begin{align*}
        u\left( x,t \right) &= \sum_{n=1}^{\infty}\left( C_n\cos\left( \frac{\pi}{2}nt \right) + D_n\sin\left( \frac{\pi}{2}n t \right) \right)B_n\sin\left( \frac{\pi}{2}n x \right).
        \intertext{Rewriting with different constants, we get}
                            &= \sum_{n=1}^{\infty} A_n\sin\left( \frac{\pi}{2}nx \right)\cos\left( \frac{\pi}{2}nt \right) + B_n\sin\left( \frac{\pi}{2}nx \right)\sin\left( \frac{\pi}{2}nt \right).
      \end{align*}
      Now, plugging in our first initial condition, with $u\left( x,0 \right) = \sin\left( 2\pi x \right)$, we get $A_4 = 1$ and $A_{n\neq 4} = 0$. This gives the narrowed expression
      \begin{align*}
        u\left( x,t \right) &= \sin\left( 2\pi x \right)\cos\left( 2\pi t \right) + \sum_{n=1}^{\infty}B_n\sin\left( \frac{\pi}{2}nx \right)\sin\left( \frac{\pi}{2}nt \right).
      \end{align*}
      Using the second initial condition of $ \pd{u}{t}\bigr\vert_{(x,0)} = 0 $, we get $B_n = \frac{1}{3\pi}$, so our particular solution is
      \begin{align*}
        u\left( x,t \right) &= \sin\left( 2\pi x \right)\cos\left( 2\pi t \right) + \frac{1}{3\pi}\sin\left( 3\pi x \right)\sin\left( 3\pi t \right).
      \end{align*}
      
  \end{enumerate}
\end{solution}

\end{document}
