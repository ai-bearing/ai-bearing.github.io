\documentclass[12pt]{mypackage}

% sans serif font:
%\usepackage{cmbright,sfmath,bbold}
%\renewcommand{\mathcal}{\mathtt}

%Euler:
\usepackage{newpxtext,eulerpx,eucal,eufrak}
\renewcommand*{\mathbb}[1]{\varmathbb{#1}}
\renewcommand*{\hbar}{\hslash}

\usepackage{homework}

\pagestyle{fancy} %better headers
\fancyhf{}
\rhead{Avinash Iyer}
\lhead{Partial Differential Equations: Homework 2}

\setcounter{secnumdepth}{0}

\begin{document}
\renewcommand{\arraystretch}{1.5}
\RaggedRight
\section{Textbook Questions}%
\begin{solution}[4.4, Problem 2]
  By the method of inspection, we get the general solution of
  \begin{align*}
    y(x) &= c_1\cos\left( \frac{3}{2}x \right) + c_2\sin\left( \frac{3}{2}x \right) + \frac{5}{3}.
  \end{align*}
\end{solution}
\begin{solution}[4.4, Problem 4]
  By the method of inspection (basically just undetermined coefficients without actually going through the full steps), we get the general solution of
  \begin{align*}
    y(x) &= c_1e^{3x} + c_2 e^{-2x} - \frac{1}{3}x + \frac{1}{8}.
  \end{align*}
\end{solution}
\begin{solution}[4.4, Problem 12]
  Using the power of inspection, we have the homogeneous solution of $k_1e^{4x} + k_2e^{-4x}$. For the particular solution, we guess 
  \begin{align*}
    y_p(x) &= \left( ax + b \right)e^{4x},
  \end{align*}
  and use the method of computation through Sage's \texttt{desolve} command to obtain the general solution of
  \begin{align*}
    y(x) &= k_1e^{4x} + k_2e^{-4x} + \frac{1}{4}xe^{4x}
  \end{align*}
  This can be independently verified by using undetermined coefficients on $y_p(x) = \left( ax + b \right)e^{4x}$, giving
  \begin{align*}
    y''-16y &= 8ae^{4x}\\
            &= 2e^{4x}.
  \end{align*}
\end{solution}
\begin{solution}[4.6, Problem 2]
  We guess $y_p(x) = u_1y_1(x) + u_2y_2(x)$, and use the variation of parameters derivation to find
  \begin{align*}
    \begin{pmatrix}u_1' \\ u_2'\end{pmatrix} &= \begin{pmatrix} - \frac{\sin^2 x}{\cos(x)\cos(2x)} \\ \frac{\sin(x)}{\cos\left( 2x \right)}\end{pmatrix}.
  \end{align*}
  After many tedious, error-prone calculations that are better performed in Mathematica, we get the particular solution of
  \begin{align*}
    y_p(x) &= -\frac{1}{2}\cos\left( x \right)\ln\left( \sin\left( x \right)  + 1\right) + \frac{1}{2}\cos(x)\ln\left( \sin(x) - 1 \right).
  \end{align*}
  This adds to the homogeneous solution of $k_1\cos(x) + k_2\sin(x)$.
\end{solution}
\begin{solution}[4.6, Problem 8]
  We find homogeneous solutions of $y_1 = e^{x}$ and $y_2 = e^x$. We use variation of parameters to find
  \begin{align*}
    y_p(x) &= \frac{1}{3}\sinh(2x),
  \end{align*}
  giving the general solution of
  \begin{align*}
    y(x) &= k_1e^x + k_2e^{-x} + \frac{1}{3}\sinh(2x).
  \end{align*}
\end{solution}
\begin{solution}[4.6, Problem 14]
  The solutions to the homogeneous equation $y'' - 2y' + y = 0$ are $e^x$ and $xe^x$. Setting up the variation of parameters equation, we get
  \begin{align*}
    \begin{pmatrix}v_1'\\v_2'\end{pmatrix} &= \frac{1}{e^x\left( xe^x + e^x \right) - \left( xe^x \right)e^x}\begin{pmatrix}-xe^x\left( e^{x}\arctan(x) \right) \\ e^{2x}\arctan(x)\end{pmatrix}\\
                                           &= \frac{1}{e^{2x}} \begin{pmatrix}-xe^{2x}\arctan(x) \\ e^{2x}\arctan(x)\end{pmatrix}\\
                                           &= \begin{pmatrix}-x\arctan(x) \\ \arctan(x)\end{pmatrix}.
  \end{align*}
  Using the power of scratch work on the chalkboard, we get
  \begin{align*}
    v_1 &= x\arctan(x) - \frac{1}{2}\ln\left( 1 + x^2 \right)\\
    v_2 &= -\frac{1}{2}x^2\arctan(x) + \frac{1}{2}x - \frac{1}{2}\arctan(x).
  \end{align*}
  Therefore, we get the general solution of
  \begin{align*}
    y(x) &= c_1e^x + c_2xe^x + e^x\left( x\arctan(x) - \frac{1}{2}\ln\left( 1 + x^2 \right) \right) + xe^x\left( -\frac{1}{2}x^2\arctan(x) + \frac{1}{2}x - \frac{1}{2}\arctan(x) \right).
  \end{align*}
\end{solution}
\begin{solution}[4.6, Problem 31]
  We will show that $y_p(x)$ is a solution to the equation
  \begin{align*}
    \diff{^2y}{x^2} + p\diff{y}{x} + qy &= f.
  \end{align*}
  Taking
  \begin{align*}
    y_p(x) &= \int_{0}^{x} \frac{y_1(t)y_2(x) - y_1(x)y_2(t)}{W(t)}f(t)\:dt\\
           &= y_2(x)\int_{0}^{x} \frac{y_1(t)f(t)}{W(t)}\:dt - y_1(x)\int_{0}^{x} \frac{y_2(t)f(t)}{W(t)}\:dt,
  \end{align*}
  and using the power of scratch work on the chalkboard, we get 
  \begin{align*}
    \diff{y_p}{x} &= \diff{y_2}{x}\int_{0}^{x} \frac{y_1(t)f(t)}{W(t)}\:dt - \diff{y_1}{x} \int_{0}^{x} \frac{y_2(t)f(t)}{W(t)}\:dt\\
    \diff{^2y_p}{x^2} &= \diff{^2y_2}{x^2}\int_{0}^{x} \frac{y_1(t)f(t)}{W(t)}\:dt - \diff{^2y_1}{x^2}\int_{0}^{x} \frac{y_2(t)f(t)}{W(t)}\:dx + f(x).
  \end{align*}
  Plugging these into the equation
  \begin{align*}
    \diff{^2y}{x^2} + p\diff{y}{x} + qy &= f(x)
  \end{align*}
  gives our desired solution.
\end{solution}
\begin{solution}[4.7, Problem 4]
  We multiply both sides of the equation by $x$ to get
  \begin{align*}
    x^2 \diff{^2y}{x^2} - 3x\diff{y}{x} &= 0.
  \end{align*}
  The auxiliary equation is, then,
  \begin{align*}
    r\left( r-1 \right) - 3r &= 0,
  \end{align*}
  giving $r = 0,r=4$. Thus, our solution is
  \begin{align*}
    y(x) &= c_1 x^{4} + c_2.
  \end{align*}
\end{solution}
\begin{solution}[4.7, Problem 10]
  Substituting into the auxiliary equation, we have
  \begin{align*}
    4r\left( r-1 \right) + 4r - 1 &= 0\\
    4r^2 - 1 &= 0,
  \end{align*}
  giving $r = \pm \frac{1}{2}$. Thus, our solution is
  \begin{align*}
    y(x) &= c_1x^{1/2} + c_2x^{-1/2}.
  \end{align*}
\end{solution}
\begin{solution}[4.7, Problem 12]
  Substituting into the auxiliary equation, we have
  \begin{align*}
    r\left( r-1 \right) + 8r + 6 &= 0\\
    r^2 + 7r + 6 &= 0,
  \end{align*}
  giving $r = -6,-1$. Thus, our solution is
  \begin{align*}
    y(x) &= c_1r^{-6} + c_2r^{-1}.
  \end{align*}
\end{solution}
\begin{solution}[4.7, Problem 14]
  Substituting into the auxiliary equation, we have
  \begin{align*}
    r\left( r-1 \right) - 7r + 41 &= 0,
  \end{align*}
  giving
  \begin{align*}
    r^2 - 8r + 41 &= 0.
  \end{align*}
  Completing the square and solving, we get
  \begin{align*}
    r &= 4\pm 5i,
  \end{align*}
  so our solution is
  \begin{align*}
    y(x) &= c_1x^4\cos\left( 5\ln x \right) + c_2x^4\sin\left( 5\ln x \right).
  \end{align*}
\end{solution}
\begin{solution}[4.7, Problem 16]
  Substituting into the auxiliary equation, we have
  \begin{align*}
    r\left( r-1 \right)\left( r-2 \right) + r - 1 &= 0\\
    r^3 - 3^2 + 3r - 1 &= 0.
  \end{align*}
  This gives $r = 1$ with multiplicity $3$, so we have solutions of
  \begin{align*}
    y(x) &= c_1 x + c_2x\ln (x) + c_3x\left( \ln(x) \right)^2.
  \end{align*}
\end{solution}
\begin{solution}[4.7, Problem 18]
  Substituting into the auxiliary equation, we have
  \begin{align*}
    r\left( r-1 \right)\left( r-2 \right)\left( r-3 \right) + 6r\left( r-1 \right)\left( r-2 \right) + 9r\left( r-1 \right) + 3r + 1 &= 0.\\
    \left( x^2 + 1 \right)^2 &= 0.
  \end{align*}
  Thus, we have roots of $\pm i$ with multiplicity $2$, giving solutions of
  \begin{align*}
    y(x) &= c_1\cos\left( \ln(x) \right) + c_2\sin\left( \ln\left( x \right) \right) + \ln\left( x \right)\left( c_3\cos\left( \ln(x) \right) + c_4\sin\left( \ln(x) \right) \right).
  \end{align*}
\end{solution}
\begin{solution}[4.7, Problem 32]
  Using the substitution $x = e^t$, we have
  \begin{align*}
    x^2 \left( \frac{1}{x^2}\left( \diff{^2y}{t^2} - \diff{y}{t} \right) \right) - 9x\left( \frac{1}{x}\diff{y}{t} \right) + 25y &= 0\\
    \diff{^2y}{t^2} - 10\diff{y}{t} + 25y &= 0.
  \end{align*}
  Solving, we get
  \begin{align*}
    y(t) &= c_1e^{5t} + c_2te^{5t},
  \end{align*}
  so
  \begin{align*}
    y(x) &= c_1x^5 + c_2x^5\ln(x).
  \end{align*}
\end{solution}
\section{Extra Problems}%
\begin{solution}
  A second-order linear homogeneous ODE that has solutions of $y(x) = \tan(x)$ and $y(x) = 1$ is
  \begin{align*}
    \diff{^2y}{x^2} - 2\tan(x)\diff{y}{x} &= 0.
  \end{align*}
\end{solution}
\begin{solution}
  A second order linear inhomogeneous ODE with solutions $y(x) = \sin^3(x)$ and $y(x) = \cos^3(x)$ is
  \begin{align*}
    \diff{^2y}{x^2} + 9y &= 9.
  \end{align*}
  I could not find a homogeneous ODE with these solutions.
\end{solution}

\end{document}
