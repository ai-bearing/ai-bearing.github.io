\documentclass[10pt]{mypackage}

% sans serif font:
%\usepackage{cmbright,sfmath,bbold}
%\renewcommand{\mathcal}{\mathtt}

%Euler:
\usepackage{newpxtext,eulerpx,eucal,eufrak}
\renewcommand*{\mathbb}[1]{\varmathbb{#1}}
\renewcommand*{\hbar}{\hslash}

%kp fonts:
%\usepackage{kpfonts}
%\renewcommand{\mathbb}{\mathds}

\pagestyle{fancy} %better headers
\fancyhf{}
\rhead{Avinash Iyer}
\lhead{Partial Differential Equations: Class Notes}

\setcounter{secnumdepth}{0}

\begin{document}
\RaggedRight
\section{Introduction}%
Consider the equations
\begin{align*}
  y''(x) + y(x) &= e^x\tag*{(1)}\\
  y^{(17)}(x) + \sin\left(y(x)\right) &= \left(x^{x}\right)^{x}\tag*{(2)}
\end{align*}
Before we want to solve these equations, we need to understand what these equations \textit{are}. 
\begin{enumerate}[(1)]
  \item This is a second order, inhomogeneous, linear ordinary differential equation.
  \item This is a 17th order, inhomogeneous, nonlinear ordinary differential equation.
\end{enumerate}
Generally, when we have a nonlinear equation, we convert it (using the Jacobian) to the ``nearest'' corresponding linear equation using Taylor approximations. In this case, converting equation (2), we have
\begin{align*}
  y^{(17)}(x) + y(x) &= \left(x^{x}\right)^{x}.\tag*{(2')}
\end{align*}
Now, equation (2') is linear, so it is able to be solved. It may not be pretty,\footnote{Citation needed.} but it can be solved, using Laplace Transforms or other methods.
\section{Ordinary Differential Equations}%
Returning to our equation (1), 
\begin{align*}
  y''(x) + y(x) &= e^x,\tag*{(1)}
\end{align*}
there is one more fact that we can see --- this is an equation with constant coefficients. The most general form of a $n$th order linear ordinary differential equation is of the form
\begin{align*}
  a_n(x)y^{(n)}(x) + a_{n-1}(x)y^{(n-1)}(x) + \cdots + a_1(x)y'(x) + a_0(x)y(x) = g(x).\tag*{(\textdagger)}
\end{align*}
Specifically, we also require $a_k(x)\in C(I)$, where $I$ is some interval (specifics will be detailed later).
\begin{theorem}[Existence and Uniqueness Theorem]
  Any ordinary differential equation of the form (\textdagger) has unique solutions in $I$.\newline

  There are $n$ linearly independent solutions for $g(x) = 0$.
\end{theorem}
The corresponding homogeneous equation for (1) is
\begin{align*}
  y''(x) + y(x) &= 0.\tag*{(1')}
\end{align*}
The equations (1) and (1') are related by the linearity principle. In particular, if $y_0(x)$ is a solution to (1'), then we can add $\alpha y_0(x)$ to any solution $y_p(x)$ of (1), then we have all the solutions for (1). In particular, the solutions to (1') are
\begin{align*}
  y_1(x) &= \sin(x)\\
  y_2(x) &= \cos(x).
\end{align*}
To evaluate that these solutions are linearly independent, we consider the differential operator $L$ from (\textdagger) defined by
\begin{align*}
  L\left[y\right] &= \sum_{k=0}^{n}a_k(x)y^{(k)}(x).
\end{align*}
We rewrite (\textdagger) as
\begin{align*}
  L\left[y\right] &= g(x).
\end{align*}
The operator $L$ is linear, so $L$ has the following properties:
\begin{itemize}
  \item $L\left[y_1 + y_2\right]$;
  \item $L\left[cy\right] = cL\left[y\right]$.
\end{itemize}
Now, in (1) and (1'), if we set $L\left[y\right] = y''(x) + y(x)$, then evaluating our solutions $y_1$ and $y_2$ to (1'), we get
\begin{align*}
  L\left[c_1y_1 + c_2y_2\right] &= c_1L\left[y_1\right] + c_2L\left[y_2\right]\\
                                &= 0.
\end{align*}
Now, we get
\begin{align*}
  y_0(x) &= c_1\sin(x) + c_2\sin(x)
\end{align*}
as our general solution to (1'). By the linearity principle, all we need is one solution to $L\left[y\right] = e^x$ to find all solutions to (1).
\end{document}
