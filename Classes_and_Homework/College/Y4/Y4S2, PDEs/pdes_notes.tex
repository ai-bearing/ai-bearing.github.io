\documentclass[10pt]{mypackage}

% sans serif font:
%\usepackage{cmbright,sfmath,bbold}
%\renewcommand{\mathcal}{\mathtt}

%Euler:
\usepackage{newpxtext,eulerpx,eucal,eufrak}
\renewcommand*{\mathbb}[1]{\varmathbb{#1}}
\renewcommand*{\hbar}{\hslash}

%kp fonts:
%\usepackage{kpfonts}
%\renewcommand{\mathbb}{\mathds}

\pagestyle{fancy} %better headers
\fancyhf{}
\rhead{Avinash Iyer}
\lhead{Partial Differential Equations: Class Notes}

\setcounter{secnumdepth}{0}

\begin{document}
\renewcommand{\arraystretch}{1.5}
\RaggedRight
\section{Introduction}%
Consider the equations
\begin{align*}
  \frac{d^2y}{dx^2} + y(x) &= e^x\tag*{(1)}\\
  \diff{^{17}y}{x^{17}}(x) + \sin\left(y(x)\right) &= \left(x^{x}\right)^{x}\tag*{(2)}
\end{align*}
Before we want to solve these equations, we need to understand what these equations \textit{are}. 
\begin{enumerate}[(1)]
  \item This is a second order, inhomogeneous, linear ordinary differential equation.
  \item This is a 17th order, inhomogeneous, nonlinear ordinary differential equation.
\end{enumerate}
Generally, when we have a nonlinear equation, we convert it (using the Jacobian) to the ``nearest'' corresponding linear equation using Taylor approximations. In this case, converting equation (2), we have
\begin{align*}
  \diff{^{17}y}{x^{1y}}(x) + y(x) &= \left(x^{x}\right)^{x}.\tag*{(2')}
\end{align*}
Now, equation (2') is linear, so it is able to be solved. It may not be pretty,\footnote{Citation needed.} but it can be solved, using Laplace Transforms or other methods.
\section{Ordinary Differential Equations}%
Returning to our equation (1), 
\begin{align*}
  \frac{d^2y}{dx^2} + y(x) &= e^x,\tag*{(1)}
\end{align*}
there is one more fact that we can see --- this is an equation with constant coefficients. The most general form of a $n$th order linear ordinary differential equation is of the form
\begin{align*}
  a_n(x)\diff{^ny}{x^n}(x) + a_{n-1}(x)\diff{^{n-1}y}{x^{n-1}}(x) + \cdots + a_1(x)\frac{dy}{dx} + a_0(x)y(x) = g(x).\label{eq:general_linear_ode}\tag{\textdagger}
\end{align*}
Specifically, we also require $a_k(x)\in C(I)$, where $I$ is some interval (specifics will be detailed later).
\begin{theorem}[Existence and Uniqueness Theorem]
  Any ordinary differential equation of the form (\textdagger) has unique solutions in the interval $I$.\newline

  There are $n$ linearly independent solutions for $g(x) = 0$.
\end{theorem}
The corresponding homogeneous equation for (1) is
\begin{align*}
  \frac{d^2y}{dx^2} + y(x) &= 0.\tag*{(1')}
\end{align*}
The equations (1) and (1') are related by the linearity principle. In particular, if $y_0(x)$ is a solution to (1'), then we can add $\alpha y_0(x)$ to any solution $y_p(x)$ of (1), then we have all the solutions for (1). In particular, the solutions to (1') are
\begin{align*}
  y_1(x) &= \sin(x)\\
  y_2(x) &= \cos(x).
\end{align*}
To evaluate that these solutions are linearly independent, we consider the differential operator $L$ from (\textdagger) defined by
\begin{align*}
  L\left[y\right] &= \sum_{k=0}^{n}a_k(x)\diff{^{k}y}{x^{k}}.
\end{align*}
We rewrite (\textdagger) as
\begin{align*}
  L\left[y\right] &= g(x).
\end{align*}
The operator $L$ is linear, so $L$ has the following properties:
\begin{itemize}
  \item $L\left[y_1 + y_2\right]$;
  \item $L\left[cy\right] = cL\left[y\right]$.
\end{itemize}
Now, in (1) and (1'), if we set $L\left[y\right] = \frac{d^2y}{dx^2} + y(x)$, then evaluating our solutions $y_1$ and $y_2$ to (1'), we get
\begin{align*}
  L\left[c_1y_1 + c_2y_2\right] &= c_1L\left[y_1\right] + c_2L\left[y_2\right]\\
                                &= 0.
\end{align*}
Now, we get
\begin{align*}
  y_0(x) &= c_1\sin(x) + c_2\sin(x)
\end{align*}
as our general solution to (1'). By the linearity principle, all we need is one solution to $L\left[y\right] = e^x$ to find all solutions to (1).\newline

Evaluating \eqref{eq:general_linear_ode} in the most general form, we have the general solution
\begin{align*}
  y(x) &= \underbrace{c_1y_1(x) + c_2y_2(x) + \cdots + c_ny_n(x)}_{\text{homogeneous solution}} + y_p(x),
\end{align*}
where $y_p(x)$ is the particular solution. In other words, our general solution is
\begin{align*}
  y(x) &= \Span\left(y_1(x),y_2(x),\dots,y_n(x)\right) + y_p(x).
\end{align*}
For this to work, we need the set $\set{y_1,\dots,y_n}$ to be linearly independent. To do this, we evaluate the Wronskian:
{
  \begin{align*}
    W(x) &= \det \begin{pmatrix}y_1(x) & y_2(x) & \cdots & y_n(x) \\ \diff{y_1}{x}& \diff{y_2}{x} & \cdots & \diff{y_n}{x} \\ \vdots & \vdots & \ddots & \vdots \\ \diff{^{n-1}y_1}{x^{n-1}} & \diff{^{n-1}y_2}{x^{n-1}} & \cdots & \diff{^{n-1}y_n}{x^{n-1}}\end{pmatrix}.
\end{align*}
}
Specifically, the set $\set{y_1,\dots,y_n}$ is linearly independent if $W(x)\neq 0$ for all $x\in I$.
\begin{example}
  Consider the equation
  \begin{align*}
    \frac{d^2y}{dx^2} - y(x) &= e^{x}\tag{1}
  \end{align*}
  We want to find the general solution to this constant coefficient equation.\newline

  We start by finding two linearly independent homogeneous solutions to the equation, take their span, then add a particular solution.\newline

  The characteristic equation of the homogeneous equation for (1) is
  \begin{align*}
    r^2 - 1 &= 0
  \end{align*}
  We get $r=\pm 1$, which by the definition of the characteristic equation yields $y_1(x) = e^{x}$ and $y_2(x) = e^{-x}$. To verify that this this solution set is linearly independent
  {\renewcommand{\arraystretch}{1.25}
    \begin{align*}
      W(x) &= \det \begin{pmatrix}e^{x} & e^{-x} \\ e^{x} & -e^{-x}\end{pmatrix}\\
           &= -2\\
           &\neq 0.
  \end{align*}
  }
  Thus, our solutions are linearly independent. We get the general form of
  \begin{align*}
    y(x) &= c_1e^{x} + c_2e^{-x} + y_p(x).
  \end{align*}
  Now, we only have to find a particular solution. This is, unfortunately, the hard part.\newline

  We begin by guessing. But, in a way that doesn't suck. Specifically, we let $y_p(x) = Axe^{x}$. Evaluating, we get
  \begin{align*}
    \diff{y_p}{x} &= A\left(x+1\right)e^{x}\\
    \diff{^2y_p}{x^2} &= A\left(x+2\right)e^{x}\\
     \diff{^2y_p}{x^2}- y_p(x) &= A\left(x+2\right)e^{x} - Axe^{x}\\
                        &= 2Ae^{x},
  \end{align*}
  so $2A = 1$, and $A = \frac{1}{2}$. Thus, we have the end result of
  \begin{align*}
    y(x) &= c_1e^{x} + c_2e^{x} + \frac{1}{2}xe^{x}.
  \end{align*}
  Evaluating in Mathematica, we take
  \begin{lstlisting}[style=mathematicastyle]
    DSolve[y''[x] - y[x] == Exp[x], y[x], x]
  \end{lstlisting}
  and we get
  \begin{align*}
    y(x) &= c_1e^{x} + c_2e^{-x} + \frac{1}{4}\left(2x-1\right)e^{x},
  \end{align*}
  corroborating our solution.\footnote{Only slightly different, but they're the same solution.}
\end{example}
\begin{example}
  Consider the equation
  \begin{align*}
    \diff{^3y}{x^3} - y(x) &= 0.
  \end{align*}
  The particular solution to this equation is $y(x) = 0$. The characteristic equation for this equation is
  \begin{align*}
    r^3 - 1 &= 0.
  \end{align*}
  Factoring, we get
  \begin{align*}
    \left(r-1\right)\left(r^2 + r + 1\right) &=0\\
    \left(r-1\right)\left(r-\zeta_3\right)\left(r-\zeta_3^2\right) &= 0.
  \end{align*}
  Thus, we get
  \begin{align*}
    r &= \set{1, e^{\frac{2\pi i}{3}}, e^{\frac{4\pi i}{3}}}.
  \end{align*}
  Thus, our solutions are of the form
  \begin{align*}
    y(x) &= c_1e^{x} + c_2e^{-\frac{1}{2}x}\cos\left(\frac{\sqrt{3}}{2}x\right) + c_3e^{-\frac{1}{2}x}\sin\left(\frac{\sqrt{3}}{2}x\right).
  \end{align*}
\end{example}
Recall that the most general second order constant-coefficient linear differential equation is 
\begin{align*}
  y'' + ay' + by &= 0,
\end{align*}
with characteristic equation 
\begin{align*}
  r^2 + ar + b &=0.
\end{align*}
The solutions to the characteristic equation are
\begin{align*}
  r &= -\frac{a}{2} \pm \frac{\sqrt{a^2 - 4b}}{2}.
\end{align*}
There are a few cases:
\begin{enumerate}[(1)]
  \item $r_1\neq r_2$ with $r_1,r_2\in \R$;
  \item $r_1 = r_2$ with $r_1,r_2\in \R$;
  \item $r_1 = c + id$, $r_2 = c - id$, where $c,d\in \R$.
\end{enumerate}
The solutions are $y_1 = c_1e^{r_1 x}$ and $y_2 = c_2e^{r_2 x}$.
\begin{example}[Solving Second-Order Equations]\hfill
  \begin{enumerate}[(1)]
    \item Let
      \begin{align*}
        y'' - 3y' + 2y &= 0.
      \end{align*}
      The characteristic equation is $r^2 - 3r + 2 = 0$, whose solutions are $r=1,r=2$. The general solution is, thus,
      \begin{align*}
        y(x) &= c_1 e^x + c_2e^{2x}\tag{\textdagger}
      \end{align*}
      The Wronskian is
      \begin{align*}
        W(x) &= \det \begin{pmatrix}e^x & e^{2x} \\ e^x & 2e^{2x}\end{pmatrix}\\
             &= 2e^{3x} - e^{3x}\\
             &= e^{3x}\\
             &\neq 0.
      \end{align*}
      Thus, the solution is indeed (\textdagger).
    \item Let
      \begin{align*}
        y'' + 6y' + 9y &=0.
      \end{align*}
      The characteristic equation is $r^2 + 6r + 9 =0$, with solution $r = -3,-3$. Currently, we only have the solution $y_1(x) = c_1e^{-3x}$.\newline

      Note that in an $n$th order linear ordinary differential equation, we always have $n$ linearly independent solutions. Let's guess. Consider the equation $y_2(x) = c_2xe^{-3x}$.\newline

      We can see that $y_2(x)$ is also a solution to this equation,\footnote{Exercise left for the reader.} but we need to verify linear independence. Taking the Wronskian, we get
      \begin{align*}
        W(x) &= \det \begin{pmatrix}e^{-3x} & xe^{-3x} \\ -3e^{-3x} & -3xe^{-3x} + e^{-3x}\end{pmatrix}\\
             &= e^{-6x} \begin{pmatrix}1 & x \\ -3 & -3x + 1\end{pmatrix}\\
             &= e^{-6x}\left(-3x + 1 + 3x\right)\\
             &= e^{-6x}\\
             &\neq 0.
      \end{align*}
      Thus, we have two linearly independent solutions, with the general solution of
      \begin{align*}
        y(x) &= c_1e^{-3x} + c_2xe^{-3x}.
      \end{align*}
    \item Let
      \begin{align*}
        y'' + 4y' + 5 &= 0.
      \end{align*}
      The characteristic equation is $r^2 + 4r + 5 = 0$, with solutions of $r = -2 \pm i$. We then have the solutions
      \begin{align*}
        y_1(x) &= e^{\left(-2 + i\right)x}\\
        y_2(x) &= e^{\left(-2-i\right)x}.
      \end{align*}
      Unfortunately, we cannot just let these equations stand on their own, because we want \textit{real} solutions. Let's use Euler's theorem, $e^{ix} = \cos x + i\sin x$. Then, we get
      \begin{align*}
        y(x) &= c_1e^{\left(-2+i\right)x} + c_2e^{\left(-2-i\right)x}\\
             &= e^{-2x}\left(c_1e^{ix} + c_2e^{-ix}\right).
      \end{align*}
      Let $f(x) = c_1e^{ix} + c_2e^{-ix}$. Using the even/odd decomposition, we get
      \begin{align*}
        f(x) &= \frac{1}{2}\left(f(x) + f\left(-x\right)\right) + \frac{1}{2}\left(f(x) - f\left(-x\right)\right)\\
             &= \left(c_1 + c_2\right)\cos\left(x\right) + i\left(c_1 - c_2\right)\sin\left(x\right).
      \end{align*}
      We ``real''-ize our solution by just dropping the value of $i$ in $f(x)$. Thus, we get the full general solution
      \begin{align*}
        y(x) &= e^{-2x}\left(d_1\cos\left(x\right) + d_2\sin\left(x\right)\right).
      \end{align*}
    \item If we have the equation
      \begin{align*}
        y^{(4)} - 25y'',
      \end{align*}
      then using a similar process, we get the solution
      \begin{align*}
        y(x) = c_1 + c_2 x + c_3e^{5x} + c_4e^{-5x}.
      \end{align*}
    \item Considering the equation
      \begin{align*}
        y^{(5)} + 4y''' + 4y' = 0,
      \end{align*}
      we take the characteristic equation $r^{5}+ 4r^3 + 4r = 0$. Factoring, we get solutions of $r=0,r=\pm i\sqrt{2}$. Thus, we get the solution of
      \begin{align*}
        y(x) = c_1 + c_2\cos\left(\sqrt{2}x\right) + c_3\sin\left(\sqrt{2}x\right) + c_4x\cos\left(\sqrt{2}x\right) + c_5x\sin\left(\sqrt{2}x\right).
      \end{align*}
  \end{enumerate}
\end{example}
\subsection{Reducing our Orders}%
Let
\begin{align*}
  \frac{d^2y}{dx^2} + p(x)\frac{dy}{dx} + q(x)y(x) &= 0.
\end{align*}
Suppose we know $y_1(x)$. Can we find $y_2(x)$? The answer is yes. We presume
\begin{align*}
  y_2(x) &= v(x)y_1(x).
\end{align*}
Now, we have
\begin{align*}
  y_2 &= vy_1\\
  y_2' &= v'y_1 + vy_1'\\
  y_2'' &= v''y_1 + 2v'y_1' + vy_1'',
\end{align*}
and inserting into the equation, we get
\begin{align*}
  0 &= v''y_1 + 2v'y_1' + vy_1'' + pv'y_1 + pvy_1' + qvy_1\\
    &= v''y_1 + 2v'y_1' + pv'y_1 + v\underbrace{\left(y_1'' + py_1' + qy_1\right)}_{=0}\\
    &= v''y_1 + 2v'y_1' + pv'y_1
\end{align*}
Now, we have
\begin{align*}
  \frac{v''}{v'} &= -2\frac{y_1'}{y_1} - p.\tag{\textasteriskcentered}
\end{align*}
Integrating, we get
\begin{align*}
  \ln\left(v'\right) &= -2\ln\left(y_1\right) - \int_{}^{} p(x)\:dx.
\end{align*}
Taking powers, we get
\begin{align*}
  v' &= e^{-2\ln\left(y_1\right) - \int_{}^{} p(x)\:dx}\\
     &= y_1^{-2} e^{-\int_{}^{} p(x)\:dx}\\
     &= \frac{e^{-\int_{}^{} p(x)\:dx}}{y_1(x)^2}\\
  v &= \int_{}^{} \frac{e^{-\int_{}^{} p(x)\:dx}}{y_1(x)^2}\:dx
\end{align*}
\begin{example}
  Consider the equation
  \begin{align*}
    \cos^2\left(x\right)\frac{d^2y}{dx^2} - \sin(x)\cos(x)y' - y(x) = 0.
  \end{align*}
  Putting our equation into standard form, we may be able to find another solution.
  \begin{align*}
    y'' - \tan(x)y' - \sec^2\left(x\right)y &= 0.
  \end{align*}
  Guessing $y(x) = \tan(x)$, we get $y' = \sec^2\left(x\right)$ and $y'' = 2\sec^2\left(x\right)\tan(x)$. This is also another solution, $y_2(x) = \tan(x)$.\newline

  We don't want to guess anymore. Let $y_2(x) = v(x)y_1(x)$. We get
  \begin{align*}
    v(x) &= \int_{}^{} \frac{e^{-\int_{}^{} p(x)\:dx}}{y_1^2\left(x\right)}\:dx.
  \end{align*}
  We have $-p(x) = \tan(x)$, so $-\int_{}^{} p(x)\:dx = \ln\left(\sec(x)\right)$. Thus, $e^{-\int_{}^{} p(x)\:dx} = \sec(x)$. Thus, we get
  \begin{align*}
    v(x) &= \int_{}^{} \frac{\sec(x)}{\tan^2\left(x\right)}\:dx\\
         &= \int_{}^{} \frac{\cos(x)}{\sin^2\left(x\right)}\:dx\\
         &= \int_{}^{} \frac{1}{u^2}\:du\tag*{$u = \sin(x)$}\\
         &= -\frac{1}{u}\\
         &= -\csc(x).
  \end{align*}
  Thus, we have $y_2(x) = -\csc(x)\tan(x) = -\sec(x)$. 
\end{example}
\begin{example}
  Consider the equation
  \begin{align*}
    x^2\left(\ln(x)-1\right)\frac{d^2y}{dx^2} -x\frac{dy}{dx} + \frac{dy}{dx} = 0.
  \end{align*}
  We can use the power of inspection to find one solution, $y_1(x) = x$. Dividing out, we have
  \begin{align*}
    y'' - \frac{1}{x\left(\ln(x) - 1\right)} y' + \frac{1}{x^2\left(\ln(x) - 1\right)}y &= 0.
  \end{align*}
  Using the reduction of order, we guess $y_2(x) = v(x)y_1(x)$, and have
  \begin{align*}
    v(x) &= \int_{}^{} \frac{e^{-\int_{}^{} p(x)\:dx}}{y_1^2}\:dx.
  \end{align*}
  Noting that $-p(x) = \frac{1}{x\left(\ln(x)-1\right)}$, we have $\int_{}^{} \frac{1}{x\left(\ln(x)-1\right)}\:dx = \ln\left(\ln(x)-1\right)$.\newline

  Now, we have
  \begin{align*}
    v(x) &= \int_{}^{} \frac{\ln(x)-1}{x^2}\:dx\\
         &= \frac{1-\ln(x)}{x} - \int_{}^{} -\frac{1}{x^2}\:dx\tag*{$u=\ln(x)-1,dv=x^{-2}$}\\
         &= \frac{-\ln(x)}{x} - \frac{1}{x}\\
         &= -\frac{\ln(x)}{x}.
  \end{align*}
  Thus, we get $y_2(x) = -\ln(x)$, and the general solution of $y(x) = c_1 x + c_2\ln(x)$.
\end{example}
\begin{example}[Cauchy--Euler Equation]
  A second-order Cauchy--Euler equation is of the form
  \begin{align*}
    ax^2\frac{d^2y}{dx^2} + bx\frac{dy}{dx} + cy(x) &= 0.
  \end{align*}
  More generally,
  \begin{align*}
    \sum_{k=0}^{n}c_kx^ky^{(k)}(x) &= 0.
  \end{align*}
  We guess $y(x) = x^r$. Then, $\frac{dy}{dx} = rx^{r-1}$ and $\frac{d^2y}{dx^2} = r(r-1)x^{r-2}$. This yields
  \begin{align*}
    a(r)(r-1)x^r + brx^{r} + cx^r &=x^r \left(a\left(r^2 - r\right) + br + c\right)\\
                                  &= 0.
  \end{align*}
\end{example}
\begin{example}[Solving a Cauchy--Euler Equation]
  Consider the equation
  \begin{align*}
    x^2y'' + xy' - y &= 0.
  \end{align*}
  Substituting the characteristic equation, we get
  \begin{align*}
    r^2 - 1 &= 0,
  \end{align*}
  so our general solution is $y(x) = c_1x + c_2/x$.
\end{example}
\begin{example}[Solving another Cauchy--Euler Equation]
  Consider the equation
  \begin{align*}
    x^2y'' - 3xy' + 4y &= 0.
  \end{align*}
  Substituting the characteristic equation, we get
  \begin{align*}
    r^2 - 4r + 4 &= 0,
  \end{align*}
  so our solutions are $x^2$ and $x^2$. This is not good enough, we need another solution.\newline

  Now, we place our equation into standard form.
  \begin{align*}
    y'' - \frac{3}{x}y' + \frac{4}{x^2}y' &= 0.
  \end{align*}
  Thus, we get $p(x) = -\frac{3}{x}$. Using reduction of order, we get $y_2(x) = v(x)y_1(x)$,
  \begin{align*}
    v(x) &= \int_{}^{} \frac{e^{-\int_{}^{} -3/x\:dx}}{x^4}\:dx\\
         &= \int_{}^{} \frac{e^{3\ln(x)}}{x^4}\:dx\\
         &= \int_{}^{} \frac{x^3}{x^4}\:dx\\
         &= \ln(x).
  \end{align*}
  Thus, we have the solution $y_2(x) = \ln(x)x^2$, and the general solution of $y(x) = c_1x^2 + c_2\ln(x)x^2$.
\end{example}
\begin{example}
  Consider the equation
  \begin{align*}
    x^2y'' + 3xy' + 5y &= 0.
  \end{align*}
  We get the characteristic equation of
  \begin{align*}
    0 &= r^2 - 4r + 5\\
    r &= 2\pm i.
  \end{align*}
  Now, we need to figure out what $x^{2\pm i}$ means.\newline

  To solve this part, we keep the positive exponent, so we only need to try to understand $y = x^{2 + i}$. Now, we get $y = x^2 x^i$. To evaluate $x^{i}$, we take $x = \left(e^{\ln x}\right)^{i} = e^{i\ln x}$. Using Euler's identity, we get
  \begin{align*}
    y &= x^2\left(\cos\left(\ln x\right) + i\sin\left(\ln x\right)\right).
  \end{align*}
  Since our solutions are real, get
  \begin{align*}
    y &= c_1x^2\cos\left(\ln x\right) + c_2x^2\sin\left(\ln x\right).
  \end{align*}
\end{example}
\begin{example}
  Consider the equation
  \begin{align*}
    x^4y^{(4)} - 2x^2y'' + y &= 2.
  \end{align*}
  We have the particular solution $y_p(x) = 2$. Substituting into our method for the Cauchy--Euler equation, we have
  \begin{align*}
    r\left(r-1\right)\left(r-2\right)\left(r-3\right) - 2r\left(r-1\right) + 1 &= 0.
  \end{align*}
  Factoring, we have
  \begin{align*}
    r\left(r-1\right)^2\left(r-4\right) + 1 &= 0.
  \end{align*}
  Unfortunately, to go forward from here we need Mathematica.\newline

  This has the solution set of of
  \begin{align*}
    r_1 &=\frac{3}{2}-\frac{1}{2} \sqrt{3+\frac{1}{3} \sqrt[3]{135-6 \sqrt{249}}+\frac{\sqrt[3]{45+2
   \sqrt{249}}}{3^{2/3}}}\\
      &-\frac{1}{2} \sqrt{6-\frac{1}{3} \sqrt[3]{135-6 \sqrt{249}}-\frac{\sqrt[3]{45+2
   \sqrt{249}}}{3^{2/3}}-\frac{8}{\sqrt{3+\frac{1}{3} \sqrt[3]{135-6 \sqrt{249}}+\frac{\sqrt[3]{45+2
   \sqrt{249}}}{3^{2/3}}}}}\\
          r_2 &=\frac{3}{2}-\frac{1}{2} \sqrt{3+\frac{1}{3} \sqrt[3]{135-6 \sqrt{249}}+\frac{\sqrt[3]{45+2
   \sqrt{249}}}{3^{2/3}}}\\
            &+\frac{1}{2} \sqrt{6-\frac{1}{3} \sqrt[3]{135-6 \sqrt{249}}-\frac{\sqrt[3]{45+2
   \sqrt{249}}}{3^{2/3}}-\frac{8}{\sqrt{3+\frac{1}{3} \sqrt[3]{135-6 \sqrt{249}}+\frac{\sqrt[3]{45+2
   \sqrt{249}}}{3^{2/3}}}}}\\
                r_3 &=\frac{3}{2}+\frac{1}{2} \sqrt{3+\frac{1}{3} \sqrt[3]{135-6 \sqrt{249}}+\frac{\sqrt[3]{45+2
   \sqrt{249}}}{3^{2/3}}}\\
                  &-\frac{1}{2} \sqrt{6-\frac{1}{3} \sqrt[3]{135-6 \sqrt{249}}-\frac{\sqrt[3]{45+2
   \sqrt{249}}}{3^{2/3}}+\frac{8}{\sqrt{3+\frac{1}{3} \sqrt[3]{135-6 \sqrt{249}}+\frac{\sqrt[3]{45+2
   \sqrt{249}}}{3^{2/3}}}}}\\
                      r_4&=\frac{3}{2}+\frac{1}{2} \sqrt{3+\frac{1}{3} \sqrt[3]{135-6 \sqrt{249}}+\frac{\sqrt[3]{45+2
   \sqrt{249}}}{3^{2/3}}}\\
                       &+\frac{1}{2} \sqrt{6-\frac{1}{3} \sqrt[3]{135-6 \sqrt{249}}-\frac{\sqrt[3]{45+2
   \sqrt{249}}}{3^{2/3}}+\frac{8}{\sqrt{3+\frac{1}{3} \sqrt[3]{135-6 \sqrt{249}}+\frac{\sqrt[3]{45+2
   \sqrt{249}}}{3^{2/3}}}}}
  \end{align*}
  
\end{example}
\subsection{Varying our Parameters}%
Given a set of $n$ linearly independent homogeneous solutions, we want to find a particular solution.\newline

To find this, we start with the general second-order inhomogeneous equation in standard form:
\begin{align*}
  \frac{d^2y}{dx^2} + p(x)\frac{dy}{dx} + q(x)y(x) = g(x).
\end{align*}
Given $y_1,y_2$, we find $y_p(x)$ by taking
\begin{align*}
  y_p &= v_1y_1 + v_2y_2.
\end{align*}
Finding the derivatives, we have
\begin{align*}
  y_p' &= v_1y_1' + v_1'y_1 + v_2y_2' + v_2'y_2\\
  y_p'' &= v_1y_1'' + 2v_1'y_1' + v_1''y_1 + v_2y_2'' + 2v_2'y_2' + v_2''y_2.
\end{align*}
Substituting, we have
\begin{align*}
  y_p'' &= v_1y_1'' + 2v_1'y_1' + v_1''y_1 + v_2y_2'' + 2v_2'y_2' + v_2''y_2\\
  py_p' &= pv_1y_1' + pv_1'y_1 + pv_2y_2' + pv_2'y_2\\
  qy_p &= qv_1y_1 + qv_2y_2\\
  \\
  g(x) &= v_1\overbrace{\left(y_1'' + py_1' + qy_1\right)}^{=0} + v_2\overbrace{\left(y_2'' + py_2' + qy_2\right)}^{=0} + v_1'\left(2y_1'+py_1\right) + v_1''y_1 + v_2\left(2y_2' + py_2\right) + v_2''y_2\\
       \\
  g(x) &= v_1'\left(2y_1' + py_1\right) + v_1''y_1 + v_2\left(2y_2' + py_2\right) + v_2''y_2.
\end{align*}
We suppose that $v_1'y_1 + v_2'y_2 = 0$. Then,
\begin{align*}
  \diff{}{x}\left(v_1'y_1 + v_2'y_2\right) &= 0\\
  v_1''y_1 + v_1'y_1' + v_2''y_2 + v_2'y_2 &= 0.
\end{align*}
Plugging into our earlier expression, we get the expression of
\begin{align*}
  v_1'y_1 + v_2'y_2 &= 0\\
  v_2'y2' + v_2'y_2' &= g(x).
\end{align*}
Plugging into matrix form, we have
\begin{align*}
  \begin{pmatrix}y_1 & y_2 \\ y_1' & y_2'\end{pmatrix} \begin{pmatrix}v_1'\\v_2'\end{pmatrix} &= \begin{pmatrix}0\\g(x)\end{pmatrix}.
\end{align*}
Since the Wronskian is nonzero, we have
\begin{align*}
  \begin{pmatrix}\diff{v_1}{x} \\ \diff{v_2}{x}\end{pmatrix} &= \begin{pmatrix}y_1 & y_2\\y_1' & y_2'\end{pmatrix}^{-1} \begin{pmatrix}0\\g(x)\end{pmatrix}\\
                                                 &= \frac{1}{y_1(x)\diff{y_2}{x} - y_2(x)\diff{y_1}{x}} \begin{pmatrix}-y_2(x)g(x) \\ y_1(x)g(x)\end{pmatrix}\label{eq:variation_parameters}\tag{\textdaggerdbl}
\end{align*}
\begin{example}
  Let
  \begin{align*}
    y'' - 2y' + y &= e^x.
  \end{align*}
  Solving the homogeneous solution, we have the characteristic equation of $r^2 - 2r + 1 = 0$. Thus, $y_1(x) = e^x$ and $y_2(x) = xe^x$.\newline

  To find $y_p(x)$, we guess $y_p(x) = x^2 e^x$. Using the power of computation in Sage, we get the answer of
  \begin{lstlisting}[style=pythonstyle,title=Avoiding Variation of Parameters]
  de = diff(y,x,2) - 2*diff(y,x) + y - e^(x)
  g = desolve(de,y)
  latex(expand(g))
  \end{lstlisting}
  
  \begin{align*}
    y_p(x) &= K_{2} x e^{x}+ K_{1} e^{x} + \frac{1}{2} x^{2} e^{x}.
  \end{align*}
  However, this is a very unsatisfying method.\newline

  Using \eqref{eq:variation_parameters}, we can find a different solution. We find
  \begin{align*}
    \diff{v_1}{x} &= \frac{1}{e^{2x}}\left( \left( -1 \right)\left( xe^{x} \right)\left( e^x \right) \right)\\
            &= -x,
  \end{align*}
  yielding
  \begin{align*}
    v_1(x) &= -\frac{x^2}{2} + c_2.
  \end{align*}
  Similarly, we get
  \begin{align*}
    \diff{v_2}{x} &= \frac{1}{e^{2x}}\left( e^x \right)\left( e^x \right)\\
    v_2(x) &= x + c_2.
  \end{align*}
  This gives
  \begin{align*}
    y_p(x) &= \frac{1}{2}x^2e^x.
  \end{align*}
\end{example}
\begin{example}
  Let
  \begin{align*}
    \diff{^3y}{x^3} - \frac{dy}{dx} &= x + e^{x}.
  \end{align*}
  Using the characteristic equation, we have $y_1(x) = 1$, $y_2(x) = e^x$, and $y_3(x) = e^{-x}$.\newline

  Now, using the Wronskian, we get
  \begin{align*}
    \begin{pmatrix}v_1' \\ v_2' \\ v_3'\end{pmatrix} &= \begin{pmatrix}1 & e^x & e^{-x} \\ 0 & e^x & -e^{-x} \\ 0 & e^{x} & e^{-x}\end{pmatrix}^{-1} \begin{pmatrix}0\\0\\x + e^{x}\end{pmatrix}.
  \end{align*}
  This would suck, but we would be able to find a solution nonetheless.
\end{example}
In the general form, with linearly independent homogeneous solutions $y_1,\dots,y_n$, we have the solution of
\begin{align*}
  \begin{pmatrix}v_1' \\ \vdots \\ v_n'\end{pmatrix} &= \begin{pmatrix}y_1 & \cdots & y_n \\ \vdots & \ddots & \vdots \\ y_1^{(n-1)} & \cdots & y_n^{(n-1)}\end{pmatrix}^{-1} \begin{pmatrix}0 \\ \vdots \\ g(x)\end{pmatrix}\\
  y(x) &= \sum_{i=1}^{n}c_iy_i(x) + \sum_{i=1}^{n}v_i(x)y_i(x).
\end{align*}
\subsection{Systems of Homogeneous Equations}%
We will consider solving systems of equations.
\begin{example}[Solving a Coupled System]
  Before we can start using variation of parameters for systems, we need to recall how to solve constant-coefficient systems.
  \begin{align*}
    x'(t) &= 3x(t) + y(t)\\
    y'(t) &= x(t) + 3y(t).
  \end{align*}
  Here, setting
  \begin{align*}
    \mathbf{x} &= \begin{pmatrix}x(t)\\y(t)\end{pmatrix},
  \end{align*}
  we get system of linear equations
  \begin{align*}
    \mathbf{x}'(t) &= \begin{pmatrix}3 & 1 \\ 1 & 3\end{pmatrix} \mathbf{x}\\
    \begin{pmatrix}x'(t) \\y'(t)\end{pmatrix} &= \begin{pmatrix}3& 1\\1&3\end{pmatrix} \begin{pmatrix}x(t)\\y(t)\end{pmatrix}.
  \end{align*}

\end{example}
\begin{remark}
In the matrix
\begin{align*}
  A &= \begin{pmatrix}a & b \\ b & a\end{pmatrix},
\end{align*}
the eigenvalues are 
\begin{align*}
  \lambda_1 &= a + b\\
  \lambda_2 &= a-b\\
  \end{align*}
  with eigenvectors of
  \begin{align*}
  \mathbf{v}_1 &= \begin{pmatrix}1\\1\end{pmatrix}\\
  \mathbf{v}_2 &= \begin{pmatrix}1\\-1\end{pmatrix}.
\end{align*}
\end{remark}
\begin{example}[General $n$-dimensional System of Differential Equations]
  Consider the system of equations defined by
  \begin{align*}
    x_1'(t) &= g_1\left( t,x_1(t),\dots,x_n(t) \right)\\
            &\vdots\\
    x_n'(t) &= g_n\left( t,x_1(t),\dots,x_n(t) \right).
  \end{align*}
  We will refine this slightly so as to be a system of \textit{linear} equations. Let
  \begin{align*}
    \mathbf{x} &= \begin{pmatrix}x_1(t)\\\vdots\\x_n(t)\end{pmatrix}\\
    \diff{\mathbf{x}}{t} &= \begin{pmatrix}x_1'(t)\\\vdots\\x_n'(t)\end{pmatrix}\\
    \mathbf{F} &= \begin{pmatrix}f_1(t) \\ \vdots \\ f_n(t)\end{pmatrix}\\
    \mathbf{x}_{t_0} &= \begin{pmatrix}x_1\left(t_0\right)\\\vdots\\x_n\left(t_0\right)\end{pmatrix}.
  \end{align*}
  Now, we have
  \begin{align*}
    \diff{\mathbf{x}}{t} &= A\mathbf{x},
  \end{align*}
  where $\mathbf{x}\left(t_0\right) = \mathbf{x}_{t_0}$ and $A$ is some matrix that represents some linear transformation.\newline

  Furthermore, we may make an inhomogeneous equation by
  \begin{align*}
    \diff{\mathbf{x}}{t} &= A\mathbf{x} + \mathbf{F}.
  \end{align*}
\end{example}
\begin{example}
  Going back to our example of
  \begin{align*}
    \diff{\mathbf{x}}{t} &= \underbrace{\begin{pmatrix}3 & 1 \\ 1 & 3\end{pmatrix}}_{A}\mathbf{x}.
  \end{align*}
  We find eigenvalues of $\lambda_1 = 4,\lambda_2 = 2$ and eigenvectors $\mathbf{v}_1 = \begin{pmatrix}1\\1\end{pmatrix}$ and $\mathbf{v}_2 = \begin{pmatrix}1\\-1\end{pmatrix}$. This gives
  \begin{align*}
    \mathbf{x}_1 &= e^{4t} \begin{pmatrix}1\\1\end{pmatrix}\\
    \mathbf{x}_2 &= e^{2t} \begin{pmatrix}1\\-1\end{pmatrix}.
  \end{align*}
  In general, if we have two distinct eigenvalues, then our solutions are
  \begin{align*}
    \mathbf{x} &= e^{\lambda t} \mathbf{v}
  \end{align*}
  Define
  \begin{align*}
    \Phi_A(t) &= \begin{pmatrix}\mathbf{x}_1 & \mathbf{x}_2\end{pmatrix}\\
                       &= \begin{pmatrix}e^{4t} & e^{2t} \\ e^{4t} & -e^{2t}\end{pmatrix}.
  \end{align*}
  We call $\Phi_A$ a fundamental matrix for $A$.\newline

  The general solution to the system is given by
  \begin{align*}
    \mathbf{x}(t) &= c_1\mathbf{x}_1(t) + c_2\mathbf{x}_2(t)\\
                  &= c_1 \begin{pmatrix}e^{4t}\\e^{4t}\end{pmatrix} + c_2 \begin{pmatrix}e^{2t} \\ -e^{2t}\end{pmatrix}\\
                  &= \begin{pmatrix}e^{4t} & e^{2t} \\ e^{4t} & -e^{2t}\end{pmatrix} \begin{pmatrix}c_1\\c_2\end{pmatrix}.
  \end{align*}
\end{example}
\begin{example}
  Consider the equation
  \begin{align*}
    \diff{\mathbf{x}}{t} &= A\mathbf{x},
  \end{align*}
  where
  \begin{align*}
    A &= \begin{pmatrix}4 & 2 & 1 \\ 0 & 4 & 2 \\ 0 & 0 & 4\end{pmatrix} \label{eq:generalized_eigenvalues_example}\tag{$A$}
  \end{align*}
  Notice that we have a triple-repeated eigenvalue,
  \begin{align*}
    \lambda_1 &= 4\\
    \lambda_2 &= 4\\
    \lambda_3 &= 4.
  \end{align*}
  Unfortunately, to find the eigenvectors, this will be a bit harder.
  \begin{align*}
    \left( A - 4I \right)\mathbf{v} &= 0\\
    \begin{pmatrix}0 & 2 & 1 \\ 0 & 0 & 2 \\ 0 & 0 & 0\end{pmatrix} \begin{pmatrix}a\\b\\c\end{pmatrix} &= \begin{pmatrix}0\\0\\0\end{pmatrix}.
  \end{align*}
  This gives
  \begin{align*}
    \begin{pmatrix}2b + c \\ 2c \\ 0 \end{pmatrix} &= \begin{pmatrix}0\\0\\0\end{pmatrix},
  \end{align*}
  so $b = c = 0$, and our eigenvector is
  \begin{align*}
    \mathbf{v}_1 &= \begin{pmatrix}1\\0\\0\end{pmatrix}.
  \end{align*}
  We may need some more eigenvectors. Currently, our solution is
  \begin{align*}
    \mathbf{x}_1(t) &= e^{4t} \begin{pmatrix}1\\0\\0\end{pmatrix}.
  \end{align*}
  We need to go into the realm of generalized eigenvectors. If $\lambda$ is repeated, we need to do the following.
  \begin{enumerate}[(1)]
    \item Find all the eigenvectors for which $\left( A - \lambda I \right)\mathbf{v} = 0$. If we come up short, then we have a defective system.
    \item For the remaining eigenvectors, we solve the system
      \begin{align*}
        \left( A - \lambda I \right)\mathbf{v}_j &= \mathbf{v}_k,
      \end{align*}
      where $\mathbf{v}_k$ is known, and we desire $\mathbf{v}_j$. The $\mathbf{v}_j$ are known as generalized eigenvectors.
    \item Continue this process until we are done. 
  \end{enumerate}
  Now, in this case, we get
  \begin{align*}
    \begin{pmatrix}0 & 2 & 1 \\ 0 & 0 & 2 \\ 0 & 0 & 0\end{pmatrix} \begin{pmatrix}a\\b\\c\end{pmatrix} &= \begin{pmatrix}1\\0\\0\end{pmatrix}.
  \end{align*}
  This gives
  \begin{align*}
    \begin{pmatrix}2b + c \\ 2c \\ 0\end{pmatrix} &= \begin{pmatrix}1\\0\\0\end{pmatrix},
  \end{align*}
  and a generalized eigenvector of
  \begin{align*}
    \mathbf{v}_2 &= \begin{pmatrix}0\\1/2\\0\end{pmatrix}.
  \end{align*}
  Going at it again, we have
  \begin{align*}
    \begin{pmatrix}0 & 2 & 1 \\ 0 & 0 & 2 \\ 0 & 0 & 0\end{pmatrix} \begin{pmatrix}a\\b\\c\end{pmatrix} &= \begin{pmatrix}0\\1/2\\0\end{pmatrix},
  \end{align*}
  giving the equation
  \begin{align*}
    \begin{pmatrix}2b + c\\2c\\0\end{pmatrix} &= \begin{pmatrix}0\\1/2\\0\end{pmatrix},
  \end{align*}
  giving
  \begin{align*}
    \mathbf{v}_3 &= \begin{pmatrix}0\\-1/8\\1/4\end{pmatrix}.
  \end{align*}
  Note that when we take generalized eigenvectors, we ``integrate'' with respect to $t$ before adding. For instance
  \begin{align*}
    \mathbf{x}_1 &= e^{\lambda t} \mathbf{v}_1\\
    \mathbf{x}_2 &= e^{\lambda t} \left( t \mathbf{v}_1 + \mathbf{v}_2 \right)\\
    \mathbf{x}_3 &= e^{\lambda t} \left( \frac{t^2}{t}\mathbf{v}_1 + t\mathbf{v}_2 + \mathbf{v}_3 \right).
  \end{align*}
  Now, our linearly independent solutions to the system in \eqref{eq:generalized_eigenvalues_example} is of the form
  \begin{align*}
    \mathbf{x}_1(t) &= e^{4t} \begin{pmatrix}1\\0\\0\end{pmatrix}\\
    \mathbf{x}_2(t) &= e^{4t} \left( t\begin{pmatrix}1\\0\\0\end{pmatrix} + \begin{pmatrix}0\\1/2\\0\end{pmatrix} \right)\\
    \mathbf{x}_3 (t) &= e^{4t} \left( \frac{t^2}{2} \begin{pmatrix}1\\0\\0\end{pmatrix} + t \begin{pmatrix}0\\1/2\\0\end{pmatrix} + \begin{pmatrix}0\\-1/8\\1/4\end{pmatrix} \right).
  \end{align*}
  This gives the fundamental matrix
  \begin{align*}
    \Phi(t) &= \begin{pmatrix}e^{4t} & te^{4t} & \frac{t^2}{2}e^{4t} \\ 0 & \frac{1}{2}e^{4t} & e^{4t}\left( \frac{t}{2} - \frac{1}{8} \right) \\ 0 & 0 & \frac{1}{4}e^{4t}\end{pmatrix}.
  \end{align*}
  The general solution is
  \begin{align*}
    \mathbf{x}(t) &= \Phi(t) \mathbf{c}.
  \end{align*}
  The general solution is, then,
  \begin{align*}
    \mathbf{x}(t) &= e^{At}\mathbf{c},
  \end{align*}
  where $\mathbf{c}$ is a constant vector, and $e^{At}$ is the matrix exponential of $A$.
\end{example}
\begin{example}
  Consider $A$ as the matrix with eigenvalue $\lambda$ and eigenvector $\mathbf{v}_1$ and generalized eigenvectors $\mathbf{v}_2$ and $\mathbf{v}_3$. Then, the solution set
  \begin{align*}
    \mathbf{x}_1(t) &= e^{\lambda t}\mathbf{v}_1\\
    \mathbf{x}_2(t) &= e^{\lambda t}\left( t\mathbf{v}_1 + \mathbf{v}_2 \right)\\
    \mathbf{x}_3(t) &= e^{\lambda t}\left( \frac{t^2}{2}\mathbf{v}_1 + t\mathbf{v}_2 + \mathbf{v}_3 \right).
  \end{align*}
  Thus, we have
  \begin{align*}
    \diff{\mathbf{x}}{t} &= \lambda e^{\lambda t} \mathbf{v}_1\\
    A\mathbf{x}_1(t) &= Ae^{\lambda t} \mathbf{v}_1\\
                     &= e^{\lambda t} A\mathbf{v}_1\\
                     &= \lambda e^{\lambda t}\mathbf{v}_1.
  \end{align*}
  Now, recalling that $A\mathbf{v}_1 = \lambda \mathbf{v}_1$ and $A\mathbf{v}_2 = \lambda \mathbf{v}_2 + \mathbf{v}_1$, we have
  \begin{align*}
    \diff{\mathbf{x}_2}{t} &= \lambda e^{\lambda t}\left( t\mathbf{v}_1 + \mathbf{v}_2 \right) + e^{\lambda t}\mathbf{v}_1\\
    A\mathbf{x}_2(t) &= Ae^{\lambda t}\left( t\mathbf{v}_1 + \mathbf{v}_2 \right)\\
                     &= e^{\lambda t}\left( tA\mathbf{v}_1 +  A\mathbf{v}_2\right)\\
                     &= e^{\lambda t}\left( t\lambda \mathbf{v}_1 + \lambda \mathbf{v}_2 + \mathbf{v}_1 \right)\\
                     &= \lambda e^{\lambda t}\left( t\mathbf{v}_1 + \mathbf{v}_2 \right) + e^{\lambda t}\mathbf{v}_1.
  \end{align*}
  Finally, we have $A\mathbf{v}_3 = \lambda \mathbf{v}_3 + \mathbf{v}_2 $.
\end{example}
\begin{example}
  We assume $A$ is a $n\times n$ real matrix. Then, all complex eigenvalues of $A$ come in conjugate pairs, $\lambda_1 = a + ib$ and $\lambda_2 = a - ib$.\newline

  Then, our eigenvectors are of the form $\mathbf{v}_1 = \mathbf{u} + i\mathbf{w}$ and $\mathbf{v}_2 = \mathbf{u} - i\mathbf{w}$.\newline

  Note that if we find the solution for $\lambda_1$ and $\mathbf{v}_2$. This gives
  \begin{align*}
    e^{\lambda t}\mathbf{v} &= e^{\left( a + ib \right)} \left( \mathbf{u} + i\mathbf{w} \right)\\
                            &= e^{at}\left( \cos\left( bt \right) + i\sin\left( bt \right) \right) \left( \mathbf{u} + i\mathbf{w} \right)\\
                            &= e^{at}\left( \left( \cos\left( bt \right) \mathbf{u} - \sin\left( bt \right)\mathbf{w} \right) + i\left( \cos\left( bt \right)\mathbf{w} + \sin\left( bt \right)\mathbf{u} \right) \right).
  \end{align*}
\end{example}
\begin{example}
  Consider the matrix
  \begin{align*}
    A &= \begin{pmatrix}1 & 0 & -4 \\ 0 & 3 & 0 \\ 2 & 0 & 5\end{pmatrix}
  \end{align*}
  for the system of equations
  \begin{align*}
    \diff{\mathbf{x}}{t} &= A\mathbf{x}.
  \end{align*}
  Using the power of computation, we have
  \begin{align*}
    \lambda_1 &= 3\\
    \lambda_2 &= 3+2i\\
    \lambda_3 &= 3-2i,
  \end{align*}
  and eigenvectors of
  \begin{align*}
    \mathbf{v}_1 &= \begin{pmatrix}0\\1\\0\end{pmatrix}\\
    \mathbf{v}_2 &= \begin{pmatrix}-4\\0\\2+2i\end{pmatrix}\\
    \mathbf{v}_3 &= \begin{pmatrix}-4\\0\\2-2i\end{pmatrix}.
  \end{align*}
  Now, we see that
  \begin{align*}
    \mathbf{x}_1(t) &= e^{\lambda_1 t}\mathbf{v}_1\\
                    &= \begin{pmatrix}0\\e^{3t}\\0\end{pmatrix},
  \end{align*}
  and
  \begin{align*}
    \mathbf{x}_2(t) &= e^{3t}\left( \cos\left( 2t \right) \begin{pmatrix}-4\\0\\2\end{pmatrix} - \sin\left( 2t \right) \begin{pmatrix}0\\0\\2\end{pmatrix} \right)\\
    \mathbf{x}_3(t) &= e^{3t} \left( \cos\left( 2t \right) \begin{pmatrix}0\\0\\2\end{pmatrix} + \sin\left( 2t \right) \begin{pmatrix}-4\\0\\2\end{pmatrix} \right).
  \end{align*}
  This gives the matrix
  \begin{align*}
    \Phi(t) &= \begin{pmatrix}0 & -4e^{3t}\cos\left( 2t \right) & -4e^{3t}\sin\left( 2t \right) \\ e^{3t} & 0 & 0 \\ 0 & 2e^{3t}\left( \cos\left( 2t \right) - \sin\left( 2t \right) \right) & 2e^{3t}\left( \sin\left( 2t \right) + \cos\left( 2t \right) \right)\end{pmatrix}\\
    W(t) &= \det\left( \Phi(t) \right)\\
         &= -e^{3t}\left( -8e^{6t}\left( \cos\left( 2t \right)\sin\left( 2t \right) + \cos^2\left( 2t \right) \right) + 8e^{6t}\left( \sin\left( 2t \right)\cos\left( 2t \right) - \sin^2\left( 2t \right) \right) \right)\\
         &= 8e^{9t}\\
         &\neq 0.
  \end{align*}
\end{example}
\begin{example}
  We wish to solve $\diff{\mathbf{x}}{t} = A\mathbf{x}$, where
  \begin{align*}
    A &= \begin{pmatrix}2 & 0 & 1 & 0 & 0 \\ 0 & 1 & 0 & 5 & 0 \\ 0 & 0 & 2 & 0 & 0 \\ 0 & -2 & 0 & -1 & 0 \\ 0 & 0 & 0 & 0 & 2\end{pmatrix}.
  \end{align*}
  To find our eigenvalues and eigenvectors, we begin by finding
  \begin{align*}
    \det\left( A - \lambda I \right) &= \left( 2-\lambda \right)^2 \det \begin{pmatrix}2-\lambda & 0 & 0 \\ 0 & 1-\lambda & 5 \\ 0 & -2 & -1-\lambda\end{pmatrix}\\
                                     &= \left( 2-\lambda \right)^3 \det \begin{pmatrix}1-\lambda & 5 \\ -2 & -1-\lambda\end{pmatrix}\\
                                     &= \left( 2-\lambda \right)^3 \left( \left( 1-\lambda \right)\left( -1-\lambda \right) + 10 \right).
  \end{align*}
  We have five eigenvalues,
  \begin{align*}
    \lambda &= \pm 3i,2,2,2.
  \end{align*}
  For $\lambda_{1,2} = \pm3i$, then
  \begin{align*}
    \mathbf{v}_{1,2} &= \begin{pmatrix}0\\1\\0\\-2\\0\end{pmatrix} \pm i \:\begin{pmatrix}0\\3\\0\\0\\0\end{pmatrix}.
  \end{align*}
  Now for $\lambda_3 = 2$, we have
  \begin{align*}
    \mathbf{v}_3 &= \begin{pmatrix}0\\0\\0\\0\\1\end{pmatrix}.
  \end{align*}
  Now, we have
  \begin{align*}
    \left( A - 2I \right)\mathbf{v}_3 &= 0\\
    \begin{pmatrix}0 & 0 & 1 & 0 & 0 \\ 0 & -1 & 0 & 5 & 0 \\ 0 & 0 & 0 & 0 & 0 \\  0 & -2 & 0 & -3 & 0 \\ 0 & 0 & 0 & 0 & 0 \end{pmatrix} \:\begin{pmatrix}a\\b\\c\\d\\3\end{pmatrix} &= 0.
  \end{align*}
  From this equation, we have
  \begin{align*}
    c &= 0\\
    -b + 5d &= 0\\
    -2b-3d &= 0.
  \end{align*}
  Now, we have independent $a$ and $e$. This gives
  \begin{align*}
    \mathbf{v}_4 &= \begin{pmatrix}1\\0\\0\\0\\0\end{pmatrix}.
  \end{align*}
  Note that both $\mathbf{v}_3$ and $\mathbf{v}_4$ are regular eigenvectors. Now, we wish to find one generalized eigenvector. We find this generalized eigenvector, $\mathbf{w}$, by observing that the $1$ in entry $A_{1,3}$ effectively ties our vector $\mathbf{v}_4$ to vector $\mathbf{v}_{1,2}$. Thus, we get
  \begin{align*}
    \left( A - 2I \right)\mathbf{w} &= \mathbf{v}_4\\
    \begin{pmatrix}0 & 0 & 1 & 0 & 0 \\ 0 & -1 & 0 & 5 & 0 \\ 0 & 0 & 0 & 0 & 0 \\ 0 & -2 & 0 & -3 & 0 \\ 0 & 0 & 0 & 0 & 0\end{pmatrix} \begin{pmatrix}a\\b\\c\\d\\e\end{pmatrix} &= \begin{pmatrix}1\\0\\0\\0\\0\end{pmatrix}.
  \end{align*}
  Now, solving this, we get $c = 1$, giving the generalized eigenvector of
  \begin{align*}
    \mathbf{w} &= \begin{pmatrix}0\\0\\1\\0\\0\end{pmatrix}.
  \end{align*}
  Now, we have
  \begin{align*}
    \mathbf{v}_{1} &= \begin{pmatrix}0\\1\\0\\-2\\0\end{pmatrix} + i \:\begin{pmatrix}0\\3\\0\\0\\0\end{pmatrix}\\
    \mathbf{v}_{2} &= \begin{pmatrix}0\\1\\0\\-2\\0\end{pmatrix} - i \:\begin{pmatrix}0\\3\\0\\0\\0\end{pmatrix}\\
    \mathbf{v}_3 &= \begin{pmatrix}0\\0\\0\\0\\1\end{pmatrix}\\
    \mathbf{v}_4 &= \begin{pmatrix}1\\0\\0\\0\\0\end{pmatrix}\\
    \mathbf{w} &= \begin{pmatrix}0\\0\\1\\0\\0\end{pmatrix},
  \end{align*}
  where $\mathbf{v}_4\rightarrow \mathbf{w}$ is a chain of length $2$. This gives the JCF of
  \begin{align*}
    A &= \begin{pmatrix}0 & 0 & 0 & 1 & 0 \\ 1+3i & 1-3i & 0 & 0 & 0 \\ 0 & 0 & 0 & 0 & 1 \\ -2 & -2 & 0 & 0 & 0 \\ 0 & 0 & 1 & 0 & 0\end{pmatrix} \begin{pmatrix}3i & 0 & 0 & 0 & 0 \\ 0 & -3i & 0 & 0 & 0 \\ 0 & 0 & 2 & 0 & 0 \\ 0 & 0 & 0 & 2 & 1 \\ 0 & 0 & 0 & 0 & 2\end{pmatrix} \begin{pmatrix}0 & 0 & 0 & 1 & 0 \\ 1+3i & 1-3i & 0 & 0 & 0 \\ 0 & 0 & 0 & 0 & 1 \\ -2 & -2 & 0 & 0 & 0 \\ 0 & 0 & 1 & 0 & 0\end{pmatrix}^{-1}.
  \end{align*}
  We get the solutions
  \begin{align*}
    \mathbf{x}_1(t) &= \begin{pmatrix}0 \\ -\cos\left( 3t \right) - 3\sin\left( 3t \right)\\0\\2\cos\left( 3t \right)\\0\end{pmatrix}\\
    \mathbf{x}_2(t) &= \begin{pmatrix}0\\3\cos\left( 3t \right)-\sin\left( 3t \right) \\ 0 \\ 2\sin\left( 3t \right)\\0\end{pmatrix}\\
    \mathbf{x}_3(t) &= \begin{pmatrix}0\\0\\0\\0\\e^{2t}\end{pmatrix}\\
    \mathbf{x}_4(t) &= \begin{pmatrix}e^{2t}\\0\\0\\0\\0\end{pmatrix}\\
    \mathbf{x}_5(t) &= \begin{pmatrix}te^{2t}\\0\\e^{2t}\\0\\0\end{pmatrix},
  \end{align*}
  where $\mathbf{x}_5(t) = e^{2t}\left( t\mathbf{v}_4 + \mathbf{v}_5 \right)$.\newline

  The fundamental solution matrix is
  \begin{align*}
    \Phi(t) &= \begin{pmatrix}0 & 0 & 0 & e^{2t} & te^{2t} \\ -\cos\left( 3t \right) + 3\sin\left( 3t \right) & 3\cos\left( 3t \right) - \sin\left( 3t \right) & 0 & 0 & 0 \\ 0 & 0 & 0 & 0 & e^{2t} \\ 2\cos\left( 3t \right) & 2\sin\left( 3t \right) & 0 & 0 & 0 \\ 0 & 0 & e^{2t} & 0 & 0\end{pmatrix}.
  \end{align*}
  Now, we want to find $\Phi(0)$, or $\Phi\left( t_0 \right)$. Furthermore, we need to find $\Phi^{-1}(0)$, or $\Phi^{-1}\left( t_0 \right)$.
\end{example}
\begin{example}[Implementing Initial Conditions]
  Looking back at our equation
  \begin{align*}
    \diff{\mathbf{x}}{t} &= A\mathbf{x},
  \end{align*}
  we may apply the initial condition of
  \begin{align*}
    \mathbf{x}\left( t_0 \right) &= \mathbf{x}_0.
  \end{align*}
  We use the matrix
  \begin{align*}
    A &= \begin{pmatrix}3 & 1 \\ 1 & 3\end{pmatrix},
  \end{align*}
  with the initial condition
  \begin{align*}
    \mathbf{x}_0 &= \begin{pmatrix}4\\15\end{pmatrix}.
  \end{align*}
  Generally our approach to solving this kind of problem, we take the eigenvectors and eigenvalues, giving
  \begin{align*}
    \lambda_1 &= 4\\
    \mathbf{v}_1 &= \begin{pmatrix}1\\1\end{pmatrix}\\
    \lambda_2 &= 2\\
    \mathbf{v}_2 &= \begin{pmatrix}1\\-1\end{pmatrix},
  \end{align*}
  and associated solutions of
  \begin{align*}
    \mathbf{x}_1 &= \begin{pmatrix}e^{4t}\\e^{4t}\end{pmatrix}\\
    \mathbf{x}_2 &= \begin{pmatrix}e^{2t} \\ -e^{2t}\end{pmatrix}.
  \end{align*}
  Then, we form a fundamental matrix of solutions:
  \begin{align*}
    \Phi(t) &= \begin{pmatrix}e^{4t} & e^{2t} \\ e^{4t} & -e^{2t}\end{pmatrix}.
  \end{align*}
  Note that, for any vector of constants $\mathbf{c}$, we have
  \begin{align*}
    \mathbf{x}(t) &= \Phi(t)\mathbf{c}
  \end{align*}
  is a solution of $\diff{\mathbf{x}}{t} = A\mathbf{x}$.\newline

  To find $\mathbf{c}$, we see that
  \begin{align*}
    \mathbf{x}_0 &= \Phi(0)\mathbf{c},
  \end{align*}
  so that
  \begin{align*}
    \mathbf{x}(t) &= \Phi(t)\Phi^{-1}(0)\mathbf{x}_0
  \end{align*}
  is the solution to our initial value problem.\newline

  Calculating
  \begin{align*}
    \Phi(0) &= \begin{pmatrix}1 & 1 \\ 1 & -1\end{pmatrix},
  \end{align*}
  we find
  \begin{align*}
    \Phi^{-1}(0) &= \frac{1}{2} \begin{pmatrix}1 & 1\\1 & -1\end{pmatrix}.
  \end{align*}
  Thus, we get the solutions of
  \begin{align*}
    \mathbf{x}(t) &= \begin{pmatrix}e^{4t} & e^{2t} \\ e^{4t} & -e^{2t} \end{pmatrix} \frac{1}{2}\begin{pmatrix}1 & 1 \\ 1 & -1\end{pmatrix} \begin{pmatrix}4\\15\end{pmatrix}\\
                  &= \begin{pmatrix}\frac{19}{2}e^{4t} - \frac{11}{2}e^{2t} \\ \frac{19}{2}e^{4t} + \frac{11}{2}e^{2t}\end{pmatrix}.
  \end{align*}
  Note that we may define
  \begin{align*}
    \Psi(t) &= \Phi(t) \Phi^{-1}(0),
  \end{align*}
  giving
  \begin{align*}
    \mathbf{x}(t) &= \Psi(t)\mathbf{x}_0.
  \end{align*}
  We may calculate
  \begin{align*}
    \Psi(t) &= \Phi(t) \Phi^{-1}(0)\\
            &= \frac{1}{2} \begin{pmatrix}e^{4t} & e^{2t} \\ e^{4t} & -e^{2t}\end{pmatrix} \begin{pmatrix}1 & 1 \\ 1 & -1\end{pmatrix}\\
            &= \frac{1}{2} \begin{pmatrix}e^{4t} + e^{2t} & e^{4t} - e^{2t} \\ e^{4t} - e^{2t} & e^{4t} + e^{2t}\end{pmatrix}.
  \end{align*}
\end{example}
\subsection{The Matrix Exponential}%
\begin{example}[The Matrix Exponential]
  When we have a single first-order equation, such as
  \begin{align*}
    \diff{y}{t} &= 3y,
  \end{align*}
  with initial condition $y(0)$, we solve it by taking $y(t) = \pi e^{3t}$.\newline

  Similarly, if we're given
  \begin{align*}
    \diff{\mathbf{x}}{t} &= A\mathbf{x},
  \end{align*}
  we may want to know if there is an analogous $e^{At}$.\newline

  In fact, there is. Using the Taylor expansion, we have
  \begin{align*}
    e^{At} &= I + At + A^2 \frac{t^2}{2} + A^{3}\frac{t^3}{6} + \cdots\\
           &= \sum_{k=0}^{\infty}A^{k}\frac{t^k}{k!}.
  \end{align*}
  Note that we may take $P$ to be the matrix of unit eigenvectors of $A$, and $D$ to be the matrix of eigenvalues corresponding to column eigenvectors
  \begin{align*}
    A &= PDP^{-1}.
  \end{align*}
  This is assuming $A$ can be diagonalized. This gives $D = P^{-1}AP$.\newline

  Now, if $A$ can be diagonalized, we can take
  \begin{align*}
    e^{At} &= I  + \left( PDP^{-1} \right)t + \left( PDP^{-1} \right)^2\frac{t^2}{2} + \cdots\\
           &= \sum_{k=0}^{\infty}\left( PDP^{-1} \right)^{k}\frac{t^k}{k!}\\
           &= \sum_{k=0}^{\infty}PD^{k}P^{-1}\frac{t^k}{k!}\\
           &= P\left( \sum_{k=0}^{\infty} D^k\frac{t^k}{k!}\right)P^{-1}.
  \end{align*}
  We can find the power on any diagonal matrix much more easily than we can on a general matrix. In particular, this gives
  \begin{align*}
    e^{At} &= P \begin{pmatrix}e^{\lambda_1 t} & \cdots & 0 \\ \vdots & \ddots & \vdots \\ 0 & \cdots & e^{\lambda_n t}\end{pmatrix}P^{-1}
  \end{align*}
\end{example}
  Given $A$ in the system $\diff{\mathbf{x}}{t} = A\mathbf{x}$, we wish to find $e^{At}$, and show that $e^{At} = \Psi(t)$, where $\Psi(t) = \Phi(t)\Phi^{-1}(0)$.
\begin{example}
  Let
  \begin{align*}
    A &= \begin{pmatrix}4 & 2 \\ 0 & 4\end{pmatrix}.
  \end{align*}
  We see that $A$ has repeated eigenvalues of $4$ and $4$.\newline

  Our first eigenvector is
  \begin{align*}
    \mathbf{v} &= \begin{pmatrix}1\\0\end{pmatrix}.
  \end{align*}
  Now, evaluating
  \begin{align*}
    \left( A - 4I \right)\mathbf{v} &= 0\\
    \begin{pmatrix}0 & 2\\0 & 0\end{pmatrix} \begin{pmatrix}a\\b\end{pmatrix} &= \begin{pmatrix}0\\0\end{pmatrix},
  \end{align*}
  giving $2b = 0$.\newline

  Thus, we're going to need a generalized eigenvector. We have
  \begin{align*}
    \begin{pmatrix}0 & 2 \\ 0 & 0\end{pmatrix} \begin{pmatrix}a\\b\end{pmatrix} &= \begin{pmatrix}1\\0\end{pmatrix},
  \end{align*}
  giving $\left( A - \lambda I \right)\mathbf{w} = \mathbf{v}_1$. Thus, we have $2b = 1$. Thus, we have
  \begin{align*}
    \mathbf{w} &= \begin{pmatrix}0\\1/2\end{pmatrix}.
  \end{align*}
  Now, we have
  \begin{align*}
    \mathbf{x}_1(t) &= e^{4t}\begin{pmatrix}1\\0\end{pmatrix}\\
                    &= \begin{pmatrix}e^{4t} \\ 0\end{pmatrix}\\
    \mathbf{x}_2 (t) &= te^{4t} \begin{pmatrix}1\\0\end{pmatrix} + e^{4t} \begin{pmatrix}0\\1/2\end{pmatrix}\\
                     &= \begin{pmatrix}te^{4t}\\\frac{1}{2}e^{4t}\end{pmatrix}.
  \end{align*}
  Therefore, we have
  \begin{align*}
    \mathbf{x}(t) &= c_1 \mathbf{x}_1(t) + c_2\mathbf{x}_2(t)\\
                  &= \begin{pmatrix}e^{4t} & te^{4t} \\ 0 & \frac{1}{2}e^{4t}\end{pmatrix} \begin{pmatrix}c_1\\c_2\end{pmatrix}.
  \end{align*}
  Now, we see that
  \begin{align*}
    \Phi(0) &= \begin{pmatrix}1 & 0 \\ 0 & 1/2\end{pmatrix}\\
    \Phi^{-1}(0) &= \begin{pmatrix}1 & 0 \\ 0 & 2\end{pmatrix}.
  \end{align*}
  Therefore, our matrix exponential is
  \begin{align*}
    \Psi(t) &= \begin{pmatrix}e^{4t} & te^{4t} \\ 0 & \frac{1}{2}e^{4t}\end{pmatrix} \begin{pmatrix}1 & 0 \\ 0 & 2\end{pmatrix}\\
            &= \begin{pmatrix}e^{4t} & 2te^{4t} \\ 0 & e^{4t}\end{pmatrix}.
  \end{align*}
  Therefore,
  \begin{align*}
    \mathbf{x}(t) &= \Psi(t) \mathbf{x}_0\\
                  &= \begin{pmatrix}e^{4t} & 2te^{4t} \\ 0 & e^{4t}\end{pmatrix} \mathbf{x}_0.
  \end{align*}
  We often refer to $\Psi(t)$ as the flow matrix.\newline

  Now, because $\Psi(0) = I$, we have
  \begin{align*}
    \Psi\left( t \right)\Psi\left( -t \right) &= I,
  \end{align*}
  meaning that $\Psi\left( -t \right) = \Psi\left( t \right)^{-1}$.
\end{example}
\begin{example}
  Consider
  \begin{align*}
    A\mathbf{v}_i &= \lambda_i \mathbf{v}_i
  \end{align*}
  as the set of eigenvector equations, giving
  \begin{align*}
    \begin{pmatrix}A\mathbf{v}_1 & A\mathbf{v}_2 & \cdots & A\mathbf{v}_n\end{pmatrix} &= \begin{pmatrix}\mathbf{v}_1\lambda_1 & \mathbf{v}_2\lambda_2 & \cdots & \mathbf{v}_n\lambda_n\end{pmatrix}\\
                                 &= \begin{pmatrix}\mathbf{v}_1 & \mathbf{v}_2 & \cdots & \mathbf{v}_n\end{pmatrix} \begin{pmatrix}\lambda_1 & & & \\ & \lambda_2 & &  \\ & & \ddots & \\ & & & \lambda_n\end{pmatrix},
  \end{align*}
  giving the expression $AP = PD$, where $P$ is the set of eigenvectors.\newline

  Now, if we have generalized eigenvectors, we have a different case.\newline

  Consider the case of a chain. We know that
  \begin{align*}
    \left( A -\lambda I \right)\mathbf{v}_1 &= 0
  \end{align*}
  is the expression of an eigenvector. Now, if we have repeated eigenvalues, we get the second equation of
  \begin{align*}
    \left( A - \lambda I \right)\mathbf{v}_2 &= \mathbf{ v }_1\\
    \left( A - \lambda I \right)\mathbf{v}_3 &= \mathbf{v}_1\\
                                             &\vdots\\
    \left( A - \lambda I \right)\mathbf{v}_{n} &= \mathbf{v}_{n-1}.
  \end{align*}
  We start by changing these equations to give
  \begin{align*}
    A\mathbf{v}_1 &= \lambda\mathbf{v}_1 \\
    A \mathbf{v}_2 &= \lambda \mathbf{v}_2 + \mathbf{v}_1\\
    A\mathbf{v}_3 &= \lambda \mathbf{v}_3 + \mathbf{v}_2\\
                  &\vdots\\
    A\mathbf{v}_n &= \lambda \mathbf{v}_n + \mathbf{v}_{n-1}.
  \end{align*}
  We see that these are effectively the eigenvalue equations with a small perturbation. Constructing the matrix, we have
  \begin{align*}
    \begin{pmatrix}A \mathbf{v}_1 & A\mathbf{v}_2 & A\mathbf{v}_3 & \cdots & A\mathbf{v}_n\end{pmatrix} &= \begin{pmatrix}\lambda \mathbf{v}_1 & \lambda \mathbf{v}_2 + \mathbf{v}_1 & \lambda \mathbf{v}_3 + \mathbf{v}_2 & \cdots \lambda \mathbf{v}_n + \mathbf{v}_{n-1}\end{pmatrix}\\
                                  &= \begin{pmatrix}\mathbf{v}_1 & \mathbf{v}_2 & \mathbf{v}_3 & \cdots & \mathbf{v}_n\end{pmatrix}  \underbrace{\begin{pmatrix} \lambda & 1 &  &  &  \\ & \lambda & 1 &  &  \\&  & \lambda & 1 &  \\ &  &  & \ddots & 1 \\&  &  &  & \lambda \end{pmatrix}}_{J}.
  \end{align*}
  We call the matrix $J$ the Jordan canonical form of $A$, and we get the expression $AP = PJ$, where $P$ is the matrix of generalized eigenvectors as columns.\newline

  Now, if we have multiple chains, we get multiple blocks. For instance, if we have the chains $\mathbf{v}_1 \rightarrow \mathbf{v}_2 \rightarrow \mathbf{v}_3$, $\mathbf{v}_4\rightarrow \mathbf{v}_5 \rightarrow \mathbf{v}_6 \to \mathbf{v}_7$, and $\mathbf{v}_8$ being standalone, all for the same eigenvalue $\lambda$. This gives the Jordan canonical form of
  \begin{align*}
    J &= \begin{pmatrix}
\lambda & 1 & 0 &  &  &  &  &  \\
0 & \lambda & 1 & 0 &  &  &  &  \\
0 & 0 & \lambda & 0 &  &  &  &  \\
 &  &  & \lambda & 1 & 0 & 0 &  \\
 &  &  & 0 & \lambda & 1 & 0 &  \\
 &  &  & 0 & 0 & \lambda & 1 &  \\
 &  &  & 0 & 0 & 0 & \lambda &  \\
 &  &  &  &  &  &  & \lambda 
\end{pmatrix} 
  \end{align*}
  The reason block matrices are useful is that they simplify calculations massively. We may consider the block matrices as
  \begin{align*}
    \begin{pmatrix}A & B \\ C & D\end{pmatrix} \begin{pmatrix}E & F \\ G & H\end{pmatrix} &= \begin{pmatrix}AE + BG & AF + BH \\ CE + DG & CF + DH\end{pmatrix}.
  \end{align*}
  For instance, if we have $4\times 4$ matrices, we convert this multiplication into 8 $2\times 2$ matrix multiplications. On first glance, this doesn't seem more efficient, but if there are a lot of zeros, it does actually become more efficient.
\end{example}
\begin{example}
  In the general case, our flow matrix is of the form
  \begin{align*}
    \Psi\left( t \right) &= e^{At}\\
                         &= Pe^{Jt}P^{-1}.
  \end{align*}
  If
  \begin{align*}
    J &= \begin{pmatrix}\lambda & 1  & 0 \\ 0 & \lambda & 1 \\ 0 & 0 & \lambda\end{pmatrix},
  \end{align*}
  we now want to find $e^{Jt}$.\newline

  Now, if we had
  \begin{align*}
    D &= \begin{pmatrix}\lambda_1 & &  \\ & \lambda_2 & \\ & & \lambda_3\end{pmatrix},
  \end{align*}
  then
  \begin{align*}
    e^{Dt} &= \begin{pmatrix}e^{\lambda_1t} & & \\ & e^{\lambda_2t} & \\ & & e^{\lambda_3t}\end{pmatrix}.
  \end{align*}
  Note that the Jordan--Chevalley decomposition allows us to take
  \begin{align*}
    e^{Jt} &= e^{\left( D + N \right)t}\\
           &= e^{Dt}e^{Nt}.
  \end{align*}
  where
  \begin{align*}
    N &= \begin{pmatrix}0 & 1 & 0 \\ 0 & 0 & 1 \\ 0 & 0 & 0\end{pmatrix}\\
    N^{2} &= \begin{pmatrix}0 & 0 & 1 \\ 0 & 0 & 0 \\ 0 & 0 & 0\end{pmatrix}\\
    N^3 &= \begin{pmatrix}0 & 0 & 0 \\ 0 & 0 & 0 \\ 0 & 0 & 0\end{pmatrix}.
  \end{align*}
  We may calculate
  \begin{align*}
    e^{Nt} &= I + Nt + \frac{N^2}{2}t^2 + \frac{N^3}{6}t^3 + \cdots\\
           &= \begin{pmatrix}1 & 0 & 0 \\ 0 & 1 & 0 \\ 0 & 0 & 1\end{pmatrix} + \begin{pmatrix}0 & t & 0 \\ 0 & 0 & t \\ 0 & 0 & 0\end{pmatrix} + \begin{pmatrix}0 & 0 & \frac{t^2}{2} \\ 0 & 0 & 0 \\ 0 & 0 & 0\end{pmatrix}.
  \end{align*}
  Therefore, we have
  \begin{align*}
    e^{Jt} &= \begin{pmatrix}e^{\lambda t} & & \\ & e^{\lambda t} & \\ & & e^{\lambda t}\end{pmatrix} \begin{pmatrix}1 & t & \frac{t^2}{2} \\ 0 & 1 & t \\ 0 & 0 & 1\end{pmatrix}
  \end{align*}
\end{example}
\begin{example}[Constructing Differential Equations]
  Consider
  \begin{align*}
    \diff{^2y}{x^2} + p(x) \diff{y}{x} + q(x)y &= 0,
  \end{align*}
  with solutions of $y_2(x) = \cos^2\left( x \right)$ and $y_2(x) = \sin^2\left( x \right)$.\newline

  We start by computing some derivatives.
  \begin{align*}
    \diff{y_1}{x} &= -2\cos(x)\sin(x)\\
            &= -\sin(2x)\\
    \diff{^2y_1}{x^2} &= -2\cos\left( 2x \right)\\
             &= -2\sin^2\left( x \right) + 2\cos^2\left( x \right)\\
             \\
    \diff{y_2}{x} &= 2\sin(x)\cos(x)\\
                  &= \sin(2x)\\
    \diff{^2y_2}{x^2} &= 2\cos(2x)\\
                      &= 2\cos^2(x)-2\sin^2(x).
  \end{align*}
  Substituting, we get
  \begin{align*}
    -2\sin^2(x) + 2\cos^2(x) - 2p(x)\cos(x)\sin(x) + q(x)\cos^2(x) &= 0\\
    2\sin^2(x) + 2\cos^2(x) + 2p(x)\cos(x)\sin(x) + q(x)\sin^2(x) &= 0.
  \end{align*}
  If we add these two equations together, we get $q(x) = 0$, meaning that a constant is a solution --- note that $\cos^2(x) + \sin^2(x) = 1$, so we know this is fair.\newline

  If we subtract the second equation from the first equation, we get
  \begin{align*}
    -4\sin^2(x) + 4\cos^2(x) - 4p(x)\cos(x)\sin(x) &= 0\\
    4\cos(2x) - 2p(x)\sin(2x) &= 0.
  \end{align*}
  Thus, we want
  \begin{align*}
    2p(x)\sin(2x) &= 4\cos(2x),
  \end{align*}
  giving
  \begin{align*}
    p(x) &= 2\cot(2x).
  \end{align*}
  Therefore, our equation is
  \begin{align*}
    \diff{^2y}{x^2} + 2\cot(2x)\diff{y}{x} &= 0.
  \end{align*}
\end{example}
\begin{example}
  Consider
  \begin{align*}
    \diff{\mathbf{x}}{t} &= \begin{pmatrix}1 & -2 \\ 5 & 3\end{pmatrix} \mathbf{x}.
  \end{align*}
  We see that
  \begin{align*}
    \lambda &= 2 \pm 3i\\
    \mathbf{v} &= \begin{pmatrix}-1\\5\end{pmatrix} \pm i \begin{pmatrix}3\\0\end{pmatrix}.
  \end{align*}
  Therefore, we have
  \begin{align*}
    e^{\lambda t} &= e^{2t}\left( \cos\left( 3t \right) + i\sin\left( 3t \right) \right)\\
    e^{\lambda t} \mathbf{v} &= e^{2t}\left( \cos\left( 3t \right) + i\sin\left( 3t \right) \right) \left( \begin{pmatrix}-1\\5\end{pmatrix} + i \begin{pmatrix}3\\0\end{pmatrix} \right)\\
    \mathbf{x}_1(t) &= e^{2t} \begin{pmatrix}-\cos(3t) - 3\sin(3t) \\ 5\cos(3t)\end{pmatrix}\\
    \mathbf{x}_2(t) &= e^{2t} \begin{pmatrix}3\cos(3t) - \sin(3t) \\ 5\sin(3t)\end{pmatrix}.
  \end{align*}
  Now, we want to find the fundamental matrix, $\Psi_A(t)$, and the matrix exponential $e^{At}$.\newline

  We start by finding $\Phi_A(t)$ to give
  \begin{align*}
    \Phi_A(t) &= \begin{pmatrix}-e^{2t}\cos(3t) - 3e^{2t}\sin(3t) & 3e^{2t}\cos(3t)-e^{2t}\sin(3t)\\ 5e^{2t}\cos(3t) & 5e^{2t}\sin(3t)\end{pmatrix}
  \end{align*}
  We have already found the solution
  \begin{align*}
    \mathbf{x}(t) &= \Phi(t) \mathbf{c}.
  \end{align*}
  However, if we want to apply initial conditions, we need $\Psi_A(t)$ to obtain $\mathbf{x}(t) = \Psi(t)\mathbf{x}_0$.
  \begin{align*}
    \Psi_A(t) &= \Phi(t)\Phi_A^{-1}(0)\\
              &= \begin{pmatrix}-e^{2t}\cos(3t) - 3e^{2t}\sin(3t) & 3e^{2t}\cos(3t)-e^{2t}\sin(3t)\\ 5e^{2t}\cos(3t) & 5e^{2t}\sin(3t)\end{pmatrix} \begin{pmatrix}-1 & 3 \\ 5 & 0\end{pmatrix}^{-1}.
              \intertext{Notice that $\Phi_A(0)$ has columns equal to the real and imaginary eigenvectors.}
              &= -\frac{1}{15} \begin{pmatrix}-e^{2t}\cos(3t) - 3e^{2t}\sin(3t) & 3e^{2t}\cos(3t)-e^{2t}\sin(3t)\\ 5e^{2t}\cos(3t) & 5e^{2t}\sin(3t)\end{pmatrix} \begin{pmatrix}0 & -3 \\ -5 & -1\end{pmatrix}.
  \end{align*}
  After many error-prone computations, we obtain
  \begin{align*}
    \Psi_A(t) &= \begin{pmatrix} e^{2t}\cos(3t) - \frac{1}{3}e^{2t}\sin(3t) & -\frac{2}{3}e^{2t}\sin(3t) \\ \frac{5}{3}e^{2t}\sin(3t) & e^{2t}\cos(3t) + \frac{1}{3}e^{2t}\sin(3t) \end{pmatrix}.
  \end{align*}
  To ensure that this is a (plausible) $\Psi(t)$, we check $\Psi_A(0)$, and see that $\Psi_A(0) = I$. Also, it can be verified\footnote{I don't think this is a fun thing to do, but it can indeed be verified.} that
  \begin{align*}
    \diff{\Psi_A}{t} &= A \Psi_A(t).
  \end{align*}
  Now, finding $\Psi_A$ was quite difficult. What if there's a better way?\newline

  Recall from the eigenvector with conjugate eigenvalues $\lambda = a \pm ib$ equation that
  \begin{align*}
    A \left( \mathbf{u} + i\mathbf{w} \right) &= \left( e^{at}\cos\left( bt \right) + ie^{at}\sin\left( bt \right)\right)\left( \mathbf{u} + i\mathbf{w} \right).
  \end{align*}
  Now, examining the real part, we have
  \begin{align*}
    A\mathbf{u} &= e^{at}\cos\left( bt \right)\mathbf{u} - e^{at}\sin\left( bt \right) \mathbf{w}\\
    A \mathbf{w} &= e^{at}\cos\left( bt \right)\mathbf{w} + e^{at}\sin\left( bt \right) \mathbf{u}.
  \end{align*}
  We will construct a similarity transform
  \begin{align*}
    AP &= PD.
  \end{align*}
  Now, we may take
  \begin{align*}
    P &= \begin{pmatrix}\re\left( \mathbf{v} \right) & \im\left( \mathbf{v} \right)\end{pmatrix}\\
      &= \begin{pmatrix}\mathbf{u} & \mathbf{w}\end{pmatrix}.
  \end{align*}
  Therefore, we have
  \begin{align*}
    D &= \begin{pmatrix}a & b \\ -b & a\end{pmatrix}.
  \end{align*}
  We want to find $e^{At} = Pe^{Dt}P^{-1}$. Note that, after much tedious calculation (or noticing that our vector $D$ is a matrix expression for the complex number $a + bi$), we obtain
  \begin{align*}
    e^{Dt} &= \begin{pmatrix}e^{at}\cos\left( bt \right) & e^{at}\sin\left( bt \right)\\-e^{at}\sin\left( bt \right) & e^{at}\cos\left( bt \right)\end{pmatrix}.
  \end{align*}
\end{example}
\subsection{Inhomogeneous Systems of Equations}%
Now, we discuss inhomogeneous systems. Consider the system
\begin{align*}
  \diff{\mathbf{x}}{t} &= A\mathbf{x} + \mathbf{F}(t).
\end{align*}
We look at the homogeneous system,
\begin{align*}
  \diff{\mathbf{x}}{t} &= A\mathbf{x},
\end{align*}
which has the solutions of
\begin{align*}
  \mathbf{x}(t) &= \Phi(t) \mathbf{c}\\
                &= e^{At}\mathbf{x}_0.
\end{align*}
When we deal with inhomogeneous solutions in the one-dimensional case, we use methods like variation of parameters or undetermined coefficients.\newline

Here, we will assume a perturbation --- i.e., our solution is close enough to our fundamental solution.
\begin{align*}
  \mathbf{x}(t) &= \Phi(t)\mathbf{u}(t)\\
  \diff{\mathbf{x}}{t} &= \diff{\Phi}{t}\mathbf{u}(t) + \Phi(t)\diff{\mathbf{u}}{t}.
\end{align*}
Plugging in, we have
\begin{align*}
  \diff{\Phi}{t}\mathbf{u}(t) + \Phi(t)\diff{\mathbf{u}}{t} &= A\Phi(t)\mathbf{u}(t) + \mathbf{F}(t).
\end{align*}
We want to cancel some things out. Recall that $\diff{\Phi}{t} = A\Phi$. Therefore, we obtain
\begin{align*}
  \Phi(t) \diff{\mathbf{u}}{t} &= \mathbf{F}(t).
\end{align*}
Multiplying on both sides by $\Phi^{-1}(t)$, we get
\begin{align*}
  \diff{\mathbf{u}}{t} &= \Phi^{-1}(t)\mathbf{F}(t)\\
  \mathbf{u}(t) &= \int_{}^{} \Phi^{-1}(t)\mathbf{F}(t)\:dt,
\end{align*}
where the integral is taken coordinatewise. Thus,
\begin{align*}
  \mathbf{x}_p(t) &= \Phi(t) \int_{}^{} \Phi^{-1}(t)\mathbf{F}(t)\:dt.
\end{align*}
Therefore, presumably, we have the general solution
\begin{align*}
  \mathbf{x}(t) &= \Phi(t) \mathbf{c} + \Phi(t) \int_{}^{} \Phi^{-1}(t)\mathbf{F}(t)\:dt.
\end{align*}
Instead, we may consider $\Psi_A(t)$ rather than $\Phi(t)$, as $\Psi_A^{-1}(t) = \Psi\left( -t \right)$. This gives
\begin{align*}
  \mathbf{x}(t) &= \Psi_A(t)\mathbf{c} + \Psi_A\left( t \right)\int_{}^{} \Psi_A\left( -t \right)F(t)\:dt.
\end{align*}
Now, if we want to implement the condition $\mathbf{x}_0 = \mathbf{x}(0)$, we want to modify the integral slightly.
\begin{align*}
  \mathbf{x}(t) &= \Psi_A(t)\mathbf{c} + \Psi_A(t) \int_{0}^{t} \Psi_A(-s)\mathbf{F}(s)\:ds\\
  \mathbf{x}(0) &= \mathbf{c}
  \intertext{meaning}
  \mathbf{x}(t) &= \Psi_A(t)\mathbf{x}_0 + \Psi_a(t) \int_{0}^{t} \Psi_A(-s)\mathbf{F}(s)\:dt.
\end{align*}
\begin{example}[Solving an Inhomogeneous System]
  Consider the system
  \begin{align*}
    \diff{x}{t} &= 3x + y + 2\\
    \diff{y}{t} &= x + 3y + e^{7t}.
  \end{align*}
  Writing in matrix form, we have
  \begin{align*}
    \diff{\mathbf{x}}{t} &= \begin{pmatrix}3 &1 \\ 1 & 3\end{pmatrix}\mathbf{x} + \begin{pmatrix}2\\e^{7t}\end{pmatrix}.
  \end{align*}
  Note that we don't have to use an initial condition, so we only need the formula
  \begin{align*}
    \mathbf{x}(t) &= \Psi(t)\mathbf{c} + \Psi(t) \int_{}^{} \Psi(-t)\mathbf{F}(t)\:dt
  \end{align*}
  Now, note that
  \begin{align*}
    \Psi(t) &= \begin{pmatrix}\frac{1}{2}\left( e^{4t} + e^{2t} \right) & \frac{1}{2}\left( e^{4t} - e^{2t} \right) \\ \frac{1}{2}\left( e^{4t} - e^{2t} \right) & \frac{1}{2}\left( e^{4t} + e^{2t} \right)\end{pmatrix}.
  \end{align*}
  Calculating the inverse, we have
  \begin{align*}
    \Psi^{-1}(t) &= \Psi\left( -t \right)\\
                 &= \begin{pmatrix}\frac{1}{2}\left( e^{-4t} + e^{-2t} \right) & \frac{1}{2}\left( e^{-4t} - e^{-2t} \right) \\ \frac{1}{2}\left( e^{-4t} - e^{-2t} \right) & \frac{1}{2}\left( e^{-4t} + e^{-2t} \right)\end{pmatrix}.
  \end{align*}
  Computing, we have
  \begin{align*}
    \Psi^{-1}\mathbf{F}(t) &= \begin{pmatrix}e^{-4t} + e^{-2t} + \frac{1}{2}\left( e^{3t} - e^{5t} \right) \\ e^{-4t} - e^{-2t} + \frac{1}{2}\left( e^{3t} + e^{5t} \right)\end{pmatrix}.
  \end{align*}
  Integrating coordinatewise, we get
  \begin{align*}
    \int_{}^{} \Psi^{-1}\mathbf{F}(t)\:dt &= \begin{pmatrix}- \frac{1}{4}e^{-4t} - \frac{1}{2}e^{-2t} + \frac{1}{6}e^{3t} - \frac{1}{10}e^{5t} \\ -\frac{1}{4}e^{-4t} + \frac{1}{2}e^{-2t} + \frac{1}{6}e^{3t} + \frac{1}{10}e^{5t}\end{pmatrix}
  \end{align*}
  We find the general solution by multiplying by $\Psi(t)$ and adding the homogeneous solution.
\end{example}
\begin{example}[Nonconstant Coefficients]
  Consider
  \begin{align*}
    \diff{x}{t} &= \left( 2t + 1 \right)x + y\\
    \diff{y}{t} &= x + \left( 2t + 1 \right).
  \end{align*}
  Writing in matrix form, we have
  \begin{align*}
    \diff{\mathbf{x}}{t} &= \begin{pmatrix}2t + 1 & 1 \\ 1 & 2t + 1\end{pmatrix},
  \end{align*}
  with eigenvalues
  \begin{align*}
    \lambda_1 &= 2t+2\\
    \lambda_2 &= 2t\\
    \mathbf{v}_1 &= \begin{pmatrix}1\\1\end{pmatrix}\\
    \mathbf{v}_2 &= \begin{pmatrix}1\\-1\end{pmatrix}.
  \end{align*}
  Solving, we can take
  \begin{align*}
    e^{At} &= Pe^{Dt}P^{-1}\\
           &= \begin{pmatrix}1 & 1 \\ 1 & -1\end{pmatrix} e^{Dt} \left( \frac{1}{2} \begin{pmatrix}1 & 1 \\ 1 & -1\end{pmatrix} \right).
  \end{align*}
  Now, calculating $e^{Dt}$, we need to take integrals of our expressions $\lambda_1(t)$ and $\lambda_2(t)$ for our exponent expressions, giving
  \begin{align*}
    e^{At} &= \frac{1}{2} \begin{pmatrix}1 & 1 \\ 1 & -1\end{pmatrix} \begin{pmatrix}e^{\int_{}^{} \lambda_1(t)\:dt} & 0 \\ 0 & e^{\int_{}^{} \lambda_2(t)\:dt}\end{pmatrix} \begin{pmatrix}1 & 1 \\ 1 & -1\end{pmatrix}\\
           &= \frac{1}{2} \begin{pmatrix}1 & 1 \\ 1 & -1\end{pmatrix} \begin{pmatrix}e^{t^2 + 2t} & 0 \\ 0 & e^{t^2}\end{pmatrix} \begin{pmatrix}1 & 1 \\ 1 & -1\end{pmatrix}.
  \end{align*}
  Therefore,
  \begin{align*}
    \mathbf{x}(t) &= Pe^{Dt}P^{-1} \mathbf{x}_0\\
                  &= \frac{1}{2} \begin{pmatrix}e^{t^2 + 2t} + e^{t^2} & e^{t^2 + 2t} - e^{t^2} \\ e^{t^2 + 2t} - e^{t^2} & e^{t^2 + 2t} + e^{t^2}\end{pmatrix} \mathbf{x}_0.
  \end{align*}
\end{example}
\section{Introducing Partial Differential Equations}%
Consider the equation
\begin{align*}
  \pd{u}{t} + \pd{u}{x} &= 0.\label{eq:pde_1}\tag{\textasteriskcentered}
\end{align*}
There are a lot of solutions to this equation.
\begin{itemize}
  \item $u(x,t) = 5$;
  \item $u(x,t) = x-t$;
  \item $u(x,t) = e^{x}e^{-t}$.
\end{itemize}
We want to try to understand solutions to this equation given an initial value. Note that this means that the ``initial values'' are actually functions of $x$. Upon defining our domains in $x$ and $t$, we have an initial condition of the form
\begin{align*}
  u\left( x,0 \right) &= f(x).
\end{align*}
Now, considering \eqref{eq:pde_1}, we may consider the initial condition
\begin{align*}
  u\left( x,0 \right) &= \sin(x).
\end{align*}
Then, all our proposed solutions from earlier no longer apply. Using the power of inspection, we may consider the solution
\begin{align*}
  u\left( x,t \right) &= \sin(x-t)\\
  \pd{u}{t} &= -\cos\left( x-t \right)\\
  \pd{u}{x} &= \cos\left( x-t \right),
\end{align*}
meaning that our solution for $u\left( x,t \right)$ works.\newline

This gives an initial value problem of
\begin{align*}
  \pd{u}{t} + a\pd{u}{x} &= 0\\
  u\left( x,0 \right) &= f(x).
\end{align*}
We now need to know if there is an existence/uniqueness condition for partial differential equations. Consider the proposed solution
\begin{align*}
  u\left( x,t \right) &= f\left( x-at \right)\\
  \pd{u}{x} &= f\left(x-at\right)\\
  \pd{u}{t} &= -af\left( x-at \right),
\end{align*}
which solves our equation. The main condition on $f$ here is that $f$ has to be differentiable.\newline

We call \eqref{eq:pde_1} the \textit{transport equation}.
\begin{example}
  Consider
  \begin{align*}
    A \pd{^2u}{x^2} + B \pd{^2u}{x\partial y} + C \pd{^2u}{y^2} + D \pd{u}{x} + E \pd{u}{y} + F u &= G,
  \end{align*}
  where $A,B,C,D,E,F,G$ are functions of $x$ and $y$.\newline

  This is the most general second-order two-variable inhomogeneous linear partial differential equation.
\end{example}
\begin{example}
  In the case of \eqref{eq:pde_1}, we see that $A = B = C = F + G = 0$. For the transport equation, we defined the initial condition $u\left( x,0 \right) \coloneq u_0\left( x \right)$, and a domain $-\infty < x < \infty$, $t \geq 0$.\newline

  The transport equation is a homogeneous first-order linear constant-coefficient partial differential equation, which is much easier to solve.\newline

  Equations of the type seen in the transport equation, which admit only initial conditions (and no boundary conditions) are known as Cauchy problems.
\end{example}
\begin{example}
  The equation
  \begin{align*}
    \pd{u}{t} + u \pd{u}{x} &= 0
  \end{align*}
  is a \textit{nonlinear} homogeneous first-order partial differential equation, known as Burgers' equation.
\end{example}
\begin{definition}
  Consider the general partial differential equation(s) of the form
  \begin{align*}
    A(x,y) \pd{^2u}{x^2} + B(x,y) \pd{^2u}{x\partial y} + C(x,y)\pd{^2 u}{y^2} &= 0.
  \end{align*}
  \begin{itemize}
    \item If $\left( B(x,y) \right)^2 - 4\left( A\left( x,y \right)C\left( x,y \right) \right) > 0$, then we call this equation a hyperbolic partial differential equation.
    \item If $\left( B(x,y) \right)^2 - 4\left( A\left( x,y \right)C\left( x,y \right) \right) = 0$, then we call this equation a parabolic partial differential equation.
    \item If $\left( B(x,y) \right)^2 - 4\left( A\left( x,y \right)C\left( x,y \right) \right) < 0$, then we call this equation an elliptic partial differential equation.
  \end{itemize}
\end{definition}
\begin{example}[The Heat Equation]
  Consider the equation
    \begin{align*}
      \pd{u}{t} &= k \pd{^2u}{x^2}.
    \end{align*}
    This is known as the heat equation (in one dimension). In multiple dimensions, the heat equation would have the form
    \begin{align*}
      \pd{u}{t} &= \sum_{i} k_i\pd{^2u}{x_i^2}.
    \end{align*}
    We will focus on solving the equation in one dimension.\newline

    For example, the following equations are solutions to the heat equation:
    \begin{align*}
      u\left( x,t \right) &= 0\\
      u\left( x,t \right) &= 3x + 5\\
      u\left( x,t \right) &= kt - k\frac{x^2}{2}.
    \end{align*}
    Unfortunately, there are an infinite number of solutions to the heat equation, which is not helpful.\newline

    We implement some conditions in order to solve this equation. We start by implementing the domain $a\leq x \leq b$ and $t\geq 0$.\newline

    Now, we may implement the boundary conditions of $u\left( a,t \right) = c_1$ and $u\left( b,t \right) = c_2$, which allow us to express the fact that we don't want the very ends of our heated rod to change.
\end{example}
\begin{example}[Solving a Heat Equation]
  Consider the heat equation of the form
  \begin{align*}
    \pd{u}{t} &= 3\pd{^2u}{x^2},
  \end{align*}
  with domain $0\leq x \leq \pi$, $t\geq 0$, and initial condition $u\left( x,0 \right) = \sin\left(2x\right)$. The initial conditions yield the boundary condition of $u\left( 0,t \right) = 0$ and $u\left( \pi,t \right) = 0$.\newline

  To solve this equation, we assume\footnote{Also known as guessing.} a solution of the form
  \begin{align*}
    u\left( x,t \right) &= X(x)T(t).
  \end{align*}
  This approach is known as separation of variables.\newline

  Solving using this assumption, we have
  \begin{align*}
    \pd{u}{t} &= X(x)\pd{T}{t}\\
    \pd{^2u}{x^2} &= \pd{^2X}{x^2}T(t).
  \end{align*}
  Thus, we get the equation of the form
  \begin{align*}
    X(x) \pd{T}{t} &= 3 \pd{^2X}{x^2}T(t).
  \end{align*}
  Dividing by $X(x)T(t)$, we have
  \begin{align*}
    \left( \pd{T}{t} \right)\left( \frac{1}{T(t)}  \right) &= 3 \left( \pd{^2X}{x^2} \right)\left( \frac{1}{X(x)} \right)\\
    \frac{1}{3}\left( \pd{T}{t} \right)\left( \frac{1}{T(t)} \right) &= \left( \pd{^2X}{x^2} \right)\left( \frac{1}{X(x)} \right)\\
                                                                     &= C.
  \end{align*}
  Now, examining the equation
  \begin{align*}
    \left( \pd{^2 X}{x^2} \right)\left( \frac{1}{X(x)} \right) &= C,
  \end{align*}
  we see that $X$ has solutions of the form
  \begin{align*}
    X(x) &= \sin\left( \sqrt{C}x \right)\\
         &= \cos\left( \sqrt{C}x \right)\\
         &= \cosh\left( \sqrt{C}x \right)\\
         &= \sinh\left( \sqrt{C}x \right)\\
         &= Ax + B\\
         &= B
  \end{align*}
  Now, we have three classes of solutions. If $C < 0$, we have the trigonometric functions, if $C > 0$, then we have the hyperbolic trigonometric functions, and if $C = 0$, then we have the linear functions. Alternatively, we write $C = \lambda^2$, giving
  \begin{align*}
    \frac{1}{X(x)} \left( \pd{^2X}{x^2} \right) &= \begin{cases}
      \lambda^2\\ -\lambda^2 \\ 0
    \end{cases}.
  \end{align*}
  Examining our cases, we want
  \begin{align*}
    0 &= X(0)T(t)\\
    0 &= X(\pi)T(t).
  \end{align*}
  We cannot have $T(t) = 0$. Thus, we have $X(0) = X(\pi) = 0$.\newline

  We start with the case that $ \pd{^2X}{x^2} = \lambda^2 X $. Thus, our boundary values for 
  \begin{align*}
    \diff{^2X}{x^2} - \lambda^2 X &= 0
  \end{align*}
  are at $X(0) = X(\pi) = 0$. We will use $X = \cosh\left( \lambda x \right)$ and $X = \sinh\left( \lambda x \right)$, rather than the traditional solutions of $e^{\pm \lambda x}$. The general solution is
  \begin{align*}
    X &= A\cosh\left( \lambda x \right) + B\sinh\left( \lambda x \right).
  \end{align*}
  Thus, we have $A = 0$. Furthermore, we have
  \begin{align*}
    0 &= B\sinh\left( \lambda \pi \right).
  \end{align*}
  Unfortunately this gives $B = 0$ too.\newline

  This case has no nontrivial solution.\newline

  Let's try a different case; let $ \pd{^2 X}{x^2} = -\lambda^2 $. Using the power of inspection, we get
  \begin{align*}
    X(x) &= A\cos\left( \lambda x \right) + B\sin\left( \lambda x \right).
  \end{align*}
  Plugging in our boundary conditions of $X(0) = X(\pi) = 0$, we get
  \begin{align*}
    0 &= A\cos\left( 0 \right),
  \end{align*}
  so that $A = 0$. Thus, we have the solution of
  \begin{align*}
    X(x) &= B\sin\left( \lambda x \right).
  \end{align*}
  To satisfy the boundary condition of $X(\pi) = 0$, we need
  \begin{align*}
    A \sin\left( \lambda \pi \right) &= 0,
  \end{align*}
  which only occurs when $\lambda\in \Z$. We write
  \begin{align*}
    X_n(x) &= A_{n}\sin\left( n x \right),
  \end{align*}
  where $\lambda = n$.\newline

  Now, we need to solve $\diff{T}{t} = -3\lambda^2 T(t)$. This gives
  \begin{align*}
    T(t) &= T(0)e^{-3n^2 t}.
  \end{align*}
  In the general solution, we have
  \begin{align*}
    u\left( x,t \right) &= \sum_{n=0}^{\infty}u_n\left( x,t \right)\\
                        &= \sum_{n=0}^{\infty}T(0)A_n\sin\left( n x \right)e^{-3n^2t}\\
                        &= \sum_{n=1}^{\infty}B_n\sin\left( n x \right) e^{-3n^2 t}
  \end{align*}
  Now, to solve for our initial condition, we have
  \begin{align*}
    u\left( x,t \right) &= A_n\sin\left( n x \right)e^{-3n^2 t}\\
    u\left( x,0 \right) &= A_n\sin\left( n x \right)\\
                        &= \sin\left( 2x \right).
  \end{align*}
  Thus, $n = 2$, so
  \begin{align*}
    u\left( x,t \right) &= \sin\left( 2x \right)e^{-12 t}.
  \end{align*}
  The heat effectively escapes through the end.\newline

  However, if we used a different boundary condition,
  \begin{align*}
    \pd{u}{x}\bigr\vert_{(0,t)}  &= 0\\
    \pd{u}{x}\bigr\vert_{(\pi,t)} &= 0,
  \end{align*}
  then this would express the case of insulation, and would yield a cosine instead of a sine.
\end{example}
\begin{example}
  Consider
  \begin{align*}
    \pd{^2u}{x^2} + 2 \pd{u}{x} &= 3 \pd{u}{t}.
  \end{align*}
  This is not the heat equation --- now, there is a first spatial derivative in our differential equation.\newline

  Analogously, if we have a simple harmonic oscillator,
  \begin{align*}
    m\diff{^2x}{t^2} + kx &= 0,
  \end{align*}
  then by adding a first derivative term,
  \begin{align*}
    m \diff{^2x}{t^2} + a \diff{x}{t} + kx &= 0,
  \end{align*}
  then we have a damped solution.\newline

  Analogously, the term $\pd{u}{x}$ represents the damping term. When we solve this, we get
  \begin{align*}
    \pd{^2X}{x^2}T + 2 \pd{X}{x}T &= 3X \pd{T}{t}.
  \end{align*}
  Dividing out by the solution function, we get
  \begin{align*}
    \frac{1}{X} \pd{^2X}{x^2} + \frac{2}{X} \pd{X}{x} &= \frac{3}{T} \pd{T}{t},
  \end{align*}
  which means we have separated this into two ODEs, of the form
  \begin{align*}
    \diff{^2X}{x^2} + 2\diff{X}{x} - CX &= 0\\
    3 \diff{T}{t} - CT &= 0,
  \end{align*}
  giving a damping term in $X$.
\end{example}
\begin{example}[Introducing the Wave Equation]
  Consider the equation
  \begin{align*}
    \pd{^2u}{t^2} - a^2 \pd{^2u}{x^2} &= 0.
  \end{align*}
  This is known as the wave equation. Since it is a hyperbolic PDE, it has two directions of propagation.\footnote{We will discuss more about this in the future.}\newline

  Considering a fake factorization,
  \begin{align*}
    \left( \pd{u}{t} - a\pd{u}{x} \right)\left( \pd{u}{t} + a\pd{u}{x} \right) &= 0,
  \end{align*}
  we see that there are two transport equations here.\newline

  Making it more of a realistic factorization as an expression of composed linear partial differential operators, we have
  \begin{align*}
    \left( \pd{}{t} - a\pd{}{x} \right) \left( \pd{}{t} + a \pd{}{x} \right) u\left( x,t \right) &= 0.
  \end{align*}
\end{example}
\begin{example}[Solving a Wave Equation]
  Consider the wave equation
  \begin{align*}
    \pd{^2u}{t^2} - a^2 \pd{^2u}{x^2} &= 0,
  \end{align*}
  with domain $0 \leq x \leq 2$ and $t\geq 0$. Consider the boundary conditions of
  \begin{align*}
    \pd{u}{x}\biggr\vert_{(0,t)} &= 0\\
    \pd{u}{x}\biggr\vert_{(2,t)} &= 0.
  \end{align*}
  This is known as the Neumann boundary condition, and essentially denotes ``insulation'' in the case of a heat equation.\newline

  For our initial conditions, we set $u\left( x,0 \right) = 0$ and
  \begin{align*}
    \pd{u}{t}\biggr\vert_{(x,0)} &= 3-\cos\left( 2\pi x \right).
  \end{align*}
  We assume via separation of variables that $u\left( x,t \right) = X(x)T(t)$. This gives
  \begin{align*}
    X(x)\diff{^2T}{t^2} &= a^2 \diff{^2X}{x^2}T(t).
  \end{align*}
  Dividing out by $a^2X(x)T(t)$, we have
  \begin{align*}
    \frac{1}{a^2T(t)} \diff{^2T}{t^2} &= \diff{^2X}{x^2}\\
                                      &= \begin{cases}
                                        \lambda^2\\0\\-\lambda^2
                                      \end{cases}.
  \end{align*}
  Now, considering our two boundary conditions, we have
  \begin{align*}
    \pd{u}{x}\biggr\vert_{(0,t)} &= \diff{X}{x}\biggr\vert_{0}T(t)\\
    \pd{u}{x}\biggr\vert_{(2,t)} &= \diff{X}{x}\biggr\vert_{2}T(t)\\
                                 &= 0.
  \end{align*}
  Note that if $T(t) = 0$, then our solution is trivial, and does not solve our initial condition. Thus, we must have $\diff{X}{x}\biggr\vert_{0} = \diff{X}{x}\biggr\vert_{2} = 0$. Therefore, we have the initial value problem(s) of
  \begin{align*}
    \diff{^2X}{x^2} &= \begin{cases}
      \lambda^2 X\\ 0 \\ -\lambda^2 X
    \end{cases},
  \end{align*}
  with the given initial condition. In the case of $\diff{^2X}{x^2} = \lambda^2 X$, we get the solution of
  \begin{align*}
    X(x) &= A\cosh\left( \lambda x \right) + B\sinh\left( \lambda x \right)\\
    \diff{X}{x} &= \lambda A \sinh\left( \lambda x \right) + \lambda B\cosh\left( \lambda x \right)\\
    \diff{X}{x}\biggr\vert_{0} &= 0\\
                               &= \lambda B,
  \end{align*}
  so $B = 0$. Implementing our other boundary condition, we have
  \begin{align*}
    \diff{X}{x}\biggr\vert_{2} &= \lambda A \sinh\left( 2\lambda \right)\\
                               &= 0.
  \end{align*}
  Since $\sinh(2\lambda)\neq 0$ and $\lambda\neq 0$, we get $A = 0$.\newline

  In the case with $\diff{^2X}{x^2} = 0$, we have $x = ax + b$. However, since $\diff{X}{x} = A$, we have $a = 0$. Therefore, the function $X(x) = b$ satisfies the boundary condition.\newline

  When $\diff{^2 X}{x^2} = -\lambda^2 X$, we have the solution of
  \begin{align*}
    X(x) &= a\sin\left( \lambda x \right) + b\cos\left( \lambda x \right)\\
    \diff{X}{x} &= \lambda a\cos\left( \lambda x \right) - \lambda b \sin\left( \lambda x \right).
  \end{align*}
  Plugging our first initial condition, we have
  \begin{align*}
    0 &= \lambda a,
  \end{align*}
  so $a = 0$. Now, plugging in our second initial condition, we have
  \begin{align*}
    0 &= -\lambda b \sin\left( 2\lambda \right).
  \end{align*}
  Since $\lambda\neq 0$, we must have $\lambda = \frac{n}{2}\pi$, when $n\in \Z_{>0}$. Thus, our expression for $X_n(x)$ when $n\in \Z_{> 0}$ is
  \begin{align*}
    X_n(x) &= B_n\cos\left( \frac{n\pi x}{2} \right),
  \end{align*}
  and when $n = 0$, we have $X_0(x) = 1$.\newline

  Shifting our focus to $T$, in the case with $\lambda = 0$, we have $\diff{^2T}{t^2} = 0$, so $T = Ct + D$. Meanwhile, if $\lambda\neq 0$, we have
  \begin{align*}
    \diff{^2 T}{t^2} &= -\lambda^2 a^2 T(t),
    \intertext{meaning}
    T_0(t) &= C_0 t + D_0\\
    T_n(t) &= C_n\cos\left( \frac{n\pi a}{2} t \right) + D_n \sin\left( \frac{n\pi a}{2}t \right).
  \end{align*}
  Putting all equations together, we have
  \begin{align*}
    u\left( x,t \right) &= X_0(x)T_0(t) + \sum_{n=1}^{\infty} X_n(x)T_n(t)\\
                        &= B_0\left( C_0 t + D_0 \right) + \sum_{n=1}^{\infty}B_n\cos\left( \frac{n\pi x}{2} \right)\left( C_n\sin\left( \frac{n\pi a t}{2} \right) + D_n\cos\left( \frac{n\pi a t}{2} \right) \right)\\
                        &= C_0 t + D_0 + \sum_{n=1}^{\infty}B_n\cos\left( \frac{n\pi x}{2} \right) + C_n\sin\left( \frac{n\pi a t}{2} \right) + \sum_{n=1}^{\infty} B_n\cos\left( \frac{n\pi x}{2} \right) + D_n\cos\left( \frac{n\pi a t}{2} \right).
  \end{align*}
  Using the initial condition of $u\left( x,0 \right) = 0$, we get $D_n = D_0 = 0$ for all $n$. Thus, refining our solution, we have
  \begin{align*}
    u\left( x,t \right) &= C_0 t + \sum_{n=1}^{\infty}K_n\sin\left( \frac{n\pi a t}{2} \right)\cos\left( \frac{n\pi x}{2} \right).
  \end{align*}
  Plugging in the other initial conditions, we have
  \begin{align*}
    u\left( x,t \right) &= 3t - \frac{1}{2\pi a}\cos\left( 2\pi x \right)\sin\left( 2\pi a t \right)
  \end{align*}
  
\end{example}
\begin{example}
  Consider yet again
  \begin{align*}
    \pd{u}{t} &= \pd{^2 u}{x^2},
  \end{align*}
  with domain $0 \leq x \leq 1$ and $t\geq 0$. However, we use initial conditiosn of $u\left( 0,t \right) = u\left( 1,t \right) = 0$, and $u\left( x,0 \right) = x-x^2$.\newline

  In the case of $-\lambda^2$, we get the solution of
  \begin{align*}
    X_n(x) &= \sin\left( n\pi x \right)\\
    T_n\left( t \right) &= e^{-n^2\pi^2 t},
  \end{align*}
  giving a general solution of
  \begin{align*}
    u\left( x,t \right) &= \sum_{n=1}^{\infty}c_n\sin\left( n\pi x \right)e^{-n^2\pi^2 t}.
  \end{align*}
  Given our initial condition, we must solve
  \begin{align*}
    x-x^2 &= \sum_{n=0}^{\infty}c_n\sin\left( n\pi x \right),
  \end{align*}
  and find the $c_n$ that satisfy this initial condition.\newline

  Recall from linear algebra that two vectors $\mathbf{v}$ and $\mathbf{w}$ are \textit{orthogonal} if their dot product (or inner product) is equal to zero. Similarly, we can talk about functions being orthogonal (with respect to a certain weight), by saying that $f$ and $g$ are orthogonal if
  \begin{align*}
    \int_{a}^{b} f(x)g(x)w(x)\:dx &= 0,
  \end{align*}
  where we assume $f$ and $g$ are real-valued.\newline

  For instance, if we define $f_0(x) = 1$ and $f_1(x) = x$, along the interval $[-1,1]$ and weight $w(x) = 1$, we get the legendre polynomials. Note that
  \begin{align*}
    \int_{-1}^{1} x\:dx &= 0.
  \end{align*}
  Similarly,
  \begin{align*}
    f_2 &= \frac{3x^2 - 1}{2}\\
    f_3 &= \frac{5x^3 - 3x}{2}
  \end{align*}
  are orthogonal to each other (these are the Legendre polynomials, see \href{https://ai.avinash-iyer.com/Classes_and_Homework/College/Y4/Y4S1,\%20Math\%20Methods/math_methods_notes.pdf}{Math Methods Notes}).
\end{example}

\end{document}
