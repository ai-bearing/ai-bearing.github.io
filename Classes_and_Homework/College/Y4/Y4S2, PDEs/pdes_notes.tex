\documentclass[10pt]{mypackage}

% sans serif font:
%\usepackage{cmbright,sfmath,bbold}
%\renewcommand{\mathcal}{\mathtt}

%Euler:
\usepackage{newpxtext,eulerpx,eucal,eufrak}
\renewcommand*{\mathbb}[1]{\varmathbb{#1}}
\renewcommand*{\hbar}{\hslash}

%kp fonts:
%\usepackage{kpfonts}
%\renewcommand{\mathbb}{\mathds}

\pagestyle{fancy} %better headers
\fancyhf{}
\rhead{Avinash Iyer}
\lhead{Partial Differential Equations: Class Notes}

\setcounter{secnumdepth}{0}

\begin{document}
\renewcommand{\arraystretch}{1.5}
\RaggedRight
\section{Introduction}%
Consider the equations
\begin{align*}
  y''(x) + y(x) &= e^x\tag*{(1)}\\
  y^{(17)}(x) + \sin\left(y(x)\right) &= \left(x^{x}\right)^{x}\tag*{(2)}
\end{align*}
Before we want to solve these equations, we need to understand what these equations \textit{are}. 
\begin{enumerate}[(1)]
  \item This is a second order, inhomogeneous, linear ordinary differential equation.
  \item This is a 17th order, inhomogeneous, nonlinear ordinary differential equation.
\end{enumerate}
Generally, when we have a nonlinear equation, we convert it (using the Jacobian) to the ``nearest'' corresponding linear equation using Taylor approximations. In this case, converting equation (2), we have
\begin{align*}
  y^{(17)}(x) + y(x) &= \left(x^{x}\right)^{x}.\tag*{(2')}
\end{align*}
Now, equation (2') is linear, so it is able to be solved. It may not be pretty,\footnote{Citation needed.} but it can be solved, using Laplace Transforms or other methods.
\section{Ordinary Differential Equations}%
Returning to our equation (1), 
\begin{align*}
  y''(x) + y(x) &= e^x,\tag*{(1)}
\end{align*}
there is one more fact that we can see --- this is an equation with constant coefficients. The most general form of a $n$th order linear ordinary differential equation is of the form
\begin{align*}
  a_n(x)y^{(n)}(x) + a_{n-1}(x)y^{(n-1)}(x) + \cdots + a_1(x)y'(x) + a_0(x)y(x) = g(x).\label{eq:general_linear_ode}\tag{\textdagger}
\end{align*}
Specifically, we also require $a_k(x)\in C(I)$, where $I$ is some interval (specifics will be detailed later).
\begin{theorem}[Existence and Uniqueness Theorem]
  Any ordinary differential equation of the form (\textdagger) has unique solutions in the interval $I$.\newline

  There are $n$ linearly independent solutions for $g(x) = 0$.
\end{theorem}
The corresponding homogeneous equation for (1) is
\begin{align*}
  y''(x) + y(x) &= 0.\tag*{(1')}
\end{align*}
The equations (1) and (1') are related by the linearity principle. In particular, if $y_0(x)$ is a solution to (1'), then we can add $\alpha y_0(x)$ to any solution $y_p(x)$ of (1), then we have all the solutions for (1). In particular, the solutions to (1') are
\begin{align*}
  y_1(x) &= \sin(x)\\
  y_2(x) &= \cos(x).
\end{align*}
To evaluate that these solutions are linearly independent, we consider the differential operator $L$ from (\textdagger) defined by
\begin{align*}
  L\left[y\right] &= \sum_{k=0}^{n}a_k(x)y^{(k)}(x).
\end{align*}
We rewrite (\textdagger) as
\begin{align*}
  L\left[y\right] &= g(x).
\end{align*}
The operator $L$ is linear, so $L$ has the following properties:
\begin{itemize}
  \item $L\left[y_1 + y_2\right]$;
  \item $L\left[cy\right] = cL\left[y\right]$.
\end{itemize}
Now, in (1) and (1'), if we set $L\left[y\right] = y''(x) + y(x)$, then evaluating our solutions $y_1$ and $y_2$ to (1'), we get
\begin{align*}
  L\left[c_1y_1 + c_2y_2\right] &= c_1L\left[y_1\right] + c_2L\left[y_2\right]\\
                                &= 0.
\end{align*}
Now, we get
\begin{align*}
  y_0(x) &= c_1\sin(x) + c_2\sin(x)
\end{align*}
as our general solution to (1'). By the linearity principle, all we need is one solution to $L\left[y\right] = e^x$ to find all solutions to (1).\newline

Evaluating \eqref{eq:general_linear_ode} in the most general form, we have the general solution
\begin{align*}
  y(x) &= \underbrace{c_1y_1(x) + c_2y_2(x) + \cdots + c_ny_n(x)}_{\text{homogeneous solution}} + y_p(x),
\end{align*}
where $y_p(x)$ is the particular solution. In other words, our general solution is
\begin{align*}
  y(x) &= \Span\left(y_1(x),y_2(x),\dots,y_n(x)\right) + y_p(x).
\end{align*}
For this to work, we need the set $\set{y_1,\dots,y_n}$ to be linearly independent. To do this, we evaluate the Wronskian:
{
  \begin{align*}
  W(x) &= \det \begin{pmatrix}y_1(x) & y_2(x) & \cdots & y_n(x) \\ y_1'(x)& y_2'(x) & \cdots & y_n'(x) \\ \vdots & \vdots & \ddots & \vdots \\ y_{1}^{(n-1)}(x) & y_2^{(n-1)}(x) & \cdots & y_{n}^{(n-1)}(x)\end{pmatrix}.
\end{align*}
}
Specifically, the set $\set{y_1,\dots,y_n}$ is linearly independent if $W(x)\neq 0$ for all $x\in I$.
\begin{example}
  Consider the equation
  \begin{align*}
    y''(x) - y(x) &= e^{x}\tag{1}
  \end{align*}
  We want to find the general solution to this constant coefficient equation.\newline

  We start by finding two linearly independent homogeneous solutions to the equation, take their span, then add a particular solution.\newline

  The characteristic equation of the homogeneous equation for (1) is
  \begin{align*}
    r^2 - 1 &= 0
  \end{align*}
  We get $r=\pm 1$, which by the definition of the characteristic equation yields $y_1(x) = e^{x}$ and $y_2(x) = e^{-x}$. To verify that this this solution set is linearly independent
  {\renewcommand{\arraystretch}{1.25}
    \begin{align*}
      W(x) &= \det \begin{pmatrix}e^{x} & e^{-x} \\ e^{x} & -e^{-x}\end{pmatrix}\\
           &= -2\\
           &\neq 0.
  \end{align*}
  }
  Thus, our solutions are linearly independent. We get the general form of
  \begin{align*}
    y(x) &= c_1e^{x} + c_2e^{-x} + y_p(x).
  \end{align*}
  Now, we only have to find a particular solution. This is, unfortunately, the hard part.\newline

  We begin by guessing. But, in a way that doesn't suck. Specifically, we let $y_p(x) = Axe^{x}$. Evaluating, we get
  \begin{align*}
    y_p'(x) &= A\left(x+1\right)e^{x}\\
    y_{p}''(x) &= A\left(x+2\right)e^{x}\\
    y_{p}''(x) - y_p(x) &= A\left(x+2\right)e^{x} - Axe^{x}\\
                        &= 2Ae^{x},
  \end{align*}
  so $2A = 1$, and $A = \frac{1}{2}$. Thus, we have the end result of
  \begin{align*}
    y(x) &= c_1e^{x} + c_2e^{x} + \frac{1}{2}xe^{x}.
  \end{align*}
  Evaluating in Mathematica, we take
  \begin{lstlisting}[style=mathematicastyle]
    DSolve[y''[x] - y[x] == Exp[x], y[x], x]
  \end{lstlisting}
  and we get
  \begin{align*}
    y(x) &= c_1e^{x} + c_2e^{-x} + \frac{1}{4}\left(2x-1\right)e^{x},
  \end{align*}
  corroborating our solution.\footnote{Only slightly different, but they're the same solution.}
\end{example}
\begin{example}
  Consider the equation
  \begin{align*}
    y'''(x) - y(x) &= 0.
  \end{align*}
  The particular solution to this equation is $y(x) = 0$. The characteristic equation for this equation is
  \begin{align*}
    r^3 - 1 &= 0.
  \end{align*}
  Factoring, we get
  \begin{align*}
    \left(r-1\right)\left(r^2 + r + 1\right) &=0\\
    \left(r-1\right)\left(r-\zeta_3\right)\left(r-\zeta_3^2\right) &= 0.
  \end{align*}
  Thus, we get
  \begin{align*}
    r &= \set{1, e^{\frac{2\pi i}{3}}, e^{\frac{4\pi i}{3}}}.
  \end{align*}
  Thus, our solutions are of the form
  \begin{align*}
    y(x) &= c_1e^{x} + c_2e^{-\frac{1}{2}x}\cos\left(\frac{\sqrt{3}}{2}x\right) + c_3e^{-\frac{1}{2}x}\sin\left(\frac{\sqrt{3}}{2}x\right).
  \end{align*}
\end{example}
Recall that the most general second order constant-coefficient linear differential equation is 
\begin{align*}
  y'' + ay' + by &= 0,
\end{align*}
with characteristic equation 
\begin{align*}
  r^2 + ar + b &=0.
\end{align*}
The solutions to the characteristic equation are
\begin{align*}
  r &= -\frac{a}{2} \pm \frac{\sqrt{a^2 - 4b}}{2}.
\end{align*}
There are a few cases:
\begin{enumerate}[(1)]
  \item $r_1\neq r_2$ with $r_1,r_2\in \R$;
  \item $r_1 = r_2$ with $r_1,r_2\in \R$;
  \item $r_1 = c + id$, $r_2 = c - id$, where $c,d\in \R$.
\end{enumerate}
The solutions are $y_1 = c_1e^{r_1 x}$ and $y_2 = c_2e^{r_2 x}$.
\begin{example}[Solving Second-Order Equations]\hfill
  \begin{enumerate}[(1)]
    \item Let
      \begin{align*}
        y'' - 3y' + 2y &= 0.
      \end{align*}
      The characteristic equation is $r^2 - 3r + 2 = 0$, whose solutions are $r=1,r=2$. The general solution is, thus,
      \begin{align*}
        y(x) &= c_1 e^x + c_2e^{2x}\tag{\textdagger}
      \end{align*}
      The Wronskian is
      \begin{align*}
        W(x) &= \det \begin{pmatrix}e^x & e^{2x} \\ e^x & 2e^{2x}\end{pmatrix}\\
             &= 2e^{3x} - e^{3x}\\
             &= e^{3x}\\
             &\neq 0.
      \end{align*}
      Thus, the solution is indeed (\textdagger).
    \item Let
      \begin{align*}
        y'' + 6y' + 9y &=0.
      \end{align*}
      The characteristic equation is $r^2 + 6r + 9 =0$, with solution $r = -3,-3$. Currently, we only have the solution $y_1(x) = c_1e^{-3x}$.\newline

      Note that in an $n$th order linear ordinary differential equation, we always have $n$ linearly independent solutions. Let's guess. Consider the equation $y_2(x) = c_2xe^{-3x}$.\newline

      We can see that $y_2(x)$ is also a solution to this equation,\footnote{Exercise left for the reader.} but we need to verify linear independence. Taking the Wronskian, we get
      \begin{align*}
        W(x) &= \det \begin{pmatrix}e^{-3x} & xe^{-3x} \\ -3e^{-3x} & -3xe^{-3x} + e^{-3x}\end{pmatrix}\\
             &= e^{-6x} \begin{pmatrix}1 & x \\ -3 & -3x + 1\end{pmatrix}\\
             &= e^{-6x}\left(-3x + 1 + 3x\right)\\
             &= e^{-6x}\\
             &\neq 0.
      \end{align*}
      Thus, we have two linearly independent solutions, with the general solution of
      \begin{align*}
        y(x) &= c_1e^{-3x} + c_2xe^{-3x}.
      \end{align*}
    \item Let
      \begin{align*}
        y'' + 4y' + 5 &= 0.
      \end{align*}
      The characteristic equation is $r^2 + 4r + 5 = 0$, with solutions of $r = -2 \pm i$. We then have the solutions
      \begin{align*}
        y_1(x) &= e^{\left(-2 + i\right)x}\\
        y_2(x) &= e^{\left(-2-i\right)x}.
      \end{align*}
      Unfortunately, we cannot just let these equations stand on their own, because we want \textit{real} solutions. Let's use Euler's theorem, $e^{ix} = \cos x + i\sin x$. Then, we get
      \begin{align*}
        y(x) &= c_1e^{\left(-2+i\right)x} + c_2e^{\left(-2-i\right)x}\\
             &= e^{-2x}\left(c_1e^{ix} + c_2e^{-ix}\right).
      \end{align*}
      Let $f(x) = c_1e^{ix} + c_2e^{-ix}$. Using the even/odd decomposition, we get
      \begin{align*}
        f(x) &= \frac{1}{2}\left(f(x) + f\left(-x\right)\right) + \frac{1}{2}\left(f(x) - f\left(-x\right)\right)\\
             &= \left(c_1 + c_2\right)\cos\left(x\right) + i\left(c_1 - c_2\right)\sin\left(x\right).
      \end{align*}
      We ``real''-ize our solution by just dropping the value of $i$ in $f(x)$. Thus, we get the full general solution
      \begin{align*}
        y(x) &= e^{-2x}\left(d_1\cos\left(x\right) + d_2\sin\left(x\right)\right).
      \end{align*}
    \item If we have the equation
      \begin{align*}
        y^{(4)} - 25y'',
      \end{align*}
      then using a similar process, we get the solution
      \begin{align*}
        y(x) = c_1 + c_2 x + c_3e^{5x} + c_4e^{-5x}.
      \end{align*}
    \item Considering the equation
      \begin{align*}
        y^{(5)} + 4y''' + 4y' = 0,
      \end{align*}
      we take the characteristic equation $r^{5}+ 4r^3 + 4r = 0$. Factoring, we get solutions of $r=0,r=\pm i\sqrt{2}$. Thus, we get the solution of
      \begin{align*}
        y(x) = c_1 + c_2\cos\left(\sqrt{2}x\right) + c_3\sin\left(\sqrt{2}x\right) + c_4x\cos\left(\sqrt{2}x\right) + c_5x\sin\left(\sqrt{2}x\right).
      \end{align*}
  \end{enumerate}
\end{example}
\subsection{Reducing our Orders}%
Let
\begin{align*}
  y''(x) + p(x)y'(x) + q(x)y(x) &= 0.
\end{align*}
Suppose we know $y_1(x)$. Can we find $y_2(x)$? The answer is yes. We presume
\begin{align*}
  y_2(x) &= v(x)y_1(x).
\end{align*}
Now, we have
\begin{align*}
  y_2 &= vy_1\\
  y_2' &= v'y_1 + vy_1'\\
  y_2'' &= v''y_1 + 2v'y_1' + vy_1'',
\end{align*}
and inserting into the equation, we get
\begin{align*}
  0 &= v''y_1 + 2v'y_1' + vy_1'' + pv'y_1 + pvy_1' + qvy_1\\
    &= v''y_1 + 2v'y_1' + pv'y_1 + v\underbrace{\left(y_1'' + py_1' + qy_1\right)}_{=0}\\
    &= v''y_1 + 2v'y_1' + pv'y_1
\end{align*}
Now, we have
\begin{align*}
  \frac{v''}{v'} &= -2\frac{y_1'}{y_1} - p.\tag{\textasteriskcentered}
\end{align*}
Integrating, we get
\begin{align*}
  \ln\left(v'\right) &= -2\ln\left(y_1\right) - \int_{}^{} p(x)\:dx.
\end{align*}
Taking powers, we get
\begin{align*}
  v' &= e^{-2\ln\left(y_1\right) - \int_{}^{} p(x)\:dx}\\
     &= y_1^{-2} e^{-\int_{}^{} p(x)\:dx}\\
     &= \frac{e^{-\int_{}^{} p(x)\:dx}}{y_1(x)^2}\\
  v &= \int_{}^{} \frac{e^{-\int_{}^{} p(x)\:dx}}{y_1(x)^2}\:dx
\end{align*}
\begin{example}
  Consider the equation
  \begin{align*}
    \cos^2\left(x\right)y''(x) - \sin(x)\cos(x)y' - y(x) = 0.
  \end{align*}
  Putting our equation into standard form, we may be able to find another solution.
  \begin{align*}
    y'' - \tan(x)y' - \sec^2\left(x\right)y &= 0.
  \end{align*}
  Guessing $y(x) = \tan(x)$, we get $y' = \sec^2\left(x\right)$ and $y'' = 2\sec^2\left(x\right)\tan(x)$. This is also another solution, $y_2(x) = \tan(x)$.\newline

  We don't want to guess anymore. Let $y_2(x) = v(x)y_1(x)$. We get
  \begin{align*}
    v(x) &= \int_{}^{} \frac{e^{-\int_{}^{} p(x)\:dx}}{y_1^2\left(x\right)}\:dx.
  \end{align*}
  We have $-p(x) = \tan(x)$, so $-\int_{}^{} p(x)\:dx = \ln\left(\sec(x)\right)$. Thus, $e^{-\int_{}^{} p(x)\:dx} = \sec(x)$. Thus, we get
  \begin{align*}
    v(x) &= \int_{}^{} \frac{\sec(x)}{\tan^2\left(x\right)}\:dx\\
         &= \int_{}^{} \frac{\cos(x)}{\sin^2\left(x\right)}\:dx\\
         &= \int_{}^{} \frac{1}{u^2}\:du\tag*{$u = \sin(x)$}\\
         &= -\frac{1}{u}\\
         &= -\csc(x).
  \end{align*}
  Thus, we have $y_2(x) = -\csc(x)\tan(x) = -\sec(x)$. 
\end{example}
\begin{example}
  Consider the equation
  \begin{align*}
    x^2\left(\ln(x)-1\right)y''(x) -xy'(x) + y'(x) = 0.
  \end{align*}
  We can use the power of inspection to find one solution, $y_1(x) = x$. Dividing out, we have
  \begin{align*}
    y'' - \frac{1}{x\left(\ln(x) - 1\right)} y' + \frac{1}{x^2\left(\ln(x) - 1\right)}y &= 0.
  \end{align*}
  Using the reduction of order, we guess $y_2(x) = v(x)y_1(x)$, and have
  \begin{align*}
    v(x) &= \int_{}^{} \frac{e^{-\int_{}^{} p(x)\:dx}}{y_1^2}\:dx.
  \end{align*}
  Noting that $-p(x) = \frac{1}{x\left(\ln(x)-1\right)}$, we have $\int_{}^{} \frac{1}{x\left(\ln(x)-1\right)}\:dx = \ln\left(\ln(x)-1\right)$.\newline

  Now, we have
  \begin{align*}
    v(x) &= \int_{}^{} \frac{\ln(x)-1}{x^2}\:dx\\
         &= \frac{1-\ln(x)}{x} - \int_{}^{} -\frac{1}{x^2}\:dx\tag*{$u=\ln(x)-1,dv=x^{-2}$}\\
         &= \frac{-\ln(x)}{x} - \frac{1}{x}\\
         &= -\frac{\ln(x)}{x}.
  \end{align*}
  Thus, we get $y_2(x) = -\ln(x)$, and the general solution of $y(x) = c_1 x + c_2\ln(x)$.
\end{example}
\begin{example}[Cauchy--Euler Equation]
  A second-order Cauchy--Euler equation is of the form
  \begin{align*}
    ax^2y''(x) + bxy'(x) + cy(x) &= 0.
  \end{align*}
  More generally,
  \begin{align*}
    \sum_{k=0}^{n}c_kx^ky^{(k)}(x) &= 0.
  \end{align*}
  We guess $y(x) = x^r$. Then, $y'(x) = rx^{r-1}$ and $y''(x) = r(r-1)x^{r-2}$. This yields
  \begin{align*}
    a(r)(r-1)x^r + brx^{r} + cx^r &=x^r \left(a\left(r^2 - r\right) + br + c\right)\\
                                  &= 0.
  \end{align*}
\end{example}
\begin{example}[Solving a Cauchy--Euler Equation]
  Consider the equation
  \begin{align*}
    x^2y'' + xy' - y &= 0.
  \end{align*}
  Substituting the characteristic equation, we get
  \begin{align*}
    r^2 - 1 &= 0,
  \end{align*}
  so our general solution is $y(x) = c_1x + c_2/x$.
\end{example}
\begin{example}[Solving another Cauchy--Euler Equation]
  Consider the equation
  \begin{align*}
    x^2y'' - 3xy' + 4y &= 0.
  \end{align*}
  Substituting the characteristic equation, we get
  \begin{align*}
    r^2 - 4r + 4 &= 0,
  \end{align*}
  so our solutions are $x^2$ and $x^2$. This is not good enough, we need another solution.\newline

  Now, we place our equation into standard form.
  \begin{align*}
    y'' - \frac{3}{x}y' + \frac{4}{x^2}y' &= 0.
  \end{align*}
  Thus, we get $p(x) = -\frac{3}{x}$. Using reduction of order, we get $y_2(x) = v(x)y_1(x)$,
  \begin{align*}
    v(x) &= \int_{}^{} \frac{e^{-\int_{}^{} -3/x\:dx}}{x^4}\:dx\\
         &= \int_{}^{} \frac{e^{3\ln(x)}}{x^4}\:dx\\
         &= \int_{}^{} \frac{x^3}{x^4}\:dx\\
         &= \ln(x).
  \end{align*}
  Thus, we have the solution $y_2(x) = \ln(x)x^2$, and the general solution of $y(x) = c_1x^2 + c_2\ln(x)x^2$.
\end{example}
\begin{example}
  Consider the equation
  \begin{align*}
    x^2y'' + 3xy' + 5y &= 0.
  \end{align*}
  We get the characteristic equation of
  \begin{align*}
    0 &= r^2 - 4r + 5\\
    r &= 2\pm i.
  \end{align*}
  Now, we need to figure out $x^{2\pm i}$.
\end{example}

\end{document}
