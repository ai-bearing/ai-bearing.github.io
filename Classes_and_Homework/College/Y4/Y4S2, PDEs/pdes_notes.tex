\documentclass[10pt]{mypackage}

% sans serif font:
%\usepackage{cmbright,sfmath,bbold}
%\renewcommand{\mathcal}{\mathtt}

%Euler:
\usepackage{newpxtext,eulerpx,eucal,eufrak}
\renewcommand*{\mathbb}[1]{\varmathbb{#1}}
\renewcommand*{\hbar}{\hslash}

%kp fonts:
%\usepackage{kpfonts}
%\renewcommand{\mathbb}{\mathds}

\pagestyle{fancy} %better headers
\fancyhf{}
\rhead{Avinash Iyer}
\lhead{Partial Differential Equations: Class Notes}

\setcounter{secnumdepth}{0}

\begin{document}
\RaggedRight
\section{Introduction}%
Consider the equations
\begin{align*}
  y''(x) + y(x) &= e^x\tag*{(1)}\\
  y^{(17)}(x) + \sin\left(y(x)\right) &= \left(x^{x}\right)^{x}\tag*{(2)}
\end{align*}
Before we want to solve these equations, we need to understand what these equations \textit{are}. 
\begin{enumerate}[(1)]
  \item This is a second order, inhomogeneous, linear ordinary differential equation.
  \item This is a 17th order, inhomogeneous, nonlinear ordinary differential equation.
\end{enumerate}
Generally, when we have a nonlinear equation, we convert it (using the Jacobian) to the ``nearest'' corresponding linear equation using Taylor approximations. In this case, converting equation (2), we have
\begin{align*}
  y^{(17)}(x) + y(x) &= \left(x^{x}\right)^{x}.\tag*{(2')}
\end{align*}
Now, equation (2') is linear, so it is able to be solved. It may not be pretty,\footnote{Citation needed.} but it can be solved, using Laplace Transforms or other methods.
\section{Ordinary Differential Equations}%
Returning to our equation (1), 
\begin{align*}
  y''(x) + y(x) &= e^x,\tag*{(1)}
\end{align*}
there is one more fact that we can see --- this is an equation with constant coefficients. The most general form of a $n$th order linear ordinary differential equation is of the form
\begin{align*}
  a_n(x)y^{(n)}(x) + a_{n-1}(x)y^{(n-1)}(x) + \cdots + a_1(x)y'(x) + a_0(x)y(x) = g(x).\label{eq:general_linear_ode}\tag{\textdagger}
\end{align*}
Specifically, we also require $a_k(x)\in C(I)$, where $I$ is some interval (specifics will be detailed later).
\begin{theorem}[Existence and Uniqueness Theorem]
  Any ordinary differential equation of the form (\textdagger) has unique solutions in the interval $I$.\newline

  There are $n$ linearly independent solutions for $g(x) = 0$.
\end{theorem}
The corresponding homogeneous equation for (1) is
\begin{align*}
  y''(x) + y(x) &= 0.\tag*{(1')}
\end{align*}
The equations (1) and (1') are related by the linearity principle. In particular, if $y_0(x)$ is a solution to (1'), then we can add $\alpha y_0(x)$ to any solution $y_p(x)$ of (1), then we have all the solutions for (1). In particular, the solutions to (1') are
\begin{align*}
  y_1(x) &= \sin(x)\\
  y_2(x) &= \cos(x).
\end{align*}
To evaluate that these solutions are linearly independent, we consider the differential operator $L$ from (\textdagger) defined by
\begin{align*}
  L\left[y\right] &= \sum_{k=0}^{n}a_k(x)y^{(k)}(x).
\end{align*}
We rewrite (\textdagger) as
\begin{align*}
  L\left[y\right] &= g(x).
\end{align*}
The operator $L$ is linear, so $L$ has the following properties:
\begin{itemize}
  \item $L\left[y_1 + y_2\right]$;
  \item $L\left[cy\right] = cL\left[y\right]$.
\end{itemize}
Now, in (1) and (1'), if we set $L\left[y\right] = y''(x) + y(x)$, then evaluating our solutions $y_1$ and $y_2$ to (1'), we get
\begin{align*}
  L\left[c_1y_1 + c_2y_2\right] &= c_1L\left[y_1\right] + c_2L\left[y_2\right]\\
                                &= 0.
\end{align*}
Now, we get
\begin{align*}
  y_0(x) &= c_1\sin(x) + c_2\sin(x)
\end{align*}
as our general solution to (1'). By the linearity principle, all we need is one solution to $L\left[y\right] = e^x$ to find all solutions to (1).\newline

Evaluating \eqref{eq:general_linear_ode} in the most general form, we have the general solution
\begin{align*}
  y(x) &= \underbrace{c_1y_1(x) + c_2y_2(x) + \cdots + c_ny_n(x)}_{\text{homogeneous solution}} + y_p(x),
\end{align*}
where $y_p(x)$ is the particular solution. In other words, our general solution is
\begin{align*}
  y(x) &= \Span\left(y_1(x),y_2(x),\dots,y_n(x)\right) + y_p(x).
\end{align*}
For this to work, we need the set $\set{y_1,\dots,y_n}$ to be linearly independent. To do this, we evaluate the Wronskian:
{\renewcommand{\arraystretch}{1.5}
  \begin{align*}
  W(x) &= \det \begin{pmatrix}y_1(x) & y_2(x) & \cdots & y_n(x) \\ y_1'(x)& y_2'(x) & \cdots & y_n'(x) \\ \vdots & \vdots & \ddots & \vdots \\ y_{1}^{(n-1)}(x) & y_2^{(n-1)}(x) & \cdots & y_{n}^{(n-1)}(x)\end{pmatrix}.
\end{align*}
}
Specifically, the set $\set{y_1,\dots,y_n}$ is linearly independent if $W(x)\neq 0$ for all $x\in I$.
\begin{example}
  Consider the equation
  \begin{align*}
    y''(x) - y(x) &= e^{x}\tag{1}
  \end{align*}
  We want to find the general solution to this constant coefficient equation.\newline

  We start by finding two linearly independent homogeneous solutions to the equation, take their span, then add a particular solution.\newline

  The characteristic equation of the homogeneous equation for (1) is
  \begin{align*}
    r^2 - 1 &= 0
  \end{align*}
  We get $r=\pm 1$, which by the definition of the characteristic equation yields $y_1(x) = e^{x}$ and $y_2(x) = e^{-x}$. To verify that this this solution set is linearly independent
  {\renewcommand{\arraystretch}{1.25}
    \begin{align*}
      W(x) &= \det \begin{pmatrix}e^{x} & e^{-x} \\ e^{x} & -e^{-x}\end{pmatrix}\\
           &= -2\\
           &\neq 0.
  \end{align*}
  }
  Thus, our solutions are linearly independent. We get the general form of
  \begin{align*}
    y(x) &= c_1e^{x} + c_2e^{-x} + y_p(x).
  \end{align*}
  Now, we only have to find a particular solution. This is, unfortunately, the hard part.\newline

  We begin by guessing. But, in a way that doesn't suck. Specifically, we let $y_p(x) = Axe^{x}$. Evaluating, we get
  \begin{align*}
    y_p'(x) &= A\left(x+1\right)e^{x}\\
    y_{p}''(x) &= A\left(x+2\right)e^{x}\\
    y_{p}''(x) - y_p(x) &= A\left(x+2\right)e^{x} - Axe^{x}\\
                        &= 2Ae^{x},
  \end{align*}
  so $2A = 1$, and $A = \frac{1}{2}$. Thus, we have the end result of
  \begin{align*}
    y(x) &= c_1e^{x} + c_2e^{x} + \frac{1}{2}xe^{x}.
  \end{align*}
  Evaluating in Mathematica, we take
  \begin{lstlisting}[style=mathematicastyle]
    DSolve[y''[x] - y[x] == Exp[x], y[x], x]
  \end{lstlisting}
  and we get
  \begin{align*}
    y(x) &= c_1e^{x} + c_2e^{-x} + \frac{1}{4}\left(2x-1\right)e^{x},
  \end{align*}
  corroborating our solution.\footnote{Only slightly different, but they're the same solution.}
\end{example}
\begin{example}
  Consider the equation
  \begin{align*}
    y'''(x) - y(x) &= 0.
  \end{align*}
  The particular solution to this equation is $y(x) = 0$. The characteristic equation for this equation is
  \begin{align*}
    r^3 - 1 &= 0.
  \end{align*}
  Factoring, we get
  \begin{align*}
    \left(r-1\right)\left(r^2 + r + 1\right) &=0\\
    \left(r-1\right)\left(r-\zeta_3\right)\left(r-\zeta_3^2\right) &= 0.
  \end{align*}
  Thus, we get
  \begin{align*}
    r &= \set{1, e^{\frac{2\pi i}{3}}, e^{\frac{4\pi i}{3}}}.
  \end{align*}
  Thus, our solutions are of the form
  \begin{align*}
    y(x) &= c_1e^{x} + c_2e^{-\frac{1}{2}x}\cos\left(\frac{\sqrt{3}}{2}x\right) + c_3e^{-\frac{1}{2}x}\sin\left(\frac{\sqrt{3}}{2}x\right).
  \end{align*}
\end{example}

\end{document}
