\documentclass{beamer}

\title[Excess Area Identities and Operator Symbols]{Excess Area Identities and Operator Symbols in Bergman Spaces}
\author{Avinash Iyer}
\institute{Occidental College}
\date{}
% Language and encoding
\usepackage[english]{babel}
\usepackage[utf8]{inputenc}
\usepackage[T1]{fontenc}
\usefonttheme{serif}
\usefonttheme{professionalfonts}
%\usepackage{cmbright}
%\usepackage{sfmath}
%\usepackage{bbold}

%%%%%%%%%%%%%%%%%%%%%%%%%%%%%%%%%%%%%%%%%%%%%%%
\usetheme{Madrid} % You can choose from various built-in themes

% Color theme (optional)
\usecolortheme{beaver} % You can choose from various built-in color themes
%%%%%%%%%%%%%%%%%%%%%%%%%%%%%%%%%%%%%%%%%%%%%%%%%%
% Math packages
\usepackage{amsmath, amssymb, amsfonts, mathtools}

% Graphics package
\usepackage{graphicx}
\usepackage{subcaption}

% Define custom colors
\usepackage{xcolor}
\definecolor{customblue}{RGB}{0, 56, 108}
%\definecolor{customgold}{RGB}{241, 178, 15}

% Beamer presentation settings
\mode<presentation>
{
  \usetheme{default}      
  \usecolortheme[named=customblue]{structure}
  \setbeamercolor{normal text}{fg=black,bg=white}
  %\setbeamercolor{alerted text}{fg=customgold}
  \setbeamercolor{example text}{fg=customblue}
  \setbeamercolor{background canvas}{bg=white}
  \setbeamercolor{frametitle}{fg=customblue}
  \setbeamercolor{title}{fg=customblue}
  \setbeamercolor{author}{fg=customblue}
  \setbeamercolor{date}{fg=customblue}
  \setbeamercolor{item}{fg=customblue}
  \setbeamercolor{itemize item}{fg=customblue}
  \setbeamertemplate{navigation symbols}{}
  \setbeamertemplate{caption}[numbered]
}

% % Custom commands and macros
% Useful symbols

%canonical sets
\newcommand{\N}{\mathbb{N}}
\newcommand{\Q}{\mathbb{Q}}
\newcommand{\Z}{\mathbb{Z}}
\newcommand{\R}{\mathbb{R}}
\newcommand{\C}{\mathbb{C}}
\newcommand{\F}{\mathbb{F}}

%common other symbols
\newcommand{\mcc}{\mathcal{C}} %cantor set
\newcommand{\mco}{\mathcal{O}} %holomorphic functions
\newcommand{\mfp}{\mathfrak{p}} %prime ideal

%inner products and norms
\newcommand{\iprod}[2]{\left\langle #1,#2 \right\rangle}
\newcommand{\norm}[1]{\left\Vert #1 \right\Vert}
\newcommand{\set}[1]{\left\{#1\right\}}

\usepackage{newpxtext,eulerpx,eucal,eufrak}
\renewcommand*{\mathbb}[1]{\varmathbb{#1}}
\AtBeginSection[]
{
  \begin{frame}
    \frametitle{Contents}
    \tableofcontents[currentsection]
  \end{frame}
}
\begin{document}

\begin{frame}
    \titlepage
\end{frame}
\begin{frame}{Summary}
  \tableofcontents
\end{frame}
\section{Definitions and Notations}
\begin{frame}{Weighted Square-Integrable Functions}
  \begin{itemize}
    \item Domain: $\Omega = \mathbb{D},D(0,r),\C$.
  \end{itemize}
  \begin{definition}[$L^{2}$ Functions]
    \begin{align*}
      L^{2}\left(\Omega\right)&= \set{f\left| \int_{\Omega}^{} \left\vert  f(z) \right\vert^2\:dA < \infty\right.}
    \end{align*}
  \end{definition}
  \begin{itemize}
    \item $\lambda(z) = \lambda(|z|)$
  \end{itemize}
  \begin{definition}[Weighted $L^{2}$ Functions]
    \begin{align*}
      L^{2}\left(\Omega,\lambda\right)&= \set{f\left| \int_{\Omega}^{} \left\vert  f(z) \right\vert^2\lambda(z)\:dA < \infty\right.}
    \end{align*}
  \end{definition}
\end{frame}
\begin{frame}{Weighted Square-Integrable Functions, Cont'd}
  \begin{itemize}
    \item $L^{2}\left(\Omega,\lambda\right)$ forms a Hilbert space.
  \end{itemize}
  \begin{definition}[Weighted $L^{2}$ Inner Product]
    For $f,g\in L^{2}\left(\Omega,\lambda\right)$,
    \begin{align*}
      \iprod{f}{g}_{L^2\left(\Omega,\lambda\right)} &= \int_{\Omega}^{} f(z)\overline{g(z)}\lambda(z)\:dA.
    \end{align*}
  \end{definition}
  \begin{definition}[Weighted $L^{2}$ Norm]
    For $f\in L^{2}\left(\Omega,\lambda\right)$,
    \begin{align*}
      \norm{f}_{L^{2}\left(\Omega,\lambda\right)}^2 &= \int_{\Omega}^{} \left\vert f(z) \right\vert^2\lambda(z)\:dA.
    \end{align*}
  \end{definition}
\end{frame}
\begin{frame}{Bergman Spaces}
  \begin{definition}[$\mco\left(\Omega\right)$]
    \begin{align*}
      f\in \mco\left(\Omega\right) &\Longleftrightarrow \frac{\partial f}{\partial \overline{z}} = 0\\
                                   &\Longleftrightarrow \frac{1}{2}\left(\frac{\partial f}{\partial x} + i\frac{\partial f}{\partial y}\right) = 0
    \end{align*}
  \end{definition}
  \begin{itemize}
    \item If $\displaystyle \Omega \underset{\clap{\tiny open}}{\subseteq} \C$, then $f$ is holomorphic iff $f$ is analytic.
  \end{itemize}
\end{frame}
\begin{frame}{Bergman Spaces, Cont'd}
  \begin{definition}[Bergman Space]
    \begin{align*}
      A^{2}\left(\Omega,\lambda\right) &= L^{2}\left(\Omega,\lambda\right) \cap \mco\left(\Omega\right)
    \end{align*}
  \end{definition}
  \begin{definition}[$A^{1,2}\left(\Omega,\lambda\right)$]
    \begin{align*}
      A^{1,2}\left(\Omega,\lambda\right) &= \set{h\in A^{2}\left(\Omega,\lambda\right)\left|\frac{\partial h}{\partial z}\in A^{2}\left(\Omega,\lambda\right)\right.}
    \end{align*}
  \end{definition}
  \begin{definition}[Image Area]
    Let $h\in A^{1,2}\left(\Omega,\lambda\right)$. Then,
    \begin{align*}
      A(h) &= \int_{\Omega}^{} \left\vert h'(z) \right\vert^2\lambda(z)\:dA
    \end{align*}
  \end{definition}
\end{frame}
\begin{frame}{Projection Operator}
  \begin{itemize}
    \item $A^{2}(\Omega)\subseteq L^{2}\left(\Omega\right)$ is closed and has a reproducing kernel $K_{z}(\cdot)$
    \item $\displaystyle f(z) = \iprod{f(\cdot)}{K_z(\cdot)}_{L^{2}(\Omega,\lambda)}$
  \end{itemize}
  \begin{definition}[Projection Operator]
    Let $h\in L^{2}\left(\Omega,\lambda\right)$. Then,
    \begin{align*}
      P^{\Omega,\lambda}\left(h\right) &= \int_{\Omega}^{} \left(h(w)\right)\left(\overline{K_z(w)}\right)\left(\lambda(w)\right)\:dA
    \end{align*}
  \end{definition}
  \begin{definition}[Toeplitz Operator]
    Let $\varphi \in L^{\infty}\left(\Omega,\lambda\right)$, $h\in L^{2}\left(\Omega,\lambda\right)$. Then,
    \begin{align*}
      T_{\varphi}^{\Omega,\lambda}\left(h\right) = P^{\Omega,\lambda}\left(\varphi h\right)=\int_{\Omega}^{} \left(\varphi(w)h(w)\right)\left(\overline{K_z(w)}\right)\left(\lambda(w)\right)\:dA
    \end{align*}
  \end{definition}
\end{frame}
\begin{frame}{Commutator and Hankel Operators}
  \small
  \begin{definition}[Commutator]
    Let $M_{\varphi}(h) = \varphi h$ for $\varphi \in L^{\infty}\left(\Omega,\lambda\right)$.\newline

    Then, $\left[P^{\Omega,\lambda},M_{\varphi}\right]: L^{2}\left(\Omega,\lambda\right) \rightarrow L^{2}\left(\Omega,\lambda\right)$ is defined by
    \begin{align*}
      \left[P^{\Omega,\lambda},M_{\varphi}\right](h)&= P^{\Omega,\lambda}\left(\varphi h\right) - \varphi P^{\Omega,\lambda}\left(h\right).
    \end{align*}
  \end{definition}
  \begin{definition}[Hankel Operator]
    Let $\varphi \in L^{\infty}$. Then, $H^{\Omega,\lambda}_{\varphi}: A^{2}\left(\Omega,\lambda\right)\rightarrow \left(A^{2}\left(\Omega,\lambda\right)\right)^{\perp}$ is defined by
    \begin{align*}
      H_{\varphi}^{\Omega,\lambda}\left(h\right) &= -\left[P^{\Omega,\lambda},M_{\varphi}\right]\biggr\vert_{A^{2}\left(\Omega,\lambda\right)}\\
                                                 &= \left(I-P^{\Omega,\lambda}\right)\left(\varphi h\right)\\
                                                 &= \left(M_{\varphi} - P^{\Omega,\lambda}\right)\left(h\right)
    \end{align*}
  \end{definition}
\end{frame}
\section{Motivation}
\begin{frame}{Abstract Motivations}
  \begin{itemize}
    \item Relationship between $L^{2}$ norms of functions and $\ell^{2}$ norms of Taylor coefficients. For $h\in A^{2}\left(\mathbb{D}\right)$, $h = \sum_{k=0}^{\infty}h_kz^k$
      \begin{align*}
        \norm{h}_{L^2}^2 &= \sum_{k=0}^{\infty}\frac{\left\vert h_k \right\vert^2}{k+1}.
      \end{align*}
    \item $\set{z^k}_{k=0}^{\infty}$ forms an orthogonal basis for $A^{2}$.
    \item On bounded domains,
      \begin{align*}
        \int_{\Omega}^{} \frac{\partial f}{\partial \overline{z}}\:d\overline{z}\wedge dz &= \int_{\partial \Omega}^{} f\:dz\\
        \int_{\Omega}^{} \frac{\partial f}{\partial z}\:dz\wedge d\overline{z} &= \int_{\partial \Omega}^{} f\:d\overline{z}.
      \end{align*}
    \item $\left[T_{\overline{z}}M_z,DM_{z}\right]\left(z^k\right) = 0$, where $D = \frac{\partial }{\partial z}$.
  \end{itemize}
\end{frame}
\begin{frame}{Literature Review}
  \begin{itemize}
    \item In \cite{d2019hermitian}, John D'Angelo proved the following identity regarding the excess area of the image of a function $h\in A^{1,2}\left(\mathbb{D}\right)$.
  \end{itemize}
  \begin{align*}
    A_{\mathbb{D}}(zh) - A_{\mathbb{D}}(h) &= \frac{1}{2}\int_{0}^{2\pi} \left\vert h\left(e^{i\theta}\right) \right\vert^2\:d\theta\\
                                            &= \pi \norm{Sh}_{L^{2}\left(\partial \mathbb{D}\right)}^2
                                            % &= \pi \sum_{k=0}^{\infty}\left\vert h_k \right\vert^2.
  \end{align*}
\end{frame}
\begin{frame}{Literature Review, Cont'd}
  \begin{itemize}
    \item In \cite{bambico2022generalization}, the excess area identity was extended to include Blaschke product multipliers.
    \item Additionally, \cite{bambico2022generalization} formulated an excess area identity of the form
      \begin{align*}
        \norm{\frac{\partial}{\partial z}\left(zu\right)}_{L^{2}\left(\mathbb{D}\right)}^2 - \norm{\frac{\partial}{\partial z}\left(u\right)}_{L^{2}\left(\mathbb{D}\right)}^2,
      \end{align*}
      where $u$ is a harmonic function.
  \end{itemize}
  \begin{itemize}
    \item In \cite{AshleyAdenCelikDanielLuke2024}, an algorithm to formulate a harmonic symbol $\varphi$ such that $T^{\mathbb{D}}_{\varphi}\left(p\right) = q$ for holomorphic polynomials $p$ and $q$, $p\neq 0$.
    \item Additionally, \cite{AshleyAdenCelikDanielLuke2024} substituted derivatives for Toeplitz operators in the excess area identity.
  \end{itemize}
\end{frame}
\section{Findings}
\section{REU Experience}
\section{Acknowledgements and References}
% Add your presentation content here
\begin{frame}[allowframebreaks]
  \frametitle{References}
  \bibliographystyle{amsalpha}
  \bibliography{reu_references.bib}
\end{frame}
\end{document}
