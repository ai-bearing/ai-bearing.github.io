\documentclass{reu_beamer}
\title[]{Observations on Excess Area Identities and Operator Symbols in Bergman Spaces}
\author{
    Avinash Iyer
    %Wynesakia Akamah (Rose-Hulman), (Occidental College), Jennifer Yuan (NYU Abu Dhabi) \\
    %Mentor: Dr. Mehmet \c{C}elik
}
\institute{Occidental College}
\beamertemplatenavigationsymbolsempty

\setbeamerfont{page number in head/foot}{size=\tiny}
\setbeamertemplate{footline}[frame number]
\usepackage{cancel}
%%%%%%%%%%
% Commands to include a figure:

%%%%%%%%%%
\date{}
%\date{Thursday, April 25th}
%%%%%%%%%%%%%
\begin{document}
%Slide #1: Title page - Group members’ names, school, course, and the professor’s name
%%%%%%%%%%%%%%%%%%%%
\begin{frame}
    \begin{figure}[h]
        \centering
        %\includegraphics[width=1in,trim={0 0.5cm 0 0.5cm},clip]{Presentation/TAMUC-V-ClearSpace-1.png}
        %\includegraphics[width=0.8in,trim={1.5cm, 4.5cm ,100cm, 1cm},clip]{TAMUC-V-ClearSpace-1.png}
        %\hspace{1in}
        %\includegraphics[width=0.8in]{1019px-NSF_logo.png}
        %\begin{subfigure}[b]{0.35\textwidth}
        %    \centering
        %    \includegraphics[width=1in]{Presentation/TAMUC-V-ClearSpace-1.png}
        %\end{subfigure}
        %\hfill
        %\begin{subfigure}[b]{0.35\textwidth}
        %    \centering
        %    \includegraphics[width=1in]{Presentation/1019px-NSF_logo.png}
        %\end{subfigure}
    \end{figure}
    \maketitle
\end{frame}
%%%%%%%%%%%%%%%%%%%%
%Slide #2 Summary of the talk
%%%%%%%%%%%%%%%%%%%%
\begin{frame}{Summary}
\tableofcontents
    %\begin{itemize}
    %    \item Definitions and Notations\bigskip
    %    
    %    \item Motivation to the Problem\bigskip
    %    
    %    \item Results and Observations\bigskip
    %    
    %    \item Remarks, Conjectures, and Future Directions\bigskip
    %    
    %    \item Acknowledgments\bigskip
    %    
    %    \item References
    %\end{itemize}
\end{frame}
\section{Definitions and Notations}
%%%%%%%%%%%%%%%%%%%%
%Slide #3 Definitions and notations will be used in the presentation
%%%%%%%%%%%%%%%%%%%%
\begin{frame}{Definition and Notations}
\begin{itemize}
    \item $\Omega $ : a region in \(\C\) e.g. $ \mathbb{D},D(0,r),\mathbb{A}(0,r,1),\C$
    \item $\lambda(z)=\lambda(|z|)\in C^{\infty}(\Omega)$: weight function
\end{itemize} 
\begin{definition}[\(\lambda\)-weighted Square-Integrable Functions]
    \begin{align*}
        L^{2}(\Omega,\lambda) &= \set{f: \Omega \rightarrow \C \left| \int_{\Omega}|f(z)|^2\lambda(z) \: d A(z) < \infty\right.}
    \end{align*}
\end{definition}

\begin{itemize}
    \item \(L^2(\Omega, \lambda)\) forms a Hilbert space with inner product
    \[\ang{f, g}_{L^2(\Omega, \lambda)} = \int_\Omega f(z) \ol{g(z)} \lambda(z)\: dA(z)\]
    inducing the norm
    \[\norm{f}_{L^{2}(\Omega,\lambda)}^2 = \int_{\Omega}|f(z)|^2\lambda(z)\:dA(z)\]
\end{itemize}
\end{frame}


\begin{frame}{Definitions and Notations}
\begin{definition}[Holomorphic Function on $\Omega$]
  We say $h\in \mathcal{O}(\Omega)$ if and only if for $z\in \Omega$,
    \begin{align*}
      \frac{\partial h(z)}{\partial \bar{z}}&= \frac{1}{2}\left(\frac{\partial h(z)}{\partial x} + i\frac{\partial h(z)}{\partial y} \right)\\
                                            &= \frac{1}{2}\left(\left(\frac{\partial u}{\partial x} - \frac{\partial v}{\partial y}\right) + i\left(\frac{\partial u}{\partial y} + \frac{\partial v}{\partial x}\right)\right)\\
                                            &= 0.
    \end{align*}
\end{definition} 
%\begin{example}[Holomorphic function on $\Omega$]
%\begin{itemize}
%    \item $h(z)= e^{\frac{1}{z}}$, $\frac{\partial f}{\partial \ol{z}} =0$  $\forall z \in \Omega = \C\setminus \{0\}$.
%    \item $h(z)= z|z|^2 = z^2\ol{z}$, $\frac{\partial f}{\partial \bar{z}}= z^2 \neq 0$, holomorphic nowhere on $\mathbb{C}$.
%\end{itemize}
%\end{example}

\begin{definition}[\(\lambda\)-weighted Bergman Space]
    \begin{align*}
        A^{2}(\Omega,\lambda) \coloneq \mathcal{O}(\Omega)\cap L^{2}(\Omega,\lambda).
    \end{align*}
\end{definition}
\end{frame}
\begin{frame}{Definitions and Notations}
    % \begin{definition}[Sobolev Space]
    % $h\in W^{1,2}(\Omega,\lambda)$ if
    % \begin{align*}
    %     h,Dh\in L^{2}(\Omega,\lambda),
    % \end{align*}
    % where $\forall \varphi(z)\in C^{\infty}(\Omega)$,
    % \begin{align*}
    %     \int_{\Omega}h(z)\varphi(z)\:dA &= -\int_{\Omega}(Dh)(z)\varphi(z)\:dA(z)
    % \end{align*}
    % \end{definition}
    \begin{definition}[$A^{1,2}(\Omega,\lambda)$]
        \[A^{1,2}(\Omega,\lambda) = \left\{h \in A^{2}(\Omega,\lambda)\ \left|\ \frac{\partial h}{\partial z} \in A^2(\Omega,\lambda)\right.\right\}\]
        % \begin{align*}
        %     A^{1,2}(\Omega,\lambda) &= A^{2}(\Omega,\lambda)\cap W^{1,2}(\Omega,\lambda)\\
        %     W^{1,2}(\Omega,\lambda)&= L^{2}(\Omega,\lambda)\cap W^{1}(\Omega,\lambda)\\
        %     W^{1}(\Omega,\lambda)&=\left\{f\in L^{2}(\Omega,\lambda)\ |\ \frac{\partial f(z)}{\partial \bar{z}}\in L^{2}(\Omega,\lambda)\right\}
        % \end{align*}
    \end{definition}

    \begin{definition}[Weighted Image-Area]
        Let $h\in A^{1,2}(\Omega,\lambda)$.
        \begin{align*}
            A_{\Omega,\lambda}(h) &= \int_{\Omega}\left\vert \frac{\partial h}{\partial z}\right\vert^2 \lambda(z)\:dA(z)\\
                                  &= \norm{\frac{\partial h}{\partial z}}_{L^{2}\left(\Omega,\lambda\right)}^{2}
        \end{align*}
    \end{definition}
\end{frame}
%\begin{frame}{Definitions and Notations}
%    \begin{block}{Definition (Sobolev space)}
%        \begin{align*}
%            W^{k,p}(u)=\left\{u\in L^{p}(\Omega)\left|\frac{\partial^{\alpha}}{\partial z^{\alpha}}u\in L^{p}(\Omega),\;\forall \alpha\leq k\right.\right\}
%        \end{align*}
%    \end{block}
%\end{frame}
\begin{frame}{Definitions and Notations}
\begin{itemize}
    \item $A^{2}(\Omega,\lambda)$ has a reproducing kernel i.e \(\exists !\: K_\Omega^\lambda(\cdot,z) \in A^{2}(\Omega,\lambda) :\) 
    \[h(z) = \ang{h(\cdot), K_\Omega^\lambda(\cdot,z)}_{L^2(\Omega, \lambda)}\]
    \item $A^{2}(\Omega,\lambda)$ is a closed subspace of \(L^2(\Omega, \lambda)\).
\end{itemize}
\begin{definition}[Orthogonal Projection]
    Let \(P^{\Omega,\lambda}: L^2(\Omega,\lambda) \to A^2(\Omega,\lambda)\)
    \begin{align*}
        \left(P^{\Omega,\lambda} h\right)(z) &\coloneq \ang{h(\cdot), K_\Omega^\lambda(\cdot,z)}_{L^2(\Omega, \lambda)} \\
        &= \int_{\Omega} h(w) \ol{K_\Omega^\lambda(w,z)} \lambda(w) \:dA(w)
    \end{align*}
\end{definition}
\end{frame}

\begin{frame}{Definitions and Notations}
\begin{definition}[Multiplication Operator]
    Let \(M_{\varphi}: L^2(\Omega,\lambda) \to L^2(\Omega,\lambda)\) where $\varphi \in L^{\infty}(\Omega)$
    \[M_{\varphi}(h) \coloneq \varphi h\]
\end{definition}
\begin{definition}[Toeplitz Operator]
    $T_{\varphi}^{\Omega,\lambda}: A^2(\Omega,\lambda)\rightarrow A^2(\Omega,\lambda)$, where $\varphi\in L^{\infty}(\Omega)$
    \[T_{\varphi}^{\Omega,\lambda} \coloneq P^{\Omega,\lambda}M_{\varphi}\]
\end{definition}
\end{frame}
\begin{frame}{Definitions and Notations}
    \begin{definition}[Commutator]
        Let \(\left[P^{\Omega,\lambda},M_\varphi \right]: L^2(\Omega,\lambda) \to L^2(\Omega,\lambda)\)
        \[\left[P^{\Omega,\lambda},M_{\varphi}\right] \coloneq P^{\Omega,\lambda}M_\varphi - M_\varphi P^{\Omega,\lambda} \]
    \end{definition}
    \begin{definition}[Hankel Operator]
        Let \(H^{\Omega,\lambda}_{\varphi} : A^{2}(\Omega,\lambda) \to (A^2(\Omega,\lambda))^{\perp}\)
        \begin{align*}
            H^{\Omega,\lambda}_{\varphi} &\coloneq -\left[P^{\Omega,\lambda},M_{\varphi}\right]\biggr\vert_{A^{2}(\Omega,\lambda)}\\
            &=\left(I-P^{\Omega,\lambda}\right)M_\varphi\\
            &=M_\varphi-P^{\Omega,\lambda}M_\varphi\\
            &= M_{\varphi} - T^{\Omega,\lambda}_\varphi
        \end{align*}
    \end{definition}
\end{frame}
%\begin{frame}{Definitions and Notations, Cont'd}
%    \begin{itemize}
%        \item $A^{2}(\Omega,\lambda)$ is a closed subspace of $L^{2}(\Omega,\lambda)$. 
%        \item To orthogonally project any function in $L^{2}(\Omega,\lambda)$ into $A^{2}(\Omega,\lambda)$, we use the projection operator.
%        \item Here, $K(z,w)$ denotes the reproducing kernel of the Weighted Bergman space. 
%    \end{itemize}
%    \begin{definition}[Projection Operator]
%        \begin{align*}
%            P^{\Omega,\lambda}(f)(z) &= \int_{\Omega}K(z,w)f(w)\lambda(w)\: dA(w).
%        \end{align*}
%    \end{definition}
%\end{frame}
%\begin{frame}{Definitions and Notations, Cont'd}
%\begin{definition}[Toeplitz Operator]
%    Let \(\varphi\) be some bounded function on \(\Omega\). The Toeplitz operator with symbol \(\varphi\) is denoted
%    \[ T^{\Omega,\lambda}_{\varphi}(f)(z) = P^{\Omega,\lambda}(\varphi f)(z)\]
%\end{definition} 
%    \begin{itemize}
%        \item In general, we take $\varphi$ to be bounded. 
%        \item Essentially, the Toeplitz operator multiplies its input by the symbol, and projects it to $A^{2}(\Omega,\lambda)$.
%    \end{itemize}
%\end{frame}
%%%%%%%%%%%%%%%%%%%%
%Slide #4 All the Problems - short versions! Please, do not write everything; write some reminders and then explain them during the presentation
%%%%%%%%%%%%%%%%%%%%
\section{Motivation and Problem}


\begin{frame}{Motivations}
    % \begin{itemize}
    % \item $\{z^n\}_{n=0}^{\infty}$ form a \textit{complete orthogonal system} for $A^{2}(\mathbb{D})$. 
    % \begin{itemize}
    %     \item %That is, 
    %     $\langle z^n,z^m\rangle_{A^{2}(\mathbb{D})}=0$ (when $n\not=m$) and  if $h\in A^{2}(\mathbb{D})$ s.t. $h\perp z^n$ for all $n\in\mathbb{N}$ then $h\equiv 0$.
    % \end{itemize}
    % \item %The beautiful aspect of this observation is that 
    % If $h\in \mathcal{O}(\mathbb{D})$ and $h=\sum_{n=0}^{\infty}h_nz^n$, then $S_N:=\sum_{n=0}^{N}h_nz^n\underbrace{\overset{\text{as } N\rightarrow\infty}{\rightrightarrows}}_{\text{on }K\subset\subset\mathbb{D}} h$. 
    % \begin{itemize}
    %     \item $S_N$ is the orthogonal projection of $h$ onto the subspace $\mathcal{P}_N$. %$\mathcal{P}_N$ is the space of polynomials with degree at most $N$
    %     Then $\{S_N\}$ converges to $h$ in the Hilbert space sense.
    % \end{itemize}
    % \item One can relate $\|h\|_{L^2(\mathbb{D})}^2$ to the $\ell^2$-norm of the Taylor coefficients of $h$. %How?
    % \begin{itemize}
    %     \item %For example, by 
    %     $\|h\|_{L^(\mathbb{D})}^2=\sum_{k=0}^{\infty}|h_k|^2\frac{\pi}{k+1}$, %we can say whether 
    %     $h$ is in $A^2(\mathbb{D})$ just by looking at the sequence $\{h_k\}$ such that $\sum_{k=0}^{\infty}|h_k|^2\frac{\pi}{k+1}$ converges. %Thus, we can verify membership in $A^2(\mathbb{D})$ based on the convergence of the series involving the Taylor coefficients of $h$.
    % \end{itemize}
    % \item The definition of $\|h\|_{L^2}$-norm, as integral on $\mathbb{D}$) also offers the opportunity to use the boundary geometry of the domain through integration by parts. Moreover,
    % \item %Let $D=\frac{\partial}{\partial z}$ with domain $A^{1,2}(\Omega)$. Then, $D$ is an unbounded linear operator. For $\Omega=\mathbb{D}$, the commutator 
    % $\left[T_{\overline{z}}^{\mathbb{D}}M_{z},DM_z\right]=0$ on %the monomials 
    % $z^m$. Thus, $T_{\overline{z}}^{\mathbb{D}}M_{z}$ and $DM_{z}$ have $z^m$ as their eigenfunctions. %In other words, when $T_{\overline{z}}^{\mathbb{D}}M_{z}$ or $DM_{z}$ operates on $z^m$, the outcome is a constant times $z^m$.
    % \end{itemize}

    \begin{itemize}
        \item $\{z^n\}_{n=0}^{\infty}$ form a \textit{complete orthogonal basis} for $A^{2}(\mathbb{D})$ % i.e. \(h \perp z^k \implies h \equiv 0\) for all \(k \in \N\)
        %\item Since $h\in \mathcal{O}(\mathbb{D})$ and $h=\sum_{n=0}^{\infty}h_nz^n$, then \[S_N:=\sum_{n=0}^{N}h_nz^n\underbrace{\overset{\text{as } N\rightarrow\infty}{\rightrightarrows}}_{\text{on }K\subset\subset\mathbb{D}} h\]
        \item If $h$ is holomorphic, then $h$ is analytic:
          \begin{align*}
            h(z) &= \sum_{n=0}^{\infty}h_nz^{n}
            \intertext{and}
            S_N &:= \sum_{n=0}^{N}h_nz^{n}
          \end{align*}
          converges uniformly on compact subsets.
        \item Relationship between $L^2$ norm of $h$ to the $\ell^2$ norm of $\set{h_k}_{k=0}^{\infty}$:
        \[\|h\|_{L^2(\mathbb{D})}^2 = \int_\disc |h(z)|^2\: dA(z) =\pi\sum_{k=0}^{\infty}\frac{|h_k|^2}{k+1}\]
        \item \(\left[T_{\overline{z}}^{\mathbb{D}}M_{z},DM_z\right]\left(z^m\right)=0\)
    \end{itemize}
\end{frame}
\begin{frame}{Problems}
    \begin{itemize}
       % \item How can we utilize the distinctive properties of Hilbert spaces of complex analytic functions to expand established identities concerning the Area of the image of the unit disk under a holomorphic function to various function spaces and structures? 
        \item How can we expand \textbf{established identities concerning the area of the image of domains} under a holomorphic map in different Bergman spaces?
        \item Can we study the \textbf{structural properties of integral operators} (such as Toeplitz and Hankel operators) using the properties of Bergman spaces?
        % \item Can we utilize the distinctive characteristics of Hilbert spaces of complex analytic functions to study the structural properties of integral operators such as Toeplitz operators and other operators related to Toeplitz operators through their symbols? 
    %\begin{itemize}
    %    \item The previous REU group showed that the symbols of Toeplitz operators can be manipulated to make the Toeplitz operator (integral operator) resemble the derivative operator. 
    %\end{itemize}
    \end{itemize}
\end{frame}
%\begin{frame}{Nevanlinna--Pick Interpolation}
%    \begin{block}{Interpolation Problem}
%       Initial data set of $ \set{\lambda_i}_{i=1}^{n}\subseteq \mathbb{D}$, target data set of $\set{z_{i}}_{i-1}^{n}\subseteq \mathbb{D}$, finding condition for $\varphi: \mathbb{D} \rightarrow \mathbb{D}$ holomorphic such that $\varphi(\lambda_i) = z_i$.
%    \end{block}
%\end{frame}
\begin{frame}{Literature Review on Previous Results I}
    \begin{itemize}
        \item D'Angelo's excess area identity \cite{d2019hermitian}\newline
        
    Let $h\in A^{1,2}(\mathbb{D})$. Then,
    \begin{align*}
        A_{\mathbb{D}}(zh)-A_{\mathbb{D}}(h)&=\norm{\frac{\partial(zh)}{\partial z}}_{L^{2}(\mathbb{D})}^2 - \norm{\frac{\partial h}{\partial z}}_{L^{2}(\mathbb{D})}^2 \\
        &=\frac{1}{2}\int_0^{2\pi}\left|f(e^{i\theta})\right|^2d\theta\\
        &= \pi \norm{Sh}^2_{L^{2}(b\mathbb{D})}
    \end{align*}
    where \(Sh\) is the restriction of \(h\) to the unit circle.
    \end{itemize}
\end{frame}
\begin{frame}{Literature Review on Previous Results II}
    \begin{itemize}
        \item Excess area identity with Blaschke product multiplier
    \item 'Excess area' identity for harmonic functions \cite{bambico2022generalization}\newline
        \item Generating symbols for Toeplitz operators for a given initial $p$ and target polynomial $q$ on unit disc and polydisc, $T^{\mathbb{D}}_{\varphi}(p) = q$ and $T^{\mathbb{D}^n}_{\varphi}(p) = q$ \cite{AshleyAdenCelikDanielLuke2024}\newline
    \item Substituted derivatives for Toeplitz operators in excess area identity \cite{AshleyAdenCelikDanielLuke2024}
    \end{itemize}
\end{frame}
%\begin{frame}{Excess Area Identity}
%    \begin{block}{Excess Area Identity}
%    Let $h\in A^{1,2}(\mathbb{D})$. Then,
%    \begin{align*}
%        %\norm{(zh)'}_{L^{2}(\mathbb{D})}^2 - \norm{h'}_{L^{2}(\mathbb{D})}^2 &= \pi \sum_{k=0}^{\infty}|h_k|^2.
%        \norm{(zh)'}_{L^{2}(\mathbb{D})}^2 - \norm{h'}_{L^{2}(\mathbb{D})}^2 &=\frac{1}{2}\int_0^{2\pi}\left|f(e^{i\theta})\right|^2d\theta = \pi \norm{Sh}^2_{L^{2}(b\mathbb{D})}
%    \end{align*}
%    \cite{d2019hermitian}
%    \end{block}
%\end{frame}
%\begin{frame}{Motivation to the Problem}
%\begin{itemize}
%    \item We are interested in weights of the form $\lambda(z,\bar{z}) = \left(1-|z|^2\right)^{\alpha}$, where $\alpha \geq 0$, since if $f,\frac{\partial f}{\partial z}\in A^{2}(\mathbb{D},\lambda)$, then so too are $T_{\varphi}(f)$ and $T_{\varphi}\left(\frac{\partial f}{\partial z}\right)$. \cite{Zeytuncu2013} 
%   
%   \item The regularity of Toeplitz operators on the weighted unit disc allows us to extend a previous result 
%
%%   \item[] Finally, state your result in one or two sentences without showing any calculations.\medskip
%
%%   \item[] If necessary, use two slides. Avoid cluttering each slide with too much information; instead, include reminders and elaborate on them during the presentation.
%\end{itemize}
%   \end{frame}
%\begin{frame}{Generating Symbols to Convert Polynomials}
%    \begin{itemize}
%        \item In \cite{AshleyAdenCelikDanielLuke2024}, it was found that, given $p$ and $q$ holomorphic polynomials with $p\neq 0$, it is possible to algorithmically generate $\varphi(z, \bar{z}) = \phi_1(z) + \overline{\phi_2(z)}$ such that
%        \begin{align*}
%            T_{\varphi}^{\mathbb{D}}(p) = q.
%        \end{align*} 
%        \item We hope to see if this algorithm can (with modification) extend to the weighted disc with a weighted Toeplitz operator, $T^{\mathbb{D},\lambda}_{\phi}(p) = q$.
%    \end{itemize}
%\end{frame}
\section{Results and Observations}
%%%%%%%%%%%%%%%%%%%%
%Slide #5 Results and observations
%%%%%%%%%%%%%%%%%%%%
\begin{frame}{Summary of Results}
\begin{enumerate}
 \item[1.] Results and Observations influenced by the Area Difference of the image of $\mathbb{D}$ between $zh$ and $h$: 
   \begin{itemize}
    \item[i.] On $\mathcal{F}^2=A^2(\mathbb{C},e^{-|z|^2})$, $A^2(\mathbb{D},\lambda)$, $A^2(D(0,r))$
    \item[ii.] On convergence of identities on certain weighted discs.
  \end{itemize}
\item[2] Results and Observations influenced by symbol-generating algorithm for Toeplitz Operators
\begin{itemize}
\item[i.] On unweighted and weighted Toeplitz operators relation
%\item[ii.] On products of Toeplitz operators on $A^2(\mathbb{D})$ (Experimenting Nevanlinna-Pick interpolation problem with Toeplitz operators on $\mathbb{D}$)
\item[ii.] On creating symbols for Unweighted and weighted Hankel operators and commutator operators on $A^2(\mathbb{D})$ %and $A^2(\mathbb{A}(0,r,1))$
%\item[iii.] On constructing symbols for Toeplitz operators on $A^{2}\left(\mathbb{A}(0,r,1)\right)$.
\end{itemize}
\end{enumerate}
\end{frame}

\begin{frame}{Methods Used}
    \begin{itemize}
        \item Relation between $L^{2}$ norms of functions and $\ell^{2}$ norms of Taylor series:
        \[\norm{h}^2_{L^2(\mathbb{D})} = \sum^\infty_{k = 0} \frac{|h_k|^2}{k+1}\]
        \item Integration by parts via Stokes's theorem on forms:
        \begin{align*}
            \oint_{b\Omega}f\: dz &= \int_{\Omega} \overline{\frac{\partial f}{\partial z}}\:d\overline{z}\wedge dz\\
            \oint_{b\Omega}f\:d\overline{z} &= \int_{\Omega}\frac{\partial f}{\partial z}dz\wedge d\overline{z}.
        \end{align*}
        \item Inequalities e.g. Cauchy-Schwarz inequality, H\"{o}lder's inequality 
        \item Beta, Gamma, and Hypergeometric functions 
    \end{itemize}
\end{frame}
\begin{frame}[allowframebreaks]{Using Integration by Parts to find Excess Area Identity: Wedge Product}
  The area is integrated with respect to $dA = dx \wedge dy$. The wedge product has the following properties:
  \begin{align*}
    \left(a+b\right)\wedge c &= a\wedge c + b\wedge c\\
    a\wedge b &= -b\wedge a\\
    a\wedge a &= 0.
  \end{align*}
  With $z = x + iy$, $\overline{z} = x-iy$, the substitution $x = \frac{z + \overline{z}}{2}$, $y = \frac{z - \overline{z}}{2i}$ yields
  \begin{align*}
    dx\wedge dy &= \frac{1}{2i}\left(d\overline{z}\wedge dz\right)\\
                &= -\frac{1}{2i}\left(dz \wedge d\overline{z}\right).
  \end{align*}
  The area integral is now rewritten as:
  \begin{align*}
    \iprod{\frac{\partial h}{\partial z}}{\frac{\partial h}{\partial z}}_{L^{2}\left(\Omega,\lambda\right)} &= \int_{\Omega}^{} \left(\overline{\frac{\partial h}{\partial z}}\right)\left(\frac{\partial h}{\partial z}\right)\lambda\left(|z|\right)\:dx\wedge dy\\
                                                                                                            &= \frac{1}{2i}\int_{\Omega}^{} \lambda\left(|z|\right)\:\left(\left(\overline{\frac{\partial h}{\partial z}}\right)d\overline{z}\right)\wedge \left(\left(\frac{\partial h}{\partial z}\right)dz\right)
  \end{align*}
\end{frame}
\begin{frame}[allowframebreaks]{Using Integration by Parts to find Excess Area Identity: Stokes's Theorem}
  \small
  In particular,
  \begin{align*}
    \overline{\frac{\partial}{\partial z}}\left(\left(\lambda\left(|z|\right)\right)\overline{h}\frac{\partial h}{\partial z}\right)d\overline{z}\wedge dz &= \underbrace{\left(\lambda\left(|z|\right)\right)\overline{\frac{\partial h}{\partial z}}d\overline{z}\wedge \frac{\partial h}{\partial z}dz}_{\text{area integrand}} + \left(\overline{\frac{\partial}{\partial z}}\lambda\left(|z|\right)\right)\overline{h}\wedge \frac{\partial h}{\partial z}dz
    \intertext{meaning}
    \frac{1}{2i}\int_{\Omega}^{} \frac{\partial h}{\partial z}\overline{\frac{\partial h}{\partial z}}\lambda\left(|z|\right)\:d\overline{z}\wedge dz &= \underbrace{\frac{1}{2i}\int_{\Omega}^{} \overline{\frac{\partial}{\partial z}}\left(\lambda\left(|z|\right)\overline{h}\frac{\partial h}{\partial z}\right)\:d\overline{z}\wedge dz}_{\text{Integral $A$}}\\
                                                                                                                                                                         &- \frac{1}{2i}\int_{\Omega}^{} \overline{h}\frac{\partial h}{\partial z}\left(\overline{\frac{\partial}{\partial z}}\lambda\left(|z|\right)\right)\:d\overline{z}\wedge dz.
  \end{align*}
  Turning our attention to Integral $A$,
  \begin{align*}
    \frac{1}{2i} &= \int_{\Omega}^{} d\left(\lambda\left(|z|\right)\overline{h}\frac{\partial h}{\partial z}\right)\:d\overline{z}\wedge dz\\
                 &= \underbrace{\int_{b\Omega}^{} \lambda\left(|z|\right)\overline{h}\frac{\partial h}{\partial z}\:dz}_{= 0}.
  \end{align*}
  With this, the area integral is now
  \begin{align*}
    \frac{1}{2i}\int_{}^{} \frac{\partial h}{\partial z}\overline{\frac{\partial h}{\partial z}}\lambda\left(|z|\right)\:d\overline{z}\wedge dz &= -\frac{1}{2i}\int_{}^{} \overline{h}\frac{\partial h}{\partial z}\left(\overline{\frac{\partial}{\partial z}}\lambda\left(|z|\right)\right)\:d\overline{z}\wedge dz
  \end{align*}
\end{frame}
\begin{frame}{Excess Area on Fock Spaces}
    \begin{center}
        \small
        D'Angelo's Identity: 
        \[A_{\mathbb{D}}(zh) - A_{\mathbb{D}}(h) = \frac{1}{2}\int_0^{2\pi}\left|f(e^{i\theta})\right|^2d\theta = \pi \norm{Sh}_{L^{2}(b\mathbb{D})}^2\]
    \end{center}
    \begin{block}{Excess Area on Fock Space}
    Given $h\in \mathcal{F}^2 $ with $\frac{\partial h}{\partial z}$,
    \begin{align*}
        &A_{\mathcal{F}^2}\left(zh\right) - A_{\mathcal{F}^2}\left(h\right) \\
        &= \pi \norm{zT^{\mathcal{F}^2}_{\ol{z}}\left(h\right)}_{\mathcal{F}^2}^2+\pi\norm{T^{\mathcal{F}^2}_{\ol{z}}\left(h\right)}_{\mathcal{F}^2}^2+ \pi\norm{H^{\mathcal{F}^2}_{\ol{z}}\left(h\right)}_{\mathcal{F}^2}^2
        %&= \pi\norm{z^{2}h}^2_{\mathcal{F}^2}-2\pi\norm{zh_\rho}^2_{\mathcal{F}^2}+\pi\norm{h_\rho}^2_{\mathcal{F}^2}
    \end{align*}
       % \[\mathcal{F}^2 \coloneq A^2\prn{\C, e^{-|z|^2}}\]
    \end{block}  
    Here, the restriction of \(h\) to the unit circle in D'Angelo's identity is replaced with the Bergman projection on $\mathbb{C}$.\\
    %Proved in two ways: Taylor series representation of holomorphic functions, and a combination of Integration by Parts and Stoke's theorem.
%    Let $0 < \rho < 1$, given some \(h \in \mathcal{F}^2\), $h_{\rho}(z) \coloneq h(\rho z)$:
%    \[A(h_{\rho}) \coloneq \pi \norm{h_\rho{}'}_{\mathcal{F}^2}^2 = \pi \left(\norm{zh_{\rho}}_{\mathcal{F}^2}^2 - \norm{h_{\rho}}_{\mathcal{F}^2}^2\right)\]  
%    \[A(zh_{\rho}) - A(h_{\rho}) = \pi\left(\norm{z^2h_{\rho}}_{\mathcal{F}^2}^2 - 2\norm{zh_{\rho}}_{\mathcal{F}^2}^2 + \norm{h_{\rho}}_{\mathcal{F}^2}^2\right)\]
\end{frame}
%\begin{frame}[allowframebreaks]{Using Integration by Parts to find Excess Area Identity: Calculation}
%  \footnotesize
%  Set $\lambda = e^{-|z|^2} = e^{-z\overline{z}}$. Then, for $h,\frac{\partial h}{\partial z} \in A^{2}\left(\mathbb{C},e^{-|z|^2}\right)$, we have
%  \begin{align*}
%    \norm{\frac{\partial h}{\partial z}}_{A^2\left(\mathbb{C},e^{-|z|^2}\right)}^2 &= \frac{1}{\pi}\int_{\C}^{} \left(\frac{\partial h}{\partial z}\right)\overline{\left(\frac{\partial h}{\partial z}\right)}e^{-|z|^2}\:dx\wedge dy\\
%                                                                                   &= \frac{1}{2\pi i}\int_{\C}^{} \overline{\left(\frac{\partial h}{\partial z}\right)}e^{-|z|^2}\:d\overline{z}\wedge \left(\frac{\partial h}{\partial z}\right) dz
%  \end{align*}
%  We now use integration by parts to move the derivative from $\overline{\left(\frac{\partial h}{\partial z}\right)}$ to $e^{-|z|^2}$.
%  \begin{align*}
%    \overline{\frac{\partial}{\partial z}}\left(\overline{h}e^{-|z|^2}\frac{\partial h}{\partial z}\right) &= \left(\left(\overline{\frac{\partial h}{\partial z}}\right)e^{-|z|^2}\right)\left(\frac{\partial h}{\partial z}\right) + \left(\overline{h}\overline{\frac{\partial}{\partial z}}\left(e^{-|z|^2}\right)\right)\left(\frac{\partial h}{\partial z}\right)\\
%                                                                                                           &+ \cancelto{0}{\left(\overline{h}e^{-|z|^2}\right)\overline{\frac{\partial}{\partial z}}\left(\frac{\partial h}{\partial z}\right)}.\\
%    \left(\frac{\partial h}{\partial z}\right)\overline{\left(\frac{\partial h}{\partial z}\right)}e^{-|z|^2} &= \overline{\frac{\partial }{\partial z}}\left(\overline{h}e^{-|z|^2}\frac{\partial h}{\partial z}\right) + \left(\overline{h}\right)\left(z\frac{\partial h}{\partial z}\right)\left(e^{-|z|^2}\right)
%  \end{align*}
%  Rewriting the integral, we now have
%  \begin{align*}
%    \norm{\frac{\partial h}{\partial z}}_{A^{2}\left(\C,e^{-|z|^2}\right)}^2 &= \underbrace{\frac{1}{2\pi i}\int_{\C}^{} \overline{\frac{\partial }{\partial z}}\left(\overline{h}e^{-|z|^2}\frac{\partial h}{\partial z}\right)\:d\overline{z}\wedge dz}_{\text{Integral $A$}}\\
%                                                                             &+ \frac{1}{2\pi i}\int_{\C}^{} \left(\overline{h}\right)\left(z\frac{\partial h}{\partial z}\right)\left(e^{-|z|^2}\right)\:d\overline{z}\wedge dz
%  \end{align*}
%  Let's turn our attention to Integral $A$.
%  \begin{align*}
%    \frac{1}{2\pi i}\int_{\C}^{} \overline{\frac{\partial }{\partial z}}\left(\overline{h}e^{-|z|^2}\frac{\partial h}{\partial z}\right)\:d\overline{z}\wedge dz &= \frac{1}{2\pi i}\lim_{r\rightarrow\infty} \int_{D(0,r)}^{} \overline{\frac{\partial }{\partial z}}\left(\overline{h}e^{-|z|^2}\frac{\partial h}{\partial z}\right)\:d\overline{z}\wedge dz\\
%                                                                                                                                                                                         &= \frac{1}{2\pi i } \lim_{r\rightarrow\infty}\int_{D(0,r)}^{} d\left(\overline{h}e^{-|z|^2}\frac{\partial h}{\partial z}\right)\:d\overline{z}\wedge dz.
%  \end{align*}
%  We can now use Stokes's Theorem to move the integral in the limit from the interior of the disc to the boundary of the disc (dropping a derivative in the process).
%  \begin{align*}
%    \frac{1}{2\pi i } \lim_{r\rightarrow\infty}\int_{D(0,r)}^{} d\left(\overline{h}e^{-|z|^2}\frac{\partial h}{\partial z}\right)\:d\overline{z}\wedge dz &= \frac{1}{2\pi i}\lim_{r\rightarrow\infty}\int_{b D(0,r)}^{} \overline{h}e^{-|z|^2}\frac{\partial h}{\partial z}\:dz\\
%                                                                                                                                                                    &= \frac{1}{2\pi i}\lim_{r\rightarrow\infty}e^{-r^2}\int_{bD(0,r)}^{} \overline{h}\frac{\partial h}{\partial z}\:dz\\
%                                                                                                                                                                                                                                                                                                                                                                                                                                                                                                                                                                    &= 0
%  \end{align*}
%  Returning to the original area integral, we now have
%  \begin{align*}
%    \norm{\frac{\partial h}{\partial z}}_{A^{2}\left(\C,e^{-|z|^2}\right)}^{2}&= \frac{1}{2\pi i}\int_{\C}^{} \left(\overline{h}\right)\left(z\frac{\partial h}{\partial z}\right)e^{-|z|^2}\:d\overline{z}\wedge dz
%  \end{align*}
%\end{frame}
\begin{frame}{Excess Area on $A^2(\mathbb{D},\lambda)$}
%\begin{align*}
%    A(zh)-A(h)=\norm{Sh}^2_{L^2(b\mathbb{D})}
%\end{align*}
    \begin{center}
        \small
        D'Angelo's Identity: 
        \[A_{\mathbb{D}}(zh) - A_{\mathbb{D}}(h) = \frac{1}{2}\int_0^{2\pi}\left|f(e^{i\theta})\right|^2d\theta = \pi \norm{Sh}_{L^{2}(b\mathbb{D})}^2\]
    \end{center}
\begin{block}{Excess Area on $A^{2}(\mathbb{D},\lambda)$}
    Let $h\in A^{1,2}(\mathbb{D},\lambda)$, $\lambda(z) = 1-|z|^2$. Then,
    \[ A_{\disc, \lambda}\left(z^{m+1}h\right)-A_{\disc, \lambda}\left(z^{m}h\right)=\pi\norm{z^mh}^2_{L^2(\mathbb{D},\lambda)}.\]
\end{block}
    % Set \(\Omega = \disc\) and \(\lambda(|z|) = 1 - |z|^2\).
    %Given some $h\in A^{1,2}(\mathbb{D},\lambda)$,  
%    \[ A_{\lambda}(h):=\norm{h'}^2_{L^2(\mathbb{D},\lambda)}\] 
    Here, the restriction of $h$ to the unit circle is replaced with the function itself.
\end{frame}
%\begin{frame}{Excess Area on $A^2(D(0,r))$}
%    \begin{block}{Excess Area on $A^2(D(0,r))$, $0 < r < 1$}
%    \small
%        Let $f_{r,a_k}(\zeta)=\left(rf_{a_k}\right)\circ\left(rf_b\right)^{-1}(\zeta)$ and $\left|f_{r,a_k}(\zeta)\right|=r$ when $\left|\zeta\right|=r$, $a_k\neq b$.\newline
%        
%        Let $B_r = \prod_{k=1}^{n}f_{r, a_k}$ be a modified finite Blaschke product. Then,
%        \begin{align*}
%            A_{D(0,r)}(B_rh) - r^{2N}A_{D(0,r)}(h) &= \pi r^{2(N-1)}\sum_{k=1}^{n}m_k\norm{Sh\left(f_{r, a_k}^{-1}\right)}_{L^{2}(bD(0,r))}^2,
%        \end{align*}
%        where $m_k$ is the multiplicity of $f_{r,a_k}$ and $N = \sum_{k=1}^{n}m_k$.
%    \end{block}
%    %\begin{block}{Excess Area on $A^2(D(0,r))$, $0<r<1$}
%    %    Let $h\in A^{1,2}(D(0,r))$,  Then,
%    %    \begin{align*}
%    %        A_{D(0,r)}(zh) - r^2 A_{D(0,r)}(h) &= \pi r^2 \norm{Sh}_{L^{2}(bD(0,r))}^2.
%    %    \end{align*}
%    %\end{block}
%    %Proved in two ways: Taylor series representation of holomorphic functions, and a combination of Integration by Parts and Stoke's theorem.
%\end{frame}
\begin{frame}{``Excess Area'' on $A^{2}(D(0,r))$}
    \begin{block}{Excess Area Identity for Harmonic Functions on $D(0,r)$, $0 < r < 1$}
    \small
        For a harmonic function $u\in L^2(D(0,r)),\;\exists v\in L^{2}\left(D(0,r)\right)$ harmonic conjugate \cite{bambico2022generalization}. Let $h=u+iv$ be the corresponding holomorphic function. Then, 
        \begin{align*}
            &\norm{\frac{\partial (zu)}{\partial z}}^2_{L^2(D(0,r))}-r^2\norm{\frac{\partial u}{\partial z}}^2_{L^2(D(0,r))}\\
            &=\frac{1}{4}\left(\underbrace{r^2\pi\norm{Sh}_{L^2(b D(0,r))}^2}_{A_{D(0,r)}(zh) - r^2A_{D(0,r)}(h)}+2r^2\pi\Re(h_0^2)+\norm{h}^2_{L^2(D(0,r))}\right).
        \end{align*}
    \end{block}
\end{frame}
\begin{frame}{Dilation and Contraction from $A^2(D(0,r))$ to $A^2(\mathbb{D})$}

    Contracting $h\in A^{1, 2}(\mathbb{D})$ by taking $h_r = h(rz)$ for some \(0 < r < 1\),
    \begin{align*}
        A_\disc\left(zh_{r}\right) - A_\disc\left(h_{r}\right) &= \pi \norm{Sh_{r}}_{L^2(b\mathbb{D})}^2\tag{1}\\
        A_{D(0,r)}(zh) - r^2 A_{D(0,r)}(h) &= \pi r^2 \norm{Sh}_{L^{2}(bD(0,r))}^2.\tag{2}
    \end{align*}
    Dilating $h \in A^{1,2}(D(0,r))$ by taking $h_{\frac{1}{r}}=h(\frac{z}{r})$ for some $0 < r < 1$
    \begin{align*}
        A_{D(0,r)}\left(zh_{1/r}\right) - r^2A_{D(0,r)}\left(h_{1/r}\right) &= \pi r^2 \norm{Sh_{1/r}}_{L^{2}(bD(0,r))}^2\tag{3}\\
        A_{\mathbb{D}}(zh) - A_{\mathbb{D}}(h) &= \pi \norm{Sh}_{L^{2}(b\mathbb{D})}^2\tag{4}
    \end{align*}
\end{frame}
\begin{frame}{Approximation for Sequences of Berezin }
    \begin{block}{Weighted Area on $D(0,r)$}
        Let $\lambda_r(z) = \chi_{D(0, r)}\left(1 - \frac{|z|^2}{r^2}\right)^{r^2}$ where \(r > 0\). Then,
        \begin{align*}
            A_{D(0, r),\lambda_r}(h) &= \int_{D(0,r)}\left\vert h'(z)\right\vert^2\left(1 - \frac{|z|^2}{r^2}\right)^{r^2}\: d A(z)
        \end{align*}
    \end{block}
    We find that, as $r\rightarrow\infty$, $A_{D(0, r),\lambda_r}(h) \rightarrow A_{\mathcal{F}^2}(h)$.\newline %Dominated Convergence in $r$ with $g = e^{-|z|^2}$.
    
    Additionally, we know that
   \begin{align*}
      A_{\mathcal{F}^2}\left(h_{\rho}\right)&=\norm{T^{\mathcal{F}^2}_{\overline{z}}h_{\rho}}^2_{\mathcal{F}^2}\\
   \end{align*}
\end{frame}
\begin{frame}{Berezin Transform Convergence}
    \begin{block}{Reproducing Kernel on $A^{2}\left(D(0,r),\lambda_r\right)$}
        \begin{align*}
            K_{D(0,r)}^{\lambda_r}(w, z) &= \frac{1}{\left(1 - \frac{\ol{z}w}{r^2}\right)^{r^2 + 2}}
        \end{align*}
    \end{block}
    $K_{D(0,r)}^{\lambda_r}(w, z)$ uniformly converges on compact subsets of $D(0,r)$.
    \begin{block}{Reproducing Kernel on Fock Space}
        \begin{align*}
            K_{\mathcal{F}^2}(w,z) &= e^{\ol{z}w}
        \end{align*}
    \end{block}
\end{frame}
\begin{frame}{Berezin Transform Convergence, Cont'd}
    \begin{definition}[Berezin Transform (\cite{ZhuBook2007}]
        Let
        \[k^{\Omega, \lambda}_z(w) \coloneq \frac{K_\Omega^ \lambda(w, z)}{\sqrt{K_\Omega^ \lambda(z, z)}}\]
        Then, for some bounded operator $T$ on \(L^2(\Omega, \lambda)\), define \(\mathcal{B}^{\Omega, \lambda} : B(L^2(\Omega, \lambda)) \to L^2(\Omega, \lambda)\)
        \[(\mathcal{B}^{\Omega, \lambda}T)(z) \coloneq \ang{T k^{\Omega, \lambda}_z, k^{\Omega, \lambda}_z}_{L^2(\Omega, \lambda)}\]
    \end{definition}
\end{frame}
\begin{frame}{Berezin Transform Convergence, Cont'd}
\begin{itemize}
    \item For $\varphi \in L^{\infty}(\Omega,\lambda)$, $\mathcal{B}^{\Omega,\lambda}T_{\varphi} = \mathcal{B}^{\Omega,\lambda}M_\varphi$. (see Axler and Zheng, \cite{AxlerZhengToeplitzOperators}).
    \item $\varphi$ is harmonic if and only if $\mathcal{B}^{\Omega,\lambda} M_\varphi = \varphi$  (proof by Engli\v s, \cite{EnglisHarmonicFunctions}).
    \item We find that, for $T^{D(0,r), \lambda_r}_{\varphi} = P^{D(0,r), \lambda_r}M_{\varphi}$, the Berezin transform $\mathcal{B}^{D(0,r), \lambda_r}T^{D(0,r), \lambda_r}_\varphi$ converges pointwise to $\mathcal{B}^{\mathcal{F}^2}T^{\mathcal{F}^2}_{\varphi}$ as $r\rightarrow\infty$ from G\"{o}\u{g}\"{u}\c{s} and \c{S}ahuto\u{g}lu (\cite{SahutogluGogus2020})
    \item This convergence is uniform on compact subsets of $\mathbb{C}$ (proof inspired by G\"{o}\u{g}\"{u}\c{s} and \c{S}ahuto\u{g}lu in \cite{SahutogluGogus2020}).
\end{itemize}
\end{frame}
\begin{frame}{Unweighted and Weighted Toeplitz Operators Relation I}
    Using an extension of \cite[Lemma 2.1]{AshleyAdenCelikDanielLuke2024}
    \newline\newline
    For weight $\lambda(z)=\left(1-|z|^2\right)^\alpha$ ($\alpha \geq 0$) on the unit disc, \(\mathbb{D} = \crl{z \in \C: \left |z| < 1\right.}\):
        \[
        \frac{T^{\disc, \lambda_\alpha}_{\overline{z}^m}(z^n)}{T^{\mathbb{D}}_{\overline{z}^m}(z^n)} =
        \begin{cases}
            \frac{\Gamma(m-n+\alpha-2)\Gamma(n+1)(m+1)}{\Gamma(m-n+2)\Gamma(n+\alpha+2)}&\text{if}\ m \leq n \\
            \text{indeterminate} &\text{else}
        \end{cases}
    \] 
    \newline
    \[T^{\disc, \lambda_\alpha}_{\overline{z}^m}(z^n)=s_{n,m,\alpha}T^{\disc}_{\overline{z}^m}(z^n),\text{ and }\limit{n}{\infty}s_{n,m,\alpha}=1\]
\end{frame}
%\begin{frame}{Unweighted and Weighted Toeplitz Operators Relation II}
%    Using an extension \cite[Lemma 3.1]{AshleyAdenCelikDanielLuke2024}
%    \newline\newline
%    For weight $\lambda\left(z_i\right)=\prod_{i=1}^{n}\left(1-\left|z_i\right|^2\right)^\alpha$ ($\alpha \geq 0$) on the polydisc, \(\mathbb{D}^n = \crl{\crl{z_1,\cdots,z_n} \in \C^n\left| |z_j| < 1 \text{ where }j\in\crl{1,\cdots,n}\right.}\)
%    \[\frac{T^{\mathbb{D}^n,\lambda_\alpha}_{\overline{z}^\gamma}(z^\beta)}{T^{\mathbb{D}^n,}_{\overline{z}^\gamma}(z^\beta)}=\begin{cases}
%        \frac{\prod_{k=1}^n\frac{B(\beta_k+1,\alpha+1)}{B(\beta_k-\gamma_k+1,\alpha+1)}}{\prod_{k=1}^n\frac{\beta_k-\gamma_k+1}{\beta_k+1}}&\text{ if }\beta_1\geq\gamma_1,\dots,\beta_n\geq\gamma_n;\\
%        \text{indeterminate}&\text{ otherwise}
%    \end{cases}\]
%    \newline
%    \[T^{\disc^n, \lambda_\alpha}_{\overline{z}^\gamma}(z^\beta)=s_{\beta,\gamma,\alpha}T^{\disc^n, \lambda_\alpha}_{\overline{z}^\gamma}(z^\beta),\text{ and }\limit{\beta}{\infty}s_{\beta,\gamma,\alpha}=1\]
%\end{frame}
%\begin{frame}{Unweighted Toeplitz Operator on Annulus}
%    \begin{block}{Boundness of Toeplitz Operator on Annulus}
%        Let $p$ be a nontrivial 'Laurent polynomial' (with root outside of the annulus) on $\mathbb{A}\left(r,1\right)$. Then, for $\varphi$ a measurable symbol such that $T^{\mathbb{A}\left(r,1\right)}_{\varphi}\left(p\right) = q$, it must be the case that $\varphi \notin L^{\infty}\left(\mathbb{A}\left(r,1\right)\right)$.
%    \end{block}
%\end{frame}
\begin{frame}{Unweighted and Weighted Commutator on $A^2(\mathbb{D})$}
    \begin{block}{Existence of Commutator Symbols}
        Given $p$ and $q$ are harmonic polynomials and $\frac{\partial}{\partial z} (p)\not=0$, there does not exist a polynomial symbol $\phi$, such that $\left[P^{\mathbb{D}},M_\phi\right](p)=q$ or $\left[P^{\mathbb{D},\lambda},M_\phi\right](p)=q$. 
    \end{block}
    Compare to \cite{AshleyAdenCelikDanielLuke2024}, who worked on constructing Toeplitz symbols mapping between holomorphic polynomials.
\end{frame}
\begin{frame}{Unweighted and Weighted Hankel Operator on $A^2(\mathbb{D})$}
    \begin{block}{Existence of Hankel Operator Symbols}
        Given some holomorphic polynomials $p,q$ where $p$ is not constant, there does not exist a polynomial symbol $\phi$ such that $H^{\mathbb{D}}_\phi(p)=\overline{q}$ or $H^{\mathbb{D},\lambda}_\phi(p)=\overline{q}$  
    \end{block}
\end{frame}
%%%%%%%%%%%%%%%%%%%%
%Slide #6 Propose some questions, suggestions, and conjectures on your part of the project.
%%%%%%%%%%%%%%%%%%%%
\section{Remarks and Future Directions}
\begin{frame}{Remarks on the Annulus}
    \begin{block}{Toeplitz Operator on Monomials on $A^{2}(\mathbb{A}(0,r,1))$}
        For all integers \(m\) and \(n\),
        \[
            T^{\mathbb{A}\left(0,r,1\right)}_{\ol{z}^m}\left(z^n\right) = 
            \begin{cases}
                \frac{2 m r^{2 m} \ln(r)}{\prn{r^{2 m} - 1}}z^{-m - 1} &\text{if}\ n = -1 \\
                \frac{r^{2 m} - 1}{2 m \ln(r)} z^{-1} &\text{if}\ n = m - 1 \\
                \frac{(n - m + 1)\prn{1 - r^{2 n+2}}}{(n+1)\prn{1 - r^{2 n - 2  m +2}}} z^{n-m} &\text{else}
            \end{cases}.
        \]
    \end{block}
    We attempted to find an algorithm to generate $\varphi\in L^{\infty}\left(\mathbb{A}\left(0,r,1\right)\right)$ such that $T^{\mathbb{A}(0,r,1)}_{\varphi}(p) = q$ for given holomorphic Laurent polynomials $p$ and $q$, but ran into trouble beyond the case where $p$ has roots outside $\overline{\mathbb{A}(0,r,1)}$.
     %this is because all functions of the form a/z^n are bounded on the annulus (even if they are unbounded and non-square integrable on the disc)
\end{frame}
\begin{frame}{Future Directions}
    \begin{itemize}
        %\item The existence (or lack thereof) for symbols $T_{\varphi}(p) = q$ on $\mathbb{A}(0,r,1)$.
        \item Existence (or lack thereof) of bounded symbols for Toeplitz operators for a given initial polynomial $p$ and target polynomial $q$ on $\mathbb{A}(0,r,1)$, $T^{\mathbb{A}(0,r,1)}_{\varphi}(p) = q$
        \newline
        \item Extension of 'excess area' identity to harmonic functions in \(L^2\prn{\C, e^{-|z|^2}}\).
        \newline
        \item Connection between non-weighted and weighted Toeplitz operators when the weight is exponential, $(1-|z|^2)^{A}
        e^{\frac{-B}{(1-|z|^2)^{\alpha}}}(A\geq 0,B>0,\alpha>0)$.%\cite{Zeytuncu2013}.
    \end{itemize}
\end{frame}

%%%%%%%%%%%%%%%%%%%%
%Slide #7 To acknowledge your project, thank any individuals or organizations that aided your project research.
%%%%%%%%%%%%%%%%%%%%
%\section{REU Experience}
%\begin{frame}{Experience at the REU}
%  \begin{itemize}
%    \item Group consisted of Sakia Akamah (Rose--Hulman Institute of Technology), Jennifer Yuan (NYU Abu Dhabi), and myself.
%    \item Spent 7 weeks doing various calculations, final week spent preparing the final report and presentation.
%    \item Other projects at the REU included machine learning, coding theory, and mathematical biology, \pause but ours was obviously the coolest.
%  \end{itemize}
%\end{frame}
\section{REU Experience}
\begin{frame}{What is Research Like?}
  \begin{itemize}
    \item The complex analysis group consisted of myself, Jennifer Yuan (NYU Abu Dhabi), and Sakia Akamah (Rose--Hulman Institute of Technology).
    \item Weeks were 9am to 5pm, mostly doing various calculations and updating our collected results document.
    \item We did not fully understand what we were doing a lot of the time.
  \end{itemize}
\end{frame}
\section{Acknowledgements and References}
    \begin{frame}{Acknowledgments}
    \centering
    This work was only possible by the guidance of Dr. \c{C}elik.\newline
    
    We would also like to express our gratitude to Texas A\&M-Commerce and its Mathematics Department for hosting the REU where we conducted this research.\newline
    
    This research is based upon work supported by the National Science Foundation under Grant DMS-2243991.\newline
    
    This presentation was originally given August 2, 2024 at the Texas A\&M University-Commerce. The presentation was prepared by Sakia Akamah, Avinash Iyer, and Jennifer Yuan.
\end{frame}
%%%%%%%%%%%%%%%%%%%%
%Slide #8 References slide. Give credit to people or sources you used while working on this project.
%%%%%%%%%%%%%%%%%%%%
%\begin{frame}{References}
%\nocite{*}
%\bibliographystyle{alpha}
%{\tiny\bibliography{allreferences}}
%\end{frame}
%------------------------------------------
%Following works but takes a long time to complie, only complie for the finial version  
%For how it looks like, refer to 'Demonstration_of_references_pages.tex'
\begin{frame}[allowframebreaks]{References}
\nocite{*}
\bibliographystyle{alpha}
{\small\bibliography{reu_references.bib}}
\end{frame}
%------------------------------------------

%%%%%%%%%%%%%%%%%%%%%%%%%%
% Commands to include a figure:
%\begin{figure}
%\includegraphics[width=\textwidth]{your-figure's-file-name}
%\caption{\label{fig:your-figure}Caption goes here.}
%\end{figure}
%%%%%%%%%%%%%%%%%%%


\end{document}
