\documentclass{beamer}
%
% Choose how your presentation looks.
%
% For more themes, color themes and font themes, see:
% http://deic.uab.es/~iblanes/beamer_gallery/index_by_theme.html
%
\mode<presentation>
{
  \usetheme{default}      % or try Darmstadt, Madrid, Warsaw, ...
  \usecolortheme{default} % or try albatross, beaver, crane, ...
  \usefonttheme{default}  % or try serif, structurebold, ...
  \setbeamertemplate{navigation symbols}{}
  \setbeamertemplate{caption}[numbered]
} 

\usepackage[english]{babel}
\usepackage[utf8]{inputenc}
\usepackage[T1]{fontenc}

%%%%%%%%
\usepackage{amsmath,amssymb,amsfonts} %math
\usepackage{graphicx}
%%%%%%%%
\title[Your Shor test]{The Title of Your Presentation}
\author{Group members’ names (in alphabetical order)\\ 
\\
}
%\institute{Department of Mathematics, Texas A\&M-Commerce}
%%%%%%%%%%
% Commands to include a figure:
\begin{figure}
\includegraphics[width=1in]{TAMUC}
%\caption{\label{fig:your-figure}Caption goes here.}
\end{figure}
%%%%%%%%%%
%\date{Thursday, April 25th, 2024}
%%%%%%%%%%%%%
\begin{document}
%Slide #1: Title page - Group members’ names, school, course, and the professor’s name
%%%%%%%%%%%%%%%%%%%%
\begin{frame}
  \titlepage
\end{frame}
%%%%%%%%%%%%%%%%%%%%
%Slide #2 Summary of the talk
%%%%%%%%%%%%%%%%%%%%
\begin{frame}{Summary}
    \begin{itemize}
        \item Definitions and Notations\bigskip
        
        \item Motivation to the Problem\bigskip
        
        \item Results and Observations\bigskip
        
        \item Remarks, Conjectures, and Future Directions\bigskip
        
        \item Acknowledgments\bigskip
        
        \item References
    \end{itemize}
\end{frame}
%%%%%%%%%%%%%%%%%%%%
%Slide #3 Definitions and notations will be used in the presentation
%%%%%%%%%%%%%%%%%%%%
\begin{frame}{Definitions and Notations}
    \begin{itemize}
        \item[] A critical aspect of a presentation is ensuring that the audience can understand the various terms and notations used throughout the talk. To achieve this, it is recommended to include a slide (or multiple slides, if necessary) that briefly explains the key definitions and notations used in the presentation. This will help the audience follow along with the content and avoid confusion or misunderstandings. \medskip
        
        \item[] So, providing a clear and concise overview of the definitions and notations used in your presentation is always a good approach. 
    \end{itemize}
        
\end{frame}
%%%%%%%%%%%%%%%%%%%%
%Slide #4 All the Problems - short versions! Please, do not write everything; write some reminders and then explain them during the presentation
%%%%%%%%%%%%%%%%%%%%
\begin{frame}{Motivation to the Problem}
\begin{itemize}
   \item[] In your presentation slide, start by briefly explaining the concepts related to the problem. Emphasize the significance of the problem and explain why it needs to be addressed. Support your explanation with examples.\medskip
   
   \item[] Next, state the problem clearly and concisely without using technical terms. Provide context and background information as needed. Consider summarizing the problem and its importance.\medskip

   \item[] Finally, state your result in one or two sentences without showing any calculations.\medskip

   \item[] If necessary, use two slides. Avoid cluttering each slide with too much information; instead, include reminders and elaborate on them during the presentation.
\end{itemize}
   \end{frame}
%%%%%%%%%%%%%%%%%%%%
%Slide #5 Results and observations
%%%%%%%%%%%%%%%%%%%%
\begin{frame}{Results and Observations}
    \begin{itemize}
        \item[] When presenting your work, it is important to thoroughly and comprehensively explain the computations and ideas that led you to your final result. \medskip

        \item[] To ensure accuracy and proper referencing of any external sources, it is recommended to use the AMS citation style (e.g. \cite{bambico2022generalization}).\medskip

        \item[] Additionally, it is crucial to highlight the Complex Analysis concepts used in your project and clearly explain where they are applied to the problem.\medskip 

        \item[] Rather than presenting computations as isolated points or ideas, it is suggested to connect them in a coherent and logical manner that flows naturally. Doing so will make your message straighter and more understandable to the audience.\medskip 

        \item[] By following these guidelines, you can effectively convey your findings and demonstrate the depth of your understanding of the subject matter.
    \end{itemize}
\end{frame}
%%%%%%%%%%%%%%%%%%%%
%Slide #6 Propose some questions, suggestions, and conjectures on your part of the project.
%%%%%%%%%%%%%%%%%%%%
\begin{frame}{Remarks, Conjectures, and Future Directions}
    \begin{itemize}
        \item[] Using this slide to effectively communicate the insights and conclusions you have drawn from your research is important. In this regard, you should discuss any noteworthy observations or patterns you have identified in your investigation and any conjectures you have formed based on these observations.\medskip

        \item[] Additionally, outline your plans for future research in this area, including any potential avenues for further investigation that you intend to pursue. By doing so, you can provide your audience with a sense of the broader context and potential significance of your work.\medskip

        \item[] If you have already solved the problem at hand, describe the generalization of your findings and the implications of your research for related areas of study. This can help others better understand the broader implications of your work and its potential impact on the field. 
    \end{itemize}
\end{frame}
%%%%%%%%%%%%%%%%%%%%
%Slide #7 To acknowledge your project, thank any individuals or organizations that aided your project research.
%%%%%%%%%%%%%%%%%%%%
\begin{frame}{Acknowledgments}
    \begin{itemize}
        \item[] When you are working on a project, it is important to show gratitude to the people (and organizations) who helped you in your research. Make sure to include specific details about their assistance, such as ideas, feedback, funding, expertise, or access to resources. This demonstrates your appreciation, acknowledges their contributions, and helps build positive relationships for future projects. \medskip
        
        \item[] Remember to be genuine and sincere in your acknowledgments, and take the time to personalize each one where possible. 
    \end{itemize}
\end{frame}
%%%%%%%%%%%%%%%%%%%%
%Slide #8 References slide. Give credit to people or sources you used while working on this project.
%%%%%%%%%%%%%%%%%%%%
\begin{frame}{References}
    To write in AMS format, follow the format given below:
\begin{itemize}
    \item Write the first initial or first name of the author, followed by their last name.
    \item Italicize the title of the journal article.
    \item Write the name of the journal in capital letters, abbreviated form, and without italics.
    \item Bold the volume number and write it along with the year of publication.
    \item Mention the issue number along with the page numbers.
    \item Finally, provide the DOI number, with a space followed by a number and without a colon.
\end{itemize}


\bibliographystyle{alpha}
{\small\bibliography{allreferences}}

\end{frame}

%%%%%%%%%%%%%%%%%%%%%%%%%%
% Commands to include a figure:
%\begin{figure}
%\includegraphics[width=\textwidth]{your-figure's-file-name}
%\caption{\label{fig:your-figure}Caption goes here.}
%\end{figure}
%%%%%%%%%%%%%%%%%%%
\begin{frame}{Suggestion (this slide is not a part of the presentation)}
    For your presentation, aim to keep it within 20-25 minutes. Once you have created the presentation, decide on who will present each part and practice presenting it repeatedly. You can reduce the presentation time by rehearsing to 20-25 minutes, even though the first practice may take 30-35 minutes. To prepare for questions from other groups, ask each other questions and practice answering them.
\end{frame}


\end{document}
