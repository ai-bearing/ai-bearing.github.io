%%%%%%%%%%%%%%%%%%%%%%%%%%%%%%%%%%%%%%%%%%%%%%%%%%%%%%
% Thanks to Xu Minghao's work                        %
% I modify it into uchicago version                  %
% to not make new bug, I don't alter "Ritsumeikan"   %
% keywords in file. Pls feel free to use             %
%%%%%%%%%%%%%%%%%%%%%%%%%%%%%%%%%%%%%%%%%%%%%%%%%%%%%%

%%%%%%%%%%%%%%%%%%%%%%%%%%%%%%%%%%%%%%%%%%%%%%%%%%%%%%
% A Beamer template for Ritsumeikan University       %
% Author: Ming-Hao Xu (Xu Minghao)                   %
% Date:   April 2022.                                %
% LPPL Licensed.                                     %
%%%%%%%%%%%%%%%%%%%%%%%%%%%%%%%%%%%%%%%%%%%%%%%%%%%%%%

\documentclass{beamer}
\usepackage{hyperref}
\usepackage[T1]{fontenc}

% other packages
\usepackage{latexsym,amsmath,xcolor,multicol,booktabs,calligra}
\usepackage{graphicx,pstricks,listings,stackengine}

% dummy text; remove it when working on this template
\usepackage{lipsum}

\author{Wynesakia Akamah}
\title{A Functional Analysis of the Heat Equation}
\subtitle{An Introduction to Semigroup Theory}
\institute{
    Department of Mathematics \\
    Rose-Hulman Institute of Technology
}
\date{\small May 22, 2024}
\usepackage{alt_reu_format}


% defs
\def\cmd#1{\texttt{\color{blue}\footnotesize $\backslash$#1}}
\def\env#1{\texttt{\color{blue}\footnotesize #1}}
\definecolor{lightblue}{rgb}{17,114,222}  % Blue
\definecolor{deepgold}{RGB}{241,178,15}  % Gold
\definecolor{deepgreen}{rgb}{0,0.5,0}
\definecolor{halfgray}{gray}{0.55}

\lstset{
    basicstyle=\ttfamily\small,
    keywordstyle=\bfseries\color{lightblue},
    emphstyle=\ttfamily\color{deepgold},    % Custom highlighting style
    stringstyle=\color{deepgreen},
    numbers=left,
    numberstyle=\small\color{halfgray},
    rulesepcolor=\color{blue!20!white!20!gold!20},
    frame=shadowbox,
}



\begin{document}

\begin{frame}
    \titlepage
    \vspace*{-0.6cm}
    \begin{figure}[htpb]
        \centering
        \includegraphics[keepaspectratio, scale=0.14]{Presentation/TAMUC-V-ClearSpace-1.png}
    \end{figure}
\end{frame}

\begin{frame}
    \tableofcontents[sectionstyle=show, subsectionstyle=show/shaded/hide, subsubsectionstyle=show/shaded/hide]
\end{frame}

\section{Introduction}

\begin{frame}{Introduction}
    \begin{itemize}[<+-| alert@+>]
        \item Extending ODEs from finite-dimensional spaces to Banach spaces
        \item Focus on applying this theory to solve the heat equation
        \item Emphasizing the role of semigroups in defining and solving these equations
    \end{itemize}
\end{frame}

\section{Theoretical Framework}

\subsection{ODE Theory}
\begin{frame}
    \begin{definition}[Matrix Exponential ]
    Let \(A\) be an endomorphism on \(\mathbb{R}^n\). The matrix exponential \(e^{t \cdot A}\) is defined as:
    \[
    e^{t \cdot A} = \sum_{k=0}^{\infty} \frac{t^k \cdot A^k}{k!}
    \]
    \end{definition}
\end{frame}

\begin{frame}{System of ODEs}
    \begin{theorem}[Fundamental Theorem of Linear ODE Systems]
    Let \(A\) be an endomorphism on \(\mathbb{R}^n\). For a given \(\mathbf{x}_0 \in \mathbb{R}^n\), the initial value problem
    \[
    \begin{cases}
        \diff{t} \mathbf{x}(t) = A \mathbf{x}(t) \\
        \mathbf{x}(0) = \mathbf{x}_0
    \end{cases}
    \]
    has the unique solution
    \[
    \mathbf{x}(t) = e^{t \cdot A} \mathbf{x}_0
    \]
    for all real numbers \(t\).
\end{theorem}
\end{frame}


\subsection{Banach Spaces and Operators}

\begin{frame}{Banach Spaces and Operators}
    \begin{itemize}[<+-| alert@+>]
        \item Banach spaces: Cauchy-complete normed vector spaces
        \item Bounded linear operators on Banach spaces
        \item Exponential of a bounded linear operator: 
        \[
            e^{t \cdot A} = \sum_{k=0}^{\infty} \frac{t^k \cdot A^k}{k!}
        \]
    \end{itemize}
\end{frame}

\begin{frame}{Generlization}
    \begin{theorem}
        Let \(A\) be a bounded endomorphism on some Banach space \(X\). For a given \(\mathbf{x}_0 \in X\), the initial value problem
        \[
        \begin{cases}
            \diff{t} \mathbf{x} = A \mathbf{x} \\
            \mathbf{x}(0) = \mathbf{x}_0
        \end{cases}
        \]
        has the unique solution
        \[
        \mathbf{x}(t) = e^{t \cdot A} \mathbf{x}_0
        \]
        for all real numbers \(t\).
    \end{theorem}
\end{frame}



\subsection{Semigroup Theory}

\begin{frame}{Semigroup Theory}
    \begin{itemize}[<+-| alert@+>]
        \item Semigroups: Associative algebraic structures
        \item Resolvent set and operator semigroups
        \item Theorem: Existence of operator semigroups for closed operators
    \end{itemize}
\end{frame}

\begin{frame}{Semigroup Theory}
    \begin{theorem}[Exsistence of Operator Semigroups]
           There exists some \(E : [0, \infty) \to B(X)\) that satisfies the following properties:
            \begin{enumerate}[I.]
                \item \(E(0) = I\) (the identity operator),
                \item \(E(s) \circ E(t) = E(s + t)\) for all \(s, t \geq 0\),
                \item \(\norm{E(t)} \leq e^{\alpha \cdot t}\) for all \(t \geq 0\),
                \item \(E\mathbf{x}\) is continuous on \([0, \infty)\) for all \(\mathbf{x} \in X\).
                \item \(E\mathbf{x}\) is differentiable on \([0, \infty)\) for all \(\mathbf{x} \in U\), and
                \[
                \diff{t}E(t)\mathbf{x} = AE(t)\mathbf{x}
                \]
                \item \(E(t) \circ (\lambda \cdot I - A)^{-1} = (\lambda \cdot I - A)^{-1} \circ E(t)\) for all \(\lambda \geq b\).
            \end{enumerate}
        \end{theorem}
\end{frame}

\begin{frame}{Terminology}
        \begin{itemize}[<+-| alert@+>]
        \item \(\{E(t) : t \geq 0\}\), the range of \(E\), is referred to as a strongly continuous, \(\alpha\)-contractive semigroup. 
        \item The operator \(A\) is referred to as the infinitesimal generator of the semigroup \(\{E(t) : t \geq 0\}\).
    \end{itemize}
\end{frame}

\section{Application to the Heat Equation}

\begin{frame}{Application to the Heat Equation}
    \begin{itemize}[<+-| alert@+>]
        \item Heat equation: 
        \[
            \begin{cases}
        \partial_t u(\mathbf{x}, t) = \nabla^2 u(\mathbf{x}, t) &\text{in}\ \Omega \times (0, T] \\
        u(\mathbf{x}, t) = 0 &\text{on}\ \partial\Omega \times [0, T] \\
        u(\mathbf{x}, t) = g &\text{in}\ \Omega \times \{0\} \\
    \end{cases}
        \]
        \item Reformulated in a functional analysis framework
        \item Laplacian operator as an infinitesimal generator of a semigroup
    \end{itemize}
\end{frame}

\begin{frame}{}
     \begin{lemma}
         \(H^1_0(\Omega, \mathbb{R}) \cap H^2(\Omega, \mathbb{R})\) is a dense subset of \(L^2\) and \(\nabla^2\)  acts as a closed linear operator from \(H^1_0(\Omega, \mathbb{R}) \cap H^2(\Omega, \mathbb{R})\) to \(L^2(\Omega, \mathbb{R})\).
     \end{lemma}
\end{frame}


\begin{frame}{}
        \begin{theorem}[The Laplacian as an Infinitesimal Generator]
    The Laplacian operator \(\nabla^2\) generates a strongly continuous, \(\gamma\)- contractive semigroup \(\{E(t) : t \geq 0\}\) on \(L^2(\Omega, \mathbb{R})\).
\end{theorem}
    \end{frame}

\begin{frame}{Weak Solutions}
    \begin{itemize}[<+-| alert@+>]
        \item Write here
    \end{itemize}
\end{frame}

\begin{frame}{Results}
    \begin{itemize}[<+-| alert@+>]
        \item Write here 
    \end{itemize}
\end{frame}


\section{Results}


\section{References}

\begin{frame}[allowframebreaks]
    \bibliography{allreferences}
    \bibliographystyle{ieeetr}
    \nocite{*} % used here because no citation happens in slides
    % if there are too many try use:
    % \tiny\bibliographystyle{alpha}
\end{frame}

\begin{frame}
    \begin{center}
        {\Huge\calligra Thank You}
    \end{center}
\end{frame}

\end{document}


\begin{document}

\begin{frame}
    \titlepage
    \vspace*{-0.6cm}
    \begin{figure}[htpb]
        \begin{center}
            \includegraphics[keepaspectratio, scale=0.14]{rose.png}
        \end{center}
    \end{figure}
\end{frame}

\begin{frame}    
\tableofcontents[sectionstyle=show,
subsectionstyle=show/shaded/hide,
subsubsectionstyle=show/shaded/hide]
\end{frame}

\section{Introduction}

\begin{frame}{Classical ODE SystemTitle}
    \begin{itemize}[<+-| alert@+>] % stepwise alerts
        \item \lipsum[1][1-4]
        \item \lipsum[1][5-8]
    \end{itemize}
\end{frame}


\section{Banach SpacesLiterature Review}

\subsection{GPT3-derived Models DALLE \& CLIP}

\begin{frame}
    \begin{itemize}
        \item \lipsum[2][1-4]
        \item \lipsum[2][5-9]
        \item Results accessible at \newline \url{https://scholar.google.com}
    \end{itemize}
\end{frame}


\section{Methods}

\subsection{Diffusion Model}

\begin{frame}{Title}
    \begin{itemize}
        \item \lipsum[3][1-4]
    \end{itemize}
    \begin{table}[h]
        \centering
        \begin{tabular}{c|c}
            Microsoft\textsuperscript{\textregistered}  Windows & Apple\textsuperscript{\textregistered}  Mac OS \\
            \hline
            Windows-Kernel & Unix-like \\
            Arm, Intel & Intel, Apple Silicon \\
            Sudden update & Stable update \\
            Less security & More security \\
            ... & ... \\
        \end{tabular}
    \end{table}
\end{frame}

\begin{frame}{Algorithms}
    \begin{exampleblock}{Non-Numbering Formula}
        \begin{equation*}
            J(\theta) = \mathbb{E}_{\pi_\theta}[G_t] = \sum_{s\in\mathcal{S}} d^\pi (s)V^\pi(s)=\sum_{s\in\mathcal{S}} d^\pi(s)\sum_{a\in\mathcal{A}}\pi_\theta(a|s)Q^\pi(s,a)
        \end{equation*}
    \end{exampleblock}
    \begin{exampleblock}{Multi-Row Formula\footnote{If text appears in the formula,use $\backslash$mathrm\{\} or $\backslash$text\{\} instead}}
        \begin{align}
            Q_\mathrm{target}&=r+\gamma Q^\pi(s^\prime, \pi_\theta(s^\prime)+\epsilon)\\
            \epsilon&\sim\mathrm{clip}(\mathcal{N}(0, \sigma), -c, c)\nonumber
        \end{align}
    \end{exampleblock}
\end{frame}

\begin{frame}
    \begin{exampleblock}{Numbered Multi-line Formula}
        % Taken from Mathmode.tex
        \begin{multline}
            A=\lim_{n\rightarrow\infty}\Delta x\left(a^{2}+\left(a^{2}+2a\Delta x+\left(\Delta x\right)^{2}\right)\right.\label{eq:reset}\\
            +\left(a^{2}+2\cdot2a\Delta x+2^{2}\left(\Delta x\right)^{2}\right)\\
            +\left(a^{2}+2\cdot3a\Delta x+3^{2}\left(\Delta x\right)^{2}\right)\\
            +\ldots\\
            \left.+\left(a^{2}+2\cdot(n-1)a\Delta x+(n-1)^{2}\left(\Delta x\right)^{2}\right)\right)\\
            =\frac{1}{3}\left(b^{3}-a^{3}\right)
        \end{multline}
    \end{exampleblock}
\end{frame}

\begin{frame}{Graphics and Columns}
    \begin{minipage}[c]{0.3\linewidth}
        \psset{unit=0.8cm}
        \begin{pspicture}(-1.75,-3)(3.25,4)
            \psline[linewidth=0.25pt](0,0)(0,4)
            \rput[tl]{0}(0.2,2){$\vec e_z$}
            \rput[tr]{0}(-0.9,1.4){$\vec e$}
            \rput[tl]{0}(2.8,-1.1){$\vec C_{ptm{ext}}$}
            \rput[br]{0}(-0.3,2.1){$\theta$}
            \rput{25}(0,0){%
            \psframe[fillstyle=solid,fillcolor=lightgray,linewidth=.8pt](-0.1,-3.2)(0.1,0)}
            \rput{25}(0,0){%
            \psellipse[fillstyle=solid,fillcolor=yellow,linewidth=3pt](0,0)(1.5,0.5)}
            \rput{25}(0,0){%
            \psframe[fillstyle=solid,fillcolor=lightgray,linewidth=.8pt](-0.1,0)(0.1,3.2)}
            \rput{25}(0,0){\psline[linecolor=red,linewidth=1.5pt]{->}(0,0)(0.,2)}
%           \psRotation{0}(0,3.5){$\dot\phi$}
%           \psRotation{25}(-1.2,2.6){$\dot\psi$}
            \psline[linecolor=red,linewidth=1.25pt]{->}(0,0)(0,2)
            \psline[linecolor=red,linewidth=1.25pt]{->}(0,0)(3,-1)
            \psline[linecolor=red,linewidth=1.25pt]{->}(0,0)(2.85,-0.95)
            \psarc{->}{2.1}{90}{112.5}
            \rput[bl](.1,.01){C}
        \end{pspicture}
    \end{minipage}\hspace{2cm}
    \begin{minipage}{0.5\linewidth}
        \medskip
        % \hspace{2cm}
        \begin{figure}[h]
            \centering
            \includegraphics[height=.4\textheight]{pic/sample.pdf}
        \end{figure}
    \end{minipage}
\end{frame}

\begin{frame}[fragile]{\LaTeX{} Common Commands}
    \begin{exampleblock}{Commands}
        \centering
        \footnotesize
        \begin{tabular}{llll}
            \cmd{chapter} & \cmd{section} & \cmd{subsection} & \cmd{paragraph} \\
            chapter & section & sub-section & paragraph \\\hline
            \cmd{centering} & \cmd{emph} & \cmd{verb} & \cmd{url} \\
            center & emphasize & original & hyperlink \\\hline
            \cmd{footnote} & \cmd{item} & \cmd{caption} & \cmd{includegraphics} \\
            footnote & list item & caption & insert image \\\hline
            \cmd{label} & \cmd{cite} & \cmd{ref} \\
            label & citation & refer\\\hline
        \end{tabular}
    \end{exampleblock}
    \begin{exampleblock}{Environment}
        \centering
        \footnotesize
        \begin{tabular}{lll}
            \env{table} & \env{figure} & \env{equation}\\
            table & figure & formula \\\hline
            \env{itemize} & \env{enumerate} & \env{description}\\
            non-numbering item & numbering item & description \\\hline
        \end{tabular}
    \end{exampleblock}
\end{frame}

\begin{frame}[fragile]{\LaTeX{} Examples of environmental commands}
    \begin{minipage}{0.5\linewidth}
\begin{lstlisting}[language=TeX]
\begin{itemize}
  \item A \item B
  \item C
  \begin{itemize}
    \item C-1
  \end{itemize}
\end{itemize}
\end{lstlisting}
    \end{minipage}\hspace{1cm}
    \begin{minipage}{0.3\linewidth}
        \begin{itemize}
            \item A
            \item B
            \item C
            \begin{itemize}
                \item C-1
            \end{itemize}
        \end{itemize}
    \end{minipage}
    \medskip
    \pause
    \begin{minipage}{0.5\linewidth}
\begin{lstlisting}[language=TeX]
\begin{enumerate}
  \item A \item B
  \item C
  \begin{itemize}
    \item[n+e]
  \end{itemize}
\end{enumerate}
\end{lstlisting}
    \end{minipage}\hspace{1cm}
    \begin{minipage}{0.3\linewidth}
        \begin{enumerate}
            \item A
            \item B
            \item C
            \begin{itemize}
                \item[n+e]
            \end{itemize}
        \end{enumerate}
    \end{minipage}
\end{frame}

\begin{frame}[fragile]{\LaTeX{} Formulas}
    \begin{columns}
        \begin{column}{.55\textwidth}
\begin{lstlisting}[language=TeX]
$V = \frac{4}{3}\pi r^3$

\[
  V = \frac{4}{3}\pi r^3
\]

\begin{equation}
  \label{eq:vsphere}
  V = \frac{4}{3}\pi r^3
\end{equation}
\end{lstlisting}
        \end{column}
        \begin{column}{.4\textwidth}
            $V = \frac{4}{3}\pi r^3$
            \[
                V = \frac{4}{3}\pi r^3
            \]
            \begin{equation}
                \label{eq:vsphere}
                V = \frac{4}{3}\pi r^3
            \end{equation}
        \end{column}
    \end{columns}
    \begin{itemize}
        \item more information \href{https://ja.overleaf.com/learn/latex/Mathematical_expressions}{\color{purple}{here}}
    \end{itemize}
\end{frame}

\begin{frame}[fragile]
    \begin{columns}
        \column{.6\textwidth}
\begin{lstlisting}[language=TeX]
\begin{table}[htbp]
  \caption{numbers & meaning}
  \label{tab:number}
  \centering
  \begin{tabular}{cl}
    \toprule
    number & meaning \\
    \midrule
    1 & 4.0 \\
    2 & 3.7 \\
    \bottomrule
  \end{tabular}
\end{table}
\end{lstlisting}
        \column{.4\textwidth}
        \begin{table}[htpb]
            \centering
            \caption{numbers \& meaning}
            \label{tab:number}
            \begin{tabular}{cl}\toprule
                numbers & meaning \\\midrule
                1 & 4.0\\
                2 & 3.7\\\bottomrule
            \end{tabular}
        \end{table}
        \normalsize formula~(\ref{eq:vsphere}) at previous slide and Table~\ref{tab:number}。
    \end{columns}
\end{frame}

\section{Results}
\begin{frame}
    \begin{itemize}
        \item \lipsum[4][1-4]
        \item \lipsum[4][5-9]
        \item \lipsum[5][1-4]
        \item \lipsum[5][5-8]
    \end{itemize}
\end{frame}

\section{References}

\begin{frame}[allowframebreaks]
    \bibliography{ref}
    \bibliographystyle{ieeetr}
    \nocite{*} % used here because no citation happens in slides
    % if there are too many try use:
    % \tiny\bibliographystyle{alpha}
\end{frame}


\begin{frame}
    \begin{center}
        {\Huge\calligra Thank You}
    \end{center}
\end{frame}

\end{document}