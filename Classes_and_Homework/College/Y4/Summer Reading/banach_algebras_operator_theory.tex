\documentclass[12pt]{extarticle}

\title{}
\author{}
\date{}
\usepackage[shortlabels]{enumitem}

%paper setup
\usepackage{geometry}
\geometry{letterpaper, portrait, margin=1in}
\usepackage{fancyhdr}
% sans serif font:
%\usepackage{cmbright}
%\usepackage{sfmath}

%symbols
\usepackage{amsmath}
\usepackage{bigints}
\usepackage{amssymb}
\usepackage{amsthm}
\usepackage{mathtools}

\usepackage[hidelinks]{hyperref} %hyperlinks
\usepackage{gensymb} %more symbols
\usepackage{multirow,array} %better tables
\usepackage{multicol} %multiple columns per page
\usepackage{bbold} %better blackboard bold
\usepackage[T1]{fontenc}
\usepackage[utf8]{inputenc}

\usepackage{pgfplots}
\usepackage{tikz}
\usetikzlibrary{cd}
\tikzset{middleweight/.style={pos = 0.5}}

\usepackage{graphicx}
\graphicspath{ {./images/} }

\setlength{\parindent}{0pt} %I don't like indentation
\usepackage{cancel} %better X-throughs
\pagestyle{fancy} %better headers
\fancyhf{}
\rhead{Avinash Iyer}
\lhead{\textit{Banach Algebra Techniques in Operator Theory}}

%canonical sets
\newcommand{\N}{\mathbb{N}}
\newcommand{\Q}{\mathbb{Q}}
\newcommand{\Z}{\mathbb{Z}}
\newcommand{\R}{\mathbb{R}}
\newcommand{\C}{\mathbb{C}}
\newcommand{\F}{\mathbb{F}}

%common other symbols
\newcommand{\mcc}{\mathcal{C}} %cantor set
\newcommand{\mco}{\mathcal{O}} %holomorphic functions
\newcommand{\mfp}{\mathfrak{p}} %prime ideal

%inner products and norms
\newcommand{\iprod}[2]{\left\langle #1,#2\right\rangle}
\newcommand{\norm}[1]{\left\Vert #1\right\Vert}
%\newcommand{\bra}[1]{\left\langle#1\right\vert}
%\newcommand{\ket}[1]{\left\vert#1\right\rangle}
%\newcommand{\braket}[2]{\left\langle#1\mid#2\right\rangle}
\newcommand{\set}[1]{\left\{#1\right\}}
\newcommand{\ve}{\varepsilon}
\DeclareMathOperator*{\esssup}{ess\,sup}

\setcounter{secnumdepth}{0}

\theoremstyle{plain}
\newtheorem*{theorem}{Theorem}
\newtheorem*{lemma}{Lemma}
\newtheorem*{corollary}{Corollary}
\newtheorem*{proposition}{Proposition}

\theoremstyle{definition}
\newtheorem*{definition}{Definition}
\newtheorem*{example}{Example}

\newtheoremstyle{note}
  {3pt}%
  {3pt}%
  {}%
  {}%
  {\bfseries}%
  {:}%
  { }%
  {}%
\theoremstyle{note}
\newtheorem*{note}{Note}
\newtheorem*{remark}{Remark}
%\newtheorem*{claim}{Claim}

\usepackage{newpxtext,eulerpx}
\renewcommand*{\mathbb}[1]{\varmathbb{#1}}
\usepackage[document]{ragged2e}
\renewcommand{\newline}{\hfill\break}
\renewcommand{\thefootnote}{\roman{footnote}}
\begin{document}
\section{Prelude}%
  My REU mentor recently bought me the book \textit{Banach Algebra Techniques in Operator Theory}, so I'm going to be reading through it here. Astute readers may already know that I am also reading through the book \textit{Quantum Theory for Mathematicians}, and may be wondering if this is going to crowd out that book. The answer is yes --- but I don't really care that much. If I come out of the summer knowing more things than I knew entering, then I will have succeeded.
\section{Prerequisite Notes}%
Since Douglas's book is very advanced, I'm going to end up going back and reading other important material in order to contextualize the parts of the book I don't fully understand.
\subsection{Tychonoff's Theorem}%
I'm drawing information for this section from Volker Runde's book \textit{A Taste of Topology}, specifically from Chapter 3.
\begin{definition}[Product Topology]
  Let $\set{\left(X_{i},\tau_{i}\right)}_{i}$ be a family of topological spaces, and $\displaystyle X = \prod_{i\in I}X_i$.\newline

  The product topology on $X$ is the coarsest topology $\tau$ on $X$ such that
  \begin{align*}
    \prod_{i}: X \rightarrow X_i;~~f\mapsto f(i)
  \end{align*}
  is continuous.
\end{definition}
The product topology's open sets are of the form
\begin{align*}
  \bigcap_{j=1}^{n}\pi_{i_j}\left(U_j\right),
\end{align*}
where $i_{j}\in I$. The product topology is the topology of coordinatewise convergence.
\begin{theorem}[Tychonoff]
  Let $\set{\left(K_i,\tau_i\right)}_{i\in I}$ be a nonempty family of compact topological spaces. Then, the product space $\displaystyle K = \prod_{i\in I}K_i$ is compact in the product topology.
\end{theorem}
\begin{proof}
  Let $\set{f_{\alpha}}_{\alpha\in A}$ be a net\footnote{See future definition of nets.} in $K$. Let $J\subseteq I$ be nonempty, and let $f\in K$.\newline

  We call $\left(J,f\right)$ a partial accumulation point of $\set{f_{\alpha}}_{\alpha \in A}$ if $f\vert_{J}$ is a accumulation point of $\set{f_{\alpha}\vert_{J}}_{\alpha \in A}$ in $\displaystyle \prod_{j\in J}K_j$. A partial accumulation point of $\set{f_{\alpha}}_{\alpha \in A}$ is a accumulation point of $\set{f_{\alpha}}_{\alpha\in A}$  if and only if $J = I$.\newline

  Let $\mathcal{P}$ be the set of partial accumulation points of $\set{f_{\alpha}}_{\alpha \in A}$ For any two $\left(J_f,f\right),\left(J_g,g\right)\in \mathcal{P}$, define the order $\left(J_f,f\right)\leq \left(J_g,g\right)$ if and only if $J_f\subseteq J_g$ and $g\vert_{J_f} = f$.\newline

  Since $K_i$ is compact for each $i\in I$, the net $\set{f_{\alpha}}_{\alpha}$ has partial accumulation points $\left(\set{i},f_i\right)$ for each $i\in I$ (since each $K_i$ is compact, the net analogue to sequential compactness holds); in particular, $\mathcal{P}$ is nonempty.\newline

  Let $\mathcal{Q}$ be a totally ordered subset of $\mathcal{P}$, and $J_g = \bigcup\set{J_f\mid \left(J_f,f\right)\in \Q}$. Define $g$ by letting $g(j) = f(j)$ for each $j\in J_f$ with $\left(J_f,f\right)\in \mathcal{Q}$, and arbitrarily on $I\setminus J_g$.\newline

  Since $\mathcal{Q}$ is totally ordered, $g$ is well-defined. We claim that $\left(J_g,g\right)$ is a partial accumulation point of $\set{f_{\alpha}}_{\alpha}$.\newline

  Let $\displaystyle N\subseteq \prod_{j\in J_g} K_j$ be a neighborhood of $g\vert_{J_g}$. We may suppose that
  \begin{align*}
    N = \pi_{j_1}^{-1}\left(U_{j_1}\right) \cap \cdots \cap \pi_{j_n}\left(U_{j_n}\right),
  \end{align*}
  where $j_1,\dots,j_n\in J_g$, and $U_{j_i} \subseteq K_{j_i}$ are open.\newline

  Let $\left(J_{h},h\right)\in \mathcal{Q}$ be such that $\set{j_1,\dots,j_n}\subseteq J_h$, which is possible since $\mathcal{Q}$ is totally ordered. Since $\left(J_h,h\right)$ is a partial accumulation point of $\set{f_{\alpha}}_{\alpha}$, there is an index $\alpha$ and a $\beta \geq \alpha$, where
  \begin{align*}
    f_{\beta}\left(j_k\right) = \pi_{j_k}\left(f_{\beta}\right)U_{j_k},
  \end{align*}
  so $f_{\beta}\in N$. Thus, $\left(J_g,g\right)$ is a partial accumulation point of $\set{f_{\alpha}}_{\alpha}$, and is an element of $\mathcal{P}$.\newline

  By Zorn's lemma,\footnote{In a partially ordered set, if every totally ordered subset contains an upper bound, the set contains a maximal element} $\mathcal{P}$ has a maximal element, $\left(J_{\text{max}},f_{\text{max}}\right)$.\newline

  Suppose toward contradiction that $J_{\text{max}}\subset I$, meaning there is an $i_0\in I\setminus J_{\text{max}}$. Since $\left(J_{\text{max}},f_{\text{max}}\right)$ is a partial accumulation point of $\set{f_{\alpha}}_{\alpha}$, there is a subnet $\set{f_{\alpha_\beta}}_{\beta}$ such that $\pi_{j}\left(f_{\alpha_\beta}\right) \rightarrow \pi_{j}\left(f_{\text{max}}\right)$ for each $j\in J_{\text{max}}$.\newline

  Since $K_{i_0}$ is compact, we find a subnet $\set{f_{\alpha_{\beta_{\gamma}}}}_{\gamma}$ such that $\pi_{i_0}\left(f_{\alpha_{\beta_{\gamma}}}\right)_{\gamma}$ converges to $x_{i_0}$ in $K_{i_0}$.\newline

  Define $\tilde{f} \in K$ by setting $\tilde{f}\vert_{J_{\text{max}}} = f_{\text{max}}$, and $\tilde{f}(i_0) = x_{i_0}$. Thus, $\left(J_{\text{max}}\cup \set{i_0},\tilde{f}\right)$ is a partial accumulation point, which contradicts the maximality of $\left(J_{\text{max}},f_{\text{max}}\right)$.
\end{proof}
\subsection{Complex Measures and the Radon--Nikodym Theorem}%
I am going to be drawing much of this information from Gerald B. Folland's text on Real Analysis.
\begin{definition}[Signed Measure]
  For $(X,\Omega)$ a measurable space, a signed measure is a function $\nu: \Omega \rightarrow [-\infty,\infty]$ such that
  \begin{itemize}
    \item $\nu\left(\emptyset\right) = 0$
    \item $\nu$ assumes at most one of $\pm\infty$
    \item For $\set{E_j}$ a sequence of disjoint sets in $\Omega$,
      \begin{align*}
        \nu\left(\bigsqcup_{j=1}^{\infty}E_j\right) &= \sum_{j=1}^{\infty}\nu\left(E_j\right),
      \end{align*}
      with the latter sum converging if $\nu\left(\bigsqcup_{j=1}^{\infty}E_j\right)$ is finite.
  \end{itemize}
\end{definition}
Traditional measures will be referred to as positive measures.\newline

If $\mu_1$ and $\mu_2$ are positive measures on $\Omega$ with at least one a finite measure, then $\nu = \mu_1 - \mu_2$ is a signed measure.\newline

For $\mu$ a measure on $\Omega$, if $f: X\rightarrow [-\infty,\infty]$ such that at least one of $\int f^{+}\:d\mu$ or $\int f^{-}\:d\mu$ is finite, we call $f$ an extended $\mu$-integrable function, with $\nu(E)\int_{E}f\:d\mu$ a signed measure.\newline

In fact, we shall soon see that every signed measure is represented in these forms.
\begin{theorem}[Hahn Decomposition]
  If $\nu$ is a signed measure on $\left(X,\Omega\right)$, then there exist a positive set $P$ and a negative set $N$ for $\nu$ such that $P\cup N = X$, and $P\cap N = 0$. If $P'$ and $N'$ are another set, then $P\triangle P'$ and $N\triangle N'$ are $\nu$-null.
\end{theorem}
\begin{proof}
  We assume that $\nu$ does not assume the value of negative infinity. Let $m$ be the supremum of $\nu(E)$ as $E$ ranges over all positive sets; let $\set{P_j}$ be the sequence of positive sets such that $\nu(P_j) \rightarrow m$.\newline

  We set $P = \bigcup_{j=1}^{\infty}P_j$; by continuity and the property that the union of a countable family of positive sets is positive, we see that $P$ is positive and $\nu(P) = m < \infty$. We claim that $N = X\setminus P$ is negative.\newline

  Suppose toward contradiction that it is not the case. First, we can see that $N$ does not contain any nonnull positive sets, as for $E\subseteq N$ positive, then $E\cup P$ is positive and $\nu(E\cup P) = \nu(E) + \nu(P) > m$. Alternatively, we can see that for any $A\subseteq N$ with $\nu(A) > 0$, we find $C\subseteq A$ with $\nu(C) < 0$ (as $A$ cannot be positive), so $B = A\setminus C$ has measure $\nu(A) - \nu(C) > \nu(A)$.\newline

  If $N$ is nonnegative, we can find subsets $\set{A_j}$ in $N$ and define $n_j$ as follows. We select $n_1$ to be the smallest integer for which there exists a set $B\subseteq N$ with $\nu(B) > \frac{1}{n_1}$; $A_1$ is the given set. Inductively, select $n_j$ the smallest integer where $B\subseteq A_{j-1}$ has measure $\nu(B) > \nu(A_{j-1}) + \frac{1}{n_j}$, with $A_j$ as the set.\newline

  Let $A = \bigcap_{j=1}^{\infty}A_j$. Then,
  \begin{align*}
    \sum_{j=1}^{\infty}\frac{1}{n_j} &< \lim_{j\rightarrow\infty}\nu\left(A_j\right)\\
                                     &<\infty,
  \end{align*}
  meaning that $n_j \rightarrow\infty$ as $j\rightarrow\infty$. However, we still have $B\subseteq A$ with $\nu(B) > \nu(A) + \frac{1}{n}$ for some $n$; for $j$ sufficiently large, we have $n < n_j$ with $B\subseteq A_{j-1}$, which contradicts the construction of $n_j$.\newline

  If $P'$ and $N'$ are another pair of sets, then $P\setminus P' \subseteq P$ and $P\setminus P' \subseteq N'$, meaning $P\setminus P'$ is measure zero.
\end{proof}
The decomposition $X = P\sqcup N$ is known as the Hahn decomposition for $\nu$ (non-unique, generally speaking).\newline

We say two measures, $\mu$ and $\nu$ on $(X,\Omega)$ are mutually singular if there exist disjoint $E,F\in \Omega$ with $E\sqcup F = X$, where $\mu(E) = 0$ and $\nu(F) = 0$. Informally speaking, $\mu$ and $\nu$ exist on disjoint sets; we denote mutual singularity as $\mu \perp \nu$.
\begin{theorem}[Jordan Decomposition]
  If $\nu$ is a signed measure, then there exist unique positive measure $\nu^{+}$ and $\nu^{-}$ such that $\nu = \nu^{+}-\nu^{-}$, and $\nu^{+}\perp \nu^{-}$.
\end{theorem}
\begin{proof}
  Let $X = P\sqcup N$ be a Hahn decomposition for $\nu$, and define $\nu^{+}\left(E\right) = \nu\left(E\cap P\right)$, $\nu^{-}\left(E\right)= -\nu\left(E\cap N\right)$. Then, we can obviously see that $\nu = \nu^{+} - \nu^{-}$, with $\nu^{+}\perp \nu^{-}$.\newline

  Suppose $\nu = \mu^{+} - \mu^{-}$ with $\nu^{+} \perp \nu^{-}$. Let $E,F\in \Omega$ be such that $E\cap F = \emptyset$, $E\cup F = X$, and $\mu^{+}\left(F\right) = \mu^{-}\left(E\right) = 0$. Then, $X = E \sqcup F$ is another Hahn decomposition for $\nu$, meaning $P\triangle E$ is $\nu$-null, meaning that for any $A\in \Omega$, $\mu^{+}\left(A\right) = \nu^{+}\left(A\cap E\right) = \nu\left(A\cap E\right) = \nu\left(A\cap P\right) = \nu^{+}\left(A\right)$, and similarly, $\nu^{-} = \mu^{-}$.
\end{proof}
\begin{definition}[Variation of Signed Measure]
  We define $\nu^{+}$ to be the positive variation of $\nu$, $\nu^{-}$ to be the negative variation of $\nu$, with the total variation of $\nu$ being defined by
  \begin{align*}
    |\nu| = \nu^{+} + \nu^{-}.
  \end{align*}
  If $\nu$ does not admit the value $\infty$, then $\nu^{+}(X) = \nu(P) < \infty$, meaning $\nu^{+}$ is a signed measure, and $\nu$ is bounded above by $\nu^{+}$.\newline

We say $\nu$ is ($\sigma$-)finite if $|\nu|$ is ($\sigma$-)finite.
\end{definition}
\begin{definition}[Integration with respect to a Signed Measure]
  To integrate with respect to the signed measure $\nu$, we take $L^{1}\left(\nu\right) = L^{1}\left(\nu^{+}\right) - L^{1}\left(\nu^{-}\right)$, and
  \begin{align*}
    \int_{X}^{} f\:d\nu &= \int_{X}^{} f\:d\nu^{+} - \int_{X}^{} f\:d\nu^{-}.
  \end{align*}
\end{definition}
\begin{definition}[Absolute Continuity]
Let $\nu$ be a signed measure, and $\mu$ a positive measure on $(X,\Omega)$. We say $\nu$ is absolutely continuous with respect to $\mu$ if $\nu(E) = 0$ for every $E\in \Omega$ where $\mu(E) = 0$. We write $\nu \ll \mu$.\newline

We can verify that $\nu \ll \mu$ if and only if $|\nu| \ll \mu$, which is true if and only if $\nu^{+}\ll \mu$ and $\nu^{-}\ll \mu$.
\end{definition}
\begin{theorem}[Definition of Absolute Continuity]
  Let $\nu$ be a signed measure and $\mu$ a positive measure on $(X,\Omega)$. Then, $\nu \ll \mu$ if and only if, for every $\ve > 0$, there exists $\delta > 0$ such that $\left\vert \nu(E) \right\vert < \ve$ when $\mu\left(E\right)< \delta$.
\end{theorem}
\begin{proof}
  Since $\nu \ll \mu$ if and only if $\left\vert \nu \right\vert\ll\mu$, and $\left\vert \nu(E) \right\vert\leq \left\vert \nu \right\vert\left(E\right)$, we only need assume $\nu$ is positive.\newline

  We can see easily that the $\ve$-$\delta$ condition implies $\nu \ll \mu$.\newline

  In the opposite direction, suppose toward contradiction that there exists $\ve_0 > 0$ such that for all $n\in \N$, we can find $E_n\in \Omega$ with $\mu\left(E_n\right) < 2^{-n}$ with $\nu\left(E_n\right) \geq \ve_0$.\newline

  Let $F_k = \bigcup_{n=k}^{\infty}E_n$, and $F = \bigcap_{k=1}^{\infty}F_k$. Then, $\mu\left(F_k\right) < 2^{1-k}$, meaning $\mu(F) = 0$, but $\nu\left(F_k\right) \geq \ve_0$ for all $k$, meaning $\nu$ is finite and $\nu\left(F\right) = \lim_{k\rightarrow\infty}\nu\left(F_k\right) \geq \ve$, meaning $\nu \nll \mu$.
\end{proof}
\begin{corollary}
  Let $\nu(E)$ be defined by $\nu(E) = \int_{E}^{} F\:d\mu$. Then, if $f\in L^{1}\left(\mu\right)$, $\nu \ll \mu$.
\end{corollary}
\begin{lemma}[Prelude to Radon--Nikodym]
  Suppose that $\nu$ and $\mu$ are finite measures on $(X,\Omega)$. Either $\nu \perp \mu$, or there exists $\ve_0 > 0$ and $E\in \Omega$ such that $\mu(E) > 0$ and $\nu \geq \ve_0 \mu$ on $E$.
\end{lemma}
\begin{proof}
  Let $X = P_n \cup N_n$ be a Hahn decomposition for $\nu - \frac{1}{n}\mu$. Let $P = \bigcup_{n=1}^{\infty}P_n$, $N = \bigcap_{n=1}^{\infty}N_n = P^{c}$.\newline

  Then, $N$ is a negative set for $\nu - \frac{1}{n}\mu$ for all $n$, meaning $0 \leq \nu(N) \leq \frac{1}{n}\mu(N)$ for all $n$, so $\nu\left(N\right) = 0$. \newline

  If $\mu(P) = 0$, then $\nu \perp \mu$. If $\mu(P) > 0$, then $\mu\left(P_n\right)$ > 0 for some $n$, and $P_n$ is a positive set for $\nu - \frac{1}{n}\mu$.
\end{proof}
\begin{theorem}[Radon--Nikodym]
  Let $\nu$ be a $\sigma$-finite signed measure, $\mu$ a $\sigma$-finite positive measure on $(X,\Omega)$. Then, there exist unique $\sigma$-finite signed measures $\lambda,\rho$ on $(X,\Omega)$ such that $\lambda \perp \mu$, $\rho \ll \mu$, and $\nu = \lambda + \rho$.\newline

  Moreover, there exists an extended $\mu$-integrable function $f: X\rightarrow \R$ such that $\rho(E) = \int_{E}^{} f\:d\mu$. The derived measure $\rho$ will be referred to by the shorthand, $d\rho = f\:d\mu$.
\end{theorem}
\begin{proof}\hfill
  \begin{description}
    \item[Case 1:] Suppose $\nu$ and $\mu$ are finite positive measures. Let
      \begin{align*}
        \mathcal{F} &= \set{f: X\rightarrow[0,\infty]\mid \int_{E}^{} f\:d\mu \leq \nu(E)~\forall E\in \Omega}.
      \end{align*}
      Then, $\mathcal{F}$ is nonempty, since $0\in \mathcal{F}$. If $f,g\in \mathcal{F}$, then $h = \max\left(f,g\right) \in \mathcal{F}$, since, for $A = \set{x\mid f(x) > g(x)}$,
      \begin{align*}
        \int_{E}^{} h\:d\mu &= \int_{E\cap A}^{} f\:d\mu + \int_{E\setminus A}^{} g\:d\mu\\
                            &\leq \nu\left(E\cap A\right) + \nu\left(E\setminus A\right)\\
                            &= \nu\left(E\right).
      \end{align*}
      Let $a = \sup\set{\int_{X}^{} f\:d\mu\mid f\in \mathcal{F}}$. Noting that $a\leq \nu(X) < \infty$, choose a sequence $\set{f_n}\in \mathcal{F}$ such that $\int_{X}f_n\:d\mu \rightarrow a$.\newline

      Let $g_n = \max\left(f_1,\dots,f_n\right)$, and $f = \sup_{n}f_n$. Then, $g_n \in \mathcal{F}$, increasing pointwise to $f$, and $\int_{X}^{} g_n\:d\mu \geq \int_{X}^{} f_n\:d\mu$. Thus, $\lim_{n\rightarrow\infty}\int_{X}^{} g_n\:d\mu = a$, meaning that by monotone convergence, $f\in \mathcal{F}$ with $\int_{X}^{} f\:d\mu = a$.\newline

      We claim that the measure $d\lambda = d\mu - f\:d\mu$ is singular with respect to $\mu$. If it isn't, then there exists $E\in \Omega$ and $\ve_0 > 0$ such that $\mu\left(E\right) > 0$ and $\lambda \geq \ve_0 \mu$ on $E$. However, $\ve_0 \chi_{E}d\mu \leq d\lambda = d\nu - f\:d\mu$, meaning $\left(f + \ve_0\chi_{E}\right)d\mu \leq d\nu$, meaning $f + \ve_0\chi_E\in \mathcal{F}$, and
      \begin{align*}
        \int_{X}^{} \left(f + \ve_0 \chi_E\right)\:d\mu &= a + \ve_0\mu(E)\\
                                                        &> a,
      \end{align*}
      contradicting the definition of $a$.\newline

      Thus, existence of $\lambda,f$, and $d\rho = f\:d\mu$ is proved. To show uniqueness, if it is also the case that $d\nu = d\lambda' + f'd\mu$, we have $d\lambda - d\lambda' = \left(f' - f\right)d\mu$. However, $\lambda - \lambda' \perp \mu$, while $\left(f' - f\right)d\mu \ll d\mu$, meaning $d\lambda - d\lambda' = \left(f'-f\right)d\mu = 0$, meaning $\lambda = \lambda '$ and $f = f'$ $\mu$-almost everywhere.
    \item[Case 2:] If $\mu$ and $\nu$ are $\sigma$-finite measures, then $X$ is a countable disjoint union of $\mu$-finite sets and a countable disjoint union of $\nu$-finite sets.\newline

      Taking intersections, we obtain a sequence of disjoint sets $\set{A_i}\subseteq \Omega$ such that $\mu\left(A_j\right)$ and $\nu\left(A_j\right)$ are finite for all $j$, and $X = \bigcup_{j=1}^{\infty}A_j$.\newline

      Define $\mu_j \left(E\right) = \mu\left(E\cap A_j\right)$, and $\nu_j\left(E\right) = \nu\left(E\cap A_j\right)$. For each $j$, we have $d\nu_j = d\lambda_j + f_j\:d\mu_j$, where $\lambda_j \perp \mu_j$ (applying Case 1 to each finite measure).\newline

      Since $\mu_j\left(A^{c}_j\right) = \nu_j\left(A^{c}_j\right) = 0$, we have $\lambda_j\left(A_j^{c}\right) = \nu_j\left(A_j^{c}\right) - \int_{A_{j}^c}^{} f\:d\mu = 0$, meaning $f_j = 0 $ on $A_{j}^c$.\footnote{$\mu$-almost everywhere, of course.}\newline

      Let $\lambda = \sum_{j=1}^{\infty}\lambda_j$ and $f = \sum_{j=1}^{\infty}f_j$. Then, $d\nu = d\lambda + f\:d\mu$, $\lambda \perp \mu$, and $d\lambda$, $f\:d\mu$ are $\sigma$-finite.\newline

      The uniqueness follows from the uniqueness of each $\lambda_j$ and $f\:d\mu_j$.
  \end{description}
  In the general case, we apply each argument to $\nu^{+}$ and $\nu^{-}$, then subtract.
\end{proof}
\begin{definition}[Radon--Nikodym Derivative]
  The composition $\nu = \lambda + \rho$, where $\lambda \perp \mu$ and $\rho \ll \mu$ is known as the Lebesgue decomposition of $\nu$ with respect to $\mu$.\newline

  When $\nu \ll \mu$, then $d\nu = f\:d\mu$ for some $f$. We call $f$ the Radon--Nikodym derivative of $\nu$ with respect to $\mu$. We write
  \begin{align*}
    d\nu &= \frac{d\nu}{d\mu} d\mu.
  \end{align*}
\end{definition}
\begin{proposition}[Chain Rule]
  Let $\nu$ be a $\sigma$-finite signed measure, where $\mu$, $\lambda$ are $\sigma$-finite measures on $(X,\Omega)$ with $\nu \ll \mu$, $\mu \ll \lambda$.
  \begin{enumerate}[(a)]
    \item If $g \in L^{1}\left(\nu\right)$, then $g\frac{d\nu}{d\mu}\in L^1\left(\mu\right)$ and
      \begin{align*}
        \int_{}^{} g\:d\nu &= \int_{}^{} g\frac{d\nu}{d\mu}\:d\mu.
      \end{align*}
    \item We have $\nu \ll \lambda$, and
      \begin{align*}
        \frac{d\nu}{d\lambda} &= \frac{d\nu}{d\mu}\frac{d\mu}{d\lambda}
      \end{align*}
      $\lambda$-almost everywhere.
  \end{enumerate}
\end{proposition}
\begin{proof}
  We may assume $\nu \geq 0$. By the definition of $\frac{d\nu}{d\mu}$, it is the case that
  \begin{align*}
    \int_{}^{} g\:d\nu &= \int_{}^{} g\frac{d\nu}{d\mu}\:d\mu
  \end{align*}
  whenever $g = \chi_E$. Thus, by linearity, the equation is true for simple functions, and then for nonnegative measurable functions by monotone convergence, then for $L^{1}(\nu)$ functions by linearity.\newline

  Replacing $\nu$ and $\mu$ with $\mu$ and $\lambda$, setting $g = \chi_E \frac{d\nu}{d\mu}$, we have
  \begin{align*}
    \nu(E) &= \int_{E}^{} \frac{d\nu}{d\mu}\:d\mu\\
           &= \int_{E}^{} \frac{d\nu}{d\mu}\frac{d\mu}{d\lambda}\:d\lambda
  \end{align*}
  for all $E\in \Omega$, meaning
  \begin{align*}
    \frac{d\nu}{d\lambda} &= \frac{d\nu}{d\mu}\frac{d\mu}{d\lambda}
  \end{align*}
  $\lambda$-almost everywhere.
\end{proof}
\begin{corollary}
  If $\mu \ll \lambda$ and $\lambda \ll \mu$, then $\frac{d\lambda}{d\mu}\frac{d\mu}{d\lambda} = 1$.
\end{corollary}
\begin{example}[Dirac $\delta$ Distribution]
  Let $\mu$ be the Lebesgue measure on $\R$, and $\nu$ the point mass at $0$ on $\left(\R,\mathcal{B}_{\R}\right)$. We can see that $\nu \perp \mu$.\newline

  The ``Radon--Nikodym derivative'' $\frac{d\nu}{d\mu}$ is the Dirac $\delta$ distribution.
\end{example}
\subsection{Essentials of Abstract Harmonic Analysis}%
In order to go further into theories of Banach algebras, we need a better understanding of the theory of topological groups and other essentials of abstract harmonic analysis. As a result, I'm going to be pulling information from Hewitt and Ross's \textit{Abstract Harmonic Analysis, Volume I}.
\subsubsection{Basic Definitions}%
\begin{definition}[Topological Group]
Let $G$ be a set that is a group with a topological structure. If
\begin{enumerate}[(i)]
  \item $\cdot: G\times G \rightarrow G$, $(x,y) \mapsto xy$;
  \item and $^{-1}: G\rightarrow G$, $x\mapsto x^{-1}$
\end{enumerate}
are continuous, then $G$ is a topological group.
\end{definition}
The continuity of group multiplication implies that for every neighborhood $U$ of $xy$, there exists a neighborhood $V$ of $x$ and $W$ of $y$ such that $VW \subseteq U$.
\begin{theorem}[Homeomorphisms]
  Let $G$ be a topological group. For $a\in G$, the maps $\set{a}\times G \rightarrow aG$, $G\times \set{a}\rightarrow Ga$, and inversion are homeomorphisms.
\end{theorem}
\begin{theorem}[Translation of Bases]
  Let $G$ be a topological group, and let $\mathcal{U}$ be a basis for $G$ at $e$. Then, the families $\mathcal{U}x$ and $x\mathcal{U}$ for every $x\in G$ are bases for $G$.
\end{theorem}
\begin{proof}
  Let $W$ be a nonempty open subset of $G$ with $a\in W$. The map $x \mapsto a^{-1}x$ takes $W$ to the set $a^{-1}W$; notice that $e\in a^{-1}W$.\newline

  Since $\mathcal{U}$ is a basis at $e$, there is a neighborhood $U\in \mathcal{U}$ such that $U\subseteq a^{-1}W$, meaning $aU\subseteq W$.\newline

  Thus, $W$ is a union of the sets $aU$, meaning $\set{xU\mid x\in G,U\in \mathcal{U}}$ is an basis for $G$.
\end{proof}
\begin{theorem}[Product Sets]
  Let $G$ be a topological group, with $A,B\subseteq G$.
  \begin{itemize}
    \item If $A$ is open, then $AB$ and $BA$ are open.
    \item If $A$ and $B$ are compact, then $AB$ is compact.
    \item If $A$ is closed and $B$ is compact, then $AB$ and $BA$ are closed.
    \item If $A$ and $B$ are closed, then $AB$ need not be closed.
  \end{itemize}
\end{theorem}
\begin{proof}
  To prove the first item, we have $AB = \bigcup_{b\in B}Ab$; since open sets are translation-invariant, this means $AB$ is a union of open sets, and thus open.\newline

  Suppose $A$ and $B$ are compact; then, by Tychonoff's theorem, $A\times B$ is compact in $G\times G$. Since group multiplication is continuous, $AB$ is compact.\newline

  Suppose $A$ is closed and $B$ is compact. Let $\set{x_{\alpha}}_{\alpha \in D}$ be a net in $AB$ converging to $x_0$ in $G$. We only need show that $x_0\in AB$. For each $x_{\alpha}$, we have $x_{\alpha} = a_{\alpha}b_{\alpha}$, where $a_{\alpha}\in A$ and $b_{\alpha}\in B$.\newline

  Since $B$ is compact, there is a subnet such that $\lim_{\beta \in E}b_{\beta}\rightarrow b_0$. It must be the case that $x_{\beta}\rightarrow x_0$, and we can see that $\left(x_{\beta},b_{\beta}\right)\rightarrow \left(x_0,b_0\right)$. Therefore, $a_{\beta} = x_{\beta}b_{\beta}$ converges to $x_0b_0^{-1}$, as it is the composition of $\left(x_{\beta},y_{\beta}\right)$ with $(x,y)\mapsto xy^{-1}$. Since $A$ is closed, and each $a_{\beta}\in A$, $a_{\beta}\rightarrow a_0\in A$, meaning
  \begin{align*}
    x_0 &= \left(x_0b_0^{-1}\right)b_0\\
        &= \left(a_0\right)b_0\\
        &\in AB.
  \end{align*}
  Similarly, we can see that $BA$ is closed.
\end{proof}
\begin{theorem}[Characterization of Topological Groups]
  Let $G$ be a topological group with $\mathcal{U}$ a basis centered at $e$. Then,
  \begin{enumerate}[(i)]
    \item for every $U\in \mathcal{U}$, there is a $V\in \mathcal{U}$ such that $V^{2} \subseteq U$;
    \item for every $U\in \mathcal{U}$, there is a $V\in \mathcal{U}$ such that $V^{-1}\subseteq U$;
    \item For every $U\in \mathcal{U}$ and every $x\in U$, there is a $V\in \mathcal{U}$ such that $xV \subseteq U$;
    \item for every $U\in \mathcal{U}$ and every $x\in G$, there is a $V\in \mathcal{U}$ such that $xVx^{-1}\subseteq U$.
  \end{enumerate}
  Conversely, if $G$ is a group and $\mathcal{U}$ is a family of subsets of $G$ with the finite intersection property and (i)--(iv), then the family of subsets $\set{xU\mid U\in \mathcal{U},x\in G}$ is a subbasis\footnote{The topology on $G$ is the smallest topology containing $\set{xU\mid U\in \mathcal{U},x\in G}$.} for a topology on $G$.\newline

  If $\mathcal{U}$ also satisfies $U,V\in \mathcal{U}\Rightarrow \exists W\in \mathcal{U}$ such that $W\subseteq U\cap V$, then $\set{xU\mid U\in \mathcal{U},x\in G}$ and $\set{Ux\mid U\in \mathcal{U},x\in G}$ are open bases for the topology on $G$.
\end{theorem}
\begin{proof}
  In the forward direction, we see that (i) implies that $(x,y)\mapsto xy$ is continuous, (ii) implies that $x\mapsto x^{-1}$ is continuous, and (iii) implies that $U$ is open. Finally, (iv) follows from the fact that $x\mapsto ax \mapsto axa^{-1}$ is a homeomorphism.\newline

  In the reverse direction, let $\mathcal{U}$ satisfy conditions (i)--(iv) and have the finite intersection property. Then, for $U\in \mathcal{U}$, there are $V,W\in\mathcal{U}$ such that $V^2 \subseteq U$ and $W^{-1}\subseteq V$. Since $V\cap W\neq \emptyset$, we have $e\in VW^{-1}\subseteq V^2 \subseteq U$. Thus, all elements of $\mathcal{U}$ must contain $e$.\newline

  Let $\tilde{\mathcal{U}}$ be the family of all sets $\bigcap_{k=1}^{n}U_k$ for $U_1,\dots,U_n \in \mathcal{U}$. For each $U_{k}$, there exist $V_k$ such that $V_k^2 \subseteq U_k$. Thus,
  \begin{align*}
    \left(\bigcap_{k=1}^{n}V_k\right)^2 &\subseteq \bigcap_{k=1}^{n}\left(V_k\right)^2\\
                                        &\subseteq \bigcap_{k=1}^{n}U_k.
  \end{align*}
  Thus, condition (i) holds for $\tilde{\mathcal{U}}$. Additionally, since $\left(\bigcap_{k=1}^{n}V_k\right)^{-1} = \bigcap_{k=1}^{n}V_k^{-1}$, (ii) holds for $\tilde{\mathcal{U}}$. Finally, since $x\left(\bigcap_{k=1}^{n}V_k\right) = \bigcap_{k=1}^{n}xV_k$ and $x\left(\bigcap_{k=1}^{n}V_k\right)x^{-1} = \bigcap_{k=1}^{n}\left(xV_kx^{-1}\right)$, properties (iii) and (iv) hold for $\tilde{\mathcal{U}}$.\newline

  Thus, the nonempty sets $\bigcap_{k=1}^{n}\left(x_kU_k\right)$ with $x_k\in G$ and $U_k\in \mathcal{U}$ form an open basis for the topology of $G$. Let $y\in \bigcap_{k=1}^{n}\left(x_kU_k\right)$, and let $V_k$ be such that $x_kyV_k \subseteq U_k$ for each $k$. Then,
  \begin{align*}
    y\left(\bigcap_{k=1}^{n}V_k\right) &= \bigcap_{k=1}^{n}\left(yV_k\right)\\
                                       &\subseteq \bigcap_{k=1}^{n}\left(x_kU_k\right),
  \end{align*}
  meaning $y\tilde{U}$ as $\tilde{U}$ runs through $\tilde{\mathcal{U}}$ forms an open basis at $y$ for each $y\in G$.\newline

  To show that $G$ is a topological group, let $a,b\in G$ and $\tilde{U} \in \tilde{\mathcal{U}}$. By (i) and (iv) on $\tilde{\mathcal{U}}$, there exist $\tilde{V},\tilde{W}\in \tilde{\mathcal{U}}$ such that $\left(b^{-1}\tilde{W}b\right)\tilde{V}\subseteq \tilde{U}$, meaning $\left(a\tilde{W}\right)\left(b\tilde{V}\right) \subseteq ab\tilde{U}$, meaning group multiplication is continuous. Similarly, from (ii) and (iv), we can see that $a^{-1}\tilde{V} \subseteq \tilde{U}$.
\end{proof}
\begin{theorem}[Symmetric Neighborhoods]
  Every topological group $G$ has a basis at $e$ consisting of neighborhoods $U$ such that $U = U^{-1}$.
\end{theorem}
\begin{proof}
  For an arbitrary neighborhood $U$ of $e$, we can see that for $V = U\cap U^{-1}$, $V = V^{-1}$ and $V$ is a neighborhood of $V$ with $V\subseteq U$.
\end{proof}
\begin{corollary}[Regularity of Topological Groups at Identity]
  Let $G$ be a topological group. For every (open) neighborhood $U$ of $e$, there is a neighborhood $V$ of $e$ such that $\overline{V} \subseteq U$.
\end{corollary}
\begin{proof}
  Let $V$ be a symmetric neighborhood of $e$ such that $V^{2}\subseteq U$. Then, for $x\in \overline{V}$, we have $xV \cap V \neq \emptyset$, meaning $xv_1 = v_2$ for some $v_1,v_2 \in V$, so $x = v_2v_1^{_1}\in VV^{-1} = V^2 \subseteq U$.
\end{proof}
\begin{theorem}[$T_3$ Property of Topological Groups]
  Let $G$ be a $T_0$ topological group.\footnote{$T_0$ is the Kolmogorov property, implying that for two points $x\neq y$, there exists an open set $O$ such that $x\in O\wedge y\notin O$ or vice versa.} Then, $G$ is regular at every point, thus it is Hausdorff.
\end{theorem}
\begin{proof}
  From the corollary, we know that $G$ is regular at $e$, and since left translation is a homeomorphism, this means $G$ is regular at every element. Thus, $G$ is Hausdorff.
\end{proof}
\begin{theorem}[Conjugate Neighborhoods in Compact Subsets]
  Let $G$ be a topological group, with $U$ a neighborhood of $e$, $F \subseteq G$ compact. Then, there is a neighborhood $e$ of $V$ such that $xVx^{-1} \subseteq U$ for all $x\in F$.
\end{theorem}
\begin{proof}
  Let $\mathcal{U}$ denote the family of symmetric neighborhoods of $e$. We will first show that for $y\in G$, there is a $V\in \mathcal{U}$ such that $x\subseteq Vy$ implies $xVx^{-1}\subseteq U$.\newline

  Let $V_1\in \mathcal{U}$ be such that $V_1^{3}\subseteq U$, and $V_2\in \mathcal{U}$ such that $yV_2 y^{-1}\subseteq V_1$. Let $V = V_1 \cap V_2$. Then, for $x\in Vy$, we have $xy^{-1}\in V\subseteq V_1$, and $yx^{-1}\in V_1^{-1} = V_1$, meaning $xVx^{-1}\subseteq xV_2x^{-1} = \left(xy^{-1}\right) y V_2 y^{-1} \left(yx^{-1}\right) \subseteq V_1^{3}\subseteq U$.\newline

  For each $y\in F$, there is a $V_y\in \mathcal{U}$ such that $xV_y y \Rightarrow xV_y x^{-1}\subseteq U$. Since $F\subseteq \bigcup_{y\in F}V_y y$, and $F$ is compact, there exist $y_1,\dots,y_n$ such that $F\subseteq \bigcup_{k=1}^{n}V_{y_k}y_k$.\newline

  Let $V = \bigcap_{k=1}^{n}V_{y_k}$. Then, for $x\in F$, $x\in V_{y_k}y_k$ for some $k$, meaning $xVx^{-1}\subseteq xV_{y_k}x^{-1}\subseteq U$.
\end{proof}
\begin{theorem}[Neighborhoods with Compact Closure]
  Let $G$ be a topological group and $U$ an open subset of $G$ such that for compact $F$, $F\subseteq U$. Then, there is a neighborhood $V$ of $e$ such that $\left(FV\right) \cup \left(VF\right) \subseteq U$. If $G$ is locally compact, then $V$ can be chosen such that $\overline{\left(FV\right) \cup \left(VF\right)}$.
\end{theorem}
\begin{proof}
  For each $x\in F$, there is a neighborhood $W_x$ of $e$ such that $xW_x\subseteq U$, and a neighborhood $V_x$ of $e$ such that $V_x^2\subseteq W_x$.\newline

  Since $F \subseteq \bigcup_{x\in F}xV_x$, there exist $x_1,\dots,x_n\in F$ such that $F\subseteq \bigcup_{k=1}^{n}x_kV_{x_k}$. Let $V_1 = \bigcap_{k=1}^{n}V_{x_k}$. Then,
  \begin{align*}
    FV_1 &\subseteq \bigcup_{k=1}^{n}x_kV_{x_k}V_1\\
         &\subseteq \bigcup_{k=1}^{n} x_kV_{x_k}^2\\
         &\subseteq \bigcup_{k=1}^{n}x_kW_{x_k}\\
         &\subseteq U.
  \end{align*}
  Similarly, there is a neighborhood $V_2$ of $e$ such that $V_2F\subseteq U$. Letting $V = V_1\cap V_2$, we get that $\left(FV\right)\cup \left(VF\right)\subseteq U$.\newline

  If $G$ is locally compact, then $V$ can be chosen such that $\overline{V}$ is compact. It follows that $F\overline{V}$ is compact; since $FV \subseteq F\overline{V}$, and $F\overline{V}$ is closed, $\overline{FV}\subseteq F\overline{V}$, meaning $F\overline{V}$ is compact. Similarly, $\overline{VF}$ is compact, meaning $\overline{\left(FV\right) \cup \left(VF\right)}$ is compact.
\end{proof}
As a result of the fact that translation is a homeomorphism, we can introduce a notion of ``uniform nearness'' of points, as well as uniform continuity for real- and complex-valued functions on $G$.\newline

Considering uniform nearness, left translating $x$ and $y$ in $G$ by $x^{-1}$, we find that $x\mapsto e$ and $y\mapsto x^{-1} y$. If $x^{-1}y$ is in a symmetric neighborhood $U$ of $e$, we can say that $x$ and $y$ are $U$-near in the sense of left translation. Similarly, if $yx^{-1}\in U$, we can say that $x$ and $y$ are $U$-near in the sense of right translation.\newline

Both of these are uniform concepts, in that they can be applied to any $x$ and $y$ in $G$. If $\varphi$ is a complex-valued function on $G$, we can say that $\varphi$ is left (right) uniformly continuous if for every $\ve > 0$, there exists a neighborhood $U$ of $e$ such that $\left\vert \varphi(x)-\varphi(y) \right\vert < \ve$ whenever $x$ and $y$ are $U$-near in the sense of left (right) translation.\newline

Thus, for left uniform continuity, we must have
\begin{align*}
  \left\vert \varphi(x) - \varphi(xu) \right\vert < \ve
\end{align*}
 for all $x\in G$ and all $u\in U$, and for right uniform continuity, we must have
 \begin{align*}
   \left\vert \varphi(x) - \varphi(ux)  \right\vert < \ve
 \end{align*}
 for all $x\in G$ and all $u\in U$.\newline

 The notions of left and right uniform continuity are natural extensions of uniform continuity for a complex-valued function defined on $\R$; however, instead of a single $\delta > 0$ that works for all $x,y\in \R$, we have a single neighborhood $U$ that works for every $x\in G$.
 \begin{definition}[Uniform Structure]
   Let $G$ be a topological group. For every neighborhood $U$ of $e$ in $G$, let $L_{U}$ be the set of points $(x,y)\in G\times G$ such that $x^{-1}y\in U$, and let $R_U$ be the set of points $(x,y)\in G\times G$ such that $yx^{-1}\in U$. The family of sets $L_U$ (or $R_U$) as $U$ runs through all neighborhoods of $e$ is written as $\mathcal{I}_{l}\left(G\right)$ (or $\mathcal{I}_{r}\left(G\right)$), and is called hte left (or right) uniform structure on $G$.
 \end{definition}
 \begin{definition}[Uniformly Continuous Mapping]
   Let $G$ and $H$ be topological groups, with $\varphi: G\rightarrow H$ a map. Let $\mathcal{U}$ and $\mathcal{V}$ denote the bases at the identities of $G$ and $H$ respectively.\newline

   Suppose that for every $V\in \mathcal{V}$, there is a $U\in \mathcal{U}$ such that $\left(\varphi(x),\varphi(y)\right)\in L_{V}$ for all $(x,y)\in L_U$. Then, $\varphi$ is said to be a uniformly continuous mapping for the pair of uniform structures $\left(\mathcal{I}_{l}\left(G\right),\mathcal{I}_{l}\left(H\right)\right)$.\newline

   Uniform continuity for the pairs of uniform structures $\left(\mathcal{I}_l\left(G\right),\mathcal{I}_r\left(H\right)\right)$, $\left(\mathcal{I}_r\left(G\right),\mathcal{I}_l\left(H\right)\right)$, and $\left(\mathcal{I}_r\left(G\right),\mathcal{I}_r\left(H\right)\right)$ are defined similarly.
 \end{definition}
\section{Banach Spaces}%
Let $X$ be a compact Hausdorff space, and let $C(X)$ denote the set of continuous functions $f: X\rightarrow \C$. For $f_1,f_2\in C(X)$ and $\lambda\in \C$, we define
\begin{enumerate}[(1)]
  \item $\displaystyle \left(f_1 + f_2\right)(x) = f_1(x) + f_2(x)$
  \item $\displaystyle \left(\lambda f_1\right)(x) = \lambda f_1(x)$
  \item $\displaystyle \left(f_1f_2\right)(x) = f_1(x)f_2(x)$
\end{enumerate}
With these operations, $C(X)$ is a commutative algebra\footnote{A vector space with multiplication.} with identity over the field $\C$.\newline

For each $f\in C(X)$, $f$ is bounded (since $X$ is compact and $f$ is continuous); thus, $\sup\left\vert f \right\vert < \infty$. We call this the norm of $f$, and denote it
\begin{align*}
  \norm{f}_{\infty} &= \sup\set{|f(x)|\mid x\in X}.
\end{align*}
\begin{proposition}[Properties of the Norm on $C(X)$]\hfill
  \begin{enumerate}[(1)]
    \item Positive Definiteness: $\displaystyle \norm{f}_{\infty} = 0 \Leftrightarrow f = 0$
    \item Absolute Homogeneity: $\displaystyle \norm{\lambda f}_{\infty} = |\lambda|\norm{f}_{\infty}$
    \item Subadditivity (Triangle Inequality): $\displaystyle \norm{f+g}_{\infty}\leq \norm{f}_{\infty} + \norm{g}_{\infty}$
    \item Submultiplicativity: $\displaystyle \norm{fg}_{\infty} \leq \norm{f}_{\infty} \norm{g}_{\infty}$
  \end{enumerate}
\end{proposition}
We define a metric $\rho$ on $C(X)$ by $\rho(f,g) = \norm{f-g}_{\infty}$.
\begin{proposition}[Properties of the Induced Metric on $C(X)$]\hfill
  \begin{enumerate}[(1)]
    \item $\displaystyle \rho(f,g) = 0 \Leftrightarrow f=g$
    \item $\displaystyle \rho(f,g) = \rho(g,f)$
    \item $\displaystyle \rho(f,h) \leq \rho(f,g) + \rho(g,h)$
  \end{enumerate}
\end{proposition}
\begin{proposition}[Completeness of $C(X)$]
  If $X$ is a compact Hausdorff space, then $C(X)$ is a complete metric space.
\end{proposition}
\begin{proof}
  Let $\set{f_n}_{n=1}^{\infty}$ be Cauchy. Then,
  \begin{align*}
    \left\vert f_n(x) - f_m(x) \right\vert &\leq \norm{f_n - f_m}_{\infty}\\
    &= \rho(f_n,f_m)
  \end{align*}
  for each $x\in X$. Thus, $\set{f_n(x)}_{n=1}^{\infty}$ is Cauchy for each $x\in X$. We define $f(x) = \lim_{n\rightarrow \infty}f_n(x)$. We will need to show that this implies $\lim_{n\rightarrow\infty}\norm{f_n-f}_{\infty} = 0$.\newline

  Let $\varepsilon > 0$; choose $N$ such that $n,m \geq N$ implies $\norm{f_n - f_m}_{\infty} < \varepsilon$. For $x_0\in X$, there exists a neighborhood $U$ such that $\left\vert f_N(x_0) - f_N(x) \right\vert < \varepsilon$ for $x\in U$.\footnote{This is by the continuity of $\set{f_n}_n$.} Thus,
  \begin{align*}
    \left\vert f(x_0) - f(x) \right\vert &= \left\vert f_n(x_0) - f_N(x_0) + f_N(x_0) - f_N(x) + f_N(x) - f_n(x) \right\vert\\
                                         &\leq \left\vert f_n(x_0) - f_N(x_0) \right\vert + \left\vert f_N(x_0) - f_N(x) \right\vert + \left\vert f_N(x) - f_n(x) \right\vert\\
                                         &\leq 3\varepsilon.
  \end{align*}
  Thus, $f$ is continuous. Additionally, for $n\geq N$ and $x\in X$, we have
  \begin{align*}
    \left\vert f_n(x) - f(x) \right\vert &= \lim_{m\rightarrow\infty}\left\vert f_n(x) - f_m(x) \right\vert\\
                                         &\leq \lim_{m\rightarrow\infty}\norm{f_n-f_m}_{\infty}\\
                                         &\leq \varepsilon.
  \end{align*}
  Thus, $\lim_{n\rightarrow\infty}\norm{f_n - f}_{\infty} = 0$, meaning $C(X)$ is complete.
\end{proof}
\begin{definition}[Banach Space]
  A Banach space is a vector space over $\C$ with a norm $\norm{\cdot}$ is complete with respect to the induced metric.
\end{definition}
\begin{proposition}[Properties of the Banach Space Operations]
  Let $\mathcal{X}$ be a Banach space. The functions
  \begin{itemize}
    \item $a: \mathcal{X}\times \mathcal{X} \rightarrow \mathcal{X};\:a(f,g) = f+g$,
    \item $s: \C\times \mathcal{X}\rightarrow \mathcal{X};\:s(\lambda,f) = \lambda f$,
    \item $n: \mathcal{X}\rightarrow \R^{+};\:n(f) = \norm{f}$
  \end{itemize}
  are continuous.
\end{proposition}
\begin{definition}[Directed Sets and Nets]
  Let $A$ be a partially ordered set with ordering $\leq$. We say $A$ is directed if for each $\alpha,\beta \in A$, there exists a $\gamma$ such that $\alpha \leq \gamma$ and $\beta \leq \gamma$.\newline

  A net is a map $\alpha \mapsto \lambda_{\alpha}$, where $\alpha\in A$ for some directed set $A$. 
\end{definition}
\begin{definition}[Convergence of Nets]
  Let $\set{\lambda_{\alpha}}$ be a net in $X$. We say the net converges to $\lambda \in X$ if for every neighborhood $U$ of $\lambda$, there exists $\alpha_{U}$ such that for $\alpha \geq \alpha_U$, every $\lambda_{\alpha}$ is contained in $U$.\footnote{The net convergence generalizes sequence convergence in a metric space to the case where $X$ does not have a metric.}
\end{definition}
\begin{definition}[Cauchy Nets in Banach Spaces]
  A net $\set{f_{\alpha}}_{\alpha}$ in a Banach space $\mathcal{X}$ is said to be a Cauchy net if for every $\ve > 0$, there exists $\alpha_0$ in $A$ such that $\alpha_1,\alpha_2 \geq \alpha_0$ implies $\norm{f_{\alpha_1} -f_{\alpha_2}} < \ve$.
\end{definition}
\begin{proposition}[Convergence of Cauchy Nets in Banach Spaces]
In a Banach space, every Cauchy net is convergent.
\end{proposition}
\begin{proof}
  Let $\set{f_{\alpha}}_{\alpha}$ be a Cauchy net in $\mathcal{X}$. Choose $\alpha_1$ such that $\alpha \geq \alpha_1$ implies $\norm{f_{\alpha} - f_{\alpha_1}} < 1$.\newline

  We iterate this process by choosing $\alpha_{n+1}\geq \alpha_n$ such that $\alpha \geq \alpha_{n+1}$ implies  $\norm{f_{\alpha} - f_{\alpha_{n+1}}} < \frac{1}{n+1}$.\newline

  The sequence $\set{f_{\alpha_n}}_{n=1}^{\infty}$ is Cauchy, and since $\mathcal{X}$ is complete, there exists $f\in \mathcal{X}$ such that $\lim_{n\rightarrow\infty} f_{\alpha_n} = f$.\newline

  We must now prove that $\lim_{\alpha\in A}f_{\alpha} = f$. Let $\ve > 0$. Choose $n$ such that $\frac{1}{n} < \frac{\ve}{2}$, and $\norm{f_{\alpha_n} - f_{\alpha}} < \frac{\ve}{2}$. Then, for $\alpha \geq \alpha_n$, we have
  \begin{align*}
    \norm{f_{\alpha} - f} &\leq \norm{f_{\alpha} - f_{\alpha_n}} + \norm{f_{\alpha_n} - f}\\
                          &< \frac{1}{n} + \frac{\ve}{2}\\
                          &< \ve.
  \end{align*}
\end{proof}
\begin{definition}[Convergence of Infinite Series]
  Let $\set{f_{\alpha}}_{\alpha}$ be a set of vectors in $\mathcal{X}$. Let $\mathcal{F} = \set{F\subseteq A\mid F\text{ finite}}$.\newline

  Define the ordering $F_1\leq F_2 \Leftrightarrow F_1 \subseteq F_2$.\footnote{the inclusion ordering} For each $F$, define
  \begin{align*}
    g_F = \sum_{\alpha \in F}f_{\alpha}.
  \end{align*}
  If $\set{g_F}_{F\in \mathcal{F}}$ converges to some $g\in \mathcal{X}$, then
  \begin{align*}
    \sum_{\alpha \in A}f_{\alpha}
  \end{align*}
  converges, and we write
  \begin{align*}
    g = \sum_{\alpha \in A}f_{\alpha}.
  \end{align*}
\end{definition}
\begin{proposition}[Absolute Convergence of Series in Banach Space]
  Let $\set{f_{\alpha}}_{\alpha}$ be a set of vectors in the Banach space $\mathcal{X}$. Suppose $\displaystyle \sum_{\alpha \in A}\norm{f_{\alpha}}$ converges in $\R$. Then, $\sum_{\alpha \in A}f_{\alpha}$ converges in $\mathcal{X}$.
\end{proposition}
\begin{proof}
  All we need show is $\set{g_{F}}_{F\in \mathcal{F}}$ is Cauchy. Since $\displaystyle \sum_{\alpha \in A}\norm{f_{\alpha}}$ converges, there exists $F_0\in \mathcal{F}$ such that $F\geq F_{0}$ implies
  \begin{align*}
    \sum_{\alpha \in F} \norm{f_{\alpha}} - \sum_{\alpha \in F_{0}}\norm{f_{\alpha}} < \ve.
  \end{align*}
  Thus, for $F_1,F_2 \geq F_0$, we have
  \begin{align*}
    \norm{g_{F_1} - g_{F_2}} &= \norm{\sum_{\alpha \in F_1}f_{\alpha} - \sum_{\alpha \in F_2}f_{\alpha}}\\
                             &= \norm{\sum_{\alpha \in F_1\setminus F_2}f_{\alpha} - \sum_{\alpha \in F_2\setminus F_1}}\\
                             &\leq \sum_{\alpha \in F_{1}\setminus F_2} \norm{f_{\alpha}} + \sum_{\alpha \in F_2\setminus F_1}\norm{f_{\alpha}}\\
                             &\leq \sum_{\alpha \in F_1\cup F_2}\norm{f_{\alpha}} - \sum_{\alpha \in F_0}\norm{f_{\alpha}}\\
                             &< \ve.
  \end{align*}
  Thus, $\set{g_{F}}_{F\in \mathcal{F}}$ is Cauchy, and thus the series is convergent.
\end{proof}
\begin{theorem}[Absolute Convergence Criterion for Banach Spaces]
  Let $\mathcal{X}$ be a normed vector space. Then, $\mathcal{X}$ is a Banach space if and only if for every sequence $\set{f_{n}}_{n=1}^{\infty}$ of vectors in $\mathcal{X}$,
  \begin{align*}
    \sum_{n=1}^{\infty}\norm{f_n} < \infty \Rightarrow \sum_{n=1}^{\infty}f_n\text{ convergent.}
  \end{align*}
\end{theorem}
\begin{proof}
  The forward direction follows from the previous proposition.\newline

  Let $\set{g_n}_{n=1}^{\infty}$ be a Cauchy sequence in a normed vector space where
  \begin{align*}
    \sum_{n=1}^{\infty}\norm{f_n} < \infty \Rightarrow \sum_{n=1}^{\infty}f_n\text{ convergent.}
  \end{align*}
  We select a subsequence $\set{g_{n_k}}_{k=1}^{\infty}$ as follows. Choose $n_1$ such that $i,j\geq n_1$ implies $\norm{g_i - g_j} < 1$; recursively, we select $n_{N+1}$ such that $\norm{g_{N+1} - g_N} < 2^{-N}$. Then,
  \begin{align*}
    \sum_{k=1}^{\infty}\norm{g_{k+1} - g_k} < \infty.
  \end{align*}
  Set $f_k = g_{n_k} - g_{n_{k-1}}$ for $k > 1$, with $f_1 = g_{n_1}$. Then,
  \begin{align*}
    \sum_{k=1}^{\infty}\norm{f_{k}} < \infty,
  \end{align*}
  meaning $\displaystyle\sum_{k=1}^{\infty}f_k$ converges. Thus, $\set{g_{n_k}}_{k=1}^{\infty}$ converges, meaning $\set{g_{n}}_{n=1}^{\infty}$ converges in $\mathcal{X}$.
\end{proof}
\begin{definition}[Bounded Linear Functional]
Let $\mathcal{X}$ be a Banach space. A function $\varphi: \mathcal{X}\rightarrow \C$ is known as a bounded linear functional if
\begin{enumerate}[(1)]
  \item $\displaystyle \varphi(\lambda_1f_1 + \lambda_2f_2) = \lambda_1\varphi(f_1) + \lambda_2\varphi(f_2)$ for each $\lambda_1,\lambda_2\in \C$ and $f_1,f_2\in \mathcal{X}$.
  \item There exists $M$ such that $|\varphi(f)| \leq M\norm{f}$ for each $f\in \mathcal{X}$.
\end{enumerate}
\end{definition}
\begin{proposition}[Equivalent Criteria for Bounded Linear Functionals]
Let $\varphi$ be a linear functional on $\mathcal{X}$. Then, the following conditions are equivalent:
\begin{enumerate}[(1)]
  \item $\varphi$ is bounded;
  \item $\varphi$ is continuous;
  \item $\varphi$ is continuous at $0$.
\end{enumerate}
\end{proposition}
\begin{proof}
  \begin{description}[font=\normalfont]
    \item[$(1)\Rightarrow (2)$:] If $\set{f_{\alpha}}_{\alpha \in A}$ is a net in $\mathcal{X}$ converging to $f$, then $\lim_{\alpha \in A}\norm{f_{\alpha} - f} = 0$. Thus,
      \begin{align*}
        \lim_{\alpha \in A}\left\vert \varphi\left(f_{\alpha}\right)-  \varphi\left(f\right) \right\vert &= \lim_{\alpha \in A}\left\vert \varphi(f_{\alpha} - f) \right\vert\\
                                                                                                         &\leq \lim_{\alpha \in F}M\norm{f_{\alpha} - f}\\
                                                                                                         &= 0
      \end{align*}
    \item[$(2)\Rightarrow (3)$:] Trivial.
    \item[$(3) \Rightarrow (1)$:] If $\varphi$ is continuous at $0$, then there exists $\delta > 0$ such that $\norm{f} < \delta \Rightarrow \left\vert \varphi(f) \right\vert < 1$. Thus, for any $g\in X$ nonzero, we have
      \begin{align*}
        \left\vert \varphi\left(g\right) \right\vert &= \frac{2\norm{g}}{\delta}\left\vert \varphi\left(\frac{\delta}{2\norm{g}}g\right) \right\vert\\
                                                     &< \frac{2}{\delta}\norm{g},
      \end{align*}
      meaning $\varphi$ is bounded.
  \end{description}
\end{proof}
\begin{definition}[Dual Space]
  Let $\mathcal{X}^{\ast}$ be the set of bounded linear functionals on $\mathcal{X}$. For each $\varphi \in \mathcal{X}^{\ast}$, define
  \begin{align*}
    \norm{\varphi} &= \sup_{\norm{f} = 1} \left\vert \varphi(f) \right\vert.
  \end{align*}
  We say $\mathcal{X}^{\ast}$ is the dual space of $\mathcal{X}$.
\end{definition}
\begin{proposition}[Completeness of the Dual Space]
  For $\mathcal{X}$ a Banach space, $\mathcal{X}^{\ast}$ is a Banach space.
\end{proposition}
\begin{proof}
  Both positive definiteness and absolute homogeneity are apparent from the definition of the norm. We will now show the triangle inequality as follows. Let $\varphi_1,\varphi_2\in \mathcal{X}^{\ast}$. Then,
  \begin{align*}
    \norm{\varphi_1 + \varphi_2} &= \sup_{\norm{f} = 1}\left\vert \varphi_1\left(f\right) +  \varphi_{2}\left(f\right)\right\vert\\
                                 &\leq \sup_{\norm{f} = 1}\left\vert \varphi_1(f) \right\vert + \sup_{\norm{f} = 1}\left\vert \varphi_2(f) \right\vert\\
                                 &= \norm{\varphi_1} + \norm{\varphi_2}.
  \end{align*}
  We must now show completeness. Let $\set{\varphi_n}_n$ be a sequence in $\mathcal{X}^{\ast}$. Then, for every $f\in \mathcal{X}$, it is the case that
  \begin{align*}
    \left\vert \varphi_n(f) - \varphi_m(f) \right\vert &\leq \norm{\varphi_n - \varphi_m}\norm{f},
  \end{align*}
  meaning $\set{\varphi_n(f)}_n$ is Cauchy for each $f$. Define $\varphi(f) = \lim_{n\rightarrow\infty}\varphi_n(f)$. It is clear that $\varphi(f)$ is linear, and for $N$ such that $n,m \geq N \Rightarrow \norm{\varphi_n - \varphi_m} < 1$,
  \begin{align*}
    \left\vert \varphi(f) \right\vert &\leq \left\vert \varphi(f) - \varphi_N(f) \right\vert + \left\vert \varphi_N(f) \right\vert\\
                                      &\leq \lim_{n\rightarrow\infty}\left\vert \varphi_n(f) - \varphi_N(f) \right\vert + \left\vert \varphi_N(f) \right\vert\\
                                      &\leq \left(\lim_{n\rightarrow\infty}\norm{\varphi_n - \varphi_N} + \norm{\varphi_N}\right)\norm{f}\\
                                      &\leq \left(1 + \norm{\varphi_N}\right)\norm{f},
  \end{align*}
  so $\varphi$ is bounded. Thus, we must show that $\lim_{n\rightarrow\infty}\norm{\varphi_n - \varphi} = 0$. Let $\ve > 0$. Set $N$ such that $n,m\geq N \Rightarrow \norm{\varphi_n - \varphi_m} < \ve$. Then, for $f\in \mathcal{X}$,
  \begin{align*}
    \left\vert \varphi(f) - \varphi_n(f) \right\vert &\leq \left\vert \varphi(f) - \varphi_m(f) \right\vert + \left\vert \varphi_m(f) - \varphi_n(f) \right\vert\\
                                                     &\leq \left\vert (\varphi - \varphi_m)(f) \right\vert + \ve\norm{f}.\\
  \end{align*}
  Since $\lim_{m\rightarrow\infty}\left\vert \left(\varphi - \varphi_m\right)(f) \right\vert = 0$, we have $\norm{\varphi - \varphi_m} < \ve$.
\end{proof}
\begin{proposition}[Banach Spaces and their Duals]\hfill
  \begin{enumerate}[(1)]
    \item The space $\ell^{\infty}$ consists of the set of bounded sequences. For $f\in \ell^{\infty}$, the norm on $f$ is computed as $\displaystyle\norm{f}_{\infty} = \sup_{n} \left\vert f(n) \right\vert$.
    \item The subspace $c_0\subseteq \ell^{\infty}$ consists of all sequences that vanish at $\infty$. The norm on $c_0$ is inherited from the norm on $\ell_{\infty}$.
    \item The space $\ell^{1}$ consists of the set of all absolutely summable sequences. For $f\in \ell^{1}$, the norm on $f$ is computed as $\displaystyle \norm{f} = \sum_{n=1}^{\infty}\left\vert f(n) \right\vert$.
  \end{enumerate}
  We claim that these are all Banach spaces.\newline

  We also claim that $c_0^{\ast} = \ell^1$, and $\left(\ell^{1}\right)^{\ast} = \ell^{\infty}$.
\end{proposition}
\begin{proof}[Proofs of Banach Space]\hfill
  \begin{description}[font = \normalfont]
    \item[$\ell^{\infty}$:]\hfill
      \begin{description}
        \item[Proof of Normed Vector Space:] Let $a,b\in \ell^{\infty}$, and $\lambda \in \C$. Then,
          \begin{align*}
            \sup_{n}|a(n)| = 0
          \end{align*}
          if and only if $a$ is the zero sequence. Additionally, we have that
          \begin{align*}
            \norm{\lambda a}_{\infty} &= \sup_{n}\left\vert \lambda a(n) \right\vert\\
                             &= |\lambda|sup_{n}|a(n)|\\
                             &= |\lambda|\norm{a}_{\infty},
          \end{align*}
          meaning $\norm{\cdot}_{\infty}$ is absolutely homogeneous. Finally,
          \begin{align*}
            \norm{a + b}_{\infty} &= \sup_{n}\left\vert a(n) + b(n) \right\vert\\
                         &\leq \sup_{n}\left\vert a(n) \right\vert + \sup_{n}\left\vert b(n) \right\vert\\
                         &= \norm{a}_{\infty} + \norm{b}_{\infty}.
          \end{align*}
        \item[Proof of Completeness:] Let $\set{a_n}_{n=1}^{\infty}$ be a Cauchy sequence of elements of $\ell^{\infty}$. Let $\ve > 0$, and let $N$ be such that $\norm{a_n - a_m}_{\infty} < \varepsilon$ for $n,m \geq N$. Then, for each $k$,
          \begin{align*}
            \left\vert a_{n}(k) - a_m(k) \right\vert &= \left\vert (a_n - a_m)(k) \right\vert\\
                                                     &\leq \norm{a_n - a_m}\\
                                                     &< \varepsilon,
          \end{align*}
          meaning that $a_n(k)$ is Cauchy in $\C$ for each $k$.\newline

          Set $a(k) = \lim_{n\rightarrow\infty}a_n(k)$. We must now show that $\lim_{n\rightarrow\infty}\norm{a - a_n} = 0$. Let $\ve > 0$, and set $N$ such that for $n,m\geq N$, $\norm{a_m - a_n} < \ve$. Then,
          \begin{align*}
            \left\vert a(k) - a_n(k) \right\vert &\leq \left\vert a(k) - a_m(k) \right\vert + \left\vert a_m(k) - a_n(k) \right\vert\\
                                                 &\leq \left\vert a(k) - a_m(k) \right\vert + \norm{a_m - a_n}\\
                                                 &< \left\vert a(k) - a_m(k) \right\vert + \ve.
          \end{align*}
          Since $\lim_{m\rightarrow\infty} \left\vert a(k) - a_m(k) \right\vert = 0$, we have $\norm{a - a_n} < \ve$.\footnote{The reason we had to go about it like this was that we defined the sequence $a$ pointwise; however, we need to show convergence \textit{in norm}.}
      \end{description}
    \item[$c_0$:]\hfill
      \begin{description}
        \item[Proof of Subspace:] Let $a,b\in c_0$, and $\lambda \in \C\setminus \set{0}$. Let $\ve > 0$. Set $N_1$ such that $|a(n)| < \frac{\ve}{2|\lambda|}$ for all $n\geq N_1$, and set $N_2$ such that $|b(n)| < \frac{\ve}{2}$ for all $n \geq N_2$.\newline

          Then, for all $n \geq \max\set{N_1,N_2}$,
          \begin{align*}
            |\lambda a(n) + b(n)| &\leq |\lambda||a(n)| + |b(n)|\\
                                  &< |\lambda|\frac{\ve}{2|\lambda|} + \frac{\ve}{2}\\
                                  &= \ve.
          \end{align*}
        \item[Proof of Completeness:] In order to show completeness, we must show that $c_0$ is closed in $\ell^{\infty}$. Let $\set{a_k}_{k=1}^{\infty}$ be a sequence in $c_0$, with $a_k \rightarrow a$.\newline

          We will need to show that $a\in c_0$.\footnote{Sequential criterion for closure.} Let $\ve > 0$, and set $K$ such that for all $k\geq K$, $\norm{a_k - a} < \ve/2$. For each $k$, choose $N$ such that $|a_k(n)| < \ve/2$ for all $n \geq N$. Then, for all $n\geq N$,
          \begin{align*}
            \left\vert a(n) \right\vert &\leq \left\vert a(n) - a_k(n) \right\vert + \left\vert a_k(n) \right\vert\\
                                        &< \norm{a - a_k} + \left\vert a_k(n) \right\vert\\
                                        &< \ve.
          \end{align*}
          Since $c_0$ is closed in $\ell^{\infty}$, it is thus complete.
      \end{description}
    \item[$\ell^{1}$:]\hfill
      \begin{description}
        \item[Proof of Normed Vector Space:] Let $a,b\in \ell^{1}$, and $\lambda \in \C$. Then,
          \begin{align*}
            \norm{\lambda a + b} &= \sum_{k=1}^{\infty}\left\vert \lambda a(k) + b(k) \right\vert\\
                                 &\leq \sum_{k=1}^{\infty}\left\vert \lambda a(k) \right\vert + \sum_{k=1}^{\infty}\left\vert b(k) \right\vert\\
                                 &= |\lambda|\sum_{k=1}^{\infty}\left\vert a(k) \right\vert + \sum_{k=1}^{\infty}\left\vert b(k) \right\vert\\
                                 &= |\lambda|\norm{a} + \norm{b}.
          \end{align*}
          Thus, $\lambda a + b\in \ell^{1}$. We have also shown both the triangle inequality and absolute homogeneity. We can also see that, if $\norm{a} = 0$,
          \begin{align*}
            \norm{a} &= \sum_{k=1}^{\infty}\left\vert a(k) \right\vert\\
            &= 0,
          \end{align*}
          which is only true if $a(k) = 0$ for all $k$.
      \end{description}
  \end{description}
\end{proof}
\begin{example}[Pointwise Convergence and Convergence in Norm]
  Consider a sequence $\set{\varphi_n}_n$ in $\mathcal{X}^{\ast}$. If the sequence converges in norm to $\varphi$, then it must also converge pointwise. However, the converse isn't true.\newline

  For each $k$, define $L_k(f) = f(k)$, where $f\in \ell^{1}$. We can see that $L_k \in \left(\ell^{1}\right)^{\ast}$, and $\lim_{k\rightarrow\infty}L_k(f) = 0$ for each $f\in \ell^{1}$. The sequence of $L_k$ thus converges to the zero functional pointwise, but since $\norm{L_k} = 1$ always, it isn't the case that $L_k$ converges to the zero functional in norm.
\end{example}
\begin{definition}[Weak Topology and $w^{\ast}$-Topology]
Let $X$ be a set, $Y$ a topological space, and $\mathcal{F}$ be a family of functions from $X$ to $Y$. The weak topology on $X$ is the topology for which all functions in $\mathcal{F}$ are continuous.\newline

For each $f$ in $\mathcal{X}$, let $\hat{f}: \mathcal{X}^{\ast}\rightarrow \C$ be defined by $\hat{f}(\varphi) = \varphi(f)$. The $w^{\ast}$-topology on $\mathcal{X}^{\ast}$ is the weak topology on $\mathcal{X}^{\ast}$ defined by the family of functions $\set{\hat{f}\mid f\in \mathcal{X}}$.\newline

If $Y$ is Hausdorff and $\mathcal{F}$ separates the points of $X$, then the weak topology is Hausdorff.\footnote{I am trying to find a source to prove this, will include the proof of this implicit proposition hopefully.}
\end{definition}
\begin{proposition}[Hausdorff Property of $w^{\ast}$-Topology]
  The $w^{\ast}$-topology on $\mathcal{X}^{\ast}$ is Hausdorff.
\end{proposition}
\begin{proof}
  If $\varphi_1 \neq \varphi_2$, then there exists at least one $f$ such that $\varphi_1(f) \neq \varphi_2(f)$, meaning $\set{\hat{f}\mid f\in \mathcal{X}}$ separates the points of $\mathcal{X}^{\ast}$, so the $w^{\ast}$-topology is Hausdorff.
\end{proof}
\begin{proposition}[Convergence in the $w^{\ast}$-Topology]
  A net $\set{\varphi_{\alpha}}_{\alpha}$ converges to $\varphi\in\mathcal{X}^{\ast}$ in the $w^{\ast}$ topology if and only if $\lim_{\alpha \in A}\varphi_{\alpha} = \varphi$.\footnote{In the special case of Hilbert space $\mathcal{H}$, we know from the Riesz Representation Theorem that each $\varphi\in \mathcal{H}^{\ast}$ is represented by $\psi$ such that $\varphi(f) = \iprod{f}{\psi}$.}
\end{proposition}
\begin{proposition}[Determination of the $w^{\ast}$-Topology]
  Let $\mathcal{M}$ be a dense subset of $\mathcal{X}$, and let $\set{\varphi_{\alpha}}_{\alpha\in A}$ be a uniformly bounded net in $\mathcal{X}^{\ast}$, where $\lim_{\alpha \in A}\varphi_{\alpha}(f) = \varphi(f)$ for each $f\in \mathcal{M}$. Then, the net $\set{\varphi_{\alpha}}_{\alpha \in A}$ converges to $\varphi$ in the $w^{\ast}$ topology.
\end{proposition}
\begin{proof}
  Let $\displaystyle M = \sup_{\alpha \in A}\max\set{\norm{\varphi_{\alpha}},\norm{\varphi}}$, and let $\ve > 0$.\newline

  Given $g\in \mathcal{X}$, choose $f\in \mathcal{M}$ such that $\norm{f - g} < \frac{\ve}{3M}$. Let $\alpha_0\in A$ such that $\alpha \geq \alpha_0$  implies $\left\vert \varphi_{\alpha}(f) - \varphi(f) \right\vert < \frac{\ve}{3}$. Then, for all $\alpha \geq \alpha_0$,
  \begin{align*}
    \left\vert \varphi_{\alpha}(g) - \varphi(g) \right\vert &\leq \left\vert \varphi_{\alpha}(g) - \varphi_{\alpha}(f) \right\vert + \left\vert \varphi_{\alpha}(f) - \varphi(f) \right\vert + \left\vert \varphi(f) - \varphi(g) \right\vert\\
                                                            &\leq \norm{\varphi_{\alpha}}\norm{f-g} + \frac{\ve}{3} + \norm{\varphi}\norm{f-g}\\
                                                            &< \ve.
  \end{align*}
\end{proof}
\begin{definition}[Unit Ball]
  For $\mathcal{X}$ a Banach space, we denote the unit ball as $B_{\mathcal{X}} = \set{ f\in \mathcal{X}\mid \norm{f}\leq 1}$.\footnote{The book uses a different notation, but I don't like that notation.}
\end{definition}
\begin{theorem}[Banach--Alaoglu]
  The set $B_{\mathcal{X}^{\ast}}$ is compact in the $w^{\ast}$-topology.
\end{theorem}
\begin{proof}
  Let $f\in B_{\mathcal{X}}$. Let $\overline{\mathbb{D}}^{f}$ denote the $f$-labeled copy of the closed unit disc in $\C$. Set
  \begin{align*}
    P &= \prod_{f\in B_{\mathcal{X}}}\overline{\mathbb{D}}^{f}.
  \end{align*}
  Then, $P$ is compact by Tychonoff's theorem.\newline

  Define $\Lambda: B_{\mathcal{X}^{\ast}} \rightarrow P$ by $\Lambda(\varphi) = \varphi\vert_{B_{\mathcal{X}}}$. Notice that $\Lambda(\varphi_1) = \Lambda(\varphi_2)$ implies that $\varphi_1 = \varphi_2$ on $B_{\mathcal{X}}$, meaning $\varphi_1 = \varphi_2$. Therefore, $\Lambda$ is injective.\newline

  Let $\set{\varphi_{\alpha}}_{\alpha \in A}$ be a net in $\mathcal{X}^{\ast}$ converging to $\varphi$ in the $w^{\ast}$-topology. Then,
  \begin{align*}
    \lim_{\alpha \in A}\varphi_{\alpha}(f) &= \varphi(f)\\
    \lim_{\alpha \in A}\left(\Lambda\left(\varphi_{\alpha}\right)\right)(f) &= \lim_{\alpha \in A}\left(\Lambda\left(\varphi\right)\right)(f),
  \end{align*}
  meaning
  \begin{align*}
    \lim_{\alpha \in A}\Lambda\left(\varphi_{\alpha}\right) = \Lambda\left(\varphi\right)
  \end{align*}
  in $P$. Since $\Lambda$ is one-to-one, we can see that $\Lambda: \mathcal{B}_{\mathcal{X}^{\ast}} \rightarrow \Lambda\left(B_{\mathcal{X}^{\ast}}\right)\subseteq P$ is a linear homeomorphism.\newline

  Let $\set{\Lambda\left(\varphi_{\alpha}\right)}_{\alpha\in A}$ be a net in $\Lambda\left(B_{\mathcal{X}^{\ast}}\right)$ converging in the product topology to $\psi$. Let $f,g \in B_{\mathcal{X}^{\ast}}$ and $\xi \in \C$ with $f+g\in B_{\mathcal{X}^{\ast}}$ and $\xi f \in B_{\mathcal{X}^{\ast}}$. Then,
  \begin{align*}
    \psi\left(f+g\right) &= \lim_{\alpha \in A}\left(\Lambda\left(\varphi_{\alpha}\right)\right)(f+g)\\
                         &= \lim_{\alpha \in A}\left(\Lambda\left(\varphi_{\alpha}\right)\right)(f) + \lim_{\alpha \in A}\left(\Lambda\left(\varphi_{\alpha}\right)\right)(g)\\
                         &= \psi(f) + \psi(g)
  \end{align*}
  and
  \begin{align*}
    \psi(\xi f) &= \lim_{\alpha \in A}\left(\Lambda\left(\varphi_{\alpha}\right)\right)(\xi f)\\
                 &= \lim_{\alpha \in A}\varphi_{\alpha}\left(\xi f\right)\\
                 &= \varphi\left(\xi f\right)\\
                 &= \xi \varphi\left(f\right)\\
                 &= \xi \left(\Lambda\left(\varphi\right)\right)(f)\\
                 &= \xi \psi(f).
  \end{align*}
  Thus, $\psi(f)$ determines $\tilde{\psi}(f) = \frac{1}{\norm{f}}\psi\left(f\right)$ in $B_{\mathcal{X}^{\ast}}$ for all $f\in \mathcal{X}\setminus \set{0}$. If $f\in B_{\mathcal{X}}$, then $\tilde{\psi} \in \mathcal{B}_{\mathcal{X}^{\ast}}$ and $\Lambda(\tilde{\psi}) = \psi$.\newline

  Thus, $\Lambda\left(B_{\mathcal{X}^{\ast}}\right)$ is closed in $P$, meaning $B_{\mathcal{X}^{\ast}}$ is compact in the $w^{\ast}$-topology.
\end{proof}
We will be able to use the Banach--Alaoglu theorem to prove that every Banach space is isomorphic to a subspace of $C(X)$ for some compact Hausdorff space $X$. However, we will need some theorems and machinery to prove that
\begin{definition}[Sublinear Functionals]
  Let $\mathcal{E}$ be a real linear space, and let $p$ be a real-valued functional on $\mathcal{E}$. We say $p$ is a sublinear functional if $p(f+g)\leq p(f) + p(g)$ for all $f,g\in \mathcal{E}$, and $p(\lambda f) = \lambda p(f)$.
\end{definition}
\begin{theorem}[Hahn--Banach Dominated Extension]
  Let $\mathcal{E}$ be a real linear space, and $p$ a (real-valued) sublinear functional on $\mathcal{E}$. Let $\mathcal{F}\subseteq \mathcal{E}$ be a subspace, and $\varphi$ a real linear functional on $\mathcal{F}$ such that $\varphi(f) \leq p(f)$ for all $f\in \mathcal{F}$.\newline

  Then, there exists a real linear functional $\Phi$ on $\mathcal{E}$ such that $\Phi(f) = \varphi(f)$ for $f\in \mathcal{F}$, and $\Phi(g) \leq p(g)$ for all $g\in \mathcal{E}$.
\end{theorem}
\begin{proof}
  Let $\mathcal{F}\subseteq \mathcal{E}$ be a nonempty subspace, and let $f\notin \mathcal{F}$. Select $\mathcal{G} = \set{g + \lambda f\mid g\in \mathcal{F},~\lambda \in \R}$.\newline

  We will extend $\varphi$ to $\Phi_{\mathcal{G}}$ by taking $\Phi(g + \lambda f)\leq p(g + \lambda f)$. Dividing by $|\lambda|$, we find that, for all $h\in \mathcal{F}$
  \begin{align*}
    \Phi(f - h) &\leq p(f - h)\\
    \intertext{and}
    -p\left(h - f\right)&\leq \Phi\left(h - f\right).
  \end{align*}
  Thus, recalling that $\Phi(h) = \varphi(h)$ for $h\in \mathcal{F}$,
  \begin{align*}
    -p\left(h-f\right) + \varphi(h) \leq \Phi\left(f\right) \leq p(f-h) + \varphi(h).
  \end{align*}
  The desired $\Phi$ only has this property if
  \begin{align*}
    \sup_{h\in \mathcal{F}}\set{\varphi(h) - p(h-f)} \leq \inf_{k\in \mathcal{F}}\set{\varphi(k) + p(f-k)}.
  \end{align*}
  However, we also have
  \begin{align*}
    \varphi(h)- \varphi(k) &= \varphi(h-k)\\
                           &\leq p(h-k)\\
                           &\leq p(f-k) + p(h-f),
  \end{align*}
  meaning
  \begin{align*}
    \varphi(h) - p(h-f) &\leq \varphi(k) + p(f-k).
  \end{align*}
  Therefore, we can thus extend $\varphi$ on $\mathcal{F}$ to $\Phi$ on $\mathcal{G}$, where $\Phi(h) \leq p(h)$. We label this as $\Phi_{\mathcal{G}}$.\newline

  Let $\mathcal{P} = \set{(\mathcal{G}_{\delta},\Phi_{\mathcal{G}_{\delta}})}_{\delta \in D}$ denote the class of extensions of $\varphi$ such that $\Phi_{\mathcal{G}_{\delta}}(h) \leq p(h)$ for all $h\in \mathcal{G}_{\delta}$.\newline

  An element of $\mathcal{P}$ contains $\mathcal{G}$ such that $\mathcal{F}\subseteq \mathcal{G}\subseteq \mathcal{E}$, where $\Phi_{\mathcal{G}}$ extends $\varphi$, meaning $\mathcal{P}$ is nonempty.\newline

  The partial order on $\mathcal{P}$ can be set by $\left(\mathcal{G}_{1},\Phi_{\mathcal{G}_2}\right)\leq \left(\mathcal{G}_{2},\Phi_{\mathcal{G}_{2}}\right)$ if $G_1\subseteq G_2$ and $\Phi_{\mathcal{G}_1}(f) = \Phi_{\mathcal{G}_{2}}(f)$ for all $f\in \mathcal{G}_{1}$.\newline

  Consider a chain\footnote{totally ordered subset} $\set{\left(\mathcal{G}_{\alpha},\Phi_{\mathcal{G}_{\alpha}}\right)}_{\alpha \in A}$. To find an upper bound, consider
  \begin{align*}
    \mathcal{G} &= \bigcup_{\alpha \in A}\mathcal{G}_{\alpha},
  \end{align*}
  where $\Phi_{\mathcal{G}}(f) = \Phi_{\mathcal{G}_{\alpha}}(f)$ for every $f\in \mathcal{G}_{\alpha}$. Then, $\Phi_{\mathcal{G}}$ is a linear functional that satisfies the given properties,\footnote{I am too lazy to prove this.} and $\left(\mathcal{G},\Phi_{\mathcal{G}}\right)$ is an upper bound for $\set{\left(\mathcal{G}_{\alpha},\Phi_{\mathcal{G}_{\alpha}}\right)}$.\newline

  Thus, by Zorn's Lemma, there is a maximal element of $\mathcal{P}$, $\left(\mathcal{G}_{\text{max}},\Phi_{\mathcal{G}_{\text{max}}}\right)$. If $\mathcal{G}_0\neq \mathcal{E}$, then we can find a $f\notin \mathcal{G}_{0}$ and repeat the process performed at the beginning of the proof, which would contradict maximality.\newline

  Thus, we have constructed a linear functional $\Phi$ such that $\Phi(f) \leq p(f)$ for all $f\in \mathcal{E}$ that extends $\varphi$.
\end{proof}
\begin{theorem}[Hahn--Banach Continuous Extension]
  Let $\mathcal{M}$ be a subspace of the Banach space $\mathcal{X}$. If $\varphi$ is a bounded linear functional on $\mathcal{M}$, then there exists $\Phi$ on $\mathcal{X}^{\ast}$ such that $\Phi(f) = \varphi(f)$ for all $f\in \mathcal{M}$ and $\norm{\Phi} = \norm{\varphi}$.
\end{theorem}
\begin{proof}
  Consider $\tilde{\mathcal{X}}$ as the real linear space on which $\norm{\cdot}$ is the sublinear functional. Set $\psi = \text{Re}\left(\varphi\right)$ on $\mathcal{M}$.\newline

  We can see that, since $\text{Re}\left(\varphi(f)\right) \leq |\varphi(f)|$, $\norm{\psi}\leq \norm{\varphi}$.\newline

  Set $p(f) = \norm{\varphi}\norm{f}$. Since $\psi(f) \leq p(f)$ for all $f\in \mathcal{X}$, by the dominated extension theorem, there exists $\Psi$ defined on $\tilde{\mathcal{X}}$ that extends $\psi$. In particular, we can see that $\Psi(f) \leq \norm{\varphi}\norm{f}$.\newline

  Define $\Phi$ on $\mathcal{X}$ by $\Phi(f) = \Psi(f) - i\Psi(if)$ for any $f\in \mathcal{X}$. We will show that $\Phi$ is a complex bounded linear functional that extends $\varphi$ and has norm $\norm{\varphi}$. We can see that
  \begin{align*}
    \Phi(f+g) &= \Psi(f + g) - i\Psi\left(i\left(f+g\right)\right) \\
              &= \Psi(f) - i\Psi(if) +\Psi(g) - i\Psi(ig)\\
              &= \Phi(f) + \Phi(g),
  \end{align*}
  and for $\lambda_1,\lambda_2\in \R$,\footnote{Notice that $\Phi(if) = \Psi(if) - i\Psi(-f) = i\Psi(f) + \Psi(if) = i\Phi(f)$}
  \begin{align*}
    \Phi\left(\left(\lambda_1 + i\lambda_2\right)f\right) &= \Phi\left(\lambda_1 f\right) + \Phi\left(i\lambda_2 f\right)
                                                          &=\left(\lambda_1 + i\lambda_2\right)\Phi(f).
  \end{align*}
  To verify that $\Phi(f)$ extends $\varphi(f)$, let $f\in \mathcal{M}$, and we can see that
  \begin{align*}
    \Phi(f) &= \Psi(f) - i\Psi(if)\\
            &= \psi(f) - i\psi(if)\\
            &= \text{Re}\left(\varphi(f)\right) - i\text{Re}\left(\varphi(if)\right)\\
            &= \text{Re}\left(\varphi(f)\right) - i\left(-\text{Im}\left(\varphi(f)\right)\right)\\
            &= \varphi(f).
  \end{align*}
  Finally, to verify that $\norm{\Phi} = \norm{\varphi}$, all we need show is that $\norm{\Phi} \leq \norm{\Psi}$. Let $\Phi(f) = re^{i\theta}$. Then,
  \begin{align*}
    \left\vert \Phi(f) \right\vert &= r\\
                                   &= e^{-i\theta}\Phi(f)\\
                                   &= \Phi\left(e^{-i\theta}f\right)\\
                                   &= \Psi\left(e^{-i\theta}f\right)\\
                                   &\leq \left\vert \Psi\left(e^{-i\theta}f\right) \right\vert\\
                                   &\leq \norm{\Psi}\norm{f},
                                   \intertext{meaning}
                                   \norm{\Phi}\norm{f} \leq \norm{\Psi}\norm{f}.
  \end{align*}
\end{proof}
\begin{corollary}[Norming Functional]
  If $f\in \mathcal{X}$, then there exists $\varphi \in \mathcal{X}^{\ast}$ such that $\norm{\varphi} = 1$ and $\varphi(f) = \norm{f}$.
\end{corollary}
\begin{proof}
  Assume $f\neq 0$. Let $\mathcal{M} = \set{\lambda f\mid \lambda \in \C}$, and define $\psi$ on $\mathcal{M}$ by $\psi(\lambda f) = \lambda \norm{f}$. Then, $\norm{\psi} = 1$ and an extension of $\psi$ to $\mathcal{X}$ has the desired properties.
\end{proof}
\begin{theorem}[Banach]
  Let $\mathcal{X}$ be any Banach space. Then, $\mathcal{X}$ is isometrically isomorphic to some closed subspace of $C(X)$ for compact Hausdorff $X$.
\end{theorem}
\begin{proof}
  Set $X = B_{\mathcal{X}^{\ast}}$ in the $w^{\ast}$-topology, which by Banach--Alaoglu, is compact.\newline

  Set $\beta: \mathcal{X}\rightarrow C(X)$ by $\beta(f)(\varphi) = \varphi(f)$. Then, for $\lambda_1,\lambda_2\in \C$, $f_1,f_2\in \mathcal{X}$,
  \begin{align*}
    \beta\left(\lambda_1f_1 + \lambda_2f_2\right)(\varphi) &= \varphi\left(\lambda_1 f_1 + \lambda_2f_2\right)\\
                                                           &= \lambda_1 \varphi(f_1) + \lambda_2\varphi\left(f_2\right)\\
                                                           &= \left(\lambda_1\beta\left(f_1\right) + \lambda_2\beta\left(f_2\right)\right)\left(\varphi\right).
  \end{align*}
  Let $f\in \mathcal{X}$. Then,
  \begin{align*}
    \norm{\beta(f)}_{\infty} &= \sup_{\varphi \in B_{\mathcal{X}^{\ast}}} \left\vert \beta(f)(\varphi) \right\vert\\
                             &= \sup_{\varphi \in B_{\mathcal{X}^{\ast}}}\left\vert \varphi(f) \right\vert\\
                             &\leq \sup_{\varphi \in B_{\mathcal{X}^{\ast}}}\norm{\varphi}\norm{f}\\
                             &\leq \norm{f}.
  \end{align*}
  Additionally, since there exists a norming functional in $B_{\mathcal{X}^{\ast}}$, we have that $\norm{\beta(f)}_{\infty} = \norm{f}$, meaning $\beta$ is an isometric isomorphism.
\end{proof}
\begin{note}
  The preceding construction cannot yield an isometric isomorphism to $C\left(B_{\mathcal{X}^{\ast}}\right)$ itself, even if $\mathcal{X} = C(Y)$ for some $Y$.\newline

  It can be shown via topological arguments that if $\mathcal{X}$ is separable, we can take $X$ to be the interval $[0,1]$.
\end{note}
Now, we turn to finding the dual space of $C([0,1])$. In particular, we will soon find out that $C([0,1]) = \text{BV}([0,1])$, which is the space of all functions of bounded variation.
\begin{definition}[Bounded Variation]
  If $\varphi$ is a complex function with domain $[0,1]$, $\varphi$ is said to be of bounded variation if for every partition $0 = t_0 < t_1 < \cdots < t_{n} < t_{n+1} = 1$, it is the case that
  \begin{align*}
    \sum_{i=0}^{n}\left\vert \varphi\left(t_{n+1}\right) - \varphi\left(t_{n}\right) \right\vert \leq M.
  \end{align*}
  The infimum of all such values of $M$ is denoted $\norm{\varphi}_{\text{BV}}$.\footnote{The book uses $\norm{\varphi}_{\nu}$, but I think that's more confusing than BV.} Henceforth, all functions of bounded variation will be referred to as BV functions.
\end{definition}
\begin{proposition}[Limits of BV Functions]
  A BV function possesses a limit from the left and right at each endpoint.
\end{proposition}
\begin{proof}
  Let $\varphi: [0,1]\rightarrow \C$ not have a limit from the left at some point $t\in (0,1]$.\newline

  Then, for any $\delta > 0$, there exist $s_1,s_2$ such that $t-\delta < s_1 < s_2 < t$ and $\left\vert \varphi(s_2) - \varphi(s_1) \right\vert \geq \ve$. Selecting $\delta_2 = t - s_2$, we inductively create a sequence $\set{s_n}_{n = 1}^{\infty}$ where $0 < s_1 < s_2 < \cdots < s_n < \cdots < t$.\newline

  Consider a partition $t_0 = 0$, and $t_k= s_k$ for $k = 1,2,\dots,N $, and $t_{N+1} = 1$, we have
  \begin{align*}
    \sum_{k=0}^{N}\left\vert \varphi(t_{k+1}) - \varphi(t_k) \right\vert &\geq \sum_{k=1}^{N}\left\vert \varphi(s_{k+1}) - \varphi(s_{k}) \right\vert\\
                                                                         &\geq N\varepsilon.
  \end{align*}
  Thus, $\varphi$ is not a BV function.
\end{proof}
\begin{corollary}[Discontinuities of a BV Function]
  Let $\varphi: [0,1]\rightarrow \C$ be a BV function. Then, $\varphi$ has countably many discontinuities.
\end{corollary}
\begin{proof}
  Notice that $\varphi$ is discontinuous at a point $t$ if and only if $\varphi(t) \neq \varphi\left(t^{+}\right)$ or $\varphi(t) \neq \varphi\left(t^{-}\right)$.\newline

  If $t_0,t_1,\dots,t_n$ are distinct points of $[0,1]$, then
  \begin{align*}
    \sum_{i=0}^{N}\left\vert \varphi(t) - \varphi(t^{+}) \right\vert + \sum_{i=0}^{N}\left\vert \varphi(t) - \varphi(t^{-}) \right\vert \leq \norm{\varphi}_{\text{BV}}.
  \end{align*}
  Thus, for every $\ve > 0$, there exist at most finitely many $t$ such that $\left\vert \varphi(t) - \varphi\left(t^{+}\right) \right\vert + \left\vert \varphi(t) - \varphi(t^{-}) \right\vert \geq \ve$, meaning there can be at most countably many discontinuities.
\end{proof}
\begin{definition}[Riemann--Stieltjes Integral]
  Let $f\in C([0,1])$, and let $\varphi \in \text{BV}\left([0,1]\right)$. Then, we denote the Riemann--Stieltjes integral
  \begin{align*}
    \int_{0}^{1}f\:d\varphi &= \sum_{i=0}^{n}f\left(t_i'\right)\left[\varphi\left(t_{i+1}\right) - \varphi\left(t_i\right)\right],
  \end{align*}
  where $\{t_i\}$ is a partition and $t_i'\in [t_i,t_{i+1}]$.
\end{definition}
\begin{proposition}[Essential properties of the Riemann--Stieltjes Integral]
  If $f\in C([0,1])$ and $\varphi\in \text{BV}\left([0,1]\right)$, then
  \begin{enumerate}[(1)]
    \item $\displaystyle \int_{0}^{1} f\:d\varphi$ exists;
    \item $\displaystyle \int_{0}^{1} \left(\lambda_1f_1 + \lambda_2f_2\right)\:d\varphi = \lambda_1 \int_{0}^{1} f_1\:d\varphi + \lambda_2 \int_{0}^{1} f_2\:d\varphi$ for $\lambda_1,\lambda_2\in \C$ and $f_1,f_2 \in C([0,1])$;
    \item $\displaystyle \int_{0}^{1} f\:d\left(\lambda_1\varphi_1 + \lambda_2\varphi_2\right) = \lambda_1\int_{0}^{1} f_1\:d\varphi_1 + \lambda_2 \int_{0}^{1} f_2\:d\varphi_2$ for $\lambda_1,\lambda_2\in \C$ and $\varphi_1,\varphi_2\in \text{BV}\left([0,1]\right)$;
    \item $\displaystyle \left\vert \int_{0}^{1} f\:d\varphi \right\vert \leq \norm{f}_{\infty}\norm{\varphi}_{\text{BV}}$ for $f\in C\left([0,1]\right)$ and $\varphi \in \text{BV}\left([0,1]\right)$.
  \end{enumerate}
\end{proposition}
\begin{proposition}[BV Function Limits and Riemann--Stieltjes Integrals]
  Let $\varphi\in \text{BV}\left([0,1]\right)$ and $\psi$ be defined by $\psi(t) = \varphi\left(t^{-}\right)$ for $t\in (0,1)$, where $\psi(0) = \varphi(0)$ and $\psi(1) = \varphi(1)$.\newline

  Then, $\psi \in \text{BV}\left([0,1]\right)$, $\norm{\psi}_{\text{BV}} \leq \norm{\varphi}_{\text{BV}}$, and
  \begin{align*}
    \int_{0}^{1} f\:d\varphi &= \int_{0}^{1} f\:d\psi
  \end{align*}
  for $f\in C\left([0,1]\right)$.
\end{proposition}
\begin{proof}
  We list the set $\set{s_i}_{i\geq 1}$ the points where $\varphi$ is discontinuous from the left. By the definition of $\psi$, we have $\psi(t) = \varphi(t)$ for $t\notin \set{s_i}_{i\geq 1}$.\newline

  Let $0 = t_0 < t_1 < \cdots < t_n < t_{n+1} = 1$ be a partition where if $t_i \in \set{s_i}_{i\geq 1}$, then neither $t_{i-1}$ nor $t_{i+1}$ is. To show that $\psi$ is BV, then we must show
  \begin{align*}
    \sum_{i=0}^{n}\left\vert \psi\left(t_{i+1}\right) - \psi\left(t_{i}\right) \right\vert \leq \norm{\varphi}_{\text{BV}}.
  \end{align*}
  Set $\ve > 0$. If $t_i \notin \set{s_i}_{i\geq 1}$, $i=0$, or $i = n + 1$, then set $t_i' = t_i$. If $t_i\in \set{s_i}_{i\geq 1}$ and $i\neq 0,n+1$, choose $t_i'\in \left(t_{i-1},t_i\right)$ such that $\left\vert \varphi\left(t_i^-\right) - \varphi\left(t_i'\right) \right\vert < \frac{\ve}{2n}$. Then, $0 = t_0' < t_1' < \cdots < t_n' < t_{n+1}' = 1$ is a partition of $0,1$ with
  \begin{align*}
    \sum_{i=0}^{n}\left\vert \psi\left(t_{i+1}\right)- \psi\left(t_i\right) \right\vert &= \sum_{i=0}^{n}\left\vert \varphi\left(t_{i+1}^{-}\right) - \varphi_{t_i^{-}} \right\vert\\
                                                                                        &\leq \sum_{i=0}^{n}\left\vert \varphi\left(t_{i+1}^{-}\right) - \varphi\left(t_{i+1}'\right)\right\vert + \sum_{i=0}^{n}\left\vert \varphi\left(t_{i+1}'\right) - \varphi(t_i') \right\vert + \sum_{i=0}^{n}\left\vert \varphi\left(t_i'\right) - \varphi\left(t_i^{-}\right) \right\vert\\
                                                                                        &\leq \frac{\ve}{2} + \norm{\varphi}_{\text{BV}} + \frac{\ve}{2}
  \end{align*}
  Since $\ve $ was arbitrary, $\psi\in \text{BV}\left([0,1]\right)$, with $\norm{\psi}_{\text{BV}} \leq \norm{\varphi}_{\text{BV}}$.\newline

  For $N$ any arbitrary integer, define $\eta_{N}(t) = 0$ for $t$ not in $\set{s_1,s_2,\dots,s_N}$, and $\eta_{N}(s_i) = \varphi(s_i) - \psi(s_i)$. Then, we can see that $\norm{\varphi - \left(\psi + \eta_N\right)}_{\text{BV}} = 0$, with $\int_{0}^{1}f\:d\eta_N = 0$. Thus,
  \begin{align*}
    \int_{0}^{1}f\:d\varphi &= \int_{0}^{1}f\:d\psi + \lim_{N\rightarrow\infty}\int_{0}^{1}f\:d\eta_N\\
                            &= \int_{0}^{1} f\:d\psi.
  \end{align*}
\end{proof}
We let $\text{BV}\left([0,1]\right)$ be the space of all BV functions with pointwise addition and scalar multiplication, with norm $\norm{\cdot}_{\text{BV}}$.\footnote{Yes, technically before now I was engaging in a gross abuse of notation.}
\begin{theorem}
  $\text{BV}\left([0,1]\right)$ is a Banach space.
\end{theorem}
\begin{proof}
  Suppose $\set{\varphi_n}_{n=1}^{\infty}$ is a sequence in $\text{BV}\left([0,1]\right)$ such that
  \begin{align*}
    \sum_{n=1}^{\infty}\norm{\varphi_n}_{\text{BV}} & < \infty.
  \end{align*}
  Additionally,
  \begin{align*}
    \left\vert \varphi_n(t) \right\vert &\leq \left\vert \varphi_n(t) - \varphi_n(0) \right\vert + \left\vert \varphi_n(1) - \varphi_n(t) \right\vert\\
                                        &\leq \norm{\varphi_n}_{\text{BV}}
  \end{align*}
  for $t\in [0,1]$, meaning
  \begin{align*}
    \sum_{n=1}^{\infty}\varphi_n(t)
  \end{align*}
  converges uniformly and absolutely to a function $\varphi$ defined on $[0,1]$. We can see that $\varphi(0) = 0$ and $\varphi$ is continuous from the left on $(0,1)$. We must now show that $\varphi$ is of bounded variation and
  \begin{align*}
    \lim_{N\rightarrow\infty}\norm{\varphi - \sum_{n=1}^{N}\varphi_n} &= 0.
  \end{align*}
  To start, let $0 = t_0 < t_1 < \cdots < t_k < t_{k+1} = 1$ be a partition of $[0,1]$. Then,
  \begin{align*}
    \sum_{i=0}^{k}\left\vert \varphi\left(t_{i+1}\right)- \varphi\left(t_i\right) \right\vert &= \sum_{i=0}^{k}\left\vert \sum_{n=1}^{\infty}\varphi_{n}(t_{i+1}) - \sum_{n=1}^{\infty}\varphi_n(t_i) \right\vert\\
                                                                                              &\leq \sum_{n=1}^{\infty}\left(\sum_{i=0}^{k}\left\vert \varphi_n(t_{i+1}) - \varphi_n(t_i) \right\vert\right)\\
                                                                                              &\leq \sum_{n=1}^{\infty}\norm{\varphi_n}_{\text{BV}}.
  \end{align*}
  Thus, $\varphi \in \text{BV}\left([0,1]\right)$. Additionally,
  \begin{align*}
    \sum_{i=0}^{k}\left\vert \left(\varphi - \sum_{n=1}^{N}\varphi_n\right)\left(t_{i+1}\right) - \left(\varphi - \sum_{n=1}^{N}\varphi_n\right)\left(t_i\right) \right\vert &= \sum_{i=0}^{k}\left\vert \sum_{n=N+1}^{\infty}\varphi_n\left(t_{i+1}\right) - \sum_{n=N+1}^{\infty}\varphi_n\left(t_{i}\right) \right\vert \\
                                                                                                                                                                                                 &\leq \sum_{i=0}^{k}\sum_{n=N+1}^{\infty}\left\vert \varphi_{n}\left(t_{i+1}\right)-\varphi_n\left(t_i\right) \right\vert\\
                                                                                                &\leq \sum_{n=N+1}^{\infty}\norm{\varphi_n}_{\text{BV}},
  \end{align*}
  meaning $\displaystyle\varphi = \sum_{n=1}^{\infty}\varphi_n$ in the BV norm.
\end{proof}
\begin{theorem}[Riesz]
  Let $\hat{\varphi}(f) = \int_{0}^{1}f\:d\varphi$. Then, $\varphi \rightarrow \hat{\varphi}$ is an isometric isomorphism between $\left(C\left([0,1]\right)\right)^{\ast}$ and $\text{BV}\left([0,1]\right)$.
\end{theorem}
\begin{proof}
  We must show that the map $\varphi\mapsto \hat\varphi$ is an isometric isomorphism.\newline

  We can see that, to start, $\hat\varphi \in \left(C\left([0,1]\right)\right)^{\ast}$, with $\norm{\hat{\varphi}} \leq \norm{\varphi}_{\text{BV}}$.\newline

  We must now show that for $L\in \left(C\left([0,1]\right)\right)^{\ast}$, there exists $\psi \in \text{BV}\left([0,1]\right)$ such that $\hat{\psi} = L$, $\norm{\hat{\psi}}_{\text{BV}} \leq \norm{L}$, and $\psi$ is unique.\newline

  Let $B([0,1])$ be the space of all \textit{bounded} functions on $[0,1]$. It is readily apparent that $C([0,1])\subseteq B([0,1])$,\footnote{Extreme Value Theorem} and we can see $B([0,1])$ is a Banach space with pointwise addition and scalar multiplication under the norm $\norm{f}_{u} = \sup_{t\in [0,1]}\left\vert f(t) \right\vert$.\footnote{Obviously $B([0,1])$ is a normed vector space. For a Cauchy sequence of functions $(f_n)_n\in B([0,1])$, completeness has pointwise convergence to $f$. Boundedness and convergence follows from the properties of the supremum.} For $E\subseteq [0,1]$, define $I_{E}$ to be the indicator function on $E$. The indicator function is always bounded.\footnote{I am using $I_E$ instead of $\mathbb{1}_{E}$ since it's easier for me to type that faster.}\newline

  Since $L$ is a bounded linear functional on $C([0,1])$, the Hahn--Banach continuous extension theorem allows us to create a (not necessarily unique) bounded linear functional $L'$ on $B([0,1])$ with $\norm{L'} = \norm{L}$.\newline

  In particular, we can choose $L'$ such that $L'\left(I_{\set{0}}\right) = 0$, by extending $L$ to the linear span of $I_{\set{0}}$ and $C([0,1])$:
  \begin{align*}
    \left\vert L'\left(f + \lambda I_{\set{0}}\right) \right\vert &= \left\vert L(f) \right\vert\\
                                                                  &\leq \norm{L}\norm{f}_{\infty}\\
                                                                  &\leq \norm{L}\norm{f + \lambda I_{\set{0}}}_{u}
  \end{align*}
  for all $f\in C([0,1])$ and $\lambda \in \C$.\newline

  FOr $0 < t \leq 1$, let $\varphi(t) = L\left(I_{(t,t+1]}\right)$, with $\varphi(0) = 0$. We aim to show that $\varphi \in \text{BV}\left([0,1]\right)$ and $\norm{\varphi}_{\text{BV}} \leq \norm{L}$.\newline

  Select a partition $0 = t_0 < t_1 < \cdots < t_n < t_{n+1} = 1$, and set
  \begin{align*}
    \lambda_k &= \frac{\varphi\left(t_{k+1}\right) - \varphi\left(t_{k}\right)}{\left\vert \varphi\left(t_{k+1}\right) - \varphi\left(t_k\right) \right\vert}
  \end{align*}
  for $\varphi\left(t_{k+1}\right) \neq \varphi\left(t_k\right)$, and $\lambda_k = 0$ otherwise. Then,
  \begin{align*}
    f &= \sum_{k=0}^{n}\lambda_k I_{\left(t_{k},t_{k+1}\right]} \in B\left([0,1]\right)
  \end{align*}
  with $\norm{f}_{u} \leq 1$, and
  \begin{align*}
    \sum_{k=0}^{n}\left\vert \varphi\left(t_{k+1}\right) - \varphi\left(t_{k}\right) \right\vert &= \sum_{k=0}^{n}\lambda_k\left(\varphi\left(t_{k+1} - t_k\right)\right)\\
       &= \sum_{k=0}^{n}L'\left(I_{\left(t_k - t_{k+1}\right]}\right)\\
           &= L'(f)\\
                             &\leq \norm{L'} = \norm{L}.
  \end{align*}
  Thus, $\norm{\varphi}_{\text{BV}} \leq \norm{L}$.\newline

  Now, we need to show that $L(g) = \int_{0}^{1} g\:d\varphi$ for every $g\in C\left([0,1]\right)$.\newline

  Let $g \in C\left([0,1]\right)$. For $\ve > 0$, set $0 = t_0 < t_1 < \cdots < t_{n} < t_{n+1} = 1$ a partition such that
  \begin{align*}
    \left\vert g(s) - g(s') \right\vert &< \frac{\ve}{2\norm{L'}}
  \end{align*}
  for every $s,s'\in \left(t_{k},t_{k+1}\right]$, and
  \begin{align*}
    \left\vert \int_{0}^{1} g\:d\varphi - \sum_{k=0}^{n}g(t_k)\left(\varphi\left(t_{k+1}\right)-\varphi\left(t_{k}\right)\right) \right\vert < \frac{\ve}{2}.
  \end{align*}
  Thus, for $f = \sum_{k=0}^{n}g(t_k)I_{\left(t_{k},t_{k+1}\right]} + g(0)I_{\set{0}}$, we have
  \begin{align*}
    \left\vert L(g) - \int_{0}^{1} g\:d\varphi \right\vert &\leq \left\vert L(g) - L'(f) \right\vert + \left\vert L'(f) - \int_{0}^{1} g\:d\varphi \right\vert\\
                                                           &\leq \norm{L'}\norm{g - f}_{u} + \left\vert \sum_{k=0}^{n}g(t_k)\left(\varphi\left(t_{k+1}\right)- \varphi\left(t_k\right)\right) - \int_{0}^{1} g\:d\varphi \right\vert\\
                                                           &\leq \frac{\ve}{2} + \frac{\ve}{2}\\
                                                           &= \ve.
  \end{align*}
  Thus, $L(g) = \int_{0}^{1} g\:d\varphi$.\newline

  We obtain $\psi \in \text{BV}\left([0,1]\right)$ with $\norm{\psi}_{\text{BV}} \leq \norm{\varphi}_{\text{BV}}\leq \norm{L}$ (see function limits), and
  \begin{align*}
    \hat{\psi}(g) &= \int_{0}^{1} g\:d\psi\\
                  &= \int_{0}^{1} g\:d\varphi\\
                  &= L(g).
  \end{align*}
  Now, we must show that the mapping $\varphi \mapsto \hat\varphi$ is injective.\newline

  Let $\varphi \in \text{BV}\left([0,1]\right)$. Fix $0 < t_0 \leq 1$, and let $f_n$ be a sequence of functions in $C\left([0,1]\right)$ defined by
  \begin{align*}
    f_n\left(t\right) &= \begin{cases}
      1 & 0 \leq t \leq \frac{n-1}{n}t_0\\
      n\left(1 - \frac{t}{t_0}\right) & \frac{n-1}{n}t_0 < t \leq t_0\\
      0 & t_0 < t \leq 1
    \end{cases}.
  \end{align*}
  The function $I_{\left(0,t_0\right]} - f_n$ is zero outside the open interval $\left(\frac{n-1}{n}t_0,t_0\right)$. If we define
  \begin{align*}
    \varphi_n(t) &= \begin{cases}
      \varphi\left(\frac{n-1}{n}t_0\right) & 0 \leq t \leq \frac{n-1}{n}t_0\\
      \varphi(t) & \frac{n-1}{n}t_0 < t \leq t_0\\
      \varphi\left(t_0\right) & t_0 < t \leq 1
    \end{cases},
  \end{align*}
  then
  \begin{align*}
    \left\vert \int_{0}^{1} \left(I_{\left(0,t_0\right]} - f_n\right)\:d\varphi \right\vert &= \left\vert \int_{0}^{1} \left(I_{\left(0,t_0\right]} - f_n\right)\:d\varphi_n \right\vert\\
                                                                                            &\leq \norm{\varphi_n}_{\text{BV}}.
  \end{align*}
  We claim that $\lim_{n\rightarrow\infty}\norm{\varphi}_{\text{BV}} = 0$.\newline

  Since $\varphi$ is left continuous at $t_0$, there exists $\delta > 0$ such that $0 < t_0 - t < \delta$ implies $\left\vert \varphi(t_0 - t) \right\vert < \frac{\ve}{2}$. Let $0 = t_0 < t_1 < \cdots < t_{k+1} = 1$ be a partition of $[0,1]$, where
  \begin{align*}
    \left\vert\norm{\varphi}_[\text{BV}] - \left(\sum_{i=0}^{k}\left\vert \varphi(t_{i+1}) - \varphi(t_i) \right\vert\right)\right\vert < \frac{\ve}{2}.
  \end{align*}
  Let $t_0 = t_{i_0}$ for some $i_0$, where $t_{i_0} - t_{i_0 - 1} < \delta$. Then,
  \begin{align*}
    \left\vert \varphi\left(t_{i_0}\right) - \varphi\left(t_{i_0 - 1}\right) \right\vert < \frac{\ve}{2},
  \end{align*}
  and $\text{Var}\left(\varphi\right)_{\left[t_{i_0 - 1} ,t_{i_0}\right]} < \ve$. Therefore,
  \begin{align*}
    \varphi\left(t_0\right) &= \int_{0}^{1} I_{\left(0,t_0\right]}\:d\varphi\\
                            &= \lim_{n\rightarrow\infty}\int_{0}^{1} f\:d\varphi,
  \end{align*}
  with $\hat{\varphi} = 0$ implying $\varphi = 0$. Thus, $\left(C\left([0,1]\right)\right)^{\ast} = \text{BV}\left([0,1]\right)$.
\end{proof}
\begin{example}[Conjugate Space of $C(X)$]
  If $X$ is any compact Hausdorff space, rather than merely $[0,1]$, it makes no sense to talk about bounded variation (since $X$ may not have an ordering on it), so to find $\left(C(X)\right)^{\ast}$ would require some extra work.\newline

  Every countably additive measure on $\mathcal{B}_{X}$ gives rise to a bounded linear functional on $C(X)$. Using the Hahn--Banach continuous extension theorem, we can extend this to the Banach space of bounded Borel functions, and obtain a Borel measure by evaluating the extended linear functional on the indicator functions of Borel subsets of $X$.\newline

  If we restrict our attention to regular measures\footnote{Inner regular means the measure of a set can be approximated by compact subsets, outer regular means the measure of a set can be approximated by open supersets, and regular means both inner and outer regular.}, the extended functional \textit{is} unique, and we can identify $\left(C(X)\right)^{\ast}$ to be $M(X)$, which is the set of complex regular Borel measures on $X$.\newline

  This result is known as the Riesz--Markov--Kakutani Representation Theorem.
\end{example}
\begin{example}[Quotient Spaces of Banach Spaces]
  Let $\mathcal{X}$ be a Banach space, and $\mathcal{M}$ be a closed subspace of $\mathcal{X}$. We will try to find a norm on $\mathcal{X}/\mathcal{M}$.\newline

  The space $\mathcal{X}/\mathcal{M}$ is the set of equivalence classes of $f\in \mathcal{X}$ where $[f] = \set{f + g\mid g\in \mathcal{M}}$. The norm can be defined by
  \begin{align*}
    \norm{[f]} &= \inf_{g\in \mathcal{M}}\norm{f-g}.
  \end{align*}
  If $\norm{[f]} = 0$, then there is a sequence $g_n$ such that $\lim_{n\rightarrow\infty}\norm{f - g_n} = 0$,\footnote{I am using $\norm{[f]} = \inf_{g\in \mathcal{M}}\norm{f-g}$ instead since that is what my professor uses.} meaning $g_n \rightarrow f$; since $\mathcal{M}$ is closed, this implies that $[f] = [0]$. In the converse direction, if $[f] = [0]$, then $0\leq \norm{[f]} \leq \norm{f-f} = 0$. Thus, $\norm{[f]}$ is positive definite.\newline

  To show homogeneity, let $f\in \mathcal{X}$ and $\lambda \in \C$. Then,
  \begin{align*}
    \norm{\lambda[f]} &= \inf_{g\in \mathcal{M}}\norm{\lambda f - g}\\
                      &= \inf_{h\in \mathcal{M}}\norm{\lambda\left(f - h\right)}\\
                      &= |\lambda|\inf_{h\in \mathcal{M}}\norm{f - h}\\
                      &= |\lambda|\norm{[f]}.
  \end{align*}
  Finally, to show the triangle inequality, let $f_1,f_2\in \mathcal{X}$. Then,
  \begin{align*}
    \norm{[f_1] + [f_2]} &= \norm{[f_1 + f_2]}\\
                         &= \inf_{g\in \mathcal{M}}\norm{\left(f_1 + f_2\right) - g}\\
                         &= \inf_{g_1,g_2\in \mathcal{M}}\norm{(f_1 - g_1) + (f_2 - g_2)}\\
                         &\leq \inf_{g_1\in \mathcal{M}}\norm{f_1 - g_1} + \inf_{g_2\in \mathcal{M}}\norm{f_2 - g_2}\\
                         &= \norm{[f_1]} + \norm{[f_2]}.
  \end{align*}
  Finally, to show completeness, we let $\set{[f_n]}_{n=1}^{\infty}$ be a Cauchy sequence in $\mathcal{X}/\mathcal{M}$. Then, there exists a subsequence $\set{\left[f_{n_k}\right]}_{k=1}^{\infty}$ such that $\norm{\left[f_{n_{k+1}}\right] - \left[f_{n_k}\right]} < \frac{1}{2^k}$.\newline

  Select $h_k \in \left[f_{n_{k+1}} - f_{n_k}\right]$ such that $\norm{h_k} < \frac{1}{2^k}$. Then, $\sum_{k=1}^{\infty}\norm{h_k} < 1 < \infty$, meaning $\sum_{k=1}^{\infty}h_k = h$ for some $h$.\newline

  Since
  \begin{align*}
    \left[f_{n_k} - f_{n_1}\right] &= \sum_{i=1}^{k-1}\left[f_{n_{i+1}} - f_{n_i}\right]\\
                                   &= \sum_{i=1}^{k-1}\left[h_i\right],
  \end{align*}
  we must have $\lim_{k\rightarrow\infty}\left[f_{n_k} - f_{n_1}\right] = [h]$, meaning $\lim_{k\rightarrow\infty}\left[f_{n_k}\right] = \left[h + f_{n_1}\right]$.
\end{example}
We can see that there is a natural (projection) map $\pi: \mathcal{X}\rightarrow \mathcal{X}/\mathcal{M}$, defined by $\pi(f) = [f]$. This is a contraction and a surjective (which we will later see to be the same as open) map.
\begin{definition}[Bounded Linear Transformation]
Let $\mathcal{X},\mathcal{Y}$ be Banach spaces. The linear transformation $T: \mathcal{X}\rightarrow \mathcal{Y}$ is bounded if
\begin{align*}
  \norm{T}_{\text{op}} &= \sup_{\norm{f} = 1}\norm{T(f)}\\
                       &< \infty
\end{align*}
The set of all bounded linear transformations from $\mathcal{X}$ to $\mathcal{Y}$ is denoted $\mathcal{L}\left(\mathcal{X},\mathcal{Y}\right)$. We have proven earlier that a linear transformation is bounded if and only if it is continuous.\footnote{This holds in all normed vector spaces, not just Banach spaces.}
\end{definition}
\begin{proposition}[Properties of $\mathcal{L}\left(\mathcal{X},\mathcal{Y}\right)$]
  The space $\mathcal{L}\left(\mathcal{X},\mathcal{Y}\right)$ is a Banach space.
\end{proposition}
\begin{proof}
  It is readily apparent that $\mathcal{L}\left(\mathcal{X},\mathcal{Y}\right)$ is a normed vector space under pointwise addition and scalar multiplication. All we need to show now is completeness. \newline

  Let $\left(T_n\right)_n$ be a Cauchy sequence of elements of $\mathcal{L}\left(\mathcal{X},\mathcal{Y}\right)$. Then, for $\ve > 0$, there exists $N$ such that for $m,n > N$,
  \begin{align*}
    \norm{T_m - T_n}_{\text{op}} < \ve.
  \end{align*}
  This means that for any $f\in \mathcal{X}$, there exists $N_f$ such that for $m,n > N_f$,
  \begin{align*}
    \norm{\left(T_m - T_n\right)(f)} &\leq \norm{f}\norm{T_m-T_n}_{\text{op}}\\
                                     &< \frac{\ve}{\norm{f}} \norm{f}\\
                                     &= \ve.
  \end{align*}
  Since for each $f$, $\left(T_n(f)\right)_n$ is Cauchy, and $\mathcal{Y}$ is complete, we define $T$ to be the pointwise limit of $\left(T_n\right)_n$.\newline

  Thus, since
  \begin{align*}
    \lim_{m\rightarrow\infty}\norm{T_m - T_n}_{\text{op}} &= \norm{T - T_n}_{\text{op}}\\
                                                          &< \ve,
  \end{align*}
  we have that $\mathcal{L}\left(\mathcal{X},\mathcal{Y}\right)$ is complete.
\end{proof}
\begin{theorem}[Open Mapping]
  Let $\mathcal{X}$, $\mathcal{Y}$ be Banach spaces, and let $T\in \mathcal{L}\left(\mathcal{X},\mathcal{Y}\right)$ be surjective. Then, $T$ is an open map.
\end{theorem}
\begin{note}
  I don't like order that Douglas's book introduces the Open Mapping/Bounded Inverse/Uniform Boundedness principle as well as the proofs, so I'm going to be drawing the following proofs mostly from Stein and Shakarchi's Functional Analysis text.
\end{note}
\begin{proof}
  We see
  \begin{align*}
    \mathcal{X} &= \bigcup_{n=1}^{\infty}U_{\mathcal{X}}(0,n),
  \end{align*}
  Since $T$ is surjective, we have
  \begin{align*}
    \mathcal{Y} &= \bigcup_{n=1}^{\infty}T\left(U_{\mathcal{X}}(0,n)\right).
  \end{align*}
  Since $Y$ is complete, the Baire Category Theorem states that there must be at least one value of $n$ such that $\overline{T\left(U_{\mathcal{X}}(0,n)\right)}^{\circ}$ is nonempty. Since $T$ is linear, in particular we can see that $\overline{T\left(U_{\mathcal{X}}(0,1)\right)}$ has a nonempty interior.\newline

  We let $U_{\mathcal{Y}}\left(y_0,\ve\right)\subseteq \overline{T\left(U_{\mathcal{X}}(0,1)\right)}$. By the definition of closure, we may select $y_1 = Tx_1$ for $x_1 \in T\left(U_{\mathcal{X}}(0,1)\right)$ such that $\norm{y_1 - y_0} < \frac{\ve}{2}$. \newline

  Inductively, we can select $y_2 = Tx_2$ for $x_2 \in T\left(U_{\mathcal{X}}(0,1/2)\right)$ such that $\norm{y_0 - y_1 - y_2} < \frac{\ve}{4}$, and so on, selecting $x_n\in T\left(U_{\mathcal{X}}\left(0,\frac{1}{2^{n-1}}\right)\right)$ such that $\norm{y_0 - \sum_{j=1}^{n}Tx_j} < \frac{\ve}{2^{n}}$.\newline

  Since $\norm{x_j} < \frac{1}{2^{j-1}}$ for $j\in \N$, it is clear that $\sum_{j=1}^{\infty}\norm{x_j}$ converges --- thus, since $X$ is a Banach space, there exists $x$ such that $x = \sum_{j=1}^{n}x_j$. Moreover, since $\norm{y_0 - \sum_{j=1}^{n}Tx_j} < \frac{\ve}{2^n}$, and $T$ is continuous, the limit of $\set{x_j}_{j=1}^{n}$ must be such that $T(x) = y_0$.\newline

  Therefore, we must have that $U_{\mathcal{Y}}\left(0,\frac{1}{2}\right)\subseteq T\left(U_{\mathcal{X}}\left(0,1\right)\right)$.
\end{proof}
\begin{theorem}[Bounded Inverse]
  Let $T: \mathcal{X}\rightarrow \mathcal{Y}$ be a bounded bijective linear transformation. Then, $T^{-1}:\mathcal{Y}\rightarrow \mathcal{X}$ is also bounded.
\end{theorem}
\begin{proof}
  Since $T$ is bijective, $T$ is an open map, meaning $T^{-1}$ must be continuous.
\end{proof}
\begin{theorem}[Uniform Boundedness Principle]
  Let $\mathcal{L}$ be a collection of continuous linear functionals on a Banach space $\mathcal{X}$. Then, if $\sup_{\varphi \in \mathcal{L}}\left\vert \varphi(f) \right\vert < \infty$ for all $f$ in a residual subset $A\subseteq \mathcal{X}$, then $\sup_{\varphi \in \mathcal{L}}\norm{\varphi} < \infty$.
\end{theorem}
\begin{proof}
  Suppose $\sup_{\varphi \in \mathcal{L}}\left\vert \varphi(f) \right\vert < \infty$ for all $f\in A$, where $A$ is residual. For every $M$, define $A_{M,\varphi} = \set{f\in \mathcal{X}\mid \left\vert \varphi(f) \right\vert\leq M}$. Each of $A_{M,\varphi}$ is closed since $\varphi$ is continuous. Define $A_{M} = \bigcap_{\varphi \in \mathcal{L}}A_{M,\varphi}$; each $A_{M}$ is closed.\newline

  We can see that
  \begin{align*}
    A &= \bigcup_{M=1}^{\infty}\bigcap_{\varphi \in \mathcal{L}}A_{M,\varphi}.
  \end{align*}
  Since $A$ is residual, there must be some $M_0$ such that $A_{M_0}$ has nonempty interior, so there exists $f_0\in \mathcal{X}$ and $r > 0$ such that $U_{\mathcal{X}}\left(f_0,r\right)\subseteq A_{M_0}$.\newline

  Thus, for every $\varphi \in \mathcal{L}$, we have $\left\vert \varphi(f) \right\vert \leq M_0$ for all $f$ where $\norm{f-f_0} < r$. Thus, for all $\norm{g} < r$ and $\varphi \in \mathcal{L}$, we have
  \begin{align*}
    \left\vert \varphi(g) \right\vert &\leq \left\vert \varphi(g + f_0) \right\vert + \left\vert \varphi(-f_0) \right\vert\\
                                      &\leq 2M_0,
  \end{align*}
  meaning $\norm{\varphi} < \infty$ for all $\varphi \in \mathcal{L}$.
\end{proof}
\begin{definition}[Lebesgue Spaces]
  Let $\mu$ be a probability measure on a $\sigma$-algebra $\Omega$ of the subsets of a set $X$.\newline

  We let $\mathcal{L}^{1}$ denote the vector space of all integrable complex-valued functions, with $\mathcal{N}\subseteq \mathcal{L}^{1}$ denoting the subspace of all $f\in \mathcal{L}^{1}$ where
  \begin{align*}
    \int_{X}^{} \left\vert f \right\vert\:d\mu = 0.
  \end{align*}
  Then, $L^{1} = \mathcal{L}^{1}/\mathcal{N}$ is the space of equivalence classes $[f]$, where $\norm{[f]}_{1} = \int_{X}|f|d\mu$.\newline

  For each $1 \leq p < \infty$, we set $\mathcal{L}^{p}$ to be the functions in $\mathcal{L}^{1}$ where $\int_{X}|f|^{p}\:d\mu < \infty$; then, defining $\mathcal{N}^{p} = \mathcal{N}\cap \mathcal{L}^{p}$, the quotient space $L^{p} = \mathcal{L}^{p}/\mathcal{N}^{p}$ is the space of equivalence classes $[f]$ with norm
  \begin{align*}
    \norm{[f]}_{p} &= \left(\int_{X}^{} |f|^{p}\:d\mu\right)^{1/p}.
  \end{align*}
  To construct $L^{\infty}$, we start by constructing $\mathcal{L}^{\infty}$, which is the set of all essentially bounded functions, where $\mu\set{x\in X\mid |f(x)| > M} = 0$ for some $M$; we say $\norm{f}_{\infty}$ is the infimum of all such $M$. Equivalently, $\norm{f}_{\infty} = \esssup |f|$. The set $\mathcal{N}^{\infty} = \mathcal{N}\cap \mathcal{L}^{\infty}$, and $L^{\infty} = \mathcal{L}^{\infty}/\mathcal{N}^{\infty}$ is the set of the equivalence classes $[f]$ where $\norm{[f]}_{\infty} = \norm{f}_{\infty} < \infty$ for $f$ a representative of $[f]$.\newline

  We can see that all the $L^{p}$ spaces are normed vector spaces; to show completeness will take more work, but we will show completeness for both $L^{1}$ and $L^{\infty}$.
\end{definition}
\begin{theorem}[Completeness of $L^{1}$]
  The space $L^{1}$ is complete with respect to the norm $\norm{[f]}_1 = \int_{X}|f|\:d\mu$.
\end{theorem}
\begin{proof}
  Let $\set{\left[f_n\right]}_{n=1}^{\infty}$ be a sequence in $L^{1}$ where $\sum_{n=1}^{\infty}\norm{[f_n]}_1 \leq M < \infty$.\newline

   Select representatives $f_n$ from each equivalence class. The sequence $\set{\sum_{n=1}^{N}f_n}_{N=1}^{\infty}$ is increasing for every $x\in X$ and non-negative, meaning
   \begin{align*}
     \int_{X}\left(\sum_{n=1}^{N}|f_n|\right)\:d\mu &= \sum_{n=1}^{N}\norm{[f_n]}_{1}\\
                                                    &\leq M,
   \end{align*}
   so by the dominated convergence theorem\footnote{The book states that they use Fatou's Lemma but I couldn't really understand where it comes into use so I decided to use the dominated convergence theorem and provide an explanation.} (with $g = M$, whose integral is finite because $\mu$ is a probability measure), we have that $\set{\sum_{n=1}^{N}|f_n|}_{N=1}^{\infty}$ is integrable and converges $\mu$-almost everywhere to $[k]\in \mathcal{L}^{1}$.\newline

   Finally,
   \begin{align*}
     \norm{[k] - \int_{n=1}^{N}}_{1} &= \int_{X}^{} \left\vert \sum_{n=1}^{\infty}f_n - \sum_{n=1}^{N}f_n \right\vert\:d\mu\\
                                     &\leq \sum_{n=N+1}^{\infty}\int_{X}^{} \left\vert f_n \right\vert\:d\mu\\
                                     &\leq \sum_{n=N+1}^{\infty}\norm{[f_n]}_1.
   \end{align*}
   Thus, $\sum_{n=1}^{\infty}[f_n] = [k]$.
\end{proof}
\begin{theorem}[Completeness of $L^{\infty}$]
  The space $L^{\infty}$ is complete with respect to the norm $\norm{[f]}_{\infty} = \esssup |f|$.\footnote{I had a proof of this in my Real Analysis II notes with Cauchy sequences. Here, I'll be going off the book's proof, which uses the absolute convergence determination criterion for Banach spaces.}
\end{theorem}
\begin{proof}
  Let $\set{[f_n]}_{n=1}^{\infty}$ be a sequence of elements of $L^{\infty}$ with $\sum_{n=1}^{\infty}\norm{[f_n]}_{\infty}\leq M < \infty$. Choose representatives $f_n$ from $[f_n]$, such that $\left\vert f_n \right\vert$ is bounded everywhere by $\norm{[f_n]}_{\infty}$.\newline

  For $x\in X$, we have
  \begin{align*}
    \sum_{n=1}^{\infty}\left\vert f_n(x) \right\vert &\leq \sum_{n=1}^{\infty}\norm{[f_n]}\\
                                                     &\leq M.
  \end{align*}
  Thus, by dominated convergence, the sequence $\sum_{n=1}^{\infty}f_n = \lim_{N\rightarrow\infty}\sum_{n=1}^{N}f_n$ converges to a measurable bounded function $h$, where
  \begin{align*}
    \left\vert h(x) \right\vert &= \left\vert \sum_{n=1}^{\infty}f_n(x) \right\vert\\
                                &\leq \sum_{n=1}^{\infty}\left\vert f_n(x) \right\vert\\
                                &\leq M.
  \end{align*}
  Thus, $h\in \mathcal{L}^{\infty}$. Finally, we can see that
  \begin{align*}
    \left\vert [h] - \sum_{n=1}^{N} \right\vert &= \left\vert \sum_{n=1}^{\infty}f_n - \sum_{n=1}^{N} f_n \right\vert\\
                                                &\leq \sum_{n=N + 1}^{\infty}\left\vert f_n\right\vert\\
                                                &\leq \sum_{n=N+1}^{\infty}\norm{f_n}_{\infty},
  \end{align*}
  meaning $\norm{[h] = \sum_{n=1}^{N}f_n}_{\infty}$ converges to $0$.
\end{proof}
The traditional abuse of notation for elements of $L^{p}$ spaces is to refer to $f\in L^{1}$ to mean the equivalence class of $\mu$-almost everywhere functions equal to $f\in \mathcal{L}^{1}$.\newline

Now, we turn our attention to the dual of $L^{1}$, $\left(L^{1}\right)^{\ast}$.
\begin{theorem}[Dual of $L^{1}$]
  Let $\hat{\varphi}$ be the linear functional defined by
  \begin{align*}
    \hat{\varphi}(f) &= \int_{X}^{} f\varphi\:d\mu
  \end{align*}
  for $f\in L^{1}$ and $\varphi \in L^{\infty}$. Then, the map $\varphi \mapsto \hat\varphi$ is an isometric isomorphism of $L^{\infty}$ onto $\left(L^{1}\right)^{\ast}$.
\end{theorem}
\begin{proof}
  If $\varphi \in L^{\infty}$, then for $f\in L^{1}$, it is the case that $\left\vert (\varphi f)(x) \right\vert \leq \norm{\varphi}_{\infty}\left\vert f(x) \right\vert$ almost everywhere. Thus, $\varphi f$ is integrable, meaning $\hat{\varphi}$ is well-defined and linear, with
  \begin{align*}
    \left\vert \hat{\varphi}(f) \right\vert &= \left\vert \int_{X}^{} f\varphi\:d\mu \right\vert\\
                                            &\leq \norm{\varphi}_{\infty}\int_{X}^{} |f|\:d\mu\\
                                            &\leq \norm{\varphi}_{\infty}\norm{f}_{1},
  \end{align*}
  meaning $\hat{\varphi}\in \left(L^{1}\right)^{\ast}$ and $\norm{\hat{\varphi}}\leq \norm{\varphi}_{\infty}$.\newline

  Let $L\in \left(L^{1}\right)^{\ast}$. For $E$ a measurable subset of $X$, $I_{E}$, the indicator function on $E$, is $L^{1}$, with $\norm{I_{E}}_{1} = \mu(E)$.\newline

  If we set $\lambda(E) = L(I_E)$, we can see that $\lambda$ is a finitely additive complex-valued measure, with $\left\vert \lambda(E) \right\vert\leq \mu(E) \norm{L}$. Moreover, for $\set{E_n}_{n=1}^{\infty}$ a nested sequence of measurable sets with $\bigcap_{n=1}^{\infty} E_n = \emptyset$, we have
  \begin{align*}
    \left\vert \lim_{n\rightarrow\infty}\lambda\left(E_n\right) \right\vert &\leq \lim_{n\rightarrow\infty}\left\vert \lambda\left(E_n\right) \right\vert\\
                                                                            &\leq \norm{L}\lim_{n\rightarrow\infty}\mu\left(E_n\right)\\
                                                                            &= 0.
  \end{align*}
  Thus, $\lambda$ is dominated by $\mu$, meaning that by the Radon--Nikodym theorem,\footnote{Someday I will actually learn this theorem for real.} there exists an integrable function $\varphi$ on $X$ such that $\lambda(E) = \int_{X}I_E\varphi \: d\mu$ for all measurable sets $E$. What we need to show now is that $\varphi$ is essentially bounded, and $L(f) = \int_{X}f\varphi\:d\mu$ for all $f\in L^1$.\newline

  Set
  \begin{align*}
    E_N &= \set{x\in X\,\bigg|\,\norm{L} + \frac{1}{N} \leq \left\vert \varphi(x) \right\vert \leq N}.
  \end{align*}
  Then, $E_N$ is measurable, and $I_{E_N} \varphi$ is bounded.\newline

  If $f = \sum_{i=1}^{k}c_iI_{E_i}$, then we can see that $L(f) = \int_{X}f\varphi\:d\mu$. We can also see that for any $f\in L_1$ with $\text{supp}(f) = E_N$, $L(f) = \int_{X}f\varphi\:d\mu$.\newline

  Let $g = \frac{\overline{\varphi(x)}}{\left\vert \varphi(x) \right\vert}$ if $x\in E_N$ and $\varphi(x) \neq 0$; otherwise, $g = 0$. Then, $g\in L^1$ with $\text{supp}(g) = E_n$ and $\norm{g}_{1} = \mu(E_N)$. Thus, we have
  \begin{align*}
    \mu(E_N)\norm{L} &\geq \left\vert L(g) \right\vert\\
                     &= \left\vert \int_{X}^{} g\varphi\:d\mu \right\vert\\
                     &= \int_{X}^{} \left\vert \varphi \right\vert I_{E_N}\:d\mu\\
                     &\geq \left(\norm{L} + \frac{1}{N}\right)\mu\left(E_N\right),
  \end{align*}
  meaning $\mu(E_N) = 0$. Thus, $\mu\left(\bigcup_{N=1}^{\infty}E_N\right) = 0$, meaning $\varphi$ is essentially bounded and $\norm{\varphi}_{\infty}\leq \norm{L}$.
\end{proof}
\begin{definition}[Hardy Spaces]
  Let $\mathbb{T}$ denote the unit circle in the complex plane, and $\mu$ the Lebesgue measure normalized such that $\mu\left(\mathbb{T}\right) = 1$. We define $L^{p}\left(\mathbb{T}\right)$ with respect to $\mu$ as the Lebesgue space on $\mathbb{T}$.\newline

  The Hardy space, $H^{p}$ is a closed subspace of $L^{p}\left(\mathbb{T}\right)$.\newline

  For $n\in \Z$, we define $\chi_n$ on $\mathbb{T}$ such that $\chi_n(z) = z^n$. We define
  \begin{align*}
    H^1 &= \set{f\in L^{1}\left(\mathbb{T}\right) \mid \frac{1}{2\pi}\int_{0}^{2\pi}f\chi_n\:dt = 0}.
  \end{align*}
  We can see that $H^1$ is a linear subspace, and is a kernel of a bounded linear functional on $L^1\left(\mathbb{T}\right)$, meaning it is closed.\newline

  For similar reasons,
  \begin{align*}
    H^{\infty} &= \set{\varphi \in L^{\infty}\left(\mathbb{T}\right) \mid \frac{1}{2\pi}\int_{0}^{2\pi}\varphi \chi_n\:dt = 0}
  \end{align*}
  is also a closed subspace of $L^{\infty}\left(\mathbb{T}\right)$. In particular, this is the kernel of the $w^{\ast}$-continuous function
  \begin{align*}
    \hat{\chi}_n\left(\varphi\right) &= \frac{1}{2\pi}\int_{0}^{2\pi} \varphi \chi_n\:dt,
  \end{align*}
  meaning $H^{\infty}$ is also $w^{\ast}$-closed.
\end{definition}
\section{Banach Algebras}%
Earlier, we showed that $C(X)$, where $X$ is a compact Hausdorff space, is a Banach space; additionally, every Banach space is isomorphic to some subspace of $C(X)$. We can also see that $C(X)$ is an algebra\footnote{Vector space with multiplication.} with multiplication continuous in the norm topology.
\begin{definition}[Multiplicative Linear Functional]
  A linear functional $\varphi: C(X) \rightarrow \C$ is multiplicative if $\varphi(fg) = \varphi(f)\varphi(g)$, meaning $\varphi(1) = 1$.\newline

  For each $x\in X$, we define $\varphi_x(f) = f(x)$.\newline

  The space of multiplicative linear functionals on $C(X)$ is denoted $M_{C(X)}$.
\end{definition}
\begin{proposition}
  Let $\psi: X\rightarrow M_{C(X)}$ be defined by $\psi(x) = \varphi_x$.\newline

  Then, $\psi$ is a homeomorphism from $X$ onto $M_{C(X)}$, where $M_{C(X)}$ is given the $w^{\ast}$-topology on $\left(C(X)\right)^{\ast}$.
\end{proposition}
\begin{proof}
  Let $\varphi \in M_{C(X)}$, and set
  \begin{align*}
    \mathfrak{R} &= \ker\varphi\\
                 &= \set{f\in C(X) \mid \varphi(f) = 0}.
  \end{align*}
  We show that there exists $x_0$ in $X$ such that $f(x_0) = 0$ for all $f\in \mathfrak{R}$.\newline

  If this were not the case, then for each $x\in X$, there would exist $f_x\in \mathfrak{R}$ such that $f_x(x) \neq 0$. Since $f_x$ is continuous, there exists a neighborhood $U_x$ of $x$ where $f_x\neq 0$. Since $X$ is compact, and $\set{U_x}_{x\in X}$ is an open cover of $X$, there exist $U_{x_1},\dots,U_{x_N}$ with $X = \bigcup_{n=1}^{N}U_{x_n}$.\newline

  If we set $g = \sum_{n=1}^{N}\overline{f_{x_n}}f_{x_n}$, then
  \begin{align*}
    \varphi(g) &= \sum_{n=1}^{n}\varphi\left(\overline{f_{x_n}}\right)\varphi\left(f_{x_n}\right)\\
               &= 0,
  \end{align*}
  implying $g\in \mathfrak{R}$. However, $g\neq 0$ on $C(X)$, meaning $g$ is invertible, implying $\varphi(1) = \varphi(g) \varphi(1/g) = 0$. Thus, there must exist $x_0\in X$ such that $f(x_0) = 0$.\newline

  If $f\in C(X)$, then $f - (1)(\varphi(f))$ is in $\mathfrak{R}$, since $\varphi\left(f - (1)\varphi(f)\right) = \varphi(f) - \varphi(f) = 0$, meaning $f(x_0) - \varphi(f) = 0$, and $\varphi = \varphi_{x_0}$.\newline

  Since each $\varphi \in M_{C(X)}$ is bounded, we can give $M_{C(X)}$ the subspace topology of the $w^{\ast}$-topology on $\left(C(X)\right)^{\ast}$.\newline

  Consider $\psi: X \rightarrow M_{C(X)}$. Since $X$ is compact and Haursdorff, it is normal, meaning that by Urysohn's lemma, there exists $f\in C(X)$ such that $f(x) \neq f(y)$, meaning
  \begin{align*}
    \psi(x)(f) &= \varphi_x(f)\\
               &= f(x)\\
               &\neq f(y)\\
               &= \varphi_y(f)\\
               &= \psi(y)(f),
  \end{align*}
  implying $\psi$ is injective.\newline

  To show $\psi$ is continuous, let $\set{x_{\alpha}}_{\alpha \in A}$ be a net in $X$ converging to $x$. Then, for every $f\in C(X)$, $\lim_{\alpha \in A}f\left(x_{\alpha}\right) = f(x)$, or $\lim_{\alpha \in A}\psi\left(x_{\alpha}\right)(f) = \psi(x)(f)$.\newline

  Thus, $\set{\psi\left(x_{\alpha}\right)}_{\alpha \in A}$ converges in the $w^{\ast}$-topology to $\psi(x)$, meaning $\psi$ is continuous.\newline

  Since $\psi$ is injective and continuous mapping a compact space onto a Hausdorff space, $\psi$ is a homeomorphism.
\end{proof}
\begin{definition}[Banach Algebra]
  A Banach algebra $\mathfrak{B}$ is an algebra over $\C$ with identity $e$ which has a norm that makes it a Banach space, where $\norm{e} = 1$ and $\norm{fg} \leq \norm{f}\norm{g}$.
\end{definition}
\begin{proposition}[Invertible Elements]
  If $f\in \mathfrak{B}$ with $\norm{e-f} < 1$, then $f$ is invertible and
  \begin{align*}
    \norm{f^{-1}}\leq \frac{1}{1-\norm{e-f}}.
  \end{align*}
\end{proposition}
\begin{proof}
  If we set $\eta = \norm{e-f} < 1$, then for $N\geq M$, we have
  \begin{align*}
    \norm{\sum_{n=0}^{N}\left(e-f\right)^{n} - \sum_{n=0}^{M}\left(e-f\right)^{n}} &= \norm{\sum_{n=M+1}^{N}\left(e-f\right)^{n}}\\
                                                                                   &\leq \sum_{n=M+1}^{N}\norm{e-f}^n\\
                                                                                   &= \sum_{n=M+1}^{N}\eta^{n}\\
                                                                                   &\leq \frac{\eta^{M+1}}{1-\eta},
  \end{align*}
  meaning the sequence of partial sums $\set{\sum_{n=0}^{N}(1-f)^n}_{N=0}^{\infty}$ is Cauchy.\newline

  If $g = \sum_{n=0}^{\infty}\left(e-f\right)^{n}$, then
  \begin{align*}
    fg &= \left(e-(e-f)\right)\left(\sum_{n=0}^{\infty}\left(e-f\right)^n\right)\\
       &= \lim_{N\rightarrow\infty}\left(\left(1-(e-f)\right)\sum_{n=0}^{N}\left(e-f\right)^{n}\right)\\
       &= \lim_{N\rightarrow\infty} \left(1-\left(e-f\right)^{N+1}\right)\\
       &= 1.
  \end{align*}
  Similarly, $gf = 1$, meaning $f$ is invertible with $f^{-1} = g$. We can also see
  \begin{align*}
    \norm{g} &= \lim_{N\rightarrow\infty}\norm{\sum_{n=0}^{N}\left(e-f\right)^n}\\
             &\leq \lim_{N\rightarrow\infty}\sum_{n=0}^{N}\norm{e-f}^n\\
             &= \frac{1}{1-\norm{e-f}}.
  \end{align*}
\end{proof}
\begin{definition}[Set of Invertible Elements]
  For a Banach algebra $\mathfrak{B}$, we denote the collection of invertible elements as $\mathcal{G}$, with $\mathcal{G}_{l}$ denoting the left-invertible elements that are not invertible, and $\mathcal{G}_{r}$ the collection of right-invertible elements that are not invertible.
\end{definition}
\begin{proposition}[Openness of Sets of Invertible Elements]
  For $\mathfrak{B}$ a Banach algebra, the sets $\mathcal{G}$, $\mathcal{G}_l$, and $\mathcal{G}_r$ are open.
\end{proposition}
\begin{proof}
  Let $f\in \mathcal{G}$. Then, if $\norm{f-g} \leq \frac{1}{\norm{f^{-1}}}$, then $1 > \norm{f^{-1}}\norm{f-g} \geq \norm{e-f^{-1}g}$, implying that $f^{-1}g \in \mathcal{G}$, and $g\in \mathcal{G}$, meaning $\mathcal{G}$ contains the open ball of radius $\frac{1}{\norm{f^{-1}}}$ about each element.\newline

  If $f\in \mathcal{G}_{l}$, then there exists $h\in \mathfrak{B}$ such that $hf = 1$; if $\norm{f-g} < \frac{1}{\norm{h}}$, then $1 > \norm{h}\norm{f-g} \geq \norm{1-hg}$, implying $hg$ is invertible and $g$ is left invertible. Thus, $\mathcal{G}_{l}$ has the open ball of radius $\frac{1}{\norm{h}}$ about every element of $f$, meaning $\mathcal{G}_{l}$ is open.\newline

  A similar argument holds for $\mathcal{G}_{r}$.
\end{proof}
\begin{corollary}[Topological Group of Invertible Elements]
  If $\mathfrak{B}$ is a Banach algebra, then $f\mapsto f^{-1}$ defined on $\mathcal{G}$ is continuous.
\end{corollary}
\begin{proof}
  If $f\in \mathcal{G}$, then $\norm{f-g} < \frac{1}{2}\norm{f^{-1}}$ implies $\norm{e-f^{-1}g} < \frac{1}{2}$. Thus,
  \begin{align*}
    \norm{g^{-1}} &\leq \norm{g^{-1}f}\norm{f^{-1}}\\
                  &= \norm{\left(f^{-1}g\right)^{-1}}\norm{f^{-1}}\\
                  &\leq 2\norm{f^{-1}}.
  \end{align*}
  Thus,
  \begin{align*}
    \norm{f^{-1}-g^{-1}} &= \norm{f^{-1}\left(f-g\right)g^{-1}}\\
                         &\leq 2\norm{f^{-1}}^2\norm{f-g},
  \end{align*}
  meaning $f\mapsto f^{-1}$ is Lipschitz.
\end{proof}
\begin{proposition}[Connected Component with Identity]
  Let $\mathfrak{B}$ be a Banach algebra, with $\mathcal{G}$ the group of invertible elements. Let $\mathcal{G}_0$ be the connected component in $\mathcal{G}$ that contains the identity.\newline

  Then, $\mathcal{G}_0$ is a clopen normal subgroup of $\mathcal{G}$, the cosets of $\mathcal{G}_0$ are the components of $\mathcal{G}$, and $\mathcal{G}/\mathcal{G}_0$ is a discrete group\footnote{Totally disconnected group.}
\end{proposition}
\begin{proof}
  Since $\mathcal{G}$ is an open subset of a locally connected space, its components are clopen subsets of $\mathcal{G}$.\newline

  If $f,g\in \mathcal{G}_0$, then $f\mathcal{G}_0$ is a connected subset of $\mathcal{G}$ which contains $fg$ and $f$, meaning $\mathcal{G}_0 \cup f\mathcal{G}_0$ is connected, and so contained in $\mathcal{G}_0$, so $fg\in \mathcal{G}_0$, and thus $\mathcal{G}_0$ is a semigroup.\newline

  Similarly, $f^{-1}\mathcal{G}_0\cup \mathcal{G}_0$ is connected, meaning it is contained in $\mathcal{G}_0$, so $\mathcal{G}_0$ is a subgroup of $\mathcal{G}$.\newline

  Finally, for $f\in \mathcal{G}$, then $f\mathcal{G}_0 f^{-1} = \mathcal{G}_0$, meaning $\mathcal{G}_0$ is normal.\newline

  Since $f\mathcal{G}_0$ is a clopen connected subset of $\mathcal{G}$ for every $f\in \mathcal{G}$, the cosets of $\mathcal{G}$ are components of $\mathcal{G}$.\newline

  Finally, $\mathcal{G}/\mathcal{G}_0$ is discrete, since $\mathcal{G}_0$ is open and closed in $\mathcal{G}$.\footnote{A result in abstract harmonic analysis holds that a quotient group over $G$ is discrete if and only if the normal subgroup is open in $G$.}
\end{proof}
\begin{definition}[Abstract Index Group]
  For $\mathfrak{B}$ a Banach algebra, the abstract index group for $\mathfrak{B}$, denoted $\Lambda_{\mathfrak{B}}$, is the quotient group $\mathcal{G}/\mathcal{G}_0$. The abstract index is the natural homomorphism $\gamma: \mathcal{G}\rightarrow \Lambda_{\mathfrak{B}}$.
\end{definition}
\begin{definition}[Exponential Map]
  Let $\mathfrak{B}$ be a Banach algebra. Then, the exponential map on $\mathfrak{B}$, denoted $\exp$, is defined by
  \begin{align*}
    \exp f &= \sum_{n=0}^{\infty}\frac{1}{n!}f^n.
  \end{align*}
\end{definition}
\begin{remark}
  The traditional properties of the exponential map, such as its absolute convergence, hold in all commutative Banach algebras, but do not necessarily hold in noncommutative Banach algebras.
\end{remark}
\begin{lemma}[Exponential Properties]
  For $f,g\in \mathfrak{B}$,
  \begin{align*}
    \exp(f+g) &= \exp(f) + \exp(g).
  \end{align*}
\end{lemma}
\begin{lemma}[Elements in Range of Exponential Map]
  If $f\in \mathfrak{B}$ is such that $\norm{e-f} < 1$, then $f\in \exp \mathfrak{B}$.
\end{lemma}
\begin{proof}
  Let $g = \sum_{n=1}^{\infty}-\frac{1}{n}(1-f)^n$. This series converges absolutely, and substituting into the expansion for $\exp g$, we find that
  \begin{align*}
    \exp g &= f.
  \end{align*}
\end{proof}
\begin{theorem}[Collection of Finite Products in $\exp\mathfrak{B}$]
  Let $\mathfrak{B}$ be a (not necessarily commutative) Banach algebra. Then, the collection of finite products of elements of $\exp \mathfrak{B}$ is $\mathcal{G}_0$.
\end{theorem}
\begin{proof}
  Let $f = \exp g$. Then, $f\exp(-g) = \exp(g-g) = 1 = \exp(-g)f$, meaning $f\in \mathcal{G}$.\newline

  The map $\varphi: [0,1]\rightarrow \exp \mathfrak{B}$ defined by $\varphi(\lambda) = \exp\left(\lambda g\right)$ is a continuous map such that $\varphi(0) = e$ and $\varphi(1) = f$, meaning $f\in \mathcal{G}_0$. Thus, $\exp \mathfrak{B}\subseteq \mathcal{G}_0$.\newline

  If $\mathcal{F}$ denotes the collection of finite products of elements of $\exp \mathfrak{B}$, then $\mathcal{F}$ is a subgroup contained in $\mathcal{G}_0$, meaning $\mathcal{F}$ is open. Finally, since each of the left cosets of $\mathcal{F}$ is open, it follows that $\mathcal{F}$ is clopen in $\mathcal{G}_0$, so $\mathcal{F} = \mathcal{G}_0$.
\end{proof}
\begin{corollary}[Collection of Finite Prodcuts of Commutative Banach Algebra]
  If $\mathfrak{B}$ is commutative, then $\exp \mathfrak{B} = \mathcal{G}_0$.
\end{corollary}
\begin{proof}
  If $\exp \mathfrak{B}$ is commutative, then $\exp \mathfrak{B}$ is a subgroup of $\mathcal{G}_0$.
\end{proof}
For a given Banach algebra $\mathfrak{B}$, the set of multiplicative linear functionals on $\mathfrak{B}$ is denoted $M_{\mathfrak{B}} = M$.\footnote{There was a small section here relating the abstract index group of $C(X)$ for a compact Hausdorff space $X$ to $\pi^1(X)$, which is the group of homotopy classes of continuous maps of $X$ to $\mathbb{T}$ (the circle group). I don't know any algebraic topology so I didn't really understand this part, and it doesn't seem to be particularly necessary outside of these facts.}
\begin{proposition}[Norm of Multiplicative Linear Functional]
  For $\mathfrak{B}$ a Banach algebra, if $\varphi \in M$, then $\norm{\varphi} = 1$.
\end{proposition}
\begin{proof}
  Let $\mathfrak{R} = \ker\varphi$. Since $\varphi(f - \varphi(f)e) = 0$, we can see that every element in $\mathfrak{B}$ can be written as $\lambda e + f$ for $\lambda \in \C$ and $f \in \mathfrak{R}$. Thus,
  \begin{align*}
    \norm{\varphi} &= \sup_{\norm{g} = 1}\left\vert \varphi(g) \right\vert\\
                   &= \sup_{\substack{f\in \mathfrak{R}\\\norm{\lambda + f} = 1}}\left\vert \varphi(\lambda + f) \right\vert\\
                   &= \sup_{\substack{f\in \mathfrak{R}\\\norm{\lambda + f} = 1}} \left\vert \varphi(\lambda) \right\vert\\
                   &= 1
  \end{align*}
\end{proof}
\begin{proposition}[Compactness of $M$ in $\mathfrak{B}^{\ast}$]
  If $\mathfrak{B}$ is a Banach algebra, then $M$ is a $w^{\ast}$-compact subset of $B_{\mathfrak{B}^{\ast}}$.
\end{proposition}
\begin{proof}
  Let $\set{\varphi_{\alpha}}_{\alpha \in A}$ be a net in $M$ that converges in the $w^{\ast}$-topology on $B_{\mathfrak{B}^{\ast}}$ to $\varphi \in B_{\mathfrak{B}^{\ast}}$.\newline

  All we need show is that $\varphi$ is multiplicative with $\varphi(e) = 1$. First, we see that
  \begin{align*}
    \varphi(e) &= \lim_{\alpha \in A}\varphi_{\alpha}(e)
               &= \lim_{\alpha \in A} 1\\
               &= 1.
  \end{align*}
  Further, for $f,g\in \mathfrak{B}$, we have
  \begin{align*}
    \varphi(fg) &= \lim_{\alpha \in A}\left(\varphi_{\alpha}(f)\varphi_{\alpha}(g)\right)\\
                &= \left(\lim_{\alpha \in A}\varphi_{\alpha}(f)\right)\left(\lim_{\alpha \in A}\varphi_{\alpha}(g)\right)\\
                &= \varphi(f)\varphi(g).
  \end{align*}
\end{proof}
Thus, $M$ is compact in the subspace $w^{\ast}$-topology. Recall that for every $f\in \mathfrak{B}$, there is a $w^{\ast}$-continuous function $\hat{f}: B_{\mathfrak{B}^{\ast}}\rightarrow \C$ given by $\hat{f}(\varphi) = \varphi(f)$.\newline

Since $M\subseteq B_{\mathfrak{B}^{\ast}}$, then $\hat{f}|_{M}$ is continuous.
\begin{definition}[Gelfand Transform]
  For $\mathfrak{B}$, if $M\neq \emptyset$, then the Gelfand transform $\Gamma: \mathfrak{B}\rightarrow C(M)$ is given by $\Gamma(f) = \hat{f}|_{M}$.
\end{definition}
\begin{proposition}[Properties of the Gelfand Transform]
  Let $\mathfrak{B}$ be a Banach algebra, and $\Gamma: \mathfrak{B}\rightarrow C(M)$ be the Gelfand transform on $\mathfrak{B}$. Then,
  \begin{enumerate}[(1)]
    \item $\Gamma$ is an algebra homomorphism;
    \item $\norm{\Gamma(f)}_{\infty} \leq \norm{f}$ for all $f\in \mathfrak{B}$.
  \end{enumerate}
\end{proposition}
\begin{proof}
  To show that $\Gamma$ is an algebra homomorphism, we show that for $f,g\in \mathfrak{B}$,
  \begin{align*}
    \Gamma(fg)(\varphi) &= \varphi(fg) \\
                        &= \varphi(f)\varphi(g)\\
                        &= \Gamma(f)(\varphi)\Gamma(g)(\varphi)\\
                        &= \left(\Gamma(f)\Gamma(g)\right)(\varphi).
  \end{align*}
  Additionally, for $f\in \mathfrak{B}$,
  \begin{align*}
    \norm{\Gamma(f)}_{\infty} &= \norm{\hat{f}|_{M}}_{\infty}\\
                              &\leq \norm{\hat{f}}_{\infty}\\
                              &= \norm{f}.
  \end{align*}
  Thus, $\Gamma$ is a contractive algebra homomorphism.
\end{proof}
\begin{remark}[Notes on the Gelfand Transform]
  Note that $\Gamma(fg - gf) = 0$, meaning that if $\mathfrak{B}$ is not commutative, then the subalgebra of $C(M)$ that is $\text{ran}\left(\Gamma\right)$ may not reflect the properties of $\mathfrak{B}$.\newline

  In the commutative case, though, $M$ is not only nonempty, but sufficiently large such that the invertibility of $f\in \mathfrak{B}$ is determined by the invertibility of $\Gamma(f)$ in $C(M)$.
\end{remark}
\begin{definition}[Spectrum of an Element]
  Let $f\in \mathfrak{B}$ for $\mathfrak{B}$ a Banach algebra. Then,
  \begin{align*}
    \sigma_{\mathfrak{B}}(f) &= \set{\lambda \in \C\mid f - \lambda e\notin \mathcal{G}}.
  \end{align*}
  The resolvent of $f$ is
  \begin{align*}
    \rho_{\mathfrak{B}}(f) &= \C \setminus \sigma_{\mathfrak{B}}\left(f\right).
  \end{align*}
  Finally, the spectral radius of $f$ is
  \begin{align*}
    r_{\mathfrak{B}}\left(f\right) &= \sup_{\lambda \in \sigma_{\mathfrak{B}}(f)}|\lambda|.
  \end{align*}
  We write $\sigma(f)$, $\rho(f)$, and $r(f)$.
\end{definition}
\begin{proposition}[Properties of the Spectrum]
  Let $\mathfrak{B}$ be a Banach algebra. Then, $\sigma(f)$ is compact in $\C$ and $r(f) \leq \norm{f}$.
\end{proposition}
\begin{proof}
  Define $\varphi: \C\rightarrow \mathfrak{B}$, $\varphi(\lambda) = f - \lambda e$. Then, $\varphi$ is continuous, and $\rho(f) = \varphi^{-1}\left(\mathcal{G}\right)$ is open. Thus, $\sigma(f)$ is closed.\newline

  If $|\lambda| > \norm{f}$, then
  \begin{align*}
    1 &> \frac{\norm{f}}{|\lambda|}\\
      &= \norm{\frac{f}{\lambda}}\\
      &= \norm{e - \left(e - \frac{f}{\lambda}\right)},
  \end{align*}
  meaning $e - \frac{f}{\lambda}$ is invertible, so $f - \lambda e$ is invertible. Thus, $\lambda \in \rho(f)$, so $\sigma(f)$ is bounded (hence compact), and $r(f) \leq \norm{f}$.
\end{proof}
\begin{theorem}[Existence of Spectrum]
  Let $f\in \mathfrak{B}$. Then, $\sigma(f)$ is nonempty.
\end{theorem}
\begin{proof}
  Consider $F: \rho(f) \rightarrow \mathfrak{B}$ defined by $F(\lambda) = \left(f - \lambda e\right)^{-1}$. We will show that $F$ is an analytic $\mathfrak{B}$-valued function on $\rho(f)$ that is bounded at infinity (thus, a contradiction).\newline

  Since inversion is continuous, we have that for $\lambda_0 \in \rho(f)$,
  \begin{align*}
    \lim_{\lambda \rightarrow \lambda_0} \frac{F(\lambda) - F(\lambda_0)}{\lambda - \lambda_0} &= \lim_{\lambda \rightarrow \lambda_0}\frac{\left(f - \lambda_0 e\right)^{-1}\left((f - \lambda-0 e) - (f - \lambda e)\right)\left(f - \lambda e\right)^{-1}}{\lambda - \lambda_0}\\
                                                                                               &= \lim_{\lambda \rightarrow \lambda_0}\left(f - \lambda_0 e\right)^{-1}\left(f - \lambda e\right)^{-1}\\
                                 &= \left(f - \lambda e\right)^{-2}.
  \end{align*}
  In particular, for $\varphi \in \mathfrak{B}^{\ast}$, the function $\varphi(F)$ is holomorphic on $\rho(f)$. Further, for $|\lambda| \geq \norm{f}$, we have that $e - \frac{f}{\lambda}$ is invertible, and
  \begin{align*}
    \norm{\left(e - \frac{f}{\lambda}\right)^{-1}} &\leq \frac{1}{1 - \norm{\frac{f}{\lambda}}},
  \end{align*}
  meaning
  \begin{align*}
    \lim_{\lambda \rightarrow\infty}\norm{F(\lambda)} &= \lim_{\lambda \rightarrow\infty}\norm{\frac{1}{\lambda}\left(\frac{f}{\lambda} - e\right)^{-1}}\\
                                                      &\leq \lim_{|\lambda|\rightarrow\infty}\sup\frac{1}{|\lambda|} \frac{1}{1-\norm{\frac{f}{\lambda}}}\\
                                                      &= 0.
  \end{align*}
  Thus, for $\varphi \in \mathfrak{B}^{\ast}$, $\lim_{\lambda \rightarrow\infty}\varphi\left(F(\lambda)\right) - 0$.\newline

  If $\sigma(f)$ is empty, then $\rho(f) = \C$, meaning that for $\varphi \in \mathfrak{B}^{\ast}$, it follows that $\varphi(F)$ is an entire function that vanishes at infinity, meaning $\varphi(F) = 0$ by Liouville's Theorem.\newline

  In particular, for $\lambda \in \C$, $\varphi\left(F(\lambda)\right) = 0$, meaning that $F(\lambda) = 0$, which contradicts $F(\lambda)$ being invertible in $\mathfrak{B}$.
\end{proof}
\begin{theorem}[Gelfand--Mazur]
  If $\mathfrak{B}$ is a Banach algebra that is also a division algebra,\footnote{Every nonzero element has a nonzero inverse.} then there exists a unique isometric isomorphism of $\mathfrak{B}$ onto $\C$.
\end{theorem}
\begin{proof}
  Let $f\in \mathfrak{B}$. Then $\sigma(f)$ is nonempty; for $\lambda_f\in \sigma(f)$, we have that $f-\lambda_f e$ is not invertible, meaning that $f - \lambda_f e = 0$ since $\mathfrak{B}$ is a division algebra.\newline

  Moreover, for $\lambda \neq \lambda_f$, $f - \lambda e = \lambda_f e - \lambda e$, which is invertible. Thus, $\sigma(f)$ consists of exactly one $\lambda_f\in \C$ for each $f$.\newline

  The map $\psi: \mathfrak{B}\rightarrow \C$ defined by $\psi(f) = \lambda_f$ is an isometric isomorphism of $\mathfrak{B}$ onto $\C$.\newline

  Moreover, for $\psi': \mathfrak{B} \rightarrow \C$, we would have that $\psi'(f) \in \sigma(f)$, meaning $\psi'(f) = \psi(f)$.
\end{proof}
\begin{definition}[Quotient Algebra]
Let $\mathfrak{B}$ be a Banach algebra, and let $\mathfrak{M}$ be a closed two-sided ideal in $\mathfrak{B}$. Since $\mathfrak{M}$ is closed in $\mathfrak{B}$, we can define a norm on $\mathfrak{B}/\mathfrak{M}$ to make it into a Banach space, and since $\mathfrak{M}$ is a two-sided ideal in $\mathfrak{B}$, we know that $\mathfrak{B}/\mathfrak{M}$ is an algebra.\newline

To verify that $\mathfrak{B}/\mathfrak{M}$ is a Banach algebra, we need to verify two facts.\newline

To show that $\norm{[e]} = 1$, we see that $\norm{[e]} = \inf_{g\in \mathfrak{M}} \norm{e-g} = 1$; if $\norm{e-g} < 1$, then $g$ is invertible.\footnote{Any proper ideal in $\mathfrak{B}$ cannot contain any invertible elements, since if $x\in \mathfrak{M}$ is invertible, then for $y\in \mathfrak{B}$, $y = \left(yx^{-1}\right)x \in \mathfrak{M}$, implying $\mathfrak{M} = \mathfrak{B}$.}\newline

For $f,g\in \mathfrak{B}$, we have
\begin{align*}
  \norm{[f][g]} &= \norm{[fg]}\\
                &= \inf_{h\in \mathfrak{M}}\norm{fg - h}\\
                &\leq \inf_{h_1,h_2\in \mathfrak{M}}\norm{(f-h_1)(g-h_2)}\\
                &\leq \inf_{h_1\in \mathfrak{M}}\norm{f-h_1}\inf_{h_2\in \mathfrak{M}}\norm{g-h_2}\\
                &= \norm{[f]}\norm{[g]}.
\end{align*}
Thus, we can see that $\mathfrak{B}/\mathfrak{M}$ is a Banach algebra, with the natural map $\pi: f \rightarrow [f]$ a contractive algebra homomorphism.
\end{definition}
\begin{proposition}[Multiplicative Linear Functionals and Maximal Ideal Space]
  If $\mathfrak{B}$ is a commutative Banach algebra, then there is a bijection between $M_{\mathfrak{B}}$ and the set of maximal two-sided ideals in $\mathfrak{B}$.
\end{proposition}
\end{document}
