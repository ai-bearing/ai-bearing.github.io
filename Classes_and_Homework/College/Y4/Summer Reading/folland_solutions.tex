\documentclass[10pt]{mypackage}

% sans serif font:
%\usepackage{cmbright}
%\usepackage{sfmath}
%\usepackage{bbold} %better blackboard bold

%serif font + different blackboard bold for serif font
\usepackage{newpxtext,eulerpx,eucal,eufrak}
\renewcommand*{\mathbb}[1]{\varmathbb{#1}}
\renewcommand*{\hbar}{\hslash}

\pagestyle{fancy} %better headers
\fancyhf{}
\rhead{Avinash Iyer}
\lhead{ \textit{Real Analysis and Applications} (Folland) Exercise Workthrough }

\setcounter{secnumdepth}{0}

\begin{document}
\RaggedRight
\section{Introduction}%
Oh hey, it's another one of those textbook notes that I never complete. I've decided to try something different in order to develop my understanding of measure theory. One of the primary for understanding measure theory is Gerald B. Folland's \textit{Real Analysis and Applications} --- and one of the benefits it has over a lot of other texts is that it has a significant number of exercises. I'm going to try to do them all --- I'll start with Chapters 1--3, and if that goes well enough, continue up through whatever chapter I end up having to tap out at. Interspersed, I will include various notes. I figure that in order to make a subject like measure theory really stick, I need to deal with it consistently.
\tableofcontents
\section{Chapter 1}%
\subsection{Section 1.2}%
\begin{definition}[$\sigma$-Algebra]
  An algebra of sets on $X$ is a nonempty collection $\mathcal{A}$ of $X$ that is closed under finite unions and complements.\newline

  A $\sigma$-algebra is an algebra that is closed under countable unions.
\end{definition}
\begin{exercise}[Exercise 1]
  A family of sets $\mathcal{R}\subseteq P(X)$ is called a ring if it is closed under finite unions and differences. A ring that is closed under countable unions is called a $\sigma$-ring.
  \begin{enumerate}[(a)]
    \item Rings ($\sigma$-rings) are closed under finite (countable) intersections.
    \item If $\mathcal{R}$ is a ring ($\sigma$-ring), then $\mathcal{R}$ is an algebra ($\sigma$-algebra) if and only if $X\in \mathcal{R}$.
    \item If $\mathcal{R}$ is a $\sigma$-ring, then $\set{E\subseteq X | E\in \mathcal{R}\text{ or }E^{c}\in \mathcal{R}}$ is a $\sigma$-algebra.
    \item If $\mathcal{R}$ is a $\sigma$-ring, then $\set{E\subseteq X | E\cap F\in \mathcal{R}\text{ for all }F\in \mathcal{R}}$ is a $\sigma$-algebra.
  \end{enumerate}
\end{exercise}
\begin{solution}\hfill
  \begin{enumerate}[(a)]
    \item Note that for any $E,F\in \mathcal{R}$, that $E\cap F = E\cup F \setminus\left(\left(E\setminus F\right)\cup \left(F\setminus E\right)\right)$.
    \item Let $\mathcal{R}$ be a $\sigma$-ring. Then, $\mathcal{R}$ is a $\sigma$-algebra if for some $E\in \mathcal{R}$, $E^{c}\in \mathcal{R}$. Since $E^{c} = X\setminus E\in \mathcal{R}$, we have $X\setminus E \cup E \in \mathcal{R}$ as $\mathcal{R}$ is closed under (countable) unions. Hence, $X\in \mathcal{R}$.\newline

      If $X\in \mathcal{R}$, then for any $E\in \mathcal{R}$, $E^{c} = X\setminus E\in \mathcal{R}$. Thus, $\mathcal{R}$ is closed under intersections.
    \item Since $\mathcal{R}$ is a $\sigma$-ring, we only need show that the set $\mathcal{A} = \set{E\subseteq X | E\in \mathcal{R}\text{ or }E^{c}\in \mathcal{R}}$ is closed under complements. We see that for any $E\in \mathcal{A}$, it is the case that either $E\in \mathcal{R}$ or $E^{c}\in \mathcal{R}$, so $E^{c}\in \mathcal{A}$ if and only if $E^{c}\in \mathcal{R}$ or $E\in \mathcal{R}$, so $\mathcal{A}$ is closed under complements.
    \item Let $\mathcal{R}$ be a $\sigma$-ring, and let $\mathcal{A} = \set{E\subseteq X | E\cap F\in \mathcal{R}\text{ for all }F\in \mathcal{R}}$. We will show that $\mathcal{A}$ is closed under unions and complements.\newline

      Let $E,F\in \mathcal{A}$. Then, for all $S\in \mathcal{R}$, we have $E\cap S\in \mathcal{R}$ and $F\cap S\in \mathcal{R}$. Since $\mathcal{R}$ is closed under unions, we must have $\left(E\cup F\right)\cap S = \left(E\cap S\right)\cup \left(F\cap S\right)\in \mathcal{R}$, so $E\cup F\in \mathcal{A}$.\newline

      Let $E\in \mathcal{A}$.
  \end{enumerate}
\end{solution}

\begin{proposition}[Proposition 1.2]
  The Borel $\sigma$-algebra, $\mathcal{B}_{\R}$, is generated by each of the following:
  \begin{enumerate}[(a)]
    \item the open intervals, $\mathcal{E}_1 = \set{(a,b) | a < b}$;
    \item the closed intervals, $\mathcal{E}_2 = \set{[a,b] | a < b}$;
    \item the half-open intervals, $\mathcal{E}_3 = \set{(a,b] | a < b}$ or $\mathcal{E}_4 = \set{[a,b) | a < b}$;
    \item the open rays, $\mathcal{E}_5 = \set{(a,\infty) | a \in \R}$ or $\mathcal{E}_6 = \set{(-\infty,a) | a\in\R}$;
    \item the closed rays, $\mathcal{E}_7 = \set{[a,\infty) | a\in\R}$ or $\mathcal{E}_8 = \set{(-\infty,a] | a\in\R}$.
  \end{enumerate}
\end{proposition}
\begin{proof}
  The elements for $\mathcal{E}_j$ for $j\neq 3,4$ are open or closed, and the elements of $\mathcal{E}_3$, $\mathcal{E}_4$ are $G_{\delta}$ sets --- for instance,
  \begin{align*}
    (a,b] = \bigcap_{n=1}^{\infty}\left(a,b + \frac{1}{n}\right).
  \end{align*}
  Thus, $\sigma\left(\mathcal{E}_j\right)\subseteq \mathcal{B}_{\R}$ for each $j$. On the other hand, every open set in $\R$ is a countable union of open intervals, so $\mathcal{B}_{\R}\subseteq \sigma\left(\mathcal{E}_1\right)$. Thus, $\mathcal{B}_{\R} = \sigma\left(\mathcal{E}_1\right)$.
\end{proof}
\subsection{Section 1.3}%
\begin{theorem}[Theorem 1.9]
  Let $\left(X,\mathcal{M},\mu\right)$ be a measure space. Let $\mathcal{N} = \set{N\in \mathcal{M} | \mu\left(N\right) = 0}$, and let $\overline{\mathcal{M}} = \set{E\cup F | E\in \mathcal{M}\text{ and }F\subseteq N\text{ for some }N\in \mathcal{N}}$. Then, $\mathcal{M}$ is a $\sigma$-algebra, and there is a unique extension $\overline{\mu}$ of $\mu$ to a complete measure on $\overline{\mathcal{M}}$.
\end{theorem}
\begin{proof}
  Since $\mathcal{M}$ and $\mathcal{N}$ are closed under countable unions, so is $\overline{\mathcal{M}}$. If $E\cup F\in \overline{\mathcal{M}}$, with $E\in \mathcal{M}$ and $F\subseteq N\in \mathcal{N}$, we may assume $E\cap N = \emptyset$ --- else, we replace $F$ with $F\setminus E$ and $N$ with $N\setminus E$. Then, $E\cup F = \left(E\cup N\right)\cap \left(N^{c}\cup F\right)$, so
  $\left(E\cup F\right)^{c} = \left(E\cup N\right)^{c}\cup \left(N\setminus F\right)$. Since $\left(E\cup N\right)^{c}\in \mathcal{M}$ and $N\setminus F\subseteq N$, we have $\left(E\cup F\right)^{c}\in \overline{\mathcal{M}}$, so $\overline{M}$ is a $\sigma$-algebra.\newline

  If $E\cup F \in \overline{\mathcal{M}}$ as above, we set $\overline{\mu}\left(E\cup F\right) = \mu\left(E\right)$. This is well-defined, since if $E_1 \cup F_1 = E_2\cup F_2$, with $F_j\subseteq N_j\in \mathcal{N}$, then $E_1 \subseteq E_2\cup N_2$, so $\mu\left(E_1\right)\leq \mu\left(E_2\right) + \mu\left(N_2\right) = \mu\left(E_2\right)$. Similarly, $\mu\left(E_2\right)\subseteq \mu\left(E_1\right)$.
\end{proof}
\begin{exercise}[Exercise 6]
Complete the proof of Theorem 1.9.
\end{exercise}
\begin{solution}
  We now wish to show that every subset of a null set in $\mathcal{M}$ is an element of $\overline{\mathcal{M}}$. This can be seen by the fact that for some $F\subseteq N\in \mathcal{N}$, we have $F = \emptyset \cup F\in \overline{\mathcal{M}}$.\newline

  To show uniqueness, we suppose there is some other measure $\nu\colon \overline{\mathcal{M}}\rightarrow [0,\infty)$ such that $\nu$ agrees with $\mu$ on $\mathcal{M}$. For some $E\in \mathcal{M}$ and $F\subseteq N\in \mathcal{N}$, we have
  \begin{align*}
    \nu\left(E\cup F\right) &= \mu\left(E\right)\\
                            &= \overline{\mu}\left(E\cup F\right).
  \end{align*}
\end{solution}
\begin{exercise}[Exercise 7]
  If $\mu_1,\dots,\mu_n$ are measures on $\left(X,\mathcal{M}\right)$, and $a_1,\dots,a_n\in [0,\infty)$, then $\mu = \sum_{j=1}^{n}a_j\mu_j$ is a measure on $\left(X,\mathcal{M}\right)$.
\end{exercise}
\begin{solution}
  It is clear that $\mu\left(\emptyset\right) = \emptyset$. If we have a sequence of disjoint sets $\set{E_i}_{i=1}^{\infty}\subseteq \mathcal{M}$, then
  \begin{align*}
    \mu\left(\bigcup_{i=1}^{\infty}E_i\right) &= \sum_{j=1}^{n}a_j\mu_j\left(\bigcup_{i=1}^{\infty}E_i\right)\\
                                              &= \sum_{j=1}^{n}a_j\sum_{i=1}^{\infty}\mu_j\left(E_i\right)\\
                                              &= \sum_{i=1}^{\infty}\left(\sum_{j=1}^{n}a_j\mu_j\right)\left(E_i\right)\\
                                              &= \sum_{i=1}^{\infty}\mu\left(E_i\right).
  \end{align*}
  
\end{solution}

\begin{exercise}[Exercise 9]
  If $\left(X,\mathcal{M},\mu\right)$ is a measure space, and $E,F\in \mathcal{M}$, then $\mu\left(E\right) + \mu\left(F\right) = \mu\left(E\cup F\right) + \mu\left(E\cap F\right)$.
\end{exercise}
\begin{solution}
  We have
  \begin{align*}
    \mu\left(E\right) &= \mu\left(\left(\left(E\cup F\right)\setminus F\right)\sqcup E\cap F\right)\\
    \mu\left(E\right) &= \mu\left(E\cup F\right) - \mu\left(F\right) + \mu\left(E\cap F\right)\\
    \mu\left(E\right) + \mu\left(F\right) &= \mu\left(E\cup F\right) + \mu\left(E\cap F\right).
  \end{align*}
\end{solution}


\end{document}
