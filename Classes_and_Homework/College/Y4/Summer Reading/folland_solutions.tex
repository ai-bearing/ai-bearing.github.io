\documentclass[10pt]{mypackage}

% sans serif font:
%\usepackage{cmbright}
%\usepackage{sfmath}
%\usepackage{bbold} %better blackboard bold

%serif font + different blackboard bold for serif font
\usepackage{newpxtext,eulerpx,eucal,eufrak}
\renewcommand*{\mathbb}[1]{\varmathbb{#1}}
\renewcommand*{\hbar}{\hslash}

\pagestyle{fancy} %better headers
\fancyhf{}
\rhead{Avinash Iyer}
\lhead{ \textit{Real Analysis and Applications} (Folland) Exercise Workthrough }

\setcounter{secnumdepth}{0}

\begin{document}
\RaggedRight
\section{Introduction}%
Oh hey, it's another one of those textbook notes that I never complete. I've decided to try something different in order to develop my understanding of measure theory. One of the primary for understanding measure theory is Gerald B. Folland's \textit{Real Analysis and Applications} --- and one of the benefits it has over a lot of other texts is that it has a significant number of exercises. I'm going to try to do them all --- I'll start with Chapters 1--3, and if that goes well enough, continue up through whatever chapter I end up having to tap out at. Interspersed, I will include various notes. I figure that in order to make a subject like measure theory really stick, I need to deal with it consistently.
\tableofcontents
\section{Chapter 1}%
\subsubsection{Section 1.2}%
\begin{exercise}[Exercise 1]
  A family of sets $\mathcal{R}\subseteq P(X)$ is called a ring if it is closed under finite unions and differences. A ring that is closed under countable unions is called a $\sigma$-ring.
  \begin{enumerate}[(a)]
    \item Rings ($\sigma$-rings) are closed under finite (countable) intersections.
    \item If $\mathcal{R}$ is a ring ($\sigma$-ring), then $\mathcal{R}$ is an algebra ($\sigma$-algebra) if and only if $X\in \mathcal{R}$.
    \item If $\mathcal{R}$ is a $\sigma$-ring, then $\set{E\subseteq X | E\in \mathcal{R}\text{ or }E^{c}\in \mathcal{R}}$ is a $\sigma$-algebra.
    \item If $\mathcal{R}$ is a $\sigma$-ring, then $\set{E\subseteq X | E\cap F\in \mathcal{R}\text{ for all }F\in R}$ is a $\sigma$-algebra.
  \end{enumerate}
\end{exercise}
\begin{proposition}[Proposition 1.2]
  The Borel $\sigma$-algebra, $\mathcal{B}_{\R}$, is generated by each of the following:
  \begin{enumerate}[(a)]
    \item the open intervals, $\mathcal{E}_1 = \set{(a,b) | a < b}$;
    \item the closed intervals, $\mathcal{E}_2 = \set{[a,b] | a < b}$;
    \item the half-open intervals, $\mathcal{E}_3 = \set{(a,b] | a < b}$ or $\mathcal{E}_4 = \set{[a,b) | a < b}$;
    \item the open rays, $\mathcal{E}_5 = \set{(a,\infty) | a \in \R}$ or $\mathcal{E}_6 = \set{(-\infty,a) | a\in\R}$;
    \item the closed rays, $\mathcal{E}_7 = \set{[a,\infty) | a\in\R}$ or $\mathcal{E}_8 = \set{(-\infty,a] | a\in\R}$.
  \end{enumerate}
\end{proposition}
\begin{proof}
  The elements for $\mathcal{E}_j$ for $j\neq 3,4$ are open or closed, and the elements of $\mathcal{E}_3$, $\mathcal{E}_4$ are $G_{\delta}$ sets --- for instance,
  \begin{align*}
    (a,b] = \bigcap_{n=1}^{\infty}\left(a,b + \frac{1}{n}\right).
  \end{align*}
  Thus, $\sigma\left(\mathcal{E}_j\right)\subseteq \mathcal{B}_{\R}$ for each $j$. On the other hand, every open set in $\R$ is a countable union of open intervals, so $\mathcal{B}_{\R}\subseteq \sigma\left(\mathcal{E}_1\right)$. Thus, $\mathcal{B}_{\R} = \sigma\left(\mathcal{E}_1\right)$.
\end{proof}

\end{document}
