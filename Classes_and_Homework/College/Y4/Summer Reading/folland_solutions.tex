\documentclass[10pt]{mypackage}

% sans serif font:
%\usepackage{cmbright}
%\usepackage{sfmath}
%\usepackage{bbold} %better blackboard bold

%serif font + different blackboard bold for serif font
\usepackage{newpxtext,eulerpx,eucal,eufrak}
\renewcommand*{\mathbb}[1]{\varmathbb{#1}}
\renewcommand*{\hbar}{\hslash}
\usepackage{homework}

\pagestyle{fancy} %better headers
\fancyhf{}
\rhead{Avinash Iyer}
\lhead{ \textit{Real Analysis and Applications} (Folland) Exercise Workthrough }

\setcounter{secnumdepth}{0}

\begin{document}
\RaggedRight
\section{Introduction}%
Oh hey, it's another one of those textbook notes that I never complete. I've decided to try something different in order to develop my understanding of measure theory. One of the primary for understanding measure theory is Gerald B. Folland's \textit{Real Analysis and Applications} --- and one of the benefits it has over a lot of other texts is that it has a significant number of exercises. I'm going to try to do them all --- I'll start with Chapters 1--3, and if that goes well enough, continue up through whatever chapter I end up having to tap out at. Interspersed, I will include various notes. I figure that in order to make a subject like measure theory really stick, I need to deal with it consistently.
\tableofcontents
\section{Chapter 1}%
\subsection{Section 1.2}%
\begin{definition}[$\sigma$-Algebra]
  An algebra of sets on $X$ is a nonempty collection $\mathcal{A}$ of $X$ that is closed under finite unions and complements.\newline

  A $\sigma$-algebra is an algebra that is closed under countable unions.
\end{definition}
\begin{exercise}[Exercise 1]
  A family of sets $\mathcal{R}\subseteq P(X)$ is called a ring if it is closed under finite unions and differences. A ring that is closed under countable unions is called a $\sigma$-ring.
  \begin{enumerate}[(a)]
    \item Rings ($\sigma$-rings) are closed under finite (countable) intersections.
    \item If $\mathcal{R}$ is a ring ($\sigma$-ring), then $\mathcal{R}$ is an algebra ($\sigma$-algebra) if and only if $X\in \mathcal{R}$.
    \item If $\mathcal{R}$ is a $\sigma$-ring, then $\set{E\subseteq X | E\in \mathcal{R}\text{ or }E^{c}\in \mathcal{R}}$ is a $\sigma$-algebra.
    \item If $\mathcal{R}$ is a $\sigma$-ring, then $\set{E\subseteq X | E\cap F\in \mathcal{R}\text{ for all }F\in \mathcal{R}}$ is a $\sigma$-algebra.
  \end{enumerate}
\end{exercise}
\begin{solution}\hfill
  \begin{enumerate}[(a)]
    \item Note that for any $E,F\in \mathcal{R}$, that $E\cap F = E\cup F \setminus\left(\left(E\setminus F\right)\cup \left(F\setminus E\right)\right)$.
    \item Let $\mathcal{R}$ be a $\sigma$-ring. Then, $\mathcal{R}$ is a $\sigma$-algebra if for some $E\in \mathcal{R}$, $E^{c}\in \mathcal{R}$. Since $E^{c} = X\setminus E\in \mathcal{R}$, we have $X\setminus E \cup E \in \mathcal{R}$ as $\mathcal{R}$ is closed under (countable) unions. Hence, $X\in \mathcal{R}$.\newline

      If $X\in \mathcal{R}$, then for any $E\in \mathcal{R}$, $E^{c} = X\setminus E\in \mathcal{R}$. Thus, $\mathcal{R}$ is closed under intersections.
    \item Since $\mathcal{R}$ is a $\sigma$-ring, we only need show that the set $\mathcal{A} = \set{E\subseteq X | E\in \mathcal{R}\text{ or }E^{c}\in \mathcal{R}}$ is closed under complements. We see that for any $E\in \mathcal{A}$, it is the case that either $E\in \mathcal{R}$ or $E^{c}\in \mathcal{R}$, so $E^{c}\in \mathcal{A}$ if and only if $E^{c}\in \mathcal{R}$ or $E\in \mathcal{R}$, so $\mathcal{A}$ is closed under complements.
    \item Let $\mathcal{R}$ be a $\sigma$-ring, and let $\mathcal{A} = \set{E\subseteq X | E\cap F\in \mathcal{R}\text{ for all }F\in \mathcal{R}}$. We will show that $\mathcal{A}$ is closed under unions and complements.\newline

      Let $E,F\in \mathcal{A}$. Then, for all $S\in \mathcal{R}$, we have $E\cap S\in \mathcal{R}$ and $F\cap S\in \mathcal{R}$. Since $\mathcal{R}$ is closed under unions, we must have $\left(E\cup F\right)\cap S = \left(E\cap S\right)\cup \left(F\cap S\right)\in \mathcal{R}$, so $E\cup F\in \mathcal{A}$.\newline

      Let $E\in \mathcal{A}$.
  \end{enumerate}
\end{solution}

\begin{proposition}[Proposition 1.2]
  The Borel $\sigma$-algebra, $\mathcal{B}_{\R}$, is generated by each of the following:
  \begin{enumerate}[(a)]
    \item the open intervals, $\mathcal{E}_1 = \set{(a,b) | a < b}$;
    \item the closed intervals, $\mathcal{E}_2 = \set{[a,b] | a < b}$;
    \item the half-open intervals, $\mathcal{E}_3 = \set{(a,b] | a < b}$ or $\mathcal{E}_4 = \set{[a,b) | a < b}$;
    \item the open rays, $\mathcal{E}_5 = \set{(a,\infty) | a \in \R}$ or $\mathcal{E}_6 = \set{(-\infty,a) | a\in\R}$;
    \item the closed rays, $\mathcal{E}_7 = \set{[a,\infty) | a\in\R}$ or $\mathcal{E}_8 = \set{(-\infty,a] | a\in\R}$.
  \end{enumerate}
\end{proposition}
\begin{proof}
  The elements for $\mathcal{E}_j$ for $j\neq 3,4$ are open or closed, and the elements of $\mathcal{E}_3$, $\mathcal{E}_4$ are $G_{\delta}$ sets --- for instance,
  \begin{align*}
    (a,b] = \bigcap_{n=1}^{\infty}\left(a,b + \frac{1}{n}\right).
  \end{align*}
  Thus, $\sigma\left(\mathcal{E}_j\right)\subseteq \mathcal{B}_{\R}$ for each $j$. On the other hand, every open set in $\R$ is a countable union of open intervals, so $\mathcal{B}_{\R}\subseteq \sigma\left(\mathcal{E}_1\right)$. Thus, $\mathcal{B}_{\R} = \sigma\left(\mathcal{E}_1\right)$.
\end{proof}
\subsection{Section 1.3}%
\begin{theorem}[Theorem 1.9]
  Let $\left(X,\mathcal{M},\mu\right)$ be a measure space. Let $\mathcal{N} = \set{N\in \mathcal{M} | \mu\left(N\right) = 0}$, and let $\overline{\mathcal{M}} = \set{E\cup F | E\in \mathcal{M}\text{ and }F\subseteq N\text{ for some }N\in \mathcal{N}}$. Then, $\mathcal{M}$ is a $\sigma$-algebra, and there is a unique extension $\overline{\mu}$ of $\mu$ to a complete measure on $\overline{\mathcal{M}}$.
\end{theorem}
\begin{proof}
  Since $\mathcal{M}$ and $\mathcal{N}$ are closed under countable unions, so is $\overline{\mathcal{M}}$. If $E\cup F\in \overline{\mathcal{M}}$, with $E\in \mathcal{M}$ and $F\subseteq N\in \mathcal{N}$, we may assume $E\cap N = \emptyset$ --- else, we replace $F$ with $F\setminus E$ and $N$ with $N\setminus E$. Then, $E\cup F = \left(E\cup N\right)\cap \left(N^{c}\cup F\right)$, so
  $\left(E\cup F\right)^{c} = \left(E\cup N\right)^{c}\cup \left(N\setminus F\right)$. Since $\left(E\cup N\right)^{c}\in \mathcal{M}$ and $N\setminus F\subseteq N$, we have $\left(E\cup F\right)^{c}\in \overline{\mathcal{M}}$, so $\overline{M}$ is a $\sigma$-algebra.\newline

  If $E\cup F \in \overline{\mathcal{M}}$ as above, we set $\overline{\mu}\left(E\cup F\right) = \mu\left(E\right)$. This is well-defined, since if $E_1 \cup F_1 = E_2\cup F_2$, with $F_j\subseteq N_j\in \mathcal{N}$, then $E_1 \subseteq E_2\cup N_2$, so $\mu\left(E_1\right)\leq \mu\left(E_2\right) + \mu\left(N_2\right) = \mu\left(E_2\right)$. Similarly, $\mu\left(E_2\right)\subseteq \mu\left(E_1\right)$.
\end{proof}
\begin{exercise}[Exercise 6]
Complete the proof of Theorem 1.9.
\end{exercise}
\begin{solution}
  We now wish to show that every subset of a null set in $\mathcal{M}$ is an element of $\overline{\mathcal{M}}$. This can be seen by the fact that for some $F\subseteq N\in \mathcal{N}$, we have $F = \emptyset \cup F\in \overline{\mathcal{M}}$.\newline

  To show uniqueness, we suppose there is some other measure $\nu\colon \overline{\mathcal{M}}\rightarrow [0,\infty)$ such that $\nu$ agrees with $\mu$ on $\mathcal{M}$. For some $E\in \mathcal{M}$ and $F\subseteq N\in \mathcal{N}$, we have
  \begin{align*}
    \nu\left(E\cup F\right) &= \mu\left(E\right)\\
                            &= \overline{\mu}\left(E\cup F\right).
  \end{align*}
\end{solution}
\begin{exercise}[Exercise 7]
  If $\mu_1,\dots,\mu_n$ are measures on $\left(X,\mathcal{M}\right)$, and $a_1,\dots,a_n\in [0,\infty)$, then $\mu = \sum_{j=1}^{n}a_j\mu_j$ is a measure on $\left(X,\mathcal{M}\right)$.
\end{exercise}
\begin{solution}
  It is clear that $\mu\left(\emptyset\right) = \emptyset$. If we have a sequence of disjoint sets $\set{E_i}_{i=1}^{\infty}\subseteq \mathcal{M}$, then
  \begin{align*}
    \mu\left(\bigcup_{i=1}^{\infty}E_i\right) &= \sum_{j=1}^{n}a_j\mu_j\left(\bigcup_{i=1}^{\infty}E_i\right)\\
                                              &= \sum_{j=1}^{n}a_j\sum_{i=1}^{\infty}\mu_j\left(E_i\right)\\
                                              &= \sum_{i=1}^{\infty}\left(\sum_{j=1}^{n}a_j\mu_j\right)\left(E_i\right)\\
                                              &= \sum_{i=1}^{\infty}\mu\left(E_i\right).
  \end{align*}
  
\end{solution}
\begin{exercise}[Exercise 8]
  If $\left( X,\mathcal{M},\mu \right)$ is a measure space, and $\set{E_j}_{j=1}^{\infty}\subseteq \mathcal{M}$, then $\mu\left( \liminf E_j \right)\leq \liminf \mu\left( E_j \right)$. Additionally, if $\mu\left( \bigcup_{j\geq 1}E_j \right) < \infty$, then $\mu\left( \limsup E_j \right)\geq \limsup \mu\left( E_j \right)$.
\end{exercise}
\begin{solution}
  Note that
  \begin{align*}
    \liminf E_j &= \bigcup_{n=1}^{\infty}\bigcap_{j=n}^{\infty}E_j.
  \end{align*}
  Labeling
  \begin{align*}
    F_n &= \bigcap_{j=n}^{\infty}E_j,
  \end{align*}
  we have a sequence of inclusions
  \begin{align*}
    F_1 \subseteq F_2 \subseteq \cdots,
  \end{align*}
  meaning that
  \begin{align*}
    \mu\left( \limsup E_j \right) &= \inf_{n\geq 1}\mu\left( F_n \right).
  \end{align*}
  Note that we have
  \begin{align*}
    \mu\left( F_n \right) &= \mu\left( \bigcup_{n=j}^{\infty}E_j \right).
  \end{align*}
  
\end{solution}

\begin{exercise}[Exercise 9]
  If $\left(X,\mathcal{M},\mu\right)$ is a measure space, and $E,F\in \mathcal{M}$, then $\mu\left(E\right) + \mu\left(F\right) = \mu\left(E\cup F\right) + \mu\left(E\cap F\right)$.
\end{exercise}
\begin{solution}
  We have
  \begin{align*}
    \mu\left(E\right) &= \mu\left(\left(\left(E\cup F\right)\setminus F\right)\sqcup E\cap F\right)\\
    \mu\left(E\right) &= \mu\left(E\cup F\right) - \mu\left(F\right) + \mu\left(E\cap F\right)\\
    \mu\left(E\right) + \mu\left(F\right) &= \mu\left(E\cup F\right) + \mu\left(E\cap F\right).
  \end{align*}
\end{solution}
\begin{exercise}[Exercise 12]
Let $\left(X,\mathcal{M},\mu\right)$ be a finite measure space.
\begin{enumerate}[(a)]
  \item If $E,F\in \mathcal{M}$ with $\mu\left(E\triangle M\right) = 0$, then $\mu\left(E\right) = \mu\left(F\right)$.
  \item Let $E\sim F$ if $\mu\left(E\triangle F\right) = 0$. Then, $\sim$ is an equivalence relation on $\mathcal{M}$.
  \item For $E,F\in \mathcal{M}$, define $\rho\left(E,F\right) = \mu\left(E\triangle F\right)$. Then, $\rho\left(E,G\right)\leq \rho\left(E,F\right) + \rho\left(F,G\right)$, hence $\rho$ defines a metric on the space $\mathcal{M}/\sim$ of equivalence classes.
\end{enumerate}
\end{exercise}
\begin{solution}\hfill
  \begin{enumerate}[(a)]
    \item Note that $E = \left(E\setminus F\right) \sqcup \left(E\cap F\right)$, and $F = \left(F\setminus E\right)\sqcup \left(F\cap E\right)$. We also have $\mu\left(E\triangle F\right) = \mu\left(E\setminus F\right) + \mu\left(F\setminus E\right) = 0$, so $\mu\left(F\setminus E\right) = \mu\left(E\setminus F\right) = 0$. Thus,
      \begin{align*}
        \mu\left(F\right) &= \mu\left(F\cap E\right)\\
        &= \mu\left(E\cap F\right)\\
        &= \mu\left(E\right).
      \end{align*}
  \end{enumerate}
\end{solution}
\begin{definition}
  Let $\left(X,\mathcal{M},\mu\right)$ be a measure space.
  \begin{itemize}
    \item If $\mu\left(X\right) < \infty$, then $\mu$ is called finite.
    \item If $X = \bigcup_{j\geq 1}E_j$, where $E_j\in \mathcal{M}$ for each $j$ and $\mu\left(E_j\right) < \infty$, then $\mu$ is called $\sigma$-finite.
    \item If for each $E\in \mathcal{M}$ with $\mu\left(E\right) = \infty$, there exists $F\in \mathcal{M}$ with $F\subseteq E$ and $0 < \mu\left(F\right) < \infty$, then we say $\mu$ is semifinite.
  \end{itemize}
\end{definition}

\begin{exercise}[Exercise 13]
  Every $\sigma$-finite measure is semifinite.
\end{exercise}
\begin{solution}
  Let $\left(X,\mathcal{M},\mu\right)$ be a measure space such that $X = \bigcup_{j\geq 1}E_j$, where $\set{E_j}_{j\geq 1}\subseteq \mathcal{M}$ and $\mu\left(E_j\right) < \infty$ for each $j$.\newline

  Suppose $\mu\left(E\right) = \infty$. Then, we may find $F\subseteq E$ by finding $j$ such that $\mu\left(E_j\right) > 0$, and taking $F = E_j\cap E$. Then, it must be the case that $0 < \mu\left(F\right) \leq \mu\left(E_j\right) < \infty$.
\end{solution}
\begin{exercise}[Exercise 14]
  If $\mu$ is a semifinite measure and $\mu\left(E\right) = \infty$, then for any $C > 0$ there exists $F\subseteq E$ such that $C < \mu\left(F\right) < \infty$.
\end{exercise}
\begin{solution}
  By the definition of a semifinite measure, there exists $F_1\subseteq E$ such that $0 < \mu\left(F_1\right) < \infty$. We let $\delta_1 = \mu\left(F_1\right)$.\newline

  Now, it must be the case that $\mu\left(E\setminus F_1\right) = \infty$, else $\infty = \mu\left(E\right) = \mu\left(E\setminus F_1\right) + \mu\left(F_1\right) < \infty$, a contradiction.\newline

  Hence, there exists $F_2\subseteq E\setminus F_1$ with $0 < \mu\left(F_2\right) < \infty$. We let $\delta_2 = \mu\left(F_2\right)$. Similarly, we find $\delta_n = \mu\left(F_n\right)$, where $F_n \subseteq E\setminus \left(F_1\cup\cdots\cup F_{n-1}\right)$.\newline

  Now, consider the series $\sum_{n\geq 1}\delta_n = \sum_{n\geq 1}\mu\left(F_n\right) = \mu\left(\bigsqcup_{n\geq 1}F_n\right)$. This series must diverge, as otherwise we would have $\mu\left(E\right) = \mu\left(\bigsqcup_{n\geq 1}F_n\right) < \infty$, which is yet again a contradiction.\newline

  Thus, for a given $C > 0$, we find $N$ so large such that $\sum_{n=1}^{N}\delta_n > C$. Then, $F = \bigsqcup_{n=1}^{N}F_n$ is our desired set.
\end{solution}
\begin{exercise}[Exercise 15]
  Let $\mu$ be a measure on $\left(X,\mathcal{M}\right)$. Define $\mu_0$ on $\mathcal{M}$ by $\mu_0\left(E\right) = \sup\set{\mu\left(F\right) | F\subseteq E\text{ and }\mu\left(F\right) < \infty}$.
  \begin{enumerate}[(a)]
    \item $\mu_0$ is a semifinite measure It is called the semifinite part of $\mu$.
    \item If $\mu$ is semifinite, then $\mu = \mu_0$.
    \item There is a measure $\nu$ on $\mathcal{M}$ which only assumes the values $0$ and $\infty$ such that $\mu = \mu_0 + \nu$.
  \end{enumerate}
\end{exercise}
\begin{solution}\hfill
  \begin{enumerate}[(a)]
    \item Let $E\in \mathcal{M}$ be such that $\mu_0\left(E\right) = \infty$. Suppose toward contradiction that $\mu_0$ is not semifinite. Then, for any $F\subseteq E$, it is the case that $\mu\left(F\right) = 0$ or $\mu\left(F\right) = \infty$, so it would then be the case that $\mu_0\left(E\right) = 0$, a contradiction.
    \item If $\mu\left(E\right) < \infty$, then $\mu_0\left(E\right) = \mu\left(E\right)$, as $E \subseteq E$ and $\mu\left(E\right) < \infty$.\newline

      If $\mu\left(E\right) = \infty$, then it is clear that $\mu_0\left(E\right) = \infty$, as for each $C > 0$ there is some $F\subseteq E$ such that $C < \mu\left(F\right) < \infty$.\newline

      Thus, $\mu = \mu_0$.
    \item We define the measure $\nu$ on $\mathcal{M}$ by taking $\nu(E) = 0$ whenever $\mu\left(E\right) < \infty$ and $\nu(E) = \infty$ whenever $\mu(E) = \infty$.
  \end{enumerate}
\end{solution}
\begin{exercise}
  Let $\left(X,\mathcal{M},\mu\right)$ be a measure space. A set $E\subseteq X$ is called locally measurable if $E\cap A \in \mathcal{M}$ for all $A\in \mathcal{M}$ such that $\mu\left(A\right) < \infty$. Let $\widetilde{\mathcal{M}}$ be the collection of all locally measurable sets.\newline

  It is obvious that $\mathcal{M}\subseteq \widetilde{\mathcal{M}}$. If $\mathcal{M} = \widetilde{\mathcal{M}}$, then $\mu$ is called saturated.
  \begin{enumerate}[(a)]
    \item If $\mu$ is $\sigma$-finite, then $\mu$ is saturated.
    \item $\widetilde{\mathcal{M}}$ is a $\sigma$-algebra.
    \item Define $\widetilde{\mu}$ on $\widetilde{\mathcal{M}}$ by $ \widetilde{\mu}\left(E\right) = \mu\left(E\right) $ if $E\in \mathcal{M}$ and $ \widetilde{\mu}\left(E\right) = \infty $ otherwise. Then, $\widetilde{\mu}$ is a saturated measure on $\widetilde{\mathcal{M}}$ called the saturation of $\mu$.
    \item If $\mu$ is complete, so is $\widetilde{\mu}$.
    \item Suppose that $\mu$ is semifinite. For $E\in \widetilde{\mathcal{M}}$, define $\underline{\mu}\left(E\right) = \sup\set{\mu\left(A\right) | A\in \mathcal{M}\text{ and }A\subseteq E}$. Then, $\underline{\mu}$ is a saturated measure on $\widetilde{\mathcal{M}}$ that extends $\mu$.
    \item Let $X_1$ and $X_2$ be disjoint uncountable sets, $X = X_1\sqcup X_2$, and $\mathcal{M}$ the $\sigma$-algebra of countable and cocountable sets in $X$. Let $\mu_0$ be the counting measure on $P\left(X_1\right)$ and define $\mu$ on $\mathcal{M}$ by $\mu\left(E\right) = \mu_0\left(E\cap X_1\right)$. Then, 
      \begin{itemize}
        \item $\mu$ is a measure on $\mathcal{M}$;
        \item $\widetilde{\mathcal{M}} = P(X)$;
        \item and $\widetilde{\mu}\neq \underline{\mu}$.
      \end{itemize}
  \end{enumerate}
\end{exercise}
\begin{solution}\hfill
  \begin{enumerate}[(a)]
    \item Let $\mu$ be $\sigma$-finite, and let $E\in \widetilde{\mathcal{M}}$. We know that $E\cap A\in \mathcal{M}$ for all $A\in \mathcal{M}$ with $\mu(A) < \infty$. In particular, we can select a disjoint collection $\set{A_j}_{j=1}^{\infty}$ such that $\mu\left(A_j\right) < \infty$ and $\bigsqcup_{j\geq 1}A_j = X$. Thus, since $E = X\cap E$, we must have $E\in \mathcal{M}$ as $E$ is locally measurable.
  \end{enumerate}
\end{solution}

\subsection{Section 1.4}%
\begin{definition}
  An outer measure on a nonempty set $X$ is a function $\mu^{\ast}\colon P(X) \rightarrow [0,\infty]$ such that
  \begin{itemize}
    \item $\mu^{\ast}\left(\emptyset\right) = 0$;
    \item $\mu^{\ast}\left(A\right)\leq \mu^{\ast}\left(B\right)$ if $A\subseteq B$;
    \item $\mu^{\ast}\left(\bigcup_{j\geq 1}A_j\right)\leq \sum_{j=1}^{\infty}\mu^{\ast}\left(A_j\right)$.
  \end{itemize}
\end{definition}
\begin{proposition}
  Let $\mathcal{E}\subseteq P(X)$, and $\rho\colon \mathcal{E}\rightarrow [0,\infty]$ be such that $\emptyset\in \mathcal{E}$, $X\in \mathcal{E}$, and $\rho\left(\emptyset\right) = 0$. For any $A\subseteq X$, define
  \begin{align*}
    \mu^{\ast}\left(A\right) &= \inf\set{\sum_{j\geq 1}\rho\left(E_j\right) | E_j\in \mathcal{E}\text{ and }A\subseteq \bigcup_{j\geq 1}E_j}.
  \end{align*}
  Then, $\mu^{\ast}$ is an outer measure.
\end{proposition}
\begin{proof}
  For any $A\subseteq X$, there exists $\set{E_j}_{j\geq 1}\subseteq \mathcal{E}$ such that $A\subseteq \bigcup_{j\geq 1}E_j$ (taking $E_j = X$). Clearly, $\mu^{\ast}\left(\emptyset \right) = \emptyset$.\newline

  Additionally, since $A\subseteq B$, we the infimum taken to define $\mu^{\ast}\left(A\right)$ includes the corresponding set in the definition of $\mu^{\ast}\left(B\right)$, so $\mu^{\ast}\left(A\right) \leq \mu^{\ast}\left(B\right)$.\newline

  Suppose $\set{A_j}_{j\geq 1}\subseteq P(X)$, and let $\ve > 0$. For each $j$, there exists $\set{E_{j,k}}_{k\geq 1}\subseteq \mathcal{E}$ such that $A_j\subseteq \bigcup_{k\geq 1}E_{j,k}$ and $\sum_{k\geq 1}\rho\left(E_{j,k}\right) \leq \mu^{\ast}\left(A_j\right) + \ve 2^{-j}$. Thus, if $A = \bigcup_{j\geq 1}A_j$, we have $A \subseteq \bigcup_{j,k\geq 1}E_{j,k}$, and $\sum_{j,k\geq 1}\rho\left(E_{j,k}\right)\leq \sum_{j\geq 1}\mu^{\ast}\left(A_j\right) + \ve$, meaning $\mu^{\ast}\left(A\right)\leq \sum_{j\geq 1}\mu^{\ast}\left(A_j\right) + \ve$. Sine this holds for all $\ve > 0$, we must have $\mu^{\ast}\left(\bigcup_{j\geq 1}A_j\right) \leq \sum_{j\geq 1}\mu^{\ast}\left(A_j\right)$.
\end{proof}
\begin{definition}
  If $\mu^{\ast}$ is an outer measure, a set $A\subseteq X$ is called $\mu^{\ast}$-measurable if
  \begin{align*}
    \mu^{\ast}\left(E\right) &= \mu^{\ast}\left(E\cap A\right) + \mu^{\ast}\left(E\cap A^{c}\right)
  \end{align*}
  for all $E\subseteq X$. In other words, $A$ is measurable if it serves as a well-behaved ``cookie cutter'' for any subset of $X$.\newline

  Note that it suffices to show that
  \begin{align*}
    \mu^{\ast}\left(E\right) \geq \mu^{\ast}\left(E\cap A\right) + \mu^{\ast}\left(E\cap A^{c}\right).
  \end{align*}
\end{definition}
\begin{definition}
  If $\mathcal{A}\subseteq P(X)$ is an algebra, a function $\mu_0\colon \mathcal{A}\rightarrow [0,\infty]$ is called a premeasure if $\mu_0\left( \emptyset \right) = 0$ and, for any sequence of disjoint sets $\set{A_j}_{j=1}^{\infty}$ in $\mathcal{A}$ such that $\bigsqcup_{j=1}^{\infty}A_j\in \mathcal{A}$, we have
  \begin{align*}
    \mu_0\left( \bigcup_{j=1}^{\infty}A_j \right) &= \sum_{j=1}^{\infty}\mu_0\left( A_j \right).
  \end{align*}
\end{definition}
A premeasure induces an outer measure on $X$ by
\begin{align*}
  \mu^{\ast}\left( E \right) &= \inf\set{\sum_{j=1}^{\infty}\mu_0\left( A_j \right) | A_j\in \mathcal{A},E\subseteq \bigcup_{j=1}^{\infty}A_j}.
\end{align*}
\begin{exercise}[Exercise 17]
  If $\mu^{\ast}$ is an outer measure on $X$ and $\set{A_j}_{j=1}^{\infty}$ is a sequence of disjoint $\mu^{\ast}$-measurable sets, then $\mu^{\ast}\left( E\cap \left( \bigsqcup_{j=1}^{n}A_j \right) \right) = \sum_{j=1}^{\infty}\mu^{\ast}\left( E\cap A_j \right)$.
\end{exercise}
\begin{solution}
  By the definition of measurability, we have
  \begin{align*}
    \mu\left( E\cap \left( \bigsqcup_{j=1}^{\infty} A_j \right) \right) &= \mu\left( \left( E\cap \left( \bigsqcup_{j=1}^{\infty}A_j \right) \right)\cap A_1 \right) + \mu\left( \left( E\cap\left( \bigsqcup_{j=1}^{\infty}A_j \right) \right)\cap A_1^{c} \right)\\
                                                                        &= \mu\left( E\cap A_1 \right) + \mu\left( E\cap \left( \bigsqcup_{j=2}^{\infty}A_j \right) \right).
  \end{align*}
  Continuing in this pattern, we get
  \begin{align*}
    \mu\left( E\cap \left( \bigsqcup_{j=1}^{\infty}A_j \right) \right) &= \sum_{j=1}^{\infty} \mu\left( E\cap A_j \right).
  \end{align*}
\end{solution}

\begin{exercise}[Exercise 18]
  Let $\mathcal{A}\subseteq P(X)$ be an algebra, $\mathcal{A}_{\sigma}$ the collection of countable unions of sets in $\mathcal{A}$, and $\mathcal{A}_{\sigma\delta}$ the collection of countable intersections in $\mathcal{A}_{\sigma}$. Let $\mu_0$ be a premeasure on $\mathcal{A}$, and let $\mu^{\ast}$ be the induced outer measure.
  \begin{enumerate}[(a)]
    \item For any $E\subseteq X$ and $\ve > 0$, there exists $A\in \mathcal{A}_{\sigma}$ with $E\subseteq A$, $\mu^{\ast}\left( A \right)\leq \mu^{\ast}\left( E \right) + \ve$.
    \item If $\mu^{\ast}\left( E \right) < \infty$, then $E$ is $\mu^{\ast}$-measurable if and only if there exists $B\in \mathcal{A}_{\sigma\delta}$ with $E\subseteq B$ and $\mu^{\ast}\left( B\setminus E \right) = 0$.
    \item If $\mu_0$ is $\sigma$-finite, then the restriction $\mu^{\ast}\left( E \right)< \infty$ in (b) is superfluous.
  \end{enumerate}
\end{exercise}
\begin{solution}\hfill
  \begin{enumerate}[(a)]
    \item We know that
      \begin{align*}
        \mu^{\ast}\left( E \right) &= \inf\set{\sum_{j=1}^{\infty}\mu_0\left( A_j \right) | A_j\in \mathcal{A},E\subseteq \bigcup_{j=1}^{\infty}A_j},
      \end{align*}
      meaning that, by the definition of infimum, for any $\ve > 0$, there exists some sequence $\set{A_j}_{j=1}^{\infty}$ in $\mathcal{A}$ such that
      \begin{align*}
        \mu_0\left( \bigcup_{j=1}^{\infty}A_j \right) &\leq \mu^{\ast}\left( E \right) + \ve.
      \end{align*}
      Defining $A = \bigcup_{j=1}^{\infty}A_j$, we have $A\in \mathcal{A}_{\sigma}$.
    \item Let $\mu^{\ast}\left( E \right) < \infty$.\newline

      Suppose $E$ is measurable. Then, for any $T\subseteq X$, we have
      \begin{align*}
        \mu^{\ast}\left( T \right) &= \mu^{\ast}\left( E\cap T \right) + \mu^{\ast}\left( E^{c}\cap T \right). 
      \end{align*}
  \end{enumerate}
\end{solution}

\section{Chapter 3}%
\subsection{Section 3.5}%
\begin{definition}
  A function $F\colon \R\rightarrow \C$ is called \textit{absolutely continuous} if, for any $\ve > 0$, there is $\delta > 0$ such that for any finite set of disjoint open intervals $\set{\left( a_j,b_j \right)}_{j=1}^{N}$ with
  \begin{align*}
    \sum_{j=1}^{N} \left( b_j-a_j \right) &< \delta,
  \end{align*}
  we have
  \begin{align*}
    \sum_{j=1}^{N}\left\vert F\left( b_j \right) - F\left( a_j \right) \right\vert &< \ve.
  \end{align*}
\end{definition}
\begin{remark}
All absolutely continuous functions are uniformly continuous.
\end{remark}

\begin{exercise}[Exercise 36]
  Let $G$ be a continuous, increasing function on $\left[ a,b \right]$, and let $G(a)=c,G(b)=d$.
  \begin{enumerate}[(a)]
    \item If $E\subseteq \left[ c,d \right]$ is a Borel set, then $m\left( E \right) = \mu_G\left( G^{-1}(E) \right)$.
    \item If $f$ is a Borel-measurable and integrable function on $\left[ c,d \right]$, then 
      \begin{align*}
        \int_{c}^{d} f(y)\:dy &= \int_{a}^{b} f\left( G(x) \right)\:dG(x).
      \end{align*}
      If $G$ is absolutely continuous, then
      \begin{align*}
        \int_{a}^{b} f(y)\:dy &= \int_{a}^{b} f\left( G(x) \right)G'(x)\:dx
      \end{align*}
    \item The validity of (b) may fail if $G$ is merely right-continuous.
  \end{enumerate}
\end{exercise}
\begin{solution}\hfill
  \begin{enumerate}[(a)]
    \item We may start by assuming that $E$ is a closed subinterval of $\left[ c,d \right]$, which we call $\left[ \alpha,\beta \right]$, with $\alpha \geq c$ and $\beta \leq d$. Then, $m\left( E \right) = \beta-\alpha$, and 
      \begin{align*}
        \mu_G\left( G^{-1}\left[ \alpha,\beta \right] \right) &= G\left( G^{-1}\left( \beta \right) \right) - G\left( G^{-1}\left( \alpha \right) \right)\\
                                                              &= \beta-\alpha.
      \end{align*}
      Using countability, we may apply this to all Borel sets.
    \item We start with the borel set $E\subseteq [c,d]$, and its corresponding indicator function, giving
      \begin{align*}
        \int_{c}^{d} \1_E(y)\:dy &= m\left( E \right)\\
                                 &= \mu_G\left( G^{-1}\left( E \right) \right)\\
                                 &= \int_{a}^{b} \1_E\left( G(x) \right)\:dG(x).
      \end{align*}
      By linearity, this applies to simple functions, and by Monotone Convergence, to all integrable $f\colon [c,d]\rightarrow \C$.\newline

      Furthermore, by Lebesgue differentiation and the Lebesgue--Radon--Nikodym theorem, we also have
      \begin{align*}
        \int_{a}^{b} f\left( G(x) \right)\:dG(x) &= \int_{a}^{b} f\left( G(x) \right)\diff{G}{x}\:dx.
      \end{align*}
  \end{enumerate}
\end{solution}
\begin{exercise}[Exercise 37]
  Let $F\colon \R\rightarrow \C$. There is a constant $M$ such that $\left\vert F(x)-F(y) \right\vert \leq M\left\vert x-y \right\vert$ for all $x,y\in \R$ (i.e., $F$ is Lipschitz) if and only if $F$ is absolutely continuous and $\left\vert F' \right\vert \leq M$ almost everywhere.
\end{exercise}
\begin{solution}
  Let $F$ be Lipschitz. Then, setting $\delta = \ve / M$, we see that $F$ is absolutely continuous, and since
  \begin{align*}
    \sup_{x,y\in\R} \frac{\left\vert F(y)-F(x) \right\vert}{\left\vert y-x \right\vert} \leq M,
  \end{align*}
  with the left-hand side including $\left\vert F'(x) \right\vert$, we have that $\left\vert F' \right\vert \leq M$ almost everywhere.\newline

Meanwhile, if $f$ is absolutely continuous with bounded derivative, then if $M = \esssup_{x\in\R} \left\vert f(x) \right\vert$, we have
\begin{align*}
  \left\vert f(y) - f(x) \right\vert &= \left\vert \int_{x}^{y} f'(t)\:dt \right\vert\\
                                     &\leq \int_{x}^{y} \left\vert f'(t) \right\vert\:dt\\
                                     &\leq \int_{x}^{y} M\:dt\\
                                     &= M\left\vert y-x \right\vert,
\end{align*}
so that $f$ is Lipschitz.
\end{solution}


\end{document}
