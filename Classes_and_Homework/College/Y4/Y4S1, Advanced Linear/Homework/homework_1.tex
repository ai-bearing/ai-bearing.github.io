\documentclass[10pt]{mypackage}

% sans serif font:
%\usepackage{cmbright}
%\usepackage{sfmath}
%\usepackage{bbold} %better blackboard bold

%serif font + different blackboard bold for serif font
\usepackage{newpxtext,eulerpx}
\renewcommand*{\mathbb}[1]{\varmathbb{#1}}

%\fancyhf{}
%\rhead{Avinash Iyer}
%\lhead{Advanced Linear Algebra: Homework 1}
%
\setcounter{secnumdepth}{0}

\begin{document}
\begin{center}
  \large
  \bfseries
  Math 395: Homework 1\\
  Name: Avinash Iyer\\
  Due: 09/05/2024\\
  Collaborators: Gianluca Crescenzo, Noah Smith, Carly Venenciano
\end{center}
\RaggedRight
%\section{Problem 1}%
%\begin{problem}
%  Define
%  \begin{align*}
%    \mathfrak{sl}_n\left(\Q\right) &= \set{X = \left(x_{i,j}\right)\in \text{Mat}_n\left(\Q\right)\left| \text{Tr}\left(X\right) = \sum_{i=1}^{n}x_{i,i} = 0\right.}.
%  \end{align*}
%  Show that $\mathfrak{sl}_n\left(\Q\right)$ is a $\Q$-vector space.
%\end{problem}
%\begin{solution}
%  To show that $\mathfrak{sl}_n\left(\Q\right)$ is a $\Q$-vector space, we show each of the necessary qualities of a $\Q$-vector space.
%  \begin{enumerate}[(1)]
%    \item We let $\mathbb{0}$ be the matrix defined by $\mathbb{0}_{i,j} = 0$ for all $0 \leq i,j \leq n$. It is clear that $\mathbb{0} \in \mathfrak{sl}_{n}\left(\Q\right)$, as $\mathbb{0}_{i,i} = 0$ for all $i$, meaning $\sum_{i=1}^{n}\mathbb{0}_{i,i} = 0$.\newline
%
%      For some $a = \left(a_{i,j}\right)\in \mathfrak{sl}_{n}\left(\Q\right)$, it is then the case that
%      \begin{align*}
%        \left(\mathbb{0} + a\right)_{i,j} &= \mathbb{0}_{i,j}+ a_{i,j}\\
%                                          &= 0 + a_{i,j}\\
%                                          &= a_{i,j},
%      \end{align*}
%      meaning that for each $0 \leq i,j \leq n$, $\left(\mathbb{0} + a\right)_{i,j} = a_{i,j}$, so $\mathbb{0} + a = a$.
%  \end{enumerate}
%\end{solution}
\section{Problem 3}%
\begin{problem}
  Let $V$ be an $\mathbb{F}$-vector space.
  \begin{enumerate}[(a)]
    \item Prove that an arbitrary intersection of subspaces of $V$ is again a subspace of $V$.
    \item Prove that the union of two subspaces of $V$ is a subspace of $V$ if and only if one of the subspaces is contained in the other.
  \end{enumerate}
\end{problem}
\begin{solution}\hfill
  \begin{enumerate}[(a)]
    \item Let $U,W\subseteq V$ be subspaces. Since $U$ and $W$ are subspaces, $0_V \in U$ and $0_V \in W$, meaning $U\cap W$ is nonempty.\newline

      Let $u,w\in U\cap W$, and let $\alpha \in \mathbb{F}$. Then, since $u\in U$ and $w\in U$, it is the case that $u + \alpha w \in U$. Similarly, since $u\in W$ and $w\in W$, it is the case that $u + \alpha w \in W$. Thus, $u + \alpha w\in U\cap W$, meaning $U\cap W$ is a subspace.\newline

      Having shown the base case, we let $\bigcap_{k=1}^{N} U_k$ be an intersection of subspaces $U_k$. By the inductive hypothesis, we have $W = \bigcap_{k=1}^{N}U_k$, where $W$ is a subspace. 
    \item Let $U,W\subseteq V$ be subspaces.\newline

      In the reverse direction, if, without loss of generality, $U\subseteq W$, then it is the case that $U\cup W = W$, meaning that $U\cup W$ is a subspace of $V$.\newline

      In the forward direction, suppose toward contradiction that there exist subspaces $U,W\in V$ such that $U\nsubseteq W$ and $W\nsubseteq U$, but $U\cup W$ is a subspace of $V$. Since $U\nsubseteq W$ and $W\nsubseteq U$, there exist non-trivial vectors $w\in W\setminus U$ and $u\in U\setminus W$. Since $w + u\in W\cup U$, it is the case that $w + u$ is contained either in $U$ or in $W$. If $w + u\in U$, then $\left(w + u\right)-u \in U$ (as $u\in U$ and $U$ is a subspace), meaning $w\in U$, which is a contradiction. Similarly, if $w + u \in W$, then $\left(w + u\right)-w \in W$, or $u\in W$, which is yet again a contradiction.\newline

      Thus, it must be the case that $W\subseteq U$ or $U\subseteq W$.
  \end{enumerate}
\end{solution}
\section{Problem 4}%
\begin{problem}
  Let $T\in \text{Hom}_{\mathbb{F}}\left(\mathbb{F},\mathbb{F}\right)$. Prove there exists $\alpha \in \mathbb{F}$ such that $T(v) = \alpha v$ for all $v\in \mathbb{F}$.
\end{problem}
\begin{solution}
  Since $\Dim_{\mathbb{F}}\left(\mathbb{F}\right) = 1$, we know that the basis of $\mathbb{F}$ is $\set{\beta}$ for some $\beta\in \F$. For $v\in \mathbb{F}$, it is then the case that $v$ is a linear combination of the basis of $\mathbb{F}$ over $\mathbb{F}$, meaning $v = v_0 \beta$ for some $v_0 \in \F$, implying $\beta = \left(v_0^{-1}\right)v$.\newline

  Considering a linear transformation $T(v)$, we have
  \begin{align*}
    T\left(v\right) &= T\left(v_0 \beta\right).
  \end{align*}
  Substituting $\beta = v_0^{-1}v$, and using the commutativity and associativity of multiplication under $\mathbb{F}$, we have
  \begin{align*}
    T\left(v\right) &= T\left(v_0\left(v_0^{-1} v\right)\right).
    \intertext{Using the fact that $T$ is linear and $v\in \F$, we have}
                    &= vT\left(v_0^{-1}v_0\right)\\
                    &= vT\left(1\right).
  \end{align*}
  Thus, $\alpha = T\left(1\right)$.
\end{solution}
\section{Problem 6}%
\begin{problem}
  Let $V$ be an $\mathbb{F}$-vector space. Prove that if $\set{v_1,\dots,v_n}$ is linearly independent, then so is the set $\set{v_1-v_2,v_2-v_3,\dots,v_{n-1}-v_{n},v_n}$.
\end{problem}
\begin{solution}
  To prove that $\set{v_1-v_2,v_2-v_3,\dots,v_{n-1}-v_{n},v_n}$ is linearly independent, we consider the sum
  \begin{align*}
    a_1\left(v_1 - v_2\right) + a_2\left(v_2 - v_3\right) + \cdots + a_{n-1}\left(v_{n-1} - v_n\right) + a_nv_n,
  \end{align*}
  and show that this sum equals zero if and only if $a_i = 0$ for each $i$. Rearranging the sum, we have
  \begin{align*}
    a_1v_1 + \left(a_2 - a_1\right)v_2 + \cdots + \left(a_{n-1} - a_{n-2}\right)v_{n-1} + \left(a_{n}-a_{n-1}\right)v_n.
  \end{align*}
  Since the set $\set{v_1,\dots,v_n}$ are linearly independent, this linear combination equals $0_V$ if and only if $a_1 = \left(a_2 - a_1\right) = \cdots = a_{n} - a_{n-1} = 0$. In particular, since $a_1 = 0$, it must be the case that $a_2 = 0$, $a_3 = 0$, and so on.\newline

  Thus, $\set{v_{1} - v_2,v_2 - v_3,\dots,v_{n-1}-v_n,v_n}$ are linearly independent.
\end{solution}
%\section{Problem 9}%
%\begin{problem}
%  Let $V$ be a finite-dimensional vector space and $T\in \text{Hom}_{\mathbb{F}}\left(V,V\right)$ with $T^2 = T$.
%  \begin{enumerate}[(a)]
%    \item Prove that $\text{im}\left(T\right)\cap \ker\left(T\right) = \set{0}$.
%    \item Prove that $V = \text{im}(T)\oplus \ker(T)$.
%    \item Let $V = \F^n$. Prove that there is a basis of $V$ such that the matrix of $T$ with respect to this basis is a diagonal matrix whose entries are all $0$ or $1$.
%  \end{enumerate}
%\end{problem}
%
\section{Problem 13}%
\begin{problem}
  Let $p$ be a prime and $V$ a dimension $n$ vector space over $\mathbb{F}_p$. Show there are
  \begin{align*}
    \left(p^n-1\right)\left(p^n - p\right)\left(p^n - p^2\right)\cdots \left(p^n - p^{n-1}\right)
  \end{align*}
  distinct bases of $V$.
\end{problem}
\begin{solution}
  We begin by constructing our basis by selecting $v_1 \in V\setminus \{0_V\}$. Since $V$ is a dimension $n$ vector space over $\F_p$, it is the case that there are $p^n - 1$ options to select $v_1$.\newline

  To select $v_2$, we find $v_2\in V\setminus \Span\set{v_1}$; since $\Card\left(\Span\set{v_1}\right) = p$, there are $p^n - p$ vectors that are linearly independent of $v_1$.\newline

  To select $v_3$, we find $v_3 \in V\setminus \Span\set{v_1,v_2}$; since $\Card\left(\Span\set{v_1,v_2}\right) = p^2$, there are $p^n - p^2$ vectors that are linearly independent of $\set{v_1,v_2}$.\newline

  Continuing down the chain, we find that to select $v_i$, one can select from $p^n - p^{i-1}$ vectors that are linearly independent of $\set{v_1,\dots,v_{i-1}}$.\newline

  Thus, the number of distinct bases of $V$ is
  \begin{align*}
    \prod_{i=0}^{n-1}\left(p^n - p^{i}\right).
  \end{align*}
\end{solution}
\end{document}
