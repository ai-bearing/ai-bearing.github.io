\documentclass[10pt]{mypackage}

% sans serif font:
%\usepackage{cmbright}
%\usepackage{sfmath}
%\usepackage{bbold} %better blackboard bold

%serif font + different blackboard bold for serif font
\usepackage{newpxtext,eulerpx}
\renewcommand*{\mathbb}[1]{\varmathbb{#1}}
\renewcommand*{\hbar}{\hslash}

%\pagestyle{fancy} %better headers
%\fancyhf{}
%\rhead{Avinash Iyer}
%\lhead{}

\setcounter{secnumdepth}{0}

\begin{document}
\RaggedRight
\begin{center}
  \textbf{Math 395: Homework 8}\break
  \textbf{Due: November 26, 2024}\break
  \textbf{Name: Avinash Iyer}\break
  \textbf{Collaborators: Carly Venenciano, Noah Smith, Chris Swanson}
\end{center}
\section{Problem 1}%
\begin{problem}
  Let $V_1,V_2$ be subspaces of $V$. Show that $V = V_1\perp V_2$ if
  \begin{enumerate}[(i)]
    \item $V = V_1\oplus V_2$;
    \item given any $v,v'\in V$, when we write $v = v_1 + v_2$ and $v' = v_1' + v_2'$, for $v_i,v_i'\in V_i$, we have
      \begin{align*}
        \varphi\left(v,v'\right) &= \varphi_1\left(v_1,v_1'\right) + \varphi_2\left(v_2,v_2'\right),
      \end{align*}
      where $\varphi_i = \varphi|_{V_i\times V_i}$.
  \end{enumerate}
\end{problem}
\begin{solution}
  By condition (i), since $V = V_1 \oplus V_2$, it is the case that $V_1 + V_2 = V$ and $V_1\cap V_2 = \set{0}$, meaning that for any $v\in V$, we can write $v = v_1 + v_2$ for unique $v_1\in V_1$ and $v_2\in V_2$. Thus, we must show that $\varphi\left(v,w\right) = 0$ for any $v\in V_1$ and $w\in V_2$.\newline

  From condition (ii), we know that
  \begin{align*}
    \varphi\left(v,v'\right) &= \varphi\left(v_1 + v_2,v_1' + v_2'\right)\\
                             &= \varphi\left(v_1,v_1'\right) + \varphi\left(v_1,v_2'\right) + \varphi\left(v_2,v_1'\right) + \varphi\left(v_2,v_2'\right)\\
                             &= \varphi_1\left(v_1,v_1'\right) + \varphi_2\left(v_2,v_2'\right).
  \end{align*}
  Since, by definition, we have $\varphi_i = \varphi$ for $v_i\in V_i$, we have $\varphi\left(v_1,v_1'\right) = \varphi_1\left(v_1,v_1'\right)$ and $\varphi\left(v_2,v_2'\right) = \varphi_2\left(v_2,v_2'\right)$. Thus, by this equality, we have
  \begin{align*}
    \varphi\left(v_1,v_2'\right) + \varphi\left(v_2,v_1'\right) &= 0.
  \end{align*}
  Considering $\varphi\left(v_1,v_2'\right)$, we uniquely decompose $v_1 = v_1 + 0$, where $0\in V_2$, and $v_2' = 0 + v_2'$, where $0\in V_1$, yielding
  \begin{align*}
    \varphi\left(v_1,v_2'\right) &= \varphi_1\left(v_1,0\right) + \varphi_2\left(0,v_2'\right)\\
                                 &= 0.
  \end{align*}
  Similarly, we have $\varphi\left(v_2,v_1'\right) = 0$, meaning that for any $v\in V_1$ and $w\in V_2$, we must have
  \begin{align*}
    \varphi\left(v,w\right) &= 0.
  \end{align*}
  Thus, $V_1$ and $V_2$ are orthogonal complements, yielding $V = V_1 \oplus V_2$.
\end{solution}

\section{Problem 2}%
\begin{problem}
  Let $T \in \Hom_{F}\left(V,V\right)$, and let $\varphi$ be a bilinear form on $V$. Prove that $\psi\left(v,w\right) = \varphi\left(T(v),w\right)$ is a bilinear form on $V$.
\end{problem}
\begin{solution}
  Let $v,v_1,v_2,w,w_1,w_2\in V$ and $\alpha\in F$. Then,
  \begin{align*}
    \psi\left(\alpha v_1 + v_2,w\right) &= \varphi\left(T\left(\alpha v_1 + v_2\right),w\right)\\
                                        &= \varphi\left(\alpha T\left(v_1\right) + T\left(v_2\right),w\right)\\
                                        &= \alpha\varphi\left(T\left(v_1\right),w\right) + \varphi\left(T\left(v_2\right),w\right)\\
                                        &= \alpha \psi\left(v_1,w\right) + \psi\left(v_2,w\right)\\
                                        \\
    \psi\left(v,\alpha w_1 + w_2\right) &= \varphi\left(T\left(v\right),\alpha w_1 + w_2 \right)\\
                                        &= \alpha\varphi\left(T\left(v\right),w_1\right) + \varphi\left(w_2\right)\\
                                        &= \alpha \varphi\left(T\left(v\right),w_1\right) + \varphi\left(T\left(v\right),w_2\right)\\
                                        &= \alpha \psi\left(v,w_1\right) + \psi\left(v,w_2\right).
  \end{align*}
  Thus, $\psi$ is a bilinear form.
\end{solution}
\section{Problem 5}%
\begin{problem}
  Let $V = \R^2$, and set $\varphi\left(\left(x_1,y_1\right),\left(x_2,y_2\right)\right) = x_1x_2$.
  \begin{enumerate}[(a)]
    \item Show this is a bilinear form. Give a matrix representing this form. Is this form nondegenerate?
    \item Let $W = \Span_{\R}\left(e_1\right)$, where $e_1$ is the standard basis element. Show that $V = W \perp W^{\perp}$.
    \item Calculate $\left(W^{\perp}\right)^{\perp}$.
  \end{enumerate}
\end{problem}
\begin{solution}\hfill
  \begin{enumerate}[(a)]
    \item We have, for $\left(x_1,y_1\right),\left(x_2,y_2\right),\left(x_3,y_3\right)\in \R^2$ and $\alpha \in \R$,
      \begin{align*}
        \varphi\left(\alpha\left(x_1,y_1\right) + \left(x_2,y_2\right),\left(x_3,y_3\right)\right) &= \varphi\left(\left(\alpha x_1 + x_2,y_1 + y_2\right),\left(x_3,y_3\right)\right)\\
                                                                                                   &= \left(\alpha x_1 + x_2\right)\left(x_3\right)\\
                                                                                                   &= \alpha \left(x_1x_3\right) + \left(x_2x_3\right)\\
                                                                                                   &= \alpha \varphi\left(\left(x_1,y_1\right),\left(x_3,y_3\right)\right) + \varphi\left(\left(x_2,y_2\right),\left(x_3,y_3\right)\right)\\
                                                                                                   \\
        \varphi\left(\left(x_1,y_1\right),\alpha\left(x_2,y_2\right) + \left(x_3,y_3\right)\right) &= \varphi\left(\left(x_1,y_1\right),\left(\alpha x_2 + x_3,y_2 + y_3\right)\right)\\
                                                                                                   &= x_1\left(\alpha x_2 + x_3\right)\\
                                                                                                   &= \alpha \left(x_1 x_2\right) + \left(x_1x_3\right)\\
                                                                                                   &= \alpha \varphi\left(\left(x_1,y_1\right),\left(x_2,y_2\right)\right) + \varphi\left(\left(x_1,y_1\right),\left(x_3,y_3\right)\right).
      \end{align*}
      Using the basis $\mathcal{B}= \set{(1,0),(0,1)}$, where $e_1 = (1,0)$ and $e_2 = (0,1)$, we have the matrix representation of
      \begin{align*}
        \left[\varphi\right]_{\mathcal{B}} &= \begin{pmatrix}1 & 0 \\ 0 & 0\end{pmatrix}.
      \end{align*}
      This is a degenerate bilinear form, since, for instance, taking $\left(x_1,y_1\right) = \left(0,5\right)$ and $\left(x_2,y_2\right) = (1,8)$, we have
      \begin{align*}
        \varphi\left(\left(x_1,y_1\right),\left(x_2,y_2\right)\right) &= 0,
      \end{align*}
      despite $\left(x_1,y_1\right),\left(x_2,y_2\right)\neq \left(0,0\right)$.
    \item Letting $W = \set{\alpha e_1 | \alpha \in \R}$, we see that for any $\left(x_2,y_2\right)\in \R^2$, that
      \begin{align*}
        \varphi\left(\left(\alpha,0\right),\left(x_2,y_2\right)\right) &= \alpha x_2,
      \end{align*}
      which equals zero whenever $\alpha = 0$ or $x_2 = 0$. Since we can select $\alpha \neq 0$, if we want $\left(x_2,y_2\right)\in W^{\perp}$, we need $x_2 = 0$. Thus, $W^{\perp} = \Span\left(e_2\right)$.\newline

      Additionally, since $W$ and $W^{\perp}$ are subspaces, $W\cap W^{\perp} = \set{0}$, and for any $v = \left(x_1,y_1\right)\in \R$, we have $\left(x_1,y_1\right) = \left(x_1,0\right) + \left(0,y_2\right)\in W_1 + W_2$, we have that $W \oplus W^{\perp} = V$.\newline

      Therefore, we must have $W_1 \perp W_2 = V$.
    \item We know that $W^{\perp} = \Span\left(e_2\right)$. Thus, we see that $\left(W^{\perp}\right)^{\perp}$ is the set of all $\left(x,y\right)\in \R^2$ such that
      \begin{align*}
        \varphi\left(\left(x,y\right),\left(0,\alpha\right)\right) &= 0.
      \end{align*}
      Since this holds for all $\left(x,y\right)\in \R^2$, we have that $\left(W^{\perp}\right)^{\perp} = V$.
  \end{enumerate}
\end{solution}
\section{Exercise}%
\begin{problem}
  If $\operatorname{char}(F) = 2$, show that $\varphi\left(v,v\right) = 0$ is a redundant condition provided $\varphi\left(w,v\right)=-\varphi\left(v,w\right)$ for all $v,w\in V$.
\end{problem}
\begin{solution}
  Since $\varphi\left(v,w\right) = -\varphi\left(w,v\right)$ for all $v,w\in V$, this applies in particular for $v =w$. Thus, we have
  \begin{align*}
    \varphi\left(v,v\right) &= -\varphi\left(v,v\right)\\
    2\varphi\left(v,v\right) &= 0\label{ref1}\tag*{(\textasteriskcentered)}\\
    \varphi\left(v,v\right) &= 0;
  \end{align*}
  where we used the property that $\operatorname{char}\left(F\right)\neq 2$ to move from the line in \ref{ref1} to the final line.
\end{solution}

\end{document}
