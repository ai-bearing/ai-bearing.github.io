\documentclass[12pt]{mypackage}

% sans serif font:
%\usepackage{cmbright}
%\usepackage{sfmath}
%\usepackage{bbold} %better blackboard bold

%serif font + different blackboard bold for serif font
\usepackage{newpxtext,eulerpx}
\renewcommand*{\mathbb}[1]{\varmathbb{#1}}
\renewcommand*{\hbar}{\hslash}

%\pagestyle{fancy} %better headers
%\fancyhf{}
%\rhead{Avinash Iyer}
%\lhead{}

\setcounter{secnumdepth}{0}

\begin{document}
\RaggedRight
\begin{center}
  \bfseries
  Math 395: Homework 2\\
  Name: Avinash Iyer\\
  Due: 09/24/2024\\
  Collaborators: Noah Smith, Gianluca Crescenzo, Carly Venenciano, Timothy Rainone, Clarissa Ly, Ben Langer Weida
\end{center}
\section{Exercise 1}%
\begin{problem}\hfill
  \begin{enumerate}[(1)]
    \item Let $\mathcal{A}$ be a basis of $U$, $\mathcal{B}$ be a basis of $V$, and $\mathcal{C}$ be a basis of $W$. Let $S\in \Hom_{\F}\left(U,V\right)$ and $T\in \Hom_{\F}\left(V,W\right)$. Show that
      \begin{align*}
        \left[T\circ S\right]_{\mathcal{A}}^{\mathcal{C}} &= \left[T\right]_{\mathcal{B}}^{\mathcal{C}}\left[S\right]_{\mathcal{A}}^{\mathcal{B}}.
      \end{align*}
    \item We know that, given $A\in \Mat_{m,p}\left(\F\right)$ and $B\in \Mat_{n,m}\left(\F\right)$, we have corresponding $T_A$ and $T_B$ linear maps. Show that you recover the definition of matrix multiplication by using part (1) to define matrix multiplication.
  \end{enumerate}
\end{problem}
\begin{solution}\hfill
  \begin{enumerate}[(1)]
    \item Assuming that $U,V,W$ are $\F$-vector spaces with dimensions of $n$, $m$, and $p$ respectively, we can see that the following diagram commutes.
      \begin{center}
        % https://tikzcd.yichuanshen.de/#N4Igdg9gJgpgziAXAbVABwnAlgFyxMJZABgBpiBdUkANwEMAbAVxiRAFUQBfU9TXfIRQBGclVqMWbAGrdeIDNjwEiAJjHV6zVohAB1OXyWCiZYeK1TdAHWsAxAHqEeRgSpGlzmyTpC3HALaGCvzKQsjqXhLabP4OaNziMFAA5vBEoABmAE4QQYhkIDgQSADM3jG6ACoA+sC2AXQ4ABYAxozAAIJcXCDUDHQARjAMAAqhJrpYYNiwwTl5SKJFJYgALBVWILX11o0t7QzAAEI9fSADw2MT7iDTs6wuIAv56itIAKybvjsNTW0dADCZ36QxG42Mt3uWDmTxeSEKxSW3zYAGV5rl8sskYg3pYfhjFohyu91qCrhC3EIQNksClmjhzvjYtYGDBMjhkKjbLT6TgKHU-gcOt0uA5dvsAUdTr04ZikBtSV8LmDrpDqbyGUyfCy2RzkFUeXSGQKJf9DiceuKhVLgMDZRQuEA
\begin{tikzcd}
U \arrow[d, "T_{\mathcal{A}}" description] \arrow[r, "S"]       & V \arrow[d, "T_{\mathcal{B}}" description] \arrow[r, "T"]       & W \arrow[d, "T_{\mathcal{C}}" description] \\
\F^n \arrow[r, "{\left[S\right]_{\mathcal{A}}^{\mathcal{B}}}"'] & \F^m \arrow[r, "{\left[T\right]_{\mathcal{B}}^{\mathcal{C}}}"'] & \F^p                                      
\end{tikzcd}
      \end{center}
      Therefore, it must be the case that $\left[T\circ S\right]_{\mathcal{A}}^{\mathcal{C}} = \left[T\right]_{\mathcal{B}}^{\mathcal{C}} \left[S\right]_{\mathcal{A}}^{\mathcal{B}}$.
    \item For $\left(a_{ij}\right) = A \in \Mat_{m,p}\left(\F\right)$ and $\left(b_{ij}\right) = B \in \Mat_{n,m}\left(\F\right)$, we have
      \begin{align*}
        T_B\left(e_j\right) &= \sum_{k=1}^{m}b_{kj}f_k\\
        T_A\left(f_k\right) &= \sum_{i=1}^{p}a_{ik}g_i.
      \end{align*}
      In particular, since we know that
      \begin{align*}
        \left[T_A\circ T_B\right]_{\mathcal{A}}^{\mathcal{C}} &= \left[T_A\right]_{\mathcal{B}}^{\mathcal{C}}\left[T_B\right]_{\mathcal{A}}^{\mathcal{B}},
      \end{align*}
      we have
      \begin{align*}
        \left[T_A\circ T_B\right]_{\mathcal{A}}^{\mathcal{C}} \left(e_j\right) &= \sum_{i=1}^{p}c_{ij}g_{i}\\
                                                                               &= \left[T_A\right]_{\mathcal{B}}^{\mathcal{C}}\left[T_B\right]_{\mathcal{A}}^{\mathcal{B}}\left(e_j\right),\\
                                                                               &= \left[T_A\right]_{\mathcal{B}}^{\mathcal{C}}\left(\sum_{k=1}^{m}b_{kj}f_k\right)\\
                                                                               &= \sum_{i=1}^{p}\underbrace{\left(\sum_{k=1}^{m}a_{ik}b_{kj}\right)}_{c_{ij}}g_i.
      \end{align*}
      Thus, we recover the definition of matrix multiplication.
  \end{enumerate}
\end{solution}
\section{Exercise 2}%
\begin{problem}
  Let $A_1,A_2\in \Mat_{m,n}\left(\F\right)$, $c\in \F$. Use the definition of the transpose to show
  \begin{align*}
    \left(A_1 + A_2\right)^{T} &= A_1^T + A_2^T\\
    \left(cA_1\right)^T &= cA_1^T.
  \end{align*}
\end{problem}
\begin{solution}
  For bases $\mathcal{E}_n = \set{e_1,\dots,e_n}$ and $\mathcal{F}_m = \set{f_1,\dots,f_m}$ for $\F^n$ and $\F^m$, and corresponding linear transformations $T_{A_1}$ and $T_{A_2}$, we have
  \begin{align*}
    \left(A_1 + A_2\right)^{T} &= \left[\left(T_{A_1} + T_{A_2}\right)'\right]_{\mathcal{F}_{m}'}^{\mathcal{E}_{n}'}\\
                               &= \left[T_{A_1}' + T_{A_2}'\right]_{\mathcal{F}_m'}^{\mathcal{E}_n'}\\
                               &= \left[T_{A_1}'\right]_{\mathcal{F}_m'}^{\mathcal{E}_n'} + \left[T_{A_2}'\right]_{\mathcal{F}_m'}^{\mathcal{E}_n'}\\
                               &= A_{1}^{T} + A_2^T\\
                               \\
    \left(cA_1\right)^{T} &+ \left[\left(T_{cA_1}\right)'\right]_{\mathcal{F}_m'}^{\mathcal{E}_n'}\\
                          &= \left[\left(cT_{A_1}\right)'\right]_{\mathcal{F}_m'}^{\mathcal{E}_n'}\\
                          &= \left[cT_{A_1}'\right]_{\mathcal{F}_m'}^{\mathcal{E}_n'}\\
                          &= c\left[T_{A_1}'\right]_{\mathcal{F}_m'}^{\mathcal{E}_n'}\\
                          &= cA_{1}^{T}.
  \end{align*}
\end{solution}
\section{Problem 1}%
\begin{problem}
  Let $V = P_{n}\left(\F\right)$. Let $\mathcal{B} = \set{1,x,\dots,x^n}$ be a basis of $V$.\newline

  Let $\lambda \in \F$, and set $\mathcal{C} = \set{1,x-\lambda,\dots,\left(x-\lambda\right)^{n-1},\left(x-\lambda\right)^n}$.\newline

  Define a linear transformation $T\in \Hom_{\F}\left(V,V\right)$ by taking $T\left(x^j\right) = \left(x-\lambda\right)^j$. Determine the matrix of this linear transformation.\newline

  Use this to conclude that $\mathcal{C}$ is also a basis of $V$.
\end{problem}
\begin{solution}
  Considering our basis $\mathcal{B} = \set{1,x,\dots,x^n}$, we evaluate $T\left(x^j\right)$ for each $j$. In particular, this yields
  \begin{align*}
    T\left(x^j\right) &= \sum_{k=0}^{j}{j\choose k}\left(-\lambda\right)^{j-k}x^k,
  \end{align*}
  meaning that our linear transformation is
  \begin{align*}
    \left[T\right]_{\mathcal{B}}^{\mathcal{B}} &= \begin{pmatrix}1 & -\lambda & \left(-\lambda\right)^2 & \cdots & \left(-\lambda\right)^{n}\\ 0 & 1 & 2\left(-\lambda\right) & \cdots & {n\choose 1}\left(-\lambda\right)^{n-1}\\ 0 & 0 & 1 & \cdots & {n\choose 2}\left(-\lambda\right)^{n-1} \\ &\vdots & \vdots & \ddots & \vdots \\ 0 & 0 & 0 & \cdots & 1\end{pmatrix}.
  \end{align*}
  We can see that $\left[T\right]_{\mathcal{B}}^{\mathcal{B}}$ is nonsingular (since it is an upper triangular matrix that is nonzero along the diagonal), meaning that $T$ is injective (and thus, bijective), so it is an isomorphism.\newline

  Since $T$ is an isomorphism, and $T\left(x^j\right) = \left(x-\lambda\right)^j$, this means $\mathcal{C}$ is a basis.
\end{solution}
\section{Problem 4}%
\begin{problem}
  Let $V = P_{5}\left(\Q\right)$ and let $\mathcal{B} = \set{1,x,\dots,x^5}$. Prove that the following are elements of $V'$< and express them as linear combinations of the dual basis.
  \begin{enumerate}[(a)]
    \item $\varphi: V\rightarrow \Q$ defined by $\varphi\left(p(x)\right) = \int_{0}^{1} t^2p(t)\:dt$.
    \item $\varphi: V\rightarrow \Q$ defined by $\varphi\left(p(x)\right) = p'(5)$, where $p'(x)$ denotes the derivative of $p(x)$.
  \end{enumerate}
\end{problem}
\begin{solution}
  We define $\mathcal{B} = \set{1,x,\dots,x^5} = \set{e_0,e_1,\dots,e_5}$.\newline

  In particular, we can see that for $p(x) =\sum_{i=0}^{5}a_ix^i$, $a_i = e_i'\left(p\right)$.
  \begin{enumerate}[(a)]
    \item Let $p(x) = \sum_{i=0}^{5}a_ix^i$. Then,
      \begin{align*}
        \int_{0}^{1} t^2p(t)\:dt &= \int_{0}^{1} t^2\sum_{i=0}^{5}a_it^i\:dt\\
                                 &= \int_{0}^{1} \sum_{i=0}^{5}a_it^{i+2}\:dt\\
                                 &= \sum_{i=0}^{5}\frac{1}{i+3}a_i\\
                                 &= \sum_{i=0}^{5}\frac{1}{i+3}e_i'\left(p\right).
      \end{align*}
    \item Let $p(x) = \sum_{i=0}^{5}a_ix^i$. Then,
      \begin{align*}
        p'(x) &= \sum_{i=1}^{5}a_ix^{i-1}\\
              &= \sum_{i=0}^{4}a_{i+1}x^{i}\\
        p'(5) &= \sum_{i=0}^{4}a_{i+1}\left(5^{i}\right)\\
              &= \sum_{i=0}^{4}\left(5^i\right)e_{i+1}'\left(p\right).
      \end{align*}
  \end{enumerate}
\end{solution}
\end{document}
