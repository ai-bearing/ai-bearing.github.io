\documentclass[10pt]{mypackage}

% sans serif font:
%\usepackage{cmbright}
%\usepackage{sfmath}
%\usepackage{bbold} %better blackboard bold

%serif font + different blackboard bold for serif font
\usepackage{newpxtext,eulerpx}
\renewcommand*{\mathbb}[1]{\varmathbb{#1}}
\renewcommand*{\hbar}{\hslash}

%\pagestyle{fancy} %better headers
%\fancyhf{}
%\rhead{Avinash Iyer}
%\lhead{}

\setcounter{secnumdepth}{0}

\begin{document}
\RaggedRight
\begin{center}
  \bfseries
  Math 395: Homework 2\\
  Name: Avinash Iyer\\
  Due: 09/24/2024\\
  Collaborators: Noah Smith, Gianluca Crescenzo, Carly Venenciano, Timothy Rainone, Clarissa Ly, Ben Langer Weida
\end{center}
\section{Problem 1}%
\begin{problem}
  Let $V = P_{n}\left(\F\right)$. Let $\mathcal{B} = \set{1,x,\dots,x^n}$ be a basis of $V$. Let $\lambda \in \F$, and set $\mathcal{C} = \set{1,x-\lambda,\dots,\left(x-\lambda\right)^{n-1},\left(x-\lambda\right)^n}$. Define a linear transformation $T\in \Hom_{\F}\left(V,V\right)$ by taking $T\left(x^j\right) = \left(x-\lambda\right)^j$. Determine the matrix of this linear transformation. Use this to conclude that $\mathcal{C}$ is also a basis of $V$.
\end{problem}
\begin{solution}
  Considering our basis $\mathcal{B} = \set{1,x,\dots,x^n}$, we evaluate $T\left(x^j\right)$ for each $j$. In particular, this yields
  \begin{align*}
    T\left(1\right) &= 1\\
    T\left(x\right) &= x-\lambda\\
                    &\vdots\\
    T\left(x^{n-1}\right) &= \left(x-\lambda\right)^{n-1}\\
    T\left(x^n\right) &= \left(x-\lambda\right)^n.
  \end{align*}
  In particular, $T\left(x^j\right) = (1)\left(x-\lambda\right)^j$, implying that our matrix is
  \begin{align*}
    \left[T\right]_{\mathcal{B}}^{\mathcal{C}} &= \begin{pmatrix} 1 & 0 & \cdots & 0 \\ 0 & 1 & \cdots & 0\\ \vdots & \vdots & \ddots & \vdots \\ 0 & 0 & \cdots & 1\end{pmatrix}\\
                                               &= I_{n}.
  \end{align*}
  In particular, since $I_{n}$ is an isomorphism, it is the case that $T$ maps one basis of $V$ to another basis of $V$, meaning $\mathcal{C}$ is a basis of $P_{n}\left(\F\right)$.
\end{solution}
\end{document}
