\documentclass[10pt]{mypackage}

% sans serif font:
%\usepackage{cmbright}
%\usepackage{sfmath}
%\usepackage{bbold} %better blackboard bold

%serif font + different blackboard bold for serif font
\usepackage{newpxtext,eulerpx}
\renewcommand*{\mathbb}[1]{\varmathbb{#1}}
\renewcommand*{\hbar}{\hslash}

%\pagestyle{fancy} %better headers
%\fancyhf{}
%\rhead{Avinash Iyer}
%\lhead{}

\setcounter{secnumdepth}{0}

\begin{document}
\begin{center}
  \textbf{Math 395: Homework 4}\break
  \textbf{Due:}  10/03/2024\break
  
  \textbf{Avinash Iyer}\break
  \textbf{Collaborators: Carly Venenciano, Gianluca Crescenzo, Noah Smith, Ben Langer Weida, Clarissa Ly}
\end{center}
\RaggedRight
\section{Problem 6}%
\begin{problem}
  Let $A\in \Mat_{n}\left(\F\right)$.
  \begin{enumerate}[(a)]
    \item Assume $A$ has eigenvalues $\lambda_1,\dots,\lambda_n$. Prove that $\det(A) = \lambda_1\cdots\lambda_n$ and $\tr(A) = \lambda_1 + \cdots + \lambda_n$.
    \item Suppose $A$ does not have $n$ distinct eigenvalues, but $c_A(x)$ splits into linear factors over $F$. Can you characterize the determinant and trace of $A$ in terms of the eigenvalues?
  \end{enumerate}
\end{problem}
\begin{solution}\hfill
  \begin{enumerate}[(a)]
    \item If $A\in \Mat_{n}\left(\F\right)$ has distinct eigenvalues $\lambda_1,\dots,\lambda_n$, then there exists $P\in \text{GL}_{n}\left(\F\right)$ such that
      \begin{align*}
        A &= P\left(\diag\left(\lambda_1,\dots,\lambda_n\right)\right)P^{-1},
      \end{align*}
      where $\diag\left(\lambda_1,\dots,\lambda_n\right)$ denote the diagonal matrix with entries $\lambda_1,\dots,\lambda_n$ at entries $a_{11},\dots,a_{nn}$. In particular, this means
      \begin{align*}
        \det(A) &= \det\left(P\left(\diag\left(\lambda_1,\dots,\lambda_n\right)\right)P^{-1}\right)\\
                &= \det\left(\diag\left(\lambda_1,\dots,\lambda_n\right)\right)\\
                &= \prod_{j=1}^{n}\lambda_j,
                \intertext{and}
        \tr\left(A\right) &= \tr\left(P\left(\diag\left(\lambda_1,\dots,\lambda_n\right)\right)P^{-1}\right)\\
                          &= \tr\left(\diag\left(\lambda_1,\dots,\lambda_n\right)\right)\\
                          &= \sum_{j=1}^{n}\lambda_j.
      \end{align*}
    \item If $c_A(x)$ splits into linear factors over $F$, then the Jordan canonical form for $A$ exists, with each of its Jordan blocks consisting of the roots of $c_A(x)$ with multiplicity.\footnote{Assistance from Wikipedia} Thus, we can characterize $\tr(A)$ to be the sum of the roots of $c_A(X)$ with multiplicity, and $\det(A)$ to be the product of the roots with multiplicity.
  \end{enumerate}
\end{solution}
\section{Problem 8}%
\begin{problem}
  Prove that if $\lambda_1,\dots,\lambda_n$ are the eigenvalues of a matrix $A\in \Mat_{n}\left(\F\right)$, the $\lambda_1^k,\dots,\lambda_n^{k}$ are the eigenvalues for $A^k$ for any $k\geq 0$.
\end{problem}
\begin{solution}
  Since $A$ has eigenvalues $\lambda_1,\dots,\lambda_n$, it is the case that there exists some $P\in \text{GL}_{n}\left(\F\right)$ such that
  \begin{align*}
    A &= P\left(\diag\left(\lambda_1,\dots,\lambda_k\right)\right)P^{-1}.
  \end{align*}
  For $k = 0$, we have
  \begin{align*}
    A^{0}  &= \left(P\left(\diag\left(\lambda_1,\dots,\lambda_n\right)\right)P^{-1}\right)^{0}
           &= I_n\\
           &= P\left(\diag\left(\lambda_1^{0},\dots,\lambda_n^{0}\right)\right)P^{-1},
  \end{align*}
  meaning $\lambda_1^k,\dots,\lambda_n^k$ are eigenvalues for $A^k$.\newline

  For $k > 0$, we have
  \begin{align*}
    A^{k} &= \underbrace{\left(P\left(\diag\left(\lambda_1,\dots,\lambda_n\right)\right)P^{-1}\right)\left(P\left(\diag\left(\lambda_1,\dots,\lambda_n\right)\right)P^{-1}\right)\cdots \left(P\left(\diag\left(\lambda_1,\dots,\lambda_n\right)\right)P^{-1}\right)}_{\text{$k$ times}}\\
          &= P\underbrace{\left(\diag\left(\lambda_1,\dots,\lambda_n\right)\right)\left(\diag\left(\lambda_1,\dots,\lambda_n\right)\right)\cdots\left(\diag\left(\lambda_1,\dots,\lambda_n\right)\right)}_{\text{$k$ times}}P^{-1}\\
          &= P\left(\diag\left(\lambda_1^{k},\dots,\lambda_n^{k}\right)\right)P^{-1},
  \end{align*}
  meaning $\lambda_1^k,\dots,\lambda_n^{k}$ are eigenvalues for $A^{k}$.
\end{solution}
\end{document}
