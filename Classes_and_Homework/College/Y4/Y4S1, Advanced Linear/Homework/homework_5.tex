\documentclass[10pt]{mypackage}

% sans serif font:
%\usepackage{cmbright}
%\usepackage{sfmath}
%\usepackage{bbold} %better blackboard bold

%serif font + different blackboard bold for serif font
\usepackage{newpxtext,eulerpx}
\renewcommand*{\mathbb}[1]{\varmathbb{#1}}
\renewcommand*{\hbar}{\hslash}

%\pagestyle{fancy} %better headers
%\fancyhf{}
%\rhead{Avinash Iyer}
%\lhead{Advanced Linear Algebra: Ho}

\setcounter{secnumdepth}{0}

\begin{document}
\RaggedRight
\begin{center}
  \textbf{Math 395: Homework 5}\break
  \textbf{Due: October 22, 2024}\break
  \textbf{Name: Avinash Iyer}\break
  \textbf{Collaborators: Carly Venenciano, Gianluca Crescenzo, Noah Smith, Ben Langer Weida}
\end{center}
\section{Problem 6}%
\begin{problem}
  Let $A\in \Mat_{n}\left(\F\right)$ be a triangular matrix. Prove that the eigenvalues of $A$ are the entries on the main diagonal.
\end{problem}
\begin{solution}
  Let
  \begin{align*}
    A &=  \begin{pmatrix}
a_1 & \cdots & \cdots \\
0 & \ddots & \vdots \\
0 & 0 & a_n 
\end{pmatrix}\\
      &\in \Mat_{n}\left(\F\right).
  \end{align*}
  Then, we have
  \begin{align*}
    c_A(x) &= \det\left(xI_{n} - A\right)\\
           &= \det\left(\begin{pmatrix}
x - a_1 & \cdots & \cdots \\
0 & \ddots & \vdots \\
0 & 0 & x - a_n 
\end{pmatrix}\right)\\
                                &= \left(x-a_1\right)\left(x-a_2\right)\cdots\left(x-a_n\right).
  \end{align*}
  Thus, $c_A(x)$ has roots at $a_1,\dots,a_n$, which are the entries on the main diagonal of $A$.
\end{solution}
\section{Problem 16}%
\begin{problem}
  Let
  \begin{align*}
    A &= \begin{pmatrix}2 & -4 & 2 & 2 \\ -2 & 0 & 1 & 3 \\ -2 & -2 & 3 & 3 \\ -2 & -6 & 3 & 7\end{pmatrix}.
  \end{align*}
  \begin{enumerate}[(a)]
    \item Find the characteristic polynomial of $A$.
    \item Compute $E_{\lambda}^{j}$ for all eigenvalues $\lambda$ of $A$ and all $j$.
    \item Give the Jordan canonical form of $A$ and the Jordan basis $\mathcal{B}$ of $\F^4$.
  \end{enumerate}
\end{problem}
\begin{solution}\hfill
  \begin{enumerate}[(a)]
    \item Using computational assistance, we find
      \begin{align*}
        c_A(x) &= \det\left(A - x I_{4}\right)\\
               &= x^4 - 12x^3 + 52x^2 - 96x + 64\\
               &= \left(x-4\right)^2\left(x-2\right)^2.
      \end{align*}
    \item The eigenvalues of $A$ are $2$ and $4$. Thus, we calculate
      \begin{align*}
        A - 2I_4 &= \begin{pmatrix}0 & -4 & 2 & 2 \\ -2 & -2 & 1 & 3 \\ -2 & -2 & 1 & 3 \\ -2 & -6 & 3 & 5\end{pmatrix}.
        \intertext{In reduced row echelon form (with computational assistance), we get}
                 &\simeq \begin{pmatrix}1 & 0 & 0 & -1 \\ 0 & 1 & -\frac{1}{2} & -\frac{1}{2} \\ 0 & 0 & 0 & 0 \\ 0 & 0 & 0 & 0\end{pmatrix}.
      \end{align*}
      Thus, $A - 2I_4$ is of rank $2$, so $\Dim\left(\ker\left(A - 2I_4\right)\right) = 2$. The vectors $v_1 = e_1 + \frac{1}{2}e_2 + e_4$ and $v_2 = \frac{1}{2}e_2 + e_3$ form a basis for $E_{2}^{1}$. Since the degree on the factor $\left(x-2\right)$ is $2$, this means $E_{2}^{\infty} = E_{2}^{1}$.\newline

      Now, we turn our attention to $A - 4I_4$. We have
      \begin{align*}
        A - 4I_4 &= \begin{pmatrix} -2 & -4 & 2 & 2 \\ -2 & -4 & 1 & 3 \\ -2 & -2 & -1 & 3 \\ -2 & -6 & 3 & 3\end{pmatrix},
        \intertext{which row-reduces to}
                &\simeq \begin{pmatrix}1 & 0 & 0 & 0 \\ 0 & 1 & 0 & -1\\ 0 & 0 & 1 & -1 \\ 0 & 0 & 0 & 0\end{pmatrix}.
      \end{align*}
      Thus, $A - 4I_4$ is of rank $3$, so $\Dim\left(\ker\left(A - 4I_4\right)\right) = 1$. The vector $v_3 = e_2 + e_3 + e_4$ forms a basis for $E_{4}^{1}$. However, since the degree on the factor $\left(x-4\right)$ is $2$, we must turn our attention to $E_{4}^{2}$.\newline

      We now examine $\left(A-I_4\right)^2$. We have
      \begin{align*}
        \left(A - 4I_4 \right)^2 &= \begin{pmatrix}4 & 8 & -4 & -4 \\ 4 & 4 & 0 & -4 \\ 4 & 0 & 4 & -4 \\ 4 & 8 & -4 & -4\end{pmatrix},
        \intertext{which row-reduces to}
                                 &\simeq \begin{pmatrix}1 & 0 & 1 & -1 \\ 0 & 1 & -1 & 0 \\ 0 & 0 & 0 & 0 \\ 0 & 0 & 0 & 0\end{pmatrix}.
      \end{align*}
      Thus, $\left(A - 4I_4\right)^2$ is of rank $2$, meaning $\Dim\left(\ker\left(\left(A - 4I_4\right)^2\right)\right)  =2$, The vector $v_4 = e_1 + e_4$, along with $v_3$, forms a basis for $E_{4}^{2}$.
    \item Thus, via finding the generalized eigenspaces $E_{2}^1$ and $E_{4}^2$, we get the Jordan basis of
      \begin{align*}
        \mathcal{B} &= \set{ \begin{pmatrix}1 \\ \frac{1}{2} \\ 0 \\ 1\end{pmatrix}, \begin{pmatrix}0 \\ \frac{1}{2} \\ 1 \\ 0\end{pmatrix}, \begin{pmatrix}0\\1\\1\\1\end{pmatrix}, \begin{pmatrix}1\\0\\0\\1\end{pmatrix}},
      \end{align*}
      with the Jordan canonical form of
      \begin{align*}
        \left[T_A\right]_{\mathcal{B}} &= \begin{pmatrix}2 &0 &0 &0 \\ 0& 2 & 0&0 \\ 0& 0& 4 & 1 \\ 0& 0& 0& 4\end{pmatrix}
      \end{align*}
  \end{enumerate}
\end{solution}
\section{Problem 21}%
\begin{problem}
Prove that the matrices
\begin{align*}
  A &= \begin{pmatrix}2 & 0 & 0 & 0 \\ -4 & -1 & -4 & 0 \\ 2 & 1 & 3 & 0 \\ -2 & 4 & 9 & 1\end{pmatrix}
\end{align*}
and
\begin{align*}
  B &= \begin{pmatrix}5 & 0 & -4 & -7 \\ 3 & -8 & 15 & -13 \\ 2 & -4 & 7 & -7 \\ 1 & 2 & -5 & 1\end{pmatrix}
\end{align*}
are similar.
\end{problem}
\begin{solution}
  We calculate the Jordan canonical form of $A$ by calculating the characteristic polynomial.
  \begin{align*}
    c_A(x) &= \left(x-1\right)^3\left(x-2\right).
  \end{align*}
  Evaluating the generalized eigenspaces, we find (with computational assistance) that
  \begin{align*}
    E_{1}^{\infty} &= E_{1}^{3}\\
    E_{2}^{\infty} &= E_{2}^{1},
    \intertext{meaning the Jordan canonical form of $A$ is}
                   &\simeq \begin{pmatrix}1 & 1 & 0 & 0 \\ 0 & 1 & 1 & 0 \\ 0 & 0 & 1 & 0 \\ 0 & 0 & 0 & 2\end{pmatrix}.
  \end{align*}
  Similarly, we find
  \begin{align*}
    c_B(x) &= \left(x-1\right)^3\left(x-2\right).
  \end{align*}
  With computational assistance, we find that
  \begin{align*}
    E_{1}^{\infty} &= E_{1}^{3}\\
    E_{2}^{\infty} &= E_{2}^{1},
    \intertext{meaning the Jordan canonical form of $B$ is}
                   &\simeq \begin{pmatrix}1 & 1 & 0 & 0 \\ 0 & 1 & 1 & 0 \\ 0 & 0 & 1 & 0 \\ 0 & 0 & 0 & 2\end{pmatrix}.
  \end{align*}
  Since $A$ and $B$ have the same Jordan canonical form, $A$ and $B$ are similar.
\end{solution}

\section{Problem 23}%
\begin{problem}
  Determine all possible Jordan canonical forms for a linear transformation with characteristic polynomial $\left(x-2\right)^3 \left(x-3\right)^2$.
\end{problem}
\begin{solution}
  The following dimensions are possible for each of the generalized eigenspaces
  \begin{align*}
    \Dim\left(E_{2}^{\infty}\right) &= 1,2,\text{ or }3,
    \intertext{and}
    \Dim\left(E_{3}^{\infty}\right) &= 1\text{ or }2.
  \end{align*}
  Thus, the potential Jordan canonical forms consist of the following matrices and the permutations of the Jordan blocks.
  \begin{align*}
   \begin{pmatrix}
  2 & 1 &  &  &  \\
   & 2 & 1 &  &  \\
   &  & 2 &  &  \\
   &  &  & 3 & 1 \\
   &  &  &  & 3 
  \end{pmatrix} \\
 \begin{pmatrix}
2 &  &  &  &  \\
 & 2 & 1 &  &  \\
 &  & 2 &  &  \\
 &  &  & 3 & 1 \\
 &  &  &  & 3 
\end{pmatrix}  \\
 \begin{pmatrix}\\
2 &  &  &  &  \\
 & 2 &  &  &  \\
 &  & 2 &  &  \\
 &  &  & 3 & 1 \\
 &  &  &  & 3 
\end{pmatrix} \\
 \begin{pmatrix}
2 & 1 &  &  &  \\
 & 2 & 1 &  &  \\
 &  & 2 &  &  \\
 &  &  & 3 &  \\
 &  &  &  & 3 
\end{pmatrix}  \\
 \begin{pmatrix}
2 &  &  &  &  \\
 & 2 & 1 &  &  \\
 &  & 2 &  &  \\
 &  &  & 3 &  \\
 &  &  &  & 3 
\end{pmatrix}\\
 \begin{pmatrix}
2 &  &  &  &  \\
 & 2 &  &  &  \\
 &  & 2 &  &  \\
 &  &  & 3 &  \\
 &  &  &  & 3 
\end{pmatrix}  
  \end{align*}
  
\end{solution}

\end{document}
