\documentclass[10pt]{mypackage}

% sans serif font:
%\usepackage{cmbright}
%\usepackage{sfmath}
%\usepackage{bbold} %better blackboard bold

%serif font + different blackboard bold for serif font
\usepackage{newpxtext,eulerpx}
\renewcommand*{\mathbb}[1]{\varmathbb{#1}}

%\pagestyle{fancy} %better headers
\fancyhf{}
\rhead{Avinash Iyer}
\lhead{}

\setcounter{secnumdepth}{0}

\begin{document}
\RaggedRight
\begin{center}
  \bfseries
  Math 395: Homework 2\\
  Name: Avinash Iyer\\
  Due: 09/12/2024\\
  Collaborators: Noah Smith, Gianluca Crescenzo, Carly Venenciano
\end{center}
\section{Problem 11}%
\begin{problem}
  Let $T\in \Hom_{\F}\left(\mathcal{P}_7\left(\F\right),\mathcal{P}_{7}\left(\R\right)\right)$ be defined by $T\left(f(x)\right) = f'(x)$, where $f'(x)$ denotes the usual derivative of a polynomial $f(x)\in \mathcal{P}_{7}\left(\F\right)$. For each of the fields below, determine a basis for the image and kernel of $T$:
  \begin{enumerate}[(a)]
    \item $\F = \R$
    \item $\F = \F_{3}$.
  \end{enumerate}
\end{problem}
\begin{solution}
  \begin{enumerate}[(a)]
    \item 
  \end{enumerate}
\end{solution}
\section{Problem 12}%
\begin{problem}
  Let $T\in \Hom_{\F}\left(V,\F\right)$. Prove that if $v\in V$ is not in $\ker(T)$, then
  \begin{align*}
    V &= \ker(T) \oplus \set{cv\mid c\in\F}.
  \end{align*}
\end{problem}
\begin{solution}
  Since $T(v)\neq 0$, there exists $\left(T(v)\right)^{-1}\in \F$. Let $w\in V$. Then,
  \begin{align*}
    T(w) &= \left(T(w)\left(T(v)\right)^{-1}\right)T(v).
  \end{align*}
  We let $c = T(w)\left(T(v)\right)^{-1}$. We have
  \begin{align*}
    T(w) &= cT(v)\\
         &= T(cv),
  \end{align*}
  meaning
  \begin{align*}
    T\left(w - cv\right) &= 0,
  \end{align*}
  so $w-cv\in \ker(T)$, or $w\in \left[cv\right]_{\sim}$, where $\sim$ is the equivalence relation defining $V/\ker(T)$.\newline

  Thus, we have $w \in \ker(T) + \set{cv\mid c\in \F}$, implying that $V \subseteq \ker(T) + \set{cv\mid c\in \F}$, so $V = \ker(T) + \set{cv\mid c\in\F}$.\newline

  For $k\in\ker(T)$, suppose
  \begin{align*}
    cv + k &= 0.
  \end{align*}
  Then,
  \begin{align*}
    T\left(cv + k\right) &= 0_V\\
    cT\left(v\right) + T(k) &= 0\\
    cT\left(v\right) &= 0.
  \end{align*}
  Since $T(v)\neq 0$ by the definition of $v$, it must be the case that $c=0$, meaning $cv = 0_V$. Thus, it is the case that $\ker(T)$ and $\set{cv\mid c\in\F}$ are independent subspaces, meaning
  \begin{align*}
    V &= \ker(T)\oplus \set{cv\mid c\in\F}.
  \end{align*}
\end{solution}
\end{document}
