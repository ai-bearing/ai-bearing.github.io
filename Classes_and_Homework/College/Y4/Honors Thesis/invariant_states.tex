\documentclass[10pt]{mypackage}

% sans serif font:
%\usepackage{cmbright}
%\usepackage{sfmath}
%\usepackage{bbold} %better blackboard bold

%serif font + different blackboard bold for serif font
\usepackage{newpxtext,eulerpx,eucal}
\renewcommand*{\mathbb}[1]{\varmathbb{#1}}
\renewcommand*{\hbar}{\hslash}
\DeclareMathOperator{\Prob}{Prob}

\pagestyle{fancy} %better headers
\fancyhf{}
\rhead{Avinash Iyer}
\lhead{Invariant States and Means on Groups}

\setcounter{secnumdepth}{0}

\begin{document}
\RaggedRight
\tableofcontents
\section{Introduction}%
This is going to be a part of my Honors thesis independent study, focused on amenability and $C^{\ast}$-algebras. This section of notes will be a deeper dive into group amenability. These notes will be taken from the notes my professor has prepared on group amenability, with supplementation from Volker Runde's \textit{Lectures on Amenability} and Pierre de la Harpe's \textit{Topics in Geometric Group Theory}.\newline

I do not claim any of this work to be original.
\section{Basics of Amenable Groups and Subgroups}%
Let $G$ be a group, with $P(G)$ denoting the power set.
\begin{definition}
  An invariant mean on $G$ is a set function $m\colon P(G)\rightarrow [0,1]$, which satisfies, for all $t\in G$ and $E,F\subseteq G$,
  \begin{enumerate}[(1)]
    \item $m(G) = 1$;
    \item $m\left(E\sqcup F\right) = M(E) + m(F)$;
    \item $m\left(tE\right) = m\left(E\right)$.
  \end{enumerate}
  We say $G$ is amenable if it admits a mean.\newline

  We can also say that $m$ is a translation-invariant probability measure on $\left(G,P(G)\right)$.
\end{definition}
\begin{proposition}[Amenability of Subgroups and Quotient Groups]
  Let $G$ be amenable, with $H\leq G$.
  \begin{enumerate}[(1)]
    \item $H$ is amenable;
    \item for $H\trianglelefteq G$, $G/H$ is amenable.
  \end{enumerate}
\end{proposition}
\begin{proof}\hfill
  \begin{enumerate}[(1)]
    \item Let $R$ be a right transversal for $H$ (i.e., selecting one element of each right coset of $H$ to make up $R$).\newline

      If $m$ is a mean for $G$, we set
      \begin{align*}
        \lambda\colon \mathcal{P}\left(H\right) \rightarrow [0,1]
      \end{align*}
      by $\lambda(E) = m\left(ER\right)$. We have
      \begin{align*}
        \lambda\left(H\right) &= m\left(HR\right)\\
                              &= m\left(G\right)\\
                              &= 1.
      \end{align*}
      We claim that if $E\cap F = \emptyset$, then $ER\cap FR = \emptyset$, since if we suppose toward contradiction that $ER \cap FR \neq \emptyset$, then $xr_1 = yr_2$ for some $ x\in E,y\in F$ and $r_1,r_2\in R$. Then, we must have $r_2r_1^{-1} = y^{-1}x \in H$, meaning $r_1 = r_2$ and $x = y$, which means $E\cap F \neq \emptyset$.\newline

      Thus, we have
      \begin{align*}
        \lambda\left(E\sqcup F\right) &= m\left(\left(E\sqcup F\right)R\right)\\
                                      &= m\left(ER\sqcup FR\right)\\
                                      &= m\left(ER\right) + m\left(FR\right)\\
                                      &= \lambda(E) + \lambda(F),
      \end{align*}
      and
      \begin{align*}
        \lambda\left(sE\right) &= m\left(sER\right)\\
                               &= m\left(ER\right)\\
                               &= \lambda\left(E\right).
      \end{align*}
    \item For the canonical projection map $\pi\colon G\rightarrow G/H$ defined by $\pi(t) = tH$, we define
      \begin{align*}
        \lambda\colon P\left(G/H\right)\rightarrow [0,1]
      \end{align*}
      by $\lambda\left(E\right) = m\left(\pi^{-1}\left(E\right)\right)$. We have
      \begin{align*}
        \lambda\left(G/H\right) &= m\left(\pi^{-1}\left(G/H\right)\right)\\
                                &= m\left(G\right)\\
                                &= 1,
      \end{align*}
      and
      \begin{align*}
        \lambda\left(E\sqcup F\right) &= m\left(\pi^{-1}\left(E\sqcup F\right)\right)\\
                                      &= m\left(\pi^{-1}\left(E\right)\sqcup \pi^{-1}\left(F\right)\right)\\
                                      &= m\left(\pi^{-1}\left(E\right)\right) + m\left(\pi^{-1}\left(F\right)\right)\\
                                      &= \lambda\left(E\right) + \lambda\left(F\right).
      \end{align*}
      To show translation-invariance, we let $sH = \pi\left(s\right)\in G/H$, and $E\subseteq G/H$. Note that
      \begin{align*}
        \pi^{-1}\left(\pi\left(s\right)E\right) &= s\pi^{-1}\left(E\right),
      \end{align*}
      since for $r\in s\pi^{-1}\left(E\right)$, we have $r = st$ for $\pi(t)\in E$, so $\pi\left(r\right) = \pi\left(st\right) = \pi\left(s\right)\pi\left(t\right) \in \pi\left(s\right)E$.\newline

      Additionally, if $r\in \pi^{-1}\left(\pi\left(s\right)E\right)$, then $\pi\left(r\right) \in \pi\left(s\right)E $, so $\pi\left(s^{-1}r\right)\in E$, and $s^{-1}r\in \pi^{-1}\left(E\right)$. Thus, we have
      \begin{align*}
        \lambda\left(\pi\left(s\right)E\right) &= m\left(\pi^{-1}\left(\pi\left(s\right)E\right)\right)\\
                                               &= m\left(s\pi^{-1}\left(E\right)\right)\\
                                               &= m\left(\pi^{-1}\left(E\right)\right)\\
                                               &= \lambda\left(E\right).
      \end{align*}
      
  \end{enumerate}
\end{proof}
\section{Understanding Free Groups}%
In the \href{https://blog.avinashiyer.xyz/Classes_and_Homework/College/Y4/Honors\%20Thesis/amenability_notes.pdf}{Tarski's Theorem} notes, we discussed a little bit the ramifications of the free group on two generators being paradoxical. In order to better understand free groups, we will draw information from Pierre de la Harpe's \textit{Topics in Geometric Group Theory} and Clara Löh's \textit{Geometric Group Theory: An Introduction}.
\subsection{Groups specified by Generating Sets}%
\begin{definition}
  Let $G$ be a group and $S\subseteq G$ be a subset. The subgroup generated by $S$ is the intersection of all subgroups of $G$ that contain $S$. We write $\left\langle S \right\rangle_{G}$. We say $S$ generates $G$ if $\left\langle S \right\rangle_{G} = G$.\newline

  A group is called finitely generated if it contains a finite subset that contains the group in question.
\end{definition}
\begin{definition}[Characterization of a Generated Subgroup]
We can characterize a generated subgroup by $S$ as follows:
\begin{align*}
  \left\langle S \right\rangle_{G} &= \set{s_1^{\ve_1}s_{2}^{\ve_2}\cdots s_n^{\ve_n} | n\in\N,~s_1,\dots,s_n\in S,~\ve_1,\dots,\ve_n\in\set{-1,1}}.
\end{align*}
\end{definition}
\begin{example}[Generating Sets]\hfill
  \begin{itemize}
    \item If $G$ is a group, then $G$ is a generating set of $G$.
    \item The trivial group is generated by the empty set.
    \item The set $\set{1}$ generates $\Z$, as does $\set{2,3}$. However, $\set{2}$ and $\set{3}$ alone do not generate $\Z$.
    \item Let $X$ be a set. The symmetric group $S_{X}$ is finitely generated if and only if $X$ is finite.
  \end{itemize}
\end{example}
\subsection{Free Groups}%
\begin{definition}
Let $S$ be a set. A group $F$ containing $S$ is said to be freely generated if, for every group $G$ and every map $\varphi\colon S\rightarrow G$, there is a unique group homomorphism $\overline{\varphi}:F\rightarrow G$ extending $\varphi$. The following diagram commutes:
\begin{center}
  % https://tikzcd.yichuanshen.de/#N4Igdg9gJgpgziAXAbVABwnAlgFyxMJZABgBpiBdUkANwEMAbAVxiRAGUQBfU9TXfIRQBGclVqMWbAOLdeIDNjwEiZYePrNWiEADFu4mFADm8IqABmAJwgBbJGRA4ISURK1sAOp-pW0ACyw5Sxt7REdnJAAmahw6LAY2fwgIAGsQagY6ACMYBgAFfmUhECssY38cDPcpHW98OOCQazto2JdEN01akG8IGhgrBiwwGGBvXwCsLmqs3IKiwTYyiqquCi4gA
\begin{tikzcd}
S \arrow[r, "\varphi"] \arrow[d, "\iota"', hook] & G \\
F \arrow[ru, "\overline{\varphi}"']              &  
\end{tikzcd}
\end{center}
A group is free if it contains a free generating set.
\end{definition}
\begin{example}\hfill
  \begin{itemize}
    \item The additive group $\Z$ is freely generated by $\set{1}$. The additive group $\Z$ is \textit{not} freely generated by $\set{2,3}$, or $\set{2}$, or $\set{3}$. In particular, not every generating set of a group contains a free generating set.
    \item The trivial group is freely generated by the empty set.
    \item Not every group is free --- the additive groups $\Z/2\Z$ and $\Z\times \Z$ are not free.
  \end{itemize}
\end{example}
We will use the universal property of free groups to show their uniqueness up to isomorphism.
\begin{proposition}
  Let $S$ be a set. Then, there is at most one group freely generated by $S$ up to isomorphism.
\end{proposition}
\begin{proof}
  Let $F$ and $F'$ be two groups freely generated by $S$, with inclusions of $\varphi$ and $\varphi'$ respectively.
    Because $F$ is freely generated by $S$, there is a group homomorphism $\overline{\varphi}'\colon F\rightarrow F'$ that extends $\varphi$ --- i.e., that $\overline{\varphi}' \circ \varphi = \varphi'$.\newline

      Similarly, there is a group homomorphism $\overline{\varphi}\colon F'\rightarrow F$ with $\overline{\varphi}\circ \varphi' = \varphi$.
      \begin{center}
        % https://tikzcd.yichuanshen.de/#N4Igdg9gJgpgziAXAbVABwnAlgFyxMJZABgBpiBdUkANwEMAbAVxiRAGUQBfU9TXfIRQBGclVqMWbAGIBybrxAZseAkTLDx9Zq0QhpCvisFEAzGOrapezjyMC1KACwWJOmYaX9VQ5Oc2Wkrr68lziMFAA5vBEoABmAE4QALZIZCA4EEiibtYgADr59AloABZYoYqJKWnUmUgATHV0WAxspRAQANYg1Ax0AEYwDAAK3iZ6CViRpTi9ucGFxWVYntWpiE0ZWYg5Vov5EDQwCQxYYDDAS3Ql5VzyfYPDY8aOIFMzc3Yg60jm20gXAs2Ndbqtvr9EP96ogAKzNVrtTo9R5DUbjN4fWbzfYgoo3FaVeJJDbwgGIIH9NEvBxCd7TbGBdx6QpHE5nC5XfFgrjcChcIA
\begin{tikzcd}
S \arrow[r, "\varphi'"] \arrow[d, "\varphi"', hook] & F' &  & S \arrow[r, "\varphi"] \arrow[d, "\varphi'"', hook] & F \\
F \arrow[ru, "\overline{\varphi}'"']                &    &  & F' \arrow[ru, "\overline{\varphi}"']                &  
\end{tikzcd}
      \end{center}
      We will show that $\overline{\varphi}\circ \overline{\varphi}' = \id_{F}$, and $\overline{\varphi}'\circ \overline{\varphi} = \id_{F'}$. The composition $\overline{\varphi}\circ \overline{\varphi}'$ is a group homomorphism that makes the following diagram commute.
      \begin{center}
% https://tikzcd.yichuanshen.de/#N4Igdg9gJgpgziAXAbVABwnAlgFyxMJZABgBpiBdUkANwEMAbAVxiRAGUQBfU9TXfIRQBGclVqMWbAGLdeIDNjwEiZYePrNWiELK7iYUAObwioAGYAnCAFskZEDghJRErWwA6H+pbQALLDkLaztEByckACZqHDosBjY-CAgAaxBqTSkdLx9-QOoGOgAjGAYABX5lIRBLLCM-HCCQK1somOdEV0KS8srBNlr6xozJbRAvCBoYSwYsMBhgHLpfAK4vAGMsS3WJqZm5haWVrC4Acm4KLiA
\begin{tikzcd}
S \arrow[r, "\varphi"] \arrow[d, "\varphi"', hook]          & F \\
F \arrow[ru, "\overline{\varphi}\circ\overline{\varphi}'"'] &  
\end{tikzcd}
      \end{center}
      Since $\id_{F}$ is a group homomorphism contained in this diagram, and $F$ is freely generated by $S$, we must have $\overline{\varphi}\circ \overline{\varphi}' = \id_{F}$. Similarly, we must have $\overline{\varphi}' \circ \overline{\varphi} = \id_{F'}$.
\end{proof}
\begin{theorem}[Existence of Free Groups]
  Let $S$ be a set. There exists a group freely generated by $S$. This group is unique up to isomorphism.
\end{theorem}
\begin{proof}
  We want to construct a group consisting of ``words'' made up of the elements of $S$ and their ``inverses,'' then modding out by the natural cancellation rules.\newline

  We consider the alphabet
  \begin{align*}
    A &= S\cup \hat{S}.
  \end{align*}
  Here, $\hat{S} = \set{\hat{s} | s\in S}$ is a disjoint copy of $S$, such that $\hat{s}$ will serve as the inverse of $s$ in the group we will construct.\newline

  We define $A^{\ast}$ to be the set of all finite sequences over the alphabet $A$, including the empty word $\epsilon$. We define the operation $A^{\ast}\times A^{\ast} \rightarrow A^{\ast}$ by concatenation. This operation is associative with neutral element $\epsilon$.\newline

  We define
  \begin{align*}
    F(S) &= A^{\ast}/\sim,
  \end{align*}
  where $\sim$ is the equivalence relation generated by, for all $x,y\in A^{\ast}$ and $s\in S$, $x s \hat{s} y \sim xy$ and $x\hat{s} s y \sim xy$.\newline

  We denote the equivalence classes with respect to $\sim$ by $\left[\cdot\right]$.\newline

  Concatenation induces a well-defined operation $F\left(S\right)\times F\left(S\right) \rightarrow F(S)$ by
  \begin{align*}
    \left[x\right] \left[y\right] &= \left[xy\right]
  \end{align*}
  for $x,y\in A^{\ast}$.\newline

  We claim that $F(S)$ with the defined concatenation is a group. We can see that $\left[\epsilon\right]$ is a neutral element for the operation, and associativity of the operation is inherited from the associativity of the operation on $A^{\ast}$.\newline

  To find inverses, we define $I\colon A^{\ast}\rightarrow A^{\ast}$ by $I\left(\epsilon\right) = \epsilon$, and
  \begin{align*}
    I\left(sx\right) &= I(x)\hat{s}\\
    I\left(\hat{s}x\right) &= I(x)s
  \end{align*}
  for all $x\in A^{\ast}$ and $s\in S$. Induction shows that $I\left(I\left(x\right)\right) = x$, and
  \begin{align*}
    \left[I\left(x\right)\right]\left[x\right] &= \left[I\left(x\right)x\right]\\
                                               &= \left[\epsilon\right]
  \end{align*}
  for all $x\in A^{\ast}$. Thus, we must also have
  \begin{align*}
    \left[x\right]\left[I(x)\right] &= \left[I\left(I\left(x\right)\right)\right]\left[I\left(x\right)\right]\\
                                    &= \left[\epsilon\right].
  \end{align*}
  Thus, we see that there are inverses in $F(S)$.\newline

  To see that $F(S)$ is freely generated by $S$, we let $\iota\colon S\rightarrow F(S)$ be the map given by sending a letter in $S\subseteq A^{\ast}$ to its equivalence class in $F(S)$. By construction, $F(S)$ is generated by the subset $\iota\left(S\right)\subseteq F(S)$.\newline

  We do not know yet that $\iota$ is injective, so we take a bit of a detour. We show that for every group $G$ and every map $\varphi\colon S\rightarrow G$, there is a unique group homomorphism $\overline{\varphi}\colon F(S) \rightarrow G$ such that $\overline{\varphi}\circ \iota = \varphi$.\newline

  We construct a map $\varphi^{\ast}\colon A^{\ast}\rightarrow G$ inductively by
  \begin{align*}
    \epsilon &\mapsto e\\
    sx &\mapsto \varphi(s)\varphi^{\ast}\left(x\right)\\
    \hat{s}x &\mapsto \left(\varphi\left(s\right)\right)^{-1}\varphi^{\ast}\left(x\right)
  \end{align*}
  for all $s\in S$ and $x\in A^{\ast}$. We can see that, since the definition of $\varphi^{\ast}$ is compatible with the generating set of $\sim$, it is compatible with the equivalence relation of $\sim$ on $A^{\ast}$. Additionally, we can see that $\varphi^{\ast}\left(xy\right) = \varphi^{\ast}\left(x\right)\varphi^{\ast}\left(y\right)$ for all $x,y\in A^{\ast}$. Thus,
  \begin{align*}
    \overline{\varphi}\colon F\left(S\right) &\rightarrow G\\
    \left[x\right] &\mapsto \left[\varphi^{\ast}\left(x\right)\right],
  \end{align*}
  is, as constructed, a group homomorphism, with $\overline{\varphi}\circ \iota = \varphi$. Since $\iota\left(S\right)$ generates $F(S)$, this group homomorphism is unique.\newline
  
  We must now show that $\iota$ is injective.\newline

  Let $s_1,s_2\in S$. Consider the map $\varphi\colon S\rightarrow \Z$ given by $\varphi\left(s_1\right) = 1$ and $\varphi\left(s_2\right) = -1$. The corresponding homomorphism $\overline{\varphi}\colon F(S)\rightarrow G$ satisfies
  \begin{align*}
    \overline{\varphi}\left(\iota\left(s_1\right)\right) &= \varphi\left(s_1\right)\\
                                                         &= 1\\
                                                         &\neq -1\\
                                                         &= \varphi\left(s_2\right)\\
                                                         &= \overline{\varphi}\left(\iota\left(s_2\right)\right),
  \end{align*}
  meaning $\iota\left(s_1\right)\neq \iota\left(s_2\right)$. Thus, $\iota$ is injective.
\end{proof}
\subsection{Free Groups, Free Products, and the Ping Pong Lemma}%
We can now consider free groups in a more categorical context --- specifically, as a special type of free object. Whereas the previous section drew information from Clara Löh's \textit{Geometric Group Theory: An Introduction}, this section will draw information from Pierre de la Harpe's \textit{Topics in Geometric Group Theory}. Specifically, we are focused on chapter $2$, which discusses free products, free groups, and the ping pong lemma (which will provide a more general proof of the paradoxicality of $SO(3)$).
\begin{definition}[Free Monoid]
  A monoid is a set with multiplication that is associative and includes a neutral element.\newline

  Given a set $A$, the free monoid on $A$ is the set $W(A)$ of finite sequences of elements of $A$ (also known as words). We write an element of $W(A)$ as $w = a_1a_2\cdots a_n$, where each $a_j\in A$. We identify $A$ with the subset of $W(A)$ of words with length $1$.
\end{definition}
\begin{definition}[Free Product]
  Let $\left(\Gamma_i\right)_{i\in I}$ be a family of groups. Set
  \begin{align*}
    A &= \coprod_{i\in I}\Gamma_{i}\\
      &= \set{\left(g_i,i\right) | g_i\in \Gamma_i,i\in I}
  \end{align*}
  to be the coproduct of this family.\newline

  Let $\sim$ be the equivalence relation generated by
  \begin{align*}
    we_i w' &\sim ww' \tag*{where $e_i\in \Gamma_i$ is the neutral element}\\
    wabw' &\sim wcw' \tag*{where $a,b,c\in \Gamma_i$, $c=ab$ for some $i\in I$}
  \end{align*}
  for all $w,w'\in W(A)$. The quotient $W(A)/\sim$ with the operation of concatenation is a group, which is known as the free product of the groups $\set{\Gamma_i}_{i\in I}$. We write it as
  \begin{align*}
    \bigstar_{i\in I} \Gamma_{i}.
  \end{align*}
  The inverse of the equivalence class for $w = a_1a_2\dots a_n$ is $w^{-1} = a_n^{-1}a_{n-1}^{-1}\cdots a_{1}^{-1}$. The neutral element is $\epsilon$, which is the empty word.\newline

  A word $w = a_1a_2\cdots a_n\in W(A)$ with $a_j\in \Gamma_{i_j}$ is said to be reduced if $i_{j + 1} \neq i_{j}$ and $a_j$ is not the neutral element of $\Gamma_{i_j}$.
\end{definition}
\begin{proposition}[Existence of the Free Product]
  Let $\set{\Gamma_i}_{i\in I}$ be a family of groups, $A = \coprod_{i\in I}\Gamma_i$, and $\bigstar_{i\in I}\Gamma_i = W(A)/\sim$ be as above.\newline

  Then, any element in the free product $\bigstar_{i\in I}\Gamma_i$ is represented by a unique reduced word in $W(A)$.
\end{proposition}
\begin{proof}\hfill
  \begin{description}[font=\normalfont\scshape,leftmargin=0pt]
    \item[Existence:] Consider an integer $n\geq 0$ and a reduced word $w = a_1a_2\cdots a_n$ in $W(A)$, an element $a\in A$, and the word $aw\in W(A)$. We set
      \begin{align*}
        \mathcal{R}\left(aw\right) &= \begin{cases}
          w & a = e_i\\
          aa_1a_2\cdots a_n & a\in \Gamma_i,a\neq e_i,i\neq k\\
          ba_2\cdots a_n & a\in \Gamma_k, aa_1 = b \neq e_k\\
          a_2\cdots a_n & a\in \Gamma_k,a_1 = a^{-1}\in \Gamma_k
        \end{cases},
      \end{align*}
      where $k$ is the index for which $a_1\in \Gamma_k$.\newline

      Then, $\mathcal{R}\left(aw\right)$ is yet another reduced word, and $\mathcal{R}\left(aw\right) \sim aw$, meaning that any word $w\in W(A)$ is equivalent to some reduced word by inducting on the length of $w$.
    \item[Uniqueness:] For each $a\in A$, Let $T(a) = \mathcal{R}\left(aw\right)$ be a self-map on the set of reduced words.\newline

      For each $w = b_1b_2\cdots b_n$, we set $T(w) = T\left(b_1\right)T\left(b_2\right)\cdots T\left(b_n\right)$. For $a,b,c\in \Gamma_i$ with $ab = c$, we have $T\left(a\right)T\left(b\right) = T\left(c\right)$, and $T\left(e_{i}\right) = \epsilon$ (the empty word) for all $i\in I$.\newline

       For each reduced word, notice that $T\left(w\right)\epsilon = w$.\newline

       Let $w$ be some word in $W(A)$ with $w_1,w_2$ reduced words equivalent to $w$. Since $w_1\sim w_2$, we have $T\left(w_1\right) = T\left(w_2\right)$, and
       \begin{align*}
         w_1 &= T\left(w_1\right)\epsilon\\
             &= T\left(w_2\right)\epsilon\\
             &= w_2.
       \end{align*}
  \end{description}
\end{proof}
\begin{corollary}
  Let $\set{\Gamma_i}_{i\in I}$ and $\Gamma = \bigstar_{i\in I}\Gamma_i$ as above. For each $i_0\in I$, the canonical inclusion
  \begin{align*}
    \iota\colon \Gamma_{i_0}\rightarrow \Gamma
  \end{align*}
  is injective.
\end{corollary}
\begin{proof}
  For any $a\in \Gamma_{i_0}\setminus \set{e_{i_0}}$, $\iota\left(a\right)$ is represented by a unique one-letter reduced word not equivalent to the empty word.
\end{proof}
Now that we have an understanding of free products, we can conceptualize the free group as a particular type of free product.
\begin{definition}[Free Groups]
  Let $X$ be a set. The free group over $X$ is the free product of a family of copies of $\Z$ indexed by $X$, denoted $F(X)$.\newline

  Equivalently, the free group over $X$ is
  \begin{align*}
    F(X) &= \bigstar_{a\in X}\left\langle a \right\rangle,
  \end{align*}
  where $\left\langle a \right\rangle$ denotes the cyclic group generated by the element $a$.\newline

  We can also identify $F(X)$ with the set of reduced words in $X\sqcup X^{-1}$ (as was done in the previous subsection).\newline

  The cardinality of $X$ is called the rank of $F(X)$.\newline

  If $\Gamma$ is a group, then a free subset of $\Gamma$ is a subset $X\subseteq \Gamma$ such that the inclusion $X\hookrightarrow F(X)$ extends to an isomorphism of $\left\langle X \right\rangle_{\Gamma}$ onto $F(X)$.
\end{definition}
We can now state and prove a universal property for free products (which naturally simplifies in the case of a free group.)
\begin{theorem}[Universal Property for Free Products]
  Let $\Gamma$ be a group, and $\set{\Gamma_i}_{i\in I}$ be a family of groups. Let $\set{h_{i}\colon \Gamma_i\rightarrow \Gamma}_{i\in I}$ be a family of homomorphisms.\newline

  Then, there exists a unique homomorphism $h\colon \bigstar_{i\in I}\Gamma_{i}\rightarrow \Gamma$ such that the following diagram commutes for each $i_0\in I$.
  \begin{center}
    % https://tikzcd.yichuanshen.de/#N4Igdg9gJgpgziAXAbVABwnAlgFyxMJZARgBoAGAXVJADcBDAGwFcYkQAdDgcXoFs+9EAF9S6TLnyEU5UsWp0mrdlwBGWAOZwc9AE4B9YFi5YwAAgCSwrrwH19WEWJAZseAkVlUaDFm0ScPPyChlj65MIiCjBQGvBEoABmuhB8SABMNDgQSLKKfuwAFqHhkaJJKWmImSDZSGS19FiMRRAQANYgNIz0qjCMAAoS7tIgupqFOF35ygEmEDpOFan1WTmIeT19g8NS7OMak9O+syCFUcJAA
\begin{tikzcd}
\Gamma_{i_0} \arrow[r, "h_{i_0}"] \arrow[d, "\iota"', hook] & \Gamma \\
\bigstar_{i\in I}\Gamma_i \arrow[ru, "h"']                  &       
\end{tikzcd}
  \end{center}
  In particular, if $\Gamma$ is a group, $X$ is a set, and $\phi\colon X\rightarrow \Gamma$ is a set map, there exists a unique homomorphism $\Phi\colon F(X)\rightarrow \Gamma$ such that $\Phi(x) = \phi(x)$ for each $x\in X$.
\end{theorem}
\begin{proof}
  For a reduced word $w = a_1a_2\cdots a_n\in \bigstar_{i\in I}\Gamma_i$ with $a_j\in \Gamma_{i_j}\setminus \set{e_{i_j}}$, and $i_{j+1}\neq i_{j}$ for each $j\in \set{1,\dots,n-1}$, we set
  \begin{align*}
    h\left(w\right) &= h_{i_1}\left(a_1\right)h_{i_2}\left(a_2\right)\cdots h_{i_n}\left(a_n\right),
  \end{align*}
  which defines $h$ uniquely in terms of $h_{i}$.
\end{proof}
Note that for any two sets $X,Y$, the universal property provides that any map $X\rightarrow Y$ extends canonically to a group homomorphism, $F(X) \rightarrow F(Y)$.
\begin{center}
  % https://tikzcd.yichuanshen.de/#N4Igdg9gJgpgziAXAbVABwnAlgFyxMJZABgBpiBdUkANwEMAbAVxiRAA0QBfU9TXfIRQBGclVqMWbAJrdeIDNjwEiZYePrNWiEADEAFOwCUcvksFFR66pqk6D0k13EwoAc3hFQAMwBOEAFskACZqHAgkAGYeH38gxDIQcKRhGJA-QKRE5MRQpLosBjYACwgIAGtTdLiUsIjESLCCop1Siu4KLiA
\begin{tikzcd}
X \arrow[r] \arrow[d, hook] & Y \arrow[d, hook] \\
F(X) \arrow[r]              & F(Y)             
\end{tikzcd}
\end{center}
We can now prove an important lemma that will be useful in understanding paradoxical groups.
\begin{theorem}[Ping Pong Lemma]
  Let $G$ be a group acting on a set $X$, and let $\Gamma_1,\Gamma_2$ be subgroups of $G$. Let $\Gamma = \left\langle \Gamma_1,\Gamma_2 \right\rangle$. Assume $\Gamma_1$ contains at least $3$ elements and $\Gamma_2$ contains at least two elements.\newline

  Suppose there exist nonempty subsets $X_1,X_2\subseteq X$ with $X_1\triangle X_2\neq \emptyset$, such that for all $\gamma_1\in \Gamma_1$ with $\gamma_1\neq e_{G}$, and for all $\gamma_2\in \Gamma_2$ with $\gamma_2\neq e_{G}$,
  \begin{align*}
    \gamma\left(X_2\right)\subseteq X_1\\
    \gamma\left(X_1\right)\subseteq X_2.
  \end{align*}
  Then, $\Gamma$ is isomorphic to the free product $\Gamma_1\star \Gamma_2$.
\end{theorem}
\begin{proof}
  Let $w$ be a nonempty reduced word spelled with letters from the disjoint union of $\Gamma_1\setminus \set{e_G}$ and $\Gamma_2\setminus \set{e_G}$. We must show that the element of $\Gamma$ defined by $w$ is not the identity.\newline

  If $w = a_1b_1a_2b_2\cdots a_k$ with $a_1,\dots,a_k\in \Gamma_1\setminus \set{e_G}$ and $b_1,\dots,b_{k-1}\in \Gamma_2\setminus \set{e_G}$. Then,
  \begin{align*}
    w\left(X_2\right) &= a_1b_1\cdots a_{k-1}b_{k-1}a_k\left(X_2\right)\\
                      &\subseteq a_1b_1\cdots a_{k-1}b_{k-1}\left(X_1\right)\\
                      &\subseteq a_1b_1\cdots a_{k-1}\left(X_2\right)\\
                      &\vdots\\
                      &\subseteq a_1\left(X_2\right)\\
                      &\subseteq X_1.
  \end{align*}
  Since $X_2\nsubseteq X_1$, this implies $w\neq e_{G}$.\newline

  If $w = b_1a_2b_2a_2\cdots b_k$, we select $a\in \Gamma_1\setminus \set{e_G}$, and apply the previous argument to $awa^{-1}$. Since $awa^{-1}\neq e_{G}$, neither is $w$.\newline

  Similarly, if $w = a_1b_1\cdots a_kb_k$, we select $a\in \Gamma_1\setminus \set{e_G,a_1^{-1}}$, and apply the argument to $awa^{-1}$, and if $w = b_1a_2b_2\cdots a_k$, we select $a\in \Gamma_1\setminus \set{e_G,a_k}$, and apply the argument to $awa^{-1}$.
\end{proof}
\begin{example}
  We can use the Ping Pong Lemma to see that
  \begin{align*}
    A &= \begin{pmatrix}1 & 2 \\ 0 & 1\end{pmatrix}\\
    B &= \begin{pmatrix}1 & 0 \\ 2 & 1\end{pmatrix}
  \end{align*}
  generate a subgroup of $SL\left(2,\Z\right)$ which is free of rank $2$.
\end{example}
\begin{corollary}
  The special orthogonal group $SO(3)$ contains a subgroup isomorphic to the free group on two generators.
\end{corollary}
To prove this, we state a different version of the Ping Pong Lemma that we will apply to a particular space.
\begin{theorem}[Ping Pong Lemma for Cyclic Groups]
  Let $G$ act on a set $X$, and suppose there exist disjoint subsets $A_{+},A_{-},B_{+},B_{-}\subseteq X$ whose union is not all of $X$. If there exist elements $a$ and $b$ in $G$ such that
  \begin{align*}
    a\cdot \left(X\setminus A_{-}\right) &\subseteq A_{+}\\
    a^{-1}\cdot \left(X\setminus A_{+}\right) &\subseteq A_{-}\\
    b \cdot \left(X\setminus B_{-}\right) &\subseteq B_{+}\\
    b\cdot \left(X\setminus B_{+}\right) &\subseteq B_{-},
  \end{align*}
  then it is the case that the group generated by $a$ and $b$ is free of rank $2$.
\end{theorem}
\begin{proof}[Proof of Corollary]
  We let
  \begin{align*}
    a &= \begin{pmatrix}3/5 & 4/5 & 0 \\ -4/5 & 3/5 & 0 \\ 0 & 0 & 1\end{pmatrix}\\
    a^{-1} &= \begin{pmatrix}3/5 & -4/5 & 0 \\ 4/5 & 3/5 & 0 \\ 0 & 0 & 1\end{pmatrix}\\
    b &= \begin{pmatrix}1 & 0 & 0 \\ 0 & 3/5 & -4/5 \\ 0 & 4/5 & 3/5\end{pmatrix}\\
    b^{-1} &= \begin{pmatrix}1 & 0 & 0 \\ 0 & 3/5 & 4/5 \\ 0 & -4/5 & 3/5\end{pmatrix}.
  \end{align*}
  We specify
  \begin{align*}
    X &= A_{+} \sqcup A_{-} \sqcup B_{+} \sqcup B_{-} \sqcup \begin{pmatrix}0\\1\\0\end{pmatrix},
  \end{align*}
  where
  \begin{align*}
    A_{+} &= \set{\frac{1}{5^{k}} \begin{pmatrix}x\\y\\z\end{pmatrix} | k\in \Z, x \equiv 3y\text{ modulo $5$}, z\equiv0\text{ modulo $5$}}\\
    A_{-} &= \set{\frac{1}{5^{k}} \begin{pmatrix}x\\y\\z\end{pmatrix} | k\in \Z, x \equiv -3y\text{ modulo $5$}, z\equiv 0\text{ modulo $5$}}\\
    B_{+} &= \set{\frac{1}{5^{k}} \begin{pmatrix}x\\y\\z\end{pmatrix} | k\in \Z, z \equiv 3y\text{ modulo $5$}, x\equiv 0\text{ modulo $5$}}\\
    B_{-} &= \set{\frac{1}{5^{k}} \begin{pmatrix}x\\y\\z\end{pmatrix} | k\in \Z, z \equiv -3y\text{ modulo $5$}, x\equiv 0\text{ modulo $5$}}.
  \end{align*}
  To verify that the conditions of the Ping Pong Lemma hold, we calculate
  \begin{align*}
    \begin{pmatrix}3/5 & 4/5 & 0 \\ -4/5 & 3/5 & 0 \\ 0 & 0 & 1\end{pmatrix}\left(\frac{1}{5^k} \begin{pmatrix}x\\y\\z\end{pmatrix}\right) &= \frac{1}{5^{k+1}} \begin{pmatrix}3x + 4y \\ -4x + 3y \\ 5z\end{pmatrix}\tag*{(1)}\\
    \begin{pmatrix}3/5 & -4/5 & 0 \\ 4/5 & 3/5 & 0 \\ 0 & 0 & 1\end{pmatrix} \left(\frac{1}{5^k} \begin{pmatrix}x\\y\\z\end{pmatrix}\right) &= \frac{1}{5^{k+1}} \begin{pmatrix}3x - 4y \\ 4x + 3y \\ 5z\end{pmatrix}\tag*{(2)}\\
    \begin{pmatrix}1 & 0 & 0 \\ 0 & 3/5 & -4/5 \\ 0 & 4/5 & 3/5\end{pmatrix}\left(\frac{1}{5^{k}} \begin{pmatrix}x\\y\\z\end{pmatrix}\right) &= \frac{1}{5^{k+1}} \begin{pmatrix}5x \\ 3y- 4z \\ 4y + 3z\end{pmatrix}\tag*{(3)}\\
    \begin{pmatrix}1 & 0 & 0 \\ 0 & 3/5 & 4/5 \\ 0 & -4/5 & 3/5\end{pmatrix} \left(\frac{1}{5^{k}} \begin{pmatrix}x\\y\\z\end{pmatrix}\right) &= \frac{1}{5^{k+1}} \begin{pmatrix}5x \\ 3y + 4z \\ -4y + 3z\end{pmatrix}.\tag*{(4)}
  \end{align*}
  We verify that the conditions for the Ping Pong Lemma hold for each of these four conditions.
  \begin{enumerate}[(1)]
    \item For any vector
      \begin{align*}
        \frac{1}{5^{k}} \begin{pmatrix}x\\y\\z\end{pmatrix} \notin A_{-},
      \end{align*}
      we see that $k+1\in \Z$, $x' = 3x + 4y \equiv 3\left(-4x + 3y\right)$  modulo $5$, and that $z' = 5z\equiv 0$ modulo $5$.
    \item For any vector
      \begin{align*}
        \frac{1}{5^{k}} \begin{pmatrix}x\\y\\z\end{pmatrix} \notin A_{+},
      \end{align*}
      we see that $k+1\in \Z$, $x' = 3x - 4y\equiv -3\left(4x + 3y\right)$ modulo $5$, and $z' = 5z \equiv 0$ modulo $5$.
    \item For any vector
      \begin{align*}
        \frac{1}{5^{k}} \begin{pmatrix}x\\y\\z\end{pmatrix}\notin B_{-},
      \end{align*}
      we see that $k+1\in \Z$, $z' = 4y + 3z \equiv 3\left(3y-4z\right)$ modulo $5$, and $x' = 5x\equiv 0$ modulo $5$.
    \item For any vector
      \begin{align*}
        \frac{1}{5^{k}} \begin{pmatrix}x\\y\\z\end{pmatrix}\notin B_{+},
      \end{align*}
      we see that $k+1\in \Z$, $z' = -4y + 3z \equiv -3\left(3y + 4z\right)$ modulo $5$, and $x' = 5x \equiv 0$ modulo $5$.
  \end{enumerate}
  Since we have verified that the conditions for the Ping Pong Lemma hold for each of the conditions, we have that $\set{a,b}\subseteq SO(3)$ generate a group isomorphic to the free group on two generators.
\end{proof}
\section{States and Means on $\ell_{\infty}(G)$}%
\begin{definition}
  Let $G$ be a group.
  \begin{enumerate}[(1)]
    \item The space $\mathcal{F}\left(G,\R\right)$ is defined by
      \begin{align*}
        \mathcal{F}\left(G,\R\right) = \set{f | f\colon G\rightarrow \R\text{ is a function}}.
      \end{align*}
    \item A function $f\in \mathcal{F}\left(G,\R\right)$ is positive if $f(x) \geq 0$ for all $x\in G$.
    \item A function $f\in \mathcal{F}\left(G,\R\right)$ is simple if $\Ran(f)$ is finite. We say
      \begin{align*}
        \Sigma &= \set{f\colon \mathcal{F}\left(G,\R\right) | f\text{ is simple}}.
      \end{align*}
  \end{enumerate}
\end{definition}
\begin{fact}
  $\Sigma\subseteq \mathcal{F}\left(G,\R\right)$ is a subspace. To see this, if $f,g$ are such that $\Ran(f),\Ran(g)$ are finite, and $\alpha \in \R$, then
  \begin{align*}
    \Ran\left(f + \alpha g\right) &\leq \Ran\left(f\right) + \Ran\left(g\right),
  \end{align*}
  so $f + \alpha g$ has finite range.
\end{fact}
\begin{definition}
  For $E\subseteq G$, set
  \begin{align*}
    \1_{E}\colon G\rightarrow \R
  \end{align*}
  defined by
  \begin{align*}
    \1_{E}\left(x\right) &= \begin{cases}
      1 & x\in E\\
      0 & x\notin E
    \end{cases}.
  \end{align*}
  This is the characteristic function of $E$.
\end{definition}
\begin{fact}
  \begin{align*}
    \Span\set{\1_{E} | E\subseteq G} &= \Sigma.
  \end{align*}
\end{fact}
\begin{proof}
  We see that $\1_{E}\in \Sigma$ for any $E\subseteq G$, and $\Sigma$ is a subspace.\newline

  If $\phi\in \Sigma$, with $\Ran\left(\phi\right) = \set{t_1,\dots,t_n}$ with $t_i$ distinct, we set
  \begin{align*}
    E_i &= \phi^{-1}\left(\set{t_i}\right),
  \end{align*}
  meaning
  \begin{align*}
    \phi &= \sum_{i=1}^{n}t_i\1_{E_i}.
  \end{align*}
\end{proof}
\begin{definition}\hfill
  \begin{enumerate}[(1)]
    \item A function $f\in \mathcal{F}\left(G,\R\right)$ is bounded if there exists $M > 0$ such that $\Ran(f) \subseteq \left[-M,M\right]$.
    \item The space $\ell_{\infty}(G)$ is defined by
      \begin{align*}
        \ell_{\infty}\left(G\right) &= \set{f\in \mathcal{F}\left(G,\R\right) | f\text{ is bounded}}.
      \end{align*}
    \item The norm on $\ell_{\infty}(G)$ is defined by
      \begin{align*}
        \norm{f} &= \sup_{x\in G}\left\vert f(x) \right\vert.
      \end{align*}
  \end{enumerate}
\end{definition}
\begin{proposition}
  The space $\ell_{\infty}(G)$ is complete, Additionally, $\overline{\Sigma} = \ell_{\infty}\left(G\right)$.
\end{proposition}
\begin{proof}
  Let $\left(f_n\right)_n$ be Cauchy. For $x\in G$, it is the case that
  \begin{align*}
    \left\vert f_n(x) - f_m(x) \right\vert &= \left\vert \left(f_n - f_m\right)(x) \right\vert\\
                                           &\leq \norm{f_n - f_m},
  \end{align*}
  meaning $\left(f_n(x)\right)_n$ is Cauchy in $\R$. We define $f(x) = \lim_{n\rightarrow\infty}f_n(x)$. We must show that $f\in \ell_{\infty}(G)$ and $\norm{f_n - f} \rightarrow 0$.
  \begin{align*}
    \left\vert f(x) \right\vert &= \left\vert \lim_{n\rightarrow\infty} f_n(x) \right\vert\\
                                &= \lim_{n\rightarrow\infty} \left\vert f_n(x) \right\vert\\
                                &\leq \limsup_{n\rightarrow\infty}\norm{f_n}\\
                                &\leq C,
  \end{align*}
  as Cauchy sequences are always bounded. Thus, $\sup_{x\in G}\left\vert f(x) \right\vert \leq C$.\newline

  Given $\ve > 0$, we find $N$ such that for all $m,n\geq N$, $\norm{f_n - f_m} \leq \ve$. Thus, for $x\in G$, we have
  \begin{align*}
    \left\vert f_n(x) - f)m(x) \right\vert &\leq \norm{f_n - f_m}\\
                                           &\leq \ve.
  \end{align*}
  Taking $m\rightarrow\infty$, we get $\left\vert f_n(x) - f(x) \right\vert \leq \ve$ for all $n\geq N$, meaning $\norm{f_n - f} \leq \ve$ for all $n\geq N$.\newline

  Now, for $f\in \ell_{\infty}(G)$, let $\Ran(f) \subseteq \left[-M,M\right]$ for some $M > 0$. Let $\ve > 0$. Since $\left[-M,M\right]$ is compact, it is totally bounded, so we can find intervals $I_1,\dots,I_n$ with $\left[-M,M\right] = \bigsqcup_{k=1}^{n}I_k$, with the length of each $I_k$ less than $\ve$.\newline

  Set $E_k = f^{-1}\left(I_k\right)$. Pick $t_k\in I_k$. Then, we set
  \begin{align*}
    \phi = \sum_{i=1}^{n}t_k \1_{E_k}.
  \end{align*}
  We see that $\norm{\phi - f} < \ve$.
\end{proof}
\begin{corollary}
  For any $f\in \ell_{\infty}(G)$, there is a sequence $\left(\phi_n\right)_n$ in $\Sigma$ with $\norm{\phi_n - f} \rightarrow 0$. If $f \geq 0$, then it is possible to select $\phi_n \geq 0$.
\end{corollary}
\begin{proposition}
  Let $G$ be a group. There is an action
  \begin{align*}
    G \xrightarrow{\lambda_s} \operatorname{Isom}\left(\ell_{\infty}(G)\right)
  \end{align*}
  defined by
  \begin{align*}
    \lambda_{s}\left(f\right)\left(t\right) &= f\left(s^{-1}t\right).
  \end{align*}
\end{proposition}
\begin{proof}
  We have
  \begin{align*}
    \lambda_{s}\left(f + \alpha g\right)\left(t\right) &= \left(f + \alpha g\right)\left(s^{-1}t\right)\\
                                                       &= f\left(s^{-1}t\right) + \alpha g\left(s^{-1}t\right)\\
                                                       &= \lambda_s\left(f\right)\left(t\right) + \alpha \lambda_s\left(g\right)\left(t\right)\\
                                                       &= \left(\lambda_s\left(f\right) + \alpha \lambda_s\left(g\right)\right)\left(t\right).
  \end{align*}
  Thus, $\lambda_s$ is a linear operator.\newline

  We have
  \begin{align*}
    \norm{\lambda_s(f)} &= \sup_{t\in G}\left\vert \lambda_s\left(f\right)\left(t\right) \right\vert\\
                        &= \sup_{t\in G}\left\vert f\left(s^{-1}t\right) \right\vert\\
                        &= \norm{f},
  \end{align*}
  hence
  \begin{align*}
    \norm{\lambda_s\left(f\right) - \lambda_s\left(f\right)} &= \norm{\lambda_s\left(f-g\right)}\\
                                                             &= \norm{f-g}.
  \end{align*}
  Thus, $\lambda_s$ is an isometry.\newline

  We have
  \begin{align*}
    \lambda_s\circ \lambda_r\left(f\right)\left(t\right) &= \lambda_r\left(f\right)\left(s^{-1}t\right)\\
                                                         &= f\left(r^{-1}s^{-1}t\right)\\
                                                         &= f\left(\left(sr\right)^{-1}t\right)\\
                                                         &= \lambda_{sr}\left(f\right)\left(t\right),
  \end{align*}
  meaning $\lambda_s\circ \lambda_r = \lambda_{sr}$.
\end{proof}
\begin{remark}
  By a similar process, we find that $\lambda_{s}\left(\1_{E}\right) = \1_{sE}$ for any subset $E\subseteq G$ and $s\in G$.
\end{remark}
\begin{definition}
  A state on $\ell_{\infty}\left(G\right)$ is a continuous linear functional $\mu\in \left(\ell_{\infty}\left(G\right)\right)^{\ast}$ that satisfies the following.
  \begin{enumerate}[(1)]
    \item $\mu$ is positive;
    \item $\mu\left(\1_{G}\right) = 1$.
  \end{enumerate}
  A state is called left-invariant if
  \begin{align*}
    \mu\left(\lambda_{s}\left(f\right)\right) &= \mu\left(f\right).
  \end{align*}
  
\end{definition}
\begin{example}
  Let $G$ be a group.
  \begin{itemize}
    \item If $x\in G$, then $\delta_{x}\colon \ell_{\infty}(G) \rightarrow \F$ defined by
      \begin{align*}
        \delta_{x}\left(f\right) &= f(x)
      \end{align*}
      is a state. However, note that it is not necessarily invariant.
      \begin{align*}
        \delta_{x}\left(\lambda_s\left(f\right)\right) &= \lambda_{s}\left(f\right)\left(x\right)\\
                                                       &= f\left(s^{-1}x\right)\\
                                                       &\neq f(x).
      \end{align*}
    \item If $G$ is finite, then
      \begin{align*}
        \mu &= \frac{1}{\left\vert G \right\vert} \sum_{x\in G}\delta_{x}
      \end{align*}
      is an invariant state.
  \end{itemize}
\end{example}
\begin{lemma}[Characterization of States]\hfill
  \begin{enumerate}[(1)]
    \item If $\mu$ is a state on $\ell_{\infty}\left(G\right)$, then
      \begin{align*}
        \norm{\mu}_{\text{op}} = 1.
      \end{align*}
    \item If $\mu\in \left(\ell_{\infty}\left(G\right)\right)^{\ast}$ is such that
      \begin{align*}
        \norm{\mu} &= \mu\left(\1_{G}\right)\\
                   &= 1,
      \end{align*}
      then $\mu$ is positive and a state.
  \end{enumerate}
\end{lemma}
\begin{proof}\hfill
  \begin{enumerate}[(1)]
    \item Given $f\in \ell_{\infty}\left(G\right)$, we have
      \begin{align*}
        \norm{f}\1_{G} - f &\geq 0\\
        \norm{f}\1_{G} + f &\geq 0,
      \end{align*}
      so
      \begin{align*}
        0 &\leq \mu\left(\norm{f}\1_{G} - f\right)\\
          &= \norm{f}\mu\left(\1_{G}\right) - \mu\left(f\right)\\
        0 &\leq \mu\left(\norm{f}\1_{G} + f\right)\\
          &= \norm{f}\mu\left(\1_{G}\right) + \mu\left(f\right).
      \end{align*}
      Thus, we have $\pm \mu\left(f\right) \leq \norm{f}\mu\left(\1_G\right) = \norm{f}$, so $\left\vert \mu\left(f\right) \right\vert \leq \norm{f}$, so $\norm{\mu}\leq 1$. Additionally, since $\mu\left(\1_{G}\right) = 1$, we must have $\norm{\mu} = 1$.
    \item Suppose $\norm{\mu} = \mu\left(\1_{G}\right) = 1$. Let $f \geq 0$. Set $g = \frac{1}{\norm{f}_{u}} f$.\newline

      Then, $\Ran\left(g\right) \subseteq [0,1]$, and $\Ran\left(g-\1_{G}\right) \subseteq [-1,1]$, so $\norm{g - \1_{G}}_{u} \leq 1$.\newline

      Since $\norm{\mu} = 1$, we must have
      \begin{align*}
        \left\vert \mu\left(g - \1_{G}\right) \right\vert &\leq 1\\
        \left\vert \mu\left(g\right) - 1 \right\vert &\leq 1,
      \end{align*}
      and since $\mu\left(\1_{G}\right) = 1$, we must have $\mu\left(g\right)\in [0,2]$, so $\mu(f) = \norm{f}\mu\left(g\right) \geq 0$.
  \end{enumerate}
\end{proof}
\begin{corollary}
  The set of states on $\left(\ell_{\infty}\left(G\right)\right)^{\ast}$ forms a $w^{\ast}$-compact subset of $B_{\left(\ell_{\infty}\left(G\right)\right)^{\ast}}$.
\end{corollary}
\begin{proof}
  It has been proven in \href{https://blog.avinashiyer.xyz/Classes_and_Homework/College/Y4/Honors%20Thesis/topological_vector_spaces.pdf}{functional analysis} that a convex subset of $\left(\ell_{\infty}\left(G\right)\right)^{\ast}$ is $w^{\ast}$-compact if it is norm bounded and $w^{\ast}$-closed. Since the set of states is convex and norm-bounded, all we need to show is that $S\left(\ell_{\infty}\left(G\right)\right)$ is $w^{\ast}$-closed.\newline

    To this end, let $f\in \ell_{\infty}(G)$ be positive and $\left(\varphi_i\right)_i$ be a net in $S\left(\ell_{\infty}\left(G\right)\right)$ with $\left(\varphi_i\right)_i\rightarrow \varphi$. We must show that $\varphi$ is positive and satisfies $\varphi\left(\1_{G}\right) = 1$. To this end, we see that
    \begin{align*}
      \varphi_i\left(f\right) &\geq 1
    \end{align*}
    for all $i\in I$, so we must necessarily have $\varphi\left(f\right) \geq 0$, and similarly, since $\varphi_i\left(\1_{G}\right) = 1$ for each $i\in I$, we also have $\varphi\left(\1_G\right) = 1$.
\end{proof}
\begin{proposition}
  If $\mu\in \left(\ell_{\infty}\left(G\right)\right)^{\ast}$ is a state, then $m\colon P(G) \rightarrow [0,1]$ defined by $m(E) = \mu\left(\1_{E}\right)$ is a finitely additive probability measure on $G$. Moreover, if $\mu$ is invariant, then $m$ is a mean.
\end{proposition}
\begin{proof}
  We have
  \begin{align*}
    m(G) &= \mu\left(\1_{G}\right)\\
         &= 1\\
    m\left(\emptyset\right) &= \mu\left(0\right)\\
                            &= 0\\
    m\left(E\sqcup F\right) &= \mu\left(\1_{E\sqcup F}\right)\\
                            &= \mu\left(\1_{E} + \1_{F}\right)\\
                            &= \mu\left(\1_{E}\right) + \mu\left(\1_{F}\right)\\
                            &= m\left(E\right) + m\left(F\right).
  \end{align*}
  Additionally, since $0\leq \1_{E}\leq \1_G$, we have $0\leq \mu\left(\1_{E}\right) \leq 1$, so $0 \leq m(E) \leq m(G) = 1$.\newline

  If $\mu$ is invariant, then
  \begin{align*}
    m\left(sE\right) &= \mu\left(\1_{sE}\right)\\
                     &= \mu\left(\lambda_{s}\left(\1_{E}\right)\right)\\
                     &= \mu\left(\1_{E}\right)\\
                     &= m\left(E\right).
  \end{align*}
\end{proof}
\begin{proposition}
  If $G$ admits a mean, then $\left(\ell_{\infty}\left(G\right)\right)^{\ast}$ admits an invariant state.
\end{proposition}
\begin{proof}
  Let $m$ be a finitely-additive probability measure. Define
  \begin{align*}
    \mu_0\colon \Sigma \rightarrow \R
  \end{align*}
  by
  \begin{align*}
    \mu_0\left(\sum_{k=1}^{n}t_k\1_{E_k}\right) &= \sum_{k=1}^{n}t_km\left(E_k\right).
  \end{align*}
  Since $m$ is finitely additive, it is the case that $\mu_0$ is well-defined, linear, and positive.\newline

  Note that $\mu_0\left(\1_{G}\right) = m(G) = 1$.\newline

  If $m$ is a mean, then for $f = \sum_{k=1}^{n}t_kE_k$,
  \begin{align*}
    \mu_0\left(\lambda_s\left(f\right)\right) &= \mu_{0}\left(\lambda_{s}\left(\sum_{k=1}^{n}t_k\1_{E_k}\right)\right)\\
                                              &= \mu_0\left(\sum_{k=1}^{n}t_k\1_{sE_k}\right)\\
                                              &= \sum_{k=1}^{n}t_km\left(sE_k\right)\\
                                              &= \sum_{k=1}^{n}t_km\left(E_k\right)\\
                                              &= \mu_0\left(f\right).
  \end{align*}
  Additionally, we see that
  \begin{align*}
    \left\vert \mu_0\left(f\right) \right\vert &= \left\vert \sum_{k=1}^{n}t_km\left(E_k\right) \right\vert\\
                                               &\leq \sum_{k=1}^{n}\left\vert t_k \right\vert m\left(E_k\right)\\
                                               &\leq \sum_{k=1}^{n}\norm{f}m\left(E_k\right)\\
                                               &= \norm{f}\sum_{k=1}^{n}m\left(E_k\right)\\
                                               &\leq \norm{f}.
  \end{align*}
  Thus, $\mu_0$ is continuous, so $\mu_0$ is uniformly continuous.\newline

  Since $\overline{\Sigma} = \ell_{\infty}(G)$, we see that $\mu_0$ extends to a continuous linear functional $\mu\colon \ell_{\infty}\left(G\right) \rightarrow \R$, with $\mu\left(\1_{G}\right) = \mu_{0}\left(\1_{G}\right) = 1$.\newline

  If $f \geq 0$, we find a sequence $\left(\phi_n\right)_n$ in $\Sigma$ with $\phi_n\geq 0$, $\norm{\phi_n - f}\xrightarrow{n\rightarrow\infty} 0$, and we set
  \begin{align*}
    \mu\left(f\right) &= \lim_{n\rightarrow\infty}\mu\left(\phi_n\right)\\
                      &= \lim_{n\rightarrow\infty}\mu_0\left(\phi_n\right)\\
                      &\geq 0,
  \end{align*}
  meaning $\mu$ is a state.\newline

  If $f\in \ell_{\infty}\left(G\right)$, $s\in G$, and $\left(\phi_n\right)_n$ in $\Sigma$ with $\left(\phi_n\right)_n\rightarrow f$, then
  \begin{align*}
    \norm{\lambda_s\left(\phi_n\right) - \lambda_s\left(f\right)} &= \norm{\lambda_s\left(\phi_n - f\right)}\\
                                                                  &= \norm{\phi_n - f}\\
                                                                  &\rightarrow 0
  \end{align*}
  Thus, we have
  \begin{align*}
    \mu\left(\lambda_s\left(\phi_n\right)\right) &= \mu_0\left(\lambda_s\left(\phi_n\right)\right)\\
                                                 &= \mu_0\left(\phi_n\right)\\
                                                 &= \mu\left(\phi_n\right)\\
                                                 &\rightarrow \mu\left(f\right),
  \end{align*}
  so $\mu\left(f\right) = \mu\left(\lambda_s\left(f\right)\right)$. Thus, $\mu\in \left(\ell_{\infty}\left(G\right)\right)^{\ast}$ is an invariant state.
\end{proof}
\section{Using Invariant States}%
\begin{proposition}
  $\Z$ is amenable.
\end{proposition}
\begin{proof}
  We know that $\lambda_{1}\colon \ell_{\infty}\left(\Z\right) \rightarrow \ell_{\infty}\left(\Z\right)$, defined by
  \begin{align*}
    \lambda_{1}\left(f\right)\left(k\right) &= f\left(k-1\right)
  \end{align*}
  is an isometry.\newline

  We set $Y = \Ran\left(\id - \lambda_{1}\right) \subseteq \ell_{\infty}\left(\Z\right)$.\newline

  We claim that $\dist_{Y}\left(\1_{\Z}\right) \geq 1$.\newline

  Suppose toward contradiction that there is $y\in Y$ with $\norm{\1_{\Z} - y}_{u} = \rho < 1$. Then, $y = f - \lambda_1\left(f\right)$ for some $f\in \ell_{\infty}\left(\Z\right)$, meaning
  \begin{align*}
    \norm{\1 - \left(f - \lambda_1\left(f\right)\right)} &= \rho.
  \end{align*}
  Thus, for all $k\in \Z$, we have
  \begin{align*}
    \left\vert 1 - \left(f(k) - f\left(k-1\right)\right) \right\vert \leq \rho,
  \end{align*}
  meaning $\left\vert f\left(k\right) - f\left(k-1\right) \right\vert \geq 1-\rho > 0$. However, such an $f$ cannot be bounded.\newline

  Since $\dist_{\overline{Y}}\left(\1_{Z}\right) = \dist_{Y}\left(\1_{Z}\right) \geq 1$, the Hahn--Banach theorem provides $\mu\in \left(\ell_{\infty}\left(\Z\right)\right)^{\ast}$ with $\norm{\mu} = 1$, $\mu|_{\overline{Y}} = 0$, and $\mu\left(\1\right) = \dist_{Y}\left(\1_{\Z}\right) \geq 1$.\newline

  Since $\norm{\mu} = 1$ and $\mu\left(\1\right) \geq 1$, we must have $\mu\left(\1\right) = 1$.\newline

  Since $\norm{\mu} = \mu\left(\1_{\Z}\right) = 1$, it is the case that $\mu$ is a state on $\ell_{\infty}\left(\Z\right)$. Since $\mu\left(y\right) = 0$ for all $y\in Y$, we have
  \begin{align*}
    \mu\left(f - \lambda_1(f)\right) = 0\\
    \mu\left(f\right) &= \mu\left(\lambda_1(f)\right), 
  \end{align*}
  so inductively, we have $\mu\left(f\right) = \mu\left(\lambda_k\left(f\right)\right)$ for all $k\in \Z$, meaning $\mu$ is an invariant state on $\ell_{\infty}\left(\Z\right)$. Thus, $\Z$ is amenable.
\end{proof}
\begin{proposition}
  If $N\trianglelefteq G$ and $G/N$ are amenable, then $G$ is amenable.
\end{proposition}
\begin{proof}
  Let $\rho\in \left(\ell_{\infty}\left(G/N\right)\right)^{\ast}$ be an invariant state, and $p\colon  P(N)\rightarrow [0,1]$. For $E\subseteq G$, we define
  \begin{align*}
    f_E\colon G/N\rightarrow \R
  \end{align*}
  by $f_E\left(tN\right) = p\left(N\cap t^{-1}E\right)$.\newline

  We verify that this is well-defined --- for $tN = sN$, we have $s^{-1}t\in N$, so
  \begin{align*}
    p\left(N\cap t^{-1}E\right) &= p\left(s^{-1}t\left(N\cap t^{-1}E\right)\right)\\
                                &= p\left(s^{-1}TN\cap s^{-1}E\right)\\
                                &= p\left(N\cap s^{-1}E\right).
  \end{align*}
  We also see that $f_E$ is bounded, and
  \begin{align*}
    f_{E\sqcup F} \left(tN\right) &= p\left(N\cap t^{-1}\left(E\sqcup F\right)\right)\\
                                  &= p\left(N\cap \left(t^{-1}E\sqcup t^{-1}F\right)\right)\\
                                  &= p\left(\left(N\cap t^{-1}E\right)\sqcup \left(N\cap t^{-1}F\right)\right)\\
                                  &= p\left(N\cap t^{_1}E\right) + p\left(N\cap t^{-1}F\right)\\
                                  &= f_E\left(tN\right) + f_{F}\left(tN\right)\\
                                  &= \left(f_E + f_F\right)\left(tN\right).
  \end{align*}
  Thus, $f_{E\sqcup F} = f_E + f_F$.\newline

  Additionally,
  \begin{align*}
    f_{sE}\left(tN\right) &= p\left(N\cap t^{-1}sE\right)\\
                          &= f_E\left(s^{-1}t N\right)\\
                          &= \lambda_{sN}\left(f_E\right)\left(tN\right),
  \end{align*}
  so $f_{sE} = \lambda_{sN}\left(f_{E}\right)$.\newline

  Finally,
  \begin{align*}
    f_G\left(tN\right) &= p\left(N\cap t^{-1}G\right)\\
                       &= p\left(N\right)\\
                       &= 1,
  \end{align*}
  so $f_{G} = \1_{G/N}$.\newline

  We define $m\colon P(G) \rightarrow [0,1]$ by
  \begin{align*}
    m(E) &= \rho\left(f_{E}\right).
  \end{align*}
  Then, we have
  \begin{align*}
    m\left(E\sqcup F\right) &= m\left(E\right) + m\left(F\right)\\
    m\left(G\right) &= 1\\
    m\left(sE\right) &= \rho\left(f_{sE}\right)\\
                     &= \rho\left(\lambda_{sN}\left(f_{E}\right)\right)\\
                     &= \rho\left(f_{E}\right)\\
                     &= m\left(E\right),
  \end{align*}
  meaning $m$ is a mean.
\end{proof}
\begin{corollary}
  The finite direct product of amenable groups is amenable.
\end{corollary}
\begin{proof}
  If $H$ and $K$ are amenable, then we know that
  \begin{align*}
    K &\cong \frac{H\times K}{H}
  \end{align*}
  is amenable, and $H$ is amenable, so $H\times K$ is amenable. 
\end{proof}
\begin{corollary}[]
  Finitely generated abelian groups are amenable.
\end{corollary}
\begin{proof}
  All finitely generated abelian groups are isomorphic to $\Z^{d}\times \Z/n_1\Z\times\cdots\times \Z/n_k\Z$ by the Fundamental Theorem of Finitely Generated Abelian Groups. Since $\Z^{d}$ is a finite direct product of $\Z$ (which is amenable), and the torsion group $\Z/n_1\Z\times\cdots \Z/n_k\Z$ is finite, we have that a finitely generated abelian group is amenable.
\end{proof}
\begin{corollary}
  If $\set{G_i}_{i\in I}$ is a directed family of amenable groups --- i.e., that for any two groups $G_j$ and $G_k$, there is $G_{\ell}$ with $G_j\subseteq G_{\ell}$ and $G_k\subseteq G_{\ell}$ --- then the direct union,
  \begin{align*}
    G &= \bigcup_{i\in I}G_i,
  \end{align*}
  is also amenable.
\end{corollary}
\begin{proof}
  Let $\mu_{i}\in \left(\ell_{\infty}\left(G_i\right)\right)^{\ast}$ be the respective invariant states.\newline

  Set
  \begin{align*}
    M_i &= \set{\mu\in S\left(\left(\ell_{\infty}\left(G\right)\right)^{\ast}\right) | \mu\left(\lambda_{s}\left(f\right)\right) = \mu\left(f\right)\text{ for all $s\in G_i$}},
  \end{align*}
  and set $\mu(f) = \mu_{i}\left(f\vert_{G_i}\right)$. We see that $M_i$ is $w^{\ast}$-closed in $B_{\left(\ell_{\infty}\left(G\right)\right)^{\ast}}$, as we have established the state space as a $w^{\ast}$-closed subset of $B_{\left(\ell_{\infty}\left(G\right)\right)^{\ast}}$.\newline

  For $i_1,\dots,i_n$, we find $G_j\supseteq G_{i_1},\dots,G_{i_n}$, which necessarily exists as $\set{G_i}_{i\in I}$ is directed. Thus, $M_j\subseteq \bigcap_{k=1}^{n}M_{i_k}$, meaning $\set{M_i}_{i\in I}$ has the finite intersection property.\newline

  By compactness, there is $\mu\in \bigcap_{i\in I}M_{i}$, meaning $\mu$ is an invariant state.
\end{proof}
\begin{corollary}
  All abelian groups are amenable.
\end{corollary}
\begin{proof}
  Every abelian group is the direct union of its finitely generated subgroups.
\end{proof}
\begin{corollary}
  All solvable groups are amenable.
\end{corollary}
\begin{proof}
  Let $e_G = G_0 \leq G_1\leq\cdots\leq G_n = G$ be such that $G_{j-1}\trianglelefteq G_j$ for $j=1,\dots,n$, and $G_j/G_{j-1}$ abelian.\newline

  Since $G_0$ is abelian, it is amenable. Similarly, $G_1/G_0$ is abelian, so it is amenable, so $G_1$ is amenable. Continuing in this fashion, we see that $G$ is amenable.
\end{proof}
\section{Følner's Condition and Invariant Approximate Means}%
\begin{definition}
  A group $G$ is said to satisfy the Følner condition if, for every $\ve > 0$, and for all $E\subseteq G$ finite, there is a nonempty $F\subseteq G$ finite such that for all $t\in E$,
  \begin{align*}
    \frac{\left\vert tF\triangle F \right\vert}{\left\vert F \right\vert} &\leq \ve.
  \end{align*}
  Equivalently, since
  \begin{align*}
    \frac{\left\vert tF\triangle F \right\vert}{\left\vert F \right\vert} &= 2\left(1-\frac{\left\vert tF\cap F \right\vert}{\left\vert F \right\vert}\right),
  \end{align*}
  we have the equivalent formulation that
  \begin{align*}
    \frac{\left\vert tF\triangle F \right\vert}{\left\vert F \right\vert} \leq \ve \text{ if and only if } 1-\frac{\left\vert tF\cap  \right\vert}{\left\vert f \right\vert} \leq \ve/2.
  \end{align*}
  Thus, $G$ satisfies the Følner condition if and only if, for all $\ve > 0$ and for all finite $E\subseteq G$, there exists a nonempty $F\subseteq G$ with
  \begin{align*}
    \frac{\left\vert tF\cap F \right\vert}{\left\vert f \right\vert} &\geq 1-\ve.
  \end{align*}
\end{definition}
\begin{example}
  All finite groups satisfy Følner's condition by taking $F = G$ for each subset $E\subseteq G$.
\end{example}
\begin{lemma}
  A countable group $G$ satisfies the Følner condition if and only if $G$ admits a Følner sequence, $\left(F_n\right)_n$ with $F_n\subseteq G$ finite, such that
  \begin{align*}
    \left(\frac{\left\vert tF_n\triangle F_n \right\vert}{\left\vert F_n \right\vert}\right)_n \xrightarrow{n\rightarrow\infty} 0,
  \end{align*}
  or equivalently,
  \begin{align*}
    \left(\frac{\left\vert tF_n\cap F_n \right\vert}{\left\vert F_n \right\vert}\right)_n \xrightarrow{n\rightarrow\infty}1,
  \end{align*}
  for all $t$ in $G$.
\end{lemma}
\begin{proof}
  Let $G$ admit a Følner sequence, $\left(F_n\right)_n$. Given $\ve > 0$, and $E\subseteq G$ finite, find $N$ such that for all $s\in E$ and $n\geq N$,
  \begin{align*}
    \frac{\left\vert sF_n\triangle F_n \right\vert}{\left\vert F_n \right\vert} \leq \ve.
  \end{align*}
  We take $F = F_N$.\newline

  Let $G$ satisfy the Følner condition. We write $G = \bigcup_{n\geq 1}E_n$, with $E_1\subseteq E_2\subseteq E_3\subseteq\cdots$, and define $F_n$ such that for all $t\in E_n$,
  \begin{align*}
    \frac{\left\vert tF_n\triangle F_n \right\vert}{\left\vert F_n \right\vert} &\leq \frac{1}{n}.
  \end{align*}
  Then, given $t\in G$, it is the case that $t\in E_N$ for some $N$, so $t\in E_n$ for all $n\geq N$, so
  \begin{align*}
    \frac{\left\vert tF_n\triangle F_n \right\vert}{\left\vert F_n \right\vert} &\leq \frac{1}{n}
  \end{align*}
  for all $n\geq N$, meaning that
  \begin{align*}
    \frac{\left\vert tF_n\triangle F_n \right\vert}{\left\vert F_n \right\vert} &\xrightarrow{n\rightarrow\infty}0.
  \end{align*}
\end{proof}
\begin{lemma}
  Let $G$ be a finitely generated group with generating set $S$ (where $S$ may not be symmetric\footnote{Closed under inversion.}). If $\left(F_n\right)_n$ is a sequence of finite subsets of $G$ such that
  \begin{align*}
    \left(\frac{\left\vert sF_n\triangle F_n \right\vert}{\left\vert F_n \right\vert}\right)_{n}\rightarrow 0
  \end{align*}
  for all $s\in S$, then $\left(F_n\right)_n$ is a Følner sequence for $G$.
\end{lemma}
\begin{proof}
  We start by showing that we can assume $S$ to be symmetric. The following are both true:
  \begin{itemize}
    \item $s\left(A\triangle B\right) = sA\triangle sB$;
    \item $A\triangle C\subseteq \left(A\triangle B\right)\cup \left(B\triangle C\right)$. 
  \end{itemize}
  Thus, if we have $s^{-1}$ rather than $s$, our assumption provides, for all $s\in S$,
  \begin{align*}
    \frac{\left\vert s^{-1}F_n\triangle F_n \right\vert}{\left\vert F_n \right\vert} &= \frac{s^{-1}\left(F_n\triangle sF_n\right)}{\left\vert F_n \right\vert}\\
                                                                                     &= \frac{\left\vert F_n\triangle sF_n \right\vert}{\left\vert F_n \right\vert}\\
                                                                                     &\xrightarrow{n\rightarrow \infty}0.
  \end{align*}
  Thus, we may assume $S$ is symmetric.\newline

  For $s,t\in F_n$, we have
  \begin{align*}
    \frac{\left\vert stF_n\triangle F_n \right\vert}{\left\vert F_n \right\vert} &\leq \frac{\left\vert StF_n\triangle sF_n \right\vert}{F_n} + \frac{\left\vert sF_n\triangle F_n \right\vert}{\left\vert F_n \right\vert}\\
                                                                                 &= \frac{\left\vert s\left(tF_n\triangle F_n\right) \right\vert}{\left\vert F_n \right\vert} + \frac{\left\vert sF_n\triangle F_n \right\vert}{\left\vert F_n \right\vert}\\
                                                                                 &= \frac{\left\vert tF_n\triangle F_n \right\vert}{\left\vert F_n \right\vert} + \frac{\left\vert sF_n\triangle F_n \right\vert}{\left\vert F_n \right\vert}\\
                                                                                 &\xrightarrow{n\rightarrow\infty} 0.
  \end{align*}
  We use induction to find the general case.
\end{proof}
\begin{example}
  Considering $\Z$ again, we remember that $\set{1}$ is the generating set for $\Z$. If we let $F_n = \set{-n,-n+1,\dots,-1,0,1,\dots,n}$, we have
  \begin{align*}
    \frac{\left\vert \left(F_{n} + 1\right)\triangle F_n \right\vert}{\left\vert F_n \right\vert} &= \frac{2}{2n+1}\\
                                                                                                  &\xrightarrow{n\rightarrow\infty} 0.
  \end{align*}
  Thus, we have that $\Z$ satisfies the Følner condition.
\end{example}
We now turn our attention to approximate means, from which with Følner's condition, we will be able to construct a different, equivalent condition for group amenability.
\begin{definition}
  If $G$ is a group, we define
  \begin{align*}
    \Prob(G) &\coloneq \set{f\colon G \rightarrow [0,\infty) | \left\vert \supp(f) \right\vert < \infty,~\sum_{t\in G}f(t) = 1 }.
  \end{align*}
\end{definition}
Note that $\Prob(G) \subseteq B_{\ell_1(G)}$. Given $f\in \Prob(G)$, we set
\begin{align*}
  \varphi_{f}\colon \ell_{\infty}\left(G\right) \rightarrow \C,
\end{align*}
defined by
\begin{align*}
  \varphi_{f}\left(g\right) &= \sum_{t\in G}g(t)f(t).
\end{align*}
We claim that $\varphi_{f}$ is a state on $\ell_{\infty}\left(G\right)$.
\begin{proof}
  If $g\geq 0$, then $\varphi_{f}\left(g\right) \geq 0$, and $\varphi_{f}\left(\1_{G}\right) = 1$. It is also clear that $\varphi_f$ is linear.\newline

  We only need show that $\norm{\varphi_f} = 1$. We see
  \begin{align*}
    \left\vert \varphi_{f}\left(g\right) \right\vert &= \left\vert \sum_{t\in G}g(t)f(t) \right\vert\\
                                                     &\leq \sum_{t\in G}\left\vert g(t) \right\vert\left\vert f(t) \right\vert\\
                                                     &\leq \norm{g}_{\infty}\sum_{t}\left\vert f_t \right\vert\\
                                                     &= \norm{g}_{\infty}.
  \end{align*}
\end{proof}
\begin{proposition}
  There is an action $\lambda\colon G\xrightarrow \Isom\left(\ell_{1}\left(G\right)\right)$ such that $\Prob(G)$ is invariant.
\end{proposition}
\begin{proof}
  We let $\lambda_{s}\left(f\right)\left(t\right) = f\left(s^{-1}t\right)$. Then,
  \begin{align*}
    \norm{\lambda_s\left(f\right)}_{1} &= \sum_{t\in G}\left\vert \lambda_s\left(f\right)\left(t\right) \right\vert\\
                                       &= \sum_{t\in G}\left\vert f\left(s^{-1}t\right) \right\vert\\
                                       &= \sum_{r\in G}\left\vert f(r) \right\vert\\
                                       &= \norm{f}.
  \end{align*}
  We also see that $\lambda_s$ is linear.\newline

  Additionally,
  \begin{align*}
    \lambda_r\circ\lambda_s\left(f\right)\left(t\right) &= \lambda_s\left(f\right)\left(r^{-1}t\right)\\
                                                        &= f\left(s^{-1}r^{-1}t\right)\\
                                                        &= f\left(\left(rs\right)^{-1}t\right)\\
                                                        &= \lambda_{rs}\left(f\right)\left(t\right).
  \end{align*}
  We see that if $f\in \Prob(G)$, then for $f\geq 0$, we have $\lambda_s\left(f\right) \geq 0$, and
  \begin{align*}
    \supp\left(\lambda_s\left(f\right)\right) &= s\left(\supp\left(f\right)\right),
  \end{align*}
  which is also finite.\newline

  Thus, 
  \begin{align*}
    \sum_{t\in G}\lambda_s\left(f\right)\left(t\right) &= \sum_{t\in G}f\left(s^{-1}t\right)\\
                                                       &= \sum_{r\in G}f(r)\\
                                                       &= 1
  \end{align*}
  for $f\in \Prob(G)$.
\end{proof}
\begin{definition}
  For a countable group $G$, a sequence $\left(f_k\right)_k$ in $\Prob(G)$ is an approximate invariant mean if, for all $s\in G$,
  \begin{align*}
    \norm{f_k - \lambda_s\left(f_k\right)}_{1}\xrightarrow{k\rightarrow\infty}0.
  \end{align*}
\end{definition}
\begin{proposition}
  If $G$ admits a Følner sequence $\left(F_k\right)_k$, then it admits an approximate mean.
\end{proposition}
\begin{proof}
  Set $f_k = \frac{1}{\left\vert F_k \right\vert}\1_{F_k}\in \Prob(G)$. Then,
  \begin{align*}
    \norm{f_k - \lambda_{s}\left(f_k\right)}_{1} &= \frac{1}{\left\vert F_k \right\vert} \norm{\1_{F_k} - \lambda_{s}\left(\1_{F_k}\right)}\\
                                                 &= \frac{1}{\left\vert F_k \right\vert}\norm{\1_{F_k} - \1_{sF_k}}\\
                                                 &= \frac{\left\vert F_k \triangle sF_k \right\vert}{\left\vert F_k \right\vert},
  \end{align*}
  which thus converges to $0$ as $k\rightarrow \infty$.
\end{proof}
\begin{proposition}
  If $G$ has an approximate mean, then $G$ is amenable.
\end{proposition}
\begin{proof}
  Let $\left(f_k\right)_k$ be an approximate mean. We define $\varphi_k = \left(\varphi_{f_k}\right)_k$ to be a sequence of states on $\ell_{\infty}\left(G\right)$.\newline

  Since the state space on $\ell_{\infty}\left(G\right)$ is $w^{\ast}$-compact, there is a state $\mu$ and a subnet $\left(\varphi_{k_j}\right)_{j}$ with $\left(\varphi_{k_j}\right)_{j}\xrightarrow{w^{\ast}}\mu$.\newline

  We only need to show that $\mu$ is invariant. Note that
  \begin{align*}
    \left\vert \mu\left(g\right) - \mu\left(\lambda_s\left(g\right)\right) \right\vert &\leq \left\vert \mu_{g} - \varphi_{k_j}\left(g\right) \right\vert + \left\vert \varphi_{k_j}\left(g\right) - \varphi_{k_j}\left(\lambda_s\left(g\right)\right) \right\vert + \left\vert \varphi_{k_j}\left(\lambda_{s}\left(g\right)\right) - \mu\left(\lambda_{s}\left(g\right)\right) \right\vert
  \end{align*}
  holds for all $g\in \ell_{\infty}\left(G\right)$, $s\in G$, and for all $j$.\newline

  Given $\ve > 0$, we find $J$ such that for $j\geq J$, we have
  \begin{align*}
    \left\vert \mu\left(g\right) - \varphi_{k_j}\left(g\right) \right\vert &< \ve/3\\
    \left\vert \mu\left(\lambda_s\left(g\right)\right) - \varphi_{k_j}\left(\lambda_s\left(g\right)\right) \right\vert &< \ve/3.
  \end{align*}
  We see that
  \begin{align*}
    \left\vert \varphi_{k_j}\left(g\right) - \varphi_{k_j}\left(\lambda_s\left(g\right)\right) \right\vert &= \left\vert \sum_{t\in G}g(t)f_{k_j}\left(t\right) - \sum_{t\in G}g\left(s^{-1}t\right)f_{k_j}\left(t\right) \right\vert\\
                                                                                                           &= \left\vert \sum_{t\in G}g(t)f_{k_j}\left(t\right) - \sum_{r\in G}g(r)f_{k_j}\left(sr\right) \right\vert \tag*{$r = s^{-1}t$}\\
                                                                                                           &= \left\vert \sum_{t\in G}g(t)\left(f_{k_j}\left(t\right)-\lambda_{s^{-1}}\left(f_{k_j}\right)\left(t\right)\right) \right\vert\\
                                                                                                           &\leq \norm{g}_{\infty}\sum_{t\in G}\left\vert f_{k_j}\left(t\right) - \lambda_{s^{-1}}\left(f_{k_j}\right)\left(t\right) \right\vert\\
                                                                                                           &= \norm{g}_{\infty}\norm{f_{k_j} - \lambda_{s^{-1}}\left(f_{k_j}\right)}\\
                                                                                                           &< \ve/3
  \end{align*}
  for large $j$. Thus, we have
  \begin{align*}
    \left\vert \mu\left(g\right) - \mu\left(\lambda_{s}\left(g\right)\right) \right\vert &< \ve,
  \end{align*}
  for all $\ve > 0$, so $\mu\left(g\right) = \mu\left(\lambda_{s}\left(g\right)\right)$.
\end{proof}
\subsection{Equivalence between Means and Approximate Means}%
We wish to show that if $G$ is amenable, then $G$ has an approximate mean.
\begin{theorem}
  Let $G$ be amenable. Then, $G$ has an approximate mean.
\end{theorem}
\begin{proof}
  Recall that a net $\left(f_i\right)_{i\in I}$ in $\Prob(G)$ has its domain as the set
  \begin{align*}
    I &= \set{\left(E,\ve\right) | E\subseteq G\text{ finite,} \ve > 0},
  \end{align*}
  directed by $\left(E,\ve\right) \leq \left(E',\ve'\right)$ if $E \subseteq E'$ and $\ve \geq \ve'$.\newline

  Suppose there is no approximate mean. Then, there exists a finite subset $E_0\subseteq G$ and $\ve_0 > 0$ such that for all $s\in E_0$ and $f\in \Prob(G)$, we have
  \begin{align*}
    \norm{f - \lambda_s\left(f\right)} \geq \ve_0.
  \end{align*}
  Let $X = \bigoplus_{\left\vert E_0 \right\vert}\ell_1\left(G\right)$ be endowed with the $1$-norm.\newline

  Consider the set
  \begin{align*}
    C &= \set{\left(f - \lambda_s\left(f\right)\right)_{s\in E_0} | f\in \Prob(G)}.
  \end{align*}
  Since $\Prob(G)$ is convex, $C$ is convex, and since $\left\vert E_0 \right\vert$ is finite, $C$ is necessarily bounded. Additionally, note that it is the case that $0\notin \overline{C}$.\newline

  By the Hahn--Banach separation theorem, there is a real-valued $\varphi\in X^{\ast}$ such that $\varphi\left(C\right) \geq 1$.\newline

  Note also that
  \begin{align*}
    X^{\ast} &\cong \bigoplus_{\left\vert E_0 \right\vert}\ell_{1}\left(G\right)^{\ast}\\
             &\cong \bigoplus_{\left\vert E_0 \right\vert}\ell_{\infty}\left(G\right)
  \end{align*}
  with the $\infty$ norm. We let $\varphi = \left(\varphi_{g_s}\right)_{s\in E_0}$ with $g_s\in \ell_{\infty}\left(G\right)$. From the duality between $\ell_{1}\left(G\right)$ and $\ell_{\infty}\left(G\right)$, for any $f\in \ell_{1}\left(G\right)$ and $s\in E_0$, we have
  \begin{align*}
    \varphi_{g_s}\left(f\right) &= \sum_{t\in G}f(t)g(t).
  \end{align*}
  Thus, for all $f\in \Prob(G)$, we have
  \begin{align*}
    1 &\leq \varphi\left(\left(f-\lambda_s\left(f\right)\right)_{s\in E_0}\right)\\
      &= \sum_{s\in E_0}\varphi_{g_s}\left(f-\lambda_s\left(f\right)\right)\\
      &= \sum_{s\in E_0}\sum_{t\in G}\left(f-\lambda_s\left(f\right)\right)\left(t\right)g_s\left(t\right)\\
      &= \sum_{s\in E_0}\left(\sum_{t\in G}f(t)g_s\left(t\right) - \sum_{t\in G}f\left(s^{-1}t\right)g_s(t)\right)\\
      &= \sum_{s\in E_0}\left(\sum_{t\in G}f(t)g_s\left(t\right) - \sum_{r\in G}f\left(r\right)g\left(sr\right)\right) \tag*{$s^{-1}t = r$}\\
      &= \sum_{s\in E_0}\left(\sum_{r\in T}f\left(r\right)g_s\left(r\right) - \sum_{r\in G}f\left(r\right)\lambda_{s^{-1}}\left(g\right)\left(r\right)\right)\\
      &= \sum_{s\in E_0}\sum_{r\in G}f\left(r\right)\left(g_s-\lambda_{s^{-1}}\left(g\right)\right)\left(r\right).
  \end{align*}
  Note that this holds for any $f\in \Prob(G)$. In particular, if $f = \delta_t$ for a given $t\in \Prob(G)$, then we must have
  \begin{align*}
    \sum_{s\in E_0}\sum_{r\in G}f\left(r\right)\left(g_s\left(r\right) - \lambda_{s^{-1}}\left(g_s\right)\left(r\right)\right) &= \sum_{s\in E_0}\sum_{r\in G}\delta_t\left(r\right)\left(g_s\left(r\right) - \lambda_{s^{-1}}\left(g_s\right)\left(r\right)\right)\\
                                                                                                                               &= \sum_{s\in E_0}\left(g_s - \lambda_{s^{-1}}g_s\right)\left(t\right)\\
                                                                                                                               &\geq 1.
  \end{align*}
  Since $G$ is amenable, there is a  mean $\mu: \ell_{\infty}\left(G\right)\rightarrow \C$ with $\mu\left(g_s\right) = \mu\left(\lambda_{s^{-1}}\left(g_s\right)\right)$, meaning
  \begin{align*}
    \mu\left(\sum_{s\in E_0}\left(g_s - \lambda_{s^{-1}}\left(g_s\right)\right)\left(t\right)\right) &= 0\\
                                                                                                     &\geq 1,
  \end{align*}
  which is a contradiction.
\end{proof}
\section{Growth Rates and Amenability}%
\begin{definition}
  Let $G$ be a group, and let $S$ be a finite symmetric generating set (i.e., $\left\langle S \right\rangle = G$ and $s\in S \Leftrightarrow s^{-1}\in S$) for $G$. We define
  \begin{align*}
    \ell_{G,S}\left(g\right) &= \min\set{n | g = s_1\cdots s_n | s_i\in S},
  \end{align*}
  where $\ell_{G,S}\left(e_G\right) = 0$.
\end{definition}
Some easy facts we can see from this definition:
\begin{itemize}
  \item $\ell_{G,S}\left(g\right) = \ell_{G,S}\left(g^{-1}\right)$;
  \item $\ell_{G,S}\left(gh\right) \leq \ell_{G,S}\left(g\right) + \ell_{G,S}\left(h\right)$.
\end{itemize}
The following is a more substantive fact that we will use in our discussion of growth rates and amenability. Specifically, this will allow us to talk about ``a'' growth rate, as all generating sets are, in a sense, equivalent.
\begin{fact}
  Let $S,T$ be finite symmetric generating sets for $G$. Then, there exists some $K\in \N$ such that, for all $g\in G$,
  \begin{align*}
    \frac{1}{K}\ell_{G,S}\left(g\right) \leq \ell_{G,T}\left(g\right) \leq K \ell_{G,S}\left(g\right).
  \end{align*}
\end{fact}
\begin{proof}
  Let
  \begin{align*}
    M &= \max\set{\ell_{G,T}\left(s\right) | s\in S}\\
    N &= \max\set{\ell_{G,S}\left(t\right) | t\in T}.
  \end{align*}
  Let $n = \ell_{G,S}\left(g\right)$, such that $g = s_1\cdots s_n$ for $s_i\in S$. Then, we see that
  \begin{align*}
    \ell_{G,T}\left(g\right) &= \ell_{G,T}\left(s_1,\dots,s_n\right)\\
                             &\leq \ell_{G,T}\left(s_1\right) + \cdots + \ell_{G,T}\left(s_n\right)\\
                             &\leq nM\\
                             &= M\ell_{G,S}\left(g\right).
  \end{align*}
  Similarly, we find that $\ell_{G,S}\left(g\right) \leq N\ell_{G,T}\left(g\right)$. We set $K = \max\left(M,N\right)$, and find 
  \begin{align*}
    \frac{1}{K}\ell_{G,S}\left(g\right) \leq \ell_{G,T}\left(g\right) \leq K\ell_{G,S}\left(g\right).
  \end{align*}
\end{proof}
Since $\ell_{G,S}\left(g\right)$ is, in a sense, a ``norm'' on $G$, we may also define a metric --- the word metric --- with respect to the generating set $S$.
\begin{fact}
  Let $S$ be a finite symmetric generating set for $G$. 
  \begin{align*}
    d_{S}\left(g,h\right) &= \ell_{G,S}\left(g^{-1}h\right),
  \end{align*}
  we obtain a metric on $G$. If $S$ and $T$ are finite symmetric generating sets for $G$, then the metrics $d_S$ and $d_T$ are equivalent.
\end{fact}
\begin{proof}
  \begin{align*}
    d_S\left(g,h\right) &= \ell_{G,S}\left(g^{-1}h\right) \\
                        &= \ell_{G,S}\left(h^{-1}g\right)\\
                        &= d_S\left(h,g\right)\\
                        \\
    d_{S}\left(g,h\right) &= \ell_{G,S}\left(g^{-1}h\right)\\
                          &= \ell_{G,S}\left(g^{-1}kk^{-1}h\right)\\
                          &\leq \ell_{G,S}\left(g^{-1}k\right) + \ell_{G,S}\left(k^{-1}h\right)\\
                          &= d_{S}\left(g,k\right) + d_{S}\left(k,h\right).\\
                          \\
    d_{S}\left(g,g\right) &= \ell_{G,S}\left(g^{-1}g\right)\\
                          &= \ell_{G,S}\left(e\right)\\
                          &= 0\\
    d_S\left(g,h\right) = 0 &\Leftrightarrow \ell_{G,S}\left(g^{-1}h\right) = 0\\
                            &\Leftrightarrow g^{-1}h = e\\
                            &\Leftrightarrow g = h.
  \end{align*}
  The equivalence of the induced metrics follows from the previous fact.
\end{proof}
\begin{definition}
  Let $G$ be a group with finite generating symmetric set $S$. Let $n\geq 0$. Let
  \begin{align*}
    B_{G,S}\left(n\right) &= \set{g\in G | \ell_{G,S}\leq n}\\
    \gamma_{G,S}\left(n\right) &= \left\vert B_{G,S}\left(n\right) \right\vert.
  \end{align*}
\end{definition}
\begin{fact}
  The following hold:
  \begin{enumerate}[(1)]
    \item $\gamma_{G,S}\left(n\right)$ is increasing;
    \item $\gamma_{G,S}\left(n + m\right) \leq \gamma_{G,S}\left(n\right) \gamma_{G,S}\left(m\right)$;
    \item $\lim_{n\rightarrow\infty}\left(\gamma_{G,S}\left(n\right)\right)^{1/n} = \rho_{G,S}$ exists;
    \item if $S,T$ are symmetric finite generating sets for $G$, then there exists $K\in \N$ such that $\gamma_{G,T}\left(n\right) \leq \gamma_{G,S}\left(Kn\right)$ for all $n\in \N$, and $\rho_{G,S} = \rho_{G,T}$.
  \end{enumerate}
\end{fact}
\begin{proof}\hfill
  \begin{enumerate}[(1)]
    \item We have $B_{G,S}\left(n\right) \subseteq B_{G,S}\left(n+1\right)$, so $\gamma_{G,S}\left(n\right)$ is increasing.
    \item We claim that $B_{G,S}\left(n\right)B_{G,S}\left(m\right) = B_{G,S}\left(n+m\right)$. Let $g\in B_{G,S}\left(n\right),h\in B_{G,S}\left(m\right)$. Then, we have $\ell_{G,S}\left(gh\right) \leq \ell_{G,S}\left(g\right) + \ell_{G,S}\left(h\right) \leq m+n$, so $B_{G,S}\left(n\right) B_{G,S}\left(m\right) \subseteq B_{G,S}\left(n+m\right)$. Meanwhile, if $g\in B_{G,S}\left(n+m\right)$, we may write $g = s_1\cdots s_k$, where $k\leq n+m$ and $s_i\in S$. Then, we may factor 
      \begin{align*}
        g = \underbrace{s_{1}\cdots s_{\ell}}_{g_1}\underbrace{s_{\ell+1}\cdots s_k}_{g_2},
      \end{align*}
      where $\ell \leq n$ and $k-\ell \leq m$, meaning $g_1\in B_{G,S}\left(n\right)$ and $g_2\in B_{G,S}\left(m\right)$. Thus, we have $B_{G,S}\left(n\right)B_{G,S}\left(m\right) = B_{G,S}\left(n+m\right)$.\newline

      We thus have
      \begin{align*}
        \gamma_{G,S}\left(n+m\right) &= \left\vert B_{G,S}\left(n+m\right) \right\vert\\
                                     &= \left\vert B_{G,S}\left(n\right)B_{G,S}\left(m\right) \right\vert\\
                                     &\leq \left\vert B_{G,S}\left(n\right) \right\vert\left\vert B_{G,S}\left(m\right) \right\vert\\
                                     &= \gamma_{G,S}\left(n\right)\gamma_{G,S}\left(m\right).
      \end{align*}
    \item We see that $\gamma_{G,S}\left(n\right) \leq \gamma_{G,S}\left(1\right)^{n}$. Inductively, we have
      \begin{align*}
        \gamma_{G,S}\left(n+1\right) &\leq \gamma_{G,S}\left(n\right)\gamma_{G,S}\left(1\right)\\
                                     &\leq \gamma_{G,S}\left(1\right)^{n}\gamma_{G,S}\left(1\right)\\
                                     &= \gamma_{G,S}\left(1\right)^{n+1},
      \end{align*}
      hence
      \begin{align*}
        1 \leq \gamma_{G,S}\left(n\right)^{1/n}\leq \gamma_{G,S}\left(1\right).
      \end{align*}
    \item We know that there exists $K$ such that
      \begin{align*}
        \frac{1}{K}\ell_{G,S} \leq \ell_{G,T} \leq K \ell_{G,S}.
      \end{align*}
      Thus, if $g\in B_{G,T}\left(n\right)$, then $\ell_{G,T}\left(g\right) \leq n$, so $\ell_{G,S}\left(g\right) \leq Kn$, so $g\in B_{G,S}\left(Kn\right)$, and $B_{G,T}\left(n\right)\subseteq B_{G,S}\left(Kn\right)$. We also have $\gamma_{G,T}\left(n\right) \leq \gamma_{G,S}\left(Kn\right)$.\newline

      Similarly, if $g\in B_{G,S}\left(n\right)$, then $\ell_{G,S}\left(g\right) \leq n$, so $\ell_{G,T}\left(g\right) \leq Kn $, and $g\in B_{G,T}\left(Kn\right)$. Therefore, $B_{G,S}\left(n\right)\subseteq B_{G,T}\left(Kn\right)$, so $\gamma_{G,S}\left(n\right)\leq \gamma_{G,T}\left(Kn\right)$. Thus, we get
      \begin{align*}
        \gamma_{G,S}\left(\frac{n}{K}\right)^{1/n} &\leq \gamma_{G,T}\left(n\right)^{1/n}\\
                                                   &\leq \left(\gamma_{G,S}\left(Kn\right)^{1/Kn}\right)^{K}.
      \end{align*}
      Sending $n\rightarrow\infty$, we get
      \begin{align*}
        \rho_{G,S} \leq \rho_{G,T}\leq \rho_{G,S},
      \end{align*}
      and $\rho_{G,S} = \rho_{G,T}$.
  \end{enumerate}
\end{proof}

\end{document}
