\documentclass[10pt]{mypackage}

% sans serif font:
%\usepackage{cmbright}
%\usepackage{sfmath}
%\usepackage{bbold} %better blackboard bold

%serif font + different blackboard bold for serif font
\usepackage{newpxtext,eulerpx}
\renewcommand*{\mathbb}[1]{\varmathbb{#1}}
\renewcommand*{\hbar}{\hslash}

\pagestyle{fancy} %better headers
\fancyhf{}
\rhead{Avinash Iyer}
\lhead{Invariant States and Means on Groups}

\setcounter{secnumdepth}{0}

\begin{document}
\RaggedRight
\tableofcontents
\section{Introduction}%
This is going to be a part of my Honors thesis independent study, focused on amenability and $C^{\ast}$-algebras. This section of notes will be a deeper dive into group amenability. These notes will be taken from the notes my professor has prepared on group amenability, with supplementation from Volker Runde's \textit{Lectures on Amenability} and Pierre de la Harpe's \textit{Topics in Geometric Group Theory}.\newline

I do not claim any of this work to be original.
\section{Amenable Groups and Subgroups}%
Let $G$ be a group, with $P(G)$ denoting the power set.
\begin{definition}
  An invariant mean on $G$ is a set function $m: P(G)\rightarrow [0,1]$, which satisfies, for all $t\in G$ and $E,F\subseteq G$,
  \begin{enumerate}[(1)]
    \item $m(G) = 1$;
    \item $m\left(E\sqcup F\right) = M(E) + m(F)$;
    \item $m\left(tE\right) = m\left(E\right)$.
  \end{enumerate}
  We say $G$ is amenable if it admits a mean.\newline

  We can also say that $m$ is a translation-invariant probability measure on $\left(G,P(G)\right)$.
\end{definition}
\begin{proposition}[Amenability of Subgroups and Quotient Groups]
  Let $G$ be amenable, with $H\leq G$.
  \begin{enumerate}[(1)]
    \item $H$ is amenable;
    \item for $H\trianglelefteq G$, $G/H$ is amenable.
  \end{enumerate}
\end{proposition}
\begin{proof}\hfill
  \begin{enumerate}[(1)]
    \item Let $R$ be a right transversal for $H$ (i.e., selecting one element of each right coset of $H$ to make up $R$).\newline

      If $m$ is a mean for $G$, we set
      \begin{align*}
        \lambda: \mathcal{P}\left(H\right) \rightarrow [0,1]
      \end{align*}
      by $\lambda(E) = m\left(ER\right)$. We have
      \begin{align*}
        \lambda\left(H\right) &= m\left(HR\right)\\
                              &= m\left(G\right)\\
                              &= 1.
      \end{align*}
      We claim that if $E\cap F = \emptyset$, then $ER\cap FR = \emptyset$, since if we suppose toward contradiction that $ER \cap FR \neq \emptyset$, then $xr_1 = yr_2$ for some $ x\in E,y\in F$ and $r_1,r_2\in R$. Then, we must have $r_2r_1^{-1} = y^{-1}x \in H$, meaning $r_1 = r_2$ and $x = y$, which means $E\cap F \neq \emptyset$.\newline

      Thus, we have
      \begin{align*}
        \lambda\left(E\sqcup F\right) &= m\left(\left(E\sqcup F\right)R\right)\\
                                      &= m\left(ER\sqcup FR\right)\\
                                      &= m\left(ER\right) + m\left(FR\right)\\
                                      &= \lambda(E) + \lambda(F),
      \end{align*}
      and
      \begin{align*}
        \lambda\left(sE\right) &= m\left(sER\right)\\
                               &= m\left(ER\right)\\
                               &= \lambda\left(E\right).
      \end{align*}
    \item For the canonical projection map $\pi: G\rightarrow G/H$ defined by $\pi(t) = tH$, we define
      \begin{align*}
        \lambda: P\left(G/H\right)\rightarrow [0,1]
      \end{align*}
      by $\lambda\left(E\right) = m\left(\pi^{-1}\left(E\right)\right)$. We have
      \begin{align*}
        \lambda\left(G/H\right) &= m\left(\pi^{-1}\left(G/H\right)\right)\\
                                &= m\left(G\right)\\
                                &= 1,
      \end{align*}
      and
      \begin{align*}
        \lambda\left(E\sqcup F\right) &= m\left(\pi^{-1}\left(E\sqcup F\right)\right)\\
                                      &= m\left(\pi^{-1}\left(E\right)\sqcup \pi^{-1}\left(F\right)\right)\\
                                      &= m\left(\pi^{-1}\left(E\right)\right) + m\left(\pi^{-1}\left(F\right)\right)\\
                                      &= \lambda\left(E\right) + \lambda\left(F\right).
      \end{align*}
      To show translation-invariance, we let $sH = \pi\left(s\right)\in G/H$, and $E\subseteq G/H$. Note that
      \begin{align*}
        \pi^{-1}\left(\pi\left(s\right)E\right) &= s\pi^{-1}\left(E\right),
      \end{align*}
      since for $r\in s\pi^{-1}\left(E\right)$, we have $r = st$ for $\pi(t)\in E$, so $\pi\left(r\right) = \pi\left(st\right) = \pi\left(s\right)\pi\left(t\right) \in \pi\left(s\right)E$.\newline

      Additionally, if $r\in \pi^{-1}\left(\pi\left(s\right)E\right)$, then $\pi\left(r\right) \in \pi\left(s\right)E $, so $\pi\left(s^{-1}r\right)\in E$, and $s^{-1}r\in \pi^{-1}\left(E\right)$. Thus, we have
      \begin{align*}
        \lambda\left(\pi\left(s\right)E\right) &= m\left(\pi^{-1}\left(\pi\left(s\right)E\right)\right)\\
                                               &= m\left(s\pi^{-1}\left(E\right)\right)\\
                                               &= m\left(\pi^{-1}\left(E\right)\right)\\
                                               &= \lambda\left(E\right).
      \end{align*}
      
  \end{enumerate}
\end{proof}
\section{Understanding Free Groups}%
In the \href{https://blog.avinashiyer.xyz/Classes_and_Homework/College/Y4/Honors\%20Thesis/amenability_notes.pdf}{Tarski's Theorem} notes, we discussed a little bit the ramifications of the free group on two generators being paradoxical. In order to better understand free groups, we will draw information from Pierre de la Harpe's \textit{Topics in Geometric Group Theory} and Clara Löh's \textit{Geometric Group Theory: An Introduction}.
\subsection{Groups specified by Generating Sets}%
\begin{definition}
  Let $G$ be a group and $S\subseteq G$ be a subset. The subgroup generated by $S$ is the intersection of all subgroups of $G$ that contain $S$. We write $\left\langle S \right\rangle_{G}$. We say $S$ generates $G$ if $\left\langle S \right\rangle_{G} = G$.\newline

  A group is called finitely generated if it contains a finite subset that contains the group in question.
\end{definition}
\begin{definition}[Characterization of a Generated Subgroup]
We can characterize a generated subgroup by $S$ as follows:
\begin{align*}
  \left\langle S \right\rangle_{G} &= \set{s_1^{\ve_1}s_{2}^{\ve_2}\cdots s_n^{\ve_n} | n\in\N,~s_1,\dots,s_n\in S,~\ve_1,\dots,\ve_n\in\set{-1,1}}.
\end{align*}
\end{definition}
\begin{example}[Generating Sets]\hfill
  \begin{itemize}
    \item If $G$ is a group, then $G$ is a generating set of $G$.
    \item The trivial group is generated by the empty set.
    \item The set $\set{1}$ generates $\Z$, as does $\set{2,3}$. However, $\set{2}$ and $\set{3}$ alone do not generate $\Z$.
    \item Let $X$ be a set. The symmetric group $S_{X}$ is finitely generated if and only if $X$ is finite.
  \end{itemize}
\end{example}
\subsection{Free Groups}%
\begin{definition}
Let $S$ be a set. A group $F$ containing $S$ is said to be freely generated if, for every group $G$ and every map $\varphi: S\rightarrow G$, there is a unique group homomorphism $\overline{\varphi}:F\rightarrow G$ extending $\varphi$. The following diagram commutes:
\begin{center}
  % https://tikzcd.yichuanshen.de/#N4Igdg9gJgpgziAXAbVABwnAlgFyxMJZABgBpiBdUkANwEMAbAVxiRAGUQBfU9TXfIRQBGclVqMWbAOLdeIDNjwEiZYePrNWiEADFu4mFADm8IqABmAJwgBbJGRA4ISURK1sAOp-pW0ACyw5Sxt7REdnJAAmahw6LAY2fwgIAGsQagY6ACMYBgAFfmUhECssY38cDPcpHW98OOCQazto2JdEN01akG8IGhgrBiwwGGBvXwCsLmqs3IKiwTYyiqquCi4gA
\begin{tikzcd}
S \arrow[r, "\varphi"] \arrow[d, "\iota"', hook] & G \\
F \arrow[ru, "\overline{\varphi}"']              &  
\end{tikzcd}
\end{center}
A group is free if it contains a free generating set.
\end{definition}
\begin{example}\hfill
  \begin{itemize}
    \item The additive group $\Z$ is freely generated by $\set{1}$. The additive group $\Z$ is \textit{not} freely generated by $\set{2,3}$, or $\set{2}$, or $\set{3}$. In particular, not every generating set of a group contains a free generating set.
    \item The trivial group is freely generated by the empty set.
    \item Not every group is free --- the additive groups $\Z/2\Z$ and $\Z\times \Z$ are not free.
  \end{itemize}
\end{example}
We will use the universal property of free groups to show their uniqueness up to isomorphism.
\begin{proposition}
  Let $S$ be a set. Then, there is at most one group freely generated by $S$ up to isomorphism.
\end{proposition}
\begin{proof}
  Let $F$ and $F'$ be two groups freely generated by $S$, with inclusions of $\varphi$ and $\varphi'$ respectively.
    Because $F$ is freely generated by $S$, there is a group homomorphism $\overline{\varphi}': F\rightarrow F'$ that extends $\varphi$ --- i.e., that $\overline{\varphi}' \circ \varphi = \varphi'$.\newline

      Similarly, there is a group homomorphism $\overline{\varphi}: F'\rightarrow F$ with $\overline{\varphi}\circ \varphi' = \varphi$.
      \begin{center}
        % https://tikzcd.yichuanshen.de/#N4Igdg9gJgpgziAXAbVABwnAlgFyxMJZABgBpiBdUkANwEMAbAVxiRAGUQBfU9TXfIRQBGclVqMWbAGIBybrxAZseAkTLDx9Zq0QhpCvisFEAzGOrapezjyMC1KACwWJOmYaX9VQ5Oc2Wkrr68lziMFAA5vBEoABmAE4QALZIZCA4EEiibtYgADr59AloABZYoYqJKWnUmUgATHV0WAxspRAQANYg1Ax0AEYwDAAK3iZ6CViRpTi9ucGFxWVYntWpiE0ZWYg5Vov5EDQwCQxYYDDAS3Ql5VzyfYPDY8aOIFMzc3Yg60jm20gXAs2Ndbqtvr9EP96ogAKzNVrtTo9R5DUbjN4fWbzfYgoo3FaVeJJDbwgGIIH9NEvBxCd7TbGBdx6QpHE5nC5XfFgrjcChcIA
\begin{tikzcd}
S \arrow[r, "\varphi'"] \arrow[d, "\varphi"', hook] & F' &  & S \arrow[r, "\varphi"] \arrow[d, "\varphi'"', hook] & F \\
F \arrow[ru, "\overline{\varphi}'"']                &    &  & F' \arrow[ru, "\overline{\varphi}"']                &  
\end{tikzcd}
      \end{center}
      We will show that $\overline{\varphi}\circ \overline{\varphi}' = \id_{F}$, and $\overline{\varphi}'\circ \overline{\varphi} = \id_{F'}$. The composition $\overline{\varphi}\circ \overline{\varphi}'$ is a group homomorphism that makes the following diagram commute.
      \begin{center}
% https://tikzcd.yichuanshen.de/#N4Igdg9gJgpgziAXAbVABwnAlgFyxMJZABgBpiBdUkANwEMAbAVxiRAGUQBfU9TXfIRQBGclVqMWbAGLdeIDNjwEiZYePrNWiELK7iYUAObwioAGYAnCAFskZEDghJRErWwA6H+pbQALLDkLaztEByckACZqHDosBjY-CAgAaxBqTSkdLx9-QOoGOgAjGAYABX5lIRBLLCM-HCCQK1somOdEV0KS8srBNlr6xozJbRAvCBoYSwYsMBhgHLpfAK4vAGMsS3WJqZm5haWVrC4Acm4KLiA
\begin{tikzcd}
S \arrow[r, "\varphi"] \arrow[d, "\varphi"', hook]          & F \\
F \arrow[ru, "\overline{\varphi}\circ\overline{\varphi}'"'] &  
\end{tikzcd}
      \end{center}
      Since $\id_{F}$ is a group homomorphism contained in this diagram, and $F$ is freely generated by $S$, we must have $\overline{\varphi}\circ \overline{\varphi}' = \id_{F}$. Similarly, we must have $\overline{\varphi}' \circ \overline{\varphi} = \id_{F'}$.
\end{proof}
\begin{theorem}[Existence of Free Groups]
  Let $S$ be a set. There exists a group freely generated by $S$. This group is unique up to isomorphism.
\end{theorem}
\begin{proof}
  We want to construct a group consisting of ``words'' made up of the elements of $S$ and their ``inverses,'' then modding out by the natural cancellation rules.\newline

  We consider the alphabet
  \begin{align*}
    A &= S\cup \hat{S}.
  \end{align*}
  Here, $\hat{S} = \set{\hat{s} | s\in S}$ is a disjoint copy of $S$, such that $\hat{s}$ will serve as the inverse of $s$ in the group we will construct.\newline

  We define $A^{\ast}$ to be the set of all finite sequences over the alphabet $A$, including the empty word $\epsilon$. We define the operation $A^{\ast}\times A^{\ast} \rightarrow A^{\ast}$ by concatenation. This operation is associative with neutral element $\epsilon$.\newline

  We define
  \begin{align*}
    F(S) &= A^{\ast}/\sim,
  \end{align*}
  where $\sim$ is the equivalence relation generated by, for all $x,y\in A^{\ast}$ and $s\in S$, $x s \hat{s} y \sim xy$ and $x\hat{s} s y \sim xy$.\newline

  We denote the equivalence classes with respect to $\sim$ by $\left[\cdot\right]$.\newline

  Concatenation induces a well-defined operation $F\left(S\right)\times F\left(S\right) \rightarrow F(S)$ by
  \begin{align*}
    \left[x\right] \left[y\right] &= \left[xy\right]
  \end{align*}
  for $x,y\in A^{\ast}$.\newline

  We claim that $F(S)$ with the defined concatenation is a group. We can see that $\left[\epsilon\right]$ is a neutral element for the operation, and associativity of the operation is inherited from the associativity of the operation on $A^{\ast}$.\newline

  To find inverses, we define $I: A^{\ast}\rightarrow A^{\ast}$ by $I\left(\epsilon\right) = \epsilon$, and
  \begin{align*}
    I\left(sx\right) &= I(x)\hat{s}\\
    I\left(\hat{s}x\right) &= I(x)s
  \end{align*}
  for all $x\in A^{\ast}$ and $s\in S$. Induction shows that $I\left(I\left(x\right)\right) = x$, and
  \begin{align*}
    \left[I\left(x\right)\right]\left[x\right] &= \left[I\left(x\right)x\right]\\
                                               &= \left[\epsilon\right]
  \end{align*}
  for all $x\in A^{\ast}$. Thus, we must also have
  \begin{align*}
    \left[x\right]\left[I(x)\right] &= \left[I\left(I\left(x\right)\right)\right]\left[I\left(x\right)\right]\\
                                    &= \left[\epsilon\right].
  \end{align*}
  Thus, we see that there are inverses in $F(S)$.\newline

  To see that $F(S)$ is freely generated by $S$, we let $\iota: S\rightarrow F(S)$ be the map given by sending a letter in $S\subseteq A^{\ast}$ to its equivalence class in $F(S)$. By construction, $F(S)$ is generated by the subset $\iota\left(S\right)\subseteq F(S)$.\newline

  We do not know yet that $\iota$ is injective, so we take a bit of a detour. We show that for every group $G$ and every map $\varphi: S\rightarrow G$, there is a unique group homomorphism $\overline{\varphi}:F(S) \rightarrow G$ such that $\overline{\varphi}\circ \iota = \varphi$.\newline

  We construct a map $\varphi^{\ast}: A^{\ast}\rightarrow G$ inductively by
  \begin{align*}
    \epsilon &\mapsto e\\
    sx &\mapsto \varphi(s)\varphi^{\ast}\left(x\right)\\
    \hat{s}x &\mapsto \left(\varphi\left(s\right)\right)^{-1}\varphi^{\ast}\left(x\right)
  \end{align*}
  for all $s\in S$ and $x\in A^{\ast}$. We can see that, since the definition of $\varphi^{\ast}$ is compatible with the generating set of $\sim$, it is compatible with the equivalence relation of $\sim$ on $A^{\ast}$. Additionally, we can see that $\varphi^{\ast}\left(xy\right) = \varphi^{\ast}\left(x\right)\varphi^{\ast}\left(y\right)$ for all $x,y\in A^{\ast}$. Thus,
  \begin{align*}
    \overline{\varphi}: F\left(S\right) &\rightarrow G\\
    \left[x\right] &\mapsto \left[\varphi^{\ast}\left(x\right)\right],
  \end{align*}
  is, as constructed, a group homomorphism, with $\overline{\varphi}\circ \iota = \varphi$. Since $\iota\left(S\right)$ generates $F(S)$, this group homomorphism is unique.\newline
  
  We must now show that $\iota$ is injective.\newline

  Let $s_1,s_2\in S$. Consider the map $\varphi: S\rightarrow \Z$ given by $\varphi\left(s_1\right) = 1$ and $\varphi\left(s_2\right) = -1$. The corresponding homomorphism $\overline{\varphi}: F(S)\rightarrow G$ satisfies
  \begin{align*}
    \overline{\varphi}\left(\iota\left(s_1\right)\right) &= \varphi\left(s_1\right)\\
                                                         &= 1\\
                                                         &\neq -1\\
                                                         &= \varphi\left(s_2\right)\\
                                                         &= \overline{\varphi}\left(\iota\left(s_2\right)\right),
  \end{align*}
  meaning $\iota\left(s_1\right)\neq \iota\left(s_2\right)$. Thus, $\iota$ is injective.
\end{proof}
\subsection{Free Groups, Free Products, and the Ping Pong Lemma}%
We can now consider free groups in a more categorical context --- specifically, as a special type of free object. Whereas the previous section drew information from Clara Löh's \textit{Geometric Group Theory: An Introduction}, this section will draw information from Pierre de la Harpe's \textit{Topics in Geometric Group Theory}. Specifically, we are focused on chapter $2$, which discusses free products, free groups, and the ping pong lemma (which will provide a more general proof of the paradoxicality of $SO(3)$).
\begin{definition}[Free Monoid]
  A monoid is a set with multiplication that is associative and includes a neutral element.\newline

  Given a set $A$, the free monoid on $A$ is the set $W(A)$ of finite sequences of elements of $A$ (also known as words). We write an element of $W(A)$ as $w = a_1a_2\cdots a_n$, where each $a_j\in A$. We identify $A$ with the subset of $W(A)$ of words with length $1$.
\end{definition}
\begin{definition}[Free Product]
  Let $\left(\Gamma_i\right)_{i\in I}$ be a family of groups. Set
  \begin{align*}
    A &= \coprod_{i\in I}\Gamma_{i}\\
      &= \set{\left(g_i,i\right) | g_i\in \Gamma_i,i\in I}
  \end{align*}
  to be the coproduct of this family.\newline

  Let $\sim$ be the equivalence relation generated by
  \begin{align*}
    we_i w' &\sim ww' \tag*{where $e_i\in \Gamma_i$ is the neutral element}\\
    wabw' &\sim wcw' \tag*{where $a,b,c\in \Gamma_i$, $c=ab$ for some $i\in I$}
  \end{align*}
  for all $w,w'\in W(A)$. The quotient $W(A)/\sim$ with the operation of concatenation is a group, which is known as the free product of the groups $\set{\Gamma_i}_{i\in I}$. We write it as
  \begin{align*}
    \bigstar_{i\in I} \Gamma_{i}.
  \end{align*}
  The inverse of the equivalence class for $w = a_1a_2\dots a_n$ is $w^{-1} = a_n^{-1}a_{n-1}^{-1}\cdots a_{1}^{-1}$. The neutral element is $\epsilon$, which is the empty word.\newline

  A word $w = a_1a_2\cdots a_n\in W(A)$ with $a_j\in \Gamma_{i_j}$ is said to be reduced if $i_{j + 1} \neq i_{j}$ and $a_j$ is not the neutral element of $\Gamma_{i_j}$.
\end{definition}
\begin{proposition}[Existence of the Free Product]
  Let $\set{\Gamma_i}_{i\in I}$ be a family of groups, $A = \coprod_{i\in I}\Gamma_i$, and $\bigstar_{i\in I}\Gamma_i = W(A)/\sim$ be as above.\newline

  Then, any element in the free product $\bigstar_{i\in I}\Gamma_i$ is represented by a unique reduced word in $W(A)$.
\end{proposition}
\begin{proof}\hfill
  \begin{description}[font=\normalfont\scshape,leftmargin=0pt]
    \item[Existence:] Consider an integer $n\geq 0$ and a reduced word $w = a_1a_2\cdots a_n$ in $W(A)$, an element $a\in A$, and the word $aw\in W(A)$. We set
      \begin{align*}
        \mathcal{R}\left(aw\right) &= \begin{cases}
          w & a = e_i\\
          aa_1a_2\cdots a_n & a\in \Gamma_i,a\neq e_i,i\neq k\\
          ba_2\cdots a_n & a\in \Gamma_k, aa_1 = b \neq e_k\\
          a_2\cdots a_n & a\in \Gamma_k,a_1 = a^{-1}\in \Gamma_k
        \end{cases},
      \end{align*}
      where $k$ is the index for which $a_1\in \Gamma_k$.\newline

      Then, $\mathcal{R}\left(aw\right)$ is yet another reduced word, and $\mathcal{R}\left(aw\right) \sim aw$, meaning that any word $w\in W(A)$ is equivalent to some reduced word by inducting on the length of $w$.
    \item[Uniqueness:] For each $a\in A$, Let $T(a) = \mathcal{R}\left(aw\right)$ be a self-map on the set of reduced words.\newline

      For each $w = b_1b_2\cdots b_n$, we set $T(w) = T\left(b_1\right)T\left(b_2\right)\cdots T\left(b_n\right)$. For $a,b,c\in \Gamma_i$ with $ab = c$, we have $T\left(a\right)T\left(b\right) = T\left(c\right)$, and $T\left(e_{i}\right) = \epsilon$ (the empty word) for all $i\in I$.\newline

       For each reduced word, notice that $T\left(w\right)\epsilon = w$.\newline

       Let $w$ be some word in $W(A)$ with $w_1,w_2$ reduced words equivalent to $w$. Since $w_1\sim w_2$, we have $T\left(w_1\right) = T\left(w_2\right)$, and
       \begin{align*}
         w_1 &= T\left(w_1\right)\epsilon\\
             &= T\left(w_2\right)\epsilon\\
             &= w_2.
       \end{align*}
  \end{description}
\end{proof}
\begin{corollary}
  Let $\set{\Gamma_i}_{i\in I}$ and $\Gamma = \bigstar_{i\in I}\Gamma_i$ as above. For each $i_0\in I$, the canonical inclusion
  \begin{align*}
    \iota: \Gamma_{i_0}\rightarrow \Gamma
  \end{align*}
  is injective.
\end{corollary}
\begin{proof}
  For any $a\in \Gamma_{i_0}\setminus \set{e_{i_0}}$, $\iota\left(a\right)$ is represented by a unique one-letter reduced word not equivalent to the empty word.
\end{proof}
Now that we have an understanding of free products, we can conceptualize the free group as a particular type of free product.
\begin{definition}[Free Groups]
  Let $X$ be a set. The free group over $X$ is the free product of a family of copies of $\Z$ indexed by $X$, denoted $F(X)$.\newline

  Equivalently, the free group over $X$ is
  \begin{align*}
    F(X) &= \bigstar_{a\in X}\left\langle a \right\rangle,
  \end{align*}
  where $\left\langle a \right\rangle$ denotes the cyclic group generated by the element $a$.\newline

  We can also identify $F(X)$ with the set of reduced words in $X\sqcup X^{-1}$ (as was done in the previous subsection).\newline

  The cardinality of $X$ is called the rank of $F(X)$.\newline

  If $\Gamma$ is a group, then a free subset of $\Gamma$ is a subset $X\subseteq \Gamma$ such that the inclusion $X\hookrightarrow F(X)$ extends to an isomorphism of $\left\langle X \right\rangle_{\Gamma}$ onto $F(X)$.
\end{definition}
We can now state and prove a universal property for free products (which naturally simplifies in the case of a free group.)
\begin{theorem}[Universal Property for Free Products]
  Let $\Gamma$ be a group, and $\set{\Gamma_i}_{i\in I}$ be a family of groups. Let $\set{h_{i}:\Gamma_i\rightarrow \Gamma}_{i\in I}$ be a family of homomorphisms.\newline

  Then, there exists a unique homomorphism $h: \bigstar_{i\in I}\Gamma_{i}\rightarrow \Gamma$ such that the following diagram commutes for each $i_0\in I$.
  \begin{center}
    % https://tikzcd.yichuanshen.de/#N4Igdg9gJgpgziAXAbVABwnAlgFyxMJZARgBoAGAXVJADcBDAGwFcYkQAdDgcXoFs+9EAF9S6TLnyEU5UsWp0mrdlwBGWAOZwc9AE4B9YFi5YwAAgCSwrrwH19WEWJAZseAkVlUaDFm0ScPPyChlj65MIiCjBQGvBEoABmuhB8SABMNDgQSLKKfuwAFqHhkaJJKWmImSDZSGS19FiMRRAQANYgNIz0qjCMAAoS7tIgupqFOF35ygEmEDpOFan1WTmIeT19g8NS7OMak9O+syCFUcJAA
\begin{tikzcd}
\Gamma_{i_0} \arrow[r, "h_{i_0}"] \arrow[d, "\iota"', hook] & \Gamma \\
\bigstar_{i\in I}\Gamma_i \arrow[ru, "h"']                  &       
\end{tikzcd}
  \end{center}
  In particular, if $\Gamma$ is a group, $X$ is a set, and $\phi: X\rightarrow \Gamma$ is a set map, there exists a unique homomorphism $\Phi: F(X)\rightarrow \Gamma$ such that $\Phi(x) = \phi(x)$ for each $x\in X$.
\end{theorem}
\begin{proof}
  For a reduced word $w = a_1a_2\cdots a_n\in \bigstar_{i\in I}\Gamma_i$ with $a_j\in \Gamma_{i_j}\setminus \set{e_{i_j}}$, and $i_{j+1}\neq i_{j}$ for each $j\in \set{1,\dots,n-1}$, we set
  \begin{align*}
    h\left(w\right) &= h_{i_1}\left(a_1\right)h_{i_2}\left(a_2\right)\cdots h_{i_n}\left(a_n\right),
  \end{align*}
  which defines $h$ uniquely in terms of $h_{i}$.
\end{proof}
Note that for any two sets $X,Y$, the universal property provides that any map $X\rightarrow Y$ extends canonically to a group homomorphism, $F(X) \rightarrow F(Y)$.
\begin{center}
  % https://tikzcd.yichuanshen.de/#N4Igdg9gJgpgziAXAbVABwnAlgFyxMJZABgBpiBdUkANwEMAbAVxiRAA0QBfU9TXfIRQBGclVqMWbAJrdeIDNjwEiZYePrNWiEADEAFOwCUcvksFFR66pqk6D0k13EwoAc3hFQAMwBOEAFskACZqHAgkAGYeH38gxDIQcKRhGJA-QKRE5MRQpLosBjYACwgIAGtTdLiUsIjESLCCop1Siu4KLiA
\begin{tikzcd}
X \arrow[r] \arrow[d, hook] & Y \arrow[d, hook] \\
F(X) \arrow[r]              & F(Y)             
\end{tikzcd}
\end{center}
We can now prove an important lemma that will be useful in understanding paradoxical groups.
\begin{theorem}[Ping Pong Lemma]
  Let $G$ be a group acting on a set $X$, and let $\Gamma_1,\Gamma_2$ be subgroups of $G$. Let $\Gamma = \left\langle \Gamma_1,\Gamma_2 \right\rangle$. Assume $\Gamma_1$ contains at least $3$ elements and $\Gamma_2$ contains at least two elements.\newline

  Suppose there exist nonempty subsets $X_1,X_2\subseteq X$ with $X_1\triangle X_2\neq \emptyset$, such that for all $\gamma_1\in \Gamma_1$ with $\gamma_1\neq e_{G}$, and for all $\gamma_2\in \Gamma_2$ with $\gamma_2\neq e_{G}$,
  \begin{align*}
    \gamma\left(X_2\right)\subseteq X_1\\
    \gamma\left(X_1\right)\subseteq X_2.
  \end{align*}
  Then, $\Gamma$ is isomorphic to the free product $\Gamma_1\star \Gamma_2$.
\end{theorem}
\begin{proof}
  Let $w$ be a nonempty reduced word spelled with letters from the disjoint union of $\Gamma_1\setminus \set{e_G}$ and $\Gamma_2\setminus \set{e_G}$. We must show that the element of $\Gamma$ defined by $w$ is not the identity.\newline

  If $w = a_1b_1a_2b_2\cdots a_k$ with $a_1,\dots,a_k\in \Gamma_1\setminus \set{e_G}$ and $b_1,\dots,b_{k-1}\in \Gamma_2\setminus \set{e_G}$. Then,
  \begin{align*}
    w\left(X_2\right) &= a_1b_1\cdots a_{k-1}b_{k-1}a_k\left(X_2\right)\\
                      &\subseteq a_1b_1\cdots a_{k-1}b_{k-1}\left(X_1\right)\\
                      &\subseteq a_1b_1\cdots a_{k-1}\left(X_2\right)\\
                      &\vdots\\
                      &\subseteq a_1\left(X_2\right)\\
                      &\subseteq X_1.
  \end{align*}
  Since $X_2\nsubseteq X_1$, this implies $w\neq e_{G}$.\newline

  If $w = b_1a_2b_2a_2\cdots b_k$, we select $a\in \Gamma_1\setminus \set{e_G}$, and apply the previous argument to $awa^{-1}$. Since $awa^{-1}\neq e_{G}$, neither is $w$.\newline

  Similarly, if $w = a_1b_1\cdots a_kb_k$, we select $a\in \Gamma_1\setminus \set{e_G,a_1^{-1}}$, and apply the argument to $awa^{-1}$, and if $w = b_1a_2b_2\cdots a_k$, we select $a\in \Gamma_1\setminus \set{e_G,a_k}$, and apply the argument to $awa^{-1}$.
\end{proof}
\begin{example}
  We can use the Ping Pong Lemma to see that
  \begin{align*}
    A &= \begin{pmatrix}1 & 2 \\ 0 & 1\end{pmatrix}\\
    B &= \begin{pmatrix}1 & 0 \\ 2 & 1\end{pmatrix}
  \end{align*}
  generate a subgroup of $SL\left(2,\Z\right)$ which is free of rank $2$.
\end{example}
\begin{corollary}
  The special orthogonal group $SO(3)$ contains a subgroup isomorphic to the free group on two generators.
\end{corollary}
To prove this, we state a different version of the Ping Pong Lemma that we will apply to a particular space.
\begin{theorem}[Ping Pong Lemma for Cyclic Groups]
  Let $G$ act on a set $X$, and suppose there exist disjoint subsets $A_{+},A_{-},B_{+},B_{-}\subseteq X$ whose union is not all of $X$. If there exist elements $a$ and $b$ in $G$ such that
  \begin{align*}
    a\cdot \left(X\setminus A_{-}\right) &\subseteq A_{+}\\
    a^{-1}\cdot \left(X\setminus A_{+}\right) &\subseteq A_{-}\\
    b \cdot \left(X\setminus B_{-}\right) &\subseteq B_{+}\\
    b\cdot \left(X\setminus B_{+}\right) &\subseteq B_{-},
  \end{align*}
  then it is the case that the group generated by $a$ and $b$ is free of rank $2$.
\end{theorem}
\begin{proof}[Proof of Corollary]
  We let
  \begin{align*}
    a &= \begin{pmatrix}3/5 & 4/5 & 0 \\ -4/5 & 3/5 & 0 \\ 0 & 0 & 1\end{pmatrix}\\
    a^{-1} &= \begin{pmatrix}3/5 & -4/5 & 0 \\ 4/5 & 3/5 & 0 \\ 0 & 0 & 1\end{pmatrix}\\
    b &= \begin{pmatrix}1 & 0 & 0 \\ 0 & 3/5 & -4/5 \\ 0 & 4/5 & 3/5\end{pmatrix}\\
    b^{-1} &= \begin{pmatrix}1 & 0 & 0 \\ 0 & 3/5 & 4/5 \\ 0 & -4/5 & 3/5\end{pmatrix}.
  \end{align*}
  We specify
  \begin{align*}
    X &= A_{+} \sqcup A_{-} \sqcup B_{+} \sqcup B_{-} \sqcup \begin{pmatrix}0\\1\\0\end{pmatrix},
  \end{align*}
  where
  \begin{align*}
    A_{+} &= \set{\frac{1}{5^{k}} \begin{pmatrix}x\\y\\z\end{pmatrix} | k\in \Z, x \equiv 3y\text{ modulo $5$}, z\equiv0\text{ modulo $5$}}\\
    A_{-} &= \set{\frac{1}{5^{k}} \begin{pmatrix}x\\y\\z\end{pmatrix} | k\in \Z, x \equiv -3y\text{ modulo $5$}, z\equiv 0\text{ modulo $5$}}\\
    B_{+} &= \set{\frac{1}{5^{k}} \begin{pmatrix}x\\y\\z\end{pmatrix} | k\in \Z, z \equiv 3y\text{ modulo $5$}, x\equiv 0\text{ modulo $5$}}\\
    B_{-} &= \set{\frac{1}{5^{k}} \begin{pmatrix}x\\y\\z\end{pmatrix} | k\in \Z, z \equiv -3y\text{ modulo $5$}, x\equiv 0\text{ modulo $5$}}.
  \end{align*}
  To verify that the conditions of the Ping Pong Lemma hold, we calculate
  \begin{align*}
    \begin{pmatrix}3/5 & 4/5 & 0 \\ -4/5 & 3/5 & 0 \\ 0 & 0 & 1\end{pmatrix}\left(\frac{1}{5^k} \begin{pmatrix}x\\y\\z\end{pmatrix}\right) &= \frac{1}{5^{k+1}} \begin{pmatrix}3x + 4y \\ -4x + 3y \\ 5z\end{pmatrix}\tag*{(1)}\\
    \begin{pmatrix}3/5 & -4/5 & 0 \\ 4/5 & 3/5 & 0 \\ 0 & 0 & 1\end{pmatrix} \left(\frac{1}{5^k} \begin{pmatrix}x\\y\\z\end{pmatrix}\right) &= \frac{1}{5^{k+1}} \begin{pmatrix}3x - 4y \\ 4x + 3y \\ 5z\end{pmatrix}\tag*{(2)}\\
    \begin{pmatrix}1 & 0 & 0 \\ 0 & 3/5 & -4/5 \\ 0 & 4/5 & 3/5\end{pmatrix}\left(\frac{1}{5^{k}} \begin{pmatrix}x\\y\\z\end{pmatrix}\right) &= \frac{1}{5^{k+1}} \begin{pmatrix}5x \\ 3y- 4z \\ 4y + 3z\end{pmatrix}\tag*{(3)}\\
    \begin{pmatrix}1 & 0 & 0 \\ 0 & 3/5 & 4/5 \\ 0 & -4/5 & 3/5\end{pmatrix} \left(\frac{1}{5^{k}} \begin{pmatrix}x\\y\\z\end{pmatrix}\right) &= \frac{1}{5^{k+1}} \begin{pmatrix}5x \\ 3y + 4z \\ -4y + 3z\end{pmatrix}.\tag*{(4)}
  \end{align*}
  We verify that the conditions for the Ping Pong Lemma hold for each of these four conditions.
  \begin{enumerate}[(1)]
    \item For any vector
      \begin{align*}
        \frac{1}{5^{k}} \begin{pmatrix}x\\y\\z\end{pmatrix} \notin A_{-},
      \end{align*}
      we see that $k+1\in \Z$, $x' = 3x + 4y \equiv 3\left(-4x + 3y\right)$  modulo $5$, and that $z' = 5z\equiv 0$ modulo $5$.
    \item For any vector
      \begin{align*}
        \frac{1}{5^{k}} \begin{pmatrix}x\\y\\z\end{pmatrix} \notin A_{+},
      \end{align*}
      we see that $k+1\in \Z$, $x' = 3x - 4y\equiv -3\left(4x + 3y\right)$ modulo $5$, and $z' = 5z \equiv 0$ modulo $5$.
    \item For any vector
      \begin{align*}
        \frac{1}{5^{k}} \begin{pmatrix}x\\y\\z\end{pmatrix}\notin B_{-},
      \end{align*}
      we see that $k+1\in \Z$, $z' = 4y + 3z \equiv 3\left(3y-4z\right)$ modulo $5$, and $x' = 5x\equiv 0$ modulo $5$.
    \item For any vector
      \begin{align*}
        \frac{1}{5^{k}} \begin{pmatrix}x\\y\\z\end{pmatrix}\notin B_{+},
      \end{align*}
      we see that $k+1\in \Z$, $z' = -4y + 3z \equiv -3\left(3y + 4z\right)$ modulo $5$, and $x' = 5x \equiv 0$ modulo $5$.
  \end{enumerate}
  Since we have verified that the conditions for the Ping Pong Lemma hold for each of the conditions, we have that $\set{a,b}\subseteq SO(3)$ generate a group isomorphic to the free group on two generators.
\end{proof}
\section{The Normed Space $\ell_{\infty}(G)$}%

\end{document}
