\documentclass[10pt]{mypackage2}

% sans serif font:
%\usepackage{cmbright,sfmath,bbold}
%\renewcommand{\mathcal}{\mathtt}

%Euler:
\usepackage{newpxtext,eulerpx,eucal,eufrak}
\renewcommand*{\mathbb}[1]{\varmathbb{#1}}
\renewcommand*{\hbar}{\hslash}
\usepackage[backend=biber,style=alphabetic,sorting=nty]{biblatex}
\addbibresource{chapters/references.bib}

%\usepackage{homework}

%\pagestyle{fancy} %better headers
%\fancyhf{}
%\rhead{Avinash Iyer}
%\lhead{}

\title{Understanding Amenability in Discrete Groups}
\author{Avinash Iyer}
\date{March 2025}
%\setcounter{secnumdepth}{0}
\setcounter{section}{-1}
\begin{document}
\maketitle
\RaggedRight
%\tableofcontents
\begin{abstract}
  We provide a brief yet thorough overview of amenability in discrete groups by using techniques from functional analysis. We discuss the definition of a mean on a group, and provide some basic characterizations for amenability, including the interplay between means and invariant states on groups, paradoxical decompositions via Tarski's Theorem, and a more combinatorial approximation property via Følner sequences. We bridge important results in group theory and functional analysis in order to prove these results using a variety of characterizations.
\end{abstract}
\section{Preliminaries}%
Here, we overview some of the results we make liberal use of throughout this thesis. We assume that all the readers are familiar with group theory and real analysis, on the level of Math 320 and Math 310 at Occidental College.
\subsection{More Group Theory}%
There's a bit more group theory that we need to cover in this section --- primarily because the finite groups that we learn in the first course of group theory are all amenable.\newline

Here, we will discuss the archetypal (some might say universal) group that can be constructed from any set. This is known as the free group. The definitions and results in section are drawn from \cite{delaHarpe_topics_in_geometric_group_theory} and \cite{loh_geometric_group_theory}.
\begin{definition}\label{def:free_group}
  Let $S$ be a set. A group $F$ containing $S$ is said to be \textit{freely generated} if, for every group $G$, and every set-map $\phi\colon S\rightarrow G$, there is a unique group homomorphism $\varphi\colon F\rightarrow G$ that extends $\varphi$. The following diagram, where $\iota$ denotes the inclusion of $S$ into $F$, commutes:
  \begin{center}
    \begin{tikzcd}
      S \arrow[d, "\iota"', hook] \arrow[r, "\phi"] & G \\
      F \arrow[ru, "\varphi"']                      &  
    \end{tikzcd}
  \end{center}
We say $F$ is the \textit{free group} generated by $S$.
\end{definition}
Free groups do exist, and by definition, are unique up to isomorphism.
\begin{theorem}
  If $S$ is a set, we may define the formal inverse of elements of $S$, $S^{-1} \coloneq \set{s^{-1} | s\in S}$. Let $W(S)$ be the set of words in the formal alphabet $S\cup S^{-1}$.\newline

  Let $F(S)$ be defined by $W(S)/\sim$, where $\sim$ is the equivalence relation generated by
  \begin{align*}
    xss^{-1}y &\sim xy\\
    xs^{-1}sy &\sim xy.
  \end{align*}
  Then, $F(S)$ is freely generated by $S$.
\end{theorem}
\begin{example}
  If we consider the set $S = \set{a,b}$, then the free group $F(a,b)$ is defined to be the set of all reduced words in the alphabet $\set{a,b,a^{-1},b^{-1}}$.
\end{example}
% free groups
The free group is an example of a more general construction --- the free product of groups. We define the free product and its universal property, and leave it as an exercise for the reader to determine the specific family of groups for which $F(S)$ is the free product.
\begin{definition}[Free Product]\label{def:free_product}
  Let $A$ be a set, and set $W(A)$ to be the set of words in $A$ equipped with the operation of concatenation. This turns $W(A)$ into a construction known as the \textit{free monoid}.\newline

  If $\set{\Gamma_i}_{i\in I}$ is a family of groups, and $A = \coprod_{i\in I}\Gamma_i$ is the coproduct (or disjoint union) of the groups $\Gamma_i$, then we define the equivalence relation $\sim$ generated by
  \begin{align*}
    we_iw' &\sim ww'\text{ where $e_i$ is the neutral element of $\Gamma_i$ for some $i\in I$}\\
    wabw' &\sim wcw'\text{ where $a,b,c\in \Gamma_i$ and $c=ab$ for some $i\in I$}.
  \end{align*}
  Then, the quotient $W(A)/\sim$ is known as the \textit{free product} of the groups $\set{\Gamma_i}_{i\in I}$, and is denoted
  \begin{align*}
    \bigstar_{i\in I}\Gamma_i.
  \end{align*}
\end{definition}
Predictably, the free group also admits a universal property.
\begin{theorem}\label{thm:universal_property}
  Let $\set{\Gamma_i}_{i\in I}$ be a family of groups, and let $h_i\colon \Gamma_i\rightarrow \Gamma$ be a family of homomorphisms for each $\Gamma_i$. Then, there is a unique homomorphism $h\colon \bigstar_{i\in I}\Gamma_i\rightarrow \Gamma$ such that the following diagram commutes for each $\Gamma_{i_0}$.
  \begin{center}
        % https://tikzcd.yichuanshen.de/#N4Igdg9gJgpgziAXAbVABwnAlgFyxMJZABgBpiBdUkANwEMAbAVxiRAB12BxOgW17oB9YFkHEAviHGl0mXPkIoyARiq1GLNpwBGWAOZwcdAE7CsnLGAAEASXGce-IVikyQGbHgJFl5NfWZWRA5uPgFBF3E1GCg9eCJQADNjCF4kMhAcCCRfdUCtdnwjMzFJagY6bRgGAAU5L0UQY30ACxwQaiMsBjYWiAgAa1cklLTEDKykACZqAM1glpKJYZBk1JzO7MQZkAqq2vqFNma9No68+ZAWqQpxIA
    \begin{tikzcd}
      \Gamma_{i_0} \arrow[d, "\iota_{i_0}"', hook] \arrow[r, "h_{i_0}"] & \Gamma_i \\
      \bigstar_{i\in I}\Gamma_i \arrow[ru, "h"']                        &         
    \end{tikzcd}
  \end{center}
\end{theorem}
One of the useful facts about the free product is that its properties allow us to find subgroups isomorphic to $F(a,b)$. This occurs through a special property of the action of a group on the set.
\begin{theorem}[Ping Pong Lemma]\label{thm:ping_pong_lemma}
  Let $G$ be a group that acts on a set $X$, and let $\Gamma_1,\Gamma_2$ be subgroups of $G$, with $\Gamma = \left\langle \Gamma_1,\Gamma_2 \right\rangle$. Assume $\Gamma_1$ contains at least three elements and assume $\Gamma_2$ contains at least two elements.\newline

  Let $\emptyset\neq X_1,X_2\subseteq X$ with $X_1\triangle X_2\neq\emptyset$. Suppose that for all $e_G\neq s\in \Gamma_1$ and for all $e_G\neq t\in \Gamma_2$, we have
  \begin{align*}
    s\cdot X_1&\subseteq X_2\\
    t\cdot X_2&\subseteq X_1.
  \end{align*}
  Then, $\Gamma$ is isomorphic to the free product $\Gamma_1\star \Gamma_2$.
\end{theorem}
% free products and ping-pong lemma
\subsection{Functional Analysis}%
% norms
% normed vector spaces and continuous duals
% weak and weak* topology
% hahn--banach separation
\section{What is Amenability?}%
\section{Paradoxical Decompositions and Amenability}%
\section{Amenability and Invariant States}%
\section{Følner's Condition and Amenability}%
\section{Remarks and Notes}%
\section{Apologies and Acknowledgments}%
This thesis is an abridged version of a longer text that I have been writing. That text would have been my honors thesis, but unfortunately it would have been a bit long. I'm writing it with the aim of creating a thorough overview that properly introduces amenability, starting from discrete groups. That text includes other characterizations of amenability, such as a discussion of the left-regular representation and results that relate properties of the group $C^{\ast}$-algebra and amenability of the underlying group. That text will always be a bit of a work in progress, as the theory of amenability is extremely deep; the case of discrete groups is only one case of the more general theory of amenability in locally compact groups, which dives deeper into functional analysis.\newline

Ultimately, the goal of this whole thesis was to provide a more clear exposition on the topic of amenability, assuming minimal prerequisites. While there are certain leaps of faith that I take for granted (as, otherwise, this thesis would certainly be too long as was my original text), I hope that I did not use any major leaps of argumentation that seemed out of hand.\newline

This entire thesis would not be possible without the assistance and guidance of professor Rainone, who put forth the idea of an independent study on Tarski's theorem, and would not have happened without one of my friends at my REU, Lisa Samoylov, telling me that Dana Williams at Dartmouth was a good graduate student advisor. Unfortunately, he's probably retiring, but Appendix A in one of his books, \textit{Crossed Product $C^{\ast}$-Algebras}, was ultimately what convinced me to study amenability for my honors thesis. It turned out to be a very good idea.
\nocite{*}
\printbibliography[title={References}]
\end{document}
