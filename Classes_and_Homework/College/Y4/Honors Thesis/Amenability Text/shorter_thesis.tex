\documentclass[10pt]{mypackage2}

% sans serif font:
%\usepackage{cmbright,sfmath,bbold}
%\renewcommand{\mathcal}{\mathtt}

%Euler:
\usepackage{newpxtext,eulerpx,eucal,eufrak}
\renewcommand*{\mathbb}[1]{\varmathbb{#1}}
\renewcommand*{\hbar}{\hslash}
\usepackage[backend=biber,style=alphabetic,sorting=nty]{biblatex}
\addbibresource{chapters/references.bib}

%\usepackage{homework}

%\pagestyle{fancy} %better headers
%\fancyhf{}
%\rhead{Avinash Iyer}
%\lhead{}

\title{Understanding Amenability in Discrete Groups}
\author{Avinash Iyer}
\date{March 2025}
%\setcounter{secnumdepth}{0}
\setcounter{section}{-1}
\begin{document}
\maketitle
\RaggedRight
%\tableofcontents
\begin{abstract}
  We provide a brief yet thorough overview of amenability in discrete groups by using techniques from functional analysis. We discuss the definition of a mean on a group, and provide some basic characterizations for amenability, including the interplay between means and invariant states on groups, paradoxical decompositions via Tarski's Theorem, and a more combinatorial approximation property via Følner sequences. We bridge important results in group theory and functional analysis in order to prove these results using a variety of characterizations.
\end{abstract}
\section{Preliminaries}\label{sec:preliminaries}%
Here, we overview some of the results we make liberal use of throughout this thesis. We assume that all the readers are familiar with real analysis and group theory, about at the level of Math 310 and Math 320, as well as their preliminaries. We also occasionally allude to results in topology.
\subsection{More Group Theory}%
There's a bit more group theory that we need to cover. These groups will provide the backbone for Section \ref{sec:banach_tarski_tarskis_theorem}\newline

Here, we will discuss the archetypal (some might say universal) group that can be constructed from any set. This is known as the free group. The definitions and results in section are drawn from \cite{delaHarpe_topics_in_geometric_group_theory} and \cite{loh_geometric_group_theory}.
\begin{definition}\label{def:free_group}
  Let $S$ be a set. A group $F$ containing $S$ is said to be \textit{freely generated} if, for every group $G$, and every set-map $\phi\colon S\rightarrow G$, there is a unique group homomorphism $\varphi\colon F\rightarrow G$ that extends $\varphi$. The following diagram, where $\iota$ denotes the inclusion of $S$ into $F$, commutes:
  \begin{center}
    \begin{tikzcd}
      S \arrow[d, "\iota"', hook] \arrow[r, "\phi"] & G \\
      F \arrow[ru, "\varphi"']                      &  
    \end{tikzcd}
  \end{center}
We say $F$ is the \textit{free group} generated by $S$.
\end{definition}
Free groups do exist, and by definition, are unique up to isomorphism.
\begin{theorem}
  If $S$ is a set, we may define the formal inverse of elements of $S$, $S^{-1} \coloneq \set{s^{-1} | s\in S}$. Let $W(S)$ be the set of words in the formal alphabet $S\cup S^{-1}$.\newline

  Let $F(S)$ be defined by $W(S)/\sim$, where $\sim$ is the equivalence relation generated by
  \begin{align*}
    xss^{-1}y &\sim xy\\
    xs^{-1}sy &\sim xy.
  \end{align*}
  Then, $F(S)$ is freely generated by $S$.
\end{theorem}
\begin{example}
  If we consider the set $S = \set{a,b}$, then the free group $F(a,b)$ is defined to be the set of all reduced words in the alphabet $\set{a,b,a^{-1},b^{-1}}$.
\end{example}
% free groups
The free group is an example of a more general construction --- the free product of groups. We define the free product and its universal property, and leave it as an exercise for the reader to determine the specific family of groups for which $F(S)$ is the free product.
\begin{definition}[Free Product]\label{def:free_product}
  Let $A$ be a set, and set $W(A)$ to be the set of words in $A$ equipped with the operation of concatenation. This turns $W(A)$ into a construction known as the \textit{free monoid.}\newline

  If $\set{\Gamma_i}_{i\in I}$ is a family of groups, and $A = \coprod_{i\in I}\Gamma_i$ is the coproduct (or disjoint union) of the groups $\Gamma_i$, then we define the equivalence relation $\sim$ generated by
  \begin{align*}
    we_iw' &\sim ww'\text{ where $e_i$ is the neutral element of $\Gamma_i$ for some $i\in I$}\\
    wabw' &\sim wcw'\text{ where $a,b,c\in \Gamma_i$ and $c=ab$ for some $i\in I$}.
  \end{align*}
  Then, the quotient $W(A)/\sim$ is known as the \textit{free product} of the groups $\set{\Gamma_i}_{i\in I}$, and is denoted
  \begin{align*}
    \bigstar_{i\in I}\Gamma_i.
  \end{align*}
\end{definition}
Predictably, the free group also admits a universal property.
\begin{theorem}\label{thm:universal_property}
  Let $\set{\Gamma_i}_{i\in I}$ be a family of groups, and let $h_i\colon \Gamma_i\rightarrow \Gamma$ be a family of homomorphisms for each $\Gamma_i$. Then, there is a unique homomorphism $h\colon \bigstar_{i\in I}\Gamma_i\rightarrow \Gamma$ such that the following diagram commutes for each $\Gamma_{i_0}$.
  \begin{center}
        % https://tikzcd.yichuanshen.de/#N4Igdg9gJgpgziAXAbVABwnAlgFyxMJZABgBpiBdUkANwEMAbAVxiRAB12BxOgW17oB9YFkHEAviHGl0mXPkIoyARiq1GLNpwBGWAOZwcdAE7CsnLGAAEASXGce-IVikyQGbHgJFl5NfWZWRA5uPgFBF3E1GCg9eCJQADNjCF4kMhAcCCRfdUCtdnwjMzFJagY6bRgGAAU5L0UQY30ACxwQaiMsBjYWiAgAa1cklLTEDKykACZqAM1glpKJYZBk1JzO7MQZkAqq2vqFNma9No68+ZAWqQpxIA
    \begin{tikzcd}
      \Gamma_{i_0} \arrow[d, "\iota_{i_0}"', hook] \arrow[r, "h_{i_0}"] & \Gamma_i \\
      \bigstar_{i\in I}\Gamma_i \arrow[ru, "h"']                        &         
    \end{tikzcd}
  \end{center}
\end{theorem}
One of the useful facts about the free product is that its properties allow us to find subgroups isomorphic to $F(a,b)$. This occurs through a special property of the action of a group on the set.
\begin{theorem}[Ping Pong Lemma]\label{thm:ping_pong_lemma}
  Let $G$ be a group that acts on a set $X$, and let $\Gamma_1,\Gamma_2$ be subgroups of $G$, with $\Gamma = \left\langle \Gamma_1,\Gamma_2 \right\rangle$. Assume $\Gamma_1$ contains at least three elements and assume $\Gamma_2$ contains at least two elements.\newline

  Let $\emptyset\neq X_1,X_2\subseteq X$ with $X_1\triangle X_2\neq\emptyset$. Suppose that for all $e_G\neq s\in \Gamma_1$ and for all $e_G\neq t\in \Gamma_2$, we have
  \begin{align*}
    s\cdot X_1&\subseteq X_2\\
    t\cdot X_2&\subseteq X_1.
  \end{align*}
  Then, $\Gamma$ is isomorphic to the free product $\Gamma_1\star \Gamma_2$.
\end{theorem}
Narrowing down, we may consider a ``doubles'' case that splits each of $X_1$ and $X_2$ and looks only at two elements of $G$.
\begin{corollary}[Ping Pong Lemma for ``Doubles'']\label{corollary:ping_pong_doubles}
  Let $G$ act on $X$, and let $A_{+}, A_{-},B_{+},B_{-}$ be disjoint subsets of $X$ whose union is not equal to $X$. Then, if
  \begin{align*}
    a\cdot \left(X\setminus A_{-}\right) &\subseteq A_{+}\\
    a^{-1}\cdot \left(X\setminus A_{+}\right) &\subseteq A_{-}\\
    b\cdot \left(X\setminus B_{-}\right) &\subseteq B_{+}\\
    b^{-1}\cdot \left(X\setminus B_{+}\right) &\subseteq B_{-},
  \end{align*}
  then it is the case that $\left\langle a,b \right\rangle$ is isomorphic to $F(a,b)$.
\end{corollary}
% free products and ping-pong lemma
\subsection{Functional Analysis}%
In Section \ref{sec:invariant_states}, we will begin discussing an alternative set of characterizations for amenability; in order to do that, we must cover some important concepts in functional analysis. Excellent resources to learn more include \cite{rudin_functional_analysis} and \cite{aliprantis_infinite_dimensional_analysis}.\newline

We assume that all vector spaces are over the complex numbers.\newline

First, we begin by discussing some important linear algebra concepts that are more geometric in nature.
\begin{definition}\label{def:vector_space_subset_operations}
  Let $X$ be a vector space.
  \begin{itemize}
    \item If $A,B\subseteq X$, then we define
      \begin{align*}
        A + B &= \set{x + y | x\in A,y\in B}.
      \end{align*}
      If $A = \set{x_0}$, we abbreviate $\set{x_0} + B$ as $x_0 + B$, which is called the translation of $B$ by $x_0$.
    \item If $A\subseteq X$, and $\alpha\in \C$, then
      \begin{align*}
        \alpha A &= \set{\alpha x | x\in A}
      \end{align*}
      is the scaling of $A$ by $\alpha$. We write $(-1)A = -A$.
    \item A subset $A\subseteq X$ is called \textit{symmetric} if $-A = A$.
    \item A subset $A\subseteq X$ is called \textit{balanced} if $\alpha A\subseteq A$ for all $\left\vert \alpha \right\vert\leq 1$.
    \item A subset $C\subseteq X$ is called \textit{convex} if for all $t\in [0,1]$ and $x_1,x_2\in C$, $\left(1-t\right)x_1 + tx_2 \in C$.
  \end{itemize}
  We define the \textit{convex hull} of $A\subseteq X$ by
  \begin{align*}
    \operatorname{conv}\left(A\right) &= \bigcap\set{C | A\subseteq C\subseteq X,C\text{ is convex}}\\
                                      &= \set{\sum_{j=1}^{n}t_ja_j | n\in\N,t_j\geq 0,\sum_{j=1}^{n}t_j = 1,a_j\in A}.
  \end{align*}
\end{definition}
\begin{definition}
  Let $X$ be a vector space. A \textit{seminorm} on $X$ is a map $p\colon X\times X\rightarrow \R$ that satisfies
  \begin{itemize}
    \item $p(x) \geq 0$;
    \item $p\left( x,y \right) \leq p\left( x \right) + p\left( y \right)$;
    \item $p\left( \alpha x \right) = \left\vert \alpha \right\vert p(x)$;
  \end{itemize}
  for all $x,y\in X$ and $\alpha\in \C$. If $p$ also satisfies
  \begin{itemize}
    \item $p\left( x \right) = 0$ if and only if $x = 0$;
  \end{itemize}
  then we say $p$ is a \textit{norm}. We usually write $\norm{\cdot}$.\newline

  The pair $\left( X,\norm[\cdot] \right)$ is known as a normed vector space.
\end{definition}
\begin{remark}
  Naturally, norms induce a metric on the vector space, given by
  \begin{align*}
    d\left( x,y \right) &= \norm{x-y}.
  \end{align*}
  It can be verified that the requirements for a metric are satisfied by this definition.
\end{remark}
\begin{example}[Some Normed Vector Spaces]\hfill
  \begin{enumerate}[(a)]
    \item The space $\R^n$, equipped with the Euclidean norm,
      \begin{align*}
        \norm{x} &= \left( \sum_{i=1}^{n}\left\vert x_i \right\vert^2 \right)^{1/2},
      \end{align*}
      is a normed vector space.
    \item The space of continuous functions, $f\colon [0,1]\rightarrow \C$, equipped with the norm
      \begin{align*}
        \norm{f}_{u} &= \sup_{x\in[0,1]}\left\vert f(x) \right\vert,
      \end{align*}
      is also a normed vector space, typically denoted $C\left( [0,1] \right)$.
    \item In general, if $\Omega$ is any set, then the space $\ell_{\infty}\left(\Omega\right)$ is the space of all functions $f\colon \Omega\rightarrow \C$ such that
      \begin{align*}
        \norm{f}_{\ell_{\infty}} &\coloneq \sup_{x\in\Omega}\left\vert f(x) \right\vert\\
                                 &< \infty.
      \end{align*}
      This is the space of bounded functions with domain $\Omega$.
  \end{enumerate}
\end{example}
\begin{definition}[Important Subsets of Normed Vector Spaces]
  Let $X$ be a normed vector space.
  \begin{itemize}
    \item We define the \textit{open ball} centered at $x\in X$ with radius $\ve > 0$ by
      \begin{align*}
        U\left( x,\ve \right) &\coloneq \set{y\in X | \norm{x-y} < \ve}.
      \end{align*}
      The open unit ball of $X$ is denoted $U_{X}\coloneq U\left( 0,1 \right)$.
    \item We define the \textit{closed ball} centered at $x\in X$ with radius $\ve > 0$ by
      \begin{align*}
        B\left( x,\ve \right) &\coloneq \set{y\in X | \norm{x-y} \leq \ve}.
      \end{align*}
      The closed unit ball of $X$ is denoted $B_{X}\coloneq B\left( 0,1 \right)$.
    \item We define the \textit{sphere} centered at $x\in X$ with radius $\ve > 0$ by
      \begin{align*}
        S\left( x,\ve \right) &\coloneq \set{y\in X | \norm{x-y} = \ve}.
      \end{align*}
      The unit sphere of $X$ is denoted $S_{X}\coloneq S\left( 0,1 \right)$.
  \end{itemize}
\end{definition}
% norms
Recall that if $X$ and $Y$ are vector spaces, then $\mathcal{L}\left( X,Y \right)$ is the vector space of all linear maps between $X$ and $Y$ when endowed with pointwise addition and scalar multiplication. If $Y = \C$, then $X' \coloneq \mathcal{L}\left( X,\C \right)$ is the space of linear functionals on $X$.\newline

However, when we deal with normed vector spaces, especially infinite-dimensional ones, we must take care to ensure the continuity of linear maps. We provide a brief overview of continuity in the context of normed vector spaces here, before moving on to one of the most important results related to continuity in normed vector spaces.
\begin{definition}
  Let $X$ and $Y$ be normed vector spaces, and let $T\colon X\rightarrow Y$ be a map.
  \begin{itemize}
    \item The function $T$ is continuous if, for all $c\in X$ and for all $\ve > 0$, there exists $\delta > 0$ such that whenever $\norm{x-c} < \delta$, then $\norm{T(x) - T(c)} < \ve$.
    \item The function $T$ is uniformly continuous if, for all $\ve > 0$, there exists $\delta > 0 $ such that for all $x,y\in X$, if $\norm{x-y} < \delta$, then $\norm{T(x) - T(y)} < \ve$.
    \item The function $T$ is Lipschitz continuous if there exists some constant $C > 0$ such that, for all $x,y\in X$, $\norm{T(x)-T(y)}\leq C\norm{x-y}$.
  \end{itemize}
\end{definition}
\begin{theorem}
  Let $X$ and $Y$ be normed vector spaces, and let $T\colon X\rightarrow Y$ be a linear map. Then, the following are equivalent:
  \begin{itemize}
    \item $T$ is continuous at $0$;
    \item $T$ is continuous;
    \item $T$ is uniformly continuous;
    \item $T$ is Lipschitz continuous;
    \item there exists some $C > 0$ such that, for all $x\in X$,
      \begin{align*}
        \norm{T(x)} &\leq C\norm{x}.
      \end{align*}
  \end{itemize}
\end{theorem}
\begin{definition}\hfill
  \begin{itemize}
    \item We say that a linear map $T\colon X\rightarrow Y$ is \textit{bounded} if $T\left( B_X \right)$ is a bounded set in $B_Y$.
    \item The operator norm of $T$ is defined by
      \begin{align*}
        \norm{T}_{\op} &\coloneq \sup_{x\in B_{X}} \norm{T(x)}.
      \end{align*}
    \item We define the collection of all continuous (or bounded) linear maps between $X$ and $Y$ by
      \begin{align*}
        \B\left( X,Y \right) &\coloneq \set{T | T\in \mathcal{L}\left( X,Y \right), \norm{T}_{\op} < \infty}.
      \end{align*}
    \item The \textit{continuous dual} of $X$ is the space
      \begin{align*}
        X^{\ast}\coloneq \B\left( X,\C \right).
      \end{align*}
      %Note that $X^{\ast}$ is a normed vector space when equipped with the operator norm.
  \end{itemize}
\end{definition}
The continuous dual, $X^{\ast}$, will feature prominently in our discussion of amenability in \ref{sec:invariant_states}, so we expand upon it a little bit here. Specifically, we discuss some topologies on $X^{\ast}$ and some prominent theorems related to the continuous dual.
\begin{definition}
  Let $X$ be a normed vector space, and let $X^{\ast}$ denote the continuous dual. Let $\left( \varphi_{\alpha} \right)_{\alpha}$ be a net (or generalized sequence) in $X^{\ast}$.
  \begin{itemize}
    \item We say $\left( \varphi_{\alpha} \right)_{\alpha}\rightarrow \varphi$ in the \textit{norm topology} if $\norm{\varphi_{\alpha} - \varphi}\rightarrow 0$.
    \item We say $\left( \varphi_{\alpha} \right)_{\alpha}\rightarrow \varphi$ in the \textit{weak* topology} if, for all $x\in X$, $\left( \varphi_{\alpha} \right)_{\alpha}\rightarrow \varphi(x)$. The weak* topology is the topology of pointwise convergence.
  \end{itemize}
\end{definition}
\begin{remark}
  Convergence in the norm topology implies convergence in the weak* topology, but not the other way around.
\end{remark}
One of the central results relating to the weak* topology is the Banach--Alaoglu theorem, which we will use to prove the existence of particular continuous linear functionals in Section \ref{sec:invariant_states}.
\begin{theorem}[Banach--Alaoglu]\label{thm:banach_alaoglu}
  Let $X$ be a normed vector space. Then, $B_{X^{\ast}}$ is compact in the weak* topology.
\end{theorem}
% normed vector spaces and continuous duals
% weak and weak* topology
% hahn--banach separation
The Banach--Alaoglu theorem provides information about the topological structure of $X^{\ast}$. Now, we turn our attention to understanding the analytic and geometric structure of $X^{\ast}$.\newline

Consider the following problem from linear algebra: if $X$ is a vector space, and $Y\subseteq X$ is a subspace, and $\varphi\in Y'$, is there a linear functional $\phi\in X$ such that $\phi|_{Y} = \varphi$?\newline

The answer is yes. We may take a basis $\mathcal{B} = \set{x_i}_{i\in I}$ for $Y$, and extend it to a basis for $X$, $\mathcal{C}$. We may then define $\Phi$ on the basis elements $\set{x_j}_{j\in J}$ of $X$ by
\begin{align*}
  \Phi\left( x_j \right) &= \begin{cases}
    \phi\left( x_j \right) & x_j\in \mathcal{B}\\
    0 & x_j\notin \mathcal{B}.
  \end{cases}
\end{align*}
However, when $X$ is a normed vector space, we also end up running into issues of continuity --- if $\varphi\in Y^{\ast}$ is continuous, how do we know that there exists a continuous $\Phi\in X^{\ast}$ such that $\Phi|_{Y} =\varphi$. For that matter, how do we know that there are any nonzero elements in $X^{\ast}$?\newline

This is the domain of the Hahn--Banach theorems. Both the extension and separation results will be eminently useful as we further study amenability.
\begin{theorem}[Hahn--Banach Continuous Extension]\label{thm:hahn_banach_extension}
  Let $X$ be a normed vector space, and let $Y\subseteq X$ be a subspace. If $\varphi\in Y^{\ast}$ is a continuous linear functional, then there is a (not necessarily unique) continuous $\Phi\in X^{\ast}$ such that $\Phi|_{Y} = \varphi$.
\end{theorem}
One of the primary uses of the Hahn--Banach extension is to establish crucial separation results.
\begin{theorem}[Hahn--Banach Separation Theorems]\label{thm:hahn_banach_separation}
  Let $X$ be a normed vector space. 
  \begin{itemize}
    \item Let $Y\subseteq X$ be a subspace. There is a continuous linear functional $\varphi\in X^{\ast}$ such that $\varphi|_{Y} = 0$ and $\varphi\left( x \right) = \dist_{Y}\left( x \right)$.
    \item If $C,K\subseteq X$ are closed and convex sets, with $K$ compact, then there is a continuous linear functional $\varphi\in X^{\ast}$, with $\varphi = u + iv$, with $t\in \R$, and $\delta > 0$, such that
      \begin{align*}
        u(x) \leq t \leq t + \delta \leq u(y)
      \end{align*}
      for all $x\in C$ and all $y\in K$.
  \end{itemize}
\end{theorem}

\section{What is Amenability?}\label{sec:intro_amenability}%
\section{Paradoxical Decompositions and Amenability}\label{sec:banach_tarski_tarskis_theorem}%
\section{Amenability and Invariant States}\label{sec:invariant_states}%
\section{Følner's Condition and Amenability}\label{sec:folner_condition}%
\section{Remarks and Notes}%
\section{Apologies and Acknowledgments}%
This thesis is an abridged version of a longer text that I have been writing. That text would have been my honors thesis, but unfortunately it would have been a bit long. I'm writing it with the aim of creating a thorough overview that properly introduces amenability, starting from discrete groups. That text includes other characterizations of amenability, such as a discussion of the left-regular representation and results that relate properties of the group $C^{\ast}$-algebra and amenability of the underlying group. That text will always be a bit of a work in progress, as the theory of amenability is extremely deep; the case of discrete groups is only one case of the more general theory of amenability in locally compact groups, which dives deeper into functional analysis.\newline

Ultimately, the goal of this whole thesis was to provide a more clear exposition on the topic of amenability, assuming minimal prerequisites. While there are certain leaps of faith that I take for granted (as, otherwise, this thesis would certainly be too long as was my original text), I hope that I did not use any major leaps of argumentation that seemed out of hand.\newline

This entire thesis would not be possible without the assistance and guidance of professor Rainone, who put forth the idea of an independent study on Tarski's theorem, and would not have happened without one of my friends at my REU, Lisa Samoylov, telling me that Dana Williams at Dartmouth was a good graduate student advisor. Unfortunately, he told me that he is probably retiring, but Appendix A in one of his books, \textit{Crossed Product $C^{\ast}$-Algebras}, was ultimately what convinced me to study amenability for my honors thesis. It turned out to be a very good idea.
\nocite{*}
\printbibliography[title={References}]
\end{document}
