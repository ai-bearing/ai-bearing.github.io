\documentclass[11pt]{package2}

%\usepackage{cmbright,sfmath,bbold}
%\renewcommand{\mathcal}{\mathtt}
\setcounter{chapter}{-1}
%\usepackage{newpxtext,eulerpx,eucal}
%\renewcommand*{\mathbb}[1]{\varmathbb{#1}}
%\renewcommand*{\hbar}{\hslash}
%\usepackage[light]{kpfonts}
%\renewcommand{\coloneq}{\coloneqq}
\renewcommand{\N}{\Z_{>0}}
\DeclareMathOperator{\cb}{cb}
\renewcommand{\mathbb}{\mathds}
\usepackage{titlesec}
\usepackage[backend=biber,style=alphabetic,sorting=nty]{biblatex}
\addbibresource{chapters/references.bib}
\DeclareMathOperator{\op}{op}
\DeclareMathOperator{\sa}{s.a.}
%\renewcommand{\coloneq}{=}
%\makeatletter
%\renewcommand{\maketitle}{%
%    \begin{center}
%        \rule{\textwidth}{0.6pt} \\[0.4em]  % Top line
%        \rule{\textwidth}{0.6pt} \\[1em]
%        {\LARGE\bfseries \@title} \\[0.5em] % Main Title
%        {\large\itshape \@subtitle} \\[1em] % Subtitle
%        {\large \@author} \\[0.5em]         % Author
%        {\large \@date} \\[1em]             % Date
%        \rule{\textwidth}{0.6pt} \\[0.4em]  % Bottom line
%        \rule{\textwidth}{0.6pt} \\[1em]
%    \end{center}
%}
%\makeatother
\title{Understanding Amenability in Discrete Groups\\\vspace{5pt}{\large A Gentle Introduction to Higher Analysis}}
%\newcommand{\subtitle}{A Gentle Introduction to Higher Analysis}
\author{Avinash Iyer}
\date{March 31, 2025}
\usepackage{microtype}
\hbadness=10000
\usepackage[hidelinks]{hyperref}
\begin{document}
\maketitle
\RaggedRight
\tableofcontents
%\part{Prelude}
\chapter{Introduction}
\section{Overview}%
In the beginning, God created the heavens and the Earth,\footnote{Well, maybe not God specifically.} and a lot of other things that are detailed in the book of Genesis. Unfortunately, those that wrote down and translated the book of Genesis failed to mention the most important feature of the universe that He (may or may not have) created --- the axiom of choice. It may be remarked that the axiom of choice is not, strictly speaking, a God-given creation, but accepting it certainly requires a leap of faith, and since we are going to be working in the realm of analysis throughout this thesis, we will be accepting it as such.\newline

Unfortunately, despite our best efforts, and the convenient results that the axiom of choice provides for (Example \ref{ex:results_follow_from_axiom_of_choice}), the axiom of choice provides some counterintuitive and downright paradoxical results --- not that it's worth throwing out, but it is certainly worth investigating and understanding. One of these counterintuitive results is detailed in Chapter \ref{ch:paradoxical_decompositions}, where we show implicitly that there does not exist a finitely additive measure on the three-dimensional real numbers that is also invariant under Euclidean isometry by proving the Banach--Tarski paradox in its most general form.\newline

Using the Banach--Tarski paradox as motivation, we then go to proving various characterizations, definitions, and proofs of amenability in groups --- i.e., we now want to understand when a group is well-behaved, rather than ill-behaved as in the case of the isometry group of $\R^3$. We first use some primarily group-theoretic techniques surrounding amenability, such as in the proof of Tarski's theorem in Chapter \ref{ch:tarskis_theorem} and the establishment of amenability in subgroups and quotient groups in the first section of Chapter \ref{ch:invariant_states}. We then use techniques from functional analysis to prove amenability, first by establishing the equivalence between amenability and the existence of an invariant state on $\ell_{\infty}(G)$, in sections 2--4 of Chapter \ref{ch:invariant_states}; we then expand on these techniques in Chapter \ref{ch:folner_condition} to understand Følner's condition and approximate means.\newline

After a quick discussion of the application of Følner's condition to geometric group theory, we discuss representations of groups as bounded operators on Hilbert spaces, using yet more techniques from functional analysis and the theory of operator algebras to show, in Chapter \ref{ch:left_regular_representation}, that a group is amenable if it satisfies certain criteria related to the left-regular representation of $\Gamma$ on the space $\ell_2\left(\Gamma\right)$. Finally, in Chapter \ref{ch:nuclearity}, we move from the representation of a group to representations of its group $\ast$-algebra, and show how properties of the group $C^{\ast}$-algebra inform us about properties of the group, and vice-versa.\newline

All vector spaces in this thesis are assumed to be over $\C$ unless otherwise specified, and all groups (unless apparent otherwise) are countable, finitely generated, and endowed with the discrete topology. However, even with this relatively limited scope, we can still establish some crucial results that provide a worthy harmonization of algebra and analysis.
\section{Apologies and Acknowledgments}%
As is evident from the file size or anyone who has a PDF viewer that reports the number of pages of a document, this thesis is certainly much, much longer than an undergraduate honors thesis generally is. Part of this is my fault --- I am more verbose and particular about spacing than the average mathematics writer --- and part of this is because the content was extremely fun to learn, and I just kept learning about it.\newline

This project's topic was originally conceived by professor Rainone in a footnote to one of the problems in problem set 8 of Real Analysis II in Spring 2024. The problem mentioned the idea of an amenable group and a paradoxical group, and had us prove the easy direction of Tarski's Theorem (Theorem \ref{thm:tarski}), with the footnote saying that the reverse direction (and an exploration of amenability more generally) was a suitable honors project. After much hemming and hawing by yours truly, an appendix in the book \textit{Crossed Products of $C^{\ast}$-Algebras} by Dana Williams eventually convinced me that amenability was a topic worth exploring and understanding.\newline

Furthermore, as I dove deeper into the functional analysis necessary to understand the more heavy results in amenability, professor Rainone's draft textbook, \textit{Functional Analysis-En Route to Operator Algebras} was an incredible resource that helped me really understand the fundamentals of a subject that I had long desired to learn, but where most of the books were quite terse and hard to follow. I hope the text gets published sometime in the future, it is an extremely valuable resource.\newline

The results and proofs in this thesis are primarily not my own, but a compilation of various sources that provide a broad and deep coverage of the subject. I have collected, simplified, exposited, and reordered them in order to understand not only for myself, but to potentially help others with the process of understanding amenability. I have attempted my best to attribute all theorems and proofs to the texts that I have obtained them from --- just because I may have forgotten to attribute a proof to someone does not mean it is my own (indeed, it is probably not).
%The groups in this thesis should be assumed countable and finitely generated unless it is otherwise clear that they are not (such as $\text{SO}\left( 3 \right)$, which is obviously not countable or finitely generated), and all are endowed with the discrete topology. This is because if I dedicated the necessary time towards understanding and including the case of topological groups with any locally compact topology, this thesis would actually be longer than the Bible. Nonetheless, the case of countable discrete groups still provides an extensive elaboration on concepts often covered in undergraduate analysis and algebra, and, as is the case with the Banach--Tarski paradox, strike at the foundations of mathematics.
\chapter{Categorical Constructions for the Unemployed Mathematician}\label{ch:categorical_constructions}
In this book, we cover certain structures --- like the free group, free $\ast$-algebra, tensor product, etc. --- that are usually not covered in the undergraduate algebra or analysis curriculum in depth. We discuss these ``free'' constructions\footnote{Hence the name of this chapter.} here, with the general theme that these constructions allow us to, in a ``universal'' manner, convert one type of map (a set-map or a bilinear map) into another type of map (a group homomorphism or a linear map).
\section{Free Groups}%
Given a set $A$, we want to know how exactly we can create a group structure from the elements in $A$ such that they extend from $A$ to a group generated by $A$ in a particularly ``natural'' way. This will be the free group.
\begin{definition}\label{def:generating_sets}
  Let $G$ be a group, and $S\subseteq G$ be a subset. We define the subgroup \textit{generated by} $S$ to be
  \begin{align*}
    \left\langle S \right\rangle_{G} &= \bigcap \set{H | S\subseteq H,~H\text{ a subgroup}}.
  \end{align*}
  We say $S$ generates $G$ if $\left\langle S \right\rangle_{G} = G$.\newline

  We say $\left\langle S \right\rangle_{G}$ is \textit{finitely generated} if $\Card(S) < \infty$.\newline

  If $S$ is such that, for any $x\in S$, we have $x^{-1}\in S$, then we say $S$ is \textit{symmetric}.
\end{definition}
\begin{fact}
  If $S = \set{s_1,\dots,s_n}\subseteq G$, then the picture of $\left\langle S \right\rangle$ is as follows:
  \begin{align*}
    \left\langle S \right\rangle_{G} &= \set{s_1^{a_1}s_2^{a_2}\cdots s_n^{a_n} | n\in\N,~s_1,\dots,s_n\in S,~a_1,\dots,a_n\in \set{-1,1}}.
  \end{align*}
\end{fact}
To construct a free group, we begin by stating its universal property --- that is, its innate nature as an ``extension'' of a set-function into a group structure. Then, we will show that a more constructive definition of the free group satisfies this universal property. The following section draws heavily from \cite{loh_geometric_group_theory}, but we will mostly focus on the construction of the free group rather than the proof of uniqueness.
\begin{definition}\label{def:free_group}
  Let $S$ be a set. A group $F$ containing $S$ is said to be \textit{freely generated} if, for every group $G$, and every map $\phi\colon S\rightarrow G$, there is a unique group homomorphism $\varphi\colon F\rightarrow G$ that extends $\varphi$. The following diagram, where $\iota$ denotes the inclusion of $S$ into $F$, commutes:
    \begin{center}
      \begin{tikzcd}
        S \arrow[d, "\iota"', hook] \arrow[r, "\phi"] & G \\
        F \arrow[ru, "\varphi"']                      &  
      \end{tikzcd}
    \end{center}
  We say $F$ is the \textit{free group} generated by $S$.
\end{definition}
Intuitively, to construct the free group, if we have $a\mapsto \phi(a)$ between $S$ and $G$, then we will define $\varphi\left(a^n\right) = \phi(a)^n$ inside $F(S)$. Uniqueness will follow from the fact that we can take two groups that satisfy the universal property, $F$ and $F'$, and apply the universal property on set-valued functions between $S$ and $F$ and $S$ and $F'$ respectively. 
\begin{theorem}\label{thm:free_group_existence}
  If $S$ is some set, then there is some freely generated group $F(S)$ that satisfies \ref{def:free_group}.
\end{theorem}
\begin{proof}
  We will construct a group consisting of ``words'' made up of elements of $S$ and their inverses. This starts by considering the alphabet $A = S\cup \widehat{S}$, where $\widehat{S}$ is a disjoint copy of $S$ --- every $\hat{s}\in \widehat{S}$ will play the role of an inverse to $s$ in our group.
  \begin{itemize}
    \item Define $A^{\ast}$ to be the set of all words over the alphabet $A$, including the empty word, $\epsilon$. We define the operation $A^{\ast}\times A^{\ast}\rightarrow A^{\ast}$ by concatenating words, which is an associative operation with neutral element $\epsilon$.
    \item Define the equivalence relation $\sim$ generated by the following two relations, where for all $x,y\in A^{\ast}$ and $s\in S$, we have
      \begin{align*}
        xs\hat{s}y &\sim xy\\
        x\hat{s}sy &\sim xy.
      \end{align*}
      The equivalence classes with respect to $\sim$ will be denoted $\left[\cdot\right]$.\newline

      We have a well-defined composition $\left[x\right]\left[y\right] = \left[xy\right]$ mapping $F(S) \times F(S) \rightarrow F(S)$ for all $x,y\in A^{\ast}$.
  \end{itemize}
  We show that $F(S)$ with the concatenation operation is a group. Here, we see that $\left[\epsilon\right]$ is the neutral element for the composition, and associativity is inherited from associativity of concatenation in $A^{\ast}$. To show the existence of inverses, we define the inverse map inductively by taking $I\left(\epsilon\right) = \epsilon$, and
  \begin{align*}
    I\left(sx\right) &= I(x)\hat{s}\\
    I\left(\hat{s}x\right) &= I(x)s
  \end{align*}
  for all $x\in A^{\ast}$  and $s\in S$. Inductively, we can see that $I\left(I\left(x\right)\right) = x$ and
  \begin{align*}
    \left[I(x)\right]\left[x\right] &= \left[I(x)x\right]\\
                                    &= \left[\epsilon\right]\\
    \left[x\right]\left[I(x)\right] &= \left[xI(x)\right]\\
                                    &= \left[\epsilon\right].
  \end{align*}
  Thus, $F(S)$ is a group.\newline

  Now, we show $F(S)$ is freely generated. Let $i\colon S\rightarrow F(S)$ be the map that sends $s\mapsto \left[s\right]$. By our construction, we know that $i(S)\subseteq F(S)$ is a generating set for $F(S)$. We will show the universal property holds for $F(S)$.\newline

  To start, let $\phi\colon S\rightarrow G$ be a set-valued map between $S$ and an arbitrary group $G$. We construct $\phi^{\ast}\colon A^{\ast}\rightarrow G$ by taking
  \begin{align*}
    \epsilon &\mapsto e\\
    sx &\mapsto \phi(s)\phi^{\ast}\left(x\right)\\
    \hat{s}x &\mapsto \left(\phi(s)\right)^{-1}\phi^{\ast}\left(x\right)
  \end{align*}
  for all $x\in A^{\ast}$ and $s\in S$. This definition of $\phi^{\ast}$ is compatible with the equivalence relation on $A^{\ast}$, and we see that $\phi^{\ast}\left(xy\right) = \phi^{\ast}\left(x\right)\phi^{\ast}\left(y\right)$. Thus, we get a well-defined map $\varphi\colon F(S)\rightarrow G$, taking $\left[x\right]\mapsto \left[\phi^{\ast}\left(x\right)\right]$.\newline

  It remains to be shown that the map $i\colon S\rightarrow F(S)$ is injective, which will show that $F(S)$ is freely generated by $S$. Let $s_1,s_2\in S$, and consider the set-function $\phi\colon S\rightarrow \Z$ given by $\phi\left(s_1\right) = 1$ and $\phi\left(s_2\right) = -1$. Then, we must have
  \begin{align*}
    \varphi\left(i\left(s_1\right)\right) &= \phi\left(s_1\right)\\
                                          &= 1\\
                                          &\neq -1\\
                                          &= \phi\left(s_2\right)\\
                                          &= \varphi\left(i\left(s_2\right)\right).
  \end{align*}
  Thus, we have $i\left(s_1\right)\neq i\left(s_2\right)$, so $i$ is injective.
\end{proof}
Most of the definitions of the free group automatically default to the characterization of $F(S)$ as the set of reduced words in $S\cup S^{-1}$. This is the characterization we will be using in the future, but it is still important to understand where exactly the ``free'' in free group comes from, and how it relates to the particular universal property that actually characterizes $F(S)$ uniquely up to isomorphism.
\section{Free Vector Spaces}%
Given a set $A$, just as we are able to construct a free group, $F(A)$, we can take any set $A$ and construct a ``universal'' vector space out of the set.\newline

The free vector space (as it is known) is the universal object that extends any set-valued function into a linear map, treating elements of the set as its basis (Definition \ref{def:basis}). We are interested in the case of the free vector space over the complex numbers, but note that the following definition of the free vector space applies over any field. 
\begin{theorem}\label{thm:free_vector_space}
  Let $\Gamma$ be a nonempty set. There is a vector space, $\C\left[\Gamma\right]$, with $\Dim\left(\C\left[\Gamma\right]\right) = \Card\left(\Gamma\right)$, and an injective map $\delta\colon \Gamma\rightarrow \C\left[\Gamma\right]$ such that the following universal property holds: if $V$ is a $\C$-vector space, and $\phi\colon \Gamma\rightarrow V$ is a set-valued function, then there is a unique linear map $T_{\phi}\colon \C\left[\Gamma\right]\rightarrow V$ such that $T_{\phi}\circ \delta = \phi$.
  \begin{center}
    % https://tikzcd.yichuanshen.de/#N4Igdg9gJgpgziAXAbVABwnAlgFyxMJZABgBpiBdUkANwEMAbAVxiRAB12BxOgW17ogAvqXSZc+QigCM5KrUYs2nAMLJOPfnQrDRIDNjwEis6fPrNWiEADVh8mFADm8IqABmAJwi8kZEDgQSLIKlsrssAw4giIe3r6IIYFIAEzUDHQARjAMAAriRlIgDDDuOCDUFkrWACoA+sCcaAAWWEK6cT5+1MmIacVZOfmGkmyeWE7N5ZWKVhzsLVj2QkA
\begin{tikzcd}
\Gamma \arrow[r, "\delta"] \arrow[rd, "\phi"'] & {\C[\Gamma]} \arrow[d, "T_{\phi}"] \\
                                               & V                                 
\end{tikzcd}
  \end{center}
\end{theorem}
\begin{proof}
  Consider the linear subspace of finitely supported functions, $\C\left[\Gamma\right]\subseteq \mathcal{F}\left(\Gamma,\C\right)$. For each $t\in \Gamma$, we define
  \begin{align*}
    \delta_t\left(s\right) &= \begin{cases}
      1 & s=t\\
      0 & \text{else}
    \end{cases}.
  \end{align*}
    We see that $\delta_t\neq \delta_s$ whenever $s\neq t$, meaning that the map $\delta\colon \Gamma\rightarrow \C\left[\Gamma\right]$, defined by $s \mapsto \delta_s$, is injective.\newline

    We will show that $\set{\delta_s}_{s\in \Gamma}$ is a linear basis for $\C\left[\Gamma\right]$. If $f\in \C\left[\Gamma\right]$, with $\supp\left(f\right) = \set{s_1,\dots,s_n}\subseteq \Gamma$, we set $\alpha_j = f\left(t_j\right)$, and see that
    \begin{align*}
      f &= \sum_{j=1}^{n}\alpha_j\delta_{s_j},
    \end{align*}
    which shows that $\set{\delta_s}_{s\in\Gamma}$ is a spanning set.\newline

    To show that $\set{\delta_s}_{s\in\Gamma}$ is linearly independent, consider $g = \sum_{j=1}^{n}\alpha_j\delta_{s_j}\in \C\left[\Gamma\right]$ such that $g = 0$. Then, $g(t) = 0$ for all $t\in\Gamma$, and in particular, $g\left(s_i\right) = 0$ for every $1 \leq i \leq n$. Thus, we have
    \begin{align*}
      0 &= g\left(s_j\right)\\
        &= \sum_{j=1}^{n}\alpha_j\delta_{s_j}\left(s_i\right)\\
        &= \alpha_i,
    \end{align*}
    so $\alpha_j = 0$ for each $j$. Thus, $\set{\delta_s}_{s\in \Gamma}$ is linearly independent.\newline

    Turning to the universal property, we define $T_{\phi}\colon \C\left[\Gamma\right]\rightarrow V$ in terms of $\phi$ as follows:
    \begin{align*}
      T_{\phi}\left(\sum_{j=1}^{n}\alpha_j\delta_{s_j}\right) &= \sum_{j=1}^{n}\alpha_j\phi\left(s_j\right).
    \end{align*}
    This yields an expression of $T_{\phi}$ uniquely in terms of $\phi$ and $\delta$.
\end{proof}
\begin{example}
  Let $z$ be an abstract variable, and consider the set of ``formal powers'' of $z$, $\set{z^k}_{k\in\N}$. Then, the free vector space generated by this set, $\C\left[z\right]$, is the set of all polynomials with coefficients in $\C$. By the universal property, we know that every polynomial $p\in \C\left[z\right]$ has a unique expression $p = \sum_{j=0}^{n}a_jz^j$.
\end{example}
One of the primary uses of the free vector space is that, via this construction, we can show that vector spaces are particularly nice algebraic objects. We often use these properties implicitly in linear algebra.
\begin{theorem}\label{thm:injective_projective_objects}
  Let $X$, $Y$, and $Z$ be vector spaces.
  \begin{enumerate}[(a)]
    \item If $\iota\colon Y\hookrightarrow X$ is an injective linear map, and $\varphi\colon Y\rightarrow Z$ is a linear map, then there is a (not necessarily unique) map $T\colon X\rightarrow Y$ such that $T\circ\iota = \varphi$.
      \begin{center}
        % https://tikzcd.yichuanshen.de/#N4Igdg9gJgpgziAXAbVABwnAlgFyxMJZABgBpiBdUkANwEMAbAVxiRGJAF9T1Nd9CKAIzkqtRizYBNLjxAZseAkRFCx9Zq0QgAWrN6KBRAEyjqGydoAaXMTCgBzeEVAAzAE4QAtkhEgcEEgAzOYSWiAAOhH4OHQg1LFYDGwAFhAQANb6IB7evgmBiKYgDHQARjAMAAp8SoIg7lgOKTjx4ppsUfTuaClY2bk+iCH+hcWlFdW1RtoMMK6toR3aACoDnkNko76cFJxAA
\begin{tikzcd}
0 \arrow[r] & Y \arrow[r, "\iota", hook] \arrow[d, "\varphi"'] & X \arrow[ld, "T"] \\
            & Z                                                &                  
\end{tikzcd}
      \end{center}
This shows that vector spaces are injective objects --- any linear map factors through an injective map.
    \item If $\pi\colon X\rightarrow Z$ is a surjective linear map, and $\varphi\colon Y\rightarrow Z$ is a linear map, then there is a (not necessarily unique) map $\delta\colon Y\rightarrow X$ such that $\pi\circ\delta = \varphi$.
      \begin{center}
        % https://tikzcd.yichuanshen.de/#N4Igdg9gJgpgziAXAbVABwnAlgFyxMJZARgBoAGAXVJADcBDAGwFcYkQBNEAX1PU1z5CKMsWp0mrdgC0efEBmx4CRcqTE0GLNohAANOfyVCiAJnXitU3eR7iYUAObwioAGYAnCAFskakDgQSOYgjPQARjCMAAoCysIgHliOABY4IJqSOiAAOjmwjDj0hiCePsE0gUhkoRFRscYqukmp6Zna7HloWCVlvoj+VYg1Vtl5DB5oKT00YZExcSa6jDBu6bzuXv01QwDM3JTcQA
        \begin{tikzcd}
                            & Y \arrow[ld, "\delta"'] \arrow[d, "\varphi"] &   \\
        X \arrow[r, "\pi"'] & Z \arrow[r]                                  & 0
        \end{tikzcd}
      \end{center}
      This shows that vector spaces are projective objects --- any linear map factors through a surjective map.
  \end{enumerate}
\end{theorem}
\begin{proof}\hfill
  \begin{enumerate}[(a)]
    \item Let $\mathcal{A}$ be a basis for $Y$. Then, since $\iota$ is an injective linear map, the set $\mathcal{B}_0 = \set{\iota(y) | y\in \mathcal{A}}$ can be extended to a basis $\mathcal{B}$ for $X$.\newline

      We set $t\colon \mathcal{B}\rightarrow Z$ to be
      \begin{align*}
        t\left(x\right) &= \begin{cases}
          \varphi(x) & x\in \mathcal{B}_0\\
          0 & x\in \mathcal{B}\setminus \mathcal{B}_0
        \end{cases}.
      \end{align*}
      By the universal property of the free vector space, this extends to a linear map $T\colon X\rightarrow Y$. Since $T\circ\iota$ agrees with $\varphi$ on $\mathcal{A}$, the universal property of the free vector space states that $T\circ\iota$ agrees with $\varphi$ on all of $Y$.
    \item Let $\set{y_i}_{i\in I}$ be a basis for $Y$. We define $d\left( y_i \right) = x_i\in  \pi^{-1}\circ \varphi\left( y_i \right)$ for each $i\in I$, where $x_i\in \pi^{-1}\circ \varphi\left( y_i \right)$ is some representative.  By the universal property of the free vector space, this extends to a unique linear map $\delta\colon Y\rightarrow X$ that agrees on the basis of $Y$.
  \end{enumerate}
\end{proof}

\section{Free Algebras}%
Later chapters of this thesis will require understanding results from the theory of operator algebras and algebras more generally. Here, we establish a purely algebraic understanding of a free construction, similar to the free vector space and free group. Just as there are free groups and free vector spaces, we can also talk about free algebras. In Chapter \ref{ch:nuclearity}, we will construct special norms on free algebras to elucidate properties of the underlying group.\newline

Similar to a free group, the free algebra (or free $\ast$-algebra) is constructed by taking a certain collection of ``words'' over a set of symbols, and then, if desired, ``modding out'' by the ideal generated by a set of relations. We formalize this in steps.
\begin{definition}\label{def:free_algebra}
  Let $E=\set{x_i}_{i\in I}$ be a collection of symbols that may not commute. The space of all polynomials over $E$ is the free vector space over the set of words formed by symbols in $E$,
  \begin{align*}
    \Gamma_E &= \set{x_{i_1}x_{i_2}\cdots x_{i_n} | n\in\N,i_1,\dots,i_n\in I}.
  \end{align*}
  We denote this space $\C \left\langle E \right\rangle$.\newline

  In the free vector space $\C \left\langle E \right\rangle$, we may define multiplication by concatenation:
  \begin{align*}
    \left(x_{i_1}x_{i_2}\cdots x_{i_n}\right)\left(x_{j_1}x_{j_2}\cdots x_{j_m}\right) &= x_{i_1}x_{i_2}\cdots x_{i_n}x_{j_1}x_{j_2}\cdots x_{j_m},
  \end{align*}
  where $i_1,\dots,i_n,j_1,\dots,j_m\in I$. The space $\C\left\langle E \right\rangle$, equipped with multiplication by concatenation, is known as the \textit{free algebra} on $E$.\newline

  To turn $\C\left\langle E \right\rangle$ into a $\ast$-algebra, we define the formal set $E^{\ast} = \set{x_{i}^{\ast}}_{i\in I}$, and define the involution on $\C\left\langle E\cup E^{\ast} \right\rangle$ by taking
  \begin{align*}
    \left(\alpha x_{i_1}^{\ve_1}x_{i_2}^{\ve_2}\cdots x_{i_n}^{\ve_n}\right)^{\ast} &= \overline{\alpha}x_{i_n}^{\delta_n}x_{i_{n-1}}^{\delta_{n-1}}\cdots x_{i_2}^{\delta_2}x_{i_1}^{\delta_1},
  \end{align*}
  where
  \begin{align*}
    \delta_j &= \begin{cases}
      \ast & \ve_j = 1\\
      1 & \ve_j = \ast
    \end{cases}.
  \end{align*}
  The set $\C\left\langle E\cup E^{\ast} \right\rangle$ with the involution defined above is known as the \textit{free $\ast$-algebra} on $E$, and is usually denoted $\mathbb{A}^{\ast}\left(E\right)$.\newline

  If $R\subseteq \mathbb{A}^{\ast}\left(E\right)$ is a collection of relations, we let $I(R) = \operatorname{ideal}\left(R\right)$. Then, the quotient algebra
  \begin{align*}
    \mathbb{A}^{\ast}\left(E|R\right) &= \mathbb{A}^{\ast}\left(E\right)/I(R)
  \end{align*}
  is known as the \textit{universal $\ast$-algebra on $E$ with relations $R$}.
\end{definition}
Evident from the name, the universal $\ast$-algebra(s) admit universal properties that characterize them as unique.
\begin{theorem}[Universal Properties]\label{thm:universal_property_free_algebra}
  Let $E = \set{x_i}_{i\in I}$ be a set of abstract symbols, and let $B$ be a $\ast$-algebra. Let $\phi\colon E\rightarrow B$ be an injective map, and define $b_i = \phi\left(x_i\right)$.
  \begin{itemize}
    \item There is a unique $\ast$-homomorphism $\varphi\colon \mathbb{A}^{\ast}\left(E\right) \rightarrow B$ such that $x_i \mapsto b_i$. The following diagram commutes.
      \begin{center}
        % https://tikzcd.yichuanshen.de/#N4Igdg9gJgpgziAXAbVABwnAlgFyxMJZABgBpiBdUkANwEMAbAVxiRAFEQBfU9TXfIRQBGclVqMWbAELdeIDNjwEiZYePrNWiEAB1dAWzo4AFgCMzwAIJcAFOwCU3cTCgBzeEVAAzAE4QDJDIQHAgkUQktNn00Eyw5H39AxGDQpAAmagY6MxgGAAV+ZSEQXyw3ExwQak0pHX18HDoEkD8AjOo0xAjs3IKiwTYyiqqayW09XXpfWPiuCi4gA
        \begin{tikzcd}
        E \arrow[r, "\phi"] \arrow[d, "\iota"'] & B \\
        \mathbb{A}^{\ast}(E) \arrow[ru, "\varphi"']    &  
        \end{tikzcd}
      \end{center}
    \item If $R\subseteq \mathbb{A}^{\ast}\left(E\right)$ is a set of relations, and $\set{b_i}_{i\in I}$ satisfies the relations $R$, then there is a unique $\ast$-homomorphism $\mathbb{A}^{\ast}\left(E|R\right) \rightarrow B$ such that $x_i + I(R) \mapsto b_i$. The following diagram commutes.
      \begin{center}
        % https://tikzcd.yichuanshen.de/#N4Igdg9gJgpgziAXAbVABwnAlgFyxMJZABgBpiBdUkANwEMAbAVxiRAFEQBfU9TXfIRQBGclVqMWbAELdeIDNjwEiZYePrNWiEAB1dAWzo4AFgCMzwAIJcAesH104OLgAp2AHwBKASm7iYKABzeCJQADMAJwgDJDIQHAgkUQktNn00Eyw5COjYxHjEpAAmagY6MxgGAAV+ZSEQSKwgkxwQak0pHX18HDockCiYkuoixBTyypq6wTYmlraOyW09XXpIzOyuCi4gA
\begin{tikzcd}
E \arrow[r, "\phi"] \arrow[d, "\iota"']       & B \\
\mathbb{A}^{\ast}(E|R) \arrow[ru, "\varphi"'] &  
\end{tikzcd}
      \end{center}
  \end{itemize}
\end{theorem}
One of the most important $\ast$-algebras we will study is generated from a group by taking the free vector space over the group.
\begin{definition}\label{def:group_star_algebra}
  Let $\Gamma$ be a group with identity element $e$, and let $\C\left[\Gamma\right]$ be the free vector space generated by $\Gamma$. We define a multiplication $f \ast g$, where $f,g\in \C\left[\Gamma\right]$ are finitely supported functions, by convolution:
  \begin{align*}
    f\ast g(s) &= \sum_{t\in\Gamma}f(t)g\left(t^{-1}s\right)\\
               &= \sum_{r\in\Gamma}f\left(sr^{-1}\right)g\left(r\right).
  \end{align*}
  The involution on $\C\left[\Gamma\right]$ is defined by $f^{\ast}\left(t\right) = \overline{f\left(t^{-1}\right)}$. The multiplicative identity is $\delta_e$, and multiplication satisfies $\delta_s\ast \delta_t = \delta_{st}$. Furthermore, this gives $\delta_{s}^{\ast} = \delta_{s^{-1}}$.\newline

  This is known as the \textit{group $\ast$-algebra}.
\end{definition}
\begin{remark}
  In Chapter \ref{ch:nuclearity}, we will endow the group $\ast$-algebra with special norms to create the group $C^{\ast}$-algebra(s).
\end{remark}
\section{Tensor Products}%
Given two vector spaces $V,W$, and a bilinear map $b\colon V\times W \rightarrow Z$ (for some vector space $Z$), it's tempting to use the property of the free vector space to find a linear map on some structure that incorporates both $V$ and $W$ and stays faithful to the bilinear map $b$. Indeed, this is what the tensor product of the vector spaces $V$ and $W$ is --- a universal construction that ``turns'' bilinear maps into linear maps.\newline

In this section, we detail the construction of the tensor product $V\otimes W$, and apply it to the specific case when $V$ and $W$ are Banach spaces (see definition \ref{def:norms}) to obtain certain norms on the tensor product that ``play nicely'' with the norms on $V$ and $W$.
\subsection{Algebraic Fundamentals}%
\begin{definition}\label{def:bilinear_map}
  Let $V,W,Z$ be vector spaces, and let $b\colon V\times W\rightarrow Z$ be a map such that, for all $\alpha\in \C$, $v,v_1,v_2\in V$, and $w,w_1,w_2\in W$,
  \begin{align*}
    b\left( \alpha v_1 + v_2,w \right) &= \alpha b\left( v_1,w \right) + b\left( v_2,w \right)\\
    b\left( v,\alpha w_1 + w_2 \right) &= b\left( v,w_1 \right) + \alpha b\left( v,w_2 \right).
  \end{align*}
  Then, we say $b$ is \textit{bilinear}. The space of bilinear maps is denoted $\operatorname{Bil}\left( V,W;Z \right)$.
\end{definition}
Just as we defined the free vector space and free group, we define the tensor product through a universal property --- and, just as with the case of the free group, we will focus more on the construction of the tensor product than on showing uniqueness.
\begin{theorem}[Universal Property of Tensor Products]\label{thm:tensor_product_existence}
  Let $V,W,Z$ be vector spaces, and let $b\colon V\times W \rightarrow Z$ be a bilinear map. Then, there exists a vector space, $V\otimes W$ and a linear map $T\colon V\otimes W \rightarrow Z$ such that for any $v\in V$ and $w\in W$, $T\left( v\otimes w \right) = b\left( v,w \right)$. The following diagram, where $\iota\colon V\times W \hookrightarrow V\otimes W$ is defined by $\left( v,w \right)\mapsto v\otimes w$, commutes.
  \begin{center}
      % https://tikzcd.yichuanshen.de/#N4Igdg9gJgpgziAXAbVABwnAlgFyxMJZABgBpiBdUkANwEMAbAVxiRADUAdTvAW3gAEAdRABfUuky58hFAEZyVWoxZsunCH0Ejxk7HgJEFcpfWatEIAFpilMKAHN4RUADMAThF5IyIHBCQAJmoGOgAjGAYABSkDWRB3LAcACxwQajNVSzCxCRAPLx9qfyQFZXM2bnwcOnS-OiwGNkgwVl18z29EMpLEYJBQiOjYmTYGGFc0jJULEAAVW1EgA
    \begin{tikzcd}
    V\times W \arrow[rd, "b"'] \arrow[r, "\iota"] & V\otimes W \arrow[d, "T"] \\
                                                  & Z                        
    \end{tikzcd}
  \end{center}
  The vector space $V\otimes W$ is unique up to linear isomorphism, and is known as the \textit{tensor product} of $V$ and $W$.
\end{theorem}
\begin{proof}
  We focus on showing existence. With $V$ and $W$ as in Theorem \ref{thm:tensor_product_existence}, we consider the free vector space (Theorem \ref{thm:free_vector_space}) on $V\times W$, $\C\left[ V\times W \right]$. Elementary elements of $V\times W$ are of the form $\delta_{(v,w)}$, where
  \begin{align*}
    \delta_{(v,w)} \left( s,t \right) &= \begin{cases}
      1 & v=s,w=t\\
      0 & \text{else}
    \end{cases}.
  \end{align*}
  Intuitively, from the way we have defined the tensor product as a linear map that extends a bilinear map, we would find the following properties of tensors desirable, for any $v,v_1,v_2\in V$, $w,w_1,w_2\in W$, and $\alpha\in \C$
  \begin{align*}
    \left( v_1 + v_2 \right)\otimes w &= v_1\otimes w + v_2\otimes w\label{eq:rel_1}\tag{1}\\
    v\otimes\left( w_1 + w_2 \right) &= v\otimes w_1 + v\otimes w_2\label{eq:rel_2}\tag{2}\\
    \left( \alpha v \right)\otimes w &= \alpha \left( v\otimes w \right)\label{eq:rel_3}\tag{3}\\
    v\otimes \left( \alpha w \right) &= \alpha \left( v\otimes w \right)\label{eq:rel_4}\tag{4}.
  \end{align*}
  With these four desirable properties in mind, we define a certain set of relations on the free vector space that we will ``mod out'' to obtain our desired tensor product. 
  \begin{itemize}
    \item To satisfy \eqref{eq:rel_1}, we define the set $N_1 = \set{\delta_{\left( v_1 + v_2,w \right)} - \delta_{\left( v_1,w \right)} - \delta_{\left( v_2,w \right)} | v_1,v_2\in V,w\in W}$, as this will be equivalent to the statement $\left( v_1 + v_2 \right)\otimes w - v_1\otimes w - v_2\otimes w = 0$.
    \item To satisfy \eqref{eq:rel_2}, we define the set $N_2 = \set{\delta_{\left( v,w_1 + w_2 \right)} - \delta_{\left( v,w_1 \right)} - \delta_{\left( v,w_2 \right)} | v\in V,w_1,w_2\in W}$, as this will be equivalent to the statement $v\otimes \left( w_1 + w_2 \right) - v\otimes w_1 - v\otimes w_2 = 0$.
    \item To satisfy \eqref{eq:rel_3}, we define the set $N_3 = \set{\delta_{\left( \alpha v,w \right)} - \alpha \delta_{\left( v,w \right)} | \alpha\in\C,v\in V,w\in W}$, as this will be equivalent to the statement $\left( \alpha v \right)\otimes w - \alpha \left( v\otimes w \right) = 0$.
    \item To satisfy \eqref{eq:rel_4}, we define the set $N_4 = \set{\delta_{\left(  v,\alpha w \right)} - \alpha \delta_{\left( v,w \right)} | \alpha\in\C,v\in V,w\in W}$, as this will be equivalent to the statement $v\otimes \left( \alpha w \right) - \alpha \left( v\otimes w \right) = 0$.
  \end{itemize}
  We define the ``zero set'' of our tensor product to be
  \begin{align*}
    N &= \Span\left( N_1 \cup N_2\cup N_3\cup N_4 \right),
  \end{align*}
  and consider the quotient space (Definition \ref{def:subspace_quotient_space_direct_sum}) $\C\left[ V\times W \right]/N$. We define
  \begin{align*}
    v\otimes w &\coloneq \delta_{(v,w)} + N.
  \end{align*}
  It can be verified that this definition is faithful to our requirements in \eqref{eq:rel_1}--\eqref{eq:rel_4}. Elements of $V\otimes W$ are of the form $\sum_{i\in I}v_i \otimes w_i$. We call elements of the form $v\otimes w$ \textit{elementary tensors}.\newline

  Define $\iota\colon V\times W \rightarrow V\otimes W$ by $\left( v,w \right) \mapsto v\otimes w$, and set $b = T\circ \iota$.\newline

  We verify that this definition satisfies the universal property of tensor products. We let $v_1,v_2,v\in V$, $w_1,w_2,w\in W$, and $\alpha\in \C$. Then,
  \begin{align*}
    b\left( v_1 + cv_2,w \right) &= T\left( \iota\left( v_1 + cv_2,w \right) \right)\\
                                 &= T\left( \left( v_1 + cv_2 \right)\otimes w \right)\\
                                 &= T\left( v_1\otimes w + c\left( v_2\otimes w \right) \right)\\
                                 &= T\left( v_1\otimes w \right) + cT\left( v_2\otimes w \right)\\
                                 &= b\left( v_1,w \right) + cb\left( v_2,w \right)\\
                                 \\
    b\left( v,w_1 + cw_2 \right) &= T\left( \iota\left( v,w_1 + cw_2 \right) \right)\\
                                 &= T\left( v\otimes \left( w_1 + cw_2 \right) \right)\\
                                 &= T\left( v\otimes w_1 + c\left( v\otimes w_2 \right) \right)\\
                                 &= T\left( v\otimes w_1 \right) + cT\left( v\otimes w_2 \right)\\
                                 &= b\left( v,w_1 \right) + cb\left( v,w_2 \right).
  \end{align*}
  Thus, by the universal property of the free vector space, there is a unique linear map $\tilde{b}\colon \C\left[ V\times W \right]\rightarrow Z$, defined by $\tilde{b}\left( \delta_{(v,w)} \right) = b\left( v,w \right)$.\newline

  Note that $\tilde{b}$ vanishes on $N$, so by the First Isomorphism Theorem there is a unique linear map $T_{b}\colon \C\left[ V\times W \right]/N\rightarrow Z$ that is defined by $T_{b} \circ \pi = \tilde{b}$, where $\pi\colon \C\left[ V\times W \right] \rightarrow \C\left[ V\times W \right]/N$ is the canonical projection.\newline

  Thus, we know that $T = T_{b}$ satisfies the universal property of tensor products.
\end{proof}
\begin{remark}
  Elements of the tensor product $X\otimes Y$ are of the form
  \begin{align*}
    t &= \sum_{k=1}^{n}x_k\otimes y_k,
  \end{align*}
  where $x_k\in X$ and $y_k\in Y$. Elements of the form $x_k\otimes y_k$ are known as \textit{elementary tensors}.\newline

  We note that any such $t$ has a variety of representations as elements of the tensor product.
\end{remark}
In linear algebra, we often use the universal property of tensor products to convert from bilinear maps to linear maps. However, we can also apply tensor products to spaces of linear maps by using the universal property.
\begin{proposition}[{\cite[Proposition E.6.14]{rainone_analysis}}]
  Let $X,Y,V,W$ be $\C$-vector spaces.
  \begin{enumerate}[(1)]
    \item If $T\in \mathcal{L}\left( X,V \right)$ and $S\in \mathcal{L}\left( Y,W \right)$, then there is a unique linear map
      \begin{align*}
        T\bar{\otimes} S \colon X\otimes Y \rightarrow V\otimes W
      \end{align*}
      that satisfies $T\bar{\otimes} S\left( x\otimes y \right) = T\left( x \right)\otimes S\left( y \right)$ for all $x\in X$ and $y\in Y$.
    \item For all $\varphi\in X'$ and $\psi\in Y'$, there is a linear map $\varphi\times\psi \in \left( X\otimes Y \right)'$ such that $\left( \varphi\times\psi \right)\left( x\otimes y \right) = \varphi\left( x \right)\psi\left( y \right)$ for all $x\in X$ and $y\in y$.
  \end{enumerate}
\end{proposition}
\begin{proof}\hfill
  \begin{enumerate}[(1)]
    \item The map $X\times Y \rightarrow V\otimes W$ that sends $\left( x,y \right)\mapsto T\left( x \right)\otimes S\left( y \right)$ is bilinear. Thus, by the universal property of tensor products, there exists a linear map $T \bar{\otimes} S \colon X\otimes Y \rightarrow V\otimes W$.
    \item Similarly, the map $X\times Y \rightarrow \C$ given by $\left( x,y \right)\mapsto \varphi(x)\psi(y)$ is bilinear, so the universal property gives us $\varphi\times\psi\colon X\otimes Y \rightarrow \C$ such that $\left( \varphi\times \psi \right)\left( x\otimes y \right) = \varphi\left( x \right)\psi\left( y \right)$.
  \end{enumerate}
\end{proof}
\begin{remark}
Technically, the map $T\bar{\otimes} S$ is an element of $\mathcal{L}\left( X\otimes Y,V\otimes W \right)$, rather than the vector space $\mathcal{L}\left( X,V \right)\otimes \mathcal{L}\left( Y,W \right)$. The next proposition will show an injection of the latter space into the former.
\end{remark}
\begin{proposition}[{\cite[Proposition E.6.17]{rainone_analysis}}]
  Let $X,Y,V,W$ be $\C$-vector spaces. There is a natural linear embedding
  \begin{align*}
    \iota\colon \mathcal{L}\left( X,V \right)\otimes \mathcal{L}\left( Y,W \right) \hookrightarrow \mathcal{L}\left( X\otimes Y,V\otimes W \right)
  \end{align*}
  such that $\iota\left( T\otimes S \right) = T\bar{\otimes}S$.
\end{proposition}
\begin{proof}
  We see that for any $T,T_1,T_2\in \mathcal{L}\left( X,V \right)$, $S,S_1,S_2\in \mathcal{L} \left( Y,W \right)$, and $\alpha \in \C$, that
  \begin{align*}
    \left( T_1 + \alpha T_2 \right)\bar{\otimes}S &= T_1\bar{\otimes}S + \alpha \left( T_2\bar{\otimes}S \right)\\
    T\bar{\otimes}\left( S_1 + \alpha S_2 \right) &= T\bar{\otimes}S_1 + \alpha \left( T\bar{\otimes}S_2 \right).
  \end{align*}
  Therefore, the map $\mathcal{L}\left( X,V \right)\times \mathcal{L}\left( Y,W \right) \rightarrow \mathcal{L}\left( X\otimes Y,V\otimes W \right)$, sending $\left( T,S \right)\mapsto T\bar{\otimes}S$. Thus, there is a map $\iota\colon \mathcal{L}\left( X,V \right)\otimes \mathcal{L}\left( Y,W \right)\rightarrow \mathcal{L}\left( X\otimes V,Y\otimes W \right)$ such that $\iota\left( T\otimes S \right) = T\bar{\otimes}S$.\newline

  Now, we will show that $\iota$ is injective. Suppose that
  \begin{align*}
    0 &= \iota\left( \sum_{k=1}^{n}T_k\otimes S_k \right)\\
      &= \sum_{k=1}^{n}T_k\bar{\otimes}S_k,
  \end{align*}
  where $T_k,S_k$ are linearly independent. Now, for any $x\in X$ and $y\in Y$, we have
  \begin{align*}
    0 &= \left( \sum_{k=1}^{n}T_k\bar{\otimes}S_k \right)\left( x\otimes y \right)\\
      &= \sum_{k=1}^{n}T_k\left( x \right)\otimes S_k\left( y \right).
  \end{align*}
  Furthermore, for any $\varphi\in V'$ and $\psi\in W'$, we have
  \begin{align*}
    0 &= \left( \varphi\times \psi \right)\left( \sum_{k=1}^{n}T_k\left( x \right)\otimes S_k\left( y \right) \right)\\
      &= \sum_{k=1}^{n}\varphi\left( T_k\left( x \right) \right)\psi\left( S_k\left( y \right) \right)\\
      &= \psi\left( \sum_{k=1}^{n}\varphi\left( T_k\left( x \right) \right)S_k\left( y \right) \right).
  \end{align*}
  Now, since $W'$ separates points, we have that
  \begin{align*}
    0 &= \sum_{k=1}^{n}\varphi\left( T_k\left( x \right) \right)S_k\left( y \right)\\
      &= \left( \sum_{k=1}^{n} T_k\left( x \right)S_k\right)\left( y \right),
  \end{align*}
  meaning $\displaystyle\sum_{k=1}^{n}\varphi\left( T_k\left( x \right) \right)S_k = 0$ in $\mathcal{L}\left( Y,W \right)$. Since we have defined $S_k$ to be linearly independent, we have $\varphi\left( T_k\left( x \right) \right) = 0$ for all $k$, $x\in X$, and $\varphi\in V'$. Since $V'$ separates points, we have $T_k(x) = 0$ for all $k$ and $x\in X$, so $T_k = 0$ for all $k$.\newline

  Thus, 
  \begin{align*}
    \sum_{k=1}^{n}T_k\otimes S_k &= 0,
  \end{align*}
  so $\ker\left( \iota \right) = \set{0}$, and $\iota$ is injective.
\end{proof}
Now that we understand how tensor products play with spaces of linear maps, we may prove some crucial results related to tensor products of algebras. Note that in all of these cases, we use the universal property of tensor products to ensure that our expression is unique.
\begin{proposition}[{\cite[Proposition F.2.24]{rainone_analysis}}]\label{prop:tensor_product_algebras}
  Let $A$ and $B$ be algebras. The vector space $A\otimes B$ admits a multiplication $\left( A\otimes B \right)\times \left( A\otimes B \right)\rightarrow A\otimes B$, given by
  \begin{align*}
    \left( a\otimes b \right)\left( c\otimes d \right) &= ac\otimes bd.
  \end{align*}
  If $A$ and $B$ are unital, then so too is $A\otimes B$. If $A$ and $B$ are $\ast$-algebras, then $A\otimes B$ admits an involution given by
  \begin{align*}
    \left( a\otimes b \right)^{\ast} &= a^{\ast}\otimes b^{\ast}.
  \end{align*}
\end{proposition}
\begin{proof}
  Fixing $a\in A$ and $b\in B$, we define linear maps $L_a\colon A\rightarrow A$ and $L_b\colon B\rightarrow B$  by
  \begin{align*}
    L_a\left( x \right) &= ax\\
    L_b\left( y \right) &= by.
  \end{align*}
  The maps $a\mapsto L_a$ and $b\mapsto L_b$ are linear by the fact that ring multiplication is left-distributive. Therefore, the map $A\times B \rightarrow \mathcal{L}\left( A \right)\otimes \mathcal{L}\left( B \right)$, given by $\left( a,b \right)\mapsto L_a\otimes L_b$ is bilinear.\newline

  By the universal property of tensor products, we have a linear map $L\colon A\otimes B \rightarrow \mathcal{L}\left( A \right)\otimes \mathcal{L}\left( B \right)$, given by $a\otimes b \mapsto L_a\otimes L_b$.\newline

  Now, by above, there is an embedding $\mathcal{L}\left( A \right)\otimes \mathcal{L}\left( B \right) \hookrightarrow \mathcal{L}\left( A\otimes B \right)$, so we may identify elements of $\mathcal{L}\left( A \right)\otimes \mathcal{L}\left( B \right)$ with linear maps on $A\otimes B$.\newline

  We define $\left( A\otimes B \right)\times \left( A\otimes B \right)\rightarrow A\otimes B$ by $\left( t,s \right)\mapsto ts \coloneq L(t)(s)$.\newline

  Since $L$ is linear, and $L(t)$ is linear for all $t\in A\otimes B$, both scalar multiplication and distributivity are preserved.\newline

  For all $a\in A$ and $b\in B$, we have
  \begin{align*}
    \left( a\otimes b \right)\left( c\otimes d \right) &= L\left( a\otimes b \right)\left( c\otimes d \right)\\
                                                       &= L_a\otimes L_b\left( c\otimes d \right)\\
                                                       &= \left( L_a\left( c \right) \right)\otimes \left( L_b\left( d \right) \right)\\
                                                       &= ac\otimes bd.
  \end{align*}
  Since multiplication in $A$ and $B$ is associative, multiplication in $A\otimes B$ is also associative.\newline

  Now, if $A$ and $B$ are unital, then $1_A\otimes 1_B$ is a unit for $A\otimes B$, as
  \begin{align*}
    \left( 1_A\otimes 1_B \right)\left( a\otimes b \right) &= \left( 1_A a \right)\otimes \left( 1_B b \right)\\
                                                           &= a\otimes b.
  \end{align*}
  Now, if $A$ and $B$ are $\ast$-algebras, we write $\overline{A\otimes B}$ to refer to the conjugate space. Regarding the conjugate space, we if $V$ is a vector space, then $\overline{V}$ is defined by
  \begin{align*}
    \overline{v} + \overline{w} &= \overline{v+w}\\
    \alpha\cdot \left( \overline{x} \right) &= \overline{\overline{\alpha}x}.
  \end{align*}
  Thus, if $\overline{A\otimes B}$ is the conjugate space, we can see from the definition of the involution that the map $A\times B \rightarrow \overline{A\otimes B}$, given by $\left( a,b \right)\mapsto \overline{a^{\ast}\otimes b^{\ast}}$ is a bilinear map.\newline

  There is a unique linear map $\psi\colon A\otimes B \rightarrow \overline{A\otimes B}$ such that $\psi\left( a\otimes b \right) = \overline{a^{\ast}\otimes b^{\ast}}$. Additionally, the map $\mu\colon \overline{A\otimes B}\rightarrow A\otimes B$, given by $\mu\left( \overline{t} \right) = t$ is conjugate linear.\newline

  Thus, the map $\nu\colon A\otimes B \rightarrow A\otimes B$, given by $a\otimes b \mapsto a^{\ast}\otimes b^{\ast}$ is conjugate linear, and we may define an involution $A\otimes B \rightarrow A\otimes B$ by $t\mapsto t^{\ast}\coloneq \nu(t)$. We see that
  \begin{align*}
    \left( \left( a\otimes b \right)\left( c\otimes d \right) \right)^{\ast} &= \left( ac\otimes bd \right)^{\ast}\\
                                                                             &= \left( ac \right)^{\ast}\otimes \left( bd \right)^{\ast}\\
                                                                             &= c^{\ast}a^{\ast} \otimes d^{\ast}b^{\ast}\\
                                                                             &= \left( c^{\ast}\otimes d^{\ast} \right)\left( a^{\ast}\otimes b^{\ast} \right)\\
                                                                             &= \left( c\otimes d \right)^{\ast}\left( a\otimes b \right)^{\ast}.
  \end{align*}
  A similar approach gives $t^{\ast\ast} = t$ for all $t\in A\otimes B$, meaning that this is a bona fide involution on $A\otimes B$, universal by definition.
\end{proof}
Now, we turn our attention towards matrix algebras. Recall that if $A$ is an algebra, then the matrix algebra $\Mat_n\left( A \right)$ is the set of $n\times n$ matrices $\left( a_{ij} \right)_{ij}$ such that $a_{ij}\in A$ for each $i,j$. This is also an algebra, but perhaps even more importantly, it is able to be expressed as a tensor product, which is often used in the definition of nuclearity for $C^{\ast}$-algebras. We will discuss more on this in Chapter \ref{ch:nuclearity}.
\begin{theorem}[{\cite[Example E.6.21, Example F.4.11]{rainone_analysis}}]
  Let $A$ be a $\ast$-algebra, and let $n\in \N$. Then, there is a $\ast$-isomorphism of $\ast$-algebras, $\varphi\colon \Mat_n\left( A \right)\rightarrow \Mat_n\left( \C \right)\otimes A$, given by
  \begin{align*}
    \varphi\left( \left( a_{ij} \right)_{ij} \right) &= \sum_{i,j=1}^{n}e_{ij}\otimes a_{ij},
  \end{align*}
  where $\set{e_{ij}}_{i,j=1}^{n}$ are the system of matrix units.
\end{theorem}
\begin{proof}
  We start by showing that $\varphi$ is an isomorphism of vector spaces. Note that by the definition of the tensor product, we know that $\varphi$ is a linear map. Now, we start by showing that $\varphi$ is injective. Suppose
  \begin{align*}
    \varphi\left( \left( a_{ij} \right)_{ij} \right) &= \sum_{i,j=1}^{n}e_{ij}a_{ij}\\
                                                     &= 0.
  \end{align*}
  Then, since the system of matrix units is linearly independent in $\Mat_n\left( \C \right)$, we have that $x_{ij} = 0$ for all $i,j$, so $\left( x_{ij} \right)_{ij} = 0$, so $\varphi$ is injective.\newline

  Now, we show that $\varphi$ is surjective. Let $t\in \Mat_n\left( \C \right)\otimes A$ be given by
  \begin{align*}
    t &= \sum_{k=1}^{m}m_k\otimes a_k,
  \end{align*}
  where $m_k\in \Mat_n\left( \C \right)$ and $a_k\in A$. Since every matrix over $\C$ is a linear combination of the matrix units, we write
  \begin{align*}
    m_k &= \sum_{i,j=1}^{n} m_k\left( i,j \right)e_{ij}.
  \end{align*}
  Substituting, we get
  \begin{align*}
    t &= \sum_{k=1}^{m}\left( \sum_{i,j=1}^{n}m_k\left( i,j \right)e_{ij} \right)\otimes a_k\\
      &= \sum_{i,j=1}^{n}e_{ij}\otimes \left( \sum_{k=1}^{m}m_k\left( i,j \right)a_k \right),
      \intertext{and defining $a_{ij}\coloneq \sum_{k=1}^{m}m_k\left( i,j \right)a_k$, we obtain}
      &= \sum_{i,j=1}^{n}e_{ij}\otimes a_{ij}.
  \end{align*}
  Thus, $\varphi\left( \left( a_{ij} \right)_{ij} \right) = t$, and $\varphi$ is surjective, hence a linear isomorphism.\newline

  Now, we show that $\varphi$ is multiplicative and preserves the involution. Let $\left( a_{ij} \right)_{ij},\left( b_{ij} \right)_{ij}\in \Mat_n\left( A \right)$. Then,
  \begin{align*}
    \varphi\left( \left( a_{ik} \right)_{ik} \right)\varphi\left( \left( b_{\ell j} \right)_{\ell j} \right) &= \left( \sum_{i,k=1}^{n}e_{ik}\otimes a_{ik} \right)\left( \sum_{\ell,j=1}^{n}e_{\ell j}\otimes b_{\ell j} \right)\\
                                                                               &= \sum_{i,j,k,\ell=1}^{n}\left( e_{ik}\otimes a_{ik} \right)\left( e_{\ell j}\otimes b_{\ell j} \right)\\
                                                                               &= \sum_{i,j,k,\ell=1}^{n}e_{ik}e_{\ell j}\otimes a_{ik}b_{\ell j}\\
                                                                               &= \sum_{i,j,k=1}^{n}e_{ik}e_{kj}\otimes a_{ik}b_{kj}\\
                                                                               &= \sum_{i,j,k=1}^{n}e_{ij}\otimes a_{ik}b_{kj}\\
                                                                               &= \sum_{i,j=1}^{n}e_{ij}\otimes \left( \sum_{k=1}^{n}a_{ik}b_{kj} \right)\\
                                                                               &= \varphi\left( \left( \sum_{k=1}^{n}a_{ik}b_{kj} \right)_{ij} \right)\\
                                                                               &= \varphi\left( \left( a_{ij} \right)_{ij}\left( b_{ij} \right)_{ij} \right),
  \end{align*}
  meaning $\varphi$ is multiplicative.\newline

  Finally, we have
  \begin{align*}
    \varphi\left( \left( a_{ij} \right)_{ij} \right)^{\ast} &= \left( \sum_{i,j=1}^{n}e_{ij}\otimes a_{ij} \right)^{\ast}\\
                                                            &= \sum_{i,j=1}^{n}\left( e_{ij}\otimes a_{ij} \right)^{\ast}\\
                                                            &= \sum_{i,j=1}^{n}e_{ij}^{\ast}\otimes a_{ij}^{\ast}\\
                                                            &= \sum_{i,j=1}^{n}e_{ji}\otimes a_{ij}^{\ast}\\
                                                            &= \sum_{i,j=1}^{n}e_{ij}\otimes a_{ji}^{\ast}\\
                                                            &= \varphi\left( \left( a_{ji}^{\ast} \right)_{ij} \right)\\
                                                            &= \varphi\left( \left( a_{ij} \right)_{ij}^{\ast} \right),
  \end{align*}
  meaning that $\varphi$ is a $\ast$-isomorphism.
\end{proof}
\subsection{Applying Norms to Tensor Products}%
When our vector spaces are equipped with a norm --- specifically, if they are Banach spaces --- not only does it matter that the tensor product preserves the vector space structure, but also that it preserves the norm structure in a particular manner. This is the domain of the injective and projective norms. The injective and projective norms will become more relevant when we discuss $C^{\ast}$-algebras, nuclearity, and amenability in Chapter \ref{ch:nuclearity}.\newline
 
Before discussing the injective and projective norms, however, we begin by elaborating on the connection between tensor products and linear maps. Recall that a linear map $T\colon X\rightarrow Y$ is called finite rank if $\Dim\left( \Ran\left( T \right) \right) < \infty$.
\begin{definition}
  Let $X$ and $Y$ be vector spaces. For each $\psi \in Y'$ and $x\in X$, we define the rank-one operator
  \begin{align*}
    L_{x,\psi}\left( y \right) &= \psi\left( y \right)x.
  \end{align*}
\end{definition}
\begin{fact}
  The map $T\colon X\times Y'\rightarrow \mathcal{L}\left( Y,X \right)$ that sends $\left( x,\psi \right)\mapsto L_{x,\psi}$ is bilinear.
\end{fact}
\begin{proof}
  We have, for a fixed $\psi\in Y'$ and for all $y\in Y$, $x_1,x_2\in X$, and $\alpha\in\C$,
  \begin{align*}
    T\left( x_1 + \alpha x_2,\psi \right)(y) &= \psi\left( y \right)\left( x_1 + \alpha x_2 \right)\\
                                             &= \psi\left( y \right)x_1 + \alpha \psi\left( y \right)x_2\\
                                             &= T\left( x_1,\psi \right)\left( y \right) + \alpha T\left( x_2,\psi \right)\left( y \right).
  \end{align*}
  Furthermore, for a fixed $x\in X$ and for all $y\in Y$, $\psi_1,\psi_2\in Y'$, and $\alpha\in \C$,
  \begin{align*}
    T\left( x,\psi_1 + \alpha \psi_2 \right)\left( y \right) &= \left( \psi_1 + \alpha \psi_2 \right)\left( y \right)x\\
                                                             &= \psi_1\left( y \right)x + \alpha \psi_2\left( y \right)x\\
                                                             &= T\left( x,\psi_1 \right)\left( y \right) + \alpha T\left( x,\psi_2 \right)\left( y \right).
  \end{align*}
\end{proof}
Note that since $T$ is bilinear, $T$ extends to a linear map $X\otimes Y' \rightarrow \mathcal{L}\left( Y,X \right)$. With a little bit more work, it can be shown that the space of finite-rank operators is linearly isomorphic to $X\otimes Y'$.\newline

Furthermore, for any $\varphi\in X'$ and $\psi\in Y'$, there is a linear functional $\varphi\times \psi\in \left( X\otimes Y \right)'$ such that $\left( \varphi\times \psi \right)\left( x\otimes y \right) = \varphi(x)\varphi(y)$.\newline

With these facts in mind, we desire an extension to the case when $X$ and $Y$ are normed spaces such that continuity and the preservation of operator norms of elements in $X^{\ast}$ and $Y^{\ast}$ are desirable. This is what the injective tensor product will resolve.
\begin{proposition}[{\cite[Proposition 3.5.1]{rainone_analysis}}]
  Let $X$ and $Y$ be normed vector spaces. Given some $t\in X\otimes Y$, we define
  \begin{align*}
    \norm{t}_{\vee} &= \sup_{\substack{\varphi\in B_{X^{\ast}}\\\psi\in B_{Y^{\ast}}}} \left\vert \left( \varphi\times \psi \right)\left( t \right) \right\vert
  \end{align*}
  as the \textit{injective norm} on $X\otimes Y$. The injective norm is cross, in that for all $x\in X$ and $y\in Y$,
  \begin{align*}
    \norm{x\otimes y} &= \norm{x}\norm{y}.
  \end{align*}
\end{proposition}
\begin{proof}
  We start by showing that the supremum is finite. Let
  \begin{align*}
    t &= \sum_{k=1}^{n} x_k\otimes y_k
  \end{align*}
  be any representation of $t\in X\otimes Y$, and let $\varphi\in B_{X^{\ast}}$ and $\psi\in B_{Y^{\ast}}$. Using Corollary \ref{cor:norm_from_functionals}, we have
  \begin{align*}
    \norm{\left( \varphi\times \psi \right)\left( t \right)} &= \left\vert \sum_{k=1}^{n}\varphi\left( x_k \right)\psi\left( y_k \right) \right\vert\\
                                                             &\leq \sum_{k=1}^{n} \left\vert \varphi\left( x_k \right) \right\vert\left\vert \psi\left( y_k \right) \right\vert\\
                                                             &\leq \sum_{k=1}^{n}\norm{x_k}\norm{y_k}.
  \end{align*}
  From the definition of the injective norm, we know that the triangle inequality and homogeneity hold, so we only need to focus on positive definiteness.\newline

  Let $\norm{t}_{\vee} = 0$. We express
  \begin{align*}
    t &= \sum_{k=1}^{n}x_k\otimes y_k,
  \end{align*}
  where we allow for $\set{y_1,\dots,y_n}$ to be linearly independent. For all $\varphi\in B_{X^{\ast}}$ and $\psi\in B_{Y^{\ast}}$, we have
  \begin{align*}
    0 &= \left( \varphi\times \psi \right)\left( t \right)\\
      &= \sum_{k=1}^{n}\varphi\left( x_k \right)\psi\left( y_k \right)\\
      &= \psi\left( \sum_{k=1}^{n}\varphi\left( x_k \right)y_k \right).
  \end{align*}
  Since this holds for all $\psi\in B_{Y^{\ast}}$, the Hahn--Banach separation (Theorem \ref{thm:hb_separation}) holds that
  \begin{align*}
    \sum_{k=1}^{n}\varphi\left( x_k \right)y_k &= 0.
  \end{align*}
  Since $\set{y_1,\dots,y_n}$ are linearly independent, then $\varphi\left( x_k \right) = 0$ for all $k$ and all $\varphi\in B_{X^{\ast}}$, so yet again by Hahn--Banach separation, we have $x_k = 0$ for all $k$, so $t = 0$.\newline

  To prove that the injective norm is cross, we know from Corollary \ref{cor:norm_from_functionals} that
  \begin{align*}
    \norm{z} &= \sup_{\varphi\in B_{Z^{\ast}}}\left\vert \varphi\left( z \right) \right\vert.
  \end{align*}
  Thus, we have
  \begin{align*}
    \norm{x\otimes y}_{\vee} &= \sup_{\substack{\varphi\in B_{X^{\ast}}\\\psi\in B_{Y^{\ast}}}}\left\vert \varphi\left( x \right) \psi\left( y \right)\right\vert\\
                             &= \sup_{\varphi\in B_{X^{\ast}}}\left\vert \varphi\left( x \right) \right\vert\sup_{\psi\in B_{Y^{\ast}}}\left\vert \psi\left( y \right) \right\vert\\
                             &= \norm{x}\norm{y}.
  \end{align*}
\end{proof}
Thus, we know that the injective norm is a valid cross norm. Using Proposition \ref{prop:completion_existence}, we may complete $X\otimes Y$ to yield a Banach space.
\begin{definition}
  If $X$ and $Y$ are normed spaces, the norm completion of $X\otimes Y$ with respect to the injective norm is called the \textit{injective tensor product} of $X$ and $Y$, denoted $X\check\otimes Y$.
\end{definition}
The injective tensor product allows us to realize the tensor product as a space of bounded linear maps, just as we are able to realize the algebraic tensor product as a space of finite-rank linear maps.
\begin{proposition}[{\cite[Proposition 3.5.5]{rainone_analysis}}]
  Let $X$ and $Y$ be normed spaces. There is an isometric embedding $X\check\otimes Y \hookrightarrow \B\left( Y^{\ast},X \right)$.
\end{proposition}
\begin{proof}
  We provide an outline of the proof rather than fill in the full details.\newline

  Start by defining the linear map $\theta_{x,y}\colon Y^{\ast}\rightarrow X$ by $\theta_{x,y}\left( \varphi \right) = \varphi\left( y \right)x$. From the Hahn--Banach theorems, we then know that $\norm{\theta_{x,y}}_{\op} = \norm{x}\norm{y}$.\newline

  After showing that the map $X\times Y \rightarrow \B\left( Y^{\ast},X \right)$ given by $\left( x,y \right)\mapsto \theta_{x,y}$ is bilinear, we use the universal property of the tensor product to find the map $u\colon X\otimes Y \rightarrow \B\left( Y^{\ast},X \right)$, where $x\otimes y \mapsto \theta_{x,y}$.\newline

  Then, it is shown via the definition of the injective tensor product and Corollary \ref{cor:norm_from_functionals} that $\norm{u(t)}_{\op} = \norm{t}$ for all $t\in X\otimes Y$.\newline

  Since $u$ is an isometry (hence uniformly continuous), we may extend it to the completion to yield an isometric embedding $U\colon X\check\otimes Y \hookrightarrow \B\left( Y^{\ast},X \right)$.
\end{proof}
\begin{remark}
  A similar process allows us to show that there is an isometric embedding $X\check\otimes Y \hookrightarrow \B\left( X^{\ast},Y \right)$.
\end{remark}
Contrasted with the injective norm's connection between the tensor product and the space of linear maps $\B\left( Y^{\ast},X \right)$, the projective norm draws upon the connection between linear maps and bilinear maps. First, we need to implement a norm on bilinear maps.
\begin{definition}[{\cite[Definition 3.5.8]{rainone_analysis}}]
  Let $b\colon X\times Y \rightarrow Z$ be a bilinear map on normed vector spaces $X,Y,Z$. Then, we say $b$ is \textit{bounded bilinear} if
  \begin{align*}
    \norm{b}_{\op} &\coloneq \sup_{\substack{x\in B_{X}\\y\in B_{Y}}}\norm{b\left( x,y \right)}
  \end{align*}
  is finite.
\end{definition}
Just as how linear maps between normed spaces are continuous if and only if they are bounded (Fact \ref{fact:continuity_of_linear_maps}), bilinear maps on normed spaces are continuous if and only if they are bounded.\newline

We begin by defining the projective norm and proving its properties before showing the connection between the projective tensor product and the space of bilinear maps.
\begin{proposition}[{\cite[Proposition 3.5.10]{rainone_analysis}}]
  Let $X$ and $Y$ be normed spaces. The \textit{projective norm} on $X\otimes Y$ is defined by
  \begin{align*}
    \norm{t}_{\wedge} &= \inf\set{\sum_{k=1}^{n}\norm{x_k}\norm{y_k} | t = \sum_{k=1}^{n}x_k\otimes y_k}.
  \end{align*}
  The projective norm is a cross norm that satisfies $\norm{t}_{\vee}\leq \norm{t}_{\wedge}$.
\end{proposition}
\begin{proof}
  The norm is homogeneous from the definition of the tensor product.\newline

  Now, if $t,t'\in X\otimes Y$ have representations of $t= \sum_{k=1}^{n}x_k\otimes y_k$ and $t' = \sum_{j=1}^{m}u_j\otimes v_j$, then
  \begin{align*}
    t + t' &= \sum_{k=1}^{n}x_k\otimes y_k + \sum_{j=1}^{m}u_j\otimes v_j,
    \intertext{giving}
    \norm{t + t'}_{\wedge} &\leq \sum_{k=1}^{n}\norm{x_k}\norm{y_k} + \sum_{j=1}^{m}\norm{u_j}\norm{v_j}.
  \end{align*}
  Taking the infimum over all representations of $t$, we have
  \begin{align*}
    \norm{t + t'}_{\wedge} &\leq \norm{t} + \sum_{j=1}^{m}\norm{u_j}\norm{v_j},
  \end{align*}
  and taking the infimum over all representations of $t'$, we get
  \begin{align*}
    \norm{t + t'} &\leq \norm{t} + \norm{t'}.
  \end{align*}
  We show that the projective norm satisfies $\norm{t}_{\vee}\leq \norm{t}_{\wedge}$. Letting $t = \sum_{k=1}^{n}x_k\otimes y_k$, we use Corollary \ref{cor:norm_from_functionals} to obtain
  \begin{align*}
    \norm{t}_{\vee} &= \sup_{\substack{\varphi\in B_{X^{\ast}}\\\psi\in B_{Y^{\ast}}}} \left\vert \sum_{k=1}^{n}\varphi\left( x_k \right)\psi\left( y_k \right) \right\vert\\
                    &\leq \sum_{k=1}^{n}\left\vert \varphi\left( x_k \right) \right\vert\left\vert \psi\left( y_k \right) \right\vert\\
                    &\leq \sum_{k=1}^{n}\norm{x_k}\norm{y_k}.
  \end{align*}
  Taking the infimum over all representations, we then obtain $\norm{t}_{\vee}\leq \norm{t}_{\wedge}$.\newline

  Thus, if $\norm{t}_{\wedge} = 0$, then $\norm{t}_{\vee} = 0$, so $t = 0$ as $\norm{\cdot}_{\vee}$ is a norm.\newline

  Finally, for $x\in X$ and $y\in Y$, we have
  \begin{align*}
    \norm{x}\norm{y} &= \norm{x\otimes y}_{\vee}\\
                     &\leq \norm{x\otimes y}_{\wedge}\\
                     &\leq \norm{x}\norm{y}.
  \end{align*}
\end{proof}
The norm completion of $X\otimes Y$ with respect to the projective norm is known as the \textit{projective tensor product}, and is denoted $X\hat\otimes Y$.\newline

Now, we may draw the connection between the projective tensor product and the space of bounded bilinear maps.
\begin{proposition}[{\cite[Proposition 3.5.13]{rainone_analysis}}]
  Let $X,Y,Z$ be normed spaces.\newline

  If $b\in \operatorname{Bil}\left( X,Y;Z \right)$ is bounded bilinear, then there is a unique $T_{b}\in \B\left( X\hat\otimes Y,Z \right)$ such that $T_b\left( x\otimes y \right)= b\left( x,y \right)$, and that $\norm{T_{b}}_{\op} = \norm{b}_{\op}$.\newline

  Furthermore, the map $b\mapsto T_b$ is a linear isometric isomorphism.
\end{proposition}
\begin{proof}
  Since $b$ is bilinear, there is a unique linear map $T_b\colon X\otimes Y \rightarrow Z$ by the universal property, where $T_b\left( x\otimes y \right) = b\left( x,y \right)$. If we let $t = \sum_{k=1}^{n}x_k\otimes y_k$, then
  \begin{align*}
    \norm{T_b(t)} &= \norm{\sum_{k=1}^{n}b\left( x_k,y_k \right)}\\
                  &\leq \sum_{k=1}^{n}\norm{b\left( x_k,y_k \right)}\\
                  &\leq \sum_{k=1}^{n}\norm{b}_{\op}\norm{x_k}\norm{y_k}.
  \end{align*}
  Taking the infimum of both sides, we get
  \begin{align*}
    \norm{T_b\left( t \right)} &\leq \norm{b}_{\op}\norm{t}_{\wedge},
  \end{align*}
  so
  \begin{align*}
    \norm{T_b}_{\op} &\leq \norm{b}_{\op}.
  \end{align*}
  Similarly, for $x\in B_X$ and $y\in B_Y$, we have $\norm{x\otimes y}_{\wedge}\leq 1$, meaning
  \begin{align*}
    \norm{b\left( x,y \right)} &= \norm{T_b\left( x\otimes y \right)}\\
                               &\leq \norm{T_b}_{\op},
  \end{align*}
  so by taking suprema, we get $\norm{b}_{\op} = \norm{T_b}_{\op}$. Since $X\hat\otimes Y$ is the completion of $X\otimes Y$ with the projective norm, there is a norm-preserving continuous extension to $X\hat\otimes Y$.\newline

  Now, if $T\colon X\otimes Y\rightarrow Z$ is any map, then we may define $b\colon X\times Y \rightarrow Z$ by $b\left( x,y \right) = T\left( x\otimes y \right)$. We have that $b$ is bounded, as
  \begin{align*}
    \norm{b\left( x,y \right)} &= \norm{T\left( x\otimes y \right)}\\
                               &\leq \norm{T}_{\op}\norm{x\otimes y}_{\wedge}\\
                               &= \norm{T}_{\op}\norm{x}\norm{y}.
  \end{align*}
  Now, since $T_b\left( x\otimes y \right) = b\left( x,y \right) = T\left( x\otimes y \right)$, $T$ agrees with $T_b$ on $X\otimes Y$, and both $T$ and $T_b$ are bounded with respect to the projective norm, they must agree on $X\hat\otimes Y$.\newline

  The map $b\mapsto T_b$ is automatically linear by definition, so since $\norm{b}_{\op} = \norm{T_b}_{\op}$, the map $b\mapsto T_b$ is an isometric isomorphism.
\end{proof}
We may also draw connections between the tensor product and spaces of linear maps.
\begin{proposition}[{\cite[Proposition 3.5.14]{rainone_analysis}}]
  Let $X$ and $Y$ be normed spaces. There is an isometric isomorphism such that $\left( X\hat\otimes Y \right)^{\ast} \cong B\left( X,Y^{\ast} \right)$.
\end{proposition}
\begin{proof}
  We provide an outline of the proof of this proposition.\newline

  First, define the map $T\colon \left( X\hat\otimes Y \right)^{\ast}\rightarrow \B\left( X,Y^{\ast} \right)$ by $\psi\mapsto T_{\psi}$, where $T_{\psi}\left( x \right)\left( y \right) = \psi\left( x\otimes y \right)$. Then, $T_{\psi}$ is a linear operator on $Y$ such that $\norm{T_{\psi}(x)}_{\op}\leq \norm{\psi}\norm{x}$, meaning $T_{\psi}\in Y^{\ast}$. Furthermore, we also establish that $T_{\psi}$ is a contraction.\newline

  In the other direction, we establish that a map $\eta\colon \B\left( X,Y^{\ast} \right)\rightarrow \left( X\hat\otimes Y \right)^{\ast}$ is also a bounded linear map that is a contraction with $\eta\left( T_{\psi} \right) = \psi$, meaning $T$ is an isometric isomorphism.
\end{proof}
\section{Remarks}%
We will use all these established foundations in later chapters, both implicitly and explicitly, to understand different properties related to groups, amenability, and results in functional analysis and operator algebras that will relate to groups and amenability.\newline

To provide a preview, we use properties of the free group in Chapter \ref{ch:paradoxical_decompositions} to establish an understanding of when a group is \textit{not} amenable. This helps establish that any group that \textsl{is} amenable does not contain a freely generated subgroup.\newline

Meanwhile, the results on free algebras, the group $\ast$-algebra, and tensor products will be used in Chapter \ref{ch:nuclearity} to understand amenability of groups via their $C^{\ast}$-algebras. We will endow the universal $\ast$-algebra(s) with certain norms that, when completed, will yield $C^{\ast}$-algebras, and we will discuss the connection between nuclearity and the tensor product in Chapter \ref{ch:nuclearity}, though we will not directly prove the theorem in \cite{choi_nuclearity} that displays the equivalence between these two definitions of nuclearity.

% Here, will run through free groups, free vector spaces, tensor products, free algebras, and the group *-algebra. We will discuss the free group in chapter 2, and the rest in chapter 8 when elaborating on nuclearity.
\chapter{How to Feed 5,000 Hungry Mathematicians: Paradoxical Decompositions}\label{ch:paradoxical_decompositions}
In the Bible, one of the miracles of Jesus is the feeding of the five thousand,\footnote{Fun fact: the feeding of the five thousand is the only other miracle of Jesus (aside from the resurrection) that is in all four gospels.} where, despite only having five loaves of bread and two fishes, a large crowd splits these morsels among themselves and eats to satisfaction after Jesus calls upon the power of God to enable them to do so. Of course, we may not be able to fully replicate this without some divine intervention --- but, mathematically, thanks to the power of the axiom of choice, we can show that something like the feeding of the five thousand is not only possible, but a fundamental feature of the isometry group of $\R^3$. This is exemplified in the most general form of the Banach--Tarski paradox.
\begin{restatable}[Strong Banach--Tarski Paradox]{proposition}{banachtarski}\label{prop:banachtarski}
  Let $A$ and $B$ be bounded subsets of $\R^3$ with nonempty interior. There is a partition of $A$ into finitely many disjoint subsets such that a sequence of isometries applied to these subsets yields $B$.
\end{restatable}
The Banach--Tarski paradox throws a wrench into a common belief that we have about $\R^3$ --- specifically, that every subset of $\R^3$ has a \textit{finitely additive} ``volume'' that is invariant under rigid motion.\footnote{Note that if we desire countable additivity, the axiom of choice shows that there does not exist a countably additive measure on $P\left(\R\right)$ that is also translation-invariant (see \cite[Section 1.1]{folland_real_analysis}). Finite additivity is a weaker condition than countable additivity that allows for the existence of well-behaved measures on $P\left(\R\right)$ and $P\left(\R^2\right)$, but even this fails in $\R^3$ and above.} This property does exist for $\R$ and $\R^2$, as their isometry groups have a property known as amenability --- in Section \ref{sec:invariant_states_remarks}, we will provide an outline for why this is true.\newline

To develop paradoxical decompositions, we will begin with the Ping Pong Lemma, which will allow us to find freely generated subgroups. We will apply this to the case of $\text{SO}(3)$ to find a freely generated subgroup. Then, we will use the fact that free groups on more than one generator have a property known as paradoxicality --- this property will provide the germ of the proof of the Banach--Tarski paradox.
\section{The Ping Pong Lemma}\label{sec:ping_pong_lemma}%
To move towards paradoxical decompositions, we need to find a simple and easily applicable criterion for knowing when an arbitrary group contains a freely generated subgroup. This is the domain of the Ping Pong Lemma, which we will prove in this section to show the existence of a freely generated subgroup of $\text{SO}(3)$. Later, this freely generated subgroup will be indispensable in proving the Banach--Tarski paradox.\newline

We begin by defining a free product of a family of groups $\set{\Gamma_i}_{i\in I}$. This will allow us to state the Ping Pong Lemma in its maximal generality.
\begin{definition}[Free Product]\label{def:free_product}
  Let $A$ be a set, and set $W(A)$ to be the set of words in $A$ equipped with the operation of concatenation. This turns $W(A)$ into a construction known as the \textit{free monoid}.\newline

  If $\set{\Gamma_i}_{i\in I}$ is a family of groups, and $A = \coprod_{i\in I}\Gamma_i$ is the coproduct (or disjoint union) of the groups $\Gamma_i$, then we define the equivalence relation $\sim$ generated by
  \begin{align*}
    we_iw' &\sim ww'\text{ where $e_i$ is the neutral element of $\Gamma_i$ for some $i\in I$}\\
    wabw' &\sim wcw'\text{ where $a,b,c\in \Gamma_i$ and $c=ab$ for some $i\in I$}.
  \end{align*}
  Then, the quotient $W(A)/\sim$ is known as the \textit{free product} of the groups $\set{\Gamma_i}_{i\in I}$, and is denoted
  \begin{align*}
    \bigstar_{i\in I}\Gamma_i.
  \end{align*}
\end{definition}
\begin{remark}
  The free group $F(S)$ is an instance of the free product where the $\Gamma_i$ are the formal cyclic groups generated by each $s\in S$.\newline

  From the way we have defined the free product, it can be shown, as in \cite[II.A.]{delaHarpe_topics_in_geometric_group_theory}, that every element of the free product is represented a unique reduced word on $W(A)$, along with the following universal property: if $\set{\Gamma_i}_{i\in I}$ is a family of groups, and $h_i\colon \Gamma_i\rightarrow\Gamma$ for some fixed group $\Gamma$, then there is a unique homomorphism $h\colon \bigstar_{i\in I}\Gamma_i \rightarrow \Gamma$ such that the following diagram commutes for each $\Gamma_{i_0}$.
  \begin{center}
        % https://tikzcd.yichuanshen.de/#N4Igdg9gJgpgziAXAbVABwnAlgFyxMJZABgBpiBdUkANwEMAbAVxiRAB12BxOgW17oB9YFkHEAviHGl0mXPkIoyARiq1GLNpwBGWAOZwcdAE7CsnLGAAEASXGce-IVikyQGbHgJFl5NfWZWRA5uPgFBF3E1GCg9eCJQADNjCF4kMhAcCCRfdUCtdnwjMzFJagY6bRgGAAU5L0UQY30ACxwQaiMsBjYWiAgAa1cklLTEDKykACZqAM1glpKJYZBk1JzO7MQZkAqq2vqFNma9No68+ZAWqQpxIA
    \begin{tikzcd}
      \Gamma_{i_0} \arrow[d, "\iota_{i_0}"', hook] \arrow[r, "h_{i_0}"] & \Gamma_i \\
      \bigstar_{i\in I}\Gamma_i \arrow[ru, "h"']                        &         
    \end{tikzcd}
  \end{center}
\end{remark}

\begin{theorem}[Ping Pong Lemma]
  Let $G$ be a group that acts on a set $X$, and let $\Gamma_1,\Gamma_2$ be subgroups of $G$. Let $\Gamma = \left\langle \Gamma_1,\Gamma_2 \right\rangle$. Assume $\Gamma_1$ contains at least $3$ elements, and $\Gamma_2$ contains at least $2$ elements.\newline

  Suppose there exist nonempty subsets $X_1,X_2\subseteq X$ with $X_1\triangle X_2 \neq \emptyset$ such that for all $\gamma\in \Gamma_1$ with $\gamma \neq e_{G}$,
  \begin{align*}
    \gamma\left(X_2\right)\subseteq X_1,
  \end{align*}
  and for all $\gamma \in \Gamma_2$ with $\gamma \neq e_G$,
  \begin{align*}
    \gamma\left(X_1\right)\subseteq X_2.
  \end{align*}
  Then, $\Gamma$ is isomorphic to the free product $\Gamma_1\star \Gamma_2$.\label{thm:ping_pong}
\end{theorem}
\begin{proof}
  Let $w$ be a nonempty reduced word with letters in the disjoint union of $\Gamma_1\setminus \set{e_G}$ and $\Gamma_2\setminus \set{e_G}$. We must show that the element of $\Gamma$ defined by $w$ is not the identity.\newline

  If $w = a_1b_1a_2b_2\cdots a_k$ with $a_1,\dots,a_k\in \Gamma_1\setminus \set{e_G}$ and $b_1,\dots,b_{k-1}\in \Gamma_{2}\setminus \set{e_G}$, then,
  \begin{align*}
    w\left(X_2\right) &= a_1b_1\cdots a_{k-1}b_{k-1}a_k\left(X_2\right)\\
                      &\subseteq a_1b_1\cdots a_{k-1}b_{k-1}\left(X_1\right)\\
                      &\subseteq a_1b_1\cdots a_{k-1}\left(X_2\right)\\
                      &\vdots\\
                      &\subseteq a_1\left(X_2\right)\\
                      &\subseteq X_1.
  \end{align*}
  Seeing as $X_2\nsubseteq X_1$ (by the definition of symmetric difference), it is the case that $w\neq e_{G}$.\newline

  If $w = b_1a_2b_2a_2\cdots b_k$, we select $a\in \Gamma_1\setminus \set{e_G}$, and we find that $awa^{-1}\neq e_G$, meaning $w\neq e_G$. Similarly, if $w = a_1b_1\cdots a_kb_k$, we select $a\in \Gamma_1\setminus \set{e_G,a_{1}^{-1}}$, similarly finding that $awa^{-1}\neq e_{G}$. If $w = b_1a_2b_2\cdots a_k$, then we select $a\in \Gamma_1\setminus \set{1,a_k}$, and find $awa^{-1}\neq e_G$.
\end{proof}

We can refine Theorem \ref{thm:ping_pong} to the case of ``doubles'' wherein we find a different (yet more readily applicable) sufficient condition for a group that contains a copy of the free group on two generators.
\begin{corollary}[Ping Pong Lemma for ``Doubles'']
  Let $G$ act on $X$, and let $A_{+}, A_{-},B_{+},B_{-}$ be disjoint subsets of $X$ whose union is not equal to $X$. Then, if
  \begin{align*}
    a\cdot \left(X\setminus A_{-}\right) &\subseteq A_{+}\\
    a^{-1}\cdot \left(X\setminus A_{+}\right) &\subseteq A_{-}\\
    b\cdot \left(X\setminus B_{-}\right) &\subseteq B_{+}\\
    b^{-1}\cdot \left(X\setminus B_{+}\right) &\subseteq B_{-},
  \end{align*}
  then it is the case that $\left\langle a,b \right\rangle$ is isomorphic to the free group on two generators.\label{corollary:ping_pong_doubles}
\end{corollary}
\begin{proof}
  We let $A = A_{+}\sqcup A_{-}$, $B = B_{+}\sqcup B_{-}$, $\Gamma_1 = \left\langle a \right\rangle$, and $\Gamma_2 = \left\langle b \right\rangle$. Then, $A,B,\Gamma_1,\Gamma_2$ satisfy the conditions for Theorem \ref{thm:ping_pong}.
\end{proof}
\begin{remark}
Instead of typing out ``the free group on two generators,'' we will henceforth use $F(a,b)$ to refer to the free group on two generators.
\end{remark}

We can apply Theorem \ref{thm:ping_pong} to show the existence of a set of isometries of $\R^n$ that is isomorphic to $F(a,b)$.
\begin{definition}[Special Orthogonal Group]
  For $n\in \N$, we define $\text{SO}(n)$ to be the group of all real $n\times n$ matrices $A$ such that $A^{T} = A^{-1}$ and $\det(A) = 1$.
\end{definition}
In terms of an isometry of $\R^3$, the group $\text{SO}(3)$ denotes the set of all rotations about any line through the origin.
\begin{theorem}\label{thm:free_group_so3}
  There are elements $a,b\in \text{SO}(3)$ such that $\left\langle a,b \right\rangle_{\text{SO}(3)} \cong F(a,b)$.
\end{theorem}
\begin{proof}
  We let
  \begin{align*}
    a &= \begin{pmatrix}3/5 & 4/5 & 0 \\ -4/5 & 3/5 & 0 \\ 0 & 0 & 1\end{pmatrix}\\
    a^{-1} &= \begin{pmatrix}3/5 & -4/5 & 0 \\ 4/5 & 3/5 & 0 \\ 0 & 0 & 1\end{pmatrix}\\
    b &= \begin{pmatrix}1 & 0 & 0 \\ 0 & 3/5 & -4/5 \\ 0 & 4/5 & 3/5\end{pmatrix}\\
    b^{-1} &= \begin{pmatrix}1 & 0 & 0 \\ 0 & 3/5 & 4/5 \\ 0 & -4/5 & 3/5\end{pmatrix}.
  \end{align*}
  We specify
  \begin{align*}
    X &= A_{+} \sqcup A_{-} \sqcup B_{+} \sqcup B_{-} \sqcup \begin{pmatrix}0\\1\\0\end{pmatrix},
  \end{align*}
  where
  \begin{align*}
    A_{+} &= \set{\frac{1}{5^{k}} \begin{pmatrix}x\\y\\z\end{pmatrix} | k\in \Z, x \equiv 3y\text{ modulo $5$}, z\equiv0\text{ modulo $5$}}\\
    A_{-} &= \set{\frac{1}{5^{k}} \begin{pmatrix}x\\y\\z\end{pmatrix} | k\in \Z, x \equiv -3y\text{ modulo $5$}, z\equiv 0\text{ modulo $5$}}\\
    B_{+} &= \set{\frac{1}{5^{k}} \begin{pmatrix}x\\y\\z\end{pmatrix} | k\in \Z, z \equiv 3y\text{ modulo $5$}, x\equiv 0\text{ modulo $5$}}\\
    B_{-} &= \set{\frac{1}{5^{k}} \begin{pmatrix}x\\y\\z\end{pmatrix} | k\in \Z, z \equiv -3y\text{ modulo $5$}, x\equiv 0\text{ modulo $5$}}.
  \end{align*}
  To verify that the conditions of Theorem \ref{thm:ping_pong} hold, we calculate
  \begin{align*}
    \begin{pmatrix}3/5 & 4/5 & 0 \\ -4/5 & 3/5 & 0 \\ 0 & 0 & 1\end{pmatrix}\left(\frac{1}{5^k} \begin{pmatrix}x\\y\\z\end{pmatrix}\right) &= \frac{1}{5^{k+1}} \begin{pmatrix}3x + 4y \\ -4x + 3y \\ 5z\end{pmatrix}\tag*{(1)}\\
    \begin{pmatrix}3/5 & -4/5 & 0 \\ 4/5 & 3/5 & 0 \\ 0 & 0 & 1\end{pmatrix} \left(\frac{1}{5^k} \begin{pmatrix}x\\y\\z\end{pmatrix}\right) &= \frac{1}{5^{k+1}} \begin{pmatrix}3x - 4y \\ 4x + 3y \\ 5z\end{pmatrix}\tag*{(2)}\\
    \begin{pmatrix}1 & 0 & 0 \\ 0 & 3/5 & -4/5 \\ 0 & 4/5 & 3/5\end{pmatrix}\left(\frac{1}{5^{k}} \begin{pmatrix}x\\y\\z\end{pmatrix}\right) &= \frac{1}{5^{k+1}} \begin{pmatrix}5x \\ 3y- 4z \\ 4y + 3z\end{pmatrix}\tag*{(3)}\\
    \begin{pmatrix}1 & 0 & 0 \\ 0 & 3/5 & 4/5 \\ 0 & -4/5 & 3/5\end{pmatrix} \left(\frac{1}{5^{k}} \begin{pmatrix}x\\y\\z\end{pmatrix}\right) &= \frac{1}{5^{k+1}} \begin{pmatrix}5x \\ 3y + 4z \\ -4y + 3z\end{pmatrix}.\tag*{(4)}
  \end{align*}
  We verify that the conditions for Corollary \ref{corollary:ping_pong_doubles} hold for each of these four conditions.
  \begin{enumerate}[(1)]
    \item For any vector
      \begin{align*}
        \frac{1}{5^{k}} \begin{pmatrix}x\\y\\z\end{pmatrix} \notin A_{-},
      \end{align*}
      we see that $k+1\in \Z$, $x' = 3x + 4y \equiv 3\left(-4x + 3y\right)$  modulo $5$, and that $z' = 5z\equiv 0$ modulo $5$.
    \item For any vector
      \begin{align*}
        \frac{1}{5^{k}} \begin{pmatrix}x\\y\\z\end{pmatrix} \notin A_{+},
      \end{align*}
      we see that $k+1\in \Z$, $x' = 3x - 4y\equiv -3\left(4x + 3y\right)$ modulo $5$, and $z' = 5z \equiv 0$ modulo $5$.
    \item For any vector
      \begin{align*}
        \frac{1}{5^{k}} \begin{pmatrix}x\\y\\z\end{pmatrix}\notin B_{-},
      \end{align*}
      we see that $k+1\in \Z$, $z' = 4y + 3z \equiv 3\left(3y-4z\right)$ modulo $5$, and $x' = 5x\equiv 0$ modulo $5$.
    \item For any vector
      \begin{align*}
        \frac{1}{5^{k}} \begin{pmatrix}x\\y\\z\end{pmatrix}\notin B_{+},
      \end{align*}
      we see that $k+1\in \Z$, $z' = -4y + 3z \equiv -3\left(3y + 4z\right)$ modulo $5$, and $x' = 5x \equiv 0$ modulo $5$.
  \end{enumerate}
  Thus, by Theorem \ref{thm:ping_pong} and Corollary \ref{corollary:ping_pong_doubles}, it is the case that $\left\langle a,b \right\rangle\cong F(a,b)$.
\end{proof}

\section{Introducing Paradoxical Decompositions}\label{sec:intro_paradoxical_decompositions}%
We now turn our attention towards ``paradoxical'' actions that seem to recreate a set by using disjoint proper subsets. This will allow us to use the result from Theorem \ref{thm:free_group_so3} to move towards the Banach--Tarski paradox.
\begin{definition}[Paradoxical Decompositions and Paradoxical Groups]
  Let $G$ be a group that acts on a set $X$, with $E\subseteq X$. We say $E$ is $G$\textit{-paradoxical} if there exist pairwise disjoint proper subsets $A_1,\dots,A_n$ and $B_1,\dots,B_m$ of $E$ and group elements $g_1,\dots,g_n,h_1,\dots,h_m\in G$ such that
  \begin{align*}
    E &= \bigcup_{j=1}^{n}g_j\cdot A_j
  \end{align*}
  and
  \begin{align*}
    E &= \bigcup_{j=1}^{m}h_j\cdot B_j.
  \end{align*}
  If $G$ acts on itself by left-multiplication, and $G$ satisfies these conditions, we say $G$ is a \textit{paradoxical group}.
\end{definition}
\begin{example}
  The free group on two generators, $F(a,b)$, is a paradoxical group.\newline
  %The free group is defined to be the set of all reduced words over the set $\set{a,b,a^{-1},b^{-1},e_{F(a,b)}}$, where $aa^{-1}$, $a^{-1}a$, $bb^{-1}$, and $b^{-1}b$ are replaced with the identity $e_{F(a,b)}$.\newline

  To see that $F(a,b)$ is a paradoxical group, we let $W(x)$ denote the set of words in $F(a,b)$ that start with $x\in \set{a,b,a^{-1},b^{-1}}$. For instance, $ba^2ba^{-1}\in W(b)$.\newline

  Since every word in $F$ is either the empty word, or starts with one of $a,b,a^{-1},b^{-1}$, we see that
  \begin{align*}
    F(a,b) &= \set{e_{F(a,b)}} \sqcup W(a) \sqcup W(b) \sqcup W\left(a^{-1}\right) \sqcup W\left(b^{-1}\right).
  \end{align*}
  If $w\in F(a,b)\setminus W(a)$, we see that $a^{-1}w\in W\left(a^{-1}\right)$. Thus, $w\in aW\left(a^{-1}\right)$. For any $t\in F(a,b)$ either $t\in W(a)$ or $t\in F(a,b)\setminus W(a) = aW\left(a^{-1}\right)$. Thus, $F\left(a,b\right) $ is equal to $ W(a)\sqcup aW\left(a^{-1}\right)$.\newline

  Similarly, if $t\in F(a,b)$ either $t\in W(b)$ or $t\in F\left(a,b\right)\setminus W(b) = bW\left(b^{-1}\right)$, so $F\left(a,b\right) $ is equal to $ W(b)\sqcup bW\left(b^{-1}\right)$.\newline

  We have thus constructed
  \begin{align*}
    F(a,b) &= W(a)\sqcup aW\left(a^{-1}\right)\\
           &= W(b)\sqcup bW\left(b^{-1}\right),
  \end{align*}
  a paradoxical decomposition of $F(a,b)$ with the action of left-multiplication.
\end{example}
Now that we understand a little more about paradoxical groups, we now want to understand the actions of paradoxical groups on sets.
\begin{proposition}
  Let $G$ be a paradoxical group that acts freely on $X$. Then, $X$ is $G$-paradoxical.
\end{proposition}
\begin{proof}
  Let $A_1,\dots,A_n,B_1,\dots,B_m\subset G$ be pairwise disjoint, and let $g_1,\dots,g_n,h_1,\dots,h_m\in G$ such that
  \begin{align*}
    G &= \bigcup_{j=1}^{n}g_jA_j\\
      &= \bigcup_{j=1}^{m}h_jB_j.
  \end{align*}
  Let $M\subseteq X$ contain exactly one element from every orbit in $X$.
  \begin{claim}
  The set $\set{g\cdot M\mid g\in G}$ is a partition of $X$.
  \end{claim}
  \begin{proof}[Proof of Claim:]
  Since $M$ contains exactly one element from every orbit in $X$, it is the case that $G\cdot M = X$, so
  \begin{align*}
    \bigcup_{g\in G} g\cdot M &= X
  \end{align*}
  Additionally, for $x,y\in M$, if $g\cdot x = h\cdot y$, then $\left(h^{-1}g\right)\cdot x = y$, meaning $y$ is in the orbit of $x$ and vice versa, implying $x = y$. Since $G$ acts freely on $X$, we must have $h^{-1}g = e_G$.\newline

  Thus, we can see that $g_1\cdot M \neq g_2\cdot M$, implying $\set{g\cdot M\mid g\in G}$ is a partition of $X$.
  \end{proof}

  We define
  \begin{align*}
    A_j^{\ast} &= \bigcup_{g\in A_j}g\cdot M,
  \end{align*}
  and similarly define
  \begin{align*}
    B_j^{\ast} &= \bigcup_{h\in B_j}h\cdot M.
  \end{align*}
  As a useful shorthand, we can also write $A_j^{\ast} = A_j\cdot M$, and similarly, $B_j^{\ast} = B_j\cdot M$, to denote the union of the elements of $A_j$ and $B_j$ respectively acting on $M$.\newline

  Since $\set{g\cdot M\mid g\in G}$ is a partition of $X$, and $A_1,\dots,A_n,B_1,\dots,B_m\subset G$ are pairwise disjoint, it must be the case that $A_1^{\ast},\dots,A_n^{\ast},B_1^{\ast},\dots,B_m^{\ast}\subset X$ are also pairwise disjoint.\newline

  For the original $g_1,\dots,g_n,h_1,\dots,h_m$ that defined the paradoxical decomposition of $G$, we thus have
  \begin{align*}
    \bigcup_{j=1}^{n}g_j\cdot A_j^{\ast} &= \bigcup_{j=1}^{n}\left(g_jA_j\right)\cdot M\\
                                         &= G\cdot M\\
                                         &= X,
  \end{align*}
  and
  \begin{align*}
    \bigcup_{j=1}^{m}h_j\cdot B_j^{\ast} &= \bigcup_{j=1}^{m}\left(h_jB_j\right)\cdot M\\
                                         &= G\cdot M\\
                                         &= X.
  \end{align*}
  Thus, $X$ is $G$-paradoxical.
\end{proof}
\begin{remark}
  This proof requires the axiom of choice, as we invoked it to define $M$ to contain exactly one element from every orbit in $X$.
\end{remark}
\section{The Weak Banach--Tarski Paradox}\label{sec:weak_banach_tarski}%
Now that we have established $F(a,b)$ as being a paradoxical group, we wish to use it to construct paradoxical decompositions of the unit sphere $S^2\subseteq \R^3$. Specifically, we will show a weak version of the Banach--Tarski paradox --- one where you can break apart the unit ball into finitely many pieces and reconstitute it into two copies of itself.
\begin{fact}
  If $H$ is a paradoxical group, and $H\leq G$, then $G$ is a paradoxical group.
\end{fact}
With this fact in mind, we will show that $\text{SO}(3)$ is a paradoxical group.
\begin{theorem}
  There are rotations $A$ and $B$ that about lines through the origin in $\R^3$ that generate a subgroup of $\text{SO}(3)$ isomorphic to $F(a,b)$
\end{theorem}
\begin{proof}
  We take $A$ and $B$ as in the proof of Theorem \ref{thm:free_group_so3}.
%  We take
%  \begin{align*}
%    A &= \begin{bmatrix}1/3 & -\frac{2\sqrt{2}}{3} & 0\\ \frac{2\sqrt{2}}{3} & 1/3 & 0 \\ 0 & 0 & 1\end{bmatrix}\\
%    A^{-1} &= \begin{bmatrix}1/3 & \frac{2\sqrt{2}}{3} & 0\\ -\frac{2\sqrt{2}}{3} & 1/3 & 0 \\ 0 & 0 & 1\end{bmatrix}\\
%    B &= \begin{bmatrix}1 & 0 & 0 \\ 0 & 1/3 & -\frac{2\sqrt{2}}{3} \\ 0 & \frac{2\sqrt{2}}{3} & 1/3\end{bmatrix}\\
%    B^{-1} &= \begin{bmatrix}1 & 0 & 0 \\ 0 & 1/3 & \frac{2\sqrt{2}}{3} \\ 0 & -\frac{2\sqrt{2}}{3} & 1/3\end{bmatrix}
%  \end{align*}
%  We let $A^{\pm}$ denote $A$ and $A^{-1}$ respectively, and similarly for $B^{\pm}$.\newline
%
%  Let $w$ be a reduced word in $\set{A,A^{-1},B,B^{-1}}$ which is not the empty word. We claim that $w$ cannot be the identity.\newline
%
%  Without loss of generality, we assume that $w$ ends in $A$ or $A^{-1}$ --- this is because if $w$ is the identity, then $AwA^{-1}$ and $A^{-1}wA$ are also the identity.\newline
%
%  We will show that there exist $a,b,c\in \Z$ with $b\not\equiv 0$ mod $3$ such that
%  \begin{align*}
%    w \cdot \begin{pmatrix}1\\0\\0\end{pmatrix} &= \frac{1}{3^k} \begin{pmatrix}a\\b\sqrt{2}\\c\end{pmatrix}.
%  \end{align*}
%  If $b\not\equiv 0$ mod $3$, and $w$ is not empty, then $w$ cannot act as the identity.\newline
%
%  We induct on the length of $w$. For $w = A^{\pm}$, we have
%  \begin{align*}
%    w\cdot \begin{pmatrix}1\\0\\0\end{pmatrix} &= \frac{1}{3}\begin{pmatrix}1\\\pm2\sqrt{2}\\0\end{pmatrix},
%  \end{align*}
%  proving the base case.\newline
%
%  Let $k > 0$, meaning $w = A^{\pm}w'$, or $w = B^{\pm}w'$, with $w'$ not equal to the empty. The inductive hypothesis says
%  \begin{align*}
%    w'\cdot \begin{pmatrix}1\\0\\0\end{pmatrix} &= \frac{1}{3^{k-1}} \begin{pmatrix}a'\\b'\sqrt{2}\\c'\end{pmatrix}
%  \end{align*}
%  for some $a',b',c'\in \Z$, and $b'\not\equiv 0$ mod $3$. In particular,
%  \begin{align*}
%    A^{\pm}w' \cdot \begin{pmatrix}1\\0\\0\end{pmatrix} &= \frac{1}{3^k} \begin{pmatrix}a\mp 4b \\ \left(b'\pm 2a'\right)\sqrt{2} \\ 3c'\end{pmatrix}\\
%    B^{\pm}w' \cdot \begin{pmatrix}1\\0\\0\end{pmatrix} &= \frac{1}{3^k} \begin{pmatrix}3a' \\ \left(b'\mp 2c'\right)\sqrt{2} \\ c'\pm 4b'\end{pmatrix}.
%  \end{align*}
%  Now, we set
%  \begin{align*}
%    w \cdot \begin{pmatrix}1\\0\\0\end{pmatrix} &= \frac{1}{3^k} \begin{pmatrix}a\\b\sqrt{2}\\c\end{pmatrix},
%  \end{align*}
%  meaning
%  \begin{align*}
%    a &= \begin{cases}
%      a'\mp 4b', & w = A^{\pm} w'\\
%      3a', & w = B^{\pm}w'
%    \end{cases}\\
%      b &= \begin{cases}
%        b'\pm 2a', & w = A^{\pm}w'\\
%        b'\mp 2c', & w = B^{\pm}w'
%      \end{cases}\\
%        c &= \begin{cases}
%          3c', & w = A^{\pm}w'\\
%          c' \pm 4b', & w = B^{\pm}w'
%        \end{cases}
%  \end{align*}
%  Let $w^{\ast}$ denote the word such that $w' = A^{\pm}w^{\ast}$ or $w' = B^{\pm}w^{\ast}$. We write
%  \begin{align*}
%    w^{\ast} &= \frac{1}{3^{k-2}} \begin{pmatrix}a''\\b''\sqrt{2}\\c''\end{pmatrix},
%  \end{align*}
%  where $a'',b'',c''\in \Z$. Note that it may not be the case that $w^{\ast}$ is a non-empty word. We examine the following four cases.
%  \begin{description}
%    \item[Case 1:] Suppose $w = A^{\pm}B^{\pm}w^{\ast}$. Then, $b = b'\mp 2a'$, where $a' = 3a''$. Since $b'\not\equiv 0$ mod $3$ (by the inductive hypothesis), it is also the case $b\equiv 0$ mod $3$.
%    \item[Case 2:] Suppose $w = B^{\pm}A^{\pm}w^{\ast}$. Then, $b = b'\mp 2c'$, where $c' = 3c''$. Since $b'\not\equiv 0$ mod $3$ (by the inductive hypothesis), it is also the case that $b\not\equiv 0$ mod $3$.
%    \item[Case 3:] Suppose $w = A^{\pm}A^{\pm}w^{\ast}$. Then, we have
%      \begin{align*}
%        b &= b' \pm 2a'\\
%          &= b' \pm 2\left(a'' \pm 4b''\right)\\
%          &= b'+ \left(b'' \pm 2a''\right) - 9b''\\
%          &= 2b' - 9b''.
%      \end{align*}
%      Thus, regardless of the value of $b''$, since $b'\not\equiv 0$ mod $3$ by the inductive hypothesis, it is the case that $b\not\equiv 0$ mod $3$.
%    \item Suppose $w = B^{\pm}B^{\pm}w^{\ast}$. Then, we have
%      \begin{align*}
%        b &= b' \mp 2c'\\
%          &= b' \mp 2\left(c'' \pm 4b''\right)\\
%          &= b' + \left(b'' \mp 2c''\right) - 9b''\\
%          &= 2b' - 9b''.
%      \end{align*}
%      Thus, regardless of the value of $b''$, since $b'\not\equiv 0$ mod $3$ by the inductive hypothesis, it is the case that $b\not\equiv 0$ mod $3$.
%  \end{description}
%  We have thus shown that any non-empty reduced word over $\set{A,A^{-1},B,B^{-1}}$ does not act as the identity. The subgroup of $\text{SO}(3)$ generated by $\set{A,A^{-1},B,B^{-1}}$ is isomorphic to $F(a,b)$.
\end{proof}
\begin{remark}
  Since $\text{SO}(n)$ contains a subgroup isomorphic to $\text{SO}(3)$ for all $n\geq 3$ (via the block matrices), it is the case that $\text{SO}(n)$ also contains a subgroup isomorphic to $F(a,b)$ for all $n\geq 3$.
\end{remark}
Since we have shown that $\text{SO}(3)$ is paradoxical, as it contains a paradoxical subgroup, we can now begin to examine the action of $\text{SO}(3)$ on subsets of $\R^3$.
\begin{theorem}[Hausdorff Paradox]
  There is a countable subset $D$ of $S^{2}$ such that $S^{2}\setminus D$ is $\text{SO}(3)$-paradoxical.
\end{theorem}
\begin{proof}
  Let $A$ and $B$ be the rotations in $\text{SO}(3)$ that serve as the generators of the subgroup isomorphic to $F(a,b)$ (as in \ref{thm:free_group_so3}).\newline

  Since $A$ and $B$ are rotations, so too is any element of $\left\langle A,B \right\rangle$. Thus, any such non-empty word contains two fixed points.\newline

  We let
  \begin{align*}
    F &= \set{x\in S^{2}\mid x\text{ is a fixed point for some word }w}.
  \end{align*}
  Since $\left\langle A,B \right\rangle$ is countably infinite, so too is $F$. Thus, the union of all these fixed points under the action of all such words $w$ is countable.
  \begin{align*}
    D &= \bigcup_{w\in \left\langle A,B \right\rangle} w\cdot F.
  \end{align*}
  Therefore, $\left\langle A,B \right\rangle$ acts freely on $S^{2}\setminus D$, so $S^{2}\setminus D$ is $\text{SO}(3)$-paradoxical.
\end{proof}

Unfortunately, the Hausdorff paradox is not enough for us to be able to prove the Banach--Tarski paradox. In order to do this, we need to be able to show that two sets are ``similar'' under the action of a group.
\begin{definition}[Equidecomposable Sets]
  Let $G$ act on $X$, and let $A,B\subseteq X$. We say $A$ and $B$ are $G$-equidecomposable if there are partitions $\set{A_j}_{j=1}^{n}$ of $A$ and $\set{B_j}_{j=1}^{n}$ of $B$, and elements $g_1,\dots,g_n\in G$, such that for all $j$,
  \begin{align*}
    B_j &= g_j\cdot A_j.
  \end{align*}
  We write $A\sim_{G}B$ if $A$ and $B$ are $G$-equidecomposable.
\end{definition}
\begin{fact}\label{fact:equidecomposability_equivalence_relation}
  The relation $\sim_{G}$ is an equivalence relation.
\end{fact}
\begin{proof}
  Let $A$, $B$, and $C$ be sets.\newline

  To show reflexivity, we can select $g_1 = g_2 = \cdots = g_n = e_G$. Thus, $A\sim_{G}A$.\newline

  To show symmetry, let $A\sim_{G} B$. Set $\set{A_j}_{j=1}^{n}$ to be the partition of $A$, and set $\set{B_j}_{j=1}^{n}$ to be the partition of $B$, such that there exist $g_1,\dots,g_n\in G$ with $g_j\cdot A_j = B_j$. Then,
  \begin{align*}
    g_j^{-1}\cdot \left(g_j\cdot A_j\right) &= g_j^{-1}\cdot B_j\\
    A_j &= g_j^{-1}\cdot B_j,
  \end{align*}
  so $B_j\sim_{G}A_j$.\newline

  To show transitivity, let $A\sim_{G} B$ and $B\sim_{G} C$. Let $\set{A_i}_{i=1}^{n}$ and $\set{B_i}_{i=1}^{n}$ be the partitions of $A$ and $B$ respectively and $g_1,\dots,g_n\in G$ such that $g_i\cdot A_i = B_i$. Let $\set{B_j}_{j=1}^{m}$ and $\set{C_j}_{j=1}^{m}$ be partitions of $B$ and $C$, and $h_1,\dots,h_m\in G$, such that $h_j\cdot B_j = C_j$.\newline

  We refine the partition of $A$ to $A_{ij}$ by taking $A_{ij} = g_i^{-1}\left(B_{i}\cap B_j\right)$, where $i = 1,\dots,n$ and $j = 1,\dots,m$. Then, $\left(h_jg_i\right)\cdot A_{ij}$ maps the refined partition of $A$ to $C$, so $A$ and $C$ are $G$-equidecomposable.
\end{proof}
\begin{fact}
  For $A\sim_{G} B$, there is a bijection $\phi\colon A\rightarrow B$ by taking $C_{i} = C\cap A_i$, and mapping $\phi\left(C_i\right) = g_i\cdot C_i$.\newline

  In particular, this means that for any subset $C\subseteq A$, it is the case that $C\sim \phi(C)$.\label{fact:bijections}
\end{fact}

We can now use this equidecomposability to glean information about the existence of paradoxical decompositions.
\begin{proposition}
  Let $G$ act on $X$, with $E,E'\subseteq X$ such that $E\sim_{G}E'$. Then, if $E$ is $G$-paradoxical, then so too is $E'$.
\end{proposition}

\begin{proof}
Let $A_1,\dots,A_n,B_1,\dots,B_m\subset E$ be pairwise disjoint, with $g_1,\dots,g_n,h_1,\dots,h_m\in G$ such that
\begin{align*}
  E &= \bigcup_{i=1}^{n}g_i\cdot A_i\\
    &= \bigcup_{j=1}^{m}h_j\cdot B_j.
\end{align*}
We let
\begin{align*}
  A &= \bigsqcup_{i=1}^{n}A_i\\
  B &= \bigsqcup_{j=1}^{m}B_j.
\end{align*}
It follows that $A\sim_{G}E$ and $B\sim_{G}E$, since we can take the partition of $A$ to be $A_1,\dots,A_n$, and partition $E$ by taking $g_i\cdot A_i$ for $i=1,\dots,n$, and similarly for $B$.\newline

Since $E\sim_{G}E'$, and $\sim_{G}$ is an equivalence relation, it follows that $A\sim_{G}E'$ and $B\sim_{G}E'$. Thus, there is a paradoxical decomposition of $E'$ in $A_1,\dots,A_n$ and $B_1,\dots,B_m$.
\end{proof}

We will now show that $S^{2}$ is $\text{SO}(3)$ paradoxical.
\begin{proposition}
  Let $D\subseteq S^{2}$ be countable. Then, $S^{2}$ and $S^{2}\setminus D$ are $\text{SO}(3)$-equidecomposable.
\end{proposition}

\begin{proof}
  Let $L$ be a line in $\R^3$ such that $L\cap D = \emptyset$. Such an $L$ must exist since $S^{2}$ is uncountable.\newline

  Define $\rho_{\theta}\in \text{SO}(3)$ to be a rotation about $L$ by an angle of $\theta$. For a fixed $n\in \N$ and fixed $\theta\in [0,2\pi)$, define $R_{n,\theta} = \set{x\in D\mid \rho^{n}_{\theta}\cdot x \in D}$. Since $D$ is countable, $R_{n,\theta}$ is necessarily countable.\newline

  We define $W_n = \set{\theta\mid R_{n,\theta}\neq \emptyset}$. Since the map $\theta \mapsto \rho_{\theta}^{n}\cdot x$ into $D$ is injective, it is the case that $W_n$ is countable. Therefore,
  \begin{align*}
    W &= \bigcup_{n\in \N}W_n
  \end{align*}
  is countable.\newline

  Thus, there must exist $\omega \in [0,2\pi)\setminus W$. We define $\rho_{\omega}$ to be a rotation about $L$ by $\omega$. Then, for every $n,m\in \N$, we have
  \begin{align*}
    \rho_{\omega}^n\cdot D \cap \rho_{\omega}^{m}\cdot D &= \emptyset.
  \end{align*}
  We define $\widetilde{D} = \bigsqcup_{n=0}^{\infty}\rho^{n}_{\omega}D$. Note that 
  \begin{align*}
    \rho_{\omega}\cdot \widetilde{D} &= \rho_{\omega}\cdot\bigsqcup_{n=0}^{\infty}\rho_{\omega}^{n}\cdot D\\
                                     &= \bigsqcup_{n=1}^{\infty}\rho_{\omega}^{n}\cdot D\\
                                     &= \widetilde{D} \setminus D,
  \end{align*}
  meaning $\widetilde{D}$ and $D$ are $\text{SO}(3)$-equidecomposable.\newline

  Thus, we have
  \begin{align*}
    S^{2} &= \widetilde{D}\sqcup \left(S^{2}\setminus \widetilde{D}\right)\\
          &\sim_{\text{SO}(3)}\left(\rho_{\omega}\cdot \widetilde{D}\right)\sqcup \left(S^{2}\setminus\widetilde{D}\right)\\
          &= \left(\widetilde{D}\setminus D\right)\sqcup \left(S^{2}\setminus\widetilde{D}\right)\\
          &= S^{2}\setminus D,
  \end{align*}
  establishing $S^{2}$ and $S^{2}\setminus D$ as $\text{SO}(3)$-equidecomposable.\newline

  In particular, this means $S^{2}$ is also $\text{SO}(3)$-paradoxical.
\end{proof}
To prove the Banach--Tarski paradox, we need a slightly larger group than $\text{SO}(3)$ --- one that includes translations in addition to the traditional rotations.
\begin{definition}[Euclidean Group]
  The {Euclidean group}, $\text{E}(n)$, consists of all isometries of a Euclidean space. An isometry of a Euclidean space consists of translations, rotations, and reflections.
\end{definition}
\begin{corollary}[Weak Banach--Tarski Paradox]
  Every closed ball in $\R^3$ is $\text{E}(3)$-paradoxical.
\end{corollary}
\begin{proof}
  We only need to show that $B(0,1)$ is $\text{E}(3)$-paradoxical. To do this, we start by showing that $B(0,1)\setminus \set{0}$ is $\text{SO}(3)$-paradoxical.\newline

  Since $S^{2}$ is $\text{SO}(3)$-paradoxical, there exists pairwise disjoint subsets $A_1,\dots,A_n,B_1,\dots,B_m\subset S^2$ and elements $g_1,\dots,g_n,h_1,\dots,h_m\in \text{SO}(3)$ such that
  \begin{align*}
    S^{2} &= \bigcup_{i=1}^{n}g_i\cdot A_i\\
          &= \bigcup_{j=1}^{m}h_j\cdot B_j.
  \end{align*}
  Define
  \begin{align*}
    A_i^{\ast} &= \set{tx\mid t\in (0,1], x\in A_i}\\
    B_j^{\ast} &= \set{ty\mid t\in (0,1], y\in B_j}.
  \end{align*}
  Then, $A_1^{\ast},\dots,A_n^{\ast},B_1^{\ast},\dots,B_m^{\ast}\subset B(0,1)\setminus \set{0}$ are pairwise disjoint, and
  \begin{align*}
    B(0,1)\setminus \set{0} &= \bigcup_{i=1}^{n}g_i\cdot A_i^{\ast}\\
                            &= \bigcup_{j=1}^{m}h_j\cdot B_j^{\ast}.
  \end{align*}
  Thus, we have established that $B(0,1)\setminus \set{0}$ is $\text{E}(3)$-paradoxical.\newline

  Now, we want to show that $B(0,1)\setminus \set{0}$ and $B(0,1)$ are $\text{E}(3)$-equidecomposable. Let $x\in B(0,1)\setminus \set{0}$, and let $\rho$ be a rotation through $x$ by a line not through the origin such that $\rho^{n}\cdot 0\neq \rho^{m}\cdot 0$ when $n\neq m$.\newline

  Let $D = \set{\rho^{n}\cdot 0\mid n\in \N}$. We can see that $\rho\cdot D = D\setminus \set{0}$, and that $D$ and $\rho\cdot D$ are $\text{E}(3)$-equidecomposable. Thus,
  \begin{align*}
    B(0,1) &= D\sqcup \left(B(0,1)\setminus D\right)\\
           &\sim_{\text{E}(3)}\left(\rho\cdot D\right) \sqcup \left(B(0,1)\setminus D\right)\\
           &= \left(D\setminus \set{0}\right)\sqcup \left(B\left(0,1\right)\setminus D\right)\\
           &= B\left(0,1\right)\setminus \set{0}.
  \end{align*}
  Therefore, $B(0,1)$ is $\text{E}(3)$-paradoxical.
\end{proof}
\section{The Strong Banach--Tarski Paradox}\label{sec:full_banach_tarski}%
In order to prove the general case of the Banach--Tarski paradox, we need one more piece of mathematical machinery.\newline

In Fact \ref{fact:equidecomposability_equivalence_relation}, we showed that the relation $A\sim_{G} B$ if and only if $A$ and $B$ are $G$-equidecomposable is an equivalence relation. Using the power of subsets, we may extend this to a preorder on any subsets $A$ and $B$ of $X$.
\begin{definition}
  Let $G$ act on a set $X$ with $A,B\subseteq X$. We write $A\preceq_{G}B$ if $A$ is equidecomposable with a subset of $B$.
\end{definition}
\begin{fact}
  The relation $\preceq_{G}$ is a reflexive and transitive relation.\label{fact:preorder}
\end{fact}
\begin{proof}
  To see reflexivity, we can see that since $A\sim_{G}A$, and $A\subseteq A$, $A\preceq_{G} A$.\newline

  To see transitivity, let $A\preceq_{G}B$ and $B\preceq_{G}C$. Then, there exist $g_1,\dots,g_n\in G$ such that $g_i\cdot A_i = B_{\alpha,i}$ for each $i$, where $A\sim_{G}B_{\alpha}\subseteq B$. Similarly, there exist $h_1,\dots,h_m\in G$ such that $h_j\cdot B_j= C_{\beta,j}$ for each $j$, where $B\sim_{G}C_{\beta}\subseteq C$.\newline

  We take a refinement of $B$ by taking intersections $B_{\alpha,ij} = B_{\alpha,i}\cap B_j$, with $i=1,\dots,n$ and $j = 1,\dots,m$. We define $C_{\beta,\alpha,ij} = h_j\cdot B_{\alpha,ij}$ for each $j = 1,\dots,m$. Then, $h_jg_i\cdot A_i = C_{\beta,\alpha,ij}$, meaning $A\sim_{G}C_{\beta,\alpha,ij}\subseteq C_{\beta}\subseteq C$, so $A\preceq_{G}C$.
\end{proof}

We know from Fact \ref{fact:bijections} that $A\preceq_{G}B$ implies the existence of a bijection $\phi\colon A\rightarrow B'\subseteq B$, meaning $\phi\colon A\hookrightarrow B$ is an injection. Similarly, if $B\preceq_{G}A$, then Fact \ref{fact:bijections} implies the existence of an injection $\psi\colon B\hookrightarrow A$.\newline

One may ask if an analogue of the Cantor--Schröder--Bernstein theorem exists in the case of the relation $\preceq_{G}$, implying that the preorder established in Fact \ref{fact:preorder} is indeed a partial order. The following theorem establishes this result.
\begin{theorem}
  Let $G$ act on $X$, and let $A,B\subseteq X$. If $A\preceq_{G}B$ and $B\preceq_{G}A$, then $A\sim_{G}B$.\label{thm:csb_for_equidecomposability}
\end{theorem}
\begin{proof}
  Let $B'\subseteq B$ with $A\sim_{G}B'$, and let $A'\subseteq A$ with $B\sim_{G}A'$. Then, we know from Fact \ref{fact:bijections} that there exist bijections $\phi\colon A\rightarrow B'$ and $\psi\colon B\rightarrow A'$.\newline

  Define $C_0 = A\setminus A'$, and $C_{n+1} = \psi\left(\phi\left(C_n\right)\right)$. We set
  \begin{align*}
    C &= \bigcup_{n\geq 0}C_{n}.
  \end{align*}
  Since $\psi^{-1}\left(\psi\left(\phi\left(C_n\right)\right)\right) = \phi\left(C_n\right)$, we have
  \begin{align*}
    \psi^{-1}\left(A\setminus C\right) &= B\setminus \phi(C).
  \end{align*}
  Having established in Fact \ref{fact:bijections} that for any subset of $C\subseteq A$, $C\sim_{G} \phi(C)$, we also see that $A\setminus C \sim_{G} B\setminus \phi(C)$.\newline

  Thus, we can see that
  \begin{align*}
    A &= \left(A\setminus C\right)\sqcup C\\
      &\sim_{G}\left(B\setminus \phi(C)\right)\sqcup \phi(C)\\
      &= B.
  \end{align*}
\end{proof}

Finally, we are able to prove Proposition \ref{prop:banachtarski}. We restate the proposition here, followed by its proof.
\begin{tcolorbox}[blanker,breakable,left=3mm,before skip=10pt, after skip=10pt, borderline west={1pt}{0pt}{blue!50!white},sharp corners,]
\banachtarski*
\end{tcolorbox}
\begin{proof}[Proof of Proposition \ref{prop:banachtarski}:]
  By symmetry, it is enough to show that $A\preceq_{\text{E}(3)} B$.\newline

  Since $A$ is bounded, there exists $r > 0$ such that $A\subseteq B(0,r)$.\newline

  Let $x_0\in B^{\circ}$. Then, there exists $\ve > 0$ such that $B\left(x_0,\ve\right) \subseteq B$.\newline

  Since $B(0,r)$ is compact (hence totally bounded), there are translations $g_1,\dots,g_n$ such that
  \begin{align*}
    B\left(0,r\right) \subseteq g_1\cdot B\left(x_0,\ve\right) \cup \cdots \cup g_n\cdot B\left(x_0,\ve\right).
  \end{align*}
  We select translations $h_1,\dots,h_n$ such that $h_j\cdot B\left(x_0,\ve\right) \cap h_k\cdot B\left(x_0,\ve\right) = \emptyset$ for $j\neq k$. We set
  \begin{align*}
    S &= \bigcup_{j=1}^{n}h_j\cdot B\left(x_0,\ve\right).
  \end{align*}
  Each $h_j\cdot B\left(x_0,\ve\right)\subseteq S$ is $\text{E}(3)$-equidecomposable with any arbitrary closed ball subset of $B\left(x_0,\ve\right)$, it is the case that $S\preceq B\left(x_0,\ve\right)$.\newline

  Thus, we have
  \begin{align*}
    A &\subseteq B\left(0,r\right)\\
      &\subseteq g_1\cdot B\left(x_0,\ve\right)\cup\cdots\cup b_n\cdot B\left(x_0,\ve\right)\\
      &\preceq S\\
      &\preceq B\left(x_0,\ve\right)\\
      &\preceq B.
  \end{align*}
\end{proof}


%\part{Fugue}
\chapter{Well-Behaved Groups of a Feather Flock Together: Tarski's Theorem}\label{ch:tarskis_theorem}
Ultimately, the reason the Banach--Tarski paradox ``works'' is because the paradoxical group $F(a,b)$ is not amenable --- specifically, its paradoxicality closes off the possibility of amenability. Before we go further into the characterizations of amenability discussed in Chapters \ref{ch:invariant_states} and \ref{ch:folner_condition}, we will show that this statement reverses. Indeed, every amenable group is \textit{non}-paradoxical.
\begin{restatable}[Tarski's Theorem, {\cite[Theorem 0.2.1]{lectures_on_amenability}}]{theorem}{tarski}
  Let $G$ be a group that acts on a set $X$, and let $E \subseteq X$ be nonempty.\newline

  There is a finitely additive measure $\mu \colon P(X) \to [0, \infty]$ with $\mu(E) \in (0, \infty)$ and $\mu\left( t\cdot E \right) = \mu(E)$ for all $t\in G$ if and only if $E$ is not $G$-paradoxical.
\label{thm:tarski}
\end{restatable}
We can prove one of the directions of Tarski's theorem now.
\begin{proof}[Proof of the Forward Direction of Theorem \ref{thm:tarski}:]
  Let $E$ be $G$-paradoxical. Suppose toward contradiction that such a translation-invariant finitely additive $\nu$ existed with $\nu(E) \in (0,\infty)$.\newline

  Let $A_1,\dots,A_n,B_1,\dots,B_m\subseteq E$ be pairwise disjoint, and let $t_1,\dots,t_n,s_1,\dots,f_m\in G$ such that
  \begin{align*}
    E &= \bigsqcup_{i=1}^{n}t_i\cdot A_i\\
      &= \bigsqcup_{j=1}^{m}s_j\cdot B_j.
  \end{align*}
  Then, it would be the case that
  \begin{align*}
    \nu(E) &= \nu\left(\bigsqcup_{i=1}^{n}t_i\cdot A_i\right)\\
           &= \sum_{i=1}^{n}\nu\left(t_i\cdot A_i\right)\\
           &= \sum_{i=1}^{n}\nu\left(A_i\right),
  \end{align*}
  and
  \begin{align*}
    \nu(E) &= \sum_{j=1}^{m}\nu\left(B_j\right).
  \end{align*}
  However, this also yields
  \begin{align*}
    \nu\left(E\right) &= \nu\left(\left(\bigsqcup_{i=1}^{n}A_i\right)\sqcup \left(\bigsqcup_{j=1}^{m}B_j\right)\right)\\
                      &= \sum_{i=1}^{n}\nu\left(A_i\right) + \sum_{j=1}^{m}\nu\left(B_j\right)\\
                      &= \sum_{i=1}^{n}\nu\left(t_i\cdot A_i\right) + \sum_{j=1}^{m}\nu\left(x_j\cdot B_j\right)\\
                      &= \nu\left(E\right) + \nu\left(E\right)\\
                      &= 2\nu\left(E\right).
  \end{align*}
  implying that $\nu(E) = 0$ or $\nu(E) = \infty$.
\end{proof}
The opposite direction, unfortunately, will be significantly harder to prove. We will need to know some results from graph theory, understand the properties of the type semigroup of an action, and use some results on commutative semigroups to show the existence of a mean.
\section{A Little Bit of Graph Theory}
To prove the reverse direction of Tarski's theorem, we need to develop some machinery from graph theory that will allow us to prove that a certain semigroup we will construct in the next section satisfies the cancellation identity.\newline

We start by defining graphs and paths, before proving a special case of Hall's theorem, ultimately extending to the infinite case with König's theorem.
\begin{definition}[Graphs and Paths, {\cite[7]{lectures_on_amenability}}]
  A \textit{graph} is a triple $\left(V,E,\phi\right)$, with $V,E$ nonempty sets and $\phi\colon E\rightarrow P_{2}(V)$ a map from $E$ to the set of all unordered subset pairs of $V$.\newline

  For $e\in E$, if $\phi(e) = \set{v,w}$, then we say $v$ and $w$ are the \textit{endpoints} of $e$, and $e$ is \textit{incident} on $v$ and $w$.\newline

  A \textit{path} in $\left(V,E,\phi\right)$ is a finite sequence $\left(e_1,\dots,e_n\right)$ of edges, with a finite sequence of vertices $\left(v_0,\dots,v_n\right)$, such that $\phi\left(e_k\right) = \set{v_{k-1},v_k}$.\newline

  The \textit{degree} of a vertex, $\deg(v)$, is the number of edges incident on $v$.\newline

  We define the \textit{neighbors} of $S\subseteq V$ to be the set of all vertices $v\in V\setminus S$ such that $v$ is an endpoint to an edge incident on $S$. We denote this set $N(S)$.
\end{definition}

\begin{definition}[Bipartite Graphs and $k$-Regularity, {\cite[Definition 0.2.2]{lectures_on_amenability}}]
  Let $\left(V,E,\phi\right)$ be a graph, with $k\in \N$.
  \begin{enumerate}[(i)]
    \item If $\deg(v) = k$ for each $v\in V$, we say $\left(V,E,\phi\right)$ is \textit{$k$-regular}.
    \item If $V = X\sqcup Y$, with each edge in $E$ having one endpoint in $X$ and one endpoint in $Y$, then we say $V$ is \textit{bipartite}, and write $\left(X,Y,E,\phi\right)$.
  \end{enumerate}
\end{definition}

\begin{definition}[Perfect Matching, {\cite[Definition 0.2.3]{lectures_on_amenability}}]
  Let $\left(X,Y,E,\phi\right)$ be a bipartite graph. Let $A\subseteq X$ and $B\subseteq Y$. A \textit{perfect matching} of $A$ and $B$ is a subset $F\subseteq E$ with
  \begin{enumerate}[(i)]
    \item each element of $A\cup B$ is an endpoint of exactly one $f\in F$;
    \item all endpoints of edges in $F$ are in $A\cup B$.
  \end{enumerate}
\end{definition}
\begin{definition}[Hall Condition, {\cite[Exercise 0.2.2]{lectures_on_amenability}}]
  We say a bipartite graph $\left(X,Y,E,\phi\right)$ satisfies the \textit{Hall condition} on $X$ if, for all $S\subseteq X$, $\left\vert N(S) \right\vert \geq \left\vert S \right\vert$.\newline

  Equivalently, we say a (finite) collection of not necessarily distinct finite sets $\mathcal{X} = \set{X_i}_{i=1}^{n}$ satisfies the Hall condition if and only if for all subcollections $\mathcal{Y}_k = \set{X_{i_k}}_{k=1}^{m}$,
  \begin{align*}
    \left\vert \mathcal{Y}_k \right\vert \leq \left\vert \bigcup_{k=1}^{m}X_{i_k} \right\vert.
  \end{align*}
\end{definition}
\begin{remark}
These two formulations of the Hall condition are equivalent regarding an $X$-perfect matching.
\end{remark}
\begin{theorem}[Hall's Theorem for Finite $k$-Regular Bipartite Graphs, {\cite[Exercise 0.2.2]{lectures_on_amenability}}]
  Let $\left(X,Y,E,\phi\right)$ be a $k$-regular bipartite graph for some $k\in \N$, and let $V = X\sqcup E$ be finite. Then, there is a perfect matching of $X$ and $Y$.\label{thm:hall_finite}
\end{theorem}
\begin{proof}
  Note that since $\left\vert E \right\vert = k\left\vert K \right\vert = k\left\vert Y \right\vert$, it is the case that $\left\vert X \right\vert = \left\vert Y \right\vert$.\newline

  Let $M\subseteq V$ be any subset. We will show that $\left\vert N(M) \right\vert\geq \left\vert M \right\vert$ --- that is, $\left(X,Y,E,\phi\right)$ satisfies the Hall condition.\newline

  Let $M_X = M\cap X$ and $M_Y = M\cap Y$, where $M = M_X\sqcup M_Y$. Let $\left[M_X,N\left(M_X\right)\right]$ be the set of edges with endpoints in $M_X$ and $N\left(M_X\right)$, and $\left[M_Y,N\left(M_Y\right)\right]$ be the set of edges with endpoints in $M_Y$ and $N\left(M_Y\right)$. We also let $\left[X,N\left(M_X\right)\right]$ denote the set of edges with endpoints in $X$ and $N\left(M_X\right)$, and similarly, $\left[Y,N\left(M_Y\right)\right]$ is the set of edges with endpoints in $Y$ and $N\left(M_Y\right)$.\newline

  We can see that $\left[M_X,N\left(M_X\right)\right]\subseteq \left[X,N\left(M_X\right)\right]$, and similarly, $\left[M_Y,N\left(M_Y\right)\right]\subseteq \left[Y,N\left(M_Y\right)\right]$.\newline

  Since $\left\vert \left[M_X,N\left(M_X\right)\right] \right\vert = k\left\vert M_X \right\vert$ and $\left\vert \left[X,N\left(M_X\right)\right] \right\vert = k\left\vert N\left(M_X\right) \right\vert$, we have
  \begin{align*}
    \left\vert M_X \right\vert\leq \left\vert N\left(M_X\right) \right\vert,
  \end{align*}
  and similarly,
  \begin{align*}
    \left\vert M_Y \right\vert\leq \left\vert N\left(M_Y\right) \right\vert.
  \end{align*}
  Thus, $\left\vert M \right\vert\leq \left\vert N\left(M\right) \right\vert$.\newline

  We will now show that there is an $X$-perfect matching. Suppose toward contradiction that $F$ is a maximal perfect matching on $A\subseteq X$ and $B\subseteq Y$ with $X\setminus A \neq \emptyset$.\newline

  Then, there is $x\in X\setminus A$. Consider $Z\subseteq V$ consisting of all vertices $z$ such that there exists a $F$-alternating path $\left(e_1,\dots,e_n\right)$ between $z\in Z$ and $x$.\newline

  It cannot be the case that $Z\cap Y$ is empty, since the number of neighbors of $x$ is greater than or equal to $1$ by the Hall condition --- if it were the case that $Z\cap Y$ were empty, we could add an edge to $F$ consisting of $x$ and one element of $N\left(\set{x}\right)$, which would contradict the maximality of $F$.\newline

  Consider a path traversing along $Z$, $\left(e_1,\dots,e_n\right)$. It must be the case that $e_n\in F$, or else we would be able to ``flip'' the matching $F$ by exchanging $e_{i}$ with $e_{i+1}$ for $e_i\in F$, which would contradict the maximality of $F$ yet again. Thus, every element of $Z\cap Y$ is satisfied by $F$, so $Z\cap Y\subseteq B$.\newline

  Since each element in $Z\cap Y$ is paired with exactly one element of $Z\cap X$ (with one left over), it is the case that $\left\vert Z\cap X \right\vert = \left\vert Z\cap Y \right\vert + 1$.\newline

  Suppose toward contradiction that there exists $y\in N\left(Z\cap X\right)$ with $y\notin Z\cap Y$. Then, there exists $v\in Z\cap X$ and $e\in E$ such that $\phi(e) = \set{v,y}$. However, this means $v$ is connected via a path to $x$, meaning $y\in Z$, so $y\in Z\cap Y$. Thus, we must have $N\left(Z\cap X\right) = Z\cap Y$.\newline

  Therefore,
  \begin{align*}
    \left\vert Z\cap X \right\vert &= \left\vert Z\cap Y \right\vert + 1\\
                                   &= \left\vert N\left(Z\cap X\right) \right\vert + 1,
  \end{align*}
  which contradicts the fact that $\left(X,Y,E,\phi\right)$ satisfies the Hall condition. Therefore, $A = X$.\newline

  By symmetry, there is a perfect matching of $X$ and $Y$ in $\left(X,Y,E,\phi\right)$.
\end{proof}
\begin{remark}
  An equivalent formulation to Hall's theorem states that there is a system of distinct representatives on the collection $\mathcal{X} = \set{X_k}_{k=1}^{n}$, which is a set $\set{x_{k}}_{k=1}^{n}$ such that $x_{k}\in X_{k}$ and $x_{i}\neq x_j$ for $i\neq j$.\newline

  This implies the existence of an injection $f\colon \mathcal{X}\hookrightarrow \bigcup_{k=1}^{n}X_{k}$, such that $f\left(X_k\right) \in X_k$.
\end{remark}
%\begin{definition}[Choice Function]
%  Let $\mathcal{X} = \set{X_{i}}_{i\in I}$ be a collection of sets. A function $f\colon \mathcal{X}\rightarrow \bigcup_{i\in I}X_i$ is called a choice function if, for each $i\in I$, $f\left(X_{i}\right)\in X{i}$.\newline
%
%  We also say $f\colon \mathcal{X}\rightarrow \bigcup_{i\in I}X_i$ is a choice function if $f\in \prod_{i\in I}X_i$.
%\end{definition}
%
%\begin{theorem}[Tychonoff's Theorem]
%  If $\set{X_{i}}_{i\in I}$ is a family of compact topological spaces
%\end{theorem}
\begin{theorem}[Infinite Hall's Theorem, {\cite{marshall_hall_thm}}]
  Let $\mathcal{G} = \set{X_{i}}_{i\in I}$ be a collection of (not necessarily distinct) finite sets. If, for every finite subcollection $\mathcal{Y} = \set{X_{i_k}}_{k=1}^{n}$,
  \begin{align*}
    n\leq \left\vert \bigcup_{k=1}^{n}X_{i_k} \right\vert,
  \end{align*}
  then there is a choice function on $G$.
\end{theorem}
\begin{proof}
  We endow each $X_i\in \set{X_{i}}_{i\in I}$ with the discrete topology. Since each $X_i$ is finite, each $X_i$ is compact.\newline

  Thus, by Tychonoff's theorem, it is the case that $\prod_{i\in I}X_{i}$ is compact.\newline

  For every finite subset $Y\subseteq \mathcal{G}$, we define
  \begin{align*}
    S_Y &= \set{\left.f\in \prod_{i\in I}X_i\right|f\vert_{Y}\text{ is injective}}.
  \end{align*}
  The injectivity of $f\vert_{Y}$ is equivalent to the existence of a system of distinct representatives on $Y$. Since $Y$ satisfies the Hall condition, each $S_{Y}$ is nonempty. Additionally, for any net of functions $f_{\alpha}\in S_{Y}$ with $\lim_{\alpha}f_{\alpha} = f$, it is the case that $f_{\alpha}\vert_{Y}$ is injective, so $f\vert_{Y}$ is injective, meaning $S_{Y}$ is closed.\newline

  We define $F = \set{S_{Y}| Y\subseteq \mathcal{G}\text{ finite}}$. For finite $Y_{1},Y_{2}\subseteq \mathcal{G}$, every system of distinct representatives in $Y_1\cup Y_2$ is necessarily a system of distinct representatives on $Y_1$ and a system of distinct representatives on $Y_{2}$, meaning $S_{Y_1\cup Y_2}\subseteq S_{Y_1}\cap S_{Y_2}$. Thus, $F$ has the finite intersection property.\newline

  Since $\prod_{i\in I}X_i$ is compact, $\bigcap F$ is nonempty, where the intersection is taken over all finite subsets of $\mathcal{G}$. For any $f\in \bigcap F$, $f$ is necessarily a choice function.
\end{proof}
\begin{remark}
  This is equivalent to the existence of an injection $f\colon \mathcal{G}\hookrightarrow \bigcup_{i\in I}X_i$.
\end{remark}
We will use this infinite case of Hall's theorem to prove König's theorem. 
\begin{theorem}[König's Theorem, {\cite[Theorem 0.2.4]{lectures_on_amenability}}]
  Let $\left(X,Y,E,\phi\right)$ be a $k$-regular bipartite graph (not necessarily finite). Then, there is a perfect matching of $X$ and $Y$.\label{thm:konig}
\end{theorem}
\begin{proof}
  If $k = 1$, it is clear that there is a perfect matching in $\left(X,Y,E,\phi\right)$ consisting of the edges in $\left(X,Y,E,\phi\right)$.\newline

  Let $k\geq 2$. Since any finite subset of $X$ satisfies the Hall condition, as displayed in the proof of Theorem \ref{thm:hall_finite}, there is some $X$-perfect matching in $\left(X,Y,E,\phi\right)$. We call this $X$-perfect matching $F$. There is an injection $f\colon X\hookrightarrow Y$ following the edges in $F$.\newline

  Similarly, since any finite subset of $Y$ satisfies the Hall condition, there is some $Y$-perfect matching in $\left(X,Y,E,\phi\right)$. We call this $Y$-perfect matching $G$. There is an injection $g\colon Y\hookrightarrow X$ following the edges of $G$.\break

  Consider the subgraph $\left(X,Y,F\cup G,\phi|_{F\cup G}\right)$. The injections $f$ and $g$ still hold in this graph. By the Cantor--Schröder--Bernstein theorem, there is a bijection $h\colon X\rightarrow Y$ in $\left(X,Y,F\cup G,\phi|_{F\cup G}\right)$, which is equivalent to the existence of a perfect matching of $X$ and $Y$.
\end{proof}
\section{Type Semigroups}%
\begin{definition}[{\cite[Definition 0.2.5]{lectures_on_amenability}}]\label{def:xstar_gstar}
  Let $G$ be a group that acts on a set $X$.
  \begin{enumerate}[(i)]
    \item We define $X^{\ast} = X\times \Z_{\geq }$, and
      \begin{align*}
        G^{\ast} &= \set{\left(g,\pi\right)| g\in G,\pi\in\sym\left(\Z_{\geq 0}\right)}.
      \end{align*}
    \item If $A\subseteq X^{\ast}$, the values of $n$ for which there is an element of $A$ whose second coordinate is $n$ are called the \textit{levels} of $A$.
  \end{enumerate}
\end{definition}
\begin{fact}[{\cite[Exercise 0.2.4]{lectures_on_amenability}}]\label{fact:type_semigroup_equidecomposability}
  If $E_1,E_2\subseteq X$, then $E_{1}\sim_{G}E_2$ if and only if $E_1\times \set{n}\sim_{G^{\ast}}E_{2}\times \set{m}$ for all $m,n\in \Z_{\geq 0}.$
\end{fact}
\begin{proof}
  Let $E_{1}\sim_{G}E_2$. Then, there exist pairwise disjoint $A_1,\dots,A_n\subset E_1$, pairwise disjoint $B_1,\dots,B_n\subset E_2$, and elements $g_1,\dots,g_n\in G$ such that $g_i\cdot A_i = B_i$. We select the permutation $\pi_{i}\in \sym\left(\Z_{\geq 0}\right)$ such that $\pi_{i}(n) = m$ and $\pi_i(m) = n$ for each $i$. Then,
  \begin{align*}
    \left(g_i,\pi_i\right)\cdot \left(A_{i}\cdot \set{n}\right) &= B_{i}\cdot \set{m}.
  \end{align*}

  Similarly, if $E_{1}\times \set{n} \sim_{G^{\ast}}E_2\times \set{m}$, then of the pairwise disjoint subsets
  \begin{align*}
    A_1\times \set{n},\dots,A_n\times \set{n}\subset E_1\times \set{n}
  \end{align*}
  and
  \begin{align*}
    B_1\times\set{m},\dots,B_n\times\set{m}\subset E_2\times \set{m},
  \end{align*}
  we set $A_1,\dots,A_n\subset E_1$ and $B_1,\dots,B_n\subset E_2$. Similarly, for
  \begin{align*}
    \left(g_1,\pi_1\right),\dots,\left(g_n,\pi_n\right)\in G^{\ast}
    \intertext{such that}
    \left(g_i,\pi_i\right)\cdot A_i\times \set{n} = B_i\times\set{m},
  \end{align*}
  we select $g_1,\dots,g_n\in G$. Then, by definition,
  \begin{align*}
    g_i\cdot A_i = B_i
  \end{align*}
  for each $i$. Thus, $E_1\sim_{G}E_2$.
\end{proof}

\begin{definition}[{\cite[Definition 0.2.6]{lectures_on_amenability}}]\label{def:type_semigroup}
  Let $G$ be a group that acts on $X$, and let $G^{\ast}$, $X^{\ast}$ be defined as in \ref{def:xstar_gstar}.
  \begin{enumerate}[(i)]
    \item A set $A\subseteq X^{\ast}$ is said to be \textit{bounded} if it has finitely many levels.
    \item If $A\subseteq X^{\ast}$ is bounded, the equivalence class of $A$ with respect to $G^{\ast}$-equidecomposability is called the \textit{type} of $A$, which is denoted $\left[A\right]$.
    \item If $E\subseteq X$, we write $\left[E\right] = \left[E\times \set{0}\right]$.
    \item Let $A,B\subseteq X^{\ast}$ be bounded with $k\in \Z_{\geq 0}$ such that for
      \begin{align*}
        B'= \set{\left(b,n+k\right)| \left(b,n\right)\in B},
      \end{align*}
      we have $B'\cap A = \emptyset$. Then, $\left[A\right] + \left[B\right] = \left[A\sqcup B'\right]$. Note that $\left[B'\right] = \left[B\right]$.
    \item We define
      \begin{align*}
        \mathcal{S} &= \set{\left[A\right]| A\subseteq X^{\ast}\text{ bounded}}
      \end{align*}
      under the addition defined in (iv) to be the \textit{type semigroup} of the action of $G$ on $X$.
  \end{enumerate}
\end{definition}

\begin{fact}[{\cite[Exercise 0.2.5]{lectures_on_amenability}}]
  Addition is well-defined in $\left(\mathcal{S},+\right)$, and $\left(\mathcal{S},+\right)$ is a well-defined commutative semigroup with identity $\left[\emptyset\right]$.\label{fact:type_semigroup_well_defined}
\end{fact}
\begin{proof}
  To show that addition is well-defined, we let $\left[A_1\right] = \left[A_2\right]$, and $\left[B_1\right] = \left[B_2\right]$. Without loss of generality, $A_1\cap B_1 = \emptyset$ and $A_2\cap B_2 = \emptyset$.\newline

  By the definition of the type, $A_1\sim_{G^{\ast}}A_2$ and $B_1\sim_{G^{\ast}}B_2$, meaning
  \begin{align*}
    A_1\sqcup B_1\sim_{G^{\ast}} A_2\sqcup B_2,
  \end{align*}
  so
  \begin{align*}
    \left[A_1\right] + \left[B_1\right] &= \left[A_1\sqcup B_1\right]\\
                                        &= \left[A_2\sqcup B_2\right]\\
                                        &= \left[A_2\right] + \left[A_2\right],
  \end{align*}
  meaning addition is well-defined.\newline

  Since addition is well-defined, and $A\sqcup B = B\sqcup A$, we can see that addition is also commutative. We also have
  \begin{align*}
    \left[A\right] + \left[\emptyset\right] &= \left[A\sqcup \emptyset\right]\\
                                            &= \left[A\right],
  \end{align*}
  so $\left[\emptyset\right]$ is the identity on $\mathcal{S}$.\newline

  Finally, since for any $\left[A\right],\left[B\right]\in \mathcal{S}$, $A$ and $B$ have finitely many levels, it is the case that $A\cup B$ has finitely many levels for any $A$ and $B$, so $\left[A\right] + \left[B\right] \in \mathcal{S}$. 
\end{proof}

\begin{definition}[{\cite[10]{lectures_on_amenability}}]
  For any commutative semigroup $\mathcal{S}$ with $\alpha \in S$ and $n\in \N$, we define
  \begin{align*}
    n\alpha = \underbrace{\alpha + \cdots + \alpha}_{\text{$n$ times}}
  \end{align*}
\end{definition}
\begin{definition}[{\cite[10]{lectures_on_amenability}}]
  For $\alpha,\beta \in \mathcal{S}$, if there exists $\gamma \in \mathcal{S}$ such that $\alpha + \gamma = \beta$, we write $\alpha \leq \beta$.
\end{definition}
\begin{fact}[{\cite[Exercise 0.2.7]{lectures_on_amenability}}]\label{fact:type_semigroup_paradoxicality}%\label{fact:type_semigroup_criterion_paradoxicality}
  If $G$ is a group acting on $X$ with corresponding type semigroup $\mathcal{S}$, then the following are true.
  \begin{enumerate}[(i)]
    \item If $\alpha,\beta\in \mathcal{S}$ with $\alpha \leq \beta$ and $\beta \leq \alpha$, then $\alpha = \beta$.
    \item A set $E\subseteq X$ is $G$-paradoxical if and only if $\left[E\right] = 2\left[E\right]$.
  \end{enumerate}
\end{fact}
\begin{proof}
  Let $G$ act on $X$, and let $\mathcal{S}$ be the corresponding type semigroup.
  \begin{enumerate}[(i)]
    \item If $\left[A\right]\leq \left[B\right]$, then there exists $C_1\in \mathcal{S}$ such that $\left[A\right] + \left[C_1\right] = \left[B\right]$. Without loss of generality, $C_1\cap A= \emptyset$, meaning $\left[B\right] = \left[A\sqcup C_1\right]$. Thus, $A\sqcup C_1 \sim_{G^{\ast}} B$, meaning $B\preceq_{G^{\ast}}A$.\newline

      Similarly, if $\left[B\right]\leq \left[A\right]$, then $B\preceq_{G^{\ast}}A$. By Theorem \ref{thm:csb_for_equidecomposability}, it is thus the case that $A\sim_{G^{\ast}}B$.
    \item Let $E$ be $G$-paradoxical. \newline

      Then, $E\sim_{G}\bigsqcup_{i=1}^{n}A_i$ and $E \sim_{G}\bigsqcup_{j=1}^{m}B_j$ for pairwise disjoint subsets $A_1,\dots,A_n,B_1,\dots,B_m\subset E$. Thus, we have
      \begin{align*}
        \left[E\right] &= \left[\left(\bigsqcup_{i=1}^{n}A_i\right)\sqcup \left(\bigsqcup_{j=1}^{m}B_j\right)\right]\\
                       &= \left[\bigsqcup_{i=1}^{n}A_i\right] + \left[\bigsqcup_{j=1}^{m}B_j\right]\\
                       &= 2\left[E\right].
      \end{align*}
      Similarly, if $\left[E\right] = 2\left[E\right]$, then there exist $A$ and $B$ such that
      \begin{align*}
        \left[E\right] &= \left[A\right] + \left[B\right]\\
                       &= \left[A\sqcup B\right],
      \end{align*}
      meaning $A$ and $B$ are each $G$-equidecomposable with $E$, so $E$ is $G$-paradoxical.
  \end{enumerate}
\end{proof}
We can now prove the cancellation identity, which we will be useful as we construct our desired finitely additive measure.
\begin{theorem}[Cancellation Identity on $\mathcal{S}$, {\cite[Theorem 0.2.7]{lectures_on_amenability}}]
  Let $\mathcal{S}$ be the type semigroup for some group action, and let $\alpha,\beta\in \mathcal{S}$, $n\in \N$ such that $n\alpha = n\beta$. Then, $\alpha = \beta$.
\end{theorem}
\begin{proof}
  Let $n\alpha = n\beta$. Then, there are two disjoint bounded subsets $E,E'\subseteq X^{\ast}$ with $E\sim_{G^{\ast}}E'$, and pairwise disjoint subsets $A_1,\dots,A_n\subseteq E$, $B_1,\dots,B_n\subseteq E'$ such that
  \begin{itemize}
    \item $E = A_1\cup\cdots\cup A_n$, $E' = B_1\cup\cdots\cup B_n$
    \item $\left[ A_j \right] = \alpha$ and $\left[B_j\right] = \beta$ for each $j=1,\dots,n$.
  \end{itemize}
  Let $\chi\colon E\rightarrow E'$ be a bijection as in Fact \ref{fact:bijections}, with $\phi_j\colon A_1\rightarrow A_j$, $\psi_j\colon B_1\rightarrow B_j$ also being bijections as in Fact \ref{fact:bijections}; here we define $\phi_1$ and $\psi_1$ to be the identity map.\newline

  For each $a\in A_1$ and $b\in B_1$, we define
  \begin{align*}
    \overline{a} &= \set{a,\phi_2(a),\dots,\phi_n(a)}\\
    \overline{b} &= \set{b,\psi_2(b),\dots,\psi_n(b)}.
  \end{align*}
  We construct a graph by letting $X = \set{\overline{a}| a\in A_1}$ and $Y = \set{\overline{b}| b\in B_1}$, and, for each $j$, define edges $\set{\overline{a},\overline{b}}$ if $\chi\left(\phi_j(a)\right)\in \overline{b}$.\newline

  Since $\chi$ is a bijection, for each $j=1,\dots,n$, $\chi\left(\phi_j(a)\right)$ must be an element of $B_k$ for some $k$, and since $\set{B_k}_{k=1}^{n}$ are disjoint, $\chi\left(\phi_j(a)\right)$ is an element of exactly one $B_k$. Thus, the graph is $n$-regular.\newline

  By Theorem \ref{thm:konig}, this graph has a perfect matching $F$. As a result, for each $\overline{a}\in X$, there is a unique $\overline{b}\in Y$ and a unique edge $\set{\overline{a},\overline{b}}\in F$ such that $\chi\left(\phi_j(a)\right) = \psi_k(b)$ for some $j,k\in \set{1,\dots,n}$.\newline

  We define
  \begin{align*}
    C_{j,k} &= \set{a\in A_1| \set{\overline{a},\overline{b}}\in F,~\chi\left(\phi_j(a)\right) = \psi_k(b)}\\
    D_{j,k} &= \set{b\in B_1| \set{\overline{a},\overline{b}}\in F,~\chi\left(\phi_j(a)\right) = \psi_k(b)}.
  \end{align*}
  Therefore, we must have $\psi_{k}^{-1}\circ \chi\circ \phi_j$ is a bijection from $C_{j,k}$ to $D_{j,k}$, so $C_{j,k}\sim_{G^{\ast}}D_{j,k}$.\newline

  Since $C_{j,k}$ and $D_{j,k}$ are partitions of $A_1$ and $B_1$ respectively, it follows that $A_1\sim_{G^{\ast}}B_1$, so $\alpha = \beta$.
\end{proof}
\begin{corollary}[{\cite[Corollary 0.2.8]{lectures_on_amenability}}]\label{cor:non_paradoxicality}
  Let $\mathcal{S}$ be the type semigroup of some group action, and let $\alpha\in \mathcal{S}$ and $n\in \N$ such that $\left(n+1\right)\alpha \leq n\alpha$. Then, $\alpha = 2\alpha$.\label{corollary:paradoxical_elements}
\end{corollary}
\begin{proof}
  We have
  \begin{align*}
    2\alpha + n\alpha &= \left(n+1\right)\alpha + \alpha\\
                      &\leq n\alpha + \alpha\\
                      &= \left(n+1\right)\alpha\\
                      &\leq n\alpha.
  \end{align*}
  Inductively repeating this argument, we get $n\alpha \geq 2n\alpha$; since $n\alpha \leq 2n\alpha$ by definition, we must have $n\alpha = 2n\alpha$, so $\alpha = 2\alpha$.
\end{proof}
\begin{remark}
  We will call such an $\alpha$ a paradoxical element.
\end{remark}
\section{Two Results on Commutative Semigroups}%
Now that we are aware of paradoxical elements and the relationship between $G$-paradoxicality and the properties of the particular elements of the type semigroup (Fact \ref{fact:type_semigroup_paradoxicality}), we will now relate these properties to finitely additive measures of sets by using the following lemma and theorem.
\begin{lemma}[{\cite[Lemma 0.2.9]{lectures_on_amenability}}]\label{lemma:set_function_existence}
  Let $\mathcal{S}$ be a commutative semigroup, with $\mathcal{S}_0\subseteq \mathcal{S}$ finite, and $\epsilon\in \mathcal{S}_0$ satisfying the following assumptions:
  \begin{enumerate}[(a)]
    \item $\left(n+1\right)\epsilon \nleq n\epsilon$ for all $n\in \N$ (i.e., that $\epsilon$ is non-paradoxical);
    \item for each $\alpha\in \mathcal{S}$, there is $n\in \N$ such that $\alpha \leq n\epsilon$.
  \end{enumerate}
  Then, there is a set function $\nu\colon \mathcal{S}_0\rightarrow [0,\infty]$ that satisfies the following conditions:
  \begin{enumerate}[(i)]
    \item $\nu\left(\epsilon\right) = 1$;
    \item for $\alpha_1,\dots,\alpha_n,\beta_1,\dots,\beta_m\in \mathcal{S}_0$ with $\alpha_1+\cdots+\alpha_n\leq \beta_1+\cdots\beta_m$,
      \begin{align*}
        \sum_{j=1}^{n}\nu\left(\alpha_j\right) \leq \sum_{j=1}^{m}\nu\left(\beta_j\right).
      \end{align*}
  \end{enumerate}
\end{lemma}
\begin{proof}
  We will prove this result by inducting on the cardinality of $\mathcal{S}_0$.\newline

  We start with $\left\vert \mathcal{S}_0 \right\vert = 1$. In that case, we define $\nu\left(\epsilon\right) = 1$, satisfying condition (i). To satisfy condition (ii), we see that for $n,m\in \N$ with $n\epsilon \leq m\epsilon$, if $n \geq m+1$, then $\left(m+1\right)\epsilon \leq n\epsilon \leq m\epsilon$, implying that $\epsilon = 2\epsilon$, which contradicts assumption (a).\newline

  Let $\alpha_0\in \mathcal{S}_0\setminus\set{\epsilon}$. The induction hypothesis says there is a set function satisfying conditions (i) and (ii), $\nu\colon \mathcal{S}_0\setminus \set{\alpha_0}\rightarrow [0,\infty]$.\newline

  For $r\in \N$, there are $\gamma_1,\dots,\gamma_p,\delta_1,\dots,\delta_q\in \mathcal{S}\setminus \set{\alpha_0}$ such that
  \begin{align*}
    \delta_{1} + \cdots + \delta_q + r\alpha_0 \leq \gamma_1 + \cdots + \gamma_p.\label{set_function_id1}\tag*{(\textdagger)}
  \end{align*}
  Consider the set $N$ defined as follows:
  \begin{align*}
    N &= \set{\frac{1}{r}\left(\sum_{j=1}^{p}\nu\left(\gamma_j\right) - \sum_{j=1}^{q}\nu\left(\delta_j\right)\right)| \text{$\gamma_j,\delta_j$ satisfy \ref{set_function_id1}}}. \label{set_function_N}\tag*{($\ddag$)}
  \end{align*}
  We define the extension of $\nu$ as follows:
  \begin{align*}
    \nu\left(\alpha_0\right) &= \inf N.
  \end{align*}
  This infimum is well-defined since, by assumption (b), there is some $n\in \N$ such that $\alpha_0 \leq n\epsilon$, and $\nu\left(\epsilon\right)$ is defined.\newline

  Now, we must show that this extension of $\nu$ satisfies condition (ii).\newline

  Let $\alpha_1,\dots,\alpha_n,\beta_1,\dots,\beta_m\in \mathcal{S}_0\setminus \set{\alpha_0}$ and $s,t\in \Z_{\geq 0}$ such that
  \begin{align*}
    \alpha_1 + \cdots + \alpha_n + s\alpha_0 \leq \beta_1 + \cdots + \beta_m + t\alpha_0.\label{set_function_conditionii}\tag*{(\textasteriskcentered)}
  \end{align*}
  We will verify condition (ii) in the three following cases.
  \begin{description}[font=\normalfont\scshape,leftmargin=0cm]
    \item[Case 0:] If $s = t = 0$, then the induction hypothesis states that \ref{set_function_conditionii} satisfies condition (ii).
    \item[Case 1:] Let $s = 0$ and $t > 0$. We want to show that
      \begin{align*}
        \sum_{j=1}^{n}\nu\left(\alpha_j\right) \leq t\nu\left(\alpha_0\right) + \sum_{j=1}^{m}\nu\left(\beta_j\right),
      \end{align*}
      which implies that
      \begin{align*}
        \nu\left(\alpha_0\right) \geq \frac{1}{t}\left(\sum_{j=1}^{n}\nu\left(\alpha_j\right) - \sum_{j=1}^{m}\nu\left(\beta_j\right)\right).
      \end{align*}
      By the definition of infimum, it suffices to show that for $r\in \N$ and $\delta_1,\dots,\delta_q,\gamma_1,\dots,\gamma_p\in \mathcal{S}\setminus \set{\alpha_0}$ satisfying \ref{set_function_id1}, it is the case that
      \begin{align*}
        \frac{1}{r}\left(\sum_{j=1}^{p}\nu\left(\gamma_j\right)-\sum_{j=1}^{q}\nu\left(\delta_j\right)\right) \geq \frac{1}{t}\left(\sum_{j=1}^{n}\nu\left(\alpha_j\right) - \sum_{j=1}^{m}\nu\left(\beta_j\right)\right).
      \end{align*}
      Multiplying \ref{set_function_conditionii} by $r$ on both sides, and adding $t\delta_1 + \cdots + t\delta_q$ to both sides, we have
      \begin{align*}
        r\alpha_1 + \cdots + r\alpha_n + t\delta_1 + \cdots + t\delta_q \leq r\beta_1 + \cdots + r\beta_m + t\left(r\alpha_0\right) + t\delta_1 + \cdots + t\delta_q.
      \end{align*}
      Substituting \ref{set_function_id1}, we find
      \begin{align*}
        r\alpha_1 + \cdots + r\alpha_n + t\delta_1 + \cdots + t\delta_q \leq r\beta_1 + \cdots + r\beta_m + t\gamma_1 + \cdots + t\gamma_p.
      \end{align*}
      Applying the induction hypothesis, we have
      \begin{align*}
        r\sum_{j=1}^{n}\nu\left(\alpha_j\right) + t\sum_{j=1}^{q}\nu\left(\delta_j\right) \leq r\sum_{j=1}^{m}\nu\left(\beta_j\right) + t\sum_{j=1}^{p}\nu\left(\gamma_j\right),
      \end{align*}
      yielding
      \begin{align*}
        \frac{1}{r}\left(\sum_{j=1}^{p}\nu\left(\gamma_j\right) - \sum_{j=1}^{q}\nu\left(\delta_j\right)\right) \geq \frac{1}{t}\left(\sum_{j=1}^{n}\nu\left(\alpha_j\right) - \sum_{j=1}^{m}\nu\left(\beta_j\right)\right).
      \end{align*}
    \item[Case 2:] Let $s > 0$. For $z_1,\dots,z_t\in N$ \ref{set_function_N}, we need to show that
      \begin{align*}
        s\nu\left(\alpha_0\right) + \sum_{j=1}^{n}\nu\left(\alpha_j\right) \leq z_1 + \cdots + z_t + \sum_{j=1}^{n}\nu\left(\beta_j\right).
      \end{align*}
      Without loss of generality, we can set $z_1,\dots,z_n = z$, as for each $z\in N$, $z \geq \nu\left(\alpha_0\right)$.\newline

      As in Case 1, we multiply \ref{set_function_conditionii} by $r$, add $t\delta_{1} + \cdots + t\delta_q$ to both sides, and substitute with \ref{set_function_id1}, yielding
      \begin{align*}
        r\alpha_1 + \cdots + r\alpha_n + rs\alpha_0 + t\delta_1 + \cdots + t\delta_q &\leq r\beta_1 + \cdots + r\beta_m + t\left(r\alpha_0\right) + t\delta_1 + \cdots + t\delta_q\\
        r\alpha_1 + \cdots + r\alpha_n + t\delta_1 + \cdots + t\delta_q + rs\alpha_0 &\leq r\beta_1 + \cdots + r\beta_m + t\gamma_1 + \cdots + t\gamma_p.
      \end{align*}
      Defining
      \begin{align*}
        z &= \frac{1}{r}\left(\sum_{j=1}^{p}\nu\left(\gamma_j\right) - \sum_{j=1}^{q}\nu\left(\delta_j\right)\right),
      \end{align*}
      we get
      \begin{align*}
        s\nu\left(\alpha_0\right) + \sum_{j=1}^{n}\nu\left(\alpha_j\right) &\leq \sum_{j=1}^{n}\nu\left(\alpha_j\right) + \frac{s}{sr}\left(r\sum_{j=1}^{m}\nu\left(\beta_j\right) - r\sum_{j=1}^{n}\nu\left(\alpha_j\right) + t\sum_{j=1}^{p}\nu\left(\gamma_j\right) - t\sum_{j=1}^{q}\nu\left(\delta_j\right)\right)\\
                                                                           &= tz + \sum_{j=1}^{m}\nu\left(\beta_j\right).
      \end{align*}
  \end{description}
  Thus, we have shown that $\nu$ extends in a manner that satisfies conditions (i) and (ii).
\end{proof}

We can ``upgrade'' our finitely additive set function to a semigroup homomorphism as follows.
\begin{theorem}[{\cite[Theorem 0.2.10]{lectures_on_amenability}}]\label{thm:homomorphism_existence}
  Let $\left(\mathcal{S},+\right)$ be a commutative semigroup with identity element $0$, and let $\epsilon\in \mathcal{S}$. Then, the following are equivalent:
  \begin{enumerate}[(i)]
    \item $\left(n+1\right)\epsilon \leq n\epsilon$ for all $n\in \N$;
    \item there is a semigroup homomorphism $\nu\colon \left(\mathcal{S},+\right)\rightarrow \left([0,\infty],+\right)$ such that $\nu(\epsilon) = 1$.
  \end{enumerate}
\end{theorem}
\begin{proof}
  To show that (ii) implies (i), we let $\nu\colon \left(\mathcal{S},+\right)\rightarrow \left([0,\infty],+\right)$ be a semigroup homomorphism with $\nu\left(\epsilon\right) = 1$. Then,
  \begin{align*}
    \nu\left(\left(n+1\right)\epsilon\right) &= \left(n+1\right)\nu\left(\epsilon\right)\\
                                             &= n+1\\
                                             &> n\\
                                             &= n\nu\left(\epsilon\right)\\
                                             &= \nu\left(n\epsilon\right),
  \end{align*}
  meaning that $\left(n+1\right)\epsilon \nleq n\epsilon$.\newline

  To show that (i) implies (ii), we suppose that for each $\alpha \in \mathcal{S}$, there is $n\in \N$ such that $\alpha \leq n\epsilon$ --- for any such $\alpha$ for which this is not the case, we define $\nu\left(\alpha\right) = \infty$.\newline

  For a finite subset $\mathcal{S}_0 \subseteq \mathcal{S}$ with $\epsilon\in \mathcal{S}_0$, we define $M_{\mathcal{S}_0}$ to be the set of all $\kappa\colon \mathcal{S}\rightarrow [0,\infty]$ such that
  \begin{itemize}
    \item $\kappa\left(\epsilon\right) = 1$;
    \item $\kappa\left(\alpha + \beta\right) = \kappa\left(\alpha\right) + \kappa\left(\beta\right)$ for $\alpha,\beta,\alpha + \beta\in \mathcal{S}_0$.
  \end{itemize}
  Since we assume condition (i), we know that such a $\kappa$ with $\kappa\left(\epsilon\right) = 1$ exists. Additionally, since
  \begin{align*}
    \alpha + \beta \leq \left(\alpha + \beta\right)
  \end{align*}
  and
  \begin{align*}
    \left(\alpha + \beta\right) \leq \alpha + \beta,
  \end{align*}
  it is the case that
  \begin{align*}
    \kappa\left(\alpha + \beta\right) \leq \kappa\left(\alpha\right) + \kappa\left(\beta\right) \leq \kappa\left(\alpha + \beta\right),
  \end{align*}
  meaning $\kappa\left(\alpha + \beta\right) = \kappa\left(\alpha\right) + \kappa\left(\beta\right)$. Thus, $M_{\mathcal{S}_0}$ is nonempty. It is also the case that $M_{\mathcal{S}_0}$ is closed, since any net of functions $\kappa_{p}\colon \mathcal{S}\rightarrow [0,\infty]$ with $\kappa_{p}\left(\epsilon\right) = 1$ and $\kappa_{p}\left(\alpha + \beta\right) = \kappa_{p}\left(\alpha\right) + \kappa_{p}\left(\beta\right)$ will necessarily satisfy these conditions in the limit.\newline

  We let $\left[0,\infty\right]^{\mathcal{S}} = \set{\kappa| \kappa:\mathcal{S}\rightarrow [0,\infty]}$ be equipped with the product topology. By Tychonoff's theorem, $\left[0,\infty\right]^{\mathcal{S}}$ is compact.\newline

  Furthermore, for any finite subcollection $\mathcal{S}_1,\dots,\mathcal{S}_n$, it is the case that
  \begin{align*}
    M_{\mathcal{S}_1\cup\cdots\cup \mathcal{S}_n} \subseteq M_{\mathcal{S}_1} \cap \cdots \cap M_{\mathcal{S}_n},
  \end{align*}
  as any such $\kappa\in M_{\mathcal{S}_1\cup\cdots\cup \mathcal{S}_n}$ must necessarily be in every $M_{\mathcal{S}_i}$.\newline

  Thus, the family
  \begin{align*}
    \mathcal{M} &= \set{M_{\mathcal{S}_0}| \mathcal{S}_0\subseteq \mathcal{S}\text{ finite}}
  \end{align*}
  has the finite intersection property. By compactness, there is some $\nu$ such that
  \begin{align*}
    \nu\in \bigcap \mathcal{M}
  \end{align*}
  with $\nu\left(\epsilon\right) = 1$ and, for all $\alpha,\beta\in \mathcal{S}$, since $\nu\in M_{\set{\alpha,\beta,\alpha + \beta}}$, $\nu\left(\alpha + \beta\right) = \nu\left(\alpha\right) + \nu\left(\beta\right)$.
\end{proof}
\section{Proof of Tarski's Theorem}%
Finally, we are able to prove the reverse direction of Tarski's Theorem. We restate the theorem before giving its proof.
\begin{tcolorbox}[blanker,breakable,left=3mm,before skip=10pt, after skip=10pt, borderline west={1pt}{0pt}{blue!50!white},sharp corners,]
\tarski*
\end{tcolorbox}
\begin{proof}[Proof of the Reverse Direction of Theorem \ref{thm:tarski}:]
  Let $\mathcal{S}$ be the type semigroup of the action of $G$ on $X$.\newline

  Suppose $E$ is not $G$-paradoxical. Then, $\left[E\right]\neq 2\left[E\right]$ by Fact \ref{fact:type_semigroup_paradoxicality}, meaning $\left(n+1\right)\left[E\right]\nleq n\left[E\right]$ for all $n\in \N$ by the contrapositive of Corollary \ref{cor:non_paradoxicality}.\newline

  Thus, by Theorem \ref{thm:homomorphism_existence}, there is a map $\nu\colon \mathcal{S}\rightarrow [0,\infty]$ with $\nu\left(\left[E\right]\right) = 1$. The map $\mu\colon P(X)\rightarrow [0,\infty]$ defined by
  \begin{align*}
    \mu\left(A\right) &= \nu\left(\left[A\right]\right)
  \end{align*}
  is the desired finitely additive measure.
\end{proof}

\chapter{The More Things Change, the More They Stay the Same: Invariant States}\label{ch:invariant_states}
\epigraph{The whole is greater than the sum of its parts.}{Aristotle, who had yet to learn about amenability in groups.}
In Chapter 4, we introduced amenability through Tarski's Theorem (Theorem \ref{thm:tarski}), where we proved the existence of a translation-invariant finitely additive set function, $\mu\colon P(X)\rightarrow [0,\infty]$, and the non-paradoxicality of $G$'s action on $X$. There, we used the type semigroup of $G$'s action on $X$ in order to prove the theorem. Now, we turn our attention towards other constructions from analysis --- as well as the intrinsic properties of the group itself --- to understand ahd prove other criteria for $G$'s amenability.\newline

In this section, we will use techniques from functional analysis to prove the equivalence between amenability and the existence of an invariant state $\mu\colon \ell_{\infty}(G) \rightarrow [0,1]$.
\section{Amenability in Subgroups and Quotient Groups}\label{sec:amenability_subgroups_quotients}%
We begin by defining a mean on $G$ --- note that this definition is slightly different from the one used in the proof in Theorem \ref{thm:tarski}. However, one can show that they are equivalent by letting $G$ act on itself by left-multiplication and taking $E = G$.
\begin{definition}
  Let $G$ be a group, with $P(G)$ denoting its power set.\newline

  An invariant mean on $G$ is a set function $m\colon P(G)\rightarrow [0,1]$ which satisfies, for all $t\in G$ and $E,F\subseteq G$,
  \begin{itemize}
    \item $m(G) = 1$;
    \item $m\left(E\sqcup F\right) = m(E) + m(F)$;
    \item $m\left(tE\right) = m\left(E\right)$.
  \end{itemize}
  We say $G$ is amenable if $G$ admits a mean.\newline

  The mean $m$ is a translation-invariant probability measure on the measurable space $\left(G,P(G)\right)$.
\end{definition}
We can establish some inheritance properties using the properties of a mean. In Proposition \ref{prop:subgroups_quotientgroups_amenability}, we will show that subgroups of amenable groups are amenable and quotients of amenable groups are amenable.
\begin{proposition}\label{prop:subgroups_quotientgroups_amenability}
  Let $G$ be an amenable group with $H\leq G$. Then, the following are true:
  \begin{enumerate}[(1)]
    \item $H$ is amenable;
    \item for $H\trianglelefteq G$, $G/H$ is amenable.
  \end{enumerate}
\end{proposition}
\begin{proof}\hfill
  \begin{enumerate}[(1)]
    \item Let $R$ be a right transversal for $H$, wherein we select one element of each right coset of $H$ to make up $R$.\newline

      If $m$ is a mean for $G$, we set $\lambda\colon P(H)\rightarrow [0,1]$ defined by
      \begin{align*}
        \lambda(E) = m\left(ER\right).
      \end{align*}
       We have
      \begin{align*}
        \lambda(H) &= m\left(HR\right)\\
                   &= m\left(G\right)\\
                   &= 1.
      \end{align*}
      We claim that if $E\cap F = \emptyset$, then $ER \cap FR = \emptyset$. Suppose toward contradiction this is not the case. Then, $xr_1 = yr_2$ for some $x\in E$, $y\in F$, and $r_1,r_2\in R$. Then, we must have $r_2r_1^{-1} = y^{-1}x \in H$, meaning $r_1 = r_2$ as, by definition, $R$ contains exactly one element of each right coset. Thus, $x=y$, so $E\cap F \neq \emptyset$.\newline

      We then have
      \begin{align*}
        \lambda\left(E\sqcup F\right) &= m\left(\left(E\sqcup F\right)R\right)\\
                                      &= m\left(ER\sqcup FR\right)\\
                                      &= m\left(ER\right) + m\left(FR\right)\\
                                      &= \lambda\left(E\right) + \lambda\left(F\right),
      \end{align*}
      and
      \begin{align*}
        \lambda\left(sE\right) &= m\left(sER\right)\\
                               &= m\left(ER\right)\\
                               &= \lambda\left(E\right).
      \end{align*}
    \item Let $\pi\colon G\rightarrow G/H$ be the canonical projection, defined by $\pi\left(t\right) = tH$. We define
      \begin{align*}
        \lambda\colon P\left(G/H\right) \rightarrow [0,1]
      \end{align*}
      by $\lambda(E) = m\left(\pi^{-1}\left(E\right)\right)$. We have
      \begin{align*}
        \lambda\left(G/H\right) &= m\left(\pi^{-1}\left(G/H\right)\right)\\
                                &= m\left(G\right)\\
                                &= 1,
      \end{align*}
      and
      \begin{align*}
        \lambda\left(E\sqcup F\right) &= m\left(\pi^{-1}\left(E\sqcup F\right)\right)\\
                                      &= m\left(\pi^{-1}\left(E\right)\sqcup \pi^{-1}\left(F\right)\right)\\
                                      &= m\left(\pi^{-1}\left(E\right)\right) + m\left(\pi^{-1}\left(F\right)\right)\\
                                      &= \lambda(E) + \lambda(F).
      \end{align*}
      To show translation-invariance, we let $sH = \pi(s)\in G/H$, and $E\subseteq G/H$. Note that
      \begin{align*}
        \pi^{-1}\left(\pi(s)E\right) &= s\pi^{-1}\left(E\right),
      \end{align*}
      since for $r\in s\pi^{-1}(E)$, we have $r = st$ for $t\in \pi(E)$, so $\pi\left(r\right) =\pi\left(st\right) = \pi\left(s\right)\pi\left(t\right)\in \pi\left(s\right)E$.\newline

      Additionally, if $r\in \pi^{-1}\left(\pi(s)E\right)$, we have $\pi(r)\in \pi(s)E$, so $\pi\left(s^{-1}r\right)\in E$, meaning $s^{-1}r\in \pi^{-1}(E)$.\newline

      Thus,
      \begin{align*}
        \lambda\left(\pi\left(s\right)E\right) &= m\left(\pi^{-1}\left(\pi\left(s\right)E\right)\right)\\
                                               &= m\left(s\pi^{-1}\left(E\right)\right)\\
                                               &= m\left(\pi^{-1}\left(E\right)\right)\\
                                               &= \lambda\left(E\right).
      \end{align*}
  \end{enumerate}
\end{proof}
\section{Establishing Amenability through Functional Analysis}\label{sec:functional_analysis_and_amenability}%
Now that we understand some useful properties of means in relation to groups and subgroups, we turn our attention toward finding means on groups. In order to do this, we turn our attention towards the space $\ell_{\infty}\left(G\right)$, which allows us to use theories from functional analysis to better understand means on $G$.
\begin{definition}
  Let $G$ be a group.
  \begin{enumerate}[(1)]
    \item The space $\mathcal{F}\left(G,\R\right)$ is defined by
      \begin{align*}
        \mathcal{F}\left(G,\R\right) &= \set{f | f\colon G\rightarrow \R\text{ is a function}}.
      \end{align*}
    \item A function $f\in \mathcal{F}\left(G,\R\right)$ is called positive if $f(x) \geq 0$ for all $x\in G$.
    \item A function $f\in \mathcal{F}\left(G,\R\right)$ is called simple if $\ran(f)$ is finite. We let
      \begin{align*}
        \Sigma &= \set{f\in \mathcal{F}\left(G,\R\right) | f\text{ is simple}}.
      \end{align*}
  \end{enumerate}
\end{definition}
\begin{fact}
  It is the case that $\Sigma \subseteq \mathcal{F}\left(G,\R\right)$ is a linear subspace.
\end{fact}
\begin{definition}
  For $E\subseteq G$, we define
  \begin{align*}
    \1_{E}\colon G\rightarrow \R
  \end{align*}
  by
  \begin{align*}
    \1_{E}\left(x\right) &= \begin{cases}
      1 & x\in E\\
      0 & x\notin E
    \end{cases}.
  \end{align*}
  This is the characteristic function of $E$.
\end{definition}
\begin{fact}
  We have
  \begin{align*}
    \Span\set{\1_{E}| E\subseteq G} &= \Sigma.
  \end{align*}
\end{fact}
\begin{proof}
  We see that $\1_{E}\in \Sigma$ for any $E\subseteq G$, and that $\Sigma$ is a subspace.\newline

  If $\phi\in \Sigma$ with $\Ran\left(\phi\right) = \set{t_1,\dots,t_n}$, where $t_i$ are distinct, we set
  \begin{align*}
    E_i &= \phi^{-1}\left(\set{t_i}\right),
  \end{align*}
  yielding
  \begin{align*}
    \phi &= \sum_{i=1}^{n}t_i\1_{E_i}.
  \end{align*}
\end{proof}
\begin{definition}\hfill
  \begin{enumerate}[(1)]
    \item A function $f\in \mathcal{F}\left(G,\R\right)$ is bounded if there exists $M > 0$ such that $\Ran\left(f\right) \subseteq \left[-M,M\right]$.
    \item The space $\ell_{\infty}\left(G\right)$ is defined by
      \begin{align*}
        \ell_{\infty}\left(G\right) &= \set{f\in \mathcal{F}\left(G,\R\right)| f\text{ is bounded}}.
      \end{align*}
    \item The norm on $\ell_{\infty}\left(G\right)$ is defined by
      \begin{align*}
        \norm{f} &= \sup_{x\in G}\left\vert f(x) \right\vert.
      \end{align*}
  \end{enumerate}
\end{definition}
\begin{remark}
  Note that this is slightly different from the definition in Definition \ref{def:three_function_spaces}. We mostly focus on the case with codomain $\R$ rather than codomain $\C$ as this more readily translates to the various proofs we will encounter in the rest of the chapter. However, the case of codomain $\R$ and codomain $\C$ are virtually identical.
\end{remark}

\begin{proposition}
  The space $\ell_{\infty}(G)$ is complete. Additionally, $\overline{\Sigma} = \ell_{\infty}\left(G\right)$.
\end{proposition}
\begin{proof}
  Let $\left(f_n\right)_n$ be $\norm{\cdot}$-Cauchy in $\ell_{\infty}\left(G\right)$. Then, for all $x\in G$, it is the case that
  \begin{align*}
    \left\vert f_n(x) - f_m(x) \right\vert &= \left\vert \left(f_n - f_m\right)\left(x\right) \right\vert\\
                                           &\leq \norm{f_n - f_m},
  \end{align*}
  meaning $\left(f_n\left(x\right)\right)_n$ is Cauchy in $\R$. We define $f(x) = \lim_{n\rightarrow\infty}f_n(x)$. We must show that $f\in \ell_{\infty}\left(G\right)$, and $\norm{f_n-f}\rightarrow 0$.\newline

  We have
  \begin{align*}
    \left\vert f(x) \right\vert &= \left\vert \lim_{n\rightarrow\infty}f_n\left(x\right) \right\vert\\
                                &= \lim_{n\rightarrow\infty}\left\vert f_n\left(x\right) \right\vert\\
                                &\leq \limsup_{n\rightarrow\infty}\norm{f_n}\\
                                &\leq C,
  \end{align*}
  as Cauchy sequences are always bounded. Thus, $\sup_{x\in G}\left\vert f(x) \right\vert\leq C$.\newline

  Given $\ve > 0$, we find $N$ such that for all $m,n\geq N$, $\norm{f_n - f_m} \leq \ve$. Thus, for $x\in G$, we have
  \begin{align*}
    \left\vert f_n(x) - f_m(x) \right\vert &\leq \norm{f_n - f_m}\\
                                           &\leq \ve.
  \end{align*}
  Taking $m\rightarrow\infty$, we get $\left\vert f_n(x) - f(x) \right\vert \leq \ve$, for all $n\geq N$, so $\norm{f_n - f}\leq \ve$ for all $n\geq N$.\newline

  For $f\in \ell_{\infty}\left(G\right)$, let $\Ran\left(f\right) \subseteq \left[-M,M\right]$ for some $M > 0$. Let $\ve > 0$. Since $\left[-M,M\right]$ is compact, it is totally bounded, so we can find intervals $I_{1},\dots,I_n$ with $\left[-M,M\right] = \bigsqcup_{k=1}^{n}I_k$, with the length of each $I_k$ less than $\ve$.\newline

  Set $E_k = f^{-1}\left(I_k\right)$. Pick some $t_k\in I_k$. We set
  \begin{align*}
    \phi &= \sum_{i=1}^{n}t_k\1_{E_k}.
  \end{align*}
  Then, it is the case that $\norm{\phi - f} < \ve$.
\end{proof}
\begin{corollary}
  For any $f\in \ell_{\infty}\left(G\right)$, there is a sequence $\left(\phi_n\right)_n$ of simple functions with $\norm{\phi_n -f}\rightarrow 0$. If $f\geq 0$, then we can select $\phi_n\geq 0$.
\end{corollary}
Now that we understand how simple functions relate to $\ell_{\infty}(G)$, we start by defining a translation action on $\ell_{\infty}(G)$, from which we will be able to convert the idea of means into invariant elements of the state space of the dual of $\ell_{\infty}\left(G\right)$.
\begin{proposition}\label{prop:translation_action}
  Let $G$ be a group. There is an action
  \begin{align*}
    \lambda_s\colon G\rightarrow \Isom\left(\ell_{\infty}\left(G\right)\right)
  \end{align*}
  defined by
  \begin{align*}
    \lambda_{s}\left(f\right)\left(t\right) &= f\left(s^{-1}t\right)
  \end{align*}
\end{proposition}
\begin{proof}
  We have
  \begin{align*}
    \lambda_s\left(f + \alpha g\right)\left(t\right) &= \left(f + \alpha g\right) \left(s^{-1}t\right)\\
                                                     &= f\left(s^{-1}t\right) \alpha g\left(s^{-1}t\right)\\
                                                     &= \lambda_s\left(f\right)\left(t\right) + \alpha \lambda_s\left(g\right)\left(t\right)\\
                                                     &= \left(\lambda_s\left(f\right) + \alpha \lambda_s\left(g\right)\right)(t).
  \end{align*}
  Thus, $\lambda_s$ is linear. Additionally,
  \begin{align*}
    \norm{\lambda_s\left(f\right)} &= \sup_{t\in G}\left\vert \lambda_s\left(f\right)\left(t\right) \right\vert\\
                                   &= \sup_{t\in G}\left\vert f\left(s^{-1}t\right) \right\vert\\
                                   &= \norm{f},
  \end{align*}
  and
  \begin{align*}
    \norm{\lambda_s\left(f\right) - \lambda_s\left(g\right)} &= \norm{\lambda_s\left(f-g\right)}\\
                                                             &= \norm{f-g},
  \end{align*}
  meaning $\lambda_s$ is an isometry.\newline

  We have
  \begin{align*}
    \lambda_s\circ \lambda_r\left(f\right)\left(t\right) &= \lambda_r\left(f\right)\left(s^{-1}t\right)\\
                                                         &= \lambda_r\left(r^{-1}s^{-1}t\right)\\
                                                         &= f\left(\left(sr\right)^{-1}t\right)\\
                                                         &= \lambda_{sr}\left(f\right)\left(t\right),
  \end{align*}
  establishing that $\lambda_s\circ \lambda_r = \lambda_{sr}$.\newline

  By a similar process, we find that $\lambda_{s}\left(\1_{E}\right) = \1_{sE}$ for any $E\subseteq G$ and $s\in G$.
\end{proof}
\begin{definition}
  A {state} on $\ell_{\infty}\left(G\right)$ is a continuous linear functional $\mu\in \ell_{\infty}\left(G\right)^{\ast}$ such that the following are true:
  \begin{itemize}
    \item $\mu$ is positive;
    \item $\mu\left(\1_{G}\right) = 1$.
  \end{itemize}
  A state is called left-invariant if
  \begin{align*}
    \mu\left(\lambda_s\left(f\right)\right) = \mu\left(f\right).
  \end{align*}
\end{definition}
\begin{example}
  The evaluation functional, $\delta_x\colon \ell_{\infty}\rightarrow \R$, defined by
  \begin{align*}
    \delta_{x}\left(f\right) &= f(x),
  \end{align*}
  is a state. However, it is not necessarily invariant, as
  \begin{align*}
    \delta_x\left(\lambda_s\left(f\right)\right) &= \lambda_s\left(f\right)\left(x\right)\\
                                                 &= f\left(s^{-1}x\right)\\
                                                 &\neq f(x).
  \end{align*}
  However, we can use the evaluation functional to create an invariant state. If $G$ is finite, we define
  \begin{align*}
    \mu &= \frac{1}{\left\vert G \right\vert} \sum_{x\in G}\delta_x,
  \end{align*}
  which is indeed an invariant state.
\end{example}
We can characterize states slightly differently, which will enable us to show the equivalence between invariant states and means.
\begin{lemma}\label{lemma:characterizing_states}\hfill
  \begin{enumerate}[(1)]
    \item If $\mu$ is a state on $\ell_{\infty}\left(G\right)$, then
      \begin{align*}
        \norm{\mu}_{\op} = 1.
      \end{align*}
    \item If $\mu\in \ell_{\infty}\left(G\right)^{\ast}$ is such that
      \begin{align*}
        \norm{\mu}_{\op} &= \mu\left(\1_{G}\right)\\
                               &= 1,
      \end{align*}
      then $\mu$ is positive and a state.
  \end{enumerate}
\end{lemma}
\begin{proof}\hfill
  \begin{enumerate}[(1)]
    \item Let $\mu$ be a state. Given $f\in \ell_{\infty}\left(G\right)$, we have
      \begin{align*}
        \norm{f}\1_{G} - f &\geq 0\\
        \norm{f}\1_{G} + f &\geq 0,
      \end{align*}
      so
      \begin{align*}
        0 &\leq \mu\left(\norm{f}\1_{G} - f\right) \\
          &= \norm{f}\mu\left(\1_{G}\right) - \mu\left(f\right)
          \intertext{meaning}
        \mu\left(f\right) &\leq \norm{f}.
        \intertext{Additionally,}
        0 &\leq \mu\left(\norm{f}\1_{G} + f\right)\\
          &= \norm{f}\mu\left(\1_{G}\right) + \mu\left(f\right),
          \intertext{meaning}
        -\mu\left(f\right) &\leq \norm{f}.
      \end{align*}
      Thus, we have $\left\vert \mu\left(f\right) \right\vert \leq \norm{f}$, so $\norm{\mu}_{\op} \leq 1$. However, since $\mu\left(\1_{G}\right) = 1$, we must have $\norm{\mu}_{\op} = 1$.
  \item Suppose $\norm{\mu}_{\op} = \mu\left(\1_{G}\right) = 1$. Let $f\geq 0$. Set $g = \frac{1}{\norm{f}_u}f$.\newline

    Then, $\Ran(g) \subseteq [0,1]$, and $\Ran\left(g - \1_{G}\right) \subseteq \left[-1,1\right]$. Thus, $\norm{g - \1_{G}}_{u} \leq 1$.\newline

    Since $\norm{\mu}_{\op} = 1$, we must have
    \begin{align*}
      \left\vert \mu\left(g - \1_{G}\right) \right\vert &\leq 1\\
      \left\vert \mu\left(g\right) - 1 \right\vert &\leq 1,
    \end{align*}
    and since $\mu\left(\1_{G}\right) = 1$, we have $\mu\left(g\right) \in [0,2]$. Thus, $\mu\left(f\right) = \norm{f}\mu\left(g\right) \geq 0$.
  \end{enumerate}
\end{proof}
%To show the equivalence between means and invariant states, we need to be able to characterize the state space on $\ell_{\infty}\left(G\right)^{\ast}$. To do this, we make use of some results from functional analysis.\newline
%
%If $X$ is a normed vector space, then the topology on $X^{\ast}$ induced by $X^{\ast\ast}$ is known as the weak* topology. The weak* topology is the topology of pointwise convergence in $X^{\ast}$ --- a net $\left(\varphi_{\alpha}\right)_{\alpha}$ converges to $\varphi$ in the weak* topology if and only if, for all $\hat{x}\in X^{\ast\ast}$, we have
%\begin{align*}
%  \left(\hat{x}\left(\varphi_{\alpha}\right)\right)_{\alpha}\rightarrow \hat{x}\left(\varphi\right),
%\end{align*}
%or by the definition of $X^{\ast\ast}$,
%\begin{align*}
%  \left(\varphi_{\alpha}\left(x\right)\right) \rightarrow \varphi\left(x\right)
%\end{align*}
%for all $x\in X$.\newline
%
%We state some important results in functional analysis here. The proofs of these results can be found in functional analysis textbooks such as \cite{rudin_functional_analysis}.
%\begin{theorem}[Hahn--Banach Continuous Extension Theorem]
%  Let $X$ be a normed vector space, $E\subseteq X$ a subspace, and $\varphi\in E^{\ast}$ a bounded linear functional. Then, there exists a continuous $\psi\in X^{\ast}$ such that $\norm{\varphi}_{\op} = \norm{\psi}_{\op}$, and $\psi|_{E} = \varphi$.
%\end{theorem}
%\begin{theorem}[Hahn--Banach Separation Theorems]
%  Let $X$ be a normed vector space.
%  \begin{enumerate}[(1)]
%    \item Given a nonzero $x_0\in X$, there is a $\varphi\in X^{\ast}$ with $\norm{\varphi}_{\op} = 1$ and $\varphi\left(x_0\right) = \norm{x}$. We call $\varphi$ a norming functional.
%    \item Given a proper closed subspace $E\subseteq X$ and $x_0\in X\setminus E$, there is a $\varphi\in X^{\ast}$ such that $\varphi|_{E} = 0$, $\norm{\varphi}_{\op} = 1$, and $\varphi\left(x\right) = \dist_{E}(x)$ for all $x\in X$.
%  \end{enumerate}
%\end{theorem}
%\begin{theorem}[Banach--Alaoglu Theorem]
%  Let $X$ be a normed vector space.
%  \begin{enumerate}[(1)]
%    \item The closed unit ball in the dual space, $B_{X^{\ast}}$, is compact in the $w^{\ast}$ topology.
%    \item A subset $C\subseteq X$ is $w^{\ast}$-compact if and only if $C$ is $w^{\ast}$-closed and norm bounded.
%  \end{enumerate}
%\end{theorem}
\begin{corollary}
  The set of states in $\ell_{\infty}\left(G\right)^{\ast}$ forms a $w^{\ast}$-compact subset of $B_{\ell_{\infty}\left(G\right)^{\ast}}$.
\end{corollary}
\begin{proof}
  From the Banach--Alaoglu Theorem (Theorem \ref{thm:banach_alaoglu}), we only need to show that the set of states, $S\left(\ell_{\infty}\left(G\right)\right)$, is $w^{\ast}$-closed, as every element of $S\left(\ell_{\infty}\left(G\right)\right)$ has norm $1$.\newline

  Let $f\in \ell_{\infty}\left(G\right)$ be positive, and let $\left(\varphi_{i}\right)_i$ be a net in $S\left(\ell_{\infty}\left(G\right)\right)$ with $\left(\varphi_{i}\right)_i\xrightarrow{w^{\ast}} \varphi\in \ell_{\infty}\left(G\right)^{\ast}$. From Lemma \ref{lemma:characterizing_states}, we must show that $\varphi$ is positive and $\varphi\left(\1_{G}\right) = 1$.\newline

  We start by seeing that, since each $\varphi_i$ is a state, we have $\varphi_{i}\left(f\right) \geq 0$ for each $i\in I$, so we must have $\varphi\left(f\right) \geq 0$.\newline

  Similarly, since $\varphi_{i}\left(\1_{G}\right) = 1$ for each $i\in I$, and $\left(\varphi_i\right)_i \xrightarrow{w^{\ast}} \varphi$, we have $\varphi\left(\1_{G}\right) = 1$. Thus, by Lemma \ref{lemma:characterizing_states}, we have that $S\left(\ell_{\infty}\left(G\right)\right)$ is $w^{\ast}$-closed.
\end{proof}

Now, we may show the correspondence between invariant states and means.
\begin{proposition}\label{prop:state_implies_mean}
  If $\mu\in \ell_{\infty}\left(G\right)^{\ast}$ is a state, then $m\colon P(G)\rightarrow [0,1]$ defined by $m(E) = \mu\left(\1_{E}\right)$ is a finitely additive probability measure on $G$.\newline

  Moreover, if $\mu$ is invariant, then $m$ is a mean.
\end{proposition}
\begin{proof}
  We have
  \begin{align*}
    m\left(G\right) &= \mu\left(\1_{G}\right)\\
                    &= 1\\
                    \\
    m\left(\emptyset\right) &= \mu\left(0\right)\\
                            &= 0\\
                            \\
    m\left(E\sqcup F\right) &= \mu\left(\1_{E\sqcup F}\right)\\
                            &= \mu\left(\1_{E} + \1_{F}\right)\\
                            &= \mu\left(\1_{E}\right) + \mu\left(\1_{F}\right)\\
                            &= m\left(E\right) + m\left(F\right).
  \end{align*}
  Additionally, since $0 \leq \1_{E}\leq \1_{G}$, we have $0 \leq \mu\left(\1_{E}\right) \leq 1$, so $0 \leq m(E) \leq 1$.\newline

  If $\mu$ is invariant, then
  \begin{align*}
    m\left(sE\right) &= \mu\left(\1_{sE}\right)\\
                     &= \mu\left(\lambda_s\left(\1_{E}\right)\right)\\
                     &= \mu\left(\1_{E}\right)\\
                     &= m\left(E\right).
  \end{align*}
\end{proof}
\begin{proposition}\label{prop:mean_implies_state}
  If $G$ admits a mean, then $\ell_{\infty}\left(G\right)^{\ast}$ admits an invariant state.
\end{proposition}
\begin{proof}
  Let $m$ be a mean. Define $\mu_0\colon \Sigma\rightarrow \R$ by
  \begin{align*}
    \mu_0\left(\sum_{k=1}^{n}t_k\1_{E_k}\right) &= \sum_{k=1}^{n}t_km\left(E_k\right).
  \end{align*}
  Since $m$ is finitely additive, it is the case that $\mu_0$ is well-defined, linear, and positive, with $\mu_0\left(\1_{G}\right) = m\left(G\right) = 1$.\newline

  Additionally, since $m$ is a mean, then for $f = \sum_{k=1}^{n}t_k\1_{E_k}$, we have
  \begin{align*}
    \mu_0\left(\lambda_s\left(f\right)\right) &= \mu_0\left(\lambda_s\left(\sum_{k=1}^{n}t_k\1_{E_k}\right)\right)\\
                                              &= \mu_0\left(\sum_{k=1}^{n}t_k\1_{sE_k}\right)\\
                                              &= \sum_{k=1}^{n}t_km\left(sE_k\right)\\
                                              &= \sum_{k=1}^{n}t_km\left(E_k\right)\\
                                              &= \mu_0\left(f\right).
  \end{align*}
  We see that
  \begin{align*}
    \left\vert \mu_0\left(f\right) \right\vert &= \left\vert \sum_{k=1}^{n}t_km\left(E_k\right) \right\vert\\
                                               &\leq \sum_{k=1}^{n}\left\vert t_k \right\vert m\left(E_k\right)\\
                                               &\leq \sum_{k=1}^{n}\norm{f}\sum_{k=1}^{n}m\left(E_k\right)\\
                                               &= \norm{f}\sum_{k=1}^{n}m\left(E_k\right)\\
                                               &\leq \norm{f},
  \end{align*}
  meaning $\mu_0$ is continuous, so $\mu_0$ is uniformly continuous.\newline

  Since $\overline{\Sigma} = \ell_{\infty}\left(G\right)$, uniform continuity provides that $\mu_0$ extends to a continuous linear functional $\mu\colon \ell_{\infty}\left(G\right)\rightarrow \R$ with $\mu\left(\1_{G}\right) = \mu_0\left(\1_{G}\right) = 1$.\newline

  For $f\geq 0$, we find a sequence $\left(\phi_n\right)_n$ in $\Sigma$ with $\phi_n\geq 0$ and $\norm{\phi_n - f} \xrightarrow{n\rightarrow\infty}0$. We set
  \begin{align*}
    \mu\left(f\right) &= \lim_{n\rightarrow\infty}\mu\left(\phi_n\right)\\
                      &= \lim_{n\rightarrow\infty}\mu_0\left(\phi_n\right)\\
                      &\geq 0,
  \end{align*}
  so $\mu$ is a state.\newline

  If $f\in \ell_{\infty}\left(G\right)$, $s\in G$, and $\left(\phi_n\right)_n$ a sequence in $\Sigma$ with $\left(\phi_n\right)_n\rightarrow f$, then
  \begin{align*}
    \norm{\lambda_s\left(\phi_n\right) - \lambda_s\left(f\right)} &= \norm{\lambda_s\left(\phi_n - f\right)}\\
                                                                  &= \norm{\phi_n - f}\\
                                                                  &\rightarrow 0.
  \end{align*}
  Thus, we have
  \begin{align*}
    \mu\left(\lambda_s\left(\phi_n\right)\right) &= \mu_0\left(\lambda_s\left(\phi_n\right)\right)\\
                                                 &= \mu_0\left(\phi_n\right)\\
                                                 &= \mu\left(\phi_n\right)\\
                                                 &\rightarrow \mu\left(f\right),
  \end{align*}
  so $\mu\left(f\right) = \mu\left(\lambda_s\left(f\right)\right)$. Thus, $\mu\in \ell_{\infty}\left(G\right)^{\ast}$ is an invariant state.
\end{proof}
\section{Establishing Amenability using Invariant States}\label{sec:amenability_invariant_states}%
Owing to the correspondence between invariant states and means, we are now able to establish amenability for large classes of groups.
\begin{proposition}
  The group of integers, $\Z$, is amenable.
\end{proposition}
\begin{proof}
  We define the left shift, $\lambda_1\colon \ell_{\infty}\left(\Z\right) \rightarrow \ell_{\infty}\left(\Z\right)$, by
  \begin{align*}
    \lambda_1\left(f\right)\left(k\right) &= f\left(k-1\right).
  \end{align*}
  This is an action as in Proposition \ref{prop:translation_action}. \newline

  We set $Y = \Ran\left(\id - \lambda_1\right)\subseteq \ell_{\infty}\left(\Z\right)$. We claim that $\dist_{Y}\left(\1_{\Z}\right) \geq 1$.\newline

  Suppose toward contradiction that there is $y\in Y$ with $\norm{\1_{\Z} - y}_{u} = r < 1$. Then, $y = f - \lambda_1 f$ for some $f\in \ell_{\infty}(\Z)$, so
  \begin{align*}
    \norm{\1_{\Z} - \left(f - \lambda_1\left(f\right)\right)} &= r.
  \end{align*}
  Thus, for all $k\in\Z$, we have
  \begin{align*}
    \left\vert 1 - \left(f(k) - f(k-1)\right) \right\vert &\leq r,
  \end{align*}
  so $\left\vert f(k) - f\left(k-1\right) \right\vert \geq 1-r > 0$. However, such an $f$ cannot be bounded.\newline

  Since $\dist_{\overline{Y}}\left(\1_{\Z}\right) = \dist_{Y}\left(\1_{\Z}\right)$, the Hahn--Banach separation theorems provide $\mu\in \left(\ell_{\infty}\left(\Z\right)\right)^{\ast}$ with $\norm{\mu}_{\op} = 1$, $\mu|_{\overline{Y}} = 0$, and $\mu\left(\1_{\Z}\right) = \dist_{Y}\left(\1_{\Z}\right) \geq 1$.\newline

  Since $\norm{\mu}_{\op} = 1$ and $\mu\left(\1_{\Z}\right) \geq 1$, we must have $\mu\left(\1_{\Z}\right) = 1$.\newline

  Additionally, since $\norm{\mu}_{\op} = \mu\left(\1_{\Z}\right) = 1$, we have that $\mu$ is a state on $\ell_{\infty}\left(\Z\right)$, and since $\mu\left(y\right) = 0$ for all $y\in Y$, we have
  \begin{align*}
    \mu\left(f - \lambda_1\left(f\right)\right) &= 0\\
    \mu\left(f\right) &= \mu\left(\lambda_1\left(f\right)\right).
  \end{align*}
  Inductively, this means that $\mu\left(f\right) = \mu\left(\lambda_k\left(f\right)\right)$ for all $k\in \Z$, so $\mu$ is an invariant state on $\ell_{\infty}\left(\Z\right)$. Thus, $\Z$ is amenable.
\end{proof}
\begin{proposition}\label{prop:normal_subgroups_quotient_groups_amenability}
  If $N\trianglelefteq G$ and $G/N$ are amenable, then $G$ is amenable.
\end{proposition}
\begin{proof}
  Let $\rho\in \left(\ell_{\infty}\left(G/N\right)\right)^{\ast}$ be an invariant state, and let $p\colon P(N)\rightarrow [0,1]$ be a mean. For $E\subseteq G$, we define $f_E\colon G/N\rightarrow \R$ by
  \begin{align*}
    f_E\left(tN\right) &= p\left(N\cap t^{-1}E\right).
  \end{align*}
  We start by verifying that $f_E$ is well-defined. For $tN = sN$, we have $s^{-1}t\in N$, so
  \begin{align*}
    p\left(N\cap t^{-1}E\right) &= p\left(s^{-1}t\left(N\cap t^{-1}E\right)\right)\\
                                &= p\left(s^{-1}tN \cap s^{-1}E\right)\\
                                &= p\left(N\cap s^{-1}E\right).
  \end{align*}
  Since $f_E$ is defined through $p$, we can see that $f_E$ is bounded. Additionally,
  \begin{align*}
    f_{E\sqcup F}\left(tN\right) &= p\left(N\cap t^{-1}\left(E\sqcup F\right)\right)\\
                                 &= p\left(N\cap \left(t^{-1}E\sqcup t^{-1}F\right)\right)\\
                                 &= p\left(\left(N\cap t^{-1}E\right) \sqcup \left(N\cap t^{-1}F\right)\right)\\
                                 &= p\left(N\cap t^{-1}E\right) + p\left(N\cap t^{-1}F\right)\\
                                 &= f_E\left(tN\right) + f_F\left(tN\right)\\
                                 &= \left(f_E + f_F\right)\left(tN\right),
  \end{align*}
  and
  \begin{align*}
    f\left(sE\right) \left(tN\right) &= p\left(N\cap t^{-1}sE\right)\\
                                     &= f_E\left(s^{-1}tN\right)\\
                                     &= \lambda_{sN}\left(f_E\right)\left(tN\right),
  \end{align*}
  so $f_{sE} = \lambda_{sN}\left(f_E\right)$. Finally,
  \begin{align*}
    f_G\left(tN\right) &= p\left(N\cap t^{-1}G\right)\\
                       &=p\left(N\right)\\
                       &= 1,
  \end{align*}
  meaning $f_G = \1_{G/N}$.\newline

  We define $m\colon P(G)\rightarrow [0,1]$ by
  \begin{align*}
    m(E) &= \rho\left(f_E\right).
  \end{align*}
  Then, we have
  \begin{align*}
    m\left(E\sqcup F\right) &= m(E) + m(F)\\
                            \\
    m\left(G\right) &= 1\\
    \\
    m\left(sE\right) &= \rho\left(f_{sE}\right)\\
                     &= \rho\left(\lambda_{sN}\left(f_{E}\right)\right)\\
                     &= \rho\left(f_E\right)\\
                     &= m(E),
  \end{align*}
  so $m$ is a mean.
\end{proof}
\begin{corollary}
  The finite direct product of amenable groups is amenable.
\end{corollary}
\begin{proof}
  If $H$ and $K$ are amenable, then $K\cong \left(H\times K\right)/H$ is amenable and $H$ is amenable, so $H\times K$ is amenable by Proposition \ref{prop:normal_subgroups_quotient_groups_amenability}. Induction provides the general case.
\end{proof}
\begin{corollary}\label{cor:finitely_generated_amenable}
  Finitely generated abelian groups are amenable.
\end{corollary}
\begin{proof}
  By the fundamental theorem of finitely generated abelian groups (see \cite[158]{dummit_and_foote}), all finitely generated abelian groups are isomorphic to $\Z^{d}\times \Z/n_1\Z\times\cdots\times \Z/{n_k}\Z$.\newline

  Since $\Z^{d}$ is a finite direct product of $\Z$, and the torsion subgroup $\Z/n_1\Z\times\cdots\times \Z/n_k\Z$ is finite, we see that a finitely generated abelian group is a direct product of two amenable groups, hence amenable.
\end{proof}
\begin{corollary}\label{cor:direct_limit_amenable}
  If $\set{G_i}_{i\in I}$ is a directed family of amenable groups, then the direct limit,
  \begin{align*}
    G &= \bigcup_{i\in I}G_i,
  \end{align*}
  is also amenable.
\end{corollary}
\begin{proof}
  Let $\mu_i\in \left(\ell_{\infty}\left(G_i\right)\right)^{\ast}$ be invariant states.\newline

  Set
  \begin{align*}
    M_i &= \set{\mu\in S\left(\ell_{\infty}\left(G\right)\right)| \mu\left(\lambda_s\left(f\right)\right) = \mu\left(f\right)\text{ for all }s\in G_i}.
  \end{align*}
  We set $\mu\left(f\right) = \mu_i\left(f|_{G_i}\right)$. Since each $G_i$ is amenable, it is the case that each $M_i$ is nonempty. Similarly, seeing as we have established the state space as $w^{\ast}$-closed in $B_{\ell_{\infty}\left(G\right)^{\ast}}$, it is the case that each $M_i$ is $w^{\ast}$-closed in $B_{\ell_{\infty}\left(G\right)^{\ast}}$.\newline

  For $i_1,\dots,i_n$, we find $G_j \supseteq G_{i_1},\dots,G_{i_n}$, which exists since $\set{G_i}_{i\in I}$ is directed. We have that $M_j\subseteq \bigcap_{k=1}^{n}M_{i_k}$, so $\set{M_i}_{i\in I}$ has the finite intersection property.\newline

  By compactness, there is $\mu\in \bigcap_{i\in I}M_i$ which is necessarily invariant on $G$.
\end{proof}
\begin{corollary}\label{cor:abelian_groups_amenable}
  All abelian groups are amenable.
\end{corollary}
\begin{proof}
  Every abelian group is the direct limit of its finitely generated subgroups.
\end{proof}
\begin{corollary}\label{cor:solvable_groups_amenable}
  All solvable groups are amenable.
\end{corollary}
\begin{proof}
  Let $e_G = G_0 \leq G_1\leq\cdots\leq G_n\leq G$ be such that $G_{j-1}\trianglelefteq G_j$ for $j=1,\dots,n$, and $G_i/G_j$ is abelian.\newline

  Since $G_0$ is abelian, it is amenable, as is $G_1/G_0$, so $G_1$ is amenable. We see then that $G_2$ is amenable as $G_1$ and $G_2/G_1$ are amenable.\newline

  Continuing in this fashion, we see that $G$ is amenable.
\end{proof}
\section{Remarks and Notes}\label{sec:invariant_states_remarks}%
The following proposition is, in a sense, a kind of converse to Proposition \ref{prop:subgroups_quotientgroups_amenability}, in that if a subgroup is amenable, we can show that the original group is also amenable, but this is only a sufficient condition if the subgroup has finite index.
\begin{proposition}\label{prop:finite_index_amenable_subgroup}
  Let $G$ be a group, and let $H\leq G$ be amenable, with $\left[G:H\right]  = n < \infty$. Then, $G$ is amenable.
\end{proposition}
\begin{proof}
  Let $H\leq G$ be amenable with $\left[G:H\right] = n$. Let $\mu$ be the mean on $H$, and let $\set{g_iH}_{i=1}^{n}$ be a partition of $G$ by the left cosets of $H$. We define the mean on $G$ by taking, for $A\subseteq G$,
  \begin{align*}
    \lambda\left(A\right) &= \frac{1}{n}\sum_{i=1}^{n}\mu\left(g_i^{-1}A\cap H\right).
  \end{align*}
  We begin by verifying that this is well-defined. Specifically, we will show that this definition is independent of the coset representatives. Suppose $g_jH = h_j H$. Then, $h_j^{-1}g_j \in H$. Now, we have $g_j^{-1}A \cap H \subseteq H$, so by left-multiplication, we get $\left(h_j^{-1}g_j\right)g_j^{-1}A\cap H \subseteq H$, so $h_j^{-1}A\cap H\subseteq H$. Since $\set{g_i H}_{i=1}^{n}$ is a partition, we get that this definition of the mean on $G$ is independent of the choice of coset representatives.\newline

  Next, we show that this is a finitely additive measure. Let $A,B\subseteq G$ be such that $A\cap B = \emptyset$. Then, we get
  \begin{align*}
    \lambda\left(A\sqcup B\right) &= \frac{1}{n}\sum_{i=1}^{n}\mu\left(g_i^{-1}\left(A\sqcup B\right)\cap H\right)\\
                                  &= \frac{1}{n}\sum_{i=1}^{n}\mu\left(\left(g_i^{-1}A\cap H\right)\sqcup \left(g_i^{-1}B\cap H\right)\right)\\
                                  &= \frac{1}{n}\left(\sum_{i=1}^{n}\mu\left(g_i^{-1}A\cap H\right) + \sum_{i=1}^{n}\mu\left(g_i^{-1}B\cap H\right)\right)\\
                                  &= \frac{1}{n}\sum_{i=1}^{n}\mu\left(g_i^{-1}A\cap H\right) + \frac{1}{n}\sum_{i=1}^{n}\mu\left(g_i^{-1}B\cap H\right)\\
                                  &= \lambda\left(A\right) + \lambda\left(B\right).
  \end{align*}
  It is relatively simple to see that $\lambda$ is a probability measure, as
  \begin{align*}
    \lambda\left(G\right) &= \frac{1}{n}\sum_{i=1}^{n}\mu\left(g_i^{-1}G\cap H\right)\\
                          &= \frac{1}{n}\sum_{i=1}^{n}\mu\left(G\cap H\right)\\
                          &= \frac{1}{n}\sum_{i=1}^{n}\mu\left(H\right)\\
                          &= 1.
  \end{align*}
  Now, we must show that $\lambda$ is translation-invariant. Let $A\subseteq G$ and $t\in G$. Then, using the translation-invariance of $\mu$, we get
  \begin{align*}
    \lambda\left(tA\right) &= \frac{1}{n}\sum_{i=1}^{n}\mu\left(g_i^{-1}tA\cap H\right)\\
                           &= \frac{1}{n}\sum_{i=1}^{n}\mu\left(g_i^{-1}\left(t\left(A\cap H\right)\right)\right)\\
                           &= \frac{1}{n}\sum_{i=1}^{n}\mu\left(g_i^{-1}A\cap H\right)\\
                           &= \lambda\left(A\right).
  \end{align*}
  Thus, $G$ is amenable.
\end{proof}
In \cite{free_subgroups_of_linear_groups}, Jacques Tits proved that in any subgroup of $\text{GL}_n\left(\F\right)$ (where $\F$ is any field with characteristic zero), then the subgroup either admits a solvable subgroup of finite index (hence amenable by Corollary \ref{cor:solvable_groups_amenable} and Proposition \ref{prop:finite_index_amenable_subgroup}) or contains a non-abelian freely generated subgroup (which is necessarily not amenable by Theorem \ref{thm:tarski}). This is also known as the Tits alternative.\newline

Note that we have found a sufficient condition to find a (non-abelian) freely generated subgroup by Theorem \ref{thm:ping_pong}, and showed in Theorem \ref{thm:free_group_so3} that $\text{SO}(3)$ contains a (necessarily) non-abelian freely generated subgroup (in this case, isomorphic to the free group with 2 generators), so we know by the Tits alternative that $\text{SO}(3)$ does not admit a solvable subgroup with a finite index. In the introduction to Chapter 3, we stated that the Banach--Tarski paradox cannot hold for $\R$ and $\R^2$, because the rotation group $\text{SO}\left(2\right)$ in $\R^2$ is abelian, and since the isometry group $\text{E}\left(2\right)$ has the abelian subgroup $\text{SO}\left(2\right)$ with finite index, $\text{E}\left(2\right)$ is amenable by Proposition \ref{prop:finite_index_amenable_subgroup}. Similarly, the isometry group $\text{E}(1)$ contains an abelian subgroup $\text{SO}(1)$\footnote{$SO(1) = \set{1}$.} with finite index.

% need snappy name for this too
\chapter{Close Enough: Approximate Means and Følner's Condition}\label{ch:folner_condition}
Having proven Tarski's theorem, we can turn our attention to a more definite understanding of amenability. We will use theorems and techniques from functional analysis to help understand the space $\ell_{\infty}\left(G\right)$, which will open a wide variety of characterizations for amenability, beyond that which was established in Tarski's theorem.
\section{Means and Invariant States}%
\begin{definition}
  Let $G$ be a group, with $P(G)$ denoting its power set.\newline

  An invariant {mean} on $G$ is a set function $m\colon P(G)\rightarrow [0,1]$ which satisfies, for all $t\in G$ and $E,F\subseteq G$,
  \begin{itemize}
    \item $m(G) = 1$;
    \item $m\left(E\sqcup F\right) = m(E) + m(F)$;
    \item $m\left(tE\right) = m\left(E\right)$.
  \end{itemize}
  We say $G$ is amenable if $G$ admits a mean.\newline

  The mean $m$ is, in other words, a translation-invariant probability measure on the measurable space $\left(G,P(G)\right)$.
\end{definition}
We have shown in the proof of Theorem \ref{thm:tarski} that an equivalent condition for amenability is that the group is not paradoxical.\newline

Using some essential results in group theory, we can establish some preliminary results on subgroups and quotient groups.
\begin{proposition}\label{prop:subgroups_quotientgroups_amenability}
  Let $G$ be an amenable group with $H\leq G$. Then, the following are true:
  \begin{enumerate}[(1)]
    \item $H$ is amenable;
    \item for $H\trianglelefteq G$, $G/H$ is amenable.
  \end{enumerate}
\end{proposition}
\begin{proof}\hfill
  \begin{enumerate}[(1)]
    \item Let $R$ be a right transversal for $H$, wherein we select one element of each right coset of $H$ to make up $R$.\newline

      If $m$ is a mean for $G$, we set $\lambda\colon P(H)\rightarrow [0,1]$ defined by
      \begin{align*}
        \lambda(E) = m\left(ER\right).
      \end{align*}
       We have
      \begin{align*}
        \lambda(H) &= m\left(HR\right)\\
                   &= m\left(G\right)\\
                   &= 1.
      \end{align*}
      We claim that if $E\cap F = \emptyset$, then $ER \cap FR = \emptyset$. Suppose toward contradiction this is not the case. Then, $xr_1 = yr_2$ for some $x\in E$, $y\in F$, and $r_1,r_2\in R$. Then, we must have $r_2r_1^{-1} = y^{-1}x \in H$, meaning $r_1 = r_2$ as, by definition, $R$ contains exactly one element of each right coset. Thus, $x=y$, so $E\cap F \neq \emptyset$.\newline

      We then have
      \begin{align*}
        \lambda\left(E\sqcup F\right) &= m\left(\left(E\sqcup F\right)R\right)\\
                                      &= m\left(ER\sqcup FR\right)\\
                                      &= m\left(ER\right) + m\left(FR\right)\\
                                      &= \lambda\left(E\right) + \lambda\left(F\right),
      \end{align*}
      and
      \begin{align*}
        \lambda\left(sE\right) &= m\left(sER\right)\\
                               &= m\left(ER\right)\\
                               &= \lambda\left(E\right).
      \end{align*}
    \item Let $\pi\colon G\rightarrow G/H$ be the canonical projection, defined by $\pi\left(t\right) = tH$. We define
      \begin{align*}
        \lambda\colon P\left(G/H\right) \rightarrow [0,1]
      \end{align*}
      by $\lambda(E) = m\left(\pi^{-1}\left(E\right)\right)$. We have
      \begin{align*}
        \lambda\left(G/H\right) &= m\left(\pi^{-1}\left(G/H\right)\right)\\
                                &= m\left(G\right)\\
                                &= 1,
      \end{align*}
      and
      \begin{align*}
        \lambda\left(E\sqcup F\right) &= m\left(\pi^{-1}\left(E\sqcup F\right)\right)\\
                                      &= m\left(\pi^{-1}\left(E\right)\sqcup \pi^{-1}\left(F\right)\right)\\
                                      &= m\left(\pi^{-1}\left(E\right)\right) + m\left(\pi^{-1}\left(F\right)\right)\\
                                      &= \lambda(E) + \lambda(F).
      \end{align*}
      To show translation-invariance, we let $sH = \pi(s)\in G/H$, and $E\subseteq G/H$. Note that
      \begin{align*}
        \pi^{-1}\left(\pi(s)E\right) &= s\pi^{-1}\left(E\right),
      \end{align*}
      since for $r\in s\pi^{-1}(E)$, we have $r = st$ for $t\in \pi(E)$, so $\pi\left(r\right) =\pi\left(st\right) = \pi\left(s\right)\pi\left(t\right)\in \pi\left(s\right)E$.\newline

      Additionally, if $r\in \pi^{-1}\left(\pi(s)E\right)$, we have $\pi(r)\in \pi(s)E$, so $\pi\left(s^{-1}r\right)\in E$, meaning $s^{-1}r\in \pi^{-1}E$.\newline

      Thus,
      \begin{align*}
        \lambda\left(\pi\left(s\right)E\right) &= m\left(\pi^{-1}\left(\pi\left(s\right)E\right)\right)\\
                                               &= m\left(s\pi^{-1}\left(E\right)\right)\\
                                               &= m\left(\pi^{-1}\left(E\right)\right)\\
                                               &= \lambda\left(E\right).
      \end{align*}
  \end{enumerate}
\end{proof}

Now that we understand some useful properties of means in relation to groups and subgroups, we turn our attention toward finding means on groups. In order to do this, we turn our attention towards the space $\ell_{\infty}\left(G\right)$, which allows us to use theories from functional analysis to better understand means on $G$.
\begin{definition}
  Let $G$ be a group.
  \begin{enumerate}[(1)]
    \item The space $\mathcal{F}\left(G,\R\right)$ is defined by
      \begin{align*}
        \mathcal{F}\left(G,\R\right) &= \set{f | f\colon G\rightarrow \R\text{ is a function}}.
      \end{align*}
    \item A function $f\in \mathcal{F}\left(G,\R\right)$ is called positive if $f(x) \geq 0$ for all $x\in G$.
    \item A function $f\in \mathcal{F}\left(G,\R\right)$ is called simple if $\ran(f)$ is finite. We let
      \begin{align*}
        \Sigma &= \set{f\in \mathcal{F}\left(G,\R\right) | f\text{ is simple}}.
      \end{align*}
  \end{enumerate}
\end{definition}
\begin{fact}
  It is the case that $\Sigma \subseteq \mathcal{F}\left(G,\R\right)$ is a linear subspace.
\end{fact}
\begin{definition}
  For $E\subseteq G$, we define
  \begin{align*}
    \1_{E}\colon G\rightarrow \R
  \end{align*}
  by
  \begin{align*}
    \1_{E}\left(x\right) &= \begin{cases}
      1 & x\in E\\
      0 & x\notin E
    \end{cases}.
  \end{align*}
  This is the characteristic function of $E$.
\end{definition}
\begin{fact}
  We have
  \begin{align*}
    \Span\set{\1_{E}| E\subseteq G} &= \Sigma.
  \end{align*}
\end{fact}
\begin{proof}
  We see that $\1_{E}\in \Sigma$ for any $E\subseteq G$, and that $\Sigma$ is a subspace.\newline

  If $\phi\in \Sigma$ with $\Ran\left(\phi\right) = \set{t_1,\dots,t_n}$, where $t_i$ are distinct, we set
  \begin{align*}
    E_i &= \phi^{-1}\left(\set{t_i}\right),
  \end{align*}
  yielding
  \begin{align*}
    \phi &= \sum_{i=1}^{n}t_i\1_{E_i}.
  \end{align*}
\end{proof}
\begin{definition}
  \begin{enumerate}[(1)]
    \item A function $f\in \mathcal{F}\left(G,\R\right)$ is bounded if there exists $M > 0$ such that $\Ran\left(f\right) \subseteq \left[-M,M\right]$.
    \item The space $\ell_{\infty}\left(G\right)$ is defined by
      \begin{align*}
        \ell_{\infty}\left(G\right) &= \set{f\in \mathcal{F}\left(G,\R\right)| f\text{ is bounded}}.
      \end{align*}
    \item The norm on $\ell_{\infty}\left(G\right)$ is defined by
      \begin{align*}
        \norm{f} &= \sup_{x\in G}\left\vert f(x) \right\vert.
      \end{align*}
  \end{enumerate}
\end{definition}
\begin{proposition}
  The space $\ell_{\infty}(G)$ is complete. Additionally, $\overline{\Sigma} = \ell_{\infty}\left(G\right)$.
\end{proposition}
\begin{proof}
  Let $\left(f_n\right)_n$ be $\norm{\cdot}$-Cauchy in $\ell_{\infty}\left(G\right)$. Then, for all $x\in G$, it is the case that
  \begin{align*}
    \left\vert f_n(x) - f_m(x) \right\vert &= \left\vert \left(f_n - f_m\right)\left(x\right) \right\vert\\
                                           &\leq \norm{f_n - f_m},
  \end{align*}
  meaning $\left(f_n\left(x\right)\right)_n$ is Cauchy in $\R$. We define $f(x) = \lim_{n\rightarrow\infty}f_n(x)$. We must show that $f\in \ell_{\infty}\left(G\right)$, and $\norm{f_n-f}\rightarrow 0$.\newline

  We have
  \begin{align*}
    \left\vert f(x) \right\vert &= \left\vert \lim_{n\rightarrow\infty}f_n\left(x\right) \right\vert\\
                                &= \lim_{n\rightarrow\infty}\left\vert f_n\left(x\right) \right\vert\\
                                &\leq \limsup_{n\rightarrow\infty}\norm{f_n}\\
                                &\leq C,
  \end{align*}
  as Cauchy sequences are always bounded. Thus, $\sup_{x\in G}\left\vert f(x) \right\vert\leq C$.\newline

  Given $\ve > 0$, we find $N$ such that for all $m,n\geq N$, $\norm{f_n - f_m} \leq \ve$. Thus, for $x\in G$, we have
  \begin{align*}
    \left\vert f_n(x) - f_m(x) \right\vert &\leq \norm{f_n - f_m}\\
                                           &\leq \ve.
  \end{align*}
  Taking $m\rightarrow\infty$, we get $\left\vert f_n(x) - f(x) \right\vert \leq \ve$, for all $n\geq N$, so $\norm{f_n - f}\leq \ve$ for all $n\geq N$.\newline

  For $f\in \ell_{\infty}\left(G\right)$, let $\Ran\left(f\right) \subseteq \left[-M,M\right]$ for some $M > 0$. Let $\ve > 0$. Since $\left[-M,M\right]$ is compact, it is totally bounded, so we can find intervals $I_{1},\dots,I_n$ with $\left[-M,M\right] = \bigsqcup_{k=1}^{n}I_k$, with the length of each $I_k$ less than $\ve$.\newline

  Set $E_k = f^{-1}\left(I_k\right)$. Pick some $t_k\in I_k$. We set
  \begin{align*}
    \phi &= \sum_{i=1}^{n}t_k\1_{E_k}.
  \end{align*}
  Then, it is the case that $\norm{\phi - f} < \ve$.
\end{proof}
\begin{corollary}
  For any $f\in \ell_{\infty}\left(G\right)$, there is a sequence $\left(\phi_n\right)_n$ with $\norm{\phi_n -f}\rightarrow 0$. If $f\geq 0$, then we can select $\phi_n\geq 0$.
\end{corollary}
Now that we understand how simple functions relate to $\ell_{\infty}(G)$, we start by defining a translation action on $\ell_{\infty}(G)$, from which we will be able to convert the idea of means into invariant elements of the state space of the dual of $\ell_{\infty}\left(G\right)$.
\begin{proposition}\label{prop:translation_action}
  Let $G$ be a group. There is an action
  \begin{align*}
    \lambda_s\colon G\rightarrow \Isom\left(\ell_{\infty}\left(G\right)\right)
  \end{align*}
  defined by
  \begin{align*}
    \lambda_{s}\left(f\right)\left(t\right) &= f\left(s^{-1}t\right)
  \end{align*}
\end{proposition}
\begin{proof}
  We have
  \begin{align*}
    \lambda_s\left(f + \alpha g\right)\left(t\right) &= \left(f + \alpha g\right) \left(s^{-1}t\right)\\
                                                     &= f\left(s^{-1}t\right) \alpha g\left(s^{-1}t\right)\\
                                                     &= \lambda_s\left(f\right)\left(t\right) + \alpha \lambda_s\left(g\right)\left(t\right)\\
                                                     &= \left(\lambda_s\left(f\right) + \alpha \lambda_s\left(g\right)\right)(t).
  \end{align*}
  Thus, $\lambda_s$ is linear. Additionally,
  \begin{align*}
    \norm{\lambda_s\left(f\right)} &= \sup_{t\in G}\left\vert \lambda_s\left(f\right)\left(t\right) \right\vert\\
                                   &= \sup_{t\in G}\left\vert f\left(s^{-1}t\right) \right\vert\\
                                   &= \norm{f},
  \end{align*}
  and
  \begin{align*}
    \norm{\lambda_s\left(f\right) - \lambda_s\left(g\right)} &= \norm{\lambda_s\left(f-g\right)}\\
                                                             &= \norm{f-g},
  \end{align*}
  meaning $\lambda_s$ is an isometry.\newline

  We have
  \begin{align*}
    \lambda_s\circ \lambda_r\left(f\right)\left(t\right) &= \lambda_r\left(f\right)\left(s^{-1}t\right)\\
                                                         &= \lambda_r\left(r^{-1}s^{-1}t\right)\\
                                                         &= f\left(\left(sr\right)^{-1}t\right)\\
                                                         &= \lambda_{sr}\left(f\right)\left(t\right),
  \end{align*}
  establishing that $\lambda_s\circ \lambda_r = \lambda_{sr}$.\newline

  By a similar process, we find that $\lambda_{s}\left(\1_{E}\right) = \1_{sE}$ for any $E\subseteq G$ and $s\in G$.
\end{proof}
\begin{definition}
  A {state} on $\ell_{\infty}\left(G\right)$ is a continuous linear functional $\mu\in \left(\ell_{\infty}\left(G\right)\right)^{\ast}$ such that the following are true:
  \begin{itemize}
    \item $\mu$ is positive;
    \item $\mu\left(\1_{G}\right) = 1$.
  \end{itemize}
  A state is called left-invariant if
  \begin{align*}
    \mu\left(\lambda_s\left(f\right)\right) = \mu\left(f\right).
  \end{align*}
\end{definition}
\begin{example}
  The evaluation functional, $\delta_x\colon \ell_{\infty}\rightarrow \R$, defined by
  \begin{align*}
    \delta_{x}\left(f\right) &= f(x),
  \end{align*}
  is a state. However, it is not necessarily invariant, as
  \begin{align*}
    \delta_x\left(\lambda_s\left(f\right)\right) &= \lambda_s\left(f\right)\left(x\right)\\
                                                 &= f\left(s^{-1}x\right)\\
                                                 &\neq f(x).
  \end{align*}
  However, we can use the evaluation functional to create an invariant state. If $G$ is finite, we define
  \begin{align*}
    \mu &= \frac{1}{\left\vert G \right\vert} \sum_{x\in G}\delta_x,
  \end{align*}
  which is indeed an invariant state.
\end{example}
We can characterize states slightly differently, which will enable us to show the equivalence between invariant states and means.
\begin{lemma}\label{lemma:characterizing_states}\hfill
  \begin{enumerate}[(1)]
    \item If $\mu$ is a state on $\ell_{\infty}\left(G\right)$, then
      \begin{align*}
        \norm{\mu}_{\op} = 1.
      \end{align*}
    \item If $\mu\in \left(\ell_{\infty}\left(G\right)\right)^{\ast}$ is such that
      \begin{align*}
        \norm{\mu}_{\op} &= \mu\left(\1_{G}\right)\\
                               &= 1,
      \end{align*}
      then $\mu$ is positive and a state.
  \end{enumerate}
\end{lemma}
\begin{proof}\hfill
  \begin{enumerate}[(1)]
    \item Let $\mu$ be a state. Given $f\in \ell_{\infty}\left(G\right)$, we have
      \begin{align*}
        \norm{f}\1_{G} - f &\geq 0\\
        \norm{f}\1_{G} + f &\geq 0,
      \end{align*}
      so
      \begin{align*}
        0 &\leq \mu\left(\norm{f}\1_{G} - f\right) \\
          &= \norm{f}\mu\left(\1_{G}\right) - \mu\left(f\right)
          \intertext{meaning}
        \mu\left(f\right) &\leq \norm{f}.
        \intertext{Additionally,}
        0 &\leq \mu\left(\norm{f}\1_{G} + f\right)\\
          &= \norm{f}\mu\left(\1_{G}\right) + \mu\left(f\right),
          \intertext{meaning}
        -\mu\left(f\right) &\leq \norm{f}.
      \end{align*}
      Thus, we have $\left\vert \mu\left(f\right) \right\vert \leq \norm{f}$, so $\norm{\mu}_{\op} \leq 1$. However, since $\mu\left(\1_{G}\right) = 1$, we must have $\norm{\mu}_{\op} = 1$.
  \item Suppose $\norm{\mu}_{\op} = \mu\left(\1_{G}\right) = 1$. Let $f\geq 0$. Set $g = \frac{1}{\norm{f}_u}f$.\newline

    Then, $\Ran(g) \subseteq [0,1]$, and $\Ran\left(g - \1_{G}\right) \subseteq \left[-1,1\right]$. Thus, $\norm{g - \1_{G}}_{u} \leq 1$.\newline

    Since $\norm{\mu}_{\op} = 1$, we must have
    \begin{align*}
      \left\vert \mu\left(g - \1_{G}\right) \right\vert &\leq 1\\
      \left\vert \mu\left(g\right) - 1 \right\vert &\leq 1,
    \end{align*}
    and since $\mu\left(\1_{G}\right) = 1$, we have $\mu\left(g\right) \in [0,2]$. Thus, $\mu\left(f\right) = \norm{f}\mu\left(g\right) \geq 0$.
  \end{enumerate}
\end{proof}

To show the equivalence between means and invariant states, we need to be able to characterize the state space on $\left(\ell_{\infty}\left(G\right)\right)^{\ast}$. To do this, we make use of some results from functional analysis.\newline

If $X$ is a normed vector space, then the topology on $X^{\ast}$ induced by $X^{\ast\ast}$ is known as the weak* topology. The weak* topology is the topology of pointwise convergence in $X^{\ast}$ --- a net $\left(\varphi_{\alpha}\right)_{\alpha}$ converges to $\varphi$ in the weak* topology if and only if, for all $\hat{x}\in X^{\ast\ast}$, we have
\begin{align*}
  \left(\hat{x}\left(\varphi_{\alpha}\right)\right)_{\alpha}\rightarrow \hat{x}\left(\varphi\right),
\end{align*}
or by the definition of $X^{\ast\ast}$, this is equivalent to
\begin{align*}
  \left(\varphi_{\alpha}\left(x\right)\right) \rightarrow \varphi\left(x\right)
\end{align*}
for all $x\in X$.\newline

We state some important results in functional analysis here without proof. The proofs of these results can be found in functional analysis textbooks such as \cite{rudin_functional_analysis}.
\begin{theorem}[Hahn--Banach Continuous Extension Theorem]
  Let $X$ be a normed vector space, $E\subseteq X$ a subspace, and $\varphi\in E^{\ast}$ a bounded linear functional. Then, there exists a continuous $\psi\in X^{\ast}$ such that $\norm{\varphi}_{\op} = \norm{\psi}_{\op}$, and $\psi|_{E} = \varphi$.
\end{theorem}
\begin{theorem}[Hahn--Banach Separation Theorems]
  Let $X$ be a normed vector space.
  \begin{enumerate}[(1)]
    \item Given a nonzero $x_0\in X$, there is a $\varphi\in X^{\ast}$ with $\norm{\varphi}_{\op} = 1$ and $\varphi\left(x_0\right) = \norm{x}$. We call $\varphi$ a norming functional.
    \item Given a proper closed subspace $E\subseteq X$ and $x_0\in X\setminus E$, there is a $\varphi\in X^{\ast}$ such that $\varphi|_{E} = 0$, $\norm{\varphi}_{\op} = 1$, and $\varphi\left(x\right) = \dist_{E}(x)$ for all $x\in X$.
  \end{enumerate}
\end{theorem}
\begin{theorem}[Banach--Alaoglu Theorem]
  Let $X$ be a normed vector space.
  \begin{enumerate}[(1)]
    \item The closed unit ball in the dual space, $B_{X^{\ast}}$, is compact in the $w^{\ast}$ topology.
    \item A subset $C\subseteq X$ is $w^{\ast}$-compact if and only if $C$ is $w^{\ast}$-closed and norm bounded.
  \end{enumerate}
\end{theorem}
\begin{corollary}
  The set of states in $\left(\ell_{\infty}\left(G\right)\right)^{\ast}$ forms a $w^{\ast}$-compact subset of $B_{\left(\ell_{\infty}\left(G\right)\right)^{\ast}}$.
\end{corollary}
\begin{proof}
  From the Banach--Alaoglu Theorem, we only need to show that the set of states, $S\left(\ell_{\infty}\left(G\right)\right)$, is $w^{\ast}$-closed, as every element of $S\left(\ell_{\infty}\left(G\right)\right)$ has norm $1$.\newline

  Let $f\in \ell_{\infty}\left(G\right)$ be positive, and let $\left(\varphi_{i}\right)_i$ be a net in $S\left(\ell_{\infty}\left(G\right)\right)$ with $\left(\varphi_{i}\right)_i\xrightarrow{w^{\ast}} \varphi\in \left(\ell_{\infty}\left(G\right)\right)^{\ast}$. From Lemma \ref{lemma:characterizing_states}, we must show that $\varphi$ is positive and $\varphi\left(\1_{G}\right) = 1$.\newline

  We start by seeing that, since each $\varphi_i$ is a state, we have $\varphi_{i}\left(f\right) \geq 0$ for each $i\in I$, so we must have $\varphi\left(f\right) \geq 0$.\newline

  Similarly, since $\varphi_{i}\left(\1_{G}\right) = 1$ for each $i\in I$, and $\left(\varphi_i\right)_i \xrightarrow{w^{\ast}} \varphi$, we have $\varphi\left(\1_{G}\right) = 1$. Thus, by Lemma \ref{lemma:characterizing_states}, we have that $S\left(\ell_{\infty}\left(G\right)\right)$ is $w^{\ast}$-closed.
\end{proof}

Now, we may show the correspondence between invariant states and means.
\begin{proposition}\label{prop:state_implies_mean}
  If $\mu\in \left(\ell_{\infty}\left(G\right)\right)^{\ast}$ is a state, then $m\colon P(G)\rightarrow [0,1]$ defined by $m(E) = \mu\left(\1_{E}\right)$ is a finitely additive probability measure on $G$.\newline

  Moreover, if $\mu$ is invariant, then $m$ is a mean.
\end{proposition}
\begin{proof}
  We have
  \begin{align*}
    m\left(G\right) &= \mu\left(\1_{G}\right)\\
                    &= 1\\
                    \\
    m\left(\emptyset\right) &= \mu\left(0\right)\\
                            &= 0\\
                            \\
    m\left(E\sqcup F\right) &= \mu\left(\1_{E\sqcup F}\right)\\
                            &= \mu\left(\1_{E} + \1_{F}\right)\\
                            &= \mu\left(\1_{E}\right) + \mu\left(\1_{F}\right)\\
                            &= m\left(E\right) + m\left(F\right).
  \end{align*}
  Additionally, since $0 \leq \1_{E}\leq \1_{G}$, we have $0 \leq \mu\left(\1_{E}\right) \leq 1$, so $0 \leq m(E) \leq 1$.\newline

  If $\mu$ is invariant, then
  \begin{align*}
    m\left(sE\right) &= \mu\left(\1_{sE}\right)\\
                     &= \mu\left(\lambda_s\left(\1_{E}\right)\right)\\
                     &= \mu\left(\1_{E}\right)\\
                     &= m\left(E\right).
  \end{align*}
\end{proof}
\begin{proposition}\label{prop:mean_implies_state}
  If $G$ admits a mean, then $\left(\ell_{\infty}\left(G\right)\right)^{\ast}$ admits an invariant state.
\end{proposition}
\begin{proof}
  Let $m$ be a mean. Define $\mu_0\colon \Sigma\rightarrow \R$ by
  \begin{align*}
    \mu_0\left(\sum_{k=1}^{n}t_k\1_{E_k}\right) &= \sum_{k=1}^{n}t_km\left(E_k\right).
  \end{align*}
  Since $m$ is finitely additive, it is the case that $\mu_0$ is well-defined, linear, and positive, with $\mu_0\left(\1_{G}\right) = m\left(G\right) = 1$.\newline

  Additionally, since $m$ is a mean, then for $f = \sum_{k=1}^{n}t_k\1_{E_k}$, we have
  \begin{align*}
    \mu_0\left(\lambda_s\left(f\right)\right) &= \mu_0\left(\lambda_s\left(\sum_{k=1}^{n}t_k\1_{E_k}\right)\right)\\
                                              &= \mu_0\left(\sum_{k=1}^{n}t_k\1_{sE_k}\right)\\
                                              &= \sum_{k=1}^{n}t_km\left(sE_k\right)\\
                                              &= \sum_{k=1}^{n}t_km\left(E_k\right)\\
                                              &= \mu_0\left(f\right).
  \end{align*}
  We see that
  \begin{align*}
    \left\vert \mu_0\left(f\right) \right\vert &= \left\vert \sum_{k=1}^{n}t_km\left(E_k\right) \right\vert\\
                                               &\leq \sum_{k=1}^{n}\left\vert t_k \right\vert m\left(E_k\right)\\
                                               &\leq \sum_{k=1}^{n}\norm{f}\sum_{k=1}^{n}m\left(E_k\right)\\
                                               &= \norm{f}\sum_{k=1}^{n}m\left(E_k\right)\\
                                               &\leq \norm{f},
  \end{align*}
  meaning $\mu_0$ is continuous, so $\mu_0$ is uniformly continuous.\newline

  Since $\overline{\Sigma} = \ell_{\infty}\left(G\right)$, uniform continuity provides that $\mu_0$ extends to a continuous linear functional $\mu\colon \ell_{\infty}\left(G\right)\rightarrow \R$ with $\mu\left(\1_{G}\right) = \mu_0\left(\1_{G}\right) = 1$.\newline

  For $f\geq 0$, we find a sequence $\left(\phi_n\right)_n$ in $\Sigma$ with $\phi_n\geq 0$ and $\norm{\phi_n - f} \xrightarrow{n\rightarrow\infty}0$. We set
  \begin{align*}
    \mu\left(f\right) &= \lim_{n\rightarrow\infty}\mu\left(\phi_n\right)\\
                      &= \lim_{n\rightarrow\infty}\mu_0\left(\phi_n\right)\\
                      &\geq 0,
  \end{align*}
  so $\mu$ is a state.\newline

  If $f\in \ell_{\infty}\left(G\right)$, $s\in G$, and $\left(\phi_n\right)_n$ a sequence in $\Sigma$ with $\left(\phi_n\right)_n\rightarrow f$, then
  \begin{align*}
    \norm{\lambda_s\left(\phi_n\right) - \lambda_s\left(f\right)} &= \norm{\lambda_s\left(\phi_n - f\right)}\\
                                                                  &= \norm{\phi_n - f}\\
                                                                  &\rightarrow 0.
  \end{align*}
  Thus, we have
  \begin{align*}
    \mu\left(\lambda_s\left(\phi_n\right)\right) &= \mu_0\left(\lambda_s\left(\phi_n\right)\right)\\
                                                 &= \mu_0\left(\phi_n\right)\\
                                                 &= \mu\left(\phi_n\right)\\
                                                 &\rightarrow \mu\left(f\right),
  \end{align*}
  so $\mu\left(f\right) = \mu\left(\lambda_s\left(f\right)\right)$. Thus, $\mu\in \left(\ell_{\infty}\left(G\right)\right)^{\ast}$ is an invariant state.
\end{proof}
\section{Establishing Amenability using Invariant States}%
Owing to the correspondence between invariant states and means, we are now able to establish the amenability of large classes of groups.
\begin{proposition}
  The group of integers, $\Z$, is amenable.
\end{proposition}
\begin{proof}
  We define the left shift, $\lambda_1\colon \ell_{\infty}\left(\Z\right) \rightarrow \ell_{\infty}\left(\Z\right)$, by
  \begin{align*}
    \ell_{\infty}\left(f\right)\left(k\right) &= f\left(k-1\right).
  \end{align*}
  This is an action as in Proposition \ref{prop:translation_action}. \newline

  We set $Y = \Ran\left(\id - \lambda_1\right)\subseteq \ell_{\infty}\left(\Z\right)$. We claim that $\dist_{Y}\left(\1_{\Z}\right) \geq 1$.\newline

  Suppose toward contradiction that there is $y\in Y$ with $\norm{\1_{\Z} - y}_{u} = r < 1$. Then, $y = f - \lambda_1 f$ for some $f\in \ell_{\infty}(\Z)$, so
  \begin{align*}
    \norm{\1_{\Z} - \left(f - \lambda_1\left(f\right)\right)} &= r.
  \end{align*}
  Thus, for all $k\in\Z$, we have
  \begin{align*}
    \left\vert 1 - \left(f(k) - f(k-1)\right) \right\vert &\leq r,
  \end{align*}
  so $\left\vert f(k) - f\left(k-1\right) \right\vert \geq 1-r > 0$. However, such an $f$ cannot be bounded.\newline

  Since $\dist_{\overline{Y}}\left(\1_{\Z}\right) = \dist_{Y}\left(\1_{\Z}\right)$, the Hahn--Banach separation theorems provide $\mu\in \left(\ell_{\infty}\left(\Z\right)\right)^{\ast}$ with $\norm{\mu}_{\op} = 1$, $\mu|_{\overline{Y}} = 0$, and $\mu\left(\1_{\Z}\right) = \dist_{Y}\left(\1_{\Z}\right) \geq 1$.\newline

  Since $\norm{\mu}_{\op} = 1$ and $\mu\left(\1_{\Z}\right) \geq 1$, we must have $\mu\left(\1_{\Z}\right) = 1$.\newline

  Additionally, since $\norm{\mu}_{\op} = \mu\left(\1_{\Z}\right) = 1$, we have that $\mu$ is a state on $\ell_{\infty}\left(\Z\right)$, and since $\mu\left(y\right) = 0$ for all $y\in Y$, we have
  \begin{align*}
    \mu\left(f - \lambda_1\left(f\right)\right) &= 0\\
    \mu\left(f\right) &= \mu\left(\lambda_1\left(f\right)\right).
  \end{align*}
  Inductively, this means that $\mu\left(f\right) = \mu\left(\lambda_k\left(f\right)\right)$ for all $k\in \Z$, so $\mu$ is an invariant state on $\ell_{\infty}\left(\Z\right)$. Thus, $\Z$ is amenable.
\end{proof}
\begin{proposition}
  If $N\trianglelefteq G$ and $G/N$ are amenable, then $G$ is amenable.
\end{proposition}
\begin{proof}
  Let $\rho\in \left(\ell_{\infty}\left(G/N\right)\right)^{\ast}$ be an invariant state, and let $p\colon P(N)\rightarrow [0,1]$ be a mean. For $E\subseteq G$, we define $f_E\colon G/N\rightarrow \R$ by
  \begin{align*}
    f_E\left(tN\right) &= p\left(N\cap t^{-1}E\right).
  \end{align*}
  We start by verifying that $f_E$ is well-defined. For $tN = sN$, we have $s^{-1}t\in N$, so
  \begin{align*}
    p\left(N\cap t^{-1}E\right) &= p\left(s^{-1}t\left(N\cap t^{-1}E\right)\right)\\
                                &= p\left(s^{-1}tN \cap s^{-1}E\right)\\
                                &= p\left(N\cap s^{-1}E\right).
  \end{align*}
  Since $f_E$ is defined through $p$, we can see that $f_E$ is bounded. Additionally,
  \begin{align*}
    f_{E\sqcup F}\left(tN\right) &= p\left(N\cap t^{-1}\left(E\sqcup F\right)\right)\\
                                 &= p\left(N\cap \left(t^{-1}E\sqcup t^{-1}F\right)\right)\\
                                 &= p\left(\left(N\cap t^{-1}E\right) \sqcup \left(N\cap t^{-1}F\right)\right)\\
                                 &= p\left(N\cap t^{-1}E\right) + p\left(N\cap t^{-1}F\right)\\
                                 &= f_E\left(tN\right) + f_F\left(tN\right)\\
                                 &= \left(f_E + f_F\right)\left(tN\right),
  \end{align*}
  and
  \begin{align*}
    f\left(sE\right) \left(tN\right) &= p\left(N\cap t^{-1}sE\right)\\
                                     &= f_E\left(s^{-1}tN\right)\\
                                     &= \lambda_{sN}\left(f_E\right)\left(tN\right),
  \end{align*}
  so $f_{sE} = \lambda_{sN}\left(f_E\right)$. Finally,
  \begin{align*}
    f_G\left(tN\right) &= p\left(N\cap t^{-1}G\right)\\
                       &=p\left(N\right)\\
                       &= 1,
  \end{align*}
  meaning $f_G = \1_{G/N}$.\newline

  We define $m\colon P(G)\rightarrow [0,1]$ by
  \begin{align*}
    m(E) &= \rho\left(f_E\right).
  \end{align*}
  Then, we have
  \begin{align*}
    m\left(E\sqcup F\right) &= m(E) + m(F)\\
                            \\
    m\left(G\right) &= 1\\
    \\
    m\left(sE\right) &= \rho\left(f_{sE}\right)\\
                     &= \rho\left(\lambda_{sN}\left(f_{E}\right)\right)\\
                     &= \rho\left(f_E\right)\\
                     &= m(E),
  \end{align*}
  so $m$ is a mean.
\end{proof}
\begin{corollary}
  The finite direct product of amenable groups is amenable.
\end{corollary}
\begin{proof}
  For $H$ and $K$ amenable groups, we know that $K\cong \left(H\times K\right)/H$ and $H$ are amenable, so $H\times K$ is amenable. Induction provides the general case.
\end{proof}
\begin{corollary}
  Finitely generated abelian groups are amenable.
\end{corollary}
\begin{proof}
  By the fundamental theorem of finitely generated abelian groups, all finitely generated abelian groups are isomorphic to $\Z^{d}\times \Z/n_1\Z\times\cdots\times \Z/{n_k}\Z$.\newline

  Since $\Z^{d}$ is a finite direct product of $\Z$, and the torsion subgroup $\Z/n_1\Z\times\cdots\times \Z/n_k\Z$ is finite, we see that a finitely generated abelian group is a direct product of two amenable groups, hence amenable.
\end{proof}
\begin{corollary}
  If $\set{G_i}_{i\in I}$ is a directed family of amenable groups, then the direct limit,
  \begin{align*}
    G &= \bigcup_{i\in I}G_i,
  \end{align*}
  is also amenable.
\end{corollary}
\begin{proof}
  Let $\mu_i\in \left(\ell_{\infty}\left(G_i\right)\right)^{\ast}$ be invariant states.\newline

  Set
  \begin{align*}
    M_i &= \set{\mu\in S\left(\ell_{\infty}\left(G\right)\right)| \mu\left(\lambda_s\left(f\right)\right) = \mu\left(f\right)\text{ for all }s\in G_i}.
  \end{align*}
  We set $\mu\left(f\right) = \mu_i\left(f|_{G_i}\right)$. Since each $G_i$ is amenable, it is the case that each $M_i$ is nonempty. Similarly, seeing as we have established the state space as $w^{\ast}$-closed in $B_{\left(\ell_{\infty}\left(G\right)\right)^{\ast}}$, it is the case that each $M_i$ is $w^{\ast}$-closed in $B_{\left(\ell_{\infty}\left(G\right)\right)^{\ast}}$.\newline

  For $i_1,\dots,i_n$, we find $G_j \supseteq G_{i_1},\dots,G_{i_n}$, which exists since $\set{G_i}_{i\in I}$ is directed. We have that $M_j\subseteq \bigcap_{k=1}^{n}M_{i_k}$, so $\set{M_i}_{i\in I}$ has the finite intersection property.\newline

  Thus, there is $\mu\in \bigcap_{i\in I}M_i$, which is necessarily invariant on $G$.
\end{proof}
\begin{corollary}
  All abelian groups are amenable.
\end{corollary}
\begin{proof}
  Every abelian group is the direct union of its finitely generated subgroups.
\end{proof}
\begin{corollary}
  All solvable groups are amenable.
\end{corollary}
\begin{proof}
  Let $e_G = G_0 \leq G_1\leq\cdots\leq G_n\leq G$ be such that $G_{j-1}\trianglelefteq G_j$ for $j=1,\dots,n$, and $G_i/G_j$ is abelian.\newline

  Since $G_0$ is abelian, it is amenable, as is $G_1/G_0$, so $G_1$ is amenable. We see then that $G_2$ is amenable as $G_1$ and $G_2/G_1$ are amenable.\newline

  Continuing in this fashion, we see that $G$ is amenable.
\end{proof}
\section{Følner's Condition and Approximate Means}%
While showing the existence of an invariant state is necessary and sufficient for showing a group is amenable, as well as showing the group is non-paradoxical, it is often difficult to establish either of these conditions.\newline

However, we can often more easily create a sequence (or net) of finitely supported functions whose limit is an invariant state. This will require the use of the Følner condition.
\begin{definition}\label{def:folner_condition}
  A group is said to satisfy the {Følner condition} if, for every $\ve > 0$ and $E\subseteq G$, there is a nonempty finite subset $F\subseteq G$ such that for all $t\in E$,
  \begin{align*}
    \frac{\left\vert tF\triangle F \right\vert}{\left\vert F \right\vert}\leq \ve.
  \end{align*}
  Equivalently, we can also say that the Følner condition is satisfied if and only if
  \begin{align*}
    \frac{\left\vert tF\cap F \right\vert}{\left\vert F \right\vert} \geq 1 - \ve
  \end{align*}
  for every $\ve > 0$.
\end{definition}
\begin{lemma}\label{lemma:folner_sequences}
  A countable group $G$ satisfies the Følner condition if and only if $G$ admits a sequence $\left(F_n\right)_n$ with $F_n\subseteq G$ finite such that
  \begin{align*}
    \left(\frac{\left\vert tF_n\triangle F_n \right\vert}{\left\vert F_n \right\vert}\right)_n \xrightarrow{n\rightarrow \infty}0
  \end{align*}
  for all $t\in G$. Such a sequence is known as a Følner sequence.
\end{lemma}
\begin{proof}
  Let $G$ admit a Følner sequence, $\left(F_n\right)_n$. Given $\ve > 0$ and $E\subseteq G$ finite, find $N$ such that for all $s\in E$ and $n\geq N$,
  \begin{align*}
    \frac{\left\vert sF_n\triangle F_n \right\vert}{\left\vert F_n \right\vert} &\leq \ve.
  \end{align*}
  We take $F = F_N$ in the definition of the Følner condition.\newline

  Let $G$ satisfy the Følner condition. We write $G = \bigcup_{n\geq 1}E_n$, with $E_1\subseteq E_2\subseteq \cdots$, and define $F_n$ such that for all $t\in E_n$,
  \begin{align*}
    \frac{\left\vert tF_n\triangle F_n \right\vert}{\left\vert F_n \right\vert} &\leq \frac{1}{n}.
  \end{align*}
  Given $t\in G$, then $t\in E_N$ for some $N$, so $t\in E_n$ For all $n\geq N$, so
  \begin{align*}
    \frac{\left\vert tF_n\triangle F_n \right\vert}{\left\vert F_n \right\vert} &\leq \frac{1}{n}
  \end{align*}
  for all $n\geq N$. Thus,
  \begin{align*}
    \left(\frac{\left\vert tF_n\triangle F_n \right\vert}{\left\vert F_n \right\vert}\right)\xrightarrow{n\rightarrow\infty}0.
  \end{align*}
\end{proof}
\begin{lemma}
  Let $G$ be a finitely generated group with generating set $S$. If $\left(F_n\right)_n$ is a sequence of finite subsets such that, for all $s\in S$,
  \begin{align*}
    \left(\frac{\left\vert sF_n\triangle F_n \right\vert}{\left\vert F_n \right\vert}\right)_n\rightarrow 0,
  \end{align*}
  then $\left(F_n\right)_n$ is a Følner sequence for $G$.
\end{lemma}
\begin{proof}
  Note that
  \begin{itemize}
    \item $s\left(A\triangle B\right) = sA\triangle sB$;
    \item $A\triangle C \subseteq \left(A\triangle B\right) \cup \left(B\triangle C\right)$.
  \end{itemize}
  We see that for any $s\in S$,
  \begin{align*}
    \frac{\left\vert s^{-1}F_n\triangle F_n \right\vert}{\left\vert F_n \right\vert} &= \frac{\left\vert s^{-1}\left(F_n\triangle sF_n\right) \right\vert}{\left\vert F_n \right\vert}\\
                                                                                     &= \frac{\left\vert F_n\triangle sF_n \right\vert}{\left\vert F_n \right\vert}\\
                                                                                     &\rightarrow 0.
  \end{align*}
  Thus, we may assume that $S$ is symmetric --- i.e., that $\set{s^{-1}| s\in S} = \set{s | s\in S}$.\newline

  For any $s,t\in S$, we have
  \begin{align*}
    \frac{\left\vert stF_n\triangle F_n \right\vert}{\left\vert F_n \right\vert} &\leq \frac{\left\vert stF_n\triangle F_n \right\vert}{\left\vert F_n \right\vert} + \frac{\left\vert sF_n\triangle F_n \right\vert}{\left\vert F_n \right\vert}\\
                                                                                 &= \frac{\left\vert s\left(tF_n\triangle F_n\right) \right\vert}{\left\vert F_n \right\vert} + \frac{\left\vert sF_n\triangle F_n \right\vert}{\left\vert F_n \right\vert}\\
                                                                                 &= \frac{\left\vert tF_n\triangle F_n \right\vert}{\left\vert F_n \right\vert} + \frac{\left\vert sF_n\triangle F_n \right\vert}{\left\vert F_n \right\vert}\\
                                                                                 &\rightarrow 0.
  \end{align*}
  We use induction to find the general case.
\end{proof}
\begin{example}
  Consider the group $\Z$. Since $\Z$ is generated by the element $\set{1}$, we see that for $F_n = [-n,n]$, that
  \begin{align*}
    \frac{\left\vert \left(F_n + 1\right)\triangle F_n \right\vert}{\left\vert F_n \right\vert} &= \frac{2}{2n+1}\\
                                                                                                &\rightarrow 0,
  \end{align*}
  meaning that $\Z$ satisfies the Følner condition.
\end{example}
We have thus far proven that $G$ satisfies the Følner condition if and only if $G$ admits a Følner sequence, and that $G$ is amenable if and only if $G$ admits an invariant state.\newline

We will now begin harmonizing these two characterizations through the use of approximate means, eventually showing that $G$ satisfies the Følner condition if and only if $G$ admits an approximate mean, and that $G$ admits an approximate mean if and only if $G$ is amenable.
\begin{definition}\label{def:state_on_prob_g}
  For a group $G$, we define
  \begin{align*}
    \Prob\left(G\right) = \set{f\colon G\rightarrow [0,\infty) | \left\vert \supp(f) \right\vert < \infty,~\sum_{t\in G}f(t) = 1}.
  \end{align*}
  Note that $\Prob(G) \subseteq B_{\ell_1\left(G\right)}$. For $f\in \prob(G)$, we set $\varphi_f\colon \ell_{\infty}(G)\rightarrow \C$ defined by
  \begin{align*}
    \varphi_f\left(g\right) &= \sum_{t\in G}g(t)f(t).
  \end{align*}
\end{definition}
\begin{fact}\label{fact:prob_g_state}
  For $f\in \prob(G)$, $\varphi_f$ is a state on $\ell_{\infty}\left(G\right)$.
\end{fact}
\begin{proof}
We can see that, by definition, $\varphi_f$ is positive, linear, and has $\varphi_f\left(\1_{G}\right) = 1$.\newline

We only need to show that $\norm{\varphi_f} = 1$. We see that
\begin{align*}
  \left\vert \varphi_f\left(g\right) \right\vert &= \left\vert \sum_{t\in G}g(t)f(t) \right\vert\\
                                                 &\leq \sum_{t\in G}\left\vert g(t) \right\vert\left\vert f(t) \right\vert\\
                                                 &\leq \norm{g}_{\infty}\sum_{t\in G}\left\vert f(t) \right\vert\\
                                                 &= \norm{g}_{\infty}.
\end{align*}
\end{proof}
\begin{proposition}
  There is an action $\lambda\colon G\xrightarrow \Isom\left(\ell_{1}\left(G\right)\right)$ such that $\prob(G)$ is invariant.
\end{proposition}
\begin{proof}
  Let $\lambda_s\left(f\right)\left(t\right) = f\left(s^{-1}t\right)$. Then,
  \begin{align*}
    \norm{\lambda_s\left(f\right)}_1 &= \sum_{t\in G}\left\vert \lambda_s\left(f\right)\left(t\right) \right\vert\\
                                     &= \sum_{t\in G}\left\vert f\left(s^{-1}t\right) \right\vert\\
                                     &= \sum_{r\in G}\left\vert f(r) \right\vert\\
                                     &= \norm{f}_{1}.
  \end{align*}
  Just as in Proposition \ref{prop:translation_action}, it is the case that $\lambda_s$ is linear. Additionally,
  \begin{align*}
    \lambda_r\circ \lambda_s\left(f\right)\left(t\right) &= \lambda_s\left(f\right)\left(r^{-1}t\right)\\
                                                         &= f\left(s^{-1}r^{-1}\left(t\right)\right)\\
                                                         &= f\left(\left(rs\right)^{-1}t\right)\\
                                                         &= \lambda_{rs}\left(f\right)\left(t\right).
  \end{align*}
  We see that if $f\in \prob(G)$, then for $f\geq 0$, we have $\lambda_s\left(f\right) \geq 0$, and
  \begin{align*}
    \sum_{t\in G}\lambda_s\left(f\right)\left(t\right) &= \sum_{t\in G}f\left(s^{-1}t\right)\\
                                                       &= \sum_{r\in G}f\left(r\right)\\
                                                       &= 1
  \end{align*}
  for any $f\in \prob(G)$.
\end{proof}
\begin{definition}\label{def:approximate_mean}
  For a countable group $G$, a sequence $\left(f_k\right)_k$ is called an approximate mean if, for all $s\in G$,
  \begin{align*}
    \norm{f_k - \lambda_s\left(f_k\right)}_{1} &\xrightarrow{k\rightarrow \infty}0.
  \end{align*}
\end{definition}
To begin the forward direction regarding the equivalence between the Følner condition, approximate means, and means, we begin by showing that the existence of a Følner sequence implies the existence of an approximate mean. Then, we will show that the existence of an approximate mean implies the existence of an invariant state (hence mean).
\begin{proposition}
  If $G$ admits a Følner sequence $\left(F_k\right)_k$, then $G$ admits an approximate mean.
\end{proposition}
\begin{proof}
  Set $f_k = \frac{1}{\left\vert F_k \right\vert}\1_{F_k}\in \prob(G)$. Then,
  \begin{align*}
    \norm{f_k - \lambda_s\left(f_k\right)}_{1} &= \frac{1}{\left\vert f_k \right\vert} \norm{\1_{F_k} - \lambda_s\left(\1_{F_k}\right)}\\
                                               &= \frac{1}{F_k}\norm{\1_{F_k} - \1_{sF_k}}\\
                                               &= \frac{\left\vert F_k\triangle sF_k \right\vert}{\left\vert F_k \right\vert}\\
                                               &\rightarrow 0.
  \end{align*}
\end{proof}
\begin{proposition}
  If $G$ admits an approximate mean, then $G$ is amenable.
\end{proposition}
\begin{proof}
  Let $\left(f_k\right)_k$ be an approximate mean. We define $\varphi_k = \left(\varphi_{f_k}\right)_k$ (as in Definition \ref{def:state_on_prob_g}) to be a sequence of states on $\ell_{\infty}\left(G\right)$.\newline

  Since the state space on $\ell_{\infty}\left(G\right)$ is $w^{\ast}$-compact, there is a state $\mu$ and a subnet $\left(\varphi_{k_j}\right)_j \xrightarrow{w^{\ast}}\mu$. \newline

  We only need to show that $\mu$ is invariant. Note that
  \begin{align*}
    \left\vert \mu\left(g\right) - \mu\left(\lambda_s\left(g\right)\right) \right\vert &\leq \left\vert \mu\left(g\right) - \varphi_{k_j}\left(g\right) \right\vert + \left\vert \varphi_{k_j}\left(g\right) - \varphi_{k_j}\left(\lambda_s\left(g\right)\right) \right\vert + \left\vert \varphi_{k_j}\left(\lambda_s\left(g\right)\right) - \mu\left(\lambda_s\left(g\right)\right) \right\vert
  \end{align*}
  for all $g\in \ell_{\infty}\left(G\right)$, $s\in G$, and all $j$.\newline

  Given $\ve > 0$, we find $J$ such that for $j\geq J$,
  \begin{align*}
    \left\vert \mu\left(g\right) - \varphi_{k_j}\left(g\right) \right\vert &< \ve/3\\
    \left\vert \mu\left(\lambda_s\left(g\right)\right) \varphi_{k_j}\left(\lambda_s\left(g\right)\right)\right\vert &< \ve/3.
  \end{align*}
  We also see that
  \begin{align*}
    \left\vert \varphi_{k_j}\left(g\right) - \varphi_{k_j}\left(\lambda_s\left(g\right)\right) \right\vert &= \left\vert \sum_{t\in G}g(t)f_{k_j}\left(t\right) - \sum_{t\in G}g\left(s^{-1}t\right)f_{k_j}\left(t\right) \right\vert\\
                                                                                                           &= \left\vert \sum_{t\in G}g(t)f_{k_j}\left(t\right) - \sum_{r\in G}g(r)f_{k_j}\left(sr\right) \right\vert \tag*{$r = s^{-1}t$}\\
                                                                                                           &= \left\vert \sum_{t\in G}g(t)\left(f_{k_j}\left(t\right)-\lambda_{s^{-1}}\left(f_{k_j}\right)\left(t\right)\right) \right\vert\\
                                                                                                           &\leq \norm{g}_{\infty}\sum_{t\in G}\left\vert f_{k_j}\left(t\right) - \lambda_{s^{-1}}\left(f_{k_j}\right)\left(t\right) \right\vert\\
                                                                                                           &= \norm{g}_{\infty}\norm{f_{k_j} - \lambda_{s^{-1}}\left(f_{k_j}\right)}_{1}\\
                                                                                                           &< \ve/3
  \end{align*}
  for large $j$. Thus, we have
  \begin{align*}
    \left\vert \mu\left(g\right) - \mu\left(\lambda_{s}\left(g\right)\right) \right\vert &< \ve,
  \end{align*}
  for all $\ve > 0$, so $\mu\left(g\right) = \mu\left(\lambda_{s}\left(g\right)\right)$.
\end{proof}

We will now commence with the reverse direction, starting by showing that amenability implies the existence of an approximate mean, and then showing that the existence of an approximate mean implies that the Følner condition is satisfied.
\begin{proposition}
  If $G$ is amenable, then $G$ admits an approximate mean.
\end{proposition}
\begin{proof}
  Suppose $G$ does not admit an approximate mean. Then, there exists a finite subset $E_0\subseteq G$ and $\ve_0 > 0$ such that for all $s\in E_0$ and all $f\in \Prob(G)$, we have $\norm{f - \lambda_s\left(f\right) \geq \ve_0}$.\newline

  Consider the set
  \begin{align*}
    X &= \bigoplus_{\left\vert E_0 \right\vert} \ell_1\left(G\right),
  \end{align*}
  endowed with the norm
  \begin{align*}
    \norm{\left(f_s\right)_{s\in E_0}} &= \sum_{s\in E_0}\sum_{t\in G}\left\vert f_s(t) \right\vert\\
                                       &= \sum_{s\in E_0}\norm{f_s}_{1},
  \end{align*}
  and let
  \begin{align*}
    C &= \set{\left(f - \lambda_s\left(f\right)\right)_{s\in E_0} | f\in \Prob(G)}.
  \end{align*}
  Since $\Prob(G)$ is convex, it is the case that $C$ is convex, and since $\left\vert E_0 \right\vert$ is finite, $C$ is necessarily bounded. Note that $0\notin \overline{C}$.\newline

  By the Hahn--Banach separation theorem for convex sets, there is a real-valued $\varphi\in X^{\ast}$ such that $\varphi\left(C\right)\geq 1$. Here,
  \begin{align*}
    X^{\ast} &\cong \bigoplus_{\left\vert E_0 \right\vert}\ell_1\left(G\right)^{\ast}\\
             &\cong \sum_{\left\vert E_0 \right\vert}\ell_{\infty}\left(G\right),
  \end{align*}
  endowed with the norm
  \begin{align*}
    \norm{\left(g_s\right)_{s\in E_0}} &= \max_{s\in E_0}\left(\sup_{t\in G}\left\vert g_s(t) \right\vert\right)\\
                                       &= \max_{s\in E_0}\norm{g_s}_{\infty}.
  \end{align*}
  We let $\varphi = \left(\varphi_{g_s}\right)_{s\in E_0}$, where $g_s\in \ell_{\infty}\left(G\right)$ is defined by the duality
  \begin{align*}
    \varphi_{g_s}\left(f\right) &= \sum_{t\in G}f(t)g_s(t).
  \end{align*}
  Thus, for all $f\in \Prob(G)$, we have
  \begin{align*}
    1 &\leq \varphi\left(\left(f - \lambda_s\left(f\right)\right)_{s\in E_0}\right)\\
      &= \sum_{s\in E_0}\varphi_{g_s}\left(f - \lambda-s\left(f\right)\right)\\
      &= \sum_{s\in E_0}\sum_{t\in G}\left(f - \lambda_s\left(f\right)\right)(t)g_s(t)\\
      &= \sum_{s\in E_0}\left(\sum_{t\in G}f(t)g_s(t) - \sum_{t\in G}f\left(s^{-1}t\right)g_s(t)\right)\\
      &= \sum_{s\in E_0}\left(\sum_{t\in G}f(t)g_s(t) - \sum_{r\in G}f\left(r\right)g_s\left(sr\right)\right)\\
      &= \sum_{s\in E_0}\left(\sum_{r\in G}f(r)g_s(r) - \sum_{r\in G}f(r)\lambda_{s^{-1}}\left(g\right)(r)\right)\\
      &= \sum_{s\in E_0}\sum_{r\in G}f(r)\left(g_s - \lambda_{s^{-1}}\left(g\right)\right)(r).
      \intertext{Note that this holds for any $f\in \Prob(G)$, including the case of $f = \delta_t$ for a given $t\in G$. We must have}
      &= \sum_{s\in E_0}\sum_{r\in G}\delta_{t}\left(r\right)\left(g_s\left(r\right) - \lambda_{s^{-1}}\left(g_s\right)\right)\left(r\right)\\
      &= \sum_{s\in E_0}\left(g_s - \lambda_{s^{-1}}\left(g\right)\right)\left(t\right).
      \intertext{In particular, we must have}
      &\geq \1_{G}.
  \end{align*}
  Since $G$ is amenable, there is a mean $\mu\colon \ell_{\infty}\left(G\right)\rightarrow \C$ with $\mu\left(g_s\right) = \mu\left(\lambda_{s^{-1}}\left(g_s\right)\right)$, meaning
  \begin{align*}
    0 &= \mu\left(\sum_{s\in E_0}\left(g_s - \lambda_{s^{-1}}\left(g_s\right)\right)\left(t\right)\right)\\
      &\geq \mu\left(\1_{G}\right)\\
      &= 1,
  \end{align*}
  which is a contradiction.
\end{proof}
To show that the existence of an approximate mean implies the Følner condition, we require the following lemma.
\begin{lemma}\label{lemma:layer_cake_representation}
  Let $f\colon S\rightarrow \R$ be finitely supported with $\sum_{s\in S}f(s) = 1$. Then, there exist subsets $\set{F_i}_{i=1}^{n}$, where $F_1\supseteq F_2\supseteq \cdots \supseteq F_n$, and constants $\set{c_i}_{i=1}^{n}$, such that
  \begin{align*}
    f &= \sum_{i=1}^{n}c_i\1_{F_i},
  \end{align*}
  where
  \begin{align*}
    \sum_{i=1}^{n}c_i\left\vert F_i \right\vert &= 1.
  \end{align*}
  This is known as the layer cake representation for $f$.
\end{lemma}
\begin{proof}[Proof of Lemma \ref{lemma:layer_cake_representation}:]
  We define $F_1 = \supp\left(f\right)$, and take $c_1 = \min\left(\Ran\left(f\right)\right)$. Taking $E_1 = f^{-1}\left(c_1\right)$ (as a set-theoretic inverse), we define $F_2 = F_1\setminus E_1$.\newline

  Take $d_1 = \min\left(f\left(F_2\right)\right)$, and define $c_2 = d_1 - c_1$. Then, defining $E_2 = f^{-1}\left(d_1\right)$, $F_3 = F_2 \setminus E_2$, and $d_2 = \min\left(f\left(F_3\right)\right)$, we define $c_3 = d_2 - c_2 - c_1$.\newline

  Continuing in this pattern, we find $d_{i-1} = \min\left(f\left(F_i\right)\right)$, $E_i = f^{-1}\left(d_{i-1}\right)$, and $c_i = d_{i-1} - \sum_{j=1}^{i-1}c_i$.\newline

  This yields a decomposition $F_1\supseteq F_2\supseteq \cdots \supseteq F_n$, where $\sum_{i=1}^{n}c_i\1_{F_i} = f$ by construction.\newline

  We now verify that $\sum_{i=1}^n c_i\left\vert F_i \right\vert = 1$.
  \begin{align*}
    1 &= \sum_{s\in S}f(s)\\
      &= \sum_{s\in S}\sum_{i=1}^{n}c_i\1_{F_i}\left(s\right).
      \intertext{By definition, if $s\in F_j$ for some $j$, we see that $c_j$ is summed for $\left\vert F_j \right\vert$ many times. Thus, we obtain}
      &= \sum_{i=1}^{n}c_i\left\vert F_i \right\vert.
  \end{align*}
\end{proof}

We will use the layer cake decomposition to prove that if $G$ admits an approximate mean, then $G$ satisfies the Følner condition.
\begin{proposition}
  Let $G$ admit an approximate mean. Then, $G$ satisfies the Følner condition.
\end{proposition}
\begin{proof}
  Let $\left(f_k\right)_k$ be an approximate mean, as in Definition \ref{def:approximate_mean}. Fix a finite nonempty set $S \subseteq G$. Then, by the definition of an approximate mean, there must exist some $N\in\N$ such that for all $k\geq N$ and all $s\in G$,
  \begin{align*}
    \norm{f_k - \lambda_s\left(f_k\right)}_{1} &\leq \frac{\ve}{|S|}.
  \end{align*}
  In particular, this holds for $f_N$ and for all $s\in S$.\newline

  Since $f_N\in \Prob(G)$ is finitely supported and $\sum_{s\in G}f_N(s) = 1$, we may use Lemma \ref{lemma:layer_cake_representation} to rewrite $f_N$ as
  \begin{align*}
    f_N &= \sum_{i=1}^{n}c_i\1_{F_i},
  \end{align*}
  where $F_1 \supseteq F_2\supseteq \cdots \supseteq F_n$, and $\sum_{i=1}^{n}c_i\left\vert F_i \right\vert = 1$. 
\end{proof}

Thus far, we have shown the following to be equivalent for a discrete group $G$:
\begin{enumerate}[(1)]
  \item $G$ is non-paradoxical;
  \item $G$ is amenable;
  \item $G$ admits an invariant state;
  \item $G$ admits an approximate mean;
  \item $G$ satisfies the Følner condition.
\end{enumerate}
The equivalence between (1) and (2) follows from Tarski's theorem (Theorem \ref{thm:tarski}), the equivalence between (2) and (3) follows from Propositions \ref{prop:state_implies_mean} and \ref{prop:mean_implies_state}, and the equivalence between (3), (4), and (5) follows from 

% Testing names: Engineering a Mean: Følner's Condition and Approximate Means
\chapter{I've Looked at Groups from Both Sides Now: the Left-Regular Representation}\label{ch:left_regular_representation}
Just as God appears in many forms (or representations) throughout the Bible, such as the Burning Bush in the book of Exodus, so too are groups often dealt with and their properties understood through their representations. The field of representation theory, for instance, focuses on the properties of groups as subgroups of groups of linear transformations, and how the properties of these groups of linear transformations can provide insights into the properties of the groups themselves.\newline

In this chapter, we will engage with the properties of groups represented as unitary operators on a Hilbert space --- this will allow us to understand and prove various important results related to groups by using techniques from functional analysis, just as we used techniques of functional analysis to prove the important results in Chapters \ref{ch:invariant_states} and \ref{ch:folner_condition}.\newline

The exposition in this chapter occasionally pulls from \cite{rainone_analysis}, with the bulk of the proofs following \cite[Appendix A]{juschenko_amenability}, with some details added.
\section{Representing a Group}%
On a Hilbert space $\mathcal{H}$, we know that the set of unitary operators, $\mathcal{U}\left(\mathcal{H}\right)$, is a group under composition.\footnote{To see this, note that $I_{\mathcal{H}}$ is the identity element, that $U^{\ast}U =UU^{\ast} = I_{\mathcal{H}}$, and that if $U,V$ are unitary, then $\left( UV \right)^{\ast}\left( UV \right) = V^{\ast}U^{\ast}UV = I_{\mathcal{H}}$ and similarly for the other way around.} Given any other group $\Gamma$, it is then tempting to consider how we can ``model'' $\Gamma$ (so to speak) as a subgroup of $\mathcal{U}\left(\mathcal{H}\right)$. This is the essence behind the idea of a unitary representation.
\begin{definition}[{\cite[Definition 6.2.18]{rainone_analysis}}]
  Let $\Gamma$ be a group. A unitary representation of $\Gamma$ is a pair, $\left(\pi,\mathcal{H}\right)$, where $\mathcal{H}$ is a Hilbert space and $\pi\colon \Gamma\rightarrow \mathcal{U}\left(\mathcal{H}\right)$ is a group homomorphism.\newline

  Furthermore, every unitary representation $U\colon \Gamma\rightarrow \mathcal{U}\left(\mathcal{H}\right)$, given by $s\mapsto U_s$, has the following properties:
  \begin{itemize}
    \item if $e$ is the identity element for $\Gamma$, then $U_{e} = I_{\mathcal{H}}$;
    \item for all $s\in \Gamma$, $U_{s}^{\ast} = U_{s^{-1}}$.
  \end{itemize}
\end{definition}
\begin{example}\label{ex:some_representations}
  One excellent example of a unitary representation is the representation $1_{\Gamma}\colon \Gamma\rightarrow \C$, defined by $1_{\Gamma}(s) = 1$ for all $s\in\Gamma$. This is known as the trivial representation, and it will play an integral role in establishing amenability.\newline

  A more substantive unitary representation is the representation of the circle group, $\mathbb{T}\rightarrow \B\left(\ell_2\left(\Z\right)\right)$, given by $\omega \mapsto d_{\omega}$. Here, $d_{\omega}$ is the multiplication operator defined by
  \begin{align*}
    d_{\omega}\left(\left(a_{k}\right)_{k\in\Z}\right) &= \left(\omega^k a_{k}\right)_{k\in\Z}.
  \end{align*}
\end{example}
Via the trivial representation, we know that any group can be unitarily represented --- however, the trivial representation is, unfortunately, quite unable to give us information about properties of the underlying group. In general, the Hilbert space we want to represent the group on should, in some way, be based on the underlying group, and the unitary representation to be based on the group's self-action by left-multiplication (see Definition \ref{def:group_action}).\newline

As for the Hilbert space, we will use the space $\ell_2\left( \Gamma \right)$, which, from Definition \ref{def:three_function_spaces}, is the space of all functions $f\colon \Gamma\rightarrow \C$ such that $\sum_{t\in \Gamma}\left\vert f(t) \right\vert^2 < \infty$.
\begin{theorem}[{\cite[Example 6.2.21]{rainone_analysis}}]\label{thm:left_regular_representation}
  Let $\Gamma$ be a group. For a fixed $t\in\Gamma$, we define $\lambda_t\colon \ell_2\left( \Gamma \right)\rightarrow \ell_2\left( \Gamma \right)$ by
  \begin{align*}
    \lambda_t\left( f \right)\left( s \right) &= f\left( t^{-1}s \right).
  \end{align*}
  Then, $\lambda_t$ is an isometry, and the map
  \begin{align*}
    \lambda\colon \Gamma\rightarrow \B\left( \ell_2\left( \Gamma \right) \right),
  \end{align*}
  given by $t\mapsto \lambda_t$, is a unitary representation of $\Gamma$. This is known as the \textit{left-regular representation} of $\Gamma$.
\end{theorem}
\begin{proof}
  For a fixed $t$, the map $\lambda_t\colon \ell_2\left( \Gamma \right)\rightarrow \ell_2\left( \Gamma \right)$ is a well-defined linear isometry, as
  \begin{align*}
    \norm{\lambda_t\left( f \right)}_{\ell_2}^2 &= \sum_{s\in\Gamma}\left\vert \lambda_t\left( f \right)\left( s \right) \right\vert^2\\
                                                &= \sum_{s\in\Gamma}\left\vert f\left( t^{-1}s \right) \right\vert^2\\
                                                &= \sum_{r\in\Gamma}\left\vert f\left( r \right) \right\vert^2\tag*{$r = t^{-1}s$}\\
                                                &= \norm{f}_{\ell_2}^2.
  \end{align*}
  Now, we know that each $\lambda_s$ has an inverse of $\lambda_{s^{-1}}$, so we know that each $\lambda_s$ is unitary, with $\lambda_s^{\ast} = \lambda_{s^{-1}}$. To evaluate that $\lambda$ is an action, we verify on the orthonormal basis of $\ell_2\left( \Gamma \right)$, $\set{\delta_t}_{t\in\Gamma}$ (see Example \ref{ex:orthonormal_bases}). This gives
  \begin{align*}
    \lambda_s\left( \delta_t \right)\left( r \right) &= \delta_t\left( s^{-1}r \right)\\
                                                     &= \begin{cases}
                                                       1 & s^{-1}r = t\\
                                                       0 & s^{-1}r \neq t
                                                     \end{cases}\\
                                                     &= \begin{cases}
                                                       1 & r = st\\
                                                       0 & r\neq st
                                                     \end{cases}\\
                                                     &= \delta_{st}\left( r \right),
  \end{align*}
  meaning $\lambda_s\left( \delta_t \right) = \delta_{st}$. Additionally, we see that
  \begin{align*}
    \lambda_{s}\circ \lambda_r\left( f \right)\left( t \right) &= \lambda_{r}\left( f \right)\left( s^{-1}t \right)\\
                                                               &= f\left( r^{-1}s^{-1}t \right)\\
                                                               &= f\left( \left( sr \right)^{-1}t \right)\\
                                                               &= \lambda_{sr}\left( f \right)\left( t \right).
  \end{align*}
  Thus, we obtain the unitary representation of $\Gamma$, $\lambda\colon\Gamma\rightarrow \mathcal{U}\left( \ell_2\left( \Gamma \right) \right)$.
\end{proof}
%Insert something here about Fell's absorption principle
\begin{remark}
  The other ``regular representation'' is, predictably, the right-regular representation, given by $s \mapsto \rho_s$, where
  \begin{align*}
    \rho_s\left( f \right)\left( t \right) &= f\left( ts \right).
  \end{align*}
  The right-regular representation acts on orthonormal basis elements by mapping $\delta_t \mapsto \delta_{ts^{-1}}$.\newline

  It can be shown that the left-regular representation and right-regular representation are isomorphic, in the sense that there is a bijective map $\lambda_s\mapsto \rho_s$ that remains faithful to the underlying group structure. However, we will be working with the left-regular representation as it is more commonly when dealing with unitary representations of groups, though it is important to underscore that this is purely personal preference rather than something innate with the left-regular representation itself.
\end{remark}
\section{Almost-Invariant Vectors in the Left-Regular Representation}%
One of the crucial aspects of the left-regular representation is that yet again we are able to use the tools of functional analysis, as in Section \ref{sec:functional_analysis_and_amenability}, to establish amenability. However, this time, rather than being forced to use the dual space of $\ell_{\infty}\left(\Gamma\right)$, we are able to use the properties of $\ell_2\left( \Gamma \right)$ itself rather than being forced to pass to the dual space.\footnote{Technically, this is because, from the Riesz Representation Theorem on Hilbert Spaces (Theorem \ref{thm:riesz_hilbert_spaces}), the space $\ell_2(\Omega)$ is (isomorphic to) its dual. Very convenient, indeed.}\newline

For a given unitary representation $\lambda$, we say a unit vector $\xi\in\ell_2\left( \Gamma \right)$ is invariant for $\lambda$ if, for all $s\in\Gamma$, we have $\lambda_s\left( \xi \right) = \xi$. As it turns out, the existence of a purely invariant vector is a sufficient condition for amenability, though not in a particularly eye-catching manner.
\begin{theorem}[{\cite[137]{juschenko_amenability}}]
  Let $\Gamma$ be a group, and let $\lambda\colon \Gamma\rightarrow \mathcal{U}\left( \ell_2\left( \Gamma \right) \right)$ be the left-regular representation. Then, $\lambda$ admits an invariant vector if and only if $\Gamma$ is finite.
\end{theorem}
\begin{proof}
  Let $\Gamma$ be finite. Since all functions $f\colon \Gamma\rightarrow \C$ are square-summable, as $\Gamma$ is finite, so too is $\xi = \1_{\Gamma}$. Since $s\Gamma = \Gamma$ for all $s\in\Gamma$, we have $\1_{\Gamma}$ is invariant for $\lambda$.\newline

  Now, let $\lambda\colon \Gamma\rightarrow \mathcal{U}\left( \ell_2\left( \Gamma \right) \right)$ be the left-regular representation, and suppose there is $\xi\in \ell_2\left( \Gamma \right)$ such that for all $s\in\Gamma$, we have
  \begin{align*}
    \lambda_s\left( \xi \right) &= \xi.
  \end{align*}
  In particular, this means that for all $t\in\Gamma$, we have
  \begin{align*}
    \lambda_s\left( \xi \right)\left( t \right) &= \xi\left( s^{-1}t \right)\\
                                                &= \xi\left( t \right).
  \end{align*}
  Now, since this holds for all $s\in\Gamma$, this means that $\xi\left( t \right) = \xi\left( s \right)$ for any $s\neq t$, as we may find $r\in \Gamma$ such that $r^{-1}t = s$ so that $\lambda_r\left( \xi \right)\left( t \right) = \xi\left( s \right)$. Therefore, $\xi = c\1_{\Gamma}$ for some $c\in\C$.\newline

  Now, since $\xi\in \ell_2\left( \Gamma \right)$, we must have that
  \begin{align*}
    \sum_{t\in\Gamma}\left\vert \xi(t) \right\vert^2 < \infty.
  \end{align*}
  This is equivalent to the condition that
  \begin{align*}
    \sum_{t\in\Gamma}\left\vert c \right\vert^2 &< \infty.
  \end{align*}
  This can only hold if $\Gamma$ is finite.
\end{proof}
Now, finite groups are amenable (by Example \ref{ex:finite_invariant_state}), but sadly that is not very interesting, and this is not helpful for the various infinite groups we hope to establish the amenability of. What is interesting, though, is that the existence of an \textit{almost}-invariant vector for $\lambda$ characterizes amenability.\newline

To prove that the existence of an almost-invariant vector for $\lambda$ is equivalent to amenability, however, we need to use a different version of the approximate mean defined in Definition \ref{def:approximate_mean}. This is also known as Reiter's condition.
\begin{theorem}[Reiter's Condition]\label{thm:reiter}
  Let $\Gamma$ be a (countable, discrete) group. Then, $\Gamma$ is amenable if and only if, for any $\ve > 0$ and for any finite subset $E\subseteq G$, there is a $\mu\in \Prob(G)$ (see Definition \ref{def:state_on_prob_g}) such that $\norm{\lambda_s\left( \mu \right) - \mu}_{\ell_1} \leq \ve$.
\end{theorem}
\begin{proof}
  We will show that Reiter's condition is equivalent to the existence of an approximate mean.\newline

  Suppose $\Gamma$ is amenable. Then, $\Gamma$ admits a sequence of (unit) vectors, $\left( f_k \right)_k$ such that 
  \begin{align*}
    \norm{\lambda_s\left( f_k \right) - f_k}_{\ell_1}\rightarrow 0
  \end{align*}
   for all $s\in\Gamma$.\newline

   If $\ve > 0$, then Reiter's condition follows from finding $K$ so large such that $\norm{\lambda_s\left( f_K \right) - f_K}_{\ell_1} < \ve$, and the result then holds for any finite $E\subseteq \Gamma$ as it must hold for all $s\in \Gamma$.\newline

   Now, we suppose that $\Gamma$ satisfies Reiter's condition. Let $\Gamma = \bigcup_{n\geq 1}E_n$, where each of the $E_n$ are finite, and $E_1\subseteq E_2\subseteq \cdots$ are nested. For each $E_n$, we may find a sequence of vectors $\left( f_{k,n} \right)_k$ such that $\norm{\lambda_s\left( f_{k,n} \right) - f_{k,n}}_{\ell_1} < 1/k$ for all $s\in E_n$.\newline

   We define $\mu_{n} = f_{n,n}$. Then, for any $s\in \Gamma$, we may find $N$ such that $s\in E_{k}$, for all $k\geq N$, and we know that by the definition, we have $\norm{\lambda_s\left( \mu_{N} \right) - \mu_{N}}_{\ell_1} < \frac{1}{N}$. The sequence $\left( \mu_{n} \right)_{n}$ then forms an approximate mean as in Definition \ref{def:approximate_mean}.
\end{proof}
\begin{definition}\label{def:almost_invariant_vector}
  Let $\Gamma$ be a group and $\lambda\colon \Gamma\rightarrow \mathcal{U}\left( \ell_2\left( \Gamma \right) \right)$ be the left-regular representation of $\Gamma$. We say $\lambda$ admits an \textit{almost-invariant vector} if there is a sequence of unit vectors $\left( \xi_n \right)_n$ in $\ell_2\left( \Gamma \right)$ such that, for all $s\in\Gamma$, we have
  \begin{align*}
    \norm{\lambda_s\left( \xi_n \right) - \xi_n}_{\ell_2} &\xrightarrow{n\rightarrow\infty} 0.
  \end{align*}
  %We call the sequence $\left( \xi_n \right)_n$ the almost-invariant vector for $\lambda$.
\end{definition}
\begin{theorem}[{\cite[Theorem A.5]{juschenko_amenability}}]\label{thm:almost_invariant_vector}
  Let $\Gamma$ be a group, and $\lambda\colon \Gamma\rightarrow \mathcal{U}\left( \ell_2\left( \Gamma \right) \right)$ be the left-regular representation of $\Gamma$.\newline

  Then, $\Gamma$ is amenable if and only if $\lambda$ admits an almost-invariant vector.
\end{theorem}
\begin{proof}
  Let $\Gamma$ be amenable. Then, by the results in Section \ref{sec:approximate_means}, we know that $\Gamma$ admits a Følner sequence, $\left( F_n \right)_{n}$, such that
  \begin{align*}
    \frac{\left\vert sF_n\triangle F_n \right\vert}{\left\vert F_n \right\vert} \rightarrow 0
  \end{align*}
  for all $s\in\Gamma$. We define the unit vectors $\xi_n$ by
  \begin{align*}
    \xi_n &= \frac{1}{\sqrt{\left\vert F_n \right\vert}} \1_{F_n}.
  \end{align*}
  Then, we have that
  \begin{align*}
    \norm{\lambda_s\left( \xi_n \right) - \xi_n}_{\ell_2}^2 &= \sum_{t\in\Gamma}\left\vert \lambda_x\left( \xi_n \right)\left( t \right) - \xi_n\left( t \right) \right\vert^2\\
                                                            &= \sum_{t\in\Gamma}\left\vert \lambda_s\left( \frac{1}{\sqrt{\left\vert F_n \right\vert}}\1_{F_n} \right)(t) - \frac{1}{\sqrt{\left\vert F_n \right\vert}}\1_{F_n}(t) \right\vert^2\\
                                                            &= \sum_{t\in\Gamma}\left\vert \frac{1}{\sqrt{\left\vert F_n \right\vert}}\1_{sF_n}(t) - \frac{1}{\sqrt{\left\vert F_n \right\vert}}\1_{F_n}(t) \right\vert^2\\
                                                            &= \frac{\left\vert sF_n\triangle F_n \right\vert}{\left\vert F_n \right\vert}\\
                                                            &\rightarrow 0.
  \end{align*}
  Thus, $\lambda$ admits an almost invariant vector.\newline

  Now, suppose there exists an almost-invariant vector $\left( \xi_n \right)_n\in \ell_2\left( \Gamma \right)$. We define $\mu_n = \xi_n^2$. From Hölder's inequality (Theorem \ref{thm:holder_inequality}), we know that $\mu_n\in \ell_1\left( \Gamma \right)$. Substituting into the definition of an approximate mean, we obtain
  \begin{align*}
    \norm{\lambda_s\left( \mu_n \right) - \mu_n}_{\ell_1} &= \sum_{t\in\Gamma}\left\vert \lambda_s\left( \xi_n^2 \right)\left( t \right) - \xi_n^2\left( t \right)\right\vert\\
                                                          &= \sum_{t\in\Gamma}\left\vert \left( \lambda_s\left( \xi_n \right)(t) - \xi_n\left( t \right) \right)\left( \lambda_s\left( \xi_n \right)(t) + \lambda_s\left( t \right) \right) \right\vert\\
                                                          &= \norm{\left( \lambda_s\left( \xi_n \right) - \xi_n \right)\left( \lambda_s\left( \xi_n \right) + \xi_n \right)}_{\ell_1}\\
                                                          &\leq\norm{\lambda_s\left( \xi_n \right) + \xi_n}_{\ell_2} \norm{\lambda_s\left( \xi_n \right) - \xi_n}_{\ell_2}\\
                                                          &\leq 2\norm{\lambda_s\left( \xi_n \right) - \xi_n}_{\ell_2}\\
                                                          &\rightarrow 0.
  \end{align*}
  Thus, $\left( \mu_n \right)_n$ is an approximate mean, hence amenable.
\end{proof}
\section{A Potpurri of Characterizations}%
Now, we may use the almost-invariant vectors criterion to prove amenability via various different methods. We start by diving into some theory behind representations and weak containment, then go into two criteria for amenability that are very intimately tied to the analytic properties of the left-regular representation.
\subsection{Weak Containment}%
Loosely speaking, weak containment is a type of approximation property for unitary representations of groups. What we will show in this subsection is that, if $\Gamma$ is a group, and the left-regular representation weakly contains the trivial representation, then the group $\Gamma$ is amenable.
\begin{definition}
  Let $\Gamma$ be a group, and let $\pi\colon \Gamma\rightarrow \mathcal{U}\left( \mathcal{H} \right)$ and $\rho\colon \Gamma\rightarrow \mathcal{U}\left( \mathcal{K} \right)$ be two unitary representations on separate Hilbert spaces $\mathcal{H}$ and $\mathcal{K}$. We say $\pi$ is weakly contained in $\rho$, written $\pi\prec \rho$, if, for any finite subset $E\subseteq \Gamma$ and any $\ve > 0$, and for all $\xi\in \mathcal{H}$, there are $\eta_1,\dots,\eta_k$ such that
  \begin{align*}
    \left\vert \iprod{\pi(g)\left( \xi \right)}{\xi} - \sum_{i=1}^{n} \iprod{\rho\left( g \right)\left( \eta_i \right)}{\eta_i} \right\vert < \ve
  \end{align*}
  for all $g \in E$.
\end{definition}
In order to prove the full weak containment result, we will need to make use of some lemmas that show certain convergence properties hold between inner products and norms.
\begin{lemma}[{\cite[Corollary F.1.5]{kazhdan_property_t}}]\label{lemma:complex_identities_left_regular_representation}
  Let $\xi$ be a unit vector, and let $\lambda_g\colon \ell_2\left( \Gamma \right)\rightarrow \ell_2\left( \Gamma \right)$ be given by $\lambda_g\left( \xi \right)\left( t \right) = \xi\left( g^{-1}t \right)$. Then,
  \begin{enumerate}[(a)]
    \item $\displaystyle \norm{\lambda_g\left( \xi \right) - \xi}_{\ell_2}^2 \leq 2\left\vert 1- \iprod{\lambda_g\left( \xi \right)}{\xi} \right\vert$
    \item $\displaystyle \left\vert 1- \iprod{\lambda_g\left( \xi \right)}{\xi} \right\vert \leq \norm{\lambda_g\left( \xi \right) - \xi}_{\ell_2}$.
  \end{enumerate}
\end{lemma}
\begin{proof}\hfill
  \begin{enumerate}[(a)]
    \item Directly calculating, we have
      \begin{align*}
        \norm{\lambda_g\left( \xi \right) - \xi}_{\ell_2}^2 &= \iprod{\lambda_g\left( \xi \right) - \xi}{ \lambda_g\left( \xi \right) - \xi }\\
                                                            &= \iprod{\lambda_g\left( \xi \right)}{\lambda_g\left( \xi \right)} + \iprod{\xi}{\xi} - \iprod{\lambda_g\left( \xi \right)}{\xi} - \iprod{\xi}{\lambda_g\left( \xi \right)}\\
                                                            &= \iprod{\lambda_g\left( \xi \right)}{\lambda_g\left( \xi \right)} + \iprod{\xi}{\xi} - 2\re\left( \iprod{\lambda_g\left( \xi \right)}{\xi} \right)\\
                                                            &= 2 - 2\re\left( \iprod{\lambda_g\left( \xi \right)}{\xi} \right)\\
                                                            &= 2\re\left( 1- \iprod{\lambda_g\left( \xi \right)}{\xi} \right)\\
                                                            &\leq 2 \left\vert 1- \iprod{\lambda_g\left( \xi \right)}{\xi} \right\vert.
      \end{align*}
    \item Similarly, direct calculation gives
      \begin{align*}
        \left\vert 1- \iprod{\lambda_g\left( \xi \right)}{\xi} \right\vert &= \left( 1- \iprod{\lambda_g\left( \xi \right)}{\xi} \right) \overline{\left( 1 - \iprod{\lambda_g\left( \xi \right)}{\xi} \right)}\\
                                                                           &= 1 - \overline{ \iprod{\lambda_g\left( \xi \right)}{\xi} } - \iprod{\lambda_g\left( \xi \right)}{\xi} + \left\vert \iprod{\lambda_g\left( \xi \right)}{\xi} \right\vert^2\\
                                                                           &\leq 2 - 2\re\left( \iprod{\lambda_g\left( \xi \right)}{\xi} \right)\\
                                                                           &= \norm{\lambda_g\left( \xi \right) - \xi}_{\ell_2}^2.
      \end{align*}
  \end{enumerate}
\end{proof}
\begin{lemma}[{\cite[Corollary F.1.5]{kazhdan_property_t}}]\label{lemma:l2_unit_vector}
  If $\lambda\colon\Gamma\rightarrow \mathcal{U}\left( \ell_2\left( \Gamma \right) \right)$ is the left-regular representation, then $1_{\Gamma}\prec \lambda$ if and only if, for every finite subset $S\subseteq \Gamma$ and every $\ve > 0$, there exists a unit vector $\xi\in \ell_2\left( \Gamma \right)$ such that
  \begin{align*}
    \norm{\lambda_s\left( \xi \right) - \xi}_{\ell_2} &< \ve.
  \end{align*}
\end{lemma}
\begin{proof}
  Suppose $1_{\Gamma}\prec \lambda$. Then, there exists a unit vector $\xi$ such that $\left\vert 1 - \iprod{\lambda_s\left( \xi \right)}{\xi} \right\vert < \ve^2/2$. So, by Lemma \ref{lemma:complex_identities_left_regular_representation} (a), we have $\norm{\lambda_s\left( \xi \right) - \xi}_{\ell_2} < \ve$.\newline

  Similarly, if $\norm{\lambda_s\left( \xi \right) - \xi}_{\ell_2} < \ve$, then we know from Lemma \ref{lemma:complex_identities_left_regular_representation} (b) that $ \left\vert 1 - \iprod{\lambda_s\left( \xi \right)}{\xi} \right\vert < \ve $.
\end{proof}
\begin{theorem}[{\cite[Theorem G.3.2]{kazhdan_property_t}}]
  Let $\Gamma$ be a discrete group. Then, $\Gamma$ is amenable if and only if $1_{\Gamma}\prec \lambda$, where $1_{\Gamma}$ is the trivial representation (see Example \ref{ex:some_representations}) and $\lambda$ is the left-regular representation.
\end{theorem}
\begin{proof}
  For one direction, we will show that $1_{\Gamma}\prec \lambda$ if and only if $\lambda$ admits an almost-invariant vector. By Theorem \ref{thm:almost_invariant_vector}, this means $\Gamma$ is amenable.\newline

  Let $\Gamma$ be amenable. Let $E\subseteq \Gamma$ be any finite set and $\ve > 0$, and let $\xi$ be almost-invariant for all $g\in E$ --- that is, $\norm{\lambda_g\left( \xi \right) - \xi}_{\ell_2} < \ve$ for all $g\in E$. Then, by Lemma \ref{lemma:complex_identities_left_regular_representation} (b), we must have
  \begin{align*}
    \left\vert 1 - \iprod{\lambda_g\left( \xi \right)}{\xi} \right\vert &\leq \norm{\lambda_g\left( \xi \right) - \xi}_{\ell_2}\\
    &< \ve.
  \end{align*}
  Thus, $1_{\Gamma}\prec \lambda$ when we take $n = 1$ and $\eta_1 = \xi$.\newline

  Now, in the reverse direction, we suppose that $1_{\Gamma}\prec \lambda$. Then, we know, by Lemma \ref{lemma:l2_unit_vector}, that for any finite subset $E\subseteq \Gamma$ and any $\ve > 0$, there is some unit vector $f\in \ell_2\left( \Gamma \right)$ such that $\norm{\lambda_s\left( f \right) - f}_{\ell_2} < \ve$ for all $s\in E$.\newline

  Set $g = \left\vert f \right\vert^2$. We have that $g\in \ell_1\left( \Gamma \right)$, and from Hölder's inequality, we obtain
  \begin{align*}
    \norm{\lambda_s\left( g \right) - g}_{\ell_1} &= \norm{\lambda_s\left( \left\vert f \right\vert^2 \right) - \left\vert f \right\vert^2}_{\ell_1}\\
                                                  &= \norm{\left( \lambda_s\left( f \right) - f \right)\left( \lambda_s\left( \overline{f} \right) + \overline{f} \right)}_{\ell_1}\\
                                                  &\leq \norm{\lambda_s\left( \overline{f} \right) + \overline{f}}_{\ell_2}\norm{\lambda_s\left( f \right) - f}_{\ell_2}\\
                                                  &\leq 2\norm{\lambda_s\left( f \right) - f}_{\ell_2}\\
                                                  & < 2\ve,
  \end{align*}
  meaning that $\Gamma$ satisfies Reiter's condition (Theorem \ref{thm:reiter}), and is thus amenable.
\end{proof}
\subsection{Kesten's Criterion}%
Kesten's criterion, expanded upon in \cite{kesten_random_walks} and \cite{kesten_means}, originated in the study of random walks on the generators of finitely generated groups.\newline

Consider a finitely supported probability measure $\mu$ on a (discrete, finitely generated) group $\Gamma$ with the property that $\mu\left( g \right) = \mu\left( g^{-1} \right)$ for all $g\in \supp\left( \mu \right)$. If our symmetric generating set $S$ is a subset of $\supp\left( \mu \right)$, then we may consider a random walk on the group by sampling elements of $S$ and concatenating them together --- then, we may, for instance, ask the probability of returning to $e_{\Gamma}$ in $n$ steps, for some $n$.\newline

Kesten showed that the probability of doing so in a certain number of steps was intimately tied to the spectral radius (or operator norm) of a matrix of probabilities.\newline

We will begin by showing some important results from the theory of self-adjoint operators before establishing Kesten's criterion for the special case where $\supp(\mu) = S$ and $\mu$ has a uniform distribution over $S$ --- i.e., $\mu\left( g \right) = \frac{1}{\left\vert S \right\vert}$. The more general case requires deeper results in spectral theory.
\begin{lemma}\label{lemma:norm_self_adjoint_operator}
  Let $\mathcal{H}$ be a Hilbert space, and let $T\in\B\left( \mathcal{H} \right)$ be a self-adjoint operator (see Definition \ref{def:distinguished_operators}). Then, the operator norm of $T$ is determined by
  \begin{align*}
    \norm{T}_{\op} &= \sup_{x\in S_{\mathcal{H}}} \left\vert \iprod{T\left( x \right)}{x} \right\vert.
  \end{align*}
\end{lemma}
\begin{proof}
  Using Cauchy--Schwarz, one of the directions immediately becomes clear:
  \begin{align*}
    \left\vert \iprod{T\left( x \right)}{x} \right\vert &\leq \norm{T\left( x \right)}\norm{x}\\
                                                        &\leq \norm{T}_{\op}\norm{x}^2\\
                                                        &= \norm{T}_{\op}.
  \end{align*}
  To establish the opposite direction requires a bit more work. First, we recall the definition of the operator norm, which states that
  \begin{align*}
    \norm{T}_{\op} &= \sup_{x,y\in S_{\mathcal{H}}}\left\vert \iprod{T\left( x \right)}{y} \right\vert.
  \end{align*}
  We set 
  \begin{align*}
    \alpha = \sup_{x\in S_{\mathcal{H}}} \left\vert \iprod{T\left( x \right)}{x} \right\vert.
  \end{align*}
  Notice that for any nonzero $x\in \mathcal{H}$, we have
  \begin{align*}
    \left\vert \iprod{T\left( \frac{x}{\norm{x}} \right)}{\frac{x}{\norm{x}}} \right\vert &\leq \alpha\\
    \left\vert \iprod{T\left( x \right)}{x} \right\vert &\leq \alpha \norm{x}^2.
  \end{align*}
  
  Recall that if $T$ is self-adjoint, then for any $x\in \mathcal{H}$, $ \iprod{T\left( x \right)}{x} $ is real. This can be shown relatively easily using the properties of adjoints and inner products:
  \begin{align*}
    \iprod{T\left( x \right)}{x} &= \iprod{x}{T^{\ast}\left( x \right)}\\
                                 &= \iprod{x}{T\left( x \right)}\\
                                 &= \overline{ \iprod{T\left( x \right)}{x} }.
  \end{align*}
  Now, let $x,y\in S_{\mathcal{H}}$. We may assume that $ \iprod{T\left( x \right)}{y} \in \R $, as we may multiply $x$ by $\sgn\left( \iprod{T\left( x \right)}{y} \right)$, where $\sgn(z) = \frac{|z|}{z}\in \C$, and by the Polarization Identity (Theorem \ref{thm:polarization}) and the fact that $T$ is self-adjoint, we get
  \begin{align*}
    \iprod{T\left( x \right)}{y} &= \frac{1}{4}\left( \iprod{T\left( x+y \right)}{x+y} - \iprod{T\left( x-y \right)}{x-y} \right).
  \end{align*}
  Thus, applying absolute values, we obtain
  \begin{align*}
    \left\vert \iprod{T\left( x \right)}{y}  \right\vert &= \left\vert \frac{1}{4}\left( \iprod{T\left( x+y \right)}{x+y} - \iprod{T\left( x-y \right)}{x-y} \right) \right\vert\\
                                                         &\leq \frac{1}{4}\left( \left\vert \iprod{T\left( x+y \right)}{x+y} \right\vert +  \left\vert \iprod{T\left( x-y \right)}{x-y} \right\vert\right)\\
                                                         &\leq \frac{\alpha}{4} \left( \norm{x+y}^2 + \norm{x-y}^2 \right)\\
                                                         &\leq \frac{\alpha}{4} \left( 2\norm{x}^2 + 2\norm{y}^2 \right)\\
                                                         &= \alpha.
  \end{align*}
  Thus,
  \begin{align*}
    \norm{T}_{\op} &= \sup_{x,y\in S_{\mathcal{H}}} \left\vert \iprod{T\left( x \right)}{y} \right\vert\\
                   &\leq \alpha,
  \end{align*}
  so
  \begin{align*}
    \norm{T}_{\op} &= \sup_{x\in S_{\mathcal{H}}} \left\vert \iprod{T\left( x \right)}{x} \right\vert.
  \end{align*}
\end{proof}
\begin{definition}[{\cite[139--140]{juschenko_amenability}}]
  Let $\lambda\colon \Gamma\rightarrow \mathcal{U}\left( \ell_2\left( \Gamma \right) \right)$ be the left-regular representation. For a finite set $E\subseteq \Gamma$, we define the \textit{Markov operator} $M(E)$ by
  \begin{align*}
    M(E) &= \frac{1}{\left\vert E \right\vert} \sum_{t\in E}\lambda_t.
  \end{align*}
\end{definition}
\begin{fact}
  For any $E\subseteq \Gamma$, $M(E)$ is a contraction.
\end{fact}
\begin{proof}
  Note that $\lambda_t$ is an isometry for any $t\in \Gamma$. This yields
  \begin{align*}
    \norm{M(E)}_{\op} &= \norm{\frac{1}{\left\vert E \right\vert} \sum_{t\in E}\lambda_t}_{\op}\\
                      &= \frac{1}{\left\vert E \right\vert}\norm{\sum_{t\in E}\lambda_t}_{\op}\\
                      &\leq \frac{1}{\left\vert E \right\vert}\sum_{t\in E}\norm{\lambda_t}_{\op}\\
                      &= 1.
  \end{align*}
\end{proof}
\begin{theorem}[{\cite[Theorem A.8]{juschenko_amenability}}]\label{thm:kesten_criterion}
  Let $\Gamma$ be a group with finite symmetric generating set $S$. Then, $\Gamma$ is amenable if and only if
  \begin{align*}
    \norm{M(S)}_{\op} &= 1.
  \end{align*}
\end{theorem}
\begin{proof}
  Let $\Gamma$ be amenable. Then, $\lambda$ admits an almost-invariant vector, $\left( \xi_n \right)_n\subseteq S_{\ell_2\left( \Gamma \right)}$. This gives
  \begin{align*}
    \norm{\lambda_s\left( \xi_n \right) - \xi_n}_{\ell_2} &\rightarrow 0
  \end{align*}
  for all $s\in \Gamma$. Therefore, by the reverse triangle inequality, we have
  \begin{align*}
    \left\vert \left(\norm{\left(\frac{1}{\left\vert S \right\vert}\sum_{t\in S}\lambda_t\right)\left(\xi_n\right)}_{\ell_2}\right)  - \norm{\xi_n}_{\ell_2}\right\vert &\leq \norm{\left(\frac{1}{\left\vert S \right\vert}\sum_{t\in S}\lambda_t\right)\left(\xi_n\right) - \xi_n}_{\ell_2}\\
                                                                                                                                                                                  &= \frac{1}{\left\vert S \right\vert}\norm{\left(\sum_{t\in S}\lambda_t\right)\left(\xi_n\right) - \left\vert S \right\vert\xi_n}_{\ell_2}\\
                                                                                                                        &\leq \frac{1}{\left\vert S \right\vert} \sum_{t\in S}\norm{\lambda_t\left(\xi_n\right) - \xi_n}_{\ell_2}\\
                                                                                                                        &\rightarrow 0,
  \end{align*}
  meaning that
  \begin{align*}
    \sup_{\xi\in S_{\ell_2\left(\Gamma\right)}} \norm{\left(\frac{1}{\left\vert S \right\vert}\sum_{t\in S}\lambda_t\right)\left(\xi\right)} &= \norm{\xi},
  \end{align*}
  and so the norm of the Markov operator is $1$.\newline

  Now, suppose $M(S) = 1$. Since $S$ is symmetric, $M(S)$ is self-adjoint, so by \ref{lemma:norm_self_adjoint_operator}, for any $n\in \N$, there is some $\xi_n\in S_{\ell_2\left( \Gamma \right)}$ such that
  \begin{align*}
    1-\frac{1}{n} &< \iprod{\left( \frac{1}{\left\vert S \right\vert}\sum_{t\in S}\lambda_t \right)\left( \xi_n \right)}{\xi_n}\\
                  &\leq \iprod{\left( \frac{1}{\left\vert S \right\vert}\sum_{t\in S}\lambda_t \right)\left( \left\vert \xi_n \right\vert \right)}{\left\vert \xi_n \right\vert}.
  \end{align*}
  Rearranging, we get
  \begin{align*}
    1- \iprod{\left( \frac{1}{\left\vert S \right\vert}\sum_{t\in S}\lambda_t \right)\left( \left\vert \xi_n \right\vert \right)}{\left\vert \xi_n \right\vert} &< \frac{1}{n}.
  \end{align*}
  Since $M(S)$ is a self-adjoint operator, we have
  \begin{align*}
    \re\left( \iprod{\left( \frac{1}{\left\vert S \right\vert}\sum_{t\in S}\lambda_t \right)\left( \left\vert \xi_n \right\vert \right)}{\left\vert \xi_n \right\vert} \right) &= \iprod{\left( \frac{1}{\left\vert S \right\vert}\sum_{t\in S}\lambda_t \right)\left( \left\vert \xi_n \right\vert \right)}{\left\vert \xi_n \right\vert}.
  \end{align*}
  From Lemma \ref{lemma:complex_identities_left_regular_representation}, we have that
  \begin{align*}
    \norm{\left( \frac{1}{\left\vert S \right\vert}\sum_{t\in S}\lambda_t \right)\left( \left\vert \xi_n \right\vert \right) - \left\vert \xi_n \right\vert} &\leq \frac{1}{\left\vert S \right\vert}\sum_{t\in S}\norm{\lambda_t\left( \left\vert \xi_n \right\vert \right) - \left\vert \xi_n \right\vert}\\
                                                                                                                                                                    &\leq \frac{1}{\left\vert S \right\vert}\sum_{t\in S}\sqrt{2} \left\vert 1 - \iprod{\lambda_t\left( \left\vert \xi_n \right\vert \right)}{\left\vert \xi_n \right\vert} \right\vert\\
                                                                                                                                                                    &= \sqrt{2}\left\vert 1 - \frac{1}{\left\vert S \right\vert} \sum_{t\in S} \iprod{\lambda_t\left( \left\vert \xi_n \right\vert \right)}{\left\vert \xi_n \right\vert} \right\vert\\
                                                                                                                                                                    &< \frac{1}{n}.
  \end{align*}
  Thus, $\lambda$ admits an almost-invariant vector, and hence is amenable by Theorem \ref{thm:almost_invariant_vector}.
\end{proof}
\subsection{Hulanicki's Criterion}%
Kesten's criterion is especially useful in establishing a similar result known as Hulanicki's criterion. Hulanicki's criterion uses a similar operator that depends on the left-regular representation, and shows that this operator's norm serves as a bound on the sum of any positive, finitely-supported function on the group $\Gamma$.
\begin{definition}[{\cite[141]{juschenko_amenability}}]
  Let $f\in \ell_1\left( \Gamma \right)$. We define the bounded operator $\lambda_{f(t)}$ by
  \begin{align*}
    \lambda_{f(t)} &= \sum_{t\in\Gamma}f(t)\lambda_t,
  \end{align*}
  where $\lambda_t$ denotes the left-regular representation of $\Gamma$ evaluated at $t$.
\end{definition}
\begin{theorem}[{\cite[Theorem A.11]{juschenko_amenability}}]
  If $\Gamma$ is a discrete group, then $\Gamma$ is amenable if and only if, for any positive, finitely-supported function $f\colon \Gamma\rightarrow \C$, we have
  \begin{align*}
    \sum_{t\in\Gamma}f(t) &\leq \norm{\lambda_{f(t)}}_{\op}.
  \end{align*}
\end{theorem}
\begin{proof}
  Suppose $\Gamma$ is amenable. Let $f\colon \Gamma\rightarrow \C$ be a positive, finitely supported function, and let $\left( F_n \right)_n$ be a Følner sequence in $\Gamma$ such that for any $s\in \supp(f)$, we have
  \begin{align*}
    \frac{\left\vert sF_n \triangle F_n\right\vert}{\left\vert F_n \right\vert} &\leq \frac{1}{n}.
  \end{align*}
  Letting
  \begin{align*}
    \xi_n &= \frac{1}{\sqrt{\left\vert F_n \right\vert}}\1_{F_n},
  \end{align*}
  we note that $\norm{\xi_n}_{\ell_2} = 1$, and that this is the exact same almost-invariant vector for $\lambda$ that we used in Theorem \ref{thm:almost_invariant_vector}.\newline

  We will now use the fact that, for any $T\in \B\left( \mathcal{H} \right)$,
  \begin{align*}
    \sup_{x\in S_{\mathcal{H}}}\left\vert \iprod{T(x)}{x} \right\vert &\leq \norm{T}_{\op},
  \end{align*}
  which follows from the definition of the operator norm on Hilbert spaces.\newline

  This gives, 
  \begin{align*}
    \left\vert \iprod{\left( \sum_{t\in\Gamma}f(t)\lambda_t \right)\left( \xi_n \right)}{\xi_n} \right\vert \leq \norm{\lambda_{f(t)}}_{\op},
  \end{align*}
  meaning that, since the quantity on the left side is positive
  \begin{align*}
    \sup \left\vert \iprod{\left( \sum_{t\in\Gamma}f(t)\lambda_t \right)\left( \xi_n \right)}{\xi_n} \right\vert &= \lim_{n\rightarrow\infty}\left\vert \iprod{\left( \sum_{t\in\Gamma} f(t)\lambda_t\right)\left( \xi_n \right)}{\xi_n} \right\vert \\
                                                                                                                 &\leq \norm{\lambda_{f(t)}}_{\op}.
  \end{align*}
  Now, notice that, since $\xi_n$ is the almost-invariant vector we constructed in the proof of Theorem \ref{thm:almost_invariant_vector}, we have that $\norm{\lambda_t\left( \xi_n \right)-\xi_n}_{\ell_2}\xrightarrow{n\rightarrow\infty} 0$, or that, for all $s\in\Gamma$, $\xi_n\left( t^{-1}s \right)\xrightarrow{n\rightarrow\infty} \xi_n\left( s \right)$. Therefore, 
  \begin{align*}
    \lim_{n\rightarrow\infty} \left\vert \iprod{\left( \sum_{t\in\Gamma}f(t)\lambda_t \right)\left( \xi_n \right)}{\xi_n} \right\vert &= \lim_{n\rightarrow\infty}\left\vert \sum_{t,s\in\Gamma}f(t)\lambda_t\left( \xi_n \right)\left( s \right)\overline{\xi_n}\left( s \right) \right\vert\\
                                                                                                                                      &= \lim_{n\rightarrow\infty}\left\vert \sum_{t,s\in\Gamma}f(t)\xi_n\left( t^{-1}s \right)\overline{\xi_n}\left( s \right) \right\vert\\
                                                                                                                                      &= \lim_{n\rightarrow\infty}\left\vert \sum_{t,s\in\Gamma}f(t)\left\vert \xi_n\left( s \right) \right\vert^2 \right\vert\\
                                                                                                                                      &= \sum_{t\in\Gamma}f(t)\left( \sum_{s\in\Gamma}\left\vert \xi_n\left( s \right) \right\vert^{2} \right)\\
                                                                                                                                      &= \sum_{t\in\Gamma}f(t).
  \end{align*}
  Therefore, there is some constant $C$ such that
  \begin{align*}
    \sum_{t\in\Gamma}f(t) &\leq C\norm{\lambda_{f(t)}}_{\op}.
  \end{align*}
  Now, to show that $C = 1$, we note that by the definition of convolution (see Definition \ref{def:group_star_algebra}), we have
  \begin{align*}
    \left( \sum_{t\in\Gamma}f(t) \right)^{n} &= \sum_{t\in\Gamma}\left( f\ast\cdots\ast f \right)\left( t \right).
  \end{align*}
  Similarly,
  \begin{align*}
    \left( \lambda_{f(t)} \right)^n &= \left( \sum_{t\in\Gamma}f(t)\lambda_t \right)^{n}\\
                                    &= \sum_{t\in\Gamma}\left( f\ast\cdots\ast f \right)\left( t \right)\lambda_t\\
                                    &= \lambda_{(f\ast\cdots\ast f)(t)}.
  \end{align*}
  Now, by the definition of the operator norm, we have
  \begin{align*}
    \norm{\lambda_{(f\ast\cdots\ast f)(t)}}_{\op} &= \norm{\left( \lambda_{f(t)} \right)^n}_{\op}\\
                                                    &\leq \norm{\lambda_{f(t)}}_{\op}^{n}.
  \end{align*}
  Thus, we have
  \begin{align*}
    \left( \sum_{t\in\Gamma}f(t) \right)^n &= \sum_{t\in\Gamma}\left( f\ast\cdots\ast f \right)\left( t \right)\\
                                           &\leq C\norm{\lambda_{\left( f\ast\cdots\ast f \right)\left( t \right)}}_{\op}\\
                                           &\leq C\left( \norm{\lambda_{f(t)}}_{\op}^n \right),
  \end{align*}
  giving
  \begin{align*}
    \sum_{t\in\Gamma}f(t) &\leq C^{1/n}\norm{\lambda_{f(t)}}_{\op}.
  \end{align*}
  Since $n$ was arbitrary, we have that $C = 1$.\newline

  Now, suppose that for all finitely supported $f$, we have
  \begin{align*}
    \sum_{t\in\Gamma}f(t) &\leq \norm{\lambda_{f(t)}}_{\op}.
  \end{align*}
  If $f = \1_{E}$ for some finite $E\subseteq \Gamma$, we see that
  \begin{align*}
    \norm{\lambda_{f(t)}}_{\op} &= \norm{\sum_{t\in\Gamma}f(t)\lambda_t}_{\op}\\
                                &= \norm{\sum_{t\in E}\lambda_t}_{\op}\\
                                &= \left\vert E \right\vert.
  \end{align*}
  Therefore, we have
  \begin{align*}
    \norm{\frac{1}{\left\vert E \right\vert}\sum_{t\in\Gamma}\lambda_t}_{\op} &= 1.
  \end{align*}
  By Kesten's criterion (Theorem \ref{thm:kesten_criterion}), we have that $\Gamma$ is amenable.
\end{proof}

\chapter{Staying Positive: Amenability in \texorpdfstring{$C^{\ast}$-Algebras}{C*-Algebras}}\label{ch:nuclearity}
Here, we will establish the equivalence between group amenability and certain properties of the group $\mathrm{C}^{\ast}$-algebra(s). The results in here will draw from a lot of theory that we discuss a bit more in depth in Chapter \ref{ch:operator_algebras}. Some excellent books on this topic include \cite{brown_and_ozawa} and \cite{completely_bounded_maps_and_operator_algebras}, both of which go deeper into the ramifications of the results we will present herein.
\section{Norms on the Group \texorpdfstring{$\ast$-Algebras}{*-Algebras}}%
From Definition \ref{def:group_star_algebra}, we know that for any group $\Gamma$, there is a free vector space, $\C\left[ \Gamma \right]$, consisting of finitely supported functions on $\Gamma$. Elements of $\C\left[ \Gamma \right]$ are finite sums of the form
\begin{align*}
  a &= \sum_{s\in\Gamma}a(s)\delta_s,
\end{align*}
where $\delta_s$ is the point mass function
\begin{align*}
  \delta_s\left( t \right) &= \begin{cases}
    1 & s = t\\
    0 & s\neq t
  \end{cases}.
\end{align*}
This admits a multiplication by convolution:
\begin{align*}
  f\ast g(s) &= \sum_{t\in\Gamma}f(t)g\left( t^{-1}s \right)\\
              &= \sum_{r\in\Gamma}f\left( sr^{-1} \right)g\left( r \right)
\end{align*}
and an involution
\begin{align*}
  f^{\ast}\left( t \right) &= \overline{f\left( t^{-1} \right)},
\end{align*}
which turn $\C\left[ \Gamma \right]$ into a $\ast$-algebra.\newline

Now, we are interested in applying norms on the group $\ast$-algebra, turning them into group $\mathrm{C}^{\ast}$-algebras. We will do this through the use of unitary representations.\newline

There is an intimate relationship between unitary representations of groups and unital representations (see Definition \ref{def:unital_representation}) of the group $\ast$-algebra generated by the group.
\begin{proposition}[{\cite[Proposition 7.2.46]{rainone_analysis}}]\label{prop:unital_unitary_representation}
  Let $\Gamma$ be a group and let $\mathcal{H}$ be a Hilbert space.
  \begin{enumerate}[(1)]
    \item If $u\colon \Gamma\rightarrow \mathcal{U}\left( \mathcal{H} \right)$ is a unitary representation of $\Gamma$, then $\pi_u\colon \C\left[ \Gamma \right]\rightarrow \B\left( \mathcal{H} \right)$, given by
      \begin{align*}
        \pi_u(a) &= \sum_{s\in\Gamma}a(s)u_s
      \end{align*}
      is a unital representation of $\Gamma$.
    \item If $\pi\colon \C\left[ \Gamma \right]\rightarrow \B\left( \mathcal{H} \right)$ is a unital representation, then $u\colon \Gamma\rightarrow \mathcal{U}\left( \mathcal{H} \right)$, given by
      \begin{align*}
        u(s) &\coloneq \pi\left( \delta_s \right),
      \end{align*}
      is a unitary representation of $\Gamma$.
  \end{enumerate}
\end{proposition}
\begin{proof}\hfill
  \begin{enumerate}[(1)]
    \item Via the universal property of the free vector space, we know that the map $s\mapsto u_s\in \B\left( \mathcal{H} \right)$ extends to a linear map $\pi_u\colon \C\left[ \Gamma \right]\rightarrow \B\left( \mathcal{H} \right)$. Now, we must ensure that this map is faithful to the underlying multiplication structure. Letting $s,t\in \Gamma$ be arbitrary, via the properties of unitary representations, we have
      \begin{align*}
        \pi_u\left( \delta_s\delta_t \right) &= \pi_u\left( \delta_{st} \right)\\
                                             &= u_{st}\\
                                             &= u_su_t\\
                                             &= \pi_u\left( \delta_s \right)\pi_u\left( \delta_t \right)\\
        \pi_u\left( \delta_{s}^{\ast} \right) &= \pi_u\left( \delta_{s^{-1}} \right)\\
                                              &= u_{s^{-1}}\\
                                              &= u_s^{\ast}\\
                                              &= \pi_u\left( \delta_s \right)^{\ast}.
      \end{align*}
      Therefore, via linearity, we obtain that $\pi_u$ is multiplicative and $\ast$-preserving.
    \item Every $\delta_s\in \C\left[ \Gamma \right]$ is a unitary element, and since unital $\ast$-homomorphisms preserve unitary elements (Fact \ref{fact:unitary_preservation}), we know that each $u(s)$ is unitary. Furthermore, for any $s,t\in\Gamma$, we have
      \begin{align*}
        u\left( st \right) &= \pi\left( \delta_{st} \right)\\
                           &= \pi\left( \delta_s\delta_t \right)\\
                           &= \pi\left( \delta_s \right)\pi\left( \delta_t \right)\\
                           &= u(s)u(t),
      \end{align*}
      meaning $u$ is a unitary representation.
  \end{enumerate}
\end{proof}
Now, using the interplay between unitary and unital representations of $\Gamma$ and $\C\left[ \Gamma \right]$ respectively, we may define two special $\mathrm{C}^{\ast}$-norms on $\C\left[ \Gamma \right]$. We will investigate the properties of their respective group $\mathrm{C}^{\ast}$-algebras.
\begin{proposition}
  Let $\Gamma$ be a group. If $\lambda\colon\Gamma\rightarrow \mathcal{U}\left( \ell_2\left( \Gamma \right) \right)$ is the left-regular representation (Theorem \ref{thm:left_regular_representation}), then $\lambda$ extends to an injective representation $\pi_{\lambda}\colon \C\left[ \Gamma \right]\rightarrow \B\left( \ell_2\left( \Gamma \right) \right)$, given by
  \begin{align*}
    \pi_{\lambda}(a) &= \sum_{s\in\Gamma}a(s)\lambda_s.
  \end{align*}
\end{proposition}
\begin{proof}
  Suppose $\pi_{\lambda}(a) = 0$ for some $a = \sum_{s\in\Gamma}a(s)\delta_s$ in $\C\left[ \Gamma \right]$. Taking the evaluation at $\delta_{e}$, we get
  \begin{align*}
    0 &= \pi_{\lambda}\left( a \right)\left( \delta_e \right)\\
      &= \left( \sum_{s\in\Gamma}a(s)\lambda_s \right)\left( \delta_e \right)\\
      &= \sum_{s\in\Gamma}a(s)\lambda_s\left( \delta_e \right)\\
      &= \sum_{s\in\Gamma}a(s)\delta_s.
  \end{align*}
  Since the $\set{\delta_t}_{t\in\Gamma}$ are linearly independent, we must have that $a(s) = 0$ for all $s\in\Gamma$, meaning $a = 0$.
\end{proof}
\begin{definition}\label{def:reduced_group_cstar_algebra}
  Define the $\mathrm{C}^{\ast}$-norm
  \begin{align*}
    \norm{a}_{\lambda} &\coloneq \norm{\pi_{\lambda}(a)}_{\op}
  \end{align*}
  on $\C\left[ \Gamma \right]$.\newline

  The completion of $\C\left[ \Gamma \right]$ with respect to $\norm{\cdot}_{\lambda}$ is known as the \textit{reduced group $\mathrm{C}^{\ast}$-algebra}, denoted $\mathrm{C}^{\ast}_{\lambda}\left( \Gamma \right)$.
\end{definition}
\begin{proposition}\label{prop:universal_group_cstar_algebra}
  Let $\Gamma$ be a group, and let $\C\left[ \Gamma \right]$ be the group $\ast$-algebra.\newline

  Define the \textit{universal norm} (or maximum norm) on $\C\left[ \Gamma \right]$ by
  \begin{align*}
    \norm{a}_{u} &\coloneq \sup\set{\norm{\pi(a)}_{\op} | \pi\colon \C\left[ \Gamma \right]\rightarrow \B\left( \mathcal{H}_{\pi} \right)\text{ is a unital representation}}.
  \end{align*}
  Then, this is a $\mathrm{C}^{\ast}$-norm on $\C\left[ \Gamma \right]$. The completion with respect to this norm yields the \textit{universal group $\mathrm{C}^{\ast}$-algebra}, and is denoted $\mathrm{C}^{\ast}\left( \Gamma \right)$.
\end{proposition}
\begin{proof}
  First, we show that the quantity
  \begin{align*}
    \norm{a}_{u}\coloneq \sup\set{\norm{\pi(a)}_{\op} | \pi\colon \C\left[ \Gamma \right]\rightarrow \B\left( \mathcal{H}_{\pi} \right)\text{ is a unital representation}}
  \end{align*}
  is finite (i.e., that the universal norm exists).\newline

  Note that for any representation $\pi\colon \C\left[ \Gamma \right]\rightarrow \B\left( \mathcal{H}_{\pi} \right)$, the elements $\pi\left( \delta_s \right)$ are unitary in $\B\left( \mathcal{H}_{\pi} \right)$, meaning they have norm $1$. Therefore, for a finitely supported function $a = \sum_{s\in\Gamma}a(s)\delta_s$, we have
  \begin{align*}
    \norm{\pi(a)}_{\op} &= \norm{\pi\left( \sum_{s\in\Gamma}a(s)\delta_s \right)}_{\op}\\
                        &= \norm{\sum_{s\in\Gamma}a(s)\pi\left( \delta_s \right)}_{\op}\\
                        &\leq \sum_{s\in\Gamma}\norm{a(s)\pi\left( \delta_s \right)}_{\op}\\
                        &= \sum_{s\in\Gamma}\left\vert a(s) \right\vert\norm{\pi\left( \delta_s \right)}_{\op}\\
                        &= \sum_{s\in\Gamma}\left\vert a(s) \right\vert,
  \end{align*}
  so that $\norm{a}_u\leq \sum_{s\in\Gamma}\left\vert a(s) \right\vert < \infty$.\newline

  That $\norm{\cdot}_{u}$ is a $\mathrm{C}^{\ast}$-seminorm follows from the fact that for any representation $\pi$ and any $a,b\in \Gamma$, we have
  \begin{align*}
    \norm{\pi\left( ab \right)}_{\op} &= \norm{\pi(a)\pi(b)}_{\op}\\
                                      &\leq \norm{\pi(a)}_{\op}\norm{\pi(b)}_{\op},
  \end{align*}
  so by taking the supremum over all representations, we obtain $\norm{ab}_{u}\leq \norm{a}_{u}\norm{b}_{u}$. Similarly, for any $a\in\Gamma$, we have
  \begin{align*}
    \norm{\pi(a)^{\ast}\pi(a)}_{\op} &= \norm{\pi(a)}_{\op}^2,
  \end{align*}
  so by taking the supremum over all representations, we obtain $\norm{a^{\ast}a} = \norm{a}^2$, showing that it is indeed a $\mathrm{C}^{\ast}$-seminorm.\newline

  To verify that $\norm{\cdot}_u$ is a norm, we note that if $\norm{a}_{u} = 0$, then since $\pi_{\lambda}$ is a representation, we must have $\norm{a}_{\lambda} = 0$, and since $\pi_{\lambda}$ is injective, it follows that $a = 0$. Thus, $\norm{\cdot}_u$ is a norm.
\end{proof}
The moniker ``universal'' is apt for the universal $\mathrm{C}^{\ast}$-algebra, as it admits a universal property.
\begin{theorem}[{\cite[Proposition 7.2.47]{rainone_analysis}}]
  Let $\Gamma$ be a discrete group. If $u\colon \Gamma\rightarrow \mathcal{U}\left( \mathcal{H} \right)$ is a unitary representation, then there is a contractive $\ast$-homomorphism $\pi_u\colon \mathrm{C}^{\ast}\left( \Gamma \right)\rightarrow \B\left( \mathcal{H} \right)$ such that $\pi_u\left( \delta_s \right) = u(s)$ for all $s\in\Gamma$.
  \begin{center}
    % https://tikzcd.yichuanshen.de/#N4Igdg9gJgpgziAXAbVABwnAlgFyxMJZABgBpiBdUkANwEMAbAVxiRAGEA9YAHR7rg4AvnwYwAZjgAUfAOJ0Atgrp8ATlgDmACxwBKEENLpMufIRQBGclVqMWbPgCFRE6X2U4tAY0bAAEiI86tp6BkYgGNh4BERWFjb0zKyIIO50nj4MwACqgWKSMjwe3r4Bapo6+obGUWZEZPHUifYpcorKBjYwUBrwRKDiqhAKSADM1DgQSABM1Ax0AEYwDAAKJtHmIME6IE12ySBMYQNDI4hkIJNIVrZJDjxoWAD6R9Ugg8NjE1PnE3RYDDYWggEAA1sd3qcZt9rn8AUCQeChBQhEA
    \begin{tikzcd}
      \mathrm{C}^{\ast}\left(\Gamma\right) \arrow[r, "\pi_u"] & \B\left(\mathcal{H}\right)                          \\
      \Gamma \arrow[r, "u"'] \arrow[u, hook]         & \mathcal{U}\left(\mathcal{H}\right) \arrow[u, hook]
    \end{tikzcd}
  \end{center}
\end{theorem}
\begin{proof}
  From Proposition \ref{prop:unital_unitary_representation}, we know that there is a unital representation $\pi_u\colon \C\left[ \Gamma \right]\rightarrow \B\left( \mathcal{H} \right)$ that extends $u\colon \Gamma\rightarrow \mathcal{U}\left( \mathcal{H} \right)$.\newline

  By the definition of the universal norm, it follows that $\norm{\pi_u(a)}_{\op}\leq \norm{a}_{u}$. The continuous extension $\pi_u\colon \mathrm{C}^{\ast}\left( \Gamma \right)\rightarrow \B\left( \mathcal{H} \right)$ is thus a contractive $\ast$-homomorphism.
\end{proof}
\section{Ordering Properties of \texorpdfstring{$\mathrm{C}^{\ast}$-Algebras}{C*-Algebras}}%
Recall from Definition \ref{def:positive_operators} that the space $\B\left( \mathcal{H} \right)_{\sa}$ admits an order structure --- we say that an operator is \textit{positive} if, for any $\xi\in \mathcal{H}$, we have $ \iprod{T\left( \xi \right)}{\xi} \geq 0 $. It can be shown that any positive operator is of the form $T = S^{\ast}S$, where $S$ is any operator on $\mathcal{H}$.\newline

Similarly, when we discussed algebras, we discussed a definition of positivity very similar to the case of bounded operators on Hilbert spaces (Definition \ref{def:distinguished_elements_of_algebras}).\newline

In this section, we investigate the ordering aspects of $\mathrm{C}^{\ast}$-algebras in depth, and how to apply the properties of their spectra towards understanding ordering and positivity. This will lead naturally to discussions of positive and completely positive (linear) maps between $\mathrm{C}^{\ast}$-algebras in the following section, paving the way to the cornucopia of characterizations of amenability that $\mathrm{C}^{\ast}$-algebras admit.
\begin{definition}
  If $A$ is a $\ast$-algebra, and $a\in A$, then $a$ can be written as $h + ik$, where $h,k\in A_{\sa}$ are defined by
  \begin{align*}
    h &= \frac{1}{2}\left( a + a^{\ast} \right)\\
    k &= \frac{i}{2}\left( a^{\ast}-a \right).
  \end{align*}
  This is known as the \textit{Cartesian decomposition} of $a$.
\end{definition}
\begin{proposition}[{\cite[Proposition 7.3.61]{rainone_analysis}}]\label{prop:self_adjoint_positive_decomposition}
  Let $A$ be a $\mathrm{C}^{\ast}$-algebra, and let $h\in A_{\sa}$. Then, there exist unique positive elements $p,q\in A_{+}$ such that
  \begin{enumerate}[(a)]
    \item $h = p-q$;
    \item $pq = 0$;
    \item $\sigma\left( p \right),\sigma\left( q \right)\subseteq [0,\infty)$.
  \end{enumerate}
\end{proposition}
\begin{proof}
  Assume $A$ is unital. Let $\phi_h\colon C\left( \sigma\left( h \right) \right)\rightarrow \mathrm{C}^{\ast}\left( h,1_A \right)$ be the continuous functional calculus at $h$. Since $\sigma\left( h \right)\subseteq \R$ (Proposition \ref{prop:spectra_cstar_algebras}), we consider the continuous functions
  \begin{align*}
    f(t) &= \max\set{t,0};\\
    g(t) &= \max\set{-t,0}.
  \end{align*}
  Note that $fg = 0$ and $\id_{\sigma\left( h \right)} = f-g$. Setting $p = \phi_h(f)$ and $q = \phi_h(g)$. Since $f$ and $g$ are positive, and the continuous functional calculus is a $\ast$-homomorphism, both $p$ and $q$ are positive.\newline

  Moreover, by spectral mapping (Theorem \ref{thm:continuous_functional_calculus}), we have
  \begin{align*}
    \sigma\left( p \right) &= \sigma\left( f\left( h \right) \right)\\
                           &= f\left( \sigma\left( h \right) \right)\\
                           &\subseteq [0,\infty),
  \end{align*}
  and similarly for $\sigma\left( q \right)$. Furthermore,
  \begin{align*}
    pq &= \phi_h\left( f \right)\phi_h\left( g \right)\\
       &= \phi_h\left( fg \right)\\
       &= \phi_h\left( 0 \right)\\
       &= 0
  \end{align*}
  and
  \begin{align*}
    h &= \phi_h\left( \id_{\sigma\left( h \right)} \right)\\
      &= \phi_h\left( f-g \right)\\
      &= \phi_h\left( f \right)-\phi_h\left( g \right)\\
      &= p-q.
  \end{align*}
  Since $\phi_h$ is also isometric, we have
  \begin{align*}
    \norm{h} &= \norm{\phi_h\left( \id_{\sigma\left( h \right)} \right)}\\
             &= \norm{\id_{\sigma\left( h \right)}}_{u}\\
             &= \max\set{\norm{f}_{u},\norm{g}_{u}}\\
             &= \max\set{\norm{\phi_h(f)},\norm{\phi_h(g)}}\\
             &= \max\set{\norm{p},\norm{q}}.
  \end{align*}
  Now we show uniqueness. Let $x,y\in A_{+}$ be such that $h = x-y$, $xy = 0$, and $\sigma\left( x \right),\sigma\left( y \right)\subseteq [0,\infty)$. By induction, we have $h^n = x^n + \left( -y \right)^n$, so for any polynomials $p\in \C\left[ x \right]$ with nonconstant term, we have
  \begin{align*}
    p\left( h \right) = p\left( x \right) + p\left( -y \right).
  \end{align*}
  Since $f(0) = 0$, there is a sequence of polynomials with nonconstant terms, $\left( p_n \right)_n$, such that $\left( p_n \right)_n\rightarrow f$ uniformly on the compact set (Theorem \ref{thm:spectrum_of_cstar_algebras})
  \begin{align*}
    K &= \sigma\left( h \right)\cup \sigma\left( x \right)\cup \sigma\left( -y \right).
  \end{align*}
  Applying the functional calculus at $h$, $x$, and $y$, we have
  \begin{align*}
    p &= f(h)\\
      &= \lim_{n\rightarrow\infty}p_n(h)\\
      &= \lim_{n\rightarrow\infty} \left( p_n\left( x \right)-p_n\left( y \right) \right)\\
      &= f(x) + f(-y).
  \end{align*}
  Now, since $\sigma\left( x \right)\subseteq [0,\infty)$, and $f(t) = t$ on $[0,\infty)$, we have $f(x) = x$. Additionally, since $\sigma\left( -y \right) = -\sigma\left( y \right)$, by Theorem \ref{thm:continuous_functional_calculus} part (3), we have $\sigma\left( -y \right)\subseteq (-\infty,0]$, so $f\left( -y \right) = 0$. Thus, $p = x$, and then $q = p-h = y$.\newline

  If $A$ is nonunital, then we use the functional calculus on $\phi_h\colon C\left( \sigma\left( h \right) \right)\rightarrow \mathrm{C}^{\ast}\left( h,1_{\widetilde{A}} \right)$ in the unitization. Since $0\in \sigma\left( h \right)$ and $f(0) = g(0) = 0$, we have that $p$ and $q$ belong to $A$, by Theorem \ref{thm:continuous_functional_calculus} part (6).
\end{proof}
We know that the self-adjoint elements of a $\mathrm{C}^{\ast}$-algebra have real spectrum --- it is possible to show that for elements that have positive spectrum (which we will soon show are the positive elements), they are preserved under the traditional operations that define a cone (Example \ref{ex:ordered_vector_space}).
\begin{lemma}[{\cite[Lemma 7.3.63]{rainone_analysis}}]\label{lemma:spectrum_conic_operations}
  Let $A$ be a $\mathrm{C}^{\ast}$-algebra.
  \begin{enumerate}[(1)]
    \item If $\sigma\left( a \right)\subseteq [0,\infty)$, and $t\geq 0$, then $\sigma\left( ta \right)\subseteq [0,\infty)$.
    \item Assume $A$ is unital, and let $a\in A_{\sa}$, $t \geq 0$, such that $\norm{a}\leq t$. Then, $\sigma\left( a \right)\subseteq [0,\infty)$ if and only if $\norm{t1_A - a} \leq t$.
    \item Let $a,b\in A_{\sa}$ with $\sigma\left( a \right),\sigma\left( b \right)\subseteq [0,\infty)$. Then, $\sigma\left( a+b \right)\subseteq [0,\infty)$.
  \end{enumerate}
\end{lemma}
\begin{proof}\hfill
  \begin{enumerate}[(1)]
    \item By Theorem \ref{thm:continuous_functional_calculus} (3), we have $\sigma\left( ta \right) = t\sigma\left( a \right)\subseteq [0,\infty)$.
    \item Since $a = a^{\ast}$ and $\norm{a}\leq t$, we have $\sigma\left( a \right)\subseteq [-t,t]$. Let $\phi_a\colon C\left( \sigma\left( a \right) \right)\rightarrow \mathrm{C}^{\ast}\left( a,1_A \right)\subseteq A$ be the continuous functional calculus at $a$. We have
      \begin{align*}
        \norm{t1_A - a} &= \norm{\phi_{a}\left( t\1_{\sigma\left( a \right)} - \iota \right)}\\
                        &= \norm{t\1_{\sigma\left( a \right)}-\iota}_{u}\\
                        &\leq 1,
      \end{align*}
      which holds if and only if $\left\vert t - \lambda \right\vert \leq t$ for all $\lambda\in \sigma\left( a \right)$, which holds if and only if $\sigma\left( a \right)\subseteq [0,2t]$.
    \item We may assume that $A$ is unital. From (2), we have that $\norm{\left( \norm{a}1_A \right) - a}\leq \norm{a}$, and $\norm{\left( \norm{b}1_B \right) - b}\leq \norm{b}$. So,
      \begin{align*}
        \norm{\left( \norm{a} + \norm{b} \right)1_A - \left( a+b \right)} &\leq \norm{\left( \norm{a}1_A \right)-a} + \norm{\left( \norm{b}1_B \right) - b}\\
                                                                          &\leq \norm{a} + \norm{b}.
      \end{align*}
      Since $\norm{a +b}\leq \norm{a} + \norm{b}$, setting $t \coloneq \norm{a} + \norm{b}$, we have $\sigma\left( a + b \right)\subseteq [0,\infty)$.
  \end{enumerate}
\end{proof}
\begin{theorem}[{\cite[Theorem 7.3.64]{rainone_analysis}}]
  Let $A$ be a $\mathrm{C}^{\ast}$-algebra. Then, $a\in A_{+}$ if and only if $a\in A_{\sa}$ and $\sigma\left( a \right)\subseteq [0,\infty)$.
\end{theorem}
\begin{proof}
  Let $a$ be self-adjoint with $\sigma\left( a \right)\subseteq [0,\infty)$. Note that $a$ is normal. Let $A$ be unital, and let $\phi_a\colon C\left( \sigma\left( a \right) \right)\rightarrow \mathrm{C}^{\ast}\left( a,1_A \right)$ be the continuous functional calculus at $a$. Since $\sigma\left( a \right)\subseteq [0,\infty)$, the function $f\left( t \right) = \sqrt{t}$ is well-defined, self-adjoint, and continuous on $\sigma\left( a \right)$. Set $b = \phi_a\left( f \right) = f(a)$. Then,
  \begin{align*}
    a &= \phi_a\left( \iota \right)\\
      &= \phi_a\left( f^2 \right)\\
      &= \phi_a\left( f \right)\phi_a\left( f \right)\\
      &= b^2.
  \end{align*}
  Since $b$ is self-adjoint, as it is the $\ast$-homomorphic image of a self-adjoint element, we have $a = b^{\ast}b\in A_{+}$. If $A$ is nonunital, then we look at the continuous functional calculus on the unitization, $\phi_a\colon C\left( \sigma\left( a \right) \right)\rightarrow \mathrm{C}^{\ast}\left( a,1_{\widetilde{A}} \right)$, and note that $f(0) = 0$, so $b\in \mathrm{C}^{\ast}\left( a \right)\subseteq A$ by Theorem \ref{thm:continuous_functional_calculus} (6).\newline

  Now, assume $a = b^{\ast}b$ for some $b\in A$. Since $a$ is self-adjoint, we must show that $\sigma\left( a \right)\subseteq [0,\infty)$. We may assume $A$ is unital, and write $a = p-q$, where $p$ and $q$ are in Proposition \ref{prop:self_adjoint_positive_decomposition}. Set $c = bq$.\newline

  Then, we have
  \begin{align*}
    c^{\ast}c &= \left( bq \right)^{\ast}bq\\
              &= q^{\ast}b^{\ast}bq\\
              &= qaq\\
              &= q\left( p-q \right)q\\
              &= -q^3,
  \end{align*}
  since $pq = 0$. Thus, from Theorem \ref{thm:continuous_functional_calculus} (3), and the result from Proposition \ref{prop:self_adjoint_positive_decomposition}, we have
  \begin{align*}
    \sigma\left( c^{\ast}c \right)&=\sigma\left( -q^3 \right)\\
                                  &= -\sigma\left( q \right)^3\\
                                  &\subseteq (-\infty,0].
  \end{align*}
  Thus, from Fact \ref{fact:spectrum_commutative} and Theorem \ref{thm:continuous_functional_calculus} (3), we have $\sigma\left( -cc^{\ast} \right)\subseteq [0,\infty)$.\newline

  We may write $c = h + ik$ via the Cartesian decomposition, and compute
  \begin{align*}
    c^{\ast}c + cc^{\ast} &= \left( h + ik \right)^{\ast}\left( h + ik \right) + \left( h + ik \right)\left( h + ik \right)^{\ast}\\
                          &= 2\left( h^2 + k^2 \right).
  \end{align*}
  Thus, we have that $c^{\ast}c = 2h^2 + 2k^2 + \left( -cc^{\ast} \right)$. Therefore, by Lemma \ref{lemma:spectrum_conic_operations}, we also have that $\sigma\left( c^{\ast}c \right)\subseteq [0,\infty)$, so $\sigma\left( c^{\ast}c \right) = \set{0}$. Thus, by Proposition \ref{prop:normal_spectral_radius}, we have
  \begin{align*}
    \norm{c}^2 &= \norm{c^{\ast}c}\\
               &= r\left( c^{\ast}c \right)\\
               &= 0,
  \end{align*}
  so $-q^3 = c^{\ast}c = 0$. Thus, $q = 0$, so $a = p$ and $\sigma\left( a \right) = \sigma\left( p \right)\subseteq [0,\infty)$.
\end{proof}
We may now show that $A_{+}$ is a cone in $A_{\sa}$, which allows us to define an ordering on $A$.
\begin{corollary}[{\cite[Corollary 7.3.66]{rainone_analysis}}]
  Let $A$ be a $\mathrm{C}^{\ast}$-algebra. The collection of positive elements, $A_{+}$, is a generating norm-closed cone in $A_{\sa}$.
\end{corollary}
\begin{proof}
  If $a,-a\in A_{+}$, then $\sigma\left( a \right)\subseteq [0,\infty)$ and $\sigma\left( -a \right) \subseteq [0,\infty)$. By Theorem \ref{thm:continuous_functional_calculus} (3), we also have $\sigma\left( a \right) = -\sigma\left( -a \right)\subseteq (-\infty,0]$, meaning that $\sigma\left( a \right) = \set{0}$. Thus, by Proposition \ref{prop:normal_spectral_radius}, we have $r(a) = \norm{a} = 0$, so $a = 0$.\newline

  From Proposition \ref{prop:self_adjoint_positive_decomposition}, we know that $A_{+}$ generates $A_{\sa}$.\newline

  Now, we show that $A_{+}$ is closed. Assume $A$ is unital, and let $\left( a_n \right)_n$ be a sequence in $A_{+}$ converging to $a\in A$. Then,
  \begin{align*}
    a &= \lim_{n\rightarrow\infty}\left( a_n \right)_n\\
      &= \lim_{n\rightarrow\infty}\left( a_n^{\ast} \right)_n\\
      &= a^{\ast},
  \end{align*}
  so $a$ is self-adjoint. Let $C > 0$ be such that $\norm{a_n}\leq C$ for all $n\geq 1$. Then, $\norm{a}\leq C$, and $\norm{C1_A - a} = \lim_{n\rightarrow\infty}\norm{C1_A - a_n}\leq C$, so $a\in A_{+}$.
\end{proof}
\begin{definition}\label{def:ordering_of_cstar_algebras}
  Let $A$ be a $\mathrm{C}^{\ast}$-algebra. We define an ordering on $A_{\sa}$ by $x\leq y$ if and only if $y-x\in A_{+}$.
\end{definition}
\begin{proposition}[{\cite[Proposition 7.3.67]{rainone_analysis}}]
  The ordering on $A_{\sa}$ satisfies the following.
  \begin{enumerate}[(1)]
    \item If $x\leq y$ in $A_{\sa}$, then for any $z\in A$, $z^{\ast}xz \leq z^{\ast}yz$.
    \item If $A$ is unital, $x\in A_{\sa}$, and $s,t\in \R$, then $s1_A \leq x \leq t1_A$ if and only if $\sigma\left( x \right)\subseteq [s,t]$.
    \item If $A$ is unital, and $x\in A_{\sa}$, then $-\norm{x}1_A \leq x \leq \norm{x}1_A$.
    \item For all $a\geq 0$, $a\leq 1_A$ if and only if $\norm{a}\leq 1$.
    \item If $0\leq a \leq b$, then $\norm{a}\leq \norm{b}$.
    \item If $a,b\in A$, then $0\leq b^{\ast}a^{\ast}ab \leq \norm{a}^2b^{\ast}b$.
  \end{enumerate}
\end{proposition}
\begin{proof}\hfill
  \begin{enumerate}[(1)]
    \item If $x\leq y$, then $y-x = a^{\ast}a$ for some $a\in A$. Thus,
      \begin{align*}
        z^{\ast}yz - z^{\ast}xz &= z^{\ast}\left( y-x \right)z\\
                                &= z^{\ast}a^{\ast}az\\
                                &= \left( az \right)^{\ast}az\\
                                &\in A_{+},
      \end{align*}
      so $z^{\ast}xz \leq z^{\ast}yz$.
    \item We have $s1_A - x$ is self-adjoint, so it has a real spectrum. Using Theorem \ref{thm:continuous_functional_calculus} (3), we have
      \begin{align*}
        s1_A \leq x &\leftrightarrow x-s1_A \in A_{+}\\
                    &\Leftrightarrow \sigma\left( x-s1_A \right)\subseteq [0,\infty)\\
                    &\Leftrightarrow \sigma\left( x \right)-s \subseteq [0,\infty)\\
                    &\Leftrightarrow \sigma\left( x \right)\subseteq [s,\infty).
      \end{align*}
      Similarly, if $x\leq t1_A$, then $\sigma\left( x \right)\subseteq (-\infty,t]$. Thus, $s1_A \leq x \leq t1_A$ if and only if $\sigma\left( x \right)\subseteq [s,t]$.
    \item Since $\sigma\left( x \right)\subseteq \left[-\norm{x},\norm{x}\right]$, this follows from (2).
    \item Taking $s = 0$ and $t = 1$ in (2), we have
      \begin{align*}
        0\leq a \leq 1 &\Leftrightarrow \sigma\left( a \right)\subseteq [0,1]\\
                       &\Leftrightarrow r(a)\leq 1\\
                       &\Leftrightarrow \norm{a}\leq 1,
      \end{align*}
      where we use Proposition \ref{prop:normal_spectral_radius}.
    \item From (3), we have that $0\leq a \leq b \leq \norm{b}1_A$, so $\sigma\left( a \right)\subseteq [0,\norm{b}]$. Since $a$ is normal, $\norm{a}\in \sigma\left( a \right)$, so $\norm{a}\leq \norm{b}$.
    \item If $A$ is unital, then by the $\mathrm{C}^{\ast}$-identity, we have $a^{\ast}a\leq \norm{a}^21_A$. We then apply (1) with $z = b$ to get $b^{\ast}a^{\ast}ab \leq \norm{a}^2b^{\ast}b$. If $A$ is nonunital, then we operate in the unitization of $A$, $\widetilde{A}$.
  \end{enumerate}
\end{proof}
\begin{corollary}\label{cor:positive_map_order_preserving}
  Let $\phi\colon A\rightarrow B$ be a linear map between $\mathrm{C}^{\ast}$-algebras. If $\phi\left( A_{+} \right)\subseteq B_{+}$, then $\phi$ is order-preserving.
\end{corollary}
\begin{proof}
  If $x\leq y$ in $A$, then $y-x\in A_{+}$. Thus, $\phi\left( y-x \right)\in B_{+}$, meaning $\phi\left( y \right)-\phi\left( x \right)\in B_{+}$, whence $\phi\left( x \right)\leq \phi\left( y \right)$.
\end{proof}

\section{Positive Maps in \texorpdfstring{$\mathrm{C}^{\ast}$-Algebras}{C*-Algebras}}%
Now that we have discussed positivity of elements in $\mathrm{C}^{\ast}$-algebras, we may discuss linear maps that preserve positivity. Note that $\ast$-homomorphisms preserve positivity and order (Fact \ref{fact:positivity_homomorphisms}), so we are interested in linear maps with a more general domain than just $\mathrm{C}^{\ast}$-algebras, and specifically a particular class of map that preserves positivity after amplification to matrix algebras.\newline

Recall that a linear map $\phi\colon A\rightarrow B$ between algebras is called positive if $\phi\left( A_{+} \right)\subseteq \phi\left( B \right)_{+}$. In this section, we will prove a result on a class of maps known as \textit{completely positive} maps --- specifically, that all such maps are compressions of $\ast$-homomorphisms to special subspaces of $\mathrm{C}^{\ast}$-algebras. This will lend itself to a discussion of nuclearity in $\mathrm{C}^{\ast}$-algebras.\newline

A lot of this section will follow the exposition in \cite{completely_bounded_maps_and_operator_algebras}.
\begin{definition}\label{def:operator_systems_operator_spaces}
  Let $A$ be a unital $\mathrm{C}^{\ast}$-algebra.
  \begin{itemize}
    \item A linear subspace $M\subseteq A$ is known as an \textit{operator space}.
    \item A linear subspace $S\subseteq A$ is known as an \textit{operator system} if $S$ is an operator space such that, for all $a\in S$, $a^{\ast}\in S$ and $1_A \in S$.
  \end{itemize}
  Note that all unital $\mathrm{C}^{\ast}$-algebras are operator spaces and operator systems.
\end{definition}
\begin{remark}\label{rem:positive_elements_operator_system}
  If $S$ is an operator system, we are always able to write $h\in S$ as a difference of two positive elements via the following decomposition:
  \begin{align*}
    h &= \frac{1}{2}\left( \norm{h}1_A + h \right) - \frac{1}{2}\left( \norm{h}1_A - h \right).
  \end{align*}
\end{remark}
The location of what we will study in this section is the amplification of a Hilbert space $\mathcal{H}$.
\begin{definition}
  Let $\mathcal{H}$ be a Hilbert space. The \textit{$n$-fold amplification} of $\mathcal{H}$, denoted $\mathcal{H}^{(n)}$, is the space of all vectors
  \begin{align*}
    \begin{pmatrix}\xi_1\\\vdots\\\xi_n\end{pmatrix}
  \end{align*}
  subject to the inner product
  \begin{align*}
    \iprod{ \begin{pmatrix}\xi_1\\\vdots\\\xi_n\end{pmatrix} }{ \begin{pmatrix}\eta_1\\\vdots\\\eta_n\end{pmatrix} } &= \sum_{j=1}^{n} \iprod{\xi_j}{\eta_j}.
  \end{align*}
\end{definition}
\begin{theorem}
  There is a $\ast$-isomorphism between $\Mat_n\left( \B\left( \mathcal{H} \right) \right)$ (as in Theorem \ref{thm:matrix_algebras_tensor_product}) and the space $\B\left( \mathcal{H}^{(n)} \right)$.
\end{theorem}
\begin{proof}
  For any $\left( T_{ij} \right)_{ij}\in \Mat_n\left( \B\left( \mathcal{H} \right) \right)$, we define the linear map $T\in \B\left( \mathcal{H}^{(n)} \right)$ by
  \begin{align*}
    T\left( \begin{pmatrix}h_1\\\vdots\\h_n\end{pmatrix} \right) &= \begin{pmatrix}\sum_{j=1}^{n}T_{1j}\left( h_j \right) \\ \vdots \\ \sum_{j=1}^{n}T_{nj}\left( h_j \right)\end{pmatrix}.
  \end{align*}
  The map $\left( T_{ij} \right)_{ij} \mapsto T$ is the desired $\ast$-homomorphism.
\end{proof}
\begin{remark}
  The positive cone on $\Mat_n\left( \B\left( \mathcal{H} \right) \right)$ is defined by $\left( T_{ij} \right)_{ij}\in \Mat_n\left( \B\left( \mathcal{H} \right) \right)_{+}$ if and only if, for all $x_1,\dots,x_n\in \mathcal{H}$ and all $v\in \C^n$,
  \begin{align*}
    \iprod{\left( \iprod{T_{ij}\left( x_j \right)}{x_i} \right)_{ij} \left( v \right)}{v} &\geq 0.
  \end{align*}
  In other words, a matrix of operators in $\Mat_n\left( \B\left( \mathcal{H} \right) \right)$ is positive if and only if its matrix representation, $\left( \iprod{T_{ij}\left( x_j \right)}{x_i} \right)_{ij}$, is positive in $\Mat_n\left( \C \right)$, for all such matrix representations.
\end{remark}
As it turns out, any $\mathrm{C}^{\ast}$-algebra can be faithfully represented as some $\mathrm{C}^{\ast}$-subalgebra of $\B\left( \mathcal{H} \right)$. This follows from the GNS construction, which we state without proof. An outline of a proof can be found in \cite[Section II.6.4]{blackadar_operator_algebras}.
\begin{theorem}[GNS Construction]\label{thm:gns_construction}
  Let $A$ be a $\mathrm{C}^{\ast}$-algebra, and let $\phi\colon A\rightarrow \C$ be a positive linear functional.\newline

  Then, there exists a Hilbert space $\mathcal{H}$, a vector $\xi_{\phi}\in \mathcal{H}$, and an isometric $\ast$-homomorphism $\pi\colon A\rightarrow \B\left( \mathcal{H} \right)$ such that $\phi(a) = \iprod{\pi\left( a \right)\left( \xi_{\phi} \right)}{\xi_{\phi}}$ for all $a\in A$, the subspace
  \begin{align*}
    \left[ \pi\left( A \right)\xi_{\phi} \right] \coloneq \set{\pi(a)\left( \xi_{\phi} \right) | a\in A}
  \end{align*}
  is dense in $\mathcal{H}$, and $\norm{\phi}_{\op} = \norm{\xi_{\phi}}^2$. We call the triplet $\left( \pi,\xi_{\phi},\mathcal{H} \right)$ a \textit{GNS representation} of the $\mathrm{C}^{\ast}$-algebra $A$.
\end{theorem}
Due to the GNS construction, we are able to turn $\Mat_n\left( A \right)$ into a $\mathrm{C}^{\ast}$-algebra by identifying $A$ as a $\ast$-subalgebra of $\Mat_n\left( \B\left( \mathcal{H} \right) \right)$, where $\mathcal{H}$ is the Hilbert space in the GNS construction of $A$. Letting $\pi_{n}\colon \Mat_n\left( A \right)\rightarrow \Mat_n\left( \B\left( \mathcal{H} \right) \right)$ be the isometric representation (hence injective, from Theorem \ref{thm:cstar_homomorphism_isometric_injective}), we are then able to define a $\mathrm{C}^{\ast}$-norm on $\Mat_n\left( A \right)$ by
\begin{align*}
  \norm{\left( a_{ij} \right)_{ij}} &= \norm{\pi_n\left( \left( a_{ij} \right)_{ij} \right)}_{\op}.
\end{align*}
Furthermore, since all $\mathrm{C}^{\ast}$-norms are equivalent on any $\mathrm{C}^{\ast}$-algebra (Proposition \ref{prop:cstar_norm_equivalent}), this uniquely defines the $\mathrm{C}^{\ast}$-norm on $\Mat_n\left( A \right)$. Thus, any $\mathrm{C}^{\ast}$-algebra carries with it a family of canonically defined norms (and order properties) via the amplification to matrix algebras.
\subsection{Amplifying Positive Maps}%
\begin{definition}
  Given two $\mathrm{C}^{\ast}$-algebras, $A$ and $B$, and a (linear) map $\phi\colon A\rightarrow B$, we define the \textit{$n$-fold amplification} of $\phi$, $\phi_n\colon \Mat_n\left( A \right)\rightarrow \Mat_n\left( B \right)$, by
  \begin{align*}
    \phi_n\left( \left( a_{ij} \right)_{ij} \right) &= \left( \phi\left( a_{ij} \right) \right)_{ij}.
  \end{align*}
  If $\phi$ is a positive map, then we say $\phi$ is \textit{$n$-positive} if $\phi_n$ is positive. We say $\phi$ is \textit{completely positive} if, for all $n$, $\phi_n$ is positive, in that $\phi_n$ maps positive elements of $\Mat_n\left( A \right)$ to the positive elements of $\Mat_n\left( B \right)$.\newline

  If $\phi$ is contractive, then we say $\phi$ is \textit{$n$-contractive} if $\phi_n$ is contractive, and if $\phi_n$ is contractive for all $n$, we say $\phi$ is completely contractive.\newline

  We say $\phi$ is \textit{completely bounded} if there exists some $C > 0$ such that for all $n$, the canonical map $\phi_n\colon \Mat_n\left( S \right)\rightarrow \Mat_n\left( B \right)$ is such that $\norm{\phi_n}_{\op}\leq C$. We write
  \begin{align*}
    \norm{\phi}_{\cb}\coloneq \sup_{n} \norm{\phi_n}_{\op}
  \end{align*}
\end{definition}
One of the useful facts about positive maps is that they are always bounded linear.
\begin{proposition}
  Let $S\subseteq A$ be an operator system, and let $B$ be a $\mathrm{C}^{\ast}$-algebra. If $\phi\colon S\rightarrow B$ is a positive map, then $\phi$ is bounded, with
  \begin{align*}
    \norm{\phi}_{\op} &\leq 2\norm{\phi\left( 1_A \right)}.
  \end{align*}
\end{proposition}
\begin{proof}
  If $p$ is positive, then $0\leq p \leq \norm{p}1_A$, so since $\phi$ is positive, Corollary \ref{cor:positive_map_order_preserving} gives $0\leq \phi\left( p \right)\leq \norm{p}\phi\left( 1_A \right)$. Therefore, we have
  \begin{align*}
    \norm{\phi\left( p \right)} &\leq \norm{p}\norm{\phi\left( 1_A \right)}.
  \end{align*}
  When $p_1,p_2$ are positive, we have $\norm{p_1 - p_2}\leq \max\set{\norm{p_1},\norm{p_2}}$. Now, if $h$ is self-adjoint, we use Remark \ref{rem:positive_elements_operator_system} and the fact that $\phi$ is order-preserving to obtain
  \begin{align*}
    \phi(h) &= \frac{1}{2}\phi\left( \norm{h}1_A + h \right) + \frac{1}{2}\phi\left( \norm{h}1_A - h \right),
  \end{align*}
  which is the difference of two positive elements in $B$, meaning
  \begin{align*}
    \norm{\phi(h)} &\leq \frac{1}{2}\max\set{\norm{\phi\left( \norm{h}1_A + h \right)},\norm{\phi\left( \norm{h}1_A - h \right)}}\\
                   &\leq \norm{h}\norm{\phi\left( 1_A \right)}.
  \end{align*}
  Now, if $a\in S$ is arbitrary, we use the Cartesian decomposition to write $a = h + ik$, where
  \begin{align*}
    h &= \frac{1}{2}\left( a + a^{\ast} \right)\\
    k &= \frac{i}{2}\left( a^{\ast}-a \right),
  \end{align*}
  and obtain
  \begin{align*}
    \norm{\phi(a)} &\leq \norm{\phi(h)} + \norm{\phi(k)}\\
                   &\leq 2\norm{a}\norm{\phi\left( 1_A \right)}.
  \end{align*}
  Thus,
  \begin{align*}
    \norm{\phi}_{\op} &\leq 2\norm{\phi\left( 1_A \right)}.
  \end{align*}
\end{proof}
\begin{remark}
  If $S\subseteq A$ is an operator system, $B$ a unital $\mathrm{C}^{\ast}$-algebra, and $\phi\colon S\rightarrow B$ is a positive map, then $\phi$ is self-adjoint --- i.e., $\phi\left( x^{\ast} \right) = \phi\left( x \right)^{\ast}$.\newline

  Furthermore, if $\phi\colon S\rightarrow B$ a unital contraction, then $\phi$ is positive --- the proof requires the fact that if $f\colon S\rightarrow \C$ is a linear functional with $f\left(1_A\right) = \norm{f} = 1$ (i.e., that $f$ is a state, see Definition \ref{def:state_linear_functional}), then $f(a)\in \overline{\conv}\left( \sigma\left( a \right) \right)$ for any normal element $a\in A$.
\end{remark}
We begin by establishing our first, ``one-dimensional'' relationship(s) between contractive maps and positive extensions.
\begin{proposition}\label{prop:extension_of_contractions}
  Let $A$ be a unital $\mathrm{C}^{\ast}$-algebra and $M\subseteq A$ an operator space. If $B$ is a unital $\mathrm{C}^{\ast}$-algebra, and $\phi\colon M\rightarrow B$ is a unital contraction, then the map $\Phi\colon M + M^{\ast}\rightarrow B$, given by $\Phi\left( a + b^{\ast} \right) = \phi\left( a \right) + \phi\left( b \right)^{\ast}$, is a well-defined, unique, positive extension of $\phi$ to $M + M^{\ast}$.
\end{proposition}
\begin{proof}
  We only need to show that this formula yields a well-defined, positive map, as its uniqueness of $\widetilde{\phi}$ follows from the fact that $\phi$ is self-adjoint.\newline

  To show that $\widetilde{\phi}$ is well-defined, it is enough to prove that if $a,a^{\ast}\in M$, then $\phi\left( a^{\ast} \right) = \phi\left( a \right)^{\ast}$. Set
  \begin{align*}
    S_1 &= \set{a | a\in M,a^{\ast}\in M}.
  \end{align*}
  Then, $S_1$ is an operator system, and $\phi$ is a unital, contractive map on $S_1$, hence positive. Since positive maps are self-adjoint, $\widetilde{\phi}$ is well-defined.\newline

  Now, we show that $\widetilde{\phi}$ is positive. It is sufficient to assume that $B = \B\left( \mathcal{H} \right)$ (else, represent faithfully represent $B$ using the GNS construction), and fix $x\in \mathcal{H}$ with $\norm{x} = 1$. We set $\widetilde{\rho}\left( a \right) = \iprod{\left( \widetilde{\phi}\left( a \right) \right)(x)}{x}$. We will show that $\widetilde{\rho}$ is positive.\newline

  Let $\rho\colon M\rightarrow \C$ be defined by $\rho(a) = \iprod{\left( \phi(a) \right)(x)}{x}$. Then, $\norm{\rho}_{\op} = 1$ (take $a = 1_A$), so by Theorem \ref{thm:hb_continuous_extension}, there is a norm-preserving extension $\rho_1\colon M + M^{\ast}\rightarrow \C$. However, since $\rho_1$ is positive, as it is a unital contraction on an operator space, we have
  \begin{align*}
    \rho_1\left( a + b^{\ast} \right) &= \rho\left( a \right) + \overline{\rho\left( b \right)}\\
                                      &= \widetilde{\rho}\left( a + b^{\ast} \right),
  \end{align*}
  so $\widetilde{\rho}$ is positive.
\end{proof}
Now, we may begin investigating amplifications of positive maps to the matrix algebras. We start a useful lemma characterizing positive matrices in $\Mat_n\left( A \right)$.
\begin{lemma}\label{lemma:positive_elements_from_matrix_algebras}
  Let $A$ be a unital $\mathrm{C}^{\ast}$-algebra. Then, the following hold.
  \begin{enumerate}[(i)]
    \item If $a\in A$, then $\norm{a}\leq 1$ if and only if
      \begin{align*}
        M &= \begin{pmatrix}1_A & a\\a^{\ast} & 1_A\end{pmatrix}.
      \end{align*}
      is positive in $\Mat_2\left( A \right)$.
    \item If $a,b\in A$, then 
      \begin{align*}
        M &= \begin{pmatrix}1_A & a \\ a^{\ast} & b\end{pmatrix}
      \end{align*}
      is positive in $\Mat_2\left( A \right)$ if and only if $a^{\ast}a \leq b$.
  \end{enumerate}
\end{lemma}
\begin{proof}\hfill
  \begin{enumerate}[(i)]
    \item Let $A$ be represented by $\pi\colon A\rightarrow \B\left( \mathcal{H} \right)$ for some Hilbert space $\mathcal{H}$, and set $T = \pi\left( a \right)$. Now, if $\norm{T}_{\op}\leq 1$, then for any $x,y\in \mathcal{H}$, we have
      \begin{align*}
        \iprod{ \begin{pmatrix}I & T \\ T^{\ast} & I\end{pmatrix} \begin{pmatrix}x\\y\end{pmatrix} }{ \begin{pmatrix}x\\y\end{pmatrix} } &= \iprod{x}{x} + \iprod{T\left( y \right)}{x} + \iprod{x}{T\left( y \right)} + \iprod{y}{y}\\
                                 &\geq \norm{x}^2 - 2\norm{T}_{\op}\norm{x}\norm{y} + \norm{y}^2\\
                                 &\geq 0.
      \end{align*}
      Conversely, if $\norm{T}_{\op} > 1$, then there exist unit vectors $x$ and $y$ such that $ \iprod{T\left( y \right)}{x} < -1 $, meaning that the inner product is negative.
    \item Assume $b\geq a^{\ast}a$, where we represent $a,b$ as elements $A,B\in \B\left( \mathcal{H} \right)$. Then, for all $y\in \mathcal{H}$, we have
      \begin{align*}
        \iprod{\left( B-A^{\ast}a \right)\left( y \right)}{y} &\geq 0,\\
        \iprod{B\left( y \right)}{y} &\geq \norm{A\left( y \right)}^2.
      \end{align*}
      Now, for $ \begin{pmatrix}x\\y\end{pmatrix}\in \mathcal{H}^{(2)} $, we have
      \begin{align*}
        \iprod{ \begin{pmatrix}I & A \\ A^{\ast} & B\end{pmatrix} \begin{pmatrix}x\\y\end{pmatrix}}{ \begin{pmatrix}x\\y\end{pmatrix} } &= \iprod{x}{x} + \iprod{A\left( y \right)}{x} + \iprod{A^{\ast}\left( x \right)}{y} + \iprod{B\left( y \right)}{y}\\
                                   &\geq \iprod{x}{x} + \iprod{A\left( y \right)}{x} + \iprod{A^{\ast}\left( x \right)}{y} + \norm{A\left( y \right)}^2\\
                                   &= \norm{x}^2 + \norm{A\left( y \right)}^2 + 2\re\left( \iprod{A\left( y \right)}{x} \right)\\
                                   &\geq \norm{x}^2 + \norm{A\left( y \right)}^2 - 2\norm{A\left( y \right)}\norm{x}\\
                                   &\geq 0,
      \end{align*}
      meaning the matrix is positive.\newline

      Now, suppose $B\ngeq A^{\ast}A$. Then, there is some $y\in \mathcal{H}$ such that $ \iprod{\left( B-A^{\ast}A \right)\left( y \right)}{y}  < 0$, giving $ \iprod{B\left( y \right)}{y} < \norm{A(y)}^2$. We may scale $y$ such that $\norm{A\left( y \right)}^2 = 1$, and setting $x = -A\left( y \right)$, we get
      \begin{align*}
        \iprod{ \begin{pmatrix}I & A \\ A^{\ast} & B\end{pmatrix} \begin{pmatrix}x\\y\end{pmatrix}}{ \begin{pmatrix}x\\y\end{pmatrix} } &= \iprod{x}{x} + \iprod{A\left( y \right)}{x} + \iprod{A^{\ast}\left( x \right)}{y} + \iprod{B\left( y \right)}{y}\\
                                 &= \iprod{x}{x} + \iprod{A\left( y \right)}{x} + \iprod{x}{A^{\ast}y} + \iprod{B\left( y \right)}{y}\\
                                 &= \iprod{-A\left( y \right)}{-A\left( y \right)} + \iprod{A\left( y \right)}{-A\left( y \right)} + \iprod{-A\left( y \right)}{A\left( y \right)} + \iprod{B\left( y \right)}{y}\\
                                 &= \norm{A\left( y \right)}^2 - 2\norm{A\left( y \right)}^2 + \iprod{B\left( y \right)}{y}\\
                                 &= -1 + \iprod{B\left( y \right)}{y}\\
                                 &< -1 + \norm{A\left( y \right)}^2\\
                                 &= 0.
      \end{align*}
      Thus, the matrix is not positive.
  \end{enumerate}
\end{proof}
\subsection{Completely Positive and Completely Contractive Maps}%
Now, we may discuss the important interplay between positive maps and contractive maps through the amplification to matrix algebras.
\begin{proposition}\label{prop:two_positive_contractive}
  Let $S$ be an operator system, $B$ a unital $\mathrm{C}^{\ast}$-algebra, and $\phi\colon S\rightarrow B$ unital $2$-positive map. Then, $\phi$ is contractive.
\end{proposition}
\begin{proof}
  Let $a\in S$ be such that $\norm{a}\leq 1$. Then,
  \begin{align*}
    \phi_2 \begin{pmatrix}1_A & a \\ a^{\ast} & 1_A\end{pmatrix} &= \begin{pmatrix} 1 & \phi\left( a \right) \\ \phi\left( a \right)^{\ast} & 1\end{pmatrix},
  \end{align*}
  is a positive matrix by the $2$-positivity of $\phi$, so by Lemma \ref{lemma:positive_elements_from_matrix_algebras} (i), we have that $\norm{\phi(a)}\leq 1$, meaning $\phi$ is contractive.
\end{proof}
\begin{proposition}[Cauchy--Schwarz for $2$-Positive Maps]
  Let $A$ and $B$ be unital $\mathrm{C}^{\ast}$-algebras, and let $\phi\colon A\rightarrow B$ be a unital $2$-positive map. Then,
  \begin{align*}
    \phi\left( a \right)^{\ast}\phi\left( a \right)\leq \phi\left( a^{\ast}a \right).
  \end{align*}
\end{proposition}
\begin{proof}
  Note that
  \begin{align*}
    \begin{pmatrix}1 & a \\ 0 & 0\end{pmatrix}^{\ast} \begin{pmatrix}1 & a \\ 0 & 0\end{pmatrix} &= \begin{pmatrix}1 & a \\ a^{\ast} & a^{\ast}a\end{pmatrix}\\
                     &\geq 0,
  \end{align*}
  so by the $2$-positivity of $\phi$, we have
  \begin{align*}
    \begin{pmatrix}1 & \phi\left( a \right) \\ \phi\left( a \right)^{\ast} & \phi\left( a^{\ast}a \right)\end{pmatrix} &\geq 0,
  \end{align*}
  so $\phi\left( a^{\ast}a \right) \geq \phi\left( a \right)^{\ast}\phi\left( a \right)$ by Lemma \ref{lemma:positive_elements_from_matrix_algebras} (ii).
\end{proof}
\begin{proposition}
  Let $A$ and $B$ be unital $\mathrm{C}^{\ast}$-algebras, and let $M\subseteq A$ be a unital operator space. Let $S = M + M^{\ast}$, where $M^{\ast}$ denotes pointwise involution of elements of $M$. Then, if $\phi\colon M\rightarrow B$ is unital and $2$-contractive, then $\widetilde{\phi}\colon S\rightarrow B$, given by $\widetilde{\phi}\left( a + b^{\ast} \right) = \phi\left( a \right) + \phi\left( b \right)^{\ast}$, is $2$-positive and contractive.
\end{proposition}
\begin{proof}
  Since $\phi$ is contractive, the map $\widetilde{\phi}$ is positive and well-defined by Proposition \ref{prop:extension_of_contractions}. Note that
  \begin{align*}
    \Mat_2\left( S \right) &= \Mat_2\left( M \right) + \Mat_2\left( M \right)^{\ast}.
  \end{align*}
  The map $\widetilde{\phi}\colon M + M^{\ast}\rightarrow B$ extends via a similar scheme to $\widetilde{\phi}_2\colon \Mat_2\left( M \right) + \Mat_2\left( M \right)^{\ast}$. Since $\phi_2$ is contractive, $\widetilde{\phi}_2$ is positive, so $\widetilde{\phi}$ is contractive by Proposition \ref{prop:two_positive_contractive}.
\end{proof}
\begin{proposition}
  Let $A$ and $B$ be unital $\mathrm{C}^{\ast}$-algebras, let $M\subseteq A$ be a unital subspace, and let $S = M + M^{\ast}$. If $\phi\colon M\rightarrow B$ is unital and completely contractive, then $\widetilde{\phi}\colon S\rightarrow B$ is completely positive and completely contractive.
\end{proposition}
\begin{proof}
  Since $\phi_n$ is unital and contractive, $\widetilde{\phi}_n$ is positive by Proposition \ref{prop:extension_of_contractions}. Now, since $\left( \widetilde{\phi}_n \right)_{2}$ is positive, $\widetilde{\phi}_n$ is contractive by Proposition \ref{prop:two_positive_contractive}. 
\end{proof}
\begin{remark}
  Since $\Mat_{2}\left( \Mat_n\left( A \right) \right)\cong \Mat_{2n}\left( A \right)$ are $\ast $-isomorphic, since $\mathrm{C}^{\ast}$-algebras have a unique norm, the norm on $\Mat_{2}\left( \Mat_n\left( A \right) \right)$ and the norm on $\Mat_{2n}\left( A \right)$ are equal.
\end{remark}
\begin{example}
  If $A$ and $B$ are $\mathrm{C}^{\ast}$-algebras, and $\pi\colon A\rightarrow B$ is a $\ast$-homomorphism, then $\pi$ is both completely positive and completely contractive. This follows from the fact that each $\pi_n\colon \Mat_n\left( A \right)\rightarrow \Mat_n\left( B \right)$ is a $\ast$-homomorphism that is defined canonically, and $\ast$-homomorphisms are positive and contractive maps.
\end{example}
\begin{remark}
  As it turns out, if $\phi\colon A\rightarrow \B\left( \mathcal{H} \right)$ is a $\ast$-homomorphism, and $v_i\colon \mathcal{H}\rightarrow \mathcal{K}$ are bounded linear operators, then all completely bounded maps are of the form $\phi\left( a \right) = v_2^{\ast}\pi(a)v_1$ for some $v_1,v_2$.\newline

  Furthermore, Stinespring's Dilation Theorem (Theorem \ref{thm:stinespring_dilation}) says that all completely positive maps are of the form $\phi\left( a \right) = V^{\ast}\pi(a)V$ for some $\ast$-homomorphism $\pi\colon A\rightarrow \B\left( \mathcal{H} \right)$ and bounded linear map $V\colon \mathcal{H}\rightarrow \mathcal{K}$. Additionally, all ``minimal'' such representations are unitarily equivalent.
\end{remark}
One of the most useful facts about completely positive maps is that they are always completely bounded.
\begin{proposition}\label{prop:norm_of_completely_positive_map}
  Let $S\subseteq A$ be an operator system, $B$ a $\mathrm{C}^{\ast}$-algebra, and $\phi\colon S\rightarrow B$ a completely positive map. Then, $\phi$ is completely bounded, and $\norm{\phi}_{\op} = \norm{\phi(1)} = \norm{\phi}_{\cb}$.
\end{proposition}
\begin{proof}
  We can see by definition that $\norm{\phi(1)}\leq \norm{\phi}_{\op}\leq \norm{\phi}_{\cb}$. It is thus sufficient to show that $\norm{\phi}_{\cb}\leq \norm{\phi}_{\op}$.\newline

  Let $T = \left( t_{ij} \right)_{ij}\in \Mat_n\left( S \right)$ be such that $\norm{T}\leq 1$. Let $I_n$ be the identity matrix in $\Mat_n\left( A \right)$. Since the matrix
  \begin{align*}
    M &= \begin{pmatrix}I_n & A \\ A^{\ast} & I_n\end{pmatrix}
  \end{align*}
  is positive by Lemma \ref{lemma:positive_elements_from_matrix_algebras} (i), we have that the application of $\phi_{2n}$ on the $2\times 2$ repeated copy of $M$, and 
  \begin{align*}
    \phi_{2n}\left( \begin{pmatrix}I_n & A \\ A^{\ast} & I_n\end{pmatrix}_{1,2} \right) &= \begin{pmatrix} \begin{pmatrix}\phi_n\left( I_n \right) & \phi_n\left( A \right) \\ \phi_n\left( A \right)^{\ast} & \phi_n\left( I_n \right)\end{pmatrix}\end{pmatrix}
  \end{align*}
  is positive. Therefore, we have $\norm{\phi_n\left( A \right)}\leq \norm{\phi_n\left( I_n \right)} = \norm{\phi(1)}$.
\end{proof}
A useful fact is that bounded linear functionals are completely bounded, and positive linear functionals are completely positive.
\begin{proposition}
  Let $S\subseteq A$ be an operator system, and let $f\colon S\rightarrow \C$ be a bounded linear functional. Then, $\norm{f}_{\cb} = \norm{f}_{\op}$, and if $f$ is positive, then $f$ is completely positive.
\end{proposition}
\begin{proof}
  Let $\left( a_{ij} \right)_{ij}\in \Mat_n\left( S \right)$, and let $x,y$ be unit vectors in $\C^n$. Then,
  \begin{align*}
    \left\vert \iprod{f_n\left( \left( a_{ij} \right)_{ij} \right)\left( x \right)}{y} \right\vert &= \left\vert \sum_{i,j=1}^{n}f\left( a_{ij} \right)x_j\overline{y_i} \right\vert\\
                                                                                                   &= \left\vert f\left( \sum_{i,j=1}^{n} a_{ij}x_j \overline{y_i} \right) \right\vert\\
                                                                                                   &\leq \norm{f}_{\op}\norm{\sum_{i,j=1}^{n}a_{ij}x_j\overline{y_i}}.
  \end{align*}
  Now, we must show that the latter element has a norm less than $\norm{\left( a_{ij} \right)_{ij}}$. Note that the sum is equal to the $(1,1)$ entry of
  \begin{align*}
    M &= \begin{pmatrix}\overline{y_1}1_A & \cdots & \overline{y_n}1_A \\ 0 & \cdots & 0 \\ \vdots & \ddots & \vdots \\ 0 & \cdots & 0\end{pmatrix} \begin{pmatrix}a_{11} & \cdots & a_{1n} \\ \vdots & \ddots & \vdots \\ a_{n1} & \cdots & a_{nn}\end{pmatrix} \begin{pmatrix}x_1 1_A & 0 & \cdots & 0 \\ \vdots & \vdots & \ddots & \vdots \\ x_n1_A  & 0 & \cdots & 0\end{pmatrix},\label{eq:matrix_positive_linear_functional}\tag{\textasteriskcentered}
  \end{align*}
  whose outer factors each have norm $1$. This shows that $f$ is completely bounded.\newline

  To see that $f$ is completely positive, we must show that
  \begin{align*}
    \iprod{f_n\left( \left( a_{ij} \right)_{ij} \right)\left( x \right)}{x} &= f\left( \sum_{i,j=1}^{n}a_{ij}x_j\overline{x_i} \right)
  \end{align*}
  is positive whenever $\left( a_{ij} \right)_{ij}$ is positive. However, using $x = y$ in \eqref{eq:matrix_positive_linear_functional}, we find that $f$ is evaluated at the $(1,1)$ entry of a positive matrix, hence is positive.
\end{proof}
Now, we discuss a little bit about the case where $\phi\colon S\rightarrow C\left( X \right)$ is a map between an operator system and a commutative $\mathrm{C}^{\ast}$-algebra (Theorem \ref{thm:gelfand_naimark}), where $X$ is a compact Hausdorff space. Every element $F = \left( f_{ij} \right)_{ij}\in \Mat_n\left( C\left( X \right) \right)$ is a continuous matrix-valued function with pointwise multiplication and $\ast$-operations. In order to convert the space $\Mat_n\left( C\left( X \right) \right)$ into a $\mathrm{C}^{\ast}$-algebra, we define the norm $\norm{F} = \sup_{x\in X}\norm{F(x)}$, and since $\mathrm{C}^{\ast}$-norms are unique, this is the only way to create a $\mathrm{C}^{\ast}$-algebra.\newline

This allows us to establish that if the range of such a $\phi$ is commutative, then $\phi$ is completely bounded, and if $\phi$ is positive, then $\phi$ is completely positive.
\begin{theorem}\label{thm:commutative_range_completely_positive}
  Let $S$ be an operator system, $\phi\colon S\rightarrow C(X)$ a bounded linear map. Then, $\norm{\phi}_{\cb} = \norm{\phi}_{\op}$, and if $\phi$ is positive, then $\phi$ is completely positive.
\end{theorem}
\begin{proof}
  For any $x\in X$, define $\phi^{x}\colon S\rightarrow \C$ by $\phi^x(a) = \phi(a)(x)$. Then, we have
  \begin{align*}
    \norm{\phi_n}_{\op} &= \sup_{x\in X}\norm{\phi_n^{x}}\\
                        &= \sup_{x\in X}\norm{\phi^{x}}\\
                        &= \norm{\phi}_{\op}.
  \end{align*}
  Similarly, $\phi_n\left( \left( a_{ij} \right)_{ij} \right)$ is positive if and only if $\phi_n^{x}\left( \left( a_{ij} \right)_{ij} \right)$ is positive for all $x\in X$.
\end{proof}
  There is a converse --- i.e., if the domain is a commutative $\mathrm{C}^{\ast}$-algebra, then any positive map is completely positive. However, this direction is a little bit more involved. We will state it without proof.
\begin{theorem}\label{thm:commutative_domain_completely_positive}
  Let $\phi\colon C\left( X \right)\rightarrow B$ be a positive map. Then, $\phi$ is completely positive.
\end{theorem}
\subsection{Extending and Approximating Completely Positive Maps}%
Now that we've established some important properties that underly completely positive, completely contractive, and completely bounded maps, we discuss two major theorems regarding completely positive maps, though we do not state their proofs.
\begin{theorem}[Stinespring's Dilation Theorem]\label{thm:stinespring_dilation}
  Let $A$ be a unital $\mathrm{C}^{\ast}$-algebra, and let $\phi\colon A\rightarrow \B\left( \mathcal{H} \right)$ be a completely positive map. Then, there exists a Hilbert space, $\mathcal{K}$, a unital $\ast$-homomorphism $\pi\colon A\rightarrow \B\left( \mathcal{K} \right)$, and a bounded operator $V\colon \mathcal{H}\rightarrow \mathcal{K}$, with $\norm{\phi(1)} = \norm{V}_{\op}^2$, such that
  \begin{align*}
    \phi(a) &= V^{\ast}\pi(a)V.
  \end{align*}
\end{theorem}
The proof can be found in \cite[Chapter 4]{completely_bounded_maps_and_operator_algebras}, and follows a similar line of argument to the proof of the GNS construction, where a sesquilinear form is defined using the completely positive map $\phi$.\newline

Arveson's Extension Theorem is one of the other major foundational results in the theory of completely positive maps.
\begin{theorem}[Arveson's Extension Theorem]\label{thm:arveson}
  Let $A$ be a $\mathrm{C}^{\ast}$-algebra, $S\subseteq A$ an operator system, and $\phi\colon S\rightarrow \B\left( \mathcal{H} \right)$ a completely positive map. Then, there exists a completely positive map $\psi\colon A\rightarrow \B\left( \mathcal{H} \right)$ that extends $\phi$. In other words, $\B\left( \mathcal{H} \right)$ is injective in the category of operator systems whose morphisms are completely positive maps:
  \begin{center}
    % https://tikzcd.yichuanshen.de/#N4Igdg9gJgpgziAXAbVABwnAlgFyxMJZABgBpiBdUkANwEMAbAVxiRGJAF9T1Nd9CKAIzkqtRizYBlLjxAZseAkQBMo6vWatEIAIKzeigURFCxmyToA6VgEI2GMAGY4AFDYC2dHAAsAxozAABKcNgBOWADmPjgAlFxiMFCR8ESgTmEQHkhkIDgQSELc6ZnZiCJ5BYgAzNQMdABGMAwACnxKgiAR0TggGhLaIDZoPlgGIBlZhdT5SCrFE6VzM1W1IPVNre3GOo4ufeJabMPYCZxAA
    \begin{tikzcd}
    0 \arrow[r] & S \arrow[d, "\phi"'] \arrow[r] & A \arrow[ld, "\psi"] \\
                & \B\left(\mathcal{H}\right)     &                     
    \end{tikzcd}
  \end{center}
\end{theorem}
Arveson's Extension Theorem is proven in \cite[Chapter 7]{completely_bounded_maps_and_operator_algebras} by proving for the special case of $\mathcal{H} = \C^n$, and then using a compactness argument.\newline

Next, we establish basic properties of maps between $\mathrm{C}^{\ast}$-algebras that can be approximated by ``factoring'' through finite-rank completely positive maps. These are known as nuclear maps, and they play an integral role in establishing amenability. The exposition from here on out will follow (loosely) the results from \cite{brown_and_ozawa}.
\begin{definition}
  Let $\theta\colon A\rightarrow B$ be a map between $\mathrm{C}^{\ast}$-algebras. We say $\theta$ is nuclear if there exist completely positive contractions $\varphi_n\colon A\rightarrow \Mat_{k(n)}\left( \C \right)$ and $\psi_n\colon \Mat_{k(n)}\left( \C \right)\rightarrow B$, such that
  \begin{align*}
    \norm{\theta(a) - \psi_n\circ \varphi_n\left( a \right)}\xrightarrow{n\rightarrow\infty} 0
  \end{align*}
  for all $a\in A$.
\end{definition}
\begin{remark}
Every nuclear map is completely positive. Furthermore, we assume that all of $A,B,\theta,\varphi_n,\psi_n$ are unital maps, as we will focus on establishing nuclearity with the group $\mathrm{C}^{\ast}$-algebra(s), which are unital $\mathrm{C}^{\ast}$-algebras.
\end{remark}
One of the most important facts about nuclearity is the fact that it is preserved under composition.
\begin{proposition}\label{prop:nuclearity_composition}
  Let $\theta\colon A\rightarrow B$ and $\sigma\colon B\rightarrow C$ be completely positive maps between $\mathrm{C}^{\ast}$-algebras. If one of $\sigma$ or $\theta$ is nuclear, then the composition $\sigma\circ\theta$ is nuclear.
\end{proposition}
\begin{proof}
  Assume that $\theta$ is nuclear. Then, there exist $\varphi_n\colon A\rightarrow \Mat_{k(n)}\left( \C \right)$ and $\psi_n\colon \Mat_{k(n)}\left( \C \right)\rightarrow B$ such that $\norm{\theta(a) - \psi_n\circ\varphi_n(a)} \rightarrow 0$. This means that for any $\ve > 0$, there exists a finite set $F\subseteq A$ such that for all $a\in F$,
  \begin{align*}
    \norm{\theta(a) - \psi_n\circ\varphi_n(a)} < \ve.
  \end{align*}
  Now, note that
  \begin{align*}
    \sigma\circ\theta(a) - \sigma\circ\psi_n\circ\varphi_n(a) &= \sigma\left( \theta(a) - \psi_n\circ \varphi_n(a) \right).
  \end{align*}
  We define $\psi_n' \coloneq \sigma\circ \psi_n\colon \Mat_{k(n)}\left( \C \right)\rightarrow C$, and see that, for all $a\in F$,
  \begin{align*}
    \norm{\sigma\circ\theta\left( a \right) - \psi_n'\circ\varphi_n\left( a \right)} &= \norm{\sigma\left( \theta(a) - \psi_n\circ\varphi_n\left( a \right) \right)}\\
                                                                                     &= \norm{\sigma\left( \left( 1_B \right)\left( \theta\left( a \right) - \psi_n\circ\varphi_n\left( a \right) \right) \right)}\\
                                                                                     &\leq \norm{\sigma\left(1_B\right)}\norm{\theta(a) - \psi_n\circ\varphi_n\left( a \right)}\\
                                                                                     &< \norm{\sigma\left(1_B\right)} \ve.
  \end{align*}
  Since $\norm{\sigma\left( 1_B \right)} = \norm{\sigma}_{\op} < \infty$ (Proposition \ref{prop:norm_of_completely_positive_map}), we see that $\sigma\circ\theta$ is nuclear for all $a$.\newline

  Now, if $\sigma$ is nuclear, then there exist $\varphi_n\colon B\rightarrow \Mat_{k(n)}\left( \C \right)$ and $\psi_n\colon \Mat_{k(n)}\left( \C \right)\rightarrow C$ such that for all $b\in B$, $\norm{\sigma(b) - \psi_n\circ\varphi_n(b)} \rightarrow 0$. In particular, this applies to $\theta(A)\subseteq B$, so by defining $\varphi_n' \coloneq \varphi_n\circ\theta\colon A \rightarrow \Mat_{k(n)}\left( \C \right)$, we have that
  \begin{align*}
    \norm{\sigma\circ\theta\left( a \right) - \psi_n\circ\varphi_n'\left( a \right)}\rightarrow 0,
  \end{align*}
  so $\sigma\circ\theta$ is nuclear.
\end{proof}
\begin{corollary}
  If $\id\colon A\rightarrow A$ is nuclear, then for any completely positive map $\theta\colon A\rightarrow B$, $\theta$ is nuclear.
\end{corollary}
\begin{proof}
  Write $\theta = \theta\circ \id$, and apply Proposition \ref{prop:nuclearity_composition}.
\end{proof}
\begin{definition}\label{def:nuclear_cstar_algebra}
  We say a $\mathrm{C}^{\ast}$-algebra $A$ is \textit{nuclear} if the identity map $\id\colon A\rightarrow A$ is nuclear --- i.e., that all completely positive maps with domain $A$ are nuclear.
\end{definition}
One broad example of nuclear $\mathrm{C}^{\ast}$-algebras is commutative ones.
\begin{proposition}
  If $A$ is a commutative $\mathrm{C}^{\ast}$-algebra, then $A$ is nuclear.
\end{proposition}
\begin{proof}
  If $A$ is commutative then $A = C\left( X \right)$ for some compact Hausdorff space $X$ (Theorem \ref{thm:gelfand_naimark}). Now, for any finite $F\subseteq A$, there is an open cover $\set{U_1,\dots,U_n}$ of $X$ such that for each $f\in F$ and $1 \leq i \leq n$, we have $\left\vert f(x) - f(y) \right\vert < \ve$ for any $x,y\in U_i$, which follows from compactness.\newline

  Let $y_i\in U_i$ be arbitrary, and let $\set{\sigma_1,\dots,\sigma_n}$ be a partition of unity subordinate to the cover $\set{U_1,\dots,U_n}$ (\cite[Proposition 4.41]{folland_real_analysis}), where $\sum_{i=1}^{n}\sigma_i = 1$ and $\supp\left( \sigma_i \right) \subseteq U_i$ for each $i$.\newline

  Define $\varphi\colon A\rightarrow \C^n$ by $\varphi\left( f \right) = \left( f\left(y_1\right),\dots,f\left( y_n \right) \right)$. Since $\varphi$ is a unital $\ast$-homomorphism, it is a completely positive contraction.\newline

  Now, define $\psi\colon \C^n\rightarrow A$ by
  \begin{align*}
    \psi\left( d_1,\dots,d_n \right) &= \sum_{i=1}^{n}d_i\sigma_i.
  \end{align*}
  Now, $\psi$ is a positive map, and since $\C^n$ is a commutative $\mathrm{C}^{\ast}$-algebra, $\psi$ is nuclear. Thus, we have
  \begin{align*}
    \norm{f - \psi\circ\phi\left( f \right)} &= \norm{\left( \sum_{i=1}^{n}\sigma_i \right)f - \sum_{i=1}^{n}f\left( y_i \right)\sigma_i}\\
                                             &= \norm{\sum_{i=1}^{n}\left( f-f\left( y_i \right) \right)\sigma_i}\\
                                             &\leq \ve,
  \end{align*}
  which holds for all $f\in F$ and any finite $F\subseteq A$, meaning that $A$ is nuclear.
\end{proof}
\section{Characterizing Amenability using \texorpdfstring{$\mathrm{C}^{\ast}$-Algebras}{C*-Algebras}}%
The ultimate goal of this section is to prove the following theorem. In the process, we will conduct a tour of various concepts in the theory of $\mathrm{C}^{\ast}$-algebras and von Neumann algebras.
\begin{theorem}[{\cite[Theorem 2.6.8]{brown_and_ozawa}}]\label{thm:amenability_cstar_algebras_formulations}
  Let $\Gamma$ be a discrete group. The following are equivalent:
  \begin{enumerate}[(i)]
    \item $\Gamma$ is amenable;
    \item $\mathrm{C}^{\ast}_{\lambda}\left( \Gamma \right)$ is nuclear;
    \item $\mathrm{C}^{\ast}_{\lambda}\left( \Gamma \right) = \mathrm{C}^{\ast}\left( \Gamma \right)$;
    \item $\mathrm{C}^{\ast}_{\lambda}\left( \Gamma \right)$ admits a character (Definition \ref{def:algebra_homomorphisms_and_characters}).
  \end{enumerate}
\end{theorem}
We will prove Theorem \ref{thm:amenability_cstar_algebras_formulations} using a variety of subtheorems.
\subsection{Amenability and Nuclearity}%
\begin{theorem}
  If $\Gamma$ is amenable, then $\mathrm{C}^{\ast}_{\lambda}\left( \Gamma \right)$ is nuclear.
\end{theorem}
\begin{proof}
  Let $\left( F_n \right)_n\subseteq \Gamma$ be a Følner sequence, and let $P_n\in \B\left( \ell_2\left( \Gamma \right) \right)$ be the orthogonal projection onto the subspace of $\ell_2\left( \Gamma \right)$ spanned by $\set{\delta_g | g\in F_n}$. Here, $\set{\delta_t}_{t\in\Gamma}$ refers to the canonical orthonormal basis for $\C\left[ \Gamma \right]$.\newline

  Note that each $P_n$ is a finite-rank projection, so each $P_n$ can be written as
  \begin{align*}
    P_n\coloneq \sum_{g\in F_n}\theta_{\delta_{g},\delta_{g}},
  \end{align*}
  where $\theta_{\delta_g,\delta_g}$ denotes the rank-one bounded operator (Definition \ref{def:rank_one_bounded_operator}). Now, since each $F_n$ is finite-dimensional, we can rewrite the orthonormal basis as $\set{e_{1},\dots,e_{r}}$ for some $r$. Note that we have
  \begin{align*}
    T \theta_{e_j,e_j}\left( e_k \right) &= \iprod{e_k}{e_j}T\left( e_j \right)\\
                                         &= \theta_{T\left( e_j \right),e_j}\left( e_k \right),\label{eq:operator_applied_to_rank_one}\tag{\textdagger}
                                         \intertext{so}
    \theta_{e_i,e_i} T \theta_{e_j,e_j}\left( e_k \right) &= \iprod{T\left( e_j \right)}{e_i}\theta_{e_i,e_j}.
  \end{align*}
  Thus, by taking the summation over the representation of $P_n$, we find that there is an isomorphism
  \begin{align*}
    P_n\B\left( \ell_2\left( \Gamma \right) \right)P_n &\cong \Mat_{\left\vert F_n \right\vert}\left( \C \right).
  \end{align*}
  Let $\set{e_{p,q}}_{p,q\in F_n}$ be the matrix units of $P_n\B\left( \ell_2\left( \Gamma \right) \right)P_n$, which are equal to the rank-one bounded operators $\theta_{\delta_p,\delta_q}$. Note that if $\lambda_s\colon \ell_2\left( \Gamma \right)\rightarrow \ell_2\left( \Gamma \right)$ denotes the left-regular representation map $\delta_t \mapsto \delta_{st}$, then
  \begin{align*}
    e_{p,p}\lambda_se_{q,q} &= \theta_{\delta_p,\delta_p}\lambda_s\theta_{\delta_q,\delta_q}\\
                            &= \theta_{\delta_p,\delta_p}\theta_{\delta_{sq},\delta_{q}}\\
                            &= \begin{cases}
                              0 & sq\neq p\\
                              \theta_{\delta_p,\delta_q} & sq = p
                            \end{cases},
  \end{align*}
  where the latter quantity is found by evaluating \eqref{eq:operator_applied_to_rank_one} with $T = \theta_{\delta_p,\delta_p}$. Now, since $P_n = \sum_{p\in F_n}e_{p,p}$, we have
  \begin{align*}
    P_n \lambda_{s} P_n &= \sum_{p,q\in F_k}e_{p,p}\lambda_s e_{q,q}\\
                        &= \sum_{p\in F_k\cap sF_k} e_{p,s^{-1}p}.
  \end{align*}
  We start by defining $\varphi_{n}\colon \mathrm{C}^{\ast}_{\lambda}\left( \Gamma \right)\rightarrow \Mat_{F_n}\left( \C \right)$ by $x\mapsto P_n x P_n$, which is a completely positive map, as projections preserve positivity.\newline

  Next, we define $\psi_n\colon \Mat_{F_n}\left( \C \right)\rightarrow \mathrm{C}^{\ast}_{\lambda}\left( \Gamma \right)$ by taking $e_{p,q} \mapsto \frac{1}{\left\vert F_n \right\vert}\lambda_{p}\lambda_{q}^{\ast}$.\newline

  Note that if $A$ is a $\mathrm{C}^{\ast}$-algebra, then any positive element of $\Mat_n\left( A \right)$ is a sum of elements of the form $\left( a_i^{\ast}a_j \right)_{ij}$ (\cite[Lemma 3.13]{completely_bounded_maps_and_operator_algebras}), so each of the $\psi_n$ maps is a completely positive contraction.\newline

  We start by evaluating on $\lambda_s$, where $s\in\Gamma$. This gives
  \begin{align*}
    \psi_n\circ\varphi_n\left( \lambda_s \right) &= \psi_n\left( \sum_{p\in F_n\cap sF_n}e_{p,s^{-1}p} \right)\\
                                                 &= \sum_{p\in F_n\cap sF_k}\frac{1}{\left\vert F_n \right\vert}\lambda_s\\
                                                 &= \frac{\left\vert F_n \cap sF_n \right\vert}{\left\vert F_n \right\vert}\lambda_s.
  \end{align*}
  Thus, by the definition of the Følner condition (Definition \ref{def:folner_condition}), we have that
  \begin{align*}
    \norm{\lambda_s - \psi_n\circ\varphi_n\left( \lambda_s \right)} &\rightarrow 0,
  \end{align*}
  Since $\Span\left( \set{\lambda_s}_{s\in\Gamma} \right)$ is norm-dense in $\mathrm{C}^{\ast}_{\lambda}\left( \Gamma \right)$, linearity and continuity show that $\mathrm{C}^{\ast}_{\lambda}\left( \Gamma \right)$ is nuclear.
\end{proof}
For the reverse direction, we need a little bit of background on the theory of von Neumann algebras, which are a special type of $\ast$-subalgebra of $\B\left( \mathcal{H} \right)$.
\begin{definition}
  A \textit{von Neumann algebra} is a $\ast$-subalgebra $A\subseteq \B\left( \mathcal{H} \right)$ that is closed in the weak operator topology (Definition \ref{def:operator_topologies}).
\end{definition}
\begin{definition}\label{def:state_linear_functional}
  Let $A$ be a $\mathrm{C}^{\ast}$-algebra, and let $\varphi\colon A\rightarrow \C$ be a linear functional. 
  \begin{itemize}
    \item We say $\varphi$ is a \textit{state} if $\varphi\left( 1_A \right) = \norm{\varphi}_{\op} = 1$.
    \item We say $\varphi$ is \textit{tracial} if $\varphi\left( ab \right) = \varphi\left( ba \right)$ for all $a,b\in A$.
    \item If $A$ is a von Neumann algebra, then we say $\varphi$ is \textit{normal} if, for all norm-bounded, monotonically increasing nets $\left( x_{\alpha} \right)_{\alpha}\subseteq A_{\sa}$, we have $\varphi\left( \sup_{\alpha}x_{\alpha} \right) = \sup_{\alpha}\varphi\left( x_{\alpha} \right)$.
  \end{itemize}
\end{definition}
\begin{example}
  If $A\subseteq \B\left( \mathcal{H} \right)$ is a $\mathrm{C}^{\ast}$-algebra, and $x\in S_{\mathcal{H}}$ is a unit vector, then the map $a\mapsto \iprod{ax}{x}$ defines a state on $A$, known as the \textit{vector state} on $A$.
\end{example}
The following is a useful structural result on von Neumann algebras, proven in \cite[Section III.2.4]{blackadar_operator_algebras}.
\begin{theorem}
  Let $M$ be a von Neumann algebra. Then, there is a unique Banach space $X$ such that $X^{\ast}\cong M$. Specifically, $X$ can be identified with the space of normal linear functionals on $M$.\newline

  Furthermore, if $A$ is a $\mathrm{C}^{\ast}$-algebra such that there exists a Banach space $X$ with $X^{\ast}\cong A$, then $A$ is a von Neumann algebra.
\end{theorem}
Since every von Neumann algebra is a dual space, we may consider the weak* topology induced by the predual of a von Neumann algebra.
\begin{definition}
  Let $M$ be a von Neumann algebra, and let $M_{\ast}$ denote the predual of $M$. The \textit{ultraweak} topology on $M$ is the weak* topology on $M$ induced by $M_{\ast}$.\newline

  If $X$ is any Banach space, and $\B\left( X,M \right)$ is the space of bounded linear maps $T\colon X\rightarrow M$, the predual $\B\left( X,M \right)_{\ast}$ is canonically identified (Definition \ref{def:double_dual_and_canonical_embedding}) with the closed linear span of $x\otimes \xi\in \B\left( X,M \right)^{\ast}$, where $x\in X$, $\xi\in M_{\ast}$, and $x\otimes \xi\left( T \right) \coloneq \xi\left( T(x) \right)$.\newline

  On bounded sets, the weak* topology on $\B\left( X,M \right)$ induced by $\B\left( X,M \right)_{\ast}$ is known as the \textit{point-ultraweak} topology, where net convergence is defined by $\left( T_{\alpha} \right)_{\alpha}\rightarrow T$ if and only if $\xi\left( T_{\alpha}\left( x \right) \right) \rightarrow \xi\left( T\left( x \right) \right)$ for all $\xi\in M_{\ast}$ and $x\in X$.
\end{definition}
Since every space $\B\left( X,M \right)$ is a dual space, the Banach--Alaoglu Theorem (Theorem \ref{thm:banach_alaoglu}) applies, meaning that every bounded net of linear maps $T_{\alpha}\colon X\rightarrow M$ admits a convergent subnet.\newline

Now, similar to the case of a $\mathrm{C}^{\ast}$-algebra generated by some set (Definition \ref{def:generated_operator_algebras}), there is such a thing as a von Neumann algebra generated by some set, which is defined similarly --- i.e., the smallest von Neumann algebra containing a given set. In particular, we are interested in the enveloping von Neumann algebra of the group $\mathrm{C}^{\ast}$-algebras.
\begin{definition}
  Let $A\subseteq \B\left( \mathcal{H} \right)$ be a von Neumann algebra. The \textit{commutant} of $A$, denoted $A'$, is the set of all operators $T\in \B\left( \mathcal{H} \right)$ such that for any $S\in A$, $ST = TS$.\newline

  We define $A'' \coloneq \left( A' \right)'$, known as the \textit{double commutant} of $A$.
\end{definition}
\begin{theorem}[Double Commutant Theorem]
  For any $\ast$-subalgebra $A\subseteq \B\left( \mathcal{H} \right)$, we have
  \begin{align*}
    \overline{A}^{\text{SOT}} = \overline{A}^{\text{WOT}} = A''.
  \end{align*}
  Furthermore, $A$ is a von Neumann algebra if and only if $A = A''$.
\end{theorem}
\begin{definition}
  If $\Gamma$ is a group with reduced group $\mathrm{C}^{\ast}$-algebra $\mathrm{C}^{\ast}_{\lambda}\left( \Gamma \right)$, then the \textit{group von Neumann algebra} is the space $L\left( \Gamma \right) \coloneq \mathrm{C}^{\ast}_{\lambda}\left( \Gamma \right)'' \subseteq \B\left( \ell_2\left( \Gamma \right) \right)$.
\end{definition}
\begin{proposition}
  The vector state $x\mapsto \iprod{x\delta_e}{\delta_e}$ defines a faithful tracial state on $\mathrm{C}^{\ast}_{\lambda}\left( \Gamma \right)$.
\end{proposition}
\begin{proof}
  Let $\lambda_s,\lambda_t\in \mathrm{C}^{\ast}_{\lambda}\left( \Gamma \right)$. Then,
  \begin{align*}
    \iprod{\lambda_s\lambda_t\left( \delta_e \right)}{\delta_e} &= \iprod{\lambda_{st}\left( \delta_e \right)}{\delta_e}\\
                                                                &= \iprod{\delta_{st}}{\delta_e}\\
                                                                &= \sum_{r\in\Gamma}\delta_e\left( r \right)\delta_{st}\left( r \right)\\
                                                                &= \delta_{st}\left( e \right)\\
                                                                &= \delta_{ts}\left( e \right)\\
                                                                &= \iprod{\delta_{ts}}{\delta_e}\\
                                                                &= \iprod{\lambda_t\lambda_s\left( \delta_e \right)}{\delta_e}.
  \end{align*}
  Thus, the vector state is tracial.
\end{proof}
One more fact we need is related to treating $\mathrm{C}^{\ast}$-algebras as particular types of modules and multiplicative domains. More information on the subject can be found in \cite[Chapter 3]{completely_bounded_maps_and_operator_algebras}.
\begin{definition}
  Let $R$ be a ring, and let $M$ be an abelian group. A \textit{left-action} of $R$ on $M$ is a map $\rho\colon R\times M \rightarrow M$ such that
  \begin{itemize}
    \item $\rho\left( r,m+n \right) = \rho\left( r,m \right) + \rho\left( r,n \right)$
    \item $\rho\left( r + s,m \right) = \rho\left( r,m \right) + \rho\left( s,m \right)$
    \item $\rho\left( rs,m \right) = \rho\left( r,\rho\left( s,m \right) \right)$
    \item $\rho\left( 1,m \right) = m$.
  \end{itemize}
  We write $r\cdot m \coloneq \rho\left( r,m \right)$. The left-action of $R$ on $M$ defines an \textit{left $R$-module} structure on $M$.\newline

  Analogously, we may define \textit{right $R$-modules} and \textit{$R$-bimodules} on $M$.
\end{definition}
Note that if $R\subseteq S$ is an inclusion of rings, then we can consider $S$ to be an $R$-module.\newline

Similarly, if $A$ is a unital $\mathrm{C}^{\ast}$-subalgebra with unital $\mathrm{C}^{\ast}$-subalgebra $C\subseteq A$, we may consider $A$ as a left $C$-module, or a right $C$-module, or a $C$-bimodule. 
\begin{definition}
  Let $\phi\colon A\rightarrow B$ be map between unital $\mathrm{C}^{\ast}$-algebras such that $C\subseteq A$ and $C\subseteq B$. Then, we say $\phi$ is a \textit{left $C$-module map} if $\phi\left( ca \right) = c\phi\left( a \right)$ for all $a\in A$ and $c\in C$. Similar definitions hold for right $C$-module maps and $C$-bimodule maps.
\end{definition}
\begin{theorem}\label{thm:preview_multiplicative_domain}
  Let $A$ and $B$ be unital $\mathrm{C}^{\ast}$-algebras, and let $\phi\colon A\rightarrow B$ be a completely positive map with $\phi\left(1_A\right) = 1_B$. Then, if
  \begin{align*}
    S &\coloneq\set{a\in A | \phi\left( a^{\ast}a \right) = \phi\left( a \right)^{\ast}\phi\left( a \right)\text{ and }\phi\left( aa^{\ast} \right) = \phi\left( a \right)\phi\left( a^{\ast} \right)}\\
      T&\coloneq \set{a\in A | \phi\left( ab \right) = \phi\left( a \right)\phi\left( b \right)\text{ and }\phi\left( ba \right) = \phi\left( b \right)\phi\left( a \right)\text{ for all }b\in A},
  \end{align*}
  we have $S = T$.\newline

  Furthermore, the set $S$ is a $\mathrm{C}^{\ast}$-subalgebra of $A$ such that $\phi$ is a $\ast$-homomorphism when restricted to this set.
\end{theorem}
A proof of this theorem can be found in \cite{completely_bounded_maps_and_operator_algebras} with modifications. 
\begin{definition}
  If $\phi\colon A\rightarrow B$ is a completely positive map, the set
  \begin{align*}
    \set{a | \phi\left( a^{\ast}a \right) = \phi\left( a \right)^{\ast}\phi\left( a \right)\text{ and }\phi\left( aa^{\ast} \right) = \phi\left( a \right)\phi\left( a \right)^{\ast}}
  \end{align*}
  is known as the \textit{multiplicative domain} for $\phi$.
\end{definition}
The multiplicative domain is essentially the $\mathrm{C}^{\ast}$-subalgebra of a domain of a completely positive map that, when restricted, makes the map a $\ast$-preserving multiplicative map. Specifically, if a completely positive map restricts to the identity on some $\mathrm{C}^{\ast}$-subalgebra, then that $\mathrm{C}^{\ast}$-subalgebra is automatically in the multiplicative domain of the map.
\begin{corollary}\label{cor:multiplicative_domain}
  Let $A$, $B$, and $C$ be unital $\mathrm{C}^{\ast}$-algebras, and suppose $C$ is a unital $\mathrm{C}^{\ast}$-subalgebra of both $A$ and $B$. If $\phi\colon A\rightarrow B$ is a completely positive map such that $\phi(c) = c$ for all $c\in C$, then $\phi$ is a $C$-bimodule map.
\end{corollary}
\begin{theorem}\label{thm:nuclearity_implies_amenability}
  Let $\mathrm{C}^{\ast}_{\lambda}\left( \Gamma \right)$ be nuclear. Then, $\Gamma$ is amenable.
\end{theorem}
\begin{proof}
  We will construct an invariant state on $\ell_{\infty}\left( \Gamma \right)$, which we consider as multiplication operators in $\B\left( \ell_2\left( \Gamma \right) \right)$ --- i.e., $M_{f}(x) = \sum_{t\in\Gamma}f(t)x(t)$ for $f\in \ell_{\infty}\left( \Gamma \right)$ and $x\in \ell_2\left( \Gamma \right)$.\newline

  Let $\varphi_n\colon \mathrm{C}^{\ast}_{\lambda}\left( \Gamma \right)\rightarrow \Mat_{k(n)}\left( \C \right)$ and $\psi_n\colon \Mat_{k(n)}\left( \C \right)\rightarrow \mathrm{C}^{\ast}_{\lambda}\left( \Gamma \right)$ be completely positive contractions such that
  \begin{align*}
    \norm{a - \psi_n\circ\varphi_n\left( a \right)}\xrightarrow{n\rightarrow\infty} 0
  \end{align*}
  for all $a\in A$.\newline

  Since $\Mat_{k(n)}\left( \C \right) \cong \B\left( \C^{k(n)} \right)$, we may use Arveson's extension theorem (Theorem \ref{thm:arveson}) to consider $\varphi_n\colon \B\left( \ell_2\left( \Gamma \right) \right)\rightarrow \Mat_{k(n)}\left( \C \right)$. Letting $\Phi_{n}\coloneq \psi_n\circ\varphi_n$, we note that $\norm{\Phi_n}\leq \norm{\psi_n}\norm{\varphi_n}\leq 1$, and that $\Phi_n\colon \B\left( \ell_2\left( \Gamma \right) \right)\rightarrow $ is such that $\Phi_n\left( x \right)\rightarrow x$ for all $x\in \mathrm{C}^{\ast}_{\lambda}\left( \Gamma \right)\subseteq L\left( \Gamma \right)$.\newline

  Now, since the $\Phi_n$ are all norm-bounded, by the Banach--Alaoglu Theorem (Theorem \ref{thm:banach_alaoglu}), we may find a subnet (or subsequence) converging to $\Phi\colon \B\left( \ell_2\left( \Gamma \right) \right)\rightarrow L\left( \Gamma \right)$ in the point-ultraweak topology, where $\Phi(x) = x$ for all $x\in \mathrm{C}^{\ast}_{\lambda}\left( \Gamma \right)$.\newline

  Letting $\tau\colon L\left( \Gamma \right)\rightarrow \C$ be the vector trace $x\mapsto \iprod{x\delta_e}{\delta_e}$, we define the state $\eta\coloneq \tau\circ \Phi$ on $\B\left( \ell_2\left( \Gamma \right) \right)$, which we restrict to $\ell_{\infty}$.\newline

  We note that left translation of the form $f(t) \mapsto f\left( s^{-1}t \right)$ (which we refer to by $\lambda_s$ in Proposition \ref{prop:translation_action}) is implemented via $M_{f}(\delta_t) \mapsto \lambda_{s}M_f\lambda_{s}^{\ast}\left( \delta_t \right)$. This can be seen by direct calculation:
  \begin{align*}
    \lambda_sM_f\lambda_s^{\ast}\left( \delta_t \right) &= \lambda_s\left( \sum_{r\in\Gamma}f(r)\lambda_{s^{-1}}\left( \delta_t \right) \left( r \right) \right)\\
                                                        &= \lambda_s\left( \sum_{r\in\Gamma}f(r)\delta_{s^{-1}t}(r) \right)\\
                                                        &= \sum_{r\in\Gamma}f\left( s^{-1}r \right)\delta_t(r)\\
                                                        &= f\left( s^{-1}t \right).
  \end{align*}
  Define $\mu\colon \ell_{\infty}\left( \Gamma \right)\rightarrow \C$ by $\mu\left( f \right) = \eta\left( M_f \right)$.\newline

  From Corollary \ref{cor:multiplicative_domain}, and the fact that for every $s\in\Gamma$, $\lambda_s\in \mathrm{C}^{\ast}_{\lambda}\left( \Gamma \right)$, we have
  \begin{align*}
    \eta\left( \lambda_s T \lambda_s^{\ast} \right) &= \tau\left( \Phi\left( \lambda_s T \lambda_s^{\ast} \right) \right)\\
                                                    &= \tau\left( \lambda_s \Phi(T) \lambda_s^{\ast}\right)\\
                                                    &= \tau\left( \lambda_s^{\ast}\lambda_s\Phi(T) \right)\label{eq:using_traciality}\tag{\textasteriskcentered}\\
                                                    &= \tau\left( \Phi\left( T \right) \right)\\
                                                    &= \eta(T),
  \end{align*}
  for all $T\in \B\left( \ell_2\left( \Gamma \right) \right)$, where in \eqref{eq:using_traciality} we use the fact that $\tau$ is tracial. In particular, this identity holds for $T= M_f$ where $f\in \ell_{\infty}\left( \Gamma \right)$. Thus,
  \begin{align*}
    \mu\left( \lambda_s(f) \right) &= \eta\left( \lambda_sM_f\lambda_s^{\ast} \right)\\
                                   &= \eta\left( M_f \right)\\
                                   &= \mu\left( f \right),
  \end{align*}
  so $\mu\colon \ell_{\infty}\left( \Gamma \right)$ is an invariant state. Thus, $\Gamma$ is amenable.
\end{proof}

\subsection{Amenability and Further Structural Properties of Group $\mathrm{C}^{\ast}$-Algebras}%
In order to prove the cycle (i) $\Rightarrow$ (iii) $\Rightarrow$ (iv) $\Rightarrow$ (i) in Theorem \ref{thm:amenability_cstar_algebras_formulations}, we will need to know a little bit more about tensor products of Hilbert spaces and operators.\newline

If $\mathcal{H}$ and $\mathcal{K}$ are Hilbert spaces (Definition \ref{def:hilbert_spaces}), we define an inner product on the tensor product $\mathcal{H}\otimes \mathcal{K}$ by
\begin{align*}
  \iprod{x\otimes y}{x'\otimes y'} &= \iprod{x}{x'} \iprod{y}{y'}.
\end{align*}
Completing $\mathcal{H}\otimes \mathcal{K}$ with respect to the norm induced by this inner product yields the Hilbert space tensor product, which we will yet again denote $\mathcal{H}\otimes \mathcal{K}$.\newline

The Hilbert space $\ell_2\left( \Gamma,\mathcal{H} \right)$ consists of all functions $f\colon \Gamma\rightarrow \mathcal{H}$ such that
\begin{align*}
  \norm{f}_{\ell_2\left( \Gamma,\mathcal{H} \right)}^2 &\coloneq \sum_{t\in\Gamma}\norm{f(t)}^2\\
                                                       &< \infty,
\end{align*}
with inner product
\begin{align*}
  \iprod{f}{g} &= \sum_{t\in\Gamma} \iprod{f(t)}{g(t)}.
\end{align*}
It can be shown that $\ell_2\left( \Gamma,\mathcal{H} \right)$ is isometrically isomorphic to $\ell_2\left( \Gamma \right)\otimes \mathcal{H}$.\footnote{For an outline, map $\ell_2\left( \Gamma \right)\times \mathcal{H}\rightarrow \ell_2\left( \Gamma,\mathcal{H} \right)$ by taking $\left( f,\xi \right)\mapsto \left( f(t)\xi \right)_{t\in\Gamma}$. Then, use the universal property to find a linear map that preserves the inner product (hence isometric). Then, extend continuously to the norm closure, and show that it is an isomorphism.}\newline

Now, if $T\in \B\left( \mathcal{H} \right)$ and $S\in \B\left( \mathcal{K} \right)$, there is a linear map $T\otimes S\in \B\left( \mathcal{H}\otimes \mathcal{K} \right)$ such that $T\otimes S \left( x\otimes y \right) = T(x)\otimes S(y)$, and $\norm{T\otimes S}_{\op} = \norm{T}_{\op}\norm{S}_{\op}$.\newline

This is the object of our study, specifically for the purpose of proving a property of the left-regular representation, known as Fell's Absorption Principle.
\begin{theorem}[Fell's Absorption Principle]
  Let $\Gamma$ be a discrete group, let $\lambda\colon \Gamma\rightarrow \mathcal{U}\left( \ell_2\left( \Gamma \right) \right)$ be the left-regular representation (Theorem \ref{thm:left_regular_representation}), and let $\pi\colon \Gamma\rightarrow \mathcal{U}\left( \mathcal{H} \right)$ be any unitary representation.\newline

  Then, there is a unitary operator $U\colon \ell_2\left( \Gamma \right)\otimes \mathcal{H} \rightarrow \ell_2\left( \Gamma \right)\otimes \mathcal{H}$ such that $U\left( \lambda\otimes 1_{\mathcal{H}} \right) U^{\ast}= \lambda\otimes \pi$, where $1_{\mathcal{H}}\colon \Gamma\rightarrow \mathcal{H}$ denotes the representation that maps every $s\in\Gamma$ to $I_{\mathcal{H}}$.
\end{theorem}
\begin{proof}
  Elements of $\ell_2\left( \Gamma \right)\otimes \mathcal{H}$ are of the form $\sum_{t\in\Gamma}\delta_t\otimes \xi_{t}$, where $\delta_t$ is the point mass at $t\in\Gamma$ and $\xi_{t}\in \mathcal{H}$ is some vector.\newline

  By linearity and continuity, it is sufficient to verify that $\lambda\otimes \pi$ and $\lambda\otimes 1_{\mathcal{H}}$ are unitarily equivalent on elementary tensors. For $\delta_t\otimes \xi_t\in \ell_2\left( \Gamma \right)\otimes \mathcal{H}$, we define
  \begin{align*}
    U\left( \delta_t\otimes \xi_t \right) &= \delta_t\otimes \pi(t)\left(\xi_t\right).
  \end{align*}
  This map is unitary since, for any $t$, $\pi(t)\in \mathcal{U}\left( \mathcal{H} \right)$ by definition. Thus, we have
  \begin{align*}
    U\left( \left( \lambda\otimes 1_{\mathcal{H}} \right)(s) \right)\left( \delta_t\otimes \xi_t \right) &= U\left( \lambda_s\left(\delta_t\right)\otimes \xi_t \right)\\
                                                                                                         &= U\left( \delta_{st}\otimes \xi_t \right)\\
                                                                                                         &= U\left( \delta_{r}\otimes \xi_{s^{-1}r} \right)\\
                                                                                                         &= \delta_{r}\otimes \pi(r)\left( \xi_{s^{-1}r} \right).
  \end{align*}
  Meanwhile,
  \begin{align*}
    \left( \lambda\otimes \pi \right)(s)\left( U\left( \delta_t\otimes \xi_t \right) \right) &= \left( \left( \lambda\otimes \pi \right)(s) \right)\left( \delta_t\otimes \pi(t)\left( \xi_t \right) \right)\\
                                                                                             &= \lambda_s\left( \delta_t \right)\otimes \pi(s)\pi(t)\left( \xi_t \right)\\
                                                                                             &- \delta_{st}\otimes \pi\left( st \right)\left( \xi_{t} \right)\\
                                                                                             &= \delta_{r}\otimes \pi\left( r \right)\left( \xi_{s^{-1}r} \right).
  \end{align*}
  Therefore, $U\left( \lambda\otimes 1_{\mathcal{H}} \right)U^{\ast} = \lambda\otimes \pi $.
\end{proof}
Now, we may prove the cycle (i) $\Rightarrow $ (iii) $\Rightarrow$ (iv) $\Rightarrow$ (i) from Theorem \ref{thm:amenability_cstar_algebras_formulations}.
\begin{theorem}
  Let $\Gamma$ be a discrete group. The following are equivalent:
  \begin{description}[font=\normalfont]
    \item[(i)] $\Gamma$ is amenable;
    \item[(iii)] $\mathrm{C}^{\ast}_{\lambda}\left( \Gamma \right) = \mathrm{C}^{\ast}\left( \Gamma \right)$;
    \item[(iv)] $\mathrm{C}^{\ast}_{\lambda}\left( \Gamma \right)$ admits a character (Definition \ref{def:algebra_homomorphisms_and_characters}).
  \end{description}
\end{theorem}
\begin{proof}
  Suppose $\Gamma$ is amenable. Then, the left-regular representation, $\lambda\colon \Gamma\rightarrow \mathcal{U}\left( \ell_2\left( \Gamma \right) \right)$, admits an almost-invariant vector, $\left( \xi_n \right)_n\subseteq S_{\ell_2\left( \Gamma \right)}$.\newline

  For any unitary representation $\widetilde{\pi}\colon \Gamma\rightarrow \mathcal{U}\left( \mathcal{H} \right)$, we let $\pi\colon \C\left[ \Gamma \right]\rightarrow \B\left( \mathcal{H} \right)$ be the corresponding unital representation (Proposition \ref{prop:unital_unitary_representation}). Then, for any $x\in \C\left[ \Gamma \right]$ and any $\eta_1,\eta_2\in S_{\mathcal{H}}$, we use the almost-invariance of $\left( \xi_n \right)_n$ and the Cauchy--Schwarz inequality to find
  \begin{align*}
    \left\vert \iprod{\pi(x)\left( \eta_1 \right)}{\eta_2} \right\vert &= \left\vert \iprod{\left( \left( \lambda\otimes \pi \right)\left( x \right) \right)\left( \xi_n\otimes \eta_1 \right)}{\xi_n\otimes \eta_2} \right\vert\\
                                                                       &\leq \norm{\left( \lambda\otimes \pi \right)(x)}_{\op}\\
                                                                       &= \norm{\left( \lambda\otimes 1_{\mathcal{H}} \right)\left( x \right)}_{\op}\\
                                                                       &= \norm{\lambda(x)}_{\op},
  \end{align*}
  where the second-to-last equality used the fact that $\lambda\otimes \pi$ and $\lambda\otimes 1_{\mathcal{H}}$ are unitarily equivalent. Now, taking the supremum over all $\eta_1,\eta_2\in S_{\mathcal{H}}$, we get
  \begin{align*}
    \norm{\pi(x)}_{\op} &\leq \norm{\lambda(x)}_{\op},
  \end{align*}
  and taking the supremum over all representations $\pi\colon \C\left[ \Gamma \right]\rightarrow \mathcal{H}$, we get
  \begin{align*}
    \norm{x}_{u} &\leq \norm{\lambda(x)}_{\op},
  \end{align*}
  so $\norm{x}_u = \norm{\lambda(x)}_{\op}$. Thus, the surjective homomorphism provided by the universal property $\varphi\colon \mathrm{C}^{\ast}\left( \Gamma \right)\rightarrow \mathrm{C}^{\ast}_{\lambda}\left( \Gamma \right)$ is isometric, so $\mathrm{C}^{\ast}\left( \Gamma \right) = \mathrm{C}^{\ast}_{\lambda}\left( \Gamma \right)$.\newline

  Let $\mathrm{C}^{\ast}_{\lambda}\left( \Gamma \right) = \mathrm{C}^{\ast}\left( \Gamma \right)$. Now, considering the trivial representation from Example \ref{ex:some_representations}, we know from the universal property of the universal group $\mathrm{C}^{\ast}$-algebra that $1_{\Gamma}$ extends to a unital $\ast$-homomorphism $\tau\colon \C\left[ \Gamma \right]\rightarrow \B\left( \C \right) = \C$. Since $\mathrm{C}^{\ast}_{\lambda}\left( \Gamma \right) = \mathrm{C}^{\ast}\left( \Gamma \right)$, we have a nonzero $\ast$-homomorphism (or a character) $\tau\colon \mathrm{C}^{\ast}_{\lambda}\left( \Gamma \right)\rightarrow \C$.\newline

  Let $\tau\colon \mathrm{C}^{\ast}_{\lambda}\left( \Gamma \right)\rightarrow \C$ be a character. By Arveson's extension theorem, we may extend $\tau$ to $\sigma\colon \B\left( \ell_2\left( \Gamma \right) \right)\rightarrow \C$; note that $\sigma\bigr\vert_{\mathrm{C}^{\ast}_{\lambda}\left( \Gamma \right)} = \tau$, so $\mathrm{C}^{\ast}_{\lambda}\left( \Gamma \right)$ is part of the multiplicative domain of $\sigma$.\newline

  Considering $\ell_{\infty}\left( \Gamma \right)\subseteq \B\left( \ell_2\left( \Gamma \right) \right)$ as multipliers, we use the fact that $\lambda_s(f) $ (as in Proposition \ref{prop:translation_action}) and $\lambda_sM_f\lambda_s^{\ast}$ are interchangeable (where $M_f$ denotes the multiplication operator). Define $\mu\colon \ell_{\infty}\left( \Gamma \right)\rightarrow \C$ by $\mu(f) = \sigma\left( M_f \right)$. Then, since $\lambda_s,\lambda_s^{\ast}\in \mathrm{C}^{\ast}_{\lambda}\left( \Gamma \right)$, we have
  \begin{align*}
    \mu\left( \lambda_s(f) \right) &= \sigma\left( \lambda_sM_f\lambda_s^{\ast} \right)\\
                                   &= \sigma\left( \lambda_s \right)\sigma\left( M_f \right)\sigma\left( \lambda_s^{\ast} \right)\\
                                   &= \sigma\left( \lambda_s\lambda_s^{\ast} \right)\sigma\left( M_f \right)\\
                                   &= \sigma\left( M_f \right)\\
                                   &= \mu\left( f \right),
  \end{align*}
  so $\mu$ is an invariant state, meaning $\Gamma$ is amenable.
\end{proof}

\section{Remarks and Notes}%
The original definition of nuclearity for $\mathrm{C}^{\ast}$-algebras, discussed in \cite{cross_norm_takesaki}, concerns norms on the tensor product $A\otimes B$ of a $\mathrm{C}^{\ast}$-algebra $A$ with any other $\mathrm{C}^{\ast}$-algebra $B$. There are two distinguished norms that can be from a tensor product of $\mathrm{C}^{\ast}$-algebras. The ``maximal norm'' is defined akin to the universal norm for the group $\mathrm{C}^{\ast}$-algebra, where one takes the supremum over all representations, and the ``minimal norm'' is defined with respect to faithful representations of the particular $\mathrm{C}^{\ast}$-algebras, similar to how the reduced group $\mathrm{C}^{\ast}$-algebra was defined with respect to the left-regular representation. Takesaki says that a $\mathrm{C}^{\ast}$-algebra is nuclear if these two norms coincide whenever the $\mathrm{C}^{\ast}$-algebra $A$ has its tensor product taken with any other $\mathrm{C}^{\ast}$-algebra $B$.\newline

By this definition, we already know that $\Mat_n\left( \C \right)$ is a nuclear $\mathrm{C}^{\ast}$-algebra, as we mentioned that for any other $\mathrm{C}^{\ast}$-algebra $A$, there is a canonically defined norm on $\Mat_n\left( A \right)\cong \Mat_n\left( \C \right)\otimes A$. Furthermore, in the case where $\mathrm{C}^{\ast}_{\lambda}\left( \Gamma \right)$ is nuclear, as in Definition \ref{def:nuclear_cstar_algebra}, it can be shown using Fell's absorption principle that this implies the fact that $\mathrm{C}^{\ast}\left( \Gamma \right) = \mathrm{C}^{\ast}_{\lambda}\left( \Gamma \right)$, which is akin to Takesaki's definition of nuclearity. However, showing the equivalence between the completely positive approximation property of Definition \ref{def:nuclear_cstar_algebra} and Takesaki's definition was shown in \cite{choi_nuclearity}.

\chapter{Closing Remarks}
\epigraph{There is always something left undone...}{Paul Halmos, ``How to Write Mathematics''}
In \cite[48]{brown_and_ozawa}, the authors remark that ``amenable groups admit approximately $10^{10^{10}}$ characterizations.'' This paper was not long enough to discuss all of them, even when restricted to the case of discrete groups.\newline

In Section \ref{sec:subexponential_growth}, we discussed an application of the Følner condition to establishing amenability for a crucial class of groups in geometric group theory (the groups of subexponential growth) --- yet another direction in amenability concerns further study into geometric group theory, including (but not limited to) discussion of how amenability of a group relates to properties of its Cayley graph (see \cite[Section 3.2]{loh_geometric_group_theory} for more discussion on Cayley graphs). There is a notion of graph amenability related to the growth of a graph's neighboring vertex set that, it can be shown, is equivalent to the Følner condition in the case of Cayley graphs (see \cite{monfared_cayley_graphs}).\newline

There are some other directions in amenability that we might be able to take this text. There is a rich theory of amenability in locally compact groups, as well as amenability in Banach algebras and von Neumann algebras. Some of the authoritative texts on this subject include \cite{kazhdan_property_t} and \cite{amenable_banach_algebras} --- we have only touched the surface of what these texts have to offer in the discussion of amenability. Amenability in locally compact groups requires a much stronger command of abstract measure theory, especially concerning the Haar measure (which is a type of translation-invariant measure on the group), as well as integration theory with respect to abstract measures.\newline

In general, most of the results we discussed in this text that pertain to amenability in discrete groups can be translated with relative ease to the case of locally compact groups, primarily by replacing sums (which are really integrals with respect to the counting measure) with integrals with respect to the Haar measure. One analogous result, for instance, is that compact groups are amenable. However, there are also other criteria --- one may define a version of amenability for Banach algebras, and then show that a locally compact group $G$ is amenable if and only if the space $L_1\left( G \right)$ is amenable as a Banach algebra, where $L_1\left( G \right)$ is the space of integrable functions with respect to the Haar measure.\newline

Another direction in amenability concerns deeper discussion of random walks on groups, which was the primary topic of \cite{kesten_means} and \cite{kesten_random_walks}. When we discussed Kesten's criterion in the text (Theorem \ref{thm:kesten_criterion}) we only looked at the basic case where $M(S)$ was defined with respect to the symmetric generating set itself, rather than the general case of a finitely supported probability measure $\mu$ with $S\subseteq \supp(\mu)$.\newline

However, despite all of this, the main regret I have is that this text was ultimately a bit too scattered --- because there are so many different definitions for amenability, this thesis touched on so many topics that to provide enough background development for all the prerequisite ideas (coming from the ambitious assumption that anyone with a background in the standard third/fourth year real analysis courses would be able to glean most of the information in the text) would have resulted in a thesis that was even longer than this one already was. There were many points while I drafted the thesis that I had to go back and add particular bits of information that I didn't expect would be necessary, and there are certainly many tidbits and background highlights that are missing.\newline

Ultimately, this project --- understanding amenability, broadly construed --- will never be fully complete. There are tons of characterizations and nuances that appear as one goes deeper into the topic of amenability, but I like to believe I gave the topics discussed herein a fair shake, and certainly in a manner that is more than deep enough for most to find the text substantial rewarding.

% need snappy name
\appendix
\chapter{Algebra and Linear Algebra}\label{ch:algebra_and_linear_algebra}
In general, as we progress through these appendices, we will consistently add additional structure to a set. First, we begin by developing groups, rings, and fields, vector spaces, and algebras. In the following appendices, we will apply metric structures, topologies, and measures, building up to the central structure of functional analysis: normed vector spaces and the operators on these normed vector spaces.\newline

These appendices were largely written to provide essential background for the techniques and results that will appear in the main body of the text. As such, they do not include detailed proofs --- occasionally, we will include outlines for certain proofs in the remarks. The proofs for many of these results can be found in relevant (and some not-as-relevant) texts.\newline

We make heavy use of results from algebra and linear algebra in this thesis. Some excellent resources to learn more about algebra and linear algebra are \cite{dummit_and_foote} and \cite{algebra_chapter_0}. Most of the theorems are presented without proof, not because we do not want to state their proofs, but because this thesis is already long enough.
\section{Group, Rings, (some) Fields}%
\subsection{Groups}%
\begin{definition}[Groups]
  Let $A$ be a set, and let $\star$ be a binary operation on $A$. We say $A$ is a group if
  \begin{itemize}
    \item $A$ is closed under the operation $\star$;
    \item $\star$ is associative, such that for all $a,b,c\in A$, $\left(a\star b\right)\star c = a\star \left(b\star c\right)$;
    \item $A$ has an identity element $e_A$, where $a\star e_A = e_A\star a = a$ for any $a\in A$;
    \item for any $a\in A$, there exists $a^{-1}\in A$ such that $a^{-1}\star a = a\star a^{-1} = e$.
  \end{itemize}
  If the operation $\star$ is such that $a\star b = b\star a$ for all $a,b\in A$, then we say $A$ is an abelian group.\newline

  Generally, we abbreviate $a\star b \coloneq ab$.
\end{definition}
\begin{definition}[Subgroups, Normal Subgroups, and Quotient Groups]
  If $G$ is a group, $H\subseteq G$ is a subgroup if $H$ is closed under the group operation and inverses. We write $H\leq G$.\newline

  If $H$ is a subgroup, a left coset of $H$ is the set $gH \coloneq \set{gh | h\in H}$, where $g\in G$. Similarly, a right coset of $H$ is the set $Hg\coloneq \set{hg | h\in H}$. The index of $H$, denoted $\left[G:H\right]$, is the number of left (or right) cosets of $H$.\newline

  If $H\leq G$ is also such that, for any $g\in G$ and $h\in H$, $ghg^{-1}\in H$, then we call $H$ a normal subgroup of $G$. We write $H\trianglelefteq G$.\newline

  Defining the equivalence relation $g\sim g'$ if and only if $g^{-1}g'\in H$, the group of equivalence classes $gH\coloneq \left[g\right]$ is known as the quotient group $G/H$.\newline

  If the only normal subgroups of a group $G$ are $G$ itself and $\set{e_{G}}$, then we say the group $G$ is simple.
\end{definition}
\begin{definition}
  Let $G$ and $H$ be groups. A map $\varphi\colon G\rightarrow H$ is called a (group) homomorphism if, $\varphi$ ``preserves the group structure,'' in the sense that
  \begin{align*}
    \varphi\left(ab\right) &= \varphi(a)\varphi(b)\\
    \varphi\left(a^{-1}\right) &= \varphi(a)^{-1}
  \end{align*}
  for all $a,b\in G$.\newline

  We define $\ker\left(\varphi\right)$ to be the set of all $g\in G$ such that $\varphi\left(g\right) = e_H$.\newline

  If $H\trianglelefteq G$, the map $\pi\colon G\rightarrow G/H$ that sends $g\mapsto gH$ is known as the canonical projection.\newline

  If $\varphi$ is a bijection, then $\varphi$ is known as an isomorphism. We write $G\cong H$ if there exists an isomorphism $\varphi\colon G\rightarrow H$.
\end{definition}
\begin{theorem}[First Isomorphism Theorem for Groups]
  Let $G$ and $H$ be groups, and let $\varphi\colon G\rightarrow H$ be a group homomorphism. Then, $\ker\left(\varphi\right)\trianglelefteq G$ is a normal subgroup, and $G/\ker\left(\varphi\right)\cong \img\left(\varphi\right)$.
\end{theorem}
There is a complete classification of finitely generated (and finite) abelian groups.\footnote{There is also a complete classification of all the finite simple groups, but we absolutely do not have enough space for that one.}
\begin{theorem}
  Let $G$ be a finitely generated Abelian group. Then, there is some $d\in \N$ and some $k_1,\dots,k_n$ such that
  \begin{align*}
    G &\cong \underbrace{\Z^{d}}_{F}\times \underbrace{\Z/k_1\Z \times \Z/k_2\Z \times \cdots \times \Z/k_n\Z}_{T}. 
  \end{align*}
  The group $F$ is known as the free subgroup of $G$, and the group $T$ is known as the torsion subgroup of $G$. If $G$ is finite, then $G$ is isomorphic to some torsion group $\Z/k_1\Z\times \cdots \times \Z/k_1\Z$.\newline

  Furthermore, we may also take each of $k_i$ in both cases to be equal to $p_i^{e_i}$ for some prime $p_i$ and some $e_i\in \N$.
\end{theorem}
\begin{definition}
  A group $G$ is solvable if if admits a finite series of normal subgroups
  \begin{align*}
    e_G = G_0\trianglelefteq G_1\trianglelefteq \cdots \trianglelefteq G_n \trianglelefteq G
  \end{align*}
  such that $G_{j}/G_{j-1}$ is abelian for each $j = 1,\dots,n$.
\end{definition}

\begin{definition}[Group Actions]
  Let $G$ be a group, and let $A$ be a set. A (left) group action of $G$ on $A$ is a map $\rho\colon G\times A \rightarrow A$ such that, for all $a\in A$,
  \begin{itemize}
    \item $\rho\left(e_G,a\right) = a$;
    \item $\rho\left(g,\rho\left(h,a\right)\right) = \rho\left(gh,a\right)$.
  \end{itemize}
  We abbreviate $\rho\left(g,a\right) = g\cdot a$.\newline

  The permutation representation of the action $\rho$ is a homomorphism $\varphi\colon G\rightarrow \sym(A)$.
\end{definition}
\begin{definition}[Kernels, Stabilizers, and Orbits]
  Let $G$ act on $A$, and let $a\in A$.
  \begin{itemize}
    \item The stabilizer of $a$ under $G$ is the set of elements in $G$ that fix $a$:
      \begin{align*}
        G_{a} \coloneq \set{g\in G | g\cdot a = a}.
      \end{align*}
    \item The kernel of the action of $G$ on $A$ is the intersection of the stabilizers of $G$:
      \begin{align*}
        \text{kernel} &\coloneq \bigcap_{a\in A}G_a\\
                      &= \set{g\in G | g\cdot a\text{ for all }a\in A}.
      \end{align*}
    \item The action is faithful if the kernel is $e_G$.
    \item The action is free if $G_a = \set{e_G}$ for all $a\in A$.
    \item The orbit of $a$ is the equivalence class $\left[a\right]_{\sim}$ under the relation $a\sim b$ if there exists some $g\in G$ such that $a = g\cdot b$:
      \begin{align*}
        G\cdot a &= \set{b\in A | b = g\cdot a\text{ for some }g\in G}.
      \end{align*}
  \end{itemize}
\end{definition}
\begin{theorem}[Orbit-Stabilizer Theorem]
  If $G$ acts on $A$, and $a\in A$, then the number of elements in the orbit of $a$ is the index of the stabilizer of $a$. In symbolic form,
  \begin{align*}
    \left\vert G\cdot a \right\vert &= \left[G:G_a\right].
  \end{align*}
\end{theorem}
\begin{remark}
  Various theorems such as the Sylow Theorems and Lagrange's Theorem fall out of the Orbit-Stabilizer theorem.
\end{remark}
\subsection{Rings and Fields}%
\begin{definition}[Rings]
  Let $A$ be a set. Specifically, let $A$ be an abelian group, letting $+$ denote the operation on $A$ and $0\coloneq e_A$. Then, $A$ is a ring if $A$ also admits a multiplication, $\cdot$, such that
  \begin{itemize}
    \item $a\cdot \left(b+c\right) = a\cdot b + a\cdot c$;
    \item $\left(a+b\right)\cdot c = a\cdot c + b\cdot c$;
    \item $\left(a\cdot b\right)\cdot c = a\cdot \left(b\cdot c\right)$.
  \end{itemize}
  If the multiplication on $A$ is commutative, then we say $A$ is a commutative ring. If $A$ admits an element $1_A$ such that $a\cdot 1_A = 1_A \cdot a = a$, then we say $A$ is a unital ring.\newline

  When referring to the abelian group of $A$ under $+$, we often write $\left(A,+\right)$.
\end{definition}
\begin{definition}[Subrings, Ideals, and Quotient Rings]
  Let $R$ be a ring. A subset $A\subseteq R$ is known as a subring if $A$ is a subgroup of $\left(R,+\right)$, and $A$ is closed under multiplication. In other words, for all $a,b\in A$, we have
  \begin{itemize}
    \item $a-b\in A$;
    \item $ab \in A$,
  \end{itemize}
  where $a-b = a + (-b)$.\newline

  If $I\subseteq R$ is a subring that also has the property that, for all $x\in I$ and $r\in R$, $rx\in I$ and $xr\in I$, then we say $I$ is an ideal.\newline

  Similar to the case of groups and normal subgroups, we can form the quotient ring $R/I$ by defining the equivalence relation $a\sim b$ if $a-b\in I$, and defining $a + I\coloneq \left[a\right]_{\sim}$.
\end{definition}
\begin{definition}[Ring Homomorphism]
  If $R$ and $S$ are rings, then a map $\varphi\colon R\rightarrow S$ is a ring homomorphism if $\varphi$ ``preserves the ring structure,'' in the sense that, for all $a,b\in R$,
  \begin{itemize}
    \item $\varphi\left(a+b\right) = \varphi\left(a\right) + \varphi\left(b\right)$;
    \item $\varphi\left(ab\right) = \varphi\left(a\right)\varphi\left(b\right)$.
  \end{itemize}
  The kernel of the ring homomorphism is defined to be the set of all elements $a\in R$ such that $\varphi\left(a\right) = 0_{S}$.\newline

  If $I \subseteq R$ is an ideal, then the map $\pi\colon R\rightarrow R/I$ that sends $a \mapsto a + I$ is known as the canonical projection.\newline

  If $\varphi$ is a bijection, then $\varphi$ is known as an isomorphism. We write $R\cong S$ if there exists an isomorphism $\varphi\colon R\rightarrow S$.
\end{definition}
Analogously, there is a first isomorphism theorem for rings.
\begin{theorem}[First Isomorphism Theorem for Rings]
  Let $R$ and $S$ be rings, and let $\varphi\colon R\rightarrow S$ be a ring homomorphism. Then, $\ker\left(\varphi\right)\subseteq R$ is an ideal, and $R/\ker\left(\varphi\right)\cong \img\left(\varphi\right)$.
\end{theorem}
\begin{definition}
  Let $R$ be a unital ring.
  \begin{itemize}
    \item If $a,b\in R$ are nonzero elements such that $ab = 0$, then we say $a$ and $b$ are zero divisors in $R$.
    \item If $R$ is commutative and does not contain any zero divisors, then we say $R$ is an integral domain.
    \item If an element $a\in R$ is such that there exists some $b$ such that $ab = ba = 1_R$, then we call $a$ a unit.
    \item If $R$ is a such that every element of $R$ is a unit, then we say $R$ is a division algebra.
    \item If $R$ is a division algebra that is commutative, then $R$ is a field.
  \end{itemize}
\end{definition}
\begin{remark}
  Generally, when we deal with fields, we will usually be dealing with the complex numbers, $\C$, unless otherwise stated.
\end{remark}
\section{Linear Algebra}%
Certain constructions in linear algebra are extremely important in understanding functional analysis. We provide an overview of the theory of vector spaces and linear transformations in the purely algebraic context. Analytic properties that result from applying norms on these vector spaces will appear in Appendix \ref{ch:functional_analysis}.
\subsection{The Structure of Vector Spaces}%
\begin{definition}
  Let $X$ be some set, and $\F$ some field (generally, we assume $\F = \C$). We say $X$ is an $\F$-vector space if $X$ is equipped with two operations:
  \begin{itemize}
    \item scalar multiplication: $m\colon \F\times X \rightarrow X$, which sends $\left(\alpha,x\right)\mapsto \alpha x$; and
    \item vector addition: $a\colon X\times X \rightarrow X$, which sends $\left(x,y\right)\mapsto x + y$.
  \end{itemize}
  In general, $\left(X,+\right)$ is an abelian group, and scalar multiplication satisfies the following identities, for all $\alpha,\beta \in \F$ and $x,y\in X$:
  \begin{itemize}
    \item $\alpha \left(\beta x\right) = \left(\alpha\beta\right)x$;
    \item $\alpha\left(x+y\right) = \alpha x + \alpha y$;
    \item $\left(\alpha + \beta\right)x = \alpha x + \beta x$;
    \item $1_{\F}x = x$;
    \item $0_{\F} x = 0_X$.
  \end{itemize}
\end{definition}
There are certain geometric properties of subsets of vector spaces that we will be using a lot, especially when we discuss locally convex topologies on these vector spaces.
\begin{definition}\label{def:vector_space_subset_operations}
  Let $X$ be a $\C$-vector space.
  \begin{itemize}
    \item If $A,B\subseteq X$, then we define
      \begin{align*}
        A + B &= \set{x + y | x\in A,y\in B}.
      \end{align*}
      If $A = \set{x_0}$, we abbreviate $\set{x_0} + B$ as $x_0 + B$, which is called the translation of $B$.
    \item If $A\subseteq X$, and $\alpha\in \C$, then
      \begin{align*}
        \alpha A &= \set{\alpha x | x\in A}
      \end{align*}
      is the scaling of $A$ by $\alpha$. We write $(-1)A = -A$.
    \item A subset $A\subseteq X$ is called symmetric if $-A = A$.
    \item A subset $A\subseteq X$ is called balanced if $\alpha A\subseteq A$ for all $\left\vert \alpha \right\vert\leq 1$.
    \item A subset $C\subseteq X$ is called convex if for all $t\in [0,1]$ and $x_1,x_2\in C$, $\left(1-t\right)x_1 + tx_2 \in C$.
  \end{itemize}
  We define
  \begin{align*}
    \operatorname{conv}\left(A\right) &= \bigcap\set{C | A\subseteq C\subseteq X,C\text{ is convex}}\\
                                      &= \set{\sum_{j=1}^{n}t_ja_j | n\in\N,t_j\geq 0,\sum_{j=1}^{n}t_j = 1,a_j\in A}.
  \end{align*}
\end{definition}
\begin{definition}
  If $X$ is a vector space, and $M\subseteq X$ is a subset such that, for all $x,y\in M$ and $\alpha\in \C$, $\alpha x + y\in M$, then $M$ is called a subspace of $X$.\newline

  If $M$ is a subspace, we may define the equivalence relation $x\sim_{M} y$ if and only if $x-y\in M$. Equivalence classes under the relation $\sim_{M}$ are written $x + M\coloneq \left[x\right]_{\sim}$, and form the quotient space $X/M$.
\end{definition}
\begin{definition}
  Let $X$ be a vector space, and let $\set{x_i}_{i\in I}\subseteq X$ be a subset.
  \begin{itemize}
    \item The set $\set{x_i}_{i\in I}$ is called linearly independent if, for any finite linear combination such that
      \begin{align*}
        \sum_{i\in I}\alpha_ix_i &= 0_X,
      \end{align*}
      it is the case that all $\alpha_i = 0$.
    \item The set $\set{x_i}_{i\in I}$ is called spanning for $X$ if the set of all finite linear combinations $\sum_{i\in I}\alpha_ix_i$ is equal to $X$.
    \item The set $\set{x_i}_{i\in I}$ is called a basis for $X$ if it is linearly independent and spanning.
  \end{itemize}
\end{definition}
\begin{definition}
  If $X$ is a vector space, and $\mathcal{B}\subseteq X$ is a basis, then $\Dim\left(X\right)\coloneq \left\vert \mathcal{B} \right\vert$.\newline

  If $M\subseteq X$ is a subspace, then the codimension of $M$ is $\Dim\left(X/M\right)$.
\end{definition}
\begin{remark}
  Every vector space has a basis. This can be proven with Zorn's Lemma (Theorem \ref{thm:zorns_lemma}) applied on the partially ordered set (Definition \ref{def:ordered_sets}) of linearly independent subsets ordered by inclusion.\newline

  Additionally, not only does every vector space have a basis, but for any set, there is a vector space with the set as its basis (see Theorem \ref{thm:free_vector_space}).
\end{remark}
\subsection{Linear Maps}%
Linear algebra is not only the study of vector spaces, but also the study of linear maps on these vector spaces. 
\begin{definition}
  Let $X$ and $Y$ be vector spaces. A map $T\colon X\rightarrow Y$ is called linear if for every $x,x_1,x_2\in X$ and $\alpha\in \C$, we have
  \begin{align*}
    T\left(x_1 + x_2\right) &= T\left(x_1\right) + T\left(x_2\right)\\
    T\left(\alpha x\right) &= \alpha T\left(x\right).
  \end{align*}
  We write $I_{X} \coloneq \id_{X}$.\newline

  The set of linear maps between $X$ and $Y$ is denoted $\mathcal{L}\left(X,Y\right)$. The set of all linear maps between $X$ and $X$ is abbreviated $\mathcal{L}\left(X\right)$.\newline

  A linear map $\varphi\colon X\rightarrow \C$ is called a linear functional on $X$. The collection of linear functionals on $X$ is called the algebraic dual of $X$, written $X' \coloneq \mathcal{L}\left(X,\C\right)$.\newline

  The space $\mathcal{L}\left(X,Y\right)$ equipped with pointwise operations is a vector space over $\C$.
\end{definition}
\begin{definition}[Four Fundamental Subspaces]
  Let $T\colon X\rightarrow Y$ be a linear map.
  \begin{itemize}
    \item The kernel of $T$, $\ker\left(T\right)$, is the set of all $x\in X$ such that $T(x) = 0_X$.
    \item The range of $T$, $\Ran\left(T\right)$, is the set of all $y\in Y$ such that there exists $x\in X$ with $T(x) = y$.
    \item The cokernel of $T$ is $\operatorname{coker}(T) = Y/\Ran\left(T\right)$.
    \item The coimage of $T$ is $\operatorname{coim}\left(T\right) = X/\ker\left(T\right)$.
  \end{itemize}
  Note that, by an analogous version of the first isomorphism theorem, we have $\operatorname{coim}\left(T\right) \cong \Ran\left(T\right)$.\newline

  Additionally, $T$ is injective if and only if $\ker\left(T\right) = \set{0}$, and $T$ is surjective if and only if $\operatorname{coker}\left(T\right) = \set{0}$.\newline

  We write $\Dim\left(\Ran\left(T\right)\right) = \operatorname{rank}\left(T\right)$, and $\Dim\left(\ker\left(T\right)\right) = \operatorname{null}\left(T\right)$.
\end{definition}
\begin{definition}
  Let $T\in \mathcal{L}\left(X\right)$, and let $\lambda\in \C$. The eigenspace for $\lambda$ is the subspace $E_{\lambda}\left(T\right) = \ker\left(T - \lambda I\right)$.\newline

  If $E_{\lambda}\left(T\right) \neq \set{0}$, then $\lambda$ is called an eigenvalue for $T$ The nonzero vectors in $E_{\lambda}\left(T\right)$ are called eigenvectors for $T$.\newline

  The set
  \begin{align*}
    \sigma_p\left(T\right) &= \set{\lambda | \lambda\text{ is an eigenvalue for }T}
  \end{align*}
  is known as the point spectrum of $T$.
\end{definition}
\begin{theorem}[Rank--Nullity]
  If $T\colon X\rightarrow Y$ is a linear map between vector spaces, then $\operatorname{rank}\left(T\right) + \operatorname{null}\left(T\right) = \Dim\left(Y\right)$.
\end{theorem}
The separation properties of linear functionals are used heavily in proofs of various results in analysis. The following theorem is refined via the Hahn--Banach theorems (see \ref{thm:hb_separation}), as in infinite dimensions, continuity becomes an issue that analysts are forced to deal with. However, we start with the algebraic case.
\begin{proposition}
  Let $X$ be a vector space. If $0\neq x_0\in X$, then there is a $\varphi\in X'$ such that $\varphi\left(x_0\right) \neq 0$.
\end{proposition}
Geometrically, linear functionals are tied to hyperplanes within vector spaces.
\begin{proposition}
  Let $X$ be a vector space with $\Dim\left(X\right) \geq 2$, and let $H\subseteq X$ be a subspace. The following are equivalent:
  \begin{enumerate}[(i)]
    \item $H=\ker\left(\varphi\right)$ for some nonzero $\varphi\in X'$;
    \item $H\subseteq X$ is a maximal \textit{proper} subspace;
    \item $\Dim\left(X/H\right) = 1$ (i.e., $H$ has codimension $1$).
  \end{enumerate}
  A subspace that satisfies any of these equivalent properties is called a hyperplane.\newline

  If $U = H + x_0$ for some fixed $x_0$, then $U$ is known as an affine hyperplane.
\end{proposition}
\section{Algebras}%
In our definition of vector spaces, we stated that they are akin to abelian groups, equipped with an operation of scalar multiplication. We may extend the analogy towards ``rings'' that include scalar multiplication, which are known as algebras.
\subsection{The Structure of Algebras}%
\begin{definition}
  Let $A$ be a $\C$-vector space. We say $A$ is an algebra if $A$ admits a multiplication, $\left(a,b\right)\mapsto a\cdot b$, that satisfies, for all $a,b,c\in A$ and $\alpha \in \C$,
  \begin{itemize}
    \item $\left(a\cdot b\right)\cdot c = a\cdot \left(b\cdot c\right)$;
    \item $a\cdot \left(b+c\right) = a\cdot b + a\cdot c$;
    \item $\left(a+b\right)\cdot c = a\cdot c + b\cdot c$;
    \item $\left(\alpha a\right)\cdot b = \alpha \left(a\cdot b\right) = a \cdot \left(\alpha b\right)$.
  \end{itemize}
  If $A$, considered as a ring, is also unital, then we say $A$ is a unital algebra. If $a\cdot b = b\cdot a$, then we say $A$ is commutative.\newline

  If $A$ also admits a unary operation $\ast\colon A\rightarrow A$ that satisfies, for all $a,b\in A$ and $\alpha \in \C$,
  \begin{itemize}
    \item $a^{\ast\ast} = a$;
    \item $\left(ab\right)^{\ast} = b^{\ast}a^{\ast}$;
    \item $\left(\alpha a + b\right)^{\ast} = \overline{\alpha}a^{\ast} + b^{\ast}$,
  \end{itemize}
  then we say $A$ is a $\ast$-algebra.
\end{definition}
\subsection{Algebra Homomorphisms}%

% Here, will discuss group theory, results like FIT/normal subgroups and the fundamental theorem for finite/finitely generated abelian groups, ring theory (homomorphisms, ideals and maximal ideals, quotient rings, etc.), abstract linear algebra (vector spaces, bases, linear functionals), algebras (definitions, homomorphisms, unitizations)
\chapter{Point-Set Topology}\label{ch:point_set_topology}
We will need a bit of background in point-set topology in order to satisfactorily understand the functional analysis behind the results in Chapters 3, 4, and 5.
\section{Axioms of Set Theory}%
In order to garner sufficient understanding of point-set topology, we need to be able to comprehend some of the essential axioms behind the objects known as ``sets.'' This is where the axioms of set theory come into play.
\begin{definition}[Zermelo--Fraenkel Axioms]
  In Zermelo--Fraenkel set theory, all objects are sets. In order to maintain convention with the way the rest of this section will refer to sets, all sets will be referred to by capital letters, and all elements of sets by lowercase letters.
  \begin{itemize}
    \item Axiom of Existence: $\exists A\left(A = A\right)$. This axiom guarantees a nonempty universe.
    \item Axiom of Extensionality: $\forall x\left(x\in A \Leftrightarrow x\in B\right)\Rightarrow A = B$. This axiom states that if two sets share the same members, then the sets are equal.
    \item Axiom Schema of Comprehension: $\exists B\:\forall x\left(x\in B\Leftrightarrow x\in A \wedge\varphi(x)\right)$. This axiom states that for any formula $\varphi(x)$, where $x$ is a free variable, there is a set $B$ such that the members of $B$ are the members of $A$ for which $\varphi$ holds.
    \item Pairing Axiom: $\forall A\:\forall B\:\exists C\left(\left(A\in Z\right)\wedge \left(B\in Z\right)\right)$. This axiom states that for any sets $A$ and $B$, there is a set $C = \set{A,B}$ that contains the sets $A$ and $B$ as elements.
    \item Power Set Axiom: $\forall A\:\exists P(A)\:\forall B\left(B\in P(A) \Leftrightarrow B\subseteq A\right)$. We use the shorthand $B\subseteq A$ to write the statement $\forall x \left(x\in B\Rightarrow x\in A\right)$. This axiom states that for any set $A$ there exists a set $P(A)$ such that any element of $P(A)$ is a subset of $A$, and any subset of $A$ is an element of $P(A)$.
    \item Union Axiom: $\forall \mathcal{A}\:\exists A\:\forall Y\:\forall x\left(\left(x\in Y\wedge Y\in \mathcal{A}\right)\Rightarrow x\in A\right)$. This axiom states that for any collection $\mathcal{A}$, there is a set $A $, denoted $ \bigcup \mathcal{A}$, that contains all the elements of all the sets in the collection $\mathcal{A}$.
    \item Axiom of Infinity: $\exists A\left(\emptyset\in A\wedge \forall x\left(x\in A\Rightarrow x\cup\set{x}\in A\right)\right)$. This axiom states that there is a set, $A$, such that the empty set is in $A$ and, for any element $x$, if $x\in A$, then so too is the successor, $x\cup \set{x}$.
    \item Axiom of regularity: $\forall X\left(X\neq\emptyset \Rightarrow\exists Y\left(Y\in X\wedge Y\cap X = \emptyset\right)\right)$. This axiom states that any nonempty set $X$ contains a set $Y$ such that $Y$ and $X$ are disjoint. As a consequence, any chain of sets descending in membership must terminate.
    \item Axiom Schema of Replacement: $\forall A\:\exists B\:\forall v\left(v\in B\Rightarrow \exists u\left(u\in A\wedge \psi\left(u,v\right)\right)\right)$. The axiom schema of replacement says that for a function-like formula (a formula such that $\psi\left(u,v\right)\wedge \psi\left(u,w\right) \Rightarrow v=w$) $\psi\left(u,v\right)$, there is a set $A$ consisting of exactly those sets/elements $v\in B$ that correspond to $u\in A$.
  \end{itemize}
\end{definition}
The final axiom, the Axiom of Choice, is special, and as a result, we state it separately, for we will be using some of its consequences in the future sections. The following is one way of interpreting the axiom of choice.
\begin{definition}[Axiom of Choice]
  Let $\set{S_i}_{i\in I}$ be an indexed collection of nonempty sets. Then, there exists an indexed set $\set{x_i}_{i\in I}$ such that $x_i\in S_i$ for each $I$.\newline

  Equivalently, if $\set{S_i}_{i\in I}$ is an indexed collection of nonempty sets, then there is some choice function
  \begin{align*}
    f\in \prod_{i\in I}S_i.
  \end{align*}
\end{definition}
On its own, this formulation of the Axiom of Choice is not particularly useful. However, there is a statement of the Axiom of Choice which is just as useful.
\begin{definition}[Preorders, Partial Orders, Total Orders, and Well-Orders]
Let $X$ be a set, and $\preceq $ be a relation on $X$. We say a relation is a preorder if it is reflexive and transitive:
\begin{itemize}
  \item $a\preceq a$
  \item $a\preceq b \wedge b\leq c\Rightarrow a\preceq c$.
\end{itemize}
We say $X$ is a directed set if, for any $a,b\in X$, there is $c\in X$ such that $a\preceq c$ and $b\preceq c$.\newline

If $\preceq$ is also antisymmetric --- that is, $a\preceq b\wedge b\preceq a \Rightarrow a = b$ --- then, we say $\preceq$ is a partial order.\newline

We say $m\in X$ is a maximal element if, for any $x\in X$ with $m\preceq x$, $m = x$.\newline

If $X$ is partially ordered by $\preceq$ and, for any two elements $a,b\in X$, either $a\preceq b$ or $b\preceq a$, then we say $\preceq$ is a total order on $X$.\newline

If $X$ is a totally ordered set that has the property that, for any nonempty $A\subseteq X$, there is some $x\in A$ such that for any $y\in A$, $x\prec y$ for all $y \in A$ with $y\neq x$, then we say $\preceq$ is a well-order on $X$.
\end{definition}
\begin{example}
  \begin{itemize}
    \item The set $\N$ with the usual ordering is a well-ordered set.
    \item If $A$ is a set, then $P(A)$ with the containment ordering, $A\preceq B$ if $A\supseteq B$, is a partially ordered set.
    \item Similarly, if $A$ is a set, then $P(A)$ with the inclusion ordering, $A\preceq B$ if $A\subseteq B$, is a partially ordered set.
    \item A collection of functions $\set{\varphi_{i}\colon Z_i\rightarrow Y}_{i\in I}$ ordered by $\varphi_{i}\preceq \varphi_j$ if $Z_i\subseteq Z_j$ and $\varphi_{j}|_{Z_i} = \varphi_i$, is a partially ordered set. This is often known as the extension ordering.
  \end{itemize}
\end{example}

We can state an equivalent formulation of the Axiom of Choice as follows.
\begin{theorem}[Zorn's Lemma]
  If $\left(X,\preceq\right)$ is a partially ordered set with the property that for all $C\subseteq X$ with $C$ totally ordered, $C$ has an upper bound, then $X$ has a maximal element.
\end{theorem}
There are many proofs of both Zorn's Lemma from the Axiom of Choice and the Axiom of Choice from Zorn's Lemma. However, we will mostly be using it for the purposes of proving other theorems. The following results can be proven using Zorn's Lemma.
\begin{example}
  \begin{itemize}
    \item Every $\F$-vector space $V$ has a basis $B\subseteq V$ such that the set of all finite linear combinations of elements of $B$ over $\F$ is $V$.
    \item If $\varphi$ is a continuous linear functional defined on a subspace $W\subseteq V$, there is an extension $\Phi$ such that $\Phi|_{W} = \varphi$. This is one of the Hahn--Banach theorems. %See: Hahn--Banach Theorems
    \item The arbitrary product of compact spaces is compact. This is known as Tychonoff's Theorem. %See Tychonoff's Theorem.
  \end{itemize}
\end{example}
\section{Metric Spaces}%
Building upon the basics of set theory, we move towards understanding metric spaces.
\subsection{Basics of Metric Spaces}%
\begin{definition}[Metrics]
  Let $X$ be a set. A distance metric is a function
  \begin{align*}
    d\colon X\times X\rightarrow [0,\infty)
  \end{align*}
  such that the following three properties are satisfied:
  \begin{itemize}
    \item if $x,y\in X$ and $d\left(x,y\right) = 0$, then $x = y$;
    \item $d\left(x,y\right) = d\left(y,x\right)$ for all $x,y\in X$;
    \item $d\left(x,z\right) \leq d\left(x,y\right) + d\left(y,z\right)$ for all $x,y,z\in X$.
  \end{itemize}
  A function that satisfies the latter two properties is called a semimetric.\newline

  Two metrics $d$ and $\rho$ on $X$ are equivalent if there exist constants $c_1,c_2\geq 0$ such that
  \begin{align*}
    d\left(x,y\right) &\leq c_1 \rho\left(x,y\right)\\
    \rho\left(x,y\right) &\leq c_2 d\left(x,y\right)
  \end{align*}
  for all $x,y\in X$.\newline

  A metric space is a pair $\left(X,d\right)$, where $d$ is a metric.
\end{definition}
\begin{example}[Some Distance Metrics]
  \begin{itemize}
    \item The discrete metric on any nonempty set is given by
      \begin{align*}
        d\left(x,y\right) & \begin{cases}
          1 & x\neq y\\
          0 & x = y
        \end{cases}
      \end{align*}
    \item The Euclidean metric between $\left(x_1,\dots,x_n\right)$ and $\left(y_1,\dots,y_n\right)$ in $\R^n$ is
      \begin{align*}
        d_{2}\left(x,y\right) &= \left(\sum_{j=1}^{n}\left\vert y_j-x_j \right\vert^2\right)^{1/2}.
      \end{align*}
    \item Other metrics on $\R^n$ include
      \begin{align*}
        d_1\left(x,y\right) &= \sum_{j=1}^{n}\left\vert y_j-x_j \right\vert\\
        d_{\infty}\left(x,y\right) &= \max_{j=1}^{n}\left\vert y_j - x_j \right\vert.
      \end{align*}
      All of $d_1,d_2,d_{\infty}$ are equivalent metrics.
    \item The Hamming distance between two strings of bits is
      \begin{align*}
        d_{H}\colon \set{0,1}^{n}\times \set{0,1}^{n}\rightarrow [0,\infty)\\
        d_{H}\left(\left(x_{j}\right)_{j=1}^{n},\left(y_j\right)_{j=1}^{n}\right) &= \left\vert \set{j\mid x_j\neq y_j} \right\vert.
      \end{align*}
    \item The set $C\left([0,1],\R\right)$ consisting of continuous real-valued functions from $[0,1]$ to $\R$ can be equipped with
      \begin{align*}
        d_u\left(f,g\right) &= \sup_{t\in [0,1]}\left\vert f(t) - g(t) \right\vert,
      \end{align*}
      which is the uniform metric, or
      \begin{align*}
        d_{1}\left(f,g\right) &= \int_{0}^{1} \left\vert f(t)-g(t) \right\vert\:dt.
      \end{align*}
    \item All subsets of a metric space $X$ equipped with the same metric is also a metric space.
    \item If $\rho$ is a metric on $X$, then we can create a distance metric
      \begin{align*}
        d\left(x,y\right) &= \frac{\rho\left(x,y\right)}{1 + \rho\left(x,y\right)}
      \end{align*}
      that is bounded on $[0,1]$.
    \item If $d_1,\dots,d_n$ are metrics on $X$ and $c_1,\dots,c_n > 0$ are constants, then
      \begin{align*}
        d\left(x,y\right) &= \sum_{k=1}^{n}c_kd_k\left(x,y\right)
      \end{align*}
      defines a metric on $X$.
    \item If $\left(\rho_k\right)_k$ is a family of separating semimetrics for $X$ --- i.e., for $x,y\in X$ distinct, there is some $\rho_{j}$ such that $\rho_j\left(x,y\right) \neq 0$ --- then, we can obtain bounded semimetrics by taking
      \begin{align*}
        d_k\left(x,y\right) &= \frac{\rho_k\left(x,y\right)}{1 + \rho_k\left(x,y\right)}
      \end{align*}
      for each $k$. From each $d_k$, we define
      \begin{align*}
        d\left(x,y\right) &= \sum_{k=1}^{n}2^{-k}d_k\left(x,y\right),
      \end{align*}
      which is a metric on $X$.
    \item If $\left(X_k,\rho_k\right)_{k\geq 1}$ is a sequence of metric spaces, then we can form the product space
      \begin{align*}
        X &= \prod_{k\geq 1}X_{k}
      \end{align*}
      with the metric
      \begin{align*}
        D\left(f,g\right) &= \sum_{k\geq 1}d_k\left(f(k),g(k)\right).
      \end{align*}
      Here, $d_k = \frac{\rho_k}{1 + \rho_k}$ is the corresponding bounded metric to $\rho_k$.
  \end{itemize}
\end{example}
\begin{definition}[Open and Closed Sets]
  Let $\left(X,d\right)$ be a metric space.
  \begin{enumerate}[(1)]
    \item For $x\in X$ and $\delta > 0$, we define
      \begin{enumerate}[(a)]
        \item the open ball at $x$ with radius $\delta > 0$
          \begin{align*}
            U\left(x,\delta\right) &= \set{y\in X\mid d\left(y,x\right) < \delta};
          \end{align*}
        \item the closed ball centered at $x$ with radius $\delta > 0$
          \begin{align*}
            B\left(x,\delta\right) &= \set{y\in X\mid d\left(y,x\right)\leq \delta};
          \end{align*}
        \item the sphere centered at $x$ with radius $\delta > 0$
          \begin{align*}
            S\left(x,\delta\right) &= \set{y\in X\mid d\left(y,x\right) = \delta}.
          \end{align*}
      \end{enumerate}
    \item A set $V\subseteq X$ is open if, for all $x\in V$, there is $\delta > 0$ such that $U\left(x,\delta\right)\subseteq V$.\newline

      A subset $C\subseteq X$ is closed if $C^{c}$ is open.
    \item If $x\in V$ and $V\subseteq X$ is open, then we say $V$ is an open neighborhood of $x$. A neighborhood of $x$ is any subset $N\subseteq X$ such that $N$ contains an open neighborhood of $x$.
    \item If $A\subseteq X$ is any subset, the interior of $A$ is
      \begin{align*}
        A^{\circ} &:= \bigcup\set{V\mid V\text{ is open, }V\subseteq A},
      \end{align*}
      the closure of $A$ is
      \begin{align*}
        \overline{A} &= \bigcap\set{C\mid C\text{ is closed, }A\subseteq C},
      \end{align*}
      and the boundary of $A$ is
      \begin{align*}
        \partial A &= \overline{A} \setminus A^{\circ}.
      \end{align*}
  \end{enumerate}
\end{definition}
We can now talk about the topology of the metric space.
\begin{fact}
  Let $\left(X,d\right)$ be a metric space, and let
  \begin{align*}
    \mathcal{U} = \set{V\mid V\subseteq X\text{ open}}.
  \end{align*}
  Then, the following are true.
  \begin{itemize}
    \item $\emptyset\in \mathcal{U},X\in \mathcal{U}$.
    \item If $\set{V_{i}}_{i\in I}$ is a family of open sets, then $\bigcup_{i\in I}V_i\in \mathcal{U}$.
    \item If $\set{V_i}_{i=1}^{n}$ is a finite collection of open sets, then $\bigcap_{i=1}^{n}V_i \in \mathcal{U}$.
  \end{itemize}
\end{fact}

\begin{definition}
  Let $\left(X,d\right)$ be a metric space. Suppose $A\subseteq X$ is a nonempty subset.
  \begin{enumerate}[(1)]
    \item The distance from a point $x\in X$ to the set $A$ is defined by
      \begin{align*}
        \dist_{A}\left(x\right) &= \inf_{a\in A}d\left(x,a\right).
      \end{align*}
    \item The diameter of $A$ is defined by
      \begin{align*}
        \diam\left(A\right) &= \sup_{x,y\in A}d\left(x,y\right).
      \end{align*}
    \item If $\diam(A) < \infty$, then we say $A$ is bounded.
    \item If, for every $\delta > 0$, there is a finite subset $F_{\delta}\subseteq X$ such that
      \begin{align*}
        A\subseteq \bigcup_{x\in F_{\delta}}U\left(x,\delta\right).
      \end{align*}
    \item For $A,B\subseteq X$, we define the Hausdorff distance between $A$ and $B$ to be
      \begin{align*}
        d_{H}\left(A,B\right) &= \max\set{\sup_{x\in A}\dist_{B}\left(x\right),\sup_{y\in B}\dist_{A}\left(y\right)}.
      \end{align*}
  \end{enumerate}
\end{definition}
\begin{example}
  Let $\Omega$ be a nonempty set, and $\left(X,d\right)$ be a metric space. A function $f\colon \Omega\rightarrow X$ is said to be bounded if $\diam\left(\ran(f)\right) < \infty$.\newline

  The collection $\operatorname{Bd}\left(\Omega,X\right)$ denotes all bounded functions with domain $\Omega$ and codomain $X$.\newline

  On $ \operatorname{Bd}\left(\Omega,X\right)$, we define the uniform metric by
  \begin{align*}
    D_{u}\left(f,g\right) &= \sup_{x\in\Omega}d\left(f(x),g(x)\right).
  \end{align*}
\end{example}
\subsection{Convergence and Continuity in Metric Spaces}%
\begin{definition}
  Let $\left(X,d\right)$ be a metric space.
  \begin{enumerate}[(1)]
    \item A sequence in $X$ is a map $x\colon \N\rightarrow X$, which we call $\left(x_{n}\right)_{n}$ or $\left(x_{n}\right)_{n\geq 1}$.
    \item A natural sequence is a strictly increasing sequence of natural numbers $\left(n_{k}\right)_{k\geq 1}$ with $n_{k}\geq k$ and $n_{k} < n_{k+1}$.
    \item If $\left(n_k\right)_{k}$ is a natural sequence, the sequence $\left(x_{n_k}\right)_{k}$ is called a subsequence of $\left(x_{n}\right)_n$.
    \item We say $\left(x_n\right)_n\rightarrow x$ if $d\left(x_n,x\right)_{n} \xrightarrow{n\rightarrow\infty} 0$. We say $x$ is the limit of $\left(x_n\right)_n$.
  \end{enumerate}
\end{definition}
\begin{example}
  \begin{itemize}
    \item If $\Omega$ is a nonempty set, and $\left(X,d\right)$ is a metric space, the sequence of functions $f_n\colon \Omega\rightarrow X$ is said to converge pointwise to $f\colon \Omega\rightarrow X$ if
      \begin{align*}
        f_n\left(x\right)\xrightarrow{n\rightarrow\infty}f(x)
      \end{align*}
      for each $x\in \Omega$.
    \item If $\left(f_n\right)_n\in \operatorname{Bd}\left(\Omega,X\right)$ is a sequence, we say $\left(f_n\right)_n\rightarrow f$ converges uniformly if
      \begin{align*}
        D_u\left(f_n,f\right)\xrightarrow{n\rightarrow\infty}0,
      \end{align*}
      or, equivalently,
      \begin{align*}
        \sup_{x\in\Omega}d\left(f_n(x),f(x)\right)\xrightarrow{n\rightarrow\infty}0.
      \end{align*}
  \end{itemize}
\end{example}
\begin{definition}[Sequential Criteria for Closure]
  If $\left(X,d\right)$ is a metric space, and $E\subseteq X$ is nonempty, then $E$ is closed if and only if, for all $\left(x_n\right)_n\rightarrow x$ with $x_n\in E$, $x\in E$.\newline

  If $E\subseteq X$ is any nonempty set, then $\overline{E}$ is precisely the set of all $x\in X$ such that $\left(x_n\right)_n\rightarrow x$ for some $\left(x_n\right)_n\subseteq E$.
\end{definition}
\begin{definition}[Completeness]
  Let $\left(X,d\right)$ be a metric space.
  \begin{itemize}
    \item If $\left(x_n\right)_n$ is a sequence in $X$ such that for all $\ve > 0$, there is $N\in \N$ such that for all $m,n\geq N$, $d\left(x_m,x_n\right) < \ve$, then we say the sequence is called Cauchy.
    \item If, for any $\left(x_n\right)_n$ Cauchy, $\left(x_n\right)_n\rightarrow x$ in $X$, then we say $X$ is complete.
    \item If $\left(X,d\right)$ is complete, then for any $A\subseteq X$ closed, $A$ is also complete.
    \item If $A\subseteq X$ is complete as a metric space, then $A$ is closed.
  \end{itemize}
\end{definition}
\begin{example}
  The metric space $\Q$ with the metric inherited from $\R$ is not complete. For instance, there is a sequence of rational numbers $\left(2,2.7,2.71,2.718,\dots\right)$ converging to $e$, but $e\notin \Q$.\newline

  The space $\operatorname{Bd}\left(\Omega,X\right)$ is complete if $X$ is complete.
\end{example}
\begin{definition}[Continuity]
  \begin{itemize}
    \item Let $\left(X,d\right)$ and $\left(Y,\rho\right)$ be metric spaces, and let $f\colon X\rightarrow Y$ be a function. We say $f$ is continuous at $x$ if, for every $\ve > 0$, there is $\delta > 0$ such that $z\in U\left(x,\delta\right)\Rightarrow \rho\left(f(x),f(z)\right) < \ve$.
    \item If $f$ is continuous at every point in $X$, then we say $f$ is continuous.
    \item If $f$ is bijective, continuous, and $f^{-1}$ is continuous, then we say $f$ is a homeomorphism.
    \item We say $f$ is uniformly continuous on $X$ if, for any $\ve > 0$, there is $\delta > 0$ such that for any $y,z\in X$, $d\left(y,z\right) < \delta \Rightarrow \rho\left(f(y),f(z)\right) < \ve$.
    \item We say $f$ is Lipschitz if there exists $C > 0$ such that $d\left(x,y\right) \leq Cd\left(f(x),f(y)\right)$ for all $x,y\in X$.
    \item We say $f$ is an isometry if $d\left(x,y\right) = d\left(f(x),f(y)\right)$ for all $x,y\in X$.
  \end{itemize}
\end{definition}
\begin{fact}
  Let $f\colon X\rightarrow Y$ be a map between metric spaces. The following are equivalent:
  \begin{enumerate}[(i)]
    \item $f$ is continuous;
    \item if $V\subseteq Y$ is open, then $f^{-1}\left(V\right)\subseteq X$ is open;
    \item if $\left(x_n\right)_n\rightarrow x$ in $X$, then $\left(f\left(x_n\right)\right)_n\rightarrow f(x)$ in $Y$.
  \end{enumerate}
\end{fact}
\begin{fact}
  If $M$ and $N$ are metric spaces with $N$ complete, and $A\subseteq M$ is dense, then if $f\colon A\rightarrow N$ is uniformly continuous, then there is a unique uniformly continuous map $\tilde{f}\colon M\rightarrow N$.
\end{fact}
\begin{definition}
  Let $\left(X,d\right)$ and $\left(Y,\rho\right)$ be metric spaces.
  \begin{enumerate}[(1)]
    \item We say $X$ and $Y$ are homeomorphic if there is a homeomorphism $f\colon X\rightarrow Y$.
    \item We say $X$ and $Y$ are uniformly isomorphic if there is a uniformly continuous bijection $f\colon X\rightarrow Y$ with $f^{-1}$ uniformly continuous. Such an $f$ is called a metric space uniformism.
    \item We say $X$ and $Y$ are isometrically isomorphic if there is a bijective isometry $f\colon M\rightarrow N$.
  \end{enumerate}
\end{definition}
\begin{fact}
  If $X$ and $Y$ are uniformly isomorphic metric spaces with $X$ complete, then so too is $Y$.\newline

  If $d$ and $\rho$ are equivalent metrics on a set $X$, then the identity map
  \begin{align*}
    \id_{X}:\left(X,\rho\right)\rightarrow \left(X,d\right)
  \end{align*}
  is a metric space uniformism.
\end{fact}
\section{Topological Spaces}%
We can now move from metric spaces to the more general setting of topological spaces. This will enable us to understand certain properties (like openness, continuity, etc.) separate from the metric structure (or lack thereof) that a certain set is endowed.
\subsection{Definitions}%
\begin{definition}
  Let $X$ be a nonempty set. A topology on $X$ is a family of subsets $\tau$ satisfying
  \begin{enumerate}[(1)]
    \item $\emptyset\in \tau,X\in \tau$;
    \item if $\set{V_i}_{i\in I}\subseteq \tau$, then $\bigcup_{i\in I}V_i\in \tau$;
    \item if $\set{V_i}_{i=1}^{n}\subseteq \tau$, then $\bigcap_{i=1}^{n}V_i \in \tau$.
  \end{enumerate}
  If $\tau$ is a topology on $X$, then $\left(X,\tau\right)$ is called a topological space. We call members of $\tau$ open sets.\newline

  If $C\subseteq X$ and $C^{c}\in \tau$, then $C$ is called.\newline

  If $E$ is closed and open, it is called clopen.\newline

  A countable union of closed sets is called an $F_{\sigma}$ set, and a countable intersection of open sets is called a $G_{\delta}$ set.
\end{definition}
\begin{definition}
  If $X$ is a nonempty set, then the definition $\tau = P(X)$ is known as the discrete topology.\newline

  If $X$ is a nonempty set, and $\tau = \set{X,\emptyset}$, then we call $\tau$ the indiscrete topology.
\end{definition}
\begin{definition}
  Let $X$ be a nonempty set. Suppose $\tau_1,\tau_2\subseteq P(X)$ are two topologies on $X$. If $\tau_1\subseteq \tau_2$, then we say $\tau_1$ is weaker (or coarser) than $\tau_2$. We say $\tau_2$ is stronger (or finer) than $\tau_1$.
\end{definition}
\begin{definition}
  Let $X$ be a nonempty set, and suppose $\mathcal{E}\subseteq P(X)$ is a family of subsets. We define the topology on $X$ generated by $\mathcal{E}$ to be
  \begin{align*}
    \tau\left(\mathcal{E}\right) &= \bigcap\set{\tau | \tau\text{ is a topology on $X$, }\mathcal{E}\subseteq \tau}.
  \end{align*}
  In other words, $\tau\left(\mathcal{E}\right)$ is the weakest topology that contains the family $\mathcal{E}$.
\end{definition}
\begin{definition}
Let $\left(X,\tau\right)$ be a topological space. If $Y\subseteq X$ is a subset, then the subspace topology on $Y$ is defined by
\begin{align*}
  \tau_{Y} &= \set{V\cap Y | V\in \tau}.
\end{align*}

\end{definition}
\begin{definition}
  Let $\left(X,\tau\right)$ be a topological space, and let $A\subseteq X$ be a subset.
  \begin{enumerate}[(1)]
    \item The interior of $A$ is the open set $A^{\circ} = \bigcup\set{V | V\in\tau,~V\subseteq A}$.
    \item The closure of $A$ is the closed set $\overline{A} = \bigcap\set{C| C\text{ closed, }A\subseteq C}$.
    \item We say $A$ is dense if $\overline{A} = X$.
    \item We say $A$ is nowhere dense if $\left(\overline{A}\right)^{\circ} = \emptyset$.
  \end{enumerate}
  If $X$ admits a countable dense subset, then we say $X$ is separable.\newline

  If $X$ is the countable union of nowhere dense subsets, then we say $X$ is meager.
\end{definition}
\begin{remark}
  A set $A$ is dense if and only if, for any $U\in \tau$ with $U\neq \emptyset$, it is the case that $A\cap U \neq \emptyset$.
\end{remark}
\begin{fact}
  If $\left(M,d\right)$ is a separable metric space, and $E\subseteq M$ is a subset, then $E$ with the subspace topology is also separable.
\end{fact}
\begin{definition}
  Let $\left(X,\tau\right)$ be a topological space.
  \begin{itemize}
    \item An open neighborhood of $x_0$ is an open set $V\in \tau$ with $x_0\in V$. We write
      \begin{align*}
        \mathcal{O}_{x_0} &= \set{V | V\in \tau,x_0\in V}
      \end{align*}
      to denote the family of all open neighborhoods of $x_0$.
    \item If $N\subseteq X$ is a subset with $x_0\in V\subseteq N$, where $V\in \mathcal{O}_{x_0}$, then we say $N$ is a neighborhood of $x_0$. We write $\mathcal{N}_{x_0}$ to be the collection of neighborhoods of $x_0$.
    \item A neighborhood base for $\tau$ at $x_0$ is a family $\mathcal{O}\subseteq \mathcal{O}_{x_0}$ with such that for all $U\in \mathcal{O}_{x_0}$, there is $V\in \mathcal{O}$ with $V \subseteq U$.
    \item We say $\left(X,\tau\right)$ is first countable if every $x\in X$ admits a countable neighborhood base.
    \item A base for $\tau$ is a family $\mathcal{B}\subseteq \tau$ that contains a neighborhood base for $\tau$ at $x_0$ For each $x_0\in X$.
    \item We say $\left(X,\tau\right)$ is second countable if it admits a countable base.
  \end{itemize}
\end{definition}
\begin{fact}
  If $\mathcal{B}$ is a base for $\tau$, then every $U\in \tau$ can be written as a union $U= \bigcup_{i\in I}B_i$, where $B_i\in \mathcal{B}$.
\end{fact}
\begin{fact}
  All metric spaces are first-countable, with a neighborhood base of
  \begin{align*}
    \mathcal{O}_{x_0} &= \set{U\left(x_0,1/n\right)|n\in \N}
  \end{align*}
  for each $x_0\in X$.
\end{fact}
\begin{fact}
  A metric space $\left(X,d\right)$ is second countable if and only if it is separable.
\end{fact}
\begin{fact}
  If $X$ is a topological space, and $x_0\in X$ has a countable neighborhood base, then there is a neighborhood base $\left(V_n\right)_{n\geq 1}$ with $V_1\supseteq V_2\supseteq \cdots$.
\end{fact}
\subsection{Continuity in Topological Spaces}%
\begin{definition}
  Let $\left(X,\tau\right)$ and $\left(Y,\sigma\right)$ be topological spaces, and let $f\colon X\rightarrow Y$ be a map.
  \begin{enumerate}[(1)]
    \item We say $f$ is continuous at $x_0\in X$ if, for every $U\in \mathcal{O}_{f\left(x_0\right)}$, there is $V\in \mathcal{O}_{x}$ with $f\left(V\right)\subseteq U$.
    \item We say $f$ is continuous if $f$ is continuous at every point in $X$.
    \item We say $f$ is a homeomorphism if $f$ is a continuous bijection with a continuous inverse.
    \item We say $f$ is an open map if $U\in \tau$ implies $f\left(U\right)\in \sigma$. Similarly, we say $f$ is a closed map if $C\subseteq X$ closed implies $f\left(C\right)\subseteq Y$ is closed.
    \item We say $f$ is a quotient map if $f$ is surjective with $V\subseteq Y$ open if and only if $f^{-1}\left(V\right)\subseteq X$ open.
    \item We say $f$ is an embedding if $f\colon X\rightarrow \Ran\left(f\right)$ is a homeomorphism, where $\Ran\left(f\right)$ is endowed with the subspace topology.
    \item We write $C\left(X,Y\right)$ to be the continuous functions from $X$ to $Y$. If $Y = \C$ with the regular topology, then we write $C\left(X\right)$.
  \end{enumerate}
\end{definition}
\begin{fact}
  A function $f\colon X\rightarrow Y$ is continuous if and only if $f^{-1}\left(U\right)\subseteq X$ is open for every open $U\subseteq Y$. Equivalently, $f$ is continuous if and only if $f^{-1}\left(C\right)\subseteq X$ is closed for every closed $C\subseteq Y$.
\end{fact}
\begin{definition}[Separation Axioms]
Let $\left(X,\tau\right)$ be a topological space.
\begin{itemize}
  \item We say $X$ is T1 if $\set{x}$ is closed for every $x\in X$.
  \item We say $X$ is T2 (or Hausdorff) if, for every $x,y\in X$ with $x\neq y$, there are $U,V\in \tau$ with $x\in U$, $y\in V$, and $U\cap V = \emptyset$.
  \item We say $X$ is T3 if, for every $x\in X$ and $B\subseteq X$ closed with $x\notin B$, there are $U,V\in \tau$ with $x\in U$, $B\subseteq V$, and $U\cap V = \emptyset$. If $X$ is T1 and T3, we say $X$ is regular.
  \item We say $X$ is T3.5 if, for every $x_0\in X$ and closed $B\subseteq X$ with $x_0\notin B$, there is a continuous function $f\colon X\rightarrow [0,1]$ with $f\left(x_0\right) = 0$ and $f\left(B\right) = 1$. If $X$ is T1 and T3.5, we say $X$ is completely regular.
  \item We say $X$ is T4 if, for every pair of closed subsets $A,B\subseteq X$ with $A\cap B = \emptyset$, there are $U,V\in \tau$ with $A\subseteq U$, $B\subseteq V$, and $U\cap V = \emptyset$. If $X$ is T1 and T4, then we say $X$ is normal.
\end{itemize}
\end{definition}
Just as we defined completely regular spaces through the existence of certain continuous functions that act to separate points, we can completely classify normality through a separating family of continuous functions.
\begin{theorem}[Urysohn's Lemma]
  Let $\left(X,\tau\right)$ be a topological space. It is the case that $X$ is normal if and only if for every pair of disjoint closed subsets $A,B\subseteq X$, there is a continuous function $f\colon X\rightarrow [0,1]$ with $f\left(A\right) = 0$ and $f\left(B\right) = 1$.
\end{theorem}
\begin{remark}
  Metric spaces are an example of normal spaces.
\end{remark}
\subsection{Initial and Final Topologies}%
\begin{definition}
  Let $X$ be a set, and suppose $\set{\left(Y_i,\tau_i\right)}_{i\in I}$ is a family of topological spaces with corresponding maps $\set{f_i\colon X\rightarrow Y_i}_{i\in I}$. Setting
  \begin{align*}
    \ve &= \set{f_i^{-1}\left(V\right) | V_i\in \tau_i},
  \end{align*}
  and letting $\tau = \tau\left(\ve\right)$ be the topology on $X$ generated by $\ve$, we say $\tau$ is the initial topology on $X$ induced by the maps $\set{f_i}_{i\in I}$.\newline

  Specifically, $\tau$ is the weakest topology on $X$ such that each $f_i$ is continuous.
\end{definition}
\begin{definition}[Product Topology]
  Let $\set{\left(X_i,\tau_i\right)}_{i\in I}$ be a family of topological spaces. The topology on the product $\prod_{i\in I}X_i$ is defined to be the initial topology induced by the family of projection maps,
  \begin{align*}
    \pi_j\colon \prod_{i\in I}X_i\rightarrow X_j,
  \end{align*}
  defined by $\pi_j\left(\left(x_i\right)_{i\in I}\right) = x_j$.\newline
  
  For each $U\subseteq X_i$ open, we have $\pi_j^{-1}\left(U\right) = \prod_{i\in I}U_I$, where $U_i = X_i$ for $i\neq j$, and $U_j = U$. A base for this topology is the collection
  \begin{align*}
    \mathcal{B} = \set{\prod_{i\in I}U_i | U_i = X_i\text{ for all but finitely many open }U_j\subseteq X_j}.
  \end{align*}
  If we consider $X_i = X$ for all $i$, there is a bijection between $X^I \coloneq \set{f | f\colon I\rightarrow X}$, the set of all functions from $I$ to $X$, and $\prod_{i\in I}X_i$, with the map $f \mapsto \left(f\left(i\right)\right)_{i\in I}$. The product topology on $X^J$ coincides with the topology of pointwise convergence.
\end{definition}
\begin{definition}[Final Topology]
  Let $\left(X,\tau\right)$ be a topological space, $Y$ a nonempty set, and suppose $q\colon X\rightarrow Y$ is a surjection. Then, the collection
  \begin{align*}
    \tau_q &\coloneq \set{V\subseteq Y | q^{-1}\left(V\right)\in \tau}
  \end{align*}
  is what is known as the final (or quotient) topology on $Y$ produced by $q$.
\end{definition}
\subsection{Convergence in Topological Spaces}%
Given a non-first-countable space $X$ and a subset $A\subseteq X$, it is not necessarily the case that $x\in \overline{A}$ is the limit of a sequence $\left(x_n\right)_n$. However, we know that the sequential characterization of properties like closure, compactness (which will be covered in an upcoming section), and continuity is useful, so we want to generalize these ideas to non-first-countable spaces. This is where we can use nets.
\begin{definition}[Nets]
  A net is a map $A\rightarrow X$, where $\alpha \mapsto x_{\alpha}$, where $A$ is a directed set. We write nets as $\left(x_{\alpha}\right)_{\alpha}$.
\end{definition}
\begin{example}[Some Directed Sets]
  \begin{enumerate}[(1)]
    \item The natural numbers, $\N$, or the real numbers, $\R$, equipped with their usual ordering, are examples of directed sets. Every totally ordered set is directed.
    \item If $S$ is any set, the collection $F(S)$ consisting of all finite subsets of $S$ is directed by inclusion.
    \item The collection of finite partitions over a closed and bounded interval, $\mathcal{P}\left(\left[a,b\right]\right)$ is by the partition norm. If $P = \set{x_j}_{j=0}^{n}$ and $Q=\set{y_j}_{j=0}^{m}$ are partitions, we define
      \begin{align*}
        \norm{P} &= \max_{1\leq j\leq n}\left\vert x_j-x_{j-1} \right\vert\\
        \norm{Q} &= \max_{1\leq j \leq m}\left\vert y_{j} - y_{j-1} \right\vert,
      \end{align*}
      and the preorder that $P \leq Q$ if and only if $\norm{P} \geq \norm{Q}$. In other words, we say $P\leq Q$ if $Q$ is finer than $P$.\newline

      For any partitions $P$ and $Q$, their common refinement is a supremum for both --- $P\vee Q \geq P,Q$ for each partition.\footnote{This is extremely useful in defining the Riemann integral.}
    \item Let $\left(X,\tau\right)$ be a topological space, and for every $x$, we order the $\mathcal{O}_{x}$ by containment. That is, for elements $U,V\in \mathcal{O}_{x}$, we set $U\leq V$ if and only if $U\supseteq V$. This is a directed set by reverse inclusion, as we can always take $U\cap V \subseteq U,V$ (since both $U$ and $V$ contain $x$).\newline

      Similarly, the neighborhood system at $x$, $\mathcal{N}_x$, is also directed by containment.
    \item If $A$ and $B$ are directed sets, then $A\times B$ with the Cartesian ordering --- $\left(\alpha_1,\beta 1\right) \leq \left(\alpha_2,\beta 2\right)$ if and only if $\alpha_1\leq \alpha_2$ and $\beta_1 \leq \beta_2$ --- is also a directed set.
  \end{enumerate}
\end{example}
\begin{example}[Some Nets]
\begin{enumerate}[(1)]
  \item Any sequence $\left(x_k\right)_{k\in \N}$ is a net.
  \item Let $F\left(\Omega\right)$ be the set of all finite subsets of $\Omega$ directed by inclusion. Let $f\colon \Omega\rightarrow \C$ be a map. Then, we have a net $\left(s_F\right)_{F\in F\left(\Omega\right)}$ defined by
    \begin{align*}
      s_F &= \sum_{x\in F}f\left(x\right).
    \end{align*}
  \item Consider the collection of partitions $\mathcal{P}\left([a,b]\right)$ directed by the partition norm. For a bounded function $f\colon [a,b]\rightarrow \R$ and a partition $P = \set{x_j}_{j=0}^{n}$, for each $j$ we set
    \begin{align*}
      M_j\left(P\right) &= \sup_{t\in \left[x_j,x_{j-1}\right]}f(t)\\
      m_j\left(P\right) &= \inf_{t\in \left[x_j,x_{j-1}\right]}f(t).
    \end{align*}
    We obtain two nets, $U,L\colon \mathcal{P}\left([a,b]\right)$, defined by
    \begin{align*}
      U\left(P\right) &= \sum_{j=1}^{n}M_j\left(P\right)\left(x_j - x_{j-1}\right)\\
      L\left(P\right) &= \sum_{j=1}^{n}m_j\left(P\right)\left(x_j-x_{j-1}\right).
    \end{align*}
    These are known as the upper and lower Darboux sums.
\end{enumerate}
\end{example}
\begin{definition}
  Let $\left(X,\tau\right)$ be a topological space, and let $\left(x_{\alpha}\right)_{\alpha}$ be a net in $X$.
  \begin{enumerate}[(1)]
    \item For a set $U\subseteq X$, we say $\left(x_{\alpha}\right)_{\alpha}$ is eventually in $U$ if there is $\alpha_0\in A$ such that $x_{\alpha}\in U$ for all $\alpha \geq \alpha_0$.
    \item We say the net $\left(x_{\alpha}\right)_{\alpha}$ converges to $x\in X$ if, for every $U\in \mathcal{O}_{x}$, $\left(x_{\alpha}\right)_{\alpha}$ is eventually in $U$. We write $\left(x_{\alpha}\right)_{\alpha}\xrightarrow{\tau}x$, though if the topology is clear from context the $\tau$ is not written.
    \item For a given $U\subseteq X$, we say $\left(x_{\alpha}\right)_{\alpha}$ is frequently in $U$ if for every $\beta \in A$, there is $\alpha \in A$ with $\alpha \geq \beta$ and $x_{\alpha}\in U$.
    \item A point $x\in X$ is known as a cluster point of the net $\left(x_{\alpha}\right)_{\alpha}$ if for every $U\in \mathcal{O}_{x}$, $\left(x_{\alpha}\right)_{\alpha}$ is frequently in $U$. That is, for all $U\in \mathcal{O}_{x}$ and for all $\beta \in A$, there exists $\alpha \in A$ with $\alpha \geq \beta$ and $x_{\alpha}\in U$.
  \end{enumerate}
\end{definition}
\begin{fact}[Characterizations Using Nets]
  Let $\left(X,\tau\right)$ and $\left(Y,\sigma\right)$ be topological spaces, $E\subseteq X$ a subset, and $f\colon X\rightarrow Y$ a map.
  \begin{itemize}
    \item It is the case that $x\in \overline{E}$ if and only if there is a net $\left(x_{\alpha}\right)_{\alpha}$ in $E$ with $\left(x_{\alpha}\right)_{\alpha}\rightarrow x$.
    \item A map $f$ is continuous if and only if for every convergent net $\left(x_{\alpha}\right)_{\alpha}\xrightarrow{\tau}x$, we have $\left(f\left(x_{\alpha}\right)\right)_{\alpha}\xrightarrow{\sigma}f(x)$.
    \item If $X$ is given by the initial topology induced by the family of maps $\set{f_i\colon X\rightarrow \left(Y_i,\tau_i\right)}_{i\in I}$, the net $\left(x_{\alpha}\right)_{\alpha}$ converges to $x$ if and only if $\left(f_i\left(x_{\alpha}\right)\right)_{\alpha}\xrightarrow{\tau_i}f_i\left(x\right)$ in $Y_i$ for each $i\in I$.
    \item If $\set{\left(X_i,\tau_i\right)}_{i\in I}$ is a family of topological spaces, with $X = \prod_{i\in I}X_i$ equipped with the product topology, then a net $\left(x_{\alpha}\right)_{\alpha}$ in $X$ converges to $x\in X$ if and only if $\left(x_{\alpha}\left(i\right)\right)_{\alpha}\xrightarrow{\tau_i} x\left(i\right)$ in $X_i$ for each $i\in I$.
  \end{itemize}
\end{fact}

\chapter{Measure Theory and Integration}\label{ch:measure_theory}
In order to properly discuss amenability, we need a strong foundation in measure theory.

\section{Constructing Measurable Spaces}%
Fix a set $\Omega$. We let $\mathcal{A} =\set{A_{i}}_{i\in I}$ be a collection of subsets of $\Omega$.
\begin{definition}[Algebra of Subsets]
  The collection $\mathcal{A} = \set{A_i}_{i\in I}$ is known as an \textit{algebra of subsets} for $\Omega$ if
  \begin{itemize}
    \item $\emptyset,\Omega \in \mathcal{A}$;
    \item for any $A_i\in \mathcal{A}$, $A_i^{c}\in \mathcal{A}$;
    \item for any $A_i,A_j\in \mathcal{A}$, $A_i \cup A_j \in \mathcal{A}$.
  \end{itemize}
\end{definition}
We can refine the concept of an algebra of subsets to consider countable unions rather than finite unions.
\begin{definition}[$\sigma$-Algebra of Subsets]
  The collection $\mathcal{A} = \set{A_i}_{i\in I}$ is known as a \textit{$\sigma$-algebra of subsets} for $\Omega$ if
  \begin{itemize}
    \item $\emptyset,\Omega \in \mathcal{A}$;
    \item for any $A_i\in \mathcal{A}$, $A_i^{c}\in \mathcal{A}$;
    \item for any countable collection $\set{A_n}_{n\geq 1}\subseteq \mathcal{A}$, $\bigcup_{n\geq 1}A_{n} \in \mathcal{A}$.
  \end{itemize}
\end{definition}
\begin{definition}[Measurable Space]
A pair $\left(\Omega,\mathcal{A}\right)$, where $\Omega$ is a set and $A\subseteq P(\Omega)$ is a $\sigma$-algebra, is called a \textit{measurable space}. Elements in the measurable space are called $\mathcal{A}$-measurable sets.
\end{definition}
\begin{definition}[Restriction of a $\sigma$-Algebra]
  For a measurable space $\left(\Omega,\mathcal{A}\right)$, with $E\in \mathcal{A}$, the family
  \begin{align*}
    \mathcal{A}_{E} &= \set{E\cap A\mid A\in \mathcal{A}}
  \end{align*}
  is a $\sigma$-algebra on $E$, known as the \textit{restriction} of $\mathcal{A}$ to $E$.
\end{definition}
\begin{definition}[Produced $\sigma$-Algebra]
Let $\left(\Omega,\mathcal{A}\right)$ be a measurable space, and $f\colon \Omega\rightarrow \Lambda$ is a map. The $\sigma$-algebra \textit{produced} by $f$ on $\Lambda$ is the collection
\begin{align*}
  \mathcal{N} &= \set{E\mid E\subseteq \Lambda,~f^{-1}(E) \in \mathcal{A}}.
\end{align*}
\end{definition}
\begin{definition}[Generated $\sigma$-Algebra]
  For a family $\mathcal{E}\subseteq P\left(\Omega\right)$, the $\sigma$-algebra \textit{generated} by $E$ is the smallest $\sigma$-algebra that contains $E$.
  \begin{align*}
    \sigma\left(\mathcal{E}\right) &= \bigcap\set{\mathcal{M} | \mathcal{E}\subseteq \mathcal{M},\mathcal{M}\text{ is a $\sigma$-algebra}}.
  \end{align*}
\end{definition}
\begin{definition}[Borel $\sigma$-Algebra]
  If $\Omega$ is a topological space with the topology $\tau\subseteq P(\Omega)$, we define
  \begin{align*}
    \mathcal{B}_{\Omega} &= \sigma\left(\tau\right)
  \end{align*}
  to be the \textit{Borel $\sigma$-algebra}.
\end{definition}
All open, closed, clopen, $F_{\sigma}$, and $G_{\delta}$ subsets of $\Omega$ are Borel.\break

We can now begin examining measurable functions.
\begin{definition}[Measurable Functions]
  Let $\left(\Omega,\mathcal{M}\right)$ and $\left(\Lambda,\mathcal{N}\right)$ be measurable spaces.
  \begin{enumerate}[(1)]
    \item We say a map $f\colon \Omega\rightarrow \Lambda$ is \textit{$\mathcal{M}$-$\mathcal{N}$-measurable} if $f^{-1}\left(E\right)\in \mathcal{M}$ for all $E\in \mathcal{N}$.
    \item We say a map $f\colon \Omega\rightarrow \R$ is measurable if it is $\mathcal{M}$-$\mathcal{B}_{\R}$-measurable.
    \item We say a map $f\colon \Omega\rightarrow \C$ is measurable if both $\re(f)$ and $\im(f)$ are measurable.
  \end{enumerate}
  The set of all measurable functions on $\left(\Omega,\mathcal{M}\right)$ is denoted $L_{0}\left(\Omega,\mathcal{M}\right)$.\newline

  The collection of all bounded measurable functions is the set
  \begin{align*}
    B_{\infty}\left(\Omega,\mathcal{M}\right) &= \set{f\in L_0\left(\Omega,\mathcal{M}\right)\mid \sup_{x\in\Omega}\left\vert f(x) \right\vert < \infty}.
  \end{align*}
\end{definition}

\begin{example}
  If $f\colon \Omega\rightarrow \Lambda$ is a continuous map between topological spaces, then $f$ is $\mathcal{B}_{\Omega}$-$\mathcal{B}_{\Lambda}$-measurable, since
  \begin{align*}
    \mathcal{F} &= \set{E\subseteq \Lambda\mid f^{-1}\left(E\right)\in \mathcal{B}_{\Omega}}
  \end{align*}
  is a $\sigma$-algebra containing every open set in $\Lambda$, so $\mathcal{F}$ contains $\mathcal{B}_{\Lambda}$.
\end{example}

\begin{example}
  If $\left(\Omega,\mathcal{M}\right)$ is a measurable space, and $f\colon \Omega\rightarrow \Lambda$ is a map, the measurable space $\left(\Lambda,\mathcal{N}\right)$ produced by $f$ is necessarily $\mathcal{M}$-$\mathcal{N}$-measurable.
\end{example}

\begin{fact}
  If $\left(\Omega,\mathcal{M}\right)$, $\left(\Lambda,\mathcal{N}\right)$, and $\left(\Sigma,\mathcal{L}\right)$ are measurable spaces, with $f\colon \Omega\rightarrow \Lambda$ and $g\colon \Lambda\rightarrow \Sigma$ measurable, then $g\circ f$ is measurable.\label{fact:composition}
\end{fact}
%\begin{proof}[Proof of Fact \ref{fact:composition}]
%  If $E\in \mathcal{L}$, then $g^{-1}\left(E\right) \in \mathcal{N}$, so $f^{-1}\left(g^{-1}\left(E\right)\right)\in \mathcal{M}$. Thus, $\left(g\circ f\right)^{-1}\left(E\right)\in \mathcal{M}$, so $g\circ f$ is measurable.
%\end{proof}

\begin{proposition}
  Let $\left(\Omega,\mathcal{M}\right)$ be a measurable space. Let $\F = \C$ or $\R$. Suppose $f,g,h_n\colon \Omega\rightarrow \F$ are measurable for $n\geq 1$.
  \begin{enumerate}[(1)]
    \item If $\alpha \in \F$, then $f + \alpha g$ is measurable.
    \item $\overline{f}$ is measurable.
    \item $fg$ is measurable.
    \item $\frac{f}{g}$ is measurable assuming it is well-defined.
    \item if $h_n$ are $\R$-valued, and $\left(h_n\left(x\right)\right)_n$ is bounded for each $x\in \Omega$, then $\sup h_n$ and $\inf h_n$ are measurable.
    \item If $f$ and $g$ are $\R$ valued, then $\max\left(f,g\right)$ and $\min\left(f,g\right)$ are measurable. In particular,
      \begin{align*}
        f_{+} &= \max\left(f,0\right)\\
        f_{-} &= \max\left(0,-f\right)
      \end{align*}
      are measurable.
    \item $\left\vert f \right\vert$ is measurable.
    \item The pointwise limit of measurable functions is measurable --- if $\lim_{n\rightarrow\infty}h_n\left(x\right)$ exists for all $x\in \Omega$, then $h = \lim_{n\rightarrow\infty}h_n$ is measurable.
  \end{enumerate}
\end{proposition}
\begin{definition}[Simple Functions]
  A \textit{simple function} $s\colon \Omega\rightarrow \F$ is a function with finite range. In other words, $s$ is of the form
  \begin{align*}
    s &= \sum_{k=1}^{n}c_k\1_{E_k}
  \end{align*}
  for $E_k\subseteq \Omega$ and $c_k\in \F$.
\end{definition}
\begin{fact}
A simple function is measurable if and only if $E_k\in \mathcal{M}$ for each $k$.
\end{fact}
\section{Constructing Measures}%
A measure assigns a nonnegative ``length'' or ``volume'' to measurable sets.
\begin{definition}[Measures on Measurable Spaces]\label{def:measure_basics}
  A \textit{measure} on a measurable space $\left(\Omega,\mathcal{M}\right)$ is a map $\mu\colon \mathcal{M}\rightarrow \left[0,\infty\right]$ that satisfies the following.
  \begin{enumerate}[(i)]
    \item $\mu\left(\emptyset\right) = 0$;
    \item $\displaystyle \mu\left(\bigsqcup_{j=1}^{\infty}E_j\right) = \sum_{j=1}^{\infty}\mu\left(E_j\right)$.
  \end{enumerate}
  The triple $\left(\Omega,\mathcal{M},\mu\right)$ is called a \textit{measure space}.\newline

  A measure $\mu$ is \textit{finite} if $\mu\left(\Omega\right) < \infty$\newline

  If $\mu\left(\Omega\right) = 1$, then $\mu$ is called a \textit{probability measure}.\newline

  A measure $\mu$ is called \textit{finitely additive} if $\mu\left(E\sqcup F\right) = \mu(E) + \mu(F)$.\newline

  A measure $\mu$ is called \textit{$\sigma$-finite} if there is a countable family $\set{E_n}_{n\geq 1}\subseteq \mathcal{M}$ such that
  \begin{align*}
    \Omega &= \bigcup_{n\geq 1}E_n
  \end{align*}
  and $\mu\left(E_n\right) < \infty$.\newline

  A measure $\mu$ on $\left(\Omega,\mathcal{M}\right)$ is called \textit{semi-finite} if, for every $E\in \mathcal{M}$ with $\mu(E) = \infty$, there exists $F\in \mathcal{M}$ with $F\subseteq E$ and $0 < \mu(F) < \infty$.
\end{definition}

\begin{lemma}
  Let $\left(\Omega,\mathcal{M},\mu\right)$ be a measure space.
  \begin{enumerate}[(1)]
    \item If $E,F\in \mathcal{M}$ with $F\subseteq E$, then $\mu\left(F\right) \subseteq \mu\left(E\right)$.
    \item If $\left(E_n\right)_{n}$ is a sequence of measurable sets, then
      \begin{align*}
        \mu\left(\bigcup_{n\geq 1}E_n\right) &\leq \sum_{n=1}^{\infty}\mu\left(E_n\right).
      \end{align*}
    \item If $\left(E_n\right)_{n\geq 1}$ is an increasing family of measurable sets, then
      \begin{align*}
        \mu\left(\bigcup_{n\geq 1}E_n\right) &= \lim_{n\rightarrow\infty}\mu\left(E_n\right).
      \end{align*}
  \end{enumerate}
\end{lemma}
%\begin{proof}\hfill
%  \begin{enumerate}[(1)]
%    \item Since $F\subseteq E$, we can write $E = F\sqcup \left(E\setminus F\right)$. Thus,
%      \begin{align*}
%        \mu\left(E\right) &= \mu\left(F\right) + \mu\left(E\setminus F\right)\\
%                          &\geq \mu\left(F\right).
%      \end{align*}
%    \item We write
%      \begin{align*}
%        F_1 &= E_1\\
%        F_2 &= E_2\setminus E_1\\
%            &\vdots\\
%        F_n &= E_n\setminus \left(\bigcup_{k=1}^{n-1}E_k\right).
%      \end{align*}
%      Since each $F_n$ is measurable, and $F_n\subseteq E_n$, we have
%      \begin{align*}
%        \mu\left(\bigcup_{n\geq 1}E_n\right) &= \mu\left(\bigsqcup_{n\geq 1}F_n\right)\\
%                                             &= \sum_{n=1}^{\infty}F_n\\
%                                             &\leq \sum_{n=1}^{\infty}\mu\left(E_n\right).
%      \end{align*}
%    \item We write $F_n$ as the respective disjoint union for $\set{E_n}_{n\geq 1}$. We have $\bigsqcup_{k=1}^{n}F_k = E_n$. Then,
%      \begin{align*}
%        \mu\left(\bigcup_{n\geq 1}E_n\right) &= \sum_{n=1}^{\infty}\mu\left(F_n\right)\\
%                                             &= \lim_{n\rightarrow\infty}\left(\sum_{k=1}^{n}\mu\left(F_k\right)\right)\\
%                                             &= \lim_{n\rightarrow\infty}\mu\left(\bigsqcup_{k=1}^{n}F_k\right)\\
%                                             &= \lim_{n\rightarrow\infty}\mu\left(E_n\right).
%      \end{align*}
%  \end{enumerate}
%\end{proof}
%\begin{definition}[Counting Measure]
%  If $\Omega$ is any set, the {counting measure} on $\left(\Omega,P\left(\Omega\right)\right)$ assigns $\left\vert A \right\vert$ for each $A\in P\left(\Omega\right)$ finite, and $\infty$ for any infinite subset.
%\end{definition}
%\begin{definition}[Restricting Measures]
%  If $\left(\Omega,\mathcal{M},\mu\right)$ is a measure space, $\mathcal{B}$ is a $\sigma$-algebra on $\Omega$ with $\mathcal{B}\subseteq \mathcal{M}$, the restriction $\mu|_{\mathcal{B}}$ is a measure on $\left(\Omega,\mathcal{B}\right)$.\newline
%
%  If $E\in \mathcal{M}$, we can restrict $\mu$ to $\mathcal{M}_{E}$ (the restriction of $\mathcal{M}$ to $E$), yielding the measure space $\left(E,\mathcal{M}_{E},\mu|_{\mathcal{M}_E}\right)$. We denote this restricted measure $\mu_{E}$, such that $\mu_{E}\left(M\cap E\right) = \mu\left(M\cap E\right)$ for all $M\in \mathcal{M}_{E}$.
%\end{definition}
\begin{definition}[Pushforward Measure]
  Let $\left(\Omega,\mathcal{M},\mu\right)$ be a measure space, and let $\left(\Lambda,\mathcal{N}\right)$ be a measurable space. Let $f\colon \Omega\rightarrow \Lambda$ be measurable. The map
  \begin{align*}
    f_{\ast}\mu\colon \mathcal{N}\rightarrow [0,\infty]
  \end{align*}
  defined by
  \begin{align*}
    f_{\ast}\mu\left(E\right) &= \mu\left(f^{-1}\left(E\right)\right)
  \end{align*}
  defines a measure on $\left(\Lambda,\mathcal{N}\right)$. This is known as the \textit{pushforward measure} of $\mu$.\newline

  If $\mathcal{N}$ on $\Lambda$ is produced by $f$, then the pushforward measure is necessarily defined on $\mathcal{N}$, and that any function $g\colon \Lambda\rightarrow \F$ is measurable if and only if $g\circ f$ is measurable.
\end{definition}
%\begin{definition}[Disjoint Union]
%  Let $\set{\left(\Omega_n,\mathcal{M}_n,\mu_n\right)}$ be a countable family of measure spaces.\newline
%
%  We define the co-product of this family by taking
%  \begin{align*}
%    \Sigma := \bigsqcup_{n=1}^{\infty}\Omega_n,
%  \end{align*}
%  to be our set equipped with the canonical inclusion map $\iota_{n}\left(x\right) = \left(x,n\right)$, such that for each $n$,
%  \begin{align*}
%    \mathcal{M} &:= \set{E\subseteq \Sigma\mid \iota_{n}^{-1}\left(E\right)\in \mathcal{M}_n}.
%  \end{align*}
%  The measure is defined by
%  \begin{align*}
%    \mu\colon \mathcal{M}\rightarrow [0,\infty]\\
%    \mu(E) := \sum_{n=1}^{\infty}\mu_{n}\left(\iota_{n}^{-1}\left(E\right)\right).
%  \end{align*}
%  We can identify each $\Omega_n$ with the subset $\Omega_{n}^{\ast} = \set{\left(x,n\right)\mid x\in \Omega_n}\subseteq \Sigma$, with $\iota_{n}^{-1}\left(E\right)\subseteq \Omega_{n}$ identified with $E\cap \Omega_{n}^{\ast}$.\newline
%
%  The family $\set{\Omega_{n}^{\ast}}_{n\geq 1}$ forms a measurable partition of $\Sigma$, and that $\mu|_{\Omega_{n}^{\ast}}$ are the pushforwards of $\mu_{n}$ by $\iota_{n}$.\newline
%
%  Note that a map $f\colon \Sigma\rightarrow \C$ is measurable if and only if $f\circ \iota_{n}\colon \Omega_n\rightarrow \C$ is measurable for all $n$.\newline
%
%  If $\left(f_n\colon \Omega_n\rightarrow \C\right)_{n}$ is a sequence of measurable maps, the disjoint union
%  \begin{align*}
%    f = \bigsqcup_{n\geq 1}f_n\colon \Sigma\rightarrow \C
%  \end{align*}
%  defined by $f\left(x,n\right) = f_n(x)$, is measurable.
%\end{definition}
\begin{definition}
  Let $\left(\Omega,\mathcal{M},\mu\right)$ be a measure space.\newline

  A \textit{null set} is a measurable set $N\in \mathcal{M}$ with $\mu\left(N\right) = 0$.\newline

  A property which holds for all $x\in \Omega\setminus N$ for some null set $N$ is said to hold \textit{$\mu$-almost everywhere,} or $\mu$-a.e.
\end{definition}
\begin{definition}
  If $\left(\Omega,\mathcal{M},\mu\right)$ is a measure space, we can define an equivalence relation on the set $L_{0}\left(\Omega,\mathcal{M},\mu\right)$, by
  \begin{align*}
    f\sim_{\mu}g \text{ if and only if } \mu\left(\set{x\mid f(x)\neq g(x)}\right) = 0.
  \end{align*}
  We define the set of all classes of measurable functions by
  \begin{align*}
    L\left(\Omega,\mu\right) &= L_{0}\left(\Omega,\mathcal{M}\right)/\sim_{\mu}\\
                             &= \set{\left[f\right]_{\mu}\mid f\in L_{0}\left(\Omega,\mathcal{M}\right)}.
  \end{align*}
\end{definition}
\begin{fact}
  The operations
  \begin{itemize}
    \item $\displaystyle \left[f\right]_{\mu} + \left[g\right]_{\mu} = \left[f + g\right]_{\mu}$;
    \item $\displaystyle \left[f\right]_{\mu}\left[g\right]_{\mu} = \left[fg\right]_{\mu}$;
    \item and $\displaystyle \alpha \left[f\right]_{\mu} = \left[\alpha f\right]_{\mu}$
  \end{itemize}
  are well-defined.
\end{fact}
\begin{definition}[Essentially Bounded Functions and Continuous Functions]
  Let $\left(\Omega,\mathcal{M},\mu\right)$ be a measure space, and $f\colon \Omega\rightarrow \C$ be measurable. We say $f$ is \textit{$\mu$-essentially bounded} if there is $C\geq 0$ such that
  \begin{align*}
    \mu\left(\set{x\in \Omega\mid \left\vert f(x) \right\vert\geq C}\right) = 0.
  \end{align*}
  We say $C$ is an essential bound for $f$. The infimum of all essential bounds is the \textit{essential supremum}, which gives the norm
  \begin{align*}
    \norm{f}_{\infty} &= \esssup(f)\\
                      &= \inf\set{C\geq 0 \mid \mu\left(\set{x\in \Omega\mid \left\vert f(x) \right\vert\geq C}\right) = 0}.
  \end{align*}
  The collection of all $\mu$-essentially bounded functions is denoted
  \begin{align*}
    L_{\infty}\left(\Omega,\mu\right) = \set{\left[f\right]_{\mu}\in L\left(\Omega,\mu\right)\mid \norm{f}_{\infty} < \infty}.
  \end{align*}
  Note that $B_{\infty}\left(\Omega,\mu\right) = L_{\infty}\left(\Omega,\mu\right)$ as sets.\newline

  For $\mu$ a measure on $\left(\Omega,\mathcal{B}_{\Omega}\right)$, the $\mu$-equivalence classes of continuous functions are
  \begin{align*}
    C\left(\Omega,\mu\right) = \set{\left[f\right]_{\mu}\mid f\in C\left(\Omega\right)}.
  \end{align*}
\end{definition}
\begin{fact}
  If $\Omega$ is a topological space, with $\mathcal{B}_{\Omega}$ the Borel $\sigma$-algebra, we have $C\left(\Omega\right)\subseteq L_{0}\left(\Omega,\mathcal{B}_{\Omega}\right)$.\newline
\end{fact}
\begin{remark}
  Members of $L\left(\Omega,\mu\right)$ and $L_{\infty}\left(\Omega,\mu\right)$ are equivalence classes of functions (rather than functions themselves), but we use the abuse of notation that $\left[f\right]_{\mu} = f$.
\end{remark}

\begin{fact}
  Let $\left(\Omega,\mathcal{M},\mu\right)$ be a measure space, and let $f,g\colon \Omega\rightarrow \C$ be measurable, and $\alpha \in \C$. Then, the following are true:
  \begin{itemize}
    \item $\displaystyle \norm{f+g}_{\infty}\leq \norm{f}_{\infty} + \norm{g}_{\infty}$;
    \item $\displaystyle \norm{\alpha f}_{\infty} = \left\vert \alpha \right\vert\norm{f}_{\infty}$;
    \item if $\displaystyle \norm{f}_{\infty} = 0$, then $f = 0$ $\mu$-a.e.;
    \item $\displaystyle \norm{f}_{\infty}\leq \norm{f}_{u}$;
    \item if $f$ is essentially bounded, then
      \begin{align*}
        \mu\left(\set{x\mid \left\vert f(x) \right\vert\geq \norm{f}_{\infty}}\right) &= 0.
      \end{align*}
  \end{itemize}
\end{fact}

\begin{definition}[Complete Measure Space]
A measure space $\left(\Omega,\mathcal{M},\mu\right)$ is said to be \textit{complete} if all subsets of null sets are measurable (and null).
\end{definition}
\section{Integration}%
\begin{definition}
  If $\phi\colon \Omega\rightarrow [0,\infty)$ is a positive, simple, and measurable function, 
  \begin{align*}
    \phi &= \sum_{k=1}^{n}c_k\1_{E_k},
  \end{align*}
  then the \textit{integral} of $\phi$ is defined as
  \begin{align*}
    \int_{\Omega}\phi\:d\mu &= \sum_{k=1}^{n}c_k\mu\left(E_k\right),
  \end{align*}
  with the convention that $0\cdot \infty = 0$.
\end{definition}
\begin{fact}
The value of this integral is not dependent on the representation of $\phi$.
\end{fact}
\begin{definition}
  If $f\colon \Omega\rightarrow\infty [0,\infty)$ is a positive measurable function, then
  \begin{align*}
    \int_{\Omega}f\:d\mu &= \sup\set{\int_{\Omega}\phi\:d\mu\mid \phi\text{ measurable and simple, $0\leq \phi \leq f$}}.
  \end{align*}
  If $E\subseteq \Omega$ is measurable, we define
  \begin{align*}
    \int_{E} f\:d\mu &= \int_{\Omega}f\1_{E}\:d\mu.
  \end{align*}
\end{definition}
\begin{proposition}
  Let $\left(\Omega,\mathcal{M}\right)$ be a measurable space, and let $f:\Omega\rightarrow \C$ be measurable. There is a sequence $\left(\phi_{n}\right)_n$ of simple, measurable functions with $\left(\phi_{n}\left(x\right)\right)_{n}\xrightarrow{n\rightarrow\infty}f(x)$.\newline

  If $f\geq 0$, we can take $\phi_n$ to be positive and pointwise increasing.\newline

  If $f$ is bounded, then this convergence is uniform, and $\left(\phi_{n}\right)_n$ can be chosen to be uniformly bounded.
\end{proposition}

\begin{theorem}[Monotone Convergence Theorem]
  Let $\left(f_n:\Omega\rightarrow [0,\infty)\right)$ be an increasing sequence of positive, measurable functions converging pointwise to $f\colon \Omega\rightarrow [0,\infty)$. Then, $f$ is measurable, and
  \begin{align*}
    \lim_{n\rightarrow\infty}\int_{\Omega}^{} f_n\:d\mu &= \int_{\Omega}^{} f\:d\mu.
  \end{align*}
\end{theorem}
\begin{definition}
  Let $\left(\Omega,\mathcal{M},\mu\right)$ be a measure space.
  \begin{enumerate}[(1)]
    \item A measurable function $f\colon \Omega\rightarrow [0,\infty)$ is \textit{integrable} if
      \begin{align*}
        \int_{\Omega}^{} f\:d\mu < \infty.
      \end{align*}
    \item A measurable function $f\colon \Omega\rightarrow \R$ is integrable if both $f_{+}$ and $f_{-}$ are integrable. We define
      \begin{align*}
        \int_{\Omega}^{} f\:d\mu &= \int_{\Omega}^{} f_{+}\:d\mu - \int_{\Omega}^{} f_{-}\:d\mu.
      \end{align*}
    \item A measurable function $f\colon \Omega\rightarrow \C$ is said to be integrable if both $\re(f)$ and $\im(f)$ are integrable. We define
      \begin{align*}
        \int_{\Omega}^{} f\:d\mu &= \int_{\Omega}^{} \re(f)\:d\mu + i\int_{\Omega}^{} \im(f)\:d\mu.
      \end{align*}
  \end{enumerate}
\end{definition}
\begin{fact}
  Let $f,g\colon \Omega\rightarrow \C$ be integrable functions, and $\alpha\in\C$. Then,
  \begin{itemize}
    \item $f + \alpha g$ is integrable, and $\displaystyle \int_{\Omega}^{} \left(f + \alpha g\right)\:d\mu = \int_{\Omega}f\:d\mu + \alpha\int_{\Omega}g\:d\mu$;
    \item if $f$ and $g$ are real-valued, and $f\leq g$, then $\displaystyle \int_{\Omega}^{} f\:d\mu \leq \int_{\Omega}^{} g\:d\mu$;
    \item $\displaystyle \left\vert \int_{\Omega}^{} f\:d\mu \right\vert\leq \int_{\Omega}^{} \left\vert f \right\vert\:d\mu$.
  \end{itemize}
\end{fact}
\begin{fact}
  If $f = g$ $\mu$-a.e., then
  \begin{align*}
    \int_{\Omega}^{} f\:d\mu &= \int_{\Omega}^{} g\:d\mu.
  \end{align*}
\end{fact}
\begin{fact}
  If $f\colon \Omega\rightarrow \C$ is measurable, then $\int_{\Omega}^{} \left\vert f \right\vert\:d\mu = 0$ if and only if $f = 0$ $\mu$-a.e.
\end{fact}
\begin{fact}
  A measurable function $f\colon \Omega\rightarrow \C$ is integrable if and only if $\left\vert f \right\vert$ is integrable.
\end{fact}
\begin{definition}[Integrable Functions]
  Let $\left(\Omega,\mathcal{M},\mu\right)$ be a measure space.
  \begin{enumerate}[(1)]
    \item We define the set of (equivalence classes of) integrable functions to be
      \begin{align*}
        L_{1}\left(\Omega,\mu\right) = \set{\left[f\right]_{\mu}\in L\left(\Omega,\mu\right)\mid \text{$f$ is integrable}}.
      \end{align*}
    \item We define the set of (equivalence classes of) square-integrable functions to be
      \begin{align*}
        L_{2}\left(\Omega,\mu\right) &= \set{\left[f\right]_{\mu}\in L\left(\Omega,\mu\right)\mid \left\vert f \right\vert^2\text{ is integrable}}.
      \end{align*}
  \end{enumerate}
\end{definition}
\begin{definition}
  Let $\left(\Omega,\mathcal{M},\mu\right)$ be a measure space. If $f$ and $\left(f_n\right)_n$ are integrable with $\norm{f-f_n}_{1}\xrightarrow{n\rightarrow\infty}0$, we say $\left(f_n\right)_n$ \textit{converges in mean} to $f$.
\end{definition}

\begin{fact}
  Let $\left(\Omega,\mathcal{M},\mu\right)$ be a measure space.
  \begin{enumerate}[(1)]
    \item For $f\in L_{1}\left(\Omega,\mu\right)$, the maps
      \begin{align*}
        \left[f\right]_{\mu} &\longmapsto \int_{\Omega}^{} f\:d\mu\\
        \left[f\right]_{\mu} &\longmapsto \int_{\Omega}^{} \left\vert f \right\vert\:d\mu
      \end{align*}
      are well-defined.
    \item For $f\in L_{1}\left(\Omega,\mu\right)$, we define
      \begin{align*}
        \norm{f}_{1} &= \int_{\Omega}^{} \left\vert f \right\vert\:d\mu.
      \end{align*}
      This is a well-defined norm.
      \begin{align*}
        \norm{f+g}_{1} &\leq \norm{f}_{1} + \norm{g}_{1}\\
        \norm{\alpha f}_1 &= \left\vert \alpha \right\vert\norm{f}_{1}\\
        \norm{f}_1 = 0 &\Leftrightarrow f= 0 \text{ $\mu$-a.e.}
      \end{align*}
    \item 
      \begin{align*}
        d_1\left(\left[f\right]_{\mu},\left[g\right]_{\mu}\right) &= \norm{f-g}_{1}
      \end{align*}
      is a metric on $L_{1}\left(\Omega,\mu\right)$.
  \end{enumerate}
\end{fact}
\begin{theorem}[Dominated Convergence Theorem]
  Let $\left(f_n:\Omega\rightarrow \C\right)_{n}$ be a sequence of measurable functions converging pointwise to a measurable function $f\colon \Omega\rightarrow \C$. If there is an integrable $g\colon \Omega \rightarrow [0,\infty)$ with $\left\vert f_n \right\vert\leq g$ for all $n$, then
  \begin{align*}
    \int_{\Omega}^{} f_n\:d\mu \xrightarrow{n\rightarrow\infty}\int_{\Omega}^{} f\:d\mu.
  \end{align*}
\end{theorem}
\begin{corollary}
  If $f\colon \Omega\rightarrow \C$ is integrable, then there is a sequence of simple integrable functions $\left(\phi_n\right)_n$ with $\norm{f - \phi_{n}}_{1}\xrightarrow{n\rightarrow\infty}0$.
\end{corollary}
\begin{corollary}
  If $f\colon \R\rightarrow\C$ is integrable, then there is a sequence $\left(f_n\right)_n$ of compactly supported integrable functions such that $\norm{f - f_n}_{1}\xrightarrow{n\rightarrow\infty} 0$.
\end{corollary}
\begin{theorem}
  If $f\colon \R\rightarrow\C$ is integrable, and $\ve > 0$, there is a continuous, compactly supported function $g$ with $\norm{f - g}_{1} < \ve$.
\end{theorem}
\begin{proposition}
  Let $\left(\Omega,\mathcal{M},\mu\right)$ be a measure space, and let $\left(\Lambda,\mathcal{N}\right)$ be a measurable space with $f\colon \Omega\rightarrow \Lambda$ a measurable map. Let $f_{\ast}\mu$ be the pushforward measure on $\left(\Lambda,\mathcal{N}\right)$. For a measurable function $g\colon \Lambda\rightarrow [0,\infty)$, then
  \begin{align*}
    \int_{\Lambda}^{} g\:d\left(f_{\ast}\mu\right) &= \int_{\Omega}^{} \left(g\circ f\right)\:d\mu.
  \end{align*}
  Moreover, if $g\colon \Lambda\rightarrow \F$ is integrable with respect to $f_{\ast}\mu$, then so too is $g\circ f$ with respect to $\mu$.
\end{proposition}
\section{Complex Measures}%
\begin{example}
  If $\left(\Omega,\mathcal{M},\mu\right)$ is a measure space, then the map $\mu_f(E) = \int_{E}^{} f\:d\mu$ is a well-defined measure.
\end{example}
\begin{definition}
  Let $\left(\Omega,\mathcal{M},\mu\right)$ be a measurable space.
  \begin{enumerate}[(1)]
    \item A \textit{complex measure} on $\left(\Omega,\mathcal{M},\mu\right)$ is a map $\mu\colon \mathcal{M}\rightarrow \C$ satisfying the following conditions.
      \begin{itemize}
        \item $\mu\left(\emptyset\right) = 0$;
        \item $\displaystyle \mu\left(\bigsqcup_{k=1}^{\infty}E_k\right) = \sum_{k=1}^{\infty}\mu\left(E_k\right)$ for $\set{E_k}_{k\geq 1}\subseteq \mathcal{M}$.
      \end{itemize}
    \item We write $M\left(\Omega\right)$ to be the set of all complex measures on $\left(\Omega,\mathcal{M}\right)$.
    \item If $\mu\in M\left(\Omega\right)$, and $\mu\left(E\right)\in \R$ for all $E\in \mathcal{M}$, then we say $\mu$ is a \textit{real measure} on $\left(\Omega,\mathcal{M}\right)$.
    \item If $\mu\in M\left(\Omega\right)$ and $\mu(E) \geq 0$ for all $E\in \mathcal{M}$, then we say $\mu$ is a \textit{positive measure} on $\left(\Omega,\mathcal{M}\right)$.
    \item If $\mu$ is a positive measure on $\left(\Omega,\mathcal{M}\right)$ with $\mu\left(\Omega\right) = 1$, we say $\mu$ is a probability measure on $\left(\Omega,\mathcal{M}\right)$. We write $\mathcal{P}\left(\Omega,\mathcal{M}\right)$ to be the collection of all probability measures on $\left(\Omega,\mathcal{M}\right)$.
    \item If $\Omega$ is a LCH space, we always let $M\left(\Omega\right)$ be the set of all complex Borel measures on $\Omega$.
  \end{enumerate}
\end{definition}
\begin{definition}
  If $\left(\Omega,\mathcal{M}\right)$ is a measurable space, and $x\in \Omega$, the \textit{Dirac measure} at $x$ is defined by
  \begin{align*}
    \delta_{x}\colon \mathcal{M}&\rightarrow [0,1]\\
    \delta_x\left(E\right) &= \begin{cases}
      1 & x\in E\\
      0 & x\notin E
    \end{cases}.
  \end{align*}
    If $x_1,\dots,x_n$ are distinct points in $\Omega$, and $t_1,\dots,t_n\in [0,1]$ with $\sum_{j=1}^{n}t_j = 1$, then
    \begin{align*}
      \mu &= \sum_{j=1}^{n}t_j\delta_{x_j}
    \end{align*}
    is a probability measure on $\left(\Omega,\mathcal{M}\right)$.
    %Every probability measure can be weakly approximated by convex combinations of Dirac measures by the Krein-Milman Theorem.
\end{definition}
\begin{fact}
  If $\mu$ is a complex measure on $\left(\Omega,\mathcal{M}\right)$, then $\overline{\mu}$, defined by $\overline{\mu}\left(E\right) = \overline{\mu\left(E\right)}$ for $E\in \mathcal{M}$, is also a complex measure. Additionally, $\re\left(\mu\right)$ and $\im\left(\mu\right)$, defined by
  \begin{align*}
    \re\left(\mu\right)\left(E\right) &= \re\left(\mu\left(E\right)\right)\\
    \im\left(\mu\right)\left(E\right) &= \im\left(\mu\left(E\right)\right)
  \end{align*}
  are real measures.
\end{fact}
\begin{definition}
  If $\mu\in M\left(\Omega\right)$, then the \textit{total variation} of $\mu$ is the quantity
  \begin{align*}
    \left\vert \mu \right\vert\colon \mathcal{M}\rightarrow [0,\infty]
    \end{align*}
    with
    \begin{align*}
    \left\vert \mu \right\vert\left(E\right) = \sup\set{\left.\sum_{j=1}^{\infty}\left\vert \mu\left(E_j\right) \right\vert\right|E = \bigsqcup_{j=1}^{\infty}E_j,~E_j\in\mathcal{M}}.
  \end{align*}
\end{definition}
\begin{fact}
  If $\mu\in M\left(\Omega\right)$, then $\left\vert \mu \right\vert$ is a positive, finite measure. Additionally, if $\mu,\nu\in M\left(\Omega\right)$ with $\alpha\in \C$, then
  \begin{enumerate}[(a)]
    \item $\left\vert \mu\left(E\right) \right\vert\leq \left\vert \mu \right\vert\left(E\right)$
    \item $\left\vert \mu + \nu \right\vert\left(E\right) \leq \left\vert \mu \right\vert\left(E\right) + \left\vert \nu \right\vert\left(E\right)$
    \item $\left\vert \alpha\mu \right\vert\left(E\right) = \left\vert \alpha \right\vert\left\vert \mu \right\vert\left(E\right)$.
  \end{enumerate}
\end{fact}
\begin{definition}[Absolute Continuity of Measures]
  Let $\left(\Omega,\mathcal{M}\right)$ be a measurable space, and let $\mu$ and $\nu$ be positive measures on this space. If $\mu(A) > 0$ implies $\nu(A) > 0$ for a given $A\in \mathcal{M}$, we say $\mu$ is \textit{absolutely continuous} with respect to $\nu$. We write $\mu \ll \nu$.
\end{definition}
\begin{theorem}[Radon--Nikodym Theorem]
  If $\mu \ll \nu$ on $\left(\Omega,\mathcal{M}\right)$, then there exists a measurable function $f\colon \Omega\rightarrow [0,\infty]$ such that
  \begin{align*}
    \nu(A) &= \int_{A}^{} f \:d\nu
  \end{align*}
  for each $A\in \mathcal{M}$.
\end{theorem}
\begin{remark}
The Radon--Nikodym theorem extends to signed and complex measures.
\end{remark}
\begin{fact}
  Let $\left(\Omega,\mathcal{M},\lambda\right)$ be a measure space, and suppose $f\in L_{1}\left(\Omega,\lambda\right)$. Then, $\mu\left(E\right) = \int_{E}^{} f\:d\lambda$ defines a complex measure. We write $f = \diff{\mu}{\lambda}$, which is the \textit{Radon--Nikodym derivative} of $\mu$ with respect to $\lambda$.\newline

  It is also the case that
  \begin{align*}
    \left\vert \mu \right\vert\left(E\right) &= \int_{E}^{} \left\vert f \right\vert\:d\lambda.
  \end{align*}
\end{fact}
\begin{fact}
  If $\mu\in M\left(\Omega\right)$, there exists a measurable function $f\colon \Omega\rightarrow \C$ such that $\left\vert f \right\vert = 1$ and $\mu\left(E\right) = \int_{E}^{} f\:d\left\vert \mu \right\vert$ for all $E\in \mathcal{M}$.
\end{fact}
\begin{definition}
  Let $\Omega$ be a LCH space equipped with the Borel $\sigma$-algebra, $\mathcal{B}_{\Omega}$.
  \begin{enumerate}[(1)]
    \item A Borel measure $\mu\colon \mathcal{B}_{\Omega}\rightarrow [0,\infty]$ is called
      \begin{itemize}
        \item \textit{inner regular} on $E\in \mathcal{B}_{\Omega}$ if
          \begin{align*}
            \mu(E) &= \sup\set{\mu\left(K\right)\mid K\subseteq E, K\text{ compact}};
          \end{align*}
        \item \textit{outer regular} on $E\in \mathcal{B}_{\Omega}$ if
          \begin{align*}
            \mu(E) &= \inf\set{\mu\left(U\right)\mid U\supseteq E,U\text{ open}};
          \end{align*}
        \item \textit{regular} on $E$ if it is inner regular and outer regular on $E$;
        \item regular if it is regular on all $E\in \mathcal{B}_{\Omega}$;
        \item \textit{Radon} if
          \begin{itemize}
            \item $\mu(K) < \infty$ for all compact $K\subseteq \Omega$;
            \item $\mu$ is inner regular on all open sets and outer regular on all Borel sets.
          \end{itemize}
      \end{itemize}
    \item A complex Borel measure $\mu\colon \mathcal{B}_{\Omega}\rightarrow \C$ is regular if $\left\vert \mu \right\vert$ is regular; $\mu$ is Radon if $\left\vert \mu \right\vert$ is Radon.
    \item We write $M_{r}\left(\Omega\right)$ to denote the set of all complex regular measures on $\left(\Omega,\mathcal{B}_{\Omega}\right)$.
  \end{enumerate}
\end{definition}
\begin{fact}
  Every positive Radon measure is regular. Thus, every complex Borel measure is regular if and only if it is Radon.\newline

  Moreover, if $\Omega$ is a second countable LCH space, then every complex Borel measure is regular.
\end{fact}

\begin{definition}
  Let $\left(\Omega,\tau\right)$ be a topological space, and suppose $\mu\colon \mathcal{B}_{\Omega}\rightarrow [0,\infty]$ is a Borel measure.
  \begin{enumerate}[(1)]
    \item The \textit{kernel} of $\mu$ is the set
      \begin{align*}
        N_{\mu} &= \bigcup\set{U\subseteq \Omega\mid U\in\tau,~\mu(U) = 0}.
      \end{align*}
    \item The \textit{support} of $\mu$ is the complement of the kernel, $\supp(\mu) = N_{\mu}^{c}$.
  \end{enumerate}
\end{definition}
\begin{fact}
  If $\mu$ is a Radon measure on a LCH space $\Omega$, then $\mu\left(N_{\mu}\right) = 0$, meaning $\mu\left(\Omega\right) = \mu\left(\supp\left(\mu\right)\right)$.
\end{fact}

\begin{theorem}[Hahn and Jordan Decomposition]
  Let $\left(\Omega,\mathcal{M}\right)$ be a measurable space, and let $\mu\colon \mathcal{M}\rightarrow \R$ be a real measure. Then, there is a measurable partition $\Omega = P\sqcup N$ such that for all $E\subseteq P$, $\mu(E) \geq 0$, and for all $E\subseteq N$, $\mu(E) \leq 0$. This partition is unique up to a $\mu$-null symmetric difference --- that is, for any $P',N'$ satisfying the conditions, $\mu\left(P'\triangle P\right) = 0$ and $\mu\left(N'\triangle N\right) = 0$.\newline

  There is a unique decomposition $\mu = \mu_{+} - \mu_{-}$, with $\mu_{\pm}$ that are positive such that if $E\subseteq P$, then $\mu_{-}\left(E\right) = 0$, and if $E\subseteq N$, $\mu_{+}\left(E\right) = 0$.
\end{theorem}
\begin{definition}
  Let $\left(\Omega,\mathcal{M}\right)$ be a measurable space, and let $f\colon \Omega\rightarrow \C$ be measurable.
  \begin{enumerate}[(1)]
    \item If $\mu\colon \mathcal{M}\rightarrow \R$ is a real measure with $\mu = \mu_{+} - \mu_{-}$, we say that $f$ is $\mu$-integrable if it is both $\mu_{+}$ and $\mu_{-}$-integrable. We define
      \begin{align*}
        \int_{\Omega}^{} f\:d\mu &= \int_{\Omega}^{} f\:d\mu_{+} - \int_{\Omega}^{} f\:d\mu_{-}.
      \end{align*}
    \item If $\mu:\mathcal{M}\rightarrow \C$ is a complex measure with $\mu_{1} = \re\left(\mu\right)$ and $\mu_{2} = \im\left(\mu\right)$, we say $f$ is $\mu$-integrable if it is both $\mu_{1}$ and $\mu_{2}$-integrable. We define
      \begin{align*}
        \int_{\Omega}^{} f\:d\mu &= \int_{\Omega}^{} f\:d\mu_{1} + i\int_{\Omega}^{} f\:d\mu_{2}.
      \end{align*}
  \end{enumerate}
\end{definition}
\begin{theorem}[Riesz Representation Theorem on $C_c\left(\Omega\right)$]
  Let $\Omega$ be a LCH space. If $\varphi\colon C_{c}\left(\Omega\right)\rightarrow \C$ is a positive linear functional, then there is a unique Radon measure $\mu$ such that
  \begin{align*}
    \varphi\left(f\right) &= \int_{\Omega}^{} f\:d\mu
  \end{align*}
  for all $f\in C_c\left(\Omega\right)$. Additionally, for every open $U\subseteq \Omega$, we have
  \begin{align*}
    \mu\left(U\right) &= \sup\set{\varphi\left(f\right) | f\in C_c\left(\Omega,[0,1]\right),\supp(f) \subseteq U},
  \end{align*}
  and for every compact $K\subseteq \Omega$, we have
  \begin{align*}
    \mu\left(K\right) &= \inf\set{\varphi(f) | f\geq \1_{K}}.
  \end{align*}
\end{theorem}
\begin{theorem}[Riesz Representation Theorem on $C\left(X\right)$]
  Let $X$ be a compact metric space, and let $\varphi\in \left(C\left(X\right)\right)^{\ast}$ be a positive linear functional with $\varphi\left(\1_{X}\right) = \norm{\varphi} = 1$. Then, for $f\in C(X)$, there is a unique Borel probability measure such that
  \begin{align*}
    \varphi\left(f\right) &= \int_{X}^{} f\:d\mu.
  \end{align*}
\end{theorem}
%\begin{theorem}[Markov--Riesz Theorem]
%  Let $\Omega$ be a LCH space. Then, $M_r\left(\Omega\right)\cong C_0\left(\Omega\right)^{\ast}$.
%\end{theorem}
%\begin{definition}
%  Let $\Omega$ be a LCH space, and let $\tau\colon \Omega\rightarrow \Omega$ be a continuous transformation. A regular Borel probability measure $\mu\in \mathcal{P}_{r}\left(\Omega\right)$ is called $\tau$-invariant if $\tau_{\ast}\mu = \mu$.
%\end{definition}

% I can trim down this section quite a lot
\chapter{Functional Analysis}\label{ch:functional_analysis}
Functional analysis plays an integral role in establishing amenability, as we relate a group $G$ to the space of bounded functions with domain $G$, as well as the dual space of $G$.
\section{Normed Vector Spaces and Algebras}%
The fundamental unit of functional analysis is functions --- specifically, collections of functions equipped with particular operations and a norm that turn them into vector spaces and algebras. This section will focus on some of the basic facts and theory surrounding normed vector spaces and algebras.
\begin{definition}[Seminorms and Norms]\label{def:norms}
  Let $X$ be a $\F$-vector space, and let $p\colon X\rightarrow [0,\infty)$ be a function. If
  \begin{itemize}
    \item $p\left(\lambda x\right) = \left\vert \lambda \right\vert p(x)$ for all $\lambda\in \F$ and $x\in X$ (homogeneity), and
    \item $p\left(x + y\right)\leq p\left(x\right) + p\left(y\right)$ for all $x,y\in X$ (triangle inequality),
  \end{itemize}
  we say $p$ is a \textit{seminorm}. If $p$ also satisfies
  \begin{itemize}
    \item $p\left(x\right) = 0 $ if and only if $x = 0$ (positive definite),
  \end{itemize}
  then $p$ is a \textit{norm}. Norms on vector spaces are usually denoted $\norm{\cdot}$.\newline

  Additionally, if $X$ is an algebra, the (semi)norm also has to be sub-multiplicative --- i.e.,
  \begin{align*}
    \norm{xy} &\leq \norm{x}\norm{y}.
  \end{align*}
  Two norms, $\norm{\cdot}_a$ and $\norm{\cdot}_b$, are said to be \textit{equivalent} if there exist constants $C_1$ and $C_2$ such that
  \begin{align*}
    \norm{x}_a &\leq C_1\norm{x}_b\\
    \norm{x}_b &\leq C_2 \norm{x}_a
  \end{align*}
  for all $x\in X$.\newline

  If $X$ is complete with respect to the metric $d\left(x,y\right) = \norm{x-y}$, we call $X$ a \textit{Banach space}. If $X$ is a normed algebra that is complete with respect to its induced metric, then we say $X$ is a \textit{Banach algebra}.
\end{definition}
There is a useful criterion for determining if a vector space is indeed a Banach space.
\begin{proposition}[{\cite[Proposition 1.3.2]{rainone_analysis}}]\label{prop:absolute_convergence_criterion}
  Let $X$ be a vector space. The following are equivalent:
  \begin{enumerate}[(i)]
    \item $X$ is a Banach space;
    \item for any set of vectors $\set{x_k}_{k=1}^{\infty}\subseteq X$, if $\sum_{k=1}\norm{x_k} < \infty$, then $\sum_{k=1}^{\infty}x_k = x$ for some $x\in X$;
    \item for any set of vectors $\set{x_k}_{k=1}^{\infty}$, if $\norm{x_k} < 2^{-k}$ for all $k$, then $\sum_{k=1}^{\infty}x_k = x$ for some $x\in X$.
  \end{enumerate}
\end{proposition}

\begin{theorem}\label{thm:quotient_space_norm}
  If $p\colon X\times X \rightarrow [0,\infty)$ is a seminorm on a vector space $X$, then 
  \begin{align*}
    N_p\coloneq \set{x\in X | p(x) = 0}
  \end{align*}
  is a subspace of $X$, and, defining $\norm{\cdot}_{X/N_p}$ by
  \begin{align*}
    \norm{x + N_p}_{X/N_p} &\coloneq p(x),
  \end{align*}
  this gives a norm on the quotient space $X/N_p$.
\end{theorem}
\begin{theorem}
  Let $X$ be a normed vector space. Then, $X$ is complete if and only if, for every sequence of vectors $\left(x_k\right)_k$, if$\sum_{k=1}^{\infty}\norm{x_k}$ converges, then $\sum_{k=1}^{\infty}x_k$ converges.
\end{theorem}
\begin{definition}[Open Balls, Closed Balls, Spheres]\label{def:open_closed_balls}
  Let $X$ be a normed vector space.
  \begin{itemize}
    \item We write 
      \begin{align*}
        U\left(x,\delta\right) &= \set{y\in X | \norm{y-x} < \delta}
      \end{align*}
      to be the open ball of radius $\delta$ centered at $x$. We write $U_X = U\left(0,1\right)$.
    \item We write
      \begin{align*}
        B\left(x,\delta\right) &= \set{y\in X | \norm{y-x} \leq \delta}
      \end{align*}
      to be the closed ball of radius $\delta$ centered at $x$. We write $B_X = B\left(0,1\right)$.
    \item We write
      \begin{align*}
        S\left(x,\delta\right) &= \set{y\in X | \norm{y-x} = \delta}
      \end{align*}
      to be the sphere of radius $\delta$ centered at $x$. We write $S_X = S\left(0,1\right)$.
  \end{itemize}
\end{definition}
\begin{definition}\label{def:total_subset}
  Let $X$ be a normed vector space. A subset $A\subseteq X$ is said to be \textit{total} if its closed linear span is equal to $X$; $\overline{\Span}\left(A\right) = X$.
\end{definition}
\begin{definition}
  Let $T\colon X\rightarrow Y$ be a linear map between normed vector spaces. We say $T$ is \textit{bounded} if its operator norm, defined by
  \begin{align*}
    \norm{T}_{\op} &= \sup_{x\in B_X}\norm{T\left(x\right)}
  \end{align*}
  is finite. We write $\B\left(X,Y\right)$ for the set of all bounded linear maps between $X$ and $Y$.\newline

  If $\norm{T}_{\op} \leq 1$, then we say $T$ is a \textit{contraction}.
\end{definition}
\begin{remark}
  Note that if $Y$ is complete, then $\B\left(X,Y\right)$ is a Banach space with pointwise addition and scalar multiplication.\newline

  A quick sketch of the proof is as follows: consider a $\norm{\cdot}_{\op}$-Cauchy sequence $\left(T_n\right)_n$ in $\B\left(X,Y\right)$, and define $T$ to be the pointwise limit of $\left(T_n\right)_n$, which exists as for any $y\in Y$, $\left(T_n\left(y\right)\right)_{y}$ is Cauchy in $Y$. Then, it can be shown that defining $T$ in this manner yields convergence in operator norm.\newline

  Furthermore, if we define $\B(X) = \B(X,X)$, then this space is a normed algebra with pointwise addition, scalar multiplication, and composition of operators. The algebra $\B(X)$ is complete if $X$ is complete.
\end{remark}
\begin{fact}\label{fact:continuity_of_linear_maps}
  The following are equivalent for a linear map $T\colon X\rightarrow Y$ on normed spaces:
  \begin{itemize}
    \item $T$ is continuous at $0$;
    \item $T$ is continuous;
    \item $T$ is uniformly continuous;
    \item $T$ is bounded.
  \end{itemize}
\end{fact}
\begin{definition}
  Let $T\colon X\rightarrow Y$ be a linear map between normed vector spaces.
  \begin{itemize}
    \item We say $T$ is \textit{bounded below} if there exists $C > 0$ such that $\norm{T(x)}\geq C\norm{x}$ for all $x\in X$.
    \item If $T$ is bounded and bounded below, we say $T$ is \textit{bicontinuous}.
    \item If $T$ is a linear isomorphism that is bicontinuous, we say $T$ is a \textit{bicontinuous isomorphism}, and say $X\cong Y$ are bicontinuously isomorphic.
    \item If $T$ is a linear isomorphism and is such that $\norm{T(x)} = \norm{x}$ for all $x$, then we say $T$ is an \textit{isometric isomorphism}.
  \end{itemize}
\end{definition}
\begin{definition}
  Let $X$ be a normed vector space. The subset $X^{\ast}\subseteq X'$, where $X'$ is the algebraic dual of $X$, is the set of all continuous linear functionals on $X$:
  \begin{align*}
    X^{\ast} &= \B\left(X,\F\right).
  \end{align*}
  We often call $X^{\ast}$ the \textit{continuous dual} of $X$.
\end{definition}
\begin{definition}[Generalized Summation]\label{def:unconditional_summability}
  If $\Omega$ is a set and $f\colon \Omega\rightarrow X$ is any function between $\Omega$ and suitable vector space $X$ (Definition \ref{def:tvs}), we say the unconditional series $\sum_{j\in\Omega}f(j)$ converges to some value $k\in X$ if the net $\left(s_{F}\right)_{F\in \mathcal{F}}$ converges to $k$, where
  \begin{align*}
    \mathcal{F} &= \set{F | F\subseteq J,\Card(F) < \infty}
  \end{align*}
  is the collection of finite subsets of $\Omega$ directed by inclusion, and
  \begin{align*}
    s_{F} &\coloneq \sum_{j\in\Omega}f(j).
  \end{align*}
\end{definition}
\begin{definition}[Three Fundamental Function Spaces]\label{def:three_function_spaces}
  Let $\Omega$ be any set.
  \begin{itemize}
    \item The space $\ell_{1}(\Omega)$ is the set of all functions $f\colon \Omega\rightarrow \C$ such that $\displaystyle \sum_{t\in\Omega}\left\vert f(t) \right\vert < \infty$.\newline

      The norm on $\ell_1\left( \Omega \right)$ is defined to be
      \begin{align*}
        \norm{f}_{\ell_1} &= \sum_{t\in\Omega}\left\vert f(t) \right\vert.
      \end{align*}
    \item The space $\ell_{2}(\Omega)$ is the set of all functions $f\colon \Omega\rightarrow \C$ such that $\displaystyle \sum_{t\in\Omega}\left\vert f(t) \right\vert^2 < \infty$.\newline

      The norm on $\ell_2\left( \Omega \right)$ is defined to be
      \begin{align*}
        \norm{f}_{\ell_2} &= \left( \sum_{t\in\Omega}\left\vert f(t) \right\vert^2 \right)^{1/2}.
      \end{align*}
    \item The space $\ell_{\infty}(\Omega)$ is the set of all functions $f\colon \Omega\rightarrow \C$ such that $\sup_{t\in\Omega}\left\vert f(t) \right\vert < \infty$.\newline

      The norm on $\ell_{\infty}(\Omega)$ is defined to be
      \begin{align*}
        \norm{f}_{\ell_{\infty}} &= \sup_{t\in\Omega}\left\vert f(t) \right\vert.
      \end{align*}
  \end{itemize}
  More generally, we define the space $\ell_p\left( \Omega \right)$, where $p\in [1,\infty)$, to be the set of functions $f\colon \Omega\rightarrow \C$ such that $\sum_{t\in\Omega}\left\vert f(t) \right\vert^p < \infty$. The norm on any of these $\ell_p$ spaces is defined to be
  \begin{align*}
    \norm{f}_{\ell_p} &= \left( \sum_{t\in\Omega}\left\vert f(t) \right\vert^{p} \right)^{1/p}.
  \end{align*}
\end{definition}
\begin{theorem}[Hölder's Inequality]\label{thm:holder_inequality}
  If $p$ and $q$ are such that $\frac{1}{p} + \frac{1}{q} = 1$, then if $f\in \ell_p$ and $g\in \ell_q$, we have that $fg\in \ell_1$, and
  \begin{align*}
    \norm{fg}_{\ell_1} &\leq \norm{f}_{\ell_p}\norm{g}_{\ell_q}.
  \end{align*}
\end{theorem}
\section{The Fundamental Theorems of Banach Spaces}%
\begin{definition}
  Let $X$ be a topological space. We say $X$ is a \textit{Baire space} if the intersection of any countable collection of open, dense subsets is also dense.
  %\begin{itemize}
  %  \item A subset $F\subseteq X$ is called nowhere dense if $\left(\overline{F}\right)^{\circ} = \emptyset$.
  %  \item If $X$ is a countable union of nowhere dense sets, we say $X$ is meager (or of the first category).
  %  \item We say $X$ is a Baire space if, for any countable collection of open, dense subsets $\set{A_n}_{n\geq 1}$, their intersection $\bigcap_{n\geq 1}A_n\subseteq X$ is also dense.
  %\end{itemize}
\end{definition}
The Baire category theorem serves as one of the central bridges between functional analysis and topology. Note that the property of being a Baire space is a purely topological definition, while completeness is an analytic concept.
\begin{theorem}[Baire Category Theorem]\label{thm:baire}
  Let $X$ be a complete metric space. Then, $X$ is a Baire space.
\end{theorem}
The Baire category theorem is used to prove many important theorems in functional analysis, such as the ones that follow. They all fundamentally rely on the completeness of Banach spaces, which is expressed through the Baire category theorem. Proofs for these theorems can be found in functional analysis textbooks such as \cite{rudin_functional_analysis}.
\begin{theorem}[Open Mapping Theorem]\label{thm:open_mapping}
  Let $T\colon X\rightarrow Y$ be a surjective linear map between Banach spaces. Then, $T$ is an open map --- i.e., if $U\subseteq X$ is open, then $V$ is also open.
\end{theorem}
\begin{corollary}[Bounded Inverse]\label{cor:bounded_inverse}
  If $T\colon X\rightarrow Y$ is a bounded linear map that is bijective, then $T^{-1}$ is also bounded.
\end{corollary}
\begin{theorem}[Closed Graph Theorem]\label{thm:closed_graph}
  Let $T\colon X\rightarrow Y$ be a linear map between Banach spaces. Then, $T$ is bounded if and only if $\graph\left(T\right) = \set{\left(x,T(x)\right)| x\in X} \subseteq X\times Y$ is closed in the product topology.
\end{theorem}
\begin{theorem}[Uniform Boundedness Principle]\label{thm:uniform_boundedness}
  Let $\set{T_i}_{i\in I}$ be a family of maps in $\B\left(X,Y\right)$ such that, for all $x\in X$, $\sup_{i\in I}\norm{T_i(x)} < \infty$. Then, $\sup_{i\in I}\norm{T_i}_{\op} < \infty$.
\end{theorem}
Now, we turn our attention towards linear functionals. Consider the following problem in algebra: suppose $X$ is a finite-dimensional vector space, and $Y\subseteq X$ is a subspace. If we have a linear functional $\varphi\colon Y\rightarrow \F$, can this linear functional be extended to the full space?\newline

The answer is yes --- if $\mathcal{B} = \set{x_1,\dots,x_m}$ is a basis for $Y$, since vector spaces are injective (Theorem \ref{thm:injective_projective_objects}), this basis can be extended to a basis for $X$, $\mathcal{C} = \set{x_1,\dots,x_m,x_{m+1},\dots,x_n}$; if we define $c_i = \varphi\left(x_i\right)$ for $1 \leq i \leq m$, we may define $\varphi\left(x_i\right) = 0$ for $m+1 \leq i \leq n$. In other words, we may always extend elements of the algebraic dual from a subspace to the full space.\newline

However, if $X$ is infinite-dimensional and equipped with a norm (or, more generally, a locally convex topology, as in Definition \ref{def:lctvs}), we also care about continuity, norm, and whether these extensions preserve continuity and norm. This is the domain of the Hahn--Banach theorems, which establish extension and separation results in normed vector spaces (and, as detailed later, locally convex topological vector spaces, as in Theorem \ref{thm:hb_continuous_extension_lctvs}).
\begin{definition}[Minkowski Functional]
  We call a map $p\colon X\rightarrow [0,\infty)$ a \textit{Minkowski functional} if
  \begin{itemize}
    \item $p\left(x + y\right)\leq p\left(x\right) + p\left(y\right)$ for all $x,y\in X$, and;
    \item $p\left(tx\right) = tp\left(x\right)$ for all $x\in X$ and $t > 0$.
  \end{itemize}
\end{definition}
\begin{theorem}[Hahn--Banach--Minkowski Extension]\label{thm:hbm_extension}
  Let $Y\subseteq X$ be a linear subspace of a normed vector space $X$, and let $\phi\in Y^{\ast}$ and $p$ a Minkowski functional be such that for all $y\in Y$, $\phi\left(y\right) \leq p\left(y\right)$. Then, there is a map $\Phi\in X^{\ast}$ such that
  \begin{itemize}
    \item $\Phi|_{Y} = \phi$, and;
    \item $\Phi\left(x\right) \leq p\left(x\right)$ for all $x\in X$.
  \end{itemize}
\end{theorem}
\begin{theorem}[Hahn--Banach Continuous Extension]\label{thm:hb_continuous_extension}
  Let $X$ be a normed vector space, and $\phi\in Y^{\ast}$, where $Y\subseteq X$ is a linear subspace. Then, there exists a linear functional $\Phi\in X^{\ast}$ such that $\norm{\Phi}_{\op} = \norm{\phi}_{\op}$, and $\Phi|_{Y} = \phi$. This extension is not necessarily unique.
\end{theorem}
The Hahn--Banach extension theorems lend themselves nicely to understanding the separation properties of linear functionals in the continuous dual space. These results allow us to know that there are ``enough'' linear functionals in the dual space of any normed vector space that allow us to distinguish points from closed subspaces and distinguish points from each other.
\begin{theorem}[Hahn--Banach Separation]\label{thm:hb_separation}
  Let $X$ be a normed vector space.
  \begin{itemize}
    \item For a fixed $x_0\in X$, there exists a linear functional $\phi\in X^{\ast}$ such that $\phi\left(x_0\right) = \norm{x_0}$.
    \item For a proper closed subspace $Y\subseteq X$ and some fixed $x_0\in X\setminus Y$, there is a $\phi\in X^{\ast}$ such that $\norm{\phi}_{\op} \leq 1$, $\phi|_{Y} = 0$, and $\phi\left(x_0\right) = \dist_{Y}\left(x_0\right)$.
  \end{itemize}
\end{theorem}
\begin{corollary}\label{cor:norm_from_functionals}
  Let $X$ be a normed space. For every $x\in X$, we have
  \begin{align*}
    \sup_{\phi\in B_{X^{\ast}}}\left\vert \varphi\left(x\right)  \right\vert = \norm{x}.
  \end{align*}
\end{corollary}
\section{Duality}%
Here, we discuss a little bit more of the theory of dual spaces.
\begin{definition}\label{def:double_dual_and_canonical_embedding}
  Let $X$ be a normed vector space. The linear functional $\hat{x}\colon X^{\ast}\rightarrow \C$, defined by
  \begin{align*}
    \hat{x}(\varphi) &= \varphi(x)
  \end{align*}
  is bounded with norm $\norm{\hat{x}}_{\op} = \norm{x}$. We define the embedding $\iota\colon X\hookrightarrow X^{\ast\ast}$ by
  \begin{align*}
    \iota(x) &= \hat{x}.
  \end{align*}
  We call $\iota$ the canonical embedding.
\end{definition}
\begin{definition}
  Let $X$ be a normed space. A norm \textit{completion} of $X$ is a pair $\left(Z,j\right)$, where $Z$ is a Banach space, $j\colon X\hookrightarrow Z$ is a linear isometry, and $\overline{\Ran}\left(j\right) = Z$.
\end{definition}
\begin{proposition}\label{prop:completion_existence}
  Let $X$ be a normed space, and set $\widetilde{X} = \overline{\iota_X\left(X\right)}^{\norm{\cdot}_{\op}}\subseteq X^{\ast\ast}$. Then, $\left(\widetilde{X},\iota_X\right)$ is a norm completion of $X$. Additionally, if $\left(Z,j\right)$ is any other norm completion of $X$, then there is an isometric isomorphism $Z\rightarrow \widetilde{X}$.
\end{proposition}
\begin{proposition}
  Let $X$ and $Y$ be normed spaces, and let $T\in \B\left(X,Y\right)$. Then, there is a unique $\widetilde{T}\in B\left(\widetilde{X},\widetilde{Y}\right)$ such that $\widetilde{T}\circ \iota_X = \iota_Y\circ T$. The diagram below commutes.
  \begin{center}
    % https://tikzcd.yichuanshen.de/#N4Igdg9gJgpgziAXAbVABwnAlgFyxMJZABgBoBGAXVJADcBDAGwFcYkQANEAX1PU1z5CKchWp0mrdgE0efEBmx4CRMsXEMWbRCAA6ugO5ZYeRrGAduc-kqFFR6mpqk79Rk1jMxg0q93EwUADm8ESgAGYAThAAtkgATDQ4EEgAzE6S2nqGxjCm5gAqfvJRsUhkIMlIohJa7AUgNIz0AEYwjAAKAsrCIJFYQQAWONYgpXGIFVWIibUu2fg49AD6XLwR0RM10+kgzW2d3XY6-UMjGXWuuosrsv7cQA
\begin{tikzcd}
\widetilde{X} \arrow[r, "\widetilde{T}"] & \widetilde{Y}           \\
X \arrow[r, "T"'] \arrow[u, "\iota_X"]   & Y \arrow[u, "\iota_Y"']
\end{tikzcd}
  \end{center}
  Furthermore, we have $\norm{T}_{\op} = \norm{\widetilde{T}}_{\op}$. If $T$ is isometric, then so is $\widetilde{T}$, and if $T$ is an isometric isomorphism, then so is $\widetilde{T}$.
\end{proposition}
\begin{definition}
  A normed space is called a \textit{dual space} if there is a normed space $Z$ such that $Z^{\ast} \cong X$ are isometrically isomorphic. We call $Z$ the \textit{predual} of $X$.
\end{definition}
\begin{example}\hfill
  \begin{itemize}
    \item We have $c_0^{\ast}\cong \ell_1$, where $c_0$ is the space of all sequence vanishing at infinity, and $\ell_1$ is the space of all absolutely summable sequences.
    \item We have $\ell_1^{\ast}\cong \ell_{\infty}$, where $\ell_{\infty}$ is space of all bounded sequences.
    \item If $\mu$ is a $\sigma$-finite measure on the measurable space $\left(\Omega,\mathcal{M}\right)$, and $p,q\in (1,\infty)$ are such that $p^{-1} + q^{-1} = 1$, then $L_{p}\left(\Omega,\mu\right)^{\ast} \cong L_q\left(\Omega,\mu\right)$. Here,
      \begin{align*}
        L_p\left(\Omega,\mu\right) &= \set{f\colon \Omega\rightarrow \C | \int_{\Omega}^{} \left\vert f \right\vert^p\:d\mu < \infty}.
      \end{align*}
      Additionally, if $\mu$ is semi-finite, then $L_1\left(\Omega,\mu\right)^{\ast}\cong L_{\infty}\left(\Omega,\mu\right)$.
  \end{itemize}
\end{example}
\begin{theorem}[Riesz--Markov Theorem]
  Let $\Omega$ be a LCH space. Then, $M_r\left(\Omega\right) \cong C_0\left(\Omega\right)^{\ast}$ are isometrically isomorphic, where $M_r\left(\Omega\right)$ is equipped with the norm $\norm{\mu} = \left\vert \mu \right\vert\left(\Omega\right)$ for $\mu\in M_r\left(\Omega\right)$.
\end{theorem}
\section{Topological Vector Spaces}%
Earlier, we discussed some of the features of normed vector spaces and Banach spaces. Here, we expand our scope to to examine the analytic properties of vector spaces whose topology is not necessarily induced by a norm. 
\begin{definition}\label{def:tvs}
  Let $X$ be a $\C$-vector space, and let $\tau$ be a topology on $X$. We say $\tau$ is \textit{compatible with the vector space structure} of $X$ if
  \begin{enumerate}[(1)]
    \item $X$ is T1 (Definition \ref{def:separation_axioms});
    \item scalar multiplication, $(\lambda,x)\mapsto \lambda x$ is continuous, where $\C\times X$ is given the product topology;
    \item vector addition, $(x,y) \mapsto x + y$ is continuous, where $X\times X$ is given the product topology.
  \end{enumerate}
  If $X$ is equipped with a topology compatible with the vector space structure of $X$, then $(X,\tau)$ is called a \textit{topological vector space}. We abbreviate it as TVS.
\end{definition}
\begin{remark}
  It can be shown that if $(X,\tau)$ is a TVS, the topology on $X$ is automatically Hausdorff.
\end{remark}
\begin{definition}\label{def:lctvs}
  A TVS $\left(X,\tau\right)$ is called \textit{locally convex} if $X$ admits a neighborhood base (Definition \ref{def:neighborhoods_and_bases}) consisting of convex sets (Definition \ref{def:vector_space_subset_operations}). We abbreviate it as LCTVS.
\end{definition}
\begin{remark}
It can be shown that every LCTVS has a neighborhood base consisting of \textsl{balanced} convex sets (Definition \ref{def:vector_space_subset_operations}).
\end{remark}
An important structural result in the theory of topological vector spaces is the fact that every locally convex topology is generated by a separating family of seminorms.
\begin{proposition}\label{prop:structure_of_lctvs}
  Let $X$ be a $\C$-vector space, and let $\mathcal{P}$ be a family of seminorms on $X$. For each $z\in X$ and $p\in \mathcal{P}$, we define $f_{p,z}\colon X\rightarrow [0,\infty)$ by
  \begin{align*}
    f_{p,z}(x) &= p\left(x-z\right).
  \end{align*}
  The topology $\tau_{\mathcal{P}}$ is the initial topology on $X$ induced by the family
  \begin{align*}
    \mathcal{F}_{\mathcal{P}} &= \set{f_{p,z} | p\in \mathcal{P},z\in X}.
  \end{align*}
  If $\mathcal{P}$ is such that for each $x\neq 0$, there is some $p$ such that $p(x)\neq 0$ (i.e., if $\mathcal{P}$ separates the points of $X$), then the family $\mathcal{F}_{\mathcal{P}}$ separates points in $X$. It is then the case that $\left(X,\tau_{\mathcal{P}}\right)$ is a LCTVS.\newline

  Convergence of nets in the topology $\tau_{\mathcal{P}}$ is defined by $\left(x_{\alpha}\right)_{\alpha}\xrightarrow{\tau_{\mathcal{P}}}x$ if and only if $p\left(x_{\alpha}-x\right)\rightarrow 0$ for all $p\in \mathcal{P}$.\newline

  Furthermore, if $\left(X,\tau\right)$ is any LCTVS, then there is a corresponding family of separating seminorms $\mathcal{P}$ such that $\id\colon\left(X,\tau\right)\rightarrow \left(X,\tau_{\mathcal{P}}\right)$ is a homeomorphism.
\end{proposition}
The Hahn--Banach theorems (such as the extension and separation results) that we established in Theorems \ref{thm:hbm_extension}, \ref{thm:hb_continuous_extension}, and \ref{thm:hb_separation} have corresponding results in topological vector spaces. The extension results are relatively straightforward.
\begin{theorem}[Hahn--Banach Continuous Extension for LCTVS]\label{thm:hb_continuous_extension_lctvs}
  Let $X$ be a LCTVS, and suppose $E\subseteq X$ is a subspace. If $\varphi\in E^{\ast}$, then there is a $\psi\in X^{\ast}$ such that $\psi|_{E} = \varphi$.
\end{theorem}
\begin{corollary}
  Let $X$ be a LCTVS. Let $\set{x_1,\dots,x_n}\subseteq X$ be linearly independent, and $\set{\alpha_1,\dots,\alpha_n}\in \C$. Then, there exists $\varphi\in X^{\ast}$ such that $\varphi\left(x_j\right) = \alpha_j$ for all $j$.
\end{corollary}
To provide some context for the Hahn--Banach separation results, consider two open, disjoint, convex subsets $A,B\subseteq \R^n$. The hyperplane separation theorem from convex optimization (see \cite[Chapter 2.6]{convex_optimization}) states that there is a nonzero vector $m\in \R^n$ and some $b\in \R$ such that the map $\varphi\colon \R^n\rightarrow \R$, defined by $\varphi(x) = m^{T}x - b$, is strictly negative for all $x\in A$ and is strictly positive for all $x\in B$. The affine hyperplane (Definition \ref{def:hyperplane}) defined by $\set{x | \varphi(x) = b}$ is known as a separating hyperplane for $A$ and $B$.\newline

What the Hahn--Banach theorems allow us to do is extend this result beyond $\R^n$ to the case of any TVS --- with a special case if the TVS is locally convex.
\begin{theorem}[Hahn--Banach Separation for TVS]\label{thm:hb_separation_tvs}
  Let $X$ be a TVS (that may or may not be locally convex) over $\C$. Let $A$ and $B$ be convex and disjoint subsets of $X$. If $A$ is open, then there exists $\varphi\in X^{\ast}$, with $\varphi = u + iv$, and $t\in \R$ such that
  \begin{align*}
    u(a) < t \leq u(b)
  \end{align*}
  for all $a\in A$ and $b\in B$.\newline

  If $A$ and $B$ are open, then the inequalities can be taken to be strict.
\end{theorem}
The requirement that $A$ and $B$ be open can be relaxed in the case of a LCTVS, where we can separate closed, disjoint, convex sets, so long as one of the sets is compact. Specifically, we are able to separate the sets by a double hyperplanes if the topology on $X$ is locally convex.
\begin{theorem}[Hahn--Banach Separation for LCTVS]\label{thm:hb_separation_lctvs}
  Let $X$ be a LCTVS, and suppose $C,K\subseteq X$ are closed, disjoint, convex subsets of $X$, with $K$ compact. Then, there exists $\varphi\in X^{\ast}$, with $\varphi = u + iv$, $t\in \R$, and $\delta > 0$ such that
  \begin{align*}
    u(x) \leq t \leq t + \delta \leq u(y)
  \end{align*}
  for all $x\in C$ and $y\in K$.
\end{theorem}
\begin{proposition}
  Let $W\subseteq X'$, where $X'$ is the algebraic dual of $X$. For each $\varphi\in W$, consider the seminorm
  \begin{align*}
    p_{\varphi}(x) &= \left\vert \varphi(x) \right\vert.
  \end{align*}
  We let $\mathcal{P}_{W} = \set{p_{\varphi} | \varphi\in W}$. If $\mathcal{P}_{W}$ separates points, then we may construct the topology $\tau_{\mathcal{P}_{W}}$ as in Proposition \ref{prop:structure_of_lctvs}.\newline

  Alternatively, we may consider the initial topology on $X$ induced by the family $W$, written $\sigma\left(X,W\right)$.\newline

  It is the case that $\id\colon \left(X,\tau_{\mathcal{P}_{W}}\right)\rightarrow \left(X,\sigma\left(X,W\right)\right)$ is a homeomorphism.
\end{proposition}
\begin{definition}[Norm Topology]\label{def:norm_topology}
  Let $X$ be a normed vector space. If $\mathcal{P} = \set{\norm{\cdot}}$, then the topology $\tau_{\mathcal{P}}$ is known as the norm topology on $X$.\newline

  Convergence is defined by $\left(x_n\right)_n\xrightarrow{\norm{\cdot}}x$ if and only if $\norm{x_n - x}\rightarrow 0$.
\end{definition}
\begin{remark}
  Normed vector spaces are metric space, and hence first countable (so sequences are sufficient to define convergence).
\end{remark}
\begin{definition}[Weak Topology]\label{def:weak_topology}
  If $X$ is a normed vector space, we say $\sigma\left(X,X^{\ast}\right)$ is the \textit{weak topology} on $X$.\newline

  Convergence is defined by $\left(x_\alpha\right)_\alpha\xrightarrow{w}x$ if and only if $\left(\varphi\left(x_\alpha\right)\right)_\alpha\rightarrow \varphi\left(x\right)$ for all $\varphi\in X^{\ast}$.
\end{definition}
\begin{definition}[Weak* Topology]\label{def:weak_star_topology}
  If $X$ is a normed vector space, we say $\sigma\left(X^{\ast},\iota(X)\right)$, where $\iota$ is the canonical embedding (see \ref{def:double_dual_and_canonical_embedding}), is the \textit{weak* topology} on $X^{\ast}$.\newline

  Convergence is defined by $\left(\varphi_{\alpha}\right)_\alpha\xrightarrow{w^{\ast}} \varphi$ if and only if $\left(\varphi_{\alpha}(x)\right)_{\alpha}\rightarrow \varphi(x)$ for all $x\in X$.
\end{definition}
\begin{theorem}[Banach--Alaoglu Theorem]\label{thm:banach_alaoglu}
  Let $X$ be a normed vector space. 
  \begin{enumerate}[(1)]
    \item The unit ball in the dual space, $B_{X^{\ast}}$, is $w^{\ast}$-compact.
    \item A subset $C\subseteq X^{\ast}$ is $w^{\ast}$-compact if and only if $C$ is $w^{\ast}$-closed and norm bounded.
  \end{enumerate}
\end{theorem}
\section{Hilbert Spaces and Operators}%
In Chapters \ref{ch:left_regular_representation} and \ref{ch:nuclearity}, we discuss the relationship between a group $\Gamma$ and the way the group is represented as an algebra of bounded operators on a Hilbert space. Here, we discuss more exactly what is meant by ``algebra of bounded operators on a Hilbert space.''
\begin{definition}\label{def:hilbert_spaces}
  Let $X$ be a vector space. A \textit{semi-inner product} on $X$ is a map $ \iprod{\cdot}{\cdot}\colon X\times X \rightarrow \C $ such that
  \begin{itemize}
    \item $ \iprod{\alpha x + y}{z} = \alpha \iprod{x}{z} + \iprod{y}{z}$;
    \item $ \iprod{x}{\alpha y + z} = \overline{\alpha}\iprod{x}{y} + \iprod{x}{z}$;
    \item $ \iprod{x}{x}\geq 0 $.
  \end{itemize}
  If $ \iprod{x}{x} = 0 $ if and only if $ x = 0 $, then $ \iprod{\cdot}{\cdot} $ is an \textit{inner product} with induced norm $\norm{x}^2 = \iprod{x}{x}$. We call $X$ an \textit{inner product space} if it is equipped with an inner product.\newline

  If $X$ is an inner product space that is complete with respect to the induced norm, then we say $X$ is a \textit{Hilbert space}. We usually denote Hilbert spaces by $ \mathcal{H} $.
\end{definition}
There are a few important structural results that are established in linear algebra relating to inner product spaces. We list a couple that are used extremely often. These facts are used implicitly throughout the study of Hilbert spaces and operators on them.
\begin{theorem}[Polarization Identity]\label{thm:polarization}
  Let $X$ be an inner product space, and let $x,y\in X$. Then,
  \begin{align*}
    \iprod{x}{y} &= \frac{1}{4}\sum_{k=0}^{3} i^k\norm{x + i^ky}^2.
  \end{align*}
\end{theorem}
\begin{theorem}[Parallelogram Law]
  Let $X$ be an inner product space, and let $x,y\in X$. Then,
  \begin{align*}
    \norm{x-y}^2 + \norm{x+y}^2 &= 2\norm{x}^2 + 2\norm{y}^2.
  \end{align*}
\end{theorem}
\begin{theorem}[Cauchy--Schwarz Inequality]
  Let $X$ be an inner product space, and let $x,y\in X$. Then,
  \begin{align*}
    \left\vert \iprod{x}{y} \right\vert &\leq \norm{x}\norm{y}.
  \end{align*}
\end{theorem}
\begin{lemma}
  If $X$ is any inner product space and $F\colon X\times X \rightarrow \C$ is a sesquilinear form --- i.e., one that satisfies the following properties for all $x,x_1,x_2,y,y_1,y_2\in X$ and $\alpha\in\C$:
  \begin{itemize}
    \item $F\left(\alpha x_1+x_2,y\right) = \alpha F\left(x_1,y\right) + F\left(x_2,y\right)$;
    \item $F\left(x,y_1 + \alpha y_2\right) = F\left(x,y_1\right) + \overline{\alpha}F\left(x,y_2\right)$;
  \end{itemize}
  then there exists some $T\in \mathcal{L}\left(X\right)$ such that
  \begin{align*}
    F\left(x,y\right) &= \iprod{T\left(x\right)}{y},
  \end{align*}
  where $ \iprod{\cdot}{\cdot} $ is the inner product on $X$.
\end{lemma}
\begin{remark}
Note that with these three results, we can show that any two sesquilinear forms $F$ and $G$ are equal along the diagonal --- i.e., if $F(x,x) = G(x,x)$ for all $x$ --- then they are equal everywhere. Additionally, the Cauchy--Schwarz inequality for sesquilinear forms is expressed as
\begin{align*}
  \left\vert F\left(x,y\right) \right\vert &\leq \left\vert F\left(x,x\right) \right\vert^{1/2}\left\vert F\left(y,y\right) \right\vert^{1/2}.
\end{align*}

\end{remark}
\begin{definition}
  Let $\mathcal{H}$ be a Hilbert space. A subset $\set{x_i}_{i\in I}$ is called \textit{orthonormal} if 
  \begin{align*}
    \iprod{x_i}{x_j} &= \begin{cases}
      1 & i=j\\
      0 & i\neq j
    \end{cases}
  \end{align*}
  A maximal orthonormal set in $\mathcal{H}$ is called an \textit{orthonormal basis}; equivalently, the set $\set{x_i}_{i\in I}$ is an orthonormal basis if and only if $\Span\left(\set{x_i}_{i\in I}\right)$ is dense in $\mathcal{H}$.
\end{definition}
\begin{example}\label{ex:orthonormal_bases}
  Considering the function space $\ell_2\left( \Omega \right)$, the orthonormal basis is the set $\set{\delta_t}_{t\in\Omega}$, where
  \begin{align*}
    \delta_{t}\left( s \right) &=  \begin{cases}
      1 & t = s\\
      0 & t\neq s
    \end{cases}.
  \end{align*}
  Similarly, the space $\ell_2\left( \Z \right)$ has the orthonormal basis of $\set{e_n}_{n\in\Z}$, where $e_n = 1$ at index $n$ and $0$ elsewhere.
\end{example}
\begin{remark}
  Every Hilbert space has an orthonormal basis. This can be found by applying Zorn's lemma on the partially ordered set of all orthonormal subsets of $\mathcal{H}$ ordered by inclusion.
\end{remark}
\begin{theorem}[Bessel's Inequality and Parseval's Identity]
  Let $\set{e_i}_{i\in I}$ be an orthonormal set in a Hilbert space $\mathcal{H}$. Then, for any $x\in \mathcal{H}$,
  \begin{align*}
    \sum_{i\in I} \left\vert \iprod{x}{e_i} \right\vert^2 &\leq \norm{x}^2.
  \end{align*}
  If $\set{e_i}_{i\in I}$ is an orthonormal basis, then
  \begin{align*}
    \sum_{i\in I} \left\vert \iprod{x}{e_i} \right\vert^2 &= \norm{x}^2.
  \end{align*}
\end{theorem}
\begin{theorem}\label{thm:projection_theorem}
  If $M\subseteq \mathcal{H}$ is a closed subspace of a Hilbert space $\mathcal{H}$, then for any $x\in \mathcal{H}$, there is a unique $y_x\in M$ such that $\norm{x-y_x}$ is minimal.\newline

  The map $P_M\colon \mathcal{H}\rightarrow M$, $x\mapsto y_x$ is known as the \textit{orthogonal projection} onto $M$. Additionally, the map $P_M$ has the following properties:
  \begin{itemize}
    \item $P_M$ is linear;
    \item $P_M^2 = P_M$;
    \item $\norm{P_M}_{\op} = 1$ (if $M = \set{0}$, then $\norm{P_M}_{\op} = 0$).
  \end{itemize}
  Setting $M^{\perp}$ to be the range of $I_{\mathcal{H}} - P_M$, it is also the case that $\mathcal{H}/M \cong M^{\perp}$, with $\mathcal{H} = M\oplus M^{\perp}$.
\end{theorem}
One of the most important structural results on Hilbert spaces relates the continuous dual of a Hilbert space to the inner product. 
\begin{theorem}[Riesz Representation Theorem for Hilbert Spaces]\label{thm:riesz_hilbert_spaces}
  Let $\mathcal{H}$ be a Hilbert space, and let $\varphi\in \mathcal{H}^{\ast}$. Then, there is a unique $f_{\varphi}\in \mathcal{H}$ such that
  \begin{align*}
    \varphi\left(g\right) &= \iprod{g}{f_{\varphi}}
  \end{align*}
  for all $g\in \mathcal{H}$.
\end{theorem}
\begin{definition}
  Let $T\in \B\left( \mathcal{H} \right)$ be an operator, and let $\mathcal{H} = \bigoplus_{i\in I}M_i$ be an internal direct sum decomposition. Define $P_i$ to be the orthogonal projection onto $M_i$. Defining $T_{ij}\coloneq P_i T P_j$, we define
  \begin{align*}
    \left[ T \right]_{\mathcal{M}}\left( \left( x_i \right)_{i} \right) &\coloneq \left( \sum_{j\in I}T_{ij}\left( x_j \right) \right)_{i}
  \end{align*}
  to be the \textit{matrix representation of $T$}.\newline

  There is a unitary map (Definition \ref{def:distinguished_operators}) $U\colon \bigoplus_{i\in I}M_i\rightarrow \mathcal{H}$ such that $U\left[ T \right]_{\mathcal{M}} U^{\ast} = T$ --- i.e., the matrix representation of $T$ is unitarily equivalent to $T$.
\end{definition}

Now that we understand the structure of Hilbert spaces and their closed subspaces, we can now begin understanding bounded operators on Hilbert spaces.
\begin{definition}\label{def:adjoint_properties}
  Let $T\colon \mathcal{H}\rightarrow \mathcal{H}$ be a bounded operator between Hilbert spaces. We define the \textit{adjoint} of $T$ to be the unique operator $T^{\ast}\colon \mathcal{H}\rightarrow \mathcal{H}$ such that
  \begin{align*}
    \iprod{T\left(x\right)}{y} &= \iprod{x}{T^{\ast}\left(y\right)}
  \end{align*}
  for all $x,y\in \mathcal{H}$. The adjoint satisfies the following properties:
  \begin{itemize}
    \item $\left(T + \lambda S\right)^{\ast} = T^{\ast} + \overline{\lambda}S^{\ast}$;
    \item $T^{\ast\ast} = T$;
    \item $\left(R\circ T\right)^{\ast} = T^{\ast}\circ R^{\ast}$;
    \item if $T$ is invertible, then $\left(T^{-1}\right)^{\ast} = \left(T^{\ast}\right)^{-1}$;
    \item $\norm{T^{\ast}}_{\op} = \norm{T}_{\op}$;
    \item $\norm{T^{\ast}T}_{\op} = \norm{T}_{\op}^2$ (known as the $C^{\ast}$-property).
  \end{itemize}
\end{definition}
\begin{lemma}\label{lemma:operator_norm_hilbert_space}
  If $\mathcal{H}$ is a Hilbert space, and $T\in \B\left( \mathcal{H} \right)$ is a bounded linear operator, then
  \begin{align*}
    \norm{T}_{\op} &= \sup_{x,y\in S_{\mathcal{H}}} \left\vert \iprod{T\left( x \right)}{y} \right\vert.
  \end{align*}
  
\end{lemma}

There are a variety of topologies one can place on the space $\B\left(\mathcal{H}\right)$. We detail three.
\begin{definition}\label{def:operator_topologies}
  Let $\left(T_{\alpha}\right)_{\alpha}$ be a net in $\B\left(\mathcal{H}\right)$.
  \begin{itemize}
    \item We say $\left(T_{\alpha}\right)_{\alpha}\xrightarrow{\norm{\cdot}_{\op}} T$ if $\norm{T_{\alpha} - T}_{\op}\rightarrow 0$. This is the \textit{norm topology} on $\B\left(\mathcal{H}\right)$.
    \item We say $\left(T_{\alpha}\right)_{\alpha}\xrightarrow{ \text{SOT} } T$ if, for all $\xi\in \mathcal{H}$, $\norm{T_{\alpha}\left(\xi\right) - T\left(\xi\right)}\rightarrow 0$. This is the \textit{strong operator topology} (or topology of pointwise convergence) on $\B\left(\mathcal{H}\right)$.
    \item We say $\left(T_{\alpha}\right)_{\alpha}\xrightarrow{\text{WOT}}T$ if, for all $\xi,\eta\in \mathcal{H}$, $ \iprod{T_{\alpha}\left(\xi\right)}{\eta}\rightarrow \iprod{T\left(\xi\right)}{\eta} $. This is the \textit{weak operator topology} on $\B\left(\mathcal{H}\right)$.
  \end{itemize}
\end{definition}
\begin{definition}\label{def:distinguished_operators}\hfill
  \begin{itemize}
    \item We say $T\in \B\left(\mathcal{H}\right)$ is \textit{normal} if $T^{\ast}T = TT^{\ast}$.
    \item We say $T\in \B\left(\mathcal{H}\right)$ is \textit{self-adjoint} if $T^{\ast}= T$. We write $\B\left(\mathcal{H}\right)_{\sa}$ to refer to the set of all self-adjoint operators in $\B\left(\mathcal{H}\right)$.
    \item We say $P\in \B\left(\mathcal{H}\right)$ is a \textit{projection} if $P^2 = P^{\ast} = P$.
    \item We say $V\in \B\left(\mathcal{H}\right)$ is an \textit{isometry} if $V^{\ast}V = I_{\mathcal{H}}$.
    \item We say $T\in \B\left(\mathcal{H}\right)$ is a \textit{partial isometry} if $TT^{\ast}T = T$.
    \item We say $U\in \B\left(\mathcal{H}\right)$ is a \textit{unitary} if $U^{\ast} = U^{-1}$.
  \end{itemize}
  We write $\mathcal{U}\left(\mathcal{H}\right)$ to refer to the set of all unitary operators on $\mathcal{H}$. Two operators $T,S\in \B\left(\mathcal{H}\right)$ are called \textit{unitarily equivalent} if there is $U\in \mathcal{U}\left(\mathcal{H}\right)$ such that $UTU^{\ast} = S$.\newline

  The space of unitary operators, $\mathcal{U}\left(\mathcal{H}\right)$, is a group with respect to operator composition.
\end{definition}
The set $\B\left(\mathcal{H}\right)_{\sa}$ admits an order structure.
\begin{definition}\label{def:positive_operators}
  Let $T\in \B\left(\mathcal{H}\right)_{\sa}$. We say $T$ is \textit{positive} if, for every $\xi\in \mathcal{H}$, we have
  \begin{align*}
    \iprod{T\left(\xi\right)}{\xi} &\geq 0.
  \end{align*}
  We write $\B\left(\mathcal{H}\right)_{+}$ to refer to the operator norm-closed cone of positive operators in $\B\left(\mathcal{H}\right)_{\sa}$.\newline

  If $T,S\in \B\left(\mathcal{H}\right)_{\sa}$, we say $T\geq S$ if $T-S \in \B\left(\mathcal{H}\right)_{+}$.
\end{definition}
\begin{remark}\label{rem:positive_operators_definition}
  It can be shown that an operator $T\in \B\left(\mathcal{H}\right)_{+}$ if and only if there is some $S\in \B\left(\mathcal{H}\right)$ such that $T = S^{\ast}S$.
\end{remark}
\begin{definition}\label{def:rank_one_bounded_operator}
  Let $x,y\in \mathcal{H}$. We define the \textit{rank-one bounded operator} $\theta_{x,y}\colon \mathcal{H}\rightarrow \mathcal{H}$ by
  \begin{align*}
    \theta_{x,y}(z) &= \iprod{z}{y}x.
  \end{align*}
  If $T\in \B\left(\mathcal{H}\right)$ is such that
  \begin{align*}
    T &= \sum_{j=1}^{n}\theta_{x_j,y_j},
  \end{align*}
  where $x_j,y_j\in \mathcal{H}$, then $T$ is of \textit{finite rank} --- i.e., $\Dim\left(\Ran\left(T\right)\right) < \infty$. We write $T\in \F\left(\mathcal{H}\right)$.\newline

  A map $T\in \B\left(\mathcal{H}\right)$ is called \textit{compact} if $T$ maps bounded sets to sets with compact closure. The space of compact operators is written $\mathbb{K}\left(\mathcal{H}\right)$.
\end{definition}
\begin{theorem}
  The operator norm-closure of the finite rank operators on a Hilbert space $\mathcal{H}$ is the compact operators. That is,
  \begin{align*}
    \overline{\F\left(\mathcal{H}\right)}^{\norm{\cdot}_{\op}} &= \mathbb{K}\left(\mathcal{H}\right).
  \end{align*}
\end{theorem}

% Get rid of the purely algebraic content in here, only focus on the analysis.
\chapter{Operator Algebras}\label{ch:operator_algebras}
In Chapter \ref{ch:nuclearity}, we will establish that the amenability of a group is equivalent to a property known as nuclearity held by the $C^{\ast}$-algebra(s) generated by the group. For this, we need a solid background in the theory of operator algebras --- specifically, in Banach algebras and $C^{\ast}$-algebras.
\section{Definitions and Examples}%
The theory of $C^{\ast}$-algebras is motivated by the fact that the adjoint operation on $\B\left( \mathcal{H} \right)$ (Definition \ref{def:adjoint_properties}) satisfies the criteria for an involution (Definition \ref{def:algebra_star_algebra}) on an algebra. However, one property that $\B\left( \mathcal{H} \right)$ has that a pure $\ast$-algebra lacks is the fact that $\B\left( \mathcal{H} \right)$ is equipped with a norm, $\norm{\cdot}_{\op}$, that turns $\B\left( \mathcal{H} \right)$ into a normed algebra (Definition \ref{def:norms}).\newline

What the theory of $C^{\ast}$-algebras allows us to do is abstract away from $\B\left( \mathcal{H} \right)$. Soon, we will see that this abstraction will allow us to focus on purely algebraic properties of $C^{\ast}$-algebras and establish fundamental analytic results on them.
\begin{definition}\label{def:banach_star_algebra}
  Let $A$ be an algebra.
  \begin{itemize}
    \item If $\norm{\cdot}$ is such that $\left( A,\norm{\cdot} \right)$ is a Banach space that satisfies $\norm{ab}\leq \norm{a}\norm{b}$ for all $a,b\in A$, then we say $\left( A,\norm{\cdot} \right)$ is a \textit{Banach algebra}.
    \item If $A$ is a $\ast$-algebra that is also a Banach algebra, and the norm on $A$ satisfies $\norm{a^{\ast}} = \norm{a}$ for all $a\in A$, then we say $A$ is a \textit{Banach $\ast$-algebra}.
    \item If $A$ is a Banach $\ast$-algebra whose norm also satisfies $\norm{a^{\ast}a} = \norm{a}^2$ for all $a\in A$, then we say $A$ is a \textit{$C^{\ast}$-algebra}. This final property is known as the $C^{\ast}$-property.
  \end{itemize}
\end{definition}
There are many $C^{\ast}$-algebras that we interact with as we study analysis.
\begin{example}\hfill
  \begin{itemize}
    \item The complex numbers, $\C$, equipped with the involution $z\mapsto \overline{z}$, are a $C^{\ast}$-algebra under the norm $\left\vert z \right\vert$.
    \item If $\mathcal{H}$ is a Hilbert space, then $\B\left( \mathcal{H} \right)$ is a $C^{\ast}$-algebra under the operator norm with the involution $T\mapsto T^{\ast}$.
    \item The space of $n\times n$ complex matrices, $\Mat_n\left( \C \right)$ under the operator norm and the involution $\left( a_{ij}^{\ast} \right)_{ij} = \left( \overline{a_{ji}} \right)_{ij}$ is a $C^{\ast}$-algebra.
    \item If $\Omega$ is any nonempty set, then the space of bounded functions, $\ell_{\infty}\left( \Omega \right)$, is a $C^{\ast}$-algebra under the norm $\norm{f}_{\ell_{\infty}} = \sup_{x\in\Omega}\left\vert f(x) \right\vert$ and the involution $f^{\ast}\left( x \right) = \overline{f(x)}$.
  \end{itemize}
\end{example}
However, there are also some Banach $\ast$-algebras that are not $C^{\ast}$-algebras.
\begin{example}
  Let 
  \begin{align*}
    \ell_1\left( \Z \right)\coloneq \set{f\colon \Z\rightarrow\C | \norm{f}_{\ell_1} \coloneq \sum_{n\in\Z}\left\vert f(n) \right\vert < \infty}
  \end{align*}
  be equipped with the involution
  \begin{align*}
    f^{\ast}\left( n \right) &= \overline{f\left( -n \right)}
  \end{align*}
  and multiplication
  \begin{align*}
    f\ast g(n) &= \sum_{k\in\Z}f(n-k)g(k).
  \end{align*}
  Then, $\ell_1(\Z)$ is a Banach $\ast$-algebra that does not satisfy the $C^{\ast}$-property.
\end{example}
The rest of this section will focus on understanding properties of $C^{\ast}$-algebras and their elements.
\section{$C^{\ast}$-Norms and Universal $C^{\ast}$-Algebras}%
We begin by constructing $C^{\ast}$-algebras.\newline

Recall that, in the case of a normed vector space, we know that (Proposition \ref{prop:completion_existence}) there is always a completion of $X$ into a Banach space, $\widetilde{X}\coloneq \overline{\iota_X\left( X \right)}^{\norm{\cdot}_{\op}}\subseteq X^{\ast\ast}$. This extends to the case of normed algebras/$\ast$-algebras and Banach algebras/Banach $\ast$-algebras.
\begin{lemma}[{\cite[Lemma 7.2.26]{rainone_analysis}}]\label{lemma:banach_algebra_completion}
  If $A_0$ is a normed algebra/$\ast$-algebra, then its Banach space completion, $A$, is a Banach algebra/Banach $\ast$-algebra. The inclusion, $A_0\hookrightarrow A$ is an isometric homomorphism/$\ast$-homomorphism of algebras/$\ast$-algebras.
\end{lemma}
If we have a normed algebra $A$ and we want its completion to be a $C^{\ast}$-algebra, then we need the norm itself to have properties analogous to the norm on a $C^{\ast}$-algebra.
\begin{definition}[{\cite[Definition 7.2.27]{rainone_analysis}}]
  Let $A_0$ be a $\ast$-algebra. A \textit{$C^{\ast}$-norm}/\textit{$C^{\ast}$-seminorm} on $A_0$ is a norm/seminorm on $A_0$ satisfying the following:
  \begin{enumerate}[(i)]
    \item $\norm{ab}\leq \norm{a}\norm{b}$;
    \item $\norm{a^{\ast}} = \norm{a}$;
    \item $\norm{a^{\ast}a} = \norm{a}^2$ (also known as the \textit{$C^{\ast}$-property})
  \end{enumerate}
  for all $a,b\in A_0$.
\end{definition}
We're able to construct $C^{\ast}$-norms by using $\ast$-homomorphisms into $C^{\ast}$-algebras.
\begin{lemma}[{\cite[Lemma 7.2.30]{rainone_analysis}}]
  Let $A_0$ be a $\ast$-algebra, and suppose $\phi\colon A_0\rightarrow B$ is a $\ast$-homomorphism into a $C^{\ast}$-algebra $B$. Then,
  \begin{align*}
    \norm{a}_{\phi} &= \norm{\phi(a)}_{\op}
  \end{align*}
  defines a $C^{\ast}$-seminorm on $A_0$. If $\phi$ is injective, then $\norm{\cdot}_{\phi}$ is a $C^{\ast}$-norm.
\end{lemma}
Just as in the case of Lemma \ref{lemma:banach_algebra_completion}, the completion of a normed algebra with a $C^{\ast}$-norm yields a $C^{\ast}$-algebra.
\begin{lemma}[{\cite[Lemma 7.2.32]{rainone_analysis}}]
  Let $\norm{\cdot}$ be a $C^{\ast}$-norm on a $\ast$-algebra $A_0$. The norm completion, $A$, is a $C^{\ast}$-algebra, and the inclusion $A_0\hookrightarrow A$ is an isometric $\ast$-homomorphism.
\end{lemma}
Recall that any seminorm on a vector space gives rise to a norm on the quotient space (Theorem \ref{thm:quotient_space_norm}) --- similarly, we may define the enveloping $C^{\ast}$-algebra on any $C^{\ast}$-seminorm on $A$.
\begin{definition}[{\cite[Definition 7.2.33]{rainone_analysis}}]\label{def:hausdorff_completion}
  Let $A_0$ be a $\ast$-algebra equipped with a $C^{\ast}$-seminorm $p$. The norm completion of $A/N_p$ with respect to $\norm{\cdot}_{A/N_p}$, where
  \begin{align*}
    N_p \coloneq \set{a\in A | p(a) = 0}
    \intertext{and}
    \norm{a + N_p} &= p(a),
  \end{align*}
  is known as the \textit{Hausdorff completion} or \textit{enveloping $C^{\ast}$-algebra} of $\left( A_0,p \right)$.
\end{definition}
We want to now understand a sort of ``maximal'' enveloping $C^{\ast}$-algebra --- preferably one that admits a universal property, similar to the universal property for the free $\ast$-algebra of Theorem \ref{thm:universal_property_free_algebra}. This will be the universal $C^{\ast}$-algebra.
\begin{definition}[{\cite[Definition 7.2.34]{rainone_analysis}}]
  Let $A_0$ be a $\ast$-algebra, and let $\mathcal{P}$ denote the collection of all $C^{\ast}$-seminorms on $A_0$. Set
  \begin{align*}
    \norm{a}_{u} &= \sup_{p\in \mathcal{P}}p(a).
  \end{align*}
  If $\norm{a}_u < \infty$ for all $a\in A_0$, then $\norm{\cdot}_u$ defines a $C^{\ast}$-seminorm on $A_0$ called the \textit{universal $C^{\ast}$-seminorm} on $A_0$.\newline

  The \textit{universal enveloping $C^{\ast}$-algebra} of $A_0$ is the enveloping $C^{\ast}$-algebra of the pair $\left( A_0,\norm{\cdot}_u \right)$.
\end{definition}
We can also define a universal $C^{\ast}$-algebra with respect to a set of relations $R$ with a similar universal property. 
\begin{definition}[{\cite[Definition 7.2.35]{rainone_analysis}}]
  Let $E$ be a set of abstract variables and suppose $R\subseteq \mathbb{A}^{\ast}\left( E \right)$ is a collection of relations. If the universal enveloping $C^{\ast}$-algebra of $\mathbb{A}^{\ast}\left( E|R \right)$ exists --- i.e., that $\norm{a}_u < \infty$ for all $a\in \mathbb{A}^{\ast}\left( E|R \right)$ --- we denote it $C^{\ast}\left( E|R \right)$. It is know as the \textit{universal $C^{\ast}$-algebra with generators $E$ and relations $R$}.
\end{definition}
\begin{proposition}[{\cite[Proposition 7.2.36]{rainone_analysis}}]\label{prop:universal_property_universal_cstar_algebra}
  Let $E = \set{x_i}_{i\in I}$ be a set of abstract variables, and let $R\subseteq \mathbb{A}^{\ast}\left( E \right)$ be a collection of relations. Suppose the universal $C^{\ast}$-algebra $C^{\ast}\left( E|R \right)$ exists.\newline

  If $B$ is a $C^{\ast}$-algebra admitting elements $\set{b_i}_{i\in I}$ that satisfy the relations $R$, then there is a unique contractive $\ast$-homomorphism, $\varphi_B\colon C^{\ast}\left( E|R \right) \rightarrow B$, such that
  \begin{align*}
    \varphi_b\left( v_i \right) = b_i,
  \end{align*}
  where $v_i \coloneq \left( x + I(R) \right) + N_u$ is a double coset with $I(R)$ as the ideal generated by the $R$ and $N_u$ is the zero set of $\norm{\cdot}_u$, as in Definition \ref{def:hausdorff_completion}.
\end{proposition}
\section{Representations and the Group $C^{\ast}$-Algebras}%
We can realize $\ast$-algebras as $\ast$-subalgebras of bounded operators on Hilbert space.\footnote{In fact, via the GNS construction (which we apologetically cannot cover here), every $C^{\ast}$-algebra can be realized as a $\ast$-subalgebra of $\B\left( \mathcal{H} \right)$ for a suitable Hilbert space $\mathcal{H}$.} 
\begin{definition}[{\cite[Definition 7.2.41]{rainone_analysis}}]
  Let $A_0$ be a $\ast$-algebra. A \textit{representation} of $A_0$ is a pair $\left( \pi_0,\mathcal{H} \right)$, where $\pi_0\colon A_0\rightarrow \B\left( \mathcal{H} \right)$ is a $\ast$-homomorphism.\newline

  If $A_0$ is unital, and $\pi_0\left( 1_A \right) = I_{\mathcal{H}}$, then we say $\pi_0$ is a \textit{unital} representation.
\end{definition}
What makes representations special is that they give us a $C^{\ast}$-norm ``for free'' in a sense.
\begin{lemma}[{\cite[Lemma 7.2.42]{rainone_analysis}}]
  Let $A_0$ be a $\ast$-algebra, and let $\left( \pi_0,\mathcal{H} \right)$ be a representation of $A_0$. Then,
  \begin{align*}
    \norm{a}_{\pi_0} &= \norm{\pi_0\left( a \right)}_{\op}
  \end{align*}
  is a $C^{\ast}$-seminorm on $A_0$. If $\pi_0$ is injective, then $\norm{\cdot}_{\pi_0}$ is a $C^{\ast}$-norm.
\end{lemma}
\begin{lemma}[{\cite[Lemma 7.2.43]{rainone_analysis}}]
  If $A_0$ and $B_0$ are normed $\ast$-algebras with completions $A$ and $B$, then any bounded $\ast$-homomorphism extends continuously to $\varphi\colon A\rightarrow B$.
\end{lemma}
\begin{corollary}[{\cite[Corollary 7.2.44]{rainone_analysis}}]
  Let $A_0$ be a $\ast$-algebra, and let $\pi\colon A_0\rightarrow \B\left( \mathcal{H} \right)$ be an injective representation.\newline

  The completion $A$ of $A_0$ with respect to the $C^{\ast}$-norm $\norm{\cdot}_{\pi_0}$ is a $C^{\ast}$-algebra, and the continuous extension $\pi\colon A\rightarrow \B\left( \mathcal{H} \right)$ is an isometric $\ast$-homomorphism.
\end{corollary}

\section{Spectra of Elements in $C^{\ast}$-Algebras}%
\section{Characters of $C^{\ast}$-Algebras}%
\section{The Continuous Functional Calculus}%

\nocite{*}
\printbibliography[heading=bibintoc,title={References}]
\end{document}
