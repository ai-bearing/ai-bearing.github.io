\documentclass[10pt]{package2}

% sans serif font:
%\usepackage[math]{iwona}
%\usepackage[cal=lucida,calscaled=0.96]{mathalpha}
%\usepackage{cmbright}
%\usepackage{eucal}
%\usepackage{sfmath}
%\usepackage{bbold} %better blackboard bold

%serif font + different blackboard bold for serif font
%\usepackage{newpxtext,eulerpx,eucal}
\usepackage[light]{kpfonts}
\newcommand{\1}{\mathds{1}}
%\newcommand{\1}{\mathbb{1}}
%\usepackage{chancery}
%\usepackage{gfsartemisia-euler}
%\renewcommand{\textlozenge}{}
%\renewcommand*{\mathbb}[1]{\varmathbb{#1}}
%\renewcommand*{\hbar}{\hslash}
\usepackage{titlesec}
\usepackage[backend=biber,style=alphabetic,sorting=ynt]{biblatex}
\addbibresource{chapters/references.bib}
\DeclareMathOperator{\op}{op}
\renewcommand{\coloneq}{=}


%\setcounter{secnumdepth}{4}
%\titleformat{\paragraph}
%{\normalfont\normalsize\bfseries}{\theparagraph}{1em}{}
%\titlespacing*{\paragraph}
%{0pt}{3.25ex plus 1ex minus .2ex}{1.5ex plus .2ex}
%\pagestyle{fancy} %better headers
%\fancyhf{}
%\rhead{Avinash Iyer}
%\lhead{}
\setcounter{chapter}{-1}
%\renewcommand{\newline}{\mynewline}
%\pagestyle{fancy}
%\fancyhead[C]{\scshape Chapter \thechapter}
%\fancyhead[R]{}
%\fancyhead[L]{}
%\renewcommand{\headrulewidth}{0pt}
%\fancyfoot[C]{\thepage}
%\usepackage{fancyhdr}

\title{Characterizing Amenability in Discrete Groups}
\author{Avinash Iyer}
\date{\today}
\setcounter{secnumdepth}{4}
\usepackage[hidelinks]{hyperref} %hyperlinks

\begin{document}
\maketitle
\RaggedRight
\tableofcontents
\chapter{Prelude}
%This thesis will be an introduction and broad overview of the theory of amenable groups, which admit a rich set of structures and characterizations. While we will not fully track the development of the theory to its research nowadays, we will discuss the origin of the theory of amenability --- namely, the Banach--Tarski paradox --- and develop different characterizations for amenability that will draw from group theory, measure theory, general topology, and functional analysis. Most of the necessary material will be covered either in the chapters themselves or in the appendix.\newline
%
%Amenability is a truly fascinating topic that, while its core is in functional analysis, allows significant insights into group theory nonetheless. This project may seem like an utterly daunting undertaking, but amenability is a very natural follow-on to the upper division mathematics curriculum (along with measure theory).
\chapter{Movement 1: Paradoxical Decompositions}
The primary goal of this section will be to introduce the idea of a paradoxical decomposition (and its effects on the analytic properties of $\R^3$) through the Banach--Tarski Paradox. The ultimate goal is to prove the following statement.
\begin{proposition}[General Banach--Tarski Paradox]
  If $A$ and $B$ are bounded subsets of $\R^3$ with nonempty interior, there is a partition of $A$ into finitely many disjoint subsets such a sequence of isometries applied to these subsets yields $B$.
\end{proposition}
The existence of the Banach--Tarski paradox throws a wrench into a major idea that we may have about subsets of $\R^3$ --- namely, that they always have some ``volume'' to them that is invariant under isometry, similar to how ``area'' in $\R^2$ is invariant under isometry.
\section{Prelude: Essential Group Actions}
We begin by discussing some of the basic properties of group actions.
\begin{definition}[Group Action]
  Let $G$ be a group, and $A$ be a set. A left group action of $G$ onto $A$ is a map $\alpha: G\times A\rightarrow A$ that satisfies
  \begin{itemize}
    \item $\alpha\left(g_1,\left(g_2,a\right)\right) = \alpha\left(g_1g_2,a\right)$ for all $g_1,g_2\in G$ and $a\in A$;
    \item $\alpha\left(e_G,a\right) = a$ for all $a\in A$.
  \end{itemize}
  For the sake of brevity, we write $\left(g,a\right) = g\cdot a$.
\end{definition}
Every group action can be represented by a permutation on $A$.
\begin{definition}[Permutation Representation]
  For each $g$, the map $\sigma_g: A\rightarrow A$ defined by $\sigma_g\left(a\right) = g\cdot a$ is a permutation of $A$. There is a homomorphism associated to these actions, $\varphi: G\rightarrow \operatorname{Sym}(A)$, where $\operatorname{Sym}(A)$ is the symmetric group on the elements of $A$.
\end{definition}
The permutation representation can run in the opposite direction in the following sense: given a nonempty set $A$ and a homomorphism $\psi: G\rightarrow \sym(A)$, we can take $g\cdot a = \psi(g)(a)$, where $\psi(g) = \sigma_g\in \sym(A)$ is a permutation.\newline

Just as we can pass group actions into permutation representations, and discuss ideas like the kernel of homomorphisms, we can also discuss the kernel of ajn action.
\begin{definition}[Kernel]
  The kernel of the action of $G$ on $A$ is the set of elements in $g$ that act trivially on $A$:
  \begin{align*}
    \set{g\in G\mid \forall a\in A,~g\cdot a = a}.
  \end{align*}
  The kernel of the group action is the kernel of the permutation representation $\varphi: G\rightarrow \sym(A)$.
\end{definition}
\begin{definition}[Stabilizer]
  For each $a\in A$, we define the stabilizer of $a$ under $G$ to be the set of elements in $G$ that fix $a$:
  \begin{align*}
    G_a &= \set{g\in G\mid g\cdot a = a}.
  \end{align*}
\end{definition}
\begin{remark}
The kernel of the group action is the intersection of the stabilizers of every element of $A$.\newline

For each $a\in A$, $G_{a}$ is a subgroup of $G$.
\end{remark}
\begin{definition}[Faithful Action]
An action is faithful if the kernel of the action is the identity, $e_G$. Equivalently, the permutation representation $\varphi: G\rightarrow \sym(A)$ is injective.
\end{definition}
The following definition will be useful in the future as we dig deeper into the idea of paradoxical groups.
\begin{definition}[Free Action]
For a set $X$ with $G$ acting on $X$, the action of $G$ on $X$ is free if, for every $x\in X$, $g\cdot x = x$ if and only if $g = e_G$.
\end{definition}
The most important theorem relating to group actions is the orbit-stabilizer theorem. As we prove the following theorem, we will reveal the definition of an orbit as a type of equivalence class.
\begin{theorem}[Orbit-Stabilizer Theorem]
  Let $G$ be a group that acts on a nonempty set $A$. We define a relation $a\sim b$ if and only if $a = g\cdot b$ for some $g\in G$. This is an equivalence relation, with the number of elements in $\left[a\right]_{\sim}$ found by taking the index of the stabilizer of $a$ in $G$, $\left\vert G:G_a \right\vert$.
\end{theorem}
\begin{proof}
  We start by seeing that $a\sim a$, as $e_G\cdot a = a$. Similarly, if $a\sim b$, then there exists $g\in G$ such that $a = g\cdot b$. Thus,
  \begin{align*}
    g^{-1}\cdot a &= g^{-1}\cdot \left(g\cdot b\right)\\
                  &= g^{-1}g\cdot b\\
                  &= e\cdot b\\
                  &= b,
  \end{align*}
  meaning that $b\sim a$. Finally, if we have $a\sim b$ and $b\sim c$, we have $a = g\cdot b$ and $b = h\cdot c$ for some $g,h\in G$. Therefore,
  \begin{align*}
    a &= g\cdot \left(h\cdot c\right)\\
      &= \left(gh\right)\cdot c,
  \end{align*}
  meaning $a\sim c$. Thus, the relation $\sim$ is reflexive, symmetric, and transitive, so it is an equivalence relation.\newline

  We claim there is a bijection between the left cosets of $G_a$ and the elements of $\left[a\right]_{\sim}$.\newline

  Define $C_a = \set{g\cdot a\mid g\in G}$, which is the set of elements in the equivalence class of $a$. Define the map $g\cdot a \mapsto gG_a$. Since $g\cdot a$ is always an element of $C_a$, this map is surjective. Additionally, since $g\cdot a = h\cdot a$ if and only if $\left(h^{-1}g\right)\cdot a = a$, we have $h^{-1}g \in G_a$, which is only true if $gG_a = hG_a$. Thus, the map is injective.\newline

  Since there is a one to one map between the equivalence classes of $a$ under the action of $G$, and the number of left cosets of $G_a$, we know that the number of equivalence classes of $a$ under the action of $G$ is $\left\vert G:G_a \right\vert$.
\end{proof}
\begin{definition}[Orbit]
Let $G$ act on $A$, and let $a\in A$. The orbit of $a$ under $G$ is the set
\begin{align*}
  G\cdot a &= \set{g\cdot a\in A\mid g\in G}
\end{align*}
\end{definition}

\chapter{Movement 2: Tarski's Theorem}
Ultimately, the reason the Banach--Tarski paradox ``works'' is because the paradoxical group $F(a,b)$, lacks a property known as amenability. Readers may be surprised to hear that amenability and non-paradoxicality are equivalent --- that is, a group is amenable if and only if it is non-paradoxical. This fact is formalized in Tarski's theorem.
\begin{theorem}[Tarski's Theorem]\label{thm:tarski}
  Let $G$ be a group that acts on a set $X$, and let $E\subseteq X$ be nonempty. There is a finitely additive translation-invariant measure $\mu: P(X)\rightarrow [0,\infty]$ with $\mu(E)\in (0,\infty)$ if and only if $E$ is not $G$-paradoxical.
\end{theorem}
In fact, we can prove one of the directions of Tarski's theorem now.
\begin{proof}[of the Forward Direction of Tarski's Theorem]
  Let $E$ be $G$-paradoxical. Suppose toward contradiction that such a translation-invariant finitely additive $\nu$ existed with $\nu(E) \in (0,\infty)$.\newline

  Let $A_1,\dots,A_n,B_1,\dots,B_m\subseteq E$ be pairwise disjoint, and let $t_1,\dots,t_n,s_1,\dots,f_m\in G$ such that
  \begin{align*}
    E &= \bigsqcup_{i=1}^{n}t_i\cdot A_i\\
      &= \bigsqcup_{j=1}^{m}s_j\cdot B_j.
  \end{align*}
  Then, it would be the case that
  \begin{align*}
    \nu(E) &= \nu\left(\bigsqcup_{i=1}^{n}t_i\cdot A_i\right)\\
           &= \sum_{i=1}^{n}\nu\left(t_i\cdot A_i\right)\\
           &= \sum_{i=1}^{n}\nu\left(A_i\right),
  \end{align*}
  and
  \begin{align*}
    \nu(E) &= \sum_{j=1}^{m}\nu\left(B_j\right).
  \end{align*}
  However, this also yields
  \begin{align*}
    \nu\left(E\right) &= \nu\left(\left(\bigsqcup_{i=1}^{n}A_i\right)\sqcup \left(\bigsqcup_{j=1}^{m}B_j\right)\right)\\
                      &= \sum_{i=1}^{n}\nu\left(A_i\right) + \sum_{j=1}^{m}\nu\left(B_j\right)\\
                      &= \sum_{i=1}^{n}\nu\left(t_i\cdot A_i\right) + \sum_{j=1}^{m}\nu\left(x_j\cdot B_j\right)\\
                      &= \nu\left(E\right) + \nu\left(E\right)\\
                      &= 2\nu\left(E\right).
  \end{align*}
  implying that $\nu(E) = 0$ or $\nu(E) = \infty$.
\end{proof}

\section{A Little Bit of Graph Theory}%
To prove the reverse direction of Tarski's theorem, we need to develop some machinery from graph theory that will allow us to prove that a certain semigroup we will construct in the next section satisfies the cancellation identity.\newline

We start by defining graphs and paths, before proving a special case of Hall's theorem, ultimately extending to the infinite case with König's theorem.
\begin{definition}[Graphs and Paths]
  A \textbf{graph} is a triple $\left(V,E,\phi\right)$, with $V,E$ nonempty sets and $\phi: E\rightarrow P_{2}(V)$ a map from $E$ to the set of all unordered subset pairs of $V$.\newline

  For $e\in E$, if $\phi(e) = \set{v,w}$, then we say $v$ and $w$ are the \textbf{endpoints} of $e$, and $e$ is \textbf{incident} on $v$ and $w$.\newline

  A \textbf{path} in $\left(V,E,\phi\right)$ is a finite sequence $\left(e_1,\dots,e_n\right)$ of edges, with a finite sequence of vertices $\left(v_0,\dots,v_n\right)$, such that $\phi\left(e_k\right) = \set{v_{k-1},v_k}$.\newline

  The \textbf{degree} of a vertex, $\deg(v)$, is the number of edges incident on $v$.\newline

  We define the \textbf{neighbors} of $S\subseteq V$ to be the set of all vertices $v\in V\setminus S$ such that $v$ is an endpoint to an edge incident on $S$. We denote this set $N(S)$.
\end{definition}

\begin{definition}[Bipartite Graphs and $k$-Regularity]
  Let $\left(V,E,\phi\right)$ be a graph, with $k\in \N$.
  \begin{enumerate}[(i)]
    \item If $\deg(v) = k$ for each $v\in V$, we say $\left(V,E,\phi\right)$ is \textbf{$k$-regular}.
    \item If $V = X\sqcup Y$, with each edge in $E$ having one endpoint in $X$ and one endpoint in $Y$, then we say $V$ is \textbf{bipartite}, and write $\left(X,Y,E,\phi\right)$.
  \end{enumerate}
\end{definition}

\begin{definition}[Perfect Matching]
  Let $\left(X,Y,E,\phi\right)$ be a bipartite graph. Let $A\subseteq X$ and $B\subseteq Y$. A \textbf{perfect matching} of $A$ and $B$ is a subset $F\subseteq E$ with
  \begin{enumerate}[(i)]
    \item each element of $A\cup B$ is an endpoint of exactly one $f\in F$;
    \item all endpoints of edges in $F$ are in $A\cup B$.
  \end{enumerate}
\end{definition}
\begin{definition}[Hall Condition]
  We say a bipartite graph $\left(X,Y,E,\phi\right)$ satisfies the \textbf{Hall Condition} on $X$ if, for all $S\subseteq X$, $\left\vert N(S) \right\vert \geq \left\vert S \right\vert$.\newline

  Equivalently, we say a (finite) collection of not necessarily distinct finite sets $\mathcal{X} = \set{X_i}_{i=1}^{n}$ satisfies the marriage condition if and only if for all subcollections $\mathcal{Y}_k = \set{X_{i_k}}_{k=1}^{m}$,
  \begin{align*}
    \left\vert \mathcal{Y}_k \right\vert \leq \left\vert \bigcup_{k=1}^{m}X_{i_k} \right\vert.
  \end{align*}
\end{definition}
\begin{remark}
These two formulations of the Hall condition are equivalent regarding an $X$-perfect matching.
\end{remark}
\begin{theorem}[Hall's Theorem for Finite $k$-Regular Bipartite Graphs]\label{thm:hall_finite}
  Let $\left(X,Y,E,\phi\right)$ be a $k$-regular bipartite graph for some $k\in \N$, and let $V = X\sqcup E$ be finite. Then, there is a perfect matching of $X$ and $Y$.
\end{theorem}
\begin{proof}
  Note that since $\left\vert E \right\vert = k\left\vert K \right\vert = k\left\vert Y \right\vert$, it is the case that $\left\vert X \right\vert = \left\vert Y \right\vert$.\newline

  Let $M\subseteq V$ be any subset. We will show that $\left\vert N(M) \right\vert\geq \left\vert M \right\vert$ --- that is, $\left(X,Y,E,\phi\right)$ satisfies the Hall condition.\newline

  Let $M_X = M\cap X$ and $M_Y = M\cap Y$, where $M = M_X\sqcup M_Y$. Let $\left[M_X,N\left(M_X\right)\right]$ be the set of edges with endpoints in $M_X$ and $N\left(M_X\right)$, and $\left[M_Y,N\left(M_Y\right)\right]$ be the set of edges with endpoints in $M_Y$ and $N\left(M_Y\right)$. We also let $\left[X,N\left(M_X\right)\right]$ denote the set of edges with endpoints in $X$ and $N\left(M_X\right)$, and similarly, $\left[Y,N\left(M_Y\right)\right]$ is the set of edges with endpoints in $Y$ and $N\left(M_Y\right)$.\newline

  We can see that $\left[M_X,N\left(M_X\right)\right]\subseteq \left[X,N\left(M_X\right)\right]$, and similarly, $\left[M_Y,N\left(M_Y\right)\right]\subseteq \left[Y,N\left(M_Y\right)\right]$.\newline

  Since $\left\vert \left[M_X,N\left(M_X\right)\right] \right\vert = k\left\vert M_X \right\vert$ and $\left\vert \left[X,N\left(M_X\right)\right] \right\vert = k\left\vert N\left(M_X\right) \right\vert$, we have
  \begin{align*}
    \left\vert M_X \right\vert\leq \left\vert N\left(M_X\right) \right\vert,
  \end{align*}
  and similarly,
  \begin{align*}
    \left\vert M_Y \right\vert\leq \left\vert N\left(M_Y\right) \right\vert.
  \end{align*}
  Thus, $\left\vert M \right\vert\leq \left\vert N\left(M\right) \right\vert$.\newline

  We will now show that there is an $X$-perfect matching. Suppose toward contradiction that $F$ is a maximal perfect matching on $A\subseteq X$ and $B\subseteq Y$ with $X\setminus A \neq \emptyset$.\newline

  Then, there is $x\in X\setminus A$. Consider $Z\subseteq V$ consisting of all vertices $z$ such that there exists a $F$-alternating path $\left(e_1,\dots,e_n\right)$ between $z\in Z$ and $x$.\newline

  It cannot be the case that $Z\cap Y$ is empty, since the number of neighbors of $x$ is greater than or equal to $1$ by the Hall condition --- if it were the case that $Z\cap Y$ were empty, we could add an edge to $F$ consisting of $x$ and one element of $N\left(\set{x}\right)$, which would contradict the maximality of $F$.\newline

  Consider a path traversing along $Z$, $\left(e_1,\dots,e_n\right)$. It must be the case that $e_n\in F$, or else we would be able to ``flip'' the matching $F$ by exchanging $e_{i}$ with $e_{i+1}$ for $e_i\in F$, which would contradict the maximality of $F$ yet again. Thus, every element of $Z\cap Y$ is satisfied by $F$, so $Z\cap Y\subseteq B$.\newline

  Since each element in $Z\cap Y$ is paired with exactly one element of $Z\cap X$ (with one left over), it is the case that $\left\vert Z\cap X \right\vert = \left\vert Z\cap Y \right\vert + 1$.\newline

  Suppose toward contradiction that there exists $y\in N\left(Z\cap X\right)$ with $y\notin Z\cap Y$. Then, there exists $v\in Z\cap X$ and $e\in E$ such that $\phi(e) = \set{v,y}$. However, this means $v$ is connected via a path to $x$, meaning $y\in Z$, so $y\in Z\cap Y$. Thus, we must have $N\left(Z\cap X\right) = Z\cap Y$.\newline

  Therefore,
  \begin{align*}
    \left\vert Z\cap X \right\vert &= \left\vert Z\cap Y \right\vert + 1\\
                                   &= \left\vert N\left(Z\cap X\right) \right\vert + 1,
  \end{align*}
  which contradicts the fact that $\left(X,Y,E,\phi\right)$ satisfies the Hall condition. Therefore, $A = X$.\newline

  By symmetry, there is a perfect matching of $X$ and $Y$ in $\left(X,Y,E,\phi\right)$.
\end{proof}
\begin{remark}
  An equivalent formulation to Hall's theorem states that there is a \textit{system of distinct representatives} on $\mathcal{X}$, which is a set $\set{x_{k}}_{k=1}^{n}$ such that $x_{k}\in X_{k}$ and $x_{i}\neq x_j$ for $i\neq j$.\newline

  This implies the existence of an injection $f: \mathcal{X}\hookrightarrow \bigcup_{k=1}^{n}X_{k}$, such that $f\left(X_k\right) \in X_k$.
\end{remark}
%\begin{definition}[Choice Function]
%  Let $\mathcal{X} = \set{X_{i}}_{i\in I}$ be a collection of sets. A function $f: \mathcal{X}\rightarrow \bigcup_{i\in I}X_i$ is called a choice function if, for each $i\in I$, $f\left(X_{i}\right)\in X{i}$.\newline
%
%  We also say $f: \mathcal{X}\rightarrow \bigcup_{i\in I}X_i$ is a choice function if $f\in \prod_{i\in I}X_i$.
%\end{definition}
%
%\begin{theorem}[Tychonoff's Theorem]
%  If $\set{X_{i}}_{i\in I}$ is a family of compact topological spaces
%\end{theorem}
\begin{theorem}[Infinite Hall's Theorem]
  Let $\mathcal{G} = \set{X_{i}}_{i\in I}$ be a collection of (not necessarily distinct) finite sets. If, for every finite subcollection $\mathcal{Y} = \set{X_{i_k}}_{k=1}^{n}$,
  \begin{align*}
    n\leq \left\vert \bigcup_{k=1}^{n}X_{i_k} \right\vert,
  \end{align*}
  then there is a choice function on $G$.
\end{theorem}
\begin{proof}
  We endow each $X_i\in \set{X_{i}}_{i\in I}$ with the discrete topology. Since each $X_i$ is finite, each $X_i$ is compact.\newline

  Thus, by Tychonoff's theorem, it is the case that $\prod_{i\in I}X_{i}$ is compact.\newline

  For every finite subset $Y\subseteq \mathcal{G}$, we define
  \begin{align*}
    S_Y &= \set{\left.f\in \prod_{i\in I}X_i\right|f\vert_{Y}\text{ is injective}}.
  \end{align*}
  The injectivity of $f\vert_{Y}$ is equivalent to the existence of a system of distinct representatives on $Y$. Since $Y$ satisfies the Hall condition, each $S_{Y}$ is nonempty. Additionally, for any net of functions $f_{\alpha}\in S_{Y}$ with $\lim_{\alpha}f_{\alpha} = f$, it is the case that $f_{\alpha}\vert_{Y}$ is injective, so $f\vert_{Y}$ is injective, meaning $S_{Y}$ is closed.\newline

  We define $F = \set{S_{Y}\mid Y\subseteq \mathcal{G}\text{ finite}}$. For finite $Y_{1},Y_{2}\subseteq \mathcal{G}$, every system of distinct representatives in $Y_1\cup Y_2$ is necessarily a system of distinct representatives on $Y_1$ and a system of distinct representatives on $Y_{2}$, meaning $S_{Y_1\cup Y_2}\subseteq S_{Y_1}\cap S_{Y_2}$. Thus, $F$ has the finite intersection property.\newline

  Since $\prod_{i\in I}X_i$ is compact, $\bigcap F$ is nonempty, where the intersection is taken over all finite subsets of $\mathcal{G}$. For any $f\in \bigcap F$, $f$ is necessarily a choice function.
\end{proof}
\begin{remark}
  This is equivalent to the existence of an injection $f: \mathcal{G}\hookrightarrow \bigcup_{i\in I}X_i$.
\end{remark}

We will use this infinite case of Hall's theorem to prove König's theorem. 
\begin{theorem}[König's Theorem]\label{thm:konig}
  Let $\left(X,Y,E,\phi\right)$ be a $k$-regular bipartite graph (not necessarily finite). Then, there is a perfect matching of $X$ and $Y$.
\end{theorem}
\begin{proof}
  If $k = 1$, it is clear that there is a perfect matching in $\left(X,Y,E,\phi\right)$ consisting of the edges in $\left(X,Y,E,\phi\right)$.\newline

  Let $k\geq 2$. Since any finite subset of $X$ satisfies the Hall condition, as displayed in the proof of Theorem \ref{thm:hall_finite}, there is some $X$-perfect matching in $\left(X,Y,E,\phi\right)$. We call this $X$-perfect matching $F$. There is an injection $f: X\hookrightarrow Y$ following the edges in $F$.\newline

  Similarly, since any finite subset of $Y$ satisfies the Hall condition, there is some $Y$-perfect matching in $\left(X,Y,E,\phi\right)$. We call this $Y$-perfect matching $G$. There is an injection $g: Y\hookrightarrow X$ following the edges of $G$.\newline

  Consider the subgraph $\left(X,Y,F\cup G,\phi|_{F\cup G}\right)$. The injections $f$ and $g$ still hold in this graph. By the Cantor--Schröder--Bernstein theorem, there is a bijection $h: X\rightarrow Y$ in $\left(X,Y,F\cup G,\phi|_{F\cup G}\right)$.
\end{proof}
\section{Type Semigroups}%
\begin{definition}
  Let $G$ be a group that acts on a set $X$.
  \begin{enumerate}[(i)]
    \item We define $X^{\ast} = X\times \N_0$, and
      \begin{align*}
        G^{\ast} &= \set{\left(g,\pi\right)\mid g\in G,\pi\in\sym\left(\N_0\right)}.
      \end{align*}
    \item If $A\subseteq X^{\ast}$, the values of $n$ for which there is an element of $A$ whose second coordinate is $n$ are called the \textbf{levels} of $A$.
  \end{enumerate}
\end{definition}
\begin{fact}\label{fact:type_semigroup_equidecomposability}
  If $E_1,E_2\subseteq X$, then $E_{1}\sim_{G}E_2$ if and only if $E_1\times \set{n}\sim_{G^{\ast}}E_{2}\times \set{m}$ for all $m,n\in \N_{0}$
\end{fact}
\begin{proof}[of Fact \ref{fact:type_semigroup_equidecomposability}]
  Let $E_{1}\sim_{G}E_2$. Then, there exist pairwise disjoint $A_1,\dots,A_n\subset E_1$, pairwise disjoint $B_1,\dots,B_n\subset E_2$, and elements $g_1,\dots,g_n\in G$ such that $g_i\cdot A_i = B_i$. We select the permutation $\pi_{i}\in \sym\left(\N_0\right)$ such that $\pi_{i}(n) = m$ and $\pi_i(m) = n$ for each $i$. Then,
  \begin{align*}
    \left(g_i,\pi_i\right)\cdot \left(A_{i}\cdot \set{n}\right) &= B_{i}\cdot \set{m}.
  \end{align*}

  Similarly, if $E_{1}\times \set{n} \sim_{G^{\ast}}E_2\times \set{m}$, then of the pairwise disjoint subsets
  \begin{align*}
    A_1\times \set{n},\dots,A_n\times \set{n}\subset E_1\times \set{n}
  \end{align*}
  and
  \begin{align*}
    B_1\times\set{m},\dots,B_n\times\set{m}\subset E_2\times \set{m},
  \end{align*}
  we set $A_1,\dots,A_n\subset E_1$ and $B_1,\dots,B_n\subset E_2$. Similarly, for
  \begin{align*}
    \left(g_1,\pi_1\right),\dots,\left(g_n,\pi_n\right)\in G^{\ast}
    \intertext{such that}
    \left(g_i,\pi_i\right)\cdot A_i\times \set{n} = B_i\times\set{m},
  \end{align*}
  we select $g_1,\dots,g_n\in G$. Then, by definition,
  \begin{align*}
    g_i\cdot A_i = B_i
  \end{align*}
  for each $i$. Thus, $E_1\sim_{G}E_2$.
\end{proof}

\begin{definition}\label{def:type_semigroup}
  Let $G$ be a group that acts on $X$, and let $G^{\ast}$, $X^{\ast}$ be defined as above.
  \begin{enumerate}[(i)]
    \item A set $A\subseteq X^{\ast}$ is said to be \textbf{bounded} if it has finitely many levels.
    \item If $A\subseteq X^{\ast}$ is bounded, the equivalence class of $A$ with respect to $G^{\ast}$-equidecomposability is called the \textbf{type} of $A$, which is denoted $\left[A\right]$.
    \item If $E\subseteq X$, we write $\left[E\right] = \left[E\times \set{0}\right]$.
    \item Let $A,B\subseteq X^{\ast}$ be bounded with $k\in \N_{0}$ such that for
      \begin{align*}
        B' := \set{\left(b,n+k\right)\mid \left(b,n\right)\in B},
      \end{align*}
      we have $B'\cap A = \emptyset$. Then, $\left[A\right] + \left[B\right] = \left[A\sqcup B'\right]$. Note that $\left[B'\right] = \left[B\right]$.
    \item We define
      \begin{align*}
        \mathcal{S} &= \set{\left[A\right]\mid A\subseteq X^{\ast}\text{ bounded}}
      \end{align*}
      under the addition defined in (iv) to be the \textbf{type semigroup} of the action of $G$ on $X$.
  \end{enumerate}
\end{definition}

\begin{fact}\label{fact:type_semigroup_well_defined}
  Addition is well-defined in $\left(\mathcal{S},+\right)$, and $\left(\mathcal{S},+\right)$ is a well-defined commutative semigroup with identity $\left[\emptyset\right]$.
\end{fact}
\begin{proof}[of Fact \ref{fact:type_semigroup_well_defined}]
  To show that addition is well-defined, we let $\left[A_1\right] = \left[A_2\right]$, and $\left[B_1\right] = \left[B_2\right]$. Without loss of generality, $A_1\cap B_1 = \emptyset$ and $A_2\cap B_2 = \emptyset$.\newline

  By the definition of the type, $A_1\sim_{G^{\ast}}A_2$ and $B_1\sim_{G^{\ast}}B_2$, meaning
  \begin{align*}
    A_1\sqcup B_1\sim_{G^{\ast}} A_2\sqcup B_2,
  \end{align*}
  so
  \begin{align*}
    \left[A_1\right] + \left[B_1\right] &= \left[A_1\sqcup B_1\right]\\
                                        &= \left[A_2\sqcup B_2\right]\\
                                        &= \left[A_2\right] + \left[A_2\right],
  \end{align*}
  meaning addition is well-defined.\newline

  Since addition is well-defined, and $A\sqcup B = B\sqcup A$, we can see that addition is also commutative. We also have
  \begin{align*}
    \left[A\right] + \left[\emptyset\right] &= \left[A\sqcup \emptyset\right]\\
                                            &= \left[A\right],
  \end{align*}
  so $\left[\emptyset\right]$ is the identity on $\mathcal{S}$.\newline

  Finally, since for any $\left[A\right],\left[B\right]\in \mathcal{S}$, $A$ and $B$ have finitely many levels, it is the case that $A\cup B$ has finitely many levels for any $A$ and $B$, so $\left[A\right] + \left[B\right] \in \mathcal{S}$. 
\end{proof}

\begin{definition}
  For any commutative semigroup $\mathcal{S}$ with $\alpha \in S$ and $n\in \N$, we define
  \begin{align*}
    n\alpha := \underbrace{\alpha + \cdots + \alpha}_{\text{$n$ times}}
  \end{align*}
\end{definition}
\begin{definition}
  For $\alpha,\beta \in \mathcal{S}$, if there exists $\gamma \in \mathcal{S}$ such that $\alpha + \gamma = \beta$, we write $\alpha \leq \beta$.
\end{definition}
\begin{fact}\label{fact:type_semigroup_paradoxicality}
  If $G$ is a group acting on $X$ with corresponding type semigroup $\mathcal{S}$, then the following are true.
  \begin{enumerate}[(i)]
    \item If $\alpha,\beta\in \mathcal{S}$ with $\alpha \leq \beta$ and $\beta \leq \alpha$, then $\alpha = \beta$.
    \item A set $E\subseteq X$ is $G$-paradoxical if and only if $\left[E\right] = 2\left[E\right]$.
  \end{enumerate}
\end{fact}
\begin{proof}[of Fact \ref{fact:type_semigroup_paradoxicality}]
  Let $G$ act on $X$, and let $\mathcal{S}$ be the corresponding type semigroup.
  \begin{enumerate}[(i)]
    \item If $\left[A\right]\leq \left[B\right]$, then there exists $C_1\in \mathcal{S}$ such that $\left[A\right] + \left[C_1\right] = \left[B\right]$. Without loss of generality, $C_1\cap A= \emptyset$, meaning $\left[B\right] = \left[A\sqcup C_1\right]$. Thus, $A\sqcup C_1 \sim_{G^{\ast}} B$, meaning $B\preceq_{G^{\ast}}A$.\newline

      Similarly, if $\left[B\right]\leq \left[A\right]$, then $B\preceq_{G^{\ast}}A$. By Theorem \ref{thm:csb_for_equidecomposability}, it is thus the case that $A\sim_{G^{\ast}}B$.
    \item Let $E$ be $G$-paradoxical. Then, $E\sim_{G}\bigsqcup_{i=1}^{n}A_i$ and $E\sim_{G}\bigsqcup_{j=1}^{m}B_j$ for some disjoint subsets $A_1,\dots,A_n,B_1,\dots,B_m\subset E$. Thus, we have
      \begin{align*}
        \left[E\right] &= \left[\left(\bigsqcup_{i=1}^{n}A_i\right)\sqcup \left(\bigsqcup_{j=1}^{m}B_j\right)\right]\\
                       &= \left[\bigsqcup_{i=1}^{n}A_i\right] + \left[\bigsqcup_{j=1}^{m}B_j\right]\\
                       &= 2\left[E\right].
      \end{align*}
      Similarly, if $\left[E\right] = 2\left[E\right]$, then there exist $A$ and $B$ such that
      \begin{align*}
        \left[E\right] &= \left[A\right] + \left[B\right]\\
                       &= \left[A\sqcup B\right],
      \end{align*}
      meaning $A$ and $B$ are each $G$-equidecomposable with $E$, so $E$ is $G$-paradoxical.
  \end{enumerate}
\end{proof}
We can now prove the cancellation identity, which we will be useful as we construct our desired finitely additive measure.
\begin{theorem}[Cancellation Identity on $\mathcal{S}$]
  Let $\mathcal{S}$ be the type semigroup for some group action, and let $\alpha,\beta\in \mathcal{S}$, $n\in \N$ such that $n\alpha = n\beta$. Then, $\alpha = \beta$.
\end{theorem}
\begin{proof}
  Let $n\alpha = n\beta$. Then, there are two disjoint bounded subsets $E,E'\subseteq X^{\ast}$ with $E\sim_{G^{\ast}}E'$, and pairwise disjoint subsets $A_1,\dots,A_n\subseteq E$, $B_1,\dots,B_n\subseteq E'$ such that
  \begin{itemize}
    \item $E = A_1\cup\cdots\cup A_n$, $E' = B_1\cup\cdots\cup B_n$
    \item $\left\vert A_j \right\vert = \alpha$ and $\left[B_j\right] = \beta$ for each $j=1,\dots,n$.
  \end{itemize}
  Let $\chi: E\rightarrow E'$ be a bijection as in Fact \ref{fact:bijections}, with $\phi_j: A_1\rightarrow A_j$, $\psi_j: B_1\rightarrow B_j$ also being bijections as in Fact \ref{fact:bijections}; here we define $\phi_1$ and $\psi_1$ to be the identity map.\newline

  For each $a\in A_1$ and $b\in B_1$, we define
  \begin{align*}
    \overline{a} &= \set{a,\phi_2(a),\dots,\phi_n(a)}\\
    \overline{b} &= \set{b,\psi_2(b),\dots,\psi_n(b)}.
  \end{align*}
  We construct a graph by letting $X = \set{\overline{a}\mid a\in A_1}$ and $Y = \set{\overline{b}\mid b\in B_1}$, and, for each $j$, define edges $\set{\overline{a},\overline{b}}$ if $\chi\left(\phi_j(a)\right)\in \overline{b}$.\newline

  Since $\chi$ is a bijection, for each $j=1,\dots,n$, $\chi\left(\phi_j(a)\right)$ must be an element of $B_k$ for some $k$, and since $\set{B_k}_{k=1}^{n}$ are disjoint, $\chi\left(\phi_j(a)\right)$ is an element of exactly one $B_k$. Thus, the graph is $n$-regular.\newline

  By Theorem \ref{thm:konig}, this graph has a perfect matching $F$. As a result, for each $\overline{a}\in X$, there is a unique $\overline{b}\in Y$ and a unique edge $\set{\overline{a},\overline{b}}\in F$ such that $\chi\left(\phi_j(a)\right) = \psi_k(b)$ for some $j,k\in \set{1,\dots,n}$.\newline

  We define
  \begin{align*}
    C_{j,k} &= \set{a\in A_1\mid \set{\overline{a},\overline{b}}\in F,~\chi\left(\phi_j(a)\right) = \psi_k(b)}\\
    D_{j,k} &= \set{b\in B_1\mid \set{\overline{a},\overline{b}}\in F,~\chi\left(\phi_j(a)\right) = \psi_k(b)}.
  \end{align*}
  Therefore, we must have $\psi_{k}^{-1}\circ \chi\circ \phi_j$ is a bijection from $C_{j,k}$ to $D_{j,k}$, so $C_{j,k}\sim_{G^{\ast}}D_{j,k}$.\newline

  Since $C_{j,k}$ and $D_{j,k}$ are partitions of $A_1$ and $B_1$ respectively, it follows that $A_1\sim_{G^{\ast}}B_1$, so $\alpha = \beta$.
\end{proof}
\begin{corollary}\label{corollary:paradoxical_elements}
  Let $\mathcal{S}$ be the type semigroup of some group action, and let $\alpha\in \mathcal{S}$ and $n\in \N$ such that $\left(n+1\right)\alpha \leq n\alpha$. Then, $\alpha = 2\alpha$.
\end{corollary}
\begin{proof}
  We have
  \begin{align*}
    2\alpha + n\alpha &= \left(n+1\right)\alpha + \alpha\\
                      &\leq n\alpha + \alpha\\
                      &= \left(n+1\right)\alpha\\
                      &\leq n\alpha.
  \end{align*}
  Inductively repeating this argument, we get $n\alpha \geq 2n\alpha$; since $n\alpha \leq 2n\alpha$ by definition, we must have $n\alpha = 2n\alpha$, so $\alpha = 2\alpha$.
\end{proof}
\begin{remark}
  We will call such an $\alpha$ a paradoxical element.
\end{remark}
\section{Two Results on Commutative Semigroups}%
Now that we are aware of paradoxical elements and the relationship between $G$-paradoxicality and the properties of the particular elements of the type semigroup (Fact \ref{fact:type_semigroup_paradoxicality}), we will now relate these properties to finitely additive measures of sets by using the following lemma and theorem.
\begin{lemma}\label{lemma:set_function_existence}
  Let $\mathcal{S}$ be a commutative semigroup, with $\mathcal{S}_0\subseteq \mathcal{S}$ finite, and $\epsilon\in \mathcal{S}_0$ satisfying the following assumptions:
  \begin{enumerate}[(a)]
    \item $\left(n+1\right)\epsilon \nleq n\epsilon$ for all $n\in \N$ (i.e., that $\epsilon$ is non-paradoxical);
    \item for each $\alpha\in \mathcal{S}$, there is $n\in \N$ such that $\alpha \leq n\epsilon$.
  \end{enumerate}
  Then, there is a set function $\nu: \mathcal{S}_0\rightarrow [0,\infty]$ that satisfies the following conditions:
  \begin{enumerate}[(i)]
    \item $\nu\left(\epsilon\right) = 1$;
    \item for $\alpha_1,\dots,\alpha_n,\beta_1,\dots,\beta_m\in \mathcal{S}_0$ with $\alpha_1+\cdots+\alpha_n\leq \beta_1+\cdots\beta_m$,
      \begin{align*}
        \sum_{j=1}^{n}\nu\left(\alpha_j\right) \leq \sum_{j=1}^{m}\nu\left(\beta_j\right).
      \end{align*}
  \end{enumerate}
\end{lemma}
\begin{proof}
  We will prove this result by inducting on the cardinality of $\mathcal{S}_0$.\newline

  We start with $\left\vert \mathcal{S}_0 \right\vert = 1$. In that case, we define $\nu\left(\epsilon\right) = 1$, satisfying condition (i). To satisfy condition (ii), we see that for $n,m\in \N$ with $n\epsilon \leq m\epsilon$, if $n \geq m+1$, then $\left(m+1\right)\epsilon \leq n\epsilon \leq m\epsilon$, implying that $\epsilon = 2\epsilon$, which contradicts assumption (a).\newline

  Let $\alpha_0\in \mathcal{S}_0\setminus\set{\epsilon}$. The induction hypothesis says there is a set function satisfying conditions (i) and (ii), $\nu: \mathcal{S}_0\setminus \set{\alpha_0}\rightarrow [0,\infty]$.\newline

  For $r\in \N$, there are $\gamma_1,\dots,\gamma_p,\delta_1,\dots,\delta_q\in \mathcal{S}\setminus \set{\alpha_0}$ such that
  \begin{align*}
    \delta_{1} + \cdots + \delta_q + r\alpha_0 \leq \gamma_1 + \cdots + \gamma_p.\label{set_function_id1}\tag*{(\textdagger)}
  \end{align*}
  Consider the set $N$ defined as follows:
  \begin{align*}
    N &= \set{\frac{1}{r}\left(\sum_{j=1}^{p}\nu\left(\gamma_j\right) - \sum_{j=1}^{q}\nu\left(\delta_j\right)\right)\mid \text{$\gamma_j,\delta_j$ satisfy \ref{set_function_id1}}}. \label{set_function_N}\tag*{($\ddag$)}
  \end{align*}
  We define the extension of $\nu$ as follows:
  \begin{align*}
    \nu\left(\alpha_0\right) &= \inf N.
  \end{align*}
  This infimum is well-defined since, by assumption (b), there is some $n\in \N$ such that $\alpha_0 \leq n\epsilon$, and $\nu\left(\epsilon\right)$ is defined.\newline

  Now, we must show that this extension of $\nu$ satisfies condition (ii).\newline

  Let $\alpha_1,\dots,\alpha_n,\beta_1,\dots,\beta_m\in \mathcal{S}_0\setminus \set{\alpha_0}$ and $s,t\in \N_0$ such that
  \begin{align*}
    \alpha_1 + \cdots + \alpha_n + s\alpha_0 \leq \beta_1 + \cdots + \beta_m + t\alpha_0.\label{set_function_conditionii}\tag*{(\textasteriskcentered)}
  \end{align*}
  We will verify condition (ii) in the three following cases.
  \begin{description}[font=\normalfont\scshape,leftmargin=0cm]
    \item[Case 0:] If $s = t = 0$, then the induction hypothesis states that \ref{set_function_conditionii} satisfies condition (ii).
    \item[Case 1:] Let $s = 0$ and $t > 0$. We want to show that
      \begin{align*}
        \sum_{j=1}^{n}\nu\left(\alpha_j\right) \leq t\nu\left(\alpha_0\right) + \sum_{j=1}^{m}\nu\left(\beta_j\right),
      \end{align*}
      which implies that
      \begin{align*}
        \nu\left(\alpha_0\right) \geq \frac{1}{t}\left(\sum_{j=1}^{n}\nu\left(\alpha_j\right) - \sum_{j=1}^{m}\nu\left(\beta_j\right)\right).
      \end{align*}
      By the definition of infimum, it suffices to show that for $r\in \N$ and $\delta_1,\dots,\delta_q,\gamma_1,\dots,\gamma_p\in \mathcal{S}\setminus \set{\alpha_0}$ satisfying \ref{set_function_id1}, it is the case that
      \begin{align*}
        \frac{1}{r}\left(\sum_{j=1}^{p}\nu\left(\gamma_j\right)-\sum_{j=1}^{q}\nu\left(\delta_j\right)\right) \geq \frac{1}{t}\left(\sum_{j=1}^{n}\nu\left(\alpha_j\right) - \sum_{j=1}^{m}\nu\left(\beta_j\right)\right).
      \end{align*}
      Multiplying \ref{set_function_conditionii} by $r$ on both sides, and adding $t\delta_1 + \cdots + t\delta_q$ to both sides, we have
      \begin{align*}
        r\alpha_1 + \cdots + r\alpha_n + t\delta_1 + \cdots + t\delta_q \leq r\beta_1 + \cdots + r\beta_m + t\left(r\alpha_0\right) + t\delta_1 + \cdots + t\delta_q.
      \end{align*}
      Substituting \ref{set_function_id1}, we find
      \begin{align*}
        r\alpha_1 + \cdots + r\alpha_n + t\delta_1 + \cdots + t\delta_q \leq r\beta_1 + \cdots + r\beta_m + t\gamma_1 + \cdots + t\gamma_p.
      \end{align*}
      Applying the induction hypothesis, we have
      \begin{align*}
        r\sum_{j=1}^{n}\nu\left(\alpha_j\right) + t\sum_{j=1}^{q}\nu\left(\delta_j\right) \leq r\sum_{j=1}^{m}\nu\left(\beta_j\right) + t\sum_{j=1}^{p}\nu\left(\gamma_j\right),
      \end{align*}
      yielding
      \begin{align*}
        \frac{1}{r}\left(\sum_{j=1}^{p}\nu\left(\gamma_j\right) - \sum_{j=1}^{q}\nu\left(\delta_j\right)\right) \geq \frac{1}{t}\left(\sum_{j=1}^{n}\nu\left(\alpha_j\right) - \sum_{j=1}^{m}\nu\left(\beta_j\right)\right).
      \end{align*}
    \item[Case 2:] Let $s > 0$. For $z_1,\dots,z_t\in N$ \ref{set_function_N}, we need to show that
      \begin{align*}
        s\nu\left(\alpha_0\right) + \sum_{j=1}^{n}\nu\left(\alpha_j\right) \leq z_1 + \cdots + z_t + \sum_{j=1}^{n}\nu\left(\beta_j\right).
      \end{align*}
      Without loss of generality, we can set $z_1,\dots,z_n = z$, as for each $z\in N$, $z \geq \nu\left(\alpha_0\right)$.\newline

      As in Case 1, we multiply \ref{set_function_conditionii} by $r$, add $t\delta_{1} + \cdots + t\delta_q$ to both sides, and substitute with \ref{set_function_id1}, yielding
      \begin{align*}
        r\alpha_1 + \cdots + r\alpha_n + rs\alpha_0 + t\delta_1 + \cdots + t\delta_q &\leq r\beta_1 + \cdots + r\beta_m + t\left(r\alpha_0\right) + t\delta_1 + \cdots + t\delta_q\\
        r\alpha_1 + \cdots + r\alpha_n + t\delta_1 + \cdots + t\delta_q + rs\alpha_0 &\leq r\beta_1 + \cdots + r\beta_m + t\gamma_1 + \cdots + t\gamma_p.
      \end{align*}
      Defining
      \begin{align*}
        z &= \frac{1}{r}\left(\sum_{j=1}^{p}\nu\left(\gamma_j\right) - \sum_{j=1}^{q}\nu\left(\delta_j\right)\right),
      \end{align*}
      we get
      \begin{align*}
        s\nu\left(\alpha_0\right) + \sum_{j=1}^{n}\nu\left(\alpha_j\right) &\leq \sum_{j=1}^{n}\nu\left(\alpha_j\right) + \frac{s}{sr}\left(r\sum_{j=1}^{m}\nu\left(\beta_j\right) - r\sum_{j=1}^{n}\nu\left(\alpha_j\right) + t\sum_{j=1}^{p}\nu\left(\gamma_j\right) - t\sum_{j=1}^{q}\nu\left(\delta_j\right)\right)\\
                                                                           &= tz + \sum_{j=1}^{m}\nu\left(\beta_j\right).
      \end{align*}
  \end{description}
  Thus, we have shown that $\nu$ extends in a manner that satisfies conditions (i) and (ii).
\end{proof}

We can ``upgrade'' our finitely additive set function to a semigroup homomorphism as follows.
\begin{theorem}\label{thm:homomorphism_existence}
  Let $\left(\mathcal{S},+\right)$ be a commutative semigroup with identity element $0$, and let $\epsilon\in \mathcal{S}$. Then, the following are equivalent:
  \begin{enumerate}[(i)]
    \item $\left(n+1\right)\epsilon \leq n\epsilon$ for all $n\in \N$;
    \item there is a semigroup homomorphism $\nu: \left(\mathcal{S},+\right)\rightarrow \left([0,\infty],+\right)$ such that $\nu(\epsilon) = 1$.
  \end{enumerate}
\end{theorem}
\begin{proof}
  To show that (ii) implies (i), we let $\nu: \left(\mathcal{S},+\right)\rightarrow \left([0,\infty],+\right)$ be a semigroup homomorphism with $\nu\left(\epsilon\right) = 1$. Then,
  \begin{align*}
    \nu\left(\left(n+1\right)\epsilon\right) &= \left(n+1\right)\nu\left(\epsilon\right)\\
                                             &= n+1\\
                                             &> n\\
                                             &= n\nu\left(\epsilon\right)\\
                                             &= \nu\left(n\epsilon\right),
  \end{align*}
  meaning that $\left(n+1\right)\epsilon \nleq n\epsilon$.\newline

  To show that (i) implies (ii), we suppose that for each $\alpha \in \mathcal{S}$, there is $n\in \N$ such that $\alpha \leq n\epsilon$ --- for any such $\alpha$ for which this is not the case, we define $\nu\left(\alpha\right) = \infty$.\newline

  For a finite subset $\mathcal{S}_0 \subseteq \mathcal{S}$ with $\epsilon\in \mathcal{S}_0$, we define $M_{\mathcal{S}_0}$ to be the set of all $\kappa: \mathcal{S}\rightarrow [0,\infty]$ such that
  \begin{itemize}
    \item $\kappa\left(\epsilon\right) = 1$;
    \item $\kappa\left(\alpha + \beta\right) = \kappa\left(\alpha\right) + \kappa\left(\beta\right)$ for $\alpha,\beta,\alpha + \beta\in \mathcal{S}_0$.
  \end{itemize}
  Since we assume condition (i), we know that such a $\kappa$ with $\kappa\left(\epsilon\right) = 1$ exists. Additionally, since
  \begin{align*}
    \alpha + \beta \leq \left(\alpha + \beta\right)
  \end{align*}
  and
  \begin{align*}
    \left(\alpha + \beta\right) \leq \alpha + \beta,
  \end{align*}
  it is the case that
  \begin{align*}
    \kappa\left(\alpha + \beta\right) \leq \kappa\left(\alpha\right) + \kappa\left(\beta\right) \leq \kappa\left(\alpha + \beta\right),
  \end{align*}
  meaning $\kappa\left(\alpha + \beta\right) = \kappa\left(\alpha\right) + \kappa\left(\beta\right)$. Thus, $M_{\mathcal{S}_0}$ is nonempty. It is also the case that $M_{\mathcal{S}_0}$ is closed, since any net of functions $\kappa_{p}: \mathcal{S}\rightarrow [0,\infty]$ with $\kappa_{p}\left(\epsilon\right) = 1$ and $\kappa_{p}\left(\alpha + \beta\right) = \kappa_{p}\left(\alpha\right) + \kappa_{p}\left(\beta\right)$ will necessarily satisfy these conditions in the limit.\newline

  We let $\left[0,\infty\right]^{\mathcal{S}} = \set{\kappa\mid \kappa:\mathcal{S}\rightarrow [0,\infty]}$ be equipped with the product topology. By Tychonoff's theorem, $\left[0,\infty\right]^{\mathcal{S}}$ is compact.\newline

  Since, for any $\mathcal{S}_1,\dots,\mathcal{S}_n$ finite, it is the case that
  \begin{align*}
    M_{\mathcal{S}_1\cup\cdots\cup \mathcal{S}_n} \subseteq M_{\mathcal{S}_1} \cap \cdots \cap M_{\mathcal{S}_n},
  \end{align*}
  since any such $\kappa\in M_{\mathcal{S}_1\cup\cdots\cup \mathcal{S}_n}$ must necessarily be in every $M_{\mathcal{S}_i}$. Thus, the family
  \begin{align*}
    \set{M_{\mathcal{S}_0}\mid \mathcal{S}_0\subseteq \mathcal{S}\text{ finite}}
  \end{align*}
  has the finite intersection property. Thus, by compactness, there is some $\nu$ such that
  \begin{align*}
    \nu\in \bigcap\set{M_{\mathcal{S}_0}\mid \mathcal{S}_0\subseteq \mathcal{S}\text{ finite}},
  \end{align*}
  with $\nu\left(\epsilon\right) = 1$ and, for all $\alpha,\beta\in \mathcal{S}$, since $\nu\in M_{\set{\alpha,\beta,\alpha + \beta}}$, $\nu\left(\alpha + \beta\right) = \nu\left(\alpha\right) + \nu\left(\beta\right)$.

\end{proof}
\section{Proof of Tarski's Theorem}%
Finally, we are able to prove Tarski's Theorem.
\begin{proof}[of Theorem \ref{thm:tarski}]
  Let $\mathcal{S}$ be the type semigroup of the action of $G$ on $X$.\newline

  Suppose $E$ is not $G$-paradoxical. Then, $\left[E\right]\neq 2\left[E\right]$, meaning $\left(n+1\right)\left[E\right]\nleq n\left[E\right]$ for all $n\in \N$.\newline

  Thus, there is a map $\nu: \mathcal{S}\rightarrow [0,\infty]$ with $\nu\left(\left[E\right]\right) = 1$. The map
  \begin{align*}
    \mu: P(X)\rightarrow [0,\infty]
  \end{align*}
  defined by
  \begin{align*}
    \nu\left(A\right) &= \nu\left(\left[A\right]\right)
  \end{align*}
  is the desired finitely additive measure.
\end{proof}


\chapter{Movement 3: Invariant States and Følner's Condition}
Amenability, as stated earlier, is defined by a particular finitely additive, translation-invariant probability measure on the group. Of the three conditions for a mean, the ``finitely additive'' and ``probability measure'' conditions are straightforward --- we may define a measure $\delta_x$ on $P(G)$ by saying that $\delta_x(E) = 1$ if $x\in E$ and $\delta_x(E) = 0$ if $x\notin E$. This is a finitely additive probability measure --- but it is not translation-invariant.\newline

The translation-invariance condition is, generally speaking, the condition that throws a wrench into our desire to establish means on various types of groups. For instance, we desired a translation-invariant, finitely additive probability measure on $F(a,b)$, but since $bW\left( b^{-1} \right)$ is equal to $F(a,b)\setminus W(b)$, we see that the translation $bW\left( b^{-1} \right)$ creates a ``bigger'' subset than we desire, closing off our ability to construct a mean.\newline

As the reader may remark by now, this is a comically unrigorous idea. What does it mean for a set to become ``bigger'' under translation, and how much ``bigger'' does it need to become in order to close off the possibility of establishing a mean on the group?\newline

The Følner condition will allow us to make the idea of ``bigness'' precise. In this chapter, we will show exactly how the Følner condition then allows to establish amenability in groups, specifically by constructing an approximate mean, then showing that ``approximate amenability'' and amenability are equivalent.
\section{Følner's Condition}%
\begin{definition}\label{def:folner_condition}
  A group is said to satisfy the \textit{Følner condition} if, for every $\ve > 0$ and $E\subseteq G$, there is a nonempty finite subset $F\subseteq G$ such that for all $t\in E$,
  \begin{align*}
    \frac{\left\vert tF\triangle F \right\vert}{\left\vert F \right\vert}\leq \ve.
  \end{align*}
  Equivalently, we can also say that the Følner condition is satisfied if and only if
  \begin{align*}
    \frac{\left\vert tF\cap F \right\vert}{\left\vert F \right\vert} \geq 1 - \ve
  \end{align*}
  for every $\ve > 0$.
\end{definition}
\begin{lemma}\label{lemma:folner_sequences}
  A countable group $G$ satisfies the Følner condition if and only if $G$ admits a sequence $\left(F_n\right)_n$ with $F_n\subseteq G$ finite such that
  \begin{align*}
    \left(\frac{\left\vert tF_n\triangle F_n \right\vert}{\left\vert F_n \right\vert}\right)_n \xrightarrow{n\rightarrow \infty}0
  \end{align*}
  for all $t\in G$. Such a sequence is known as a \textit{Følner sequence}.
\end{lemma}
\begin{proof}
  Let $G$ admit a Følner sequence, $\left(F_n\right)_n$. Given $\ve > 0$ and $E\subseteq G$ finite, find $N$ such that for all $s\in E$ and $n\geq N$,
  \begin{align*}
    \frac{\left\vert sF_n\triangle F_n \right\vert}{\left\vert F_n \right\vert} &\leq \ve.
  \end{align*}
  We take $F = F_N$ in the definition of the Følner condition.\newline

  Let $G$ satisfy the Følner condition. We write $G = \bigcup_{n\geq 1}E_n$, with $E_1\subseteq E_2\subseteq \cdots$, and define $F_n$ such that for all $t\in E_n$,
  \begin{align*}
    \frac{\left\vert tF_n\triangle F_n \right\vert}{\left\vert F_n \right\vert} &\leq \frac{1}{n}.
  \end{align*}
  Given $t\in G$, then $t\in E_N$ for some $N$, so $t\in E_n$ For all $n\geq N$, so
  \begin{align*}
    \frac{\left\vert tF_n\triangle F_n \right\vert}{\left\vert F_n \right\vert} &\leq \frac{1}{n}
  \end{align*}
  for all $n\geq N$. Thus,
  \begin{align*}
    \left(\frac{\left\vert tF_n\triangle F_n \right\vert}{\left\vert F_n \right\vert}\right)\xrightarrow{n\rightarrow\infty}0.
  \end{align*}
\end{proof}
\begin{lemma}\label{lemma:folner_condition_generating_set}
  Let $G$ be a finitely generated group with generating set $S$ (see Definition \ref{def:generating_sets}). If $\left(F_n\right)_n$ is a sequence of finite subsets such that, for all $s\in S$,
  \begin{align*}
    \left(\frac{\left\vert sF_n\triangle F_n \right\vert}{\left\vert F_n \right\vert}\right)_n\rightarrow 0,
  \end{align*}
  then $\left(F_n\right)_n$ is a Følner sequence for $G$.
\end{lemma}
\begin{proof}
  Note that
  \begin{itemize}
    \item $s\left(A\triangle B\right) = sA\triangle sB$;
    \item $A\triangle C \subseteq \left(A\triangle B\right) \cup \left(B\triangle C\right)$.
  \end{itemize}
  We see that for any $s\in S$,
  \begin{align*}
    \frac{\left\vert s^{-1}F_n\triangle F_n \right\vert}{\left\vert F_n \right\vert} &= \frac{\left\vert s^{-1}\left(F_n\triangle sF_n\right) \right\vert}{\left\vert F_n \right\vert}\\
                                                                                     &= \frac{\left\vert F_n\triangle sF_n \right\vert}{\left\vert F_n \right\vert}\\
                                                                                     &\rightarrow 0.
  \end{align*}
  Thus, we may assume that $S$ is symmetric --- i.e., that $\set{s^{-1}| s\in S} = \set{s | s\in S}$.\newline

  For any $s,t\in S$, we have
  \begin{align*}
    \frac{\left\vert stF_n\triangle F_n \right\vert}{\left\vert F_n \right\vert} &\leq \frac{\left\vert stF_n\triangle F_n \right\vert}{\left\vert F_n \right\vert} + \frac{\left\vert sF_n\triangle F_n \right\vert}{\left\vert F_n \right\vert}\\
                                                                                 &= \frac{\left\vert s\left(tF_n\triangle F_n\right) \right\vert}{\left\vert F_n \right\vert} + \frac{\left\vert sF_n\triangle F_n \right\vert}{\left\vert F_n \right\vert}\\
                                                                                 &= \frac{\left\vert tF_n\triangle F_n \right\vert}{\left\vert F_n \right\vert} + \frac{\left\vert sF_n\triangle F_n \right\vert}{\left\vert F_n \right\vert}\\
                                                                                 &\rightarrow 0.
  \end{align*}
  We use induction to find the general case.
\end{proof}
\begin{example}
  Consider the group $\Z$. Since $\Z$ is generated by the element $\set{1}$, we see that for the sets $F_n = \set{-n,-n+1,\dots,n-1,n}$, that
  \begin{align*}
    \frac{\left\vert \left(F_n + 1\right)\triangle F_n \right\vert}{\left\vert F_n \right\vert} &= \frac{2}{2n+1}\\
                                                                                                &\rightarrow 0,
  \end{align*}
  meaning that $\Z$ satisfies the Følner condition.
\end{example}
\section{From Følner's Condition to Amenability}\label{sec:approximate_means}%
We have thus far proven that $G$ satisfies the Følner condition if and only if $G$ admits a Følner sequence, and that $G$ is amenable if and only if $G$ admits an invariant state.\newline

We will now begin harmonizing these two characterizations through the use of approximate means, eventually showing that $G$ satisfies the Følner condition if and only if $G$ admits an approximate mean, and that $G$ admits an approximate mean if and only if $G$ is amenable.
\begin{definition}\label{def:state_on_prob_g}
  For a group $G$, we define
  \begin{align*}
    \Prob\left(G\right) = \set{f\colon G\rightarrow [0,\infty) | \Card\left(\supp(f)\right)  < \infty,~\sum_{t\in G}f(t) = 1}.
  \end{align*}
  Note that $\Prob(G) \subseteq B_{\ell_1\left(G\right)}$. For $f\in \prob(G)$, we set $\varphi_f\colon \ell_{\infty}(G)\rightarrow \C$ defined by
  \begin{align*}
    \varphi_f\left(g\right) &= \sum_{t\in G}g(t)f(t).
  \end{align*}
\end{definition}
\begin{fact}\label{fact:prob_g_state}
  For $f\in \prob(G)$, $\varphi_f$ is a state on $\ell_{\infty}\left(G\right)$.
\end{fact}
\begin{proof}
We can see that, by definition, $\varphi_f$ is positive, linear, and has $\varphi_f\left(\1_{G}\right) = 1$.\newline

We only need to show that $\norm{\varphi_f}_{\op} = 1$. We see that
\begin{align*}
  \left\vert \varphi_f\left(g\right) \right\vert &= \left\vert \sum_{t\in G}g(t)f(t) \right\vert\\
                                                 &\leq \sum_{t\in G}\left\vert g(t) \right\vert\left\vert f(t) \right\vert\\
                                                 &\leq \norm{g}_{\ell_\infty}\sum_{t\in G}\left\vert f(t) \right\vert\\
                                                 &= \norm{g}_{\ell_\infty},
\end{align*}
so $\norm{\varphi_f}_{\op} \leq 1$. Since $\varphi_f\left(\1_G\right) = 1$, we must have $\norm{\varphi_f}_{\op} = 1$.
\end{proof}
\begin{proposition}
  There is an action $\lambda\colon G\rightarrow \Isom\left(\ell_{1}\left(G\right)\right)$ such that $\prob(G)$ is invariant.
\end{proposition}
\begin{proof}
  Let $\lambda_s\left(f\right)\left(t\right) = f\left(s^{-1}t\right)$. Then,
  \begin{align*}
    \norm{\lambda_s\left(f\right)}_{\ell_1} &= \sum_{t\in G}\left\vert \lambda_s\left(f\right)\left(t\right) \right\vert\\
                                     &= \sum_{t\in G}\left\vert f\left(s^{-1}t\right) \right\vert\\
                                     &= \sum_{r\in G}\left\vert f(r) \right\vert\\
                                     &= \norm{f}_{\ell_1}.
  \end{align*}
  Just as in Proposition \ref{prop:translation_action}, it is the case that $\lambda_s$ is linear. Additionally,
  \begin{align*}
    \lambda_r\circ \lambda_s\left(f\right)\left(t\right) &= \lambda_s\left(f\right)\left(r^{-1}t\right)\\
                                                         &= f\left(s^{-1}r^{-1}\left(t\right)\right)\\
                                                         &= f\left(\left(rs\right)^{-1}t\right)\\
                                                         &= \lambda_{rs}\left(f\right)\left(t\right).
  \end{align*}
  We see that if $f\in \prob(G)$, then for $f\geq 0$, we have $\lambda_s\left(f\right) \geq 0$, and
  \begin{align*}
    \sum_{t\in G}\lambda_s\left(f\right)\left(t\right) &= \sum_{t\in G}f\left(s^{-1}t\right)\\
                                                       &= \sum_{r\in G}f\left(r\right)\\
                                                       &= 1
  \end{align*}
  for any $f\in \prob(G)$.
\end{proof}
\begin{definition}\label{def:approximate_mean}
  For a countable group $G$, a sequence $\left(f_k\right)_k$ is called an approximate mean if, for all $s\in G$,
  \begin{align*}
    \norm{f_k - \lambda_s\left(f_k\right)}_{\ell_1} &\xrightarrow{k\rightarrow \infty}0.
  \end{align*}
\end{definition}
%Reorganize this one so we do Følner if and only if approximate mean, and approximate mean if and only if amenable. That allows us to include some commentary about what implicit conditions/establish facts are used
To begin the forward direction regarding the equivalence between the Følner condition, approximate means, and means, we begin by showing that the existence of a Følner sequence implies the existence of an approximate mean. Then, we will show that the existence of an approximate mean implies the existence of an invariant state (hence mean).
\begin{proposition}\label{prop:folner_implies_approx_mean}
  If $G$ admits a Følner sequence $\left(F_k\right)_k$, then $G$ admits an approximate mean.
\end{proposition}
\begin{proof}
  Set $f_k = \frac{1}{\left\vert F_k \right\vert}\1_{F_k}\in \prob(G)$. Then,
  \begin{align*}
    \norm{f_k - \lambda_s\left(f_k\right)}_{\ell_1} &= \frac{1}{\left\vert F_k \right\vert} \norm{\1_{F_k} - \lambda_s\left(\1_{F_k}\right)}_{\ell_1}\\
                                                    &= \frac{1}{F_k}\norm{\1_{F_k} - \1_{sF_k}}_{\ell_1}\\
                                               &= \frac{\left\vert F_k\triangle sF_k \right\vert}{\left\vert F_k \right\vert}\\
                                               &\rightarrow 0.
  \end{align*}
\end{proof}
% reverse direction here
To show that the existence of an approximate mean implies the Følner condition, we require the following lemma.
\begin{lemma}\label{lemma:layer_cake_representation}
  Let $f\colon S\rightarrow \R$ be finitely supported with $\sum_{s\in S}f(s) = 1$. Then, there exist subsets $\set{F_i}_{i=1}^{n}$, where $F_1\supseteq F_2\supseteq \cdots \supseteq F_n$, and constants $\set{c_i}_{i=1}^{n}$, such that
  \begin{align*}
    f &= \sum_{i=1}^{n}c_i\1_{F_i},
  \end{align*}
  where
  \begin{align*}
    \sum_{i=1}^{n}c_i\left\vert F_i \right\vert &= 1.
  \end{align*}
  This is known as the layer cake representation for $f$.
\end{lemma}
\begin{proof}
  We define $F_1 = \supp\left(f\right)$, and take $c_1 = \min\left(\Ran\left(f\right)\right)$. Taking $E_1 = f^{-1}\left(c_1\right)$ (as a set-theoretic inverse), we define $F_2 = F_1\setminus E_1$.\newline

  Take $d_1 = \min\left(f\left(F_2\right)\right)$, and define $c_2 = d_1 - c_1$. Then, defining $E_2 = f^{-1}\left(d_1\right)$, $F_3 = F_2 \setminus E_2$, and $d_2 = \min\left(f\left(F_3\right)\right)$, we define $c_3 = d_2 - c_2 - c_1$.\newline

  Continuing in this pattern, we find $d_{i-1} = \min\left(f\left(F_i\right)\right)$, $E_i = f^{-1}\left(d_{i-1}\right)$, and $c_i = d_{i-1} - \sum_{j=1}^{i-1}c_i$.\newline

  This yields a decomposition $F_1\supseteq F_2\supseteq \cdots \supseteq F_n$, where $\sum_{i=1}^{n}c_i\1_{F_i} = f$ by construction.\newline

  We now verify that $\sum_{i=1}^n c_i\left\vert F_i \right\vert = 1$.
  \begin{align*}
    1 &= \sum_{s\in S}f(s)\\
      &= \sum_{s\in S}\sum_{i=1}^{n}c_i\1_{F_i}\left(s\right).
      \intertext{By definition, if $s\in F_j$ for some $j$, we see that $c_j$ is summed for $\left\vert F_j \right\vert$ many times. Thus, we obtain}
      &= \sum_{i=1}^{n}c_i\left\vert F_i \right\vert.
  \end{align*}
\end{proof}
\begin{remark}
  Instead of using this construction where we take set-theoretic inverses and remove ``residual'' sets, there is an alternative method of construction that involves ordering the range as $r_1 < r_2< \cdots < r_n$, and constructing the set $F_k = \set{s | f(s) \geq r_k}$.
\end{remark}
We will use the layer cake decomposition to prove that if $G$ admits an approximate mean, then $G$ satisfies the Følner condition.
\begin{proposition}\label{prop:approx_mean_implies_folner}
  Let $G$ admit an approximate mean. Then, $G$ satisfies the Følner condition.
\end{proposition}
\begin{proof}
  Let $\left(f_k\right)_k$ be an approximate mean, as in Definition \ref{def:approximate_mean}. Fix a finite nonempty set $S \subseteq G$. Then, by the definition of an approximate mean, there must exist some $N\in\N$ such that for all $k\geq N$ and all $s\in G$,
  \begin{align*}
    \norm{f_k - \lambda_s\left(f_k\right)}_{\ell_1} &\leq \frac{\ve}{|S|}.
  \end{align*}
  In particular, this holds for $f_N$ and for all $s\in S$.\newline

  Since $f_N\in \Prob(G)$ is finitely supported and $\sum_{s\in G}f_N(s) = 1$, we may use Lemma \ref{lemma:layer_cake_representation} to rewrite $f_N$ as
  \begin{align*}
    f_N &= \sum_{i=1}^{n}c_i\1_{F_i},
  \end{align*}
  where $F_1 \supseteq F_2\supseteq \cdots \supseteq F_n$, and $\sum_{i=1}^{n}c_i\left\vert F_i \right\vert = 1$.\newline

  For a given $1 \leq i \leq n$, for each $s\in S$ and $t\in sF_i\triangle F_i$, we have
  \begin{align*}
    f_N\left(t\right) - f_N\left(s^{-1}t\right) &= \begin{cases}
      c_i & t\in F_i\setminus sF_i\\
      -c_i & t\in sF_i \setminus F_i
    \end{cases}.
  \end{align*}
  Thus, we see that $\left\vert f_N\left(t\right)-\lambda_s\left(f_N\right)\left(t\right) \right\vert\geq c_i$ on $sF_i\triangle F_i$. Thus, for each $s\in S$,
  \begin{align*}
    \sum_{i=1}^{n}c_i\left\vert sF_i \triangle F_i \right\vert &\leq \sum_{t\in S}\left\vert f_N\left(t\right) -  \lambda_s\left(f\right)\left(t\right)\right\vert\\
                                                               &< \frac{\ve}{\left\vert S \right\vert}\\
                                                               &= \frac{\ve}{\left\vert S \right\vert} \sum_{i=1}^{n}c_i\left\vert F_i \right\vert.
  \end{align*}
  Therefore, we have
  \begin{align*}
    \sum_{s\in S}\sum_{i=1}^{n}c_i\left\vert sF_i\triangle F_i \right\vert &< \frac{\ve}{\left\vert S \right\vert}\sum_{s\in S}\sum_{i=1}^{n}c_i\left\vert F_i \right\vert\\
                                                                           &= \ve \sum_{i=1}^{n}c_i\left\vert F_i \right\vert.
  \end{align*}
  Thus, by the pigeonhole principle, there must exist some $1\leq i \leq n$ for which
  \begin{align*}
    \sum_{s\in S}c_i\left\vert sF_i\triangle F_i \right\vert < \ve c_i\left\vert F_i \right\vert.
  \end{align*}
  Setting $F = F_i$, we find that, for all $s\in S$,
  \begin{align*}
    \frac{\left\vert sF\triangle F \right\vert}{\left\vert F \right\vert} &\leq \sum_{s\in S}\frac{\left\vert sF\triangle F \right\vert}{\left\vert F \right\vert}\\
                                                                          &< \ve.
  \end{align*}
\end{proof}
Now, we show that approximate amenability and amenability are equivalent. This will require some more heavy lifting from functional analysis.
\begin{proposition}\label{prop:approx_mean_implies_amenable}
  If $G$ admits an approximate mean, then $G$ is amenable.
\end{proposition}
\begin{proof}
  Let $\left(f_k\right)_k$ be an approximate mean.\newline

  We define $\varphi_k = \left(\varphi_{f_k}\right)_k$ (as in Definition \ref{def:state_on_prob_g}) to be a net of states on $\ell_{\infty}\left(G\right)$.\newline

  Since the state space on $\ell_{\infty}\left(G\right)$ is $w^{\ast}$-compact (Corollary \ref{cor:state_space_compact}), there is a state $\mu$ and a subnet $\left(\varphi_{k_j}\right)_j \xrightarrow{w^{\ast}}\mu$. \newline

  We only need to show that $\mu$ is invariant. Note that
  \begin{align*}
    \left\vert \mu\left(g\right) - \mu\left(\lambda_s\left(g\right)\right) \right\vert &\leq \left\vert \mu\left(g\right) - \varphi_{k_j}\left(g\right) \right\vert + \left\vert \varphi_{k_j}\left(g\right) - \varphi_{k_j}\left(\lambda_s\left(g\right)\right) \right\vert + \left\vert \varphi_{k_j}\left(\lambda_s\left(g\right)\right) - \mu\left(\lambda_s\left(g\right)\right) \right\vert
  \end{align*}
  for all $g\in \ell_{\infty}\left(G\right)$, $s\in G$, and all $j$.\newline

  Given $\ve > 0$, we find $J$ such that for $j\geq J$,
  \begin{align*}
    \left\vert \mu\left(g\right) - \varphi_{k_j}\left(g\right) \right\vert &< \ve/3\\
    \left\vert \mu\left(\lambda_s\left(g\right)\right) - \varphi_{k_j}\left(\lambda_s\left(g\right)\right)\right\vert &< \ve/3.
  \end{align*}
  We also see that
  \begin{align*}
    \left\vert \varphi_{k_j}\left(g\right) - \varphi_{k_j}\left(\lambda_s\left(g\right)\right) \right\vert &= \left\vert \sum_{t\in G}g(t)f_{k_j}\left(t\right) - \sum_{t\in G}g\left(s^{-1}t\right)f_{k_j}\left(t\right) \right\vert\\
                                                                                                           &= \left\vert \sum_{t\in G}g(t)f_{k_j}\left(t\right) - \sum_{r\in G}g(r)f_{k_j}\left(sr\right) \right\vert \tag*{$r = s^{-1}t$}\\
                                                                                                           &= \left\vert \sum_{t\in G}g(t)\left(f_{k_j}\left(t\right)-\lambda_{s^{-1}}\left(f_{k_j}\right)\left(t\right)\right) \right\vert\\
                                                                                                           &\leq \norm{g}_{\ell_\infty}\sum_{t\in G}\left\vert f_{k_j}\left(t\right) - \lambda_{s^{-1}}\left(f_{k_j}\right)\left(t\right) \right\vert\\
                                                                                                           &= \norm{g}_{\ell_\infty}\norm{f_{k_j} - \lambda_{s^{-1}}\left(f_{k_j}\right)}_{\ell_1}\\
                                                                                                           &< \ve/3
  \end{align*}
  for large $j$. Thus, we have
  \begin{align*}
    \left\vert \mu\left(g\right) - \mu\left(\lambda_{s}\left(g\right)\right) \right\vert &< \ve,
  \end{align*}
  for all $\ve > 0$, so $\mu\left(g\right) = \mu\left(\lambda_{s}\left(g\right)\right)$.
\end{proof}
%We will now commence with the reverse direction, starting by showing that amenability implies the existence of an approximate mean, and then showing that the existence of an approximate mean implies that the Følner condition is satisfied.
\begin{proposition}\label{prop:amenable_implies_approx_mean}
  If $G$ is amenable, then $G$ admits an approximate mean.
\end{proposition}
\begin{proof}
  Suppose $G$ does not admit an approximate mean. Then, there exists a finite subset $E_0\subseteq G$ and $\ve_0 > 0$ such that for all $s\in E_0$ and all $f\in \Prob(G)$, we have $\norm{f - \lambda_s\left(f\right)} \geq \ve_0$.\newline

  Consider the set
  \begin{align*}
    X &= \bigoplus_{\left\vert E_0 \right\vert} \ell_1\left(G\right),
  \end{align*}
  endowed with the norm
  \begin{align*}
    \norm{\left(f_s\right)_{s\in E_0}}_{\ell_1} &= \sum_{s\in E_0}\sum_{t\in G}\left\vert f_s(t) \right\vert\\
                                       &= \sum_{s\in E_0}\norm{f_s}_{\ell_1},
  \end{align*}
  and let
  \begin{align*}
    C &= \set{\left(f - \lambda_s\left(f\right)\right)_{s\in E_0} | f\in \Prob(G)}.
  \end{align*}
  Since $\Prob(G)$ is convex, it is the case that $C$ is convex, and since $\left\vert E_0 \right\vert$ is finite, $C$ is necessarily bounded. Note that $0\notin \overline{C}$.\newline

  By the Hahn--Banach separation for convex sets (Theorem \ref{thm:hb_separation_lctvs}), there is a real-valued $\varphi\in X^{\ast}$ such that $\varphi\left(C\right)\geq 1$. Here,
  \begin{align*}
    X^{\ast} &\cong \bigoplus_{\left\vert E_0 \right\vert}\ell_1\left(G\right)^{\ast}\\
             &\cong \sum_{\left\vert E_0 \right\vert}\ell_{\infty}\left(G\right),
  \end{align*}
  endowed with the norm
  \begin{align*}
    \norm{\left(g_s\right)_{s\in E_0}}_{\ell_{\infty}} &= \max_{s\in E_0}\left(\sup_{t\in G}\left\vert g_s(t) \right\vert\right)\\
                                                       &= \max_{s\in E_0}\norm{g_s}_{\ell_{\infty}}.
  \end{align*}
  We let $\varphi = \left(\varphi_{g_s}\right)_{s\in E_0}$, where $g_s\in \ell_{\infty}\left(G\right)$ is defined by the duality
  \begin{align*}
    \varphi_{g_s}\left(f\right) &= \sum_{t\in G}f(t)g_s(t).
  \end{align*}
  Thus, for all $f\in \Prob(G)$, we have
  \begin{align*}
    1 &\leq \varphi\left(\left(f - \lambda_s\left(f\right)\right)_{s\in E_0}\right)\\
      &= \sum_{s\in E_0}\varphi_{g_s}\left(f - \lambda-s\left(f\right)\right)\\
      &= \sum_{s\in E_0}\sum_{t\in G}\left(f - \lambda_s\left(f\right)\right)(t)g_s(t)\\
      &= \sum_{s\in E_0}\left(\sum_{t\in G}f(t)g_s(t) - \sum_{t\in G}f\left(s^{-1}t\right)g_s(t)\right)\\
      &= \sum_{s\in E_0}\left(\sum_{t\in G}f(t)g_s(t) - \sum_{r\in G}f\left(r\right)g_s\left(sr\right)\right)\\
      &= \sum_{s\in E_0}\left(\sum_{r\in G}f(r)g_s(r) - \sum_{r\in G}f(r)\lambda_{s^{-1}}\left(g\right)(r)\right)\\
      &= \sum_{s\in E_0}\sum_{r\in G}f(r)\left(g_s - \lambda_{s^{-1}}\left(g_s\right)\right)(r).
      \intertext{Note that this holds for any $f\in \Prob(G)$, including the case of $f = \delta_t$ for a given $t\in G$. We must have}
      &= \sum_{s\in E_0}\sum_{r\in G}\delta_{t}\left(r\right)\left(g_s\left(r\right) - \lambda_{s^{-1}}\left(g_s\right)\right)\left(r\right)\\
      &= \sum_{s\in E_0}\left(g_s - \lambda_{s^{-1}}\left(g_s\right)\right)\left(t\right).
  \end{align*}
  This gives
  \begin{align*}
    \1_{G} &\leq \sum_{s\in E_0}\left( g_s - \lambda_{s^{-1}}\left(g_s\right) \right)(t).
  \end{align*}
  Since $G$ is amenable, there is a mean $\mu\colon \ell_{\infty}\left(G\right)\rightarrow \C$ with $\mu\left(g_s\right) = \mu\left(\lambda_{s^{-1}}\left(g_s\right)\right)$. Therefore, we have
  \begin{align*}
    0 &= \mu\left(\sum_{s\in E_0}\left(g_s - \lambda_{s^{-1}}\left(g_s\right)\right)\left(t\right)\right)\\
      &\geq \mu\left(\1_{G}\right)\\
      &= 1,
  \end{align*}
  which is a contradiction. Therefore, $G$ admits an approximate mean.
\end{proof}
\section{Applying Følner's Condition: Groups of Subexponential Growth}\label{sec:subexponential_growth}%
Before we discuss representations of groups inside the algebra of bounded operators on a Hilbert space, we will provide an application of Følner's condition by taking a tour into geometric group theory. In this section, we will establish the amenability of yet another wide class of groups (just as we established that all abelian groups are amenable in Chapter 5) --- the groups of subexponential growth.\newline

First, we construct a little bit of machinery to understand the growth rate of a group, then we prove that Følner's condition holds for these special classes of groups.
\begin{definition}\label{def:word_metric}
  Let $G$ be a group with finite symmetric generating set $S$ (see Definition \ref{def:generating_sets}). Then, we define the word length of $g\in G$ with respect to $S$ to be
  \begin{align*}
    \ell_{G,S}\left(g\right) &= \min\set{n | g = s_1\dots s_n,~s_i\in S},
  \end{align*}
  taking $\ell_{G,S}\left(e_G\right) = 0$. We define the word metric on $G$ with respect to $S$ by taking
  \begin{align*}
    d_{S}\left(g,h\right) &= \ell_{G,S}\left(g^{-1}h\right).
  \end{align*}
\end{definition}
\begin{fact}\label{fact:word_metric_equivalent_metrics}
  If $S$ and $T$ are finite symmetric generating sets for $G$, then the respective word metrics $d_{S}$ and $d_{T}$ are equivalent (as in the sense of Definition \ref{def:metrics_and_equivalent_metrics}).
\end{fact}
\begin{proof}
  We start by showing that $d_S$ is indeed a metric. Notice that the following facts necessarily hold by our definition of the word length:
  \begin{itemize}
    \item $\ell_{G,S}\left(g\right) = \ell_{G,S}\left(g^{-1}\right)$;
    \item $\ell_{G,S}\left(gh\right) \leq \ell_{G,S}\left(g\right) + \ell_{G,S}\left(h\right)$.
  \end{itemize}
  We thus have:
  \begin{align*}
    d_{S}\left(g,h\right) &= \ell_{G,S}\left(g^{-1}h\right)\\
                          &= \ell_{G,S}\left(h^{-1}g\right)\\
                          &= d_S\left(h,g\right)\\
                          \\
    d_{S}\left(g,h\right) &= \ell_{G,S}\left(g^{-1}h\right)\\
                          &= \ell_{G.S}\left(g^{-1}kk^{-1}h\right)\\
                          &\leq \ell_{G,S}\left(g^{-1}k\right) + \ell_{G,S}\left(k^{-1}h\right)\\
                          &= d_{S}\left(g,k\right) + d_{S}\left(k,h\right)\\
                          \\
    d_{S}\left(g,g\right) &= \ell_{G,S}\left(g^{-1}g\right)\\
                          &= \ell_{G,S}\left(e_G\right)\\
                          &= 0\\
    d_{S}\left(g,h\right) = 0 &\Leftrightarrow \ell_{G,S}\left(g^{-1}h\right) = 0\\
                              &\Leftrightarrow g^{-1}h = e_{G}\\
                              &\Leftrightarrow g = h.
  \end{align*}
  Thus, $d_S$ is indeed a metric.\newline

  Let $S$ and $T$ be finite symmetric generating sets for $G$. It is sufficient to show that there exists some $k\in \N$ such that, for all $g\in G$,
  \begin{align*}
    \frac{1}{k}\ell_{G,S}\left(g\right) \leq \ell_{G,T}\left(g\right) \leq k\ell_{G,S}\left(g\right).
  \end{align*}
  Set
  \begin{align*}
    M &= \max\set{\ell_{G,T}\left(s\right) | s\in S}\\
    N &= \max\set{\ell_{G,S}\left(t\right) | t\in T}.
  \end{align*}
  Now, let $n = \ell_{G,S}\left(g\right)$, such that $g = s_1\cdots s_n$, where $s_i\in S$. Then, we have
  \begin{align*}
    \ell_{G,T}\left(g\right) &= \ell_{G,T}\left(s_1\cdots s_n\right)\\
                             &\leq \ell_{G,T}\left(s_1\right) + \cdots + \ell_{G,T}\left(s_n\right)\\
                             &\leq M\ell_{G,S}\left(g\right),
  \end{align*}
  and similarly, $\ell_{G,S}\left(g\right) \leq N\ell_{G,T}\left(g\right)$. Setting $k = \max\left(M,N\right)$, we get
  \begin{align*}
    \frac{1}{k}\ell_{G,S}\left(g\right) \leq \ell_{G,T}\left(g\right) \leq k\ell_{G,S}\left(g\right).
  \end{align*}
\end{proof}
Now, we may begin defining the growth rate of a group. We will use the fact that all word metrics with respect to a generating set are symmetric in order to show that the growth rate is well-defined (i.e., independent of the generating set for $G$).
\begin{definition}
  Let $G$ be a group with finite symmetric generating set $S$. We define
  \begin{align*}
    B_{G,S}\left(n\right) &= \set{g\in G | \ell_{G,S}\left(g\right) \leq n};\\
    \gamma_{G,S}\left(n\right) &= \left\vert B_{G,S}\left(n\right) \right\vert.
  \end{align*}
\end{definition}
The following facts hold for $\gamma$.
\begin{fact}\label{fact:properties_of_gamma_generating_set}
  Let $G$ be a finitely generated group. The following facts hold:
  \begin{enumerate}[(1)]
    \item $\gamma_{G,S}\left(n\right)$ is an increasing function;
    \item $\gamma_{G,S}\left(n+m\right)\leq \gamma_{G,S}\left(n\right)\gamma_{G,S}\left(m\right)$;
    \item $\displaystyle \lim_{n\rightarrow\infty}\left(\gamma_{G,S}\left(n\right)\right)^{1/n} = \rho_{G,S}$ exists;
    \item if $S$ and $T$ are finite symmetric generating sets for $G$, then there exists $k\in \N$ such that $\gamma_{G,T}\left(n\right)\leq \gamma_{G,S}\left(kn\right)$ for all $n\in\N$, and $\rho_{G,S} = \rho_{G,T}$.
  \end{enumerate}
\end{fact}
\begin{proof}\hfill
  \begin{enumerate}[(1)]
    \item Since $B_{G,S}\left(n\right)\subseteq B_{G,S}\left(n+1\right)$, we have $\gamma_{G,S}\left(n\right) \leq \gamma_{G,S}\left(n+1\right)$, so $\gamma_{G,S}$ is increasing.
    \item We start by showing that $B_{G,S}\left(n\right)B_{G,S}\left(m\right) = B_{G,S}\left(n+m\right)$. First, if $g\in B_{G,S}\left(n\right)$ and $h\in B_{G,S}\left(m\right)$, we know that $\ell_{G,S}\left(gh\right) \leq \ell_{G,S}\left(g\right) + \ell_{G,S}\left(h\right)\leq m+n$, so $B_{G,S}\left(n\right)B_{G,S}\left(n\right) \subseteq B_{G,S}\left(n+m\right)$. Additionally, if $g\in B_{G,S}\left(n+m\right)$, we may write
      \begin{align*}
        g &= \underbrace{s_{1}\cdots s_{\ell}}_{g_1}\underbrace{s_{\ell+1}\cdots s_{k}}_{g_2},
      \end{align*}
      where $k\leq n+m$, $\ell \leq n$, and $k-\ell \leq m$, so $g_1\in B_{G,S}\left(n\right)$ and $g_2\in B_{G,S}\left(m\right)$. Thus, we have $B_{G,S}\left(n\right)B_{G,S}\left(m\right) = B_{G,S}\left(n+m\right)$.\newline

      Now, we have
      \begin{align*}
        \gamma_{G,S}\left(n+m\right) &= \left\vert B_{G,S}\left(n+m\right) \right\vert\\
                                     &= \left\vert B_{G,S}\left(n\right)B_{G,S}\left(m\right) \right\vert\\
                                     &\leq \left\vert B_{G,S}\left(n\right) \right\vert\left\vert B_{G,S}\left(m\right) \right\vert\\
                                     &= \gamma_{G,S}\left(n\right)\gamma_{G,S}\left(m\right).
      \end{align*}
    \item From (2), we know that $\gamma_{G,S}\left(n\right) \leq \gamma_{G,S}\left(1\right)^{n}$. Inductively, we have
      \begin{align*}
        \gamma_{G,S}\left(n+1\right) &\leq \gamma_{G,S}\left(1\right)^{n+1},
      \end{align*}
      and thus,
      \begin{align*}
        1 \leq \gamma_{G,S}\left(n\right)^{1/n}\leq \gamma_{G,S}\left(1\right).
      \end{align*}
    \item We know that there exists $k$ such that $\frac{1}{k}\ell_{G,S} \leq \ell_{G,T}\leq k\ell_{G,S}$ by the proof of Fact \ref{fact:word_metric_equivalent_metrics}. Thus, if $g\in B_{G,T}\left(n\right)$, then $\ell_{G,T}\left(g\right) \leq n$, so $\ell_{G,S}\left(g\right) \leq kn$, so $g\in B_{G,S}\left(kn\right)$ and $B_{G,T}\left(n\right)\subseteq B_{G,T}\left(kn\right)$. We have $\gamma_{G,T}\left(n\right)\leq \gamma_{G,S}\left(kn\right)$.\newline

      Similarly, if $g\in B_{G,S}\left(n\right)$, then $\ell_{G,S}\left(g\right)\leq n$, so $\ell_{G,T}\left(g\right) \leq kn$, and $g\in B_{G,T}\left(kn\right)$. Thus, we get $B_{G,S}\left(n\right)\subseteq B_{G,T}\left(kn\right)$, so $\gamma_{G,S}\left(n\right)\leq \gamma_{G,T}\left(kn\right)$.\newline

      It follows that
      \begin{align*}
        \gamma_{G,S}\left(\frac{n}{k}\right)^{1/n} \leq \gamma_{G,T}\left(n\right)^{1/n} \leq \left(\gamma_{G,S}\left(kn\right)^{k}\right)^{1/kn}.
      \end{align*}
      Sending $n\rightarrow\infty$, we get $\rho_{G,S}\leq \rho_{G,T}\leq \rho_{G,S}$, so $\rho_{G,S} = \rho_{G,T}$.
  \end{enumerate}
\end{proof}
\begin{definition}
  Let $G$ be a group with finite symmetric generating set $S$. The quantity
  \begin{align*}
    \rho_{G} &= \limsup_{n\rightarrow\infty}\gamma_{G,S}\left(n\right)^{1/n}
  \end{align*}
  is known as the growth rate of the group $G$. If we have $\rho = 1$, then we say $G$ is of subexponential growth.
\end{definition}
\begin{fact}\label{fact:finite_groups_subexponential_growth}
  All finite groups are of subexponential growth.
\end{fact}
\begin{proof}
Note that since $\rho$ is independent of the generating set (as we proved in Fact \ref{fact:properties_of_gamma_generating_set}), we can set $S = G$, and we have $\limsup_{n\rightarrow\infty} \left\vert G \right\vert^{1/n} = 1$.
\end{proof}
\begin{fact}\label{fact:finitely_generated_abelian_groups_subexponential_growth}
  Let $\Gamma$ be a finitely generated abelian group. Then, $\Gamma$ is of subexponential growth.
\end{fact}
\begin{proof}
  We start by showing that $G = \Z^d$ is of subexponential growth. Notice that every element of $\Z^d$ is some linear combination of the set
  \begin{align*}
    S &= \set{e_1,e_2,\dots,e_d},\label{eq:generating_set_free_abelian_group}\tag{\textasteriskcentered}
  \end{align*}
  where
  \begin{align*}
    e_{j} &= (0,0,\dots,\underbrace{1}_{\text{position $j$}},0,0,\dots).
  \end{align*}
  Additionally, we see that any element of $B_{G,S}(n)$ is of the form $e_1^{i_1}e_2^{i_2}\dots e_d^{i_d}$, where $\sum_{j=1}^{d} i_j \leq n$. Thus, we must have $\gamma_{G,S}(n) \leq n^{d}$, meaning that 
  \begin{align*}
    \rho &= \limsup_{n\rightarrow\infty} \gamma_{G,S}(n)^{1/n}\\
         &= \limsup_{n\rightarrow\infty}n^{d/n}\\
         &= 1,
  \end{align*}
  so $\Z^d$ is of subexponential growth.\newline

  Now, if $G' = \Z^d\times \Z/k_1\Z\times \cdots \times \Z/k_r\Z$, then since there is a torsion subgroup in $G'$, we must have $\gamma_{G',S'}(n) \leq \gamma_{\Z^{d+r},T}(n)$ for any $n$, where $T$ is a generating set for $\Z^{d+r}$ and $S'$ is a generating set for $G'$. Since
  \begin{align*}
    \rho_{\Z^{d+r}} &= \limsup_{n\rightarrow\infty}\gamma_{\Z^{d+r},T}(n)^{1/n}\\
                    &= 1,
  \end{align*}
  and $1 \leq \gamma_{G',S'}(n)$, we must have $\rho_{G'} = 1$.\newline

  Since, by the fundamental theorem of finitely generated abelian groups (Theorem \ref{thm:fundamental_thm_abelian_gps}), it is the case that $\Gamma\cong \Z^{d}\times \Z/k_1\Z\times\cdots\times \Z/k_r\Z$ for some $d,k_1,\dots,k_r\in \N$, $\Gamma$ is of subexponential growth.
\end{proof}
To prove that the groups of subexponential growth are amenable, we use the following lemma from real analysis.
\begin{lemma}
  Let $\left(a_n\right)_n$ be a sequence such that $a_n > 0$ for each $n$. Then,
  \begin{align*}
    \lim_{n\rightarrow\infty}\frac{a_{n+1}}{a_n} &= \lim_{n\rightarrow\infty} \left(a_n\right)^{1/n}.
  \end{align*}
  Similarly,
  \begin{align*}
    \limsup_{n\rightarrow\infty}\frac{a_{n+1}}{a_n} &= \limsup_{n\rightarrow\infty}\left(a_n\right)^{1/n}.
  \end{align*}
\end{lemma}
\begin{theorem}\label{thm:subexponential_growth_implies_amenable}
  Let $\Gamma$ be a finitely generated group of subexponential growth. Then, $\Gamma$ is amenable.
\end{theorem}
\begin{proof}
  To prove that $\Gamma$ is amenable, we show that it satisfies the Følner condition. From the results in Section \ref{sec:approximate_means}, we know that this implies that $\Gamma$ is amenable. Let $S$ be a finite symmetric generating set for $\Gamma$.\newline

  For any $\ve > 0$, we see that there is some $k\in \N$ such that
  \begin{align*}
    \left\vert B_{\Gamma,S}\left(k\right) \right\vert^{1/k} &\leq 1 + \ve.
  \end{align*}
  Thus, by the lemma above, we must have
  \begin{align*}
    \frac{\left\vert B_{\Gamma,S}\left(k+1\right) \right\vert}{\left\vert B_{\Gamma,S} \left(k\right)\right\vert} \leq 1 + \ve.
  \end{align*}
  Note that, by Lemma \ref{lemma:folner_condition_generating_set}, we only need to verify that the Følner condition holds on $S$. For any $s\in S$, we have
  \begin{align*}
    \frac{\left\vert sB_{\Gamma,S}\left(k\right)\triangle B_{\Gamma,S}\left(k\right) \right\vert}{\left\vert B_{\Gamma,S}(k) \right\vert} &\leq \frac{2\left(\left\vert B_{G,S}\left(k+1\right) \right\vert - \left\vert B_{\Gamma,S}(k) \right\vert\right)}{\left\vert B_{\Gamma,S}(k) \right\vert}\\
                                                                                                                    &\leq 2\ve.
  \end{align*}
  Therefore, $\Gamma$ satisfies the Følner condition, hence is amenable.
\end{proof}
\begin{remark}
  %The result in Theorem \ref{thm:subexponential_growth_implies_amenable} can be used along with Fact \ref{fact:finitely_generated_abelian_groups_subexponential_growth} and Corollary \ref{cor:direct_limit_amenable} to prove Corollary \ref{cor:abelian_groups_amenable}.
  An alternative way to show that abelian groups are amenable (Corollary \ref{cor:abelian_groups_amenable}) is by using the fact that the union of a directed system of amenable groups is amenable (Corollary \ref{cor:direct_limit_amenable}) and that finitely generated abelian groups are of subexponential growth (Fact \ref{fact:finitely_generated_abelian_groups_subexponential_growth}).
\end{remark}
\section{Remarks and Notes}%
In \cite[Appendix A.3]{juschenko_amenability}, it is shown that amenability through the Følner condition only need require a constant $C < 2$ such that, for all finite $S\subseteq \Gamma$, there exists a finite $F\subseteq \Gamma$ such that for all $s\in S$,
\begin{align*}
  \frac{\left\vert sF\triangle F \right\vert}{\left\vert F \right\vert} &\leq C.
\end{align*}
The existence of such a $C$ follows from the definition of the Følner condition (Definition \ref{def:folner_condition}) --- however, the opposite direction is a bit more involved, and makes use of some results from Chapter \ref{ch:left_regular_representation}, as well as some concepts from topology.\newline

A \textit{filter} on a set $X$ is a family of subsets $\mathcal{F}\subseteq P(X)$ that does not contain $\emptyset$ and is directed by containment --- that is, if $A$ and $B$ are in $\mathcal{F}$, then $A\cap B\in \mathcal{F}$ and if $A\subseteq B$, then $B\in \mathcal{F}$. An \textit{ultrafilter} is a maximal \textit{proper} filter --- if $\mathcal{U}$ is an ultrafilter, then for any $A\in P(X)$, either $A\in \mathcal{U}$ or $A^{c}\in \mathcal{U}$.\newline

In Appendix \ref{ch:point_set_topology}, we discussed nets --- however, (\cite[Theorem 2.25]{aliprantis_infinite_dimensional_analysis}) it is actually the case that every net generates an associated filter. Similar to the case of nets, we can talk about concepts like cluster points (Definition \ref{def:cluster_points}) and limits along filters. Similarly, limits may be taken along ultrafilters.\newline

The \textit{ultraproduct} of a family of Banach spaces, $\left( X_i \right)_{i\in I}$ is defined with respect to an ultrafilter $\mathcal{U}$ on $I$. Recall that the product of a family of Banach spaces is $\prod_{i\in I}X_i$, whose elements are $\left( x_i \right)_{i\in I}$ where $\sup_{i\in I}\norm{x_i} < \infty$. To obtain the ultraproduct, we define a subspace, $N = \set{\left( x_i \right)_{i\in I} | \lim_{\mathcal{U}}\norm{x_i} = 0}$ consisting of all ``effectively zero'' elements, and then take the quotient $\prod_{i\in I}X_i/N$ to obtain the ultraproduct. The ultraproduct is usually denoted $\prod_{i\in I}X_i/\mathcal{U}$.\newline

To prove that this weakened Følner condition implies amenability, it is first proven that a unitary representation $\pi\colon \Gamma\rightarrow \mathcal{U}\left( \mathcal{H} \right)$ admits an invariant vector under the condition that $C < 2$; then, the ultraproduct of the left-regular representation (Theorem \ref{thm:left_regular_representation}) with respect to an ultrafilter $\mathcal{U}$ on $\N$ is shown to admit an invariant vector, which implies the existence of an almost-invariant vector (Definition \ref{def:almost_invariant_vector}) for the left-regular representation. Then, this implies that the group $\Gamma$ is amenable (Theorem \ref{thm:almost_invariant_vector}).

\chapter{Movement 3.5: Fixed Points}
\chapter{Movement 4: Nuclearity and Amenability in $C^{\ast}$-Algebras}
\appendix
\chapter{Point-Set Topology}
We will need a bit of background in point-set topology in order to satisfactorily understand the functional analysis behind the results in Chapters 3, 4, and 5.
\section{Axioms of Set Theory}%
In order to garner sufficient understanding of point-set topology, we need to be able to comprehend some of the essential axioms behind the objects known as ``sets.'' This is where the axioms of set theory come into play.
\begin{definition}[Zermelo--Fraenkel Axioms]
  In Zermelo--Fraenkel set theory, all objects are sets. In order to maintain convention with the way the rest of this section will refer to sets, all sets will be referred to by capital letters, and all elements of sets by lowercase letters.
  \begin{itemize}
    \item Axiom of Existence: $\exists A\left(A = A\right)$. This axiom guarantees a nonempty universe.
    \item Axiom of Extensionality: $\forall x\left(x\in A \Leftrightarrow x\in B\right)\Rightarrow A = B$. This axiom states that if two sets share the same members, then the sets are equal.
    \item Axiom Schema of Comprehension: $\exists B\:\forall x\left(x\in B\Leftrightarrow x\in A \wedge\varphi(x)\right)$. This axiom states that for any formula $\varphi(x)$, where $x$ is a free variable, there is a set $B$ such that the members of $B$ are the members of $A$ for which $\varphi$ holds.
    \item Pairing Axiom: $\forall A\:\forall B\:\exists C\left(\left(A\in Z\right)\wedge \left(B\in Z\right)\right)$. This axiom states that for any sets $A$ and $B$, there is a set $C = \set{A,B}$ that contains the sets $A$ and $B$ as elements.
    \item Power Set Axiom: $\forall A\:\exists P(A)\:\forall B\left(B\in P(A) \Leftrightarrow B\subseteq A\right)$. We use the shorthand $B\subseteq A$ to mean $\forall x\left(x\in B\Rightarrow x\in A\right)$. This axiom states that for any set $A$ there exists a set $P(A)$ such that any element of $P(A)$ is a subset of $A$, and any subset of $A$ is an element of $P(A)$.
    \item Union Axiom: $\forall \mathcal{A}\:\exists A\:\forall Y\:\forall x\left(\left(x\in Y\wedge Y\in \mathcal{A}\right)\Rightarrow x\in A\right)$. This axiom states that for any collection $\mathcal{A}$, there is a set $A $, denoted $ \bigcup \mathcal{A}$, that contains all the elements of all the sets in the collection $\mathcal{A}$.
    \item Axiom of Infinity: $\exists A\left(\emptyset\in A\wedge \forall x\left(x\in A\Rightarrow x\cup\set{x}\in A\right)\right)$. This axiom states that there is a set, $A$, such that the empty set is in $A$ and, for any element $x$, if $x\in A$, then so too is the successor, $x\cup \set{x}$.
    \item Axiom of regularity: $\forall X\left(X\neq\emptyset \Rightarrow\exists Y\left(Y\in X\wedge Y\cap X = \emptyset\right)\right)$. This axiom states that any nonempty set $X$ contains a set $Y$ such that $Y$ and $X$ are disjoint. As a consequence, any chain of sets descending in membership must terminate.
    \item Axiom Schema of Replacement: $\forall A\:\exists B\:\forall v\left(v\in B\Rightarrow \exists u\left(u\in A\wedge \psi\left(u,v\right)\right)\right)$. The axiom schema of replacement says that for a function-like formula (a formula such that $\psi\left(u,v\right)\wedge \psi\left(u,w\right) \Rightarrow v=w$) $\psi\left(u,v\right)$, there is a set $A$ consisting of exactly those sets/elements $v\in B$ that correspond to $u\in A$.
  \end{itemize}
\end{definition}
The final axiom, the Axiom of Choice, is special, and as a result, we state it separately, for we will be using some of its consequences in the future sections. The following is one way of interpreting the axiom of choice.
\begin{definition}[Axiom of Choice]
  Let $\set{S_i}_{i\in I}$ be an indexed collection of nonempty sets. Then, there exists an indexed set $\set{x_i}_{i\in I}$ such that $x_i\in S_i$ for each $I$.\newline

  Equivalently, if $\set{S_i}_{i\in I}$ is an indexed collection of nonempty sets, then there is some choice function
  \begin{align*}
    f\in \prod_{i\in I}S_i.
  \end{align*}
\end{definition}
On its own, this formulation of the Axiom of Choice is not particularly useful. However, there is a statement of the Axiom of Choice which is just as useful.
\begin{definition}[Preorders, Partial Orders, Total Orders, and Well-Orders]
Let $X$ be a set, and $\preceq $ be a relation on $X$. We say a relation is a preorder if it is reflexive and transitive:
\begin{itemize}
  \item $a\preceq a$
  \item $a\preceq b \wedge b\leq c\Rightarrow a\preceq c$.
\end{itemize}
We say $X$ is a directed set if, for any $a,b\in X$, there is $c\in X$ such that $a\preceq c$ and $b\preceq c$.\newline

If $\preceq$ is also antisymmetric --- that is, $a\preceq b\wedge b\preceq a \Rightarrow a = b$ --- then, we say $\preceq$ is a partial order.\newline

We say $m\in X$ is a maximal element if, for any $x\in X$ with $m\preceq x$, $m = x$.\newline

If $X$ is partially ordered by $\preceq$ and, for any two elements $a,b\in X$, either $a\preceq b$ or $b\preceq a$, then we say $\preceq$ is a total order on $X$.\newline

If $X$ is a totally ordered set that has the property that, for any nonempty $A\subseteq X$, there is some $x\in A$ such that for any $y\in A$, $x\prec y$ for all $y \in A$ with $y\neq x$, then we say $\preceq$ is a well-order on $X$.
\end{definition}
\begin{example}
  \begin{itemize}
    \item The set $\N$ with the usual ordering is a well-ordered set.
    \item If $A$ is a set, then $P(A)$ with the containment ordering, $A\preceq B$ if $A\supseteq B$, is a partially ordered set.
    \item Similarly, if $A$ is a set, then $P(A)$ with the inclusion ordering, $A\preceq B$ if $A\subseteq B$, is a partially ordered set.
    \item A collection of functions $\set{\varphi_{i}: Z_i\rightarrow Y}_{i\in I}$ ordered by $\varphi_{i}\preceq \varphi_j$ if $Z_i\subseteq Z_j$ and $\varphi_{j}|_{Z_i} = \varphi_i$, is a partially ordered set. This is often known as the extension ordering.
  \end{itemize}
\end{example}

We can state an equivalent formulation of the Axiom of Choice as follows.
\begin{theorem}[Zorn's Lemma]
  If $\left(X,\preceq\right)$ is a partially ordered set with the property that for all $C\subseteq X$ with $C$ totally ordered, $C$ has an upper bound, then $X$ has a maximal element.
\end{theorem}
There are many proofs of both Zorn's Lemma from the Axiom of Choice and the Axiom of Choice from Zorn's Lemma. However, we will mostly be using it for the purposes of proving other theorems. The following results can be proven using Zorn's Lemma.
\begin{example}
  \begin{itemize}
    \item Every $\F$-vector space $V$ has a basis $B\subseteq V$ such that the set of all finite linear combinations of elements of $B$ over $\F$ is $V$.
    \item If $\varphi$ is a continuous linear functional defined on a subspace $W\subseteq V$, there is an extension $\Phi$ such that $\Phi|_{W} = \varphi$. %See: Hahn--Banach Theorems
    \item The arbitrary product of compact spaces is compact. %See Tychonoff's Theorem.
  \end{itemize}
\end{example}
\section{Metric Spaces}%
Building upon the basics of set theory, we move towards understanding metric spaces.
\subsection{Basics of Metric Spaces}%
\begin{definition}[Metrics]
  Let $X$ be a set. A distance metric is a function
  \begin{align*}
    d: X\times X\rightarrow [0,\infty)
  \end{align*}
  such that the following three properties are satisfied:
  \begin{itemize}
    \item if $x,y\in X$ and $d\left(x,y\right) = 0$, then $x = y$;
    \item $d\left(x,y\right) = d\left(y,x\right)$ for all $x,y\in X$;
    \item $d\left(x,z\right) \leq d\left(x,y\right) + d\left(y,z\right)$ for all $x,y,z\in X$.
  \end{itemize}
  A function that satisfies the latter two properties is called a semimetric.\newline

  Two metrics $d$ and $\rho$ on $X$ are equivalent if there exist constants $c_1,c_2\geq 0$ such that
  \begin{align*}
    d\left(x,y\right) &\leq c_1 \rho\left(x,y\right)\\
    \rho\left(x,y\right) &\leq c_2 d\left(x,y\right)
  \end{align*}
  for all $x,y\in X$.\newline

  A metric space is a pair $\left(X,d\right)$, where $d$ is a metric.
\end{definition}
\begin{example}[Some Distance Metrics]
  \begin{itemize}
    \item The discrete metric on any nonempty set is given by
      \begin{align*}
        d\left(x,y\right) & \begin{cases}
          1 & x\neq y\\
          0 & x = y
        \end{cases}
      \end{align*}
    \item The Euclidean metric between $\left(x_1,\dots,x_n\right)$ and $\left(y_1,\dots,y_n\right)$ in $\R^n$ is
      \begin{align*}
        d_{2}\left(x,y\right) &= \left(\sum_{j=1}^{n}\left\vert y_j-x_j \right\vert^2\right)^{1/2}.
      \end{align*}
    \item Other metrics on $\R^n$ include
      \begin{align*}
        d_1\left(x,y\right) &= \sum_{j=1}^{n}\left\vert y_j-x_j \right\vert\\
        d_{\infty}\left(x,y\right) &= \max_{j=1}^{n}\left\vert y_j - x_j \right\vert.
      \end{align*}
      All of $d_1,d_2,d_{\infty}$ are equivalent metrics.
    \item The Hamming distance between two strings of bits is
      \begin{align*}
        d_{H}: \set{0,1}^{n}\times \set{0,1}^{n}\rightarrow [0,\infty)\\
        d_{H}\left(\left(x_{j}\right)_{j=1}^{n},\left(y_j\right)_{j=1}^{n}\right) &= \left\vert \set{j\mid x_j\neq y_j} \right\vert.
      \end{align*}
    \item The set $C\left([0,1],\R\right)$ consisting of continuous real-valued functions from $[0,1]$ to $\R$ can be equipped with
      \begin{align*}
        d_u\left(f,g\right) &= \sup_{t\in [0,1]}\left\vert f(t) - g(t) \right\vert,
      \end{align*}
      which is the uniform metric, or
      \begin{align*}
        d_{1}\left(f,g\right) &= \int_{0}^{1} \left\vert f(t)-g(t) \right\vert\:dt.
      \end{align*}
    \item All subsets of a metric space $X$ equipped with the same metric is also a metric space.
    \item If $\rho$ is a metric on $X$, then we can create a distance metric
      \begin{align*}
        d\left(x,y\right) &= \frac{\rho\left(x,y\right)}{1 + \rho\left(x,y\right)}
      \end{align*}
      that is bounded on $[0,1]$.
    \item If $d_1,\dots,d_n$ are metrics on $X$ and $c_1,\dots,c_n > 0$ are constants, then
      \begin{align*}
        d\left(x,y\right) &= \sum_{k=1}^{n}c_kd_k\left(x,y\right)
      \end{align*}
      defines a metric on $X$.
    \item If $\left(\rho_k\right)_k$ is a family of separating semimetrics for $X$ --- i.e., for $x,y\in X$ distinct, there is some $\rho_{j}$ such that $\rho_j\left(x,y\right) \neq 0$ --- then, we can obtain bounded semimetrics by taking
      \begin{align*}
        d_k\left(x,y\right) &= \frac{\rho_k\left(x,y\right)}{1 + \rho_k\left(x,y\right)}
      \end{align*}
      for each $k$. From each $d_k$, we define
      \begin{align*}
        d\left(x,y\right) &= \sum_{k=1}^{n}2^{-k}d_k\left(x,y\right),
      \end{align*}
      which is a metric on $X$.
    \item If $\left(X_k,\rho_k\right)_{k\geq 1}$ is a sequence of metric spaces, then we can form the product space
      \begin{align*}
        X &= \prod_{k\geq 1}X_{k}
      \end{align*}
      with the metric
      \begin{align*}
        D\left(f,g\right) &= \sum_{k\geq 1}d_k\left(f(k),g(k)\right).
      \end{align*}
      Here, $d_k = \frac{\rho_k}{1 + \rho_k}$ is the corresponding bounded metric to $\rho_k$.
  \end{itemize}
\end{example}
\begin{definition}[Open and Closed Sets]
  Let $\left(X,d\right)$ be a metric space.
  \begin{enumerate}[(1)]
    \item For $x\in X$ and $\delta > 0$, we define
      \begin{enumerate}[(a)]
        \item the open ball at $x$ with radius $\delta > 0$
          \begin{align*}
            U\left(x,\delta\right) &= \set{y\in X\mid d\left(y,x\right) < \delta};
          \end{align*}
        \item the closed ball centered at $x$ with radius $\delta > 0$
          \begin{align*}
            B\left(x,\delta\right) &= \set{y\in X\mid d\left(y,x\right)\leq \delta};
          \end{align*}
        \item the sphere centered at $x$ with radius $\delta > 0$
          \begin{align*}
            S\left(x,\delta\right) &= \set{y\in X\mid d\left(y,x\right) = \delta}.
          \end{align*}
      \end{enumerate}
    \item A set $V\subseteq X$ is open if, for all $x\in V$, there is $\delta > 0$ such that $U\left(x,\delta\right)\subseteq V$.\newline

      A subset $C\subseteq X$ is closed if $C^{c}$ is open.
    \item If $x\in V$ and $V\subseteq X$ is open, then we say $V$ is an open neighborhood of $x$. A neighborhood of $x$ is any subset $N\subseteq X$ such that $N$ contains an open neighborhood of $x$.
    \item If $A\subseteq X$ is any subset, the interior of $A$ is
      \begin{align*}
        A^{\circ} &:= \bigcup\set{V\mid V\text{ is open, }V\subseteq A},
      \end{align*}
      the closure of $A$ is
      \begin{align*}
        \overline{A} &= \bigcap\set{C\mid C\text{ is closed, }A\subseteq C},
      \end{align*}
      and the boundary of $A$ is
      \begin{align*}
        \partial A &= \overline{A} \setminus A^{\circ}.
      \end{align*}
  \end{enumerate}
\end{definition}
We can now talk about the topology of the metric space.
\begin{fact}
  Let $\left(X,d\right)$ be a metric space, and let
  \begin{align*}
    \mathcal{U} = \set{V\mid V\subseteq X\text{ open}}.
  \end{align*}
  Then, the following are true.
  \begin{itemize}
    \item $\emptyset\in \mathcal{U},X\in \mathcal{U}$.
    \item If $\set{V_{i}}_{i\in I}$ is a family of open sets, then $\bigcup_{i\in I}V_i\in \mathcal{U}$.
    \item If $\set{V_i}_{i=1}^{n}$ is a finite collection of open sets, then $\bigcap_{i=1}^{n}V_i \in \mathcal{U}$.
  \end{itemize}
\end{fact}

\begin{definition}
  Let $\left(X,d\right)$ be a metric space. Suppose $A\subseteq X$ is a nonempty subset.
  \begin{enumerate}[(1)]
    \item The distance from a point $x\in X$ to the set $A$ is defined by
      \begin{align*}
        \dist_{A}\left(x\right) &= \inf_{a\in A}d\left(x,a\right).
      \end{align*}
    \item The diameter of $A$ is defined by
      \begin{align*}
        \diam\left(A\right) &= \sup_{x,y\in A}d\left(x,y\right).
      \end{align*}
    \item If $\diam(A) < \infty$, then we say $A$ is bounded.
    \item If, for every $\delta > 0$, there is a finite subset $F_{\delta}\subseteq X$ such that
      \begin{align*}
        A\subseteq \bigcup_{x\in F_{\delta}}U\left(x,\delta\right).
      \end{align*}
    \item For $A,B\subseteq X$, we define the Hausdorff distance between $A$ and $B$ to be
      \begin{align*}
        d_{H}\left(A,B\right) &= \max\set{\sup_{x\in A}\dist_{B}\left(x\right),\sup_{y\in B}\dist_{A}\left(y\right)}.
      \end{align*}
  \end{enumerate}
\end{definition}
\begin{example}
  Let $\Omega$ be a nonempty set, and $\left(X,d\right)$ be a metric space. A function $f: \Omega\rightarrow X$ is said to be bounded if $\diam\left(\ran(f)\right) < \infty$.\newline

  The collection $\operatorname{Bd}\left(\Omega,X\right)$ denotes all bounded functions with domain $\Omega$ and codomain $X$.\newline

  On $ \operatorname{Bd}\left(\Omega,X\right)$, we define the uniform metric by
  \begin{align*}
    D_{u}\left(f,g\right) &= \sup_{x\in\Omega}d\left(f(x),g(x)\right).
  \end{align*}
\end{example}
\subsection{Convergence and Continuity in Metric Spaces}%
\begin{definition}[Crash Course on Sequences]
  Let $\left(X,d\right)$ be a metric space.
  \begin{enumerate}[(1)]
    \item A sequence in $X$ is a map $x: \N\rightarrow X$, which we call $\left(x_{n}\right)_{n}$ or $\left(x_{n}\right)_{n\geq 1}$.
    \item A natural sequence is a strictly increasing sequence of natural numbers $\left(n_{k}\right)_{k\geq 1}$ with $n_{k}\geq k$ and $n_{k} < n_{k+1}$.
    \item If $\left(n_k\right)_{k}$ is a natural sequence, the sequence $\left(x_{n_k}\right)_{k}$ is called a subsequence of $\left(x_{n}\right)_n$.
    \item We say $\left(x_n\right)_n\rightarrow x$ if $d\left(x_n,x\right)_{n} \xrightarrow{n\rightarrow\infty} 0$. We say $x$ is the limit of $\left(x_n\right)_n$.
  \end{enumerate}
\end{definition}
\begin{example}[Convergence in Metric Spaces of Functions]
  \begin{itemize}
    \item If $\Omega$ is a nonempty set, and $\left(X,d\right)$ is a metric space, the sequence of functions $f_n: \Omega\rightarrow X$ is said to converge pointwise to $f: \Omega\rightarrow X$ if
      \begin{align*}
        f_n\left(x\right)\xrightarrow{n\rightarrow\infty}f(x)
      \end{align*}
      for each $x\in \Omega$.
    \item If $\left(f_n\right)_n\in \operatorname{Bd}\left(\Omega,X\right)$ is a sequence, we say $\left(f_n\right)_n\rightarrow f$ converges uniformly if
      \begin{align*}
        D_u\left(f_n,f\right)\xrightarrow{n\rightarrow\infty}0,
      \end{align*}
      or, equivalently,
      \begin{align*}
        \sup_{x\in\Omega}d\left(f_n(x),f(x)\right)\xrightarrow{n\rightarrow\infty}0.
      \end{align*}
  \end{itemize}
\end{example}
\begin{definition}[Sequential Criteria for Closure]
  If $\left(X,d\right)$ is a metric space, and $E\subseteq X$ is nonempty, then $E$ is closed if and only if, for all $\left(x_n\right)_n\rightarrow x$ with $x_n\in E$, $x\in E$.\newline

  If $E\subseteq X$ is any nonempty set, then $\overline{E}$ is precisely the set of all $x\in X$ such that $\left(x_n\right)_n\rightarrow x$ for some $\left(x_n\right)_n\subseteq E$.
\end{definition}
\begin{definition}[Completeness]
  Let $\left(X,d\right)$ be a metric space.
  \begin{itemize}
    \item If $\left(x_n\right)_n$ is a sequence in $X$ such that for all $\ve > 0$, there is $N\in \N$ such that for all $m,n\geq N$, $d\left(x_m,x_n\right) < \ve$, then we say the sequence is called Cauchy.
    \item If, for any $\left(x_n\right)_n$ Cauchy, $\left(x_n\right)_n\rightarrow x$ in $X$, then we say $X$ is complete.
    \item If $\left(X,d\right)$ is complete, then for any $A\subseteq X$ closed, $A$ is also complete.
    \item If $A\subseteq X$ is complete as a metric space, then $A$ is closed.
  \end{itemize}
\end{definition}
\begin{example}
  The metric space $\Q$ with the metric inherited from $\R$ is not complete. For instance, there is a sequence of rational numbers $\left(2,2.7,2.71,2.718,\dots\right)$ converging to $e$, but $e\notin \Q$.\newline

  The space $\operatorname{Bd}\left(\Omega,X\right)$ is complete if $X$ is complete.
\end{example}
\begin{definition}[Continuity]
  \begin{itemize}
    \item Let $\left(X,d\right)$ and $\left(Y,\rho\right)$ be metric spaces, and let $f: X\rightarrow Y$ be a function. We say $f$ is continuous at $x$ if, for every $\ve > 0$, there is $\delta > 0$ such that $z\in U\left(x,\delta\right)\Rightarrow \rho\left(f(x),f(z)\right) < \ve$.
    \item If $f$ is continuous at every point in $X$, then we say $f$ is continuous.
    \item If $f$ is bijective, continuous, and $f^{-1}$ is continuous, then we say $f$ is a homeomorphism.
    \item We say $f$ is uniformly continuous on $X$ if, for any $\ve > 0$, there is $\delta > 0$ such that for any $y,z\in X$, $d\left(y,z\right) < \delta \Rightarrow \rho\left(f(y),f(z)\right) < \ve$.
    \item We say $f$ is Lipschitz if there exists $C > 0$ such that $d\left(x,y\right) \leq Cd\left(f(x),f(y)\right)$ for all $x,y\in X$.
    \item We say $f$ is an isometry if $d\left(x,y\right) = d\left(f(x),f(y)\right)$ for all $x,y\in X$.
  \end{itemize}
\end{definition}
\begin{fact}
  Let $f: X\rightarrow Y$ be a map between metric spaces. The following are equivalent:
  \begin{enumerate}[(i)]
    \item $f$ is continuous;
    \item if $V\subseteq Y$ is open, then $f^{-1}\left(V\right)\subseteq X$ is open;
    \item if $\left(x_n\right)_n\rightarrow x$ in $X$, then $\left(f\left(x_n\right)\right)_n\rightarrow f(x)$ in $Y$.
  \end{enumerate}
\end{fact}
\begin{fact}
  If $M$ and $N$ are metric spaces with $N$ complete, and $A\subseteq M$ is dense, then if $f: A\rightarrow N$ is uniformly continuous, then there is a unique uniformly continuous map $\tilde{f}: M\rightarrow N$.
\end{fact}


\chapter{Measure Theory and Integration}
In order to properly discuss amenability, we need a strong foundation in measure theory.

\section{Constructing Measurable Spaces}%
Fix a set $\Omega$. We let $\mathcal{A} =\set{A_{i}}_{i\in I}$ be a collection of subsets of $\Omega$.
\begin{definition}[Algebra of Subsets]
  The collection $\mathcal{A} = \set{A_i}_{i\in I}$ is known as an algebra of subsets for $\Omega$ if
  \begin{itemize}
    \item $\emptyset,\Omega \in \mathcal{A}$;
    \item for any $A_i\in \mathcal{A}$, $A_i^{c}\in \mathcal{A}$;
    \item for any $A_i,A_j\in \mathcal{A}$, $A_i \cup A_j \in \mathcal{A}$.
  \end{itemize}
\end{definition}
We can refine the concept of an algebra of subsets to consider countable unions rather than finite unions. This is known as a $\sigma$-algebra.
\begin{definition}[$\sigma$-Algebra of Subsets]
  The collection $\mathcal{A} = \set{A_i}_{i\in I}$ is known as a $\sigma$-algebra of subsets for $\Omega$ if
  \begin{itemize}
    \item $\emptyset,\Omega \in \mathcal{A}$;
    \item for any $A_i\in \mathcal{A}$, $A_i^{c}\in \mathcal{A}$;
    \item for any countable collection $\set{A_n}_{n\geq 1}\subseteq \mathcal{A}$, $\bigcup_{n\geq 1}A_{n} \in \mathcal{A}$.
  \end{itemize}
\end{definition}
\begin{definition}[Measurable Space]
A pair $\left(\Omega,\mathcal{A}\right)$, where $\Omega$ is a set and $A\subseteq P(\Omega)$ is a $\sigma$-algebra, is called a measurable space. Elements in the measurable space are called $\mathcal{A}$-measurable sets.
\end{definition}
\begin{definition}[Restriction of a $\sigma$-Algebra]
  For a measurable space $\left(\Omega,\mathcal{A}\right)$, with $E\in \mathcal{A}$, the family
  \begin{align*}
    \mathcal{A}_{E} &= \set{E\cap A\mid A\in \mathcal{A}}
  \end{align*}
  is a $\sigma$-algebra on $E$, known as the restriction of $\mathcal{A}$ to $E$.
\end{definition}
\begin{definition}[Produced $\sigma$-Algebra]
Let $\left(\Omega,\mathcal{A}\right)$ be a measurable space, and $f: \Omega\rightarrow \Lambda$ is a map. The $\sigma$-algebra produced by $f$ on $\Lambda$ is the collection
\begin{align*}
  \mathcal{N} &= \set{E\mid E\subseteq \Lambda,~f^{-1}(E) \in \mathcal{A}}.
\end{align*}

\end{definition}

\begin{definition}[Generated $\sigma$-Algebra]
  For a family $\mathcal{E}\subseteq P\left(\Omega\right)$, the $\sigma$-algebra generated by $E$ is the smallest $\sigma$-algebra that contains $E$.
  \begin{align*}
    \sigma\left(\mathcal{E}\right) &= \bigcap_{\substack{\mathcal{E}\in \mathcal{M}_i \\ \text{$\mathcal{M}_i$ $\sigma$-Algebra}}} \mathcal{M}_i
  \end{align*}
\end{definition}
\begin{definition}[Borel $\sigma$-Algebra]
  If $\Omega$ is a topological space with topology $\tau\subseteq P(\Omega)$, we define
  \begin{align*}
    \mathcal{B}_{\Omega} &= \sigma\left(\tau\right)
  \end{align*}
  to be the Borel $\sigma$-algebra.
\end{definition}
All open, closed, clopen, $F_{\sigma}$, and $G_{\delta}$ subsets of $\Omega$ are Borel.\break

We can now begin examining measurable functions.
\begin{definition}[Measurable Functions]
  Let $\left(\Omega,\mathcal{M}\right)$ and $\left(\Lambda,\mathcal{N}\right)$ be measurable spaces.
  \begin{enumerate}[(1)]
    \item We say a map $f: \Omega\rightarrow \Lambda$ is $\mathcal{M}$-$\mathcal{N}$-measurable if $f^{-1}\left(E\right)\in \mathcal{M}$ for all $E\in \mathcal{N}$.
    \item We say a map $f: \Omega\rightarrow \R$ is measurable if it is $\mathcal{M}$-$\mathcal{B}_{\R}$-measurable.
    \item We say a map $f: \Omega\rightarrow \C$ is measurable if both $\re(f)$ and $\im(f)$ are measurable.
  \end{enumerate}
  The set of all measurable functions on $\left(\Omega,\mathcal{M}\right)$ is denoted $L_{0}\left(\Omega,\mathcal{M}\right)$.\newline

  The collection of all bounded measurable functions is the set
  \begin{align*}
    B_{\infty}\left(\Omega,\mathcal{M}\right) &= \set{f\in L_0\left(\Omega,\mathcal{M}\right)\mid \sup_{x\in\Omega}\left\vert f(x) \right\vert < \infty}.
  \end{align*}
\end{definition}

\begin{example}
  If $f: \Omega\rightarrow \Lambda$ is a continuous map between topological spaces, then $f$ is $\mathcal{B}_{\Omega}$-$\mathcal{B}_{\Lambda}$-measurable, since
  \begin{align*}
    \mathcal{F} &= \set{E\subseteq \Lambda\mid f^{-1}\left(E\right)\in \mathcal{B}_{\Omega}}
  \end{align*}
  is a $\sigma$-algebra containing every open set in $\Lambda$, so $\mathcal{F}$ contains $\mathcal{B}_{\Lambda}$.
\end{example}

\begin{example}
  If $\left(\Omega,\mathcal{M}\right)$ is a measurable space, and $f: \Omega\rightarrow \Lambda$ is a map, the measurable space $\left(\Lambda,\mathcal{N}\right)$ produced by $f$ is necessarily measurable.
\end{example}

\begin{fact}\label{fact:composition}
  If $\left(\Omega,\mathcal{M}\right)$, $\left(\Lambda,\mathcal{N}\right)$, and $\left(\Sigma,\mathcal{L}\right)$ are measurable spaces, with $f: \Omega\rightarrow \Lambda$ and $g: \Lambda\rightarrow \Sigma$ measurable, then $g\circ f$ is measurable.
\end{fact}
\begin{proof}[Proof of Fact \ref{fact:composition}]
  If $E\in \mathcal{L}$, then $g^{-1}\left(E\right) \in \mathcal{N}$, so $f^{-1}\left(g^{-1}\left(E\right)\right)\in \mathcal{M}$. Thus, $\left(g\circ f\right)^{-1}\left(E\right)\in \mathcal{M}$, so $g\circ f$ is measurable.
\end{proof}

\begin{proposition}
  Let $\left(\Omega,\mathcal{M}\right)$ be a measurable space. Let $\F = \C$ or $\R$. Suppose $f,g,h_n: \Omega\rightarrow \F$ are measurable for $n\geq 1$.
  \begin{enumerate}[(1)]
    \item If $\alpha \in \F$, then $f + \alpha g$ is measurable.
    \item $\overline{f}$ is measurable.
    \item $fg$ is measurable.
    \item $\frac{f}{g}$ is measurable assuming it is well-defined.
    \item if $h_n$ are $\R$-valued, and $\left(h_n\left(x\right)\right)_n$ is bounded for each $x\in \Omega$, then $\sup h_n$ and $\inf h_n$ are measurable.
    \item If $f$ and $g$ are $\R$ valued, then $\max\left(f,g\right)$ and $\min\left(f,g\right)$ are measurable. In particular,
      \begin{align*}
        f_{+} &= \max\left(f,0\right)\\
        f_{-} &= \max\left(0,-f\right)
      \end{align*}
      are measurable.
    \item $\left\vert f \right\vert$ is measurable.
    \item The pointwise limit of measurable functions is measurable --- if $\lim_{n\rightarrow\infty}h_n\left(x\right)$ exists for all $x\in \Omega$, then $h = \lim_{n\rightarrow\infty}h_n$ is measurable.
  \end{enumerate}
\end{proposition}
\begin{definition}[Simple Functions]
  A simple function $s: \Omega\rightarrow \F$ is a function with finite range. In other words, $s$ is of the form
  \begin{align*}
    s &= \sum_{k=1}^{n}c_k\1_{E_k}
  \end{align*}
  for $E_k\subseteq \Omega$ and $c_k\in \F$.\newline

  A simple function is measurable if and only if $E_k\in \mathcal{M}$ for each $k$.
\end{definition}

\section{Constructing Measures}%
A measure assigns a nonnegative ``length'' or ``volume'' to measurable sets.
\begin{definition}[Basics of Measures]
  A measure on a measurable space $\left(\Omega,\mathcal{M}\right)$ is a map $\mu: \mathcal{M}\rightarrow \left[0,\infty\right]$ that satisfies the following.
  \begin{enumerate}[(i)]
    \item $\mu\left(\emptyset\right) = 0$;
    \item $\displaystyle \mu\left(\bigsqcup_{j=1}^{\infty}E_j\right) = \sum_{j=1}^{\infty}\mu\left(E_j\right)$.
  \end{enumerate}
  The triple $\left(\Omega,\mathcal{M},\mu\right)$ is called a measure space.\newline

  A measure $\mu$ is finite if $\mu\left(\Omega\right) < \infty$\newline

  If $\mu\left(\Omega\right) = 1$, then $\mu$ is called a probability measure.\newline

  A measure $\mu$ is called finitely additive if $\mu\left(E\sqcup F\right) = \mu(E) + \mu(F)$.\newline

  A measure $\mu$ is called $\sigma$-finite if there is a countable family $\set{E_n}_{n\geq 1}\subseteq \mathcal{M}$ such that
  \begin{align*}
    \Omega &= \bigcup_{n\geq 1}E_n
  \end{align*}
  and $\mu\left(E_n\right) < \infty$.\newline

  A measure $\mu$ on $\left(\Omega,\mathcal{M}\right)$ is called semi-finite if, for every $E\in \mathcal{M}$ with $\mu(E) = \infty$, there exists $F\in \mathcal{M}$ with $F\subseteq E$ and $0 < \mu(F) < \infty$.
\end{definition}

\begin{lemma}
  Let $\left(\Omega,\mathcal{M},\mu\right)$ be a measure space.
  \begin{enumerate}[(1)]
    \item If $E,F\in \mathcal{M}$ with $F\subseteq E$, then $\mu\left(F\right) \subseteq \mu\left(E\right)$.
    \item If $\left(E_n\right)_{n}$ is a sequence of measurable sets, then
      \begin{align*}
        \mu\left(\bigcup_{n\geq 1}E_n\right) &\leq \sum_{n=1}^{\infty}\mu\left(E_n\right).
      \end{align*}
    \item If $\left(E_n\right)_{n\geq 1}$ is an increasing family of measurable sets, then
      \begin{align*}
        \mu\left(\bigcup_{n\geq 1}E_n\right) &= \lim_{n\rightarrow\infty}\mu\left(E_n\right).
      \end{align*}
  \end{enumerate}
\end{lemma}
\begin{proof}\hfill
  \begin{enumerate}[(1)]
    \item Since $F\subseteq E$, we can write $E = F\sqcup \left(E\setminus F\right)$. Thus,
      \begin{align*}
        \mu\left(E\right) &= \mu\left(F\right) + \mu\left(E\setminus F\right)\\
                          &\geq \mu\left(F\right).
      \end{align*}
    \item We write
      \begin{align*}
        F_1 &= E_1\\
        F_2 &= E_2\setminus E_1\\
            &\vdots\\
        F_n &= E_n\setminus \left(\bigcup_{k=1}^{n-1}E_k\right).
      \end{align*}
      Since each $F_n$ is measurable, and $F_n\subseteq E_n$, we have
      \begin{align*}
        \mu\left(\bigcup_{n\geq 1}E_n\right) &= \mu\left(\bigsqcup_{n\geq 1}F_n\right)\\
                                             &= \sum_{n=1}^{\infty}F_n\\
                                             &\leq \sum_{n=1}^{\infty}\mu\left(E_n\right).
      \end{align*}
    \item We write $F_n$ as the respective disjoint union for $\set{E_n}_{n\geq 1}$. We have $\bigsqcup_{k=1}^{n}F_k = E_n$. Then,
      \begin{align*}
        \mu\left(\bigcup_{n\geq 1}E_n\right) &= \sum_{n=1}^{\infty}\mu\left(F_n\right)\\
                                             &= \lim_{n\rightarrow\infty}\left(\sum_{k=1}^{n}\mu\left(F_k\right)\right)\\
                                             &= \lim_{n\rightarrow\infty}\mu\left(\bigsqcup_{k=1}^{n}F_k\right)\\
                                             &= \lim_{n\rightarrow\infty}\mu\left(E_n\right).
      \end{align*}
  \end{enumerate}
\end{proof}
\begin{definition}[Counting Measure]
  If $\Omega$ is any set, the \textbf{counting measure} on $\left(\Omega,P\left(\Omega\right)\right)$ assigns $\left\vert A \right\vert$ for each $A\in P\left(\Omega\right)$ finite, and $\infty$ for any infinite subset.
\end{definition}
\begin{definition}[Restricting Measures]
  If $\left(\Omega,\mathcal{M},\mu\right)$ is a measure space, $\mathcal{B}$ is a $\sigma$-algebra on $\Omega$ with $\mathcal{B}\subseteq \mathcal{M}$, the restriction $\mu|_{\mathcal{B}}$ is a measure on $\left(\Omega,\mathcal{B}\right)$.\newline

  If $E\in \mathcal{M}$, we can restrict $\mu$ to $\mathcal{M}_{E}$ (the restriction of $\mathcal{M}$ to $E$), yielding the measure space $\left(E,\mathcal{M}_{E},\mu|_{\mathcal{M}_E}\right)$. We denote this restricted measure $\mu_{E}$, such that $\mu_{E}\left(M\cap E\right) = \mu\left(M\cap E\right)$ for all $M\in \mathcal{M}_{E}$.
\end{definition}
\begin{definition}[Pushforward Measure]
  Let $\left(\Omega,\mathcal{M},\mu\right)$ be a measure space, and let $\left(\Lambda,\mathcal{N}\right)$ be a measurable space. Let $f: \Omega\rightarrow \Lambda$ be measurable. The map
  \begin{align*}
    f_{\ast}\mu: \mathcal{N}\rightarrow [0,\infty]
  \end{align*}
  defined by
  \begin{align*}
    f_{\ast}\mu\left(E\right) &= \mu\left(f^{-1}\left(E\right)\right)
  \end{align*}
  defines a measure on $\left(\Lambda,\mathcal{N}\right)$. This is known as the pushforward measure of $\mu$.\newline

  If $\mathcal{N}$ on $\Lambda$ is produced by $f$, then the pushforward measure is necessarily defined on $\mathcal{N}$, and that any function $g: \Lambda\rightarrow \F$ is measurable if and only if $g\circ f$ is measurable.
\end{definition}
\begin{definition}[Disjoint Union]
  Let $\set{\left(\Omega_n,\mathcal{M}_n,\mu_n\right)}$ be a countable family of measure spaces.\newline

  We define the co-product of this family by taking
  \begin{align*}
    \Sigma := \bigsqcup_{n=1}^{\infty}\Omega_n,
  \end{align*}
  to be our set equipped with the canonical inclusion map $\iota_{n}\left(x\right) = \left(x,n\right)$, such that for each $n$,
  \begin{align*}
    \mathcal{M} &:= \set{E\subseteq \Sigma\mid \iota_{n}^{-1}\left(E\right)\in \mathcal{M}_n}.
  \end{align*}
  The measure is defined by
  \begin{align*}
    \mu: \mathcal{M}\rightarrow [0,\infty]\\
    \mu(E) := \sum_{n=1}^{\infty}\mu_{n}\left(\iota_{n}^{-1}\left(E\right)\right).
  \end{align*}
  We can identify each $\Omega_n$ with the subset $\Omega_{n}^{\ast} = \set{\left(x,n\right)\mid x\in \Omega_n}\subseteq \Sigma$, with $\iota_{n}^{-1}\left(E\right)\subseteq \Omega_{n}$ identified with $E\cap \Omega_{n}^{\ast}$.\newline

  The family $\set{\Omega_{n}^{\ast}}_{n\geq 1}$ forms a measurable partition of $\Sigma$, and that $\mu|_{\Omega_{n}^{\ast}}$ are the pushforwards of $\mu_{n}$ by $\iota_{n}$.\newline

  Note that a map $f: \Sigma\rightarrow \C$ is measurable if and only if $f\circ \iota_{n}: \Omega_n\rightarrow \C$ is measurable for all $n$.\newline

  If $\left(f_n: \Omega_n\rightarrow \C\right)_{n}$ is a sequence of measurable maps, the disjoint union
  \begin{align*}
    f = \bigsqcup_{n\geq 1}f_n: \Sigma\rightarrow \C
  \end{align*}
  defined by $f\left(x,n\right) = f_n(x)$, is measurable.
\end{definition}

\chapter{Functional Analysis}
\nocite{*}
\printbibliography
\end{document}
