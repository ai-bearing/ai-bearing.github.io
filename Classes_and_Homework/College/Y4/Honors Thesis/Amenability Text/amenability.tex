\documentclass[11pt]{package2}

%\usepackage{cmbright,sfmath,bbold}
%\renewcommand{\mathcal}{\mathtt}
\setcounter{chapter}{-1}
%\usepackage{newpxtext,eulerpx,eucal}
%\renewcommand*{\mathbb}[1]{\varmathbb{#1}}
%\renewcommand*{\hbar}{\hslash}
\usepackage[light]{kpfonts}
\renewcommand{\coloneq}{\coloneqq}
\renewcommand{\N}{\Z_{>0}}
\usepackage{titlesec}
\usepackage[backend=biber,style=alphabetic,sorting=nty]{biblatex}
\addbibresource{chapters/references.bib}
\DeclareMathOperator{\op}{op}
\DeclareMathOperator{\sa}{s.a.}
%\renewcommand{\coloneq}{=}
%\makeatletter
%\renewcommand{\maketitle}{%
%    \begin{center}
%        \rule{\textwidth}{0.6pt} \\[0.4em]  % Top line
%        \rule{\textwidth}{0.6pt} \\[1em]
%        {\LARGE\bfseries \@title} \\[0.5em] % Main Title
%        {\large\itshape \@subtitle} \\[1em] % Subtitle
%        {\large \@author} \\[0.5em]         % Author
%        {\large \@date} \\[1em]             % Date
%        \rule{\textwidth}{0.6pt} \\[0.4em]  % Bottom line
%        \rule{\textwidth}{0.6pt} \\[1em]
%    \end{center}
%}
%\makeatother
\title{Understanding Amenability in Discrete Groups\\\vspace{5pt}{\large A Gentle Introduction to Higher Analysis}}
%\newcommand{\subtitle}{A Gentle Introduction to Higher Analysis}
\author{Avinash Iyer}
\date{March 31, 2025}
\usepackage{microtype}
\hbadness=10000
\usepackage[hidelinks]{hyperref}
\begin{document}
\maketitle
\RaggedRight
\tableofcontents
%\part{Prelude}
\chapter{Introduction}
\section{Overview}%
In the beginning, God created the heavens and the Earth,\footnote{Well, maybe not God specifically.} and a lot of other things that are detailed in the book of Genesis. Unfortunately, those that wrote down and translated the book of Genesis failed to mention the most important feature of the universe that He (may or may not have) created --- the axiom of choice. It may be remarked that the axiom of choice is not, strictly speaking, a God-given creation, but accepting it certainly requires a leap of faith --- after all, Paul Cohen and Kurt Gödel showed that it is independent of the rest of the axioms of set theory --- but since we are going to be working in the realm of analysis throughout this thesis, we will be accepting it as such.\newline

Unfortunately, despite our best efforts, and the convenient results that the axiom of choice provides for (see Example \ref{ex:results_follow_from_axiom_of_choice}), the axiom of choice provides some counterintuitive and downright paradoxical results --- not that it's worth throwing out, but it is certainly worth investigating and understanding. One of these counterintuitive results is detailed in Chapter \ref{ch:paradoxical_decompositions}, where we show implicitly that there does not exist a finitely additive measure on the three-dimensional real numbers that is also invariant under Euclidean isometry by proving the Banach--Tarski paradox in its most general form.\newline

Using the Banach--Tarski paradox as motivation, we then go to proving various characterizations, definitions, and proofs of amenability in groups --- i.e., we now want to understand when a group is well-behaved, rather than ill-behaved as in the case of the isometry group of $\R^3$. We first use some primarily group-theoretic techniques surrounding amenability, such as in the proof of Tarski's theorem in Chapter \ref{ch:tarskis_theorem} and the establishment of amenability in subgroups and quotient groups in the first section of Chapter \ref{ch:invariant_states}. We then use techniques from functional analysis to prove amenability, first by establishing the equivalence between amenability and the existence of an invariant state on $\ell_{\infty}(G)$, in sections 2--4 of Chapter \ref{ch:invariant_states}; we then expand on these techniques in Chapter \ref{ch:folner_condition} to understand Følner's condition and approximate means.\newline

After a quick discussion of the application of Følner's condition to geometric group theory, we discuss representations of groups as bounded operators on Hilbert spaces, using yet more techniques from functional analysis and the theory of operator algebras to show, in Chapter \ref{ch:left_regular_representation}, that a group is amenable if it satisfies certain criteria related to the left-regular representation of $\Gamma$ on the space $\ell_2\left(\Gamma\right)$. Finally, in Chapter \ref{ch:nuclearity}, we move from the representation of a group to representations of its group $\ast$-algebra, and show how properties of the group $C^{\ast}$-algebra inform us about properties of the group, and vice-versa.\newline

All vector spaces in this thesis are assumed to be over $\C$ unless otherwise specified, $0\in\N$, all groups are endowed with the discrete topology, and unless it is apparent otherwise, should be assumed countable and finitely generated. Nonetheless, even with this relatively limited scope, we can still establish some crucial results that provide a worthy harmonization of algebra and analysis.
\section{Apologies and Acknowledgments}%
As is evident from the file size or anyone who has a PDF viewer that reports the number of pages of a document, this thesis is certainly much, much longer than an undergraduate honors thesis generally is. Part of this is my fault --- I am more verbose and particular about spacing than the average mathematics writer --- and part of this is because the content was extremely fun to learn, and I just kept learning about it.\newline

This project's topic was originally conceived by professor Rainone in a footnote to one of the problems in problem set 8 of Real Analysis II in Spring 2024. The problem mentioned the idea of an amenable group and a paradoxical group, and had us prove the easy direction of Tarski's Theorem (Theorem \ref{thm:tarski}), with the footnote saying that the previous direction was a suitable honors project. After much hemming and hawing by yours truly, an appendix in the book \textit{Crossed Products of $C^{\ast}$-Algebras} by Dana Williams eventually convinced me that amenability was a topic worth exploring and understanding. It was a very good idea.\newline

Furthermore, as I dove deeper into the functional analysis necessary to understand the more heavy results in amenability, professor Rainone's draft textbook, \textit{Functional Analysis-En Route to Operator Algebras} was an incredible resource that helped me really understand the fundamentals of a subject that I had long desired to learn, but where most of the books were quite terse and hard to follow. I hope the text gets published sometime in the future, it is an extremely valuable resource.\newline

I have always believed that an expository text in mathematics need not be a bland affair. When it is a topic that the author is interested in, such as myself with the concepts, theories, and ideas surrounding amenability, I believe it is of paramount importance that the author make the subject as enjoyable for the reader as it is for them. Thus, I have included light humor throughout this thesis, hopefully without interfering with the substance of the mathematical content, with the purpose of bringing a smile to the readers' face just as this topic has brought many a smile to mine.\newline

The results and proofs in this thesis are primarily not my own, but a compilation of various sources that provide a broad and deep coverage of the subject. I have collected, simplified, exposited, and reordered them in order to understand not only for myself, but to potentially help others with the process of understanding amenability. I have attempted my best to attribute all theorems and proofs to the texts that I have obtained them from --- just because I may have forgotten to attribute a proof to someone does not mean it is my own (indeed, it is probably not).
%The groups in this thesis should be assumed countable and finitely generated unless it is otherwise clear that they are not (such as $\text{SO}\left( 3 \right)$, which is obviously not countable or finitely generated), and all are endowed with the discrete topology. This is because if I dedicated the necessary time towards understanding and including the case of topological groups with any locally compact topology, this thesis would actually be longer than the Bible. Nonetheless, the case of countable discrete groups still provides an extensive elaboration on concepts often covered in undergraduate analysis and algebra, and, as is the case with the Banach--Tarski paradox, strike at the foundations of mathematics.
\chapter{Categorical Constructions for the Unemployed Mathematician}\label{ch:categorical_constructions}
In this book, we cover certain structures --- like the free group, free $\ast$-algebra, tensor product, etc. --- that are usually not covered in the undergraduate algebra or analysis curriculum in depth. We discuss these ``free'' constructions\footnote{Hence the name of this chapter.} here, with the general theme that these constructions allow us to, in a ``universal'' manner, convert one type of map (a set-map or a bilinear map) into another type of map (a group homomorphism or a linear map).
\section{Free Groups}%
Given a set $A$, we want to know how exactly we can create a group structure from the elements in $A$ such that they extend from $A$ to a group generated by $A$ in a particularly ``natural'' way. This will be the free group, whose properties we will discuss in Chapter \ref{ch:paradoxical_decompositions}.
\begin{definition}\label{def:generating_sets}
  Let $G$ be a group, and $S\subseteq G$ be a subset. We define the subgroup \textit{generated by} $S$ to be
  \begin{align*}
    \left\langle S \right\rangle_{G} &= \bigcap \set{H | S\subseteq H,~H\text{ a subgroup}}.
  \end{align*}
  We say $S$ generates $G$ if $\left\langle S \right\rangle_{G} = G$.\newline

  We say $\left\langle S \right\rangle_{G}$ is \textit{finitely generated} if $\Card(S) < \infty$.\newline

  If $S$ is such that, for any $x\in S$, we have $x^{-1}\in S$, then we say $S$ is \textit{symmetric}.
\end{definition}
\begin{fact}
  If $S = \set{s_1,\dots,s_n}\subseteq G$, then the picture of $\left\langle S \right\rangle$ is as follows:
  \begin{align*}
    \left\langle S \right\rangle_{G} &= \set{s_1^{a_1}s_2^{a_2}\cdots s_n^{a_n} | n\in\N,~s_1,\dots,s_n\in S,~a_1,\dots,a_n\in \set{-1,1}}.
  \end{align*}
\end{fact}
To construct a free group, we begin by stating its universal property --- that is, its innate nature as an ``extension'' of a set-function into a group structure. Then, we will show that a more constructive definition of the free group satisfies this universal property. The following section draws heavily from \cite{loh_geometric_group_theory}, but we will mostly focus on the construction of the free group rather than the proof of uniqueness.
\begin{definition}\label{def:free_group}
  Let $S$ be a set. A group $F$ containing $S$ is said to be \textit{freely generated} if, for every group $G$, and every map $\phi\colon S\rightarrow G$, there is a unique group homomorphism $\varphi\colon F\rightarrow G$ that extends $\varphi$. The following diagram, where $\iota$ denotes the inclusion of $S$ into $F$, commutes:
    \begin{center}
      \begin{tikzcd}
        S \arrow[d, "\iota"', hook] \arrow[r, "\phi"] & G \\
        F \arrow[ru, "\varphi"']                      &  
      \end{tikzcd}
    \end{center}
  We say $F$ is the \textit{free group} generated by $S$.
\end{definition}
Intuitively, to construct the free group, if we have $a\mapsto \phi(a)$ between $S$ and $G$, then we will define $\varphi\left(a^n\right) = \phi(a)^n$ inside $F(S)$. Uniqueness will follow from the fact that we can take two groups that satisfy the universal property, $F$ and $F'$, and apply the universal property on set-valued functions between $S$ and $F$ and $S$ and $F'$ respectively. 
\begin{theorem}\label{thm:free_group_existence}
  If $S$ is some set, then there is some freely generated group $F(S)$ that satisfies \ref{def:free_group}.
\end{theorem}
\begin{proof}
  We will construct a group consisting of ``words'' made up of elements of $S$ and their inverses. This starts by considering the alphabet $A = S\cup \widehat{S}$, where $\widehat{S}$ is a disjoint copy of $S$ --- every $\hat{s}\in \widehat{S}$ will play the role of an inverse to $s$ in our group.
  \begin{itemize}
    \item Define $A^{\ast}$ to be the set of all words over the alphabet $A$, including the empty word, $\epsilon$. We define the operation $A^{\ast}\times A^{\ast}\rightarrow A^{\ast}$ by concatenating words, which is an associative operation with neutral element $\epsilon$.
    \item Define the equivalence relation $\sim$ generated by the following two relations, where for all $x,y\in A^{\ast}$ and $s\in S$, we have
      \begin{align*}
        xs\hat{s}y &\sim xy\\
        x\hat{s}sy &\sim xy.
      \end{align*}
      The equivalence classes with respect to $\sim$ will be denoted $\left[\cdot\right]$.\newline

      We have a well-defined composition $\left[x\right]\left[y\right] = \left[xy\right]$ mapping $F(S) \times F(S) \rightarrow F(S)$ for all $x,y\in A^{\ast}$.
  \end{itemize}
  We show that $F(S)$ with the concatenation operation is a group. Here, we see that $\left[\epsilon\right]$ is the neutral element for the composition, and associativity is inherited from associativity of concatenation in $A^{\ast}$. To show the existence of inverses, we define the inverse map inductively by taking $I\left(\epsilon\right) = \epsilon$, and
  \begin{align*}
    I\left(sx\right) &= I(x)\hat{s}\\
    I\left(\hat{s}x\right) &= I(x)s
  \end{align*}
  for all $x\in A^{\ast}$  and $s\in S$. Inductively, we can see that $I\left(I\left(x\right)\right) = x$ and
  \begin{align*}
    \left[I(x)\right]\left[x\right] &= \left[I(x)x\right]\\
                                    &= \left[\epsilon\right]\\
    \left[x\right]\left[I(x)\right] &= \left[xI(x)\right]\\
                                    &= \left[\epsilon\right].
  \end{align*}
  Thus, $F(S)$ is a group.\newline

  Now, we show $F(S)$ is freely generated. Let $i\colon S\rightarrow F(S)$ be the map that sends $s\mapsto \left[s\right]$. By our construction, we know that $i(S)\subseteq F(S)$ is a generating set for $F(S)$. We will show the universal property holds for $F(S)$.\newline

  To start, let $\phi\colon S\rightarrow G$ be a set-valued map between $S$ and an arbitrary group $G$. We construct $\phi^{\ast}\colon A^{\ast}\rightarrow G$ by taking
  \begin{align*}
    \epsilon &\mapsto e\\
    sx &\mapsto \phi(s)\phi^{\ast}\left(x\right)\\
    \hat{s}x &\mapsto \left(\phi(s)\right)^{-1}\phi^{\ast}\left(x\right)
  \end{align*}
  for all $x\in A^{\ast}$ and $s\in S$. This definition of $\phi^{\ast}$ is compatible with the equivalence relation on $A^{\ast}$, and we see that $\phi^{\ast}\left(xy\right) = \phi^{\ast}\left(x\right)\phi^{\ast}\left(y\right)$. Thus, we get a well-defined map $\varphi\colon F(S)\rightarrow G$, taking $\left[x\right]\mapsto \left[\phi^{\ast}\left(x\right)\right]$.\newline

  It remains to be shown that the map $i\colon S\rightarrow F(S)$ is injective, which will show that $F(S)$ is freely generated by $S$. Let $s_1,s_2\in S$, and consider the set-function $\phi\colon S\rightarrow \Z$ given by $\phi\left(s_1\right) = 1$ and $\phi\left(s_2\right) = -1$. Then, we must have
  \begin{align*}
    \varphi\left(i\left(s_1\right)\right) &= \phi\left(s_1\right)\\
                                          &= 1\\
                                          &\neq -1\\
                                          &= \phi\left(s_2\right)\\
                                          &= \varphi\left(i\left(s_2\right)\right).
  \end{align*}
  Thus, we have $i\left(s_1\right)\neq i\left(s_2\right)$, so $i$ is injective.
\end{proof}
Most of the definitions of the free group automatically default to the characterization of $F(S)$ as the set of reduced words in $S\cup S^{-1}$. This is the characterization we will be using in the future, but it is still important to understand where exactly the ``free'' in free group comes from, and how it relates to the particular universal property that actually characterizes $F(S)$ uniquely up to isomorphism.
\section{Free Vector Spaces}%
Given a set $A$, just as we are able to construct a free group, $F(A)$, we can take any set $A$ and construct a ``universal'' vector space out of the set.\newline

The free vector space (as it is known) is the universal object that extends any set-valued function into a linear map, treating elements of the set as its basis (see Definition \ref{def:basis}). We are interested in the case of the free vector space over the complex numbers, but note that the following definition of the free vector space applies over any field. 
\begin{theorem}\label{thm:free_vector_space}
  Let $\Gamma$ be a nonempty set. There is a vector space, $\C\left[\Gamma\right]$, with $\Dim\left(\C\left[\Gamma\right]\right) = \Card\left(\Gamma\right)$, and an injective map $\delta\colon \Gamma\rightarrow \C\left[\Gamma\right]$ such that the following universal property holds: if $V$ is a $\C$-vector space, and $\phi\colon \Gamma\rightarrow V$ is a set-valued function, then there is a unique linear map $T_{\phi}\colon \C\left[\Gamma\right]\rightarrow V$ such that $T_{\phi}\circ \delta = \phi$.
  \begin{center}
    % https://tikzcd.yichuanshen.de/#N4Igdg9gJgpgziAXAbVABwnAlgFyxMJZABgBpiBdUkANwEMAbAVxiRAB12BxOgW17ogAvqXSZc+QigCM5KrUYs2nAMLJOPfnQrDRIDNjwEis6fPrNWiEADVh8mFADm8IqABmAJwi8kZEDgQSLIKlsrssAw4giIe3r6IIYFIAEzUDHQARjAMAAriRlIgDDDuOCDUFkrWACoA+sCcaAAWWEK6cT5+1MmIacVZOfmGkmyeWE7N5ZWKVhzsLVj2QkA
\begin{tikzcd}
\Gamma \arrow[r, "\delta"] \arrow[rd, "\phi"'] & {\C[\Gamma]} \arrow[d, "T_{\phi}"] \\
                                               & V                                 
\end{tikzcd}
  \end{center}
\end{theorem}
\begin{proof}
  Consider the linear subspace of finitely supported functions, $\C\left[\Gamma\right]\subseteq \mathcal{F}\left(\Gamma,\C\right)$. For each $t\in \Gamma$, we define
  \begin{align*}
    \delta_t\left(s\right) &= \begin{cases}
      1 & s=t\\
      0 & \text{else}
    \end{cases}.
  \end{align*}
    We see that $\delta_t\neq \delta_s$ whenever $s\neq t$, meaning that the map $\delta\colon \Gamma\rightarrow \C\left[\Gamma\right]$, defined by $s \mapsto \delta_s$, is injective.\newline

    We will show that $\set{\delta_s}_{s\in \Gamma}$ is a linear basis for $\C\left[\Gamma\right]$. If $f\in \C\left[\Gamma\right]$, with $\supp\left(f\right) = \set{s_1,\dots,s_n}\subseteq \Gamma$, we set $\alpha_j = f\left(t_j\right)$, and see that
    \begin{align*}
      f &= \sum_{j=1}^{n}\alpha_j\delta_{s_j},
    \end{align*}
    which shows that $\set{\delta_s}_{s\in\Gamma}$ is a spanning set.\newline

    To show that $\set{\delta_s}_{s\in\Gamma}$ is linearly independent, consider $g = \sum_{j=1}^{n}\alpha_j\delta_{s_j}\in \C\left[\Gamma\right]$ such that $g = 0$. Then, $g(t) = 0$ for all $t\in\Gamma$, and in particular, $g\left(s_i\right) = 0$ for every $1 \leq i \leq n$. Thus, we have
    \begin{align*}
      0 &= g\left(s_j\right)\\
        &= \sum_{j=1}^{n}\alpha_j\delta_{s_j}\left(s_i\right)\\
        &= \alpha_i,
    \end{align*}
    so $\alpha_j = 0$ for each $j$. Thus, $\set{\delta_s}_{s\in \Gamma}$ is linearly independent.\newline

    Turning to the universal property, we define $T_{\phi}\colon \C\left[\Gamma\right]\rightarrow V$ in terms of $\phi$ as follows:
    \begin{align*}
      T_{\phi}\left(\sum_{j=1}^{n}\alpha_j\delta_{s_j}\right) &= \sum_{j=1}^{n}\alpha_j\phi\left(s_j\right).
    \end{align*}
    This yields an expression of $T_{\phi}$ uniquely in terms of $\phi$ and $\delta$.
\end{proof}
\begin{example}
  Let $z$ be an abstract variable, and consider the set of ``formal powers'' of $z$, $\set{z^k}_{k\in\N}$. Then, the free vector space generated by this set, $\C\left[z\right]$, is the set of all polynomials with coefficients in $\C$. By the universal property, we know that every polynomial $p\in \C\left[z\right]$ has a unique expression $p = \sum_{j=0}^{n}a_jz^j$.
\end{example}
One of the primary uses of the free vector space is that, via this construction, we can show that vector spaces are particularly nice algebraic objects. We often use these properties implicitly in linear algebra.
\begin{theorem}\label{thm:injective_projective_objects}
  Let $X$, $Y$, and $Z$ be vector spaces.
  \begin{enumerate}[(a)]
    \item If $\iota\colon Y\hookrightarrow X$ is an injective linear map, and $\varphi\colon Y\rightarrow Z$ is a linear map, then there is a (not necessarily unique) map $T\colon X\rightarrow Y$ such that $T\circ\iota = \varphi$.
      \begin{center}
        % https://tikzcd.yichuanshen.de/#N4Igdg9gJgpgziAXAbVABwnAlgFyxMJZABgBpiBdUkANwEMAbAVxiRGJAF9T1Nd9CKAIzkqtRizYBNLjxAZseAkRFCx9Zq0QgAWrN6KBRAEyjqGydoAaXMTCgBzeEVAAzAE4QAtkhEgcEEgAzOYSWiAAOhH4OHQg1LFYDGwAFhAQANb6IB7evgmBiKYgDHQARjAMAAp8SoIg7lgOKTjx4ppsUfTuaClY2bk+iCH+hcWlFdW1RtoMMK6toR3aACoDnkNko76cFJxAA
\begin{tikzcd}
0 \arrow[r] & Y \arrow[r, "\iota", hook] \arrow[d, "\varphi"'] & X \arrow[ld, "T"] \\
            & Z                                                &                  
\end{tikzcd}
      \end{center}
This shows that vector spaces are injective objects --- any linear map factors through an injective map.
    \item If $\pi\colon X\rightarrow Z$ is a surjective linear map, and $\varphi\colon Y\rightarrow Z$ is a linear map, then there is a (not necessarily unique) map $\delta\colon Y\rightarrow X$ such that $\pi\circ\delta = \varphi$.
      \begin{center}
        % https://tikzcd.yichuanshen.de/#N4Igdg9gJgpgziAXAbVABwnAlgFyxMJZARgBoAGAXVJADcBDAGwFcYkQBNEAX1PU1z5CKMsWp0mrdgC0efEBmx4CRcqTE0GLNohAANOfyVCiAJnXitU3eR7iYUAObwioAGYAnCAFskakDgQSOYgjPQARjCMAAoCysIgHliOABY4IJqSOiAAOjmwjDj0hiCePsE0gUhkoRFRscYqukmp6Zna7HloWCVlvoj+VYg1Vtl5DB5oKT00YZExcSa6jDBu6bzuXv01QwDM3JTcQA
        \begin{tikzcd}
                            & Y \arrow[ld, "\delta"'] \arrow[d, "\varphi"] &   \\
        X \arrow[r, "\pi"'] & Z \arrow[r]                                  & 0
        \end{tikzcd}
      \end{center}
      This shows that vector spaces are projective objects --- any linear map factors through a surjective map.
  \end{enumerate}
\end{theorem}
\begin{proof}\hfill
  \begin{enumerate}[(a)]
    \item Let $\mathcal{A}$ be a basis for $Y$. Then, since $\iota$ is an injective linear map, the set $\mathcal{B}_0 = \set{\iota(y) | y\in \mathcal{A}}$ can be extended to a basis $\mathcal{B}$ for $X$.\newline

      We set $t\colon \mathcal{B}\rightarrow Z$ to be
      \begin{align*}
        t\left(x\right) &= \begin{cases}
          \varphi(x) & x\in \mathcal{B}_0\\
          0 & x\in \mathcal{B}\setminus \mathcal{B}_0
        \end{cases}.
      \end{align*}
      By the universal property of the free vector space, this extends to a linear map $T\colon X\rightarrow Y$. Since $T\circ\iota$ agrees with $\varphi$ on $\mathcal{A}$, the universal property of the free vector space states that $T\circ\iota$ agrees with $\varphi$ on all of $Y$.
    \item Let $\set{y_i}_{i\in I}$ be a basis for $Y$. We define $d\left( y_i \right) = x_i\in  \pi^{-1}\circ \varphi\left( y_i \right)$ for each $i\in I$, where $x_i\in \pi^{-1}\circ \varphi\left( y_i \right)$ is some representative.  By the universal property of the free vector space, this extends to a unique linear map $\delta\colon Y\rightarrow X$ that agrees on the basis of $Y$.
  \end{enumerate}
\end{proof}

\section{Free Algebras}%
Chapter 8 of this thesis will be focused on understanding the properties of the (reduced) group $C^{\ast}$-algebra. This will require some background in the theory of algebras, so we will understand the purely algebraic properties here before diving into the analytic properties in Chapter 5.
%\begin{definition}\label{def:star_algebra}
%  Let $A$ be an algebra over $\C$ (see Definition \ref{def:vector_space_and_algebra}). An involution on $A$ is a unary operation $\ast\colon A\rightarrow A$ that satisfies the following, for all $a,b\in A$ and $\alpha\in\C$:
%  \begin{itemize}
%    \item $\left(a^{\ast}\right)^{\ast} = a$;
%    \item $\left(a+b\right)^{\ast} = a^{\ast} + b^{\ast}$;
%    \item $\left(\alpha a\right)^{\ast} = \overline{\alpha} a^{\ast}$;
%    \item $\left(ab\right)^{\ast} = b^{\ast}a^{\ast}$.
%  \end{itemize}
%  If $A$ is equipped with an involution, we say $A$ is a $\ast$-algebra.
%\end{definition}
%\begin{example}
%  Consider the algebra of $n\times n$ matrices over $\C$, $\Mat_n\left(\C\right)$, with element-wise addition and scalar multiplication, as well as traditional matrix multiplication. From linear algebra, we know that if $T\in \Mat_n\left(\C\right)$, then $T$ admits a unique adjoint map (or conjugate transpose), $T^{\ast}$, such that for any $x,y\in \C^n$, $ \iprod{T\left(x\right)}{y} =  \iprod{x}{T^{\ast}\left(y\right)}$, where $ \iprod{\cdot}{\cdot} $ denotes the complex inner product. The map $\ast\colon \Mat_n\left(\C\right)\rightarrow \Mat_n\left(\C\right)$, given by $T\mapsto T^{\ast}$, satisfies the definition of an involution.\newline
%
%  If $x,y\in \C^n$, $\alpha\in\C$, and $T,S\in \Mat_n\left(\C\right)$,
%  \begin{align*}
%    \iprod{T\left(x\right)}{y} &= \iprod{x}{T^{\ast}\left(y\right)}\\
%                               &= \overline{ \iprod{T^{\ast}\left(y\right)}{x} }\\
%                               &= \overline{ \iprod{y}{T^{\ast\ast}\left(x\right)} }\\
%                               &= \iprod{T^{\ast\ast}\left(x\right)}{y}\\
%                               \\
%    \iprod{\left(T+S\right)\left(x\right)}{y} &= \iprod{T\left(x\right)}{y} + \iprod{S\left(x\right)}{y}\\
%                                              &= \iprod{x}{T^{\ast}\left(y\right)} + \iprod{x}{S^{\ast}\left(y\right)}\\
%                                              &= \iprod{x}{\left(T^{\ast} + S^{\ast}\right)\left(y\right)}\\
%                                              \\
%    \iprod{\alpha T\left(x\right)}{y} &= \iprod{T\left(\alpha x\right)}{y}\\
%                                      &= \iprod{\alpha x}{T^{\ast}\left(y\right)}\\
%                                      &= \iprod{x}{\overline{\alpha}T^{\ast}\left(y\right)}\\
%                                      \\
%    \iprod{TS\left(x\right)}{y} &= \iprod{S\left(x\right)}{T^{\ast}\left(y\right)}\\
%                                &= \iprod{x}{S^{\ast}T^{\ast}\left(y\right)}.
%  \end{align*}
%  Thus, we can see that $\Mat_n\left(\C\right)$ is a $\ast$-algebra.
%\end{example}
%\begin{definition}
%  Let $A$ be a $\ast$-algebra, and let $B\subseteq A$.
%  \begin{itemize}
%    \item We say $B$ is self-adjoint (or $\ast$-closed) if for any $x\in B$, $x^{\ast}\in B$.
%    \item We say $B$ is a subalgebra of $A$ if $B\subseteq A$ is a linear subspace and for any $b_1,b_2\in B$, $b_1b_2\in B$. If $B$ is $\ast$-closed, then we say $B$ is a $\ast$-subalgebra.
%    \item If $B$ is a subalgebra such that, for any $b\in B$ and $a\in A$, $ab\in B$ and $ba\in B$, then $B$ is an ideal. If $B$ is $\ast$-closed, then we say $B$ is a $\ast$-ideal.
%  \end{itemize}
%  If $J\subseteq A$ is a $\ast$-ideal, the linear space $A/J$ admits multiplication and involution, defined by
%  \begin{align*}
%    \left(a+J\right)\left(b+J\right) &= ab + J\\
%    \left(a+J\right)^{\ast} &= a^{\ast}+J.
%  \end{align*}
%  For any $a\in A$, the $\ast$-ideal generated by $a$ is denoted
%  \begin{align*}
%    \operatorname{ideal}\left(a\right) &= \bigcap\set{J\subseteq A | a\in J\text{ and }J\text{ is a $\ast$-ideal} }.
%  \end{align*}
%  We say an ideal $J\subseteq A$ is maximal if $J$ is a proper ideal in $A$ and, if $J\subseteq I$ for some ideal $I$, then $I = J$ or $I = A$.
%\end{definition}
%\begin{definition}
%  Let $A$ and $B$ be $\ast$-algebras.
%  \begin{itemize}
%    \item An algebra homomorphism between $A$ and $B$ is a linear map $\varphi\colon A\rightarrow B$ that preserves the multiplication structure.
%      \begin{align*}
%        \varphi\left(a + \alpha b\right) &= \varphi\left(a\right) + \alpha \varphi\left(b\right)\\
%        \varphi\left(ab\right) &= \varphi\left(a\right)\varphi\left(b\right)
%      \end{align*}
%      for all $a,b\in A$ and $\alpha\in \C$.
%    \item If the algebra homomorphism $\varphi\colon A\rightarrow B$ preserves the involution structure --- i.e., $\varphi\left(a^{\ast}\right) = \varphi\left(a\right)^{\ast}$ --- then we say $\varphi$ is a $\ast$-homomorphism.
%    \item A ($\ast$)-isomorphism is a bijective ($\ast$)-homomorphism.
%    \item A $\ast$-automorphism is a $\ast$-isomorphism $\varphi\colon A\rightarrow A$. We write $\operatorname{Aut}\left(A\right)$ to denote the set of all $\ast$-automorphisms.
%  \end{itemize}
%\end{definition}
Just as there are free groups and free vector spaces, we can also talk about free algebras. In Chapter 8, we will construct special norms on free algebras to elucidate properties of the underlying group.\newline

Similar to a free group, the free algebra (or free $\ast$-algebra) is constructed by taking a certain collection of ``words'' over a set of symbols, and then, if desired, ``modding out'' by the ideal generated by a set of relations. We formalize this in steps.
\begin{definition}
  Let $E=\set{x_i}_{i\in I}$ be a collection of symbols that may not commute. The space of all polynomials over $E$ is the free vector space over the set of words formed by symbols in $E$,
  \begin{align*}
    \Gamma_E &= \set{x_{i_1}x_{i_2}\cdots x_{i_n} | n\in\N,i_1,\dots,i_n\in I}.
  \end{align*}
  We denote this space $\C \left\langle E \right\rangle$.\newline

  In the free vector space $\C \left\langle E \right\rangle$, we may define multiplication by concatenation:
  \begin{align*}
    \left(x_{i_1}x_{i_2}\cdots x_{i_n}\right)\left(x_{j_1}x_{j_2}\cdots x_{j_m}\right) &= x_{i_1}x_{i_2}\cdots x_{i_n}x_{j_1}x_{j_2}\cdots x_{j_m},
  \end{align*}
  where $i_1,\dots,i_n,j_1,\dots,j_m\in I$. The space $\C\left\langle E \right\rangle$, equipped with multiplication by concatenation, is known as the \textit{free algebra} on $E$.\newline

  To turn $\C\left\langle E \right\rangle$ into a $\ast$-algebra, we define the formal set $E^{\ast} = \set{x_{i}^{\ast}}_{i\in I}$, and define the involution on $\C\left\langle E\cup E^{\ast} \right\rangle$ by taking
  \begin{align*}
    \left(\alpha x_{i_1}^{\ve_1}x_{i_2}^{\ve_2}\cdots x_{i_n}^{\ve_n}\right)^{\ast} &= \overline{\alpha}x_{i_n}^{\delta_n}x_{i_{n-1}}^{\delta_{n-1}}\cdots x_{i_2}^{\delta_2}x_{i_1}^{\delta_1},
  \end{align*}
  where
  \begin{align*}
    \delta_j &= \begin{cases}
      \ast & \ve_j = 1\\
      1 & \ve_j = \ast
    \end{cases}.
  \end{align*}
  The set $\C\left\langle E\cup E^{\ast} \right\rangle$ with the involution defined above is known as the \textit{free $\ast$-algebra} on $E$, and is usually denoted $\mathbb{A}^{\ast}\left(E\right)$.\newline

  If $R\subseteq \mathbb{A}^{\ast}\left(E\right)$ is a collection of relations, we let $I(R) = \operatorname{ideal}\left(R\right)$. Then, the quotient algebra
  \begin{align*}
    \mathbb{A}^{\ast}\left(E|R\right) &= \mathbb{A}^{\ast}\left(E\right)/I(R)
  \end{align*}
  is known as the \textit{universal $\ast$-algebra on $E$ with relations $R$}.
\end{definition}
Evident from the name, the universal $\ast$-algebra(s) admit universal properties that characterize them as unique.
\begin{theorem}[Universal Properties]
  Let $E = \set{x_i}_{i\in I}$ be a set of abstract symbols, and let $B$ be a $\ast$-algebra. Let $\phi\colon E\rightarrow B$ be an injective map, and define $b_i = \phi\left(x_i\right)$.
  \begin{itemize}
    \item There is a unique $\ast$-homomorphism $\varphi\colon \mathbb{A}^{\ast}\left(E\right) \rightarrow B$ such that $x_i \mapsto b_i$. The following diagram commutes.
      \begin{center}
        % https://tikzcd.yichuanshen.de/#N4Igdg9gJgpgziAXAbVABwnAlgFyxMJZABgBpiBdUkANwEMAbAVxiRAFEQBfU9TXfIRQBGclVqMWbAELdeIDNjwEiZYePrNWiEAB1dAWzo4AFgCMzwAIJcAFOwCU3cTCgBzeEVAAzAE4QDJDIQHAgkUQktNn00Eyw5H39AxGDQpAAmagY6MxgGAAV+ZSEQXyw3ExwQak0pHX18HDoEkD8AjOo0xAjs3IKiwTYyiqqayW09XXpfWPiuCi4gA
        \begin{tikzcd}
        E \arrow[r, "\phi"] \arrow[d, "\iota"'] & B \\
        \mathbb{A}^{\ast}(E) \arrow[ru, "\varphi"']    &  
        \end{tikzcd}
      \end{center}
    \item If $R\subseteq \mathbb{A}^{\ast}\left(E\right)$ is a set of relations, and $\set{b_i}_{i\in I}$ satisfies the relations $R$, then there is a unique $\ast$-homomorphism $\mathbb{A}^{\ast}\left(E|R\right) \rightarrow B$ such that $x_i + I(R) \mapsto b_i$. The following diagram commutes.
      \begin{center}
        % https://tikzcd.yichuanshen.de/#N4Igdg9gJgpgziAXAbVABwnAlgFyxMJZABgBpiBdUkANwEMAbAVxiRAFEQBfU9TXfIRQBGclVqMWbAELdeIDNjwEiZYePrNWiEAB1dAWzo4AFgCMzwAIJcAesH104OLgAp2AHwBKASm7iYKABzeCJQADMAJwgDJDIQHAgkUQktNn00Eyw5COjYxHjEpAAmagY6MxgGAAV+ZSEQSKwgkxwQak0pHX18HDockCiYkuoixBTyypq6wTYmlraOyW09XXpIzOyuCi4gA
\begin{tikzcd}
E \arrow[r, "\phi"] \arrow[d, "\iota"']       & B \\
\mathbb{A}^{\ast}(E|R) \arrow[ru, "\varphi"'] &  
\end{tikzcd}
      \end{center}
  \end{itemize}
\end{theorem}
One of the most important $\ast$-algebras we will study is generated from a group by taking the free vector space over the group.
\begin{definition}
  Let $\Gamma$ be a group with identity element $e$, and let $\C\left[\Gamma\right]$ be the free vector space generated by $\Gamma$. We define a multiplication $f \ast g$, where $f,g\in \C\left[\Gamma\right]$ are finitely supported functions, by convolution:
  \begin{align*}
    f\ast g(s) &= \sum_{t\in\Gamma}f(t)g\left(t^{-1}s\right)\\
               &= \sum_{r\in\Gamma}f\left(sr^{-1}\right)g\left(r\right).
  \end{align*}
  The involution on $\C\left[\Gamma\right]$ is defined by $f^{\ast}\left(t\right) = \overline{f\left(t^{-1}\right)}$. The multiplicative identity is $\delta_e$, and multiplication satisfies $\delta_s\ast \delta_t = \delta_{st}$.
\end{definition}
\section{Tensor Products}%
Given two vector spaces $V,W$, and a bilinear map $b\colon V\times W \rightarrow Z$ (for some vector space $Z$), it's tempting to use the property of the free vector space to find a linear map on some structure that incorporates both $V$ and $W$ and stays faithful to the bilinear map $b$. Indeed, this is what the tensor product of the vector spaces $V$ and $W$ is --- a universal construction that ``turns'' bilinear maps into linear maps.\newline

In this section, we detail the construction of the tensor product $V\otimes W$, and apply it to the specific case when $V$ and $W$ are Banach spaces (see definition \ref{def:norms}) to obtain certain norms on the tensor product that ``play nicely'' with the norms on $V$ and $W$.
\begin{definition}\label{def:bilinear_map}
  Let $V,W,Z$ be vector spaces, and let $b\colon V\times W\rightarrow Z$ be a map such that, for all $\alpha\in \C$, $v,v_1,v_2\in V$, and $w,w_1,w_2\in W$,
  \begin{align*}
    b\left( \alpha v_1 + v_2,w \right) &= \alpha b\left( v_1,w \right) + b\left( v_2,w \right)\\
    b\left( v,\alpha w_1 + w_2 \right) &= b\left( v,w_1 \right) + \alpha b\left( v,w_2 \right).
  \end{align*}
  Then, we say $b$ is \textit{bilinear}.\newline

  If $V$ and $W$ are normed vector spaces, then we say $b$ is \textit{bounded} bilinear if
  \begin{align*}
    \norm{b}_{\op} &\coloneq \sup_{\substack{v\in B_{V}\\w\in B_{W}}}\norm{b\left( v,w \right)}\\
                   &< \infty.
  \end{align*}
\end{definition}
Just as we defined the free vector space and free group, we define the tensor product through a universal property --- and, just as with the case of the free group, we will focus more on the construction of the tensor product than on showing uniqueness.
\begin{theorem}[Universal Property of Tensor Products]\label{thm:tensor_product_existence}
  Let $V,W,Z$ be vector spaces, and let $b\colon V\times W \rightarrow Z$ be a bilinear map. Then, there exists a vector space, $V\otimes W$ and a linear map $T\colon V\otimes W \rightarrow Z$ such that for any $v\in V$ and $w\in W$, $T\left( v\otimes w \right) = b\left( v,w \right)$. The following diagram, where $\iota\colon V\times W \hookrightarrow V\otimes W$ is defined by $\left( v,w \right)\mapsto v\otimes w$, commutes.
  \begin{center}
      % https://tikzcd.yichuanshen.de/#N4Igdg9gJgpgziAXAbVABwnAlgFyxMJZABgBpiBdUkANwEMAbAVxiRADUAdTvAW3gAEAdRABfUuky58hFAEZyVWoxZsunCH0Ejxk7HgJEFcpfWatEIAFpilMKAHN4RUADMAThF5IyIHBCQAJmoGOgAjGAYABSkDWRB3LAcACxwQajNVSzCxCRAPLx9qfyQFZXM2bnwcOnS-OiwGNkgwVl18z29EMpLEYJBQiOjYmTYGGFc0jJULEAAVW1EgA
    \begin{tikzcd}
    V\times W \arrow[rd, "b"'] \arrow[r, "\iota"] & V\otimes W \arrow[d, "T"] \\
                                                  & Z                        
    \end{tikzcd}
  \end{center}
  The vector space $V\otimes W$ is unique up to linear isomorphism, and is known as the \textit{tensor product} of $V$ and $W$.
\end{theorem}
\begin{proof}
  We focus on showing existence. With $V$ and $W$ as in Theorem \ref{thm:tensor_product_existence}, we consider the free vector space (Theorem \ref{thm:free_vector_space}) on $V\times W$, $\C\left[ V\times W \right]$. Elementary elements of $V\times W$ are of the form $\delta_{(v,w)}$, where
  \begin{align*}
    \delta_{(v,w)} \left( s,t \right) &= \begin{cases}
      1 & v=s,w=t\\
      0 & \text{else}
    \end{cases}.
  \end{align*}
  Intuitively, from the way we have defined the tensor product as a linear map that extends a bilinear map, we would find the following properties of tensors desirable, for any $v,v_1,v_2\in V$, $w,w_1,w_2\in W$, and $\alpha\in \C$
  \begin{align*}
    \left( v_1 + v_2 \right)\otimes w &= v_1\otimes w + v_2\otimes w\label{eq:rel_1}\tag{1}\\
    v\otimes\left( w_1 + w_2 \right) &= v\otimes w_1 + v\otimes w_2\label{eq:rel_2}\tag{2}\\
    \left( \alpha v \right)\otimes w &= \alpha \left( v\otimes w \right)\label{eq:rel_3}\tag{3}\\
    v\otimes \left( \alpha w \right) &= \alpha \left( v\otimes w \right)\label{eq:rel_4}\tag{4}.
  \end{align*}
  With these four desirable properties in mind, we define a certain set of relations on the free vector space that we will ``mod out'' to obtain our desired tensor product. 
  \begin{itemize}
    \item To satisfy \eqref{eq:rel_1}, we define the set $N_1 = \set{\delta_{\left( v_1 + v_2,w \right)} - \delta_{\left( v_1,w \right)} - \delta_{\left( v_2,w \right)} | v_1,v_2\in V,w\in W}$, as this will be equivalent to the statement $\left( v_1 + v_2 \right)\otimes w - v_1\otimes w - v_2\otimes w = 0$.
    \item To satisfy \eqref{eq:rel_2}, we define the set $N_2 = \set{\delta_{\left( v,w_1 + w_2 \right)} - \delta_{\left( v,w_1 \right)} - \delta_{\left( v,w_2 \right)} | v\in V,w_1,w_2\in W}$, as this will be equivalent to the statement $v\otimes \left( w_1 + w_2 \right) - v\otimes w_1 - v\otimes w_2 = 0$.
    \item To satisfy \eqref{eq:rel_3}, we define the set $N_3 = \set{\delta_{\left( \alpha v,w \right)} - \alpha \delta_{\left( v,w \right)} | \alpha\in\C,v\in V,w\in W}$, as this will be equivalent to the statement $\left( \alpha v \right)\otimes w - \alpha \left( v\otimes w \right) = 0$.
    \item To satisfy \eqref{eq:rel_4}, we define the set $N_4 = \set{\delta_{\left(  v,\alpha w \right)} - \alpha \delta_{\left( v,w \right)} | \alpha\in\C,v\in V,w\in W}$, as this will be equivalent to the statement $v\otimes \left( \alpha w \right) - \alpha \left( v\otimes w \right) = 0$.
  \end{itemize}
  We define the ``zero set'' of our tensor product to be
  \begin{align*}
    N &= \Span\left( N_1 \cup N_2\cup N_3\cup N_4 \right),
  \end{align*}
  and consider the quotient space (Definition \ref{def:subspace_quotient_space_direct_sum}) $\C\left[ V\times W \right]/N$. We define
  \begin{align*}
    v\otimes w &\coloneq \delta_{(v,w)} + N.
  \end{align*}
  It can be verified that this definition is faithful to our requirements in \eqref{eq:rel_1}--\eqref{eq:rel_4}. Elements of $V\otimes W$ are of the form $\sum_{i\in I}v_i \otimes w_i$. We call elements of the form $v\otimes w$ \textit{elementary tensors}.\newline

  Define $\iota\colon V\times W \rightarrow V\otimes W$ by $\left( v,w \right) \mapsto v\otimes w$, and set $b = T\circ \iota$.\newline

  We verify that this definition satisfies the universal property of tensor products. We let $v_1,v_2,v\in V$, $w_1,w_2,w\in W$, and $\alpha\in \C$. Then,
  \begin{align*}
    b\left( v_1 + cv_2,w \right) &= T\left( \iota\left( v_1 + cv_2,w \right) \right)\\
                                 &= T\left( \left( v_1 + cv_2 \right)\otimes w \right)\\
                                 &= T\left( v_1\otimes w + c\left( v_2\otimes w \right) \right)\\
                                 &= T\left( v_1\otimes w \right) + cT\left( v_2\otimes w \right)\\
                                 &= b\left( v_1,w \right) + cb\left( v_2,w \right)\\
                                 \\
    b\left( v,w_1 + cw_2 \right) &= T\left( \iota\left( v,w_1 + cw_2 \right) \right)\\
                                 &= T\left( v\otimes \left( w_1 + cw_2 \right) \right)\\
                                 &= T\left( v\otimes w_1 + c\left( v\otimes w_2 \right) \right)\\
                                 &= T\left( v\otimes w_1 \right) + cT\left( v\otimes w_2 \right)\\
                                 &= b\left( v,w_1 \right) + cb\left( v,w_2 \right).
  \end{align*}
  Thus, by the universal property of the free vector space, there is a unique linear map $\tilde{b}\colon \C\left[ V\times W \right]\rightarrow Z$, defined by $\tilde{b}\left( \delta_{(v,w)} \right) = b\left( v,w \right)$. Note that $\tilde{b}$ vanishes on $N$, so by the first isomorphism theorem there is a unique linear map $T_{b}\colon \C\left[ V\times W \right]/N\rightarrow Z$ that is defined by $T_{b} \circ \pi = \tilde{b}$, where $\pi\colon \C\left[ V\times W \right] \rightarrow \C\left[ V\times W \right]/N$ is the canonical projection.\newline

  Thus, we know that $T = T_{b}$ satisfies the universal property of tensor products.
\end{proof}
In linear algebra, we often use the universal property of tensor products to convert from bilinear maps to linear maps. This universal property is actually used to show that the tensor product is compatible with the multiplication structure on algebras. We state the structure on the tensor product of algebras here, but do not provide its full proof.
\begin{proposition}\label{prop:tensor_product_algebras}
  Let $A$ and $B$ be algebras. The vector space $A\otimes B$ admits a multiplication $\left( A\otimes B \right)\times \left( A\otimes B \right)\rightarrow A\otimes B$, given by
  \begin{align*}
    \left( a\otimes b \right)\left( c\otimes d \right) &= ac\otimes bd.
  \end{align*}
  If $A$ and $B$ are $\ast$-algebras, then $A\otimes B$ admits an involution given by
  \begin{align*}
    \left( a\otimes b \right)^{\ast} &= a^{\ast}\otimes b^{\ast}.
  \end{align*}
\end{proposition}
The proof of Proposition \ref{prop:tensor_product_algebras} is carried out by ``amplifying'' to the vector spaces $\mathcal{L}\left( A \right)$ and $\mathcal{L}\left( B \right)$, and establishing the linear map $L\colon A\otimes B \rightarrow \mathcal{L}\left( A \right)\otimes \mathcal{L}\left( B \right)$.\newline

When our vector spaces are equipped with a norm --- specifically, if they are Banach spaces --- not only does it matter that the tensor product preserves the vector space structure, but also that it preserves the norm structure in a particular manner. This is the domain of the injective and projective norms. The injective and projective norms will become more relevant when we discuss $C^{\ast}$-algebras, nuclearity, and amenability in Chapter \ref{ch:nuclearity}. First, we provide some background before elaborating on their definitions.

% Here, will run through free groups, free vector spaces, tensor products, free algebras, and the group *-algebra. We will discuss the free group in chapter 2, and the rest in chapter 8 when elaborating on nuclearity.
\chapter{How to Feed 5,000 Hungry Mathematicians: Paradoxical Decompositions}\label{ch:paradoxical_decompositions}
The primary goal of this section will be to introduce the idea of a paradoxical decomposition (and its effects on the analytic properties of $\R^3$) through the Banach--Tarski Paradox. The ultimate goal is to prove the following statement.
\begin{proposition}[General Banach--Tarski Paradox]
  If $A$ and $B$ are bounded subsets of $\R^3$ with nonempty interior, there is a partition of $A$ into finitely many disjoint subsets such a sequence of isometries applied to these subsets yields $B$.
\end{proposition}
The existence of the Banach--Tarski paradox throws a wrench into a major idea that we may have about subsets of $\R^3$ --- namely, that they always have some ``volume'' to them that is invariant under isometry, similar to how ``area'' in $\R^2$ is invariant under isometry.
\section{Prelude: Essential Group Actions}
We begin by discussing some of the basic properties of group actions.
\begin{definition}[Group Action]
  Let $G$ be a group, and $A$ be a set. A left group action of $G$ onto $A$ is a map $\alpha: G\times A\rightarrow A$ that satisfies
  \begin{itemize}
    \item $\alpha\left(g_1,\left(g_2,a\right)\right) = \alpha\left(g_1g_2,a\right)$ for all $g_1,g_2\in G$ and $a\in A$;
    \item $\alpha\left(e_G,a\right) = a$ for all $a\in A$.
  \end{itemize}
  For the sake of brevity, we write $\left(g,a\right) = g\cdot a$.
\end{definition}
Every group action can be represented by a permutation on $A$.
\begin{definition}[Permutation Representation]
  For each $g$, the map $\sigma_g: A\rightarrow A$ defined by $\sigma_g\left(a\right) = g\cdot a$ is a permutation of $A$. There is a homomorphism associated to these actions, $\varphi: G\rightarrow \operatorname{Sym}(A)$, where $\operatorname{Sym}(A)$ is the symmetric group on the elements of $A$.
\end{definition}
The permutation representation can run in the opposite direction in the following sense: given a nonempty set $A$ and a homomorphism $\psi: G\rightarrow \sym(A)$, we can take $g\cdot a = \psi(g)(a)$, where $\psi(g) = \sigma_g\in \sym(A)$ is a permutation.\newline

Just as we can pass group actions into permutation representations, and discuss ideas like the kernel of homomorphisms, we can also discuss the kernel of ajn action.
\begin{definition}[Kernel]
  The kernel of the action of $G$ on $A$ is the set of elements in $g$ that act trivially on $A$:
  \begin{align*}
    \set{g\in G\mid \forall a\in A,~g\cdot a = a}.
  \end{align*}
  The kernel of the group action is the kernel of the permutation representation $\varphi: G\rightarrow \sym(A)$.
\end{definition}
\begin{definition}[Stabilizer]
  For each $a\in A$, we define the stabilizer of $a$ under $G$ to be the set of elements in $G$ that fix $a$:
  \begin{align*}
    G_a &= \set{g\in G\mid g\cdot a = a}.
  \end{align*}
\end{definition}
\begin{remark}
The kernel of the group action is the intersection of the stabilizers of every element of $A$.\newline

For each $a\in A$, $G_{a}$ is a subgroup of $G$.
\end{remark}
\begin{definition}[Faithful Action]
An action is faithful if the kernel of the action is the identity, $e_G$. Equivalently, the permutation representation $\varphi: G\rightarrow \sym(A)$ is injective.
\end{definition}
The following definition will be useful in the future as we dig deeper into the idea of paradoxical groups.
\begin{definition}[Free Action]
For a set $X$ with $G$ acting on $X$, the action of $G$ on $X$ is free if, for every $x\in X$, $g\cdot x = x$ if and only if $g = e_G$.
\end{definition}
The most important theorem relating to group actions is the orbit-stabilizer theorem. As we prove the following theorem, we will reveal the definition of an orbit as a type of equivalence class.
\begin{theorem}[Orbit-Stabilizer Theorem]
  Let $G$ be a group that acts on a nonempty set $A$. We define a relation $a\sim b$ if and only if $a = g\cdot b$ for some $g\in G$. This is an equivalence relation, with the number of elements in $\left[a\right]_{\sim}$ found by taking the index of the stabilizer of $a$ in $G$, $\left\vert G:G_a \right\vert$.
\end{theorem}
\begin{proof}
  We start by seeing that $a\sim a$, as $e_G\cdot a = a$. Similarly, if $a\sim b$, then there exists $g\in G$ such that $a = g\cdot b$. Thus,
  \begin{align*}
    g^{-1}\cdot a &= g^{-1}\cdot \left(g\cdot b\right)\\
                  &= g^{-1}g\cdot b\\
                  &= e\cdot b\\
                  &= b,
  \end{align*}
  meaning that $b\sim a$. Finally, if we have $a\sim b$ and $b\sim c$, we have $a = g\cdot b$ and $b = h\cdot c$ for some $g,h\in G$. Therefore,
  \begin{align*}
    a &= g\cdot \left(h\cdot c\right)\\
      &= \left(gh\right)\cdot c,
  \end{align*}
  meaning $a\sim c$. Thus, the relation $\sim$ is reflexive, symmetric, and transitive, so it is an equivalence relation.\newline

  We claim there is a bijection between the left cosets of $G_a$ and the elements of $\left[a\right]_{\sim}$.\newline

  Define $C_a = \set{g\cdot a\mid g\in G}$, which is the set of elements in the equivalence class of $a$. Define the map $g\cdot a \mapsto gG_a$. Since $g\cdot a$ is always an element of $C_a$, this map is surjective. Additionally, since $g\cdot a = h\cdot a$ if and only if $\left(h^{-1}g\right)\cdot a = a$, we have $h^{-1}g \in G_a$, which is only true if $gG_a = hG_a$. Thus, the map is injective.\newline

  Since there is a one to one map between the equivalence classes of $a$ under the action of $G$, and the number of left cosets of $G_a$, we know that the number of equivalence classes of $a$ under the action of $G$ is $\left\vert G:G_a \right\vert$.
\end{proof}
\begin{definition}[Orbit]
Let $G$ act on $A$, and let $a\in A$. The orbit of $a$ under $G$ is the set
\begin{align*}
  G\cdot a &= \set{g\cdot a\in A\mid g\in G}
\end{align*}
\end{definition}

%\part{Fugue}
\chapter{Well-Behaved Groups of a Feather Flock Together: Tarski's Theorem}\label{ch:tarskis_theorem}
Ultimately, the reason the Banach--Tarski paradox ``works'' is because the paradoxical group $F(a,b)$, lacks a property known as amenability. Readers may be surprised to hear that amenability and non-paradoxicality are equivalent --- that is, a group is amenable if and only if it is non-paradoxical. This fact is formalized in Tarski's theorem.
\begin{theorem}[Tarski's Theorem]\label{thm:tarski}
  Let $G$ be a group that acts on a set $X$, and let $E\subseteq X$ be nonempty. There is a finitely additive translation-invariant measure $\mu: P(X)\rightarrow [0,\infty]$ with $\mu(E)\in (0,\infty)$ if and only if $E$ is not $G$-paradoxical.
\end{theorem}
In fact, we can prove one of the directions of Tarski's theorem now.
\begin{proof}[of the Forward Direction of Tarski's Theorem]
  Let $E$ be $G$-paradoxical. Suppose toward contradiction that such a translation-invariant finitely additive $\nu$ existed with $\nu(E) \in (0,\infty)$.\newline

  Let $A_1,\dots,A_n,B_1,\dots,B_m\subseteq E$ be pairwise disjoint, and let $t_1,\dots,t_n,s_1,\dots,f_m\in G$ such that
  \begin{align*}
    E &= \bigsqcup_{i=1}^{n}t_i\cdot A_i\\
      &= \bigsqcup_{j=1}^{m}s_j\cdot B_j.
  \end{align*}
  Then, it would be the case that
  \begin{align*}
    \nu(E) &= \nu\left(\bigsqcup_{i=1}^{n}t_i\cdot A_i\right)\\
           &= \sum_{i=1}^{n}\nu\left(t_i\cdot A_i\right)\\
           &= \sum_{i=1}^{n}\nu\left(A_i\right),
  \end{align*}
  and
  \begin{align*}
    \nu(E) &= \sum_{j=1}^{m}\nu\left(B_j\right).
  \end{align*}
  However, this also yields
  \begin{align*}
    \nu\left(E\right) &= \nu\left(\left(\bigsqcup_{i=1}^{n}A_i\right)\sqcup \left(\bigsqcup_{j=1}^{m}B_j\right)\right)\\
                      &= \sum_{i=1}^{n}\nu\left(A_i\right) + \sum_{j=1}^{m}\nu\left(B_j\right)\\
                      &= \sum_{i=1}^{n}\nu\left(t_i\cdot A_i\right) + \sum_{j=1}^{m}\nu\left(x_j\cdot B_j\right)\\
                      &= \nu\left(E\right) + \nu\left(E\right)\\
                      &= 2\nu\left(E\right).
  \end{align*}
  implying that $\nu(E) = 0$ or $\nu(E) = \infty$.
\end{proof}

\section{A Little Bit of Graph Theory}%
To prove the reverse direction of Tarski's theorem, we need to develop some machinery from graph theory that will allow us to prove that a certain semigroup we will construct in the next section satisfies the cancellation identity.\newline

We start by defining graphs and paths, before proving a special case of Hall's theorem, ultimately extending to the infinite case with König's theorem.
\begin{definition}[Graphs and Paths]
  A \textbf{graph} is a triple $\left(V,E,\phi\right)$, with $V,E$ nonempty sets and $\phi: E\rightarrow P_{2}(V)$ a map from $E$ to the set of all unordered subset pairs of $V$.\newline

  For $e\in E$, if $\phi(e) = \set{v,w}$, then we say $v$ and $w$ are the \textbf{endpoints} of $e$, and $e$ is \textbf{incident} on $v$ and $w$.\newline

  A \textbf{path} in $\left(V,E,\phi\right)$ is a finite sequence $\left(e_1,\dots,e_n\right)$ of edges, with a finite sequence of vertices $\left(v_0,\dots,v_n\right)$, such that $\phi\left(e_k\right) = \set{v_{k-1},v_k}$.\newline

  The \textbf{degree} of a vertex, $\deg(v)$, is the number of edges incident on $v$.\newline

  We define the \textbf{neighbors} of $S\subseteq V$ to be the set of all vertices $v\in V\setminus S$ such that $v$ is an endpoint to an edge incident on $S$. We denote this set $N(S)$.
\end{definition}

\begin{definition}[Bipartite Graphs and $k$-Regularity]
  Let $\left(V,E,\phi\right)$ be a graph, with $k\in \N$.
  \begin{enumerate}[(i)]
    \item If $\deg(v) = k$ for each $v\in V$, we say $\left(V,E,\phi\right)$ is \textbf{$k$-regular}.
    \item If $V = X\sqcup Y$, with each edge in $E$ having one endpoint in $X$ and one endpoint in $Y$, then we say $V$ is \textbf{bipartite}, and write $\left(X,Y,E,\phi\right)$.
  \end{enumerate}
\end{definition}

\begin{definition}[Perfect Matching]
  Let $\left(X,Y,E,\phi\right)$ be a bipartite graph. Let $A\subseteq X$ and $B\subseteq Y$. A \textbf{perfect matching} of $A$ and $B$ is a subset $F\subseteq E$ with
  \begin{enumerate}[(i)]
    \item each element of $A\cup B$ is an endpoint of exactly one $f\in F$;
    \item all endpoints of edges in $F$ are in $A\cup B$.
  \end{enumerate}
\end{definition}
\begin{definition}[Hall Condition]
  We say a bipartite graph $\left(X,Y,E,\phi\right)$ satisfies the \textbf{Hall Condition} on $X$ if, for all $S\subseteq X$, $\left\vert N(S) \right\vert \geq \left\vert S \right\vert$.\newline

  Equivalently, we say a (finite) collection of not necessarily distinct finite sets $\mathcal{X} = \set{X_i}_{i=1}^{n}$ satisfies the marriage condition if and only if for all subcollections $\mathcal{Y}_k = \set{X_{i_k}}_{k=1}^{m}$,
  \begin{align*}
    \left\vert \mathcal{Y}_k \right\vert \leq \left\vert \bigcup_{k=1}^{m}X_{i_k} \right\vert.
  \end{align*}
\end{definition}
\begin{remark}
These two formulations of the Hall condition are equivalent regarding an $X$-perfect matching.
\end{remark}
\begin{theorem}[Hall's Theorem for Finite $k$-Regular Bipartite Graphs]\label{thm:hall_finite}
  Let $\left(X,Y,E,\phi\right)$ be a $k$-regular bipartite graph for some $k\in \N$, and let $V = X\sqcup E$ be finite. Then, there is a perfect matching of $X$ and $Y$.
\end{theorem}
\begin{proof}
  Note that since $\left\vert E \right\vert = k\left\vert K \right\vert = k\left\vert Y \right\vert$, it is the case that $\left\vert X \right\vert = \left\vert Y \right\vert$.\newline

  Let $M\subseteq V$ be any subset. We will show that $\left\vert N(M) \right\vert\geq \left\vert M \right\vert$ --- that is, $\left(X,Y,E,\phi\right)$ satisfies the Hall condition.\newline

  Let $M_X = M\cap X$ and $M_Y = M\cap Y$, where $M = M_X\sqcup M_Y$. Let $\left[M_X,N\left(M_X\right)\right]$ be the set of edges with endpoints in $M_X$ and $N\left(M_X\right)$, and $\left[M_Y,N\left(M_Y\right)\right]$ be the set of edges with endpoints in $M_Y$ and $N\left(M_Y\right)$. We also let $\left[X,N\left(M_X\right)\right]$ denote the set of edges with endpoints in $X$ and $N\left(M_X\right)$, and similarly, $\left[Y,N\left(M_Y\right)\right]$ is the set of edges with endpoints in $Y$ and $N\left(M_Y\right)$.\newline

  We can see that $\left[M_X,N\left(M_X\right)\right]\subseteq \left[X,N\left(M_X\right)\right]$, and similarly, $\left[M_Y,N\left(M_Y\right)\right]\subseteq \left[Y,N\left(M_Y\right)\right]$.\newline

  Since $\left\vert \left[M_X,N\left(M_X\right)\right] \right\vert = k\left\vert M_X \right\vert$ and $\left\vert \left[X,N\left(M_X\right)\right] \right\vert = k\left\vert N\left(M_X\right) \right\vert$, we have
  \begin{align*}
    \left\vert M_X \right\vert\leq \left\vert N\left(M_X\right) \right\vert,
  \end{align*}
  and similarly,
  \begin{align*}
    \left\vert M_Y \right\vert\leq \left\vert N\left(M_Y\right) \right\vert.
  \end{align*}
  Thus, $\left\vert M \right\vert\leq \left\vert N\left(M\right) \right\vert$.\newline

  We will now show that there is an $X$-perfect matching. Suppose toward contradiction that $F$ is a maximal perfect matching on $A\subseteq X$ and $B\subseteq Y$ with $X\setminus A \neq \emptyset$.\newline

  Then, there is $x\in X\setminus A$. Consider $Z\subseteq V$ consisting of all vertices $z$ such that there exists a $F$-alternating path $\left(e_1,\dots,e_n\right)$ between $z\in Z$ and $x$.\newline

  It cannot be the case that $Z\cap Y$ is empty, since the number of neighbors of $x$ is greater than or equal to $1$ by the Hall condition --- if it were the case that $Z\cap Y$ were empty, we could add an edge to $F$ consisting of $x$ and one element of $N\left(\set{x}\right)$, which would contradict the maximality of $F$.\newline

  Consider a path traversing along $Z$, $\left(e_1,\dots,e_n\right)$. It must be the case that $e_n\in F$, or else we would be able to ``flip'' the matching $F$ by exchanging $e_{i}$ with $e_{i+1}$ for $e_i\in F$, which would contradict the maximality of $F$ yet again. Thus, every element of $Z\cap Y$ is satisfied by $F$, so $Z\cap Y\subseteq B$.\newline

  Since each element in $Z\cap Y$ is paired with exactly one element of $Z\cap X$ (with one left over), it is the case that $\left\vert Z\cap X \right\vert = \left\vert Z\cap Y \right\vert + 1$.\newline

  Suppose toward contradiction that there exists $y\in N\left(Z\cap X\right)$ with $y\notin Z\cap Y$. Then, there exists $v\in Z\cap X$ and $e\in E$ such that $\phi(e) = \set{v,y}$. However, this means $v$ is connected via a path to $x$, meaning $y\in Z$, so $y\in Z\cap Y$. Thus, we must have $N\left(Z\cap X\right) = Z\cap Y$.\newline

  Therefore,
  \begin{align*}
    \left\vert Z\cap X \right\vert &= \left\vert Z\cap Y \right\vert + 1\\
                                   &= \left\vert N\left(Z\cap X\right) \right\vert + 1,
  \end{align*}
  which contradicts the fact that $\left(X,Y,E,\phi\right)$ satisfies the Hall condition. Therefore, $A = X$.\newline

  By symmetry, there is a perfect matching of $X$ and $Y$ in $\left(X,Y,E,\phi\right)$.
\end{proof}
\begin{remark}
  An equivalent formulation to Hall's theorem states that there is a \textit{system of distinct representatives} on $\mathcal{X}$, which is a set $\set{x_{k}}_{k=1}^{n}$ such that $x_{k}\in X_{k}$ and $x_{i}\neq x_j$ for $i\neq j$.\newline

  This implies the existence of an injection $f: \mathcal{X}\hookrightarrow \bigcup_{k=1}^{n}X_{k}$, such that $f\left(X_k\right) \in X_k$.
\end{remark}
%\begin{definition}[Choice Function]
%  Let $\mathcal{X} = \set{X_{i}}_{i\in I}$ be a collection of sets. A function $f: \mathcal{X}\rightarrow \bigcup_{i\in I}X_i$ is called a choice function if, for each $i\in I$, $f\left(X_{i}\right)\in X{i}$.\newline
%
%  We also say $f: \mathcal{X}\rightarrow \bigcup_{i\in I}X_i$ is a choice function if $f\in \prod_{i\in I}X_i$.
%\end{definition}
%
%\begin{theorem}[Tychonoff's Theorem]
%  If $\set{X_{i}}_{i\in I}$ is a family of compact topological spaces
%\end{theorem}
\begin{theorem}[Infinite Hall's Theorem]
  Let $\mathcal{G} = \set{X_{i}}_{i\in I}$ be a collection of (not necessarily distinct) finite sets. If, for every finite subcollection $\mathcal{Y} = \set{X_{i_k}}_{k=1}^{n}$,
  \begin{align*}
    n\leq \left\vert \bigcup_{k=1}^{n}X_{i_k} \right\vert,
  \end{align*}
  then there is a choice function on $G$.
\end{theorem}
\begin{proof}
  We endow each $X_i\in \set{X_{i}}_{i\in I}$ with the discrete topology. Since each $X_i$ is finite, each $X_i$ is compact.\newline

  Thus, by Tychonoff's theorem, it is the case that $\prod_{i\in I}X_{i}$ is compact.\newline

  For every finite subset $Y\subseteq \mathcal{G}$, we define
  \begin{align*}
    S_Y &= \set{\left.f\in \prod_{i\in I}X_i\right|f\vert_{Y}\text{ is injective}}.
  \end{align*}
  The injectivity of $f\vert_{Y}$ is equivalent to the existence of a system of distinct representatives on $Y$. Since $Y$ satisfies the Hall condition, each $S_{Y}$ is nonempty. Additionally, for any net of functions $f_{\alpha}\in S_{Y}$ with $\lim_{\alpha}f_{\alpha} = f$, it is the case that $f_{\alpha}\vert_{Y}$ is injective, so $f\vert_{Y}$ is injective, meaning $S_{Y}$ is closed.\newline

  We define $F = \set{S_{Y}\mid Y\subseteq \mathcal{G}\text{ finite}}$. For finite $Y_{1},Y_{2}\subseteq \mathcal{G}$, every system of distinct representatives in $Y_1\cup Y_2$ is necessarily a system of distinct representatives on $Y_1$ and a system of distinct representatives on $Y_{2}$, meaning $S_{Y_1\cup Y_2}\subseteq S_{Y_1}\cap S_{Y_2}$. Thus, $F$ has the finite intersection property.\newline

  Since $\prod_{i\in I}X_i$ is compact, $\bigcap F$ is nonempty, where the intersection is taken over all finite subsets of $\mathcal{G}$. For any $f\in \bigcap F$, $f$ is necessarily a choice function.
\end{proof}
\begin{remark}
  This is equivalent to the existence of an injection $f: \mathcal{G}\hookrightarrow \bigcup_{i\in I}X_i$.
\end{remark}

We will use this infinite case of Hall's theorem to prove König's theorem. 
\begin{theorem}[König's Theorem]\label{thm:konig}
  Let $\left(X,Y,E,\phi\right)$ be a $k$-regular bipartite graph (not necessarily finite). Then, there is a perfect matching of $X$ and $Y$.
\end{theorem}
\begin{proof}
  If $k = 1$, it is clear that there is a perfect matching in $\left(X,Y,E,\phi\right)$ consisting of the edges in $\left(X,Y,E,\phi\right)$.\newline

  Let $k\geq 2$. Since any finite subset of $X$ satisfies the Hall condition, as displayed in the proof of Theorem \ref{thm:hall_finite}, there is some $X$-perfect matching in $\left(X,Y,E,\phi\right)$. We call this $X$-perfect matching $F$. There is an injection $f: X\hookrightarrow Y$ following the edges in $F$.\newline

  Similarly, since any finite subset of $Y$ satisfies the Hall condition, there is some $Y$-perfect matching in $\left(X,Y,E,\phi\right)$. We call this $Y$-perfect matching $G$. There is an injection $g: Y\hookrightarrow X$ following the edges of $G$.\newline

  Consider the subgraph $\left(X,Y,F\cup G,\phi|_{F\cup G}\right)$. The injections $f$ and $g$ still hold in this graph. By the Cantor--Schröder--Bernstein theorem, there is a bijection $h: X\rightarrow Y$ in $\left(X,Y,F\cup G,\phi|_{F\cup G}\right)$.
\end{proof}
\section{Type Semigroups}%
\begin{definition}
  Let $G$ be a group that acts on a set $X$.
  \begin{enumerate}[(i)]
    \item We define $X^{\ast} = X\times \N_0$, and
      \begin{align*}
        G^{\ast} &= \set{\left(g,\pi\right)\mid g\in G,\pi\in\sym\left(\N_0\right)}.
      \end{align*}
    \item If $A\subseteq X^{\ast}$, the values of $n$ for which there is an element of $A$ whose second coordinate is $n$ are called the \textbf{levels} of $A$.
  \end{enumerate}
\end{definition}
\begin{fact}\label{fact:type_semigroup_equidecomposability}
  If $E_1,E_2\subseteq X$, then $E_{1}\sim_{G}E_2$ if and only if $E_1\times \set{n}\sim_{G^{\ast}}E_{2}\times \set{m}$ for all $m,n\in \N_{0}$
\end{fact}
\begin{proof}[of Fact \ref{fact:type_semigroup_equidecomposability}]
  Let $E_{1}\sim_{G}E_2$. Then, there exist pairwise disjoint $A_1,\dots,A_n\subset E_1$, pairwise disjoint $B_1,\dots,B_n\subset E_2$, and elements $g_1,\dots,g_n\in G$ such that $g_i\cdot A_i = B_i$. We select the permutation $\pi_{i}\in \sym\left(\N_0\right)$ such that $\pi_{i}(n) = m$ and $\pi_i(m) = n$ for each $i$. Then,
  \begin{align*}
    \left(g_i,\pi_i\right)\cdot \left(A_{i}\cdot \set{n}\right) &= B_{i}\cdot \set{m}.
  \end{align*}

  Similarly, if $E_{1}\times \set{n} \sim_{G^{\ast}}E_2\times \set{m}$, then of the pairwise disjoint subsets
  \begin{align*}
    A_1\times \set{n},\dots,A_n\times \set{n}\subset E_1\times \set{n}
  \end{align*}
  and
  \begin{align*}
    B_1\times\set{m},\dots,B_n\times\set{m}\subset E_2\times \set{m},
  \end{align*}
  we set $A_1,\dots,A_n\subset E_1$ and $B_1,\dots,B_n\subset E_2$. Similarly, for
  \begin{align*}
    \left(g_1,\pi_1\right),\dots,\left(g_n,\pi_n\right)\in G^{\ast}
    \intertext{such that}
    \left(g_i,\pi_i\right)\cdot A_i\times \set{n} = B_i\times\set{m},
  \end{align*}
  we select $g_1,\dots,g_n\in G$. Then, by definition,
  \begin{align*}
    g_i\cdot A_i = B_i
  \end{align*}
  for each $i$. Thus, $E_1\sim_{G}E_2$.
\end{proof}

\begin{definition}\label{def:type_semigroup}
  Let $G$ be a group that acts on $X$, and let $G^{\ast}$, $X^{\ast}$ be defined as above.
  \begin{enumerate}[(i)]
    \item A set $A\subseteq X^{\ast}$ is said to be \textbf{bounded} if it has finitely many levels.
    \item If $A\subseteq X^{\ast}$ is bounded, the equivalence class of $A$ with respect to $G^{\ast}$-equidecomposability is called the \textbf{type} of $A$, which is denoted $\left[A\right]$.
    \item If $E\subseteq X$, we write $\left[E\right] = \left[E\times \set{0}\right]$.
    \item Let $A,B\subseteq X^{\ast}$ be bounded with $k\in \N_{0}$ such that for
      \begin{align*}
        B' := \set{\left(b,n+k\right)\mid \left(b,n\right)\in B},
      \end{align*}
      we have $B'\cap A = \emptyset$. Then, $\left[A\right] + \left[B\right] = \left[A\sqcup B'\right]$. Note that $\left[B'\right] = \left[B\right]$.
    \item We define
      \begin{align*}
        \mathcal{S} &= \set{\left[A\right]\mid A\subseteq X^{\ast}\text{ bounded}}
      \end{align*}
      under the addition defined in (iv) to be the \textbf{type semigroup} of the action of $G$ on $X$.
  \end{enumerate}
\end{definition}

\begin{fact}\label{fact:type_semigroup_well_defined}
  Addition is well-defined in $\left(\mathcal{S},+\right)$, and $\left(\mathcal{S},+\right)$ is a well-defined commutative semigroup with identity $\left[\emptyset\right]$.
\end{fact}
\begin{proof}[of Fact \ref{fact:type_semigroup_well_defined}]
  To show that addition is well-defined, we let $\left[A_1\right] = \left[A_2\right]$, and $\left[B_1\right] = \left[B_2\right]$. Without loss of generality, $A_1\cap B_1 = \emptyset$ and $A_2\cap B_2 = \emptyset$.\newline

  By the definition of the type, $A_1\sim_{G^{\ast}}A_2$ and $B_1\sim_{G^{\ast}}B_2$, meaning
  \begin{align*}
    A_1\sqcup B_1\sim_{G^{\ast}} A_2\sqcup B_2,
  \end{align*}
  so
  \begin{align*}
    \left[A_1\right] + \left[B_1\right] &= \left[A_1\sqcup B_1\right]\\
                                        &= \left[A_2\sqcup B_2\right]\\
                                        &= \left[A_2\right] + \left[A_2\right],
  \end{align*}
  meaning addition is well-defined.\newline

  Since addition is well-defined, and $A\sqcup B = B\sqcup A$, we can see that addition is also commutative. We also have
  \begin{align*}
    \left[A\right] + \left[\emptyset\right] &= \left[A\sqcup \emptyset\right]\\
                                            &= \left[A\right],
  \end{align*}
  so $\left[\emptyset\right]$ is the identity on $\mathcal{S}$.\newline

  Finally, since for any $\left[A\right],\left[B\right]\in \mathcal{S}$, $A$ and $B$ have finitely many levels, it is the case that $A\cup B$ has finitely many levels for any $A$ and $B$, so $\left[A\right] + \left[B\right] \in \mathcal{S}$. 
\end{proof}

\begin{definition}
  For any commutative semigroup $\mathcal{S}$ with $\alpha \in S$ and $n\in \N$, we define
  \begin{align*}
    n\alpha := \underbrace{\alpha + \cdots + \alpha}_{\text{$n$ times}}
  \end{align*}
\end{definition}
\begin{definition}
  For $\alpha,\beta \in \mathcal{S}$, if there exists $\gamma \in \mathcal{S}$ such that $\alpha + \gamma = \beta$, we write $\alpha \leq \beta$.
\end{definition}
\begin{fact}\label{fact:type_semigroup_paradoxicality}
  If $G$ is a group acting on $X$ with corresponding type semigroup $\mathcal{S}$, then the following are true.
  \begin{enumerate}[(i)]
    \item If $\alpha,\beta\in \mathcal{S}$ with $\alpha \leq \beta$ and $\beta \leq \alpha$, then $\alpha = \beta$.
    \item A set $E\subseteq X$ is $G$-paradoxical if and only if $\left[E\right] = 2\left[E\right]$.
  \end{enumerate}
\end{fact}
\begin{proof}[of Fact \ref{fact:type_semigroup_paradoxicality}]
  Let $G$ act on $X$, and let $\mathcal{S}$ be the corresponding type semigroup.
  \begin{enumerate}[(i)]
    \item If $\left[A\right]\leq \left[B\right]$, then there exists $C_1\in \mathcal{S}$ such that $\left[A\right] + \left[C_1\right] = \left[B\right]$. Without loss of generality, $C_1\cap A= \emptyset$, meaning $\left[B\right] = \left[A\sqcup C_1\right]$. Thus, $A\sqcup C_1 \sim_{G^{\ast}} B$, meaning $B\preceq_{G^{\ast}}A$.\newline

      Similarly, if $\left[B\right]\leq \left[A\right]$, then $B\preceq_{G^{\ast}}A$. By Theorem \ref{thm:csb_for_equidecomposability}, it is thus the case that $A\sim_{G^{\ast}}B$.
    \item Let $E$ be $G$-paradoxical. Then, $E\sim_{G}\bigsqcup_{i=1}^{n}A_i$ and $E\sim_{G}\bigsqcup_{j=1}^{m}B_j$ for some disjoint subsets $A_1,\dots,A_n,B_1,\dots,B_m\subset E$. Thus, we have
      \begin{align*}
        \left[E\right] &= \left[\left(\bigsqcup_{i=1}^{n}A_i\right)\sqcup \left(\bigsqcup_{j=1}^{m}B_j\right)\right]\\
                       &= \left[\bigsqcup_{i=1}^{n}A_i\right] + \left[\bigsqcup_{j=1}^{m}B_j\right]\\
                       &= 2\left[E\right].
      \end{align*}
      Similarly, if $\left[E\right] = 2\left[E\right]$, then there exist $A$ and $B$ such that
      \begin{align*}
        \left[E\right] &= \left[A\right] + \left[B\right]\\
                       &= \left[A\sqcup B\right],
      \end{align*}
      meaning $A$ and $B$ are each $G$-equidecomposable with $E$, so $E$ is $G$-paradoxical.
  \end{enumerate}
\end{proof}
We can now prove the cancellation identity, which we will be useful as we construct our desired finitely additive measure.
\begin{theorem}[Cancellation Identity on $\mathcal{S}$]
  Let $\mathcal{S}$ be the type semigroup for some group action, and let $\alpha,\beta\in \mathcal{S}$, $n\in \N$ such that $n\alpha = n\beta$. Then, $\alpha = \beta$.
\end{theorem}
\begin{proof}
  Let $n\alpha = n\beta$. Then, there are two disjoint bounded subsets $E,E'\subseteq X^{\ast}$ with $E\sim_{G^{\ast}}E'$, and pairwise disjoint subsets $A_1,\dots,A_n\subseteq E$, $B_1,\dots,B_n\subseteq E'$ such that
  \begin{itemize}
    \item $E = A_1\cup\cdots\cup A_n$, $E' = B_1\cup\cdots\cup B_n$
    \item $\left\vert A_j \right\vert = \alpha$ and $\left[B_j\right] = \beta$ for each $j=1,\dots,n$.
  \end{itemize}
  Let $\chi: E\rightarrow E'$ be a bijection as in Fact \ref{fact:bijections}, with $\phi_j: A_1\rightarrow A_j$, $\psi_j: B_1\rightarrow B_j$ also being bijections as in Fact \ref{fact:bijections}; here we define $\phi_1$ and $\psi_1$ to be the identity map.\newline

  For each $a\in A_1$ and $b\in B_1$, we define
  \begin{align*}
    \overline{a} &= \set{a,\phi_2(a),\dots,\phi_n(a)}\\
    \overline{b} &= \set{b,\psi_2(b),\dots,\psi_n(b)}.
  \end{align*}
  We construct a graph by letting $X = \set{\overline{a}\mid a\in A_1}$ and $Y = \set{\overline{b}\mid b\in B_1}$, and, for each $j$, define edges $\set{\overline{a},\overline{b}}$ if $\chi\left(\phi_j(a)\right)\in \overline{b}$.\newline

  Since $\chi$ is a bijection, for each $j=1,\dots,n$, $\chi\left(\phi_j(a)\right)$ must be an element of $B_k$ for some $k$, and since $\set{B_k}_{k=1}^{n}$ are disjoint, $\chi\left(\phi_j(a)\right)$ is an element of exactly one $B_k$. Thus, the graph is $n$-regular.\newline

  By Theorem \ref{thm:konig}, this graph has a perfect matching $F$. As a result, for each $\overline{a}\in X$, there is a unique $\overline{b}\in Y$ and a unique edge $\set{\overline{a},\overline{b}}\in F$ such that $\chi\left(\phi_j(a)\right) = \psi_k(b)$ for some $j,k\in \set{1,\dots,n}$.\newline

  We define
  \begin{align*}
    C_{j,k} &= \set{a\in A_1\mid \set{\overline{a},\overline{b}}\in F,~\chi\left(\phi_j(a)\right) = \psi_k(b)}\\
    D_{j,k} &= \set{b\in B_1\mid \set{\overline{a},\overline{b}}\in F,~\chi\left(\phi_j(a)\right) = \psi_k(b)}.
  \end{align*}
  Therefore, we must have $\psi_{k}^{-1}\circ \chi\circ \phi_j$ is a bijection from $C_{j,k}$ to $D_{j,k}$, so $C_{j,k}\sim_{G^{\ast}}D_{j,k}$.\newline

  Since $C_{j,k}$ and $D_{j,k}$ are partitions of $A_1$ and $B_1$ respectively, it follows that $A_1\sim_{G^{\ast}}B_1$, so $\alpha = \beta$.
\end{proof}
\begin{corollary}\label{corollary:paradoxical_elements}
  Let $\mathcal{S}$ be the type semigroup of some group action, and let $\alpha\in \mathcal{S}$ and $n\in \N$ such that $\left(n+1\right)\alpha \leq n\alpha$. Then, $\alpha = 2\alpha$.
\end{corollary}
\begin{proof}
  We have
  \begin{align*}
    2\alpha + n\alpha &= \left(n+1\right)\alpha + \alpha\\
                      &\leq n\alpha + \alpha\\
                      &= \left(n+1\right)\alpha\\
                      &\leq n\alpha.
  \end{align*}
  Inductively repeating this argument, we get $n\alpha \geq 2n\alpha$; since $n\alpha \leq 2n\alpha$ by definition, we must have $n\alpha = 2n\alpha$, so $\alpha = 2\alpha$.
\end{proof}
\begin{remark}
  We will call such an $\alpha$ a paradoxical element.
\end{remark}
\section{Two Results on Commutative Semigroups}%
Now that we are aware of paradoxical elements and the relationship between $G$-paradoxicality and the properties of the particular elements of the type semigroup (Fact \ref{fact:type_semigroup_paradoxicality}), we will now relate these properties to finitely additive measures of sets by using the following lemma and theorem.
\begin{lemma}\label{lemma:set_function_existence}
  Let $\mathcal{S}$ be a commutative semigroup, with $\mathcal{S}_0\subseteq \mathcal{S}$ finite, and $\epsilon\in \mathcal{S}_0$ satisfying the following assumptions:
  \begin{enumerate}[(a)]
    \item $\left(n+1\right)\epsilon \nleq n\epsilon$ for all $n\in \N$ (i.e., that $\epsilon$ is non-paradoxical);
    \item for each $\alpha\in \mathcal{S}$, there is $n\in \N$ such that $\alpha \leq n\epsilon$.
  \end{enumerate}
  Then, there is a set function $\nu: \mathcal{S}_0\rightarrow [0,\infty]$ that satisfies the following conditions:
  \begin{enumerate}[(i)]
    \item $\nu\left(\epsilon\right) = 1$;
    \item for $\alpha_1,\dots,\alpha_n,\beta_1,\dots,\beta_m\in \mathcal{S}_0$ with $\alpha_1+\cdots+\alpha_n\leq \beta_1+\cdots\beta_m$,
      \begin{align*}
        \sum_{j=1}^{n}\nu\left(\alpha_j\right) \leq \sum_{j=1}^{m}\nu\left(\beta_j\right).
      \end{align*}
  \end{enumerate}
\end{lemma}
\begin{proof}
  We will prove this result by inducting on the cardinality of $\mathcal{S}_0$.\newline

  We start with $\left\vert \mathcal{S}_0 \right\vert = 1$. In that case, we define $\nu\left(\epsilon\right) = 1$, satisfying condition (i). To satisfy condition (ii), we see that for $n,m\in \N$ with $n\epsilon \leq m\epsilon$, if $n \geq m+1$, then $\left(m+1\right)\epsilon \leq n\epsilon \leq m\epsilon$, implying that $\epsilon = 2\epsilon$, which contradicts assumption (a).\newline

  Let $\alpha_0\in \mathcal{S}_0\setminus\set{\epsilon}$. The induction hypothesis says there is a set function satisfying conditions (i) and (ii), $\nu: \mathcal{S}_0\setminus \set{\alpha_0}\rightarrow [0,\infty]$.\newline

  For $r\in \N$, there are $\gamma_1,\dots,\gamma_p,\delta_1,\dots,\delta_q\in \mathcal{S}\setminus \set{\alpha_0}$ such that
  \begin{align*}
    \delta_{1} + \cdots + \delta_q + r\alpha_0 \leq \gamma_1 + \cdots + \gamma_p.\label{set_function_id1}\tag*{(\textdagger)}
  \end{align*}
  Consider the set $N$ defined as follows:
  \begin{align*}
    N &= \set{\frac{1}{r}\left(\sum_{j=1}^{p}\nu\left(\gamma_j\right) - \sum_{j=1}^{q}\nu\left(\delta_j\right)\right)\mid \text{$\gamma_j,\delta_j$ satisfy \ref{set_function_id1}}}. \label{set_function_N}\tag*{($\ddag$)}
  \end{align*}
  We define the extension of $\nu$ as follows:
  \begin{align*}
    \nu\left(\alpha_0\right) &= \inf N.
  \end{align*}
  This infimum is well-defined since, by assumption (b), there is some $n\in \N$ such that $\alpha_0 \leq n\epsilon$, and $\nu\left(\epsilon\right)$ is defined.\newline

  Now, we must show that this extension of $\nu$ satisfies condition (ii).\newline

  Let $\alpha_1,\dots,\alpha_n,\beta_1,\dots,\beta_m\in \mathcal{S}_0\setminus \set{\alpha_0}$ and $s,t\in \N_0$ such that
  \begin{align*}
    \alpha_1 + \cdots + \alpha_n + s\alpha_0 \leq \beta_1 + \cdots + \beta_m + t\alpha_0.\label{set_function_conditionii}\tag*{(\textasteriskcentered)}
  \end{align*}
  We will verify condition (ii) in the three following cases.
  \begin{description}[font=\normalfont\scshape,leftmargin=0cm]
    \item[Case 0:] If $s = t = 0$, then the induction hypothesis states that \ref{set_function_conditionii} satisfies condition (ii).
    \item[Case 1:] Let $s = 0$ and $t > 0$. We want to show that
      \begin{align*}
        \sum_{j=1}^{n}\nu\left(\alpha_j\right) \leq t\nu\left(\alpha_0\right) + \sum_{j=1}^{m}\nu\left(\beta_j\right),
      \end{align*}
      which implies that
      \begin{align*}
        \nu\left(\alpha_0\right) \geq \frac{1}{t}\left(\sum_{j=1}^{n}\nu\left(\alpha_j\right) - \sum_{j=1}^{m}\nu\left(\beta_j\right)\right).
      \end{align*}
      By the definition of infimum, it suffices to show that for $r\in \N$ and $\delta_1,\dots,\delta_q,\gamma_1,\dots,\gamma_p\in \mathcal{S}\setminus \set{\alpha_0}$ satisfying \ref{set_function_id1}, it is the case that
      \begin{align*}
        \frac{1}{r}\left(\sum_{j=1}^{p}\nu\left(\gamma_j\right)-\sum_{j=1}^{q}\nu\left(\delta_j\right)\right) \geq \frac{1}{t}\left(\sum_{j=1}^{n}\nu\left(\alpha_j\right) - \sum_{j=1}^{m}\nu\left(\beta_j\right)\right).
      \end{align*}
      Multiplying \ref{set_function_conditionii} by $r$ on both sides, and adding $t\delta_1 + \cdots + t\delta_q$ to both sides, we have
      \begin{align*}
        r\alpha_1 + \cdots + r\alpha_n + t\delta_1 + \cdots + t\delta_q \leq r\beta_1 + \cdots + r\beta_m + t\left(r\alpha_0\right) + t\delta_1 + \cdots + t\delta_q.
      \end{align*}
      Substituting \ref{set_function_id1}, we find
      \begin{align*}
        r\alpha_1 + \cdots + r\alpha_n + t\delta_1 + \cdots + t\delta_q \leq r\beta_1 + \cdots + r\beta_m + t\gamma_1 + \cdots + t\gamma_p.
      \end{align*}
      Applying the induction hypothesis, we have
      \begin{align*}
        r\sum_{j=1}^{n}\nu\left(\alpha_j\right) + t\sum_{j=1}^{q}\nu\left(\delta_j\right) \leq r\sum_{j=1}^{m}\nu\left(\beta_j\right) + t\sum_{j=1}^{p}\nu\left(\gamma_j\right),
      \end{align*}
      yielding
      \begin{align*}
        \frac{1}{r}\left(\sum_{j=1}^{p}\nu\left(\gamma_j\right) - \sum_{j=1}^{q}\nu\left(\delta_j\right)\right) \geq \frac{1}{t}\left(\sum_{j=1}^{n}\nu\left(\alpha_j\right) - \sum_{j=1}^{m}\nu\left(\beta_j\right)\right).
      \end{align*}
    \item[Case 2:] Let $s > 0$. For $z_1,\dots,z_t\in N$ \ref{set_function_N}, we need to show that
      \begin{align*}
        s\nu\left(\alpha_0\right) + \sum_{j=1}^{n}\nu\left(\alpha_j\right) \leq z_1 + \cdots + z_t + \sum_{j=1}^{n}\nu\left(\beta_j\right).
      \end{align*}
      Without loss of generality, we can set $z_1,\dots,z_n = z$, as for each $z\in N$, $z \geq \nu\left(\alpha_0\right)$.\newline

      As in Case 1, we multiply \ref{set_function_conditionii} by $r$, add $t\delta_{1} + \cdots + t\delta_q$ to both sides, and substitute with \ref{set_function_id1}, yielding
      \begin{align*}
        r\alpha_1 + \cdots + r\alpha_n + rs\alpha_0 + t\delta_1 + \cdots + t\delta_q &\leq r\beta_1 + \cdots + r\beta_m + t\left(r\alpha_0\right) + t\delta_1 + \cdots + t\delta_q\\
        r\alpha_1 + \cdots + r\alpha_n + t\delta_1 + \cdots + t\delta_q + rs\alpha_0 &\leq r\beta_1 + \cdots + r\beta_m + t\gamma_1 + \cdots + t\gamma_p.
      \end{align*}
      Defining
      \begin{align*}
        z &= \frac{1}{r}\left(\sum_{j=1}^{p}\nu\left(\gamma_j\right) - \sum_{j=1}^{q}\nu\left(\delta_j\right)\right),
      \end{align*}
      we get
      \begin{align*}
        s\nu\left(\alpha_0\right) + \sum_{j=1}^{n}\nu\left(\alpha_j\right) &\leq \sum_{j=1}^{n}\nu\left(\alpha_j\right) + \frac{s}{sr}\left(r\sum_{j=1}^{m}\nu\left(\beta_j\right) - r\sum_{j=1}^{n}\nu\left(\alpha_j\right) + t\sum_{j=1}^{p}\nu\left(\gamma_j\right) - t\sum_{j=1}^{q}\nu\left(\delta_j\right)\right)\\
                                                                           &= tz + \sum_{j=1}^{m}\nu\left(\beta_j\right).
      \end{align*}
  \end{description}
  Thus, we have shown that $\nu$ extends in a manner that satisfies conditions (i) and (ii).
\end{proof}

We can ``upgrade'' our finitely additive set function to a semigroup homomorphism as follows.
\begin{theorem}\label{thm:homomorphism_existence}
  Let $\left(\mathcal{S},+\right)$ be a commutative semigroup with identity element $0$, and let $\epsilon\in \mathcal{S}$. Then, the following are equivalent:
  \begin{enumerate}[(i)]
    \item $\left(n+1\right)\epsilon \leq n\epsilon$ for all $n\in \N$;
    \item there is a semigroup homomorphism $\nu: \left(\mathcal{S},+\right)\rightarrow \left([0,\infty],+\right)$ such that $\nu(\epsilon) = 1$.
  \end{enumerate}
\end{theorem}
\begin{proof}
  To show that (ii) implies (i), we let $\nu: \left(\mathcal{S},+\right)\rightarrow \left([0,\infty],+\right)$ be a semigroup homomorphism with $\nu\left(\epsilon\right) = 1$. Then,
  \begin{align*}
    \nu\left(\left(n+1\right)\epsilon\right) &= \left(n+1\right)\nu\left(\epsilon\right)\\
                                             &= n+1\\
                                             &> n\\
                                             &= n\nu\left(\epsilon\right)\\
                                             &= \nu\left(n\epsilon\right),
  \end{align*}
  meaning that $\left(n+1\right)\epsilon \nleq n\epsilon$.\newline

  To show that (i) implies (ii), we suppose that for each $\alpha \in \mathcal{S}$, there is $n\in \N$ such that $\alpha \leq n\epsilon$ --- for any such $\alpha$ for which this is not the case, we define $\nu\left(\alpha\right) = \infty$.\newline

  For a finite subset $\mathcal{S}_0 \subseteq \mathcal{S}$ with $\epsilon\in \mathcal{S}_0$, we define $M_{\mathcal{S}_0}$ to be the set of all $\kappa: \mathcal{S}\rightarrow [0,\infty]$ such that
  \begin{itemize}
    \item $\kappa\left(\epsilon\right) = 1$;
    \item $\kappa\left(\alpha + \beta\right) = \kappa\left(\alpha\right) + \kappa\left(\beta\right)$ for $\alpha,\beta,\alpha + \beta\in \mathcal{S}_0$.
  \end{itemize}
  Since we assume condition (i), we know that such a $\kappa$ with $\kappa\left(\epsilon\right) = 1$ exists. Additionally, since
  \begin{align*}
    \alpha + \beta \leq \left(\alpha + \beta\right)
  \end{align*}
  and
  \begin{align*}
    \left(\alpha + \beta\right) \leq \alpha + \beta,
  \end{align*}
  it is the case that
  \begin{align*}
    \kappa\left(\alpha + \beta\right) \leq \kappa\left(\alpha\right) + \kappa\left(\beta\right) \leq \kappa\left(\alpha + \beta\right),
  \end{align*}
  meaning $\kappa\left(\alpha + \beta\right) = \kappa\left(\alpha\right) + \kappa\left(\beta\right)$. Thus, $M_{\mathcal{S}_0}$ is nonempty. It is also the case that $M_{\mathcal{S}_0}$ is closed, since any net of functions $\kappa_{p}: \mathcal{S}\rightarrow [0,\infty]$ with $\kappa_{p}\left(\epsilon\right) = 1$ and $\kappa_{p}\left(\alpha + \beta\right) = \kappa_{p}\left(\alpha\right) + \kappa_{p}\left(\beta\right)$ will necessarily satisfy these conditions in the limit.\newline

  We let $\left[0,\infty\right]^{\mathcal{S}} = \set{\kappa\mid \kappa:\mathcal{S}\rightarrow [0,\infty]}$ be equipped with the product topology. By Tychonoff's theorem, $\left[0,\infty\right]^{\mathcal{S}}$ is compact.\newline

  Since, for any $\mathcal{S}_1,\dots,\mathcal{S}_n$ finite, it is the case that
  \begin{align*}
    M_{\mathcal{S}_1\cup\cdots\cup \mathcal{S}_n} \subseteq M_{\mathcal{S}_1} \cap \cdots \cap M_{\mathcal{S}_n},
  \end{align*}
  since any such $\kappa\in M_{\mathcal{S}_1\cup\cdots\cup \mathcal{S}_n}$ must necessarily be in every $M_{\mathcal{S}_i}$. Thus, the family
  \begin{align*}
    \set{M_{\mathcal{S}_0}\mid \mathcal{S}_0\subseteq \mathcal{S}\text{ finite}}
  \end{align*}
  has the finite intersection property. Thus, by compactness, there is some $\nu$ such that
  \begin{align*}
    \nu\in \bigcap\set{M_{\mathcal{S}_0}\mid \mathcal{S}_0\subseteq \mathcal{S}\text{ finite}},
  \end{align*}
  with $\nu\left(\epsilon\right) = 1$ and, for all $\alpha,\beta\in \mathcal{S}$, since $\nu\in M_{\set{\alpha,\beta,\alpha + \beta}}$, $\nu\left(\alpha + \beta\right) = \nu\left(\alpha\right) + \nu\left(\beta\right)$.

\end{proof}
\section{Proof of Tarski's Theorem}%
Finally, we are able to prove Tarski's Theorem.
\begin{proof}[of Theorem \ref{thm:tarski}]
  Let $\mathcal{S}$ be the type semigroup of the action of $G$ on $X$.\newline

  Suppose $E$ is not $G$-paradoxical. Then, $\left[E\right]\neq 2\left[E\right]$, meaning $\left(n+1\right)\left[E\right]\nleq n\left[E\right]$ for all $n\in \N$.\newline

  Thus, there is a map $\nu: \mathcal{S}\rightarrow [0,\infty]$ with $\nu\left(\left[E\right]\right) = 1$. The map
  \begin{align*}
    \mu: P(X)\rightarrow [0,\infty]
  \end{align*}
  defined by
  \begin{align*}
    \nu\left(A\right) &= \nu\left(\left[A\right]\right)
  \end{align*}
  is the desired finitely additive measure.
\end{proof}


\chapter{The More Things Change, the More They Stay the Same: Invariant States}\label{ch:invariant_states}
\epigraph{The whole is greater than the sum of its parts.}{Aristotle, who had yet to learn about amenability in groups.}
Tarski's Theorem is one of our first criteria establishing amenability --- that is, a group is amenable if and only if it is non-paradoxical. Tarski's Theorem, while informative about the nature of amenable groups, is unfortunately quite uninformative when it comes to establishing amenability for broader classes of groups. How might we know if a group admits a paradoxical decomposition, or if a group admits \textit{no} paradoxical decompositions?\newline

To establish the amenability of a large class of groups --- as we will do with abelian and solvable groups in this chapter --- we need tools from functional analysis. Rather than focusing on $G$, we will focus on the space $\ell_{\infty}(G)$, and prove the existence of a mean on $G$ by proving the existence of an analogous construct on $\ell_{\infty}(G)$, known as an invariant state.
\section{Amenability in Subgroups and Quotient Groups}\label{sec:amenability_subgroups_quotients}%
We begin by defining a mean on $G$ --- note that this definition is slightly different from the one used in the proof in Theorem \ref{thm:tarski}. However, one can show that they are equivalent by letting $G$ act on itself by left-multiplication and taking $E = G$.
\begin{definition}
  Let $G$ be a group, with $P(G)$ denoting its power set.\newline

  An invariant mean on $G$ is a set function $m\colon P(G)\rightarrow [0,1]$ which satisfies, for all $t\in G$ and $E,F\subseteq G$,
  \begin{itemize}
    \item $m(G) = 1$;
    \item $m\left(E\sqcup F\right) = m(E) + m(F)$;
    \item $m\left(tE\right) = m\left(E\right)$.
  \end{itemize}
  We say $G$ is amenable if $G$ admits a mean.\newline

  The mean $m$ is a translation-invariant probability measure on the measurable space $\left(G,P(G)\right)$.
\end{definition}
We can establish some inheritance properties using the properties of a mean. In Proposition \ref{prop:subgroups_quotientgroups_amenability}, we will show that subgroups of amenable groups are amenable and quotients of amenable groups are amenable.
\begin{proposition}\label{prop:subgroups_quotientgroups_amenability}
  Let $G$ be an amenable group with $H\leq G$. Then, the following are true:
  \begin{enumerate}[(1)]
    \item $H$ is amenable;
    \item for $H\trianglelefteq G$, $G/H$ is amenable.
  \end{enumerate}
\end{proposition}
\begin{proof}\hfill
  \begin{enumerate}[(1)]
    \item Let $R$ be a right transversal for $H$, wherein we select one element of each right coset of $H$ to make up $R$.\newline

      If $m$ is a mean for $G$, we set $\lambda\colon P(H)\rightarrow [0,1]$ defined by
      \begin{align*}
        \lambda(E) = m\left(ER\right).
      \end{align*}
       We have
      \begin{align*}
        \lambda(H) &= m\left(HR\right)\\
                   &= m\left(G\right)\\
                   &= 1.
      \end{align*}
      We claim that if $E\cap F = \emptyset$, then $ER \cap FR = \emptyset$. Suppose toward contradiction this is not the case. Then, $xr_1 = yr_2$ for some $x\in E$, $y\in F$, and $r_1,r_2\in R$. Then, we must have $r_2r_1^{-1} = y^{-1}x \in H$, meaning $r_1 = r_2$ as, by definition, $R$ contains exactly one element of each right coset. Thus, $x=y$, so $E\cap F \neq \emptyset$.\newline

      We then have
      \begin{align*}
        \lambda\left(E\sqcup F\right) &= m\left(\left(E\sqcup F\right)R\right)\\
                                      &= m\left(ER\sqcup FR\right)\\
                                      &= m\left(ER\right) + m\left(FR\right)\\
                                      &= \lambda\left(E\right) + \lambda\left(F\right),
      \end{align*}
      and
      \begin{align*}
        \lambda\left(sE\right) &= m\left(sER\right)\\
                               &= m\left(ER\right)\\
                               &= \lambda\left(E\right).
      \end{align*}
    \item Let $\pi\colon G\rightarrow G/H$ be the canonical projection, defined by $\pi\left(t\right) = tH$. We define
      \begin{align*}
        \lambda\colon P\left(G/H\right) \rightarrow [0,1]
      \end{align*}
      by $\lambda(E) = m\left(\pi^{-1}\left(E\right)\right)$. We have
      \begin{align*}
        \lambda\left(G/H\right) &= m\left(\pi^{-1}\left(G/H\right)\right)\\
                                &= m\left(G\right)\\
                                &= 1,
      \end{align*}
      and
      \begin{align*}
        \lambda\left(E\sqcup F\right) &= m\left(\pi^{-1}\left(E\sqcup F\right)\right)\\
                                      &= m\left(\pi^{-1}\left(E\right)\sqcup \pi^{-1}\left(F\right)\right)\\
                                      &= m\left(\pi^{-1}\left(E\right)\right) + m\left(\pi^{-1}\left(F\right)\right)\\
                                      &= \lambda(E) + \lambda(F).
      \end{align*}
      To show translation-invariance, we let $sH = \pi(s)\in G/H$, and $E\subseteq G/H$. Note that
      \begin{align*}
        \pi^{-1}\left(\pi(s)E\right) &= s\pi^{-1}\left(E\right),
      \end{align*}
      since for $r\in s\pi^{-1}(E)$, we have $r = st$ for $t\in \pi(E)$, so $\pi\left(r\right) =\pi\left(st\right) = \pi\left(s\right)\pi\left(t\right)\in \pi\left(s\right)E$.\newline

      Additionally, if $r\in \pi^{-1}\left(\pi(s)E\right)$, we have $\pi(r)\in \pi(s)E$, so $\pi\left(s^{-1}r\right)\in E$, meaning $s^{-1}r\in \pi^{-1}(E)$.\newline

      Thus,
      \begin{align*}
        \lambda\left(\pi\left(s\right)E\right) &= m\left(\pi^{-1}\left(\pi\left(s\right)E\right)\right)\\
                                               &= m\left(s\pi^{-1}\left(E\right)\right)\\
                                               &= m\left(\pi^{-1}\left(E\right)\right)\\
                                               &= \lambda\left(E\right).
      \end{align*}
  \end{enumerate}
\end{proof}
\section{Establishing Amenability through Functional Analysis}\label{sec:functional_analysis_and_amenability}%
Now that we understand some useful properties of means in relation to groups and subgroups, we turn our attention toward finding means on groups. In order to do this, we turn our attention towards the space $\ell_{\infty}\left(G\right)$, which allows us to use theories from functional analysis to better understand means on $G$. For more elaboration on these ideas, we encourage the reader to review the results in Chapters \ref{ch:measure_theory} and \ref{ch:functional_analysis}.
\begin{definition}
  Let $G$ be a group.
  \begin{enumerate}[(1)]
    \item The space $\mathcal{F}\left(G\right)$ is defined by
      \begin{align*}
        \mathcal{F}\left(G\right) &= \set{f | f\colon G\rightarrow \C\text{ is a function}}.
      \end{align*}
    \item A function $f\in \mathcal{F}\left(G\right)$ is called positive if $f(x) \geq 0$ for all $x\in G$.
    \item A function $f\in \mathcal{F}\left(G\right)$ is called simple if $\ran(f)$ is finite. We let
      \begin{align*}
        \Sigma &= \set{f\in \mathcal{F}\left(G\right) | f\text{ is simple}}.
      \end{align*}
  \end{enumerate}
\end{definition}
\begin{fact}
  It is the case that $\Sigma \subseteq \mathcal{F}\left(G\right)$ is a linear subspace.
\end{fact}
\begin{definition}
  For $E\subseteq G$, we define
  \begin{align*}
    \1_{E}\colon G\rightarrow \C
  \end{align*}
  by
  \begin{align*}
    \1_{E}\left(x\right) &= \begin{cases}
      1 & x\in E\\
      0 & x\notin E
    \end{cases}.
  \end{align*}
  This is the characteristic function of $E$.
\end{definition}
\begin{fact}
  We have
  \begin{align*}
    \Span\set{\1_{E}| E\subseteq G} &= \Sigma.
  \end{align*}
\end{fact}
\begin{proof}
  We see that $\1_{E}\in \Sigma$ for any $E\subseteq G$, and that $\Sigma$ is a subspace.\newline

  If $\phi\in \Sigma$ with $\Ran\left(\phi\right) = \set{t_1,\dots,t_n}$, where $t_i$ are distinct, we set
  \begin{align*}
    E_i &= \phi^{-1}\left(\set{t_i}\right),
  \end{align*}
  yielding
  \begin{align*}
    \phi &= \sum_{i=1}^{n}t_i\1_{E_i}.
  \end{align*}
\end{proof}
\begin{definition}\hfill
  \begin{enumerate}[(1)]
    \item A function $f\in \mathcal{F}\left(G\right)$ is bounded if there exists $M > 0$ such that $\left\vert f(g) \right\vert \leq M$ for all $g\in G$.
    \item The space $\ell_{\infty}\left(G\right)$ is defined by
      \begin{align*}
        \ell_{\infty}\left(G\right) &= \set{f\in \mathcal{F}\left(G\right)| f\text{ is bounded}}.
      \end{align*}
    \item The norm on $\ell_{\infty}\left(G\right)$ is defined by
      \begin{align*}
        \norm{f}_{\ell_{\infty}} &= \sup_{x\in G}\left\vert f(x) \right\vert.
      \end{align*}
  \end{enumerate}
\end{definition}
\begin{proposition}
  The space $\ell_{\infty}(G)$ is complete. Additionally, $\overline{\Sigma} = \ell_{\infty}\left(G\right)$.
\end{proposition}
\begin{proof}
  Let $\left(f_n\right)_n$ be $\norm{\cdot}$-Cauchy in $\ell_{\infty}\left(G\right)$. Then, for all $x\in G$, it is the case that
  \begin{align*}
    \left\vert f_n(x) - f_m(x) \right\vert &= \left\vert \left(f_n - f_m\right)\left(x\right) \right\vert\\
                                           &\leq \norm{f_n - f_m}_{\ell_{\infty}},
  \end{align*}
  meaning $\left(f_n\left(x\right)\right)_n$ is Cauchy in $\C$. We define $f(x) = \lim_{n\rightarrow\infty}f_n(x)$. We must show that $f\in \ell_{\infty}\left(G\right)$, and $\norm{f_n-f}_{\ell_{\infty}}\rightarrow 0$.\newline

  We have
  \begin{align*}
    \left\vert f(x) \right\vert &= \left\vert \lim_{n\rightarrow\infty}f_n\left(x\right) \right\vert\\
                                &= \lim_{n\rightarrow\infty}\left\vert f_n\left(x\right) \right\vert\\
                                &\leq \limsup_{n\rightarrow\infty}\norm{f_n}_{\ell_{\infty}}\\
                                &\leq C,
  \end{align*}
  as Cauchy sequences are always bounded. Thus, $\sup_{x\in G}\left\vert f(x) \right\vert\leq C$.\newline

  Given $\ve > 0$, we find $N$ such that for all $m,n\geq N$, $\norm{f_n - f_m}_{\ell_{\infty}} \leq \ve$. Thus, for $x\in G$, we have
  \begin{align*}
    \left\vert f_n(x) - f_m(x) \right\vert &\leq \norm{f_n - f_m}_{\ell_{\infty}}\\
                                           &\leq \ve.
  \end{align*}
  Taking $m\rightarrow\infty$, we get $\left\vert f_n(x) - f(x) \right\vert \leq \ve$, for all $n\geq N$, so $\norm{f_n - f}_{\ell_{\infty}}\leq \ve$ for all $n\geq N$.\newline

  For real-valued $f\in \ell_{\infty}\left(G\right)$, let $\left\vert f \right\vert \subseteq \left[-M,M\right]$ for some $M > 0$. Let $\ve > 0$. Since $\left[-M,M\right]$ is compact, it is totally bounded, so we can find intervals $I_{1},\dots,I_n$ with $\left[-M,M\right] = \bigsqcup_{k=1}^{n}I_k$, with the length of each $I_k$ less than $\ve$.\newline

  Set $E_k = f^{-1}\left(I_k\right)$. Pick some $t_k\in I_k$. We set
  \begin{align*}
    \phi &= \sum_{i=1}^{n}t_k\1_{E_k}.
  \end{align*}
  Then, it is the case that $\norm{\phi - f}_{\ell_{\infty}} < \ve$.\newline

  If $f\in \ell_{\infty}(G)$ is complex-valued, we apply this process separately to $\re\left(f\right)$ and $\im\left(f\right)$.
\end{proof}
\begin{corollary}
  For any $f\in \ell_{\infty}\left(G\right)$, there is a sequence $\left(\phi_n\right)_n$ of simple functions with $\norm{\phi_n -f}_{\ell_{\infty}}\rightarrow 0$. If $f\geq 0$, then we can select $\phi_n\geq 0$.
\end{corollary}
Now that we understand how simple functions relate to $\ell_{\infty}(G)$, we start by defining a translation action on $\ell_{\infty}(G)$, from which we will be able to convert the idea of means into invariant elements of the state space of the dual of $\ell_{\infty}\left(G\right)$.
\begin{proposition}\label{prop:translation_action}
  Let $G$ be a group. There is an action
  \begin{align*}
    \lambda\colon G\rightarrow \Isom\left(\ell_{\infty}\left(G\right)\right),
  \end{align*}
  where $\lambda(s) = \lambda_s$, defined by
  \begin{align*}
    \lambda_{s}\left(f\right)\left(t\right) &= f\left(s^{-1}t\right)
  \end{align*}
\end{proposition}
\begin{proof}
  We have
  \begin{align*}
    \lambda_s\left(f + \alpha g\right)\left(t\right) &= \left(f + \alpha g\right) \left(s^{-1}t\right)\\
                                                     &= f\left(s^{-1}t\right) \alpha g\left(s^{-1}t\right)\\
                                                     &= \lambda_s\left(f\right)\left(t\right) + \alpha \lambda_s\left(g\right)\left(t\right)\\
                                                     &= \left(\lambda_s\left(f\right) + \alpha \lambda_s\left(g\right)\right)(t).
  \end{align*}
  Thus, $\lambda_s$ is linear. Additionally,
  \begin{align*}
    \norm{\lambda_s\left(f\right)}_{\ell_{\infty}} &= \sup_{t\in G}\left\vert \lambda_s\left(f\right)\left(t\right) \right\vert\\
                                   &= \sup_{t\in G}\left\vert f\left(s^{-1}t\right) \right\vert\\
                                   &= \norm{f}_{\ell_{\infty}},
  \end{align*}
  and
  \begin{align*}
    \norm{\lambda_s\left(f\right) - \lambda_s\left(g\right)}_{\ell_{\infty}} &= \norm{\lambda_s\left(f-g\right)}\\
                                                                             &= \norm{f-g}_{\ell_{\infty}},
  \end{align*}
  meaning $\lambda_s$ is an isometry.\newline

  We have
  \begin{align*}
    \lambda_s\circ \lambda_r\left(f\right)\left(t\right) &= \lambda_r\left(f\right)\left(s^{-1}t\right)\\
                                                         &= \lambda_r\left(r^{-1}s^{-1}t\right)\\
                                                         &= f\left(\left(sr\right)^{-1}t\right)\\
                                                         &= \lambda_{sr}\left(f\right)\left(t\right),
  \end{align*}
  establishing that $\lambda_s\circ \lambda_r = \lambda_{sr}$.\newline

  By a similar process, we find that $\lambda_{s}\left(\1_{E}\right) = \1_{sE}$ for any $E\subseteq G$ and $s\in G$.
\end{proof}
\begin{definition}
  A {state} on $\ell_{\infty}\left(G\right)$ is a continuous linear functional $\mu\in \ell_{\infty}\left(G\right)^{\ast}$ such that the following are true:
  \begin{itemize}
    \item $\mu$ is positive;
    \item $\mu\left(\1_{G}\right) = 1$.
  \end{itemize}
  A state is called left-invariant if
  \begin{align*}
    \mu\left(\lambda_s\left(f\right)\right) = \mu\left(f\right).
  \end{align*}
\end{definition}
\begin{example}\label{ex:finite_invariant_state}
  The evaluation functional, $\delta_x\colon \ell_{\infty}\rightarrow \R$, defined by
  \begin{align*}
    \delta_{x}\left(f\right) &= f(x),
  \end{align*}
  is a state. However, it is not necessarily invariant, as
  \begin{align*}
    \delta_x\left(\lambda_s\left(f\right)\right) &= \lambda_s\left(f\right)\left(x\right)\\
                                                 &= f\left(s^{-1}x\right)\\
                                                 &\neq f(x).
  \end{align*}
  However, we can use the evaluation functional to create an invariant state. If $G$ is finite, we define
  \begin{align*}
    \mu &= \frac{1}{\left\vert G \right\vert} \sum_{x\in G}\delta_x,
  \end{align*}
  which is indeed an invariant state.
\end{example}
We can characterize states slightly differently, which will enable us to show the equivalence between invariant states and means.
\begin{lemma}\label{lemma:characterizing_states}\hfill
  \begin{enumerate}[(1)]
    \item If $\mu$ is a state on $\ell_{\infty}\left(G\right)$, then
      \begin{align*}
        \norm{\mu}_{\op} = 1.
      \end{align*}
    \item If $\mu\in \ell_{\infty}\left(G\right)^{\ast}$ is such that
      \begin{align*}
        \norm{\mu}_{\op} &= \mu\left(\1_{G}\right)\\
                               &= 1,
      \end{align*}
      then $\mu$ is positive and a state.
  \end{enumerate}
\end{lemma}
\begin{proof}\hfill
  \begin{enumerate}[(1)]
    \item Let $\mu$ be a state. Given $f\in \ell_{\infty}\left(G\right)$, we have
      \begin{align*}
        \norm{f}_{\ell_{\infty}}\1_{G} - f &\geq 0\\
        \norm{f}_{\ell_{\infty}}\1_{G} + f &\geq 0,
      \end{align*}
      so
      \begin{align*}
        0 &\leq \mu\left(\norm{f}_{\ell_{\infty}}\1_{G} - f\right) \\
          &= \norm{f}_{\ell_{\infty}}\mu\left(\1_{G}\right) - \mu\left(f\right)
          \intertext{meaning}
        \mu\left(f\right) &\leq \norm{f}_{\ell_{\infty}}.
        \intertext{Additionally,}
        0 &\leq \mu\left(\norm{f}_{\ell_{\infty}}\1_{G} + f\right)\\
          &= \norm{f}_{\ell_{\infty}}\mu\left(\1_{G}\right) + \mu\left(f\right),
          \intertext{meaning}
        -\mu\left(f\right) &\leq \norm{f}_{\ell_{\infty}}.
      \end{align*}
      Thus, we have $\left\vert \mu\left(f\right) \right\vert \leq \norm{f}_{\ell_{\infty}}$, so $\norm{\mu}_{\op} \leq 1$. However, since $\mu\left(\1_{G}\right) = 1$, we must have $\norm{\mu}_{\op} = 1$.
    \item Suppose $\norm{\mu}_{\op} = \mu\left(\1_{G}\right) = 1$. Let $f\geq 0$. Set $g = \frac{1}{\norm{f}_{\ell_{\infty}}}f$.\newline

      Then, $\Ran(g) \subseteq [0,1]$, and $\Ran\left(g - \1_{G}\right) \subseteq \left[-1,1\right]$. Thus, $\norm{g - \1_{G}}_{\ell_{\infty}} \leq 1$.\newline

    Since $\norm{\mu}_{\op} = 1$, we must have
    \begin{align*}
      \left\vert \mu\left(g - \1_{G}\right) \right\vert &\leq 1\\
      \left\vert \mu\left(g\right) - 1 \right\vert &\leq 1,
    \end{align*}
    and since $\mu\left(\1_{G}\right) = 1$, we have $\mu\left(g\right) \in [0,2]$. Thus, $\mu\left(f\right) = \norm{f}_{\ell_{\infty}}\mu\left(g\right) \geq 0$.
  \end{enumerate}
\end{proof}
%To show the equivalence between means and invariant states, we need to be able to characterize the state space on $\ell_{\infty}\left(G\right)^{\ast}$. To do this, we make use of some results from functional analysis.\newline
%
%If $X$ is a normed vector space, then the topology on $X^{\ast}$ induced by $X^{\ast\ast}$ is known as the weak* topology. The weak* topology is the topology of pointwise convergence in $X^{\ast}$ --- a net $\left(\varphi_{\alpha}\right)_{\alpha}$ converges to $\varphi$ in the weak* topology if and only if, for all $\hat{x}\in X^{\ast\ast}$, we have
%\begin{align*}
%  \left(\hat{x}\left(\varphi_{\alpha}\right)\right)_{\alpha}\rightarrow \hat{x}\left(\varphi\right),
%\end{align*}
%or by the definition of $X^{\ast\ast}$,
%\begin{align*}
%  \left(\varphi_{\alpha}\left(x\right)\right) \rightarrow \varphi\left(x\right)
%\end{align*}
%for all $x\in X$.\newline
%
%We state some important results in functional analysis here. The proofs of these results can be found in functional analysis textbooks such as \cite{rudin_functional_analysis}.
%\begin{theorem}[Hahn--Banach Continuous Extension Theorem]
%  Let $X$ be a normed vector space, $E\subseteq X$ a subspace, and $\varphi\in E^{\ast}$ a bounded linear functional. Then, there exists a continuous $\psi\in X^{\ast}$ such that $\norm{\varphi}_{\op} = \norm{\psi}_{\op}$, and $\psi|_{E} = \varphi$.
%\end{theorem}
%\begin{theorem}[Hahn--Banach Separation Theorems]
%  Let $X$ be a normed vector space.
%  \begin{enumerate}[(1)]
%    \item Given a nonzero $x_0\in X$, there is a $\varphi\in X^{\ast}$ with $\norm{\varphi}_{\op} = 1$ and $\varphi\left(x_0\right) = \norm{x}$. We call $\varphi$ a norming functional.
%    \item Given a proper closed subspace $E\subseteq X$ and $x_0\in X\setminus E$, there is a $\varphi\in X^{\ast}$ such that $\varphi|_{E} = 0$, $\norm{\varphi}_{\op} = 1$, and $\varphi\left(x\right) = \dist_{E}(x)$ for all $x\in X$.
%  \end{enumerate}
%\end{theorem}
%\begin{theorem}[Banach--Alaoglu Theorem]
%  Let $X$ be a normed vector space.
%  \begin{enumerate}[(1)]
%    \item The closed unit ball in the dual space, $B_{X^{\ast}}$, is compact in the $w^{\ast}$ topology.
%    \item A subset $C\subseteq X$ is $w^{\ast}$-compact if and only if $C$ is $w^{\ast}$-closed and norm bounded.
%  \end{enumerate}
%\end{theorem}
\begin{corollary}
  The set of states in $\ell_{\infty}\left(G\right)^{\ast}$ forms a $w^{\ast}$-compact subset of $B_{\ell_{\infty}\left(G\right)^{\ast}}$.
\end{corollary}
\begin{proof}
  From the Banach--Alaoglu Theorem (Theorem \ref{thm:banach_alaoglu}), we only need to show that the set of states, $S\left(\ell_{\infty}\left(G\right)\right)$, is $w^{\ast}$-closed, as every element of $S\left(\ell_{\infty}\left(G\right)\right)$ has norm $1$.\newline

  Let $f\in \ell_{\infty}\left(G\right)$ be positive, and let $\left(\varphi_{i}\right)_i$ be a net in $S\left(\ell_{\infty}\left(G\right)\right)$ with $\left(\varphi_{i}\right)_i\xrightarrow{w^{\ast}} \varphi\in \ell_{\infty}\left(G\right)^{\ast}$. From Lemma \ref{lemma:characterizing_states}, we must show that $\varphi$ is positive and $\varphi\left(\1_{G}\right) = 1$.\newline

  We start by seeing that, since each $\varphi_i$ is a state, we have $\varphi_{i}\left(f\right) \geq 0$ for each $i\in I$, so we must have $\varphi\left(f\right) \geq 0$.\newline

  Similarly, since $\varphi_{i}\left(\1_{G}\right) = 1$ for each $i\in I$, and $\left(\varphi_i\right)_i \xrightarrow{w^{\ast}} \varphi$, we have $\varphi\left(\1_{G}\right) = 1$. Thus, by Lemma \ref{lemma:characterizing_states}, we have that $S\left(\ell_{\infty}\left(G\right)\right)$ is $w^{\ast}$-closed.
\end{proof}

Now, we may show the correspondence between invariant states and means.
\begin{proposition}\label{prop:state_implies_mean}
  If $\mu\in \ell_{\infty}\left(G\right)^{\ast}$ is a state, then $m\colon P(G)\rightarrow [0,1]$ defined by $m(E) = \mu\left(\1_{E}\right)$ is a finitely additive probability measure on $G$.\newline

  Moreover, if $\mu$ is invariant, then $m$ is a mean.
\end{proposition}
\begin{proof}
  We have
  \begin{align*}
    m\left(G\right) &= \mu\left(\1_{G}\right)\\
                    &= 1\\
                    \\
    m\left(\emptyset\right) &= \mu\left(0\right)\\
                            &= 0\\
                            \\
    m\left(E\sqcup F\right) &= \mu\left(\1_{E\sqcup F}\right)\\
                            &= \mu\left(\1_{E} + \1_{F}\right)\\
                            &= \mu\left(\1_{E}\right) + \mu\left(\1_{F}\right)\\
                            &= m\left(E\right) + m\left(F\right).
  \end{align*}
  Additionally, since $0 \leq \1_{E}\leq \1_{G}$, we have $0 \leq \mu\left(\1_{E}\right) \leq 1$, so $0 \leq m(E) \leq 1$.\newline

  If $\mu$ is invariant, then
  \begin{align*}
    m\left(sE\right) &= \mu\left(\1_{sE}\right)\\
                     &= \mu\left(\lambda_s\left(\1_{E}\right)\right)\\
                     &= \mu\left(\1_{E}\right)\\
                     &= m\left(E\right).
  \end{align*}
\end{proof}
\begin{proposition}\label{prop:mean_implies_state}
  If $G$ admits a mean, then $\ell_{\infty}\left(G\right)^{\ast}$ admits an invariant state.
\end{proposition}
\begin{proof}
  Let $m$ be a mean. Define $\mu_0\colon \Sigma\rightarrow \R$ by
  \begin{align*}
    \mu_0\left(\sum_{k=1}^{n}t_k\1_{E_k}\right) &= \sum_{k=1}^{n}t_km\left(E_k\right).
  \end{align*}
  Since $m$ is finitely additive, it is the case that $\mu_0$ is well-defined, linear, and positive, with $\mu_0\left(\1_{G}\right) = m\left(G\right) = 1$.\newline

  Additionally, since $m$ is a mean, then for $f = \sum_{k=1}^{n}t_k\1_{E_k}$, we have
  \begin{align*}
    \mu_0\left(\lambda_s\left(f\right)\right) &= \mu_0\left(\lambda_s\left(\sum_{k=1}^{n}t_k\1_{E_k}\right)\right)\\
                                              &= \mu_0\left(\sum_{k=1}^{n}t_k\1_{sE_k}\right)\\
                                              &= \sum_{k=1}^{n}t_km\left(sE_k\right)\\
                                              &= \sum_{k=1}^{n}t_km\left(E_k\right)\\
                                              &= \mu_0\left(f\right).
  \end{align*}
  We see that
  \begin{align*}
    \left\vert \mu_0\left(f\right) \right\vert &= \left\vert \sum_{k=1}^{n}t_km\left(E_k\right) \right\vert\\
                                               &\leq \sum_{k=1}^{n}\left\vert t_k \right\vert m\left(E_k\right)\\
                                               &\leq \sum_{k=1}^{n}\norm{f}_{\ell_{\infty}}\sum_{k=1}^{n}m\left(E_k\right)\\
                                               &= \norm{f}_{\ell_{\infty}}\sum_{k=1}^{n}m\left(E_k\right)\\
                                               &\leq \norm{f}_{\ell_{\infty}},
  \end{align*}
  meaning $\mu_0$ is continuous, so $\mu_0$ is uniformly continuous.\newline

  Since $\overline{\Sigma} = \ell_{\infty}\left(G\right)$, uniform continuity provides that $\mu_0$ extends to a continuous linear functional $\mu\colon \ell_{\infty}\left(G\right)\rightarrow \R$ with $\mu\left(\1_{G}\right) = \mu_0\left(\1_{G}\right) = 1$.\newline

  For $f\geq 0$, we find a sequence $\left(\phi_n\right)_n$ in $\Sigma$ with $\phi_n\geq 0$ and $\norm{\phi_n - f}_{\ell_{\infty}} \xrightarrow{n\rightarrow\infty}0$. We set
  \begin{align*}
    \mu\left(f\right) &= \lim_{n\rightarrow\infty}\mu\left(\phi_n\right)\\
                      &= \lim_{n\rightarrow\infty}\mu_0\left(\phi_n\right)\\
                      &\geq 0,
  \end{align*}
  so $\mu$ is a state.\newline

  If $f\in \ell_{\infty}\left(G\right)$, $s\in G$, and $\left(\phi_n\right)_n$ a sequence in $\Sigma$ with $\left(\phi_n\right)_n\rightarrow f$, then
  \begin{align*}
    \norm{\lambda_s\left(\phi_n\right) - \lambda_s\left(f\right)}_{\ell_{\infty}} &= \norm{\lambda_s\left(\phi_n - f\right)}_{\ell_{\infty}}\\
                                                                                  &= \norm{\phi_n - f}_{\ell_{\infty}}\\
                                                                  &\rightarrow 0.
  \end{align*}
  Thus, we have
  \begin{align*}
    \mu\left(\lambda_s\left(\phi_n\right)\right) &= \mu_0\left(\lambda_s\left(\phi_n\right)\right)\\
                                                 &= \mu_0\left(\phi_n\right)\\
                                                 &= \mu\left(\phi_n\right)\\
                                                 &\rightarrow \mu\left(f\right),
  \end{align*}
  so $\mu\left(f\right) = \mu\left(\lambda_s\left(f\right)\right)$. Thus, $\mu\in \ell_{\infty}\left(G\right)^{\ast}$ is an invariant state.
\end{proof}
\section{Establishing Amenability using Invariant States}\label{sec:amenability_invariant_states}%
Owing to the correspondence between invariant states and means, we are now able to establish amenability for large classes of groups.
\begin{proposition}
  The group of integers, $\Z$, is amenable.
\end{proposition}
\begin{proof}
  We define the left shift, $\lambda_1\colon \ell_{\infty}\left(\Z\right) \rightarrow \ell_{\infty}\left(\Z\right)$, by
  \begin{align*}
    \lambda_1\left(f\right)\left(k\right) &= f\left(k-1\right).
  \end{align*}
  This is an action as in Proposition \ref{prop:translation_action}. \newline

  We set $Y = \Ran\left(\id - \lambda_1\right)\subseteq \ell_{\infty}\left(\Z\right)$. We claim that $\dist_{Y}\left(\1_{\Z}\right) \geq 1$.\newline

  Suppose toward contradiction that there is $y\in Y$ with $\norm{\1_{\Z} - y}_{\ell_{\infty}} = r < 1$. Then, $y = f - \lambda_1 f$ for some $f\in \ell_{\infty}(\Z)$, so
  \begin{align*}
    \norm{\1_{\Z} - \left(f - \lambda_1\left(f\right)\right)}_{\ell_{\infty}} &= r.
  \end{align*}
  Thus, for all $k\in\Z$, we have
  \begin{align*}
    \left\vert 1 - \left(f(k) - f(k-1)\right) \right\vert &\leq r,
  \end{align*}
  so $\left\vert f(k) - f\left(k-1\right) \right\vert \geq 1-r > 0$. However, such an $f$ cannot be bounded.\newline

  Since $\dist_{\overline{Y}}\left(\1_{\Z}\right) = \dist_{Y}\left(\1_{\Z}\right)$, the Hahn--Banach separation theorems provide $\mu\in \left(\ell_{\infty}\left(\Z\right)\right)^{\ast}$ with $\norm{\mu}_{\op} = 1$, $\mu|_{\overline{Y}} = 0$, and $\mu\left(\1_{\Z}\right) = \dist_{Y}\left(\1_{\Z}\right) \geq 1$.\newline

  Since $\norm{\mu}_{\op} = 1$ and $\mu\left(\1_{\Z}\right) \geq 1$, we must have $\mu\left(\1_{\Z}\right) = 1$.\newline

  Additionally, since $\norm{\mu}_{\op} = \mu\left(\1_{\Z}\right) = 1$, we have that $\mu$ is a state on $\ell_{\infty}\left(\Z\right)$, and since $\mu\left(y\right) = 0$ for all $y\in Y$, we have
  \begin{align*}
    \mu\left(f - \lambda_1\left(f\right)\right) &= 0\\
    \mu\left(f\right) &= \mu\left(\lambda_1\left(f\right)\right).
  \end{align*}
  Inductively, this means that $\mu\left(f\right) = \mu\left(\lambda_k\left(f\right)\right)$ for all $k\in \Z$, so $\mu$ is an invariant state on $\ell_{\infty}\left(\Z\right)$. Thus, $\Z$ is amenable.
\end{proof}
\begin{proposition}\label{prop:normal_subgroups_quotient_groups_amenability}
  If $N\trianglelefteq G$ and $G/N$ are amenable, then $G$ is amenable.
\end{proposition}
\begin{proof}
  Let $\rho\in \left(\ell_{\infty}\left(G/N\right)\right)^{\ast}$ be an invariant state, and let $p\colon P(N)\rightarrow [0,1]$ be a mean. For $E\subseteq G$, we define $f_E\colon G/N\rightarrow \R$ by
  \begin{align*}
    f_E\left(tN\right) &= p\left(N\cap t^{-1}E\right).
  \end{align*}
  We start by verifying that $f_E$ is well-defined. For $tN = sN$, we have $s^{-1}t\in N$, so
  \begin{align*}
    p\left(N\cap t^{-1}E\right) &= p\left(s^{-1}t\left(N\cap t^{-1}E\right)\right)\\
                                &= p\left(s^{-1}tN \cap s^{-1}E\right)\\
                                &= p\left(N\cap s^{-1}E\right).
  \end{align*}
  Since $f_E$ is defined through $p$, we can see that $f_E$ is bounded. Additionally,
  \begin{align*}
    f_{E\sqcup F}\left(tN\right) &= p\left(N\cap t^{-1}\left(E\sqcup F\right)\right)\\
                                 &= p\left(N\cap \left(t^{-1}E\sqcup t^{-1}F\right)\right)\\
                                 &= p\left(\left(N\cap t^{-1}E\right) \sqcup \left(N\cap t^{-1}F\right)\right)\\
                                 &= p\left(N\cap t^{-1}E\right) + p\left(N\cap t^{-1}F\right)\\
                                 &= f_E\left(tN\right) + f_F\left(tN\right)\\
                                 &= \left(f_E + f_F\right)\left(tN\right),
  \end{align*}
  and
  \begin{align*}
    f\left(sE\right) \left(tN\right) &= p\left(N\cap t^{-1}sE\right)\\
                                     &= f_E\left(s^{-1}tN\right)\\
                                     &= \lambda_{sN}\left(f_E\right)\left(tN\right),
  \end{align*}
  so $f_{sE} = \lambda_{sN}\left(f_E\right)$. Finally,
  \begin{align*}
    f_G\left(tN\right) &= p\left(N\cap t^{-1}G\right)\\
                       &=p\left(N\right)\\
                       &= 1,
  \end{align*}
  meaning $f_G = \1_{G/N}$.\newline

  We define $m\colon P(G)\rightarrow [0,1]$ by
  \begin{align*}
    m(E) &= \rho\left(f_E\right).
  \end{align*}
  Then, we have
  \begin{align*}
    m\left(E\sqcup F\right) &= m(E) + m(F)\\
                            \\
    m\left(G\right) &= 1\\
    \\
    m\left(sE\right) &= \rho\left(f_{sE}\right)\\
                     &= \rho\left(\lambda_{sN}\left(f_{E}\right)\right)\\
                     &= \rho\left(f_E\right)\\
                     &= m(E),
  \end{align*}
  so $m$ is a mean.
\end{proof}
\begin{corollary}
  The finite direct product of amenable groups is amenable.
\end{corollary}
\begin{proof}
  If $H$ and $K$ are amenable, then $K\cong \left(H\times K\right)/H$ is amenable and $H$ is amenable, so $H\times K$ is amenable by Proposition \ref{prop:normal_subgroups_quotient_groups_amenability}. Induction provides the general case.
\end{proof}
\begin{corollary}\label{cor:finitely_generated_amenable}
  Finitely generated abelian groups are amenable.
\end{corollary}
\begin{proof}
  By the fundamental theorem of finitely generated abelian groups (Theorem \ref{thm:fundamental_thm_abelian_gps}), all finitely generated abelian groups are isomorphic to $\Z^{d}\times \Z/n_1\Z\times\cdots\times \Z/{n_k}\Z$.\newline

  Since $\Z^{d}$ is a finite direct product of $\Z$, and the torsion subgroup $\Z/n_1\Z\times\cdots\times \Z/n_k\Z$ is finite (hence amenable by \ref{ex:finite_invariant_state}), we see that a finitely generated abelian group is a direct product of two amenable groups, hence amenable.
\end{proof}
\begin{corollary}\label{cor:direct_limit_amenable}
  If $\set{G_i}_{i\in I}$ is a directed family of amenable groups, then the direct limit,
  \begin{align*}
    G &= \bigcup_{i\in I}G_i,
  \end{align*}
  is also amenable.
\end{corollary}
\begin{proof}
  Let $\mu_i\in \left(\ell_{\infty}\left(G_i\right)\right)^{\ast}$ be invariant states.\newline

  Set
  \begin{align*}
    M_i &= \set{\mu\in S\left(\ell_{\infty}\left(G\right)\right)| \mu\left(\lambda_s\left(f\right)\right) = \mu\left(f\right)\text{ for all }s\in G_i}.
  \end{align*}
  We set $\mu\left(f\right) = \mu_i\left(f|_{G_i}\right)$. Since each $G_i$ is amenable, it is the case that each $M_i$ is nonempty. Similarly, seeing as we have established the state space as $w^{\ast}$-closed in $B_{\ell_{\infty}\left(G\right)^{\ast}}$, it is the case that each $M_i$ is $w^{\ast}$-closed in $B_{\ell_{\infty}\left(G\right)^{\ast}}$.\newline

  For $i_1,\dots,i_n$, we find $G_j \supseteq G_{i_1},\dots,G_{i_n}$, which exists since $\set{G_i}_{i\in I}$ is directed. We have that $M_j\subseteq \bigcap_{k=1}^{n}M_{i_k}$, so $\set{M_i}_{i\in I}$ has the finite intersection property.\newline

  By compactness, there is $\mu\in \bigcap_{i\in I}M_i$ which is necessarily invariant on $G$.
\end{proof}
\begin{corollary}\label{cor:abelian_groups_amenable}
  All abelian groups are amenable.
\end{corollary}
\begin{proof}
  Every abelian group is the direct limit of its finitely generated subgroups.
\end{proof}
\begin{corollary}\label{cor:solvable_groups_amenable}
  All solvable groups are amenable.
\end{corollary}
\begin{proof}
  Let $e_G = G_0 \leq G_1\leq\cdots\leq G_n\leq G$ be such that $G_{j-1}\trianglelefteq G_j$ for $j=1,\dots,n$, and $G_i/G_j$ is abelian.\newline

  Since $G_0$ is abelian, it is amenable, as is $G_1/G_0$, so $G_1$ is amenable. We see then that $G_2$ is amenable as $G_1$ and $G_2/G_1$ are amenable.\newline

  Continuing in this fashion, we see that $G$ is amenable.
\end{proof}
\section{Remarks and Notes}\label{sec:invariant_states_remarks}%
The following proposition is, in a sense, a kind of converse to Proposition \ref{prop:subgroups_quotientgroups_amenability}, in that if a subgroup is amenable, we can show that the original group is also amenable, but this is only a sufficient condition if the subgroup has finite index.
\begin{proposition}\label{prop:finite_index_amenable_subgroup}
  Let $G$ be a group, and let $H\leq G$ be amenable, with $\left[G:H\right]  = n < \infty$. Then, $G$ is amenable.
\end{proposition}
\begin{proof}
  Let $H\leq G$ be amenable with $\left[G:H\right] = n$. Let $\mu$ be the mean on $H$, and let $\set{g_iH}_{i=1}^{n}$ be a partition of $G$ by the left cosets of $H$. We define the mean on $G$ by taking, for $A\subseteq G$,
  \begin{align*}
    \lambda\left(A\right) &= \frac{1}{n}\sum_{i=1}^{n}\mu\left(g_i^{-1}A\cap H\right).
  \end{align*}
  We begin by verifying that this is well-defined. Specifically, we will show that this definition is independent of the coset representatives. Suppose $g_jH = h_j H$. Then, $h_j^{-1}g_j \in H$. Now, we have $g_j^{-1}A \cap H \subseteq H$, so by left-multiplication, we get $\left(h_j^{-1}g_j\right)g_j^{-1}A\cap H \subseteq H$, so $h_j^{-1}A\cap H\subseteq H$. Since $\set{g_i H}_{i=1}^{n}$ is a partition, we get that this definition of the mean on $G$ is independent of the choice of coset representatives.\newline

  Next, we show that this is a finitely additive measure. Let $A,B\subseteq G$ be such that $A\cap B = \emptyset$. Then, we get
  \begin{align*}
    \lambda\left(A\sqcup B\right) &= \frac{1}{n}\sum_{i=1}^{n}\mu\left(g_i^{-1}\left(A\sqcup B\right)\cap H\right)\\
                                  &= \frac{1}{n}\sum_{i=1}^{n}\mu\left(\left(g_i^{-1}A\cap H\right)\sqcup \left(g_i^{-1}B\cap H\right)\right)\\
                                  &= \frac{1}{n}\left(\sum_{i=1}^{n}\mu\left(g_i^{-1}A\cap H\right) + \sum_{i=1}^{n}\mu\left(g_i^{-1}B\cap H\right)\right)\\
                                  &= \frac{1}{n}\sum_{i=1}^{n}\mu\left(g_i^{-1}A\cap H\right) + \frac{1}{n}\sum_{i=1}^{n}\mu\left(g_i^{-1}B\cap H\right)\\
                                  &= \lambda\left(A\right) + \lambda\left(B\right).
  \end{align*}
  It is relatively simple to see that $\lambda$ is a probability measure, as
  \begin{align*}
    \lambda\left(G\right) &= \frac{1}{n}\sum_{i=1}^{n}\mu\left(g_i^{-1}G\cap H\right)\\
                          &= \frac{1}{n}\sum_{i=1}^{n}\mu\left(G\cap H\right)\\
                          &= \frac{1}{n}\sum_{i=1}^{n}\mu\left(H\right)\\
                          &= 1.
  \end{align*}
  Now, we must show that $\lambda$ is translation-invariant. Let $A\subseteq G$ and $t\in G$. Then, using the translation-invariance of $\mu$, we get
  \begin{align*}
    \lambda\left(tA\right) &= \frac{1}{n}\sum_{i=1}^{n}\mu\left(g_i^{-1}tA\cap H\right)\\
                           &= \frac{1}{n}\sum_{i=1}^{n}\mu\left(g_i^{-1}\left(t\left(A\cap H\right)\right)\right)\\
                           &= \frac{1}{n}\sum_{i=1}^{n}\mu\left(g_i^{-1}A\cap H\right)\\
                           &= \lambda\left(A\right).
  \end{align*}
  Thus, $G$ is amenable.
\end{proof}
In Chapter \ref{ch:tarskis_theorem}, we proved that all the amenable groups are precisely those that are non-paradoxical, while in \ref{ch:paradoxical_decompositions}, we proved the Banach--Tarski paradox finding a subgroup of $\text{SO}(3)$ that is isomorphic to $F(a,b)$. This raises an interesting question: are all non-amenable (hence paradoxical) groups ones that contain subgroups isomorphic to $F(a,b)$?\newline

This is the substance of the von Neumann conjecture --- and as it turns out, it is false. There are some groups that are not amenable, but do not contain a subgroup isomorphic to $F(a,b)$. However, at the same time, in \cite{free_subgroups_of_linear_groups}, Jacques Tits proved that in any subgroup of $\text{GL}_n\left(\F\right)$ (where $\F$ is any field with characteristic zero), a subgroup either admits a solvable subgroup of finite index (hence amenable by Corollary \ref{cor:solvable_groups_amenable} and Proposition \ref{prop:finite_index_amenable_subgroup}) or contains a non-abelian freely generated subgroup (which is necessarily not amenable by Theorem \ref{thm:tarski}). This is known as the Tits alternative. In other words, the von Neumann conjecture \textit{is} true for linear groups, so it necessarily means that we cannot represent the counterexample groups to the von Neumann conjecture as linear groups.\newline

In the introduction to Chapter 3, we stated that the Banach--Tarski paradox cannot hold for $\R$ and $\R^2$. This is because the rotation group $\text{SO}\left(2\right)$ in $\R^2$ is abelian, and since the isometry group $\text{E}\left(2\right)$ has the abelian subgroup $\text{SO}\left(2\right)$ with finite index, $\text{E}\left(2\right)$ is amenable by Proposition \ref{prop:finite_index_amenable_subgroup}. Similarly, the isometry group $\text{E}(1)$ contains an abelian subgroup $\text{SO}(1)$\footnote{$\text{SO}(1) = \set{1}$.} with finite index.

% need snappy name for this too
\chapter{Close Enough: Approximate Means and Følner's Condition}\label{ch:folner_condition}
Amenability, as stated earlier, is defined by a particular finitely additive, translation-invariant probability measure on the group. Of the three conditions for a mean, the ``finitely additive'' and ``probability measure'' conditions are straightforward --- we may define a measure $\delta_x$ on $P(G)$ by saying that $\delta_x(E) = 1$ if $x\in E$ and $\delta_x(E) = 0$ if $x\notin E$. This is a finitely additive probability measure --- but it is not translation-invariant.\newline

The translation-invariance condition is, generally speaking, the condition that throws a wrench into our desire to establish means on various types of groups. For instance, we desired a translation-invariant, finitely additive probability measure on $F(a,b)$, but since $bW\left( b^{-1} \right)$ is equal to $F(a,b)\setminus W(b)$, we see that the translation $bW\left( b^{-1} \right)$ creates a ``bigger'' subset than we desire, closing off our ability to construct a mean.\newline

As the reader may remark by now, this is a comically unrigorous idea. What does it mean for a set to become ``bigger'' under translation, and how much ``bigger'' does it need to become in order to close off the possibility of establishing a mean on the group?\newline

The Følner condition will allow us to make the idea of ``bigness'' precise. In this chapter, we will show exactly how the Følner condition then allows to establish amenability in groups, specifically by constructing an approximate mean, then showing that ``approximate amenability'' and amenability are equivalent.
\section{Følner's Condition}%
\begin{definition}\label{def:folner_condition}
  A group is said to satisfy the \textit{Følner condition} if, for every $\ve > 0$ and $E\subseteq G$, there is a nonempty finite subset $F\subseteq G$ such that for all $t\in E$,
  \begin{align*}
    \frac{\left\vert tF\triangle F \right\vert}{\left\vert F \right\vert}\leq \ve.
  \end{align*}
  Equivalently, we can also say that the Følner condition is satisfied if and only if
  \begin{align*}
    \frac{\left\vert tF\cap F \right\vert}{\left\vert F \right\vert} \geq 1 - \ve
  \end{align*}
  for every $\ve > 0$.
\end{definition}
\begin{lemma}\label{lemma:folner_sequences}
  A countable group $G$ satisfies the Følner condition if and only if $G$ admits a sequence $\left(F_n\right)_n$ with $F_n\subseteq G$ finite such that
  \begin{align*}
    \left(\frac{\left\vert tF_n\triangle F_n \right\vert}{\left\vert F_n \right\vert}\right)_n \xrightarrow{n\rightarrow \infty}0
  \end{align*}
  for all $t\in G$. Such a sequence is known as a \textit{Følner sequence}.
\end{lemma}
\begin{proof}
  Let $G$ admit a Følner sequence, $\left(F_n\right)_n$. Given $\ve > 0$ and $E\subseteq G$ finite, find $N$ such that for all $s\in E$ and $n\geq N$,
  \begin{align*}
    \frac{\left\vert sF_n\triangle F_n \right\vert}{\left\vert F_n \right\vert} &\leq \ve.
  \end{align*}
  We take $F = F_N$ in the definition of the Følner condition.\newline

  Let $G$ satisfy the Følner condition. We write $G = \bigcup_{n\geq 1}E_n$, with $E_1\subseteq E_2\subseteq \cdots$, and define $F_n$ such that for all $t\in E_n$,
  \begin{align*}
    \frac{\left\vert tF_n\triangle F_n \right\vert}{\left\vert F_n \right\vert} &\leq \frac{1}{n}.
  \end{align*}
  Given $t\in G$, then $t\in E_N$ for some $N$, so $t\in E_n$ For all $n\geq N$, so
  \begin{align*}
    \frac{\left\vert tF_n\triangle F_n \right\vert}{\left\vert F_n \right\vert} &\leq \frac{1}{n}
  \end{align*}
  for all $n\geq N$. Thus,
  \begin{align*}
    \left(\frac{\left\vert tF_n\triangle F_n \right\vert}{\left\vert F_n \right\vert}\right)\xrightarrow{n\rightarrow\infty}0.
  \end{align*}
\end{proof}
\begin{lemma}\label{lemma:folner_condition_generating_set}
  Let $G$ be a finitely generated group with generating set $S$ (see Definition \ref{def:generating_sets}). If $\left(F_n\right)_n$ is a sequence of finite subsets such that, for all $s\in S$,
  \begin{align*}
    \left(\frac{\left\vert sF_n\triangle F_n \right\vert}{\left\vert F_n \right\vert}\right)_n\rightarrow 0,
  \end{align*}
  then $\left(F_n\right)_n$ is a Følner sequence for $G$.
\end{lemma}
\begin{proof}
  Note that
  \begin{itemize}
    \item $s\left(A\triangle B\right) = sA\triangle sB$;
    \item $A\triangle C \subseteq \left(A\triangle B\right) \cup \left(B\triangle C\right)$.
  \end{itemize}
  We see that for any $s\in S$,
  \begin{align*}
    \frac{\left\vert s^{-1}F_n\triangle F_n \right\vert}{\left\vert F_n \right\vert} &= \frac{\left\vert s^{-1}\left(F_n\triangle sF_n\right) \right\vert}{\left\vert F_n \right\vert}\\
                                                                                     &= \frac{\left\vert F_n\triangle sF_n \right\vert}{\left\vert F_n \right\vert}\\
                                                                                     &\rightarrow 0.
  \end{align*}
  Thus, we may assume that $S$ is symmetric --- i.e., that $\set{s^{-1}| s\in S} = \set{s | s\in S}$.\newline

  For any $s,t\in S$, we have
  \begin{align*}
    \frac{\left\vert stF_n\triangle F_n \right\vert}{\left\vert F_n \right\vert} &\leq \frac{\left\vert stF_n\triangle F_n \right\vert}{\left\vert F_n \right\vert} + \frac{\left\vert sF_n\triangle F_n \right\vert}{\left\vert F_n \right\vert}\\
                                                                                 &= \frac{\left\vert s\left(tF_n\triangle F_n\right) \right\vert}{\left\vert F_n \right\vert} + \frac{\left\vert sF_n\triangle F_n \right\vert}{\left\vert F_n \right\vert}\\
                                                                                 &= \frac{\left\vert tF_n\triangle F_n \right\vert}{\left\vert F_n \right\vert} + \frac{\left\vert sF_n\triangle F_n \right\vert}{\left\vert F_n \right\vert}\\
                                                                                 &\rightarrow 0.
  \end{align*}
  We use induction to find the general case.
\end{proof}
\begin{example}
  Consider the group $\Z$. Since $\Z$ is generated by the element $\set{1}$, we see that for the sets $F_n = \set{-n,-n+1,\dots,n-1,n}$, that
  \begin{align*}
    \frac{\left\vert \left(F_n + 1\right)\triangle F_n \right\vert}{\left\vert F_n \right\vert} &= \frac{2}{2n+1}\\
                                                                                                &\rightarrow 0,
  \end{align*}
  meaning that $\Z$ satisfies the Følner condition.
\end{example}
\section{From Følner's Condition to Amenability}\label{sec:approximate_means}%
We have thus far proven that $G$ satisfies the Følner condition if and only if $G$ admits a Følner sequence, and that $G$ is amenable if and only if $G$ admits an invariant state.\newline

We will now begin harmonizing these two characterizations through the use of approximate means, eventually showing that $G$ satisfies the Følner condition if and only if $G$ admits an approximate mean, and that $G$ admits an approximate mean if and only if $G$ is amenable.
\begin{definition}\label{def:state_on_prob_g}
  For a group $G$, we define
  \begin{align*}
    \Prob\left(G\right) = \set{f\colon G\rightarrow [0,\infty) | \Card\left(\supp(f)\right)  < \infty,~\sum_{t\in G}f(t) = 1}.
  \end{align*}
  Note that $\Prob(G) \subseteq B_{\ell_1\left(G\right)}$. For $f\in \prob(G)$, we set $\varphi_f\colon \ell_{\infty}(G)\rightarrow \C$ defined by
  \begin{align*}
    \varphi_f\left(g\right) &= \sum_{t\in G}g(t)f(t).
  \end{align*}
\end{definition}
\begin{fact}\label{fact:prob_g_state}
  For $f\in \prob(G)$, $\varphi_f$ is a state on $\ell_{\infty}\left(G\right)$.
\end{fact}
\begin{proof}
We can see that, by definition, $\varphi_f$ is positive, linear, and has $\varphi_f\left(\1_{G}\right) = 1$.\newline

We only need to show that $\norm{\varphi_f}_{\op} = 1$. We see that
\begin{align*}
  \left\vert \varphi_f\left(g\right) \right\vert &= \left\vert \sum_{t\in G}g(t)f(t) \right\vert\\
                                                 &\leq \sum_{t\in G}\left\vert g(t) \right\vert\left\vert f(t) \right\vert\\
                                                 &\leq \norm{g}_{\ell_\infty}\sum_{t\in G}\left\vert f(t) \right\vert\\
                                                 &= \norm{g}_{\ell_\infty},
\end{align*}
so $\norm{\varphi_f}_{\op} \leq 1$. Since $\varphi_f\left(\1_G\right) = 1$, we must have $\norm{\varphi_f}_{\op} = 1$.
\end{proof}
\begin{proposition}
  There is an action $\lambda\colon G\rightarrow \Isom\left(\ell_{1}\left(G\right)\right)$ such that $\prob(G)$ is invariant.
\end{proposition}
\begin{proof}
  Let $\lambda_s\left(f\right)\left(t\right) = f\left(s^{-1}t\right)$. Then,
  \begin{align*}
    \norm{\lambda_s\left(f\right)}_{\ell_1} &= \sum_{t\in G}\left\vert \lambda_s\left(f\right)\left(t\right) \right\vert\\
                                     &= \sum_{t\in G}\left\vert f\left(s^{-1}t\right) \right\vert\\
                                     &= \sum_{r\in G}\left\vert f(r) \right\vert\\
                                     &= \norm{f}_{\ell_1}.
  \end{align*}
  Just as in Proposition \ref{prop:translation_action}, it is the case that $\lambda_s$ is linear. Additionally,
  \begin{align*}
    \lambda_r\circ \lambda_s\left(f\right)\left(t\right) &= \lambda_s\left(f\right)\left(r^{-1}t\right)\\
                                                         &= f\left(s^{-1}r^{-1}\left(t\right)\right)\\
                                                         &= f\left(\left(rs\right)^{-1}t\right)\\
                                                         &= \lambda_{rs}\left(f\right)\left(t\right).
  \end{align*}
  We see that if $f\in \prob(G)$, then for $f\geq 0$, we have $\lambda_s\left(f\right) \geq 0$, and
  \begin{align*}
    \sum_{t\in G}\lambda_s\left(f\right)\left(t\right) &= \sum_{t\in G}f\left(s^{-1}t\right)\\
                                                       &= \sum_{r\in G}f\left(r\right)\\
                                                       &= 1
  \end{align*}
  for any $f\in \prob(G)$.
\end{proof}
\begin{definition}\label{def:approximate_mean}
  For a countable group $G$, a sequence $\left(f_k\right)_k$ is called an approximate mean if, for all $s\in G$,
  \begin{align*}
    \norm{f_k - \lambda_s\left(f_k\right)}_{\ell_1} &\xrightarrow{k\rightarrow \infty}0.
  \end{align*}
\end{definition}
%Reorganize this one so we do Følner if and only if approximate mean, and approximate mean if and only if amenable. That allows us to include some commentary about what implicit conditions/establish facts are used
To begin the forward direction regarding the equivalence between the Følner condition, approximate means, and means, we begin by showing that the existence of a Følner sequence implies the existence of an approximate mean. Then, we will show that the existence of an approximate mean implies the existence of an invariant state (hence mean).
\begin{proposition}\label{prop:folner_implies_approx_mean}
  If $G$ admits a Følner sequence $\left(F_k\right)_k$, then $G$ admits an approximate mean.
\end{proposition}
\begin{proof}
  Set $f_k = \frac{1}{\left\vert F_k \right\vert}\1_{F_k}\in \prob(G)$. Then,
  \begin{align*}
    \norm{f_k - \lambda_s\left(f_k\right)}_{\ell_1} &= \frac{1}{\left\vert F_k \right\vert} \norm{\1_{F_k} - \lambda_s\left(\1_{F_k}\right)}_{\ell_1}\\
                                                    &= \frac{1}{F_k}\norm{\1_{F_k} - \1_{sF_k}}_{\ell_1}\\
                                               &= \frac{\left\vert F_k\triangle sF_k \right\vert}{\left\vert F_k \right\vert}\\
                                               &\rightarrow 0.
  \end{align*}
\end{proof}
% reverse direction here
To show that the existence of an approximate mean implies the Følner condition, we require the following lemma.
\begin{lemma}\label{lemma:layer_cake_representation}
  Let $f\colon S\rightarrow \R$ be finitely supported with $\sum_{s\in S}f(s) = 1$. Then, there exist subsets $\set{F_i}_{i=1}^{n}$, where $F_1\supseteq F_2\supseteq \cdots \supseteq F_n$, and constants $\set{c_i}_{i=1}^{n}$, such that
  \begin{align*}
    f &= \sum_{i=1}^{n}c_i\1_{F_i},
  \end{align*}
  where
  \begin{align*}
    \sum_{i=1}^{n}c_i\left\vert F_i \right\vert &= 1.
  \end{align*}
  This is known as the layer cake representation for $f$.
\end{lemma}
\begin{proof}
  We define $F_1 = \supp\left(f\right)$, and take $c_1 = \min\left(\Ran\left(f\right)\right)$. Taking $E_1 = f^{-1}\left(c_1\right)$ (as a set-theoretic inverse), we define $F_2 = F_1\setminus E_1$.\newline

  Take $d_1 = \min\left(f\left(F_2\right)\right)$, and define $c_2 = d_1 - c_1$. Then, defining $E_2 = f^{-1}\left(d_1\right)$, $F_3 = F_2 \setminus E_2$, and $d_2 = \min\left(f\left(F_3\right)\right)$, we define $c_3 = d_2 - c_2 - c_1$.\newline

  Continuing in this pattern, we find $d_{i-1} = \min\left(f\left(F_i\right)\right)$, $E_i = f^{-1}\left(d_{i-1}\right)$, and $c_i = d_{i-1} - \sum_{j=1}^{i-1}c_i$.\newline

  This yields a decomposition $F_1\supseteq F_2\supseteq \cdots \supseteq F_n$, where $\sum_{i=1}^{n}c_i\1_{F_i} = f$ by construction.\newline

  We now verify that $\sum_{i=1}^n c_i\left\vert F_i \right\vert = 1$.
  \begin{align*}
    1 &= \sum_{s\in S}f(s)\\
      &= \sum_{s\in S}\sum_{i=1}^{n}c_i\1_{F_i}\left(s\right).
      \intertext{By definition, if $s\in F_j$ for some $j$, we see that $c_j$ is summed for $\left\vert F_j \right\vert$ many times. Thus, we obtain}
      &= \sum_{i=1}^{n}c_i\left\vert F_i \right\vert.
  \end{align*}
\end{proof}
\begin{remark}
  Instead of using this construction where we take set-theoretic inverses and remove ``residual'' sets, there is an alternative method of construction that involves ordering the range as $r_1 < r_2< \cdots < r_n$, and constructing the set $F_k = \set{s | f(s) \geq r_k}$.
\end{remark}
We will use the layer cake decomposition to prove that if $G$ admits an approximate mean, then $G$ satisfies the Følner condition.
\begin{proposition}\label{prop:approx_mean_implies_folner}
  Let $G$ admit an approximate mean. Then, $G$ satisfies the Følner condition.
\end{proposition}
\begin{proof}
  Let $\left(f_k\right)_k$ be an approximate mean, as in Definition \ref{def:approximate_mean}. Fix a finite nonempty set $S \subseteq G$. Then, by the definition of an approximate mean, there must exist some $N\in\N$ such that for all $k\geq N$ and all $s\in G$,
  \begin{align*}
    \norm{f_k - \lambda_s\left(f_k\right)}_{\ell_1} &\leq \frac{\ve}{|S|}.
  \end{align*}
  In particular, this holds for $f_N$ and for all $s\in S$.\newline

  Since $f_N\in \Prob(G)$ is finitely supported and $\sum_{s\in G}f_N(s) = 1$, we may use Lemma \ref{lemma:layer_cake_representation} to rewrite $f_N$ as
  \begin{align*}
    f_N &= \sum_{i=1}^{n}c_i\1_{F_i},
  \end{align*}
  where $F_1 \supseteq F_2\supseteq \cdots \supseteq F_n$, and $\sum_{i=1}^{n}c_i\left\vert F_i \right\vert = 1$.\newline

  For a given $1 \leq i \leq n$, for each $s\in S$ and $t\in sF_i\triangle F_i$, we have
  \begin{align*}
    f_N\left(t\right) - f_N\left(s^{-1}t\right) &= \begin{cases}
      c_i & t\in F_i\setminus sF_i\\
      -c_i & t\in sF_i \setminus F_i
    \end{cases}.
  \end{align*}
  Thus, we see that $\left\vert f_N\left(t\right)-\lambda_s\left(f_N\right)\left(t\right) \right\vert\geq c_i$ on $sF_i\triangle F_i$. Thus, for each $s\in S$,
  \begin{align*}
    \sum_{i=1}^{n}c_i\left\vert sF_i \triangle F_i \right\vert &\leq \sum_{t\in S}\left\vert f_N\left(t\right) -  \lambda_s\left(f\right)\left(t\right)\right\vert\\
                                                               &< \frac{\ve}{\left\vert S \right\vert}\\
                                                               &= \frac{\ve}{\left\vert S \right\vert} \sum_{i=1}^{n}c_i\left\vert F_i \right\vert.
  \end{align*}
  Therefore, we have
  \begin{align*}
    \sum_{s\in S}\sum_{i=1}^{n}c_i\left\vert sF_i\triangle F_i \right\vert &< \frac{\ve}{\left\vert S \right\vert}\sum_{s\in S}\sum_{i=1}^{n}c_i\left\vert F_i \right\vert\\
                                                                           &= \ve \sum_{i=1}^{n}c_i\left\vert F_i \right\vert.
  \end{align*}
  Thus, by the pigeonhole principle, there must exist some $1\leq i \leq n$ for which
  \begin{align*}
    \sum_{s\in S}c_i\left\vert sF_i\triangle F_i \right\vert < \ve c_i\left\vert F_i \right\vert.
  \end{align*}
  Setting $F = F_i$, we find that, for all $s\in S$,
  \begin{align*}
    \frac{\left\vert sF\triangle F \right\vert}{\left\vert F \right\vert} &\leq \sum_{s\in S}\frac{\left\vert sF\triangle F \right\vert}{\left\vert F \right\vert}\\
                                                                          &< \ve.
  \end{align*}
\end{proof}
Now, we show that approximate amenability and amenability are equivalent. This will require some more heavy lifting from functional analysis.
\begin{proposition}\label{prop:approx_mean_implies_amenable}
  If $G$ admits an approximate mean, then $G$ is amenable.
\end{proposition}
\begin{proof}
  Let $\left(f_k\right)_k$ be an approximate mean.\newline

  We define $\varphi_k = \left(\varphi_{f_k}\right)_k$ (as in Definition \ref{def:state_on_prob_g}) to be a net of states on $\ell_{\infty}\left(G\right)$.\newline

  Since the state space on $\ell_{\infty}\left(G\right)$ is $w^{\ast}$-compact (Corollary \ref{cor:state_space_compact}), there is a state $\mu$ and a subnet $\left(\varphi_{k_j}\right)_j \xrightarrow{w^{\ast}}\mu$. \newline

  We only need to show that $\mu$ is invariant. Note that
  \begin{align*}
    \left\vert \mu\left(g\right) - \mu\left(\lambda_s\left(g\right)\right) \right\vert &\leq \left\vert \mu\left(g\right) - \varphi_{k_j}\left(g\right) \right\vert + \left\vert \varphi_{k_j}\left(g\right) - \varphi_{k_j}\left(\lambda_s\left(g\right)\right) \right\vert + \left\vert \varphi_{k_j}\left(\lambda_s\left(g\right)\right) - \mu\left(\lambda_s\left(g\right)\right) \right\vert
  \end{align*}
  for all $g\in \ell_{\infty}\left(G\right)$, $s\in G$, and all $j$.\newline

  Given $\ve > 0$, we find $J$ such that for $j\geq J$,
  \begin{align*}
    \left\vert \mu\left(g\right) - \varphi_{k_j}\left(g\right) \right\vert &< \ve/3\\
    \left\vert \mu\left(\lambda_s\left(g\right)\right) - \varphi_{k_j}\left(\lambda_s\left(g\right)\right)\right\vert &< \ve/3.
  \end{align*}
  We also see that
  \begin{align*}
    \left\vert \varphi_{k_j}\left(g\right) - \varphi_{k_j}\left(\lambda_s\left(g\right)\right) \right\vert &= \left\vert \sum_{t\in G}g(t)f_{k_j}\left(t\right) - \sum_{t\in G}g\left(s^{-1}t\right)f_{k_j}\left(t\right) \right\vert\\
                                                                                                           &= \left\vert \sum_{t\in G}g(t)f_{k_j}\left(t\right) - \sum_{r\in G}g(r)f_{k_j}\left(sr\right) \right\vert \tag*{$r = s^{-1}t$}\\
                                                                                                           &= \left\vert \sum_{t\in G}g(t)\left(f_{k_j}\left(t\right)-\lambda_{s^{-1}}\left(f_{k_j}\right)\left(t\right)\right) \right\vert\\
                                                                                                           &\leq \norm{g}_{\ell_\infty}\sum_{t\in G}\left\vert f_{k_j}\left(t\right) - \lambda_{s^{-1}}\left(f_{k_j}\right)\left(t\right) \right\vert\\
                                                                                                           &= \norm{g}_{\ell_\infty}\norm{f_{k_j} - \lambda_{s^{-1}}\left(f_{k_j}\right)}_{\ell_1}\\
                                                                                                           &< \ve/3
  \end{align*}
  for large $j$. Thus, we have
  \begin{align*}
    \left\vert \mu\left(g\right) - \mu\left(\lambda_{s}\left(g\right)\right) \right\vert &< \ve,
  \end{align*}
  for all $\ve > 0$, so $\mu\left(g\right) = \mu\left(\lambda_{s}\left(g\right)\right)$.
\end{proof}
%We will now commence with the reverse direction, starting by showing that amenability implies the existence of an approximate mean, and then showing that the existence of an approximate mean implies that the Følner condition is satisfied.
\begin{proposition}\label{prop:amenable_implies_approx_mean}
  If $G$ is amenable, then $G$ admits an approximate mean.
\end{proposition}
\begin{proof}
  Suppose $G$ does not admit an approximate mean. Then, there exists a finite subset $E_0\subseteq G$ and $\ve_0 > 0$ such that for all $s\in E_0$ and all $f\in \Prob(G)$, we have $\norm{f - \lambda_s\left(f\right)} \geq \ve_0$.\newline

  Consider the set
  \begin{align*}
    X &= \bigoplus_{\left\vert E_0 \right\vert} \ell_1\left(G\right),
  \end{align*}
  endowed with the norm
  \begin{align*}
    \norm{\left(f_s\right)_{s\in E_0}}_{\ell_1} &= \sum_{s\in E_0}\sum_{t\in G}\left\vert f_s(t) \right\vert\\
                                       &= \sum_{s\in E_0}\norm{f_s}_{\ell_1},
  \end{align*}
  and let
  \begin{align*}
    C &= \set{\left(f - \lambda_s\left(f\right)\right)_{s\in E_0} | f\in \Prob(G)}.
  \end{align*}
  Since $\Prob(G)$ is convex, it is the case that $C$ is convex, and since $\left\vert E_0 \right\vert$ is finite, $C$ is necessarily bounded. Note that $0\notin \overline{C}$.\newline

  By the Hahn--Banach separation for convex sets (Theorem \ref{thm:hb_separation_lctvs}), there is a real-valued $\varphi\in X^{\ast}$ such that $\varphi\left(C\right)\geq 1$. Here,
  \begin{align*}
    X^{\ast} &\cong \bigoplus_{\left\vert E_0 \right\vert}\ell_1\left(G\right)^{\ast}\\
             &\cong \sum_{\left\vert E_0 \right\vert}\ell_{\infty}\left(G\right),
  \end{align*}
  endowed with the norm
  \begin{align*}
    \norm{\left(g_s\right)_{s\in E_0}}_{\ell_{\infty}} &= \max_{s\in E_0}\left(\sup_{t\in G}\left\vert g_s(t) \right\vert\right)\\
                                                       &= \max_{s\in E_0}\norm{g_s}_{\ell_{\infty}}.
  \end{align*}
  We let $\varphi = \left(\varphi_{g_s}\right)_{s\in E_0}$, where $g_s\in \ell_{\infty}\left(G\right)$ is defined by the duality
  \begin{align*}
    \varphi_{g_s}\left(f\right) &= \sum_{t\in G}f(t)g_s(t).
  \end{align*}
  Thus, for all $f\in \Prob(G)$, we have
  \begin{align*}
    1 &\leq \varphi\left(\left(f - \lambda_s\left(f\right)\right)_{s\in E_0}\right)\\
      &= \sum_{s\in E_0}\varphi_{g_s}\left(f - \lambda-s\left(f\right)\right)\\
      &= \sum_{s\in E_0}\sum_{t\in G}\left(f - \lambda_s\left(f\right)\right)(t)g_s(t)\\
      &= \sum_{s\in E_0}\left(\sum_{t\in G}f(t)g_s(t) - \sum_{t\in G}f\left(s^{-1}t\right)g_s(t)\right)\\
      &= \sum_{s\in E_0}\left(\sum_{t\in G}f(t)g_s(t) - \sum_{r\in G}f\left(r\right)g_s\left(sr\right)\right)\\
      &= \sum_{s\in E_0}\left(\sum_{r\in G}f(r)g_s(r) - \sum_{r\in G}f(r)\lambda_{s^{-1}}\left(g\right)(r)\right)\\
      &= \sum_{s\in E_0}\sum_{r\in G}f(r)\left(g_s - \lambda_{s^{-1}}\left(g_s\right)\right)(r).
      \intertext{Note that this holds for any $f\in \Prob(G)$, including the case of $f = \delta_t$ for a given $t\in G$. We must have}
      &= \sum_{s\in E_0}\sum_{r\in G}\delta_{t}\left(r\right)\left(g_s\left(r\right) - \lambda_{s^{-1}}\left(g_s\right)\right)\left(r\right)\\
      &= \sum_{s\in E_0}\left(g_s - \lambda_{s^{-1}}\left(g_s\right)\right)\left(t\right).
  \end{align*}
  This gives
  \begin{align*}
    \1_{G} &\leq \sum_{s\in E_0}\left( g_s - \lambda_{s^{-1}}\left(g_s\right) \right)(t).
  \end{align*}
  Since $G$ is amenable, there is a mean $\mu\colon \ell_{\infty}\left(G\right)\rightarrow \C$ with $\mu\left(g_s\right) = \mu\left(\lambda_{s^{-1}}\left(g_s\right)\right)$. Therefore, we have
  \begin{align*}
    0 &= \mu\left(\sum_{s\in E_0}\left(g_s - \lambda_{s^{-1}}\left(g_s\right)\right)\left(t\right)\right)\\
      &\geq \mu\left(\1_{G}\right)\\
      &= 1,
  \end{align*}
  which is a contradiction. Therefore, $G$ admits an approximate mean.
\end{proof}
\section{Applying Følner's Condition: Groups of Subexponential Growth}\label{sec:subexponential_growth}%
Before we discuss representations of groups inside the algebra of bounded operators on a Hilbert space, we will provide an application of Følner's condition by taking a tour into geometric group theory. In this section, we will establish the amenability of yet another wide class of groups (just as we established that all abelian groups are amenable in Chapter 5) --- the groups of subexponential growth.\newline

First, we construct a little bit of machinery to understand the growth rate of a group, then we prove that Følner's condition holds for these special classes of groups.
\begin{definition}\label{def:word_metric}
  Let $G$ be a group with finite symmetric generating set $S$ (see Definition \ref{def:generating_sets}). Then, we define the word length of $g\in G$ with respect to $S$ to be
  \begin{align*}
    \ell_{G,S}\left(g\right) &= \min\set{n | g = s_1\dots s_n,~s_i\in S},
  \end{align*}
  taking $\ell_{G,S}\left(e_G\right) = 0$. We define the word metric on $G$ with respect to $S$ by taking
  \begin{align*}
    d_{S}\left(g,h\right) &= \ell_{G,S}\left(g^{-1}h\right).
  \end{align*}
\end{definition}
\begin{fact}\label{fact:word_metric_equivalent_metrics}
  If $S$ and $T$ are finite symmetric generating sets for $G$, then the respective word metrics $d_{S}$ and $d_{T}$ are equivalent (as in the sense of Definition \ref{def:metrics_and_equivalent_metrics}).
\end{fact}
\begin{proof}
  We start by showing that $d_S$ is indeed a metric. Notice that the following facts necessarily hold by our definition of the word length:
  \begin{itemize}
    \item $\ell_{G,S}\left(g\right) = \ell_{G,S}\left(g^{-1}\right)$;
    \item $\ell_{G,S}\left(gh\right) \leq \ell_{G,S}\left(g\right) + \ell_{G,S}\left(h\right)$.
  \end{itemize}
  We thus have:
  \begin{align*}
    d_{S}\left(g,h\right) &= \ell_{G,S}\left(g^{-1}h\right)\\
                          &= \ell_{G,S}\left(h^{-1}g\right)\\
                          &= d_S\left(h,g\right)\\
                          \\
    d_{S}\left(g,h\right) &= \ell_{G,S}\left(g^{-1}h\right)\\
                          &= \ell_{G.S}\left(g^{-1}kk^{-1}h\right)\\
                          &\leq \ell_{G,S}\left(g^{-1}k\right) + \ell_{G,S}\left(k^{-1}h\right)\\
                          &= d_{S}\left(g,k\right) + d_{S}\left(k,h\right)\\
                          \\
    d_{S}\left(g,g\right) &= \ell_{G,S}\left(g^{-1}g\right)\\
                          &= \ell_{G,S}\left(e_G\right)\\
                          &= 0\\
    d_{S}\left(g,h\right) = 0 &\Leftrightarrow \ell_{G,S}\left(g^{-1}h\right) = 0\\
                              &\Leftrightarrow g^{-1}h = e_{G}\\
                              &\Leftrightarrow g = h.
  \end{align*}
  Thus, $d_S$ is indeed a metric.\newline

  Let $S$ and $T$ be finite symmetric generating sets for $G$. It is sufficient to show that there exists some $k\in \N$ such that, for all $g\in G$,
  \begin{align*}
    \frac{1}{k}\ell_{G,S}\left(g\right) \leq \ell_{G,T}\left(g\right) \leq k\ell_{G,S}\left(g\right).
  \end{align*}
  Set
  \begin{align*}
    M &= \max\set{\ell_{G,T}\left(s\right) | s\in S}\\
    N &= \max\set{\ell_{G,S}\left(t\right) | t\in T}.
  \end{align*}
  Now, let $n = \ell_{G,S}\left(g\right)$, such that $g = s_1\cdots s_n$, where $s_i\in S$. Then, we have
  \begin{align*}
    \ell_{G,T}\left(g\right) &= \ell_{G,T}\left(s_1\cdots s_n\right)\\
                             &\leq \ell_{G,T}\left(s_1\right) + \cdots + \ell_{G,T}\left(s_n\right)\\
                             &\leq M\ell_{G,S}\left(g\right),
  \end{align*}
  and similarly, $\ell_{G,S}\left(g\right) \leq N\ell_{G,T}\left(g\right)$. Setting $k = \max\left(M,N\right)$, we get
  \begin{align*}
    \frac{1}{k}\ell_{G,S}\left(g\right) \leq \ell_{G,T}\left(g\right) \leq k\ell_{G,S}\left(g\right).
  \end{align*}
\end{proof}
Now, we may begin defining the growth rate of a group. We will use the fact that all word metrics with respect to a generating set are symmetric in order to show that the growth rate is well-defined (i.e., independent of the generating set for $G$).
\begin{definition}
  Let $G$ be a group with finite symmetric generating set $S$. We define
  \begin{align*}
    B_{G,S}\left(n\right) &= \set{g\in G | \ell_{G,S}\left(g\right) \leq n};\\
    \gamma_{G,S}\left(n\right) &= \left\vert B_{G,S}\left(n\right) \right\vert.
  \end{align*}
\end{definition}
The following facts hold for $\gamma$.
\begin{fact}\label{fact:properties_of_gamma_generating_set}
  Let $G$ be a finitely generated group. The following facts hold:
  \begin{enumerate}[(1)]
    \item $\gamma_{G,S}\left(n\right)$ is an increasing function;
    \item $\gamma_{G,S}\left(n+m\right)\leq \gamma_{G,S}\left(n\right)\gamma_{G,S}\left(m\right)$;
    \item $\displaystyle \lim_{n\rightarrow\infty}\left(\gamma_{G,S}\left(n\right)\right)^{1/n} = \rho_{G,S}$ exists;
    \item if $S$ and $T$ are finite symmetric generating sets for $G$, then there exists $k\in \N$ such that $\gamma_{G,T}\left(n\right)\leq \gamma_{G,S}\left(kn\right)$ for all $n\in\N$, and $\rho_{G,S} = \rho_{G,T}$.
  \end{enumerate}
\end{fact}
\begin{proof}\hfill
  \begin{enumerate}[(1)]
    \item Since $B_{G,S}\left(n\right)\subseteq B_{G,S}\left(n+1\right)$, we have $\gamma_{G,S}\left(n\right) \leq \gamma_{G,S}\left(n+1\right)$, so $\gamma_{G,S}$ is increasing.
    \item We start by showing that $B_{G,S}\left(n\right)B_{G,S}\left(m\right) = B_{G,S}\left(n+m\right)$. First, if $g\in B_{G,S}\left(n\right)$ and $h\in B_{G,S}\left(m\right)$, we know that $\ell_{G,S}\left(gh\right) \leq \ell_{G,S}\left(g\right) + \ell_{G,S}\left(h\right)\leq m+n$, so $B_{G,S}\left(n\right)B_{G,S}\left(n\right) \subseteq B_{G,S}\left(n+m\right)$. Additionally, if $g\in B_{G,S}\left(n+m\right)$, we may write
      \begin{align*}
        g &= \underbrace{s_{1}\cdots s_{\ell}}_{g_1}\underbrace{s_{\ell+1}\cdots s_{k}}_{g_2},
      \end{align*}
      where $k\leq n+m$, $\ell \leq n$, and $k-\ell \leq m$, so $g_1\in B_{G,S}\left(n\right)$ and $g_2\in B_{G,S}\left(m\right)$. Thus, we have $B_{G,S}\left(n\right)B_{G,S}\left(m\right) = B_{G,S}\left(n+m\right)$.\newline

      Now, we have
      \begin{align*}
        \gamma_{G,S}\left(n+m\right) &= \left\vert B_{G,S}\left(n+m\right) \right\vert\\
                                     &= \left\vert B_{G,S}\left(n\right)B_{G,S}\left(m\right) \right\vert\\
                                     &\leq \left\vert B_{G,S}\left(n\right) \right\vert\left\vert B_{G,S}\left(m\right) \right\vert\\
                                     &= \gamma_{G,S}\left(n\right)\gamma_{G,S}\left(m\right).
      \end{align*}
    \item From (2), we know that $\gamma_{G,S}\left(n\right) \leq \gamma_{G,S}\left(1\right)^{n}$. Inductively, we have
      \begin{align*}
        \gamma_{G,S}\left(n+1\right) &\leq \gamma_{G,S}\left(1\right)^{n+1},
      \end{align*}
      and thus,
      \begin{align*}
        1 \leq \gamma_{G,S}\left(n\right)^{1/n}\leq \gamma_{G,S}\left(1\right).
      \end{align*}
    \item We know that there exists $k$ such that $\frac{1}{k}\ell_{G,S} \leq \ell_{G,T}\leq k\ell_{G,S}$ by the proof of Fact \ref{fact:word_metric_equivalent_metrics}. Thus, if $g\in B_{G,T}\left(n\right)$, then $\ell_{G,T}\left(g\right) \leq n$, so $\ell_{G,S}\left(g\right) \leq kn$, so $g\in B_{G,S}\left(kn\right)$ and $B_{G,T}\left(n\right)\subseteq B_{G,T}\left(kn\right)$. We have $\gamma_{G,T}\left(n\right)\leq \gamma_{G,S}\left(kn\right)$.\newline

      Similarly, if $g\in B_{G,S}\left(n\right)$, then $\ell_{G,S}\left(g\right)\leq n$, so $\ell_{G,T}\left(g\right) \leq kn$, and $g\in B_{G,T}\left(kn\right)$. Thus, we get $B_{G,S}\left(n\right)\subseteq B_{G,T}\left(kn\right)$, so $\gamma_{G,S}\left(n\right)\leq \gamma_{G,T}\left(kn\right)$.\newline

      It follows that
      \begin{align*}
        \gamma_{G,S}\left(\frac{n}{k}\right)^{1/n} \leq \gamma_{G,T}\left(n\right)^{1/n} \leq \left(\gamma_{G,S}\left(kn\right)^{k}\right)^{1/kn}.
      \end{align*}
      Sending $n\rightarrow\infty$, we get $\rho_{G,S}\leq \rho_{G,T}\leq \rho_{G,S}$, so $\rho_{G,S} = \rho_{G,T}$.
  \end{enumerate}
\end{proof}
\begin{definition}
  Let $G$ be a group with finite symmetric generating set $S$. The quantity
  \begin{align*}
    \rho_{G} &= \limsup_{n\rightarrow\infty}\gamma_{G,S}\left(n\right)^{1/n}
  \end{align*}
  is known as the growth rate of the group $G$. If we have $\rho = 1$, then we say $G$ is of subexponential growth.
\end{definition}
\begin{fact}\label{fact:finite_groups_subexponential_growth}
  All finite groups are of subexponential growth.
\end{fact}
\begin{proof}
Note that since $\rho$ is independent of the generating set (as we proved in Fact \ref{fact:properties_of_gamma_generating_set}), we can set $S = G$, and we have $\limsup_{n\rightarrow\infty} \left\vert G \right\vert^{1/n} = 1$.
\end{proof}
\begin{fact}\label{fact:finitely_generated_abelian_groups_subexponential_growth}
  Let $\Gamma$ be a finitely generated abelian group. Then, $\Gamma$ is of subexponential growth.
\end{fact}
\begin{proof}
  We start by showing that $G = \Z^d$ is of subexponential growth. Notice that every element of $\Z^d$ is some linear combination of the set
  \begin{align*}
    S &= \set{e_1,e_2,\dots,e_d},\label{eq:generating_set_free_abelian_group}\tag{\textasteriskcentered}
  \end{align*}
  where
  \begin{align*}
    e_{j} &= (0,0,\dots,\underbrace{1}_{\text{position $j$}},0,0,\dots).
  \end{align*}
  Additionally, we see that any element of $B_{G,S}(n)$ is of the form $e_1^{i_1}e_2^{i_2}\dots e_d^{i_d}$, where $\sum_{j=1}^{d} i_j \leq n$. Thus, we must have $\gamma_{G,S}(n) \leq n^{d}$, meaning that 
  \begin{align*}
    \rho &= \limsup_{n\rightarrow\infty} \gamma_{G,S}(n)^{1/n}\\
         &= \limsup_{n\rightarrow\infty}n^{d/n}\\
         &= 1,
  \end{align*}
  so $\Z^d$ is of subexponential growth.\newline

  Now, if $G' = \Z^d\times \Z/k_1\Z\times \cdots \times \Z/k_r\Z$, then since there is a torsion subgroup in $G'$, we must have $\gamma_{G',S'}(n) \leq \gamma_{\Z^{d+r},T}(n)$ for any $n$, where $T$ is a generating set for $\Z^{d+r}$ and $S'$ is a generating set for $G'$. Since
  \begin{align*}
    \rho_{\Z^{d+r}} &= \limsup_{n\rightarrow\infty}\gamma_{\Z^{d+r},T}(n)^{1/n}\\
                    &= 1,
  \end{align*}
  and $1 \leq \gamma_{G',S'}(n)$, we must have $\rho_{G'} = 1$.\newline

  Since, by the fundamental theorem of finitely generated abelian groups (Theorem \ref{thm:fundamental_thm_abelian_gps}), it is the case that $\Gamma\cong \Z^{d}\times \Z/k_1\Z\times\cdots\times \Z/k_r\Z$ for some $d,k_1,\dots,k_r\in \N$, $\Gamma$ is of subexponential growth.
\end{proof}
To prove that the groups of subexponential growth are amenable, we use the following lemma from real analysis.
\begin{lemma}
  Let $\left(a_n\right)_n$ be a sequence such that $a_n > 0$ for each $n$. Then,
  \begin{align*}
    \lim_{n\rightarrow\infty}\frac{a_{n+1}}{a_n} &= \lim_{n\rightarrow\infty} \left(a_n\right)^{1/n}.
  \end{align*}
  Similarly,
  \begin{align*}
    \limsup_{n\rightarrow\infty}\frac{a_{n+1}}{a_n} &= \limsup_{n\rightarrow\infty}\left(a_n\right)^{1/n}.
  \end{align*}
\end{lemma}
\begin{theorem}\label{thm:subexponential_growth_implies_amenable}
  Let $\Gamma$ be a finitely generated group of subexponential growth. Then, $\Gamma$ is amenable.
\end{theorem}
\begin{proof}
  To prove that $\Gamma$ is amenable, we show that it satisfies the Følner condition. From the results in Section \ref{sec:approximate_means}, we know that this implies that $\Gamma$ is amenable. Let $S$ be a finite symmetric generating set for $\Gamma$.\newline

  For any $\ve > 0$, we see that there is some $k\in \N$ such that
  \begin{align*}
    \left\vert B_{\Gamma,S}\left(k\right) \right\vert^{1/k} &\leq 1 + \ve.
  \end{align*}
  Thus, by the lemma above, we must have
  \begin{align*}
    \frac{\left\vert B_{\Gamma,S}\left(k+1\right) \right\vert}{\left\vert B_{\Gamma,S} \left(k\right)\right\vert} \leq 1 + \ve.
  \end{align*}
  Note that, by Lemma \ref{lemma:folner_condition_generating_set}, we only need to verify that the Følner condition holds on $S$. For any $s\in S$, we have
  \begin{align*}
    \frac{\left\vert sB_{\Gamma,S}\left(k\right)\triangle B_{\Gamma,S}\left(k\right) \right\vert}{\left\vert B_{\Gamma,S}(k) \right\vert} &\leq \frac{2\left(\left\vert B_{G,S}\left(k+1\right) \right\vert - \left\vert B_{\Gamma,S}(k) \right\vert\right)}{\left\vert B_{\Gamma,S}(k) \right\vert}\\
                                                                                                                    &\leq 2\ve.
  \end{align*}
  Therefore, $\Gamma$ satisfies the Følner condition, hence is amenable.
\end{proof}
\begin{remark}
  %The result in Theorem \ref{thm:subexponential_growth_implies_amenable} can be used along with Fact \ref{fact:finitely_generated_abelian_groups_subexponential_growth} and Corollary \ref{cor:direct_limit_amenable} to prove Corollary \ref{cor:abelian_groups_amenable}.
  An alternative way to show that abelian groups are amenable (Corollary \ref{cor:abelian_groups_amenable}) is by using the fact that the union of a directed system of amenable groups is amenable (Corollary \ref{cor:direct_limit_amenable}) and that finitely generated abelian groups are of subexponential growth (Fact \ref{fact:finitely_generated_abelian_groups_subexponential_growth}).
\end{remark}
\section{Remarks and Notes}%
In \cite[Appendix A.3]{juschenko_amenability}, it is shown that amenability through the Følner condition only need require a constant $C < 2$ such that, for all finite $S\subseteq \Gamma$, there exists a finite $F\subseteq \Gamma$ such that for all $s\in S$,
\begin{align*}
  \frac{\left\vert sF\triangle F \right\vert}{\left\vert F \right\vert} &\leq C.
\end{align*}
The existence of such a $C$ follows from the definition of the Følner condition (Definition \ref{def:folner_condition}) --- however, the opposite direction is a bit more involved, and makes use of some results from Chapter \ref{ch:left_regular_representation}, as well as some concepts from topology.\newline

A \textit{filter} on a set $X$ is a family of subsets $\mathcal{F}\subseteq P(X)$ that does not contain $\emptyset$ and is directed by containment --- that is, if $A$ and $B$ are in $\mathcal{F}$, then $A\cap B\in \mathcal{F}$ and if $A\subseteq B$, then $B\in \mathcal{F}$. An \textit{ultrafilter} is a maximal \textit{proper} filter --- if $\mathcal{U}$ is an ultrafilter, then for any $A\in P(X)$, either $A\in \mathcal{U}$ or $A^{c}\in \mathcal{U}$.\newline

In Appendix \ref{ch:point_set_topology}, we discussed nets --- however, (\cite[Theorem 2.25]{aliprantis_infinite_dimensional_analysis}) it is actually the case that every net generates an associated filter. Similar to the case of nets, we can talk about concepts like cluster points (Definition \ref{def:cluster_points}) and limits along filters. Similarly, limits may be taken along ultrafilters.\newline

The \textit{ultraproduct} of a family of Banach spaces, $\left( X_i \right)_{i\in I}$ is defined with respect to an ultrafilter $\mathcal{U}$ on $I$. Recall that the product of a family of Banach spaces is $\prod_{i\in I}X_i$, whose elements are $\left( x_i \right)_{i\in I}$ where $\sup_{i\in I}\norm{x_i} < \infty$. To obtain the ultraproduct, we define a subspace, $N = \set{\left( x_i \right)_{i\in I} | \lim_{\mathcal{U}}\norm{x_i} = 0}$ consisting of all ``effectively zero'' elements, and then take the quotient $\prod_{i\in I}X_i/N$ to obtain the ultraproduct. The ultraproduct is usually denoted $\prod_{i\in I}X_i/\mathcal{U}$.\newline

To prove that this weakened Følner condition implies amenability, it is first proven that a unitary representation $\pi\colon \Gamma\rightarrow \mathcal{U}\left( \mathcal{H} \right)$ admits an invariant vector under the condition that $C < 2$; then, the ultraproduct of the left-regular representation (Theorem \ref{thm:left_regular_representation}) with respect to an ultrafilter $\mathcal{U}$ on $\N$ is shown to admit an invariant vector, which implies the existence of an almost-invariant vector (Definition \ref{def:almost_invariant_vector}) for the left-regular representation. Then, this implies that the group $\Gamma$ is amenable (Theorem \ref{thm:almost_invariant_vector}).

% Testing names: Engineering a Mean: Følner's Condition and Approximate Means
\chapter{I've Looked at Groups from Both Sides Now: the Left-Regular Representation}\label{ch:left_regular_representation}
Just as God appears in many forms (or representations) throughout the Bible, such as the Burning Bush in the book of Exodus, so too are groups often dealt with through their representations. The field of representation theory, for instance, focuses on the properties of groups as subgroups of groups of linear transformations, and how the properties of these groups of linear transformations can provide insights into the properties of the groups themselves. Similarly, we will be engaging with the properties of groups through their representations as unitary operators acting on a Hilbert space. Soon enough, we will see that there are impressive insights about a group to be found through these unitary representations.
\section{Representing a Group}%
On a Hilbert space $\mathcal{H}$, we know that the set of unitary operators, $\mathcal{U}\left(\mathcal{H}\right)$, is a group under composition.\footnote{To see this, note that $I_{\mathcal{H}}$ is the identity element, that $U^{\ast}U =UU^{\ast} = I_{\mathcal{H}}$, and that unitarity is preserved under composition.} Given any other group $\Gamma$, it is then tempting to consider how we can ``model'' $\Gamma$ (so to speak) as a subgroup of $\mathcal{U}\left(\mathcal{H}\right)$. This is the essence behind the idea of a unitary representation.
\begin{definition}
  Let $\Gamma$ be a group. A unitary representation of $\Gamma$ is a pair, $\left(\pi,\mathcal{H}\right)$, where $\mathcal{H}$ is a Hilbert space and $\pi\colon \Gamma\rightarrow \mathcal{U}\left(\mathcal{H}\right)$ is a group homomorphism.\newline

  Furthermore, every unitary representation $U\colon \Gamma\rightarrow \mathcal{U}\left(\mathcal{H}\right)$, given by $s\mapsto U_s$, has the following properties:
  \begin{itemize}
    \item if $e$ is the identity element for $\Gamma$, then $U_{e} = I_{\mathcal{H}}$;
    \item for all $s\in \Gamma$, $U_{s}^{\ast} = U_{s^{-1}}$.
  \end{itemize}
\end{definition}
\begin{example}
  One excellent example of a unitary representation is the representation $1_{\Gamma}\colon \Gamma\rightarrow \C$, defined by $1_{\Gamma}(s) = 1$ for all $s\in\Gamma$. This is known as the trivial representation, and it will play an integral role in establishing amenability.\newline

  A more substantive unitary representation is the representation of the circle group, $\mathbb{T}\rightarrow \B\left(\ell_2\left(\Z\right)\right)$, given by $\omega \mapsto d_{\omega}$. Here, $d_{\omega}$ is the multiplication operator defined by
  \begin{align*}
    d_{\omega}\left(\left(a_{k}\right)_{k\in\Z}\right) &= \left(\omega^k a_{k}\right)_{k\in\Z}.
  \end{align*}
\end{example}

\chapter{Staying Positive: Amenability in \texorpdfstring{$C^{\ast}$-Algebras}{C*-Algebras}}\label{ch:nuclearity}
\section{Positive Maps in \texorpdfstring{$C^{\ast}$-Algebras}{C*-Algebras}}%

\section{Nuclearity and Amenability}%

\section{More Characterizations using $C^{\ast}$-Algebras}%


\chapter{Closing Remarks}
\epigraph{There is always something left undone...}{Paul Halmos, ``How to Write Mathematics''}
In \cite[48]{brown_and_ozawa}, the authors remark that ``amenable groups admit approximately $10^{10^{10}}$ characterizations.'' This paper was not long enough to discuss all of them, even when restricted to the case of discrete groups.\newline

In Section \ref{sec:subexponential_growth}, we discussed an application of the Følner condition to establishing amenability for a crucial class of groups in geometric group theory (the groups of subexponential growth) --- yet another direction in amenability concerns further study into geometric group theory, including (but not limited to) discussion of how amenability of a group relates to properties of its Cayley graph (see \cite[Section 3.2]{loh_geometric_group_theory} for more discussion on Cayley graphs). There is a notion of graph amenability related to the growth of a graph's neighboring vertex set that, it can be shown, is equivalent to the Følner condition in the case of Cayley graphs (see \cite{monfared_cayley_graphs}).\newline

There are some other directions in amenability that we might be able to take this text. There is a rich theory of amenability in locally compact groups, as well as amenability in Banach algebras and von Neumann algebras. Some of the authoritative texts on this subject include \cite{kazhdan_property_t} and \cite{amenable_banach_algebras} --- we have only touched the surface of what these texts have to offer in the discussion of amenability. Amenability in locally compact groups requires a much stronger command of abstract measure theory, especially concerning the Haar measure (which is a type of translation-invariant measure on the group), as well as integration theory with respect to abstract measures.\newline

In general, most of the results we discussed in this text that pertain to amenability in discrete groups can be translated with relative ease to the case of locally compact groups, primarily by replacing sums (which are really integrals with respect to the counting measure) with integrals with respect to the Haar measure. One analogous result, for instance, is that compact groups are amenable. However, there are also other criteria --- one may define a version of amenability for Banach algebras, and then show that a locally compact group $G$ is amenable if and only if the space $L_1\left( G \right)$ is amenable as a Banach algebra, where $L_1\left( G \right)$ is the space of integrable functions with respect to the Haar measure.\newline

Another direction in amenability concerns deeper discussion of random walks on groups, which was the primary topic of \cite{kesten_means} and \cite{kesten_random_walks}. When we discussed Kesten's criterion in the text (Theorem \ref{thm:kesten_criterion}) we only looked at the basic case where $M(S)$ was defined with respect to the symmetric generating set itself, rather than the general case of a finitely supported probability measure $\mu$ with $S\subseteq \supp(\mu)$.\newline

However, despite all of this, the main regret I have is that this text was ultimately a bit too scattered --- because there are so many different definitions for amenability, this thesis touched on so many topics that to provide enough background development for all the prerequisite ideas (coming from the ambitious assumption that anyone with a background in the standard third/fourth year real analysis courses would be able to glean most of the information in the text) would have resulted in a thesis that was even longer than this one already was. There were many points while I drafted the thesis that I had to go back and add particular bits of information that I didn't expect would be necessary, and there are certainly many tidbits and background highlights that are missing.\newline

Ultimately, this project --- understanding amenability, broadly construed --- will never be fully complete. There are tons of characterizations and nuances that appear as one goes deeper into the topic of amenability, but I like to believe I gave the topics discussed herein a fair shake, and certainly in a manner that is more than deep enough for most to find the text substantial rewarding.

% need snappy name
\appendix
\chapter{Algebra and Linear Algebra}\label{ch:algebra_and_linear_algebra}
In general, as we progress through these appendices, we will consistently add additional structure to a set. First, we begin by developing groups, rings, and fields, vector spaces, and algebras. In the following appendices, we will apply metric structures, topologies, and measures, building up to the central structure of functional analysis: normed vector spaces and the operators on these normed vector spaces.\newline

These appendices were largely written to provide essential background for the techniques and results that will appear in the main body of the text. As such, they do not include detailed proofs --- occasionally, we will include outlines for certain proofs in the remarks. The proofs for many of these results can be found in relevant canonical texts.\newline

We make heavy use of results from algebra and linear algebra in this thesis. Some excellent resources to learn more about algebra and linear algebra are \cite{dummit_and_foote} and \cite{algebra_chapter_0}.
\section{Group, Rings, (some) Fields}%
\begin{definition}[Groups]
  Let $A$ be a set, and let $\star$ be a binary operation on $A$. We say $A$ is a group if
  \begin{itemize}
    \item $A$ is closed under the operation $\star$;
    \item $A$ has an identity element $e_A$, where $a\star e_A = e_A\star a = a$ for any $a\in A$;
    \item for any $a\in A$, there exists $a^{-1}\in A$ such that $a^{-1}\star a = a\star a^{-1} = e$.
  \end{itemize}
  If the operation $\star$ is such that $a\star b = b\star a$ for all $a,b\in A$, then we say $A$ is an abelian group.\newline

  Generally, we abbreviate $a\star b \coloneq ab$.
\end{definition}
\begin{definition}[Subgroups, Normal Subgroups, and Quotient Groups]
  If $G$ is a group, $H\subseteq G$ is a subgroup if $H$ is closed under the group operation and inverses. We write $H\leq G$.\newline

  If $H$ is a subgroup, a left coset of $H$ is the set $gH \coloneq \set{gh | h\in H}$, where $g\in G$. Similarly, a right coset of $H$ is the set $Hg\coloneq \set{hg | h\in H}$. The index of $H$, denoted $\left[G:H\right]$, is the number of left (or right) cosets of $H$.\newline

  If $H\leq G$ is also such that, for any $g\in G$ and $h\in H$, $ghg^{-1}\in H$, then we call $H$ a normal subgroup of $G$. We write $H\trianglelefteq G$.\newline

  Defining the equivalence relation $g\sim g'$ if and only if $g^{-1}g'\in H$, the group of equivalence classes $gH\coloneq \left[g\right]$ is known as the quotient group $G/H$.
\end{definition}
\begin{definition}
  Let $G$ and $H$ be groups. A map $\varphi\colon G\rightarrow H$ is called a (group) homomorphism if, $\varphi$ preserves the group structure, in the sense that
  \begin{align*}
    \varphi\left(ab\right) &= \varphi(a)\varphi(b)\\
    \varphi\left(a^{-1}\right) &= \varphi(a)^{-1}
  \end{align*}
  for all $a,b\in G$.\newline

  We define $\ker\left(\varphi\right)$ to be the set of all $g\in G$ such that $\varphi\left(g\right) = e_H$.\newline

  If $H\trianglelefteq G$, the map $\pi\colon G\rightarrow G/H$ that sends $g\mapsto gH$ is known as the canonical projection.\newline

  If $\varphi$ is a bijection, then $\varphi$ is known as an isomorphism. We write $G\cong H$ if there exists an isomorphism $\varphi\colon G\rightarrow H$.
\end{definition}
\begin{theorem}[First Isomorphism Theorem for Groups]
  Let $G$ and $H$ be groups, and let $\varphi\colon G\rightarrow H$ be a group homomorphism. Then, $\ker\left(\varphi\right)\trianglelefteq G$ is a normal subgroup, and $G/\ker\left(\varphi\right)\cong \img\left(\varphi\right)$.
\end{theorem}
\begin{definition}[Rings]
  Let $A$ be a set. Specifically, let $A$ be an abelian group, letting $+$ denote the operation on $A$ and $0\coloneq e_A$. Then, $A$ is a ring if $A$ also admits a multiplication, $\cdot$, such that
  \begin{itemize}
    \item $a\cdot \left(b+c\right) = a\cdot b + a\cdot c$;
    \item $\left(a+b\right)\cdot c = a\cdot c + b\cdot c$;
    \item $\left(a\cdot b\right)\cdot c = a\cdot \left(b\cdot c\right)$.
  \end{itemize}
  If the multiplication on $A$ is commutative, then we say $A$ is a commutative ring. If $A$ admits an element $1_A$ such that $a\cdot 1_A = 1_A \cdot a = a$, then we say $A$ is a unital ring.
\end{definition}

% Here, will discuss group theory, results like FIT/normal subgroups and the fundamental theorem for finite/finitely generated abelian groups, ring theory (homomorphisms, ideals and maximal ideals, quotient rings, etc.), abstract linear algebra (vector spaces, bases, linear functionals), algebras (definitions, homomorphisms, unitizations)
\chapter{Point-Set Topology}\label{ch:point_set_topology}
We will need a bit of background in point-set topology in order to satisfactorily understand the functional analysis behind the results in Chapters 3, 4, and 5.
\section{Axioms of Set Theory}%
In order to garner sufficient understanding of point-set topology, we need to be able to comprehend some of the essential axioms behind the objects known as ``sets.'' This is where the axioms of set theory come into play.
\begin{definition}[Zermelo--Fraenkel Axioms]
  In Zermelo--Fraenkel set theory, all objects are sets. In order to maintain convention with the way the rest of this section will refer to sets, all sets will be referred to by capital letters, and all elements of sets by lowercase letters.
  \begin{itemize}
    \item Axiom of Existence: $\exists A\left(A = A\right)$. This axiom guarantees a nonempty universe.
    \item Axiom of Extensionality: $\forall x\left(x\in A \Leftrightarrow x\in B\right)\Rightarrow A = B$. This axiom states that if two sets share the same members, then the sets are equal.
    \item Axiom Schema of Comprehension: $\exists B\:\forall x\left(x\in B\Leftrightarrow x\in A \wedge\varphi(x)\right)$. This axiom states that for any formula $\varphi(x)$, where $x$ is a free variable, there is a set $B$ such that the members of $B$ are the members of $A$ for which $\varphi$ holds.
    \item Pairing Axiom: $\forall A\:\forall B\:\exists C\left(\left(A\in Z\right)\wedge \left(B\in Z\right)\right)$. This axiom states that for any sets $A$ and $B$, there is a set $C = \set{A,B}$ that contains the sets $A$ and $B$ as elements.
    \item Power Set Axiom: $\forall A\:\exists P(A)\:\forall B\left(B\in P(A) \Leftrightarrow B\subseteq A\right)$. We use the shorthand $B\subseteq A$ to mean $\forall x\left(x\in B\Rightarrow x\in A\right)$. This axiom states that for any set $A$ there exists a set $P(A)$ such that any element of $P(A)$ is a subset of $A$, and any subset of $A$ is an element of $P(A)$.
    \item Union Axiom: $\forall \mathcal{A}\:\exists A\:\forall Y\:\forall x\left(\left(x\in Y\wedge Y\in \mathcal{A}\right)\Rightarrow x\in A\right)$. This axiom states that for any collection $\mathcal{A}$, there is a set $A $, denoted $ \bigcup \mathcal{A}$, that contains all the elements of all the sets in the collection $\mathcal{A}$.
    \item Axiom of Infinity: $\exists A\left(\emptyset\in A\wedge \forall x\left(x\in A\Rightarrow x\cup\set{x}\in A\right)\right)$. This axiom states that there is a set, $A$, such that the empty set is in $A$ and, for any element $x$, if $x\in A$, then so too is the successor, $x\cup \set{x}$.
    \item Axiom of regularity: $\forall X\left(X\neq\emptyset \Rightarrow\exists Y\left(Y\in X\wedge Y\cap X = \emptyset\right)\right)$. This axiom states that any nonempty set $X$ contains a set $Y$ such that $Y$ and $X$ are disjoint. As a consequence, any chain of sets descending in membership must terminate.
    \item Axiom Schema of Replacement: $\forall A\:\exists B\:\forall v\left(v\in B\Rightarrow \exists u\left(u\in A\wedge \psi\left(u,v\right)\right)\right)$. The axiom schema of replacement says that for a function-like formula (a formula such that $\psi\left(u,v\right)\wedge \psi\left(u,w\right) \Rightarrow v=w$) $\psi\left(u,v\right)$, there is a set $A$ consisting of exactly those sets/elements $v\in B$ that correspond to $u\in A$.
  \end{itemize}
\end{definition}
The final axiom, the Axiom of Choice, is special, and as a result, we state it separately, for we will be using some of its consequences in the future sections. The following is one way of interpreting the axiom of choice.
\begin{definition}[Axiom of Choice]
  Let $\set{S_i}_{i\in I}$ be an indexed collection of nonempty sets. Then, there exists an indexed set $\set{x_i}_{i\in I}$ such that $x_i\in S_i$ for each $I$.\newline

  Equivalently, if $\set{S_i}_{i\in I}$ is an indexed collection of nonempty sets, then there is some choice function
  \begin{align*}
    f\in \prod_{i\in I}S_i.
  \end{align*}
\end{definition}
On its own, this formulation of the Axiom of Choice is not particularly useful. However, there is a statement of the Axiom of Choice which is just as useful.
\begin{definition}[Preorders, Partial Orders, Total Orders, and Well-Orders]
Let $X$ be a set, and $\preceq $ be a relation on $X$. We say a relation is a preorder if it is reflexive and transitive:
\begin{itemize}
  \item $a\preceq a$
  \item $a\preceq b \wedge b\leq c\Rightarrow a\preceq c$.
\end{itemize}
We say $X$ is a directed set if, for any $a,b\in X$, there is $c\in X$ such that $a\preceq c$ and $b\preceq c$.\newline

If $\preceq$ is also antisymmetric --- that is, $a\preceq b\wedge b\preceq a \Rightarrow a = b$ --- then, we say $\preceq$ is a partial order.\newline

We say $m\in X$ is a maximal element if, for any $x\in X$ with $m\preceq x$, $m = x$.\newline

If $X$ is partially ordered by $\preceq$ and, for any two elements $a,b\in X$, either $a\preceq b$ or $b\preceq a$, then we say $\preceq$ is a total order on $X$.\newline

If $X$ is a totally ordered set that has the property that, for any nonempty $A\subseteq X$, there is some $x\in A$ such that for any $y\in A$, $x\prec y$ for all $y \in A$ with $y\neq x$, then we say $\preceq$ is a well-order on $X$.
\end{definition}
\begin{example}
  \begin{itemize}
    \item The set $\N$ with the usual ordering is a well-ordered set.
    \item If $A$ is a set, then $P(A)$ with the containment ordering, $A\preceq B$ if $A\supseteq B$, is a partially ordered set.
    \item Similarly, if $A$ is a set, then $P(A)$ with the inclusion ordering, $A\preceq B$ if $A\subseteq B$, is a partially ordered set.
    \item A collection of functions $\set{\varphi_{i}: Z_i\rightarrow Y}_{i\in I}$ ordered by $\varphi_{i}\preceq \varphi_j$ if $Z_i\subseteq Z_j$ and $\varphi_{j}|_{Z_i} = \varphi_i$, is a partially ordered set. This is often known as the extension ordering.
  \end{itemize}
\end{example}

We can state an equivalent formulation of the Axiom of Choice as follows.
\begin{theorem}[Zorn's Lemma]
  If $\left(X,\preceq\right)$ is a partially ordered set with the property that for all $C\subseteq X$ with $C$ totally ordered, $C$ has an upper bound, then $X$ has a maximal element.
\end{theorem}
There are many proofs of both Zorn's Lemma from the Axiom of Choice and the Axiom of Choice from Zorn's Lemma. However, we will mostly be using it for the purposes of proving other theorems. The following results can be proven using Zorn's Lemma.
\begin{example}
  \begin{itemize}
    \item Every $\F$-vector space $V$ has a basis $B\subseteq V$ such that the set of all finite linear combinations of elements of $B$ over $\F$ is $V$.
    \item If $\varphi$ is a continuous linear functional defined on a subspace $W\subseteq V$, there is an extension $\Phi$ such that $\Phi|_{W} = \varphi$. %See: Hahn--Banach Theorems
    \item The arbitrary product of compact spaces is compact. %See Tychonoff's Theorem.
  \end{itemize}
\end{example}
\section{Metric Spaces}%
Building upon the basics of set theory, we move towards understanding metric spaces.
\subsection{Basics of Metric Spaces}%
\begin{definition}[Metrics]
  Let $X$ be a set. A distance metric is a function
  \begin{align*}
    d: X\times X\rightarrow [0,\infty)
  \end{align*}
  such that the following three properties are satisfied:
  \begin{itemize}
    \item if $x,y\in X$ and $d\left(x,y\right) = 0$, then $x = y$;
    \item $d\left(x,y\right) = d\left(y,x\right)$ for all $x,y\in X$;
    \item $d\left(x,z\right) \leq d\left(x,y\right) + d\left(y,z\right)$ for all $x,y,z\in X$.
  \end{itemize}
  A function that satisfies the latter two properties is called a semimetric.\newline

  Two metrics $d$ and $\rho$ on $X$ are equivalent if there exist constants $c_1,c_2\geq 0$ such that
  \begin{align*}
    d\left(x,y\right) &\leq c_1 \rho\left(x,y\right)\\
    \rho\left(x,y\right) &\leq c_2 d\left(x,y\right)
  \end{align*}
  for all $x,y\in X$.\newline

  A metric space is a pair $\left(X,d\right)$, where $d$ is a metric.
\end{definition}
\begin{example}[Some Distance Metrics]
  \begin{itemize}
    \item The discrete metric on any nonempty set is given by
      \begin{align*}
        d\left(x,y\right) & \begin{cases}
          1 & x\neq y\\
          0 & x = y
        \end{cases}
      \end{align*}
    \item The Euclidean metric between $\left(x_1,\dots,x_n\right)$ and $\left(y_1,\dots,y_n\right)$ in $\R^n$ is
      \begin{align*}
        d_{2}\left(x,y\right) &= \left(\sum_{j=1}^{n}\left\vert y_j-x_j \right\vert^2\right)^{1/2}.
      \end{align*}
    \item Other metrics on $\R^n$ include
      \begin{align*}
        d_1\left(x,y\right) &= \sum_{j=1}^{n}\left\vert y_j-x_j \right\vert\\
        d_{\infty}\left(x,y\right) &= \max_{j=1}^{n}\left\vert y_j - x_j \right\vert.
      \end{align*}
      All of $d_1,d_2,d_{\infty}$ are equivalent metrics.
    \item The Hamming distance between two strings of bits is
      \begin{align*}
        d_{H}: \set{0,1}^{n}\times \set{0,1}^{n}\rightarrow [0,\infty)\\
        d_{H}\left(\left(x_{j}\right)_{j=1}^{n},\left(y_j\right)_{j=1}^{n}\right) &= \left\vert \set{j\mid x_j\neq y_j} \right\vert.
      \end{align*}
    \item The set $C\left([0,1],\R\right)$ consisting of continuous real-valued functions from $[0,1]$ to $\R$ can be equipped with
      \begin{align*}
        d_u\left(f,g\right) &= \sup_{t\in [0,1]}\left\vert f(t) - g(t) \right\vert,
      \end{align*}
      which is the uniform metric, or
      \begin{align*}
        d_{1}\left(f,g\right) &= \int_{0}^{1} \left\vert f(t)-g(t) \right\vert\:dt.
      \end{align*}
    \item All subsets of a metric space $X$ equipped with the same metric is also a metric space.
    \item If $\rho$ is a metric on $X$, then we can create a distance metric
      \begin{align*}
        d\left(x,y\right) &= \frac{\rho\left(x,y\right)}{1 + \rho\left(x,y\right)}
      \end{align*}
      that is bounded on $[0,1]$.
    \item If $d_1,\dots,d_n$ are metrics on $X$ and $c_1,\dots,c_n > 0$ are constants, then
      \begin{align*}
        d\left(x,y\right) &= \sum_{k=1}^{n}c_kd_k\left(x,y\right)
      \end{align*}
      defines a metric on $X$.
    \item If $\left(\rho_k\right)_k$ is a family of separating semimetrics for $X$ --- i.e., for $x,y\in X$ distinct, there is some $\rho_{j}$ such that $\rho_j\left(x,y\right) \neq 0$ --- then, we can obtain bounded semimetrics by taking
      \begin{align*}
        d_k\left(x,y\right) &= \frac{\rho_k\left(x,y\right)}{1 + \rho_k\left(x,y\right)}
      \end{align*}
      for each $k$. From each $d_k$, we define
      \begin{align*}
        d\left(x,y\right) &= \sum_{k=1}^{n}2^{-k}d_k\left(x,y\right),
      \end{align*}
      which is a metric on $X$.
    \item If $\left(X_k,\rho_k\right)_{k\geq 1}$ is a sequence of metric spaces, then we can form the product space
      \begin{align*}
        X &= \prod_{k\geq 1}X_{k}
      \end{align*}
      with the metric
      \begin{align*}
        D\left(f,g\right) &= \sum_{k\geq 1}d_k\left(f(k),g(k)\right).
      \end{align*}
      Here, $d_k = \frac{\rho_k}{1 + \rho_k}$ is the corresponding bounded metric to $\rho_k$.
  \end{itemize}
\end{example}
\begin{definition}[Open and Closed Sets]
  Let $\left(X,d\right)$ be a metric space.
  \begin{enumerate}[(1)]
    \item For $x\in X$ and $\delta > 0$, we define
      \begin{enumerate}[(a)]
        \item the open ball at $x$ with radius $\delta > 0$
          \begin{align*}
            U\left(x,\delta\right) &= \set{y\in X\mid d\left(y,x\right) < \delta};
          \end{align*}
        \item the closed ball centered at $x$ with radius $\delta > 0$
          \begin{align*}
            B\left(x,\delta\right) &= \set{y\in X\mid d\left(y,x\right)\leq \delta};
          \end{align*}
        \item the sphere centered at $x$ with radius $\delta > 0$
          \begin{align*}
            S\left(x,\delta\right) &= \set{y\in X\mid d\left(y,x\right) = \delta}.
          \end{align*}
      \end{enumerate}
    \item A set $V\subseteq X$ is open if, for all $x\in V$, there is $\delta > 0$ such that $U\left(x,\delta\right)\subseteq V$.\newline

      A subset $C\subseteq X$ is closed if $C^{c}$ is open.
    \item If $x\in V$ and $V\subseteq X$ is open, then we say $V$ is an open neighborhood of $x$. A neighborhood of $x$ is any subset $N\subseteq X$ such that $N$ contains an open neighborhood of $x$.
    \item If $A\subseteq X$ is any subset, the interior of $A$ is
      \begin{align*}
        A^{\circ} &:= \bigcup\set{V\mid V\text{ is open, }V\subseteq A},
      \end{align*}
      the closure of $A$ is
      \begin{align*}
        \overline{A} &= \bigcap\set{C\mid C\text{ is closed, }A\subseteq C},
      \end{align*}
      and the boundary of $A$ is
      \begin{align*}
        \partial A &= \overline{A} \setminus A^{\circ}.
      \end{align*}
  \end{enumerate}
\end{definition}
We can now talk about the topology of the metric space.
\begin{fact}
  Let $\left(X,d\right)$ be a metric space, and let
  \begin{align*}
    \mathcal{U} = \set{V\mid V\subseteq X\text{ open}}.
  \end{align*}
  Then, the following are true.
  \begin{itemize}
    \item $\emptyset\in \mathcal{U},X\in \mathcal{U}$.
    \item If $\set{V_{i}}_{i\in I}$ is a family of open sets, then $\bigcup_{i\in I}V_i\in \mathcal{U}$.
    \item If $\set{V_i}_{i=1}^{n}$ is a finite collection of open sets, then $\bigcap_{i=1}^{n}V_i \in \mathcal{U}$.
  \end{itemize}
\end{fact}

\begin{definition}
  Let $\left(X,d\right)$ be a metric space. Suppose $A\subseteq X$ is a nonempty subset.
  \begin{enumerate}[(1)]
    \item The distance from a point $x\in X$ to the set $A$ is defined by
      \begin{align*}
        \dist_{A}\left(x\right) &= \inf_{a\in A}d\left(x,a\right).
      \end{align*}
    \item The diameter of $A$ is defined by
      \begin{align*}
        \diam\left(A\right) &= \sup_{x,y\in A}d\left(x,y\right).
      \end{align*}
    \item If $\diam(A) < \infty$, then we say $A$ is bounded.
    \item If, for every $\delta > 0$, there is a finite subset $F_{\delta}\subseteq X$ such that
      \begin{align*}
        A\subseteq \bigcup_{x\in F_{\delta}}U\left(x,\delta\right).
      \end{align*}
    \item For $A,B\subseteq X$, we define the Hausdorff distance between $A$ and $B$ to be
      \begin{align*}
        d_{H}\left(A,B\right) &= \max\set{\sup_{x\in A}\dist_{B}\left(x\right),\sup_{y\in B}\dist_{A}\left(y\right)}.
      \end{align*}
  \end{enumerate}
\end{definition}
\begin{example}
  Let $\Omega$ be a nonempty set, and $\left(X,d\right)$ be a metric space. A function $f: \Omega\rightarrow X$ is said to be bounded if $\diam\left(\ran(f)\right) < \infty$.\newline

  The collection $\operatorname{Bd}\left(\Omega,X\right)$ denotes all bounded functions with domain $\Omega$ and codomain $X$.\newline

  On $ \operatorname{Bd}\left(\Omega,X\right)$, we define the uniform metric by
  \begin{align*}
    D_{u}\left(f,g\right) &= \sup_{x\in\Omega}d\left(f(x),g(x)\right).
  \end{align*}
\end{example}
\subsection{Convergence and Continuity in Metric Spaces}%
\begin{definition}[Crash Course on Sequences]
  Let $\left(X,d\right)$ be a metric space.
  \begin{enumerate}[(1)]
    \item A sequence in $X$ is a map $x: \N\rightarrow X$, which we call $\left(x_{n}\right)_{n}$ or $\left(x_{n}\right)_{n\geq 1}$.
    \item A natural sequence is a strictly increasing sequence of natural numbers $\left(n_{k}\right)_{k\geq 1}$ with $n_{k}\geq k$ and $n_{k} < n_{k+1}$.
    \item If $\left(n_k\right)_{k}$ is a natural sequence, the sequence $\left(x_{n_k}\right)_{k}$ is called a subsequence of $\left(x_{n}\right)_n$.
    \item We say $\left(x_n\right)_n\rightarrow x$ if $d\left(x_n,x\right)_{n} \xrightarrow{n\rightarrow\infty} 0$. We say $x$ is the limit of $\left(x_n\right)_n$.
  \end{enumerate}
\end{definition}
\begin{example}[Convergence in Metric Spaces of Functions]
  \begin{itemize}
    \item If $\Omega$ is a nonempty set, and $\left(X,d\right)$ is a metric space, the sequence of functions $f_n: \Omega\rightarrow X$ is said to converge pointwise to $f: \Omega\rightarrow X$ if
      \begin{align*}
        f_n\left(x\right)\xrightarrow{n\rightarrow\infty}f(x)
      \end{align*}
      for each $x\in \Omega$.
    \item If $\left(f_n\right)_n\in \operatorname{Bd}\left(\Omega,X\right)$ is a sequence, we say $\left(f_n\right)_n\rightarrow f$ converges uniformly if
      \begin{align*}
        D_u\left(f_n,f\right)\xrightarrow{n\rightarrow\infty}0,
      \end{align*}
      or, equivalently,
      \begin{align*}
        \sup_{x\in\Omega}d\left(f_n(x),f(x)\right)\xrightarrow{n\rightarrow\infty}0.
      \end{align*}
  \end{itemize}
\end{example}
\begin{definition}[Sequential Criteria for Closure]
  If $\left(X,d\right)$ is a metric space, and $E\subseteq X$ is nonempty, then $E$ is closed if and only if, for all $\left(x_n\right)_n\rightarrow x$ with $x_n\in E$, $x\in E$.\newline

  If $E\subseteq X$ is any nonempty set, then $\overline{E}$ is precisely the set of all $x\in X$ such that $\left(x_n\right)_n\rightarrow x$ for some $\left(x_n\right)_n\subseteq E$.
\end{definition}
\begin{definition}[Completeness]
  Let $\left(X,d\right)$ be a metric space.
  \begin{itemize}
    \item If $\left(x_n\right)_n$ is a sequence in $X$ such that for all $\ve > 0$, there is $N\in \N$ such that for all $m,n\geq N$, $d\left(x_m,x_n\right) < \ve$, then we say the sequence is called Cauchy.
    \item If, for any $\left(x_n\right)_n$ Cauchy, $\left(x_n\right)_n\rightarrow x$ in $X$, then we say $X$ is complete.
    \item If $\left(X,d\right)$ is complete, then for any $A\subseteq X$ closed, $A$ is also complete.
    \item If $A\subseteq X$ is complete as a metric space, then $A$ is closed.
  \end{itemize}
\end{definition}
\begin{example}
  The metric space $\Q$ with the metric inherited from $\R$ is not complete. For instance, there is a sequence of rational numbers $\left(2,2.7,2.71,2.718,\dots\right)$ converging to $e$, but $e\notin \Q$.\newline

  The space $\operatorname{Bd}\left(\Omega,X\right)$ is complete if $X$ is complete.
\end{example}
\begin{definition}[Continuity]
  \begin{itemize}
    \item Let $\left(X,d\right)$ and $\left(Y,\rho\right)$ be metric spaces, and let $f: X\rightarrow Y$ be a function. We say $f$ is continuous at $x$ if, for every $\ve > 0$, there is $\delta > 0$ such that $z\in U\left(x,\delta\right)\Rightarrow \rho\left(f(x),f(z)\right) < \ve$.
    \item If $f$ is continuous at every point in $X$, then we say $f$ is continuous.
    \item If $f$ is bijective, continuous, and $f^{-1}$ is continuous, then we say $f$ is a homeomorphism.
    \item We say $f$ is uniformly continuous on $X$ if, for any $\ve > 0$, there is $\delta > 0$ such that for any $y,z\in X$, $d\left(y,z\right) < \delta \Rightarrow \rho\left(f(y),f(z)\right) < \ve$.
    \item We say $f$ is Lipschitz if there exists $C > 0$ such that $d\left(x,y\right) \leq Cd\left(f(x),f(y)\right)$ for all $x,y\in X$.
    \item We say $f$ is an isometry if $d\left(x,y\right) = d\left(f(x),f(y)\right)$ for all $x,y\in X$.
  \end{itemize}
\end{definition}
\begin{fact}
  Let $f: X\rightarrow Y$ be a map between metric spaces. The following are equivalent:
  \begin{enumerate}[(i)]
    \item $f$ is continuous;
    \item if $V\subseteq Y$ is open, then $f^{-1}\left(V\right)\subseteq X$ is open;
    \item if $\left(x_n\right)_n\rightarrow x$ in $X$, then $\left(f\left(x_n\right)\right)_n\rightarrow f(x)$ in $Y$.
  \end{enumerate}
\end{fact}
\begin{fact}
  If $M$ and $N$ are metric spaces with $N$ complete, and $A\subseteq M$ is dense, then if $f: A\rightarrow N$ is uniformly continuous, then there is a unique uniformly continuous map $\tilde{f}: M\rightarrow N$.
\end{fact}


\chapter{Measure Theory and Integration}\label{ch:measure_theory}
In order to properly discuss amenability, we need a strong foundation in measure theory.

\section{Constructing Measurable Spaces}%
Fix a set $\Omega$. We let $\mathcal{A} =\set{A_{i}}_{i\in I}$ be a collection of subsets of $\Omega$.
\begin{definition}[Algebra of Subsets]
  The collection $\mathcal{A} = \set{A_i}_{i\in I}$ is known as an algebra of subsets for $\Omega$ if
  \begin{itemize}
    \item $\emptyset,\Omega \in \mathcal{A}$;
    \item for any $A_i\in \mathcal{A}$, $A_i^{c}\in \mathcal{A}$;
    \item for any $A_i,A_j\in \mathcal{A}$, $A_i \cup A_j \in \mathcal{A}$.
  \end{itemize}
\end{definition}
We can refine the concept of an algebra of subsets to consider countable unions rather than finite unions. This is known as a $\sigma$-algebra.
\begin{definition}[$\sigma$-Algebra of Subsets]
  The collection $\mathcal{A} = \set{A_i}_{i\in I}$ is known as a $\sigma$-algebra of subsets for $\Omega$ if
  \begin{itemize}
    \item $\emptyset,\Omega \in \mathcal{A}$;
    \item for any $A_i\in \mathcal{A}$, $A_i^{c}\in \mathcal{A}$;
    \item for any countable collection $\set{A_n}_{n\geq 1}\subseteq \mathcal{A}$, $\bigcup_{n\geq 1}A_{n} \in \mathcal{A}$.
  \end{itemize}
\end{definition}
\begin{definition}[Measurable Space]
A pair $\left(\Omega,\mathcal{A}\right)$, where $\Omega$ is a set and $A\subseteq P(\Omega)$ is a $\sigma$-algebra, is called a measurable space. Elements in the measurable space are called $\mathcal{A}$-measurable sets.
\end{definition}
\begin{definition}[Restriction of a $\sigma$-Algebra]
  For a measurable space $\left(\Omega,\mathcal{A}\right)$, with $E\in \mathcal{A}$, the family
  \begin{align*}
    \mathcal{A}_{E} &= \set{E\cap A\mid A\in \mathcal{A}}
  \end{align*}
  is a $\sigma$-algebra on $E$, known as the restriction of $\mathcal{A}$ to $E$.
\end{definition}
\begin{definition}[Produced $\sigma$-Algebra]
Let $\left(\Omega,\mathcal{A}\right)$ be a measurable space, and $f: \Omega\rightarrow \Lambda$ is a map. The $\sigma$-algebra produced by $f$ on $\Lambda$ is the collection
\begin{align*}
  \mathcal{N} &= \set{E\mid E\subseteq \Lambda,~f^{-1}(E) \in \mathcal{A}}.
\end{align*}

\end{definition}

\begin{definition}[Generated $\sigma$-Algebra]
  For a family $\mathcal{E}\subseteq P\left(\Omega\right)$, the $\sigma$-algebra generated by $E$ is the smallest $\sigma$-algebra that contains $E$.
  \begin{align*}
    \sigma\left(\mathcal{E}\right) &= \bigcap_{\substack{\mathcal{E}\in \mathcal{M}_i \\ \text{$\mathcal{M}_i$ $\sigma$-Algebra}}} \mathcal{M}_i
  \end{align*}
\end{definition}
\begin{definition}[Borel $\sigma$-Algebra]
  If $\Omega$ is a topological space with topology $\tau\subseteq P(\Omega)$, we define
  \begin{align*}
    \mathcal{B}_{\Omega} &= \sigma\left(\tau\right)
  \end{align*}
  to be the Borel $\sigma$-algebra.
\end{definition}
All open, closed, clopen, $F_{\sigma}$, and $G_{\delta}$ subsets of $\Omega$ are Borel.\break

We can now begin examining measurable functions.
\begin{definition}[Measurable Functions]
  Let $\left(\Omega,\mathcal{M}\right)$ and $\left(\Lambda,\mathcal{N}\right)$ be measurable spaces.
  \begin{enumerate}[(1)]
    \item We say a map $f: \Omega\rightarrow \Lambda$ is $\mathcal{M}$-$\mathcal{N}$-measurable if $f^{-1}\left(E\right)\in \mathcal{M}$ for all $E\in \mathcal{N}$.
    \item We say a map $f: \Omega\rightarrow \R$ is measurable if it is $\mathcal{M}$-$\mathcal{B}_{\R}$-measurable.
    \item We say a map $f: \Omega\rightarrow \C$ is measurable if both $\re(f)$ and $\im(f)$ are measurable.
  \end{enumerate}
  The set of all measurable functions on $\left(\Omega,\mathcal{M}\right)$ is denoted $L_{0}\left(\Omega,\mathcal{M}\right)$.\newline

  The collection of all bounded measurable functions is the set
  \begin{align*}
    B_{\infty}\left(\Omega,\mathcal{M}\right) &= \set{f\in L_0\left(\Omega,\mathcal{M}\right)\mid \sup_{x\in\Omega}\left\vert f(x) \right\vert < \infty}.
  \end{align*}
\end{definition}

\begin{example}
  If $f: \Omega\rightarrow \Lambda$ is a continuous map between topological spaces, then $f$ is $\mathcal{B}_{\Omega}$-$\mathcal{B}_{\Lambda}$-measurable, since
  \begin{align*}
    \mathcal{F} &= \set{E\subseteq \Lambda\mid f^{-1}\left(E\right)\in \mathcal{B}_{\Omega}}
  \end{align*}
  is a $\sigma$-algebra containing every open set in $\Lambda$, so $\mathcal{F}$ contains $\mathcal{B}_{\Lambda}$.
\end{example}

\begin{example}
  If $\left(\Omega,\mathcal{M}\right)$ is a measurable space, and $f: \Omega\rightarrow \Lambda$ is a map, the measurable space $\left(\Lambda,\mathcal{N}\right)$ produced by $f$ is necessarily measurable.
\end{example}

\begin{fact}\label{fact:composition}
  If $\left(\Omega,\mathcal{M}\right)$, $\left(\Lambda,\mathcal{N}\right)$, and $\left(\Sigma,\mathcal{L}\right)$ are measurable spaces, with $f: \Omega\rightarrow \Lambda$ and $g: \Lambda\rightarrow \Sigma$ measurable, then $g\circ f$ is measurable.
\end{fact}
\begin{proof}[Proof of Fact \ref{fact:composition}]
  If $E\in \mathcal{L}$, then $g^{-1}\left(E\right) \in \mathcal{N}$, so $f^{-1}\left(g^{-1}\left(E\right)\right)\in \mathcal{M}$. Thus, $\left(g\circ f\right)^{-1}\left(E\right)\in \mathcal{M}$, so $g\circ f$ is measurable.
\end{proof}

\begin{proposition}
  Let $\left(\Omega,\mathcal{M}\right)$ be a measurable space. Let $\F = \C$ or $\R$. Suppose $f,g,h_n: \Omega\rightarrow \F$ are measurable for $n\geq 1$.
  \begin{enumerate}[(1)]
    \item If $\alpha \in \F$, then $f + \alpha g$ is measurable.
    \item $\overline{f}$ is measurable.
    \item $fg$ is measurable.
    \item $\frac{f}{g}$ is measurable assuming it is well-defined.
    \item if $h_n$ are $\R$-valued, and $\left(h_n\left(x\right)\right)_n$ is bounded for each $x\in \Omega$, then $\sup h_n$ and $\inf h_n$ are measurable.
    \item If $f$ and $g$ are $\R$ valued, then $\max\left(f,g\right)$ and $\min\left(f,g\right)$ are measurable. In particular,
      \begin{align*}
        f_{+} &= \max\left(f,0\right)\\
        f_{-} &= \max\left(0,-f\right)
      \end{align*}
      are measurable.
    \item $\left\vert f \right\vert$ is measurable.
    \item The pointwise limit of measurable functions is measurable --- if $\lim_{n\rightarrow\infty}h_n\left(x\right)$ exists for all $x\in \Omega$, then $h = \lim_{n\rightarrow\infty}h_n$ is measurable.
  \end{enumerate}
\end{proposition}
\begin{definition}[Simple Functions]
  A simple function $s: \Omega\rightarrow \F$ is a function with finite range. In other words, $s$ is of the form
  \begin{align*}
    s &= \sum_{k=1}^{n}c_k\1_{E_k}
  \end{align*}
  for $E_k\subseteq \Omega$ and $c_k\in \F$.\newline

  A simple function is measurable if and only if $E_k\in \mathcal{M}$ for each $k$.
\end{definition}

\section{Constructing Measures}%
A measure assigns a nonnegative ``length'' or ``volume'' to measurable sets.
\begin{definition}[Basics of Measures]
  A measure on a measurable space $\left(\Omega,\mathcal{M}\right)$ is a map $\mu: \mathcal{M}\rightarrow \left[0,\infty\right]$ that satisfies the following.
  \begin{enumerate}[(i)]
    \item $\mu\left(\emptyset\right) = 0$;
    \item $\displaystyle \mu\left(\bigsqcup_{j=1}^{\infty}E_j\right) = \sum_{j=1}^{\infty}\mu\left(E_j\right)$.
  \end{enumerate}
  The triple $\left(\Omega,\mathcal{M},\mu\right)$ is called a measure space.\newline

  A measure $\mu$ is finite if $\mu\left(\Omega\right) < \infty$\newline

  If $\mu\left(\Omega\right) = 1$, then $\mu$ is called a probability measure.\newline

  A measure $\mu$ is called finitely additive if $\mu\left(E\sqcup F\right) = \mu(E) + \mu(F)$.\newline

  A measure $\mu$ is called $\sigma$-finite if there is a countable family $\set{E_n}_{n\geq 1}\subseteq \mathcal{M}$ such that
  \begin{align*}
    \Omega &= \bigcup_{n\geq 1}E_n
  \end{align*}
  and $\mu\left(E_n\right) < \infty$.\newline

  A measure $\mu$ on $\left(\Omega,\mathcal{M}\right)$ is called semi-finite if, for every $E\in \mathcal{M}$ with $\mu(E) = \infty$, there exists $F\in \mathcal{M}$ with $F\subseteq E$ and $0 < \mu(F) < \infty$.
\end{definition}

\begin{lemma}
  Let $\left(\Omega,\mathcal{M},\mu\right)$ be a measure space.
  \begin{enumerate}[(1)]
    \item If $E,F\in \mathcal{M}$ with $F\subseteq E$, then $\mu\left(F\right) \subseteq \mu\left(E\right)$.
    \item If $\left(E_n\right)_{n}$ is a sequence of measurable sets, then
      \begin{align*}
        \mu\left(\bigcup_{n\geq 1}E_n\right) &\leq \sum_{n=1}^{\infty}\mu\left(E_n\right).
      \end{align*}
    \item If $\left(E_n\right)_{n\geq 1}$ is an increasing family of measurable sets, then
      \begin{align*}
        \mu\left(\bigcup_{n\geq 1}E_n\right) &= \lim_{n\rightarrow\infty}\mu\left(E_n\right).
      \end{align*}
  \end{enumerate}
\end{lemma}
\begin{proof}\hfill
  \begin{enumerate}[(1)]
    \item Since $F\subseteq E$, we can write $E = F\sqcup \left(E\setminus F\right)$. Thus,
      \begin{align*}
        \mu\left(E\right) &= \mu\left(F\right) + \mu\left(E\setminus F\right)\\
                          &\geq \mu\left(F\right).
      \end{align*}
    \item We write
      \begin{align*}
        F_1 &= E_1\\
        F_2 &= E_2\setminus E_1\\
            &\vdots\\
        F_n &= E_n\setminus \left(\bigcup_{k=1}^{n-1}E_k\right).
      \end{align*}
      Since each $F_n$ is measurable, and $F_n\subseteq E_n$, we have
      \begin{align*}
        \mu\left(\bigcup_{n\geq 1}E_n\right) &= \mu\left(\bigsqcup_{n\geq 1}F_n\right)\\
                                             &= \sum_{n=1}^{\infty}F_n\\
                                             &\leq \sum_{n=1}^{\infty}\mu\left(E_n\right).
      \end{align*}
    \item We write $F_n$ as the respective disjoint union for $\set{E_n}_{n\geq 1}$. We have $\bigsqcup_{k=1}^{n}F_k = E_n$. Then,
      \begin{align*}
        \mu\left(\bigcup_{n\geq 1}E_n\right) &= \sum_{n=1}^{\infty}\mu\left(F_n\right)\\
                                             &= \lim_{n\rightarrow\infty}\left(\sum_{k=1}^{n}\mu\left(F_k\right)\right)\\
                                             &= \lim_{n\rightarrow\infty}\mu\left(\bigsqcup_{k=1}^{n}F_k\right)\\
                                             &= \lim_{n\rightarrow\infty}\mu\left(E_n\right).
      \end{align*}
  \end{enumerate}
\end{proof}
\begin{definition}[Counting Measure]
  If $\Omega$ is any set, the \textbf{counting measure} on $\left(\Omega,P\left(\Omega\right)\right)$ assigns $\left\vert A \right\vert$ for each $A\in P\left(\Omega\right)$ finite, and $\infty$ for any infinite subset.
\end{definition}
\begin{definition}[Restricting Measures]
  If $\left(\Omega,\mathcal{M},\mu\right)$ is a measure space, $\mathcal{B}$ is a $\sigma$-algebra on $\Omega$ with $\mathcal{B}\subseteq \mathcal{M}$, the restriction $\mu|_{\mathcal{B}}$ is a measure on $\left(\Omega,\mathcal{B}\right)$.\newline

  If $E\in \mathcal{M}$, we can restrict $\mu$ to $\mathcal{M}_{E}$ (the restriction of $\mathcal{M}$ to $E$), yielding the measure space $\left(E,\mathcal{M}_{E},\mu|_{\mathcal{M}_E}\right)$. We denote this restricted measure $\mu_{E}$, such that $\mu_{E}\left(M\cap E\right) = \mu\left(M\cap E\right)$ for all $M\in \mathcal{M}_{E}$.
\end{definition}
\begin{definition}[Pushforward Measure]
  Let $\left(\Omega,\mathcal{M},\mu\right)$ be a measure space, and let $\left(\Lambda,\mathcal{N}\right)$ be a measurable space. Let $f: \Omega\rightarrow \Lambda$ be measurable. The map
  \begin{align*}
    f_{\ast}\mu: \mathcal{N}\rightarrow [0,\infty]
  \end{align*}
  defined by
  \begin{align*}
    f_{\ast}\mu\left(E\right) &= \mu\left(f^{-1}\left(E\right)\right)
  \end{align*}
  defines a measure on $\left(\Lambda,\mathcal{N}\right)$. This is known as the pushforward measure of $\mu$.\newline

  If $\mathcal{N}$ on $\Lambda$ is produced by $f$, then the pushforward measure is necessarily defined on $\mathcal{N}$, and that any function $g: \Lambda\rightarrow \F$ is measurable if and only if $g\circ f$ is measurable.
\end{definition}
\begin{definition}[Disjoint Union]
  Let $\set{\left(\Omega_n,\mathcal{M}_n,\mu_n\right)}$ be a countable family of measure spaces.\newline

  We define the co-product of this family by taking
  \begin{align*}
    \Sigma := \bigsqcup_{n=1}^{\infty}\Omega_n,
  \end{align*}
  to be our set equipped with the canonical inclusion map $\iota_{n}\left(x\right) = \left(x,n\right)$, such that for each $n$,
  \begin{align*}
    \mathcal{M} &:= \set{E\subseteq \Sigma\mid \iota_{n}^{-1}\left(E\right)\in \mathcal{M}_n}.
  \end{align*}
  The measure is defined by
  \begin{align*}
    \mu: \mathcal{M}\rightarrow [0,\infty]\\
    \mu(E) := \sum_{n=1}^{\infty}\mu_{n}\left(\iota_{n}^{-1}\left(E\right)\right).
  \end{align*}
  We can identify each $\Omega_n$ with the subset $\Omega_{n}^{\ast} = \set{\left(x,n\right)\mid x\in \Omega_n}\subseteq \Sigma$, with $\iota_{n}^{-1}\left(E\right)\subseteq \Omega_{n}$ identified with $E\cap \Omega_{n}^{\ast}$.\newline

  The family $\set{\Omega_{n}^{\ast}}_{n\geq 1}$ forms a measurable partition of $\Sigma$, and that $\mu|_{\Omega_{n}^{\ast}}$ are the pushforwards of $\mu_{n}$ by $\iota_{n}$.\newline

  Note that a map $f: \Sigma\rightarrow \C$ is measurable if and only if $f\circ \iota_{n}: \Omega_n\rightarrow \C$ is measurable for all $n$.\newline

  If $\left(f_n: \Omega_n\rightarrow \C\right)_{n}$ is a sequence of measurable maps, the disjoint union
  \begin{align*}
    f = \bigsqcup_{n\geq 1}f_n: \Sigma\rightarrow \C
  \end{align*}
  defined by $f\left(x,n\right) = f_n(x)$, is measurable.
\end{definition}

% I can trim down this section quite a lot
\chapter{Functional Analysis}\label{ch:functional_analysis}
\documentclass[10pt]{mypackage}

% sans serif font:
%\usepackage{cmbright}
%\usepackage{sfmath}
%\usepackage{bbold} %better blackboard bold

%serif font + different blackboard bold for serif font
\usepackage{newpxtext,eulerpx}
\renewcommand*{\mathbb}[1]{\varmathbb{#1}}
\renewcommand*{\hbar}{\hslash}

\pagestyle{fancy} %better headers
\fancyhf{}
\rhead{Avinash Iyer}
\lhead{A Foray into Functional Analysis}

\setcounter{secnumdepth}{0}

\begin{document}
\RaggedRight
\tableofcontents
\section{Introduction}%
This is going to be part of the notes for my Honors thesis independent study, which will be focused on amenability and $C^{\ast}$-algebras. This section of notes will be focused on the essential results in functional analysis, starting from normed vector spaces, working our way up through $C^{\ast}$-algebras.\newline

The primary source for this section is going to be Timothy Rainone's \textit{Functional Analysis-En Route to Operator Algebras}, which has not been published yet.\newline

I do not claim any of this work to be original.
\section{Normed Vector Spaces}%
\subsection{Vector Spaces, Norms, and Basic Properties}%
All vector spaces are defined over $\C$. Most of the information here is in my Real Analysis II notes, so I'm going to skip to some of the more important content.
\begin{definition}[Vector Space]
  A vector space $V$ is a set closed under two operations
  \begin{align*}
    a: V\times V \rightarrow V,~\left(v_1,v_2\right)\mapsto v_1 + v_2\\
    m: \C\times V\rightarrow V,~\left(\lambda,v\right) \mapsto \lambda v.
  \end{align*}
  We refer to $a$ as addition, and $m$ as scalar multiplication; $(V,+)$ is an abelian ring.
\end{definition}
\begin{definition}[Norm]
  A norm is a function
  \begin{align*}
    \norm{\cdot}: V \rightarrow \R^+,~x\mapsto \norm{x}
  \end{align*}
  that satisfies the following properties:
  \begin{itemize}
    \item Positive definiteness: $\norm{v} = 0$ if and only if $v = 0_V$.
    \item Triangle inequality: $\norm{v+w} \leq \norm{v} + \norm{w}$.
    \item Absolute Homogeneity: $\norm{\lambda v} = \left\vert \lambda \right\vert\norm{v}$, for $\lambda \in \C$.
  \end{itemize}
  If a function $p: V\rightarrow \R^+$ satisfies the triangle inequality and absolute homogeneity, we say $p$ is a seminorm.
\end{definition}
We say the pair $\left(V,\norm{\cdot}\right)$ is a normed vector space.
\begin{definition}[Balls and Spheres]
  Let $X$ be a normed vector space, $x\in X$, and $\delta > 0$. Then,
  \begin{align*}
    U(x,\delta) &= \set{y\in X\mid d(x,y) < \delta}\\
    B(x,\delta) &= \set{y\in X\mid d(x,y) \leq \delta}\\
    S(x,\delta) &= \set{y\in X\mid d(x,y) = \delta}.
  \end{align*}
  For a normed vector space, we will use the following conventions for common sets:
  \begin{align*}
    U_X &= U(0,1)\\
    B_X &= B(0,1)\\
    S_X &= S(0,1)\\
    \mathbb{D} &= U_{\C}\\
    \mathbb{T} &= S_{\C}.
  \end{align*}
\end{definition}
\begin{definition}[Equivalent Norms]
  Two norms on $V$, $\norm{\cdot}_{a}$ and $\norm{\cdot}_{b}$ are said to be equivalent if there are two constants $C_1$ and $C_2$ such that
  \begin{align*}
    \norm{v}_{a} &\leq C_1\norm{v}_b\\
    \norm{v}_{b} &\leq C_2\norm{v}_a
  \end{align*}
  for all $v\in V$. We say $\norm{\cdot}_{a}\sim \norm{\cdot}_{b}$.
\end{definition}
\subsection{Examples}%
\begin{example}[Finite-Dimensional Vector Spaces]
  The vector space $\C^n$ is with the $p$-norm is denoted $\ell_{p}^{n}$, where for $p \in [1,\infty]$, the $p$-norm is defined by
  \begin{align*}
    \norm{x}_{p} &= \left(\sum_{i=1}^{n}\left\vert x_i \right\vert^p\right)^{1/p}.
  \end{align*}
  In the case with $p=2$, this gives the traditional Euclidean norm, and with $p = \infty$, this gives the $\sup$ norm:
  \begin{align*}
    \norm{x}_{\infty} &= \max_{1\leq i \leq n}\left\vert x_i \right\vert.
  \end{align*}
\end{example}
\begin{example}[A Sequence Space]
  We let $\ell_{p} = \set{\left(x_{n}\right)_n\mid x_n\in \C,\norm{x}_p < \infty}$ be the collection of sequences in $\C$ with finite $p$-norm. Here,
  \begin{align*}
    \norm{x}_p &= \left(\sum_{n=1}^{\infty}\left\vert x_n \right\vert^p\right)^{1/p}.
  \end{align*}
  In the case with $p = \infty$, this gives the sequence space $\ell_{\infty}$, which has norm
  \begin{align*}
    \norm{x}_{\infty} &= \sup_{n\in \N}\left\vert x_n \right\vert.
  \end{align*}
\end{example}
\begin{example}[A Function Space]
  We let $\ell^{\infty}\left(\Omega\right)$ denote the set of all bounded functions $f: \Omega \rightarrow \C$, equipped with the norm
  \begin{align*}
    \norm{f}_{\infty} &= \sup_{x\in \Omega}\left\vert f(x) \right\vert.
  \end{align*}
  If $\Omega = \left(\Omega,\mathcal{M},\mu\right)$ is a measure space, then we let $L^{\infty}\left(\Omega\right)$ be the space of $\mu$-a.e. equal essentially bounded measurable functions, under the norm
  \begin{align*}
    \norm{f}_{\infty} &= \esssup_{x\in \Omega}\left\vert f(x) \right\vert.
  \end{align*}
\end{example}
\subsection{Series Convergence and Completeness}%
\begin{proposition}[Criteria for Banach Spaces]
Let $X$ be a normed vector space. The following are equivalent:
\begin{enumerate}[(i)]
  \item $X$ is a Banach space.\footnote{Complete normed vector space.}
  \item If $\left(x_k\right)_k$ is a sequence of vectors such that $\sum_{k=1}^{\infty}\norm{x_k}$ converges, then $\sum_{k=1}^{\infty}x_k$ converges.
  \item If $\left(x_k\right)_k$ is a sequence in $X$ such that $\norm{x_k} < 2^{-k}$, then $\sum_{k=1}^{\infty}x_k$ converges.
\end{enumerate}
\end{proposition}
\begin{proof}
  To show (i) implies (ii), for $n > m > N$, we have
  \begin{align*}
    \norm{s_n - s_m} &= \norm{\sum_{k=m+1}^{n}x_k}\\
                     &\leq \sum_{k=m+1}^{n}\norm{x_k}\\
                     &< \epsilon,
  \end{align*}
  implying that $s_n$ is Cauchy, and thus converges since $X$ is complete.\newline

  Since $\sum_{k=1}^{\infty}2^{-k}$ converges, it is clear that (ii) implies (iii).\newline

  To show (iii) implies (i), we let $\left(x_n\right)_n$ be a Cauchy sequence in $X$. We only need construct a convergent subsequence in order to show that $\left(x_n\right)_n$ converges.\newline

  Chose $n_1\in \N$ such that for $n,m\geq n_1$, $ \norm{x_m - x_n} < \frac{1}{2^2}$, and inductively define $n_j > n_{j-1}$ such that $n,m\geq n_j$ implies $\norm{x_m - x_n} < \frac{1}{2^{j+1}}$.\newline

  Let $y_1 = x_{n_1}$, $y_{j} = x_{n_j} - x_{n_{j-1}}$. Then,
  \begin{align*}
    \norm{y_j} &= \norm{x_{n_j} - x_{n_{j-1}}}\\
               &< \frac{1}{2^{j}},
  \end{align*}
  so $\sum_{j=1}^{\infty}y_j$ converges by our assumption. By telescoping, we see that $\sum_{j=1}^{k}y_j = x_{n_k}$, so $\left(x_{n_{k}}\right)_k$ converges.
\end{proof}
\subsection{Quotient Spaces}%
Let $X$ be a normed vector space. Then, for $E\subseteq X$ a subspace, there is a quotient space $X/E$ with the projection map $\pi: X\rightarrow X/E$, $x\mapsto x + E$. We want to make $X/E$ into a normed space --- in order to do this, we use the distance function:
\begin{align*}
  \dist_{E}(x) &= \inf_{y\in E}d(x,y),
\end{align*}
which is uniformly continuous. For $E$ closed, then $\dist_{E}(x) = 0$ if and only if $x\in E$.
\begin{proposition}[Quotient Space Norm]
  Let $X$ be a normed vector space, and $E\subseteq X$ a subspace. Set
  \begin{align*}
    \norm{x + E}_{X/E} &= \dist_{E}(x).
  \end{align*}
  Then,
  \begin{enumerate}[(1)]
    \item $\norm{\cdot}_{X/E}$ is a well-defined seminorm on $X/E$.
    \item If $E$ is closed, then $\norm{\cdot}_{X/E}$ is a norm on $X/E$.
    \item $\norm{x+E}_{X/E} \leq \norm{x}$ for all $x\in X$.
    \item If $E$ is closed, then $\pi: X\rightarrow X/E$ is Lipschitz.
    \item If $X$ is a Banach space and $E$ is closed, then $X/E$ is also a Banach space.
  \end{enumerate}
\end{proposition}
\begin{proof}\hfill
  \begin{enumerate}[(1)]
    \item We will show that $\norm{\cdot}_{X/E}$ is well-defined. If $x + E = x' + E$, $x'-x\in E$, so for every $y\in E$, $x'-x + y\in E$. Thus,
    \begin{align*}
      \norm{x-y} &= \norm{x'-\left(x'-x+y\right)}\\
                 &\geq \inf_{z\in E}\norm{x' - z}\\
                 &= \norm{x' + E}_{X/E}.
    \end{align*}
    Thus, $\norm{x + E}_{X/E} \geq \norm{x' + E}_{X/E}$, and vice versa.\newline

    Let $\lambda \in \C\setminus \set{0}$, and $x\in X$. Then,
    \begin{align*}
      \norm{\lambda\left(x + E\right)}_{X/E} &= \norm{\lambda x + E}_{X/E}\\
                                             &= \inf_{y\in E}\norm{\lambda x - y}\\
                                             &= |\lambda|\inf_{y\in E}\norm{x - \lambda^{-1}y}\\
                                             &= |\lambda|\inf_{y'\in E}\norm{x-y}\\
                                             &= |\lambda|\norm{x + E}_{X/E}
    \end{align*}
    Given $x,x'\in X$ and a fixed $\ve > 0$, we have
    \begin{align*}
      \norm{x + E} + \frac{\ve}{2} &> \norm{x-y}
    \end{align*}
    for some $y\in E$, and
    \begin{align*}
      \norm{x' + E} + \frac{\ve}{2} &> \norm{x'-y'}
    \end{align*}
    for some $y'\in E$. Thus,
    \begin{align*}
      \norm{\left(x+x'\right)-\left(y+y'\right)} &\leq \norm{x-y} + \norm{x' - y'}\\
                                                 &< \ve + \norm{x + E} + \norm{x' + E}.
    \end{align*}
    Since $y + y'\in E$, we have
    \begin{align*}
      \norm{\left(x+E\right) + \left(x' + E\right)}_{X/E} &= \norm{x + x' + E}_{X/E}\\
                                                    &\leq \norm{\left(x+x'\right) - \left(y+y'\right)}\\
                                                    &< \ve + \norm{x + E}_{X/E} + \norm{x' + E}_{X/E},
    \end{align*}
    meaning
    \begin{align*}
      \norm{\left(x+E\right) + \left(x' + E\right)} \leq \norm{x + E} + \norm{x' + E}.
    \end{align*}
  \item If $E$ is closed, and $\norm{x + E} = 0$, then $x\in E$ so $x + E = 0_{X/E}$.
  \item For $x\in X$,
    \begin{align*}
      \norm{x + E}_{X/E} &= \inf_{y\in E}\norm{x-y}\\
                         &\leq \norm{x}.
    \end{align*}
  \item We have
    \begin{align*}
      \norm{\left(x+E\right) - \left(x' + E\right)}_{X/E} &= \norm{x-x' + E}_{X/E}\\
                                                          &\leq \norm{x-x'}.
    \end{align*}
  \item Let $X$ be complete and $E\subseteq X$ be closed. Let $\left(x_k + E\right)_k$ be a sequence in $X/E$ with $\norm{x_k + E} < 2^{-k}$. We want to show that $\sum_{k=1}^{\infty}\left(x_k + E\right)$ converges.\newline

    For each $k$, since $\norm{x_k + E} < 2^{-k}$, there exists $y_k\in E$ such that $\norm{x_k - y_k} < 2^{-k}$. Since $X$ is complete, $\sum_{k=1}^{\infty}x_k - y_k$ converges.\newline

    Let $\left(\sum_{k=1}^{n}x_k - y_k\right)_n \rightarrow x$ in $X$. Applying the canonical projection map, $\pi$, to both sides, we get
    \begin{align*}
      \sum_{k=1}^{n}\left(x_k + E\right) &= \sum_{k=1}^{n}\pi\left(x_k\right)\\
                                         &= \pi\left(\sum_{k=1}^{n}\left(x_k - y_k\right)\right)\\
                                         &\rightarrow \pi(x),
    \end{align*}
    implying that $\sum_{k=1}^{\infty}\left(x_k + E\right)$ converges.
  \end{enumerate}
\end{proof}
\begin{exercise}
  Consider $\ell_{\infty}$ and its closed subspace $c_0$. If $\pi: \ell_{\infty}\rightarrow \ell_{\infty}/c_0$ denotes the canonical quotient map, with $\left(z_k\right)_k\in \ell_{\infty}$, show that
  \begin{align*}
    \norm{\left(z_k\right)_k + c_0} &= \limsup_{k\rightarrow\infty}\left\vert z_k \right\vert
  \end{align*}
\end{exercise}
\begin{solution}
  By the definition of the quotient norm, we have
  \begin{align*}
    \norm{\left(z_k\right)_k + c_0}_{\ell_{\infty}/c_0} &= \inf_{\left(a_k\right)_k\in c_0}\norm{\left(z_k\right)_k - \left(a_k\right)_k}\\
                                                        &= \inf_{\left(a_k\right)_k\in c_0}\sup_{k\in \N}\left\vert z_k - a_k \right\vert\\
                                                        &= \limsup_{k\rightarrow\infty}\left\vert z_k \right\vert.
  \end{align*}
\end{solution}
\subsection{Bounded Linear Operators}%
\begin{definition}[Continuous Functions]
  A function $f: \left(X, d_X\right)\rightarrow \left(Y,d_Y\right)$ is called Lipschitz if there is a constant $C>0$ such that
  \begin{align*}
    d_Y\left(f(x),f(x')\right) \leq Cd_x\left(x,x'\right)
  \end{align*}
  for all $x,x'\in X$.\newline

  If $C \leq 1$, a Lipschitz map is known as a contraction.\newline

  If
  \begin{align*}
    d_Y\left(f(x),f\left(x'\right)\right) = d_X\left(x,x'\right)
  \end{align*}
  for all $x,x'\in X$, then $f$ is known as an isometry.
\end{definition}
\begin{proposition}[Categorization of Continuous Linear Maps]
  Let $X$ and $Y$ be normed vector spaces, and let $T: X\rightarrow Y$ be a linear map. The following are equivalent:
  \begin{enumerate}[(i)]
    \item $T$ is continuous at $0$.
    \item $T$ is continuous.
    \item $T$ is uniformly continuous.
    \item $T$ is Lipschitz.
    \item There exists a constant $C > 0$ such that $\norm{T(x)}\leq C\norm{x}$ for all $x\in X$.
  \end{enumerate}
\end{proposition}
\begin{definition}[Bounded Linear Operator]
  Let $X$ and $Y$ be normed vector spaces, and let $T: X\rightarrow Y$ be a linear map.
  \begin{enumerate}[(1)]
    \item $T$ is bounded if $T\left(B_X\right)$ is bounded in $Y$. Equivalently, $T$ is bounded if and only if
      \begin{align*}
        \sup_{x\in B_X}\norm{T(x)} < \infty,
      \end{align*}
      or that $\exists r > 0$ such that $T\left(B_X\right) \subseteq B_Y\left(0,r\right)$.
    \item The operator norm of $T$ is the value
      \begin{align*}
        \norm{T}_{\text{op}} &= \sup_{x\in B_X}\norm{T(x)}.
      \end{align*}
  \end{enumerate}
\end{definition}
\begin{lemma}
  Let $T: X\rightarrow Y$ be a linear map between normed vector spaces. Then,
  \begin{align*}
    \norm{T}_{\text{op}} &= \sup_{x\in S_X}\norm{T(x)}
    \intertext{and for all $x\in X$,}
    \norm{T(x)} \leq \norm{T}_{\text{op}}\norm{x}.
  \end{align*}
\end{lemma}
\begin{lemma}
  Let $T: X\rightarrow Y$ be a bounded linear map between normed vector spaces. Then, for any $x\in X$ and $r > 0$,
  \begin{align*}
    r\norm{T}_{\text{op}}\leq \sup_{y\in B\left(x,r\right)}\norm{T(y)}
  \end{align*}
\end{lemma}
\begin{proof}
  Let $C = \sup_{y\in B\left(x,r\right)}\norm{T(y)}$. If $z\in B\left(0,r\right)$, then $z+x,z-x\in B(x,r)$, meaning
  \begin{align*}
    2T\left(z\right) &= T\left(z+x\right) + T\left(z-x\right),
  \end{align*}
  so by the triangle inequality, we get
  \begin{align*}
    2\norm{T(z)} &\leq \norm{T(z+x)} + \norm{T(z-x)}\\
                 &\leq 2\max\set{\norm{T(z+x)},\norm{T\left(z-x\right)}}\\
                 &\leq 2C.
  \end{align*}
  Thus,
  \begin{align*}
    \norm{T(z)} \leq \sup_{y\in B\left(x,r\right)}\norm{T(y)},
  \end{align*}
  meaning
  \begin{align*}
    r\norm{T}_{\text{op}} \leq \sup_{y\in B\left(x,r\right)}\norm{T(y)}.
  \end{align*}
\end{proof}
\begin{remark}
For a linear map $T: X\rightarrow Y$, the following are equivalent:
\begin{enumerate}[(1)]
  \item $T$ is continuous.
  \item $T$ is bounded.
  \item $\norm{T}_{\text{op}} < \infty$.
\end{enumerate}
\end{remark}
\begin{definition}
  Let $X$ and $Y$ be normed spaces, $T: X\rightarrow Y$ a linear map.
  \begin{enumerate}[(1)]
    \item $T$ is bounded below if there exists $C_2$ such that $\norm{T(x)}\geq C_2\norm{x}$ for all $x\in X$.
    \item $T$ is bicontinuous if $T$ is bounded and bounded below.
      \begin{align*}
        C_2\norm{x} \leq \norm{T(x)}\leq C_1\norm{x}
      \end{align*}
    \item $T$ is a bicontinuous isomorphism if $T$ is bijective, linear, and bicontinuous. We say $X$ and $Y$ are bicontinuously isomorphic.
    \item We say $T$ is an isometric isomorphism if $T$ is bijective, linear, and an isometry.
  \end{enumerate}
\end{definition}
\begin{example}
  Let $\rho$ be the continuous surjective wrapping function $\rho: [0,2\pi]\rightarrow \mathbb{T}$, $\rho(t) = e^{it}$. There is an induced isometry
  \begin{align*}
    T_{\rho}: C\left(\mathbb{T}\right) \rightarrow C\left([0,2\pi]\right),
  \end{align*}
  defined by $T_{\rho}\left(f\right)(t) = f\circ \rho\left(t\right) = f\left(e^{it}\right)$.\newline

  The range of $T_{\rho}$ is $C= \set{G\in C\left([0,2\pi]\right)\mid g(0) = g(2\pi)}$, which means that $C\left(\mathbb{T}\right) $ and $ C$ are isometrically isomorphic Banach spaces.
\end{example}
\begin{proposition}
  Let $X$ and $Y$ be normed spaces, and $T:X\rightarrow Y$ be a linear map. The following are equivalent.
  \begin{enumerate}[(i)]
    \item $T$ is bicontinuous.
    \item $T: X\rightarrow \ran(T)$ is a linear isomorphism and homeomorphism.
  \end{enumerate}
\end{proposition}
\begin{proof}
  Let $T$ be bicontinuous. Then, $T$ is linear, injective, and surjective onto $\ran(T)$. Since $T$ is continuous, $T$ is bounded. Let $S: \ran(T) \rightarrow X$ be defined by $S\left(T(x)\right) = x$. We can see that $S$ is well-defined, since $T: X\rightarrow \ran(T)$ is surjective, and so has a left inverse. Similarly, since $\norm{S(T(x))} = \norm{x} \leq \frac{1}{C_2}\norm{T(x)}$, $S$ is continuous.\newline

  Let $S: \ran(T) \rightarrow X$ be defined by $S(T(x)) = x$. Since $T$ is continuous, it is bounded, so
  \begin{align*}
    \norm{T(x)}\leq \norm{T}_{\text{op}}\norm{x}.
  \end{align*}
  Since $S$ is bounded,
  \begin{align*}
    \norm{x} &= \norm{S(T(x))}\\
             &= \norm{S}_{\text{op}}\norm{T(x)},
  \end{align*}
  so $\frac{1}{\norm{S}_{\text{op}}}\norm{x} \leq \norm{T(x)}$.
\end{proof}
\begin{corollary}
  Let $X$ be a vector space with $\norm{\cdot}$ and $\norm{\cdot}'$ two norms. The following are equivalent:
  \begin{enumerate}[(i)]
    \item The norms $\norm{\cdot}$ and $\norm{\cdot}'$ are equivalent.
    \item The map $\id_{X}:\left(X,\norm{\cdot}\right)\rightarrow \left(X,\norm{\cdot}'\right)$.
  \end{enumerate}
\end{corollary}
\begin{proposition}[Properties of Bounded Linear Operators]
  Let $X,Y,Z$ be normed spaces, $T: X\rightarrow Y$, $S: X\rightarrow Y$, and $R:Y\rightarrow Z$ be linear maps.
  \begin{enumerate}[(1)]
    \item $\norm{\alpha T}_{\text{op}}= \left\vert \alpha \right\vert\norm{T}_{\text{op}}$
    \item $\norm{T + S}_{\text{op}}\leq \norm{T}_{\text{op}} + \norm{S}_{\text{op}}$
    \item $\norm{T}_{\text{op}}  = 0$ if and only if $T = 0$
    \item $\norm{R\circ T}_{\text{op}}\leq \norm{R}_{\text{op}}\norm{T}_{\text{op}}$
    \item $\norm{\id_{X}}_{\text{op}} = 1$
    \item If $E\subseteq X$ is a subspace, then $\norm{T|_{E}}_{\text{op}}\leq \norm{T}_{\text{op}}$
  \end{enumerate}
\end{proposition}
\begin{proof}
  We will prove (4) here. For $x\in B_{X}$, we have
  \begin{align*}
    \norm{R\circ T(x)} &= \norm{R\left(T(x)\right)}\\
                       &\leq \norm{R}_{\text{op}}\norm{T(x)}\\
                       &\leq \norm{R}_{\text{op}}\norm{T}_{\text{op}}.
  \end{align*}
  Taking the supremum, we obtain $\norm{R\circ T}_{\text{op}}\leq \norm{R}_{\text{op}}\norm{T}_{\text{op}}$.
\end{proof}
\begin{recall}
  $\mathcal{L}(X,Y)$ is the set of all linear operators with domain $X$ and codomain $Y$.
\end{recall}
\begin{proposition}
  Let $X$ and $Y$ be normed spaces.
  \begin{enumerate}[(1)]
    \item The collection $\mathcal{B}(X,Y) = \set{T\in \mathcal{L}\left(X,Y\right)\mid \norm{T}_{\text{op}} < \infty}$ equipped with the operator norm is a normed space known as the space of bounded linear operators between $X$ and $Y$.
    \item If $Y$ is a Banach space, then $\mathcal{B}\left(X,Y\right)$ is a Banach space.
    \item The continuous dual space, $X^{\ast} = \mathcal{B}\left(X,\C\right)$ is a Banach space.
  \end{enumerate}
\end{proposition}
\begin{proof}
  We will prove (2). Let $\left(T_n\right)_n$ be Cauchy under $\norm{\cdot}_{\text{op}}$. Since Cauchy sequences are bounded, there is some $C > 0$ such that $\norm{T_n}_{\text{op}}\leq C$ for all $n\geq 1$. For $x\in X$,
  \begin{align*}
    \norm{T_n(x) - T_m(x)} \leq \norm{T_n - T_m}_{\text{op}}\norm{x},
  \end{align*}
  meaning $\left(T_n(x)\right)_{n}$ is Cauchy in $Y$. Since $Y$ is complete, we define
  \begin{align*}
    T(x) &= \lim_{n\rightarrow\infty}T_n(x)
  \end{align*}
  in $Y$. If $x\in B_X$, we have
  \begin{align*}
    \norm{T(x)} &= \norm{\lim_{n\rightarrow\infty}T_n(x)}\\
                &= \lim_{n\rightarrow\infty}\norm{T_n(x)}\\
                &\leq \limsup_{n\rightarrow\infty}\norm{T_n(x)}\\
                &\leq C\norm{x},
  \end{align*}
  meaning $\norm{T}_{\text{op}} \leq C$.\newline

  Let $\ve > 0$, and $N\in \N$ large such that $n,m\geq N$, $\norm{T_n - T_m}_{\text{op}} \leq \ve$. For $x\in B_X$,
  \begin{align*}
    \norm{T_n(x)-T(x)} &= \lim_{m\rightarrow\infty}\norm{T_n(x)-T_m(x)}\\
                      &\leq \limsup_{m\rightarrow\infty}\norm{T_n - T_m}_{\text{op}}\norm{x}\\
                      &< \ve.
  \end{align*}
  Thus, $\norm{T - T_n}_{\text{op}} < \ve$ for all $n\geq N$.
\end{proof}
\begin{definition}[Algebras]
  Let $A$ be an algebra over $\C$.
  \begin{enumerate}[(1)]
    \item If $A$ admits a norm $\norm{\cdot}$ satisfying $\norm{ab} \leq \norm{a}\norm{b}$, then $A$ is a normed algebra. If $A$ is unital, then $\norm{1_A} = 1$.
    \item If $A$ is complete with respect to its norm, then $A$ is called a Banach algebra, and if $A$ is unital, then $A$ is a unital Banach algebra.
  \end{enumerate}
\end{definition}
\begin{lemma}
  In a normed algebra $A$, the map $\cdot: A\times A \rightarrow A,(a,b)\mapsto ab$ is continuous.
\end{lemma}
\begin{proposition}
  Let $X$ be a normed space. The set of bounded operators $\mathcal{B}\left(X,X\right) = \mathcal{B}(X)$ is a unital normed algebra. Moreover, if $X$ is a Banach space, then $\mathcal{B}\left(X\right)$ is a Banach algebra.
\end{proposition}
\begin{proposition}
  Let $A$ be a unital Banach algebra, $a\in A$. The series
  \begin{align*}
    \exp(a) &= \sum_{n=0}^{\infty}\frac{a^n}{n!}
  \end{align*}
  converges absolutely in $A$. We call $\exp(a) $ the exponential of $a$.
  \begin{enumerate}[(1)]
    \item $\exp(0) = 1_A$
    \item If $A$ is commutative, then $\exp(a+b) = \exp(a)\exp(b)$.
    \item We have $\exp(a)\in \text{GL}(A)$ with $\exp(a)^{-1} = \exp(-a)$.
    \item $\norm{\exp(a)}\leq \exp\left(\norm{a}\right)$.
  \end{enumerate}
\end{proposition}
\subsection{Quotient Maps}%
\begin{definition}
  A map $f: X\rightarrow Y$ is called open if $U\subseteq X$ is open implies $f(U)\subseteq Y$ is open.
\end{definition}
\begin{proposition}
  Let $X$ and $Y$ be normed spaces, $T: X\rightarrow Y$ a linear map. The following are equivalent:
  \begin{enumerate}[(i)]
    \item $T$ is surjective and open.
    \item $T\left(U_X\right)\subseteq Y$ is open.
    \item There exists $\delta > 0$ such that $\delta U_Y \subseteq T\left(U_X\right)$.
    \item There exists $\delta$ such that $\delta B_Y \subseteq T\left(B_X\right)$.
    \item There exists $M > 0$ such that for all $y\in Y$, there exists $x\in X$ with $T(x) = y$ and $\norm{x} \leq M\norm{y}$.
  \end{enumerate}
\end{proposition}
\begin{proof}\hfill
  To see (i) implies (ii), if $T$ is surjective and open, then it is clear that $T\left(U_X\right)$, which is the image of an open set, is open.\newline

  To see (ii) implies (iii), if $T\left(U_X\right)$ is open, we have $0_Y\in T\left(U_X\right)$, so there is some $\delta$ such that $U\left(0,\delta\right) \subseteq T\left(U_X\right)$, meaning $\delta U_{Y} \subseteq T\left(U_X\right)$.\newline

  Assuming (iii), we see that $\frac{\delta}{2}B_Y \subseteq \delta U_Y \subseteq T\left(U_X\right)\subseteq T\left(B_X\right)$.\newline

  To see (iv) implies (v), let $\delta$ be such that $\delta B_Y\subseteq T\left(B_X\right)$, and set $M = \frac{1}{\delta}$. Note that for $y\in Y,y\neq 0$, $\frac{\delta}{\norm{y}}y\in \delta B_Y$, meaning $\frac{\delta}{\norm{y}} y = T(x)$ for some $x\in B_X$, implying that $T\left(\frac{\norm{y}}{\delta}x\right) = y$. Finally, since $x\in B_X$, $\frac{\norm{y}}{\delta}\norm{x} \leq \frac{1}{\delta}\norm{y} = M\norm{y}$.\newline

  To see (v) implies (i), we can see that $T$ is surjective by the assumption. Let $U\subseteq X$ be open, $y_0\in T(U)$. Then, there exists $x_0$ such that $T\left(x_0\right) = y_0$, and $\delta > 0$ such that $U\left(x_0,\delta\right)\subseteq U$. Note that $U\left(x_0,\delta\right) = x_0 + \delta U_X$, so $x_0 + \delta U_X \subseteq U$. Applying $T$, we get $T\left(x_0 + \delta U_X\right)\subseteq T(U)$, or $y_0 + \delta T\left(U_X\right)\subseteq T(U)$. By assumption, since given $y\in U_Y$, there exists $x\in X$ such that $\norm{x} \leq M\norm{y}$, meaning $\norm{x}\leq M$, we have $U_Y\subseteq T\left(MU_X\right)$. Thus, $\frac{1}{M}U_Y\subseteq T\left(U_X\right)$, meaning $y_0 + \frac{\delta}{M}U_Y\subseteq y_0\delta T\left(U_X\right)\subseteq T(U)$, so $U_Y\left(y_0,\frac{\delta}{M}\right)\subseteq T(U)$.
\end{proof}
\begin{definition}
  Let $X$ and $Y$ be normed vector spaces.
  \begin{enumerate}[(1)]
    \item A bounded linear map $T: X\rightarrow Y$ that is surjective and open is known as a quotient map.
    \item If $T\left(U_X\right) = U_Y$, then $T$ is called a $1$-quotient map.
  \end{enumerate}
\end{definition}
\begin{exercise}
  If $T\left(B_X\right) = B_Y$, show that $T\left(U_X\right) = U_Y$.
\end{exercise}
\begin{solution}
  Since $T\left(B_X\right) = B_Y$, it is the case that $\left(T\left(B_X\right)\right)^{\circ} = B_Y^{\circ}$. Since $T$ is an open map, $T$ is continuous, meaning $\left(T\left(B_X\right)\right)^{\circ} = T\left(B_X^{\circ}\right)$. Thus, $T\left(U_X\right) = U_Y$.
\end{solution}
\begin{proposition}
  Let $X$ and $Y$ be normed vector spaces with $T: X\rightarrow Y$ a quotient map. If $X$ is a Banach space, then $Y$ is a Banach space.
\end{proposition}
\begin{proof}
  We will show that $Y$ is complete by showing that an absolutely convergent series converges.\newline

  Let $\left(y_k\right)_k$ be a sequence in $Y$ with $\sum_{k=1}^{\infty}\norm{y_k} < \infty$. Since $T$ is a quotient map, there is a universal $M > 0$ such that for all $k$, there is $x_k\in X$ such that $T\left(x_k\right) = y_k$ and $\norm{x_k} \leq M\norm{y_k}$. Thus,
  \begin{align*}
    \sum_{k=1}^{\infty} &\leq M\sum_{k=1}^{\infty}\norm{y_k}\\
    &< \infty.
  \end{align*}
  Since $X$ is complete, $\sum_{k=1}^{\infty}x_k$ converges. Let $\sum_{k=1}^{\infty}x_k = x$. Then, $\left(T\left(\sum_{k=1}^{n}x_k\right)\right)_n\xrightarrow{n\rightarrow\infty}T(x) $, meaning $\sum_{k=1}^{\infty}y_k = T(x)$. Thus, $\sum_{k=1}^{\infty}y_k$ converges in $Y$, so $Y$ is a Banach space.
\end{proof}
\begin{proposition}
  Let $X$ be a normed vector space, $E\subseteq X$ a closed subspace. The canonical quotient map, $\pi: X\rightarrow X/E$ is a $1$-quotient map.
\end{proposition}
\begin{proof}
  We know that $\norm{\pi\left(x\right)} \leq \norm{x}$, meaning $\pi\left(U_X\right)\subseteq U_{X/E}$.\newline

  Let $\pi(x) = x+E \subseteq U_{X/E}$. Then, $\inf_{y\in E}\norm{x-y}\leq 1$, meaning there exists some $y$ such that $\norm{x-y} < 1$, meaning $\pi\left(x-y\right) = \pi(x)$.
\end{proof}
\begin{corollary}
  If $X$ is a Banach space, $E\subseteq X$ a closed subspace, then $X/E$ is a Banach space.
\end{corollary}
\begin{corollary}
  Let $X$ be a normed vector space and $E\subseteq X$ be closed. If two of $X,E,X/E$ are complete, the third is also complete.
\end{corollary}
\begin{proof}
  We have shown that if $X$ is complete, then $E$ is necessarily complete (since $E$ is closed) and $X/E$ is complete as shown above.\newline

  Let $E$ and $X/E$ be complete. We now want to show that $X$ is complete. Let $\left(x_k\right)_k$ be Cauchy in $X$.\newline

  For each $k$, let $x_k = s_k + y_k$, where $y_k\in E$ and $s_k + E = \pi\left(x_k\right)$. Notice that, since $x_k$ is Cauchy, so too is $s_k$, as $\norm{s_k} \leq \norm{x_k}$ for all $k$. Additionally, for $m,n \geq N$, we have
  \begin{align*}
    \norm{x_{m} - x_{n}} &= \norm{s_m + y_m - \left(s_n + y_n\right)}\\
                         &\leq \norm{s_m - s_n} + \norm{y_m - y_n}\\
                         &< \ve,
  \end{align*}
  implying that $\left(y_k\right)_k$ is Cauchy in $E$. Since $X/E$ and $E$ are complete, we define $x = \lim_{k\rightarrow\infty}s_k + \lim_{k\rightarrow\infty}y_k$. Finally, for $m,n\geq N$, we have
  \begin{align*}
    \norm{x - x_n} &= \lim_{m\rightarrow\infty}\norm{x_m - x_n}\\
                   &\leq \ve,
  \end{align*}
  meaning $\left(x_k\right)_k \xrightarrow{k\rightarrow\infty} x$, so $X$ is complete.
\end{proof}
\begin{proposition}
  Let $X$ and $Y$ be normed spaces, $E\subseteq X$ a closed subspace, and $T: X\rightarrow Y$ bounded linear with $E\subseteq \ker(T)$. Then, there exists a unique bounded linear map $\overline{T}: X/E\rightarrow Y$ such that $\overline{T}\circ \pi = T$. Moreover, $\overline{T}$ is injective if and only if $E = \ker(T)$ and $\norm{\overline{T}}=\norm{T}$.
\end{proposition}
\begin{proof}
  The existence and uniqueness of $\overline{T}: X/E\rightarrow Y$ such that $\overline{T}\circ \pi = T$ follows from the First Isomorphism Theorem for vector spaces, as does the fact that $\overline{T}$ is injective and only if $\ker(T) = E$.\newline

  Let $x+E\in X/E$. For $y\in E$, we have
  \begin{align*}
    \norm{\overline{T}\left(x+E\right)} &= \norm{\overline{T}\left(x-y+E\right)}\\
                                        &= \norm{T\left(x-y\right)}\\
                                        &\leq \norm{T}\norm{x-y}.
  \end{align*}
  Taking infimum over all $y\in E$, we get $\norm{\overline{T}\left(x+E\right)} \leq \norm{T}\norm{x+E}$, meaning $\norm{\overline{T}}\leq \norm{T}$. Additionally,
  \begin{align*}
    \norm{T} &= \norm{\overline{T}\circ \pi}\\
             &\leq \norm{\overline{T}}\norm{\pi}\\
             &= \norm{\overline{T}}.
  \end{align*}
\end{proof}
\begin{theorem}[First Isomorphism Theorem for Normed Vector Spaces]
  Let $X$ and $Y$ be normed vector spaces, $T\in \mathcal{B}\left(X,Y\right)$.
  \begin{enumerate}[(1)]
    \item $T$ is a quotient map if and only if $\overline{T}: X/\ker(T) \rightarrow Y$ is a bicontinuous isomorphism.
    \item $T$ is a $1$-quotient map if and only if $\overline{T}: X/\ker(T) \rightarrow Y$ is an isometric isomorphism.
  \end{enumerate}
\end{theorem}
\end{document}

% Get rid of the purely algebraic content in here, only focus on the analysis.
\chapter{Operator Algebras}\label{ch:operator_algebras}
In Chapter \ref{ch:nuclearity}, we will establish that the amenability of a group is equivalent to a property known as nuclearity held by the $C^{\ast}$-algebra(s) generated by the group. For this, we need a solid background in the theory of operator algebras --- specifically, in Banach algebras and $C^{\ast}$-algebras.
\section{Definitions and Examples}%
The theory of $C^{\ast}$-algebras is motivated by the fact that the adjoint operation on $\B\left( \mathcal{H} \right)$ (Definition \ref{def:adjoint_properties}) satisfies the criteria for an involution (Definition \ref{def:algebra_star_algebra}) on an algebra. However, one property that $\B\left( \mathcal{H} \right)$ has that a pure $\ast$-algebra lacks is the fact that $\B\left( \mathcal{H} \right)$ is equipped with a norm, $\norm{\cdot}_{\op}$, that turns $\B\left( \mathcal{H} \right)$ into a normed algebra (Definition \ref{def:norms}).\newline

What the theory of $C^{\ast}$-algebras allows us to do is abstract away from $\B\left( \mathcal{H} \right)$. Soon, we will see that this abstraction will allow us to focus on purely algebraic properties of $C^{\ast}$-algebras and establish fundamental analytic results on them.
\begin{definition}\label{def:banach_star_algebra}
  Let $A$ be an algebra.
  \begin{itemize}
    \item If $\norm{\cdot}$ is such that $\left( A,\norm{\cdot} \right)$ is a Banach space that satisfies $\norm{ab}\leq \norm{a}\norm{b}$ for all $a,b\in A$, then we say $\left( A,\norm{\cdot} \right)$ is a \textit{Banach algebra}.
    \item If $A$ is a $\ast$-algebra that is also a Banach algebra, and the norm on $A$ satisfies $\norm{a^{\ast}} = \norm{a}$ for all $a\in A$, then we say $A$ is a \textit{Banach $\ast$-algebra}.
    \item If $A$ is a Banach $\ast$-algebra whose norm also satisfies $\norm{a^{\ast}a} = \norm{a}^2$ for all $a\in A$, then we say $A$ is a \textit{$C^{\ast}$-algebra}. This final property is known as the $C^{\ast}$-property.
  \end{itemize}
\end{definition}
There are many $C^{\ast}$-algebras that we interact with as we study analysis.
\begin{example}\hfill
  \begin{itemize}
    \item The complex numbers, $\C$, equipped with the involution $z\mapsto \overline{z}$, are a $C^{\ast}$-algebra under the norm $\left\vert z \right\vert$.
    \item If $\mathcal{H}$ is a Hilbert space, then $\B\left( \mathcal{H} \right)$ is a $C^{\ast}$-algebra under the operator norm with the involution $T\mapsto T^{\ast}$.
    \item The space of $n\times n$ complex matrices, $\Mat_n\left( \C \right)$ under the operator norm and the involution $\left( a_{ij}^{\ast} \right)_{ij} = \left( \overline{a_{ji}} \right)_{ij}$ is a $C^{\ast}$-algebra.
    \item If $\Omega$ is any nonempty set, then the space of bounded functions, $\ell_{\infty}\left( \Omega \right)$, is a $C^{\ast}$-algebra under the norm $\norm{f}_{\ell_{\infty}} = \sup_{x\in\Omega}\left\vert f(x) \right\vert$ and the involution $f^{\ast}\left( x \right) = \overline{f(x)}$.
  \end{itemize}
\end{example}
However, there are also some Banach $\ast$-algebras that are not $C^{\ast}$-algebras.
\begin{example}
  Let 
  \begin{align*}
    \ell_1\left( \Z \right)\coloneq \set{f\colon \Z\rightarrow\C | \norm{f}_{\ell_1} \coloneq \sum_{n\in\Z}\left\vert f(n) \right\vert < \infty}
  \end{align*}
  be equipped with the involution
  \begin{align*}
    f^{\ast}\left( n \right) &= \overline{f\left( -n \right)}
  \end{align*}
  and multiplication
  \begin{align*}
    f\ast g(n) &= \sum_{k\in\Z}f(n-k)g(k).
  \end{align*}
  Then, $\ell_1(\Z)$ is a Banach $\ast$-algebra that does not satisfy the $C^{\ast}$-property.
\end{example}
The rest of this section will focus on understanding properties of $C^{\ast}$-algebras and their elements.
\section{\texorpdfstring{$C^{\ast}$-Norms}{C*-Norms}, Universal \texorpdfstring{$C^{\ast}$-Algebras}{C*-Algebras}, and Representations}%
We begin by constructing $C^{\ast}$-algebras.\newline

Recall that, in the case of a normed vector space, we know that (Proposition \ref{prop:completion_existence}) there is always a completion of $X$ into a Banach space, $\widetilde{X}\coloneq \overline{\iota_X\left( X \right)}^{\norm{\cdot}_{\op}}\subseteq X^{\ast\ast}$. This extends to the case of normed algebras/$\ast$-algebras and Banach algebras/Banach $\ast$-algebras.
\begin{lemma}[{\cite[Lemma 7.2.26]{rainone_analysis}}]\label{lemma:banach_algebra_completion}
  If $A_0$ is a normed algebra/$\ast$-algebra, then its Banach space completion, $A$, is a Banach algebra/Banach $\ast$-algebra. The inclusion, $A_0\hookrightarrow A$ is an isometric homomorphism/$\ast$-homomorphism of algebras/$\ast$-algebras.
\end{lemma}
If we have a normed algebra $A$ and we want its completion to be a $C^{\ast}$-algebra, then we need the norm itself to have properties analogous to the norm on a $C^{\ast}$-algebra.
\begin{definition}[{\cite[Definition 7.2.27]{rainone_analysis}}]
  Let $A_0$ be a $\ast$-algebra. A \textit{$C^{\ast}$-norm}/\textit{$C^{\ast}$-seminorm} on $A_0$ is a norm/seminorm on $A_0$ satisfying the following:
  \begin{enumerate}[(i)]
    \item $\norm{ab}\leq \norm{a}\norm{b}$;
    \item $\norm{a^{\ast}} = \norm{a}$;
    \item $\norm{a^{\ast}a} = \norm{a}^2$ (also known as the \textit{$C^{\ast}$-property})
  \end{enumerate}
  for all $a,b\in A_0$.
\end{definition}
We're able to construct $C^{\ast}$-norms by using $\ast$-homomorphisms into $C^{\ast}$-algebras.
\begin{lemma}[{\cite[Lemma 7.2.30]{rainone_analysis}}]
  Let $A_0$ be a $\ast$-algebra, and suppose $\phi\colon A_0\rightarrow B$ is a $\ast$-homomorphism into a $C^{\ast}$-algebra $B$. Then,
  \begin{align*}
    \norm{a}_{\phi} &= \norm{\phi(a)}_{\op}
  \end{align*}
  defines a $C^{\ast}$-seminorm on $A_0$. If $\phi$ is injective, then $\norm{\cdot}_{\phi}$ is a $C^{\ast}$-norm.
\end{lemma}
Just as in the case of Lemma \ref{lemma:banach_algebra_completion}, the completion of a normed algebra with a $C^{\ast}$-norm yields a $C^{\ast}$-algebra.
\begin{lemma}[{\cite[Lemma 7.2.32]{rainone_analysis}}]
  Let $\norm{\cdot}$ be a $C^{\ast}$-norm on a $\ast$-algebra $A_0$. The norm completion, $A$, is a $C^{\ast}$-algebra, and the inclusion $A_0\hookrightarrow A$ is an isometric $\ast$-homomorphism.
\end{lemma}
Recall that any seminorm on a vector space gives rise to a norm on the quotient space (Theorem \ref{thm:quotient_space_norm}) --- similarly, we may define the enveloping $C^{\ast}$-algebra on any $C^{\ast}$-seminorm on $A$.
\begin{definition}[{\cite[Definition 7.2.33]{rainone_analysis}}]\label{def:hausdorff_completion}
  Let $A_0$ be a $\ast$-algebra equipped with a $C^{\ast}$-seminorm $p$. The norm completion of $A/N_p$ with respect to $\norm{\cdot}_{A/N_p}$, where
  \begin{align*}
    N_p \coloneq \set{a\in A | p(a) = 0}
    \intertext{and}
    \norm{a + N_p} &= p(a),
  \end{align*}
  is known as the \textit{Hausdorff completion} or \textit{enveloping $C^{\ast}$-algebra} of $\left( A_0,p \right)$.
\end{definition}
We want to now understand a sort of ``maximal'' enveloping $C^{\ast}$-algebra --- preferably one that admits a universal property, similar to the universal property for the free $\ast$-algebra of Theorem \ref{thm:universal_property_free_algebra}. This will be the universal $C^{\ast}$-algebra.
\begin{definition}[{\cite[Definition 7.2.34]{rainone_analysis}}]
  Let $A_0$ be a $\ast$-algebra, and let $\mathcal{P}$ denote the collection of all $C^{\ast}$-seminorms on $A_0$. Set
  \begin{align*}
    \norm{a}_{u} &= \sup_{p\in \mathcal{P}}p(a).
  \end{align*}
  If $\norm{a}_u < \infty$ for all $a\in A_0$, then $\norm{\cdot}_u$ defines a $C^{\ast}$-seminorm on $A_0$ called the \textit{universal $C^{\ast}$-seminorm} on $A_0$.\newline

  The \textit{universal enveloping $C^{\ast}$-algebra} of $A_0$ is the enveloping $C^{\ast}$-algebra of the pair $\left( A_0,\norm{\cdot}_u \right)$.
\end{definition}
We can also define a universal $C^{\ast}$-algebra with respect to a set of relations $R$ with a similar universal property. 
\begin{definition}[{\cite[Definition 7.2.35]{rainone_analysis}}]
  Let $E$ be a set of abstract variables and suppose $R\subseteq \mathbb{A}^{\ast}\left( E \right)$ is a collection of relations. If the universal enveloping $C^{\ast}$-algebra of $\mathbb{A}^{\ast}\left( E|R \right)$ exists --- i.e., that $\norm{a}_u < \infty$ for all $a\in \mathbb{A}^{\ast}\left( E|R \right)$ --- we denote it $C^{\ast}\left( E|R \right)$. It is know as the \textit{universal $C^{\ast}$-algebra with generators $E$ and relations $R$}.
\end{definition}
\begin{proposition}[{\cite[Proposition 7.2.36]{rainone_analysis}}]\label{prop:universal_property_universal_cstar_algebra}
  Let $E = \set{x_i}_{i\in I}$ be a set of abstract variables, and let $R\subseteq \mathbb{A}^{\ast}\left( E \right)$ be a collection of relations. Suppose the universal $C^{\ast}$-algebra $C^{\ast}\left( E|R \right)$ exists.\newline

  If $B$ is a $C^{\ast}$-algebra admitting elements $\set{b_i}_{i\in I}$ that satisfy the relations $R$, then there is a unique contractive $\ast$-homomorphism, $\varphi_B\colon C^{\ast}\left( E|R \right) \rightarrow B$, such that
  \begin{align*}
    \varphi_b\left( v_i \right) = b_i,
  \end{align*}
  where $v_i \coloneq \left( x + I(R) \right) + N_u$ is a double coset with $I(R)$ as the ideal generated by the $R$ and $N_u$ is the zero set of $\norm{\cdot}_u$, as in Definition \ref{def:hausdorff_completion}.
\end{proposition}
We can realize $\ast$-algebras as $\ast$-subalgebras of bounded operators on Hilbert space.\footnote{In fact, via the GNS construction (which we apologetically cannot cover here), every $C^{\ast}$-algebra can be realized as a $\ast$-subalgebra of $\B\left( \mathcal{H} \right)$ for a suitable Hilbert space $\mathcal{H}$.} 
\begin{definition}[{\cite[Definition 7.2.41]{rainone_analysis}}]\label{def:unital_representation}
  Let $A_0$ be a $\ast$-algebra. A \textit{representation} of $A_0$ is a pair $\left( \pi_0,\mathcal{H} \right)$, where $\pi_0\colon A_0\rightarrow \B\left( \mathcal{H} \right)$ is a $\ast$-homomorphism.\newline

  If $A_0$ is unital, and $\pi_0\left( 1_A \right) = I_{\mathcal{H}}$, then we say $\pi_0$ is a \textit{unital} representation.
\end{definition}
What makes representations special is that they give us a $C^{\ast}$-norm ``for free'' in a sense.
\begin{lemma}[{\cite[Lemma 7.2.42]{rainone_analysis}}]
  Let $A_0$ be a $\ast$-algebra, and let $\left( \pi_0,\mathcal{H} \right)$ be a representation of $A_0$. Then,
  \begin{align*}
    \norm{a}_{\pi_0} &= \norm{\pi_0\left( a \right)}_{\op}
  \end{align*}
  is a $C^{\ast}$-seminorm on $A_0$. If $\pi_0$ is injective, then $\norm{\cdot}_{\pi_0}$ is a $C^{\ast}$-norm.
\end{lemma}
\begin{lemma}[{\cite[Lemma 7.2.43]{rainone_analysis}}]
  If $A_0$ and $B_0$ are normed $\ast$-algebras with completions $A$ and $B$, then any bounded $\ast$-homomorphism extends continuously to $\varphi\colon A\rightarrow B$.
\end{lemma}
\begin{corollary}[{\cite[Corollary 7.2.44]{rainone_analysis}}]
  Let $A_0$ be a $\ast$-algebra, and let $\pi\colon A_0\rightarrow \B\left( \mathcal{H} \right)$ be an injective representation.\newline

  The completion $A$ of $A_0$ with respect to the $C^{\ast}$-norm $\norm{\cdot}_{\pi_0}$ is a $C^{\ast}$-algebra, and the continuous extension $\pi\colon A\rightarrow \B\left( \mathcal{H} \right)$ is an isometric $\ast$-homomorphism.
\end{corollary}
\section{Spectra of Elements in \texorpdfstring{$C^{\ast}$-Algebras}{C*-Algebras}}%
\section{Characters of \texorpdfstring{$C^{\ast}$-Algebras}{C*-Algebras}}%
\section{The Continuous Functional Calculus}%

\nocite{*}
\printbibliography[heading=bibintoc,title={References}]
\end{document}
