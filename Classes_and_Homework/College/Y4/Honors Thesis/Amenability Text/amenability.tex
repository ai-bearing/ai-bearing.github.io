\documentclass[10pt]{package2}

% sans serif font:
%\usepackage[math]{iwona}
%\usepackage[cal=lucida,calscaled=0.96]{mathalpha}
%\usepackage{cmbright}
%\usepackage{eucal}
%\usepackage{sfmath}
%\usepackage{bbold} %better blackboard bold

%serif font + different blackboard bold for serif font
%\usepackage{newpxtext,eulerpx,eucal}
\usepackage[light]{kpfonts}
\newcommand{\1}{\mathds{1}}
%\newcommand{\1}{\mathbb{1}}
%\usepackage{chancery}
%\usepackage{gfsartemisia-euler}
%\renewcommand{\textlozenge}{}
%\renewcommand*{\mathbb}[1]{\varmathbb{#1}}
%\renewcommand*{\hbar}{\hslash}
\usepackage{titlesec}
\usepackage[backend=biber,style=alphabetic,sorting=ynt]{biblatex}
\addbibresource{chapters/references.bib}
\DeclareMathOperator{\op}{op}
\renewcommand{\coloneq}{=}


%\setcounter{secnumdepth}{4}
%\titleformat{\paragraph}
%{\normalfont\normalsize\bfseries}{\theparagraph}{1em}{}
%\titlespacing*{\paragraph}
%{0pt}{3.25ex plus 1ex minus .2ex}{1.5ex plus .2ex}
%\pagestyle{fancy} %better headers
%\fancyhf{}
%\rhead{Avinash Iyer}
%\lhead{}
\setcounter{chapter}{-1}
\pagestyle{fancy}
%\fancyhead[C]{\scshape Chapter \thechapter}
\fancyhead[R]{}
\fancyhead[L]{}
\renewcommand{\headrulewidth}{0pt}
\fancyfoot[C]{\thepage}

\title{Amenability in Discrete Groups\\ {\large Conditions and Applications}}
\author{Avinash Iyer}
\date{\today}
\setcounter{secnumdepth}{4}

\begin{document}
\maketitle
\RaggedRight
\tableofcontents
\chapter{Prelude}
This thesis will be an introduction and broad overview of the theory of amenable groups, which admit a rich set of structures and characterizations. While we will not fully track the development of the theory to its research nowadays, we will discuss the origin of the theory of amenability --- namely, the Banach--Tarski paradox --- and develop different characterizations for amenability that will draw from group theory, measure theory, general topology, and functional analysis. Most of the necessary material will be covered either in the chapters themselves or in the appendix.\newline

Amenability is a truly fascinating topic that, while its core is in functional analysis, allows significant insights into group theory nonetheless. This project may seem like an utterly daunting undertaking, but amenability is a very natural follow-on to the upper division mathematics curriculum (along with measure theory).
\chapter{Introduction: Paradoxical Decompositions}
In the Bible, one of the miracles of Jesus is the feeding of the five thousand,\footnote{Fun fact: the feeding of the five thousand is the only other miracle of Jesus (aside from the resurrection) that is in all four gospels.} where, despite only having five loaves of bread and two fishes, a large crowd splits these morsels among themselves and eats to satisfaction after Jesus calls upon the power of God to enable them to do so. Of course, we may not be able to fully replicate this without some divine intervention --- but, mathematically, thanks to the power of the axiom of choice, we can show that something like the feeding of the five thousand is not only possible, but a fundamental feature of the isometry group of $\R^3$. This is exemplified in the most general form of the Banach--Tarski paradox.
\begin{restatable}[Strong Banach--Tarski Paradox]{proposition}{banachtarski}\label{prop:banachtarski}
  Let $A$ and $B$ be bounded subsets of $\R^3$ with nonempty interior. There is a partition of $A$ into finitely many disjoint subsets such that a sequence of isometries applied to these subsets yields $B$.
\end{restatable}
The Banach--Tarski paradox throws a wrench into a common belief that we have about $\R^3$ --- specifically, that every subset of $\R^3$ has a \textit{finitely additive} ``volume'' that is invariant under rigid motion.\footnote{Note that if we desire countable additivity, the axiom of choice shows that there does not exist a countably additive measure on $P\left(\R\right)$ that is also translation-invariant (see \cite[Section 1.1]{folland_real_analysis}). Finite additivity is a weaker condition than countable additivity that allows for the existence of well-behaved measures on $P\left(\R\right)$ and $P\left(\R^2\right)$, but even this fails in $\R^3$ and above.} This property does exist for $\R$ and $\R^2$, as their isometry groups have a property known as amenability --- in Section \ref{sec:invariant_states_remarks}, we will provide an outline for why this is true.\newline

To develop paradoxical decompositions, we will begin with the Ping Pong Lemma, which will allow us to find freely generated subgroups. We will apply this to the case of $\text{SO}(3)$ to find a freely generated subgroup. Then, we will use the fact that free groups on more than one generator have a property known as paradoxicality --- this property will provide the germ of the proof of the Banach--Tarski paradox.
\section{The Ping Pong Lemma}\label{sec:ping_pong_lemma}%
To move towards paradoxical decompositions, we need to find a simple and easily applicable criterion for knowing when an arbitrary group contains a freely generated subgroup. This is the domain of the Ping Pong Lemma, which we will prove in this section to show the existence of a freely generated subgroup of $\text{SO}(3)$. Later, this freely generated subgroup will be indispensable in proving the Banach--Tarski paradox.\newline

We begin by defining a free product of a family of groups $\set{\Gamma_i}_{i\in I}$. This will allow us to state the Ping Pong Lemma in its maximal generality.
\begin{definition}[Free Product]\label{def:free_product}
  Let $A$ be a set, and set $W(A)$ to be the set of words in $A$ equipped with the operation of concatenation. This turns $W(A)$ into a construction known as the \textit{free monoid}.\newline

  If $\set{\Gamma_i}_{i\in I}$ is a family of groups, and $A = \coprod_{i\in I}\Gamma_i$ is the coproduct (or disjoint union) of the groups $\Gamma_i$, then we define the equivalence relation $\sim$ generated by
  \begin{align*}
    we_iw' &\sim ww'\text{ where $e_i$ is the neutral element of $\Gamma_i$ for some $i\in I$}\\
    wabw' &\sim wcw'\text{ where $a,b,c\in \Gamma_i$ and $c=ab$ for some $i\in I$}.
  \end{align*}
  Then, the quotient $W(A)/\sim$ is known as the \textit{free product} of the groups $\set{\Gamma_i}_{i\in I}$, and is denoted
  \begin{align*}
    \bigstar_{i\in I}\Gamma_i.
  \end{align*}
\end{definition}
\begin{remark}
  The free group $F(S)$ is an instance of the free product where the $\Gamma_i$ are the formal cyclic groups generated by each $s\in S$.\newline

  From the way we have defined the free product, it can be shown, as in \cite[II.A.]{delaHarpe_topics_in_geometric_group_theory}, that every element of the free product is represented a unique reduced word on $W(A)$, along with the following universal property: if $\set{\Gamma_i}_{i\in I}$ is a family of groups, and $h_i\colon \Gamma_i\rightarrow\Gamma$ for some fixed group $\Gamma$, then there is a unique homomorphism $h\colon \bigstar_{i\in I}\Gamma_i \rightarrow \Gamma$ such that the following diagram commutes for each $\Gamma_{i_0}$.
  \begin{center}
        % https://tikzcd.yichuanshen.de/#N4Igdg9gJgpgziAXAbVABwnAlgFyxMJZABgBpiBdUkANwEMAbAVxiRAB12BxOgW17oB9YFkHEAviHGl0mXPkIoyARiq1GLNpwBGWAOZwcdAE7CsnLGAAEASXGce-IVikyQGbHgJFl5NfWZWRA5uPgFBF3E1GCg9eCJQADNjCF4kMhAcCCRfdUCtdnwjMzFJagY6bRgGAAU5L0UQY30ACxwQaiMsBjYWiAgAa1cklLTEDKykACZqAM1glpKJYZBk1JzO7MQZkAqq2vqFNma9No68+ZAWqQpxIA
    \begin{tikzcd}
      \Gamma_{i_0} \arrow[d, "\iota_{i_0}"', hook] \arrow[r, "h_{i_0}"] & \Gamma_i \\
      \bigstar_{i\in I}\Gamma_i \arrow[ru, "h"']                        &         
    \end{tikzcd}
  \end{center}
\end{remark}

\begin{theorem}[Ping Pong Lemma]
  Let $G$ be a group that acts on a set $X$, and let $\Gamma_1,\Gamma_2$ be subgroups of $G$. Let $\Gamma = \left\langle \Gamma_1,\Gamma_2 \right\rangle$. Assume $\Gamma_1$ contains at least $3$ elements, and $\Gamma_2$ contains at least $2$ elements.\newline

  Suppose there exist nonempty subsets $X_1,X_2\subseteq X$ with $X_1\triangle X_2 \neq \emptyset$ such that for all $\gamma\in \Gamma_1$ with $\gamma \neq e_{G}$,
  \begin{align*}
    \gamma\left(X_2\right)\subseteq X_1,
  \end{align*}
  and for all $\gamma \in \Gamma_2$ with $\gamma \neq e_G$,
  \begin{align*}
    \gamma\left(X_1\right)\subseteq X_2.
  \end{align*}
  Then, $\Gamma$ is isomorphic to the free product $\Gamma_1\star \Gamma_2$.\label{thm:ping_pong}
\end{theorem}
\begin{proof}
  Let $w$ be a nonempty reduced word with letters in the disjoint union of $\Gamma_1\setminus \set{e_G}$ and $\Gamma_2\setminus \set{e_G}$. We must show that the element of $\Gamma$ defined by $w$ is not the identity.\newline

  If $w = a_1b_1a_2b_2\cdots a_k$ with $a_1,\dots,a_k\in \Gamma_1\setminus \set{e_G}$ and $b_1,\dots,b_{k-1}\in \Gamma_{2}\setminus \set{e_G}$, then,
  \begin{align*}
    w\left(X_2\right) &= a_1b_1\cdots a_{k-1}b_{k-1}a_k\left(X_2\right)\\
                      &\subseteq a_1b_1\cdots a_{k-1}b_{k-1}\left(X_1\right)\\
                      &\subseteq a_1b_1\cdots a_{k-1}\left(X_2\right)\\
                      &\vdots\\
                      &\subseteq a_1\left(X_2\right)\\
                      &\subseteq X_1.
  \end{align*}
  Seeing as $X_2\nsubseteq X_1$ (by the definition of symmetric difference), it is the case that $w\neq e_{G}$.\newline

  If $w = b_1a_2b_2a_2\cdots b_k$, we select $a\in \Gamma_1\setminus \set{e_G}$, and we find that $awa^{-1}\neq e_G$, meaning $w\neq e_G$. Similarly, if $w = a_1b_1\cdots a_kb_k$, we select $a\in \Gamma_1\setminus \set{e_G,a_{1}^{-1}}$, similarly finding that $awa^{-1}\neq e_{G}$. If $w = b_1a_2b_2\cdots a_k$, then we select $a\in \Gamma_1\setminus \set{1,a_k}$, and find $awa^{-1}\neq e_G$.
\end{proof}

We can refine Theorem \ref{thm:ping_pong} to the case of ``doubles'' wherein we find a different (yet more readily applicable) sufficient condition for a group that contains a copy of the free group on two generators.
\begin{corollary}[Ping Pong Lemma for ``Doubles'']
  Let $G$ act on $X$, and let $A_{+}, A_{-},B_{+},B_{-}$ be disjoint subsets of $X$ whose union is not equal to $X$. Then, if
  \begin{align*}
    a\cdot \left(X\setminus A_{-}\right) &\subseteq A_{+}\\
    a^{-1}\cdot \left(X\setminus A_{+}\right) &\subseteq A_{-}\\
    b\cdot \left(X\setminus B_{-}\right) &\subseteq B_{+}\\
    b^{-1}\cdot \left(X\setminus B_{+}\right) &\subseteq B_{-},
  \end{align*}
  then it is the case that $\left\langle a,b \right\rangle$ is isomorphic to the free group on two generators.\label{corollary:ping_pong_doubles}
\end{corollary}
\begin{proof}
  We let $A = A_{+}\sqcup A_{-}$, $B = B_{+}\sqcup B_{-}$, $\Gamma_1 = \left\langle a \right\rangle$, and $\Gamma_2 = \left\langle b \right\rangle$. Then, $A,B,\Gamma_1,\Gamma_2$ satisfy the conditions for Theorem \ref{thm:ping_pong}.
\end{proof}
\begin{remark}
Instead of typing out ``the free group on two generators,'' we will henceforth use $F(a,b)$ to refer to the free group on two generators.
\end{remark}

We can apply Theorem \ref{thm:ping_pong} to show the existence of a set of isometries of $\R^n$ that is isomorphic to $F(a,b)$.
\begin{definition}[Special Orthogonal Group]
  For $n\in \N$, we define $\text{SO}(n)$ to be the group of all real $n\times n$ matrices $A$ such that $A^{T} = A^{-1}$ and $\det(A) = 1$.
\end{definition}
In terms of an isometry of $\R^3$, the group $\text{SO}(3)$ denotes the set of all rotations about any line through the origin.
\begin{theorem}\label{thm:free_group_so3}
  There are elements $a,b\in \text{SO}(3)$ such that $\left\langle a,b \right\rangle_{\text{SO}(3)} \cong F(a,b)$.
\end{theorem}
\begin{proof}
  We let
  \begin{align*}
    a &= \begin{pmatrix}3/5 & 4/5 & 0 \\ -4/5 & 3/5 & 0 \\ 0 & 0 & 1\end{pmatrix}\\
    a^{-1} &= \begin{pmatrix}3/5 & -4/5 & 0 \\ 4/5 & 3/5 & 0 \\ 0 & 0 & 1\end{pmatrix}\\
    b &= \begin{pmatrix}1 & 0 & 0 \\ 0 & 3/5 & -4/5 \\ 0 & 4/5 & 3/5\end{pmatrix}\\
    b^{-1} &= \begin{pmatrix}1 & 0 & 0 \\ 0 & 3/5 & 4/5 \\ 0 & -4/5 & 3/5\end{pmatrix}.
  \end{align*}
  We specify
  \begin{align*}
    X &= A_{+} \sqcup A_{-} \sqcup B_{+} \sqcup B_{-} \sqcup \begin{pmatrix}0\\1\\0\end{pmatrix},
  \end{align*}
  where
  \begin{align*}
    A_{+} &= \set{\frac{1}{5^{k}} \begin{pmatrix}x\\y\\z\end{pmatrix} | k\in \Z, x \equiv 3y\text{ modulo $5$}, z\equiv0\text{ modulo $5$}}\\
    A_{-} &= \set{\frac{1}{5^{k}} \begin{pmatrix}x\\y\\z\end{pmatrix} | k\in \Z, x \equiv -3y\text{ modulo $5$}, z\equiv 0\text{ modulo $5$}}\\
    B_{+} &= \set{\frac{1}{5^{k}} \begin{pmatrix}x\\y\\z\end{pmatrix} | k\in \Z, z \equiv 3y\text{ modulo $5$}, x\equiv 0\text{ modulo $5$}}\\
    B_{-} &= \set{\frac{1}{5^{k}} \begin{pmatrix}x\\y\\z\end{pmatrix} | k\in \Z, z \equiv -3y\text{ modulo $5$}, x\equiv 0\text{ modulo $5$}}.
  \end{align*}
  To verify that the conditions of Theorem \ref{thm:ping_pong} hold, we calculate
  \begin{align*}
    \begin{pmatrix}3/5 & 4/5 & 0 \\ -4/5 & 3/5 & 0 \\ 0 & 0 & 1\end{pmatrix}\left(\frac{1}{5^k} \begin{pmatrix}x\\y\\z\end{pmatrix}\right) &= \frac{1}{5^{k+1}} \begin{pmatrix}3x + 4y \\ -4x + 3y \\ 5z\end{pmatrix}\tag*{(1)}\\
    \begin{pmatrix}3/5 & -4/5 & 0 \\ 4/5 & 3/5 & 0 \\ 0 & 0 & 1\end{pmatrix} \left(\frac{1}{5^k} \begin{pmatrix}x\\y\\z\end{pmatrix}\right) &= \frac{1}{5^{k+1}} \begin{pmatrix}3x - 4y \\ 4x + 3y \\ 5z\end{pmatrix}\tag*{(2)}\\
    \begin{pmatrix}1 & 0 & 0 \\ 0 & 3/5 & -4/5 \\ 0 & 4/5 & 3/5\end{pmatrix}\left(\frac{1}{5^{k}} \begin{pmatrix}x\\y\\z\end{pmatrix}\right) &= \frac{1}{5^{k+1}} \begin{pmatrix}5x \\ 3y- 4z \\ 4y + 3z\end{pmatrix}\tag*{(3)}\\
    \begin{pmatrix}1 & 0 & 0 \\ 0 & 3/5 & 4/5 \\ 0 & -4/5 & 3/5\end{pmatrix} \left(\frac{1}{5^{k}} \begin{pmatrix}x\\y\\z\end{pmatrix}\right) &= \frac{1}{5^{k+1}} \begin{pmatrix}5x \\ 3y + 4z \\ -4y + 3z\end{pmatrix}.\tag*{(4)}
  \end{align*}
  We verify that the conditions for Corollary \ref{corollary:ping_pong_doubles} hold for each of these four conditions.
  \begin{enumerate}[(1)]
    \item For any vector
      \begin{align*}
        \frac{1}{5^{k}} \begin{pmatrix}x\\y\\z\end{pmatrix} \notin A_{-},
      \end{align*}
      we see that $k+1\in \Z$, $x' = 3x + 4y \equiv 3\left(-4x + 3y\right)$  modulo $5$, and that $z' = 5z\equiv 0$ modulo $5$.
    \item For any vector
      \begin{align*}
        \frac{1}{5^{k}} \begin{pmatrix}x\\y\\z\end{pmatrix} \notin A_{+},
      \end{align*}
      we see that $k+1\in \Z$, $x' = 3x - 4y\equiv -3\left(4x + 3y\right)$ modulo $5$, and $z' = 5z \equiv 0$ modulo $5$.
    \item For any vector
      \begin{align*}
        \frac{1}{5^{k}} \begin{pmatrix}x\\y\\z\end{pmatrix}\notin B_{-},
      \end{align*}
      we see that $k+1\in \Z$, $z' = 4y + 3z \equiv 3\left(3y-4z\right)$ modulo $5$, and $x' = 5x\equiv 0$ modulo $5$.
    \item For any vector
      \begin{align*}
        \frac{1}{5^{k}} \begin{pmatrix}x\\y\\z\end{pmatrix}\notin B_{+},
      \end{align*}
      we see that $k+1\in \Z$, $z' = -4y + 3z \equiv -3\left(3y + 4z\right)$ modulo $5$, and $x' = 5x \equiv 0$ modulo $5$.
  \end{enumerate}
  Thus, by Theorem \ref{thm:ping_pong} and Corollary \ref{corollary:ping_pong_doubles}, it is the case that $\left\langle a,b \right\rangle\cong F(a,b)$.
\end{proof}

\section{Introducing Paradoxical Decompositions}\label{sec:intro_paradoxical_decompositions}%
We now turn our attention towards ``paradoxical'' actions that seem to recreate a set by using disjoint proper subsets. This will allow us to use the result from Theorem \ref{thm:free_group_so3} to move towards the Banach--Tarski paradox.
\begin{definition}[Paradoxical Decompositions and Paradoxical Groups]
  Let $G$ be a group that acts on a set $X$, with $E\subseteq X$. We say $E$ is $G$\textit{-paradoxical} if there exist pairwise disjoint proper subsets $A_1,\dots,A_n$ and $B_1,\dots,B_m$ of $E$ and group elements $g_1,\dots,g_n,h_1,\dots,h_m\in G$ such that
  \begin{align*}
    E &= \bigcup_{j=1}^{n}g_j\cdot A_j
  \end{align*}
  and
  \begin{align*}
    E &= \bigcup_{j=1}^{m}h_j\cdot B_j.
  \end{align*}
  If $G$ acts on itself by left-multiplication, and $G$ satisfies these conditions, we say $G$ is a \textit{paradoxical group}.
\end{definition}
\begin{example}
  The free group on two generators, $F(a,b)$, is a paradoxical group.\newline
  %The free group is defined to be the set of all reduced words over the set $\set{a,b,a^{-1},b^{-1},e_{F(a,b)}}$, where $aa^{-1}$, $a^{-1}a$, $bb^{-1}$, and $b^{-1}b$ are replaced with the identity $e_{F(a,b)}$.\newline

  To see that $F(a,b)$ is a paradoxical group, we let $W(x)$ denote the set of words in $F(a,b)$ that start with $x\in \set{a,b,a^{-1},b^{-1}}$. For instance, $ba^2ba^{-1}\in W(b)$.\newline

  Since every word in $F$ is either the empty word, or starts with one of $a,b,a^{-1},b^{-1}$, we see that
  \begin{align*}
    F(a,b) &= \set{e_{F(a,b)}} \sqcup W(a) \sqcup W(b) \sqcup W\left(a^{-1}\right) \sqcup W\left(b^{-1}\right).
  \end{align*}
  If $w\in F(a,b)\setminus W(a)$, we see that $a^{-1}w\in W\left(a^{-1}\right)$. Thus, $w\in aW\left(a^{-1}\right)$. For any $t\in F(a,b)$ either $t\in W(a)$ or $t\in F(a,b)\setminus W(a) = aW\left(a^{-1}\right)$. Thus, $F\left(a,b\right) $ is equal to $ W(a)\sqcup aW\left(a^{-1}\right)$.\newline

  Similarly, if $t\in F(a,b)$ either $t\in W(b)$ or $t\in F\left(a,b\right)\setminus W(b) = bW\left(b^{-1}\right)$, so $F\left(a,b\right) $ is equal to $ W(b)\sqcup bW\left(b^{-1}\right)$.\newline

  We have thus constructed
  \begin{align*}
    F(a,b) &= W(a)\sqcup aW\left(a^{-1}\right)\\
           &= W(b)\sqcup bW\left(b^{-1}\right),
  \end{align*}
  a paradoxical decomposition of $F(a,b)$ with the action of left-multiplication.
\end{example}
Now that we understand a little more about paradoxical groups, we now want to understand the actions of paradoxical groups on sets.
\begin{proposition}
  Let $G$ be a paradoxical group that acts freely on $X$. Then, $X$ is $G$-paradoxical.
\end{proposition}
\begin{proof}
  Let $A_1,\dots,A_n,B_1,\dots,B_m\subset G$ be pairwise disjoint, and let $g_1,\dots,g_n,h_1,\dots,h_m\in G$ such that
  \begin{align*}
    G &= \bigcup_{j=1}^{n}g_jA_j\\
      &= \bigcup_{j=1}^{m}h_jB_j.
  \end{align*}
  Let $M\subseteq X$ contain exactly one element from every orbit in $X$.
  \begin{claim}
  The set $\set{g\cdot M\mid g\in G}$ is a partition of $X$.
  \end{claim}
  \begin{proof}[Proof of Claim:]
  Since $M$ contains exactly one element from every orbit in $X$, it is the case that $G\cdot M = X$, so
  \begin{align*}
    \bigcup_{g\in G} g\cdot M &= X
  \end{align*}
  Additionally, for $x,y\in M$, if $g\cdot x = h\cdot y$, then $\left(h^{-1}g\right)\cdot x = y$, meaning $y$ is in the orbit of $x$ and vice versa, implying $x = y$. Since $G$ acts freely on $X$, we must have $h^{-1}g = e_G$.\newline

  Thus, we can see that $g_1\cdot M \neq g_2\cdot M$, implying $\set{g\cdot M\mid g\in G}$ is a partition of $X$.
  \end{proof}

  We define
  \begin{align*}
    A_j^{\ast} &= \bigcup_{g\in A_j}g\cdot M,
  \end{align*}
  and similarly define
  \begin{align*}
    B_j^{\ast} &= \bigcup_{h\in B_j}h\cdot M.
  \end{align*}
  As a useful shorthand, we can also write $A_j^{\ast} = A_j\cdot M$, and similarly, $B_j^{\ast} = B_j\cdot M$, to denote the union of the elements of $A_j$ and $B_j$ respectively acting on $M$.\newline

  Since $\set{g\cdot M\mid g\in G}$ is a partition of $X$, and $A_1,\dots,A_n,B_1,\dots,B_m\subset G$ are pairwise disjoint, it must be the case that $A_1^{\ast},\dots,A_n^{\ast},B_1^{\ast},\dots,B_m^{\ast}\subset X$ are also pairwise disjoint.\newline

  For the original $g_1,\dots,g_n,h_1,\dots,h_m$ that defined the paradoxical decomposition of $G$, we thus have
  \begin{align*}
    \bigcup_{j=1}^{n}g_j\cdot A_j^{\ast} &= \bigcup_{j=1}^{n}\left(g_jA_j\right)\cdot M\\
                                         &= G\cdot M\\
                                         &= X,
  \end{align*}
  and
  \begin{align*}
    \bigcup_{j=1}^{m}h_j\cdot B_j^{\ast} &= \bigcup_{j=1}^{m}\left(h_jB_j\right)\cdot M\\
                                         &= G\cdot M\\
                                         &= X.
  \end{align*}
  Thus, $X$ is $G$-paradoxical.
\end{proof}
\begin{remark}
  This proof requires the axiom of choice, as we invoked it to define $M$ to contain exactly one element from every orbit in $X$.
\end{remark}
\section{The Weak Banach--Tarski Paradox}\label{sec:weak_banach_tarski}%
Now that we have established $F(a,b)$ as being a paradoxical group, we wish to use it to construct paradoxical decompositions of the unit sphere $S^2\subseteq \R^3$. Specifically, we will show a weak version of the Banach--Tarski paradox --- one where you can break apart the unit ball into finitely many pieces and reconstitute it into two copies of itself.
\begin{fact}
  If $H$ is a paradoxical group, and $H\leq G$, then $G$ is a paradoxical group.
\end{fact}
With this fact in mind, we will show that $\text{SO}(3)$ is a paradoxical group.
\begin{theorem}
  There are rotations $A$ and $B$ that about lines through the origin in $\R^3$ that generate a subgroup of $\text{SO}(3)$ isomorphic to $F(a,b)$
\end{theorem}
\begin{proof}
  We take $A$ and $B$ as in the proof of Theorem \ref{thm:free_group_so3}.
%  We take
%  \begin{align*}
%    A &= \begin{bmatrix}1/3 & -\frac{2\sqrt{2}}{3} & 0\\ \frac{2\sqrt{2}}{3} & 1/3 & 0 \\ 0 & 0 & 1\end{bmatrix}\\
%    A^{-1} &= \begin{bmatrix}1/3 & \frac{2\sqrt{2}}{3} & 0\\ -\frac{2\sqrt{2}}{3} & 1/3 & 0 \\ 0 & 0 & 1\end{bmatrix}\\
%    B &= \begin{bmatrix}1 & 0 & 0 \\ 0 & 1/3 & -\frac{2\sqrt{2}}{3} \\ 0 & \frac{2\sqrt{2}}{3} & 1/3\end{bmatrix}\\
%    B^{-1} &= \begin{bmatrix}1 & 0 & 0 \\ 0 & 1/3 & \frac{2\sqrt{2}}{3} \\ 0 & -\frac{2\sqrt{2}}{3} & 1/3\end{bmatrix}
%  \end{align*}
%  We let $A^{\pm}$ denote $A$ and $A^{-1}$ respectively, and similarly for $B^{\pm}$.\newline
%
%  Let $w$ be a reduced word in $\set{A,A^{-1},B,B^{-1}}$ which is not the empty word. We claim that $w$ cannot be the identity.\newline
%
%  Without loss of generality, we assume that $w$ ends in $A$ or $A^{-1}$ --- this is because if $w$ is the identity, then $AwA^{-1}$ and $A^{-1}wA$ are also the identity.\newline
%
%  We will show that there exist $a,b,c\in \Z$ with $b\not\equiv 0$ mod $3$ such that
%  \begin{align*}
%    w \cdot \begin{pmatrix}1\\0\\0\end{pmatrix} &= \frac{1}{3^k} \begin{pmatrix}a\\b\sqrt{2}\\c\end{pmatrix}.
%  \end{align*}
%  If $b\not\equiv 0$ mod $3$, and $w$ is not empty, then $w$ cannot act as the identity.\newline
%
%  We induct on the length of $w$. For $w = A^{\pm}$, we have
%  \begin{align*}
%    w\cdot \begin{pmatrix}1\\0\\0\end{pmatrix} &= \frac{1}{3}\begin{pmatrix}1\\\pm2\sqrt{2}\\0\end{pmatrix},
%  \end{align*}
%  proving the base case.\newline
%
%  Let $k > 0$, meaning $w = A^{\pm}w'$, or $w = B^{\pm}w'$, with $w'$ not equal to the empty. The inductive hypothesis says
%  \begin{align*}
%    w'\cdot \begin{pmatrix}1\\0\\0\end{pmatrix} &= \frac{1}{3^{k-1}} \begin{pmatrix}a'\\b'\sqrt{2}\\c'\end{pmatrix}
%  \end{align*}
%  for some $a',b',c'\in \Z$, and $b'\not\equiv 0$ mod $3$. In particular,
%  \begin{align*}
%    A^{\pm}w' \cdot \begin{pmatrix}1\\0\\0\end{pmatrix} &= \frac{1}{3^k} \begin{pmatrix}a\mp 4b \\ \left(b'\pm 2a'\right)\sqrt{2} \\ 3c'\end{pmatrix}\\
%    B^{\pm}w' \cdot \begin{pmatrix}1\\0\\0\end{pmatrix} &= \frac{1}{3^k} \begin{pmatrix}3a' \\ \left(b'\mp 2c'\right)\sqrt{2} \\ c'\pm 4b'\end{pmatrix}.
%  \end{align*}
%  Now, we set
%  \begin{align*}
%    w \cdot \begin{pmatrix}1\\0\\0\end{pmatrix} &= \frac{1}{3^k} \begin{pmatrix}a\\b\sqrt{2}\\c\end{pmatrix},
%  \end{align*}
%  meaning
%  \begin{align*}
%    a &= \begin{cases}
%      a'\mp 4b', & w = A^{\pm} w'\\
%      3a', & w = B^{\pm}w'
%    \end{cases}\\
%      b &= \begin{cases}
%        b'\pm 2a', & w = A^{\pm}w'\\
%        b'\mp 2c', & w = B^{\pm}w'
%      \end{cases}\\
%        c &= \begin{cases}
%          3c', & w = A^{\pm}w'\\
%          c' \pm 4b', & w = B^{\pm}w'
%        \end{cases}
%  \end{align*}
%  Let $w^{\ast}$ denote the word such that $w' = A^{\pm}w^{\ast}$ or $w' = B^{\pm}w^{\ast}$. We write
%  \begin{align*}
%    w^{\ast} &= \frac{1}{3^{k-2}} \begin{pmatrix}a''\\b''\sqrt{2}\\c''\end{pmatrix},
%  \end{align*}
%  where $a'',b'',c''\in \Z$. Note that it may not be the case that $w^{\ast}$ is a non-empty word. We examine the following four cases.
%  \begin{description}
%    \item[Case 1:] Suppose $w = A^{\pm}B^{\pm}w^{\ast}$. Then, $b = b'\mp 2a'$, where $a' = 3a''$. Since $b'\not\equiv 0$ mod $3$ (by the inductive hypothesis), it is also the case $b\equiv 0$ mod $3$.
%    \item[Case 2:] Suppose $w = B^{\pm}A^{\pm}w^{\ast}$. Then, $b = b'\mp 2c'$, where $c' = 3c''$. Since $b'\not\equiv 0$ mod $3$ (by the inductive hypothesis), it is also the case that $b\not\equiv 0$ mod $3$.
%    \item[Case 3:] Suppose $w = A^{\pm}A^{\pm}w^{\ast}$. Then, we have
%      \begin{align*}
%        b &= b' \pm 2a'\\
%          &= b' \pm 2\left(a'' \pm 4b''\right)\\
%          &= b'+ \left(b'' \pm 2a''\right) - 9b''\\
%          &= 2b' - 9b''.
%      \end{align*}
%      Thus, regardless of the value of $b''$, since $b'\not\equiv 0$ mod $3$ by the inductive hypothesis, it is the case that $b\not\equiv 0$ mod $3$.
%    \item Suppose $w = B^{\pm}B^{\pm}w^{\ast}$. Then, we have
%      \begin{align*}
%        b &= b' \mp 2c'\\
%          &= b' \mp 2\left(c'' \pm 4b''\right)\\
%          &= b' + \left(b'' \mp 2c''\right) - 9b''\\
%          &= 2b' - 9b''.
%      \end{align*}
%      Thus, regardless of the value of $b''$, since $b'\not\equiv 0$ mod $3$ by the inductive hypothesis, it is the case that $b\not\equiv 0$ mod $3$.
%  \end{description}
%  We have thus shown that any non-empty reduced word over $\set{A,A^{-1},B,B^{-1}}$ does not act as the identity. The subgroup of $\text{SO}(3)$ generated by $\set{A,A^{-1},B,B^{-1}}$ is isomorphic to $F(a,b)$.
\end{proof}
\begin{remark}
  Since $\text{SO}(n)$ contains a subgroup isomorphic to $\text{SO}(3)$ for all $n\geq 3$ (via the block matrices), it is the case that $\text{SO}(n)$ also contains a subgroup isomorphic to $F(a,b)$ for all $n\geq 3$.
\end{remark}
Since we have shown that $\text{SO}(3)$ is paradoxical, as it contains a paradoxical subgroup, we can now begin to examine the action of $\text{SO}(3)$ on subsets of $\R^3$.
\begin{theorem}[Hausdorff Paradox]
  There is a countable subset $D$ of $S^{2}$ such that $S^{2}\setminus D$ is $\text{SO}(3)$-paradoxical.
\end{theorem}
\begin{proof}
  Let $A$ and $B$ be the rotations in $\text{SO}(3)$ that serve as the generators of the subgroup isomorphic to $F(a,b)$ (as in \ref{thm:free_group_so3}).\newline

  Since $A$ and $B$ are rotations, so too is any element of $\left\langle A,B \right\rangle$. Thus, any such non-empty word contains two fixed points.\newline

  We let
  \begin{align*}
    F &= \set{x\in S^{2}\mid x\text{ is a fixed point for some word }w}.
  \end{align*}
  Since $\left\langle A,B \right\rangle$ is countably infinite, so too is $F$. Thus, the union of all these fixed points under the action of all such words $w$ is countable.
  \begin{align*}
    D &= \bigcup_{w\in \left\langle A,B \right\rangle} w\cdot F.
  \end{align*}
  Therefore, $\left\langle A,B \right\rangle$ acts freely on $S^{2}\setminus D$, so $S^{2}\setminus D$ is $\text{SO}(3)$-paradoxical.
\end{proof}

Unfortunately, the Hausdorff paradox is not enough for us to be able to prove the Banach--Tarski paradox. In order to do this, we need to be able to show that two sets are ``similar'' under the action of a group.
\begin{definition}[Equidecomposable Sets]
  Let $G$ act on $X$, and let $A,B\subseteq X$. We say $A$ and $B$ are $G$-equidecomposable if there are partitions $\set{A_j}_{j=1}^{n}$ of $A$ and $\set{B_j}_{j=1}^{n}$ of $B$, and elements $g_1,\dots,g_n\in G$, such that for all $j$,
  \begin{align*}
    B_j &= g_j\cdot A_j.
  \end{align*}
  We write $A\sim_{G}B$ if $A$ and $B$ are $G$-equidecomposable.
\end{definition}
\begin{fact}\label{fact:equidecomposability_equivalence_relation}
  The relation $\sim_{G}$ is an equivalence relation.
\end{fact}
\begin{proof}
  Let $A$, $B$, and $C$ be sets.\newline

  To show reflexivity, we can select $g_1 = g_2 = \cdots = g_n = e_G$. Thus, $A\sim_{G}A$.\newline

  To show symmetry, let $A\sim_{G} B$. Set $\set{A_j}_{j=1}^{n}$ to be the partition of $A$, and set $\set{B_j}_{j=1}^{n}$ to be the partition of $B$, such that there exist $g_1,\dots,g_n\in G$ with $g_j\cdot A_j = B_j$. Then,
  \begin{align*}
    g_j^{-1}\cdot \left(g_j\cdot A_j\right) &= g_j^{-1}\cdot B_j\\
    A_j &= g_j^{-1}\cdot B_j,
  \end{align*}
  so $B_j\sim_{G}A_j$.\newline

  To show transitivity, let $A\sim_{G} B$ and $B\sim_{G} C$. Let $\set{A_i}_{i=1}^{n}$ and $\set{B_i}_{i=1}^{n}$ be the partitions of $A$ and $B$ respectively and $g_1,\dots,g_n\in G$ such that $g_i\cdot A_i = B_i$. Let $\set{B_j}_{j=1}^{m}$ and $\set{C_j}_{j=1}^{m}$ be partitions of $B$ and $C$, and $h_1,\dots,h_m\in G$, such that $h_j\cdot B_j = C_j$.\newline

  We refine the partition of $A$ to $A_{ij}$ by taking $A_{ij} = g_i^{-1}\left(B_{i}\cap B_j\right)$, where $i = 1,\dots,n$ and $j = 1,\dots,m$. Then, $\left(h_jg_i\right)\cdot A_{ij}$ maps the refined partition of $A$ to $C$, so $A$ and $C$ are $G$-equidecomposable.
\end{proof}
\begin{fact}
  For $A\sim_{G} B$, there is a bijection $\phi\colon A\rightarrow B$ by taking $C_{i} = C\cap A_i$, and mapping $\phi\left(C_i\right) = g_i\cdot C_i$.\newline

  In particular, this means that for any subset $C\subseteq A$, it is the case that $C\sim \phi(C)$.\label{fact:bijections}
\end{fact}

We can now use this equidecomposability to glean information about the existence of paradoxical decompositions.
\begin{proposition}
  Let $G$ act on $X$, with $E,E'\subseteq X$ such that $E\sim_{G}E'$. Then, if $E$ is $G$-paradoxical, then so too is $E'$.
\end{proposition}

\begin{proof}
Let $A_1,\dots,A_n,B_1,\dots,B_m\subset E$ be pairwise disjoint, with $g_1,\dots,g_n,h_1,\dots,h_m\in G$ such that
\begin{align*}
  E &= \bigcup_{i=1}^{n}g_i\cdot A_i\\
    &= \bigcup_{j=1}^{m}h_j\cdot B_j.
\end{align*}
We let
\begin{align*}
  A &= \bigsqcup_{i=1}^{n}A_i\\
  B &= \bigsqcup_{j=1}^{m}B_j.
\end{align*}
It follows that $A\sim_{G}E$ and $B\sim_{G}E$, since we can take the partition of $A$ to be $A_1,\dots,A_n$, and partition $E$ by taking $g_i\cdot A_i$ for $i=1,\dots,n$, and similarly for $B$.\newline

Since $E\sim_{G}E'$, and $\sim_{G}$ is an equivalence relation, it follows that $A\sim_{G}E'$ and $B\sim_{G}E'$. Thus, there is a paradoxical decomposition of $E'$ in $A_1,\dots,A_n$ and $B_1,\dots,B_m$.
\end{proof}

We will now show that $S^{2}$ is $\text{SO}(3)$ paradoxical.
\begin{proposition}
  Let $D\subseteq S^{2}$ be countable. Then, $S^{2}$ and $S^{2}\setminus D$ are $\text{SO}(3)$-equidecomposable.
\end{proposition}

\begin{proof}
  Let $L$ be a line in $\R^3$ such that $L\cap D = \emptyset$. Such an $L$ must exist since $S^{2}$ is uncountable.\newline

  Define $\rho_{\theta}\in \text{SO}(3)$ to be a rotation about $L$ by an angle of $\theta$. For a fixed $n\in \N$ and fixed $\theta\in [0,2\pi)$, define $R_{n,\theta} = \set{x\in D\mid \rho^{n}_{\theta}\cdot x \in D}$. Since $D$ is countable, $R_{n,\theta}$ is necessarily countable.\newline

  We define $W_n = \set{\theta\mid R_{n,\theta}\neq \emptyset}$. Since the map $\theta \mapsto \rho_{\theta}^{n}\cdot x$ into $D$ is injective, it is the case that $W_n$ is countable. Therefore,
  \begin{align*}
    W &= \bigcup_{n\in \N}W_n
  \end{align*}
  is countable.\newline

  Thus, there must exist $\omega \in [0,2\pi)\setminus W$. We define $\rho_{\omega}$ to be a rotation about $L$ by $\omega$. Then, for every $n,m\in \N$, we have
  \begin{align*}
    \rho_{\omega}^n\cdot D \cap \rho_{\omega}^{m}\cdot D &= \emptyset.
  \end{align*}
  We define $\widetilde{D} = \bigsqcup_{n=0}^{\infty}\rho^{n}_{\omega}D$. Note that 
  \begin{align*}
    \rho_{\omega}\cdot \widetilde{D} &= \rho_{\omega}\cdot\bigsqcup_{n=0}^{\infty}\rho_{\omega}^{n}\cdot D\\
                                     &= \bigsqcup_{n=1}^{\infty}\rho_{\omega}^{n}\cdot D\\
                                     &= \widetilde{D} \setminus D,
  \end{align*}
  meaning $\widetilde{D}$ and $D$ are $\text{SO}(3)$-equidecomposable.\newline

  Thus, we have
  \begin{align*}
    S^{2} &= \widetilde{D}\sqcup \left(S^{2}\setminus \widetilde{D}\right)\\
          &\sim_{\text{SO}(3)}\left(\rho_{\omega}\cdot \widetilde{D}\right)\sqcup \left(S^{2}\setminus\widetilde{D}\right)\\
          &= \left(\widetilde{D}\setminus D\right)\sqcup \left(S^{2}\setminus\widetilde{D}\right)\\
          &= S^{2}\setminus D,
  \end{align*}
  establishing $S^{2}$ and $S^{2}\setminus D$ as $\text{SO}(3)$-equidecomposable.\newline

  In particular, this means $S^{2}$ is also $\text{SO}(3)$-paradoxical.
\end{proof}
To prove the Banach--Tarski paradox, we need a slightly larger group than $\text{SO}(3)$ --- one that includes translations in addition to the traditional rotations.
\begin{definition}[Euclidean Group]
  The {Euclidean group}, $\text{E}(n)$, consists of all isometries of a Euclidean space. An isometry of a Euclidean space consists of translations, rotations, and reflections.
\end{definition}
\begin{corollary}[Weak Banach--Tarski Paradox]
  Every closed ball in $\R^3$ is $\text{E}(3)$-paradoxical.
\end{corollary}
\begin{proof}
  We only need to show that $B(0,1)$ is $\text{E}(3)$-paradoxical. To do this, we start by showing that $B(0,1)\setminus \set{0}$ is $\text{SO}(3)$-paradoxical.\newline

  Since $S^{2}$ is $\text{SO}(3)$-paradoxical, there exists pairwise disjoint subsets $A_1,\dots,A_n,B_1,\dots,B_m\subset S^2$ and elements $g_1,\dots,g_n,h_1,\dots,h_m\in \text{SO}(3)$ such that
  \begin{align*}
    S^{2} &= \bigcup_{i=1}^{n}g_i\cdot A_i\\
          &= \bigcup_{j=1}^{m}h_j\cdot B_j.
  \end{align*}
  Define
  \begin{align*}
    A_i^{\ast} &= \set{tx\mid t\in (0,1], x\in A_i}\\
    B_j^{\ast} &= \set{ty\mid t\in (0,1], y\in B_j}.
  \end{align*}
  Then, $A_1^{\ast},\dots,A_n^{\ast},B_1^{\ast},\dots,B_m^{\ast}\subset B(0,1)\setminus \set{0}$ are pairwise disjoint, and
  \begin{align*}
    B(0,1)\setminus \set{0} &= \bigcup_{i=1}^{n}g_i\cdot A_i^{\ast}\\
                            &= \bigcup_{j=1}^{m}h_j\cdot B_j^{\ast}.
  \end{align*}
  Thus, we have established that $B(0,1)\setminus \set{0}$ is $\text{E}(3)$-paradoxical.\newline

  Now, we want to show that $B(0,1)\setminus \set{0}$ and $B(0,1)$ are $\text{E}(3)$-equidecomposable. Let $x\in B(0,1)\setminus \set{0}$, and let $\rho$ be a rotation through $x$ by a line not through the origin such that $\rho^{n}\cdot 0\neq \rho^{m}\cdot 0$ when $n\neq m$.\newline

  Let $D = \set{\rho^{n}\cdot 0\mid n\in \N}$. We can see that $\rho\cdot D = D\setminus \set{0}$, and that $D$ and $\rho\cdot D$ are $\text{E}(3)$-equidecomposable. Thus,
  \begin{align*}
    B(0,1) &= D\sqcup \left(B(0,1)\setminus D\right)\\
           &\sim_{\text{E}(3)}\left(\rho\cdot D\right) \sqcup \left(B(0,1)\setminus D\right)\\
           &= \left(D\setminus \set{0}\right)\sqcup \left(B\left(0,1\right)\setminus D\right)\\
           &= B\left(0,1\right)\setminus \set{0}.
  \end{align*}
  Therefore, $B(0,1)$ is $\text{E}(3)$-paradoxical.
\end{proof}
\section{The Strong Banach--Tarski Paradox}\label{sec:full_banach_tarski}%
In order to prove the general case of the Banach--Tarski paradox, we need one more piece of mathematical machinery.\newline

In Fact \ref{fact:equidecomposability_equivalence_relation}, we showed that the relation $A\sim_{G} B$ if and only if $A$ and $B$ are $G$-equidecomposable is an equivalence relation. Using the power of subsets, we may extend this to a preorder on any subsets $A$ and $B$ of $X$.
\begin{definition}
  Let $G$ act on a set $X$ with $A,B\subseteq X$. We write $A\preceq_{G}B$ if $A$ is equidecomposable with a subset of $B$.
\end{definition}
\begin{fact}
  The relation $\preceq_{G}$ is a reflexive and transitive relation.\label{fact:preorder}
\end{fact}
\begin{proof}
  To see reflexivity, we can see that since $A\sim_{G}A$, and $A\subseteq A$, $A\preceq_{G} A$.\newline

  To see transitivity, let $A\preceq_{G}B$ and $B\preceq_{G}C$. Then, there exist $g_1,\dots,g_n\in G$ such that $g_i\cdot A_i = B_{\alpha,i}$ for each $i$, where $A\sim_{G}B_{\alpha}\subseteq B$. Similarly, there exist $h_1,\dots,h_m\in G$ such that $h_j\cdot B_j= C_{\beta,j}$ for each $j$, where $B\sim_{G}C_{\beta}\subseteq C$.\newline

  We take a refinement of $B$ by taking intersections $B_{\alpha,ij} = B_{\alpha,i}\cap B_j$, with $i=1,\dots,n$ and $j = 1,\dots,m$. We define $C_{\beta,\alpha,ij} = h_j\cdot B_{\alpha,ij}$ for each $j = 1,\dots,m$. Then, $h_jg_i\cdot A_i = C_{\beta,\alpha,ij}$, meaning $A\sim_{G}C_{\beta,\alpha,ij}\subseteq C_{\beta}\subseteq C$, so $A\preceq_{G}C$.
\end{proof}

We know from Fact \ref{fact:bijections} that $A\preceq_{G}B$ implies the existence of a bijection $\phi\colon A\rightarrow B'\subseteq B$, meaning $\phi\colon A\hookrightarrow B$ is an injection. Similarly, if $B\preceq_{G}A$, then Fact \ref{fact:bijections} implies the existence of an injection $\psi\colon B\hookrightarrow A$.\newline

One may ask if an analogue of the Cantor--Schröder--Bernstein theorem exists in the case of the relation $\preceq_{G}$, implying that the preorder established in Fact \ref{fact:preorder} is indeed a partial order. The following theorem establishes this result.
\begin{theorem}
  Let $G$ act on $X$, and let $A,B\subseteq X$. If $A\preceq_{G}B$ and $B\preceq_{G}A$, then $A\sim_{G}B$.\label{thm:csb_for_equidecomposability}
\end{theorem}
\begin{proof}
  Let $B'\subseteq B$ with $A\sim_{G}B'$, and let $A'\subseteq A$ with $B\sim_{G}A'$. Then, we know from Fact \ref{fact:bijections} that there exist bijections $\phi\colon A\rightarrow B'$ and $\psi\colon B\rightarrow A'$.\newline

  Define $C_0 = A\setminus A'$, and $C_{n+1} = \psi\left(\phi\left(C_n\right)\right)$. We set
  \begin{align*}
    C &= \bigcup_{n\geq 0}C_{n}.
  \end{align*}
  Since $\psi^{-1}\left(\psi\left(\phi\left(C_n\right)\right)\right) = \phi\left(C_n\right)$, we have
  \begin{align*}
    \psi^{-1}\left(A\setminus C\right) &= B\setminus \phi(C).
  \end{align*}
  Having established in Fact \ref{fact:bijections} that for any subset of $C\subseteq A$, $C\sim_{G} \phi(C)$, we also see that $A\setminus C \sim_{G} B\setminus \phi(C)$.\newline

  Thus, we can see that
  \begin{align*}
    A &= \left(A\setminus C\right)\sqcup C\\
      &\sim_{G}\left(B\setminus \phi(C)\right)\sqcup \phi(C)\\
      &= B.
  \end{align*}
\end{proof}

Finally, we are able to prove Proposition \ref{prop:banachtarski}. We restate the proposition here, followed by its proof.
\begin{tcolorbox}[blanker,breakable,left=3mm,before skip=10pt, after skip=10pt, borderline west={1pt}{0pt}{blue!50!white},sharp corners,]
\banachtarski*
\end{tcolorbox}
\begin{proof}[Proof of Proposition \ref{prop:banachtarski}:]
  By symmetry, it is enough to show that $A\preceq_{\text{E}(3)} B$.\newline

  Since $A$ is bounded, there exists $r > 0$ such that $A\subseteq B(0,r)$.\newline

  Let $x_0\in B^{\circ}$. Then, there exists $\ve > 0$ such that $B\left(x_0,\ve\right) \subseteq B$.\newline

  Since $B(0,r)$ is compact (hence totally bounded), there are translations $g_1,\dots,g_n$ such that
  \begin{align*}
    B\left(0,r\right) \subseteq g_1\cdot B\left(x_0,\ve\right) \cup \cdots \cup g_n\cdot B\left(x_0,\ve\right).
  \end{align*}
  We select translations $h_1,\dots,h_n$ such that $h_j\cdot B\left(x_0,\ve\right) \cap h_k\cdot B\left(x_0,\ve\right) = \emptyset$ for $j\neq k$. We set
  \begin{align*}
    S &= \bigcup_{j=1}^{n}h_j\cdot B\left(x_0,\ve\right).
  \end{align*}
  Each $h_j\cdot B\left(x_0,\ve\right)\subseteq S$ is $\text{E}(3)$-equidecomposable with any arbitrary closed ball subset of $B\left(x_0,\ve\right)$, it is the case that $S\preceq B\left(x_0,\ve\right)$.\newline

  Thus, we have
  \begin{align*}
    A &\subseteq B\left(0,r\right)\\
      &\subseteq g_1\cdot B\left(x_0,\ve\right)\cup\cdots\cup b_n\cdot B\left(x_0,\ve\right)\\
      &\preceq S\\
      &\preceq B\left(x_0,\ve\right)\\
      &\preceq B.
  \end{align*}
\end{proof}


\chapter{Characterizations through Paradoxicality (or lack thereof): Tarski's Theorem}
Ultimately, the reason the Banach--Tarski paradox ``works'' is because the paradoxical group $F(a,b)$ is not amenable --- specifically, its paradoxicality closes off the possibility of amenability. Before we go further into the characterizations of amenability discussed in Chapters \ref{ch:invariant_states} and \ref{ch:folner_condition}, we will show that this statement reverses. Indeed, every amenable group is \textit{non}-paradoxical.
\begin{restatable}[Tarski's Theorem, {\cite[Theorem 0.2.1]{lectures_on_amenability}}]{theorem}{tarski}
  Let $G$ be a group that acts on a set $X$, and let $E \subseteq X$ be nonempty.\newline

  There is a finitely additive measure $\mu \colon P(X) \to [0, \infty]$ with $\mu(E) \in (0, \infty)$ and $\mu\left( t\cdot E \right) = \mu(E)$ for all $t\in G$ if and only if $E$ is not $G$-paradoxical.
\label{thm:tarski}
\end{restatable}
We can prove one of the directions of Tarski's theorem now.
\begin{proof}[Proof of the Forward Direction of Theorem \ref{thm:tarski}:]
  Let $E$ be $G$-paradoxical. Suppose toward contradiction that such a translation-invariant finitely additive $\nu$ existed with $\nu(E) \in (0,\infty)$.\newline

  Let $A_1,\dots,A_n,B_1,\dots,B_m\subseteq E$ be pairwise disjoint, and let $t_1,\dots,t_n,s_1,\dots,f_m\in G$ such that
  \begin{align*}
    E &= \bigsqcup_{i=1}^{n}t_i\cdot A_i\\
      &= \bigsqcup_{j=1}^{m}s_j\cdot B_j.
  \end{align*}
  Then, it would be the case that
  \begin{align*}
    \nu(E) &= \nu\left(\bigsqcup_{i=1}^{n}t_i\cdot A_i\right)\\
           &= \sum_{i=1}^{n}\nu\left(t_i\cdot A_i\right)\\
           &= \sum_{i=1}^{n}\nu\left(A_i\right),
  \end{align*}
  and
  \begin{align*}
    \nu(E) &= \sum_{j=1}^{m}\nu\left(B_j\right).
  \end{align*}
  However, this also yields
  \begin{align*}
    \nu\left(E\right) &= \nu\left(\left(\bigsqcup_{i=1}^{n}A_i\right)\sqcup \left(\bigsqcup_{j=1}^{m}B_j\right)\right)\\
                      &= \sum_{i=1}^{n}\nu\left(A_i\right) + \sum_{j=1}^{m}\nu\left(B_j\right)\\
                      &= \sum_{i=1}^{n}\nu\left(t_i\cdot A_i\right) + \sum_{j=1}^{m}\nu\left(x_j\cdot B_j\right)\\
                      &= \nu\left(E\right) + \nu\left(E\right)\\
                      &= 2\nu\left(E\right).
  \end{align*}
  implying that $\nu(E) = 0$ or $\nu(E) = \infty$.
\end{proof}
The opposite direction, unfortunately, will be significantly harder to prove. We will need to know some results from graph theory, understand the properties of the type semigroup of an action, and use some results on commutative semigroups to show the existence of a mean.
\section{A Little Bit of Graph Theory}
To prove the reverse direction of Tarski's theorem, we need to develop some machinery from graph theory that will allow us to prove that a certain semigroup we will construct in the next section satisfies the cancellation identity.\newline

We start by defining graphs and paths, before proving a special case of Hall's theorem, ultimately extending to the infinite case with König's theorem.
\begin{definition}[Graphs and Paths, {\cite[7]{lectures_on_amenability}}]
  A \textit{graph} is a triple $\left(V,E,\phi\right)$, with $V,E$ nonempty sets and $\phi\colon E\rightarrow P_{2}(V)$ a map from $E$ to the set of all unordered subset pairs of $V$.\newline

  For $e\in E$, if $\phi(e) = \set{v,w}$, then we say $v$ and $w$ are the \textit{endpoints} of $e$, and $e$ is \textit{incident} on $v$ and $w$.\newline

  A \textit{path} in $\left(V,E,\phi\right)$ is a finite sequence $\left(e_1,\dots,e_n\right)$ of edges, with a finite sequence of vertices $\left(v_0,\dots,v_n\right)$, such that $\phi\left(e_k\right) = \set{v_{k-1},v_k}$.\newline

  The \textit{degree} of a vertex, $\deg(v)$, is the number of edges incident on $v$.\newline

  We define the \textit{neighbors} of $S\subseteq V$ to be the set of all vertices $v\in V\setminus S$ such that $v$ is an endpoint to an edge incident on $S$. We denote this set $N(S)$.
\end{definition}

\begin{definition}[Bipartite Graphs and $k$-Regularity, {\cite[Definition 0.2.2]{lectures_on_amenability}}]
  Let $\left(V,E,\phi\right)$ be a graph, with $k\in \N$.
  \begin{enumerate}[(i)]
    \item If $\deg(v) = k$ for each $v\in V$, we say $\left(V,E,\phi\right)$ is \textit{$k$-regular}.
    \item If $V = X\sqcup Y$, with each edge in $E$ having one endpoint in $X$ and one endpoint in $Y$, then we say $V$ is \textit{bipartite}, and write $\left(X,Y,E,\phi\right)$.
  \end{enumerate}
\end{definition}

\begin{definition}[Perfect Matching, {\cite[Definition 0.2.3]{lectures_on_amenability}}]
  Let $\left(X,Y,E,\phi\right)$ be a bipartite graph. Let $A\subseteq X$ and $B\subseteq Y$. A \textit{perfect matching} of $A$ and $B$ is a subset $F\subseteq E$ with
  \begin{enumerate}[(i)]
    \item each element of $A\cup B$ is an endpoint of exactly one $f\in F$;
    \item all endpoints of edges in $F$ are in $A\cup B$.
  \end{enumerate}
\end{definition}
\begin{definition}[Hall Condition, {\cite[Exercise 0.2.2]{lectures_on_amenability}}]
  We say a bipartite graph $\left(X,Y,E,\phi\right)$ satisfies the \textit{Hall condition} on $X$ if, for all $S\subseteq X$, $\left\vert N(S) \right\vert \geq \left\vert S \right\vert$.\newline

  Equivalently, we say a (finite) collection of not necessarily distinct finite sets $\mathcal{X} = \set{X_i}_{i=1}^{n}$ satisfies the Hall condition if and only if for all subcollections $\mathcal{Y}_k = \set{X_{i_k}}_{k=1}^{m}$,
  \begin{align*}
    \left\vert \mathcal{Y}_k \right\vert \leq \left\vert \bigcup_{k=1}^{m}X_{i_k} \right\vert.
  \end{align*}
\end{definition}
\begin{remark}
These two formulations of the Hall condition are equivalent regarding an $X$-perfect matching.
\end{remark}
\begin{theorem}[Hall's Theorem for Finite $k$-Regular Bipartite Graphs, {\cite[Exercise 0.2.2]{lectures_on_amenability}}]
  Let $\left(X,Y,E,\phi\right)$ be a $k$-regular bipartite graph for some $k\in \N$, and let $V = X\sqcup E$ be finite. Then, there is a perfect matching of $X$ and $Y$.\label{thm:hall_finite}
\end{theorem}
\begin{proof}
  Note that since $\left\vert E \right\vert = k\left\vert K \right\vert = k\left\vert Y \right\vert$, it is the case that $\left\vert X \right\vert = \left\vert Y \right\vert$.\newline

  Let $M\subseteq V$ be any subset. We will show that $\left\vert N(M) \right\vert\geq \left\vert M \right\vert$ --- that is, $\left(X,Y,E,\phi\right)$ satisfies the Hall condition.\newline

  Let $M_X = M\cap X$ and $M_Y = M\cap Y$, where $M = M_X\sqcup M_Y$. Let $\left[M_X,N\left(M_X\right)\right]$ be the set of edges with endpoints in $M_X$ and $N\left(M_X\right)$, and $\left[M_Y,N\left(M_Y\right)\right]$ be the set of edges with endpoints in $M_Y$ and $N\left(M_Y\right)$. We also let $\left[X,N\left(M_X\right)\right]$ denote the set of edges with endpoints in $X$ and $N\left(M_X\right)$, and similarly, $\left[Y,N\left(M_Y\right)\right]$ is the set of edges with endpoints in $Y$ and $N\left(M_Y\right)$.\newline

  We can see that $\left[M_X,N\left(M_X\right)\right]\subseteq \left[X,N\left(M_X\right)\right]$, and similarly, $\left[M_Y,N\left(M_Y\right)\right]\subseteq \left[Y,N\left(M_Y\right)\right]$.\newline

  Since $\left\vert \left[M_X,N\left(M_X\right)\right] \right\vert = k\left\vert M_X \right\vert$ and $\left\vert \left[X,N\left(M_X\right)\right] \right\vert = k\left\vert N\left(M_X\right) \right\vert$, we have
  \begin{align*}
    \left\vert M_X \right\vert\leq \left\vert N\left(M_X\right) \right\vert,
  \end{align*}
  and similarly,
  \begin{align*}
    \left\vert M_Y \right\vert\leq \left\vert N\left(M_Y\right) \right\vert.
  \end{align*}
  Thus, $\left\vert M \right\vert\leq \left\vert N\left(M\right) \right\vert$.\newline

  We will now show that there is an $X$-perfect matching. Suppose toward contradiction that $F$ is a maximal perfect matching on $A\subseteq X$ and $B\subseteq Y$ with $X\setminus A \neq \emptyset$.\newline

  Then, there is $x\in X\setminus A$. Consider $Z\subseteq V$ consisting of all vertices $z$ such that there exists a $F$-alternating path $\left(e_1,\dots,e_n\right)$ between $z\in Z$ and $x$.\newline

  It cannot be the case that $Z\cap Y$ is empty, since the number of neighbors of $x$ is greater than or equal to $1$ by the Hall condition --- if it were the case that $Z\cap Y$ were empty, we could add an edge to $F$ consisting of $x$ and one element of $N\left(\set{x}\right)$, which would contradict the maximality of $F$.\newline

  Consider a path traversing along $Z$, $\left(e_1,\dots,e_n\right)$. It must be the case that $e_n\in F$, or else we would be able to ``flip'' the matching $F$ by exchanging $e_{i}$ with $e_{i+1}$ for $e_i\in F$, which would contradict the maximality of $F$ yet again. Thus, every element of $Z\cap Y$ is satisfied by $F$, so $Z\cap Y\subseteq B$.\newline

  Since each element in $Z\cap Y$ is paired with exactly one element of $Z\cap X$ (with one left over), it is the case that $\left\vert Z\cap X \right\vert = \left\vert Z\cap Y \right\vert + 1$.\newline

  Suppose toward contradiction that there exists $y\in N\left(Z\cap X\right)$ with $y\notin Z\cap Y$. Then, there exists $v\in Z\cap X$ and $e\in E$ such that $\phi(e) = \set{v,y}$. However, this means $v$ is connected via a path to $x$, meaning $y\in Z$, so $y\in Z\cap Y$. Thus, we must have $N\left(Z\cap X\right) = Z\cap Y$.\newline

  Therefore,
  \begin{align*}
    \left\vert Z\cap X \right\vert &= \left\vert Z\cap Y \right\vert + 1\\
                                   &= \left\vert N\left(Z\cap X\right) \right\vert + 1,
  \end{align*}
  which contradicts the fact that $\left(X,Y,E,\phi\right)$ satisfies the Hall condition. Therefore, $A = X$.\newline

  By symmetry, there is a perfect matching of $X$ and $Y$ in $\left(X,Y,E,\phi\right)$.
\end{proof}
\begin{remark}
  An equivalent formulation to Hall's theorem states that there is a system of distinct representatives on the collection $\mathcal{X} = \set{X_k}_{k=1}^{n}$, which is a set $\set{x_{k}}_{k=1}^{n}$ such that $x_{k}\in X_{k}$ and $x_{i}\neq x_j$ for $i\neq j$.\newline

  This implies the existence of an injection $f\colon \mathcal{X}\hookrightarrow \bigcup_{k=1}^{n}X_{k}$, such that $f\left(X_k\right) \in X_k$.
\end{remark}
%\begin{definition}[Choice Function]
%  Let $\mathcal{X} = \set{X_{i}}_{i\in I}$ be a collection of sets. A function $f\colon \mathcal{X}\rightarrow \bigcup_{i\in I}X_i$ is called a choice function if, for each $i\in I$, $f\left(X_{i}\right)\in X{i}$.\newline
%
%  We also say $f\colon \mathcal{X}\rightarrow \bigcup_{i\in I}X_i$ is a choice function if $f\in \prod_{i\in I}X_i$.
%\end{definition}
%
%\begin{theorem}[Tychonoff's Theorem]
%  If $\set{X_{i}}_{i\in I}$ is a family of compact topological spaces
%\end{theorem}
\begin{theorem}[Infinite Hall's Theorem, {\cite{marshall_hall_thm}}]
  Let $\mathcal{G} = \set{X_{i}}_{i\in I}$ be a collection of (not necessarily distinct) finite sets. If, for every finite subcollection $\mathcal{Y} = \set{X_{i_k}}_{k=1}^{n}$,
  \begin{align*}
    n\leq \left\vert \bigcup_{k=1}^{n}X_{i_k} \right\vert,
  \end{align*}
  then there is a choice function on $G$.
\end{theorem}
\begin{proof}
  We endow each $X_i\in \set{X_{i}}_{i\in I}$ with the discrete topology. Since each $X_i$ is finite, each $X_i$ is compact.\newline

  Thus, by Tychonoff's theorem, it is the case that $\prod_{i\in I}X_{i}$ is compact.\newline

  For every finite subset $Y\subseteq \mathcal{G}$, we define
  \begin{align*}
    S_Y &= \set{\left.f\in \prod_{i\in I}X_i\right|f\vert_{Y}\text{ is injective}}.
  \end{align*}
  The injectivity of $f\vert_{Y}$ is equivalent to the existence of a system of distinct representatives on $Y$. Since $Y$ satisfies the Hall condition, each $S_{Y}$ is nonempty. Additionally, for any net of functions $f_{\alpha}\in S_{Y}$ with $\lim_{\alpha}f_{\alpha} = f$, it is the case that $f_{\alpha}\vert_{Y}$ is injective, so $f\vert_{Y}$ is injective, meaning $S_{Y}$ is closed.\newline

  We define $F = \set{S_{Y}| Y\subseteq \mathcal{G}\text{ finite}}$. For finite $Y_{1},Y_{2}\subseteq \mathcal{G}$, every system of distinct representatives in $Y_1\cup Y_2$ is necessarily a system of distinct representatives on $Y_1$ and a system of distinct representatives on $Y_{2}$, meaning $S_{Y_1\cup Y_2}\subseteq S_{Y_1}\cap S_{Y_2}$. Thus, $F$ has the finite intersection property.\newline

  Since $\prod_{i\in I}X_i$ is compact, $\bigcap F$ is nonempty, where the intersection is taken over all finite subsets of $\mathcal{G}$. For any $f\in \bigcap F$, $f$ is necessarily a choice function.
\end{proof}
\begin{remark}
  This is equivalent to the existence of an injection $f\colon \mathcal{G}\hookrightarrow \bigcup_{i\in I}X_i$.
\end{remark}
We will use this infinite case of Hall's theorem to prove König's theorem. 
\begin{theorem}[König's Theorem, {\cite[Theorem 0.2.4]{lectures_on_amenability}}]
  Let $\left(X,Y,E,\phi\right)$ be a $k$-regular bipartite graph (not necessarily finite). Then, there is a perfect matching of $X$ and $Y$.\label{thm:konig}
\end{theorem}
\begin{proof}
  If $k = 1$, it is clear that there is a perfect matching in $\left(X,Y,E,\phi\right)$ consisting of the edges in $\left(X,Y,E,\phi\right)$.\newline

  Let $k\geq 2$. Since any finite subset of $X$ satisfies the Hall condition, as displayed in the proof of Theorem \ref{thm:hall_finite}, there is some $X$-perfect matching in $\left(X,Y,E,\phi\right)$. We call this $X$-perfect matching $F$. There is an injection $f\colon X\hookrightarrow Y$ following the edges in $F$.\newline

  Similarly, since any finite subset of $Y$ satisfies the Hall condition, there is some $Y$-perfect matching in $\left(X,Y,E,\phi\right)$. We call this $Y$-perfect matching $G$. There is an injection $g\colon Y\hookrightarrow X$ following the edges of $G$.\break

  Consider the subgraph $\left(X,Y,F\cup G,\phi|_{F\cup G}\right)$. The injections $f$ and $g$ still hold in this graph. By the Cantor--Schröder--Bernstein theorem, there is a bijection $h\colon X\rightarrow Y$ in $\left(X,Y,F\cup G,\phi|_{F\cup G}\right)$, which is equivalent to the existence of a perfect matching of $X$ and $Y$.
\end{proof}
\section{Type Semigroups}%
\begin{definition}[{\cite[Definition 0.2.5]{lectures_on_amenability}}]\label{def:xstar_gstar}
  Let $G$ be a group that acts on a set $X$.
  \begin{enumerate}[(i)]
    \item We define $X^{\ast} = X\times \Z_{\geq }$, and
      \begin{align*}
        G^{\ast} &= \set{\left(g,\pi\right)| g\in G,\pi\in\sym\left(\Z_{\geq 0}\right)}.
      \end{align*}
    \item If $A\subseteq X^{\ast}$, the values of $n$ for which there is an element of $A$ whose second coordinate is $n$ are called the \textit{levels} of $A$.
  \end{enumerate}
\end{definition}
\begin{fact}[{\cite[Exercise 0.2.4]{lectures_on_amenability}}]\label{fact:type_semigroup_equidecomposability}
  If $E_1,E_2\subseteq X$, then $E_{1}\sim_{G}E_2$ if and only if $E_1\times \set{n}\sim_{G^{\ast}}E_{2}\times \set{m}$ for all $m,n\in \Z_{\geq 0}.$
\end{fact}
\begin{proof}
  Let $E_{1}\sim_{G}E_2$. Then, there exist pairwise disjoint $A_1,\dots,A_n\subset E_1$, pairwise disjoint $B_1,\dots,B_n\subset E_2$, and elements $g_1,\dots,g_n\in G$ such that $g_i\cdot A_i = B_i$. We select the permutation $\pi_{i}\in \sym\left(\Z_{\geq 0}\right)$ such that $\pi_{i}(n) = m$ and $\pi_i(m) = n$ for each $i$. Then,
  \begin{align*}
    \left(g_i,\pi_i\right)\cdot \left(A_{i}\cdot \set{n}\right) &= B_{i}\cdot \set{m}.
  \end{align*}

  Similarly, if $E_{1}\times \set{n} \sim_{G^{\ast}}E_2\times \set{m}$, then of the pairwise disjoint subsets
  \begin{align*}
    A_1\times \set{n},\dots,A_n\times \set{n}\subset E_1\times \set{n}
  \end{align*}
  and
  \begin{align*}
    B_1\times\set{m},\dots,B_n\times\set{m}\subset E_2\times \set{m},
  \end{align*}
  we set $A_1,\dots,A_n\subset E_1$ and $B_1,\dots,B_n\subset E_2$. Similarly, for
  \begin{align*}
    \left(g_1,\pi_1\right),\dots,\left(g_n,\pi_n\right)\in G^{\ast}
    \intertext{such that}
    \left(g_i,\pi_i\right)\cdot A_i\times \set{n} = B_i\times\set{m},
  \end{align*}
  we select $g_1,\dots,g_n\in G$. Then, by definition,
  \begin{align*}
    g_i\cdot A_i = B_i
  \end{align*}
  for each $i$. Thus, $E_1\sim_{G}E_2$.
\end{proof}

\begin{definition}[{\cite[Definition 0.2.6]{lectures_on_amenability}}]\label{def:type_semigroup}
  Let $G$ be a group that acts on $X$, and let $G^{\ast}$, $X^{\ast}$ be defined as in \ref{def:xstar_gstar}.
  \begin{enumerate}[(i)]
    \item A set $A\subseteq X^{\ast}$ is said to be \textit{bounded} if it has finitely many levels.
    \item If $A\subseteq X^{\ast}$ is bounded, the equivalence class of $A$ with respect to $G^{\ast}$-equidecomposability is called the \textit{type} of $A$, which is denoted $\left[A\right]$.
    \item If $E\subseteq X$, we write $\left[E\right] = \left[E\times \set{0}\right]$.
    \item Let $A,B\subseteq X^{\ast}$ be bounded with $k\in \Z_{\geq 0}$ such that for
      \begin{align*}
        B'= \set{\left(b,n+k\right)| \left(b,n\right)\in B},
      \end{align*}
      we have $B'\cap A = \emptyset$. Then, $\left[A\right] + \left[B\right] = \left[A\sqcup B'\right]$. Note that $\left[B'\right] = \left[B\right]$.
    \item We define
      \begin{align*}
        \mathcal{S} &= \set{\left[A\right]| A\subseteq X^{\ast}\text{ bounded}}
      \end{align*}
      under the addition defined in (iv) to be the \textit{type semigroup} of the action of $G$ on $X$.
  \end{enumerate}
\end{definition}

\begin{fact}[{\cite[Exercise 0.2.5]{lectures_on_amenability}}]
  Addition is well-defined in $\left(\mathcal{S},+\right)$, and $\left(\mathcal{S},+\right)$ is a well-defined commutative semigroup with identity $\left[\emptyset\right]$.\label{fact:type_semigroup_well_defined}
\end{fact}
\begin{proof}
  To show that addition is well-defined, we let $\left[A_1\right] = \left[A_2\right]$, and $\left[B_1\right] = \left[B_2\right]$. Without loss of generality, $A_1\cap B_1 = \emptyset$ and $A_2\cap B_2 = \emptyset$.\newline

  By the definition of the type, $A_1\sim_{G^{\ast}}A_2$ and $B_1\sim_{G^{\ast}}B_2$, meaning
  \begin{align*}
    A_1\sqcup B_1\sim_{G^{\ast}} A_2\sqcup B_2,
  \end{align*}
  so
  \begin{align*}
    \left[A_1\right] + \left[B_1\right] &= \left[A_1\sqcup B_1\right]\\
                                        &= \left[A_2\sqcup B_2\right]\\
                                        &= \left[A_2\right] + \left[A_2\right],
  \end{align*}
  meaning addition is well-defined.\newline

  Since addition is well-defined, and $A\sqcup B = B\sqcup A$, we can see that addition is also commutative. We also have
  \begin{align*}
    \left[A\right] + \left[\emptyset\right] &= \left[A\sqcup \emptyset\right]\\
                                            &= \left[A\right],
  \end{align*}
  so $\left[\emptyset\right]$ is the identity on $\mathcal{S}$.\newline

  Finally, since for any $\left[A\right],\left[B\right]\in \mathcal{S}$, $A$ and $B$ have finitely many levels, it is the case that $A\cup B$ has finitely many levels for any $A$ and $B$, so $\left[A\right] + \left[B\right] \in \mathcal{S}$. 
\end{proof}

\begin{definition}[{\cite[10]{lectures_on_amenability}}]
  For any commutative semigroup $\mathcal{S}$ with $\alpha \in S$ and $n\in \N$, we define
  \begin{align*}
    n\alpha = \underbrace{\alpha + \cdots + \alpha}_{\text{$n$ times}}
  \end{align*}
\end{definition}
\begin{definition}[{\cite[10]{lectures_on_amenability}}]
  For $\alpha,\beta \in \mathcal{S}$, if there exists $\gamma \in \mathcal{S}$ such that $\alpha + \gamma = \beta$, we write $\alpha \leq \beta$.
\end{definition}
\begin{fact}[{\cite[Exercise 0.2.7]{lectures_on_amenability}}]\label{fact:type_semigroup_paradoxicality}%\label{fact:type_semigroup_criterion_paradoxicality}
  If $G$ is a group acting on $X$ with corresponding type semigroup $\mathcal{S}$, then the following are true.
  \begin{enumerate}[(i)]
    \item If $\alpha,\beta\in \mathcal{S}$ with $\alpha \leq \beta$ and $\beta \leq \alpha$, then $\alpha = \beta$.
    \item A set $E\subseteq X$ is $G$-paradoxical if and only if $\left[E\right] = 2\left[E\right]$.
  \end{enumerate}
\end{fact}
\begin{proof}
  Let $G$ act on $X$, and let $\mathcal{S}$ be the corresponding type semigroup.
  \begin{enumerate}[(i)]
    \item If $\left[A\right]\leq \left[B\right]$, then there exists $C_1\in \mathcal{S}$ such that $\left[A\right] + \left[C_1\right] = \left[B\right]$. Without loss of generality, $C_1\cap A= \emptyset$, meaning $\left[B\right] = \left[A\sqcup C_1\right]$. Thus, $A\sqcup C_1 \sim_{G^{\ast}} B$, meaning $B\preceq_{G^{\ast}}A$.\newline

      Similarly, if $\left[B\right]\leq \left[A\right]$, then $B\preceq_{G^{\ast}}A$. By Theorem \ref{thm:csb_for_equidecomposability}, it is thus the case that $A\sim_{G^{\ast}}B$.
    \item Let $E$ be $G$-paradoxical. \newline

      Then, $E\sim_{G}\bigsqcup_{i=1}^{n}A_i$ and $E \sim_{G}\bigsqcup_{j=1}^{m}B_j$ for pairwise disjoint subsets $A_1,\dots,A_n,B_1,\dots,B_m\subset E$. Thus, we have
      \begin{align*}
        \left[E\right] &= \left[\left(\bigsqcup_{i=1}^{n}A_i\right)\sqcup \left(\bigsqcup_{j=1}^{m}B_j\right)\right]\\
                       &= \left[\bigsqcup_{i=1}^{n}A_i\right] + \left[\bigsqcup_{j=1}^{m}B_j\right]\\
                       &= 2\left[E\right].
      \end{align*}
      Similarly, if $\left[E\right] = 2\left[E\right]$, then there exist $A$ and $B$ such that
      \begin{align*}
        \left[E\right] &= \left[A\right] + \left[B\right]\\
                       &= \left[A\sqcup B\right],
      \end{align*}
      meaning $A$ and $B$ are each $G$-equidecomposable with $E$, so $E$ is $G$-paradoxical.
  \end{enumerate}
\end{proof}
We can now prove the cancellation identity, which we will be useful as we construct our desired finitely additive measure.
\begin{theorem}[Cancellation Identity on $\mathcal{S}$, {\cite[Theorem 0.2.7]{lectures_on_amenability}}]
  Let $\mathcal{S}$ be the type semigroup for some group action, and let $\alpha,\beta\in \mathcal{S}$, $n\in \N$ such that $n\alpha = n\beta$. Then, $\alpha = \beta$.
\end{theorem}
\begin{proof}
  Let $n\alpha = n\beta$. Then, there are two disjoint bounded subsets $E,E'\subseteq X^{\ast}$ with $E\sim_{G^{\ast}}E'$, and pairwise disjoint subsets $A_1,\dots,A_n\subseteq E$, $B_1,\dots,B_n\subseteq E'$ such that
  \begin{itemize}
    \item $E = A_1\cup\cdots\cup A_n$, $E' = B_1\cup\cdots\cup B_n$
    \item $\left[ A_j \right] = \alpha$ and $\left[B_j\right] = \beta$ for each $j=1,\dots,n$.
  \end{itemize}
  Let $\chi\colon E\rightarrow E'$ be a bijection as in Fact \ref{fact:bijections}, with $\phi_j\colon A_1\rightarrow A_j$, $\psi_j\colon B_1\rightarrow B_j$ also being bijections as in Fact \ref{fact:bijections}; here we define $\phi_1$ and $\psi_1$ to be the identity map.\newline

  For each $a\in A_1$ and $b\in B_1$, we define
  \begin{align*}
    \overline{a} &= \set{a,\phi_2(a),\dots,\phi_n(a)}\\
    \overline{b} &= \set{b,\psi_2(b),\dots,\psi_n(b)}.
  \end{align*}
  We construct a graph by letting $X = \set{\overline{a}| a\in A_1}$ and $Y = \set{\overline{b}| b\in B_1}$, and, for each $j$, define edges $\set{\overline{a},\overline{b}}$ if $\chi\left(\phi_j(a)\right)\in \overline{b}$.\newline

  Since $\chi$ is a bijection, for each $j=1,\dots,n$, $\chi\left(\phi_j(a)\right)$ must be an element of $B_k$ for some $k$, and since $\set{B_k}_{k=1}^{n}$ are disjoint, $\chi\left(\phi_j(a)\right)$ is an element of exactly one $B_k$. Thus, the graph is $n$-regular.\newline

  By Theorem \ref{thm:konig}, this graph has a perfect matching $F$. As a result, for each $\overline{a}\in X$, there is a unique $\overline{b}\in Y$ and a unique edge $\set{\overline{a},\overline{b}}\in F$ such that $\chi\left(\phi_j(a)\right) = \psi_k(b)$ for some $j,k\in \set{1,\dots,n}$.\newline

  We define
  \begin{align*}
    C_{j,k} &= \set{a\in A_1| \set{\overline{a},\overline{b}}\in F,~\chi\left(\phi_j(a)\right) = \psi_k(b)}\\
    D_{j,k} &= \set{b\in B_1| \set{\overline{a},\overline{b}}\in F,~\chi\left(\phi_j(a)\right) = \psi_k(b)}.
  \end{align*}
  Therefore, we must have $\psi_{k}^{-1}\circ \chi\circ \phi_j$ is a bijection from $C_{j,k}$ to $D_{j,k}$, so $C_{j,k}\sim_{G^{\ast}}D_{j,k}$.\newline

  Since $C_{j,k}$ and $D_{j,k}$ are partitions of $A_1$ and $B_1$ respectively, it follows that $A_1\sim_{G^{\ast}}B_1$, so $\alpha = \beta$.
\end{proof}
\begin{corollary}[{\cite[Corollary 0.2.8]{lectures_on_amenability}}]\label{cor:non_paradoxicality}
  Let $\mathcal{S}$ be the type semigroup of some group action, and let $\alpha\in \mathcal{S}$ and $n\in \N$ such that $\left(n+1\right)\alpha \leq n\alpha$. Then, $\alpha = 2\alpha$.\label{corollary:paradoxical_elements}
\end{corollary}
\begin{proof}
  We have
  \begin{align*}
    2\alpha + n\alpha &= \left(n+1\right)\alpha + \alpha\\
                      &\leq n\alpha + \alpha\\
                      &= \left(n+1\right)\alpha\\
                      &\leq n\alpha.
  \end{align*}
  Inductively repeating this argument, we get $n\alpha \geq 2n\alpha$; since $n\alpha \leq 2n\alpha$ by definition, we must have $n\alpha = 2n\alpha$, so $\alpha = 2\alpha$.
\end{proof}
\begin{remark}
  We will call such an $\alpha$ a paradoxical element.
\end{remark}
\section{Two Results on Commutative Semigroups}%
Now that we are aware of paradoxical elements and the relationship between $G$-paradoxicality and the properties of the particular elements of the type semigroup (Fact \ref{fact:type_semigroup_paradoxicality}), we will now relate these properties to finitely additive measures of sets by using the following lemma and theorem.
\begin{lemma}[{\cite[Lemma 0.2.9]{lectures_on_amenability}}]\label{lemma:set_function_existence}
  Let $\mathcal{S}$ be a commutative semigroup, with $\mathcal{S}_0\subseteq \mathcal{S}$ finite, and $\epsilon\in \mathcal{S}_0$ satisfying the following assumptions:
  \begin{enumerate}[(a)]
    \item $\left(n+1\right)\epsilon \nleq n\epsilon$ for all $n\in \N$ (i.e., that $\epsilon$ is non-paradoxical);
    \item for each $\alpha\in \mathcal{S}$, there is $n\in \N$ such that $\alpha \leq n\epsilon$.
  \end{enumerate}
  Then, there is a set function $\nu\colon \mathcal{S}_0\rightarrow [0,\infty]$ that satisfies the following conditions:
  \begin{enumerate}[(i)]
    \item $\nu\left(\epsilon\right) = 1$;
    \item for $\alpha_1,\dots,\alpha_n,\beta_1,\dots,\beta_m\in \mathcal{S}_0$ with $\alpha_1+\cdots+\alpha_n\leq \beta_1+\cdots\beta_m$,
      \begin{align*}
        \sum_{j=1}^{n}\nu\left(\alpha_j\right) \leq \sum_{j=1}^{m}\nu\left(\beta_j\right).
      \end{align*}
  \end{enumerate}
\end{lemma}
\begin{proof}
  We will prove this result by inducting on the cardinality of $\mathcal{S}_0$.\newline

  We start with $\left\vert \mathcal{S}_0 \right\vert = 1$. In that case, we define $\nu\left(\epsilon\right) = 1$, satisfying condition (i). To satisfy condition (ii), we see that for $n,m\in \N$ with $n\epsilon \leq m\epsilon$, if $n \geq m+1$, then $\left(m+1\right)\epsilon \leq n\epsilon \leq m\epsilon$, implying that $\epsilon = 2\epsilon$, which contradicts assumption (a).\newline

  Let $\alpha_0\in \mathcal{S}_0\setminus\set{\epsilon}$. The induction hypothesis says there is a set function satisfying conditions (i) and (ii), $\nu\colon \mathcal{S}_0\setminus \set{\alpha_0}\rightarrow [0,\infty]$.\newline

  For $r\in \N$, there are $\gamma_1,\dots,\gamma_p,\delta_1,\dots,\delta_q\in \mathcal{S}\setminus \set{\alpha_0}$ such that
  \begin{align*}
    \delta_{1} + \cdots + \delta_q + r\alpha_0 \leq \gamma_1 + \cdots + \gamma_p.\label{set_function_id1}\tag*{(\textdagger)}
  \end{align*}
  Consider the set $N$ defined as follows:
  \begin{align*}
    N &= \set{\frac{1}{r}\left(\sum_{j=1}^{p}\nu\left(\gamma_j\right) - \sum_{j=1}^{q}\nu\left(\delta_j\right)\right)| \text{$\gamma_j,\delta_j$ satisfy \ref{set_function_id1}}}. \label{set_function_N}\tag*{($\ddag$)}
  \end{align*}
  We define the extension of $\nu$ as follows:
  \begin{align*}
    \nu\left(\alpha_0\right) &= \inf N.
  \end{align*}
  This infimum is well-defined since, by assumption (b), there is some $n\in \N$ such that $\alpha_0 \leq n\epsilon$, and $\nu\left(\epsilon\right)$ is defined.\newline

  Now, we must show that this extension of $\nu$ satisfies condition (ii).\newline

  Let $\alpha_1,\dots,\alpha_n,\beta_1,\dots,\beta_m\in \mathcal{S}_0\setminus \set{\alpha_0}$ and $s,t\in \Z_{\geq 0}$ such that
  \begin{align*}
    \alpha_1 + \cdots + \alpha_n + s\alpha_0 \leq \beta_1 + \cdots + \beta_m + t\alpha_0.\label{set_function_conditionii}\tag*{(\textasteriskcentered)}
  \end{align*}
  We will verify condition (ii) in the three following cases.
  \begin{description}[font=\normalfont\scshape,leftmargin=0cm]
    \item[Case 0:] If $s = t = 0$, then the induction hypothesis states that \ref{set_function_conditionii} satisfies condition (ii).
    \item[Case 1:] Let $s = 0$ and $t > 0$. We want to show that
      \begin{align*}
        \sum_{j=1}^{n}\nu\left(\alpha_j\right) \leq t\nu\left(\alpha_0\right) + \sum_{j=1}^{m}\nu\left(\beta_j\right),
      \end{align*}
      which implies that
      \begin{align*}
        \nu\left(\alpha_0\right) \geq \frac{1}{t}\left(\sum_{j=1}^{n}\nu\left(\alpha_j\right) - \sum_{j=1}^{m}\nu\left(\beta_j\right)\right).
      \end{align*}
      By the definition of infimum, it suffices to show that for $r\in \N$ and $\delta_1,\dots,\delta_q,\gamma_1,\dots,\gamma_p\in \mathcal{S}\setminus \set{\alpha_0}$ satisfying \ref{set_function_id1}, it is the case that
      \begin{align*}
        \frac{1}{r}\left(\sum_{j=1}^{p}\nu\left(\gamma_j\right)-\sum_{j=1}^{q}\nu\left(\delta_j\right)\right) \geq \frac{1}{t}\left(\sum_{j=1}^{n}\nu\left(\alpha_j\right) - \sum_{j=1}^{m}\nu\left(\beta_j\right)\right).
      \end{align*}
      Multiplying \ref{set_function_conditionii} by $r$ on both sides, and adding $t\delta_1 + \cdots + t\delta_q$ to both sides, we have
      \begin{align*}
        r\alpha_1 + \cdots + r\alpha_n + t\delta_1 + \cdots + t\delta_q \leq r\beta_1 + \cdots + r\beta_m + t\left(r\alpha_0\right) + t\delta_1 + \cdots + t\delta_q.
      \end{align*}
      Substituting \ref{set_function_id1}, we find
      \begin{align*}
        r\alpha_1 + \cdots + r\alpha_n + t\delta_1 + \cdots + t\delta_q \leq r\beta_1 + \cdots + r\beta_m + t\gamma_1 + \cdots + t\gamma_p.
      \end{align*}
      Applying the induction hypothesis, we have
      \begin{align*}
        r\sum_{j=1}^{n}\nu\left(\alpha_j\right) + t\sum_{j=1}^{q}\nu\left(\delta_j\right) \leq r\sum_{j=1}^{m}\nu\left(\beta_j\right) + t\sum_{j=1}^{p}\nu\left(\gamma_j\right),
      \end{align*}
      yielding
      \begin{align*}
        \frac{1}{r}\left(\sum_{j=1}^{p}\nu\left(\gamma_j\right) - \sum_{j=1}^{q}\nu\left(\delta_j\right)\right) \geq \frac{1}{t}\left(\sum_{j=1}^{n}\nu\left(\alpha_j\right) - \sum_{j=1}^{m}\nu\left(\beta_j\right)\right).
      \end{align*}
    \item[Case 2:] Let $s > 0$. For $z_1,\dots,z_t\in N$ \ref{set_function_N}, we need to show that
      \begin{align*}
        s\nu\left(\alpha_0\right) + \sum_{j=1}^{n}\nu\left(\alpha_j\right) \leq z_1 + \cdots + z_t + \sum_{j=1}^{n}\nu\left(\beta_j\right).
      \end{align*}
      Without loss of generality, we can set $z_1,\dots,z_n = z$, as for each $z\in N$, $z \geq \nu\left(\alpha_0\right)$.\newline

      As in Case 1, we multiply \ref{set_function_conditionii} by $r$, add $t\delta_{1} + \cdots + t\delta_q$ to both sides, and substitute with \ref{set_function_id1}, yielding
      \begin{align*}
        r\alpha_1 + \cdots + r\alpha_n + rs\alpha_0 + t\delta_1 + \cdots + t\delta_q &\leq r\beta_1 + \cdots + r\beta_m + t\left(r\alpha_0\right) + t\delta_1 + \cdots + t\delta_q\\
        r\alpha_1 + \cdots + r\alpha_n + t\delta_1 + \cdots + t\delta_q + rs\alpha_0 &\leq r\beta_1 + \cdots + r\beta_m + t\gamma_1 + \cdots + t\gamma_p.
      \end{align*}
      Defining
      \begin{align*}
        z &= \frac{1}{r}\left(\sum_{j=1}^{p}\nu\left(\gamma_j\right) - \sum_{j=1}^{q}\nu\left(\delta_j\right)\right),
      \end{align*}
      we get
      \begin{align*}
        s\nu\left(\alpha_0\right) + \sum_{j=1}^{n}\nu\left(\alpha_j\right) &\leq \sum_{j=1}^{n}\nu\left(\alpha_j\right) + \frac{s}{sr}\left(r\sum_{j=1}^{m}\nu\left(\beta_j\right) - r\sum_{j=1}^{n}\nu\left(\alpha_j\right) + t\sum_{j=1}^{p}\nu\left(\gamma_j\right) - t\sum_{j=1}^{q}\nu\left(\delta_j\right)\right)\\
                                                                           &= tz + \sum_{j=1}^{m}\nu\left(\beta_j\right).
      \end{align*}
  \end{description}
  Thus, we have shown that $\nu$ extends in a manner that satisfies conditions (i) and (ii).
\end{proof}

We can ``upgrade'' our finitely additive set function to a semigroup homomorphism as follows.
\begin{theorem}[{\cite[Theorem 0.2.10]{lectures_on_amenability}}]\label{thm:homomorphism_existence}
  Let $\left(\mathcal{S},+\right)$ be a commutative semigroup with identity element $0$, and let $\epsilon\in \mathcal{S}$. Then, the following are equivalent:
  \begin{enumerate}[(i)]
    \item $\left(n+1\right)\epsilon \leq n\epsilon$ for all $n\in \N$;
    \item there is a semigroup homomorphism $\nu\colon \left(\mathcal{S},+\right)\rightarrow \left([0,\infty],+\right)$ such that $\nu(\epsilon) = 1$.
  \end{enumerate}
\end{theorem}
\begin{proof}
  To show that (ii) implies (i), we let $\nu\colon \left(\mathcal{S},+\right)\rightarrow \left([0,\infty],+\right)$ be a semigroup homomorphism with $\nu\left(\epsilon\right) = 1$. Then,
  \begin{align*}
    \nu\left(\left(n+1\right)\epsilon\right) &= \left(n+1\right)\nu\left(\epsilon\right)\\
                                             &= n+1\\
                                             &> n\\
                                             &= n\nu\left(\epsilon\right)\\
                                             &= \nu\left(n\epsilon\right),
  \end{align*}
  meaning that $\left(n+1\right)\epsilon \nleq n\epsilon$.\newline

  To show that (i) implies (ii), we suppose that for each $\alpha \in \mathcal{S}$, there is $n\in \N$ such that $\alpha \leq n\epsilon$ --- for any such $\alpha$ for which this is not the case, we define $\nu\left(\alpha\right) = \infty$.\newline

  For a finite subset $\mathcal{S}_0 \subseteq \mathcal{S}$ with $\epsilon\in \mathcal{S}_0$, we define $M_{\mathcal{S}_0}$ to be the set of all $\kappa\colon \mathcal{S}\rightarrow [0,\infty]$ such that
  \begin{itemize}
    \item $\kappa\left(\epsilon\right) = 1$;
    \item $\kappa\left(\alpha + \beta\right) = \kappa\left(\alpha\right) + \kappa\left(\beta\right)$ for $\alpha,\beta,\alpha + \beta\in \mathcal{S}_0$.
  \end{itemize}
  Since we assume condition (i), we know that such a $\kappa$ with $\kappa\left(\epsilon\right) = 1$ exists. Additionally, since
  \begin{align*}
    \alpha + \beta \leq \left(\alpha + \beta\right)
  \end{align*}
  and
  \begin{align*}
    \left(\alpha + \beta\right) \leq \alpha + \beta,
  \end{align*}
  it is the case that
  \begin{align*}
    \kappa\left(\alpha + \beta\right) \leq \kappa\left(\alpha\right) + \kappa\left(\beta\right) \leq \kappa\left(\alpha + \beta\right),
  \end{align*}
  meaning $\kappa\left(\alpha + \beta\right) = \kappa\left(\alpha\right) + \kappa\left(\beta\right)$. Thus, $M_{\mathcal{S}_0}$ is nonempty. It is also the case that $M_{\mathcal{S}_0}$ is closed, since any net of functions $\kappa_{p}\colon \mathcal{S}\rightarrow [0,\infty]$ with $\kappa_{p}\left(\epsilon\right) = 1$ and $\kappa_{p}\left(\alpha + \beta\right) = \kappa_{p}\left(\alpha\right) + \kappa_{p}\left(\beta\right)$ will necessarily satisfy these conditions in the limit.\newline

  We let $\left[0,\infty\right]^{\mathcal{S}} = \set{\kappa| \kappa:\mathcal{S}\rightarrow [0,\infty]}$ be equipped with the product topology. By Tychonoff's theorem, $\left[0,\infty\right]^{\mathcal{S}}$ is compact.\newline

  Furthermore, for any finite subcollection $\mathcal{S}_1,\dots,\mathcal{S}_n$, it is the case that
  \begin{align*}
    M_{\mathcal{S}_1\cup\cdots\cup \mathcal{S}_n} \subseteq M_{\mathcal{S}_1} \cap \cdots \cap M_{\mathcal{S}_n},
  \end{align*}
  as any such $\kappa\in M_{\mathcal{S}_1\cup\cdots\cup \mathcal{S}_n}$ must necessarily be in every $M_{\mathcal{S}_i}$.\newline

  Thus, the family
  \begin{align*}
    \mathcal{M} &= \set{M_{\mathcal{S}_0}| \mathcal{S}_0\subseteq \mathcal{S}\text{ finite}}
  \end{align*}
  has the finite intersection property. By compactness, there is some $\nu$ such that
  \begin{align*}
    \nu\in \bigcap \mathcal{M}
  \end{align*}
  with $\nu\left(\epsilon\right) = 1$ and, for all $\alpha,\beta\in \mathcal{S}$, since $\nu\in M_{\set{\alpha,\beta,\alpha + \beta}}$, $\nu\left(\alpha + \beta\right) = \nu\left(\alpha\right) + \nu\left(\beta\right)$.
\end{proof}
\section{Proof of Tarski's Theorem}%
Finally, we are able to prove the reverse direction of Tarski's Theorem. We restate the theorem before giving its proof.
\begin{tcolorbox}[blanker,breakable,left=3mm,before skip=10pt, after skip=10pt, borderline west={1pt}{0pt}{blue!50!white},sharp corners,]
\tarski*
\end{tcolorbox}
\begin{proof}[Proof of the Reverse Direction of Theorem \ref{thm:tarski}:]
  Let $\mathcal{S}$ be the type semigroup of the action of $G$ on $X$.\newline

  Suppose $E$ is not $G$-paradoxical. Then, $\left[E\right]\neq 2\left[E\right]$ by Fact \ref{fact:type_semigroup_paradoxicality}, meaning $\left(n+1\right)\left[E\right]\nleq n\left[E\right]$ for all $n\in \N$ by the contrapositive of Corollary \ref{cor:non_paradoxicality}.\newline

  Thus, by Theorem \ref{thm:homomorphism_existence}, there is a map $\nu\colon \mathcal{S}\rightarrow [0,\infty]$ with $\nu\left(\left[E\right]\right) = 1$. The map $\mu\colon P(X)\rightarrow [0,\infty]$ defined by
  \begin{align*}
    \mu\left(A\right) &= \nu\left(\left[A\right]\right)
  \end{align*}
  is the desired finitely additive measure.
\end{proof}

\chapter{Characterizations using Invariant States: the Følner Condition}
Having proven Tarski's theorem, we can turn our attention to a more definite understanding of amenability. We will use theorems and techniques from functional analysis to help understand the space $\ell_{\infty}\left(G\right)$, which will open a wide variety of characterizations for amenability, beyond that which was established in Tarski's theorem.
\section{Means and Invariant States}%
\begin{definition}
  Let $G$ be a group, with $P(G)$ denoting its power set.\newline

  An invariant {mean} on $G$ is a set function $m\colon P(G)\rightarrow [0,1]$ which satisfies, for all $t\in G$ and $E,F\subseteq G$,
  \begin{itemize}
    \item $m(G) = 1$;
    \item $m\left(E\sqcup F\right) = m(E) + m(F)$;
    \item $m\left(tE\right) = m\left(E\right)$.
  \end{itemize}
  We say $G$ is amenable if $G$ admits a mean.\newline

  The mean $m$ is, in other words, a translation-invariant probability measure on the measurable space $\left(G,P(G)\right)$.
\end{definition}
We have shown in the proof of Theorem \ref{thm:tarski} that an equivalent condition for amenability is that the group is not paradoxical.\newline

Using some essential results in group theory, we can establish some preliminary results on subgroups and quotient groups.
\begin{proposition}\label{prop:subgroups_quotientgroups_amenability}
  Let $G$ be an amenable group with $H\leq G$. Then, the following are true:
  \begin{enumerate}[(1)]
    \item $H$ is amenable;
    \item for $H\trianglelefteq G$, $G/H$ is amenable.
  \end{enumerate}
\end{proposition}
\begin{proof}\hfill
  \begin{enumerate}[(1)]
    \item Let $R$ be a right transversal for $H$, wherein we select one element of each right coset of $H$ to make up $R$.\newline

      If $m$ is a mean for $G$, we set $\lambda\colon P(H)\rightarrow [0,1]$ defined by
      \begin{align*}
        \lambda(E) = m\left(ER\right).
      \end{align*}
       We have
      \begin{align*}
        \lambda(H) &= m\left(HR\right)\\
                   &= m\left(G\right)\\
                   &= 1.
      \end{align*}
      We claim that if $E\cap F = \emptyset$, then $ER \cap FR = \emptyset$. Suppose toward contradiction this is not the case. Then, $xr_1 = yr_2$ for some $x\in E$, $y\in F$, and $r_1,r_2\in R$. Then, we must have $r_2r_1^{-1} = y^{-1}x \in H$, meaning $r_1 = r_2$ as, by definition, $R$ contains exactly one element of each right coset. Thus, $x=y$, so $E\cap F \neq \emptyset$.\newline

      We then have
      \begin{align*}
        \lambda\left(E\sqcup F\right) &= m\left(\left(E\sqcup F\right)R\right)\\
                                      &= m\left(ER\sqcup FR\right)\\
                                      &= m\left(ER\right) + m\left(FR\right)\\
                                      &= \lambda\left(E\right) + \lambda\left(F\right),
      \end{align*}
      and
      \begin{align*}
        \lambda\left(sE\right) &= m\left(sER\right)\\
                               &= m\left(ER\right)\\
                               &= \lambda\left(E\right).
      \end{align*}
    \item Let $\pi\colon G\rightarrow G/H$ be the canonical projection, defined by $\pi\left(t\right) = tH$. We define
      \begin{align*}
        \lambda\colon P\left(G/H\right) \rightarrow [0,1]
      \end{align*}
      by $\lambda(E) = m\left(\pi^{-1}\left(E\right)\right)$. We have
      \begin{align*}
        \lambda\left(G/H\right) &= m\left(\pi^{-1}\left(G/H\right)\right)\\
                                &= m\left(G\right)\\
                                &= 1,
      \end{align*}
      and
      \begin{align*}
        \lambda\left(E\sqcup F\right) &= m\left(\pi^{-1}\left(E\sqcup F\right)\right)\\
                                      &= m\left(\pi^{-1}\left(E\right)\sqcup \pi^{-1}\left(F\right)\right)\\
                                      &= m\left(\pi^{-1}\left(E\right)\right) + m\left(\pi^{-1}\left(F\right)\right)\\
                                      &= \lambda(E) + \lambda(F).
      \end{align*}
      To show translation-invariance, we let $sH = \pi(s)\in G/H$, and $E\subseteq G/H$. Note that
      \begin{align*}
        \pi^{-1}\left(\pi(s)E\right) &= s\pi^{-1}\left(E\right),
      \end{align*}
      since for $r\in s\pi^{-1}(E)$, we have $r = st$ for $t\in \pi(E)$, so $\pi\left(r\right) =\pi\left(st\right) = \pi\left(s\right)\pi\left(t\right)\in \pi\left(s\right)E$.\newline

      Additionally, if $r\in \pi^{-1}\left(\pi(s)E\right)$, we have $\pi(r)\in \pi(s)E$, so $\pi\left(s^{-1}r\right)\in E$, meaning $s^{-1}r\in \pi^{-1}E$.\newline

      Thus,
      \begin{align*}
        \lambda\left(\pi\left(s\right)E\right) &= m\left(\pi^{-1}\left(\pi\left(s\right)E\right)\right)\\
                                               &= m\left(s\pi^{-1}\left(E\right)\right)\\
                                               &= m\left(\pi^{-1}\left(E\right)\right)\\
                                               &= \lambda\left(E\right).
      \end{align*}
  \end{enumerate}
\end{proof}

Now that we understand some useful properties of means in relation to groups and subgroups, we turn our attention toward finding means on groups. In order to do this, we turn our attention towards the space $\ell_{\infty}\left(G\right)$, which allows us to use theories from functional analysis to better understand means on $G$.
\begin{definition}
  Let $G$ be a group.
  \begin{enumerate}[(1)]
    \item The space $\mathcal{F}\left(G,\R\right)$ is defined by
      \begin{align*}
        \mathcal{F}\left(G,\R\right) &= \set{f | f\colon G\rightarrow \R\text{ is a function}}.
      \end{align*}
    \item A function $f\in \mathcal{F}\left(G,\R\right)$ is called positive if $f(x) \geq 0$ for all $x\in G$.
    \item A function $f\in \mathcal{F}\left(G,\R\right)$ is called simple if $\ran(f)$ is finite. We let
      \begin{align*}
        \Sigma &= \set{f\in \mathcal{F}\left(G,\R\right) | f\text{ is simple}}.
      \end{align*}
  \end{enumerate}
\end{definition}
\begin{fact}
  It is the case that $\Sigma \subseteq \mathcal{F}\left(G,\R\right)$ is a linear subspace.
\end{fact}
\begin{definition}
  For $E\subseteq G$, we define
  \begin{align*}
    \1_{E}\colon G\rightarrow \R
  \end{align*}
  by
  \begin{align*}
    \1_{E}\left(x\right) &= \begin{cases}
      1 & x\in E\\
      0 & x\notin E
    \end{cases}.
  \end{align*}
  This is the characteristic function of $E$.
\end{definition}
\begin{fact}
  We have
  \begin{align*}
    \Span\set{\1_{E}| E\subseteq G} &= \Sigma.
  \end{align*}
\end{fact}
\begin{proof}
  We see that $\1_{E}\in \Sigma$ for any $E\subseteq G$, and that $\Sigma$ is a subspace.\newline

  If $\phi\in \Sigma$ with $\Ran\left(\phi\right) = \set{t_1,\dots,t_n}$, where $t_i$ are distinct, we set
  \begin{align*}
    E_i &= \phi^{-1}\left(\set{t_i}\right),
  \end{align*}
  yielding
  \begin{align*}
    \phi &= \sum_{i=1}^{n}t_i\1_{E_i}.
  \end{align*}
\end{proof}
\begin{definition}
  \begin{enumerate}[(1)]
    \item A function $f\in \mathcal{F}\left(G,\R\right)$ is bounded if there exists $M > 0$ such that $\Ran\left(f\right) \subseteq \left[-M,M\right]$.
    \item The space $\ell_{\infty}\left(G\right)$ is defined by
      \begin{align*}
        \ell_{\infty}\left(G\right) &= \set{f\in \mathcal{F}\left(G,\R\right)| f\text{ is bounded}}.
      \end{align*}
    \item The norm on $\ell_{\infty}\left(G\right)$ is defined by
      \begin{align*}
        \norm{f} &= \sup_{x\in G}\left\vert f(x) \right\vert.
      \end{align*}
  \end{enumerate}
\end{definition}
\begin{proposition}
  The space $\ell_{\infty}(G)$ is complete. Additionally, $\overline{\Sigma} = \ell_{\infty}\left(G\right)$.
\end{proposition}
\begin{proof}
  Let $\left(f_n\right)_n$ be $\norm{\cdot}$-Cauchy in $\ell_{\infty}\left(G\right)$. Then, for all $x\in G$, it is the case that
  \begin{align*}
    \left\vert f_n(x) - f_m(x) \right\vert &= \left\vert \left(f_n - f_m\right)\left(x\right) \right\vert\\
                                           &\leq \norm{f_n - f_m},
  \end{align*}
  meaning $\left(f_n\left(x\right)\right)_n$ is Cauchy in $\R$. We define $f(x) = \lim_{n\rightarrow\infty}f_n(x)$. We must show that $f\in \ell_{\infty}\left(G\right)$, and $\norm{f_n-f}\rightarrow 0$.\newline

  We have
  \begin{align*}
    \left\vert f(x) \right\vert &= \left\vert \lim_{n\rightarrow\infty}f_n\left(x\right) \right\vert\\
                                &= \lim_{n\rightarrow\infty}\left\vert f_n\left(x\right) \right\vert\\
                                &\leq \limsup_{n\rightarrow\infty}\norm{f_n}\\
                                &\leq C,
  \end{align*}
  as Cauchy sequences are always bounded. Thus, $\sup_{x\in G}\left\vert f(x) \right\vert\leq C$.\newline

  Given $\ve > 0$, we find $N$ such that for all $m,n\geq N$, $\norm{f_n - f_m} \leq \ve$. Thus, for $x\in G$, we have
  \begin{align*}
    \left\vert f_n(x) - f_m(x) \right\vert &\leq \norm{f_n - f_m}\\
                                           &\leq \ve.
  \end{align*}
  Taking $m\rightarrow\infty$, we get $\left\vert f_n(x) - f(x) \right\vert \leq \ve$, for all $n\geq N$, so $\norm{f_n - f}\leq \ve$ for all $n\geq N$.\newline

  For $f\in \ell_{\infty}\left(G\right)$, let $\Ran\left(f\right) \subseteq \left[-M,M\right]$ for some $M > 0$. Let $\ve > 0$. Since $\left[-M,M\right]$ is compact, it is totally bounded, so we can find intervals $I_{1},\dots,I_n$ with $\left[-M,M\right] = \bigsqcup_{k=1}^{n}I_k$, with the length of each $I_k$ less than $\ve$.\newline

  Set $E_k = f^{-1}\left(I_k\right)$. Pick some $t_k\in I_k$. We set
  \begin{align*}
    \phi &= \sum_{i=1}^{n}t_k\1_{E_k}.
  \end{align*}
  Then, it is the case that $\norm{\phi - f} < \ve$.
\end{proof}
\begin{corollary}
  For any $f\in \ell_{\infty}\left(G\right)$, there is a sequence $\left(\phi_n\right)_n$ with $\norm{\phi_n -f}\rightarrow 0$. If $f\geq 0$, then we can select $\phi_n\geq 0$.
\end{corollary}
Now that we understand how simple functions relate to $\ell_{\infty}(G)$, we start by defining a translation action on $\ell_{\infty}(G)$, from which we will be able to convert the idea of means into invariant elements of the state space of the dual of $\ell_{\infty}\left(G\right)$.
\begin{proposition}\label{prop:translation_action}
  Let $G$ be a group. There is an action
  \begin{align*}
    \lambda_s\colon G\rightarrow \Isom\left(\ell_{\infty}\left(G\right)\right)
  \end{align*}
  defined by
  \begin{align*}
    \lambda_{s}\left(f\right)\left(t\right) &= f\left(s^{-1}t\right)
  \end{align*}
\end{proposition}
\begin{proof}
  We have
  \begin{align*}
    \lambda_s\left(f + \alpha g\right)\left(t\right) &= \left(f + \alpha g\right) \left(s^{-1}t\right)\\
                                                     &= f\left(s^{-1}t\right) \alpha g\left(s^{-1}t\right)\\
                                                     &= \lambda_s\left(f\right)\left(t\right) + \alpha \lambda_s\left(g\right)\left(t\right)\\
                                                     &= \left(\lambda_s\left(f\right) + \alpha \lambda_s\left(g\right)\right)(t).
  \end{align*}
  Thus, $\lambda_s$ is linear. Additionally,
  \begin{align*}
    \norm{\lambda_s\left(f\right)} &= \sup_{t\in G}\left\vert \lambda_s\left(f\right)\left(t\right) \right\vert\\
                                   &= \sup_{t\in G}\left\vert f\left(s^{-1}t\right) \right\vert\\
                                   &= \norm{f},
  \end{align*}
  and
  \begin{align*}
    \norm{\lambda_s\left(f\right) - \lambda_s\left(g\right)} &= \norm{\lambda_s\left(f-g\right)}\\
                                                             &= \norm{f-g},
  \end{align*}
  meaning $\lambda_s$ is an isometry.\newline

  We have
  \begin{align*}
    \lambda_s\circ \lambda_r\left(f\right)\left(t\right) &= \lambda_r\left(f\right)\left(s^{-1}t\right)\\
                                                         &= \lambda_r\left(r^{-1}s^{-1}t\right)\\
                                                         &= f\left(\left(sr\right)^{-1}t\right)\\
                                                         &= \lambda_{sr}\left(f\right)\left(t\right),
  \end{align*}
  establishing that $\lambda_s\circ \lambda_r = \lambda_{sr}$.\newline

  By a similar process, we find that $\lambda_{s}\left(\1_{E}\right) = \1_{sE}$ for any $E\subseteq G$ and $s\in G$.
\end{proof}
\begin{definition}
  A {state} on $\ell_{\infty}\left(G\right)$ is a continuous linear functional $\mu\in \left(\ell_{\infty}\left(G\right)\right)^{\ast}$ such that the following are true:
  \begin{itemize}
    \item $\mu$ is positive;
    \item $\mu\left(\1_{G}\right) = 1$.
  \end{itemize}
  A state is called left-invariant if
  \begin{align*}
    \mu\left(\lambda_s\left(f\right)\right) = \mu\left(f\right).
  \end{align*}
\end{definition}
\begin{example}
  The evaluation functional, $\delta_x\colon \ell_{\infty}\rightarrow \R$, defined by
  \begin{align*}
    \delta_{x}\left(f\right) &= f(x),
  \end{align*}
  is a state. However, it is not necessarily invariant, as
  \begin{align*}
    \delta_x\left(\lambda_s\left(f\right)\right) &= \lambda_s\left(f\right)\left(x\right)\\
                                                 &= f\left(s^{-1}x\right)\\
                                                 &\neq f(x).
  \end{align*}
  However, we can use the evaluation functional to create an invariant state. If $G$ is finite, we define
  \begin{align*}
    \mu &= \frac{1}{\left\vert G \right\vert} \sum_{x\in G}\delta_x,
  \end{align*}
  which is indeed an invariant state.
\end{example}
We can characterize states slightly differently, which will enable us to show the equivalence between invariant states and means.
\begin{lemma}\label{lemma:characterizing_states}\hfill
  \begin{enumerate}[(1)]
    \item If $\mu$ is a state on $\ell_{\infty}\left(G\right)$, then
      \begin{align*}
        \norm{\mu}_{\op} = 1.
      \end{align*}
    \item If $\mu\in \left(\ell_{\infty}\left(G\right)\right)^{\ast}$ is such that
      \begin{align*}
        \norm{\mu}_{\op} &= \mu\left(\1_{G}\right)\\
                               &= 1,
      \end{align*}
      then $\mu$ is positive and a state.
  \end{enumerate}
\end{lemma}
\begin{proof}\hfill
  \begin{enumerate}[(1)]
    \item Let $\mu$ be a state. Given $f\in \ell_{\infty}\left(G\right)$, we have
      \begin{align*}
        \norm{f}\1_{G} - f &\geq 0\\
        \norm{f}\1_{G} + f &\geq 0,
      \end{align*}
      so
      \begin{align*}
        0 &\leq \mu\left(\norm{f}\1_{G} - f\right) \\
          &= \norm{f}\mu\left(\1_{G}\right) - \mu\left(f\right)
          \intertext{meaning}
        \mu\left(f\right) &\leq \norm{f}.
        \intertext{Additionally,}
        0 &\leq \mu\left(\norm{f}\1_{G} + f\right)\\
          &= \norm{f}\mu\left(\1_{G}\right) + \mu\left(f\right),
          \intertext{meaning}
        -\mu\left(f\right) &\leq \norm{f}.
      \end{align*}
      Thus, we have $\left\vert \mu\left(f\right) \right\vert \leq \norm{f}$, so $\norm{\mu}_{\op} \leq 1$. However, since $\mu\left(\1_{G}\right) = 1$, we must have $\norm{\mu}_{\op} = 1$.
  \item Suppose $\norm{\mu}_{\op} = \mu\left(\1_{G}\right) = 1$. Let $f\geq 0$. Set $g = \frac{1}{\norm{f}_u}f$.\newline

    Then, $\Ran(g) \subseteq [0,1]$, and $\Ran\left(g - \1_{G}\right) \subseteq \left[-1,1\right]$. Thus, $\norm{g - \1_{G}}_{u} \leq 1$.\newline

    Since $\norm{\mu}_{\op} = 1$, we must have
    \begin{align*}
      \left\vert \mu\left(g - \1_{G}\right) \right\vert &\leq 1\\
      \left\vert \mu\left(g\right) - 1 \right\vert &\leq 1,
    \end{align*}
    and since $\mu\left(\1_{G}\right) = 1$, we have $\mu\left(g\right) \in [0,2]$. Thus, $\mu\left(f\right) = \norm{f}\mu\left(g\right) \geq 0$.
  \end{enumerate}
\end{proof}

To show the equivalence between means and invariant states, we need to be able to characterize the state space on $\left(\ell_{\infty}\left(G\right)\right)^{\ast}$. To do this, we make use of some results from functional analysis.\newline

If $X$ is a normed vector space, then the topology on $X^{\ast}$ induced by $X^{\ast\ast}$ is known as the weak* topology. The weak* topology is the topology of pointwise convergence in $X^{\ast}$ --- a net $\left(\varphi_{\alpha}\right)_{\alpha}$ converges to $\varphi$ in the weak* topology if and only if, for all $\hat{x}\in X^{\ast\ast}$, we have
\begin{align*}
  \left(\hat{x}\left(\varphi_{\alpha}\right)\right)_{\alpha}\rightarrow \hat{x}\left(\varphi\right),
\end{align*}
or by the definition of $X^{\ast\ast}$, this is equivalent to
\begin{align*}
  \left(\varphi_{\alpha}\left(x\right)\right) \rightarrow \varphi\left(x\right)
\end{align*}
for all $x\in X$.\newline

We state some important results in functional analysis here without proof. The proofs of these results can be found in functional analysis textbooks such as \cite{rudin_functional_analysis}.
\begin{theorem}[Hahn--Banach Continuous Extension Theorem]
  Let $X$ be a normed vector space, $E\subseteq X$ a subspace, and $\varphi\in E^{\ast}$ a bounded linear functional. Then, there exists a continuous $\psi\in X^{\ast}$ such that $\norm{\varphi}_{\op} = \norm{\psi}_{\op}$, and $\psi|_{E} = \varphi$.
\end{theorem}
\begin{theorem}[Hahn--Banach Separation Theorems]
  Let $X$ be a normed vector space.
  \begin{enumerate}[(1)]
    \item Given a nonzero $x_0\in X$, there is a $\varphi\in X^{\ast}$ with $\norm{\varphi}_{\op} = 1$ and $\varphi\left(x_0\right) = \norm{x}$. We call $\varphi$ a norming functional.
    \item Given a proper closed subspace $E\subseteq X$ and $x_0\in X\setminus E$, there is a $\varphi\in X^{\ast}$ such that $\varphi|_{E} = 0$, $\norm{\varphi}_{\op} = 1$, and $\varphi\left(x\right) = \dist_{E}(x)$ for all $x\in X$.
  \end{enumerate}
\end{theorem}
\begin{theorem}[Banach--Alaoglu Theorem]
  Let $X$ be a normed vector space.
  \begin{enumerate}[(1)]
    \item The closed unit ball in the dual space, $B_{X^{\ast}}$, is compact in the $w^{\ast}$ topology.
    \item A subset $C\subseteq X$ is $w^{\ast}$-compact if and only if $C$ is $w^{\ast}$-closed and norm bounded.
  \end{enumerate}
\end{theorem}
\begin{corollary}
  The set of states in $\left(\ell_{\infty}\left(G\right)\right)^{\ast}$ forms a $w^{\ast}$-compact subset of $B_{\left(\ell_{\infty}\left(G\right)\right)^{\ast}}$.
\end{corollary}
\begin{proof}
  From the Banach--Alaoglu Theorem, we only need to show that the set of states, $S\left(\ell_{\infty}\left(G\right)\right)$, is $w^{\ast}$-closed, as every element of $S\left(\ell_{\infty}\left(G\right)\right)$ has norm $1$.\newline

  Let $f\in \ell_{\infty}\left(G\right)$ be positive, and let $\left(\varphi_{i}\right)_i$ be a net in $S\left(\ell_{\infty}\left(G\right)\right)$ with $\left(\varphi_{i}\right)_i\xrightarrow{w^{\ast}} \varphi\in \left(\ell_{\infty}\left(G\right)\right)^{\ast}$. From Lemma \ref{lemma:characterizing_states}, we must show that $\varphi$ is positive and $\varphi\left(\1_{G}\right) = 1$.\newline

  We start by seeing that, since each $\varphi_i$ is a state, we have $\varphi_{i}\left(f\right) \geq 0$ for each $i\in I$, so we must have $\varphi\left(f\right) \geq 0$.\newline

  Similarly, since $\varphi_{i}\left(\1_{G}\right) = 1$ for each $i\in I$, and $\left(\varphi_i\right)_i \xrightarrow{w^{\ast}} \varphi$, we have $\varphi\left(\1_{G}\right) = 1$. Thus, by Lemma \ref{lemma:characterizing_states}, we have that $S\left(\ell_{\infty}\left(G\right)\right)$ is $w^{\ast}$-closed.
\end{proof}

Now, we may show the correspondence between invariant states and means.
\begin{proposition}\label{prop:state_implies_mean}
  If $\mu\in \left(\ell_{\infty}\left(G\right)\right)^{\ast}$ is a state, then $m\colon P(G)\rightarrow [0,1]$ defined by $m(E) = \mu\left(\1_{E}\right)$ is a finitely additive probability measure on $G$.\newline

  Moreover, if $\mu$ is invariant, then $m$ is a mean.
\end{proposition}
\begin{proof}
  We have
  \begin{align*}
    m\left(G\right) &= \mu\left(\1_{G}\right)\\
                    &= 1\\
                    \\
    m\left(\emptyset\right) &= \mu\left(0\right)\\
                            &= 0\\
                            \\
    m\left(E\sqcup F\right) &= \mu\left(\1_{E\sqcup F}\right)\\
                            &= \mu\left(\1_{E} + \1_{F}\right)\\
                            &= \mu\left(\1_{E}\right) + \mu\left(\1_{F}\right)\\
                            &= m\left(E\right) + m\left(F\right).
  \end{align*}
  Additionally, since $0 \leq \1_{E}\leq \1_{G}$, we have $0 \leq \mu\left(\1_{E}\right) \leq 1$, so $0 \leq m(E) \leq 1$.\newline

  If $\mu$ is invariant, then
  \begin{align*}
    m\left(sE\right) &= \mu\left(\1_{sE}\right)\\
                     &= \mu\left(\lambda_s\left(\1_{E}\right)\right)\\
                     &= \mu\left(\1_{E}\right)\\
                     &= m\left(E\right).
  \end{align*}
\end{proof}
\begin{proposition}\label{prop:mean_implies_state}
  If $G$ admits a mean, then $\left(\ell_{\infty}\left(G\right)\right)^{\ast}$ admits an invariant state.
\end{proposition}
\begin{proof}
  Let $m$ be a mean. Define $\mu_0\colon \Sigma\rightarrow \R$ by
  \begin{align*}
    \mu_0\left(\sum_{k=1}^{n}t_k\1_{E_k}\right) &= \sum_{k=1}^{n}t_km\left(E_k\right).
  \end{align*}
  Since $m$ is finitely additive, it is the case that $\mu_0$ is well-defined, linear, and positive, with $\mu_0\left(\1_{G}\right) = m\left(G\right) = 1$.\newline

  Additionally, since $m$ is a mean, then for $f = \sum_{k=1}^{n}t_k\1_{E_k}$, we have
  \begin{align*}
    \mu_0\left(\lambda_s\left(f\right)\right) &= \mu_0\left(\lambda_s\left(\sum_{k=1}^{n}t_k\1_{E_k}\right)\right)\\
                                              &= \mu_0\left(\sum_{k=1}^{n}t_k\1_{sE_k}\right)\\
                                              &= \sum_{k=1}^{n}t_km\left(sE_k\right)\\
                                              &= \sum_{k=1}^{n}t_km\left(E_k\right)\\
                                              &= \mu_0\left(f\right).
  \end{align*}
  We see that
  \begin{align*}
    \left\vert \mu_0\left(f\right) \right\vert &= \left\vert \sum_{k=1}^{n}t_km\left(E_k\right) \right\vert\\
                                               &\leq \sum_{k=1}^{n}\left\vert t_k \right\vert m\left(E_k\right)\\
                                               &\leq \sum_{k=1}^{n}\norm{f}\sum_{k=1}^{n}m\left(E_k\right)\\
                                               &= \norm{f}\sum_{k=1}^{n}m\left(E_k\right)\\
                                               &\leq \norm{f},
  \end{align*}
  meaning $\mu_0$ is continuous, so $\mu_0$ is uniformly continuous.\newline

  Since $\overline{\Sigma} = \ell_{\infty}\left(G\right)$, uniform continuity provides that $\mu_0$ extends to a continuous linear functional $\mu\colon \ell_{\infty}\left(G\right)\rightarrow \R$ with $\mu\left(\1_{G}\right) = \mu_0\left(\1_{G}\right) = 1$.\newline

  For $f\geq 0$, we find a sequence $\left(\phi_n\right)_n$ in $\Sigma$ with $\phi_n\geq 0$ and $\norm{\phi_n - f} \xrightarrow{n\rightarrow\infty}0$. We set
  \begin{align*}
    \mu\left(f\right) &= \lim_{n\rightarrow\infty}\mu\left(\phi_n\right)\\
                      &= \lim_{n\rightarrow\infty}\mu_0\left(\phi_n\right)\\
                      &\geq 0,
  \end{align*}
  so $\mu$ is a state.\newline

  If $f\in \ell_{\infty}\left(G\right)$, $s\in G$, and $\left(\phi_n\right)_n$ a sequence in $\Sigma$ with $\left(\phi_n\right)_n\rightarrow f$, then
  \begin{align*}
    \norm{\lambda_s\left(\phi_n\right) - \lambda_s\left(f\right)} &= \norm{\lambda_s\left(\phi_n - f\right)}\\
                                                                  &= \norm{\phi_n - f}\\
                                                                  &\rightarrow 0.
  \end{align*}
  Thus, we have
  \begin{align*}
    \mu\left(\lambda_s\left(\phi_n\right)\right) &= \mu_0\left(\lambda_s\left(\phi_n\right)\right)\\
                                                 &= \mu_0\left(\phi_n\right)\\
                                                 &= \mu\left(\phi_n\right)\\
                                                 &\rightarrow \mu\left(f\right),
  \end{align*}
  so $\mu\left(f\right) = \mu\left(\lambda_s\left(f\right)\right)$. Thus, $\mu\in \left(\ell_{\infty}\left(G\right)\right)^{\ast}$ is an invariant state.
\end{proof}
\section{Establishing Amenability using Invariant States}%
Owing to the correspondence between invariant states and means, we are now able to establish the amenability of large classes of groups.
\begin{proposition}
  The group of integers, $\Z$, is amenable.
\end{proposition}
\begin{proof}
  We define the left shift, $\lambda_1\colon \ell_{\infty}\left(\Z\right) \rightarrow \ell_{\infty}\left(\Z\right)$, by
  \begin{align*}
    \ell_{\infty}\left(f\right)\left(k\right) &= f\left(k-1\right).
  \end{align*}
  This is an action as in Proposition \ref{prop:translation_action}. \newline

  We set $Y = \Ran\left(\id - \lambda_1\right)\subseteq \ell_{\infty}\left(\Z\right)$. We claim that $\dist_{Y}\left(\1_{\Z}\right) \geq 1$.\newline

  Suppose toward contradiction that there is $y\in Y$ with $\norm{\1_{\Z} - y}_{u} = r < 1$. Then, $y = f - \lambda_1 f$ for some $f\in \ell_{\infty}(\Z)$, so
  \begin{align*}
    \norm{\1_{\Z} - \left(f - \lambda_1\left(f\right)\right)} &= r.
  \end{align*}
  Thus, for all $k\in\Z$, we have
  \begin{align*}
    \left\vert 1 - \left(f(k) - f(k-1)\right) \right\vert &\leq r,
  \end{align*}
  so $\left\vert f(k) - f\left(k-1\right) \right\vert \geq 1-r > 0$. However, such an $f$ cannot be bounded.\newline

  Since $\dist_{\overline{Y}}\left(\1_{\Z}\right) = \dist_{Y}\left(\1_{\Z}\right)$, the Hahn--Banach separation theorems provide $\mu\in \left(\ell_{\infty}\left(\Z\right)\right)^{\ast}$ with $\norm{\mu}_{\op} = 1$, $\mu|_{\overline{Y}} = 0$, and $\mu\left(\1_{\Z}\right) = \dist_{Y}\left(\1_{\Z}\right) \geq 1$.\newline

  Since $\norm{\mu}_{\op} = 1$ and $\mu\left(\1_{\Z}\right) \geq 1$, we must have $\mu\left(\1_{\Z}\right) = 1$.\newline

  Additionally, since $\norm{\mu}_{\op} = \mu\left(\1_{\Z}\right) = 1$, we have that $\mu$ is a state on $\ell_{\infty}\left(\Z\right)$, and since $\mu\left(y\right) = 0$ for all $y\in Y$, we have
  \begin{align*}
    \mu\left(f - \lambda_1\left(f\right)\right) &= 0\\
    \mu\left(f\right) &= \mu\left(\lambda_1\left(f\right)\right).
  \end{align*}
  Inductively, this means that $\mu\left(f\right) = \mu\left(\lambda_k\left(f\right)\right)$ for all $k\in \Z$, so $\mu$ is an invariant state on $\ell_{\infty}\left(\Z\right)$. Thus, $\Z$ is amenable.
\end{proof}
\begin{proposition}
  If $N\trianglelefteq G$ and $G/N$ are amenable, then $G$ is amenable.
\end{proposition}
\begin{proof}
  Let $\rho\in \left(\ell_{\infty}\left(G/N\right)\right)^{\ast}$ be an invariant state, and let $p\colon P(N)\rightarrow [0,1]$ be a mean. For $E\subseteq G$, we define $f_E\colon G/N\rightarrow \R$ by
  \begin{align*}
    f_E\left(tN\right) &= p\left(N\cap t^{-1}E\right).
  \end{align*}
  We start by verifying that $f_E$ is well-defined. For $tN = sN$, we have $s^{-1}t\in N$, so
  \begin{align*}
    p\left(N\cap t^{-1}E\right) &= p\left(s^{-1}t\left(N\cap t^{-1}E\right)\right)\\
                                &= p\left(s^{-1}tN \cap s^{-1}E\right)\\
                                &= p\left(N\cap s^{-1}E\right).
  \end{align*}
  Since $f_E$ is defined through $p$, we can see that $f_E$ is bounded. Additionally,
  \begin{align*}
    f_{E\sqcup F}\left(tN\right) &= p\left(N\cap t^{-1}\left(E\sqcup F\right)\right)\\
                                 &= p\left(N\cap \left(t^{-1}E\sqcup t^{-1}F\right)\right)\\
                                 &= p\left(\left(N\cap t^{-1}E\right) \sqcup \left(N\cap t^{-1}F\right)\right)\\
                                 &= p\left(N\cap t^{-1}E\right) + p\left(N\cap t^{-1}F\right)\\
                                 &= f_E\left(tN\right) + f_F\left(tN\right)\\
                                 &= \left(f_E + f_F\right)\left(tN\right),
  \end{align*}
  and
  \begin{align*}
    f\left(sE\right) \left(tN\right) &= p\left(N\cap t^{-1}sE\right)\\
                                     &= f_E\left(s^{-1}tN\right)\\
                                     &= \lambda_{sN}\left(f_E\right)\left(tN\right),
  \end{align*}
  so $f_{sE} = \lambda_{sN}\left(f_E\right)$. Finally,
  \begin{align*}
    f_G\left(tN\right) &= p\left(N\cap t^{-1}G\right)\\
                       &=p\left(N\right)\\
                       &= 1,
  \end{align*}
  meaning $f_G = \1_{G/N}$.\newline

  We define $m\colon P(G)\rightarrow [0,1]$ by
  \begin{align*}
    m(E) &= \rho\left(f_E\right).
  \end{align*}
  Then, we have
  \begin{align*}
    m\left(E\sqcup F\right) &= m(E) + m(F)\\
                            \\
    m\left(G\right) &= 1\\
    \\
    m\left(sE\right) &= \rho\left(f_{sE}\right)\\
                     &= \rho\left(\lambda_{sN}\left(f_{E}\right)\right)\\
                     &= \rho\left(f_E\right)\\
                     &= m(E),
  \end{align*}
  so $m$ is a mean.
\end{proof}
\begin{corollary}
  The finite direct product of amenable groups is amenable.
\end{corollary}
\begin{proof}
  For $H$ and $K$ amenable groups, we know that $K\cong \left(H\times K\right)/H$ and $H$ are amenable, so $H\times K$ is amenable. Induction provides the general case.
\end{proof}
\begin{corollary}
  Finitely generated abelian groups are amenable.
\end{corollary}
\begin{proof}
  By the fundamental theorem of finitely generated abelian groups, all finitely generated abelian groups are isomorphic to $\Z^{d}\times \Z/n_1\Z\times\cdots\times \Z/{n_k}\Z$.\newline

  Since $\Z^{d}$ is a finite direct product of $\Z$, and the torsion subgroup $\Z/n_1\Z\times\cdots\times \Z/n_k\Z$ is finite, we see that a finitely generated abelian group is a direct product of two amenable groups, hence amenable.
\end{proof}
\begin{corollary}
  If $\set{G_i}_{i\in I}$ is a directed family of amenable groups, then the direct limit,
  \begin{align*}
    G &= \bigcup_{i\in I}G_i,
  \end{align*}
  is also amenable.
\end{corollary}
\begin{proof}
  Let $\mu_i\in \left(\ell_{\infty}\left(G_i\right)\right)^{\ast}$ be invariant states.\newline

  Set
  \begin{align*}
    M_i &= \set{\mu\in S\left(\ell_{\infty}\left(G\right)\right)| \mu\left(\lambda_s\left(f\right)\right) = \mu\left(f\right)\text{ for all }s\in G_i}.
  \end{align*}
  We set $\mu\left(f\right) = \mu_i\left(f|_{G_i}\right)$. Since each $G_i$ is amenable, it is the case that each $M_i$ is nonempty. Similarly, seeing as we have established the state space as $w^{\ast}$-closed in $B_{\left(\ell_{\infty}\left(G\right)\right)^{\ast}}$, it is the case that each $M_i$ is $w^{\ast}$-closed in $B_{\left(\ell_{\infty}\left(G\right)\right)^{\ast}}$.\newline

  For $i_1,\dots,i_n$, we find $G_j \supseteq G_{i_1},\dots,G_{i_n}$, which exists since $\set{G_i}_{i\in I}$ is directed. We have that $M_j\subseteq \bigcap_{k=1}^{n}M_{i_k}$, so $\set{M_i}_{i\in I}$ has the finite intersection property.\newline

  Thus, there is $\mu\in \bigcap_{i\in I}M_i$, which is necessarily invariant on $G$.
\end{proof}
\begin{corollary}
  All abelian groups are amenable.
\end{corollary}
\begin{proof}
  Every abelian group is the direct union of its finitely generated subgroups.
\end{proof}
\begin{corollary}
  All solvable groups are amenable.
\end{corollary}
\begin{proof}
  Let $e_G = G_0 \leq G_1\leq\cdots\leq G_n\leq G$ be such that $G_{j-1}\trianglelefteq G_j$ for $j=1,\dots,n$, and $G_i/G_j$ is abelian.\newline

  Since $G_0$ is abelian, it is amenable, as is $G_1/G_0$, so $G_1$ is amenable. We see then that $G_2$ is amenable as $G_1$ and $G_2/G_1$ are amenable.\newline

  Continuing in this fashion, we see that $G$ is amenable.
\end{proof}
\section{Følner's Condition and Approximate Means}%
While showing the existence of an invariant state is necessary and sufficient for showing a group is amenable, as well as showing the group is non-paradoxical, it is often difficult to establish either of these conditions.\newline

However, we can often more easily create a sequence (or net) of finitely supported functions whose limit is an invariant state. This will require the use of the Følner condition.
\begin{definition}\label{def:folner_condition}
  A group is said to satisfy the {Følner condition} if, for every $\ve > 0$ and $E\subseteq G$, there is a nonempty finite subset $F\subseteq G$ such that for all $t\in E$,
  \begin{align*}
    \frac{\left\vert tF\triangle F \right\vert}{\left\vert F \right\vert}\leq \ve.
  \end{align*}
  Equivalently, we can also say that the Følner condition is satisfied if and only if
  \begin{align*}
    \frac{\left\vert tF\cap F \right\vert}{\left\vert F \right\vert} \geq 1 - \ve
  \end{align*}
  for every $\ve > 0$.
\end{definition}
\begin{lemma}\label{lemma:folner_sequences}
  A countable group $G$ satisfies the Følner condition if and only if $G$ admits a sequence $\left(F_n\right)_n$ with $F_n\subseteq G$ finite such that
  \begin{align*}
    \left(\frac{\left\vert tF_n\triangle F_n \right\vert}{\left\vert F_n \right\vert}\right)_n \xrightarrow{n\rightarrow \infty}0
  \end{align*}
  for all $t\in G$. Such a sequence is known as a Følner sequence.
\end{lemma}
\begin{proof}
  Let $G$ admit a Følner sequence, $\left(F_n\right)_n$. Given $\ve > 0$ and $E\subseteq G$ finite, find $N$ such that for all $s\in E$ and $n\geq N$,
  \begin{align*}
    \frac{\left\vert sF_n\triangle F_n \right\vert}{\left\vert F_n \right\vert} &\leq \ve.
  \end{align*}
  We take $F = F_N$ in the definition of the Følner condition.\newline

  Let $G$ satisfy the Følner condition. We write $G = \bigcup_{n\geq 1}E_n$, with $E_1\subseteq E_2\subseteq \cdots$, and define $F_n$ such that for all $t\in E_n$,
  \begin{align*}
    \frac{\left\vert tF_n\triangle F_n \right\vert}{\left\vert F_n \right\vert} &\leq \frac{1}{n}.
  \end{align*}
  Given $t\in G$, then $t\in E_N$ for some $N$, so $t\in E_n$ For all $n\geq N$, so
  \begin{align*}
    \frac{\left\vert tF_n\triangle F_n \right\vert}{\left\vert F_n \right\vert} &\leq \frac{1}{n}
  \end{align*}
  for all $n\geq N$. Thus,
  \begin{align*}
    \left(\frac{\left\vert tF_n\triangle F_n \right\vert}{\left\vert F_n \right\vert}\right)\xrightarrow{n\rightarrow\infty}0.
  \end{align*}
\end{proof}
\begin{lemma}
  Let $G$ be a finitely generated group with generating set $S$. If $\left(F_n\right)_n$ is a sequence of finite subsets such that, for all $s\in S$,
  \begin{align*}
    \left(\frac{\left\vert sF_n\triangle F_n \right\vert}{\left\vert F_n \right\vert}\right)_n\rightarrow 0,
  \end{align*}
  then $\left(F_n\right)_n$ is a Følner sequence for $G$.
\end{lemma}
\begin{proof}
  Note that
  \begin{itemize}
    \item $s\left(A\triangle B\right) = sA\triangle sB$;
    \item $A\triangle C \subseteq \left(A\triangle B\right) \cup \left(B\triangle C\right)$.
  \end{itemize}
  We see that for any $s\in S$,
  \begin{align*}
    \frac{\left\vert s^{-1}F_n\triangle F_n \right\vert}{\left\vert F_n \right\vert} &= \frac{\left\vert s^{-1}\left(F_n\triangle sF_n\right) \right\vert}{\left\vert F_n \right\vert}\\
                                                                                     &= \frac{\left\vert F_n\triangle sF_n \right\vert}{\left\vert F_n \right\vert}\\
                                                                                     &\rightarrow 0.
  \end{align*}
  Thus, we may assume that $S$ is symmetric --- i.e., that $\set{s^{-1}| s\in S} = \set{s | s\in S}$.\newline

  For any $s,t\in S$, we have
  \begin{align*}
    \frac{\left\vert stF_n\triangle F_n \right\vert}{\left\vert F_n \right\vert} &\leq \frac{\left\vert stF_n\triangle F_n \right\vert}{\left\vert F_n \right\vert} + \frac{\left\vert sF_n\triangle F_n \right\vert}{\left\vert F_n \right\vert}\\
                                                                                 &= \frac{\left\vert s\left(tF_n\triangle F_n\right) \right\vert}{\left\vert F_n \right\vert} + \frac{\left\vert sF_n\triangle F_n \right\vert}{\left\vert F_n \right\vert}\\
                                                                                 &= \frac{\left\vert tF_n\triangle F_n \right\vert}{\left\vert F_n \right\vert} + \frac{\left\vert sF_n\triangle F_n \right\vert}{\left\vert F_n \right\vert}\\
                                                                                 &\rightarrow 0.
  \end{align*}
  We use induction to find the general case.
\end{proof}
\begin{example}
  Consider the group $\Z$. Since $\Z$ is generated by the element $\set{1}$, we see that for $F_n = [-n,n]$, that
  \begin{align*}
    \frac{\left\vert \left(F_n + 1\right)\triangle F_n \right\vert}{\left\vert F_n \right\vert} &= \frac{2}{2n+1}\\
                                                                                                &\rightarrow 0,
  \end{align*}
  meaning that $\Z$ satisfies the Følner condition.
\end{example}
We have thus far proven that $G$ satisfies the Følner condition if and only if $G$ admits a Følner sequence, and that $G$ is amenable if and only if $G$ admits an invariant state.\newline

We will now begin harmonizing these two characterizations through the use of approximate means, eventually showing that $G$ satisfies the Følner condition if and only if $G$ admits an approximate mean, and that $G$ admits an approximate mean if and only if $G$ is amenable.
\begin{definition}\label{def:state_on_prob_g}
  For a group $G$, we define
  \begin{align*}
    \Prob\left(G\right) = \set{f\colon G\rightarrow [0,\infty) | \left\vert \supp(f) \right\vert < \infty,~\sum_{t\in G}f(t) = 1}.
  \end{align*}
  Note that $\Prob(G) \subseteq B_{\ell_1\left(G\right)}$. For $f\in \prob(G)$, we set $\varphi_f\colon \ell_{\infty}(G)\rightarrow \C$ defined by
  \begin{align*}
    \varphi_f\left(g\right) &= \sum_{t\in G}g(t)f(t).
  \end{align*}
\end{definition}
\begin{fact}\label{fact:prob_g_state}
  For $f\in \prob(G)$, $\varphi_f$ is a state on $\ell_{\infty}\left(G\right)$.
\end{fact}
\begin{proof}
We can see that, by definition, $\varphi_f$ is positive, linear, and has $\varphi_f\left(\1_{G}\right) = 1$.\newline

We only need to show that $\norm{\varphi_f} = 1$. We see that
\begin{align*}
  \left\vert \varphi_f\left(g\right) \right\vert &= \left\vert \sum_{t\in G}g(t)f(t) \right\vert\\
                                                 &\leq \sum_{t\in G}\left\vert g(t) \right\vert\left\vert f(t) \right\vert\\
                                                 &\leq \norm{g}_{\infty}\sum_{t\in G}\left\vert f(t) \right\vert\\
                                                 &= \norm{g}_{\infty}.
\end{align*}
\end{proof}
\begin{proposition}
  There is an action $\lambda\colon G\xrightarrow \Isom\left(\ell_{1}\left(G\right)\right)$ such that $\prob(G)$ is invariant.
\end{proposition}
\begin{proof}
  Let $\lambda_s\left(f\right)\left(t\right) = f\left(s^{-1}t\right)$. Then,
  \begin{align*}
    \norm{\lambda_s\left(f\right)}_1 &= \sum_{t\in G}\left\vert \lambda_s\left(f\right)\left(t\right) \right\vert\\
                                     &= \sum_{t\in G}\left\vert f\left(s^{-1}t\right) \right\vert\\
                                     &= \sum_{r\in G}\left\vert f(r) \right\vert\\
                                     &= \norm{f}_{1}.
  \end{align*}
  Just as in Proposition \ref{prop:translation_action}, it is the case that $\lambda_s$ is linear. Additionally,
  \begin{align*}
    \lambda_r\circ \lambda_s\left(f\right)\left(t\right) &= \lambda_s\left(f\right)\left(r^{-1}t\right)\\
                                                         &= f\left(s^{-1}r^{-1}\left(t\right)\right)\\
                                                         &= f\left(\left(rs\right)^{-1}t\right)\\
                                                         &= \lambda_{rs}\left(f\right)\left(t\right).
  \end{align*}
  We see that if $f\in \prob(G)$, then for $f\geq 0$, we have $\lambda_s\left(f\right) \geq 0$, and
  \begin{align*}
    \sum_{t\in G}\lambda_s\left(f\right)\left(t\right) &= \sum_{t\in G}f\left(s^{-1}t\right)\\
                                                       &= \sum_{r\in G}f\left(r\right)\\
                                                       &= 1
  \end{align*}
  for any $f\in \prob(G)$.
\end{proof}
\begin{definition}\label{def:approximate_mean}
  For a countable group $G$, a sequence $\left(f_k\right)_k$ is called an approximate mean if, for all $s\in G$,
  \begin{align*}
    \norm{f_k - \lambda_s\left(f_k\right)}_{1} &\xrightarrow{k\rightarrow \infty}0.
  \end{align*}
\end{definition}
To begin the forward direction regarding the equivalence between the Følner condition, approximate means, and means, we begin by showing that the existence of a Følner sequence implies the existence of an approximate mean. Then, we will show that the existence of an approximate mean implies the existence of an invariant state (hence mean).
\begin{proposition}
  If $G$ admits a Følner sequence $\left(F_k\right)_k$, then $G$ admits an approximate mean.
\end{proposition}
\begin{proof}
  Set $f_k = \frac{1}{\left\vert F_k \right\vert}\1_{F_k}\in \prob(G)$. Then,
  \begin{align*}
    \norm{f_k - \lambda_s\left(f_k\right)}_{1} &= \frac{1}{\left\vert f_k \right\vert} \norm{\1_{F_k} - \lambda_s\left(\1_{F_k}\right)}\\
                                               &= \frac{1}{F_k}\norm{\1_{F_k} - \1_{sF_k}}\\
                                               &= \frac{\left\vert F_k\triangle sF_k \right\vert}{\left\vert F_k \right\vert}\\
                                               &\rightarrow 0.
  \end{align*}
\end{proof}
\begin{proposition}
  If $G$ admits an approximate mean, then $G$ is amenable.
\end{proposition}
\begin{proof}
  Let $\left(f_k\right)_k$ be an approximate mean. We define $\varphi_k = \left(\varphi_{f_k}\right)_k$ (as in Definition \ref{def:state_on_prob_g}) to be a sequence of states on $\ell_{\infty}\left(G\right)$.\newline

  Since the state space on $\ell_{\infty}\left(G\right)$ is $w^{\ast}$-compact, there is a state $\mu$ and a subnet $\left(\varphi_{k_j}\right)_j \xrightarrow{w^{\ast}}\mu$. \newline

  We only need to show that $\mu$ is invariant. Note that
  \begin{align*}
    \left\vert \mu\left(g\right) - \mu\left(\lambda_s\left(g\right)\right) \right\vert &\leq \left\vert \mu\left(g\right) - \varphi_{k_j}\left(g\right) \right\vert + \left\vert \varphi_{k_j}\left(g\right) - \varphi_{k_j}\left(\lambda_s\left(g\right)\right) \right\vert + \left\vert \varphi_{k_j}\left(\lambda_s\left(g\right)\right) - \mu\left(\lambda_s\left(g\right)\right) \right\vert
  \end{align*}
  for all $g\in \ell_{\infty}\left(G\right)$, $s\in G$, and all $j$.\newline

  Given $\ve > 0$, we find $J$ such that for $j\geq J$,
  \begin{align*}
    \left\vert \mu\left(g\right) - \varphi_{k_j}\left(g\right) \right\vert &< \ve/3\\
    \left\vert \mu\left(\lambda_s\left(g\right)\right) \varphi_{k_j}\left(\lambda_s\left(g\right)\right)\right\vert &< \ve/3.
  \end{align*}
  We also see that
  \begin{align*}
    \left\vert \varphi_{k_j}\left(g\right) - \varphi_{k_j}\left(\lambda_s\left(g\right)\right) \right\vert &= \left\vert \sum_{t\in G}g(t)f_{k_j}\left(t\right) - \sum_{t\in G}g\left(s^{-1}t\right)f_{k_j}\left(t\right) \right\vert\\
                                                                                                           &= \left\vert \sum_{t\in G}g(t)f_{k_j}\left(t\right) - \sum_{r\in G}g(r)f_{k_j}\left(sr\right) \right\vert \tag*{$r = s^{-1}t$}\\
                                                                                                           &= \left\vert \sum_{t\in G}g(t)\left(f_{k_j}\left(t\right)-\lambda_{s^{-1}}\left(f_{k_j}\right)\left(t\right)\right) \right\vert\\
                                                                                                           &\leq \norm{g}_{\infty}\sum_{t\in G}\left\vert f_{k_j}\left(t\right) - \lambda_{s^{-1}}\left(f_{k_j}\right)\left(t\right) \right\vert\\
                                                                                                           &= \norm{g}_{\infty}\norm{f_{k_j} - \lambda_{s^{-1}}\left(f_{k_j}\right)}_{1}\\
                                                                                                           &< \ve/3
  \end{align*}
  for large $j$. Thus, we have
  \begin{align*}
    \left\vert \mu\left(g\right) - \mu\left(\lambda_{s}\left(g\right)\right) \right\vert &< \ve,
  \end{align*}
  for all $\ve > 0$, so $\mu\left(g\right) = \mu\left(\lambda_{s}\left(g\right)\right)$.
\end{proof}

We will now commence with the reverse direction, starting by showing that amenability implies the existence of an approximate mean, and then showing that the existence of an approximate mean implies that the Følner condition is satisfied.
\begin{proposition}
  If $G$ is amenable, then $G$ admits an approximate mean.
\end{proposition}
\begin{proof}
  Suppose $G$ does not admit an approximate mean. Then, there exists a finite subset $E_0\subseteq G$ and $\ve_0 > 0$ such that for all $s\in E_0$ and all $f\in \Prob(G)$, we have $\norm{f - \lambda_s\left(f\right) \geq \ve_0}$.\newline

  Consider the set
  \begin{align*}
    X &= \bigoplus_{\left\vert E_0 \right\vert} \ell_1\left(G\right),
  \end{align*}
  endowed with the norm
  \begin{align*}
    \norm{\left(f_s\right)_{s\in E_0}} &= \sum_{s\in E_0}\sum_{t\in G}\left\vert f_s(t) \right\vert\\
                                       &= \sum_{s\in E_0}\norm{f_s}_{1},
  \end{align*}
  and let
  \begin{align*}
    C &= \set{\left(f - \lambda_s\left(f\right)\right)_{s\in E_0} | f\in \Prob(G)}.
  \end{align*}
  Since $\Prob(G)$ is convex, it is the case that $C$ is convex, and since $\left\vert E_0 \right\vert$ is finite, $C$ is necessarily bounded. Note that $0\notin \overline{C}$.\newline

  By the Hahn--Banach separation theorem for convex sets, there is a real-valued $\varphi\in X^{\ast}$ such that $\varphi\left(C\right)\geq 1$. Here,
  \begin{align*}
    X^{\ast} &\cong \bigoplus_{\left\vert E_0 \right\vert}\ell_1\left(G\right)^{\ast}\\
             &\cong \sum_{\left\vert E_0 \right\vert}\ell_{\infty}\left(G\right),
  \end{align*}
  endowed with the norm
  \begin{align*}
    \norm{\left(g_s\right)_{s\in E_0}} &= \max_{s\in E_0}\left(\sup_{t\in G}\left\vert g_s(t) \right\vert\right)\\
                                       &= \max_{s\in E_0}\norm{g_s}_{\infty}.
  \end{align*}
  We let $\varphi = \left(\varphi_{g_s}\right)_{s\in E_0}$, where $g_s\in \ell_{\infty}\left(G\right)$ is defined by the duality
  \begin{align*}
    \varphi_{g_s}\left(f\right) &= \sum_{t\in G}f(t)g_s(t).
  \end{align*}
  Thus, for all $f\in \Prob(G)$, we have
  \begin{align*}
    1 &\leq \varphi\left(\left(f - \lambda_s\left(f\right)\right)_{s\in E_0}\right)\\
      &= \sum_{s\in E_0}\varphi_{g_s}\left(f - \lambda-s\left(f\right)\right)\\
      &= \sum_{s\in E_0}\sum_{t\in G}\left(f - \lambda_s\left(f\right)\right)(t)g_s(t)\\
      &= \sum_{s\in E_0}\left(\sum_{t\in G}f(t)g_s(t) - \sum_{t\in G}f\left(s^{-1}t\right)g_s(t)\right)\\
      &= \sum_{s\in E_0}\left(\sum_{t\in G}f(t)g_s(t) - \sum_{r\in G}f\left(r\right)g_s\left(sr\right)\right)\\
      &= \sum_{s\in E_0}\left(\sum_{r\in G}f(r)g_s(r) - \sum_{r\in G}f(r)\lambda_{s^{-1}}\left(g\right)(r)\right)\\
      &= \sum_{s\in E_0}\sum_{r\in G}f(r)\left(g_s - \lambda_{s^{-1}}\left(g\right)\right)(r).
      \intertext{Note that this holds for any $f\in \Prob(G)$, including the case of $f = \delta_t$ for a given $t\in G$. We must have}
      &= \sum_{s\in E_0}\sum_{r\in G}\delta_{t}\left(r\right)\left(g_s\left(r\right) - \lambda_{s^{-1}}\left(g_s\right)\right)\left(r\right)\\
      &= \sum_{s\in E_0}\left(g_s - \lambda_{s^{-1}}\left(g\right)\right)\left(t\right).
      \intertext{In particular, we must have}
      &\geq \1_{G}.
  \end{align*}
  Since $G$ is amenable, there is a mean $\mu\colon \ell_{\infty}\left(G\right)\rightarrow \C$ with $\mu\left(g_s\right) = \mu\left(\lambda_{s^{-1}}\left(g_s\right)\right)$, meaning
  \begin{align*}
    0 &= \mu\left(\sum_{s\in E_0}\left(g_s - \lambda_{s^{-1}}\left(g_s\right)\right)\left(t\right)\right)\\
      &\geq \mu\left(\1_{G}\right)\\
      &= 1,
  \end{align*}
  which is a contradiction.
\end{proof}
To show that the existence of an approximate mean implies the Følner condition, we require the following lemma.
\begin{lemma}\label{lemma:layer_cake_representation}
  Let $f\colon S\rightarrow \R$ be finitely supported with $\sum_{s\in S}f(s) = 1$. Then, there exist subsets $\set{F_i}_{i=1}^{n}$, where $F_1\supseteq F_2\supseteq \cdots \supseteq F_n$, and constants $\set{c_i}_{i=1}^{n}$, such that
  \begin{align*}
    f &= \sum_{i=1}^{n}c_i\1_{F_i},
  \end{align*}
  where
  \begin{align*}
    \sum_{i=1}^{n}c_i\left\vert F_i \right\vert &= 1.
  \end{align*}
  This is known as the layer cake representation for $f$.
\end{lemma}
\begin{proof}[Proof of Lemma \ref{lemma:layer_cake_representation}:]
  We define $F_1 = \supp\left(f\right)$, and take $c_1 = \min\left(\Ran\left(f\right)\right)$. Taking $E_1 = f^{-1}\left(c_1\right)$ (as a set-theoretic inverse), we define $F_2 = F_1\setminus E_1$.\newline

  Take $d_1 = \min\left(f\left(F_2\right)\right)$, and define $c_2 = d_1 - c_1$. Then, defining $E_2 = f^{-1}\left(d_1\right)$, $F_3 = F_2 \setminus E_2$, and $d_2 = \min\left(f\left(F_3\right)\right)$, we define $c_3 = d_2 - c_2 - c_1$.\newline

  Continuing in this pattern, we find $d_{i-1} = \min\left(f\left(F_i\right)\right)$, $E_i = f^{-1}\left(d_{i-1}\right)$, and $c_i = d_{i-1} - \sum_{j=1}^{i-1}c_i$.\newline

  This yields a decomposition $F_1\supseteq F_2\supseteq \cdots \supseteq F_n$, where $\sum_{i=1}^{n}c_i\1_{F_i} = f$ by construction.\newline

  We now verify that $\sum_{i=1}^n c_i\left\vert F_i \right\vert = 1$.
  \begin{align*}
    1 &= \sum_{s\in S}f(s)\\
      &= \sum_{s\in S}\sum_{i=1}^{n}c_i\1_{F_i}\left(s\right).
      \intertext{By definition, if $s\in F_j$ for some $j$, we see that $c_j$ is summed for $\left\vert F_j \right\vert$ many times. Thus, we obtain}
      &= \sum_{i=1}^{n}c_i\left\vert F_i \right\vert.
  \end{align*}
\end{proof}

We will use the layer cake decomposition to prove that if $G$ admits an approximate mean, then $G$ satisfies the Følner condition.
\begin{proposition}
  Let $G$ admit an approximate mean. Then, $G$ satisfies the Følner condition.
\end{proposition}
\begin{proof}
  Let $\left(f_k\right)_k$ be an approximate mean, as in Definition \ref{def:approximate_mean}. Fix a finite nonempty set $S \subseteq G$. Then, by the definition of an approximate mean, there must exist some $N\in\N$ such that for all $k\geq N$ and all $s\in G$,
  \begin{align*}
    \norm{f_k - \lambda_s\left(f_k\right)}_{1} &\leq \frac{\ve}{|S|}.
  \end{align*}
  In particular, this holds for $f_N$ and for all $s\in S$.\newline

  Since $f_N\in \Prob(G)$ is finitely supported and $\sum_{s\in G}f_N(s) = 1$, we may use Lemma \ref{lemma:layer_cake_representation} to rewrite $f_N$ as
  \begin{align*}
    f_N &= \sum_{i=1}^{n}c_i\1_{F_i},
  \end{align*}
  where $F_1 \supseteq F_2\supseteq \cdots \supseteq F_n$, and $\sum_{i=1}^{n}c_i\left\vert F_i \right\vert = 1$. 
\end{proof}

Thus far, we have shown the following to be equivalent for a discrete group $G$:
\begin{enumerate}[(1)]
  \item $G$ is non-paradoxical;
  \item $G$ is amenable;
  \item $G$ admits an invariant state;
  \item $G$ admits an approximate mean;
  \item $G$ satisfies the Følner condition.
\end{enumerate}
The equivalence between (1) and (2) follows from Tarski's theorem (Theorem \ref{thm:tarski}), the equivalence between (2) and (3) follows from Propositions \ref{prop:state_implies_mean} and \ref{prop:mean_implies_state}, and the equivalence between (3), (4), and (5) follows from 

\chapter{Characterizations through Fixed Points}
\chapter{Characterizations using $C^{\ast}$-Algebras}
\appendix
\chapter{Point-Set Topology}
We will need a bit of background in point-set topology in order to satisfactorily understand the functional analysis behind the results in Chapters 3, 4, and 5.
\section{Axioms of Set Theory}%
In order to garner sufficient understanding of point-set topology, we need to be able to comprehend some of the essential axioms behind the objects known as ``sets.'' This is where the axioms of set theory come into play.
\begin{definition}[Zermelo--Fraenkel Axioms]
  In Zermelo--Fraenkel set theory, all objects are sets. In order to maintain convention with the way the rest of this section will refer to sets, all sets will be referred to by capital letters, and all elements of sets by lowercase letters.
  \begin{itemize}
    \item Axiom of Existence: $\exists A\left(A = A\right)$. This axiom guarantees a nonempty universe.
    \item Axiom of Extensionality: $\forall x\left(x\in A \Leftrightarrow x\in B\right)\Rightarrow A = B$. This axiom states that if two sets share the same members, then the sets are equal.
    \item Axiom Schema of Comprehension: $\exists B\:\forall x\left(x\in B\Leftrightarrow x\in A \wedge\varphi(x)\right)$. This axiom states that for any formula $\varphi(x)$, where $x$ is a free variable, there is a set $B$ such that the members of $B$ are the members of $A$ for which $\varphi$ holds.
    \item Pairing Axiom: $\forall A\:\forall B\:\exists C\left(\left(A\in Z\right)\wedge \left(B\in Z\right)\right)$. This axiom states that for any sets $A$ and $B$, there is a set $C = \set{A,B}$ that contains the sets $A$ and $B$ as elements.
    \item Power Set Axiom: $\forall A\:\exists P(A)\:\forall B\left(B\in P(A) \Leftrightarrow B\subseteq A\right)$. We use the shorthand $B\subseteq A$ to write the statement $\forall x \left(x\in B\Rightarrow x\in A\right)$. This axiom states that for any set $A$ there exists a set $P(A)$ such that any element of $P(A)$ is a subset of $A$, and any subset of $A$ is an element of $P(A)$.
    \item Union Axiom: $\forall \mathcal{A}\:\exists A\:\forall Y\:\forall x\left(\left(x\in Y\wedge Y\in \mathcal{A}\right)\Rightarrow x\in A\right)$. This axiom states that for any collection $\mathcal{A}$, there is a set $A $, denoted $ \bigcup \mathcal{A}$, that contains all the elements of all the sets in the collection $\mathcal{A}$.
    \item Axiom of Infinity: $\exists A\left(\emptyset\in A\wedge \forall x\left(x\in A\Rightarrow x\cup\set{x}\in A\right)\right)$. This axiom states that there is a set, $A$, such that the empty set is in $A$ and, for any element $x$, if $x\in A$, then so too is the successor, $x\cup \set{x}$.
    \item Axiom of regularity: $\forall X\left(X\neq\emptyset \Rightarrow\exists Y\left(Y\in X\wedge Y\cap X = \emptyset\right)\right)$. This axiom states that any nonempty set $X$ contains a set $Y$ such that $Y$ and $X$ are disjoint. As a consequence, any chain of sets descending in membership must terminate.
    \item Axiom Schema of Replacement: $\forall A\:\exists B\:\forall v\left(v\in B\Rightarrow \exists u\left(u\in A\wedge \psi\left(u,v\right)\right)\right)$. The axiom schema of replacement says that for a function-like formula (a formula such that $\psi\left(u,v\right)\wedge \psi\left(u,w\right) \Rightarrow v=w$) $\psi\left(u,v\right)$, there is a set $A$ consisting of exactly those sets/elements $v\in B$ that correspond to $u\in A$.
  \end{itemize}
\end{definition}
The final axiom, the Axiom of Choice, is special, and as a result, we state it separately, for we will be using some of its consequences in the future sections. The following is one way of interpreting the axiom of choice.
\begin{definition}[Axiom of Choice]
  Let $\set{S_i}_{i\in I}$ be an indexed collection of nonempty sets. Then, there exists an indexed set $\set{x_i}_{i\in I}$ such that $x_i\in S_i$ for each $I$.\newline

  Equivalently, if $\set{S_i}_{i\in I}$ is an indexed collection of nonempty sets, then there is some choice function
  \begin{align*}
    f\in \prod_{i\in I}S_i.
  \end{align*}
\end{definition}
On its own, this formulation of the Axiom of Choice is not particularly useful. However, there is a statement of the Axiom of Choice which is just as useful.
\begin{definition}[Preorders, Partial Orders, Total Orders, and Well-Orders]
Let $X$ be a set, and $\preceq $ be a relation on $X$. We say a relation is a preorder if it is reflexive and transitive:
\begin{itemize}
  \item $a\preceq a$
  \item $a\preceq b \wedge b\leq c\Rightarrow a\preceq c$.
\end{itemize}
We say $X$ is a directed set if, for any $a,b\in X$, there is $c\in X$ such that $a\preceq c$ and $b\preceq c$.\newline

If $\preceq$ is also antisymmetric --- that is, $a\preceq b\wedge b\preceq a \Rightarrow a = b$ --- then, we say $\preceq$ is a partial order.\newline

We say $m\in X$ is a maximal element if, for any $x\in X$ with $m\preceq x$, $m = x$.\newline

If $X$ is partially ordered by $\preceq$ and, for any two elements $a,b\in X$, either $a\preceq b$ or $b\preceq a$, then we say $\preceq$ is a total order on $X$.\newline

If $X$ is a totally ordered set that has the property that, for any nonempty $A\subseteq X$, there is some $x\in A$ such that for any $y\in A$, $x\prec y$ for all $y \in A$ with $y\neq x$, then we say $\preceq$ is a well-order on $X$.
\end{definition}
\begin{example}
  \begin{itemize}
    \item The set $\N$ with the usual ordering is a well-ordered set.
    \item If $A$ is a set, then $P(A)$ with the containment ordering, $A\preceq B$ if $A\supseteq B$, is a partially ordered set.
    \item Similarly, if $A$ is a set, then $P(A)$ with the inclusion ordering, $A\preceq B$ if $A\subseteq B$, is a partially ordered set.
    \item A collection of functions $\set{\varphi_{i}\colon Z_i\rightarrow Y}_{i\in I}$ ordered by $\varphi_{i}\preceq \varphi_j$ if $Z_i\subseteq Z_j$ and $\varphi_{j}|_{Z_i} = \varphi_i$, is a partially ordered set. This is often known as the extension ordering.
  \end{itemize}
\end{example}

We can state an equivalent formulation of the Axiom of Choice as follows.
\begin{theorem}[Zorn's Lemma]
  If $\left(X,\preceq\right)$ is a partially ordered set with the property that for all $C\subseteq X$ with $C$ totally ordered, $C$ has an upper bound, then $X$ has a maximal element.
\end{theorem}
There are many proofs of both Zorn's Lemma from the Axiom of Choice and the Axiom of Choice from Zorn's Lemma. However, we will mostly be using it for the purposes of proving other theorems. The following results can be proven using Zorn's Lemma.
\begin{example}
  \begin{itemize}
    \item Every $\F$-vector space $V$ has a basis $B\subseteq V$ such that the set of all finite linear combinations of elements of $B$ over $\F$ is $V$.
    \item If $\varphi$ is a continuous linear functional defined on a subspace $W\subseteq V$, there is an extension $\Phi$ such that $\Phi|_{W} = \varphi$. This is one of the Hahn--Banach theorems. %See: Hahn--Banach Theorems
    \item The arbitrary product of compact spaces is compact. This is known as Tychonoff's Theorem. %See Tychonoff's Theorem.
  \end{itemize}
\end{example}
\section{Metric Spaces}%
Building upon the basics of set theory, we move towards understanding metric spaces.
\subsection{Basics of Metric Spaces}%
\begin{definition}[Metrics]
  Let $X$ be a set. A distance metric is a function
  \begin{align*}
    d\colon X\times X\rightarrow [0,\infty)
  \end{align*}
  such that the following three properties are satisfied:
  \begin{itemize}
    \item if $x,y\in X$ and $d\left(x,y\right) = 0$, then $x = y$;
    \item $d\left(x,y\right) = d\left(y,x\right)$ for all $x,y\in X$;
    \item $d\left(x,z\right) \leq d\left(x,y\right) + d\left(y,z\right)$ for all $x,y,z\in X$.
  \end{itemize}
  A function that satisfies the latter two properties is called a semimetric.\newline

  Two metrics $d$ and $\rho$ on $X$ are equivalent if there exist constants $c_1,c_2\geq 0$ such that
  \begin{align*}
    d\left(x,y\right) &\leq c_1 \rho\left(x,y\right)\\
    \rho\left(x,y\right) &\leq c_2 d\left(x,y\right)
  \end{align*}
  for all $x,y\in X$.\newline

  A metric space is a pair $\left(X,d\right)$, where $d$ is a metric.
\end{definition}
\begin{example}[Some Distance Metrics]
  \begin{itemize}
    \item The discrete metric on any nonempty set is given by
      \begin{align*}
        d\left(x,y\right) & \begin{cases}
          1 & x\neq y\\
          0 & x = y
        \end{cases}
      \end{align*}
    \item The Euclidean metric between $\left(x_1,\dots,x_n\right)$ and $\left(y_1,\dots,y_n\right)$ in $\R^n$ is
      \begin{align*}
        d_{2}\left(x,y\right) &= \left(\sum_{j=1}^{n}\left\vert y_j-x_j \right\vert^2\right)^{1/2}.
      \end{align*}
    \item Other metrics on $\R^n$ include
      \begin{align*}
        d_1\left(x,y\right) &= \sum_{j=1}^{n}\left\vert y_j-x_j \right\vert\\
        d_{\infty}\left(x,y\right) &= \max_{j=1}^{n}\left\vert y_j - x_j \right\vert.
      \end{align*}
      All of $d_1,d_2,d_{\infty}$ are equivalent metrics.
    \item The Hamming distance between two strings of bits is
      \begin{align*}
        d_{H}\colon \set{0,1}^{n}\times \set{0,1}^{n}\rightarrow [0,\infty)\\
        d_{H}\left(\left(x_{j}\right)_{j=1}^{n},\left(y_j\right)_{j=1}^{n}\right) &= \left\vert \set{j\mid x_j\neq y_j} \right\vert.
      \end{align*}
    \item The set $C\left([0,1],\R\right)$ consisting of continuous real-valued functions from $[0,1]$ to $\R$ can be equipped with
      \begin{align*}
        d_u\left(f,g\right) &= \sup_{t\in [0,1]}\left\vert f(t) - g(t) \right\vert,
      \end{align*}
      which is the uniform metric, or
      \begin{align*}
        d_{1}\left(f,g\right) &= \int_{0}^{1} \left\vert f(t)-g(t) \right\vert\:dt.
      \end{align*}
    \item All subsets of a metric space $X$ equipped with the same metric is also a metric space.
    \item If $\rho$ is a metric on $X$, then we can create a distance metric
      \begin{align*}
        d\left(x,y\right) &= \frac{\rho\left(x,y\right)}{1 + \rho\left(x,y\right)}
      \end{align*}
      that is bounded on $[0,1]$.
    \item If $d_1,\dots,d_n$ are metrics on $X$ and $c_1,\dots,c_n > 0$ are constants, then
      \begin{align*}
        d\left(x,y\right) &= \sum_{k=1}^{n}c_kd_k\left(x,y\right)
      \end{align*}
      defines a metric on $X$.
    \item If $\left(\rho_k\right)_k$ is a family of separating semimetrics for $X$ --- i.e., for $x,y\in X$ distinct, there is some $\rho_{j}$ such that $\rho_j\left(x,y\right) \neq 0$ --- then, we can obtain bounded semimetrics by taking
      \begin{align*}
        d_k\left(x,y\right) &= \frac{\rho_k\left(x,y\right)}{1 + \rho_k\left(x,y\right)}
      \end{align*}
      for each $k$. From each $d_k$, we define
      \begin{align*}
        d\left(x,y\right) &= \sum_{k=1}^{n}2^{-k}d_k\left(x,y\right),
      \end{align*}
      which is a metric on $X$.
    \item If $\left(X_k,\rho_k\right)_{k\geq 1}$ is a sequence of metric spaces, then we can form the product space
      \begin{align*}
        X &= \prod_{k\geq 1}X_{k}
      \end{align*}
      with the metric
      \begin{align*}
        D\left(f,g\right) &= \sum_{k\geq 1}d_k\left(f(k),g(k)\right).
      \end{align*}
      Here, $d_k = \frac{\rho_k}{1 + \rho_k}$ is the corresponding bounded metric to $\rho_k$.
  \end{itemize}
\end{example}
\begin{definition}[Open and Closed Sets]
  Let $\left(X,d\right)$ be a metric space.
  \begin{enumerate}[(1)]
    \item For $x\in X$ and $\delta > 0$, we define
      \begin{enumerate}[(a)]
        \item the open ball at $x$ with radius $\delta > 0$
          \begin{align*}
            U\left(x,\delta\right) &= \set{y\in X\mid d\left(y,x\right) < \delta};
          \end{align*}
        \item the closed ball centered at $x$ with radius $\delta > 0$
          \begin{align*}
            B\left(x,\delta\right) &= \set{y\in X\mid d\left(y,x\right)\leq \delta};
          \end{align*}
        \item the sphere centered at $x$ with radius $\delta > 0$
          \begin{align*}
            S\left(x,\delta\right) &= \set{y\in X\mid d\left(y,x\right) = \delta}.
          \end{align*}
      \end{enumerate}
    \item A set $V\subseteq X$ is open if, for all $x\in V$, there is $\delta > 0$ such that $U\left(x,\delta\right)\subseteq V$.\newline

      A subset $C\subseteq X$ is closed if $C^{c}$ is open.
    \item If $x\in V$ and $V\subseteq X$ is open, then we say $V$ is an open neighborhood of $x$. A neighborhood of $x$ is any subset $N\subseteq X$ such that $N$ contains an open neighborhood of $x$.
    \item If $A\subseteq X$ is any subset, the interior of $A$ is
      \begin{align*}
        A^{\circ} &:= \bigcup\set{V\mid V\text{ is open, }V\subseteq A},
      \end{align*}
      the closure of $A$ is
      \begin{align*}
        \overline{A} &= \bigcap\set{C\mid C\text{ is closed, }A\subseteq C},
      \end{align*}
      and the boundary of $A$ is
      \begin{align*}
        \partial A &= \overline{A} \setminus A^{\circ}.
      \end{align*}
  \end{enumerate}
\end{definition}
We can now talk about the topology of the metric space.
\begin{fact}
  Let $\left(X,d\right)$ be a metric space, and let
  \begin{align*}
    \mathcal{U} = \set{V\mid V\subseteq X\text{ open}}.
  \end{align*}
  Then, the following are true.
  \begin{itemize}
    \item $\emptyset\in \mathcal{U},X\in \mathcal{U}$.
    \item If $\set{V_{i}}_{i\in I}$ is a family of open sets, then $\bigcup_{i\in I}V_i\in \mathcal{U}$.
    \item If $\set{V_i}_{i=1}^{n}$ is a finite collection of open sets, then $\bigcap_{i=1}^{n}V_i \in \mathcal{U}$.
  \end{itemize}
\end{fact}

\begin{definition}
  Let $\left(X,d\right)$ be a metric space. Suppose $A\subseteq X$ is a nonempty subset.
  \begin{enumerate}[(1)]
    \item The distance from a point $x\in X$ to the set $A$ is defined by
      \begin{align*}
        \dist_{A}\left(x\right) &= \inf_{a\in A}d\left(x,a\right).
      \end{align*}
    \item The diameter of $A$ is defined by
      \begin{align*}
        \diam\left(A\right) &= \sup_{x,y\in A}d\left(x,y\right).
      \end{align*}
    \item If $\diam(A) < \infty$, then we say $A$ is bounded.
    \item If, for every $\delta > 0$, there is a finite subset $F_{\delta}\subseteq X$ such that
      \begin{align*}
        A\subseteq \bigcup_{x\in F_{\delta}}U\left(x,\delta\right).
      \end{align*}
    \item For $A,B\subseteq X$, we define the Hausdorff distance between $A$ and $B$ to be
      \begin{align*}
        d_{H}\left(A,B\right) &= \max\set{\sup_{x\in A}\dist_{B}\left(x\right),\sup_{y\in B}\dist_{A}\left(y\right)}.
      \end{align*}
  \end{enumerate}
\end{definition}
\begin{example}
  Let $\Omega$ be a nonempty set, and $\left(X,d\right)$ be a metric space. A function $f\colon \Omega\rightarrow X$ is said to be bounded if $\diam\left(\ran(f)\right) < \infty$.\newline

  The collection $\operatorname{Bd}\left(\Omega,X\right)$ denotes all bounded functions with domain $\Omega$ and codomain $X$.\newline

  On $ \operatorname{Bd}\left(\Omega,X\right)$, we define the uniform metric by
  \begin{align*}
    D_{u}\left(f,g\right) &= \sup_{x\in\Omega}d\left(f(x),g(x)\right).
  \end{align*}
\end{example}
\subsection{Convergence and Continuity in Metric Spaces}%
\begin{definition}
  Let $\left(X,d\right)$ be a metric space.
  \begin{enumerate}[(1)]
    \item A sequence in $X$ is a map $x\colon \N\rightarrow X$, which we call $\left(x_{n}\right)_{n}$ or $\left(x_{n}\right)_{n\geq 1}$.
    \item A natural sequence is a strictly increasing sequence of natural numbers $\left(n_{k}\right)_{k\geq 1}$ with $n_{k}\geq k$ and $n_{k} < n_{k+1}$.
    \item If $\left(n_k\right)_{k}$ is a natural sequence, the sequence $\left(x_{n_k}\right)_{k}$ is called a subsequence of $\left(x_{n}\right)_n$.
    \item We say $\left(x_n\right)_n\rightarrow x$ if $d\left(x_n,x\right)_{n} \xrightarrow{n\rightarrow\infty} 0$. We say $x$ is the limit of $\left(x_n\right)_n$.
  \end{enumerate}
\end{definition}
\begin{example}
  \begin{itemize}
    \item If $\Omega$ is a nonempty set, and $\left(X,d\right)$ is a metric space, the sequence of functions $f_n\colon \Omega\rightarrow X$ is said to converge pointwise to $f\colon \Omega\rightarrow X$ if
      \begin{align*}
        f_n\left(x\right)\xrightarrow{n\rightarrow\infty}f(x)
      \end{align*}
      for each $x\in \Omega$.
    \item If $\left(f_n\right)_n\in \operatorname{Bd}\left(\Omega,X\right)$ is a sequence, we say $\left(f_n\right)_n\rightarrow f$ converges uniformly if
      \begin{align*}
        D_u\left(f_n,f\right)\xrightarrow{n\rightarrow\infty}0,
      \end{align*}
      or, equivalently,
      \begin{align*}
        \sup_{x\in\Omega}d\left(f_n(x),f(x)\right)\xrightarrow{n\rightarrow\infty}0.
      \end{align*}
  \end{itemize}
\end{example}
\begin{definition}[Sequential Criteria for Closure]
  If $\left(X,d\right)$ is a metric space, and $E\subseteq X$ is nonempty, then $E$ is closed if and only if, for all $\left(x_n\right)_n\rightarrow x$ with $x_n\in E$, $x\in E$.\newline

  If $E\subseteq X$ is any nonempty set, then $\overline{E}$ is precisely the set of all $x\in X$ such that $\left(x_n\right)_n\rightarrow x$ for some $\left(x_n\right)_n\subseteq E$.
\end{definition}
\begin{definition}[Completeness]
  Let $\left(X,d\right)$ be a metric space.
  \begin{itemize}
    \item If $\left(x_n\right)_n$ is a sequence in $X$ such that for all $\ve > 0$, there is $N\in \N$ such that for all $m,n\geq N$, $d\left(x_m,x_n\right) < \ve$, then we say the sequence is called Cauchy.
    \item If, for any $\left(x_n\right)_n$ Cauchy, $\left(x_n\right)_n\rightarrow x$ in $X$, then we say $X$ is complete.
    \item If $\left(X,d\right)$ is complete, then for any $A\subseteq X$ closed, $A$ is also complete.
    \item If $A\subseteq X$ is complete as a metric space, then $A$ is closed.
  \end{itemize}
\end{definition}
\begin{example}
  The metric space $\Q$ with the metric inherited from $\R$ is not complete. For instance, there is a sequence of rational numbers $\left(2,2.7,2.71,2.718,\dots\right)$ converging to $e$, but $e\notin \Q$.\newline

  The space $\operatorname{Bd}\left(\Omega,X\right)$ is complete if $X$ is complete.
\end{example}
\begin{definition}[Continuity]
  \begin{itemize}
    \item Let $\left(X,d\right)$ and $\left(Y,\rho\right)$ be metric spaces, and let $f\colon X\rightarrow Y$ be a function. We say $f$ is continuous at $x$ if, for every $\ve > 0$, there is $\delta > 0$ such that $z\in U\left(x,\delta\right)\Rightarrow \rho\left(f(x),f(z)\right) < \ve$.
    \item If $f$ is continuous at every point in $X$, then we say $f$ is continuous.
    \item If $f$ is bijective, continuous, and $f^{-1}$ is continuous, then we say $f$ is a homeomorphism.
    \item We say $f$ is uniformly continuous on $X$ if, for any $\ve > 0$, there is $\delta > 0$ such that for any $y,z\in X$, $d\left(y,z\right) < \delta \Rightarrow \rho\left(f(y),f(z)\right) < \ve$.
    \item We say $f$ is Lipschitz if there exists $C > 0$ such that $d\left(x,y\right) \leq Cd\left(f(x),f(y)\right)$ for all $x,y\in X$.
    \item We say $f$ is an isometry if $d\left(x,y\right) = d\left(f(x),f(y)\right)$ for all $x,y\in X$.
  \end{itemize}
\end{definition}
\begin{fact}
  Let $f\colon X\rightarrow Y$ be a map between metric spaces. The following are equivalent:
  \begin{enumerate}[(i)]
    \item $f$ is continuous;
    \item if $V\subseteq Y$ is open, then $f^{-1}\left(V\right)\subseteq X$ is open;
    \item if $\left(x_n\right)_n\rightarrow x$ in $X$, then $\left(f\left(x_n\right)\right)_n\rightarrow f(x)$ in $Y$.
  \end{enumerate}
\end{fact}
\begin{fact}
  If $M$ and $N$ are metric spaces with $N$ complete, and $A\subseteq M$ is dense, then if $f\colon A\rightarrow N$ is uniformly continuous, then there is a unique uniformly continuous map $\tilde{f}\colon M\rightarrow N$.
\end{fact}
\begin{definition}
  Let $\left(X,d\right)$ and $\left(Y,\rho\right)$ be metric spaces.
  \begin{enumerate}[(1)]
    \item We say $X$ and $Y$ are homeomorphic if there is a homeomorphism $f\colon X\rightarrow Y$.
    \item We say $X$ and $Y$ are uniformly isomorphic if there is a uniformly continuous bijection $f\colon X\rightarrow Y$ with $f^{-1}$ uniformly continuous. Such an $f$ is called a metric space uniformism.
    \item We say $X$ and $Y$ are isometrically isomorphic if there is a bijective isometry $f\colon M\rightarrow N$.
  \end{enumerate}
\end{definition}
\begin{fact}
  If $X$ and $Y$ are uniformly isomorphic metric spaces with $X$ complete, then so too is $Y$.\newline

  If $d$ and $\rho$ are equivalent metrics on a set $X$, then the identity map
  \begin{align*}
    \id_{X}:\left(X,\rho\right)\rightarrow \left(X,d\right)
  \end{align*}
  is a metric space uniformism.
\end{fact}
\section{Topological Spaces}%
We can now move from metric spaces to the more general setting of topological spaces. This will enable us to understand certain properties (like openness, continuity, etc.) separate from the metric structure (or lack thereof) that a certain set is endowed.
\subsection{Definitions}%
\begin{definition}
  Let $X$ be a nonempty set. A topology on $X$ is a family of subsets $\tau$ satisfying
  \begin{enumerate}[(1)]
    \item $\emptyset\in \tau,X\in \tau$;
    \item if $\set{V_i}_{i\in I}\subseteq \tau$, then $\bigcup_{i\in I}V_i\in \tau$;
    \item if $\set{V_i}_{i=1}^{n}\subseteq \tau$, then $\bigcap_{i=1}^{n}V_i \in \tau$.
  \end{enumerate}
  If $\tau$ is a topology on $X$, then $\left(X,\tau\right)$ is called a topological space. We call members of $\tau$ open sets.\newline

  If $C\subseteq X$ and $C^{c}\in \tau$, then $C$ is called.\newline

  If $E$ is closed and open, it is called clopen.\newline

  A countable union of closed sets is called an $F_{\sigma}$ set, and a countable intersection of open sets is called a $G_{\delta}$ set.
\end{definition}
\begin{definition}
  If $X$ is a nonempty set, then the definition $\tau = P(X)$ is known as the discrete topology.\newline

  If $X$ is a nonempty set, and $\tau = \set{X,\emptyset}$, then we call $\tau$ the indiscrete topology.
\end{definition}
\begin{definition}
  Let $X$ be a nonempty set. Suppose $\tau_1,\tau_2\subseteq P(X)$ are two topologies on $X$. If $\tau_1\subseteq \tau_2$, then we say $\tau_1$ is weaker (or coarser) than $\tau_2$. We say $\tau_2$ is stronger (or finer) than $\tau_1$.
\end{definition}
\begin{definition}
  Let $X$ be a nonempty set, and suppose $\mathcal{E}\subseteq P(X)$ is a family of subsets. We define the topology on $X$ generated by $\mathcal{E}$ to be
  \begin{align*}
    \tau\left(\mathcal{E}\right) &= \bigcap\set{\tau | \tau\text{ is a topology on $X$, }\mathcal{E}\subseteq \tau}.
  \end{align*}
  In other words, $\tau\left(\mathcal{E}\right)$ is the weakest topology that contains the family $\mathcal{E}$.
\end{definition}
\begin{definition}
Let $\left(X,\tau\right)$ be a topological space. If $Y\subseteq X$ is a subset, then the subspace topology on $Y$ is defined by
\begin{align*}
  \tau_{Y} &= \set{V\cap Y | V\in \tau}.
\end{align*}

\end{definition}
\begin{definition}
  Let $\left(X,\tau\right)$ be a topological space, and let $A\subseteq X$ be a subset.
  \begin{enumerate}[(1)]
    \item The interior of $A$ is the open set $A^{\circ} = \bigcup\set{V | V\in\tau,~V\subseteq A}$.
    \item The closure of $A$ is the closed set $\overline{A} = \bigcap\set{C| C\text{ closed, }A\subseteq C}$.
    \item We say $A$ is dense if $\overline{A} = X$.
    \item We say $A$ is nowhere dense if $\left(\overline{A}\right)^{\circ} = \emptyset$.
  \end{enumerate}
  If $X$ admits a countable dense subset, then we say $X$ is separable.\newline

  If $X$ is the countable union of nowhere dense subsets, then we say $X$ is meager.
\end{definition}
\begin{remark}
  A set $A$ is dense if and only if, for any $U\in \tau$ with $U\neq \emptyset$, it is the case that $A\cap U \neq \emptyset$.
\end{remark}
\begin{fact}
  If $\left(M,d\right)$ is a separable metric space, and $E\subseteq M$ is a subset, then $E$ with the subspace topology is also separable.
\end{fact}
\begin{definition}
  Let $\left(X,\tau\right)$ be a topological space.
  \begin{itemize}
    \item An open neighborhood of $x_0$ is an open set $V\in \tau$ with $x_0\in V$. We write
      \begin{align*}
        \mathcal{O}_{x_0} &= \set{V | V\in \tau,x_0\in V}
      \end{align*}
      to denote the family of all open neighborhoods of $x_0$.
    \item If $N\subseteq X$ is a subset with $x_0\in V\subseteq N$, where $V\in \mathcal{O}_{x_0}$, then we say $N$ is a neighborhood of $x_0$. We write $\mathcal{N}_{x_0}$ to be the collection of neighborhoods of $x_0$.
    \item A neighborhood base for $\tau$ at $x_0$ is a family $\mathcal{O}\subseteq \mathcal{O}_{x_0}$ with such that for all $U\in \mathcal{O}_{x_0}$, there is $V\in \mathcal{O}$ with $V \subseteq U$.
    \item We say $\left(X,\tau\right)$ is first countable if every $x\in X$ admits a countable neighborhood base.
    \item A base for $\tau$ is a family $\mathcal{B}\subseteq \tau$ that contains a neighborhood base for $\tau$ at $x_0$ For each $x_0\in X$.
    \item We say $\left(X,\tau\right)$ is second countable if it admits a countable base.
  \end{itemize}
\end{definition}
\begin{fact}
  If $\mathcal{B}$ is a base for $\tau$, then every $U\in \tau$ can be written as a union $U= \bigcup_{i\in I}B_i$, where $B_i\in \mathcal{B}$.
\end{fact}
\begin{fact}
  All metric spaces are first-countable, with a neighborhood base of
  \begin{align*}
    \mathcal{O}_{x_0} &= \set{U\left(x_0,1/n\right)|n\in \N}
  \end{align*}
  for each $x_0\in X$.
\end{fact}
\begin{fact}
  A metric space $\left(X,d\right)$ is second countable if and only if it is separable.
\end{fact}
\begin{fact}
  If $X$ is a topological space, and $x_0\in X$ has a countable neighborhood base, then there is a neighborhood base $\left(V_n\right)_{n\geq 1}$ with $V_1\supseteq V_2\supseteq \cdots$.
\end{fact}
\subsection{Continuity in Topological Spaces}%
\begin{definition}
  Let $\left(X,\tau\right)$ and $\left(Y,\sigma\right)$ be topological spaces, and let $f\colon X\rightarrow Y$ be a map.
  \begin{enumerate}[(1)]
    \item We say $f$ is continuous at $x_0\in X$ if, for every $U\in \mathcal{O}_{f\left(x_0\right)}$, there is $V\in \mathcal{O}_{x}$ with $f\left(V\right)\subseteq U$.
    \item We say $f$ is continuous if $f$ is continuous at every point in $X$.
    \item We say $f$ is a homeomorphism if $f$ is a continuous bijection with a continuous inverse.
    \item We say $f$ is an open map if $U\in \tau$ implies $f\left(U\right)\in \sigma$. Similarly, we say $f$ is a closed map if $C\subseteq X$ closed implies $f\left(C\right)\subseteq Y$ is closed.
    \item We say $f$ is a quotient map if $f$ is surjective with $V\subseteq Y$ open if and only if $f^{-1}\left(V\right)\subseteq X$ open.
    \item We say $f$ is an embedding if $f\colon X\rightarrow \Ran\left(f\right)$ is a homeomorphism, where $\Ran\left(f\right)$ is endowed with the subspace topology.
    \item We write $C\left(X,Y\right)$ to be the continuous functions from $X$ to $Y$. If $Y = \C$ with the regular topology, then we write $C\left(X\right)$.
  \end{enumerate}
\end{definition}
\begin{fact}
  A function $f\colon X\rightarrow Y$ is continuous if and only if $f^{-1}\left(U\right)\subseteq X$ is open for every open $U\subseteq Y$. Equivalently, $f$ is continuous if and only if $f^{-1}\left(C\right)\subseteq X$ is closed for every closed $C\subseteq Y$.
\end{fact}
\begin{definition}[Separation Axioms]
Let $\left(X,\tau\right)$ be a topological space.
\begin{itemize}
  \item We say $X$ is T1 if $\set{x}$ is closed for every $x\in X$.
  \item We say $X$ is T2 (or Hausdorff) if, for every $x,y\in X$ with $x\neq y$, there are $U,V\in \tau$ with $x\in U$, $y\in V$, and $U\cap V = \emptyset$.
  \item We say $X$ is T3 if, for every $x\in X$ and $B\subseteq X$ closed with $x\notin B$, there are $U,V\in \tau$ with $x\in U$, $B\subseteq V$, and $U\cap V = \emptyset$. If $X$ is T1 and T3, we say $X$ is regular.
  \item We say $X$ is T3.5 if, for every $x_0\in X$ and closed $B\subseteq X$ with $x_0\notin B$, there is a continuous function $f\colon X\rightarrow [0,1]$ with $f\left(x_0\right) = 0$ and $f\left(B\right) = 1$. If $X$ is T1 and T3.5, we say $X$ is completely regular.
  \item We say $X$ is T4 if, for every pair of closed subsets $A,B\subseteq X$ with $A\cap B = \emptyset$, there are $U,V\in \tau$ with $A\subseteq U$, $B\subseteq V$, and $U\cap V = \emptyset$. If $X$ is T1 and T4, then we say $X$ is normal.
\end{itemize}
\end{definition}
Just as we defined completely regular spaces through the existence of certain continuous functions that act to separate points, we can completely classify normality through a separating family of continuous functions.
\begin{theorem}[Urysohn's Lemma]
  Let $\left(X,\tau\right)$ be a topological space. It is the case that $X$ is normal if and only if for every pair of disjoint closed subsets $A,B\subseteq X$, there is a continuous function $f\colon X\rightarrow [0,1]$ with $f\left(A\right) = 0$ and $f\left(B\right) = 1$.
\end{theorem}
\begin{remark}
  Metric spaces are an example of normal spaces.
\end{remark}
\subsection{Initial and Final Topologies}%
\begin{definition}
  Let $X$ be a set, and suppose $\set{\left(Y_i,\tau_i\right)}_{i\in I}$ is a family of topological spaces with corresponding maps $\set{f_i\colon X\rightarrow Y_i}_{i\in I}$. Setting
  \begin{align*}
    \ve &= \set{f_i^{-1}\left(V\right) | V_i\in \tau_i},
  \end{align*}
  and letting $\tau = \tau\left(\ve\right)$ be the topology on $X$ generated by $\ve$, we say $\tau$ is the initial topology on $X$ induced by the maps $\set{f_i}_{i\in I}$.\newline

  Specifically, $\tau$ is the weakest topology on $X$ such that each $f_i$ is continuous.
\end{definition}
\begin{definition}[Product Topology]
  Let $\set{\left(X_i,\tau_i\right)}_{i\in I}$ be a family of topological spaces. The topology on the product $\prod_{i\in I}X_i$ is defined to be the initial topology induced by the family of projection maps,
  \begin{align*}
    \pi_j\colon \prod_{i\in I}X_i\rightarrow X_j,
  \end{align*}
  defined by $\pi_j\left(\left(x_i\right)_{i\in I}\right) = x_j$.\newline
  
  For each $U\subseteq X_i$ open, we have $\pi_j^{-1}\left(U\right) = \prod_{i\in I}U_I$, where $U_i = X_i$ for $i\neq j$, and $U_j = U$. A base for this topology is the collection
  \begin{align*}
    \mathcal{B} = \set{\prod_{i\in I}U_i | U_i = X_i\text{ for all but finitely many open }U_j\subseteq X_j}.
  \end{align*}
  If we consider $X_i = X$ for all $i$, there is a bijection between $X^I \coloneq \set{f | f\colon I\rightarrow X}$, the set of all functions from $I$ to $X$, and $\prod_{i\in I}X_i$, with the map $f \mapsto \left(f\left(i\right)\right)_{i\in I}$. The product topology on $X^J$ coincides with the topology of pointwise convergence.
\end{definition}
\begin{definition}[Final Topology]
  Let $\left(X,\tau\right)$ be a topological space, $Y$ a nonempty set, and suppose $q\colon X\rightarrow Y$ is a surjection. Then, the collection
  \begin{align*}
    \tau_q &\coloneq \set{V\subseteq Y | q^{-1}\left(V\right)\in \tau}
  \end{align*}
  is what is known as the final (or quotient) topology on $Y$ produced by $q$.
\end{definition}
\subsection{Convergence in Topological Spaces}%
Given a non-first-countable space $X$ and a subset $A\subseteq X$, it is not necessarily the case that $x\in \overline{A}$ is the limit of a sequence $\left(x_n\right)_n$. However, we know that the sequential characterization of properties like closure, compactness (which will be covered in an upcoming section), and continuity is useful, so we want to generalize these ideas to non-first-countable spaces. This is where we can use nets.
\begin{definition}[Nets]
  A net is a map $A\rightarrow X$, where $\alpha \mapsto x_{\alpha}$, where $A$ is a directed set. We write nets as $\left(x_{\alpha}\right)_{\alpha}$.
\end{definition}
\begin{example}[Some Directed Sets]
  \begin{enumerate}[(1)]
    \item The natural numbers, $\N$, or the real numbers, $\R$, equipped with their usual ordering, are examples of directed sets. Every totally ordered set is directed.
    \item If $S$ is any set, the collection $F(S)$ consisting of all finite subsets of $S$ is directed by inclusion.
    \item The collection of finite partitions over a closed and bounded interval, $\mathcal{P}\left(\left[a,b\right]\right)$ is by the partition norm. If $P = \set{x_j}_{j=0}^{n}$ and $Q=\set{y_j}_{j=0}^{m}$ are partitions, we define
      \begin{align*}
        \norm{P} &= \max_{1\leq j\leq n}\left\vert x_j-x_{j-1} \right\vert\\
        \norm{Q} &= \max_{1\leq j \leq m}\left\vert y_{j} - y_{j-1} \right\vert,
      \end{align*}
      and the preorder that $P \leq Q$ if and only if $\norm{P} \geq \norm{Q}$. In other words, we say $P\leq Q$ if $Q$ is finer than $P$.\newline

      For any partitions $P$ and $Q$, their common refinement is a supremum for both --- $P\vee Q \geq P,Q$ for each partition.\footnote{This is extremely useful in defining the Riemann integral.}
    \item Let $\left(X,\tau\right)$ be a topological space, and for every $x$, we order the $\mathcal{O}_{x}$ by containment. That is, for elements $U,V\in \mathcal{O}_{x}$, we set $U\leq V$ if and only if $U\supseteq V$. This is a directed set by reverse inclusion, as we can always take $U\cap V \subseteq U,V$ (since both $U$ and $V$ contain $x$).\newline

      Similarly, the neighborhood system at $x$, $\mathcal{N}_x$, is also directed by containment.
    \item If $A$ and $B$ are directed sets, then $A\times B$ with the Cartesian ordering --- $\left(\alpha_1,\beta 1\right) \leq \left(\alpha_2,\beta 2\right)$ if and only if $\alpha_1\leq \alpha_2$ and $\beta_1 \leq \beta_2$ --- is also a directed set.
  \end{enumerate}
\end{example}
\begin{example}[Some Nets]
\begin{enumerate}[(1)]
  \item Any sequence $\left(x_k\right)_{k\in \N}$ is a net.
  \item Let $F\left(\Omega\right)$ be the set of all finite subsets of $\Omega$ directed by inclusion. Let $f\colon \Omega\rightarrow \C$ be a map. Then, we have a net $\left(s_F\right)_{F\in F\left(\Omega\right)}$ defined by
    \begin{align*}
      s_F &= \sum_{x\in F}f\left(x\right).
    \end{align*}
  \item Consider the collection of partitions $\mathcal{P}\left([a,b]\right)$ directed by the partition norm. For a bounded function $f\colon [a,b]\rightarrow \R$ and a partition $P = \set{x_j}_{j=0}^{n}$, for each $j$ we set
    \begin{align*}
      M_j\left(P\right) &= \sup_{t\in \left[x_j,x_{j-1}\right]}f(t)\\
      m_j\left(P\right) &= \inf_{t\in \left[x_j,x_{j-1}\right]}f(t).
    \end{align*}
    We obtain two nets, $U,L\colon \mathcal{P}\left([a,b]\right)$, defined by
    \begin{align*}
      U\left(P\right) &= \sum_{j=1}^{n}M_j\left(P\right)\left(x_j - x_{j-1}\right)\\
      L\left(P\right) &= \sum_{j=1}^{n}m_j\left(P\right)\left(x_j-x_{j-1}\right).
    \end{align*}
    These are known as the upper and lower Darboux sums.
\end{enumerate}
\end{example}
\begin{definition}
  Let $\left(X,\tau\right)$ be a topological space, and let $\left(x_{\alpha}\right)_{\alpha}$ be a net in $X$.
  \begin{enumerate}[(1)]
    \item For a set $U\subseteq X$, we say $\left(x_{\alpha}\right)_{\alpha}$ is eventually in $U$ if there is $\alpha_0\in A$ such that $x_{\alpha}\in U$ for all $\alpha \geq \alpha_0$.
    \item We say the net $\left(x_{\alpha}\right)_{\alpha}$ converges to $x\in X$ if, for every $U\in \mathcal{O}_{x}$, $\left(x_{\alpha}\right)_{\alpha}$ is eventually in $U$. We write $\left(x_{\alpha}\right)_{\alpha}\xrightarrow{\tau}x$, though if the topology is clear from context the $\tau$ is not written.
    \item For a given $U\subseteq X$, we say $\left(x_{\alpha}\right)_{\alpha}$ is frequently in $U$ if for every $\beta \in A$, there is $\alpha \in A$ with $\alpha \geq \beta$ and $x_{\alpha}\in U$.
    \item A point $x\in X$ is known as a cluster point of the net $\left(x_{\alpha}\right)_{\alpha}$ if for every $U\in \mathcal{O}_{x}$, $\left(x_{\alpha}\right)_{\alpha}$ is frequently in $U$. That is, for all $U\in \mathcal{O}_{x}$ and for all $\beta \in A$, there exists $\alpha \in A$ with $\alpha \geq \beta$ and $x_{\alpha}\in U$.
  \end{enumerate}
\end{definition}
\begin{fact}[Characterizations Using Nets]
  Let $\left(X,\tau\right)$ and $\left(Y,\sigma\right)$ be topological spaces, $E\subseteq X$ a subset, and $f\colon X\rightarrow Y$ a map.
  \begin{itemize}
    \item It is the case that $x\in \overline{E}$ if and only if there is a net $\left(x_{\alpha}\right)_{\alpha}$ in $E$ with $\left(x_{\alpha}\right)_{\alpha}\rightarrow x$.
    \item A map $f$ is continuous if and only if for every convergent net $\left(x_{\alpha}\right)_{\alpha}\xrightarrow{\tau}x$, we have $\left(f\left(x_{\alpha}\right)\right)_{\alpha}\xrightarrow{\sigma}f(x)$.
    \item If $X$ is given by the initial topology induced by the family of maps $\set{f_i\colon X\rightarrow \left(Y_i,\tau_i\right)}_{i\in I}$, the net $\left(x_{\alpha}\right)_{\alpha}$ converges to $x$ if and only if $\left(f_i\left(x_{\alpha}\right)\right)_{\alpha}\xrightarrow{\tau_i}f_i\left(x\right)$ in $Y_i$ for each $i\in I$.
    \item If $\set{\left(X_i,\tau_i\right)}_{i\in I}$ is a family of topological spaces, with $X = \prod_{i\in I}X_i$ equipped with the product topology, then a net $\left(x_{\alpha}\right)_{\alpha}$ in $X$ converges to $x\in X$ if and only if $\left(x_{\alpha}\left(i\right)\right)_{\alpha}\xrightarrow{\tau_i} x\left(i\right)$ in $X_i$ for each $i\in I$.
  \end{itemize}
\end{fact}

\chapter{Measure Theory and Integration}
In order to properly discuss amenability, we need a strong foundation in measure theory.

\section{Constructing Measurable Spaces}%
Fix a set $\Omega$. We let $\mathcal{A} =\set{A_{i}}_{i\in I}$ be a collection of subsets of $\Omega$.
\begin{definition}[Algebra of Subsets]
  The collection $\mathcal{A} = \set{A_i}_{i\in I}$ is known as an \textit{algebra of subsets} for $\Omega$ if
  \begin{itemize}
    \item $\emptyset,\Omega \in \mathcal{A}$;
    \item for any $A_i\in \mathcal{A}$, $A_i^{c}\in \mathcal{A}$;
    \item for any $A_i,A_j\in \mathcal{A}$, $A_i \cup A_j \in \mathcal{A}$.
  \end{itemize}
\end{definition}
We can refine the concept of an algebra of subsets to consider countable unions rather than finite unions.
\begin{definition}[$\sigma$-Algebra of Subsets]
  The collection $\mathcal{A} = \set{A_i}_{i\in I}$ is known as a \textit{$\sigma$-algebra of subsets} for $\Omega$ if
  \begin{itemize}
    \item $\emptyset,\Omega \in \mathcal{A}$;
    \item for any $A_i\in \mathcal{A}$, $A_i^{c}\in \mathcal{A}$;
    \item for any countable collection $\set{A_n}_{n\geq 1}\subseteq \mathcal{A}$, $\bigcup_{n\geq 1}A_{n} \in \mathcal{A}$.
  \end{itemize}
\end{definition}
\begin{definition}[Measurable Space]
A pair $\left(\Omega,\mathcal{A}\right)$, where $\Omega$ is a set and $A\subseteq P(\Omega)$ is a $\sigma$-algebra, is called a \textit{measurable space}. Elements in the measurable space are called $\mathcal{A}$-measurable sets.
\end{definition}
\begin{definition}[Restriction of a $\sigma$-Algebra]
  For a measurable space $\left(\Omega,\mathcal{A}\right)$, with $E\in \mathcal{A}$, the family
  \begin{align*}
    \mathcal{A}_{E} &= \set{E\cap A\mid A\in \mathcal{A}}
  \end{align*}
  is a $\sigma$-algebra on $E$, known as the \textit{restriction} of $\mathcal{A}$ to $E$.
\end{definition}
\begin{definition}[Produced $\sigma$-Algebra]
Let $\left(\Omega,\mathcal{A}\right)$ be a measurable space, and $f\colon \Omega\rightarrow \Lambda$ is a map. The $\sigma$-algebra \textit{produced} by $f$ on $\Lambda$ is the collection
\begin{align*}
  \mathcal{N} &= \set{E\mid E\subseteq \Lambda,~f^{-1}(E) \in \mathcal{A}}.
\end{align*}
\end{definition}
\begin{definition}[Generated $\sigma$-Algebra]
  For a family $\mathcal{E}\subseteq P\left(\Omega\right)$, the $\sigma$-algebra \textit{generated} by $E$ is the smallest $\sigma$-algebra that contains $E$.
  \begin{align*}
    \sigma\left(\mathcal{E}\right) &= \bigcap\set{\mathcal{M} | \mathcal{E}\subseteq \mathcal{M},\mathcal{M}\text{ is a $\sigma$-algebra}}.
  \end{align*}
\end{definition}
\begin{definition}[Borel $\sigma$-Algebra]
  If $\Omega$ is a topological space with the topology $\tau\subseteq P(\Omega)$, we define
  \begin{align*}
    \mathcal{B}_{\Omega} &= \sigma\left(\tau\right)
  \end{align*}
  to be the \textit{Borel $\sigma$-algebra}.
\end{definition}
All open, closed, clopen, $F_{\sigma}$, and $G_{\delta}$ subsets of $\Omega$ are Borel.\break

We can now begin examining measurable functions.
\begin{definition}[Measurable Functions]
  Let $\left(\Omega,\mathcal{M}\right)$ and $\left(\Lambda,\mathcal{N}\right)$ be measurable spaces.
  \begin{enumerate}[(1)]
    \item We say a map $f\colon \Omega\rightarrow \Lambda$ is \textit{$\mathcal{M}$-$\mathcal{N}$-measurable} if $f^{-1}\left(E\right)\in \mathcal{M}$ for all $E\in \mathcal{N}$.
    \item We say a map $f\colon \Omega\rightarrow \R$ is measurable if it is $\mathcal{M}$-$\mathcal{B}_{\R}$-measurable.
    \item We say a map $f\colon \Omega\rightarrow \C$ is measurable if both $\re(f)$ and $\im(f)$ are measurable.
  \end{enumerate}
  The set of all measurable functions on $\left(\Omega,\mathcal{M}\right)$ is denoted $L_{0}\left(\Omega,\mathcal{M}\right)$.\newline

  The collection of all bounded measurable functions is the set
  \begin{align*}
    B_{\infty}\left(\Omega,\mathcal{M}\right) &= \set{f\in L_0\left(\Omega,\mathcal{M}\right)\mid \sup_{x\in\Omega}\left\vert f(x) \right\vert < \infty}.
  \end{align*}
\end{definition}

\begin{example}
  If $f\colon \Omega\rightarrow \Lambda$ is a continuous map between topological spaces, then $f$ is $\mathcal{B}_{\Omega}$-$\mathcal{B}_{\Lambda}$-measurable, since
  \begin{align*}
    \mathcal{F} &= \set{E\subseteq \Lambda\mid f^{-1}\left(E\right)\in \mathcal{B}_{\Omega}}
  \end{align*}
  is a $\sigma$-algebra containing every open set in $\Lambda$, so $\mathcal{F}$ contains $\mathcal{B}_{\Lambda}$.
\end{example}

\begin{example}
  If $\left(\Omega,\mathcal{M}\right)$ is a measurable space, and $f\colon \Omega\rightarrow \Lambda$ is a map, the measurable space $\left(\Lambda,\mathcal{N}\right)$ produced by $f$ is necessarily $\mathcal{M}$-$\mathcal{N}$-measurable.
\end{example}

\begin{fact}
  If $\left(\Omega,\mathcal{M}\right)$, $\left(\Lambda,\mathcal{N}\right)$, and $\left(\Sigma,\mathcal{L}\right)$ are measurable spaces, with $f\colon \Omega\rightarrow \Lambda$ and $g\colon \Lambda\rightarrow \Sigma$ measurable, then $g\circ f$ is measurable.\label{fact:composition}
\end{fact}
%\begin{proof}[Proof of Fact \ref{fact:composition}]
%  If $E\in \mathcal{L}$, then $g^{-1}\left(E\right) \in \mathcal{N}$, so $f^{-1}\left(g^{-1}\left(E\right)\right)\in \mathcal{M}$. Thus, $\left(g\circ f\right)^{-1}\left(E\right)\in \mathcal{M}$, so $g\circ f$ is measurable.
%\end{proof}

\begin{proposition}
  Let $\left(\Omega,\mathcal{M}\right)$ be a measurable space. Let $\F = \C$ or $\R$. Suppose $f,g,h_n\colon \Omega\rightarrow \F$ are measurable for $n\geq 1$.
  \begin{enumerate}[(1)]
    \item If $\alpha \in \F$, then $f + \alpha g$ is measurable.
    \item $\overline{f}$ is measurable.
    \item $fg$ is measurable.
    \item $\frac{f}{g}$ is measurable assuming it is well-defined.
    \item if $h_n$ are $\R$-valued, and $\left(h_n\left(x\right)\right)_n$ is bounded for each $x\in \Omega$, then $\sup h_n$ and $\inf h_n$ are measurable.
    \item If $f$ and $g$ are $\R$ valued, then $\max\left(f,g\right)$ and $\min\left(f,g\right)$ are measurable. In particular,
      \begin{align*}
        f_{+} &= \max\left(f,0\right)\\
        f_{-} &= \max\left(0,-f\right)
      \end{align*}
      are measurable.
    \item $\left\vert f \right\vert$ is measurable.
    \item The pointwise limit of measurable functions is measurable --- if $\lim_{n\rightarrow\infty}h_n\left(x\right)$ exists for all $x\in \Omega$, then $h = \lim_{n\rightarrow\infty}h_n$ is measurable.
  \end{enumerate}
\end{proposition}
\begin{definition}[Simple Functions]
  A \textit{simple function} $s\colon \Omega\rightarrow \F$ is a function with finite range. In other words, $s$ is of the form
  \begin{align*}
    s &= \sum_{k=1}^{n}c_k\1_{E_k}
  \end{align*}
  for $E_k\subseteq \Omega$ and $c_k\in \F$.
\end{definition}
\begin{fact}
A simple function is measurable if and only if $E_k\in \mathcal{M}$ for each $k$.
\end{fact}
\section{Constructing Measures}%
A measure assigns a nonnegative ``length'' or ``volume'' to measurable sets.
\begin{definition}[Measures on Measurable Spaces]\label{def:measure_basics}
  A \textit{measure} on a measurable space $\left(\Omega,\mathcal{M}\right)$ is a map $\mu\colon \mathcal{M}\rightarrow \left[0,\infty\right]$ that satisfies the following.
  \begin{enumerate}[(i)]
    \item $\mu\left(\emptyset\right) = 0$;
    \item $\displaystyle \mu\left(\bigsqcup_{j=1}^{\infty}E_j\right) = \sum_{j=1}^{\infty}\mu\left(E_j\right)$.
  \end{enumerate}
  The triple $\left(\Omega,\mathcal{M},\mu\right)$ is called a \textit{measure space}.\newline

  A measure $\mu$ is \textit{finite} if $\mu\left(\Omega\right) < \infty$\newline

  If $\mu\left(\Omega\right) = 1$, then $\mu$ is called a \textit{probability measure}.\newline

  A measure $\mu$ is called \textit{finitely additive} if $\mu\left(E\sqcup F\right) = \mu(E) + \mu(F)$.\newline

  A measure $\mu$ is called \textit{$\sigma$-finite} if there is a countable family $\set{E_n}_{n\geq 1}\subseteq \mathcal{M}$ such that
  \begin{align*}
    \Omega &= \bigcup_{n\geq 1}E_n
  \end{align*}
  and $\mu\left(E_n\right) < \infty$.\newline

  A measure $\mu$ on $\left(\Omega,\mathcal{M}\right)$ is called \textit{semi-finite} if, for every $E\in \mathcal{M}$ with $\mu(E) = \infty$, there exists $F\in \mathcal{M}$ with $F\subseteq E$ and $0 < \mu(F) < \infty$.
\end{definition}

\begin{lemma}
  Let $\left(\Omega,\mathcal{M},\mu\right)$ be a measure space.
  \begin{enumerate}[(1)]
    \item If $E,F\in \mathcal{M}$ with $F\subseteq E$, then $\mu\left(F\right) \subseteq \mu\left(E\right)$.
    \item If $\left(E_n\right)_{n}$ is a sequence of measurable sets, then
      \begin{align*}
        \mu\left(\bigcup_{n\geq 1}E_n\right) &\leq \sum_{n=1}^{\infty}\mu\left(E_n\right).
      \end{align*}
    \item If $\left(E_n\right)_{n\geq 1}$ is an increasing family of measurable sets, then
      \begin{align*}
        \mu\left(\bigcup_{n\geq 1}E_n\right) &= \lim_{n\rightarrow\infty}\mu\left(E_n\right).
      \end{align*}
  \end{enumerate}
\end{lemma}
%\begin{proof}\hfill
%  \begin{enumerate}[(1)]
%    \item Since $F\subseteq E$, we can write $E = F\sqcup \left(E\setminus F\right)$. Thus,
%      \begin{align*}
%        \mu\left(E\right) &= \mu\left(F\right) + \mu\left(E\setminus F\right)\\
%                          &\geq \mu\left(F\right).
%      \end{align*}
%    \item We write
%      \begin{align*}
%        F_1 &= E_1\\
%        F_2 &= E_2\setminus E_1\\
%            &\vdots\\
%        F_n &= E_n\setminus \left(\bigcup_{k=1}^{n-1}E_k\right).
%      \end{align*}
%      Since each $F_n$ is measurable, and $F_n\subseteq E_n$, we have
%      \begin{align*}
%        \mu\left(\bigcup_{n\geq 1}E_n\right) &= \mu\left(\bigsqcup_{n\geq 1}F_n\right)\\
%                                             &= \sum_{n=1}^{\infty}F_n\\
%                                             &\leq \sum_{n=1}^{\infty}\mu\left(E_n\right).
%      \end{align*}
%    \item We write $F_n$ as the respective disjoint union for $\set{E_n}_{n\geq 1}$. We have $\bigsqcup_{k=1}^{n}F_k = E_n$. Then,
%      \begin{align*}
%        \mu\left(\bigcup_{n\geq 1}E_n\right) &= \sum_{n=1}^{\infty}\mu\left(F_n\right)\\
%                                             &= \lim_{n\rightarrow\infty}\left(\sum_{k=1}^{n}\mu\left(F_k\right)\right)\\
%                                             &= \lim_{n\rightarrow\infty}\mu\left(\bigsqcup_{k=1}^{n}F_k\right)\\
%                                             &= \lim_{n\rightarrow\infty}\mu\left(E_n\right).
%      \end{align*}
%  \end{enumerate}
%\end{proof}
%\begin{definition}[Counting Measure]
%  If $\Omega$ is any set, the {counting measure} on $\left(\Omega,P\left(\Omega\right)\right)$ assigns $\left\vert A \right\vert$ for each $A\in P\left(\Omega\right)$ finite, and $\infty$ for any infinite subset.
%\end{definition}
%\begin{definition}[Restricting Measures]
%  If $\left(\Omega,\mathcal{M},\mu\right)$ is a measure space, $\mathcal{B}$ is a $\sigma$-algebra on $\Omega$ with $\mathcal{B}\subseteq \mathcal{M}$, the restriction $\mu|_{\mathcal{B}}$ is a measure on $\left(\Omega,\mathcal{B}\right)$.\newline
%
%  If $E\in \mathcal{M}$, we can restrict $\mu$ to $\mathcal{M}_{E}$ (the restriction of $\mathcal{M}$ to $E$), yielding the measure space $\left(E,\mathcal{M}_{E},\mu|_{\mathcal{M}_E}\right)$. We denote this restricted measure $\mu_{E}$, such that $\mu_{E}\left(M\cap E\right) = \mu\left(M\cap E\right)$ for all $M\in \mathcal{M}_{E}$.
%\end{definition}
\begin{definition}[Pushforward Measure]
  Let $\left(\Omega,\mathcal{M},\mu\right)$ be a measure space, and let $\left(\Lambda,\mathcal{N}\right)$ be a measurable space. Let $f\colon \Omega\rightarrow \Lambda$ be measurable. The map
  \begin{align*}
    f_{\ast}\mu\colon \mathcal{N}\rightarrow [0,\infty]
  \end{align*}
  defined by
  \begin{align*}
    f_{\ast}\mu\left(E\right) &= \mu\left(f^{-1}\left(E\right)\right)
  \end{align*}
  defines a measure on $\left(\Lambda,\mathcal{N}\right)$. This is known as the \textit{pushforward measure} of $\mu$.\newline

  If $\mathcal{N}$ on $\Lambda$ is produced by $f$, then the pushforward measure is necessarily defined on $\mathcal{N}$, and that any function $g\colon \Lambda\rightarrow \F$ is measurable if and only if $g\circ f$ is measurable.
\end{definition}
%\begin{definition}[Disjoint Union]
%  Let $\set{\left(\Omega_n,\mathcal{M}_n,\mu_n\right)}$ be a countable family of measure spaces.\newline
%
%  We define the co-product of this family by taking
%  \begin{align*}
%    \Sigma := \bigsqcup_{n=1}^{\infty}\Omega_n,
%  \end{align*}
%  to be our set equipped with the canonical inclusion map $\iota_{n}\left(x\right) = \left(x,n\right)$, such that for each $n$,
%  \begin{align*}
%    \mathcal{M} &:= \set{E\subseteq \Sigma\mid \iota_{n}^{-1}\left(E\right)\in \mathcal{M}_n}.
%  \end{align*}
%  The measure is defined by
%  \begin{align*}
%    \mu\colon \mathcal{M}\rightarrow [0,\infty]\\
%    \mu(E) := \sum_{n=1}^{\infty}\mu_{n}\left(\iota_{n}^{-1}\left(E\right)\right).
%  \end{align*}
%  We can identify each $\Omega_n$ with the subset $\Omega_{n}^{\ast} = \set{\left(x,n\right)\mid x\in \Omega_n}\subseteq \Sigma$, with $\iota_{n}^{-1}\left(E\right)\subseteq \Omega_{n}$ identified with $E\cap \Omega_{n}^{\ast}$.\newline
%
%  The family $\set{\Omega_{n}^{\ast}}_{n\geq 1}$ forms a measurable partition of $\Sigma$, and that $\mu|_{\Omega_{n}^{\ast}}$ are the pushforwards of $\mu_{n}$ by $\iota_{n}$.\newline
%
%  Note that a map $f\colon \Sigma\rightarrow \C$ is measurable if and only if $f\circ \iota_{n}\colon \Omega_n\rightarrow \C$ is measurable for all $n$.\newline
%
%  If $\left(f_n\colon \Omega_n\rightarrow \C\right)_{n}$ is a sequence of measurable maps, the disjoint union
%  \begin{align*}
%    f = \bigsqcup_{n\geq 1}f_n\colon \Sigma\rightarrow \C
%  \end{align*}
%  defined by $f\left(x,n\right) = f_n(x)$, is measurable.
%\end{definition}
\begin{definition}
  Let $\left(\Omega,\mathcal{M},\mu\right)$ be a measure space.\newline

  A \textit{null set} is a measurable set $N\in \mathcal{M}$ with $\mu\left(N\right) = 0$.\newline

  A property which holds for all $x\in \Omega\setminus N$ for some null set $N$ is said to hold \textit{$\mu$-almost everywhere,} or $\mu$-a.e.
\end{definition}
\begin{definition}
  If $\left(\Omega,\mathcal{M},\mu\right)$ is a measure space, we can define an equivalence relation on the set $L_{0}\left(\Omega,\mathcal{M},\mu\right)$, by
  \begin{align*}
    f\sim_{\mu}g \text{ if and only if } \mu\left(\set{x\mid f(x)\neq g(x)}\right) = 0.
  \end{align*}
  We define the set of all classes of measurable functions by
  \begin{align*}
    L\left(\Omega,\mu\right) &= L_{0}\left(\Omega,\mathcal{M}\right)/\sim_{\mu}\\
                             &= \set{\left[f\right]_{\mu}\mid f\in L_{0}\left(\Omega,\mathcal{M}\right)}.
  \end{align*}
\end{definition}
\begin{fact}
  The operations
  \begin{itemize}
    \item $\displaystyle \left[f\right]_{\mu} + \left[g\right]_{\mu} = \left[f + g\right]_{\mu}$;
    \item $\displaystyle \left[f\right]_{\mu}\left[g\right]_{\mu} = \left[fg\right]_{\mu}$;
    \item and $\displaystyle \alpha \left[f\right]_{\mu} = \left[\alpha f\right]_{\mu}$
  \end{itemize}
  are well-defined.
\end{fact}
\begin{definition}[Essentially Bounded Functions and Continuous Functions]
  Let $\left(\Omega,\mathcal{M},\mu\right)$ be a measure space, and $f\colon \Omega\rightarrow \C$ be measurable. We say $f$ is \textit{$\mu$-essentially bounded} if there is $C\geq 0$ such that
  \begin{align*}
    \mu\left(\set{x\in \Omega\mid \left\vert f(x) \right\vert\geq C}\right) = 0.
  \end{align*}
  We say $C$ is an essential bound for $f$. The infimum of all essential bounds is the \textit{essential supremum}, which gives the norm
  \begin{align*}
    \norm{f}_{\infty} &= \esssup(f)\\
                      &= \inf\set{C\geq 0 \mid \mu\left(\set{x\in \Omega\mid \left\vert f(x) \right\vert\geq C}\right) = 0}.
  \end{align*}
  The collection of all $\mu$-essentially bounded functions is denoted
  \begin{align*}
    L_{\infty}\left(\Omega,\mu\right) = \set{\left[f\right]_{\mu}\in L\left(\Omega,\mu\right)\mid \norm{f}_{\infty} < \infty}.
  \end{align*}
  Note that $B_{\infty}\left(\Omega,\mu\right) = L_{\infty}\left(\Omega,\mu\right)$ as sets.\newline

  For $\mu$ a measure on $\left(\Omega,\mathcal{B}_{\Omega}\right)$, the $\mu$-equivalence classes of continuous functions are
  \begin{align*}
    C\left(\Omega,\mu\right) = \set{\left[f\right]_{\mu}\mid f\in C\left(\Omega\right)}.
  \end{align*}
\end{definition}
\begin{fact}
  If $\Omega$ is a topological space, with $\mathcal{B}_{\Omega}$ the Borel $\sigma$-algebra, we have $C\left(\Omega\right)\subseteq L_{0}\left(\Omega,\mathcal{B}_{\Omega}\right)$.\newline
\end{fact}
\begin{remark}
  Members of $L\left(\Omega,\mu\right)$ and $L_{\infty}\left(\Omega,\mu\right)$ are equivalence classes of functions (rather than functions themselves), but we use the abuse of notation that $\left[f\right]_{\mu} = f$.
\end{remark}

\begin{fact}
  Let $\left(\Omega,\mathcal{M},\mu\right)$ be a measure space, and let $f,g\colon \Omega\rightarrow \C$ be measurable, and $\alpha \in \C$. Then, the following are true:
  \begin{itemize}
    \item $\displaystyle \norm{f+g}_{\infty}\leq \norm{f}_{\infty} + \norm{g}_{\infty}$;
    \item $\displaystyle \norm{\alpha f}_{\infty} = \left\vert \alpha \right\vert\norm{f}_{\infty}$;
    \item if $\displaystyle \norm{f}_{\infty} = 0$, then $f = 0$ $\mu$-a.e.;
    \item $\displaystyle \norm{f}_{\infty}\leq \norm{f}_{u}$;
    \item if $f$ is essentially bounded, then
      \begin{align*}
        \mu\left(\set{x\mid \left\vert f(x) \right\vert\geq \norm{f}_{\infty}}\right) &= 0.
      \end{align*}
  \end{itemize}
\end{fact}

\begin{definition}[Complete Measure Space]
A measure space $\left(\Omega,\mathcal{M},\mu\right)$ is said to be \textit{complete} if all subsets of null sets are measurable (and null).
\end{definition}
\section{Integration}%
\begin{definition}
  If $\phi\colon \Omega\rightarrow [0,\infty)$ is a positive, simple, and measurable function, 
  \begin{align*}
    \phi &= \sum_{k=1}^{n}c_k\1_{E_k},
  \end{align*}
  then the \textit{integral} of $\phi$ is defined as
  \begin{align*}
    \int_{\Omega}\phi\:d\mu &= \sum_{k=1}^{n}c_k\mu\left(E_k\right),
  \end{align*}
  with the convention that $0\cdot \infty = 0$.
\end{definition}
\begin{fact}
The value of this integral is not dependent on the representation of $\phi$.
\end{fact}
\begin{definition}
  If $f\colon \Omega\rightarrow\infty [0,\infty)$ is a positive measurable function, then
  \begin{align*}
    \int_{\Omega}f\:d\mu &= \sup\set{\int_{\Omega}\phi\:d\mu\mid \phi\text{ measurable and simple, $0\leq \phi \leq f$}}.
  \end{align*}
  If $E\subseteq \Omega$ is measurable, we define
  \begin{align*}
    \int_{E} f\:d\mu &= \int_{\Omega}f\1_{E}\:d\mu.
  \end{align*}
\end{definition}
\begin{proposition}
  Let $\left(\Omega,\mathcal{M}\right)$ be a measurable space, and let $f:\Omega\rightarrow \C$ be measurable. There is a sequence $\left(\phi_{n}\right)_n$ of simple, measurable functions with $\left(\phi_{n}\left(x\right)\right)_{n}\xrightarrow{n\rightarrow\infty}f(x)$.\newline

  If $f\geq 0$, we can take $\phi_n$ to be positive and pointwise increasing.\newline

  If $f$ is bounded, then this convergence is uniform, and $\left(\phi_{n}\right)_n$ can be chosen to be uniformly bounded.
\end{proposition}

\begin{theorem}[Monotone Convergence Theorem]
  Let $\left(f_n:\Omega\rightarrow [0,\infty)\right)$ be an increasing sequence of positive, measurable functions converging pointwise to $f\colon \Omega\rightarrow [0,\infty)$. Then, $f$ is measurable, and
  \begin{align*}
    \lim_{n\rightarrow\infty}\int_{\Omega}^{} f_n\:d\mu &= \int_{\Omega}^{} f\:d\mu.
  \end{align*}
\end{theorem}
\begin{definition}
  Let $\left(\Omega,\mathcal{M},\mu\right)$ be a measure space.
  \begin{enumerate}[(1)]
    \item A measurable function $f\colon \Omega\rightarrow [0,\infty)$ is \textit{integrable} if
      \begin{align*}
        \int_{\Omega}^{} f\:d\mu < \infty.
      \end{align*}
    \item A measurable function $f\colon \Omega\rightarrow \R$ is integrable if both $f_{+}$ and $f_{-}$ are integrable. We define
      \begin{align*}
        \int_{\Omega}^{} f\:d\mu &= \int_{\Omega}^{} f_{+}\:d\mu - \int_{\Omega}^{} f_{-}\:d\mu.
      \end{align*}
    \item A measurable function $f\colon \Omega\rightarrow \C$ is said to be integrable if both $\re(f)$ and $\im(f)$ are integrable. We define
      \begin{align*}
        \int_{\Omega}^{} f\:d\mu &= \int_{\Omega}^{} \re(f)\:d\mu + i\int_{\Omega}^{} \im(f)\:d\mu.
      \end{align*}
  \end{enumerate}
\end{definition}
\begin{fact}
  Let $f,g\colon \Omega\rightarrow \C$ be integrable functions, and $\alpha\in\C$. Then,
  \begin{itemize}
    \item $f + \alpha g$ is integrable, and $\displaystyle \int_{\Omega}^{} \left(f + \alpha g\right)\:d\mu = \int_{\Omega}f\:d\mu + \alpha\int_{\Omega}g\:d\mu$;
    \item if $f$ and $g$ are real-valued, and $f\leq g$, then $\displaystyle \int_{\Omega}^{} f\:d\mu \leq \int_{\Omega}^{} g\:d\mu$;
    \item $\displaystyle \left\vert \int_{\Omega}^{} f\:d\mu \right\vert\leq \int_{\Omega}^{} \left\vert f \right\vert\:d\mu$.
  \end{itemize}
\end{fact}
\begin{fact}
  If $f = g$ $\mu$-a.e., then
  \begin{align*}
    \int_{\Omega}^{} f\:d\mu &= \int_{\Omega}^{} g\:d\mu.
  \end{align*}
\end{fact}
\begin{fact}
  If $f\colon \Omega\rightarrow \C$ is measurable, then $\int_{\Omega}^{} \left\vert f \right\vert\:d\mu = 0$ if and only if $f = 0$ $\mu$-a.e.
\end{fact}
\begin{fact}
  A measurable function $f\colon \Omega\rightarrow \C$ is integrable if and only if $\left\vert f \right\vert$ is integrable.
\end{fact}
\begin{definition}[Integrable Functions]
  Let $\left(\Omega,\mathcal{M},\mu\right)$ be a measure space.
  \begin{enumerate}[(1)]
    \item We define the set of (equivalence classes of) integrable functions to be
      \begin{align*}
        L_{1}\left(\Omega,\mu\right) = \set{\left[f\right]_{\mu}\in L\left(\Omega,\mu\right)\mid \text{$f$ is integrable}}.
      \end{align*}
    \item We define the set of (equivalence classes of) square-integrable functions to be
      \begin{align*}
        L_{2}\left(\Omega,\mu\right) &= \set{\left[f\right]_{\mu}\in L\left(\Omega,\mu\right)\mid \left\vert f \right\vert^2\text{ is integrable}}.
      \end{align*}
  \end{enumerate}
\end{definition}
\begin{definition}
  Let $\left(\Omega,\mathcal{M},\mu\right)$ be a measure space. If $f$ and $\left(f_n\right)_n$ are integrable with $\norm{f-f_n}_{1}\xrightarrow{n\rightarrow\infty}0$, we say $\left(f_n\right)_n$ \textit{converges in mean} to $f$.
\end{definition}

\begin{fact}
  Let $\left(\Omega,\mathcal{M},\mu\right)$ be a measure space.
  \begin{enumerate}[(1)]
    \item For $f\in L_{1}\left(\Omega,\mu\right)$, the maps
      \begin{align*}
        \left[f\right]_{\mu} &\longmapsto \int_{\Omega}^{} f\:d\mu\\
        \left[f\right]_{\mu} &\longmapsto \int_{\Omega}^{} \left\vert f \right\vert\:d\mu
      \end{align*}
      are well-defined.
    \item For $f\in L_{1}\left(\Omega,\mu\right)$, we define
      \begin{align*}
        \norm{f}_{1} &= \int_{\Omega}^{} \left\vert f \right\vert\:d\mu.
      \end{align*}
      This is a well-defined norm.
      \begin{align*}
        \norm{f+g}_{1} &\leq \norm{f}_{1} + \norm{g}_{1}\\
        \norm{\alpha f}_1 &= \left\vert \alpha \right\vert\norm{f}_{1}\\
        \norm{f}_1 = 0 &\Leftrightarrow f= 0 \text{ $\mu$-a.e.}
      \end{align*}
    \item 
      \begin{align*}
        d_1\left(\left[f\right]_{\mu},\left[g\right]_{\mu}\right) &= \norm{f-g}_{1}
      \end{align*}
      is a metric on $L_{1}\left(\Omega,\mu\right)$.
  \end{enumerate}
\end{fact}
\begin{theorem}[Dominated Convergence Theorem]
  Let $\left(f_n:\Omega\rightarrow \C\right)_{n}$ be a sequence of measurable functions converging pointwise to a measurable function $f\colon \Omega\rightarrow \C$. If there is an integrable $g\colon \Omega \rightarrow [0,\infty)$ with $\left\vert f_n \right\vert\leq g$ for all $n$, then
  \begin{align*}
    \int_{\Omega}^{} f_n\:d\mu \xrightarrow{n\rightarrow\infty}\int_{\Omega}^{} f\:d\mu.
  \end{align*}
\end{theorem}
\begin{corollary}
  If $f\colon \Omega\rightarrow \C$ is integrable, then there is a sequence of simple integrable functions $\left(\phi_n\right)_n$ with $\norm{f - \phi_{n}}_{1}\xrightarrow{n\rightarrow\infty}0$.
\end{corollary}
\begin{corollary}
  If $f\colon \R\rightarrow\C$ is integrable, then there is a sequence $\left(f_n\right)_n$ of compactly supported integrable functions such that $\norm{f - f_n}_{1}\xrightarrow{n\rightarrow\infty} 0$.
\end{corollary}
\begin{theorem}
  If $f\colon \R\rightarrow\C$ is integrable, and $\ve > 0$, there is a continuous, compactly supported function $g$ with $\norm{f - g}_{1} < \ve$.
\end{theorem}
\begin{proposition}
  Let $\left(\Omega,\mathcal{M},\mu\right)$ be a measure space, and let $\left(\Lambda,\mathcal{N}\right)$ be a measurable space with $f\colon \Omega\rightarrow \Lambda$ a measurable map. Let $f_{\ast}\mu$ be the pushforward measure on $\left(\Lambda,\mathcal{N}\right)$. For a measurable function $g\colon \Lambda\rightarrow [0,\infty)$, then
  \begin{align*}
    \int_{\Lambda}^{} g\:d\left(f_{\ast}\mu\right) &= \int_{\Omega}^{} \left(g\circ f\right)\:d\mu.
  \end{align*}
  Moreover, if $g\colon \Lambda\rightarrow \F$ is integrable with respect to $f_{\ast}\mu$, then so too is $g\circ f$ with respect to $\mu$.
\end{proposition}
\section{Complex Measures}%
\begin{example}
  If $\left(\Omega,\mathcal{M},\mu\right)$ is a measure space, then the map $\mu_f(E) = \int_{E}^{} f\:d\mu$ is a well-defined measure.
\end{example}
\begin{definition}
  Let $\left(\Omega,\mathcal{M},\mu\right)$ be a measurable space.
  \begin{enumerate}[(1)]
    \item A \textit{complex measure} on $\left(\Omega,\mathcal{M},\mu\right)$ is a map $\mu\colon \mathcal{M}\rightarrow \C$ satisfying the following conditions.
      \begin{itemize}
        \item $\mu\left(\emptyset\right) = 0$;
        \item $\displaystyle \mu\left(\bigsqcup_{k=1}^{\infty}E_k\right) = \sum_{k=1}^{\infty}\mu\left(E_k\right)$ for $\set{E_k}_{k\geq 1}\subseteq \mathcal{M}$.
      \end{itemize}
    \item We write $M\left(\Omega\right)$ to be the set of all complex measures on $\left(\Omega,\mathcal{M}\right)$.
    \item If $\mu\in M\left(\Omega\right)$, and $\mu\left(E\right)\in \R$ for all $E\in \mathcal{M}$, then we say $\mu$ is a \textit{real measure} on $\left(\Omega,\mathcal{M}\right)$.
    \item If $\mu\in M\left(\Omega\right)$ and $\mu(E) \geq 0$ for all $E\in \mathcal{M}$, then we say $\mu$ is a \textit{positive measure} on $\left(\Omega,\mathcal{M}\right)$.
    \item If $\mu$ is a positive measure on $\left(\Omega,\mathcal{M}\right)$ with $\mu\left(\Omega\right) = 1$, we say $\mu$ is a probability measure on $\left(\Omega,\mathcal{M}\right)$. We write $\mathcal{P}\left(\Omega,\mathcal{M}\right)$ to be the collection of all probability measures on $\left(\Omega,\mathcal{M}\right)$.
    \item If $\Omega$ is a LCH space, we always let $M\left(\Omega\right)$ be the set of all complex Borel measures on $\Omega$.
  \end{enumerate}
\end{definition}
\begin{definition}
  If $\left(\Omega,\mathcal{M}\right)$ is a measurable space, and $x\in \Omega$, the \textit{Dirac measure} at $x$ is defined by
  \begin{align*}
    \delta_{x}\colon \mathcal{M}&\rightarrow [0,1]\\
    \delta_x\left(E\right) &= \begin{cases}
      1 & x\in E\\
      0 & x\notin E
    \end{cases}.
  \end{align*}
    If $x_1,\dots,x_n$ are distinct points in $\Omega$, and $t_1,\dots,t_n\in [0,1]$ with $\sum_{j=1}^{n}t_j = 1$, then
    \begin{align*}
      \mu &= \sum_{j=1}^{n}t_j\delta_{x_j}
    \end{align*}
    is a probability measure on $\left(\Omega,\mathcal{M}\right)$.
    %Every probability measure can be weakly approximated by convex combinations of Dirac measures by the Krein-Milman Theorem.
\end{definition}
\begin{fact}
  If $\mu$ is a complex measure on $\left(\Omega,\mathcal{M}\right)$, then $\overline{\mu}$, defined by $\overline{\mu}\left(E\right) = \overline{\mu\left(E\right)}$ for $E\in \mathcal{M}$, is also a complex measure. Additionally, $\re\left(\mu\right)$ and $\im\left(\mu\right)$, defined by
  \begin{align*}
    \re\left(\mu\right)\left(E\right) &= \re\left(\mu\left(E\right)\right)\\
    \im\left(\mu\right)\left(E\right) &= \im\left(\mu\left(E\right)\right)
  \end{align*}
  are real measures.
\end{fact}
\begin{definition}
  If $\mu\in M\left(\Omega\right)$, then the \textit{total variation} of $\mu$ is the quantity
  \begin{align*}
    \left\vert \mu \right\vert\colon \mathcal{M}\rightarrow [0,\infty]
    \end{align*}
    with
    \begin{align*}
    \left\vert \mu \right\vert\left(E\right) = \sup\set{\left.\sum_{j=1}^{\infty}\left\vert \mu\left(E_j\right) \right\vert\right|E = \bigsqcup_{j=1}^{\infty}E_j,~E_j\in\mathcal{M}}.
  \end{align*}
\end{definition}
\begin{fact}
  If $\mu\in M\left(\Omega\right)$, then $\left\vert \mu \right\vert$ is a positive, finite measure. Additionally, if $\mu,\nu\in M\left(\Omega\right)$ with $\alpha\in \C$, then
  \begin{enumerate}[(a)]
    \item $\left\vert \mu\left(E\right) \right\vert\leq \left\vert \mu \right\vert\left(E\right)$
    \item $\left\vert \mu + \nu \right\vert\left(E\right) \leq \left\vert \mu \right\vert\left(E\right) + \left\vert \nu \right\vert\left(E\right)$
    \item $\left\vert \alpha\mu \right\vert\left(E\right) = \left\vert \alpha \right\vert\left\vert \mu \right\vert\left(E\right)$.
  \end{enumerate}
\end{fact}
\begin{definition}[Absolute Continuity of Measures]
  Let $\left(\Omega,\mathcal{M}\right)$ be a measurable space, and let $\mu$ and $\nu$ be positive measures on this space. If $\mu(A) > 0$ implies $\nu(A) > 0$ for a given $A\in \mathcal{M}$, we say $\mu$ is \textit{absolutely continuous} with respect to $\nu$. We write $\mu \ll \nu$.
\end{definition}
\begin{theorem}[Radon--Nikodym Theorem]
  If $\mu \ll \nu$ on $\left(\Omega,\mathcal{M}\right)$, then there exists a measurable function $f\colon \Omega\rightarrow [0,\infty]$ such that
  \begin{align*}
    \nu(A) &= \int_{A}^{} f \:d\nu
  \end{align*}
  for each $A\in \mathcal{M}$.
\end{theorem}
\begin{remark}
The Radon--Nikodym theorem extends to signed and complex measures.
\end{remark}
\begin{fact}
  Let $\left(\Omega,\mathcal{M},\lambda\right)$ be a measure space, and suppose $f\in L_{1}\left(\Omega,\lambda\right)$. Then, $\mu\left(E\right) = \int_{E}^{} f\:d\lambda$ defines a complex measure. We write $f = \diff{\mu}{\lambda}$, which is the \textit{Radon--Nikodym derivative} of $\mu$ with respect to $\lambda$.\newline

  It is also the case that
  \begin{align*}
    \left\vert \mu \right\vert\left(E\right) &= \int_{E}^{} \left\vert f \right\vert\:d\lambda.
  \end{align*}
\end{fact}
\begin{fact}
  If $\mu\in M\left(\Omega\right)$, there exists a measurable function $f\colon \Omega\rightarrow \C$ such that $\left\vert f \right\vert = 1$ and $\mu\left(E\right) = \int_{E}^{} f\:d\left\vert \mu \right\vert$ for all $E\in \mathcal{M}$.
\end{fact}
\begin{definition}
  Let $\Omega$ be a LCH space equipped with the Borel $\sigma$-algebra, $\mathcal{B}_{\Omega}$.
  \begin{enumerate}[(1)]
    \item A Borel measure $\mu\colon \mathcal{B}_{\Omega}\rightarrow [0,\infty]$ is called
      \begin{itemize}
        \item \textit{inner regular} on $E\in \mathcal{B}_{\Omega}$ if
          \begin{align*}
            \mu(E) &= \sup\set{\mu\left(K\right)\mid K\subseteq E, K\text{ compact}};
          \end{align*}
        \item \textit{outer regular} on $E\in \mathcal{B}_{\Omega}$ if
          \begin{align*}
            \mu(E) &= \inf\set{\mu\left(U\right)\mid U\supseteq E,U\text{ open}};
          \end{align*}
        \item \textit{regular} on $E$ if it is inner regular and outer regular on $E$;
        \item regular if it is regular on all $E\in \mathcal{B}_{\Omega}$;
        \item \textit{Radon} if
          \begin{itemize}
            \item $\mu(K) < \infty$ for all compact $K\subseteq \Omega$;
            \item $\mu$ is inner regular on all open sets and outer regular on all Borel sets.
          \end{itemize}
      \end{itemize}
    \item A complex Borel measure $\mu\colon \mathcal{B}_{\Omega}\rightarrow \C$ is regular if $\left\vert \mu \right\vert$ is regular; $\mu$ is Radon if $\left\vert \mu \right\vert$ is Radon.
    \item We write $M_{r}\left(\Omega\right)$ to denote the set of all complex regular measures on $\left(\Omega,\mathcal{B}_{\Omega}\right)$.
  \end{enumerate}
\end{definition}
\begin{fact}
  Every positive Radon measure is regular. Thus, every complex Borel measure is regular if and only if it is Radon.\newline

  Moreover, if $\Omega$ is a second countable LCH space, then every complex Borel measure is regular.
\end{fact}

\begin{definition}
  Let $\left(\Omega,\tau\right)$ be a topological space, and suppose $\mu\colon \mathcal{B}_{\Omega}\rightarrow [0,\infty]$ is a Borel measure.
  \begin{enumerate}[(1)]
    \item The \textit{kernel} of $\mu$ is the set
      \begin{align*}
        N_{\mu} &= \bigcup\set{U\subseteq \Omega\mid U\in\tau,~\mu(U) = 0}.
      \end{align*}
    \item The \textit{support} of $\mu$ is the complement of the kernel, $\supp(\mu) = N_{\mu}^{c}$.
  \end{enumerate}
\end{definition}
\begin{fact}
  If $\mu$ is a Radon measure on a LCH space $\Omega$, then $\mu\left(N_{\mu}\right) = 0$, meaning $\mu\left(\Omega\right) = \mu\left(\supp\left(\mu\right)\right)$.
\end{fact}

\begin{theorem}[Hahn and Jordan Decomposition]
  Let $\left(\Omega,\mathcal{M}\right)$ be a measurable space, and let $\mu\colon \mathcal{M}\rightarrow \R$ be a real measure. Then, there is a measurable partition $\Omega = P\sqcup N$ such that for all $E\subseteq P$, $\mu(E) \geq 0$, and for all $E\subseteq N$, $\mu(E) \leq 0$. This partition is unique up to a $\mu$-null symmetric difference --- that is, for any $P',N'$ satisfying the conditions, $\mu\left(P'\triangle P\right) = 0$ and $\mu\left(N'\triangle N\right) = 0$.\newline

  There is a unique decomposition $\mu = \mu_{+} - \mu_{-}$, with $\mu_{\pm}$ that are positive such that if $E\subseteq P$, then $\mu_{-}\left(E\right) = 0$, and if $E\subseteq N$, $\mu_{+}\left(E\right) = 0$.
\end{theorem}
\begin{definition}
  Let $\left(\Omega,\mathcal{M}\right)$ be a measurable space, and let $f\colon \Omega\rightarrow \C$ be measurable.
  \begin{enumerate}[(1)]
    \item If $\mu\colon \mathcal{M}\rightarrow \R$ is a real measure with $\mu = \mu_{+} - \mu_{-}$, we say that $f$ is $\mu$-integrable if it is both $\mu_{+}$ and $\mu_{-}$-integrable. We define
      \begin{align*}
        \int_{\Omega}^{} f\:d\mu &= \int_{\Omega}^{} f\:d\mu_{+} - \int_{\Omega}^{} f\:d\mu_{-}.
      \end{align*}
    \item If $\mu:\mathcal{M}\rightarrow \C$ is a complex measure with $\mu_{1} = \re\left(\mu\right)$ and $\mu_{2} = \im\left(\mu\right)$, we say $f$ is $\mu$-integrable if it is both $\mu_{1}$ and $\mu_{2}$-integrable. We define
      \begin{align*}
        \int_{\Omega}^{} f\:d\mu &= \int_{\Omega}^{} f\:d\mu_{1} + i\int_{\Omega}^{} f\:d\mu_{2}.
      \end{align*}
  \end{enumerate}
\end{definition}
\begin{theorem}[Riesz Representation Theorem on $C_c\left(\Omega\right)$]
  Let $\Omega$ be a LCH space. If $\varphi\colon C_{c}\left(\Omega\right)\rightarrow \C$ is a positive linear functional, then there is a unique Radon measure $\mu$ such that
  \begin{align*}
    \varphi\left(f\right) &= \int_{\Omega}^{} f\:d\mu
  \end{align*}
  for all $f\in C_c\left(\Omega\right)$. Additionally, for every open $U\subseteq \Omega$, we have
  \begin{align*}
    \mu\left(U\right) &= \sup\set{\varphi\left(f\right) | f\in C_c\left(\Omega,[0,1]\right),\supp(f) \subseteq U},
  \end{align*}
  and for every compact $K\subseteq \Omega$, we have
  \begin{align*}
    \mu\left(K\right) &= \inf\set{\varphi(f) | f\geq \1_{K}}.
  \end{align*}
\end{theorem}
\begin{theorem}[Riesz Representation Theorem on $C\left(X\right)$]
  Let $X$ be a compact metric space, and let $\varphi\in \left(C\left(X\right)\right)^{\ast}$ be a positive linear functional with $\varphi\left(\1_{X}\right) = \norm{\varphi} = 1$. Then, for $f\in C(X)$, there is a unique Borel probability measure such that
  \begin{align*}
    \varphi\left(f\right) &= \int_{X}^{} f\:d\mu.
  \end{align*}
\end{theorem}
%\begin{theorem}[Markov--Riesz Theorem]
%  Let $\Omega$ be a LCH space. Then, $M_r\left(\Omega\right)\cong C_0\left(\Omega\right)^{\ast}$.
%\end{theorem}
%\begin{definition}
%  Let $\Omega$ be a LCH space, and let $\tau\colon \Omega\rightarrow \Omega$ be a continuous transformation. A regular Borel probability measure $\mu\in \mathcal{P}_{r}\left(\Omega\right)$ is called $\tau$-invariant if $\tau_{\ast}\mu = \mu$.
%\end{definition}

\chapter{Functional Analysis}
\nocite{*}
\printbibliography
\end{document}
