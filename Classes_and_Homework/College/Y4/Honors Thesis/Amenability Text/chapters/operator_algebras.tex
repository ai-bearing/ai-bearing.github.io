In Chapter \ref{ch:nuclearity}, we will establish that the amenability of a group is equivalent to a property known as nuclearity held by the $C^{\ast}$-algebra(s) generated by the group. For this, we need a solid background in the theory of operator algebras --- specifically, in Banach algebras and $C^{\ast}$-algebras.
\section{Definitions and Examples}%
The theory of $C^{\ast}$-algebras is motivated by the fact that the adjoint operation on $\B\left( \mathcal{H} \right)$ (Definition \ref{def:adjoint_properties}) satisfies the criteria for an involution (Definition \ref{def:algebra_star_algebra}) on an algebra. However, one property that $\B\left( \mathcal{H} \right)$ has that a pure $\ast$-algebra lacks is the fact that $\B\left( \mathcal{H} \right)$ is equipped with a norm, $\norm{\cdot}_{\op}$, that turns $\B\left( \mathcal{H} \right)$ into a normed algebra (Definition \ref{def:norms}).\newline

What the theory of $C^{\ast}$-algebras allows us to do is abstract away from $\B\left( \mathcal{H} \right)$. Soon, we will see that this abstraction will allow us to focus on purely algebraic properties of $C^{\ast}$-algebras and establish fundamental analytic results on them.
\begin{definition}\label{def:banach_star_algebra}
  Let $A$ be an algebra.
  \begin{itemize}
    \item If $\norm{\cdot}$ is such that $\left( A,\norm{\cdot} \right)$ is a Banach space that satisfies $\norm{ab}\leq \norm{a}\norm{b}$ for all $a,b\in A$, then we say $\left( A,\norm{\cdot} \right)$ is a \textit{Banach algebra}.
    \item If $A$ is a $\ast$-algebra that is also a Banach algebra, and the norm on $A$ satisfies $\norm{a^{\ast}} = \norm{a}$ for all $a\in A$, then we say $A$ is a \textit{Banach $\ast$-algebra}.
    \item If $A$ is a Banach $\ast$-algebra whose norm also satisfies $\norm{a^{\ast}a} = \norm{a}^2$ for all $a\in A$, then we say $A$ is a \textit{$C^{\ast}$-algebra}. This final property is known as the $C^{\ast}$-property.
  \end{itemize}
\end{definition}
There are many $C^{\ast}$-algebras that we interact with as we study analysis.
\begin{example}\hfill
  \begin{itemize}
    \item The complex numbers, $\C$, equipped with the involution $z\mapsto \overline{z}$, are a $C^{\ast}$-algebra under the norm $\left\vert z \right\vert$.
    \item If $\mathcal{H}$ is a Hilbert space, then $\B\left( \mathcal{H} \right)$ is a $C^{\ast}$-algebra under the operator norm with the involution $T\mapsto T^{\ast}$.
    \item The space of $n\times n$ complex matrices, $\Mat_n\left( \C \right)$ under the operator norm and the involution $\left( a_{ij}^{\ast} \right)_{ij} = \left( \overline{a_{ji}} \right)_{ij}$ is a $C^{\ast}$-algebra.
    \item If $\Omega$ is any nonempty set, then the space of bounded functions, $\ell_{\infty}\left( \Omega \right)$, is a $C^{\ast}$-algebra under the norm $\norm{f}_{\ell_{\infty}} = \sup_{x\in\Omega}\left\vert f(x) \right\vert$ and the involution $f^{\ast}\left( x \right) = \overline{f(x)}$.
  \end{itemize}
\end{example}
However, there are also some Banach $\ast$-algebras that are not $C^{\ast}$-algebras.
\begin{example}
  Let 
  \begin{align*}
    \ell_1\left( \Z \right)\coloneq \set{f\colon \Z\rightarrow\C | \norm{f}_{\ell_1} \coloneq \sum_{n\in\Z}\left\vert f(n) \right\vert < \infty}
  \end{align*}
  be equipped with the involution
  \begin{align*}
    f^{\ast}\left( n \right) &= \overline{f\left( -n \right)}
  \end{align*}
  and multiplication
  \begin{align*}
    f\ast g(n) &= \sum_{k\in\Z}f(n-k)g(k).
  \end{align*}
  Then, $\ell_1(\Z)$ is a Banach $\ast$-algebra that does not satisfy the $C^{\ast}$-property.
\end{example}
The rest of this section will focus on understanding properties of $C^{\ast}$-algebras and their elements.
\section{\texorpdfstring{$C^{\ast}$-Norms}{C*-Norms}, Universal \texorpdfstring{$C^{\ast}$-Algebras}{C*-Algebras}, and Representations}%
We begin by constructing $C^{\ast}$-algebras.\newline

Recall that, in the case of a normed vector space, we know that (Proposition \ref{prop:completion_existence}) there is always a completion of $X$ into a Banach space, $\widetilde{X}\coloneq \overline{\iota_X\left( X \right)}^{\norm{\cdot}_{\op}}\subseteq X^{\ast\ast}$. This extends to the case of normed algebras/$\ast$-algebras and Banach algebras/Banach $\ast$-algebras.
\begin{lemma}[{\cite[Lemma 7.2.26]{rainone_analysis}}]\label{lemma:banach_algebra_completion}
  If $A_0$ is a normed algebra/$\ast$-algebra, then its Banach space completion, $A$, is a Banach algebra/Banach $\ast$-algebra. The inclusion, $A_0\hookrightarrow A$ is an isometric homomorphism/$\ast$-homomorphism of algebras/$\ast$-algebras.
\end{lemma}
If we have a normed algebra $A$ and we want its completion to be a $C^{\ast}$-algebra, then we need the norm itself to have properties analogous to the norm on a $C^{\ast}$-algebra.
\begin{definition}[{\cite[Definition 7.2.27]{rainone_analysis}}]
  Let $A_0$ be a $\ast$-algebra. A \textit{$C^{\ast}$-norm}/\textit{$C^{\ast}$-seminorm} on $A_0$ is a norm/seminorm on $A_0$ satisfying the following:
  \begin{enumerate}[(i)]
    \item $\norm{ab}\leq \norm{a}\norm{b}$;
    \item $\norm{a^{\ast}} = \norm{a}$;
    \item $\norm{a^{\ast}a} = \norm{a}^2$ (also known as the \textit{$C^{\ast}$-property})
  \end{enumerate}
  for all $a,b\in A_0$.
\end{definition}
We're able to construct $C^{\ast}$-norms by using $\ast$-homomorphisms into $C^{\ast}$-algebras.
\begin{lemma}[{\cite[Lemma 7.2.30]{rainone_analysis}}]
  Let $A_0$ be a $\ast$-algebra, and suppose $\phi\colon A_0\rightarrow B$ is a $\ast$-homomorphism into a $C^{\ast}$-algebra $B$. Then,
  \begin{align*}
    \norm{a}_{\phi} &= \norm{\phi(a)}_{\op}
  \end{align*}
  defines a $C^{\ast}$-seminorm on $A_0$. If $\phi$ is injective, then $\norm{\cdot}_{\phi}$ is a $C^{\ast}$-norm.
\end{lemma}
Just as in the case of Lemma \ref{lemma:banach_algebra_completion}, the completion of a normed algebra with a $C^{\ast}$-norm yields a $C^{\ast}$-algebra.
\begin{lemma}[{\cite[Lemma 7.2.32]{rainone_analysis}}]
  Let $\norm{\cdot}$ be a $C^{\ast}$-norm on a $\ast$-algebra $A_0$. The norm completion, $A$, is a $C^{\ast}$-algebra, and the inclusion $A_0\hookrightarrow A$ is an isometric $\ast$-homomorphism.
\end{lemma}
Recall that any seminorm on a vector space gives rise to a norm on the quotient space (Theorem \ref{thm:quotient_space_norm}) --- similarly, we may define the enveloping $C^{\ast}$-algebra on any $C^{\ast}$-seminorm on $A$.
\begin{definition}[{\cite[Definition 7.2.33]{rainone_analysis}}]\label{def:hausdorff_completion}
  Let $A_0$ be a $\ast$-algebra equipped with a $C^{\ast}$-seminorm $p$. The norm completion of $A/N_p$ with respect to $\norm{\cdot}_{A/N_p}$, where
  \begin{align*}
    N_p \coloneq \set{a\in A | p(a) = 0}
    \intertext{and}
    \norm{a + N_p} &= p(a),
  \end{align*}
  is known as the \textit{Hausdorff completion} or \textit{enveloping $C^{\ast}$-algebra} of $\left( A_0,p \right)$.
\end{definition}
We want to now understand a sort of ``maximal'' enveloping $C^{\ast}$-algebra --- preferably one that admits a universal property, similar to the universal property for the free $\ast$-algebra of Theorem \ref{thm:universal_property_free_algebra}. This will be the universal $C^{\ast}$-algebra.
\begin{definition}[{\cite[Definition 7.2.34]{rainone_analysis}}]
  Let $A_0$ be a $\ast$-algebra, and let $\mathcal{P}$ denote the collection of all $C^{\ast}$-seminorms on $A_0$. Set
  \begin{align*}
    \norm{a}_{u} &= \sup_{p\in \mathcal{P}}p(a).
  \end{align*}
  If $\norm{a}_u < \infty$ for all $a\in A_0$, then $\norm{\cdot}_u$ defines a $C^{\ast}$-seminorm on $A_0$ called the \textit{universal $C^{\ast}$-seminorm} on $A_0$.\newline

  The \textit{universal enveloping $C^{\ast}$-algebra} of $A_0$ is the enveloping $C^{\ast}$-algebra of the pair $\left( A_0,\norm{\cdot}_u \right)$.
\end{definition}
We can also define a universal $C^{\ast}$-algebra with respect to a set of relations $R$ with a similar universal property. 
\begin{definition}[{\cite[Definition 7.2.35]{rainone_analysis}}]
  Let $E$ be a set of abstract variables and suppose $R\subseteq \mathbb{A}^{\ast}\left( E \right)$ is a collection of relations. If the universal enveloping $C^{\ast}$-algebra of $\mathbb{A}^{\ast}\left( E|R \right)$ exists --- i.e., that $\norm{a}_u < \infty$ for all $a\in \mathbb{A}^{\ast}\left( E|R \right)$ --- we denote it $C^{\ast}\left( E|R \right)$. It is know as the \textit{universal $C^{\ast}$-algebra with generators $E$ and relations $R$}.
\end{definition}
\begin{proposition}[{\cite[Proposition 7.2.36]{rainone_analysis}}]\label{prop:universal_property_universal_cstar_algebra}
  Let $E = \set{x_i}_{i\in I}$ be a set of abstract variables, and let $R\subseteq \mathbb{A}^{\ast}\left( E \right)$ be a collection of relations. Suppose the universal $C^{\ast}$-algebra $C^{\ast}\left( E|R \right)$ exists.\newline

  If $B$ is a $C^{\ast}$-algebra admitting elements $\set{b_i}_{i\in I}$ that satisfy the relations $R$, then there is a unique contractive $\ast$-homomorphism, $\varphi_B\colon C^{\ast}\left( E|R \right) \rightarrow B$, such that
  \begin{align*}
    \varphi_b\left( v_i \right) = b_i,
  \end{align*}
  where $v_i \coloneq \left( x + I(R) \right) + N_u$ is a double coset with $I(R)$ as the ideal generated by the $R$ and $N_u$ is the zero set of $\norm{\cdot}_u$, as in Definition \ref{def:hausdorff_completion}.
\end{proposition}
We can realize $\ast$-algebras as $\ast$-subalgebras of bounded operators on Hilbert space.\footnote{In fact, via the GNS construction (which we apologetically cannot cover here), every $C^{\ast}$-algebra can be realized as a $\ast$-subalgebra of $\B\left( \mathcal{H} \right)$ for a suitable Hilbert space $\mathcal{H}$.} 
\begin{definition}[{\cite[Definition 7.2.41]{rainone_analysis}}]\label{def:unital_representation}
  Let $A_0$ be a $\ast$-algebra. A \textit{representation} of $A_0$ is a pair $\left( \pi_0,\mathcal{H} \right)$, where $\pi_0\colon A_0\rightarrow \B\left( \mathcal{H} \right)$ is a $\ast$-homomorphism.\newline

  If $A_0$ is unital, and $\pi_0\left( 1_A \right) = I_{\mathcal{H}}$, then we say $\pi_0$ is a \textit{unital} representation.
\end{definition}
What makes representations special is that they give us a $C^{\ast}$-norm ``for free'' in a sense.
\begin{lemma}[{\cite[Lemma 7.2.42]{rainone_analysis}}]
  Let $A_0$ be a $\ast$-algebra, and let $\left( \pi_0,\mathcal{H} \right)$ be a representation of $A_0$. Then,
  \begin{align*}
    \norm{a}_{\pi_0} &= \norm{\pi_0\left( a \right)}_{\op}
  \end{align*}
  is a $C^{\ast}$-seminorm on $A_0$. If $\pi_0$ is injective, then $\norm{\cdot}_{\pi_0}$ is a $C^{\ast}$-norm.
\end{lemma}
\begin{lemma}[{\cite[Lemma 7.2.43]{rainone_analysis}}]
  If $A_0$ and $B_0$ are normed $\ast$-algebras with completions $A$ and $B$, then any bounded $\ast$-homomorphism extends continuously to $\varphi\colon A\rightarrow B$.
\end{lemma}
\begin{corollary}[{\cite[Corollary 7.2.44]{rainone_analysis}}]
  Let $A_0$ be a $\ast$-algebra, and let $\pi\colon A_0\rightarrow \B\left( \mathcal{H} \right)$ be an injective representation.\newline

  The completion $A$ of $A_0$ with respect to the $C^{\ast}$-norm $\norm{\cdot}_{\pi_0}$ is a $C^{\ast}$-algebra, and the continuous extension $\pi\colon A\rightarrow \B\left( \mathcal{H} \right)$ is an isometric $\ast$-homomorphism.
\end{corollary}
\section{Spectra of Elements in \texorpdfstring{$C^{\ast}$-Algebras}{C*-Algebras}}%
In a $C^{\ast}$-algebra, a lot of the purely algebraic characterizations of elements also admit an analytic component. We begin to show this by discussing the spectrum.\newline

First, we develop and analytic understanding of invertibility.
\begin{proposition}[{\cite[Proposition 7.3.1]{rainone_analysis}}]
  Let $A$ be a unital Banach algebra. If $x\in A$ with $\norm{x}\leq 1$, then $1_A - x\in \text{GL}\left( A \right)$, and
  \begin{align*}
    \left( 1-x \right)^{-1} &= \sum_{k=0}^{\infty}x^k.
  \end{align*}
  Moreover,
  \begin{align*}
    \norm{\left( 1_A - x \right)^{-1}} &\leq \frac{1}{1-\norm{x}}
  \end{align*}
\end{proposition}
The Carl Neumann series gives a family of useful corollaries.
\begin{corollary}[{\cite[Corollary 7.3.2]{rainone_analysis}}]
  Let $A$ be a unital Banach algebra, and let $a\in A$ be such that $\norm{1_A - a} < 1$. Then, $a\in \text{GL}\left( A \right)$, and
  \begin{align*}
    a^{-1} &= \sum_{k=0}^{\infty}\left( 1_A - a \right)^{k}\\
    \norm{a^{-1}} &\leq \frac{1}{1-\norm{1_A - a}}.
  \end{align*}
\end{corollary}
\begin{proposition}[{\cite[Proposition 7.3.3]{rainone_analysis}}]
  Let $A$ be a unital Banach algebra.
  \begin{enumerate}[(1)]
    \item The group of invertible elements, $\text{GL}\left( A \right)\subseteq A$, is open.
    \item The inverse map, $i\colon \text{GL}\left( A \right)\rightarrow \text{GL}\left( A \right)$, given by $a\mapsto a^{-1}$, is a homeomorphism.
  \end{enumerate}
\end{proposition}
Now, we discuss some of the analytic properties of the spectrum.
\begin{theorem}[{\cite[Theorem 7.3.4]{rainone_analysis}}]
  Let $A$ be a Banach algebra, and let $a\in A$.
  \begin{enumerate}[(1)]
    \item The resolvent, $\rho\left( a \right)\subseteq \C$, is open, and the spectrum, $\sigma\left( a \right)\subseteq \C$, is closed (see Definition \ref{def:spectrum}).
    \item If $\lambda\in\C$ is such that $\left\vert \lambda \right\vert > \norm{a}$, then $\lambda\in \rho(a)$, and 
      \begin{align*}
        \sigma\left( a \right) &\subseteq \set{z\in\C | \left\vert z \right\vert\leq \norm{a}},
      \end{align*}
      meaning $\sigma\left( a \right)\subseteq \C$ is compact.
    \item If $A$ is unital, then the map $R_a\colon \rho\left( a \right)\rightarrow A$, defined by $R_a\left( \lambda \right) = \left( a - \lambda 1_A \right)^{-1}$, is holomorphic.
    \item The spectrum $\sigma\left( a \right)$ is nonempty --- if $A$ is nonunital, then $0\in \sigma\left( a \right)$.
  \end{enumerate}
\end{theorem}
\begin{remark}
  In all cases, if $A$ is nonunital, then we consider $\rho\left( a \right)$ in the unitization of $A$.
\end{remark}
\begin{definition}{\cite[Definition 7.3.5]{rainone_analysis}}
  If $A$ is a Banach algebra, then for each $a\in A$, the \textit{spectral radius} of $A$ is defined by
  \begin{align*}
    r(a) &\coloneq \sup_{\lambda\in \sigma\left( a \right)}\left\vert \lambda \right\vert.
  \end{align*}
\end{definition}
The spectrum, and spectral radius, allow us to glean much information about $C^{\ast}$-algebras.
\begin{proposition}[{\cite[Proposition 7.3.12]{rainone_analysis}}]
  Let $A$ be a $C^{\ast}$-algebra. If $b\in A$ is normal, its norm and spectral radius are equivalent --- i.e., $r(a) =\norm{a} $. Consequently, there is some $\lambda\in \sigma\left( a \right)$ such that $r(a) = \left\vert \lambda \right\vert$.
\end{proposition}
\begin{proposition}[{\cite[Proposition 7.3.13]{rainone_analysis}}]
  Let $A$ be a ${\ast}$-algebra such that $\norm{\cdot}_1$ and $\norm{\cdot}_2$ are $C^{\ast}$-norms for $A$. Then, $\norm{\cdot}_1 = \norm{\cdot}_2$.
\end{proposition}
\begin{proposition}[{\cite[Proposition 7.3.16]{rainone_analysis}}]
  If $\varphi\colon A\rightarrow B$ is a $\ast$-homomorphism between $C^{\ast}$-algebras, then $\varphi$ is contractive --- i.e., $\norm{\varphi}_{\op}\leq 1$.
\end{proposition}
\section{Analyzing the Character Space}%
A lot of information about $C^{\ast}$-algebras can be gleaned from their characters. We will begin by discussing some of the properties of characters before, in the next section, we show their power with the continuous functional calculus.\newline

Recall that, on any algebra $A$, a character $h\colon A\rightarrow \C$ is a nonzero algebra homomorphism --- since $h$ is a nonzero linear functional, it is automatically surjective. However, when $A$ is a normed algebra --- specifically, a Banach algebra --- characters are automatically continuous.
\begin{proposition}[{\cite[Proposition 7.3.21]{rainone_analysis}}]
  Let $A$ be a Banach algebra. Every character on $A$ is bounded --- furthermore, if $h\in \Omega\left( A \right)$, then $\norm{h}_{\op}\leq 1$, and if $A$ is unital, then $\norm{h}_{\op} = 1$.
\end{proposition}
Note that $\Omega\left( A \right)\subseteq B_{A^{\ast}}\subseteq A^{\ast}$ --- we know of a particular topology on $A^{\ast}$ that endows $B_{A^{\ast}}$ with desirable properties: the weak* topology. This is what we endow $\Omega(A)$ with.
\begin{definition}[{\cite[Definition 7.3.22]{rainone_analysis}}]
   Let $A$ be a Banach algebra, and let $\Omega\left( A \right)\neq \emptyset$. We call the topological space $\left( \Omega\left( A \right),w^{\ast} \right)$ the \textit{character space} of $A$.
\end{definition}

%\begin{definition}
% 
%\end{definition}
\begin{proposition}[{\cite[Proposition 7.3.23]{rainone_analysis}}]
  Let $A$ be a Banach algebra.
  \begin{enumerate}[(1)]
    \item If $A$ is unital, then $\left( \Omega\left( A \right),w^{\ast} \right)$ is a compact Hausdorff space.
    \item If $A$ is nonunital, then $\left( \Omega\left( A \right),w^{\ast} \right)$ is a locally compact Hausdorff space, and the one-point compactification is homeomorphic to the character space of the unitization of $A$, $\left( \Omega\left( \widetilde{A} \right),w^{\ast} \right)$.
  \end{enumerate}
\end{proposition}
Similar to how hyperplanes in vector spaces correspond to kernels of linear functionals (Definition \ref{def:hyperplane}), so too do maximal ideals in commutative and unital Banach algebras correspond to kernels of characters.
\begin{theorem}[{\cite[Theorem 7.3.29]{rainone_analysis}}]
  Let $A$ be a commutative and unital Banach algebra. There is a one-to-one correspondence between the collection of all maximal ideals in $A$,
  \begin{align*}
    \mathcal{M} &\coloneq \set{M\subseteq A | M\text{ is a maximal ideal}},
  \end{align*}
  and the set of characters on $A$, $\Omega\left( A \right)$, given by $h\leftrightarrow \ker\left( h \right)$.
\end{theorem}
Recall from Corollary \ref{cor:characters_algebras} that, given an element $a\in A$ of an algebra, we have
\begin{align*}
  \set{h(a) | h\in \Omega\left( A \right)}\subseteq \sigma\left( a \right).
\end{align*}
When $A$ is a commutative Banach algebra, this yields a stronger result.
\begin{theorem}[{\cite[Theorem 7.3.30]{rainone_analysis}}]
  Let $A$ be a commutative Banach algebra.
  \begin{enumerate}[(1)]
    \item If $A$ is unital, then $\Omega\left( A \right)\neq \emptyset$, and
      \begin{align*}
        \sigma\left( a \right) &= \set{h(a) | h\in \Omega\left( A \right)}
      \end{align*}
      for all $a\in A$.
    \item If $A$ is nonunital, then for all $a\in A$,
      \begin{align*}
        \sigma\left( a \right) &= \set{h(a) | h\in \Omega\left( A \right)} \cup \set{0}.
      \end{align*}
  \end{enumerate}
\end{theorem}
Now, we can discuss the specific case of $C^{\ast}$-algebras. To do this, we need to use a special type of duality, known as the Banach--Stone theorem. We also need to recall some definitions of special continuous function spaces.
\begin{theorem}[{\cite[Theorem 4.8.22]{rainone_analysis}}]
  Let $\Lambda$ and $\Omega$ be compact Hausdorff spaces.
  \begin{enumerate}[(1)]
    \item If $\tau\colon \Omega\rightarrow \Lambda$ is a continuous map, then $T_{\tau}\colon C\left( \Lambda \right)\rightarrow C\left( \Omega \right)$, given by $T_{\tau}\left( f \right) = f\circ \tau$, is a linear contraction, with $\norm{T_{\tau}}_{\op} = 1$. Furthermore,
      \begin{enumerate}[(a)]
        \item $\tau$ is surjective if and only if $T_{\tau}$ is injective, and $T_{\tau}$ is injective if and only if it is isometric;
        \item $\tau$ is injective if and only if $T_{\tau}$ is surjective;
        \item $\tau$ is a homeomorphism if and only if $T_{\tau}$ is an isometric isomorphism of Banach spaces.
      \end{enumerate}
    \item Let $T\colon C\left( \Lambda \right)\rightarrow C\left( \Omega \right)$ be an isometric isomorphism of Banach spaces. Then, there exists a homeomorphism $\tau\colon \Omega\rightarrow \Lambda$ and a continuous map $\alpha\colon \Omega\rightarrow \mathbb{T}$ such that for every $x\in\Omega$ and $g\in C\left( \Lambda \right)$,
      \begin{align*}
        T\left( g \right)\left( x \right) &= \alpha(x)g\left( \tau\left( x \right) \right).
      \end{align*}
  \end{enumerate}
\end{theorem}
\section{The Continuous Functional Calculus}%
