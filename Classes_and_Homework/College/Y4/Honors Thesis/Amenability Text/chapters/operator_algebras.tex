In Chapter \ref{ch:nuclearity}, we will establish that the amenability of a group is equivalent to a property known as nuclearity held by the $C^{\ast}$-algebra(s) generated by the group. For this, we need a solid background in the theory of operator algebras --- specifically, in Banach algebras and $C^{\ast}$-algebras.
\section{Definitions and Examples}%
The theory of $C^{\ast}$-algebras is motivated by the fact that the adjoint operation on $\B\left( \mathcal{H} \right)$ (Definition \ref{def:adjoint_properties}) satisfies the criteria for an involution (Definition \ref{def:algebra_star_algebra}) on an algebra. However, one property that $\B\left( \mathcal{H} \right)$ has that a pure $\ast$-algebra lacks is the fact that $\B\left( \mathcal{H} \right)$ is equipped with a norm, $\norm{\cdot}_{\op}$, that turns $\B\left( \mathcal{H} \right)$ into a normed algebra (Definition \ref{def:norms}).\newline

What the theory of $C^{\ast}$-algebras allows us to do is abstract away from $\B\left( \mathcal{H} \right)$. Soon, we will see that this abstraction will allow us to focus on purely algebraic properties of $C^{\ast}$-algebras and establish fundamental analytic results on them.
\begin{definition}\label{def:banach_star_algebra}
  Let $A$ be an algebra.
  \begin{itemize}
    \item If $\norm{\cdot}$ is such that $\left( A,\norm{\cdot} \right)$ is a Banach space that satisfies $\norm{ab}\leq \norm{a}\norm{b}$ for all $a,b\in A$, then we say $\left( A,\norm{\cdot} \right)$ is a \textit{Banach algebra}.
    \item If $A$ is a $\ast$-algebra that is also a Banach algebra, and the norm on $A$ satisfies $\norm{a^{\ast}} = \norm{a}$ for all $a\in A$, then we say $A$ is a \textit{Banach $\ast$-algebra}.
    \item If $A$ is a Banach $\ast$-algebra whose norm also satisfies $\norm{a^{\ast}a} = \norm{a}^2$ for all $a\in A$, then we say $A$ is a \textit{$C^{\ast}$-algebra}. This final property is known as the $C^{\ast}$-property.
  \end{itemize}
\end{definition}
There are many $C^{\ast}$-algebras that we interact with as we study analysis.
\begin{example}\hfill
  \begin{itemize}
    \item The complex numbers, $\C$, equipped with the involution $z\mapsto \overline{z}$, are a $C^{\ast}$-algebra under the norm $\left\vert z \right\vert$.
    \item If $\mathcal{H}$ is a Hilbert space, then $\B\left( \mathcal{H} \right)$ is a $C^{\ast}$-algebra under the operator norm with the involution $T\mapsto T^{\ast}$.
    \item The space of $n\times n$ complex matrices, $\Mat_n\left( \C \right)$ under the operator norm and the involution $\left( a_{ij}^{\ast} \right)_{ij} = \left( \overline{a_{ji}} \right)_{ij}$ is a $C^{\ast}$-algebra.
    \item If $\Omega$ is any nonempty set, then the space of bounded functions, $\ell_{\infty}\left( \Omega \right)$, is a $C^{\ast}$-algebra under the norm $\norm{f}_{\ell_{\infty}} = \sup_{x\in\Omega}\left\vert f(x) \right\vert$ and the involution $f^{\ast}\left( x \right) = \overline{f(x)}$.
  \end{itemize}
\end{example}
However, there are also some Banach $\ast$-algebras that are not $C^{\ast}$-algebras.
\begin{example}
  Let 
  \begin{align*}
    \ell_1\left( \Z \right)\coloneq \set{f\colon \Z\rightarrow\C | \norm{f}_{\ell_1} \coloneq \sum_{n\in\Z}\left\vert f(n) \right\vert < \infty}
  \end{align*}
  be equipped with the involution
  \begin{align*}
    f^{\ast}\left( n \right) &= \overline{f\left( -n \right)}
  \end{align*}
  and multiplication
  \begin{align*}
    f\ast g(n) &= \sum_{k\in\Z}f(n-k)g(k).
  \end{align*}
  Then, $\ell_1(\Z)$ is a Banach $\ast$-algebra that does not satisfy the $C^{\ast}$-property.
\end{example}
The rest of this section will focus on understanding properties of $C^{\ast}$-algebras and their elements.
\section{\texorpdfstring{$C^{\ast}$-Norms}{C*-Norms}, Universal \texorpdfstring{$C^{\ast}$-Algebras}{C*-Algebras}, and Representations}%
We begin by constructing $C^{\ast}$-algebras.\newline

Recall that, in the case of a normed vector space, we know that (Proposition \ref{prop:completion_existence}) there is always a completion of $X$ into a Banach space, $\widetilde{X}\coloneq \overline{\iota_X\left( X \right)}^{\norm{\cdot}_{\op}}\subseteq X^{\ast\ast}$. This extends to the case of normed algebras/$\ast$-algebras and Banach algebras/Banach $\ast$-algebras.
\begin{lemma}[{\cite[Lemma 7.2.26]{rainone_analysis}}]\label{lemma:banach_algebra_completion}
  If $A_0$ is a normed algebra/$\ast$-algebra, then its Banach space completion, $A$, is a Banach algebra/Banach $\ast$-algebra. The inclusion, $A_0\hookrightarrow A$ is an isometric homomorphism/$\ast$-homomorphism of algebras/$\ast$-algebras.
\end{lemma}
If we have a normed algebra $A$ and we want its completion to be a $C^{\ast}$-algebra, then we need the norm itself to have properties analogous to the norm on a $C^{\ast}$-algebra.
\begin{definition}[{\cite[Definition 7.2.27]{rainone_analysis}}]
  Let $A_0$ be a $\ast$-algebra. A \textit{$C^{\ast}$-norm}/\textit{$C^{\ast}$-seminorm} on $A_0$ is a norm/seminorm on $A_0$ satisfying the following:
  \begin{enumerate}[(i)]
    \item $\norm{ab}\leq \norm{a}\norm{b}$;
    \item $\norm{a^{\ast}} = \norm{a}$;
    \item $\norm{a^{\ast}a} = \norm{a}^2$ (also known as the \textit{$C^{\ast}$-property})
  \end{enumerate}
  for all $a,b\in A_0$.
\end{definition}
We're able to construct $C^{\ast}$-norms by using $\ast$-homomorphisms into $C^{\ast}$-algebras.
\begin{lemma}[{\cite[Lemma 7.2.30]{rainone_analysis}}]
  Let $A_0$ be a $\ast$-algebra, and suppose $\phi\colon A_0\rightarrow B$ is a $\ast$-homomorphism into a $C^{\ast}$-algebra $B$. Then,
  \begin{align*}
    \norm{a}_{\phi} &= \norm{\phi(a)}_{\op}
  \end{align*}
  defines a $C^{\ast}$-seminorm on $A_0$. If $\phi$ is injective, then $\norm{\cdot}_{\phi}$ is a $C^{\ast}$-norm.
\end{lemma}
Just as in the case of Lemma \ref{lemma:banach_algebra_completion}, the completion of a normed algebra with a $C^{\ast}$-norm yields a $C^{\ast}$-algebra.
\begin{lemma}[{\cite[Lemma 7.2.32]{rainone_analysis}}]
  Let $\norm{\cdot}$ be a $C^{\ast}$-norm on a $\ast$-algebra $A_0$. The norm completion, $A$, is a $C^{\ast}$-algebra, and the inclusion $A_0\hookrightarrow A$ is an isometric $\ast$-homomorphism.
\end{lemma}
Recall that any seminorm on a vector space gives rise to a norm on the quotient space (Theorem \ref{thm:quotient_space_norm}) --- similarly, we may define the enveloping $C^{\ast}$-algebra on any $C^{\ast}$-seminorm on $A$.
\begin{definition}[{\cite[Definition 7.2.33]{rainone_analysis}}]\label{def:hausdorff_completion}
  Let $A_0$ be a $\ast$-algebra equipped with a $C^{\ast}$-seminorm $p$. The norm completion of $A/N_p$ with respect to $\norm{\cdot}_{A/N_p}$, where
  \begin{align*}
    N_p \coloneq \set{a\in A | p(a) = 0}
    \intertext{and}
    \norm{a + N_p} &= p(a),
  \end{align*}
  is known as the \textit{Hausdorff completion} or \textit{enveloping $C^{\ast}$-algebra} of $\left( A_0,p \right)$.
\end{definition}
We want to now understand a sort of ``maximal'' enveloping $C^{\ast}$-algebra --- preferably one that admits a universal property, similar to the universal property for the free $\ast$-algebra of Theorem \ref{thm:universal_property_free_algebra}. This will be the universal $C^{\ast}$-algebra.
\begin{definition}[{\cite[Definition 7.2.34]{rainone_analysis}}]
  Let $A_0$ be a $\ast$-algebra, and let $\mathcal{P}$ denote the collection of all $C^{\ast}$-seminorms on $A_0$. Set
  \begin{align*}
    \norm{a}_{u} &= \sup_{p\in \mathcal{P}}p(a).
  \end{align*}
  If $\norm{a}_u < \infty$ for all $a\in A_0$, then $\norm{\cdot}_u$ defines a $C^{\ast}$-seminorm on $A_0$ called the \textit{universal $C^{\ast}$-seminorm} on $A_0$.\newline

  The \textit{universal enveloping $C^{\ast}$-algebra} of $A_0$ is the enveloping $C^{\ast}$-algebra of the pair $\left( A_0,\norm{\cdot}_u \right)$.
\end{definition}
We can also define a universal $C^{\ast}$-algebra with respect to a set of relations $R$ with a similar universal property. 
\begin{definition}[{\cite[Definition 7.2.35]{rainone_analysis}}]
  Let $E$ be a set of abstract variables and suppose $R\subseteq \mathbb{A}^{\ast}\left( E \right)$ is a collection of relations. If the universal enveloping $C^{\ast}$-algebra of $\mathbb{A}^{\ast}\left( E|R \right)$ exists --- i.e., that $\norm{a}_u < \infty$ for all $a\in \mathbb{A}^{\ast}\left( E|R \right)$ --- we denote it $C^{\ast}\left( E|R \right)$. It is know as the \textit{universal $C^{\ast}$-algebra with generators $E$ and relations $R$}.
\end{definition}
\begin{proposition}[{\cite[Proposition 7.2.36]{rainone_analysis}}]\label{prop:universal_property_universal_cstar_algebra}
  Let $E = \set{x_i}_{i\in I}$ be a set of abstract variables, and let $R\subseteq \mathbb{A}^{\ast}\left( E \right)$ be a collection of relations. Suppose the universal $C^{\ast}$-algebra $C^{\ast}\left( E|R \right)$ exists.\newline

  If $B$ is a $C^{\ast}$-algebra admitting elements $\set{b_i}_{i\in I}$ that satisfy the relations $R$, then there is a unique contractive $\ast$-homomorphism, $\varphi_B\colon C^{\ast}\left( E|R \right) \rightarrow B$, such that
  \begin{align*}
    \varphi_b\left( v_i \right) = b_i,
  \end{align*}
  where $v_i \coloneq \left( x + I(R) \right) + N_u$ is a double coset with $I(R)$ as the ideal generated by the $R$ and $N_u$ is the zero set of $\norm{\cdot}_u$, as in Definition \ref{def:hausdorff_completion}.
\end{proposition}
We can realize $\ast$-algebras as $\ast$-subalgebras of bounded operators on Hilbert space.\footnote{In fact, via the GNS construction (which we apologetically cannot cover here), every $C^{\ast}$-algebra can be realized as a $\ast$-subalgebra of $\B\left( \mathcal{H} \right)$ for a suitable Hilbert space $\mathcal{H}$.} 
\begin{definition}[{\cite[Definition 7.2.41]{rainone_analysis}}]\label{def:unital_representation}
  Let $A_0$ be a $\ast$-algebra. A \textit{representation} of $A_0$ is a pair $\left( \pi_0,\mathcal{H} \right)$, where $\pi_0\colon A_0\rightarrow \B\left( \mathcal{H} \right)$ is a $\ast$-homomorphism.\newline

  If $A_0$ is unital, and $\pi_0\left( 1_A \right) = I_{\mathcal{H}}$, then we say $\pi_0$ is a \textit{unital} representation.
\end{definition}
What makes representations special is that they give us a $C^{\ast}$-norm ``for free'' in a sense.
\begin{lemma}[{\cite[Lemma 7.2.42]{rainone_analysis}}]
  Let $A_0$ be a $\ast$-algebra, and let $\left( \pi_0,\mathcal{H} \right)$ be a representation of $A_0$. Then,
  \begin{align*}
    \norm{a}_{\pi_0} &= \norm{\pi_0\left( a \right)}_{\op}
  \end{align*}
  is a $C^{\ast}$-seminorm on $A_0$. If $\pi_0$ is injective, then $\norm{\cdot}_{\pi_0}$ is a $C^{\ast}$-norm.
\end{lemma}
\begin{lemma}[{\cite[Lemma 7.2.43]{rainone_analysis}}]
  If $A_0$ and $B_0$ are normed $\ast$-algebras with completions $A$ and $B$, then any bounded $\ast$-homomorphism extends continuously to $\varphi\colon A\rightarrow B$.
\end{lemma}
\begin{corollary}[{\cite[Corollary 7.2.44]{rainone_analysis}}]
  Let $A_0$ be a $\ast$-algebra, and let $\pi\colon A_0\rightarrow \B\left( \mathcal{H} \right)$ be an injective representation.\newline

  The completion $A$ of $A_0$ with respect to the $C^{\ast}$-norm $\norm{\cdot}_{\pi_0}$ is a $C^{\ast}$-algebra, and the continuous extension $\pi\colon A\rightarrow \B\left( \mathcal{H} \right)$ is an isometric $\ast$-homomorphism.
\end{corollary}
\section{Spectra of Elements in \texorpdfstring{$C^{\ast}$-Algebras}{C*-Algebras}}%
\section{Characters of \texorpdfstring{$C^{\ast}$-Algebras}{C*-Algebras}}%
\section{The Continuous Functional Calculus}%
