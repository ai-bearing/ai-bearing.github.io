Just as God appears in many forms (or representations) throughout the Bible, such as the Burning Bush in the book of Exodus, so too are groups often dealt with and their properties understood through their representations. The field of representation theory, for instance, focuses on the properties of groups as subgroups of groups of linear transformations, and how the properties of these groups of linear transformations can provide insights into the properties of the groups themselves.\newline

In this chapter, we will engage with the properties of groups represented as unitary operators on a Hilbert space --- this will allow us to understand and prove various important results related to groups by using techniques from functional analysis, just as we used techniques of functional analysis to prove the important results in Chapters \ref{ch:invariant_states} and \ref{ch:folner_condition}.
\section{Representing a Group}%
On a Hilbert space $\mathcal{H}$, we know that the set of unitary operators, $\mathcal{U}\left(\mathcal{H}\right)$, is a group under composition.\footnote{To see this, note that $I_{\mathcal{H}}$ is the identity element, that $U^{\ast}U =UU^{\ast} = I_{\mathcal{H}}$, and that if $U,V$ are unitary, then $\left( UV \right)^{\ast}\left( UV \right) = V^{\ast}U^{\ast}UV = I_{\mathcal{H}}$ and similarly for the other way around.} Given any other group $\Gamma$, it is then tempting to consider how we can ``model'' $\Gamma$ (so to speak) as a subgroup of $\mathcal{U}\left(\mathcal{H}\right)$. This is the essence behind the idea of a unitary representation.
\begin{definition}
  Let $\Gamma$ be a group. A unitary representation of $\Gamma$ is a pair, $\left(\pi,\mathcal{H}\right)$, where $\mathcal{H}$ is a Hilbert space and $\pi\colon \Gamma\rightarrow \mathcal{U}\left(\mathcal{H}\right)$ is a group homomorphism.\newline

  Furthermore, every unitary representation $U\colon \Gamma\rightarrow \mathcal{U}\left(\mathcal{H}\right)$, given by $s\mapsto U_s$, has the following properties:
  \begin{itemize}
    \item if $e$ is the identity element for $\Gamma$, then $U_{e} = I_{\mathcal{H}}$;
    \item for all $s\in \Gamma$, $U_{s}^{\ast} = U_{s^{-1}}$.
  \end{itemize}
\end{definition}
\begin{example}\label{ex:some_representations}
  One excellent example of a unitary representation is the representation $1_{\Gamma}\colon \Gamma\rightarrow \C$, defined by $1_{\Gamma}(s) = 1$ for all $s\in\Gamma$. This is known as the trivial representation, and it will play an integral role in establishing amenability.\newline

  A more substantive unitary representation is the representation of the circle group, $\mathbb{T}\rightarrow \B\left(\ell_2\left(\Z\right)\right)$, given by $\omega \mapsto d_{\omega}$. Here, $d_{\omega}$ is the multiplication operator defined by
  \begin{align*}
    d_{\omega}\left(\left(a_{k}\right)_{k\in\Z}\right) &= \left(\omega^k a_{k}\right)_{k\in\Z}.
  \end{align*}
\end{example}
Via the trivial representation, we know that any group can be unitarily represented --- however, the trivial representation is, unfortunately, quite unable to give us information about properties of the underlying group. In general, the Hilbert space we want to represent the group on should, in some way, be based on the underlying group, and the unitary representation to be based on the group's self-action by left-multiplication (see Definition \ref{def:group_action}).\newline

As for the Hilbert space, we will use the space $\ell_2\left( \Gamma \right)$, which, from Definition \ref{def:three_function_spaces}, is the space of all functions $f\colon \Gamma\rightarrow \C$ such that $\sum_{t\in \Gamma}\left\vert f(t) \right\vert^2 < \infty$.
\begin{theorem}[The Left-Regular Representation]\label{thm:left_regular_representation}
  Let $\Gamma$ be a group. For a fixed $t\in\Gamma$, we define $\lambda_t\colon \ell_2\left( \Gamma \right)\rightarrow \ell_2\left( \Gamma \right)$ by
  \begin{align*}
    \lambda_t\left( f \right)\left( s \right) &= f\left( t^{-1}s \right).
  \end{align*}
  Then, $\lambda_t$ is an isometry, and the map
  \begin{align*}
    \lambda\colon \Gamma\rightarrow \B\left( \ell_2\left( \Gamma \right) \right),
  \end{align*}
  given by $t\mapsto \lambda_t$, is a unitary representation of $\Gamma$. This is known as the \textit{left-regular representation} of $\Gamma$.
\end{theorem}
\begin{proof}
  For a fixed $t$, the map $\lambda_t\colon \ell_2\left( \Gamma \right)\rightarrow \ell_2\left( \Gamma \right)$ is a well-defined linear isometry, as
  \begin{align*}
    \norm{\lambda_t\left( f \right)}_{\ell_2}^2 &= \sum_{s\in\Gamma}\left\vert \lambda_t\left( f \right)\left( s \right) \right\vert^2\\
                                                &= \sum_{s\in\Gamma}\left\vert f\left( t^{-1}s \right) \right\vert^2\\
                                                &= \sum_{r\in\Gamma}\left\vert f\left( r \right) \right\vert^2\tag*{$r = t^{-1}s$}\\
                                                &= \norm{f}_{\ell_2}^2.
  \end{align*}
  Now, we know that each $\lambda_s$ has an inverse of $\lambda_{s^{-1}}$, so we know that each $\lambda_s$ is unitary, with $\lambda_s^{\ast} = \lambda_{s^{-1}}$. To evaluate that $\lambda$ is an action, we verify on the orthonormal basis of $\ell_2\left( \Gamma \right)$, $\set{\delta_t}_{t\in\Gamma}$ (see Example \ref{ex:orthonormal_bases}). This gives
  \begin{align*}
    \lambda_s\left( \delta_t \right)\left( r \right) &= \delta_t\left( s^{-1}r \right)\\
                                                     &= \begin{cases}
                                                       1 & s^{-1}r = t\\
                                                       0 & s^{-1}r \neq t
                                                     \end{cases}\\
                                                     &= \begin{cases}
                                                       1 & r = st\\
                                                       0 & r\neq st
                                                     \end{cases}\\
                                                     &= \delta_{st}\left( r \right),
  \end{align*}
  meaning $\lambda_s\left( \delta_t \right) = \delta_{st}$. Additionally, we see that
  \begin{align*}
    \lambda_{s}\circ \lambda_r\left( f \right)\left( t \right) &= \lambda_{r}\left( f \right)\left( s^{-1}t \right)\\
                                                               &= f\left( r^{-1}s^{-1}t \right)\\
                                                               &= f\left( \left( sr \right)^{-1}t \right)\\
                                                               &= \lambda_{sr}\left( f \right)\left( t \right).
  \end{align*}
  Thus, we obtain the unitary representation of $\Gamma$, $\lambda\colon\Gamma\rightarrow \mathcal{U}\left( \ell_2\left( \Gamma \right) \right)$.
\end{proof}
%Insert something here about Fell's absorption principle
\begin{remark}
  The other ``regular representation'' is, predictably, the right-regular representation, given by $s \mapsto \rho_s$, where
  \begin{align*}
    \rho_s\left( f \right)\left( t \right) &= f\left( ts \right).
  \end{align*}
  The right-regular representation acts on orthonormal basis elements by mapping $\delta_t \mapsto \delta_{ts^{-1}}$.\newline

  It can be shown that the left-regular representation and right-regular representation are isomorphic, in the sense that there is a bijective map $\lambda_s\mapsto \rho_s$ that remains faithful to the underlying group structure. However, we will be working with the left-regular representation as it is more commonly when dealing with unitary representations of groups, though it is important to underscore that this is purely personal preference rather than something innate with the left-regular representation itself.
\end{remark}
\section{Almost-Invariant Vectors in the Left-Regular Representation}%
One of the crucial aspects of the left-regular representation is that yet again we are able to use the tools of functional analysis, as in Section \ref{sec:functional_analysis_and_amenability}, to establish amenability. However, this time, rather than being forced to use the dual space of $\ell_{\infty}\left(\Gamma\right)$, we are able to use the properties of $\ell_2\left( \Gamma \right)$ itself rather than being forced to pass to the dual space.\footnote{Technically, this is because, from the Riesz Representation Theorem on Hilbert Spaces (Theorem \ref{thm:riesz_hilbert_spaces}), the space $\ell_2(\Omega)$ is (isomorphic to) its dual. Very convenient, indeed.}\newline

For a given unitary representation $\lambda$, we say a unit vector $\xi\in\ell_2\left( \Gamma \right)$ is invariant for $\lambda$ if, for all $s\in\Gamma$, we have $\lambda_s\left( \xi \right) = \xi$. As it turns out, the existence of a purely invariant vector is a sufficient condition for amenability, though not in a particularly eye-catching manner.

 %If $\lambda\colon \Gamma\rightarrow \mathcal{U}\left( \ell_2\left( \Gamma \right) \right)$ is a unitary representation, then we say a (unit) vector $\xi\in\ell_2\left( \Gamma \right)$ is invariant for $\lambda$ if $\lambda_s\left( \xi \right) = \xi$ for all $s\in\Gamma$. The existence of a purely invariant vector is actually a sufficient condition for amenability --- unfortunately, though, not in a manner that is especially interesting.
\begin{theorem}
  Let $\Gamma$ be a group, and let $\lambda\colon \Gamma\rightarrow \mathcal{U}\left( \ell_2\left( \Gamma \right) \right)$ be the left-regular representation. Then, $\lambda$ admits an invariant vector if and only if $\Gamma$ is finite.
\end{theorem}
\begin{proof}
  Let $\Gamma$ be finite. Since all functions $f\colon \Gamma\rightarrow \C$ are square-summable, as $\Gamma$ is finite, so too is $\xi = \1_{\Gamma}$. Since $s\Gamma = \Gamma$ for all $s\in\Gamma$, we have $\1_{\Gamma}$ is invariant for $\lambda$.\newline

  Now, let $\lambda\colon \Gamma\rightarrow \mathcal{U}\left( \ell_2\left( \Gamma \right) \right)$ be the left-regular representation, and suppose there is $\xi\in \ell_2\left( \Gamma \right)$ such that for all $s\in\Gamma$, we have
  \begin{align*}
    \lambda_s\left( \xi \right) &= \xi.
  \end{align*}
  In particular, this means that for all $t\in\Gamma$, we have
  \begin{align*}
    \lambda_s\left( \xi \right)\left( t \right) &= \xi\left( s^{-1}t \right)\\
                                                &= \xi\left( t \right).
  \end{align*}
  Now, since this holds for all $s\in\Gamma$, this means that $\xi\left( t \right) = \xi\left( s \right)$ for any $s\neq t$, as we may find $r\in \Gamma$ such that $r^{-1}t = s$ so that $\lambda_r\left( \xi \right)\left( t \right) = \xi\left( s \right)$. Therefore, $\xi = c\1_{\Gamma}$ for some $c\in\C$.\newline

  Now, since $\xi\in \ell_2\left( \Gamma \right)$, we must have that
  \begin{align*}
    \sum_{t\in\Gamma}\left\vert \xi(t) \right\vert^2 < \infty.
  \end{align*}
  This is equivalent to the condition that
  \begin{align*}
    \sum_{t\in\Gamma}\left\vert c \right\vert^2 &< \infty.
  \end{align*}
  This can only hold if $\Gamma$ is finite.
\end{proof}
Now, finite groups are amenable (by Example \ref{ex:finite_invariant_state}), but sadly that is not very interesting, and this is not helpful for the various infinite groups we hope to establish the amenability of. What is interesting, though, is that the existence of an \textit{almost}-invariant vector for $\lambda$ characterizes amenability.\newline

To prove that the existence of an almost-invariant vector for $\lambda$ is equivalent to amenability, however, we need to use a different version of the approximate mean defined in Definition \ref{def:approximate_mean}. This is also known as Reiter's condition.
\begin{theorem}[Reiter's Condition]\label{thm:reiter}
  Let $\Gamma$ be a (countable, discrete) group. Then, $\Gamma$ is amenable if and only if, for any $\ve > 0$ and for any finite subset $E\subseteq G$, there is a $\mu\in \Prob(G)$ (see Definition \ref{def:state_on_prob_g}) such that $\norm{\lambda_s\left( \mu \right) - \mu}_{\ell_1} \leq \ve$.
\end{theorem}
\begin{proof}
  We will show that Reiter's condition is equivalent to the existence of an approximate mean.\newline

  Suppose $\Gamma$ is amenable. Then, $\Gamma$ admits a sequence of (unit) vectors, $\left( f_k \right)_k$ such that 
  \begin{align*}
    \norm{\lambda_s\left( f_k \right) - f_k}_{\ell_1}\rightarrow 0
  \end{align*}
   for all $s\in\Gamma$.\newline

   If $\ve > 0$, then Reiter's condition follows from finding $K$ so large such that $\norm{\lambda_s\left( f_K \right) - f_K}_{\ell_1} < \ve$, and the result then holds for any finite $E\subseteq \Gamma$ as it must hold for all $s\in \Gamma$.\newline

   Now, we suppose that $\Gamma$ satisfies Reiter's condition. Let $\Gamma = \bigcup_{n\geq 1}E_n$, where each of the $E_n$ are finite, and $E_1\subseteq E_2\subseteq \cdots$ are nested. For each $E_n$, we may find a sequence of vectors $\left( f_{k,n} \right)_k$ such that $\norm{\lambda_s\left( f_{k,n} \right) - f_{k,n}}_{\ell_1} < 1/k$ for all $s\in E_n$.\newline

   We define $\mu_{n} = f_{n,n}$. Then, for any $s\in \Gamma$, we may find $N$ such that $s\in E_{k}$, for all $k\geq N$, and we know that by the definition, we have $\norm{\lambda_s\left( \mu_{N} \right) - \mu_{N}}_{\ell_1} < \frac{1}{N}$. The sequence $\left( \mu_{n} \right)_{n}$ then forms an approximate mean as in Definition \ref{def:approximate_mean}.
\end{proof}
\begin{definition}\label{def:almost_invariant_vector}
  Let $\Gamma$ be a group and $\lambda\colon \Gamma\rightarrow \mathcal{U}\left( \ell_2\left( \Gamma \right) \right)$ be the left-regular representation of $\Gamma$. We say $\lambda$ admits an \textit{almost-invariant vector} if there is a sequence of unit vectors $\left( \xi_n \right)_n$ in $\ell_2\left( \Gamma \right)$ such that, for all $s\in\Gamma$, we have
  \begin{align*}
    \norm{\lambda_s\left( \xi_n \right) - \xi_n}_{\ell_2} &\xrightarrow{n\rightarrow\infty} 0.
  \end{align*}
  %We call the sequence $\left( \xi_n \right)_n$ the almost-invariant vector for $\lambda$.
\end{definition}
\begin{theorem}\label{thm:almost_invariant_vector}
  Let $\Gamma$ be a group, and $\lambda\colon \Gamma\rightarrow \mathcal{U}\left( \ell_2\left( \Gamma \right) \right)$ be the left-regular representation of $\Gamma$.\newline

  Then, $\Gamma$ is amenable if and only if $\lambda$ admits an almost-invariant vector.
\end{theorem}
\begin{proof}
  Let $\Gamma$ be amenable. Then, by the results in Section \ref{sec:approximate_means}, we know that $\Gamma$ admits a Følner sequence, $\left( F_n \right)_{n}$, such that
  \begin{align*}
    \frac{\left\vert sF_n\triangle F_n \right\vert}{\left\vert F_n \right\vert} \rightarrow 0
  \end{align*}
  for all $s\in\Gamma$. We define the unit vectors $\xi_n$ by
  \begin{align*}
    \xi_n &= \frac{1}{\sqrt{\left\vert F_n \right\vert}} \1_{F_n}.
  \end{align*}
  Then, we have that
  \begin{align*}
    \norm{\lambda_s\left( \xi_n \right) - \xi_n}_{\ell_2}^2 &= \sum_{t\in\Gamma}\left\vert \lambda_x\left( \xi_n \right)\left( t \right) - \xi_n\left( t \right) \right\vert^2\\
                                                            &= \sum_{t\in\Gamma}\left\vert \lambda_s\left( \frac{1}{\sqrt{\left\vert F_n \right\vert}}\1_{F_n} \right)(t) - \frac{1}{\sqrt{\left\vert F_n \right\vert}}\1_{F_n}(t) \right\vert^2\\
                                                            &= \sum_{t\in\Gamma}\left\vert \frac{1}{\sqrt{\left\vert F_n \right\vert}}\1_{sF_n}(t) - \frac{1}{\sqrt{\left\vert F_n \right\vert}}\1_{F_n}(t) \right\vert^2\\
                                                            &= \frac{\left\vert sF_n\triangle F_n \right\vert}{\left\vert F_n \right\vert}\\
                                                            &\rightarrow 0.
  \end{align*}
  Thus, $\lambda$ admits an almost invariant vector.\newline

  Now, suppose there exists an almost-invariant vector $\left( \xi_n \right)_n\in \ell_2\left( \Gamma \right)$. We define $\mu_n = \xi_n^2$. From Hölder's inequality (Theorem \ref{thm:holder_inequality}), we know that $\mu_n\in \ell_1\left( \Gamma \right)$. Substituting into the definition of an approximate mean, we obtain
  \begin{align*}
    \norm{\lambda_s\left( \mu_n \right) - \mu_n}_{\ell_1} &= \sum_{t\in\Gamma}\left\vert \lambda_s\left( \xi_n^2 \right)\left( t \right) - \xi_n^2\left( t \right)\right\vert\\
                                                          &= \sum_{t\in\Gamma}\left\vert \left( \lambda_s\left( \xi_n \right)(t) - \xi_n\left( t \right) \right)\left( \lambda_s\left( \xi_n \right)(t) + \lambda_s\left( t \right) \right) \right\vert\\
                                                          &= \norm{\left( \lambda_s\left( \xi_n \right) - \xi_n \right)\left( \lambda_s\left( \xi_n \right) + \xi_n \right)}_{\ell_1}\\
                                                          &\leq\norm{\lambda_s\left( \xi_n \right) + \xi_n}_{\ell_2} \norm{\lambda_s\left( \xi_n \right) - \xi_n}_{\ell_2}\\
                                                          &\leq 2\norm{\lambda_s\left( \xi_n \right) - \xi_n}_{\ell_2}\\
                                                          &\rightarrow 0.
  \end{align*}
  Thus, $\left( \mu_n \right)_n$ is an approximate mean, hence amenable.
\end{proof}
\section{A Potpurri of Characterizations}%
Now, we may use the almost-invariant vectors criterion to prove amenability via various different methods. We start by diving into some theory behind representations and weak containment, then go into two criteria for amenability that are very intimately tied to the analytic properties of the left-regular representation.
\subsection{Weak Containment}%
Loosely speaking, weak containment is a type of approximation property for unitary representations of groups. What we will show in this subsection is that, if $\Gamma$ is a group, and the left-regular representation weakly contains the trivial representation, then the group $\Gamma$ is amenable.
\begin{definition}
  Let $\Gamma$ be a group, and let $\pi\colon \Gamma\rightarrow \mathcal{U}\left( \mathcal{H} \right)$ and $\rho\colon \Gamma\rightarrow \mathcal{U}\left( \mathcal{K} \right)$ be two unitary representations on separate Hilbert spaces $\mathcal{H}$ and $\mathcal{K}$. We say $\pi$ is weakly contained in $\rho$, written $\pi\prec \rho$, if, for any finite subset $E\subseteq \Gamma$ and any $\ve > 0$, and for all $\xi\in \mathcal{H}$, there are $\eta_1,\dots,\eta_k$ such that
  \begin{align*}
    \left\vert \iprod{\pi(g)\left( \xi \right)}{\xi} - \sum_{i=1}^{n} \iprod{\rho\left( g \right)\left( \eta_i \right)}{\eta_i} \right\vert < \ve
  \end{align*}
  for all $g \in E$.
\end{definition}
In order to prove the full weak containment result, we will need to make use of some lemmas that show certain convergence properties hold between inner products and norms.
\begin{lemma}\label{lemma:complex_identities_left_regular_representation}
  Let $\xi$ be a unit vector, and let $\lambda_g\colon \ell_2\left( \Gamma \right)\rightarrow \ell_2\left( \Gamma \right)$ be given by $\lambda_g\left( \xi \right)\left( t \right) = \xi\left( g^{-1}t \right)$. Then,
  \begin{enumerate}[(a)]
    \item $\displaystyle \norm{\lambda_g\left( \xi \right) - \xi}_{\ell_2}^2 \leq 2\left\vert 1- \iprod{\lambda_g\left( \xi \right)}{\xi} \right\vert$
    \item $\displaystyle \left\vert 1- \iprod{\lambda_g\left( \xi \right)}{\xi} \right\vert \leq \norm{\lambda_g\left( \xi \right) - \xi}_{\ell_2}$.
  \end{enumerate}
\end{lemma}
\begin{proof}\hfill
  \begin{enumerate}[(a)]
    \item Directly calculating, we have
      \begin{align*}
        \norm{\lambda_g\left( \xi \right) - \xi}_{\ell_2}^2 &= \iprod{\lambda_g\left( \xi \right) - \xi}{ \lambda_g\left( \xi \right) - \xi }\\
                                                            &= \iprod{\lambda_g\left( \xi \right)}{\lambda_g\left( \xi \right)} + \iprod{\xi}{\xi} - \iprod{\lambda_g\left( \xi \right)}{\xi} - \iprod{\xi}{\lambda_g\left( \xi \right)}\\
                                                            &= \iprod{\lambda_g\left( \xi \right)}{\lambda_g\left( \xi \right)} + \iprod{\xi}{\xi} - 2\re\left( \iprod{\lambda_g\left( \xi \right)}{\xi} \right)\\
                                                            &= 2 - 2\re\left( \iprod{\lambda_g\left( \xi \right)}{\xi} \right)\\
                                                            &= 2\re\left( 1- \iprod{\lambda_g\left( \xi \right)}{\xi} \right)\\
                                                            &\leq 2 \left\vert 1- \iprod{\lambda_g\left( \xi \right)}{\xi} \right\vert.
      \end{align*}
    \item Similarly, direct calculation gives
      \begin{align*}
        \left\vert 1- \iprod{\lambda_g\left( \xi \right)}{\xi} \right\vert &= \left( 1- \iprod{\lambda_g\left( \xi \right)}{\xi} \right) \overline{\left( 1 - \iprod{\lambda_g\left( \xi \right)}{\xi} \right)}\\
                                                                           &= 1 - \overline{ \iprod{\lambda_g\left( \xi \right)}{\xi} } - \iprod{\lambda_g\left( \xi \right)}{\xi} + \left\vert \iprod{\lambda_g\left( \xi \right)}{\xi} \right\vert^2\\
                                                                           &\leq 2 - 2\re\left( \iprod{\lambda_g\left( \xi \right)}{\xi} \right)\\
                                                                           &= \norm{\lambda_g\left( \xi \right) - \xi}_{\ell_2}^2.
      \end{align*}
  \end{enumerate}
\end{proof}
\begin{lemma}\label{lemma:l2_unit_vector}
  If $\lambda\colon\Gamma\rightarrow \mathcal{U}\left( \ell_2\left( \Gamma \right) \right)$ is the left-regular representation, then $1_{\Gamma}\prec \lambda$ if and only if, for every finite subset $S\subseteq \Gamma$ and every $\ve > 0$, there exists a unit vector $\xi\in \ell_2\left( \Gamma \right)$ such that
  \begin{align*}
    \norm{\lambda_s\left( \xi \right) - \xi}_{\ell_2} &< \ve.
  \end{align*}
\end{lemma}
\begin{proof}
  Suppose $1_{\Gamma}\prec \lambda$. Then, there exists a unit vector $\xi$ such that $\left\vert 1 - \iprod{\lambda_s\left( \xi \right)}{\xi} \right\vert < \ve^2/2$. So, by Lemma \ref{lemma:complex_identities_left_regular_representation} (a), we have $\norm{\lambda_s\left( \xi \right) - \xi}_{\ell_2} < \ve$.\newline

  Similarly, if $\norm{\lambda_s\left( \xi \right) - \xi}_{\ell_2} < \ve$, then we know from Lemma \ref{lemma:complex_identities_left_regular_representation} (b) that $ \left\vert 1 - \iprod{\lambda_s\left( \xi \right)}{\xi} \right\vert < \ve $.
\end{proof}
\begin{theorem}
  Let $\Gamma$ be a discrete group. Then, $\Gamma$ is amenable if and only if $1_{\Gamma}\prec \lambda$, where $1_{\Gamma}$ is the trivial representation (see Example \ref{ex:some_representations}) and $\lambda$ is the left-regular representation.
\end{theorem}
\begin{proof}
  For one direction, we will show that $1_{\Gamma}\prec \lambda$ if and only if $\lambda$ admits an almost-invariant vector. By Theorem \ref{thm:almost_invariant_vector}, this means $\Gamma$ is amenable.\newline

  Let $\Gamma$ be amenable. Let $E\subseteq \Gamma$ be any finite set and $\ve > 0$, and let $\xi$ be almost-invariant for all $g\in E$ --- that is, $\norm{\lambda_g\left( \xi \right) - \xi}_{\ell_2} < \ve$ for all $g\in E$. Then, by Lemma \ref{lemma:complex_identities_left_regular_representation} (b), we must have
  \begin{align*}
    \left\vert 1 - \iprod{\lambda_g\left( \xi \right)}{\xi} \right\vert &\leq \norm{\lambda_g\left( \xi \right) - \xi}_{\ell_2}\\
    &< \ve.
  \end{align*}
  Thus, $1_{\Gamma}\prec \lambda$ when we take $n = 1$ and $\eta_1 = \xi$.\newline

  Now, in the reverse direction, we suppose that $1_{\Gamma}\prec \lambda$. Then, we know, by Lemma \ref{lemma:l2_unit_vector}, that for any finite subset $E\subseteq \Gamma$ and any $\ve > 0$, there is some unit vector $f\in \ell_2\left( \Gamma \right)$ such that $\norm{\lambda_s\left( f \right) - f}_{\ell_2} < \ve$ for all $s\in E$.\newline

  Set $g = \left\vert f \right\vert^2$. We have that $g\in \ell_1\left( \Gamma \right)$, and from Hölder's inequality, we obtain
  \begin{align*}
    \norm{\lambda_s\left( g \right) - g}_{\ell_1} &= \norm{\lambda_s\left( \left\vert f \right\vert^2 \right) - \left\vert f \right\vert^2}_{\ell_1}\\
                                                  &= \norm{\left( \lambda_s\left( f \right) - f \right)\left( \lambda_s\left( \overline{f} \right) + \overline{f} \right)}_{\ell_1}\\
                                                  &\leq \norm{\lambda_s\left( \overline{f} \right) + \overline{f}}_{\ell_2}\norm{\lambda_s\left( f \right) - f}_{\ell_2}\\
                                                  &\leq 2\norm{\lambda_s\left( f \right) - f}_{\ell_2}\\
                                                  & < 2\ve,
  \end{align*}
  meaning that $\Gamma$ satisfies Reiter's condition (Theorem \ref{thm:reiter}), and is thus amenable.
\end{proof}
\subsection{Kesten's Criterion}%
Kesten's criterion, expanded upon in \cite{kesten_random_walks} and \cite{kesten_means}, originated in the study of random walks on the generators of finitely generated groups.\newline

Consider a finitely supported probability measure $\mu$ on a (discrete, finitely generated) group $\Gamma$ with the property that $\mu\left( g \right) = \mu\left( g^{-1} \right)$ for all $g\in \supp\left( \mu \right)$. If our symmetric generating set $S$ is a subset of $\supp\left( \mu \right)$, then we may consider a random walk on the group by sampling elements of $S$ and concatenating them together --- then, we may, for instance, ask the probability of returning to $e_{\Gamma}$ in $n$ steps, for some $n$.\newline

Kesten showed that the probability of doing so in a certain number of steps was intimately tied to the spectral radius (or operator norm) of a matrix of probabilities.\newline

We will begin by showing some important results from the theory of self-adjoint operators before establishing Kesten's criterion for the special case where $\supp(\mu) = S$ and $\mu$ has a uniform distribution over $S$ --- i.e., $\mu\left( g \right) = \frac{1}{\left\vert S \right\vert}$. The more general case requires deeper results in spectral theory.
\begin{lemma}\label{lemma:norm_self_adjoint_operator}
  Let $\mathcal{H}$ be a Hilbert space, and let $T\in\B\left( \mathcal{H} \right)$ be a self-adjoint operator (see Definition \ref{def:distinguished_operators}). Then, the operator norm of $T$ is determined by
  \begin{align*}
    \norm{T}_{\op} &= \sup_{x\in S_{\mathcal{H}}} \left\vert \iprod{T\left( x \right)}{x} \right\vert.
  \end{align*}
\end{lemma}
\begin{proof}
  Using Cauchy--Schwarz, one of the directions immediately becomes clear:
  \begin{align*}
    \left\vert \iprod{T\left( x \right)}{x} \right\vert &\leq \norm{T\left( x \right)}\norm{x}\\
                                                        &\leq \norm{T}_{\op}\norm{x}^2\\
                                                        &= \norm{T}_{\op}.
  \end{align*}
  To establish the opposite direction requires a bit more work. First, we recall the definition of the operator norm, which states that
  \begin{align*}
    \norm{T}_{\op} &= \sup_{x,y\in S_{\mathcal{H}}}\left\vert \iprod{T\left( x \right)}{y} \right\vert.
  \end{align*}
  We set 
  \begin{align*}
    \alpha = \sup_{x\in S_{\mathcal{H}}} \left\vert \iprod{T\left( x \right)}{x} \right\vert.
  \end{align*}
  Notice that for any nonzero $x\in \mathcal{H}$, we have
  \begin{align*}
    \left\vert \iprod{T\left( \frac{x}{\norm{x}} \right)}{\frac{x}{\norm{x}}} \right\vert &\leq \alpha\\
    \left\vert \iprod{T\left( x \right)}{x} \right\vert &\leq \alpha \norm{x}^2.
  \end{align*}
  
  Recall that if $T$ is self-adjoint, then for any $x\in \mathcal{H}$, $ \iprod{T\left( x \right)}{x} $ is real. This can be shown relatively easily using the properties of adjoints and inner products:
  \begin{align*}
    \iprod{T\left( x \right)}{x} &= \iprod{x}{T^{\ast}\left( x \right)}\\
                                 &= \iprod{x}{T\left( x \right)}\\
                                 &= \overline{ \iprod{T\left( x \right)}{x} }.
  \end{align*}
  Now, let $x,y\in S_{\mathcal{H}}$. We may assume that $ \iprod{T\left( x \right)}{y} \in \R $, as we may multiply $x$ by $\sgn\left( \iprod{T\left( x \right)}{y} \right)$, where $\sgn(z) = \frac{|z|}{z}\in \C$, and by the Polarization Identity (Theorem \ref{thm:polarization}) and the fact that $T$ is self-adjoint, we get
  \begin{align*}
    \iprod{T\left( x \right)}{y} &= \frac{1}{4}\left( \iprod{T\left( x+y \right)}{x+y} - \iprod{T\left( x-y \right)}{x-y} \right).
  \end{align*}
  Thus, applying absolute values, we obtain
  \begin{align*}
    \left\vert \iprod{T\left( x \right)}{y}  \right\vert &= \left\vert \frac{1}{4}\left( \iprod{T\left( x+y \right)}{x+y} - \iprod{T\left( x-y \right)}{x-y} \right) \right\vert\\
                                                         &\leq \frac{1}{4}\left( \left\vert \iprod{T\left( x+y \right)}{x+y} \right\vert +  \left\vert \iprod{T\left( x-y \right)}{x-y} \right\vert\right)\\
                                                         &\leq \frac{\alpha}{4} \left( \norm{x+y}^2 + \norm{x-y}^2 \right)\\
                                                         &\leq \frac{\alpha}{4} \left( 2\norm{x}^2 + 2\norm{y}^2 \right)\\
                                                         &= \alpha.
  \end{align*}
  Thus,
  \begin{align*}
    \norm{T}_{\op} &= \sup_{x,y\in S_{\mathcal{H}}} \left\vert \iprod{T\left( x \right)}{y} \right\vert\\
                   &\leq \alpha,
  \end{align*}
  so
  \begin{align*}
    \norm{T}_{\op} &= \sup_{x\in S_{\mathcal{H}}} \left\vert \iprod{T\left( x \right)}{x} \right\vert.
  \end{align*}
\end{proof}
\begin{definition}
  Let $\lambda\colon \Gamma\rightarrow \mathcal{U}\left( \ell_2\left( \Gamma \right) \right)$ be the left-regular representation. For a finite set $E\subseteq \Gamma$, we define the \textit{Markov operator} $M(E)$ by
  \begin{align*}
    M(E) &= \frac{1}{\left\vert E \right\vert} \sum_{t\in E}\lambda_t.
  \end{align*}
\end{definition}
\begin{fact}
  For any $E\subseteq \Gamma$, $M(E)$ is a contraction.
\end{fact}
\begin{proof}
  Note that $\lambda_t$ is an isometry for any $t\in \Gamma$. This yields
  \begin{align*}
    \norm{M(E)}_{\op} &= \norm{\frac{1}{\left\vert E \right\vert} \sum_{t\in E}\lambda_t}_{\op}\\
                      &= \frac{1}{\left\vert E \right\vert}\norm{\sum_{t\in E}\lambda_t}_{\op}\\
                      &\leq \frac{1}{\left\vert E \right\vert}\sum_{t\in E}\norm{\lambda_t}_{\op}\\
                      &= 1.
  \end{align*}
\end{proof}
\begin{theorem}[Kesten's Criterion]\label{thm:kesten_criterion}
  Let $\Gamma$ be a group with finite symmetric generating set $S$. Then, $\Gamma$ is amenable if and only if
  \begin{align*}
    \norm{M(S)}_{\op} &= 1.
  \end{align*}
\end{theorem}
\begin{proof}
  Let $\Gamma$ be amenable. Then, $\lambda$ admits an almost-invariant vector, $\left( \xi_n \right)_n\subseteq S_{\ell_2\left( \Gamma \right)}$. This gives
  \begin{align*}
    \norm{\lambda_s\left( \xi_n \right) - \xi_n}_{\ell_2} &\rightarrow 0
  \end{align*}
  for all $s\in \Gamma$. Therefore, by the reverse triangle inequality, we have
  \begin{align*}
    \left\vert \left(\norm{\left(\frac{1}{\left\vert S \right\vert}\sum_{t\in S}\lambda_t\right)\left(\xi_n\right)}_{\ell_2}\right)  - \norm{\xi_n}_{\ell_2}\right\vert &\leq \norm{\left(\frac{1}{\left\vert S \right\vert}\sum_{t\in S}\lambda_t\right)\left(\xi_n\right) - \xi_n}_{\ell_2}\\
                                                                                                                                                                                  &= \frac{1}{\left\vert S \right\vert}\norm{\left(\sum_{t\in S}\lambda_t\right)\left(\xi_n\right) - \left\vert S \right\vert\xi_n}_{\ell_2}\\
                                                                                                                        &\leq \frac{1}{\left\vert S \right\vert} \sum_{t\in S}\norm{\lambda_t\left(\xi_n\right) - \xi_n}_{\ell_2}\\
                                                                                                                        &\rightarrow 0,
  \end{align*}
  meaning that
  \begin{align*}
    \sup_{\xi\in S_{\ell_2\left(\Gamma\right)}} \norm{\left(\frac{1}{\left\vert S \right\vert}\sum_{t\in S}\lambda_t\right)\left(\xi\right)} &= \norm{\xi},
  \end{align*}
  and so the norm of the Markov operator is $1$.\newline

  Now, suppose $M(S) = 1$. Since $S$ is symmetric, $M(S)$ is self-adjoint, so by \ref{lemma:norm_self_adjoint_operator}, for any $n\in \N$, there is some $\xi_n\in S_{\ell_2\left( \Gamma \right)}$ such that
  \begin{align*}
    1-\frac{1}{n} &< \iprod{\left( \frac{1}{\left\vert S \right\vert}\sum_{t\in S}\lambda_t \right)\left( \xi_n \right)}{\xi_n}\\
                  &\leq \iprod{\left( \frac{1}{\left\vert S \right\vert}\sum_{t\in S}\lambda_t \right)\left( \left\vert \xi_n \right\vert \right)}{\left\vert \xi_n \right\vert}.
  \end{align*}
  Rearranging, we get
  \begin{align*}
    1- \iprod{\left( \frac{1}{\left\vert S \right\vert}\sum_{t\in S}\lambda_t \right)\left( \left\vert \xi_n \right\vert \right)}{\left\vert \xi_n \right\vert} &< \frac{1}{n}.
  \end{align*}
  Since $M(S)$ is a self-adjoint operator, we have
  \begin{align*}
    \re\left( \iprod{\left( \frac{1}{\left\vert S \right\vert}\sum_{t\in S}\lambda_t \right)\left( \left\vert \xi_n \right\vert \right)}{\left\vert \xi_n \right\vert} \right) &= \iprod{\left( \frac{1}{\left\vert S \right\vert}\sum_{t\in S}\lambda_t \right)\left( \left\vert \xi_n \right\vert \right)}{\left\vert \xi_n \right\vert}.
  \end{align*}
  From Lemma \ref{lemma:complex_identities_left_regular_representation}, we have that
  \begin{align*}
    \norm{\left( \frac{1}{\left\vert S \right\vert}\sum_{t\in S}\lambda_t \right)\left( \left\vert \xi_n \right\vert \right) - \left\vert \xi_n \right\vert} &\leq \frac{1}{\left\vert S \right\vert}\sum_{t\in S}\norm{\lambda_t\left( \left\vert \xi_n \right\vert \right) - \left\vert \xi_n \right\vert}\\
                                                                                                                                                                    &\leq \frac{1}{\left\vert S \right\vert}\sum_{t\in S}\sqrt{2} \left\vert 1 - \iprod{\lambda_t\left( \left\vert \xi_n \right\vert \right)}{\left\vert \xi_n \right\vert} \right\vert\\
                                                                                                                                                                    &= \sqrt{2}\left\vert 1 - \frac{1}{\left\vert S \right\vert} \sum_{t\in S} \iprod{\lambda_t\left( \left\vert \xi_n \right\vert \right)}{\left\vert \xi_n \right\vert} \right\vert\\
                                                                                                                                                                    &< \frac{1}{n}.
  \end{align*}
  Thus, $\lambda$ admits an almost-invariant vector, and hence is amenable by Theorem \ref{thm:almost_invariant_vector}.
\end{proof}
\subsection{Hulanicki's Criterion}%
Kesten's criterion is especially useful in establishing a similar result known as Hulanicki's criterion. Hulanicki's criterion uses a similar operator that depends on the left-regular representation, and shows that this operator's norm serves as a bound on the sum of any positive, finitely-supported function on the group $\Gamma$.
\begin{definition}[{\cite[141]{juschenko_amenability}}]
  Let $f\in \ell_1\left( \Gamma \right)$. We define the bounded operator $\lambda_{f(t)}$ by
  \begin{align*}
    \lambda_{f(t)} &= \sum_{t\in\Gamma}f(t)\lambda_t,
  \end{align*}
  where $\lambda_t$ denotes the left-regular representation of $\Gamma$ evaluated at $t$.
\end{definition}
\begin{theorem}[{\cite[Theorem A.11]{juschenko_amenability}}]
  If $\Gamma$ is a discrete group, then $\Gamma$ is amenable if and only if, for any positive, finitely-supported function $f\colon \Gamma\rightarrow \C$, we have
  \begin{align*}
    \sum_{t\in\Gamma}f(t) &\leq \norm{\lambda_{f(t)}}_{\op}.
  \end{align*}
\end{theorem}
\begin{proof}
  Suppose $\Gamma$ is amenable. Let $f\colon \Gamma\rightarrow \C$ be a positive, finitely supported function, and let $\left( F_n \right)_n$ be a Følner sequence in $\Gamma$ such that for any $s\in \supp(f)$, we have
  \begin{align*}
    \frac{\left\vert sF_n \triangle F_n\right\vert}{\left\vert F_n \right\vert} &\leq \frac{1}{n}.
  \end{align*}
  Letting
  \begin{align*}
    \xi_n &= \frac{1}{\sqrt{\left\vert F_n \right\vert}}\1_{F_n},
  \end{align*}
  we note that $\norm{\xi_n}_{\ell_2} = 1$, and that this is the exact same almost-invariant vector for $\lambda$ that we used in Theorem \ref{thm:almost_invariant_vector}.\newline

  We will now use the fact that, for any $T\in \B\left( \mathcal{H} \right)$,
  \begin{align*}
    \sup_{x\in S_{\mathcal{H}}}\left\vert \iprod{T(x)}{x} \right\vert &\leq \norm{T}_{\op},
  \end{align*}
  which follows from the definition of the operator norm on Hilbert spaces.\newline

  This gives, 
  \begin{align*}
    \left\vert \iprod{\left( \sum_{t\in\Gamma}f(t)\lambda_t \right)\left( \xi_n \right)}{\xi_n} \right\vert \leq \norm{\lambda_{f(t)}}_{\op},
  \end{align*}
  meaning that, since the quantity on the left side is positive
  \begin{align*}
    \sup \left\vert \iprod{\left( \sum_{t\in\Gamma}f(t)\lambda_t \right)\left( \xi_n \right)}{\xi_n} \right\vert &= \lim_{n\rightarrow\infty}\left\vert \iprod{\left( \sum_{t\in\Gamma} f(t)\lambda_t\right)\left( \xi_n \right)}{\xi_n} \right\vert \\
                                                                                                                 &\leq \norm{\lambda_{f(t)}}_{\op}.
  \end{align*}
  Now, notice that, since $\xi_n$ is the almost-invariant vector we constructed in the proof of Theorem \ref{thm:almost_invariant_vector}, we have that $\norm{\lambda_t\left( \xi_n \right)-\xi_n}_{\ell_2}\xrightarrow{n\rightarrow\infty} 0$, or that, for all $s\in\Gamma$, $\xi_n\left( t^{-1}s \right)\xrightarrow{n\rightarrow\infty} \xi_n\left( s \right)$. Therefore, 
  \begin{align*}
    \lim_{n\rightarrow\infty} \left\vert \iprod{\left( \sum_{t\in\Gamma}f(t)\lambda_t \right)\left( \xi_n \right)}{\xi_n} \right\vert &= \lim_{n\rightarrow\infty}\left\vert \sum_{t,s\in\Gamma}f(t)\lambda_t\left( \xi_n \right)\left( s \right)\overline{\xi_n}\left( s \right) \right\vert\\
                                                                                                                                      &= \lim_{n\rightarrow\infty}\left\vert \sum_{t,s\in\Gamma}f(t)\xi_n\left( t^{-1}s \right)\overline{\xi_n}\left( s \right) \right\vert\\
                                                                                                                                      &= \lim_{n\rightarrow\infty}\left\vert \sum_{t,s\in\Gamma}f(t)\left\vert \xi_n\left( s \right) \right\vert^2 \right\vert\\
                                                                                                                                      &= \sum_{t\in\Gamma}f(t)\left( \sum_{s\in\Gamma}\left\vert \xi_n\left( s \right) \right\vert^{2} \right)\\
                                                                                                                                      &= \sum_{t\in\Gamma}f(t).
  \end{align*}
  Therefore, there is some constant $C$ such that
  \begin{align*}
    \sum_{t\in\Gamma}f(t) &\leq C\norm{\lambda_{f(t)}}_{\op}.
  \end{align*}
  Now, to show that $C = 1$, we note that by the definition of convolution (see Definition \ref{def:group_star_algebra}), we have
  \begin{align*}
    \left( \sum_{t\in\Gamma}f(t) \right)^{n} &= \sum_{t\in\Gamma}\left( f\ast\cdots\ast f \right)\left( t \right).
  \end{align*}
  Similarly,
  \begin{align*}
    \left( \lambda_{f(t)} \right)^n &= \left( \sum_{t\in\Gamma}f(t)\lambda_t \right)^{n}\\
                                    &= \sum_{t\in\Gamma}\left( f\ast\cdots\ast f \right)\left( t \right)\lambda_t\\
                                    &= \lambda_{(f\ast\cdots\ast f)(t)}.
  \end{align*}
  Now, by the definition of the operator norm, we have
  \begin{align*}
    \norm{\lambda_{(f\ast\cdots\ast f)(t)}}_{\op} &= \norm{\left( \lambda_{f(t)} \right)^n}_{\op}\\
                                                    &\leq \norm{\lambda_{f(t)}}_{\op}^{n}.
  \end{align*}
  Thus, we have
  \begin{align*}
    \left( \sum_{t\in\Gamma}f(t) \right)^n &= \sum_{t\in\Gamma}\left( f\ast\cdots\ast f \right)\left( t \right)\\
                                           &\leq C\norm{\lambda_{\left( f\ast\cdots\ast f \right)\left( t \right)}}_{\op}\\
                                           &\leq C\left( \norm{\lambda_{f(t)}}_{\op}^n \right),
  \end{align*}
  giving
  \begin{align*}
    \sum_{t\in\Gamma}f(t) &\leq C^{1/n}\norm{\lambda_{f(t)}}_{\op}.
  \end{align*}
  Since $n$ was arbitrary, we have that $C = 1$.\newline

  Now, suppose that for all finitely supported $f$, we have
  \begin{align*}
    \sum_{t\in\Gamma}f(t) &\leq \norm{\lambda_{f(t)}}_{\op}.
  \end{align*}
  If $f = \1_{E}$ for some finite $E\subseteq \Gamma$, we see that
  \begin{align*}
    \norm{\lambda_{f(t)}}_{\op} &= \norm{\sum_{t\in\Gamma}f(t)\lambda_t}_{\op}\\
                                &= \norm{\sum_{t\in E}\lambda_t}_{\op}\\
                                &= \left\vert E \right\vert.
  \end{align*}
  Therefore, we have
  \begin{align*}
    \norm{\frac{1}{\left\vert E \right\vert}\sum_{t\in\Gamma}\lambda_t}_{\op} &= 1.
  \end{align*}
  By Kesten's criterion (Theorem \ref{thm:kesten_criterion}), we have that $\Gamma$ is amenable.
\end{proof}

