\epigraph{The whole is greater than the sum of its parts.}{Aristotle, who had yet to learn about amenability in groups.}
In Chapter 4, we introduced amenability through Tarski's Theorem (Theorem \ref{thm:tarski}), where we proved the existence of a translation-invariant finitely additive set function, $\mu\colon P(X)\rightarrow [0,\infty]$, and the non-paradoxicality of $G$'s action on $X$. There, we used the type semigroup of $G$'s action on $X$ in order to prove the theorem. Now, we turn our attention towards other constructions from analysis --- as well as the intrinsic properties of the group itself --- to understand and prove other criteria for $G$'s amenability.\newline

In this section, we will use techniques from functional analysis to prove the equivalence between amenability and the existence of an invariant state $\mu\colon \ell_{\infty}(G) \rightarrow [0,1]$.
\section{Amenability in Subgroups and Quotient Groups}\label{sec:amenability_subgroups_quotients}%
We begin by defining a mean on $G$ --- note that this definition is slightly different from the one used in the proof in Theorem \ref{thm:tarski}. However, one can show that they are equivalent by letting $G$ act on itself by left-multiplication and taking $E = G$.
\begin{definition}
  Let $G$ be a group, with $P(G)$ denoting its power set.\newline

  An invariant mean on $G$ is a set function $m\colon P(G)\rightarrow [0,1]$ which satisfies, for all $t\in G$ and $E,F\subseteq G$,
  \begin{itemize}
    \item $m(G) = 1$;
    \item $m\left(E\sqcup F\right) = m(E) + m(F)$;
    \item $m\left(tE\right) = m\left(E\right)$.
  \end{itemize}
  We say $G$ is amenable if $G$ admits a mean.\newline

  The mean $m$ is a translation-invariant probability measure on the measurable space $\left(G,P(G)\right)$.
\end{definition}
We can establish some inheritance properties using the properties of a mean. In Proposition \ref{prop:subgroups_quotientgroups_amenability}, we will show that subgroups of amenable groups are amenable and quotients of amenable groups are amenable.
\begin{proposition}\label{prop:subgroups_quotientgroups_amenability}
  Let $G$ be an amenable group with $H\leq G$. Then, the following are true:
  \begin{enumerate}[(1)]
    \item $H$ is amenable;
    \item for $H\trianglelefteq G$, $G/H$ is amenable.
  \end{enumerate}
\end{proposition}
\begin{proof}\hfill
  \begin{enumerate}[(1)]
    \item Let $R$ be a right transversal for $H$, wherein we select one element of each right coset of $H$ to make up $R$.\newline

      If $m$ is a mean for $G$, we set $\lambda\colon P(H)\rightarrow [0,1]$ defined by
      \begin{align*}
        \lambda(E) = m\left(ER\right).
      \end{align*}
       We have
      \begin{align*}
        \lambda(H) &= m\left(HR\right)\\
                   &= m\left(G\right)\\
                   &= 1.
      \end{align*}
      We claim that if $E\cap F = \emptyset$, then $ER \cap FR = \emptyset$. Suppose toward contradiction this is not the case. Then, $xr_1 = yr_2$ for some $x\in E$, $y\in F$, and $r_1,r_2\in R$. Then, we must have $r_2r_1^{-1} = y^{-1}x \in H$, meaning $r_1 = r_2$ as, by definition, $R$ contains exactly one element of each right coset. Thus, $x=y$, so $E\cap F \neq \emptyset$.\newline

      We then have
      \begin{align*}
        \lambda\left(E\sqcup F\right) &= m\left(\left(E\sqcup F\right)R\right)\\
                                      &= m\left(ER\sqcup FR\right)\\
                                      &= m\left(ER\right) + m\left(FR\right)\\
                                      &= \lambda\left(E\right) + \lambda\left(F\right),
      \end{align*}
      and
      \begin{align*}
        \lambda\left(sE\right) &= m\left(sER\right)\\
                               &= m\left(ER\right)\\
                               &= \lambda\left(E\right).
      \end{align*}
    \item Let $\pi\colon G\rightarrow G/H$ be the canonical projection, defined by $\pi\left(t\right) = tH$. We define
      \begin{align*}
        \lambda\colon P\left(G/H\right) \rightarrow [0,1]
      \end{align*}
      by $\lambda(E) = m\left(\pi^{-1}\left(E\right)\right)$. We have
      \begin{align*}
        \lambda\left(G/H\right) &= m\left(\pi^{-1}\left(G/H\right)\right)\\
                                &= m\left(G\right)\\
                                &= 1,
      \end{align*}
      and
      \begin{align*}
        \lambda\left(E\sqcup F\right) &= m\left(\pi^{-1}\left(E\sqcup F\right)\right)\\
                                      &= m\left(\pi^{-1}\left(E\right)\sqcup \pi^{-1}\left(F\right)\right)\\
                                      &= m\left(\pi^{-1}\left(E\right)\right) + m\left(\pi^{-1}\left(F\right)\right)\\
                                      &= \lambda(E) + \lambda(F).
      \end{align*}
      To show translation-invariance, we let $sH = \pi(s)\in G/H$, and $E\subseteq G/H$. Note that
      \begin{align*}
        \pi^{-1}\left(\pi(s)E\right) &= s\pi^{-1}\left(E\right),
      \end{align*}
      since for $r\in s\pi^{-1}(E)$, we have $r = st$ for $t\in \pi(E)$, so $\pi\left(r\right) =\pi\left(st\right) = \pi\left(s\right)\pi\left(t\right)\in \pi\left(s\right)E$.\newline

      Additionally, if $r\in \pi^{-1}\left(\pi(s)E\right)$, we have $\pi(r)\in \pi(s)E$, so $\pi\left(s^{-1}r\right)\in E$, meaning $s^{-1}r\in \pi^{-1}(E)$.\newline

      Thus,
      \begin{align*}
        \lambda\left(\pi\left(s\right)E\right) &= m\left(\pi^{-1}\left(\pi\left(s\right)E\right)\right)\\
                                               &= m\left(s\pi^{-1}\left(E\right)\right)\\
                                               &= m\left(\pi^{-1}\left(E\right)\right)\\
                                               &= \lambda\left(E\right).
      \end{align*}
  \end{enumerate}
\end{proof}
\section{Establishing Amenability through Functional Analysis}\label{sec:functional_analysis_and_amenability}%
Now that we understand some useful properties of means in relation to groups and subgroups, we turn our attention toward finding means on groups. In order to do this, we turn our attention towards the space $\ell_{\infty}\left(G\right)$, which allows us to use theories from functional analysis to better understand means on $G$. For more elaboration on these ideas, we encourage the reader to review the results in Chapters \ref{ch:measure_theory} and \ref{ch:functional_analysis}.
\begin{definition}
  Let $G$ be a group.
  \begin{enumerate}[(1)]
    \item The space $\mathcal{F}\left(G\right)$ is defined by
      \begin{align*}
        \mathcal{F}\left(G\right) &= \set{f | f\colon G\rightarrow \C\text{ is a function}}.
      \end{align*}
    \item A function $f\in \mathcal{F}\left(G\right)$ is called positive if $f(x) \geq 0$ for all $x\in G$.
    \item A function $f\in \mathcal{F}\left(G\right)$ is called simple if $\ran(f)$ is finite. We let
      \begin{align*}
        \Sigma &= \set{f\in \mathcal{F}\left(G\right) | f\text{ is simple}}.
      \end{align*}
  \end{enumerate}
\end{definition}
\begin{fact}
  It is the case that $\Sigma \subseteq \mathcal{F}\left(G\right)$ is a linear subspace.
\end{fact}
\begin{definition}
  For $E\subseteq G$, we define
  \begin{align*}
    \1_{E}\colon G\rightarrow \C
  \end{align*}
  by
  \begin{align*}
    \1_{E}\left(x\right) &= \begin{cases}
      1 & x\in E\\
      0 & x\notin E
    \end{cases}.
  \end{align*}
  This is the characteristic function of $E$.
\end{definition}
\begin{fact}
  We have
  \begin{align*}
    \Span\set{\1_{E}| E\subseteq G} &= \Sigma.
  \end{align*}
\end{fact}
\begin{proof}
  We see that $\1_{E}\in \Sigma$ for any $E\subseteq G$, and that $\Sigma$ is a subspace.\newline

  If $\phi\in \Sigma$ with $\Ran\left(\phi\right) = \set{t_1,\dots,t_n}$, where $t_i$ are distinct, we set
  \begin{align*}
    E_i &= \phi^{-1}\left(\set{t_i}\right),
  \end{align*}
  yielding
  \begin{align*}
    \phi &= \sum_{i=1}^{n}t_i\1_{E_i}.
  \end{align*}
\end{proof}
\begin{definition}\hfill
  \begin{enumerate}[(1)]
    \item A function $f\in \mathcal{F}\left(G\right)$ is bounded if there exists $M > 0$ such that $\left\vert f(g) \right\vert \leq M$ for all $g\in G$.
    \item The space $\ell_{\infty}\left(G\right)$ is defined by
      \begin{align*}
        \ell_{\infty}\left(G\right) &= \set{f\in \mathcal{F}\left(G\right)| f\text{ is bounded}}.
      \end{align*}
    \item The norm on $\ell_{\infty}\left(G\right)$ is defined by
      \begin{align*}
        \norm{f}_{\ell_{\infty}} &= \sup_{x\in G}\left\vert f(x) \right\vert.
      \end{align*}
  \end{enumerate}
\end{definition}
\begin{proposition}
  The space $\ell_{\infty}(G)$ is complete. Additionally, $\overline{\Sigma} = \ell_{\infty}\left(G\right)$.
\end{proposition}
\begin{proof}
  Let $\left(f_n\right)_n$ be $\norm{\cdot}$-Cauchy in $\ell_{\infty}\left(G\right)$. Then, for all $x\in G$, it is the case that
  \begin{align*}
    \left\vert f_n(x) - f_m(x) \right\vert &= \left\vert \left(f_n - f_m\right)\left(x\right) \right\vert\\
                                           &\leq \norm{f_n - f_m}_{\ell_{\infty}},
  \end{align*}
  meaning $\left(f_n\left(x\right)\right)_n$ is Cauchy in $\C$. We define $f(x) = \lim_{n\rightarrow\infty}f_n(x)$. We must show that $f\in \ell_{\infty}\left(G\right)$, and $\norm{f_n-f}_{\ell_{\infty}}\rightarrow 0$.\newline

  We have
  \begin{align*}
    \left\vert f(x) \right\vert &= \left\vert \lim_{n\rightarrow\infty}f_n\left(x\right) \right\vert\\
                                &= \lim_{n\rightarrow\infty}\left\vert f_n\left(x\right) \right\vert\\
                                &\leq \limsup_{n\rightarrow\infty}\norm{f_n}_{\ell_{\infty}}\\
                                &\leq C,
  \end{align*}
  as Cauchy sequences are always bounded. Thus, $\sup_{x\in G}\left\vert f(x) \right\vert\leq C$.\newline

  Given $\ve > 0$, we find $N$ such that for all $m,n\geq N$, $\norm{f_n - f_m}_{\ell_{\infty}} \leq \ve$. Thus, for $x\in G$, we have
  \begin{align*}
    \left\vert f_n(x) - f_m(x) \right\vert &\leq \norm{f_n - f_m}_{\ell_{\infty}}\\
                                           &\leq \ve.
  \end{align*}
  Taking $m\rightarrow\infty$, we get $\left\vert f_n(x) - f(x) \right\vert \leq \ve$, for all $n\geq N$, so $\norm{f_n - f}_{\ell_{\infty}}\leq \ve$ for all $n\geq N$.\newline

  For real-valued $f\in \ell_{\infty}\left(G\right)$, let $\left\vert f \right\vert \subseteq \left[-M,M\right]$ for some $M > 0$. Let $\ve > 0$. Since $\left[-M,M\right]$ is compact, it is totally bounded, so we can find intervals $I_{1},\dots,I_n$ with $\left[-M,M\right] = \bigsqcup_{k=1}^{n}I_k$, with the length of each $I_k$ less than $\ve$.\newline

  Set $E_k = f^{-1}\left(I_k\right)$. Pick some $t_k\in I_k$. We set
  \begin{align*}
    \phi &= \sum_{i=1}^{n}t_k\1_{E_k}.
  \end{align*}
  Then, it is the case that $\norm{\phi - f}_{\ell_{\infty}} < \ve$.\newline

  If $f\in \ell_{\infty}(G)$ is complex-valued, we apply this process separately to $\re\left(f\right)$ and $\im\left(f\right)$.
\end{proof}
\begin{corollary}
  For any $f\in \ell_{\infty}\left(G\right)$, there is a sequence $\left(\phi_n\right)_n$ of simple functions with $\norm{\phi_n -f}_{\ell_{\infty}}\rightarrow 0$. If $f\geq 0$, then we can select $\phi_n\geq 0$.
\end{corollary}
Now that we understand how simple functions relate to $\ell_{\infty}(G)$, we start by defining a translation action on $\ell_{\infty}(G)$, from which we will be able to convert the idea of means into invariant elements of the state space of the dual of $\ell_{\infty}\left(G\right)$.
\begin{proposition}\label{prop:translation_action}
  Let $G$ be a group. There is an action
  \begin{align*}
    \lambda\colon G\rightarrow \Isom\left(\ell_{\infty}\left(G\right)\right),
  \end{align*}
  where $\lambda(s) = \lambda_s$, defined by
  \begin{align*}
    \lambda_{s}\left(f\right)\left(t\right) &= f\left(s^{-1}t\right)
  \end{align*}
\end{proposition}
\begin{proof}
  We have
  \begin{align*}
    \lambda_s\left(f + \alpha g\right)\left(t\right) &= \left(f + \alpha g\right) \left(s^{-1}t\right)\\
                                                     &= f\left(s^{-1}t\right) \alpha g\left(s^{-1}t\right)\\
                                                     &= \lambda_s\left(f\right)\left(t\right) + \alpha \lambda_s\left(g\right)\left(t\right)\\
                                                     &= \left(\lambda_s\left(f\right) + \alpha \lambda_s\left(g\right)\right)(t).
  \end{align*}
  Thus, $\lambda_s$ is linear. Additionally,
  \begin{align*}
    \norm{\lambda_s\left(f\right)}_{\ell_{\infty}} &= \sup_{t\in G}\left\vert \lambda_s\left(f\right)\left(t\right) \right\vert\\
                                   &= \sup_{t\in G}\left\vert f\left(s^{-1}t\right) \right\vert\\
                                   &= \norm{f}_{\ell_{\infty}},
  \end{align*}
  and
  \begin{align*}
    \norm{\lambda_s\left(f\right) - \lambda_s\left(g\right)}_{\ell_{\infty}} &= \norm{\lambda_s\left(f-g\right)}\\
                                                                             &= \norm{f-g}_{\ell_{\infty}},
  \end{align*}
  meaning $\lambda_s$ is an isometry.\newline

  We have
  \begin{align*}
    \lambda_s\circ \lambda_r\left(f\right)\left(t\right) &= \lambda_r\left(f\right)\left(s^{-1}t\right)\\
                                                         &= \lambda_r\left(r^{-1}s^{-1}t\right)\\
                                                         &= f\left(\left(sr\right)^{-1}t\right)\\
                                                         &= \lambda_{sr}\left(f\right)\left(t\right),
  \end{align*}
  establishing that $\lambda_s\circ \lambda_r = \lambda_{sr}$.\newline

  By a similar process, we find that $\lambda_{s}\left(\1_{E}\right) = \1_{sE}$ for any $E\subseteq G$ and $s\in G$.
\end{proof}
\begin{definition}
  A {state} on $\ell_{\infty}\left(G\right)$ is a continuous linear functional $\mu\in \ell_{\infty}\left(G\right)^{\ast}$ such that the following are true:
  \begin{itemize}
    \item $\mu$ is positive;
    \item $\mu\left(\1_{G}\right) = 1$.
  \end{itemize}
  A state is called left-invariant if
  \begin{align*}
    \mu\left(\lambda_s\left(f\right)\right) = \mu\left(f\right).
  \end{align*}
\end{definition}
\begin{example}\label{ex:finite_invariant_state}
  The evaluation functional, $\delta_x\colon \ell_{\infty}\rightarrow \R$, defined by
  \begin{align*}
    \delta_{x}\left(f\right) &= f(x),
  \end{align*}
  is a state. However, it is not necessarily invariant, as
  \begin{align*}
    \delta_x\left(\lambda_s\left(f\right)\right) &= \lambda_s\left(f\right)\left(x\right)\\
                                                 &= f\left(s^{-1}x\right)\\
                                                 &\neq f(x).
  \end{align*}
  However, we can use the evaluation functional to create an invariant state. If $G$ is finite, we define
  \begin{align*}
    \mu &= \frac{1}{\left\vert G \right\vert} \sum_{x\in G}\delta_x,
  \end{align*}
  which is indeed an invariant state.
\end{example}
We can characterize states slightly differently, which will enable us to show the equivalence between invariant states and means.
\begin{lemma}\label{lemma:characterizing_states}\hfill
  \begin{enumerate}[(1)]
    \item If $\mu$ is a state on $\ell_{\infty}\left(G\right)$, then
      \begin{align*}
        \norm{\mu}_{\op} = 1.
      \end{align*}
    \item If $\mu\in \ell_{\infty}\left(G\right)^{\ast}$ is such that
      \begin{align*}
        \norm{\mu}_{\op} &= \mu\left(\1_{G}\right)\\
                               &= 1,
      \end{align*}
      then $\mu$ is positive and a state.
  \end{enumerate}
\end{lemma}
\begin{proof}\hfill
  \begin{enumerate}[(1)]
    \item Let $\mu$ be a state. Given $f\in \ell_{\infty}\left(G\right)$, we have
      \begin{align*}
        \norm{f}_{\ell_{\infty}}\1_{G} - f &\geq 0\\
        \norm{f}_{\ell_{\infty}}\1_{G} + f &\geq 0,
      \end{align*}
      so
      \begin{align*}
        0 &\leq \mu\left(\norm{f}_{\ell_{\infty}}\1_{G} - f\right) \\
          &= \norm{f}_{\ell_{\infty}}\mu\left(\1_{G}\right) - \mu\left(f\right)
          \intertext{meaning}
        \mu\left(f\right) &\leq \norm{f}_{\ell_{\infty}}.
        \intertext{Additionally,}
        0 &\leq \mu\left(\norm{f}_{\ell_{\infty}}\1_{G} + f\right)\\
          &= \norm{f}_{\ell_{\infty}}\mu\left(\1_{G}\right) + \mu\left(f\right),
          \intertext{meaning}
        -\mu\left(f\right) &\leq \norm{f}_{\ell_{\infty}}.
      \end{align*}
      Thus, we have $\left\vert \mu\left(f\right) \right\vert \leq \norm{f}_{\ell_{\infty}}$, so $\norm{\mu}_{\op} \leq 1$. However, since $\mu\left(\1_{G}\right) = 1$, we must have $\norm{\mu}_{\op} = 1$.
    \item Suppose $\norm{\mu}_{\op} = \mu\left(\1_{G}\right) = 1$. Let $f\geq 0$. Set $g = \frac{1}{\norm{f}_{\ell_{\infty}}}f$.\newline

      Then, $\Ran(g) \subseteq [0,1]$, and $\Ran\left(g - \1_{G}\right) \subseteq \left[-1,1\right]$. Thus, $\norm{g - \1_{G}}_{\ell_{\infty}} \leq 1$.\newline

    Since $\norm{\mu}_{\op} = 1$, we must have
    \begin{align*}
      \left\vert \mu\left(g - \1_{G}\right) \right\vert &\leq 1\\
      \left\vert \mu\left(g\right) - 1 \right\vert &\leq 1,
    \end{align*}
    and since $\mu\left(\1_{G}\right) = 1$, we have $\mu\left(g\right) \in [0,2]$. Thus, $\mu\left(f\right) = \norm{f}_{\ell_{\infty}}\mu\left(g\right) \geq 0$.
  \end{enumerate}
\end{proof}
%To show the equivalence between means and invariant states, we need to be able to characterize the state space on $\ell_{\infty}\left(G\right)^{\ast}$. To do this, we make use of some results from functional analysis.\newline
%
%If $X$ is a normed vector space, then the topology on $X^{\ast}$ induced by $X^{\ast\ast}$ is known as the weak* topology. The weak* topology is the topology of pointwise convergence in $X^{\ast}$ --- a net $\left(\varphi_{\alpha}\right)_{\alpha}$ converges to $\varphi$ in the weak* topology if and only if, for all $\hat{x}\in X^{\ast\ast}$, we have
%\begin{align*}
%  \left(\hat{x}\left(\varphi_{\alpha}\right)\right)_{\alpha}\rightarrow \hat{x}\left(\varphi\right),
%\end{align*}
%or by the definition of $X^{\ast\ast}$,
%\begin{align*}
%  \left(\varphi_{\alpha}\left(x\right)\right) \rightarrow \varphi\left(x\right)
%\end{align*}
%for all $x\in X$.\newline
%
%We state some important results in functional analysis here. The proofs of these results can be found in functional analysis textbooks such as \cite{rudin_functional_analysis}.
%\begin{theorem}[Hahn--Banach Continuous Extension Theorem]
%  Let $X$ be a normed vector space, $E\subseteq X$ a subspace, and $\varphi\in E^{\ast}$ a bounded linear functional. Then, there exists a continuous $\psi\in X^{\ast}$ such that $\norm{\varphi}_{\op} = \norm{\psi}_{\op}$, and $\psi|_{E} = \varphi$.
%\end{theorem}
%\begin{theorem}[Hahn--Banach Separation Theorems]
%  Let $X$ be a normed vector space.
%  \begin{enumerate}[(1)]
%    \item Given a nonzero $x_0\in X$, there is a $\varphi\in X^{\ast}$ with $\norm{\varphi}_{\op} = 1$ and $\varphi\left(x_0\right) = \norm{x}$. We call $\varphi$ a norming functional.
%    \item Given a proper closed subspace $E\subseteq X$ and $x_0\in X\setminus E$, there is a $\varphi\in X^{\ast}$ such that $\varphi|_{E} = 0$, $\norm{\varphi}_{\op} = 1$, and $\varphi\left(x\right) = \dist_{E}(x)$ for all $x\in X$.
%  \end{enumerate}
%\end{theorem}
%\begin{theorem}[Banach--Alaoglu Theorem]
%  Let $X$ be a normed vector space.
%  \begin{enumerate}[(1)]
%    \item The closed unit ball in the dual space, $B_{X^{\ast}}$, is compact in the $w^{\ast}$ topology.
%    \item A subset $C\subseteq X$ is $w^{\ast}$-compact if and only if $C$ is $w^{\ast}$-closed and norm bounded.
%  \end{enumerate}
%\end{theorem}
\begin{corollary}
  The set of states in $\ell_{\infty}\left(G\right)^{\ast}$ forms a $w^{\ast}$-compact subset of $B_{\ell_{\infty}\left(G\right)^{\ast}}$.
\end{corollary}
\begin{proof}
  From the Banach--Alaoglu Theorem (Theorem \ref{thm:banach_alaoglu}), we only need to show that the set of states, $S\left(\ell_{\infty}\left(G\right)\right)$, is $w^{\ast}$-closed, as every element of $S\left(\ell_{\infty}\left(G\right)\right)$ has norm $1$.\newline

  Let $f\in \ell_{\infty}\left(G\right)$ be positive, and let $\left(\varphi_{i}\right)_i$ be a net in $S\left(\ell_{\infty}\left(G\right)\right)$ with $\left(\varphi_{i}\right)_i\xrightarrow{w^{\ast}} \varphi\in \ell_{\infty}\left(G\right)^{\ast}$. From Lemma \ref{lemma:characterizing_states}, we must show that $\varphi$ is positive and $\varphi\left(\1_{G}\right) = 1$.\newline

  We start by seeing that, since each $\varphi_i$ is a state, we have $\varphi_{i}\left(f\right) \geq 0$ for each $i\in I$, so we must have $\varphi\left(f\right) \geq 0$.\newline

  Similarly, since $\varphi_{i}\left(\1_{G}\right) = 1$ for each $i\in I$, and $\left(\varphi_i\right)_i \xrightarrow{w^{\ast}} \varphi$, we have $\varphi\left(\1_{G}\right) = 1$. Thus, by Lemma \ref{lemma:characterizing_states}, we have that $S\left(\ell_{\infty}\left(G\right)\right)$ is $w^{\ast}$-closed.
\end{proof}

Now, we may show the correspondence between invariant states and means.
\begin{proposition}\label{prop:state_implies_mean}
  If $\mu\in \ell_{\infty}\left(G\right)^{\ast}$ is a state, then $m\colon P(G)\rightarrow [0,1]$ defined by $m(E) = \mu\left(\1_{E}\right)$ is a finitely additive probability measure on $G$.\newline

  Moreover, if $\mu$ is invariant, then $m$ is a mean.
\end{proposition}
\begin{proof}
  We have
  \begin{align*}
    m\left(G\right) &= \mu\left(\1_{G}\right)\\
                    &= 1\\
                    \\
    m\left(\emptyset\right) &= \mu\left(0\right)\\
                            &= 0\\
                            \\
    m\left(E\sqcup F\right) &= \mu\left(\1_{E\sqcup F}\right)\\
                            &= \mu\left(\1_{E} + \1_{F}\right)\\
                            &= \mu\left(\1_{E}\right) + \mu\left(\1_{F}\right)\\
                            &= m\left(E\right) + m\left(F\right).
  \end{align*}
  Additionally, since $0 \leq \1_{E}\leq \1_{G}$, we have $0 \leq \mu\left(\1_{E}\right) \leq 1$, so $0 \leq m(E) \leq 1$.\newline

  If $\mu$ is invariant, then
  \begin{align*}
    m\left(sE\right) &= \mu\left(\1_{sE}\right)\\
                     &= \mu\left(\lambda_s\left(\1_{E}\right)\right)\\
                     &= \mu\left(\1_{E}\right)\\
                     &= m\left(E\right).
  \end{align*}
\end{proof}
\begin{proposition}\label{prop:mean_implies_state}
  If $G$ admits a mean, then $\ell_{\infty}\left(G\right)^{\ast}$ admits an invariant state.
\end{proposition}
\begin{proof}
  Let $m$ be a mean. Define $\mu_0\colon \Sigma\rightarrow \R$ by
  \begin{align*}
    \mu_0\left(\sum_{k=1}^{n}t_k\1_{E_k}\right) &= \sum_{k=1}^{n}t_km\left(E_k\right).
  \end{align*}
  Since $m$ is finitely additive, it is the case that $\mu_0$ is well-defined, linear, and positive, with $\mu_0\left(\1_{G}\right) = m\left(G\right) = 1$.\newline

  Additionally, since $m$ is a mean, then for $f = \sum_{k=1}^{n}t_k\1_{E_k}$, we have
  \begin{align*}
    \mu_0\left(\lambda_s\left(f\right)\right) &= \mu_0\left(\lambda_s\left(\sum_{k=1}^{n}t_k\1_{E_k}\right)\right)\\
                                              &= \mu_0\left(\sum_{k=1}^{n}t_k\1_{sE_k}\right)\\
                                              &= \sum_{k=1}^{n}t_km\left(sE_k\right)\\
                                              &= \sum_{k=1}^{n}t_km\left(E_k\right)\\
                                              &= \mu_0\left(f\right).
  \end{align*}
  We see that
  \begin{align*}
    \left\vert \mu_0\left(f\right) \right\vert &= \left\vert \sum_{k=1}^{n}t_km\left(E_k\right) \right\vert\\
                                               &\leq \sum_{k=1}^{n}\left\vert t_k \right\vert m\left(E_k\right)\\
                                               &\leq \sum_{k=1}^{n}\norm{f}_{\ell_{\infty}}\sum_{k=1}^{n}m\left(E_k\right)\\
                                               &= \norm{f}_{\ell_{\infty}}\sum_{k=1}^{n}m\left(E_k\right)\\
                                               &\leq \norm{f}_{\ell_{\infty}},
  \end{align*}
  meaning $\mu_0$ is continuous, so $\mu_0$ is uniformly continuous.\newline

  Since $\overline{\Sigma} = \ell_{\infty}\left(G\right)$, uniform continuity provides that $\mu_0$ extends to a continuous linear functional $\mu\colon \ell_{\infty}\left(G\right)\rightarrow \R$ with $\mu\left(\1_{G}\right) = \mu_0\left(\1_{G}\right) = 1$.\newline

  For $f\geq 0$, we find a sequence $\left(\phi_n\right)_n$ in $\Sigma$ with $\phi_n\geq 0$ and $\norm{\phi_n - f}_{\ell_{\infty}} \xrightarrow{n\rightarrow\infty}0$. We set
  \begin{align*}
    \mu\left(f\right) &= \lim_{n\rightarrow\infty}\mu\left(\phi_n\right)\\
                      &= \lim_{n\rightarrow\infty}\mu_0\left(\phi_n\right)\\
                      &\geq 0,
  \end{align*}
  so $\mu$ is a state.\newline

  If $f\in \ell_{\infty}\left(G\right)$, $s\in G$, and $\left(\phi_n\right)_n$ a sequence in $\Sigma$ with $\left(\phi_n\right)_n\rightarrow f$, then
  \begin{align*}
    \norm{\lambda_s\left(\phi_n\right) - \lambda_s\left(f\right)}_{\ell_{\infty}} &= \norm{\lambda_s\left(\phi_n - f\right)}_{\ell_{\infty}}\\
                                                                                  &= \norm{\phi_n - f}_{\ell_{\infty}}\\
                                                                  &\rightarrow 0.
  \end{align*}
  Thus, we have
  \begin{align*}
    \mu\left(\lambda_s\left(\phi_n\right)\right) &= \mu_0\left(\lambda_s\left(\phi_n\right)\right)\\
                                                 &= \mu_0\left(\phi_n\right)\\
                                                 &= \mu\left(\phi_n\right)\\
                                                 &\rightarrow \mu\left(f\right),
  \end{align*}
  so $\mu\left(f\right) = \mu\left(\lambda_s\left(f\right)\right)$. Thus, $\mu\in \ell_{\infty}\left(G\right)^{\ast}$ is an invariant state.
\end{proof}
\section{Establishing Amenability using Invariant States}\label{sec:amenability_invariant_states}%
Owing to the correspondence between invariant states and means, we are now able to establish amenability for large classes of groups.
\begin{proposition}
  The group of integers, $\Z$, is amenable.
\end{proposition}
\begin{proof}
  We define the left shift, $\lambda_1\colon \ell_{\infty}\left(\Z\right) \rightarrow \ell_{\infty}\left(\Z\right)$, by
  \begin{align*}
    \lambda_1\left(f\right)\left(k\right) &= f\left(k-1\right).
  \end{align*}
  This is an action as in Proposition \ref{prop:translation_action}. \newline

  We set $Y = \Ran\left(\id - \lambda_1\right)\subseteq \ell_{\infty}\left(\Z\right)$. We claim that $\dist_{Y}\left(\1_{\Z}\right) \geq 1$.\newline

  Suppose toward contradiction that there is $y\in Y$ with $\norm{\1_{\Z} - y}_{\ell_{\infty}} = r < 1$. Then, $y = f - \lambda_1 f$ for some $f\in \ell_{\infty}(\Z)$, so
  \begin{align*}
    \norm{\1_{\Z} - \left(f - \lambda_1\left(f\right)\right)}_{\ell_{\infty}} &= r.
  \end{align*}
  Thus, for all $k\in\Z$, we have
  \begin{align*}
    \left\vert 1 - \left(f(k) - f(k-1)\right) \right\vert &\leq r,
  \end{align*}
  so $\left\vert f(k) - f\left(k-1\right) \right\vert \geq 1-r > 0$. However, such an $f$ cannot be bounded.\newline

  Since $\dist_{\overline{Y}}\left(\1_{\Z}\right) = \dist_{Y}\left(\1_{\Z}\right)$, the Hahn--Banach separation theorems provide $\mu\in \left(\ell_{\infty}\left(\Z\right)\right)^{\ast}$ with $\norm{\mu}_{\op} = 1$, $\mu|_{\overline{Y}} = 0$, and $\mu\left(\1_{\Z}\right) = \dist_{Y}\left(\1_{\Z}\right) \geq 1$.\newline

  Since $\norm{\mu}_{\op} = 1$ and $\mu\left(\1_{\Z}\right) \geq 1$, we must have $\mu\left(\1_{\Z}\right) = 1$.\newline

  Additionally, since $\norm{\mu}_{\op} = \mu\left(\1_{\Z}\right) = 1$, we have that $\mu$ is a state on $\ell_{\infty}\left(\Z\right)$, and since $\mu\left(y\right) = 0$ for all $y\in Y$, we have
  \begin{align*}
    \mu\left(f - \lambda_1\left(f\right)\right) &= 0\\
    \mu\left(f\right) &= \mu\left(\lambda_1\left(f\right)\right).
  \end{align*}
  Inductively, this means that $\mu\left(f\right) = \mu\left(\lambda_k\left(f\right)\right)$ for all $k\in \Z$, so $\mu$ is an invariant state on $\ell_{\infty}\left(\Z\right)$. Thus, $\Z$ is amenable.
\end{proof}
\begin{proposition}\label{prop:normal_subgroups_quotient_groups_amenability}
  If $N\trianglelefteq G$ and $G/N$ are amenable, then $G$ is amenable.
\end{proposition}
\begin{proof}
  Let $\rho\in \left(\ell_{\infty}\left(G/N\right)\right)^{\ast}$ be an invariant state, and let $p\colon P(N)\rightarrow [0,1]$ be a mean. For $E\subseteq G$, we define $f_E\colon G/N\rightarrow \R$ by
  \begin{align*}
    f_E\left(tN\right) &= p\left(N\cap t^{-1}E\right).
  \end{align*}
  We start by verifying that $f_E$ is well-defined. For $tN = sN$, we have $s^{-1}t\in N$, so
  \begin{align*}
    p\left(N\cap t^{-1}E\right) &= p\left(s^{-1}t\left(N\cap t^{-1}E\right)\right)\\
                                &= p\left(s^{-1}tN \cap s^{-1}E\right)\\
                                &= p\left(N\cap s^{-1}E\right).
  \end{align*}
  Since $f_E$ is defined through $p$, we can see that $f_E$ is bounded. Additionally,
  \begin{align*}
    f_{E\sqcup F}\left(tN\right) &= p\left(N\cap t^{-1}\left(E\sqcup F\right)\right)\\
                                 &= p\left(N\cap \left(t^{-1}E\sqcup t^{-1}F\right)\right)\\
                                 &= p\left(\left(N\cap t^{-1}E\right) \sqcup \left(N\cap t^{-1}F\right)\right)\\
                                 &= p\left(N\cap t^{-1}E\right) + p\left(N\cap t^{-1}F\right)\\
                                 &= f_E\left(tN\right) + f_F\left(tN\right)\\
                                 &= \left(f_E + f_F\right)\left(tN\right),
  \end{align*}
  and
  \begin{align*}
    f\left(sE\right) \left(tN\right) &= p\left(N\cap t^{-1}sE\right)\\
                                     &= f_E\left(s^{-1}tN\right)\\
                                     &= \lambda_{sN}\left(f_E\right)\left(tN\right),
  \end{align*}
  so $f_{sE} = \lambda_{sN}\left(f_E\right)$. Finally,
  \begin{align*}
    f_G\left(tN\right) &= p\left(N\cap t^{-1}G\right)\\
                       &=p\left(N\right)\\
                       &= 1,
  \end{align*}
  meaning $f_G = \1_{G/N}$.\newline

  We define $m\colon P(G)\rightarrow [0,1]$ by
  \begin{align*}
    m(E) &= \rho\left(f_E\right).
  \end{align*}
  Then, we have
  \begin{align*}
    m\left(E\sqcup F\right) &= m(E) + m(F)\\
                            \\
    m\left(G\right) &= 1\\
    \\
    m\left(sE\right) &= \rho\left(f_{sE}\right)\\
                     &= \rho\left(\lambda_{sN}\left(f_{E}\right)\right)\\
                     &= \rho\left(f_E\right)\\
                     &= m(E),
  \end{align*}
  so $m$ is a mean.
\end{proof}
\begin{corollary}
  The finite direct product of amenable groups is amenable.
\end{corollary}
\begin{proof}
  If $H$ and $K$ are amenable, then $K\cong \left(H\times K\right)/H$ is amenable and $H$ is amenable, so $H\times K$ is amenable by Proposition \ref{prop:normal_subgroups_quotient_groups_amenability}. Induction provides the general case.
\end{proof}
\begin{corollary}\label{cor:finitely_generated_amenable}
  Finitely generated abelian groups are amenable.
\end{corollary}
\begin{proof}
  By the fundamental theorem of finitely generated abelian groups (Theorem \ref{thm:fundamental_thm_abelian_gps}), all finitely generated abelian groups are isomorphic to $\Z^{d}\times \Z/n_1\Z\times\cdots\times \Z/{n_k}\Z$.\newline

  Since $\Z^{d}$ is a finite direct product of $\Z$, and the torsion subgroup $\Z/n_1\Z\times\cdots\times \Z/n_k\Z$ is finite (hence amenable by \ref{ex:finite_invariant_state}), we see that a finitely generated abelian group is a direct product of two amenable groups, hence amenable.
\end{proof}
\begin{corollary}\label{cor:direct_limit_amenable}
  If $\set{G_i}_{i\in I}$ is a directed family of amenable groups, then the direct limit,
  \begin{align*}
    G &= \bigcup_{i\in I}G_i,
  \end{align*}
  is also amenable.
\end{corollary}
\begin{proof}
  Let $\mu_i\in \left(\ell_{\infty}\left(G_i\right)\right)^{\ast}$ be invariant states.\newline

  Set
  \begin{align*}
    M_i &= \set{\mu\in S\left(\ell_{\infty}\left(G\right)\right)| \mu\left(\lambda_s\left(f\right)\right) = \mu\left(f\right)\text{ for all }s\in G_i}.
  \end{align*}
  We set $\mu\left(f\right) = \mu_i\left(f|_{G_i}\right)$. Since each $G_i$ is amenable, it is the case that each $M_i$ is nonempty. Similarly, seeing as we have established the state space as $w^{\ast}$-closed in $B_{\ell_{\infty}\left(G\right)^{\ast}}$, it is the case that each $M_i$ is $w^{\ast}$-closed in $B_{\ell_{\infty}\left(G\right)^{\ast}}$.\newline

  For $i_1,\dots,i_n$, we find $G_j \supseteq G_{i_1},\dots,G_{i_n}$, which exists since $\set{G_i}_{i\in I}$ is directed. We have that $M_j\subseteq \bigcap_{k=1}^{n}M_{i_k}$, so $\set{M_i}_{i\in I}$ has the finite intersection property.\newline

  By compactness, there is $\mu\in \bigcap_{i\in I}M_i$ which is necessarily invariant on $G$.
\end{proof}
\begin{corollary}\label{cor:abelian_groups_amenable}
  All abelian groups are amenable.
\end{corollary}
\begin{proof}
  Every abelian group is the direct limit of its finitely generated subgroups.
\end{proof}
\begin{corollary}\label{cor:solvable_groups_amenable}
  All solvable groups are amenable.
\end{corollary}
\begin{proof}
  Let $e_G = G_0 \leq G_1\leq\cdots\leq G_n\leq G$ be such that $G_{j-1}\trianglelefteq G_j$ for $j=1,\dots,n$, and $G_i/G_j$ is abelian.\newline

  Since $G_0$ is abelian, it is amenable, as is $G_1/G_0$, so $G_1$ is amenable. We see then that $G_2$ is amenable as $G_1$ and $G_2/G_1$ are amenable.\newline

  Continuing in this fashion, we see that $G$ is amenable.
\end{proof}
\section{Remarks and Notes}\label{sec:invariant_states_remarks}%
The following proposition is, in a sense, a kind of converse to Proposition \ref{prop:subgroups_quotientgroups_amenability}, in that if a subgroup is amenable, we can show that the original group is also amenable, but this is only a sufficient condition if the subgroup has finite index.
\begin{proposition}\label{prop:finite_index_amenable_subgroup}
  Let $G$ be a group, and let $H\leq G$ be amenable, with $\left[G:H\right]  = n < \infty$. Then, $G$ is amenable.
\end{proposition}
\begin{proof}
  Let $H\leq G$ be amenable with $\left[G:H\right] = n$. Let $\mu$ be the mean on $H$, and let $\set{g_iH}_{i=1}^{n}$ be a partition of $G$ by the left cosets of $H$. We define the mean on $G$ by taking, for $A\subseteq G$,
  \begin{align*}
    \lambda\left(A\right) &= \frac{1}{n}\sum_{i=1}^{n}\mu\left(g_i^{-1}A\cap H\right).
  \end{align*}
  We begin by verifying that this is well-defined. Specifically, we will show that this definition is independent of the coset representatives. Suppose $g_jH = h_j H$. Then, $h_j^{-1}g_j \in H$. Now, we have $g_j^{-1}A \cap H \subseteq H$, so by left-multiplication, we get $\left(h_j^{-1}g_j\right)g_j^{-1}A\cap H \subseteq H$, so $h_j^{-1}A\cap H\subseteq H$. Since $\set{g_i H}_{i=1}^{n}$ is a partition, we get that this definition of the mean on $G$ is independent of the choice of coset representatives.\newline

  Next, we show that this is a finitely additive measure. Let $A,B\subseteq G$ be such that $A\cap B = \emptyset$. Then, we get
  \begin{align*}
    \lambda\left(A\sqcup B\right) &= \frac{1}{n}\sum_{i=1}^{n}\mu\left(g_i^{-1}\left(A\sqcup B\right)\cap H\right)\\
                                  &= \frac{1}{n}\sum_{i=1}^{n}\mu\left(\left(g_i^{-1}A\cap H\right)\sqcup \left(g_i^{-1}B\cap H\right)\right)\\
                                  &= \frac{1}{n}\left(\sum_{i=1}^{n}\mu\left(g_i^{-1}A\cap H\right) + \sum_{i=1}^{n}\mu\left(g_i^{-1}B\cap H\right)\right)\\
                                  &= \frac{1}{n}\sum_{i=1}^{n}\mu\left(g_i^{-1}A\cap H\right) + \frac{1}{n}\sum_{i=1}^{n}\mu\left(g_i^{-1}B\cap H\right)\\
                                  &= \lambda\left(A\right) + \lambda\left(B\right).
  \end{align*}
  It is relatively simple to see that $\lambda$ is a probability measure, as
  \begin{align*}
    \lambda\left(G\right) &= \frac{1}{n}\sum_{i=1}^{n}\mu\left(g_i^{-1}G\cap H\right)\\
                          &= \frac{1}{n}\sum_{i=1}^{n}\mu\left(G\cap H\right)\\
                          &= \frac{1}{n}\sum_{i=1}^{n}\mu\left(H\right)\\
                          &= 1.
  \end{align*}
  Now, we must show that $\lambda$ is translation-invariant. Let $A\subseteq G$ and $t\in G$. Then, using the translation-invariance of $\mu$, we get
  \begin{align*}
    \lambda\left(tA\right) &= \frac{1}{n}\sum_{i=1}^{n}\mu\left(g_i^{-1}tA\cap H\right)\\
                           &= \frac{1}{n}\sum_{i=1}^{n}\mu\left(g_i^{-1}\left(t\left(A\cap H\right)\right)\right)\\
                           &= \frac{1}{n}\sum_{i=1}^{n}\mu\left(g_i^{-1}A\cap H\right)\\
                           &= \lambda\left(A\right).
  \end{align*}
  Thus, $G$ is amenable.
\end{proof}
In Chapter \ref{ch:tarskis_theorem}, we proved that all the amenable groups are precisely those that are non-paradoxical, while in \ref{ch:paradoxical_decompositions}, we proved the Banach--Tarski paradox finding a subgroup of $\text{SO}(3)$ that is isomorphic to $F(a,b)$. This raises an interesting question: are all non-amenable (hence paradoxical) groups ones that contain subgroups isomorphic to $F(a,b)$?\newline

This is the substance of the von Neumann conjecture --- and as it turns out, it is false. There are some groups that are not amenable, but do not contain a subgroup isomorphic to $F(a,b)$. However, at the same time, in \cite{free_subgroups_of_linear_groups}, Jacques Tits proved that in any subgroup of $\text{GL}_n\left(\F\right)$ (where $\F$ is any field with characteristic zero), a subgroup either admits a solvable subgroup of finite index (hence amenable by Corollary \ref{cor:solvable_groups_amenable} and Proposition \ref{prop:finite_index_amenable_subgroup}) or contains a non-abelian freely generated subgroup (which is necessarily not amenable by Theorem \ref{thm:tarski}). This is known as the Tits alternative. In other words, the von Neumann conjecture \textit{is} true for linear groups, so it necessarily means that we cannot represent the counterexample groups to the von Neumann conjecture as linear groups.\newline

Note that we have found a sufficient condition to find a (non-abelian) freely generated subgroup by Theorem \ref{thm:ping_pong}, and showed in Theorem \ref{thm:free_group_so3} that $\text{SO}(3)$ contains a (necessarily) non-abelian freely generated subgroup --- in this case, isomorphic to $F(a,b)$ --- so we know by the Tits alternative that $\text{SO}(3)$ does not admit a solvable subgroup with a finite index. In the introduction to Chapter 3, we stated that the Banach--Tarski paradox cannot hold for $\R$ and $\R^2$, because the rotation group $\text{SO}\left(2\right)$ in $\R^2$ is abelian, and since the isometry group $\text{E}\left(2\right)$ has the abelian subgroup $\text{SO}\left(2\right)$ with finite index, $\text{E}\left(2\right)$ is amenable by Proposition \ref{prop:finite_index_amenable_subgroup}. Similarly, the isometry group $\text{E}(1)$ contains an abelian subgroup $\text{SO}(1)$\footnote{$\text{SO}(1) = \set{1}$.} with finite index.
