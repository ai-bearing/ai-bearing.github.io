The primary goal of this section will be to introduce the idea of a paradoxical decomposition (and its effects on the analytic properties of $\R^3$) through the Banach--Tarski Paradox. The ultimate goal is to prove the following statement.
\begin{proposition}[General Banach--Tarski Paradox]
  If $A$ and $B$ are bounded subsets of $\R^3$ with nonempty interior, there is a partition of $A$ into finitely many disjoint subsets such a sequence of isometries applied to these subsets yields $B$.
\end{proposition}
The existence of the Banach--Tarski paradox throws a wrench into a major idea that we may have about subsets of $\R^3$ --- namely, that they always have some ``volume'' to them that is invariant under isometry, similar to how ``area'' in $\R^2$ is invariant under isometry.
\section{Prelude: Essential Group Actions}
We begin by discussing some of the basic properties of group actions.
\begin{definition}[Group Action]
  Let $G$ be a group, and $A$ be a set. A left group action of $G$ onto $A$ is a map $\alpha: G\times A\rightarrow A$ that satisfies
  \begin{itemize}
    \item $\alpha\left(g_1,\left(g_2,a\right)\right) = \alpha\left(g_1g_2,a\right)$ for all $g_1,g_2\in G$ and $a\in A$;
    \item $\alpha\left(e_G,a\right) = a$ for all $a\in A$.
  \end{itemize}
  For the sake of brevity, we write $\left(g,a\right) = g\cdot a$.
\end{definition}
Every group action can be represented by a permutation on $A$.
\begin{definition}[Permutation Representation]
  For each $g$, the map $\sigma_g: A\rightarrow A$ defined by $\sigma_g\left(a\right) = g\cdot a$ is a permutation of $A$. There is a homomorphism associated to these actions, $\varphi: G\rightarrow \operatorname{Sym}(A)$, where $\operatorname{Sym}(A)$ is the symmetric group on the elements of $A$.
\end{definition}
The permutation representation can run in the opposite direction in the following sense: given a nonempty set $A$ and a homomorphism $\psi: G\rightarrow \sym(A)$, we can take $g\cdot a = \psi(g)(a)$, where $\psi(g) = \sigma_g\in \sym(A)$ is a permutation.\newline

Just as we can pass group actions into permutation representations, and discuss ideas like the kernel of homomorphisms, we can also discuss the kernel of ajn action.
\begin{definition}[Kernel]
  The kernel of the action of $G$ on $A$ is the set of elements in $g$ that act trivially on $A$:
  \begin{align*}
    \set{g\in G\mid \forall a\in A,~g\cdot a = a}.
  \end{align*}
  The kernel of the group action is the kernel of the permutation representation $\varphi: G\rightarrow \sym(A)$.
\end{definition}
\begin{definition}[Stabilizer]
  For each $a\in A$, we define the stabilizer of $a$ under $G$ to be the set of elements in $G$ that fix $a$:
  \begin{align*}
    G_a &= \set{g\in G\mid g\cdot a = a}.
  \end{align*}
\end{definition}
\begin{remark}
The kernel of the group action is the intersection of the stabilizers of every element of $A$.\newline

For each $a\in A$, $G_{a}$ is a subgroup of $G$.
\end{remark}
\begin{definition}[Faithful Action]
An action is faithful if the kernel of the action is the identity, $e_G$. Equivalently, the permutation representation $\varphi: G\rightarrow \sym(A)$ is injective.
\end{definition}
The following definition will be useful in the future as we dig deeper into the idea of paradoxical groups.
\begin{definition}[Free Action]
For a set $X$ with $G$ acting on $X$, the action of $G$ on $X$ is free if, for every $x\in X$, $g\cdot x = x$ if and only if $g = e_G$.
\end{definition}
The most important theorem relating to group actions is the orbit-stabilizer theorem. As we prove the following theorem, we will reveal the definition of an orbit as a type of equivalence class.
\begin{theorem}[Orbit-Stabilizer Theorem]
  Let $G$ be a group that acts on a nonempty set $A$. We define a relation $a\sim b$ if and only if $a = g\cdot b$ for some $g\in G$. This is an equivalence relation, with the number of elements in $\left[a\right]_{\sim}$ found by taking the index of the stabilizer of $a$ in $G$, $\left\vert G:G_a \right\vert$.
\end{theorem}
\begin{proof}
  We start by seeing that $a\sim a$, as $e_G\cdot a = a$. Similarly, if $a\sim b$, then there exists $g\in G$ such that $a = g\cdot b$. Thus,
  \begin{align*}
    g^{-1}\cdot a &= g^{-1}\cdot \left(g\cdot b\right)\\
                  &= g^{-1}g\cdot b\\
                  &= e\cdot b\\
                  &= b,
  \end{align*}
  meaning that $b\sim a$. Finally, if we have $a\sim b$ and $b\sim c$, we have $a = g\cdot b$ and $b = h\cdot c$ for some $g,h\in G$. Therefore,
  \begin{align*}
    a &= g\cdot \left(h\cdot c\right)\\
      &= \left(gh\right)\cdot c,
  \end{align*}
  meaning $a\sim c$. Thus, the relation $\sim$ is reflexive, symmetric, and transitive, so it is an equivalence relation.\newline

  We claim there is a bijection between the left cosets of $G_a$ and the elements of $\left[a\right]_{\sim}$.\newline

  Define $C_a = \set{g\cdot a\mid g\in G}$, which is the set of elements in the equivalence class of $a$. Define the map $g\cdot a \mapsto gG_a$. Since $g\cdot a$ is always an element of $C_a$, this map is surjective. Additionally, since $g\cdot a = h\cdot a$ if and only if $\left(h^{-1}g\right)\cdot a = a$, we have $h^{-1}g \in G_a$, which is only true if $gG_a = hG_a$. Thus, the map is injective.\newline

  Since there is a one to one map between the equivalence classes of $a$ under the action of $G$, and the number of left cosets of $G_a$, we know that the number of equivalence classes of $a$ under the action of $G$ is $\left\vert G:G_a \right\vert$.
\end{proof}
\begin{definition}[Orbit]
Let $G$ act on $A$, and let $a\in A$. The orbit of $a$ under $G$ is the set
\begin{align*}
  G\cdot a &= \set{g\cdot a\in A\mid g\in G}
\end{align*}
\end{definition}
