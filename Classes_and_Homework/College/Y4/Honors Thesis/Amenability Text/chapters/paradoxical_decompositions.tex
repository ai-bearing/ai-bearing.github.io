In the Bible, one of the miracles of Jesus is the feeding of the five thousand,\footnote{Fun fact: the feeding of the five thousand is the only other miracle of Jesus (aside from the resurrection) that is in all four gospels.} where, despite only having five loaves of bread and two fishes, a large crowd splits these morsels among themselves and eats to satisfaction after Jesus calls upon the power of God to enable them to do so. Of course, we may not be able to fully replicate this without some divine intervention --- but, mathematically, thanks to the power of the axiom of choice, we can show that something like the feeding of the five thousand is not only possible, but a fundamental feature of the isometry group of $\R^3$. This is exemplified in the most general form of the Banach--Tarski paradox.
\begin{restatable}[Strong Banach--Tarski Paradox]{proposition}{banachtarski}\label{prop:banachtarski}
  Let $A$ and $B$ be bounded subsets of $\R^3$ with nonempty interior. There is a partition of $A$ into finitely many disjoint subsets such that a sequence of isometries applied to these subsets yields $B$.
\end{restatable}
The Banach--Tarski paradox throws a wrench into a common belief that we have about $\R^3$ --- specifically, that every subset of $\R^3$ has a \textit{finitely additive} ``volume'' that is invariant under rigid motion.\footnote{Note that if we desire countable additivity, the axiom of choice shows that there does not exist a countably additive measure on $P\left(\R\right)$ that is also translation-invariant (see \cite[Section 1.1]{folland_real_analysis}). Finite additivity is a weaker condition than countable additivity that allows for the existence of well-behaved measures on $P\left(\R\right)$ and $P\left(\R^2\right)$, but even this fails in $\R^3$ and above.} This property does exist for $\R$ and $\R^2$, as their isometry groups have a property known as amenability --- in Section \ref{sec:invariant_states_remarks}, we will provide an outline for why this is true.\newline

To develop paradoxical decompositions, we will begin with the Ping Pong Lemma, which will allow us to find freely generated subgroups. We will apply this to the case of $\text{SO}(3)$ to find a freely generated subgroup. Then, we will use the fact that free groups on more than one generator have a property known as paradoxicality --- this property will provide the germ of the proof of the Banach--Tarski paradox.
\section{The Ping Pong Lemma}\label{sec:ping_pong_lemma}%
To move towards paradoxical decompositions, we need to find a simple and easily applicable criterion for knowing when an arbitrary group contains a freely generated subgroup. This is the domain of the Ping Pong Lemma, which we will prove in this section to show the existence of a freely generated subgroup of $\text{SO}(3)$. Later, this freely generated subgroup will be indispensable in proving the Banach--Tarski paradox.\newline

We begin by defining a free product of a family of groups $\set{\Gamma_i}_{i\in I}$. This will allow us to state the Ping Pong Lemma in its maximal generality.
\begin{definition}[Free Product]\label{def:free_product}
  Let $A$ be a set, and set $W(A)$ to be the set of words in $A$ equipped with the operation of concatenation. This turns $W(A)$ into a construction known as the \textit{free monoid}.\newline

  If $\set{\Gamma_i}_{i\in I}$ is a family of groups, and $A = \coprod_{i\in I}\Gamma_i$ is the coproduct (or disjoint union) of the groups $\Gamma_i$, then we define the equivalence relation $\sim$ generated by
  \begin{align*}
    we_iw' &\sim ww'\text{ where $e_i$ is the neutral element of $\Gamma_i$ for some $i\in I$}\\
    wabw' &\sim wcw'\text{ where $a,b,c\in \Gamma_i$ and $c=ab$ for some $i\in I$}.
  \end{align*}
  Then, the quotient $W(A)/\sim$ is known as the \textit{free product} of the groups $\set{\Gamma_i}_{i\in I}$, and is denoted
  \begin{align*}
    \bigstar_{i\in I}\Gamma_i.
  \end{align*}
\end{definition}
\begin{remark}
  The free group $F(S)$ is an instance of the free product where the $\Gamma_i$ are the formal cyclic groups generated by each $s\in S$.\newline

  From the way we have defined the free product, it can be shown, as in \cite[II.A.]{delaHarpe_topics_in_geometric_group_theory}, that every element of the free product is represented a unique reduced word on $W(A)$, along with the following universal property: if $\set{\Gamma_i}_{i\in I}$ is a family of groups, and $h_i\colon \Gamma_i\rightarrow\Gamma$ for some fixed group $\Gamma$, then there is a unique homomorphism $h\colon \bigstar_{i\in I}\Gamma_i \rightarrow \Gamma$ such that the following diagram commutes for each $\Gamma_{i_0}$.
  \begin{center}
        % https://tikzcd.yichuanshen.de/#N4Igdg9gJgpgziAXAbVABwnAlgFyxMJZABgBpiBdUkANwEMAbAVxiRAB12BxOgW17oB9YFkHEAviHGl0mXPkIoyARiq1GLNpwBGWAOZwcdAE7CsnLGAAEASXGce-IVikyQGbHgJFl5NfWZWRA5uPgFBF3E1GCg9eCJQADNjCF4kMhAcCCRfdUCtdnwjMzFJagY6bRgGAAU5L0UQY30ACxwQaiMsBjYWiAgAa1cklLTEDKykACZqAM1glpKJYZBk1JzO7MQZkAqq2vqFNma9No68+ZAWqQpxIA
    \begin{tikzcd}
      \Gamma_{i_0} \arrow[d, "\iota_{i_0}"', hook] \arrow[r, "h_{i_0}"] & \Gamma_i \\
      \bigstar_{i\in I}\Gamma_i \arrow[ru, "h"']                        &         
    \end{tikzcd}
  \end{center}
\end{remark}

\begin{theorem}[Ping Pong Lemma]
  Let $G$ be a group that acts on a set $X$, and let $\Gamma_1,\Gamma_2$ be subgroups of $G$. Let $\Gamma = \left\langle \Gamma_1,\Gamma_2 \right\rangle$. Assume $\Gamma_1$ contains at least $3$ elements, and $\Gamma_2$ contains at least $2$ elements.\newline

  Suppose there exist nonempty subsets $X_1,X_2\subseteq X$ with $X_1\triangle X_2 \neq \emptyset$ such that for all $\gamma\in \Gamma_1$ with $\gamma \neq e_{G}$,
  \begin{align*}
    \gamma\left(X_2\right)\subseteq X_1,
  \end{align*}
  and for all $\gamma \in \Gamma_2$ with $\gamma \neq e_G$,
  \begin{align*}
    \gamma\left(X_1\right)\subseteq X_2.
  \end{align*}
  Then, $\Gamma$ is isomorphic to the free product $\Gamma_1\star \Gamma_2$.\label{thm:ping_pong}
\end{theorem}
\begin{proof}
  Let $w$ be a nonempty reduced word with letters in the disjoint union of $\Gamma_1\setminus \set{e_G}$ and $\Gamma_2\setminus \set{e_G}$. We must show that the element of $\Gamma$ defined by $w$ is not the identity.\newline

  If $w = a_1b_1a_2b_2\cdots a_k$ with $a_1,\dots,a_k\in \Gamma_1\setminus \set{e_G}$ and $b_1,\dots,b_{k-1}\in \Gamma_{2}\setminus \set{e_G}$, then,
  \begin{align*}
    w\left(X_2\right) &= a_1b_1\cdots a_{k-1}b_{k-1}a_k\left(X_2\right)\\
                      &\subseteq a_1b_1\cdots a_{k-1}b_{k-1}\left(X_1\right)\\
                      &\subseteq a_1b_1\cdots a_{k-1}\left(X_2\right)\\
                      &\vdots\\
                      &\subseteq a_1\left(X_2\right)\\
                      &\subseteq X_1.
  \end{align*}
  Seeing as $X_2\nsubseteq X_1$ (by the definition of symmetric difference), it is the case that $w\neq e_{G}$.\newline

  If $w = b_1a_2b_2a_2\cdots b_k$, we select $a\in \Gamma_1\setminus \set{e_G}$, and we find that $awa^{-1}\neq e_G$, meaning $w\neq e_G$. Similarly, if $w = a_1b_1\cdots a_kb_k$, we select $a\in \Gamma_1\setminus \set{e_G,a_{1}^{-1}}$, similarly finding that $awa^{-1}\neq e_{G}$. If $w = b_1a_2b_2\cdots a_k$, then we select $a\in \Gamma_1\setminus \set{1,a_k}$, and find $awa^{-1}\neq e_G$.
\end{proof}

We can refine Theorem \ref{thm:ping_pong} to the case of ``doubles'' wherein we find a different (yet more readily applicable) sufficient condition for a group that contains a copy of the free group on two generators.
\begin{corollary}[Ping Pong Lemma for ``Doubles'']
  Let $G$ act on $X$, and let $A_{+}, A_{-},B_{+},B_{-}$ be disjoint subsets of $X$ whose union is not equal to $X$. Then, if
  \begin{align*}
    a\cdot \left(X\setminus A_{-}\right) &\subseteq A_{+}\\
    a^{-1}\cdot \left(X\setminus A_{+}\right) &\subseteq A_{-}\\
    b\cdot \left(X\setminus B_{-}\right) &\subseteq B_{+}\\
    b^{-1}\cdot \left(X\setminus B_{+}\right) &\subseteq B_{-},
  \end{align*}
  then it is the case that $\left\langle a,b \right\rangle$ is isomorphic to the free group on two generators.\label{corollary:ping_pong_doubles}
\end{corollary}
\begin{proof}
  We let $A = A_{+}\sqcup A_{-}$, $B = B_{+}\sqcup B_{-}$, $\Gamma_1 = \left\langle a \right\rangle$, and $\Gamma_2 = \left\langle b \right\rangle$. Then, $A,B,\Gamma_1,\Gamma_2$ satisfy the conditions for Theorem \ref{thm:ping_pong}.
\end{proof}
\begin{remark}
Instead of typing out ``the free group on two generators,'' we will henceforth use $F(a,b)$ to refer to the free group on two generators.
\end{remark}

We can apply Theorem \ref{thm:ping_pong} to show the existence of a set of isometries of $\R^n$ that is isomorphic to $F(a,b)$.
\begin{definition}[Special Orthogonal Group]
  For $n\in \N$, we define $\text{SO}(n)$ to be the group of all real $n\times n$ matrices $A$ such that $A^{T} = A^{-1}$ and $\det(A) = 1$.
\end{definition}
In terms of an isometry of $\R^3$, the group $\text{SO}(3)$ denotes the set of all rotations about any line through the origin.
\begin{theorem}\label{thm:free_group_so3}
  There are elements $a,b\in \text{SO}(3)$ such that $\left\langle a,b \right\rangle_{\text{SO}(3)} \cong F(a,b)$.
\end{theorem}
\begin{proof}
  We let
  \begin{align*}
    a &= \begin{pmatrix}3/5 & 4/5 & 0 \\ -4/5 & 3/5 & 0 \\ 0 & 0 & 1\end{pmatrix}\\
    a^{-1} &= \begin{pmatrix}3/5 & -4/5 & 0 \\ 4/5 & 3/5 & 0 \\ 0 & 0 & 1\end{pmatrix}\\
    b &= \begin{pmatrix}1 & 0 & 0 \\ 0 & 3/5 & -4/5 \\ 0 & 4/5 & 3/5\end{pmatrix}\\
    b^{-1} &= \begin{pmatrix}1 & 0 & 0 \\ 0 & 3/5 & 4/5 \\ 0 & -4/5 & 3/5\end{pmatrix}.
  \end{align*}
  We specify
  \begin{align*}
    X &= A_{+} \sqcup A_{-} \sqcup B_{+} \sqcup B_{-} \sqcup \begin{pmatrix}0\\1\\0\end{pmatrix},
  \end{align*}
  where
  \begin{align*}
    A_{+} &= \set{\frac{1}{5^{k}} \begin{pmatrix}x\\y\\z\end{pmatrix} | k\in \Z, x \equiv 3y\text{ modulo $5$}, z\equiv0\text{ modulo $5$}}\\
    A_{-} &= \set{\frac{1}{5^{k}} \begin{pmatrix}x\\y\\z\end{pmatrix} | k\in \Z, x \equiv -3y\text{ modulo $5$}, z\equiv 0\text{ modulo $5$}}\\
    B_{+} &= \set{\frac{1}{5^{k}} \begin{pmatrix}x\\y\\z\end{pmatrix} | k\in \Z, z \equiv 3y\text{ modulo $5$}, x\equiv 0\text{ modulo $5$}}\\
    B_{-} &= \set{\frac{1}{5^{k}} \begin{pmatrix}x\\y\\z\end{pmatrix} | k\in \Z, z \equiv -3y\text{ modulo $5$}, x\equiv 0\text{ modulo $5$}}.
  \end{align*}
  To verify that the conditions of Theorem \ref{thm:ping_pong} hold, we calculate
  \begin{align*}
    \begin{pmatrix}3/5 & 4/5 & 0 \\ -4/5 & 3/5 & 0 \\ 0 & 0 & 1\end{pmatrix}\left(\frac{1}{5^k} \begin{pmatrix}x\\y\\z\end{pmatrix}\right) &= \frac{1}{5^{k+1}} \begin{pmatrix}3x + 4y \\ -4x + 3y \\ 5z\end{pmatrix}\tag*{(1)}\\
    \begin{pmatrix}3/5 & -4/5 & 0 \\ 4/5 & 3/5 & 0 \\ 0 & 0 & 1\end{pmatrix} \left(\frac{1}{5^k} \begin{pmatrix}x\\y\\z\end{pmatrix}\right) &= \frac{1}{5^{k+1}} \begin{pmatrix}3x - 4y \\ 4x + 3y \\ 5z\end{pmatrix}\tag*{(2)}\\
    \begin{pmatrix}1 & 0 & 0 \\ 0 & 3/5 & -4/5 \\ 0 & 4/5 & 3/5\end{pmatrix}\left(\frac{1}{5^{k}} \begin{pmatrix}x\\y\\z\end{pmatrix}\right) &= \frac{1}{5^{k+1}} \begin{pmatrix}5x \\ 3y- 4z \\ 4y + 3z\end{pmatrix}\tag*{(3)}\\
    \begin{pmatrix}1 & 0 & 0 \\ 0 & 3/5 & 4/5 \\ 0 & -4/5 & 3/5\end{pmatrix} \left(\frac{1}{5^{k}} \begin{pmatrix}x\\y\\z\end{pmatrix}\right) &= \frac{1}{5^{k+1}} \begin{pmatrix}5x \\ 3y + 4z \\ -4y + 3z\end{pmatrix}.\tag*{(4)}
  \end{align*}
  We verify that the conditions for Corollary \ref{corollary:ping_pong_doubles} hold for each of these four conditions.
  \begin{enumerate}[(1)]
    \item For any vector
      \begin{align*}
        \frac{1}{5^{k}} \begin{pmatrix}x\\y\\z\end{pmatrix} \notin A_{-},
      \end{align*}
      we see that $k+1\in \Z$, $x' = 3x + 4y \equiv 3\left(-4x + 3y\right)$  modulo $5$, and that $z' = 5z\equiv 0$ modulo $5$.
    \item For any vector
      \begin{align*}
        \frac{1}{5^{k}} \begin{pmatrix}x\\y\\z\end{pmatrix} \notin A_{+},
      \end{align*}
      we see that $k+1\in \Z$, $x' = 3x - 4y\equiv -3\left(4x + 3y\right)$ modulo $5$, and $z' = 5z \equiv 0$ modulo $5$.
    \item For any vector
      \begin{align*}
        \frac{1}{5^{k}} \begin{pmatrix}x\\y\\z\end{pmatrix}\notin B_{-},
      \end{align*}
      we see that $k+1\in \Z$, $z' = 4y + 3z \equiv 3\left(3y-4z\right)$ modulo $5$, and $x' = 5x\equiv 0$ modulo $5$.
    \item For any vector
      \begin{align*}
        \frac{1}{5^{k}} \begin{pmatrix}x\\y\\z\end{pmatrix}\notin B_{+},
      \end{align*}
      we see that $k+1\in \Z$, $z' = -4y + 3z \equiv -3\left(3y + 4z\right)$ modulo $5$, and $x' = 5x \equiv 0$ modulo $5$.
  \end{enumerate}
  Thus, by Theorem \ref{thm:ping_pong} and Corollary \ref{corollary:ping_pong_doubles}, it is the case that $\left\langle a,b \right\rangle\cong F(a,b)$.
\end{proof}

\section{Introducing Paradoxical Decompositions}\label{sec:intro_paradoxical_decompositions}%
We now turn our attention towards ``paradoxical'' actions that seem to recreate a set by using disjoint proper subsets. This will allow us to use the result from Theorem \ref{thm:free_group_so3} to move towards the Banach--Tarski paradox.
\begin{definition}[Paradoxical Decompositions and Paradoxical Groups]
  Let $G$ be a group that acts on a set $X$, with $E\subseteq X$. We say $E$ is $G$\textit{-paradoxical} if there exist pairwise disjoint proper subsets $A_1,\dots,A_n$ and $B_1,\dots,B_m$ of $E$ and group elements $g_1,\dots,g_n,h_1,\dots,h_m\in G$ such that
  \begin{align*}
    E &= \bigcup_{j=1}^{n}g_j\cdot A_j
  \end{align*}
  and
  \begin{align*}
    E &= \bigcup_{j=1}^{m}h_j\cdot B_j.
  \end{align*}
  If $G$ acts on itself by left-multiplication, and $G$ satisfies these conditions, we say $G$ is a \textit{paradoxical group}.
\end{definition}
\begin{example}
  The free group on two generators, $F(a,b)$, is a paradoxical group.\newline
  %The free group is defined to be the set of all reduced words over the set $\set{a,b,a^{-1},b^{-1},e_{F(a,b)}}$, where $aa^{-1}$, $a^{-1}a$, $bb^{-1}$, and $b^{-1}b$ are replaced with the identity $e_{F(a,b)}$.\newline

  To see that $F(a,b)$ is a paradoxical group, we let $W(x)$ denote the set of words in $F(a,b)$ that start with $x\in \set{a,b,a^{-1},b^{-1}}$. For instance, $ba^2ba^{-1}\in W(b)$.\newline

  Since every word in $F$ is either the empty word, or starts with one of $a,b,a^{-1},b^{-1}$, we see that
  \begin{align*}
    F(a,b) &= \set{e_{F(a,b)}} \sqcup W(a) \sqcup W(b) \sqcup W\left(a^{-1}\right) \sqcup W\left(b^{-1}\right).
  \end{align*}
  If $w\in F(a,b)\setminus W(a)$, we see that $a^{-1}w\in W\left(a^{-1}\right)$. Thus, $w\in aW\left(a^{-1}\right)$. For any $t\in F(a,b)$ either $t\in W(a)$ or $t\in F(a,b)\setminus W(a) = aW\left(a^{-1}\right)$. Thus, $F\left(a,b\right) $ is equal to $ W(a)\sqcup aW\left(a^{-1}\right)$.\newline

  Similarly, if $t\in F(a,b)$ either $t\in W(b)$ or $t\in F\left(a,b\right)\setminus W(b) = bW\left(b^{-1}\right)$, so $F\left(a,b\right) $ is equal to $ W(b)\sqcup bW\left(b^{-1}\right)$.\newline

  We have thus constructed
  \begin{align*}
    F(a,b) &= W(a)\sqcup aW\left(a^{-1}\right)\\
           &= W(b)\sqcup bW\left(b^{-1}\right),
  \end{align*}
  a paradoxical decomposition of $F(a,b)$ with the action of left-multiplication.
\end{example}
Now that we understand a little more about paradoxical groups, we now want to understand the actions of paradoxical groups on sets.
\begin{proposition}
  Let $G$ be a paradoxical group that acts freely on $X$. Then, $X$ is $G$-paradoxical.
\end{proposition}
\begin{proof}
  Let $A_1,\dots,A_n,B_1,\dots,B_m\subset G$ be pairwise disjoint, and let $g_1,\dots,g_n,h_1,\dots,h_m\in G$ such that
  \begin{align*}
    G &= \bigcup_{j=1}^{n}g_jA_j\\
      &= \bigcup_{j=1}^{m}h_jB_j.
  \end{align*}
  Let $M\subseteq X$ contain exactly one element from every orbit in $X$.
  \begin{claim}
  The set $\set{g\cdot M\mid g\in G}$ is a partition of $X$.
  \end{claim}
  \begin{proof}[Proof of Claim:]
  Since $M$ contains exactly one element from every orbit in $X$, it is the case that $G\cdot M = X$, so
  \begin{align*}
    \bigcup_{g\in G} g\cdot M &= X
  \end{align*}
  Additionally, for $x,y\in M$, if $g\cdot x = h\cdot y$, then $\left(h^{-1}g\right)\cdot x = y$, meaning $y$ is in the orbit of $x$ and vice versa, implying $x = y$. Since $G$ acts freely on $X$, we must have $h^{-1}g = e_G$.\newline

  Thus, we can see that $g_1\cdot M \neq g_2\cdot M$, implying $\set{g\cdot M\mid g\in G}$ is a partition of $X$.
  \end{proof}

  We define
  \begin{align*}
    A_j^{\ast} &= \bigcup_{g\in A_j}g\cdot M,
  \end{align*}
  and similarly define
  \begin{align*}
    B_j^{\ast} &= \bigcup_{h\in B_j}h\cdot M.
  \end{align*}
  As a useful shorthand, we can also write $A_j^{\ast} = A_j\cdot M$, and similarly, $B_j^{\ast} = B_j\cdot M$, to denote the union of the elements of $A_j$ and $B_j$ respectively acting on $M$.\newline

  Since $\set{g\cdot M\mid g\in G}$ is a partition of $X$, and $A_1,\dots,A_n,B_1,\dots,B_m\subset G$ are pairwise disjoint, it must be the case that $A_1^{\ast},\dots,A_n^{\ast},B_1^{\ast},\dots,B_m^{\ast}\subset X$ are also pairwise disjoint.\newline

  For the original $g_1,\dots,g_n,h_1,\dots,h_m$ that defined the paradoxical decomposition of $G$, we thus have
  \begin{align*}
    \bigcup_{j=1}^{n}g_j\cdot A_j^{\ast} &= \bigcup_{j=1}^{n}\left(g_jA_j\right)\cdot M\\
                                         &= G\cdot M\\
                                         &= X,
  \end{align*}
  and
  \begin{align*}
    \bigcup_{j=1}^{m}h_j\cdot B_j^{\ast} &= \bigcup_{j=1}^{m}\left(h_jB_j\right)\cdot M\\
                                         &= G\cdot M\\
                                         &= X.
  \end{align*}
  Thus, $X$ is $G$-paradoxical.
\end{proof}
\begin{remark}
  This proof requires the axiom of choice, as we invoked it to define $M$ to contain exactly one element from every orbit in $X$.
\end{remark}
\section{The Weak Banach--Tarski Paradox}\label{sec:weak_banach_tarski}%
Now that we have established $F(a,b)$ as being a paradoxical group, we wish to use it to construct paradoxical decompositions of the unit sphere $S^2\subseteq \R^3$. Specifically, we will show a weak version of the Banach--Tarski paradox --- one where you can break apart the unit ball into finitely many pieces and reconstitute it into two copies of itself.
\begin{fact}
  If $H$ is a paradoxical group, and $H\leq G$, then $G$ is a paradoxical group.
\end{fact}
With this fact in mind, we will show that $\text{SO}(3)$ is a paradoxical group.
\begin{theorem}
  There are rotations $A$ and $B$ that about lines through the origin in $\R^3$ that generate a subgroup of $\text{SO}(3)$ isomorphic to $F(a,b)$
\end{theorem}
\begin{proof}
  We take $A$ and $B$ as in the proof of Theorem \ref{thm:free_group_so3}.
%  We take
%  \begin{align*}
%    A &= \begin{bmatrix}1/3 & -\frac{2\sqrt{2}}{3} & 0\\ \frac{2\sqrt{2}}{3} & 1/3 & 0 \\ 0 & 0 & 1\end{bmatrix}\\
%    A^{-1} &= \begin{bmatrix}1/3 & \frac{2\sqrt{2}}{3} & 0\\ -\frac{2\sqrt{2}}{3} & 1/3 & 0 \\ 0 & 0 & 1\end{bmatrix}\\
%    B &= \begin{bmatrix}1 & 0 & 0 \\ 0 & 1/3 & -\frac{2\sqrt{2}}{3} \\ 0 & \frac{2\sqrt{2}}{3} & 1/3\end{bmatrix}\\
%    B^{-1} &= \begin{bmatrix}1 & 0 & 0 \\ 0 & 1/3 & \frac{2\sqrt{2}}{3} \\ 0 & -\frac{2\sqrt{2}}{3} & 1/3\end{bmatrix}
%  \end{align*}
%  We let $A^{\pm}$ denote $A$ and $A^{-1}$ respectively, and similarly for $B^{\pm}$.\newline
%
%  Let $w$ be a reduced word in $\set{A,A^{-1},B,B^{-1}}$ which is not the empty word. We claim that $w$ cannot be the identity.\newline
%
%  Without loss of generality, we assume that $w$ ends in $A$ or $A^{-1}$ --- this is because if $w$ is the identity, then $AwA^{-1}$ and $A^{-1}wA$ are also the identity.\newline
%
%  We will show that there exist $a,b,c\in \Z$ with $b\not\equiv 0$ mod $3$ such that
%  \begin{align*}
%    w \cdot \begin{pmatrix}1\\0\\0\end{pmatrix} &= \frac{1}{3^k} \begin{pmatrix}a\\b\sqrt{2}\\c\end{pmatrix}.
%  \end{align*}
%  If $b\not\equiv 0$ mod $3$, and $w$ is not empty, then $w$ cannot act as the identity.\newline
%
%  We induct on the length of $w$. For $w = A^{\pm}$, we have
%  \begin{align*}
%    w\cdot \begin{pmatrix}1\\0\\0\end{pmatrix} &= \frac{1}{3}\begin{pmatrix}1\\\pm2\sqrt{2}\\0\end{pmatrix},
%  \end{align*}
%  proving the base case.\newline
%
%  Let $k > 0$, meaning $w = A^{\pm}w'$, or $w = B^{\pm}w'$, with $w'$ not equal to the empty. The inductive hypothesis says
%  \begin{align*}
%    w'\cdot \begin{pmatrix}1\\0\\0\end{pmatrix} &= \frac{1}{3^{k-1}} \begin{pmatrix}a'\\b'\sqrt{2}\\c'\end{pmatrix}
%  \end{align*}
%  for some $a',b',c'\in \Z$, and $b'\not\equiv 0$ mod $3$. In particular,
%  \begin{align*}
%    A^{\pm}w' \cdot \begin{pmatrix}1\\0\\0\end{pmatrix} &= \frac{1}{3^k} \begin{pmatrix}a\mp 4b \\ \left(b'\pm 2a'\right)\sqrt{2} \\ 3c'\end{pmatrix}\\
%    B^{\pm}w' \cdot \begin{pmatrix}1\\0\\0\end{pmatrix} &= \frac{1}{3^k} \begin{pmatrix}3a' \\ \left(b'\mp 2c'\right)\sqrt{2} \\ c'\pm 4b'\end{pmatrix}.
%  \end{align*}
%  Now, we set
%  \begin{align*}
%    w \cdot \begin{pmatrix}1\\0\\0\end{pmatrix} &= \frac{1}{3^k} \begin{pmatrix}a\\b\sqrt{2}\\c\end{pmatrix},
%  \end{align*}
%  meaning
%  \begin{align*}
%    a &= \begin{cases}
%      a'\mp 4b', & w = A^{\pm} w'\\
%      3a', & w = B^{\pm}w'
%    \end{cases}\\
%      b &= \begin{cases}
%        b'\pm 2a', & w = A^{\pm}w'\\
%        b'\mp 2c', & w = B^{\pm}w'
%      \end{cases}\\
%        c &= \begin{cases}
%          3c', & w = A^{\pm}w'\\
%          c' \pm 4b', & w = B^{\pm}w'
%        \end{cases}
%  \end{align*}
%  Let $w^{\ast}$ denote the word such that $w' = A^{\pm}w^{\ast}$ or $w' = B^{\pm}w^{\ast}$. We write
%  \begin{align*}
%    w^{\ast} &= \frac{1}{3^{k-2}} \begin{pmatrix}a''\\b''\sqrt{2}\\c''\end{pmatrix},
%  \end{align*}
%  where $a'',b'',c''\in \Z$. Note that it may not be the case that $w^{\ast}$ is a non-empty word. We examine the following four cases.
%  \begin{description}
%    \item[Case 1:] Suppose $w = A^{\pm}B^{\pm}w^{\ast}$. Then, $b = b'\mp 2a'$, where $a' = 3a''$. Since $b'\not\equiv 0$ mod $3$ (by the inductive hypothesis), it is also the case $b\equiv 0$ mod $3$.
%    \item[Case 2:] Suppose $w = B^{\pm}A^{\pm}w^{\ast}$. Then, $b = b'\mp 2c'$, where $c' = 3c''$. Since $b'\not\equiv 0$ mod $3$ (by the inductive hypothesis), it is also the case that $b\not\equiv 0$ mod $3$.
%    \item[Case 3:] Suppose $w = A^{\pm}A^{\pm}w^{\ast}$. Then, we have
%      \begin{align*}
%        b &= b' \pm 2a'\\
%          &= b' \pm 2\left(a'' \pm 4b''\right)\\
%          &= b'+ \left(b'' \pm 2a''\right) - 9b''\\
%          &= 2b' - 9b''.
%      \end{align*}
%      Thus, regardless of the value of $b''$, since $b'\not\equiv 0$ mod $3$ by the inductive hypothesis, it is the case that $b\not\equiv 0$ mod $3$.
%    \item Suppose $w = B^{\pm}B^{\pm}w^{\ast}$. Then, we have
%      \begin{align*}
%        b &= b' \mp 2c'\\
%          &= b' \mp 2\left(c'' \pm 4b''\right)\\
%          &= b' + \left(b'' \mp 2c''\right) - 9b''\\
%          &= 2b' - 9b''.
%      \end{align*}
%      Thus, regardless of the value of $b''$, since $b'\not\equiv 0$ mod $3$ by the inductive hypothesis, it is the case that $b\not\equiv 0$ mod $3$.
%  \end{description}
%  We have thus shown that any non-empty reduced word over $\set{A,A^{-1},B,B^{-1}}$ does not act as the identity. The subgroup of $\text{SO}(3)$ generated by $\set{A,A^{-1},B,B^{-1}}$ is isomorphic to $F(a,b)$.
\end{proof}
\begin{remark}
  Since $\text{SO}(n)$ contains a subgroup isomorphic to $\text{SO}(3)$ for all $n\geq 3$ (via the block matrices), it is the case that $\text{SO}(n)$ also contains a subgroup isomorphic to $F(a,b)$ for all $n\geq 3$.
\end{remark}
Since we have shown that $\text{SO}(3)$ is paradoxical, as it contains a paradoxical subgroup, we can now begin to examine the action of $\text{SO}(3)$ on subsets of $\R^3$.
\begin{theorem}[Hausdorff Paradox]
  There is a countable subset $D$ of $S^{2}$ such that $S^{2}\setminus D$ is $\text{SO}(3)$-paradoxical.
\end{theorem}
\begin{proof}
  Let $A$ and $B$ be the rotations in $\text{SO}(3)$ that serve as the generators of the subgroup isomorphic to $F(a,b)$ (as in \ref{thm:free_group_so3}).\newline

  Since $A$ and $B$ are rotations, so too is any element of $\left\langle A,B \right\rangle$. Thus, any such non-empty word contains two fixed points.\newline

  We let
  \begin{align*}
    F &= \set{x\in S^{2}\mid x\text{ is a fixed point for some word }w}.
  \end{align*}
  Since $\left\langle A,B \right\rangle$ is countably infinite, so too is $F$. Thus, the union of all these fixed points under the action of all such words $w$ is countable.
  \begin{align*}
    D &= \bigcup_{w\in \left\langle A,B \right\rangle} w\cdot F.
  \end{align*}
  Therefore, $\left\langle A,B \right\rangle$ acts freely on $S^{2}\setminus D$, so $S^{2}\setminus D$ is $\text{SO}(3)$-paradoxical.
\end{proof}

Unfortunately, the Hausdorff paradox is not enough for us to be able to prove the Banach--Tarski paradox. In order to do this, we need to be able to show that two sets are ``similar'' under the action of a group.
\begin{definition}[Equidecomposable Sets]
  Let $G$ act on $X$, and let $A,B\subseteq X$. We say $A$ and $B$ are $G$-equidecomposable if there are partitions $\set{A_j}_{j=1}^{n}$ of $A$ and $\set{B_j}_{j=1}^{n}$ of $B$, and elements $g_1,\dots,g_n\in G$, such that for all $j$,
  \begin{align*}
    B_j &= g_j\cdot A_j.
  \end{align*}
  We write $A\sim_{G}B$ if $A$ and $B$ are $G$-equidecomposable.
\end{definition}
\begin{fact}\label{fact:equidecomposability_equivalence_relation}
  The relation $\sim_{G}$ is an equivalence relation.
\end{fact}
\begin{proof}
  Let $A$, $B$, and $C$ be sets.\newline

  To show reflexivity, we can select $g_1 = g_2 = \cdots = g_n = e_G$. Thus, $A\sim_{G}A$.\newline

  To show symmetry, let $A\sim_{G} B$. Set $\set{A_j}_{j=1}^{n}$ to be the partition of $A$, and set $\set{B_j}_{j=1}^{n}$ to be the partition of $B$, such that there exist $g_1,\dots,g_n\in G$ with $g_j\cdot A_j = B_j$. Then,
  \begin{align*}
    g_j^{-1}\cdot \left(g_j\cdot A_j\right) &= g_j^{-1}\cdot B_j\\
    A_j &= g_j^{-1}\cdot B_j,
  \end{align*}
  so $B_j\sim_{G}A_j$.\newline

  To show transitivity, let $A\sim_{G} B$ and $B\sim_{G} C$. Let $\set{A_i}_{i=1}^{n}$ and $\set{B_i}_{i=1}^{n}$ be the partitions of $A$ and $B$ respectively and $g_1,\dots,g_n\in G$ such that $g_i\cdot A_i = B_i$. Let $\set{B_j}_{j=1}^{m}$ and $\set{C_j}_{j=1}^{m}$ be partitions of $B$ and $C$, and $h_1,\dots,h_m\in G$, such that $h_j\cdot B_j = C_j$.\newline

  We refine the partition of $A$ to $A_{ij}$ by taking $A_{ij} = g_i^{-1}\left(B_{i}\cap B_j\right)$, where $i = 1,\dots,n$ and $j = 1,\dots,m$. Then, $\left(h_jg_i\right)\cdot A_{ij}$ maps the refined partition of $A$ to $C$, so $A$ and $C$ are $G$-equidecomposable.
\end{proof}
\begin{fact}
  For $A\sim_{G} B$, there is a bijection $\phi\colon A\rightarrow B$ by taking $C_{i} = C\cap A_i$, and mapping $\phi\left(C_i\right) = g_i\cdot C_i$.\newline

  In particular, this means that for any subset $C\subseteq A$, it is the case that $C\sim \phi(C)$.\label{fact:bijections}
\end{fact}

We can now use this equidecomposability to glean information about the existence of paradoxical decompositions.
\begin{proposition}
  Let $G$ act on $X$, with $E,E'\subseteq X$ such that $E\sim_{G}E'$. Then, if $E$ is $G$-paradoxical, then so too is $E'$.
\end{proposition}

\begin{proof}
Let $A_1,\dots,A_n,B_1,\dots,B_m\subset E$ be pairwise disjoint, with $g_1,\dots,g_n,h_1,\dots,h_m\in G$ such that
\begin{align*}
  E &= \bigcup_{i=1}^{n}g_i\cdot A_i\\
    &= \bigcup_{j=1}^{m}h_j\cdot B_j.
\end{align*}
We let
\begin{align*}
  A &= \bigsqcup_{i=1}^{n}A_i\\
  B &= \bigsqcup_{j=1}^{m}B_j.
\end{align*}
It follows that $A\sim_{G}E$ and $B\sim_{G}E$, since we can take the partition of $A$ to be $A_1,\dots,A_n$, and partition $E$ by taking $g_i\cdot A_i$ for $i=1,\dots,n$, and similarly for $B$.\newline

Since $E\sim_{G}E'$, and $\sim_{G}$ is an equivalence relation, it follows that $A\sim_{G}E'$ and $B\sim_{G}E'$. Thus, there is a paradoxical decomposition of $E'$ in $A_1,\dots,A_n$ and $B_1,\dots,B_m$.
\end{proof}

We will now show that $S^{2}$ is $\text{SO}(3)$ paradoxical.
\begin{proposition}
  Let $D\subseteq S^{2}$ be countable. Then, $S^{2}$ and $S^{2}\setminus D$ are $\text{SO}(3)$-equidecomposable.
\end{proposition}

\begin{proof}
  Let $L$ be a line in $\R^3$ such that $L\cap D = \emptyset$. Such an $L$ must exist since $S^{2}$ is uncountable.\newline

  Define $\rho_{\theta}\in \text{SO}(3)$ to be a rotation about $L$ by an angle of $\theta$. For a fixed $n\in \N$ and fixed $\theta\in [0,2\pi)$, define $R_{n,\theta} = \set{x\in D\mid \rho^{n}_{\theta}\cdot x \in D}$. Since $D$ is countable, $R_{n,\theta}$ is necessarily countable.\newline

  We define $W_n = \set{\theta\mid R_{n,\theta}\neq \emptyset}$. Since the map $\theta \mapsto \rho_{\theta}^{n}\cdot x$ into $D$ is injective, it is the case that $W_n$ is countable. Therefore,
  \begin{align*}
    W &= \bigcup_{n\in \N}W_n
  \end{align*}
  is countable.\newline

  Thus, there must exist $\omega \in [0,2\pi)\setminus W$. We define $\rho_{\omega}$ to be a rotation about $L$ by $\omega$. Then, for every $n,m\in \N$, we have
  \begin{align*}
    \rho_{\omega}^n\cdot D \cap \rho_{\omega}^{m}\cdot D &= \emptyset.
  \end{align*}
  We define $\widetilde{D} = \bigsqcup_{n=0}^{\infty}\rho^{n}_{\omega}D$. Note that 
  \begin{align*}
    \rho_{\omega}\cdot \widetilde{D} &= \rho_{\omega}\cdot\bigsqcup_{n=0}^{\infty}\rho_{\omega}^{n}\cdot D\\
                                     &= \bigsqcup_{n=1}^{\infty}\rho_{\omega}^{n}\cdot D\\
                                     &= \widetilde{D} \setminus D,
  \end{align*}
  meaning $\widetilde{D}$ and $D$ are $\text{SO}(3)$-equidecomposable.\newline

  Thus, we have
  \begin{align*}
    S^{2} &= \widetilde{D}\sqcup \left(S^{2}\setminus \widetilde{D}\right)\\
          &\sim_{\text{SO}(3)}\left(\rho_{\omega}\cdot \widetilde{D}\right)\sqcup \left(S^{2}\setminus\widetilde{D}\right)\\
          &= \left(\widetilde{D}\setminus D\right)\sqcup \left(S^{2}\setminus\widetilde{D}\right)\\
          &= S^{2}\setminus D,
  \end{align*}
  establishing $S^{2}$ and $S^{2}\setminus D$ as $\text{SO}(3)$-equidecomposable.\newline

  In particular, this means $S^{2}$ is also $\text{SO}(3)$-paradoxical.
\end{proof}
To prove the Banach--Tarski paradox, we need a slightly larger group than $\text{SO}(3)$ --- one that includes translations in addition to the traditional rotations.
\begin{definition}[Euclidean Group]
  The {Euclidean group}, $\text{E}(n)$, consists of all isometries of a Euclidean space. An isometry of a Euclidean space consists of translations, rotations, and reflections.
\end{definition}
\begin{corollary}[Weak Banach--Tarski Paradox]
  Every closed ball in $\R^3$ is $\text{E}(3)$-paradoxical.
\end{corollary}
\begin{proof}
  We only need to show that $B(0,1)$ is $\text{E}(3)$-paradoxical. To do this, we start by showing that $B(0,1)\setminus \set{0}$ is $\text{SO}(3)$-paradoxical.\newline

  Since $S^{2}$ is $\text{SO}(3)$-paradoxical, there exists pairwise disjoint subsets $A_1,\dots,A_n,B_1,\dots,B_m\subset S^2$ and elements $g_1,\dots,g_n,h_1,\dots,h_m\in \text{SO}(3)$ such that
  \begin{align*}
    S^{2} &= \bigcup_{i=1}^{n}g_i\cdot A_i\\
          &= \bigcup_{j=1}^{m}h_j\cdot B_j.
  \end{align*}
  Define
  \begin{align*}
    A_i^{\ast} &= \set{tx\mid t\in (0,1], x\in A_i}\\
    B_j^{\ast} &= \set{ty\mid t\in (0,1], y\in B_j}.
  \end{align*}
  Then, $A_1^{\ast},\dots,A_n^{\ast},B_1^{\ast},\dots,B_m^{\ast}\subset B(0,1)\setminus \set{0}$ are pairwise disjoint, and
  \begin{align*}
    B(0,1)\setminus \set{0} &= \bigcup_{i=1}^{n}g_i\cdot A_i^{\ast}\\
                            &= \bigcup_{j=1}^{m}h_j\cdot B_j^{\ast}.
  \end{align*}
  Thus, we have established that $B(0,1)\setminus \set{0}$ is $\text{E}(3)$-paradoxical.\newline

  Now, we want to show that $B(0,1)\setminus \set{0}$ and $B(0,1)$ are $\text{E}(3)$-equidecomposable. Let $x\in B(0,1)\setminus \set{0}$, and let $\rho$ be a rotation through $x$ by a line not through the origin such that $\rho^{n}\cdot 0\neq \rho^{m}\cdot 0$ when $n\neq m$.\newline

  Let $D = \set{\rho^{n}\cdot 0\mid n\in \N}$. We can see that $\rho\cdot D = D\setminus \set{0}$, and that $D$ and $\rho\cdot D$ are $\text{E}(3)$-equidecomposable. Thus,
  \begin{align*}
    B(0,1) &= D\sqcup \left(B(0,1)\setminus D\right)\\
           &\sim_{\text{E}(3)}\left(\rho\cdot D\right) \sqcup \left(B(0,1)\setminus D\right)\\
           &= \left(D\setminus \set{0}\right)\sqcup \left(B\left(0,1\right)\setminus D\right)\\
           &= B\left(0,1\right)\setminus \set{0}.
  \end{align*}
  Therefore, $B(0,1)$ is $\text{E}(3)$-paradoxical.
\end{proof}
\section{The Strong Banach--Tarski Paradox}\label{sec:full_banach_tarski}%
In order to prove the general case of the Banach--Tarski paradox, we need one more piece of mathematical machinery.\newline

In Fact \ref{fact:equidecomposability_equivalence_relation}, we showed that the relation $A\sim_{G} B$ if and only if $A$ and $B$ are $G$-equidecomposable is an equivalence relation. Using the power of subsets, we may extend this to a preorder on any subsets $A$ and $B$ of $X$.
\begin{definition}
  Let $G$ act on a set $X$ with $A,B\subseteq X$. We write $A\preceq_{G}B$ if $A$ is equidecomposable with a subset of $B$.
\end{definition}
\begin{fact}
  The relation $\preceq_{G}$ is a reflexive and transitive relation.\label{fact:preorder}
\end{fact}
\begin{proof}
  To see reflexivity, we can see that since $A\sim_{G}A$, and $A\subseteq A$, $A\preceq_{G} A$.\newline

  To see transitivity, let $A\preceq_{G}B$ and $B\preceq_{G}C$. Then, there exist $g_1,\dots,g_n\in G$ such that $g_i\cdot A_i = B_{\alpha,i}$ for each $i$, where $A\sim_{G}B_{\alpha}\subseteq B$. Similarly, there exist $h_1,\dots,h_m\in G$ such that $h_j\cdot B_j= C_{\beta,j}$ for each $j$, where $B\sim_{G}C_{\beta}\subseteq C$.\newline

  We take a refinement of $B$ by taking intersections $B_{\alpha,ij} = B_{\alpha,i}\cap B_j$, with $i=1,\dots,n$ and $j = 1,\dots,m$. We define $C_{\beta,\alpha,ij} = h_j\cdot B_{\alpha,ij}$ for each $j = 1,\dots,m$. Then, $h_jg_i\cdot A_i = C_{\beta,\alpha,ij}$, meaning $A\sim_{G}C_{\beta,\alpha,ij}\subseteq C_{\beta}\subseteq C$, so $A\preceq_{G}C$.
\end{proof}

We know from Fact \ref{fact:bijections} that $A\preceq_{G}B$ implies the existence of a bijection $\phi\colon A\rightarrow B'\subseteq B$, meaning $\phi\colon A\hookrightarrow B$ is an injection. Similarly, if $B\preceq_{G}A$, then Fact \ref{fact:bijections} implies the existence of an injection $\psi\colon B\hookrightarrow A$.\newline

One may ask if an analogue of the Cantor--Schröder--Bernstein theorem exists in the case of the relation $\preceq_{G}$, implying that the preorder established in Fact \ref{fact:preorder} is indeed a partial order. The following theorem establishes this result.
\begin{theorem}
  Let $G$ act on $X$, and let $A,B\subseteq X$. If $A\preceq_{G}B$ and $B\preceq_{G}A$, then $A\sim_{G}B$.\label{thm:csb_for_equidecomposability}
\end{theorem}
\begin{proof}
  Let $B'\subseteq B$ with $A\sim_{G}B'$, and let $A'\subseteq A$ with $B\sim_{G}A'$. Then, we know from Fact \ref{fact:bijections} that there exist bijections $\phi\colon A\rightarrow B'$ and $\psi\colon B\rightarrow A'$.\newline

  Define $C_0 = A\setminus A'$, and $C_{n+1} = \psi\left(\phi\left(C_n\right)\right)$. We set
  \begin{align*}
    C &= \bigcup_{n\geq 0}C_{n}.
  \end{align*}
  Since $\psi^{-1}\left(\psi\left(\phi\left(C_n\right)\right)\right) = \phi\left(C_n\right)$, we have
  \begin{align*}
    \psi^{-1}\left(A\setminus C\right) &= B\setminus \phi(C).
  \end{align*}
  Having established in Fact \ref{fact:bijections} that for any subset of $C\subseteq A$, $C\sim_{G} \phi(C)$, we also see that $A\setminus C \sim_{G} B\setminus \phi(C)$.\newline

  Thus, we can see that
  \begin{align*}
    A &= \left(A\setminus C\right)\sqcup C\\
      &\sim_{G}\left(B\setminus \phi(C)\right)\sqcup \phi(C)\\
      &= B.
  \end{align*}
\end{proof}

Finally, we are able to prove Proposition \ref{prop:banachtarski}. We restate the proposition here, followed by its proof.
\begin{tcolorbox}[blanker,breakable,left=3mm,before skip=10pt, after skip=10pt, borderline west={1pt}{0pt}{blue!50!white},sharp corners,]
\banachtarski*
\end{tcolorbox}
\begin{proof}[Proof of Proposition \ref{prop:banachtarski}:]
  By symmetry, it is enough to show that $A\preceq_{\text{E}(3)} B$.\newline

  Since $A$ is bounded, there exists $r > 0$ such that $A\subseteq B(0,r)$.\newline

  Let $x_0\in B^{\circ}$. Then, there exists $\ve > 0$ such that $B\left(x_0,\ve\right) \subseteq B$.\newline

  Since $B(0,r)$ is compact (hence totally bounded), there are translations $g_1,\dots,g_n$ such that
  \begin{align*}
    B\left(0,r\right) \subseteq g_1\cdot B\left(x_0,\ve\right) \cup \cdots \cup g_n\cdot B\left(x_0,\ve\right).
  \end{align*}
  We select translations $h_1,\dots,h_n$ such that $h_j\cdot B\left(x_0,\ve\right) \cap h_k\cdot B\left(x_0,\ve\right) = \emptyset$ for $j\neq k$. We set
  \begin{align*}
    S &= \bigcup_{j=1}^{n}h_j\cdot B\left(x_0,\ve\right).
  \end{align*}
  Each $h_j\cdot B\left(x_0,\ve\right)\subseteq S$ is $\text{E}(3)$-equidecomposable with any arbitrary closed ball subset of $B\left(x_0,\ve\right)$, it is the case that $S\preceq B\left(x_0,\ve\right)$.\newline

  Thus, we have
  \begin{align*}
    A &\subseteq B\left(0,r\right)\\
      &\subseteq g_1\cdot B\left(x_0,\ve\right)\cup\cdots\cup b_n\cdot B\left(x_0,\ve\right)\\
      &\preceq S\\
      &\preceq B\left(x_0,\ve\right)\\
      &\preceq B.
  \end{align*}
\end{proof}

