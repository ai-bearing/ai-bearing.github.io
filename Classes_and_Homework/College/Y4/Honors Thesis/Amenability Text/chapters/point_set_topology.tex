\section{Ordering, the Axiom of Choice, and Zorn's Lemma}%
\begin{definition}[Preorders, Partial Orders, Total Orders, and Well-Orders]
Let $X$ be a set, and $\preceq $ be a relation on $X$. We say a relation is a preorder if it is reflexive and transitive:
\begin{itemize}
  \item $a\preceq a$
  \item $a\preceq b \wedge b\leq c\Rightarrow a\preceq c$.
\end{itemize}
We say $X$ is a directed set if, for any $a,b\in X$, there is $c\in X$ such that $a\preceq c$ and $b\preceq c$.\newline

If $\preceq$ is also antisymmetric --- that is, $a\preceq b\wedge b\preceq a \Rightarrow a = b$ --- then, we say $\preceq$ is a partial order.\newline

We say $m\in X$ is a maximal element if, for any $x\in X$ with $m\preceq x$, $m = x$.\newline

If $X$ is partially ordered by $\preceq$ and, for any two elements $a,b\in X$, either $a\preceq b$ or $b\preceq a$, then we say $\preceq$ is a total order on $X$.\newline

If $X$ is a totally ordered set that has the property that, for any nonempty $A\subseteq X$, there is some $x\in A$ such that for any $y\in A$, $x\prec y$ for all $y \in A$ with $y\neq x$, then we say $\preceq$ is a well-order on $X$.
\end{definition}
\begin{example}\hfill
  \begin{itemize}
    \item The set $\N$ with the usual ordering is a well-ordered set.
    \item If $A$ is a set, then $P(A)$ with the ordering $A\preceq B$ if $A\supseteq B$ is a partially ordered set. This is known as the containment ordering.
    \item Similarly, if $A$ is a set, then $P(A)$ with the ordering $A\preceq B$ if $A\subseteq B$ is a partially ordered set. This is known as the inclusion ordering.
    \item A collection of functions $\set{\varphi_{i}\colon Z_i\rightarrow Y}_{i\in I}$ ordered by $\varphi_{i}\preceq \varphi_j$ if $Z_i\subseteq Z_j$ and $\varphi_{j}|_{Z_i} = \varphi_i$, is a partially ordered set. This is often known as the extension ordering.
  \end{itemize}
\end{example}
The axiom of choice, stated below, is a load-bearing part of topology and analysis.
\begin{definition}
  Let $\mathcal{A} = \set{A_i}_{i\in I}$ be an indexed collection of sets. There exists an indexed set $\set{x_i}_{i\in I}$ such that $x_i\in S_i$ for each $i\in I$.
\end{definition}
Using some of the basic results in order theory, we may state an equivalent formulation of the Axiom of Choice that is also more readily used to prove important results in analysis.
\begin{theorem}[Zorn's Lemma]
  If $\left(X,\preceq\right)$ is a nonempty partially ordered set with the property that for all $C\subseteq X$ with $C$ totally ordered, $C$ has an upper bound, then $X$ has a maximal element.
\end{theorem}
Zorn's lemma can be used to prove the following theorems.
\begin{example}\hfill
  \begin{itemize}
    \item Every $\F$-vector space $V$ has a basis $B\subseteq V$ such that the set of all finite linear combinations of elements of $B$ over $\F$ is $V$.
    \item If $\varphi$ is a continuous linear functional defined on a subspace $W\subseteq V$, there is an extension $\Phi$ such that $\Phi|_{W} = \varphi$. This is one of the Hahn--Banach theorems. %See: Hahn--Banach Theorems
    \item The arbitrary product of compact spaces is compact. This is known as Tychonoff's Theorem (Theorem \ref{thm:tychonoff}). %See Tychonoff's Theorem.
  \end{itemize}
\end{example}
\section{Metric Spaces}%
Building upon the basics of sets and orders, we move towards understanding metric spaces.
\subsection{Basics of Metric Spaces}%
\begin{definition}[Metrics]\label{def:metrics_and_equivalent_metrics}
  Let $X$ be a set. A distance metric is a function
  \begin{align*}
    d\colon X\times X\rightarrow [0,\infty)
  \end{align*}
  such that the following three properties are satisfied:
  \begin{itemize}
    \item if $x,y\in X$ and $d\left(x,y\right) = 0$, then $x = y$;
    \item $d\left(x,y\right) = d\left(y,x\right)$ for all $x,y\in X$;
    \item $d\left(x,z\right) \leq d\left(x,y\right) + d\left(y,z\right)$ for all $x,y,z\in X$.
  \end{itemize}
  A function that satisfies the latter two properties is called a semimetric.\newline

  Two metrics $d$ and $\rho$ on $X$ are equivalent if there exist constants $c_1,c_2\geq 0$ such that
  \begin{align*}
    d\left(x,y\right) &\leq c_1 \rho\left(x,y\right)\\
    \rho\left(x,y\right) &\leq c_2 d\left(x,y\right)
  \end{align*}
  for all $x,y\in X$.\newline

  A metric space is a pair $\left(X,d\right)$, where $d$ is a metric.
\end{definition}
\begin{example}[Some Distance Metrics]\hfill
  \begin{itemize}
    \item The discrete metric on any nonempty set is given by
      \begin{align*}
        d\left(x,y\right) & \begin{cases}
          1 & x\neq y\\
          0 & x = y
        \end{cases}
      \end{align*}
    \item The Euclidean metric between $\left(x_1,\dots,x_n\right)$ and $\left(y_1,\dots,y_n\right)$ in $\R^n$ is
      \begin{align*}
        d_{2}\left(x,y\right) &= \left(\sum_{j=1}^{n}\left\vert y_j-x_j \right\vert^2\right)^{1/2}.
      \end{align*}
    \item Other metrics on $\R^n$ include
      \begin{align*}
        d_1\left(x,y\right) &= \sum_{j=1}^{n}\left\vert y_j-x_j \right\vert\\
        d_{\infty}\left(x,y\right) &= \max_{j=1}^{n}\left\vert y_j - x_j \right\vert.
      \end{align*}
      All of $d_1,d_2,d_{\infty}$ are equivalent metrics.
    \item The Hamming distance between two strings of bits is
      \begin{align*}
        d_{H}\colon \set{0,1}^{n}\times \set{0,1}^{n}\rightarrow [0,\infty)\\
        d_{H}\left(\left(x_{j}\right)_{j=1}^{n},\left(y_j\right)_{j=1}^{n}\right) &= \left\vert \set{j\mid x_j\neq y_j} \right\vert.
      \end{align*}
    \item The set $C\left([0,1],\R\right)$ consisting of continuous real-valued functions from $[0,1]$ to $\R$ can be equipped with
      \begin{align*}
        d_u\left(f,g\right) &= \sup_{t\in [0,1]}\left\vert f(t) - g(t) \right\vert,
      \end{align*}
      which is the uniform metric, or
      \begin{align*}
        d_{1}\left(f,g\right) &= \int_{0}^{1} \left\vert f(t)-g(t) \right\vert\:dt.
      \end{align*}
    \item All subsets of a metric space $X$ equipped with the same metric is also a metric space.
    \item If $\rho$ is a metric on $X$, then we can create a distance metric
      \begin{align*}
        d\left(x,y\right) &= \frac{\rho\left(x,y\right)}{1 + \rho\left(x,y\right)}
      \end{align*}
      that is bounded on $[0,1]$.
    \item If $d_1,\dots,d_n$ are metrics on $X$ and $c_1,\dots,c_n > 0$ are constants, then
      \begin{align*}
        d\left(x,y\right) &= \sum_{k=1}^{n}c_kd_k\left(x,y\right)
      \end{align*}
      defines a metric on $X$.
    \item If $\left(\rho_k\right)_k$ is a family of separating semimetrics for $X$ --- i.e., for $x,y\in X$ distinct, there is some $\rho_{j}$ such that $\rho_j\left(x,y\right) \neq 0$ --- then, we can obtain bounded semimetrics by taking
      \begin{align*}
        d_k\left(x,y\right) &= \frac{\rho_k\left(x,y\right)}{1 + \rho_k\left(x,y\right)}
      \end{align*}
      for each $k$. From each $d_k$, we define
      \begin{align*}
        d\left(x,y\right) &= \sum_{k=1}^{n}2^{-k}d_k\left(x,y\right),
      \end{align*}
      which is a metric on $X$.
    \item If $\left(X_k,\rho_k\right)_{k\geq 1}$ is a sequence of metric spaces, then we can form the product space
      \begin{align*}
        X &= \prod_{k\geq 1}X_{k}
      \end{align*}
      with the metric
      \begin{align*}
        D\left(f,g\right) &= \sum_{k\geq 1}d_k\left(f(k),g(k)\right).
      \end{align*}
      Here, $d_k = \frac{\rho_k}{1 + \rho_k}$ is the corresponding bounded metric to $\rho_k$.
  \end{itemize}
\end{example}
\begin{definition}[Open and Closed Sets]
  Let $\left(X,d\right)$ be a metric space.
  \begin{enumerate}[(1)]
    \item For $x\in X$ and $\delta > 0$, we define
      \begin{enumerate}[(a)]
        \item the open ball at $x$ with radius $\delta > 0$
          \begin{align*}
            U\left(x,\delta\right) &= \set{y\in X\mid d\left(y,x\right) < \delta};
          \end{align*}
        \item the closed ball centered at $x$ with radius $\delta > 0$
          \begin{align*}
            B\left(x,\delta\right) &= \set{y\in X\mid d\left(y,x\right)\leq \delta};
          \end{align*}
        \item the sphere centered at $x$ with radius $\delta > 0$
          \begin{align*}
            S\left(x,\delta\right) &= \set{y\in X\mid d\left(y,x\right) = \delta}.
          \end{align*}
      \end{enumerate}
    \item A set $V\subseteq X$ is open if, for all $x\in V$, there is $\delta > 0$ such that $U\left(x,\delta\right)\subseteq V$.\newline

      A subset $C\subseteq X$ is closed if $C^{c}$ is open.
    \item If $x\in V$ and $V\subseteq X$ is open, then we say $V$ is an open neighborhood of $x$. A neighborhood of $x$ is any subset $N\subseteq X$ such that $N$ contains an open neighborhood of $x$.
    \item If $A\subseteq X$ is any subset, the interior of $A$ is
      \begin{align*}
        A^{\circ} &:= \bigcup\set{V\mid V\text{ is open, }V\subseteq A},
      \end{align*}
      the closure of $A$ is
      \begin{align*}
        \overline{A} &= \bigcap\set{C\mid C\text{ is closed, }A\subseteq C},
      \end{align*}
      and the boundary of $A$ is
      \begin{align*}
        \partial A &= \overline{A} \setminus A^{\circ}.
      \end{align*}
  \end{enumerate}
\end{definition}
We can now talk about the topology of the metric space.
\begin{fact}
  Let $\left(X,d\right)$ be a metric space, and let
  \begin{align*}
    \mathcal{U} = \set{V\mid V\subseteq X\text{ open}}.
  \end{align*}
  Then, the following are true.
  \begin{itemize}
    \item $\emptyset\in \mathcal{U},X\in \mathcal{U}$.
    \item If $\set{V_{i}}_{i\in I}$ is a family of open sets, then $\bigcup_{i\in I}V_i\in \mathcal{U}$.
    \item If $\set{V_i}_{i=1}^{n}$ is a finite collection of open sets, then $\bigcap_{i=1}^{n}V_i \in \mathcal{U}$.
  \end{itemize}
\end{fact}

\begin{definition}
  Let $\left(X,d\right)$ be a metric space. Suppose $A\subseteq X$ is a nonempty subset.
  \begin{enumerate}[(1)]
    \item The distance from a point $x\in X$ to the set $A$ is defined by
      \begin{align*}
        \dist_{A}\left(x\right) &= \inf_{a\in A}d\left(x,a\right).
      \end{align*}
    \item The diameter of $A$ is defined by
      \begin{align*}
        \diam\left(A\right) &= \sup_{x,y\in A}d\left(x,y\right).
      \end{align*}
    \item If $\diam(A) < \infty$, then we say $A$ is bounded.
    \item If, for every $\delta > 0$, there is a finite subset $F_{\delta}\subseteq X$ such that
      \begin{align*}
        A\subseteq \bigcup_{x\in F_{\delta}}U\left(x,\delta\right).
      \end{align*}
    \item For $A,B\subseteq X$, we define the Hausdorff distance between $A$ and $B$ to be
      \begin{align*}
        d_{H}\left(A,B\right) &= \max\set{\sup_{x\in A}\dist_{B}\left(x\right),\sup_{y\in B}\dist_{A}\left(y\right)}.
      \end{align*}
  \end{enumerate}
\end{definition}
\begin{example}
  Let $\Omega$ be a nonempty set, and $\left(X,d\right)$ be a metric space. A function $f\colon \Omega\rightarrow X$ is said to be bounded if $\diam\left(\ran(f)\right) < \infty$.\newline

  The collection $\operatorname{Bd}\left(\Omega,X\right)$ denotes all bounded functions with domain $\Omega$ and codomain $X$.\newline

  On $ \operatorname{Bd}\left(\Omega,X\right)$, we define the uniform metric by
  \begin{align*}
    D_{u}\left(f,g\right) &= \sup_{x\in\Omega}d\left(f(x),g(x)\right).
  \end{align*}
\end{example}
\subsection{Convergence and Continuity in Metric Spaces}%
\begin{definition}
  Let $\left(X,d\right)$ be a metric space.
  \begin{enumerate}[(1)]
    \item A sequence in $X$ is a map $x\colon \N\rightarrow X$, which we call $\left(x_{n}\right)_{n}$ or $\left(x_{n}\right)_{n\geq 1}$.
    \item A natural sequence is a strictly increasing sequence of natural numbers $\left(n_{k}\right)_{k\geq 1}$ with $n_{k}\geq k$ and $n_{k} < n_{k+1}$.
    \item If $\left(n_k\right)_{k}$ is a natural sequence, the sequence $\left(x_{n_k}\right)_{k}$ is called a subsequence of $\left(x_{n}\right)_n$.
    \item We say $\left(x_n\right)_n\rightarrow x$ if $d\left(x_n,x\right)_{n} \xrightarrow{n\rightarrow\infty} 0$. We say $x$ is the limit of $\left(x_n\right)_n$.
  \end{enumerate}
\end{definition}
\begin{example}\hfill
  \begin{itemize}
    \item If $\Omega$ is a nonempty set, and $\left(X,d\right)$ is a metric space, the sequence of functions $f_n\colon \Omega\rightarrow X$ is said to converge pointwise to $f\colon \Omega\rightarrow X$ if
      \begin{align*}
        f_n\left(x\right)\xrightarrow{n\rightarrow\infty}f(x)
      \end{align*}
      for each $x\in \Omega$.
    \item If $\left(f_n\right)_n\in \operatorname{Bd}\left(\Omega,X\right)$ is a sequence, we say $\left(f_n\right)_n\rightarrow f$ converges uniformly if
      \begin{align*}
        D_u\left(f_n,f\right)\xrightarrow{n\rightarrow\infty}0,
      \end{align*}
      or, equivalently,
      \begin{align*}
        \sup_{x\in\Omega}d\left(f_n(x),f(x)\right)\xrightarrow{n\rightarrow\infty}0.
      \end{align*}
  \end{itemize}
\end{example}
\begin{definition}[Sequential Criteria for Closure]
  If $\left(X,d\right)$ is a metric space, and $E\subseteq X$ is nonempty, then $E$ is closed if and only if, for all $\left(x_n\right)_n\rightarrow x$ with $x_n\in E$, $x\in E$.\newline

  If $E\subseteq X$ is any nonempty set, then $\overline{E}$ is precisely the set of all $x\in X$ such that $\left(x_n\right)_n\rightarrow x$ for some $\left(x_n\right)_n\subseteq E$.
\end{definition}
\begin{definition}[Completeness]
  Let $\left(X,d\right)$ be a metric space.
  \begin{itemize}
    \item If $\left(x_n\right)_n$ is a sequence in $X$ such that for all $\ve > 0$, there is $N\in \N$ such that for all $m,n\geq N$, $d\left(x_m,x_n\right) < \ve$, then we say the sequence is called Cauchy.
    \item If, for any $\left(x_n\right)_n$ Cauchy, $\left(x_n\right)_n\rightarrow x$ in $X$, then we say $X$ is complete.
    \item If $\left(X,d\right)$ is complete, then for any $A\subseteq X$ closed, $A$ is also complete.
    \item If $A\subseteq X$ is complete as a metric space, then $A$ is closed.
  \end{itemize}
\end{definition}
\begin{example}
  The metric space $\Q$ with the metric inherited from $\R$ is not complete. For instance, there is a sequence of rational numbers $\left(2,2.7,2.71,2.718,\dots\right)$ converging to $e$, but $e\notin \Q$.\newline

  The space $\operatorname{Bd}\left(\Omega,X\right)$ is complete if $X$ is complete.
\end{example}
\begin{definition}[Continuity]\hfill
  \begin{itemize}
    \item Let $\left(X,d\right)$ and $\left(Y,\rho\right)$ be metric spaces, and let $f\colon X\rightarrow Y$ be a function. We say $f$ is continuous at $x$ if, for every $\ve > 0$, there is $\delta > 0$ such that $z\in U\left(x,\delta\right)\Rightarrow \rho\left(f(x),f(z)\right) < \ve$.
    \item If $f$ is continuous at every point in $X$, then we say $f$ is continuous.
    \item If $f$ is bijective, continuous, and $f^{-1}$ is continuous, then we say $f$ is a homeomorphism.
    \item We say $f$ is uniformly continuous on $X$ if, for any $\ve > 0$, there is $\delta > 0$ such that for any $y,z\in X$, $d\left(y,z\right) < \delta \Rightarrow \rho\left(f(y),f(z)\right) < \ve$.
    \item We say $f$ is Lipschitz if there exists $C > 0$ such that $d\left(x,y\right) \leq Cd\left(f(x),f(y)\right)$ for all $x,y\in X$.
    \item We say $f$ is an isometry if $d\left(x,y\right) = d\left(f(x),f(y)\right)$ for all $x,y\in X$.
  \end{itemize}
\end{definition}
\begin{fact}
  Let $f\colon X\rightarrow Y$ be a map between metric spaces. The following are equivalent:
  \begin{enumerate}[(i)]
    \item $f$ is continuous;
    \item if $V\subseteq Y$ is open, then $f^{-1}\left(V\right)\subseteq X$ is open;
    \item if $\left(x_n\right)_n\rightarrow x$ in $X$, then $\left(f\left(x_n\right)\right)_n\rightarrow f(x)$ in $Y$.
  \end{enumerate}
\end{fact}
\begin{fact}
  If $M$ and $N$ are metric spaces with $N$ complete, and $A\subseteq M$ is dense, then if $f\colon A\rightarrow N$ is uniformly continuous, then there is a unique uniformly continuous map $\tilde{f}\colon M\rightarrow N$.
\end{fact}
\begin{definition}
  Let $\left(X,d\right)$ and $\left(Y,\rho\right)$ be metric spaces.
  \begin{enumerate}[(1)]
    \item We say $X$ and $Y$ are homeomorphic if there is a homeomorphism $f\colon X\rightarrow Y$.
    \item We say $X$ and $Y$ are uniformly isomorphic if there is a uniformly continuous bijection $f\colon X\rightarrow Y$ with $f^{-1}$ uniformly continuous. Such an $f$ is called a metric space uniformism.
    \item We say $X$ and $Y$ are isometrically isomorphic if there is a bijective isometry $f\colon M\rightarrow N$.
  \end{enumerate}
\end{definition}
\begin{fact}
  If $X$ and $Y$ are uniformly isomorphic metric spaces with $X$ complete, then so too is $Y$.\newline

  If $d$ and $\rho$ are equivalent metrics on a set $X$, then the identity map
  \begin{align*}
    \id_{X}:\left(X,\rho\right)\rightarrow \left(X,d\right)
  \end{align*}
  is a metric space uniformism.
\end{fact}
\section{Topological Spaces}%
We can now move from metric spaces to the more general setting of topological spaces. This will enable us to understand certain properties (like openness, continuity, etc.) separate from the metric structure (or lack thereof) that a certain set is endowed.
\subsection{Definitions}%
\begin{definition}
  Let $X$ be a nonempty set. A topology on $X$ is a family of subsets $\tau$ satisfying
  \begin{enumerate}[(1)]
    \item $\emptyset\in \tau,X\in \tau$;
    \item if $\set{V_i}_{i\in I}\subseteq \tau$, then $\bigcup_{i\in I}V_i\in \tau$;
    \item if $\set{V_i}_{i=1}^{n}\subseteq \tau$, then $\bigcap_{i=1}^{n}V_i \in \tau$.
  \end{enumerate}
  If $\tau$ is a topology on $X$, then $\left(X,\tau\right)$ is called a topological space. We call members of $\tau$ open sets.\newline

  If $C\subseteq X$ and $C^{c}\in \tau$, then $C$ is called.\newline

  If $E$ is closed and open, it is called clopen.\newline

  A countable union of closed sets is called an $F_{\sigma}$ set, and a countable intersection of open sets is called a $G_{\delta}$ set.
\end{definition}
\begin{definition}
  If $X$ is a nonempty set, then the definition $\tau = P(X)$ is known as the discrete topology.\newline

  If $X$ is a nonempty set, and $\tau = \set{X,\emptyset}$, then we call $\tau$ the indiscrete topology.
\end{definition}
\begin{definition}
  Let $X$ be a nonempty set. Suppose $\tau_1,\tau_2\subseteq P(X)$ are two topologies on $X$. If $\tau_1\subseteq \tau_2$, then we say $\tau_1$ is weaker (or coarser) than $\tau_2$. We say $\tau_2$ is stronger (or finer) than $\tau_1$.
\end{definition}
\begin{definition}
  Let $X$ be a nonempty set, and suppose $\mathcal{E}\subseteq P(X)$ is a family of subsets. We define the topology on $X$ generated by $\mathcal{E}$ to be
  \begin{align*}
    \tau\left(\mathcal{E}\right) &= \bigcap\set{\tau | \tau\text{ is a topology on $X$, }\mathcal{E}\subseteq \tau}.
  \end{align*}
  In other words, $\tau\left(\mathcal{E}\right)$ is the weakest topology that contains the family $\mathcal{E}$.
\end{definition}
\begin{definition}
Let $\left(X,\tau\right)$ be a topological space. If $Y\subseteq X$ is a subset, then the subspace topology on $Y$ is defined by
\begin{align*}
  \tau_{Y} &= \set{V\cap Y | V\in \tau}.
\end{align*}

\end{definition}
\begin{definition}
  Let $\left(X,\tau\right)$ be a topological space, and let $A\subseteq X$ be a subset.
  \begin{enumerate}[(1)]
    \item The interior of $A$ is the open set $A^{\circ} = \bigcup\set{V | V\in\tau,~V\subseteq A}$.
    \item The closure of $A$ is the closed set $\overline{A} = \bigcap\set{C| C\text{ closed, }A\subseteq C}$.
    \item We say $A$ is dense if $\overline{A} = X$.
    \item We say $A$ is nowhere dense if $\left(\overline{A}\right)^{\circ} = \emptyset$.
  \end{enumerate}
  If $X$ admits a countable dense subset, then we say $X$ is separable.\newline

  If $X$ is the countable union of nowhere dense subsets, then we say $X$ is meager.
\end{definition}
\begin{remark}
  A set $A$ is dense if and only if, for any $U\in \tau$ with $U\neq \emptyset$, it is the case that $A\cap U \neq \emptyset$.
\end{remark}
\begin{fact}
  If $\left(M,d\right)$ is a separable metric space, and $E\subseteq M$ is a subset, then $E$ with the subspace topology is also separable.
\end{fact}
\begin{definition}
  Let $\left(X,\tau\right)$ be a topological space.
  \begin{itemize}
    \item An open neighborhood of $x_0$ is an open set $V\in \tau$ with $x_0\in V$. We write
      \begin{align*}
        \mathcal{O}_{x_0} &= \set{V | V\in \tau,x_0\in V}
      \end{align*}
      to denote the family of all open neighborhoods of $x_0$.
    \item If $N\subseteq X$ is a subset with $x_0\in V\subseteq N$, where $V\in \mathcal{O}_{x_0}$, then we say $N$ is a neighborhood of $x_0$. We write $\mathcal{N}_{x_0}$ to be the collection of neighborhoods of $x_0$.
    \item A neighborhood base for $\tau$ at $x_0$ is a family $\mathcal{O}\subseteq \mathcal{O}_{x_0}$ with such that for all $U\in \mathcal{O}_{x_0}$, there is $V\in \mathcal{O}$ with $V \subseteq U$.
    \item We say $\left(X,\tau\right)$ is first countable if every $x\in X$ admits a countable neighborhood base.
    \item A base for $\tau$ is a family $\mathcal{B}\subseteq \tau$ that contains a neighborhood base for $\tau$ at $x_0$ For each $x_0\in X$.
    \item We say $\left(X,\tau\right)$ is second countable if it admits a countable base.
  \end{itemize}
\end{definition}
\begin{fact}
  If $\mathcal{B}$ is a base for $\tau$, then every $U\in \tau$ can be written as a union $U= \bigcup_{i\in I}B_i$, where $B_i\in \mathcal{B}$.
\end{fact}
\begin{fact}
  All metric spaces are first-countable, with a neighborhood base of
  \begin{align*}
    \mathcal{O}_{x_0} &= \set{U\left(x_0,1/n\right)|n\in \N}
  \end{align*}
  for each $x_0\in X$.
\end{fact}
\begin{fact}
  A metric space $\left(X,d\right)$ is second countable if and only if it is separable.
\end{fact}
\begin{fact}
  If $X$ is a topological space, and $x_0\in X$ has a countable neighborhood base, then there is a neighborhood base $\left(V_n\right)_{n\geq 1}$ with $V_1\supseteq V_2\supseteq \cdots$.
\end{fact}
\subsection{Continuity in Topological Spaces}%
\begin{definition}
  Let $\left(X,\tau\right)$ and $\left(Y,\sigma\right)$ be topological spaces, and let $f\colon X\rightarrow Y$ be a map.
  \begin{enumerate}[(1)]
    \item We say $f$ is continuous at $x_0\in X$ if, for every $U\in \mathcal{O}_{f\left(x_0\right)}$, there is $V\in \mathcal{O}_{x}$ with $f\left(V\right)\subseteq U$.
    \item We say $f$ is continuous if $f$ is continuous at every point in $X$.
    \item We say $f$ is a homeomorphism if $f$ is a continuous bijection with a continuous inverse.
    \item We say $f$ is an open map if $U\in \tau$ implies $f\left(U\right)\in \sigma$. Similarly, we say $f$ is a closed map if $C\subseteq X$ closed implies $f\left(C\right)\subseteq Y$ is closed.
    \item We say $f$ is a quotient map if $f$ is surjective with $V\subseteq Y$ open if and only if $f^{-1}\left(V\right)\subseteq X$ open.
    \item We say $f$ is an embedding if $f\colon X\rightarrow \Ran\left(f\right)$ is a homeomorphism, where $\Ran\left(f\right)$ is endowed with the subspace topology.
    \item We write $C\left(X,Y\right)$ to be the continuous functions from $X$ to $Y$. If $Y = \C$ with the regular topology, then we write $C\left(X\right)$.
  \end{enumerate}
\end{definition}
\begin{fact}
  A function $f\colon X\rightarrow Y$ is continuous if and only if $f^{-1}\left(U\right)\subseteq X$ is open for every open $U\subseteq Y$. Equivalently, $f$ is continuous if and only if $f^{-1}\left(C\right)\subseteq X$ is closed for every closed $C\subseteq Y$.
\end{fact}
\begin{definition}[Separation Axioms]
Let $\left(X,\tau\right)$ be a topological space.
\begin{itemize}
  \item We say $X$ is T1 if $\set{x}$ is closed for every $x\in X$.
  \item We say $X$ is T2 (or Hausdorff) if, for every $x,y\in X$ with $x\neq y$, there are $U,V\in \tau$ with $x\in U$, $y\in V$, and $U\cap V = \emptyset$.
  \item We say $X$ is T3 if, for every $x\in X$ and $B\subseteq X$ closed with $x\notin B$, there are $U,V\in \tau$ with $x\in U$, $B\subseteq V$, and $U\cap V = \emptyset$. If $X$ is T1 and T3, we say $X$ is regular.
  \item We say $X$ is T3.5 if, for every $x_0\in X$ and closed $B\subseteq X$ with $x_0\notin B$, there is a continuous function $f\colon X\rightarrow [0,1]$ with $f\left(x_0\right) = 0$ and $f\left(B\right) = 1$. If $X$ is T1 and T3.5, we say $X$ is completely regular.
  \item We say $X$ is T4 if, for every pair of closed subsets $A,B\subseteq X$ with $A\cap B = \emptyset$, there are $U,V\in \tau$ with $A\subseteq U$, $B\subseteq V$, and $U\cap V = \emptyset$. If $X$ is T1 and T4, then we say $X$ is normal.
\end{itemize}
\end{definition}
Just as we defined completely regular spaces through the existence of certain continuous functions that act to separate points, we can completely classify normality through a separating family of continuous functions.
\begin{theorem}[Urysohn's Lemma]
  Let $\left(X,\tau\right)$ be a topological space. It is the case that $X$ is normal if and only if for every pair of disjoint closed subsets $A,B\subseteq X$, there is a continuous function $f\colon X\rightarrow [0,1]$ with $f\left(A\right) = 0$ and $f\left(B\right) = 1$.
\end{theorem}
\begin{remark}
  Metric spaces are an example of normal spaces.
\end{remark}
\subsection{Initial and Final Topologies}%
\begin{definition}
  Let $X$ be a set, and suppose $\set{\left(Y_i,\tau_i\right)}_{i\in I}$ is a family of topological spaces. Let $\set{f_i\colon X\rightarrow Y_i}$ be a family of maps. Setting
  \begin{align*}
    \ve &= \set{f_i^{-1}\left(V\right) | V_i\in \tau_i},
  \end{align*}
  and letting $\tau = \tau\left(\ve\right)$ be the topology on $X$ generated by $\ve$, we say $\tau$ is the initial topology on $X$ induced by the maps $\set{f_i}_{i\in I}$.\newline

  Specifically, $\tau$ is the weakest topology on $X$ such that each $f_i$ is continuous.
\end{definition}
\begin{definition}[Product Topology]
  Let $\set{\left(X_i,\tau_i\right)}_{i\in I}$ be a family of topological spaces. The topology on the product $\prod_{i\in I}X_i$ is defined to be the initial topology induced by the family of projection maps,
  \begin{align*}
    \pi_j\colon \prod_{i\in I}X_i\rightarrow X_j,
  \end{align*}
  defined by $\pi_j\left(\left(x_i\right)_{i\in I}\right) = x_j$.\newline
  
  For each $U\subseteq X_i$ open, we have $\pi_j^{-1}\left(U\right) = \prod_{i\in I}U_I$, where $U_i = X_i$ for $i\neq j$, and $U_j = U$. A base for this topology is the collection
  \begin{align*}
    \mathcal{B} = \set{\prod_{i\in I}U_i | U_i = X_i\text{ for all but finitely many open }U_i\subseteq X_i}.
  \end{align*}
  If we consider $X_i = X$ for all $i$, there is a bijection between $X^I \coloneq \set{f | f\colon I\rightarrow X}$, the set of all functions from $I$ to $X$, and $\prod_{i\in I}X_i$, with the map $f \mapsto \left(f\left(i\right)\right)_{i\in I}$. The product topology on $X^I$ coincides with the topology of pointwise convergence.
\end{definition}
\begin{definition}[Final Topology]
  Let $\left(X,\tau\right)$ be a topological space, $Y$ a nonempty set, and suppose $q\colon X\rightarrow Y$ is a surjection. Then, the collection
  \begin{align*}
    \tau_q &\coloneq \set{V\subseteq Y | q^{-1}\left(V\right)\in \tau}
  \end{align*}
  is what is known as the final (or quotient) topology on $Y$ produced by $q$.
\end{definition}
\subsection{Convergence in Topological Spaces}%
Given a non-first-countable space $X$ and a subset $A\subseteq X$, it is not necessarily the case that $x\in \overline{A}$ is the limit of a sequence $\left(x_n\right)_n$. However, we know that the sequential characterization of properties like closure, compactness (which will be covered in an upcoming section), and continuity is useful, so we want to generalize these ideas to non-first-countable spaces. This is where we can use nets.
\begin{definition}[Nets]
  A net is a map $A\rightarrow X$, where $\alpha \mapsto x_{\alpha}$, where $A$ is a directed set. We write nets as $\left(x_{\alpha}\right)_{\alpha}$.
\end{definition}
\begin{example}[Some Directed Sets]\hfill
  \begin{enumerate}[(1)]
    \item The natural numbers, $\N$, or the real numbers, $\R$, equipped with their usual ordering, are examples of directed sets. Every totally ordered set is directed.
    \item If $S$ is any set, the collection $F(S)$ consisting of all finite subsets of $S$ is directed by inclusion.
    \item The collection of finite partitions over a closed and bounded interval, $\mathcal{P}\left(\left[a,b\right]\right)$ is by the partition norm. If $P = \set{x_j}_{j=0}^{n}$ and $Q=\set{y_j}_{j=0}^{m}$ are partitions, we define
      \begin{align*}
        \norm{P} &= \max_{1\leq j\leq n}\left\vert x_j-x_{j-1} \right\vert\\
        \norm{Q} &= \max_{1\leq j \leq m}\left\vert y_{j} - y_{j-1} \right\vert,
      \end{align*}
      and the preorder that $P \leq Q$ if and only if $\norm{P} \geq \norm{Q}$. In other words, we say $P\leq Q$ if $Q$ is finer than $P$.\newline

      For any partitions $P$ and $Q$, their common refinement is a supremum for both --- $P\vee Q \geq P,Q$ for each partition.\footnote{This is extremely useful in defining the Riemann integral.}
    \item Let $\left(X,\tau\right)$ be a topological space, and for every $x$, we order the $\mathcal{O}_{x}$ by containment. That is, for elements $U,V\in \mathcal{O}_{x}$, we set $U\leq V$ if and only if $U\supseteq V$. This is a directed ordering, as we can always take $U\cap V \subseteq U,V$ (since both $U$ and $V$ contain $x$). Similarly, the neighborhood system at $x$, $\mathcal{N}_x$, is also directed by containment.
    \item If $A$ and $B$ are directed sets, then $A\times B$ with the Cartesian ordering --- $\left(\alpha_1,\beta 1\right) \leq \left(\alpha_2,\beta 2\right)$ if and only if $\alpha_1\leq \alpha_2$ and $\beta_1 \leq \beta_2$ --- is also a directed set.
  \end{enumerate}
\end{example}
\begin{example}[Some Nets]\hfill
\begin{enumerate}[(1)]
  \item Any sequence $\left(x_k\right)_{k\in \N}$ is a net.
  \item Let $F\left(\Omega\right)$ be the set of all finite subsets of $\Omega$ directed by inclusion. Let $f\colon \Omega\rightarrow \C$ be a map. Then, we have a net $\left(s_F\right)_{F\in F\left(\Omega\right)}$ defined by
    \begin{align*}
      s_F &= \sum_{x\in F}f\left(x\right).
    \end{align*}
  \item Consider the collection of partitions $\mathcal{P}\left([a,b]\right)$ directed by the partition norm. For a bounded function $f\colon [a,b]\rightarrow \R$ and a partition $P = \set{x_j}_{j=0}^{n}$, for each $j$ we set
    \begin{align*}
      M_j\left(P\right) &= \sup_{t\in \left[x_j,x_{j-1}\right]}f(t)\\
      m_j\left(P\right) &= \inf_{t\in \left[x_j,x_{j-1}\right]}f(t).
    \end{align*}
    We obtain two nets, $U,L\colon \mathcal{P}\left([a,b]\right)$, defined by
    \begin{align*}
      U\left(P\right) &= \sum_{j=1}^{n}M_j\left(P\right)\left(x_j - x_{j-1}\right)\\
      L\left(P\right) &= \sum_{j=1}^{n}m_j\left(P\right)\left(x_j-x_{j-1}\right).
    \end{align*}
    These are known as the upper and lower Darboux sums.
\end{enumerate}
\end{example}
\begin{definition}
  Let $\left(X,\tau\right)$ be a topological space, and let $\left(x_{\alpha}\right)_{\alpha}$ be a net in $X$.
  \begin{enumerate}[(1)]
    \item For a set $U\subseteq X$, we say $\left(x_{\alpha}\right)_{\alpha}$ is eventually in $U$ if there is $\alpha_0\in A$ such that $x_{\alpha}\in U$ for all $\alpha \geq \alpha_0$.
    \item We say the net $\left(x_{\alpha}\right)_{\alpha}$ converges to $x\in X$ if, for every $U\in \mathcal{O}_{x}$, $\left(x_{\alpha}\right)_{\alpha}$ is eventually in $U$. We write $\left(x_{\alpha}\right)_{\alpha}\xrightarrow{\tau}x$, though if the topology is clear from context the $\tau$ is not written.
    \item For a given $U\subseteq X$, we say $\left(x_{\alpha}\right)_{\alpha}$ is frequently in $U$ if for every $\beta \in A$, there is $\alpha \in A$ with $\alpha \geq \beta$ and $x_{\alpha}\in U$.
    \item A point $x\in X$ is known as a cluster point of the net $\left(x_{\alpha}\right)_{\alpha}$ if for every $U\in \mathcal{O}_{x}$, $\left(x_{\alpha}\right)_{\alpha}$ is frequently in $U$. That is, for all $U\in \mathcal{O}_{x}$ and for all $\beta \in A$, there exists $\alpha \in A$ with $\alpha \geq \beta$ and $x_{\alpha}\in U$.
  \end{enumerate}
\end{definition}
\begin{fact}[Characterizations Using Nets]
  Let $\left(X,\tau\right)$ and $\left(Y,\sigma\right)$ be topological spaces, $E\subseteq X$ a subset, and $f\colon X\rightarrow Y$ a map.
  \begin{itemize}
    \item It is the case that $x\in \overline{E}$ if and only if there is a net $\left(x_{\alpha}\right)_{\alpha}$ in $E$ with $\left(x_{\alpha}\right)_{\alpha}\rightarrow x$.
    \item A map $f$ is continuous if and only if for every convergent net $\left(x_{\alpha}\right)_{\alpha}\xrightarrow{\tau}x$, we have $\left(f\left(x_{\alpha}\right)\right)_{\alpha}\xrightarrow{\sigma}f(x)$.
    \item If $X$ is given by the initial topology induced by the family of maps $\set{f_i\colon X\rightarrow \left(Y_i,\tau_i\right)}_{i\in I}$, the net $\left(x_{\alpha}\right)_{\alpha}$ converges to $x$ if and only if $\left(f_i\left(x_{\alpha}\right)\right)_{\alpha}\xrightarrow{\tau_i}f_i\left(x\right)$ in $Y_i$ for each $i\in I$.
    \item If $\set{\left(X_i,\tau_i\right)}_{i\in I}$ is a family of topological spaces, with $X = \prod_{i\in I}X_i$ equipped with the product topology, then a net $\left(x_{\alpha}\right)_{\alpha}$ in $X$ converges to $x\in X$ if and only if $\left(x_{\alpha}\left(i\right)\right)_{\alpha}\xrightarrow{\tau_i} x\left(i\right)$ in $X_i$ for each $i\in I$.
  \end{itemize}
\end{fact}
\begin{definition}
  Let $A$ and $B$ be directed sets.
  \begin{enumerate}[(1)]
    \item A subset $J\subseteq A$ is said to be cofinal if for every $\alpha \in A$, there is $\gamma\in J$ with $\gamma \geq \alpha$.
    \item A map $\sigma\colon B\rightarrow A$ is monotone increasing if it preserves the order of the sets --- i.e., if $\beta_1\leq \beta_2$ in $B$, then $\sigma\left(\beta_1\right) \leq \sigma\left(\beta_2\right)$ in $A$.
    \item A map $\sigma\colon B\rightarrow A$ is called natural if it is monotone increasing and $\sigma\left(B\right)\subseteq A$ is cofinal.
  \end{enumerate}
  If $X$ is a topological space, and $\left(x_{\alpha}\right)_{\alpha}$ is a net in $X$, a subnet, $\left(x_{\sigma\left(\beta\right)}\right)_{\beta\in B}$, is a net where $B$ is a directed set and $\sigma\colon B\rightarrow A$ is a natural map. We will write $\alpha_{\beta} = \sigma\left(\beta\right)$.
\end{definition}
\begin{fact}
  Let $\left(x_{\alpha}\right)_{\alpha}$ be a net in $\left(X,\tau\right)$. A point $x\in X$ is a cluster point of $\left(x_{\alpha}\right)_{\alpha}$ if and only if $\left(x_{\alpha}\right)_{\alpha}$ admits a subnet converging to $x$.
\end{fact}
An important fact about nets is that they can be used to characterize the topology of certain spaces.
\begin{fact}
  Let $\left(X,\tau\right)$ be a completely regular space, and $\left(x_{\alpha}\right)_{\alpha}$ a net in $X$. Then, $\left(x_{\alpha}\right)_{\alpha}\rightarrow x$ if and only if $\left(f\left(x_{\alpha}\right)\right)_{\alpha}\rightarrow f(x)$ for every $f\in C\left(X,\left[0,1\right]\right)$.
\end{fact}
\subsection{Compactness}%
\begin{definition}
  Let $\left(X,\tau\right)$ be a topological space. A subspace $K\subseteq X$ is called compact if, for every collection $\set{U_i}_{i\in I}\subseteq \tau$ with $K\subseteq \bigcup_{i\in I}U_i$, there exists a finite subset $F\subseteq I$ such that $K\subseteq \bigcup_{i\in F}U_i$.\newline

  Colloquially, we say that $K$ is compact if every open cover of $K$ admits a finite subcover.
\end{definition}
\begin{proposition}
  Let $\left(X,\tau\right)$ be a topological space, and let $K\subseteq X$ be closed. The following are equivalent:
  \begin{enumerate}[(i)]
    \item $K$ is compact;
    \item for any family $\set{C_i}_{i\in I}$ of closed subsets in $K$ with the finite intersection property --- i.e., for all finite $F\subseteq I$, $\bigcap_{i\in F}C_i \neq \emptyset$ --- it is the case that $\bigcap_{i\in I}C_i \neq \emptyset$;
    \item every net $\left(x_{\alpha}\right)_{\alpha}$ in $K$ has a cluster point in $K$;
    \item every net $\left(x_{\alpha}\right)_{\alpha}$ in $K$ admits a convergent subnet.
  \end{enumerate}
\end{proposition}
\begin{fact}
  Let $X$ and $Y$ be topological spaces. If $f\colon X\rightarrow Y$ is continuous, then for any compact $K\subseteq X$, $f(K)\subseteq Y$ is compact.
\end{fact}
\begin{fact}
  If $X$ and $Y$ are topological spaces with $X$ compact and $Y$ Hausdorff, then any continuous bijection $f\colon X\rightarrow Y$ is a homeomorphism.
\end{fact}
\begin{fact}
  Let $M$ and $N$ be metric spaces. If $f\colon M\rightarrow N$ is continuous, and $M$ is compact, then $f$ is uniformly continuous.
\end{fact}
\begin{fact}
  If $\left(M,d\right)$ is a metric space, the following are equivalent:
  \begin{enumerate}[(i)]
    \item $M$ is compact;
    \item $M$ is sequentially compact --- every sequence in $M$ admits a convergent subsequence;
    \item $M$ is complete and totally bounded --- for any $\ve > 0$, there exists $\set{x_1,\dots,x_n}\subseteq M$ such that $M\subseteq \bigcup_{i=1}^{n}U\left(x_i,\ve\right)$.
  \end{enumerate}
\end{fact}
\begin{definition}
  Let $\Omega$ be a compact Hausdorff space. A subset $\mathcal{F}\subseteq C\left(\Omega\right)$ is
  \begin{enumerate}[(a)]
    \item pointwise bounded if, for all $x\in \Omega$, $\sup_{f\in \mathcal{F}}\left\vert f(x) \right\vert < \infty$;
    \item equicontinuous at $x\in \Omega$ if for every $\ve > 0$, there is a neighborhood $U_x$ of $x$ such that $\left\vert f(y) - f(x) \right\vert < \ve$ for all $f\in \mathcal{F}$ and all $y\in U_x$;
    \item equicontinuous if $\mathcal{F}$ is equicontinuous at every $x\in \Omega$.
  \end{enumerate}
\end{definition}
\begin{theorem}[Arzela--Ascoli Theorem]\label{thm:arzela_ascoli}
  Let $\Omega$ be a compact Hausdorff space. If $\mathcal{F}\subseteq C\left(\Omega\right)$ is pointwise bounded and equicontinuous, then the uniform closure $\overline{\mathcal{F}}^{\norm{\cdot}_u}\subseteq C\left(\Omega\right)$ is compact in $C\left(\Omega\right)$.
\end{theorem}
\begin{theorem}[Tychonoff's Theorem]\label{thm:tychonoff}
  Let $\left(X_i,\tau_i\right)_{i\in I}$ be a family of compact topological spaces. Then, the product space $\prod_{i\in I}X_i$, equipped with the product topology, is also compact.
\end{theorem}
\begin{definition}
  A topological space $\left(X,\tau\right)$ is said to be locally compact if for any $x\in U\in \tau$, there is $V\in \tau$ with compact closure such that $x\in V\subseteq \overline{V}\subseteq U$.\newline

  If $X$ is locally compact and Hausdorff, we say it is a LCH space.
\end{definition}
\begin{theorem}[Urysohn's Lemma for LCH spaces]
  Let $X$ be a LCH space. Suppose $C,K\subseteq X$ are closed disjoint subsets with $K$ compact. Then, there is a continuous function $f\colon X\rightarrow [0,1]$ with $f|_{C} = 0$ and $f|_{K} = 1$.\newline

  If $K\subseteq U\subseteq X$ is such that $K$ is compact and $U$ is open, then there is a compactly supported continuous function $f\colon X\rightarrow [0,1]$ such that $f|_{K} = 1$ and $\supp\left(f\right)\subseteq U$.
\end{theorem}
\begin{definition}
  Let $X$ be a LCH space. A compactification of $X$ is a pair $\left(Z,\iota\right)$, with
  \begin{enumerate}[(i)]
    \item $Z$ compact and Hausdorff;
    \item $\iota\colon X\rightarrow Z$ an embedding;
    \item $\Ran\left(\iota\right)\subseteq Z$ dense.
  \end{enumerate}
\end{definition}
\begin{theorem}[One-Point Compactification]
  Let $X$ be a noncompact LCH space. There is a compactification $\left(X_{\alpha},\iota\right)$ of $X$, with
  \begin{enumerate}[(1)]
    \item $X_{\infty}\setminus \iota(X)$ is exactly one point;
    \item for any other compactification of $X$, $\left(Z,j\right)$, where $Z\setminus j(X)$ is one point, $Z$ is homeomorphic to $X_{\infty}$.
  \end{enumerate}
\end{theorem}
