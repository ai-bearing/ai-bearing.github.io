We will need a bit of background in point-set topology in order to satisfactorily understand the functional analysis behind the results in Chapters 3, 4, and 5.
\section{Axioms of Set Theory}%
In order to garner sufficient understanding of point-set topology, we need to be able to comprehend some of the essential axioms behind the objects known as ``sets.'' This is where the axioms of set theory come into play.
\begin{definition}[Zermelo--Fraenkel Axioms]
  In Zermelo--Fraenkel set theory, all objects are sets. In order to maintain convention with the way the rest of this section will refer to sets, all sets will be referred to by capital letters, and all elements of sets by lowercase letters.
  \begin{itemize}
    \item Axiom of Existence: $\exists A\left(A = A\right)$. This axiom guarantees a nonempty universe.
    \item Axiom of Extensionality: $\forall x\left(x\in A \Leftrightarrow x\in B\right)\Rightarrow A = B$. This axiom states that if two sets share the same members, then the sets are equal.
    \item Axiom Schema of Comprehension: $\exists B\:\forall x\left(x\in B\Leftrightarrow x\in A \wedge\varphi(x)\right)$. This axiom states that for any formula $\varphi(x)$, where $x$ is a free variable, there is a set $B$ such that the members of $B$ are the members of $A$ for which $\varphi$ holds.
    \item Pairing Axiom: $\forall A\:\forall B\:\exists C\left(\left(A\in Z\right)\wedge \left(B\in Z\right)\right)$. This axiom states that for any sets $A$ and $B$, there is a set $C = \set{A,B}$ that contains the sets $A$ and $B$ as elements.
    \item Power Set Axiom: $\forall A\:\exists P(A)\:\forall B\left(B\in P(A) \Leftrightarrow B\subseteq A\right)$. We use the shorthand $B\subseteq A$ to mean $\forall x\left(x\in B\Rightarrow x\in A\right)$. This axiom states that for any set $A$ there exists a set $P(A)$ such that any element of $P(A)$ is a subset of $A$, and any subset of $A$ is an element of $P(A)$.
    \item Union Axiom: $\forall \mathcal{A}\:\exists A\:\forall Y\:\forall x\left(\left(x\in Y\wedge Y\in \mathcal{A}\right)\Rightarrow x\in A\right)$. This axiom states that for any collection $\mathcal{A}$, there is a set $A $, denoted $ \bigcup \mathcal{A}$, that contains all the elements of all the sets in the collection $\mathcal{A}$.
    \item Axiom of Infinity: $\exists A\left(\emptyset\in A\wedge \forall x\left(x\in A\Rightarrow x\cup\set{x}\in A\right)\right)$. This axiom states that there is a set, $A$, such that the empty set is in $A$ and, for any element $x$, if $x\in A$, then so too is the successor, $x\cup \set{x}$.
    \item Axiom of regularity: $\forall X\left(X\neq\emptyset \Rightarrow\exists Y\left(Y\in X\wedge Y\cap X = \emptyset\right)\right)$. This axiom states that any nonempty set $X$ contains a set $Y$ such that $Y$ and $X$ are disjoint. As a consequence, any chain of sets descending in membership must terminate.
    \item Axiom Schema of Replacement: $\forall A\:\exists B\:\forall v\left(v\in B\Rightarrow \exists u\left(u\in A\wedge \psi\left(u,v\right)\right)\right)$. The axiom schema of replacement says that for a function-like formula (a formula such that $\psi\left(u,v\right)\wedge \psi\left(u,w\right) \Rightarrow v=w$) $\psi\left(u,v\right)$, there is a set $A$ consisting of exactly those sets/elements $v\in B$ that correspond to $u\in A$.
  \end{itemize}
\end{definition}
The final axiom, the Axiom of Choice, is special, and as a result, we state it separately, for we will be using some of its consequences in the future sections. The following is one way of interpreting the axiom of choice.
\begin{definition}[Axiom of Choice]
  Let $\set{S_i}_{i\in I}$ be an indexed collection of nonempty sets. Then, there exists an indexed set $\set{x_i}_{i\in I}$ such that $x_i\in S_i$ for each $I$.\newline

  Equivalently, if $\set{S_i}_{i\in I}$ is an indexed collection of nonempty sets, then there is some choice function
  \begin{align*}
    f\in \prod_{i\in I}S_i.
  \end{align*}
\end{definition}
On its own, this formulation of the Axiom of Choice is not particularly useful. However, there is a statement of the Axiom of Choice which is just as useful.
\begin{definition}[Preorders, Partial Orders, Total Orders, and Well-Orders]
Let $X$ be a set, and $\preceq $ be a relation on $X$. We say a relation is a preorder if it is reflexive and transitive:
\begin{itemize}
  \item $a\preceq a$
  \item $a\preceq b \wedge b\leq c\Rightarrow a\preceq c$.
\end{itemize}
We say $X$ is a directed set if, for any $a,b\in X$, there is $c\in X$ such that $a\preceq c$ and $b\preceq c$.\newline

If $\preceq$ is also antisymmetric --- that is, $a\preceq b\wedge b\preceq a \Rightarrow a = b$ --- then, we say $\preceq$ is a partial order.\newline

We say $m\in X$ is a maximal element if, for any $x\in X$ with $m\preceq x$, $m = x$.\newline

If $X$ is partially ordered by $\preceq$ and, for any two elements $a,b\in X$, either $a\preceq b$ or $b\preceq a$, then we say $\preceq$ is a total order on $X$.\newline

If $X$ is a totally ordered set that has the property that, for any nonempty $A\subseteq X$, there is some $x\in A$ such that for any $y\in A$, $x\prec y$ for all $y \in A$ with $y\neq x$, then we say $\preceq$ is a well-order on $X$.
\end{definition}
\begin{example}
  \begin{itemize}
    \item The set $\N$ with the usual ordering is a well-ordered set.
    \item If $A$ is a set, then $P(A)$ with the containment ordering, $A\preceq B$ if $A\supseteq B$, is a partially ordered set.
    \item Similarly, if $A$ is a set, then $P(A)$ with the inclusion ordering, $A\preceq B$ if $A\subseteq B$, is a partially ordered set.
    \item A collection of functions $\set{\varphi_{i}: Z_i\rightarrow Y}_{i\in I}$ ordered by $\varphi_{i}\preceq \varphi_j$ if $Z_i\subseteq Z_j$ and $\varphi_{j}|_{Z_i} = \varphi_i$, is a partially ordered set. This is often known as the extension ordering.
  \end{itemize}
\end{example}

We can state an equivalent formulation of the Axiom of Choice as follows.
\begin{theorem}[Zorn's Lemma]
  If $\left(X,\preceq\right)$ is a partially ordered set with the property that for all $C\subseteq X$ with $C$ totally ordered, $C$ has an upper bound, then $X$ has a maximal element.
\end{theorem}
There are many proofs of both Zorn's Lemma from the Axiom of Choice and the Axiom of Choice from Zorn's Lemma. However, we will mostly be using it for the purposes of proving other theorems. The following results can be proven using Zorn's Lemma.
\begin{example}
  \begin{itemize}
    \item Every $\F$-vector space $V$ has a basis $B\subseteq V$ such that the set of all finite linear combinations of elements of $B$ over $\F$ is $V$.
    \item If $\varphi$ is a continuous linear functional defined on a subspace $W\subseteq V$, there is an extension $\Phi$ such that $\Phi|_{W} = \varphi$. %See: Hahn--Banach Theorems
    \item The arbitrary product of compact spaces is compact. %See Tychonoff's Theorem.
  \end{itemize}
\end{example}
\section{Metric Spaces}%
Building upon the basics of set theory, we move towards understanding metric spaces.
\subsection{Basics of Metric Spaces}%
\begin{definition}[Metrics]
  Let $X$ be a set. A distance metric is a function
  \begin{align*}
    d: X\times X\rightarrow [0,\infty)
  \end{align*}
  such that the following three properties are satisfied:
  \begin{itemize}
    \item if $x,y\in X$ and $d\left(x,y\right) = 0$, then $x = y$;
    \item $d\left(x,y\right) = d\left(y,x\right)$ for all $x,y\in X$;
    \item $d\left(x,z\right) \leq d\left(x,y\right) + d\left(y,z\right)$ for all $x,y,z\in X$.
  \end{itemize}
  A function that satisfies the latter two properties is called a semimetric.\newline

  Two metrics $d$ and $\rho$ on $X$ are equivalent if there exist constants $c_1,c_2\geq 0$ such that
  \begin{align*}
    d\left(x,y\right) &\leq c_1 \rho\left(x,y\right)\\
    \rho\left(x,y\right) &\leq c_2 d\left(x,y\right)
  \end{align*}
  for all $x,y\in X$.\newline

  A metric space is a pair $\left(X,d\right)$, where $d$ is a metric.
\end{definition}
\begin{example}[Some Distance Metrics]
  \begin{itemize}
    \item The discrete metric on any nonempty set is given by
      \begin{align*}
        d\left(x,y\right) & \begin{cases}
          1 & x\neq y\\
          0 & x = y
        \end{cases}
      \end{align*}
    \item The Euclidean metric between $\left(x_1,\dots,x_n\right)$ and $\left(y_1,\dots,y_n\right)$ in $\R^n$ is
      \begin{align*}
        d_{2}\left(x,y\right) &= \left(\sum_{j=1}^{n}\left\vert y_j-x_j \right\vert^2\right)^{1/2}.
      \end{align*}
    \item Other metrics on $\R^n$ include
      \begin{align*}
        d_1\left(x,y\right) &= \sum_{j=1}^{n}\left\vert y_j-x_j \right\vert\\
        d_{\infty}\left(x,y\right) &= \max_{j=1}^{n}\left\vert y_j - x_j \right\vert.
      \end{align*}
      All of $d_1,d_2,d_{\infty}$ are equivalent metrics.
    \item The Hamming distance between two strings of bits is
      \begin{align*}
        d_{H}: \set{0,1}^{n}\times \set{0,1}^{n}\rightarrow [0,\infty)\\
        d_{H}\left(\left(x_{j}\right)_{j=1}^{n},\left(y_j\right)_{j=1}^{n}\right) &= \left\vert \set{j\mid x_j\neq y_j} \right\vert.
      \end{align*}
    \item The set $C\left([0,1],\R\right)$ consisting of continuous real-valued functions from $[0,1]$ to $\R$ can be equipped with
      \begin{align*}
        d_u\left(f,g\right) &= \sup_{t\in [0,1]}\left\vert f(t) - g(t) \right\vert,
      \end{align*}
      which is the uniform metric, or
      \begin{align*}
        d_{1}\left(f,g\right) &= \int_{0}^{1} \left\vert f(t)-g(t) \right\vert\:dt.
      \end{align*}
    \item All subsets of a metric space $X$ equipped with the same metric is also a metric space.
    \item If $\rho$ is a metric on $X$, then we can create a distance metric
      \begin{align*}
        d\left(x,y\right) &= \frac{\rho\left(x,y\right)}{1 + \rho\left(x,y\right)}
      \end{align*}
      that is bounded on $[0,1]$.
    \item If $d_1,\dots,d_n$ are metrics on $X$ and $c_1,\dots,c_n > 0$ are constants, then
      \begin{align*}
        d\left(x,y\right) &= \sum_{k=1}^{n}c_kd_k\left(x,y\right)
      \end{align*}
      defines a metric on $X$.
    \item If $\left(\rho_k\right)_k$ is a family of separating semimetrics for $X$ --- i.e., for $x,y\in X$ distinct, there is some $\rho_{j}$ such that $\rho_j\left(x,y\right) \neq 0$ --- then, we can obtain bounded semimetrics by taking
      \begin{align*}
        d_k\left(x,y\right) &= \frac{\rho_k\left(x,y\right)}{1 + \rho_k\left(x,y\right)}
      \end{align*}
      for each $k$. From each $d_k$, we define
      \begin{align*}
        d\left(x,y\right) &= \sum_{k=1}^{n}2^{-k}d_k\left(x,y\right),
      \end{align*}
      which is a metric on $X$.
    \item If $\left(X_k,\rho_k\right)_{k\geq 1}$ is a sequence of metric spaces, then we can form the product space
      \begin{align*}
        X &= \prod_{k\geq 1}X_{k}
      \end{align*}
      with the metric
      \begin{align*}
        D\left(f,g\right) &= \sum_{k\geq 1}d_k\left(f(k),g(k)\right).
      \end{align*}
      Here, $d_k = \frac{\rho_k}{1 + \rho_k}$ is the corresponding bounded metric to $\rho_k$.
  \end{itemize}
\end{example}
\begin{definition}[Open and Closed Sets]
  Let $\left(X,d\right)$ be a metric space.
  \begin{enumerate}[(1)]
    \item For $x\in X$ and $\delta > 0$, we define
      \begin{enumerate}[(a)]
        \item the open ball at $x$ with radius $\delta > 0$
          \begin{align*}
            U\left(x,\delta\right) &= \set{y\in X\mid d\left(y,x\right) < \delta};
          \end{align*}
        \item the closed ball centered at $x$ with radius $\delta > 0$
          \begin{align*}
            B\left(x,\delta\right) &= \set{y\in X\mid d\left(y,x\right)\leq \delta};
          \end{align*}
        \item the sphere centered at $x$ with radius $\delta > 0$
          \begin{align*}
            S\left(x,\delta\right) &= \set{y\in X\mid d\left(y,x\right) = \delta}.
          \end{align*}
      \end{enumerate}
    \item A set $V\subseteq X$ is open if, for all $x\in V$, there is $\delta > 0$ such that $U\left(x,\delta\right)\subseteq V$.\newline

      A subset $C\subseteq X$ is closed if $C^{c}$ is open.
    \item If $x\in V$ and $V\subseteq X$ is open, then we say $V$ is an open neighborhood of $x$. A neighborhood of $x$ is any subset $N\subseteq X$ such that $N$ contains an open neighborhood of $x$.
    \item If $A\subseteq X$ is any subset, the interior of $A$ is
      \begin{align*}
        A^{\circ} &:= \bigcup\set{V\mid V\text{ is open, }V\subseteq A},
      \end{align*}
      the closure of $A$ is
      \begin{align*}
        \overline{A} &= \bigcap\set{C\mid C\text{ is closed, }A\subseteq C},
      \end{align*}
      and the boundary of $A$ is
      \begin{align*}
        \partial A &= \overline{A} \setminus A^{\circ}.
      \end{align*}
  \end{enumerate}
\end{definition}
We can now talk about the topology of the metric space.
\begin{fact}
  Let $\left(X,d\right)$ be a metric space, and let
  \begin{align*}
    \mathcal{U} = \set{V\mid V\subseteq X\text{ open}}.
  \end{align*}
  Then, the following are true.
  \begin{itemize}
    \item $\emptyset\in \mathcal{U},X\in \mathcal{U}$.
    \item If $\set{V_{i}}_{i\in I}$ is a family of open sets, then $\bigcup_{i\in I}V_i\in \mathcal{U}$.
    \item If $\set{V_i}_{i=1}^{n}$ is a finite collection of open sets, then $\bigcap_{i=1}^{n}V_i \in \mathcal{U}$.
  \end{itemize}
\end{fact}

\begin{definition}
  Let $\left(X,d\right)$ be a metric space. Suppose $A\subseteq X$ is a nonempty subset.
  \begin{enumerate}[(1)]
    \item The distance from a point $x\in X$ to the set $A$ is defined by
      \begin{align*}
        \dist_{A}\left(x\right) &= \inf_{a\in A}d\left(x,a\right).
      \end{align*}
    \item The diameter of $A$ is defined by
      \begin{align*}
        \diam\left(A\right) &= \sup_{x,y\in A}d\left(x,y\right).
      \end{align*}
    \item If $\diam(A) < \infty$, then we say $A$ is bounded.
    \item If, for every $\delta > 0$, there is a finite subset $F_{\delta}\subseteq X$ such that
      \begin{align*}
        A\subseteq \bigcup_{x\in F_{\delta}}U\left(x,\delta\right).
      \end{align*}
    \item For $A,B\subseteq X$, we define the Hausdorff distance between $A$ and $B$ to be
      \begin{align*}
        d_{H}\left(A,B\right) &= \max\set{\sup_{x\in A}\dist_{B}\left(x\right),\sup_{y\in B}\dist_{A}\left(y\right)}.
      \end{align*}
  \end{enumerate}
\end{definition}
\begin{example}
  Let $\Omega$ be a nonempty set, and $\left(X,d\right)$ be a metric space. A function $f: \Omega\rightarrow X$ is said to be bounded if $\diam\left(\ran(f)\right) < \infty$.\newline

  The collection $\operatorname{Bd}\left(\Omega,X\right)$ denotes all bounded functions with domain $\Omega$ and codomain $X$.\newline

  On $ \operatorname{Bd}\left(\Omega,X\right)$, we define the uniform metric by
  \begin{align*}
    D_{u}\left(f,g\right) &= \sup_{x\in\Omega}d\left(f(x),g(x)\right).
  \end{align*}
\end{example}
\subsection{Convergence and Continuity in Metric Spaces}%
\begin{definition}[Crash Course on Sequences]
  Let $\left(X,d\right)$ be a metric space.
  \begin{enumerate}[(1)]
    \item A sequence in $X$ is a map $x: \N\rightarrow X$, which we call $\left(x_{n}\right)_{n}$ or $\left(x_{n}\right)_{n\geq 1}$.
    \item A natural sequence is a strictly increasing sequence of natural numbers $\left(n_{k}\right)_{k\geq 1}$ with $n_{k}\geq k$ and $n_{k} < n_{k+1}$.
    \item If $\left(n_k\right)_{k}$ is a natural sequence, the sequence $\left(x_{n_k}\right)_{k}$ is called a subsequence of $\left(x_{n}\right)_n$.
    \item We say $\left(x_n\right)_n\rightarrow x$ if $d\left(x_n,x\right)_{n} \xrightarrow{n\rightarrow\infty} 0$. We say $x$ is the limit of $\left(x_n\right)_n$.
  \end{enumerate}
\end{definition}
\begin{example}[Convergence in Metric Spaces of Functions]
  \begin{itemize}
    \item If $\Omega$ is a nonempty set, and $\left(X,d\right)$ is a metric space, the sequence of functions $f_n: \Omega\rightarrow X$ is said to converge pointwise to $f: \Omega\rightarrow X$ if
      \begin{align*}
        f_n\left(x\right)\xrightarrow{n\rightarrow\infty}f(x)
      \end{align*}
      for each $x\in \Omega$.
    \item If $\left(f_n\right)_n\in \operatorname{Bd}\left(\Omega,X\right)$ is a sequence, we say $\left(f_n\right)_n\rightarrow f$ converges uniformly if
      \begin{align*}
        D_u\left(f_n,f\right)\xrightarrow{n\rightarrow\infty}0,
      \end{align*}
      or, equivalently,
      \begin{align*}
        \sup_{x\in\Omega}d\left(f_n(x),f(x)\right)\xrightarrow{n\rightarrow\infty}0.
      \end{align*}
  \end{itemize}
\end{example}
\begin{definition}[Sequential Criteria for Closure]
  If $\left(X,d\right)$ is a metric space, and $E\subseteq X$ is nonempty, then $E$ is closed if and only if, for all $\left(x_n\right)_n\rightarrow x$ with $x_n\in E$, $x\in E$.\newline

  If $E\subseteq X$ is any nonempty set, then $\overline{E}$ is precisely the set of all $x\in X$ such that $\left(x_n\right)_n\rightarrow x$ for some $\left(x_n\right)_n\subseteq E$.
\end{definition}
\begin{definition}[Completeness]
  Let $\left(X,d\right)$ be a metric space.
  \begin{itemize}
    \item If $\left(x_n\right)_n$ is a sequence in $X$ such that for all $\ve > 0$, there is $N\in \N$ such that for all $m,n\geq N$, $d\left(x_m,x_n\right) < \ve$, then we say the sequence is called Cauchy.
    \item If, for any $\left(x_n\right)_n$ Cauchy, $\left(x_n\right)_n\rightarrow x$ in $X$, then we say $X$ is complete.
    \item If $\left(X,d\right)$ is complete, then for any $A\subseteq X$ closed, $A$ is also complete.
    \item If $A\subseteq X$ is complete as a metric space, then $A$ is closed.
  \end{itemize}
\end{definition}
\begin{example}
  The metric space $\Q$ with the metric inherited from $\R$ is not complete. For instance, there is a sequence of rational numbers $\left(2,2.7,2.71,2.718,\dots\right)$ converging to $e$, but $e\notin \Q$.\newline

  The space $\operatorname{Bd}\left(\Omega,X\right)$ is complete if $X$ is complete.
\end{example}
\begin{definition}[Continuity]
  \begin{itemize}
    \item Let $\left(X,d\right)$ and $\left(Y,\rho\right)$ be metric spaces, and let $f: X\rightarrow Y$ be a function. We say $f$ is continuous at $x$ if, for every $\ve > 0$, there is $\delta > 0$ such that $z\in U\left(x,\delta\right)\Rightarrow \rho\left(f(x),f(z)\right) < \ve$.
    \item If $f$ is continuous at every point in $X$, then we say $f$ is continuous.
    \item If $f$ is bijective, continuous, and $f^{-1}$ is continuous, then we say $f$ is a homeomorphism.
    \item We say $f$ is uniformly continuous on $X$ if, for any $\ve > 0$, there is $\delta > 0$ such that for any $y,z\in X$, $d\left(y,z\right) < \delta \Rightarrow \rho\left(f(y),f(z)\right) < \ve$.
    \item We say $f$ is Lipschitz if there exists $C > 0$ such that $d\left(x,y\right) \leq Cd\left(f(x),f(y)\right)$ for all $x,y\in X$.
    \item We say $f$ is an isometry if $d\left(x,y\right) = d\left(f(x),f(y)\right)$ for all $x,y\in X$.
  \end{itemize}
\end{definition}
\begin{fact}
  Let $f: X\rightarrow Y$ be a map between metric spaces. The following are equivalent:
  \begin{enumerate}[(i)]
    \item $f$ is continuous;
    \item if $V\subseteq Y$ is open, then $f^{-1}\left(V\right)\subseteq X$ is open;
    \item if $\left(x_n\right)_n\rightarrow x$ in $X$, then $\left(f\left(x_n\right)\right)_n\rightarrow f(x)$ in $Y$.
  \end{enumerate}
\end{fact}
\begin{fact}
  If $M$ and $N$ are metric spaces with $N$ complete, and $A\subseteq M$ is dense, then if $f: A\rightarrow N$ is uniformly continuous, then there is a unique uniformly continuous map $\tilde{f}: M\rightarrow N$.
\end{fact}

