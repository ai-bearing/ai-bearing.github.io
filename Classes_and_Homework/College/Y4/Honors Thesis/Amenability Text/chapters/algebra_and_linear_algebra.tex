In general, as we progress through these appendices, we will consistently add additional structure to a set. First, we begin by developing groups, rings, and fields, vector spaces, and algebras. In the following appendices, we will apply metric structures, topologies, and measures, building up to the central structure of functional analysis: normed vector spaces and the operators on these normed vector spaces.\newline

These appendices were largely written to provide essential background for the techniques and results that will appear in the main body of the text. As such, they do not include detailed proofs --- occasionally, we will include outlines for certain proofs in the remarks. The proofs for many of these results can be found in relevant canonical texts.\newline

We make heavy use of results from algebra and linear algebra in this thesis. Some excellent resources to learn more about algebra and linear algebra are \cite{dummit_and_foote} and \cite{algebra_chapter_0}.
\section{Group, Rings, (some) Fields}%
\begin{definition}[Groups]
  Let $A$ be a set, and let $\star$ be a binary operation on $A$. We say $A$ is a group if
  \begin{itemize}
    \item $A$ is closed under the operation $\star$;
    \item $A$ has an identity element $e_A$, where $a\star e_A = e_A\star a = a$ for any $a\in A$;
    \item for any $a\in A$, there exists $a^{-1}\in A$ such that $a^{-1}\star a = a\star a^{-1} = e$.
  \end{itemize}
  If the operation $\star$ is such that $a\star b = b\star a$ for all $a,b\in A$, then we say $A$ is an abelian group.\newline

  Generally, we abbreviate $a\star b \coloneq ab$.
\end{definition}
\begin{definition}[Subgroups, Normal Subgroups, and Quotient Groups]
  If $G$ is a group, $H\subseteq G$ is a subgroup if $H$ is closed under the group operation and inverses. We write $H\leq G$.\newline

  If $H$ is a subgroup, a left coset of $H$ is the set $gH \coloneq \set{gh | h\in H}$, where $g\in G$. Similarly, a right coset of $H$ is the set $Hg\coloneq \set{hg | h\in H}$. The index of $H$, denoted $\left[G:H\right]$, is the number of left (or right) cosets of $H$.\newline

  If $H\leq G$ is also such that, for any $g\in G$ and $h\in H$, $ghg^{-1}\in H$, then we call $H$ a normal subgroup of $G$. We write $H\trianglelefteq G$.\newline

  Defining the equivalence relation $g\sim g'$ if and only if $g^{-1}g'\in H$, the group of equivalence classes $gH\coloneq \left[g\right]$ is known as the quotient group $G/H$.
\end{definition}
\begin{definition}
  Let $G$ and $H$ be groups. A map $\varphi\colon G\rightarrow H$ is called a (group) homomorphism if, $\varphi$ preserves the group structure, in the sense that
  \begin{align*}
    \varphi\left(ab\right) &= \varphi(a)\varphi(b)\\
    \varphi\left(a^{-1}\right) &= \varphi(a)^{-1}
  \end{align*}
  for all $a,b\in G$.\newline

  We define $\ker\left(\varphi\right)$ to be the set of all $g\in G$ such that $\varphi\left(g\right) = e_H$.\newline

  If $H\trianglelefteq G$, the map $\pi\colon G\rightarrow G/H$ that sends $g\mapsto gH$ is known as the canonical projection.\newline

  If $\varphi$ is a bijection, then $\varphi$ is known as an isomorphism. We write $G\cong H$ if there exists an isomorphism $\varphi\colon G\rightarrow H$.
\end{definition}
\begin{theorem}[First Isomorphism Theorem for Groups]
  Let $G$ and $H$ be groups, and let $\varphi\colon G\rightarrow H$ be a group homomorphism. Then, $\ker\left(\varphi\right)\trianglelefteq G$ is a normal subgroup, and $G/\ker\left(\varphi\right)\cong \img\left(\varphi\right)$.
\end{theorem}
\begin{definition}[Rings]
  Let $A$ be a set. Specifically, let $A$ be an abelian group, letting $+$ denote the operation on $A$ and $0\coloneq e_A$. Then, $A$ is a ring if $A$ also admits a multiplication, $\cdot$, such that
  \begin{itemize}
    \item $a\cdot \left(b+c\right) = a\cdot b + a\cdot c$;
    \item $\left(a+b\right)\cdot c = a\cdot c + b\cdot c$;
    \item $\left(a\cdot b\right)\cdot c = a\cdot \left(b\cdot c\right)$.
  \end{itemize}
  If the multiplication on $A$ is commutative, then we say $A$ is a commutative ring. If $A$ admits an element $1_A$ such that $a\cdot 1_A = 1_A \cdot a = a$, then we say $A$ is a unital ring.
\end{definition}
