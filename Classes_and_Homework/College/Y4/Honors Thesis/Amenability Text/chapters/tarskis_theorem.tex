Ultimately, the reason the Banach--Tarski paradox ``works'' is because the paradoxical group $F(a,b)$ is not amenable --- specifically, its paradoxicality closes off the possibility of amenability. Before we go further into the characterizations of amenability discussed in Chapters \ref{ch:invariant_states} and \ref{ch:folner_condition}, we will show that this statement reverses. Indeed, every amenable group is \textit{non}-paradoxical.
\begin{restatable}[Tarski's Theorem, {\cite[Theorem 0.2.1]{lectures_on_amenability}}]{theorem}{tarski}
  Let $G$ be a group that acts on a set $X$, and let $E \subseteq X$ be nonempty.\newline

  There is a finitely additive measure $\mu \colon P(X) \to [0, \infty]$ with $\mu(E) \in (0, \infty)$ and $\mu\left( t\cdot E \right) = \mu(E)$ for all $t\in G$ if and only if $E$ is not $G$-paradoxical.
\label{thm:tarski}
\end{restatable}
We can prove one of the directions of Tarski's theorem now.
\begin{proof}[Proof of the Forward Direction of Theorem \ref{thm:tarski}:]
  Let $E$ be $G$-paradoxical. Suppose toward contradiction that such a translation-invariant finitely additive $\nu$ existed with $\nu(E) \in (0,\infty)$.\newline

  Let $A_1,\dots,A_n,B_1,\dots,B_m\subseteq E$ be pairwise disjoint, and let $t_1,\dots,t_n,s_1,\dots,f_m\in G$ such that
  \begin{align*}
    E &= \bigsqcup_{i=1}^{n}t_i\cdot A_i\\
      &= \bigsqcup_{j=1}^{m}s_j\cdot B_j.
  \end{align*}
  Then, it would be the case that
  \begin{align*}
    \nu(E) &= \nu\left(\bigsqcup_{i=1}^{n}t_i\cdot A_i\right)\\
           &= \sum_{i=1}^{n}\nu\left(t_i\cdot A_i\right)\\
           &= \sum_{i=1}^{n}\nu\left(A_i\right),
  \end{align*}
  and
  \begin{align*}
    \nu(E) &= \sum_{j=1}^{m}\nu\left(B_j\right).
  \end{align*}
  However, this also yields
  \begin{align*}
    \nu\left(E\right) &= \nu\left(\left(\bigsqcup_{i=1}^{n}A_i\right)\sqcup \left(\bigsqcup_{j=1}^{m}B_j\right)\right)\\
                      &= \sum_{i=1}^{n}\nu\left(A_i\right) + \sum_{j=1}^{m}\nu\left(B_j\right)\\
                      &= \sum_{i=1}^{n}\nu\left(t_i\cdot A_i\right) + \sum_{j=1}^{m}\nu\left(x_j\cdot B_j\right)\\
                      &= \nu\left(E\right) + \nu\left(E\right)\\
                      &= 2\nu\left(E\right).
  \end{align*}
  implying that $\nu(E) = 0$ or $\nu(E) = \infty$.
\end{proof}
The opposite direction, unfortunately, will be significantly harder to prove. We will need to know some results from graph theory, understand the properties of the type semigroup of an action, and use some results on commutative semigroups to show the existence of a mean.
\section{A Little Bit of Graph Theory}
To prove the reverse direction of Tarski's theorem, we need to develop some machinery from graph theory that will allow us to prove that a certain semigroup we will construct in the next section satisfies the cancellation identity.\newline

We start by defining graphs and paths, before proving a special case of Hall's theorem, ultimately extending to the infinite case with König's theorem.
\begin{definition}[Graphs and Paths, {\cite[7]{lectures_on_amenability}}]
  A \textit{graph} is a triple $\left(V,E,\phi\right)$, with $V,E$ nonempty sets and $\phi\colon E\rightarrow P_{2}(V)$ a map from $E$ to the set of all unordered subset pairs of $V$.\newline

  For $e\in E$, if $\phi(e) = \set{v,w}$, then we say $v$ and $w$ are the \textit{endpoints} of $e$, and $e$ is \textit{incident} on $v$ and $w$.\newline

  A \textit{path} in $\left(V,E,\phi\right)$ is a finite sequence $\left(e_1,\dots,e_n\right)$ of edges, with a finite sequence of vertices $\left(v_0,\dots,v_n\right)$, such that $\phi\left(e_k\right) = \set{v_{k-1},v_k}$.\newline

  The \textit{degree} of a vertex, $\deg(v)$, is the number of edges incident on $v$.\newline

  We define the \textit{neighbors} of $S\subseteq V$ to be the set of all vertices $v\in V\setminus S$ such that $v$ is an endpoint to an edge incident on $S$. We denote this set $N(S)$.
\end{definition}

\begin{definition}[Bipartite Graphs and $k$-Regularity, {\cite[Definition 0.2.2]{lectures_on_amenability}}]
  Let $\left(V,E,\phi\right)$ be a graph, with $k\in \N$.
  \begin{enumerate}[(i)]
    \item If $\deg(v) = k$ for each $v\in V$, we say $\left(V,E,\phi\right)$ is \textit{$k$-regular}.
    \item If $V = X\sqcup Y$, with each edge in $E$ having one endpoint in $X$ and one endpoint in $Y$, then we say $V$ is \textit{bipartite}, and write $\left(X,Y,E,\phi\right)$.
  \end{enumerate}
\end{definition}

\begin{definition}[Perfect Matching, {\cite[Definition 0.2.3]{lectures_on_amenability}}]
  Let $\left(X,Y,E,\phi\right)$ be a bipartite graph. Let $A\subseteq X$ and $B\subseteq Y$. A \textit{perfect matching} of $A$ and $B$ is a subset $F\subseteq E$ with
  \begin{enumerate}[(i)]
    \item each element of $A\cup B$ is an endpoint of exactly one $f\in F$;
    \item all endpoints of edges in $F$ are in $A\cup B$.
  \end{enumerate}
\end{definition}
\begin{definition}[Hall Condition, {\cite[Exercise 0.2.2]{lectures_on_amenability}}]
  We say a bipartite graph $\left(X,Y,E,\phi\right)$ satisfies the \textit{Hall condition} on $X$ if, for all $S\subseteq X$, $\left\vert N(S) \right\vert \geq \left\vert S \right\vert$.\newline

  Equivalently, we say a (finite) collection of not necessarily distinct finite sets $\mathcal{X} = \set{X_i}_{i=1}^{n}$ satisfies the Hall condition if and only if for all subcollections $\mathcal{Y}_k = \set{X_{i_k}}_{k=1}^{m}$,
  \begin{align*}
    \left\vert \mathcal{Y}_k \right\vert \leq \left\vert \bigcup_{k=1}^{m}X_{i_k} \right\vert.
  \end{align*}
\end{definition}
\begin{remark}
These two formulations of the Hall condition are equivalent regarding an $X$-perfect matching.
\end{remark}
\begin{theorem}[Hall's Theorem for Finite $k$-Regular Bipartite Graphs, {\cite[Exercise 0.2.2]{lectures_on_amenability}}]
  Let $\left(X,Y,E,\phi\right)$ be a $k$-regular bipartite graph for some $k\in \N$, and let $V = X\sqcup E$ be finite. Then, there is a perfect matching of $X$ and $Y$.\label{thm:hall_finite}
\end{theorem}
\begin{proof}
  Note that since $\left\vert E \right\vert = k\left\vert K \right\vert = k\left\vert Y \right\vert$, it is the case that $\left\vert X \right\vert = \left\vert Y \right\vert$.\newline

  Let $M\subseteq V$ be any subset. We will show that $\left\vert N(M) \right\vert\geq \left\vert M \right\vert$ --- that is, $\left(X,Y,E,\phi\right)$ satisfies the Hall condition.\newline

  Let $M_X = M\cap X$ and $M_Y = M\cap Y$, where $M = M_X\sqcup M_Y$. Let $\left[M_X,N\left(M_X\right)\right]$ be the set of edges with endpoints in $M_X$ and $N\left(M_X\right)$, and $\left[M_Y,N\left(M_Y\right)\right]$ be the set of edges with endpoints in $M_Y$ and $N\left(M_Y\right)$. We also let $\left[X,N\left(M_X\right)\right]$ denote the set of edges with endpoints in $X$ and $N\left(M_X\right)$, and similarly, $\left[Y,N\left(M_Y\right)\right]$ is the set of edges with endpoints in $Y$ and $N\left(M_Y\right)$.\newline

  We can see that $\left[M_X,N\left(M_X\right)\right]\subseteq \left[X,N\left(M_X\right)\right]$, and similarly, $\left[M_Y,N\left(M_Y\right)\right]\subseteq \left[Y,N\left(M_Y\right)\right]$.\newline

  Since $\left\vert \left[M_X,N\left(M_X\right)\right] \right\vert = k\left\vert M_X \right\vert$ and $\left\vert \left[X,N\left(M_X\right)\right] \right\vert = k\left\vert N\left(M_X\right) \right\vert$, we have
  \begin{align*}
    \left\vert M_X \right\vert\leq \left\vert N\left(M_X\right) \right\vert,
  \end{align*}
  and similarly,
  \begin{align*}
    \left\vert M_Y \right\vert\leq \left\vert N\left(M_Y\right) \right\vert.
  \end{align*}
  Thus, $\left\vert M \right\vert\leq \left\vert N\left(M\right) \right\vert$.\newline

  We will now show that there is an $X$-perfect matching. Suppose toward contradiction that $F$ is a maximal perfect matching on $A\subseteq X$ and $B\subseteq Y$ with $X\setminus A \neq \emptyset$.\newline

  Then, there is $x\in X\setminus A$. Consider $Z\subseteq V$ consisting of all vertices $z$ such that there exists a $F$-alternating path $\left(e_1,\dots,e_n\right)$ between $z\in Z$ and $x$.\newline

  It cannot be the case that $Z\cap Y$ is empty, since the number of neighbors of $x$ is greater than or equal to $1$ by the Hall condition --- if it were the case that $Z\cap Y$ were empty, we could add an edge to $F$ consisting of $x$ and one element of $N\left(\set{x}\right)$, which would contradict the maximality of $F$.\newline

  Consider a path traversing along $Z$, $\left(e_1,\dots,e_n\right)$. It must be the case that $e_n\in F$, or else we would be able to ``flip'' the matching $F$ by exchanging $e_{i}$ with $e_{i+1}$ for $e_i\in F$, which would contradict the maximality of $F$ yet again. Thus, every element of $Z\cap Y$ is satisfied by $F$, so $Z\cap Y\subseteq B$.\newline

  Since each element in $Z\cap Y$ is paired with exactly one element of $Z\cap X$ (with one left over), it is the case that $\left\vert Z\cap X \right\vert = \left\vert Z\cap Y \right\vert + 1$.\newline

  Suppose toward contradiction that there exists $y\in N\left(Z\cap X\right)$ with $y\notin Z\cap Y$. Then, there exists $v\in Z\cap X$ and $e\in E$ such that $\phi(e) = \set{v,y}$. However, this means $v$ is connected via a path to $x$, meaning $y\in Z$, so $y\in Z\cap Y$. Thus, we must have $N\left(Z\cap X\right) = Z\cap Y$.\newline

  Therefore,
  \begin{align*}
    \left\vert Z\cap X \right\vert &= \left\vert Z\cap Y \right\vert + 1\\
                                   &= \left\vert N\left(Z\cap X\right) \right\vert + 1,
  \end{align*}
  which contradicts the fact that $\left(X,Y,E,\phi\right)$ satisfies the Hall condition. Therefore, $A = X$.\newline

  By symmetry, there is a perfect matching of $X$ and $Y$ in $\left(X,Y,E,\phi\right)$.
\end{proof}
\begin{remark}
  An equivalent formulation to Hall's theorem states that there is a system of distinct representatives on the collection $\mathcal{X} = \set{X_k}_{k=1}^{n}$, which is a set $\set{x_{k}}_{k=1}^{n}$ such that $x_{k}\in X_{k}$ and $x_{i}\neq x_j$ for $i\neq j$.\newline

  This implies the existence of an injection $f\colon \mathcal{X}\hookrightarrow \bigcup_{k=1}^{n}X_{k}$, such that $f\left(X_k\right) \in X_k$.
\end{remark}
%\begin{definition}[Choice Function]
%  Let $\mathcal{X} = \set{X_{i}}_{i\in I}$ be a collection of sets. A function $f\colon \mathcal{X}\rightarrow \bigcup_{i\in I}X_i$ is called a choice function if, for each $i\in I$, $f\left(X_{i}\right)\in X{i}$.\newline
%
%  We also say $f\colon \mathcal{X}\rightarrow \bigcup_{i\in I}X_i$ is a choice function if $f\in \prod_{i\in I}X_i$.
%\end{definition}
%
%\begin{theorem}[Tychonoff's Theorem]
%  If $\set{X_{i}}_{i\in I}$ is a family of compact topological spaces
%\end{theorem}
\begin{theorem}[Infinite Hall's Theorem, {\cite{marshall_hall_thm}}]
  Let $\mathcal{G} = \set{X_{i}}_{i\in I}$ be a collection of (not necessarily distinct) finite sets. If, for every finite subcollection $\mathcal{Y} = \set{X_{i_k}}_{k=1}^{n}$,
  \begin{align*}
    n\leq \left\vert \bigcup_{k=1}^{n}X_{i_k} \right\vert,
  \end{align*}
  then there is a choice function on $G$.
\end{theorem}
\begin{proof}
  We endow each $X_i\in \set{X_{i}}_{i\in I}$ with the discrete topology. Since each $X_i$ is finite, each $X_i$ is compact.\newline

  Thus, by Tychonoff's theorem, it is the case that $\prod_{i\in I}X_{i}$ is compact.\newline

  For every finite subset $Y\subseteq \mathcal{G}$, we define
  \begin{align*}
    S_Y &= \set{\left.f\in \prod_{i\in I}X_i\right|f\vert_{Y}\text{ is injective}}.
  \end{align*}
  The injectivity of $f\vert_{Y}$ is equivalent to the existence of a system of distinct representatives on $Y$. Since $Y$ satisfies the Hall condition, each $S_{Y}$ is nonempty. Additionally, for any net of functions $f_{\alpha}\in S_{Y}$ with $\lim_{\alpha}f_{\alpha} = f$, it is the case that $f_{\alpha}\vert_{Y}$ is injective, so $f\vert_{Y}$ is injective, meaning $S_{Y}$ is closed.\newline

  We define $F = \set{S_{Y}| Y\subseteq \mathcal{G}\text{ finite}}$. For finite $Y_{1},Y_{2}\subseteq \mathcal{G}$, every system of distinct representatives in $Y_1\cup Y_2$ is necessarily a system of distinct representatives on $Y_1$ and a system of distinct representatives on $Y_{2}$, meaning $S_{Y_1\cup Y_2}\subseteq S_{Y_1}\cap S_{Y_2}$. Thus, $F$ has the finite intersection property.\newline

  Since $\prod_{i\in I}X_i$ is compact, $\bigcap F$ is nonempty, where the intersection is taken over all finite subsets of $\mathcal{G}$. For any $f\in \bigcap F$, $f$ is necessarily a choice function.
\end{proof}
\begin{remark}
  This is equivalent to the existence of an injection $f\colon \mathcal{G}\hookrightarrow \bigcup_{i\in I}X_i$.
\end{remark}
We will use this infinite case of Hall's theorem to prove König's theorem. 
\begin{theorem}[König's Theorem, {\cite[Theorem 0.2.4]{lectures_on_amenability}}]
  Let $\left(X,Y,E,\phi\right)$ be a $k$-regular bipartite graph (not necessarily finite). Then, there is a perfect matching of $X$ and $Y$.\label{thm:konig}
\end{theorem}
\begin{proof}
  If $k = 1$, it is clear that there is a perfect matching in $\left(X,Y,E,\phi\right)$ consisting of the edges in $\left(X,Y,E,\phi\right)$.\newline

  Let $k\geq 2$. Since any finite subset of $X$ satisfies the Hall condition, as displayed in the proof of Theorem \ref{thm:hall_finite}, there is some $X$-perfect matching in $\left(X,Y,E,\phi\right)$. We call this $X$-perfect matching $F$. There is an injection $f\colon X\hookrightarrow Y$ following the edges in $F$.\newline

  Similarly, since any finite subset of $Y$ satisfies the Hall condition, there is some $Y$-perfect matching in $\left(X,Y,E,\phi\right)$. We call this $Y$-perfect matching $G$. There is an injection $g\colon Y\hookrightarrow X$ following the edges of $G$.\break

  Consider the subgraph $\left(X,Y,F\cup G,\phi|_{F\cup G}\right)$. The injections $f$ and $g$ still hold in this graph. By the Cantor--Schröder--Bernstein theorem, there is a bijection $h\colon X\rightarrow Y$ in $\left(X,Y,F\cup G,\phi|_{F\cup G}\right)$, which is equivalent to the existence of a perfect matching of $X$ and $Y$.
\end{proof}
\section{Type Semigroups}%
\begin{definition}[{\cite[Definition 0.2.5]{lectures_on_amenability}}]\label{def:xstar_gstar}
  Let $G$ be a group that acts on a set $X$.
  \begin{enumerate}[(i)]
    \item We define $X^{\ast} = X\times \N_0$, and
      \begin{align*}
        G^{\ast} &= \set{\left(g,\pi\right)| g\in G,\pi\in\sym\left(\N_0\right)}.
      \end{align*}
    \item If $A\subseteq X^{\ast}$, the values of $n$ for which there is an element of $A$ whose second coordinate is $n$ are called the \textit{levels} of $A$.
  \end{enumerate}
\end{definition}
\begin{fact}[{\cite[Exercise 0.2.4]{lectures_on_amenability}}]\label{fact:type_semigroup_equidecomposability}
  If $E_1,E_2\subseteq X$, then $E_{1}\sim_{G}E_2$ if and only if $E_1\times \set{n}\sim_{G^{\ast}}E_{2}\times \set{m}$ for all $m,n\in \N_{0}.$
\end{fact}
\begin{proof}
  Let $E_{1}\sim_{G}E_2$. Then, there exist pairwise disjoint $A_1,\dots,A_n\subset E_1$, pairwise disjoint $B_1,\dots,B_n\subset E_2$, and elements $g_1,\dots,g_n\in G$ such that $g_i\cdot A_i = B_i$. We select the permutation $\pi_{i}\in \sym\left(\N_0\right)$ such that $\pi_{i}(n) = m$ and $\pi_i(m) = n$ for each $i$. Then,
  \begin{align*}
    \left(g_i,\pi_i\right)\cdot \left(A_{i}\cdot \set{n}\right) &= B_{i}\cdot \set{m}.
  \end{align*}

  Similarly, if $E_{1}\times \set{n} \sim_{G^{\ast}}E_2\times \set{m}$, then of the pairwise disjoint subsets
  \begin{align*}
    A_1\times \set{n},\dots,A_n\times \set{n}\subset E_1\times \set{n}
  \end{align*}
  and
  \begin{align*}
    B_1\times\set{m},\dots,B_n\times\set{m}\subset E_2\times \set{m},
  \end{align*}
  we set $A_1,\dots,A_n\subset E_1$ and $B_1,\dots,B_n\subset E_2$. Similarly, for
  \begin{align*}
    \left(g_1,\pi_1\right),\dots,\left(g_n,\pi_n\right)\in G^{\ast}
    \intertext{such that}
    \left(g_i,\pi_i\right)\cdot A_i\times \set{n} = B_i\times\set{m},
  \end{align*}
  we select $g_1,\dots,g_n\in G$. Then, by definition,
  \begin{align*}
    g_i\cdot A_i = B_i
  \end{align*}
  for each $i$. Thus, $E_1\sim_{G}E_2$.
\end{proof}

\begin{definition}[{\cite[Definition 0.2.6]{lectures_on_amenability}}]\label{def:type_semigroup}
  Let $G$ be a group that acts on $X$, and let $G^{\ast}$, $X^{\ast}$ be defined as in \ref{def:xstar_gstar}.
  \begin{enumerate}[(i)]
    \item A set $A\subseteq X^{\ast}$ is said to be \textit{bounded} if it has finitely many levels.
    \item If $A\subseteq X^{\ast}$ is bounded, the equivalence class of $A$ with respect to $G^{\ast}$-equidecomposability is called the \textit{type} of $A$, which is denoted $\left[A\right]$.
    \item If $E\subseteq X$, we write $\left[E\right] = \left[E\times \set{0}\right]$.
    \item Let $A,B\subseteq X^{\ast}$ be bounded with $k\in \N_{0}$ such that for
      \begin{align*}
        B'= \set{\left(b,n+k\right)| \left(b,n\right)\in B},
      \end{align*}
      we have $B'\cap A = \emptyset$. Then, $\left[A\right] + \left[B\right] = \left[A\sqcup B'\right]$. Note that $\left[B'\right] = \left[B\right]$.
    \item We define
      \begin{align*}
        \mathcal{S} &= \set{\left[A\right]| A\subseteq X^{\ast}\text{ bounded}}
      \end{align*}
      under the addition defined in (iv) to be the \textit{type semigroup} of the action of $G$ on $X$.
  \end{enumerate}
\end{definition}

\begin{fact}[{\cite[Exercise 0.2.5]{lectures_on_amenability}}]
  Addition is well-defined in $\left(\mathcal{S},+\right)$, and $\left(\mathcal{S},+\right)$ is a well-defined commutative semigroup with identity $\left[\emptyset\right]$.\label{fact:type_semigroup_well_defined}
\end{fact}
\begin{proof}
  To show that addition is well-defined, we let $\left[A_1\right] = \left[A_2\right]$, and $\left[B_1\right] = \left[B_2\right]$. Without loss of generality, $A_1\cap B_1 = \emptyset$ and $A_2\cap B_2 = \emptyset$.\newline

  By the definition of the type, $A_1\sim_{G^{\ast}}A_2$ and $B_1\sim_{G^{\ast}}B_2$, meaning
  \begin{align*}
    A_1\sqcup B_1\sim_{G^{\ast}} A_2\sqcup B_2,
  \end{align*}
  so
  \begin{align*}
    \left[A_1\right] + \left[B_1\right] &= \left[A_1\sqcup B_1\right]\\
                                        &= \left[A_2\sqcup B_2\right]\\
                                        &= \left[A_2\right] + \left[A_2\right],
  \end{align*}
  meaning addition is well-defined.\newline

  Since addition is well-defined, and $A\sqcup B = B\sqcup A$, we can see that addition is also commutative. We also have
  \begin{align*}
    \left[A\right] + \left[\emptyset\right] &= \left[A\sqcup \emptyset\right]\\
                                            &= \left[A\right],
  \end{align*}
  so $\left[\emptyset\right]$ is the identity on $\mathcal{S}$.\newline

  Finally, since for any $\left[A\right],\left[B\right]\in \mathcal{S}$, $A$ and $B$ have finitely many levels, it is the case that $A\cup B$ has finitely many levels for any $A$ and $B$, so $\left[A\right] + \left[B\right] \in \mathcal{S}$. 
\end{proof}

\begin{definition}[{\cite[10]{lectures_on_amenability}}]
  For any commutative semigroup $\mathcal{S}$ with $\alpha \in S$ and $n\in \N$, we define
  \begin{align*}
    n\alpha = \underbrace{\alpha + \cdots + \alpha}_{\text{$n$ times}}
  \end{align*}
\end{definition}
\begin{definition}[{\cite[10]{lectures_on_amenability}}]
  For $\alpha,\beta \in \mathcal{S}$, if there exists $\gamma \in \mathcal{S}$ such that $\alpha + \gamma = \beta$, we write $\alpha \leq \beta$.
\end{definition}
\begin{fact}[{\cite[Exercise 0.2.7]{lectures_on_amenability}}]\label{fact:type_semigroup_paradoxicality}%\label{fact:type_semigroup_criterion_paradoxicality}
  If $G$ is a group acting on $X$ with corresponding type semigroup $\mathcal{S}$, then the following are true.
  \begin{enumerate}[(i)]
    \item If $\alpha,\beta\in \mathcal{S}$ with $\alpha \leq \beta$ and $\beta \leq \alpha$, then $\alpha = \beta$.
    \item A set $E\subseteq X$ is $G$-paradoxical if and only if $\left[E\right] = 2\left[E\right]$.
  \end{enumerate}
\end{fact}
\begin{proof}
  Let $G$ act on $X$, and let $\mathcal{S}$ be the corresponding type semigroup.
  \begin{enumerate}[(i)]
    \item If $\left[A\right]\leq \left[B\right]$, then there exists $C_1\in \mathcal{S}$ such that $\left[A\right] + \left[C_1\right] = \left[B\right]$. Without loss of generality, $C_1\cap A= \emptyset$, meaning $\left[B\right] = \left[A\sqcup C_1\right]$. Thus, $A\sqcup C_1 \sim_{G^{\ast}} B$, meaning $B\preceq_{G^{\ast}}A$.\newline

      Similarly, if $\left[B\right]\leq \left[A\right]$, then $B\preceq_{G^{\ast}}A$. By Theorem \ref{thm:csb_for_equidecomposability}, it is thus the case that $A\sim_{G^{\ast}}B$.
    \item Let $E$ be $G$-paradoxical. \newline

      Then, $E\sim_{G}\bigsqcup_{i=1}^{n}A_i$ and $E \sim_{G}\bigsqcup_{j=1}^{m}B_j$ for pairwise disjoint subsets $A_1,\dots,A_n,B_1,\dots,B_m\subset E$. Thus, we have
      \begin{align*}
        \left[E\right] &= \left[\left(\bigsqcup_{i=1}^{n}A_i\right)\sqcup \left(\bigsqcup_{j=1}^{m}B_j\right)\right]\\
                       &= \left[\bigsqcup_{i=1}^{n}A_i\right] + \left[\bigsqcup_{j=1}^{m}B_j\right]\\
                       &= 2\left[E\right].
      \end{align*}
      Similarly, if $\left[E\right] = 2\left[E\right]$, then there exist $A$ and $B$ such that
      \begin{align*}
        \left[E\right] &= \left[A\right] + \left[B\right]\\
                       &= \left[A\sqcup B\right],
      \end{align*}
      meaning $A$ and $B$ are each $G$-equidecomposable with $E$, so $E$ is $G$-paradoxical.
  \end{enumerate}
\end{proof}
We can now prove the cancellation identity, which we will be useful as we construct our desired finitely additive measure.
\begin{theorem}[Cancellation Identity on $\mathcal{S}$, {\cite[Theorem 0.2.7]{lectures_on_amenability}}]
  Let $\mathcal{S}$ be the type semigroup for some group action, and let $\alpha,\beta\in \mathcal{S}$, $n\in \N$ such that $n\alpha = n\beta$. Then, $\alpha = \beta$.
\end{theorem}
\begin{proof}
  Let $n\alpha = n\beta$. Then, there are two disjoint bounded subsets $E,E'\subseteq X^{\ast}$ with $E\sim_{G^{\ast}}E'$, and pairwise disjoint subsets $A_1,\dots,A_n\subseteq E$, $B_1,\dots,B_n\subseteq E'$ such that
  \begin{itemize}
    \item $E = A_1\cup\cdots\cup A_n$, $E' = B_1\cup\cdots\cup B_n$
    \item $\left[ A_j \right] = \alpha$ and $\left[B_j\right] = \beta$ for each $j=1,\dots,n$.
  \end{itemize}
  Let $\chi\colon E\rightarrow E'$ be a bijection as in Fact \ref{fact:bijections}, with $\phi_j\colon A_1\rightarrow A_j$, $\psi_j\colon B_1\rightarrow B_j$ also being bijections as in Fact \ref{fact:bijections}; here we define $\phi_1$ and $\psi_1$ to be the identity map.\newline

  For each $a\in A_1$ and $b\in B_1$, we define
  \begin{align*}
    \overline{a} &= \set{a,\phi_2(a),\dots,\phi_n(a)}\\
    \overline{b} &= \set{b,\psi_2(b),\dots,\psi_n(b)}.
  \end{align*}
  We construct a graph by letting $X = \set{\overline{a}| a\in A_1}$ and $Y = \set{\overline{b}| b\in B_1}$, and, for each $j$, define edges $\set{\overline{a},\overline{b}}$ if $\chi\left(\phi_j(a)\right)\in \overline{b}$.\newline

  Since $\chi$ is a bijection, for each $j=1,\dots,n$, $\chi\left(\phi_j(a)\right)$ must be an element of $B_k$ for some $k$, and since $\set{B_k}_{k=1}^{n}$ are disjoint, $\chi\left(\phi_j(a)\right)$ is an element of exactly one $B_k$. Thus, the graph is $n$-regular.\newline

  By Theorem \ref{thm:konig}, this graph has a perfect matching $F$. As a result, for each $\overline{a}\in X$, there is a unique $\overline{b}\in Y$ and a unique edge $\set{\overline{a},\overline{b}}\in F$ such that $\chi\left(\phi_j(a)\right) = \psi_k(b)$ for some $j,k\in \set{1,\dots,n}$.\newline

  We define
  \begin{align*}
    C_{j,k} &= \set{a\in A_1| \set{\overline{a},\overline{b}}\in F,~\chi\left(\phi_j(a)\right) = \psi_k(b)}\\
    D_{j,k} &= \set{b\in B_1| \set{\overline{a},\overline{b}}\in F,~\chi\left(\phi_j(a)\right) = \psi_k(b)}.
  \end{align*}
  Therefore, we must have $\psi_{k}^{-1}\circ \chi\circ \phi_j$ is a bijection from $C_{j,k}$ to $D_{j,k}$, so $C_{j,k}\sim_{G^{\ast}}D_{j,k}$.\newline

  Since $C_{j,k}$ and $D_{j,k}$ are partitions of $A_1$ and $B_1$ respectively, it follows that $A_1\sim_{G^{\ast}}B_1$, so $\alpha = \beta$.
\end{proof}
\begin{corollary}[{\cite[Corollary 0.2.8]{lectures_on_amenability}}]\label{cor:non_paradoxicality}
  Let $\mathcal{S}$ be the type semigroup of some group action, and let $\alpha\in \mathcal{S}$ and $n\in \N$ such that $\left(n+1\right)\alpha \leq n\alpha$. Then, $\alpha = 2\alpha$.\label{corollary:paradoxical_elements}
\end{corollary}
\begin{proof}
  We have
  \begin{align*}
    2\alpha + n\alpha &= \left(n+1\right)\alpha + \alpha\\
                      &\leq n\alpha + \alpha\\
                      &= \left(n+1\right)\alpha\\
                      &\leq n\alpha.
  \end{align*}
  Inductively repeating this argument, we get $n\alpha \geq 2n\alpha$; since $n\alpha \leq 2n\alpha$ by definition, we must have $n\alpha = 2n\alpha$, so $\alpha = 2\alpha$.
\end{proof}
\begin{remark}
  We will call such an $\alpha$ a paradoxical element.
\end{remark}
\section{Two Results on Commutative Semigroups}%
Now that we are aware of paradoxical elements and the relationship between $G$-paradoxicality and the properties of the particular elements of the type semigroup (Fact \ref{fact:type_semigroup_paradoxicality}), we will now relate these properties to finitely additive measures of sets by using the following lemma and theorem.
\begin{lemma}[{\cite[Lemma 0.2.9]{lectures_on_amenability}}]\label{lemma:set_function_existence}
  Let $\mathcal{S}$ be a commutative semigroup, with $\mathcal{S}_0\subseteq \mathcal{S}$ finite, and $\epsilon\in \mathcal{S}_0$ satisfying the following assumptions:
  \begin{enumerate}[(a)]
    \item $\left(n+1\right)\epsilon \nleq n\epsilon$ for all $n\in \N$ (i.e., that $\epsilon$ is non-paradoxical);
    \item for each $\alpha\in \mathcal{S}$, there is $n\in \N$ such that $\alpha \leq n\epsilon$.
  \end{enumerate}
  Then, there is a set function $\nu\colon \mathcal{S}_0\rightarrow [0,\infty]$ that satisfies the following conditions:
  \begin{enumerate}[(i)]
    \item $\nu\left(\epsilon\right) = 1$;
    \item for $\alpha_1,\dots,\alpha_n,\beta_1,\dots,\beta_m\in \mathcal{S}_0$ with $\alpha_1+\cdots+\alpha_n\leq \beta_1+\cdots\beta_m$,
      \begin{align*}
        \sum_{j=1}^{n}\nu\left(\alpha_j\right) \leq \sum_{j=1}^{m}\nu\left(\beta_j\right).
      \end{align*}
  \end{enumerate}
\end{lemma}
\begin{proof}
  We will prove this result by inducting on the cardinality of $\mathcal{S}_0$.\newline

  We start with $\left\vert \mathcal{S}_0 \right\vert = 1$. In that case, we define $\nu\left(\epsilon\right) = 1$, satisfying condition (i). To satisfy condition (ii), we see that for $n,m\in \N$ with $n\epsilon \leq m\epsilon$, if $n \geq m+1$, then $\left(m+1\right)\epsilon \leq n\epsilon \leq m\epsilon$, implying that $\epsilon = 2\epsilon$, which contradicts assumption (a).\newline

  Let $\alpha_0\in \mathcal{S}_0\setminus\set{\epsilon}$. The induction hypothesis says there is a set function satisfying conditions (i) and (ii), $\nu\colon \mathcal{S}_0\setminus \set{\alpha_0}\rightarrow [0,\infty]$.\newline

  For $r\in \N$, there are $\gamma_1,\dots,\gamma_p,\delta_1,\dots,\delta_q\in \mathcal{S}\setminus \set{\alpha_0}$ such that
  \begin{align*}
    \delta_{1} + \cdots + \delta_q + r\alpha_0 \leq \gamma_1 + \cdots + \gamma_p.\label{set_function_id1}\tag*{(\textdagger)}
  \end{align*}
  Consider the set $N$ defined as follows:
  \begin{align*}
    N &= \set{\frac{1}{r}\left(\sum_{j=1}^{p}\nu\left(\gamma_j\right) - \sum_{j=1}^{q}\nu\left(\delta_j\right)\right)| \text{$\gamma_j,\delta_j$ satisfy \ref{set_function_id1}}}. \label{set_function_N}\tag*{($\ddag$)}
  \end{align*}
  We define the extension of $\nu$ as follows:
  \begin{align*}
    \nu\left(\alpha_0\right) &= \inf N.
  \end{align*}
  This infimum is well-defined since, by assumption (b), there is some $n\in \N$ such that $\alpha_0 \leq n\epsilon$, and $\nu\left(\epsilon\right)$ is defined.\newline

  Now, we must show that this extension of $\nu$ satisfies condition (ii).\newline

  Let $\alpha_1,\dots,\alpha_n,\beta_1,\dots,\beta_m\in \mathcal{S}_0\setminus \set{\alpha_0}$ and $s,t\in \N_0$ such that
  \begin{align*}
    \alpha_1 + \cdots + \alpha_n + s\alpha_0 \leq \beta_1 + \cdots + \beta_m + t\alpha_0.\label{set_function_conditionii}\tag*{(\textasteriskcentered)}
  \end{align*}
  We will verify condition (ii) in the three following cases.
  \begin{description}[font=\normalfont\scshape,leftmargin=0cm]
    \item[Case 0:] If $s = t = 0$, then the induction hypothesis states that \ref{set_function_conditionii} satisfies condition (ii).
    \item[Case 1:] Let $s = 0$ and $t > 0$. We want to show that
      \begin{align*}
        \sum_{j=1}^{n}\nu\left(\alpha_j\right) \leq t\nu\left(\alpha_0\right) + \sum_{j=1}^{m}\nu\left(\beta_j\right),
      \end{align*}
      which implies that
      \begin{align*}
        \nu\left(\alpha_0\right) \geq \frac{1}{t}\left(\sum_{j=1}^{n}\nu\left(\alpha_j\right) - \sum_{j=1}^{m}\nu\left(\beta_j\right)\right).
      \end{align*}
      By the definition of infimum, it suffices to show that for $r\in \N$ and $\delta_1,\dots,\delta_q,\gamma_1,\dots,\gamma_p\in \mathcal{S}\setminus \set{\alpha_0}$ satisfying \ref{set_function_id1}, it is the case that
      \begin{align*}
        \frac{1}{r}\left(\sum_{j=1}^{p}\nu\left(\gamma_j\right)-\sum_{j=1}^{q}\nu\left(\delta_j\right)\right) \geq \frac{1}{t}\left(\sum_{j=1}^{n}\nu\left(\alpha_j\right) - \sum_{j=1}^{m}\nu\left(\beta_j\right)\right).
      \end{align*}
      Multiplying \ref{set_function_conditionii} by $r$ on both sides, and adding $t\delta_1 + \cdots + t\delta_q$ to both sides, we have
      \begin{align*}
        r\alpha_1 + \cdots + r\alpha_n + t\delta_1 + \cdots + t\delta_q \leq r\beta_1 + \cdots + r\beta_m + t\left(r\alpha_0\right) + t\delta_1 + \cdots + t\delta_q.
      \end{align*}
      Substituting \ref{set_function_id1}, we find
      \begin{align*}
        r\alpha_1 + \cdots + r\alpha_n + t\delta_1 + \cdots + t\delta_q \leq r\beta_1 + \cdots + r\beta_m + t\gamma_1 + \cdots + t\gamma_p.
      \end{align*}
      Applying the induction hypothesis, we have
      \begin{align*}
        r\sum_{j=1}^{n}\nu\left(\alpha_j\right) + t\sum_{j=1}^{q}\nu\left(\delta_j\right) \leq r\sum_{j=1}^{m}\nu\left(\beta_j\right) + t\sum_{j=1}^{p}\nu\left(\gamma_j\right),
      \end{align*}
      yielding
      \begin{align*}
        \frac{1}{r}\left(\sum_{j=1}^{p}\nu\left(\gamma_j\right) - \sum_{j=1}^{q}\nu\left(\delta_j\right)\right) \geq \frac{1}{t}\left(\sum_{j=1}^{n}\nu\left(\alpha_j\right) - \sum_{j=1}^{m}\nu\left(\beta_j\right)\right).
      \end{align*}
    \item[Case 2:] Let $s > 0$. For $z_1,\dots,z_t\in N$ \ref{set_function_N}, we need to show that
      \begin{align*}
        s\nu\left(\alpha_0\right) + \sum_{j=1}^{n}\nu\left(\alpha_j\right) \leq z_1 + \cdots + z_t + \sum_{j=1}^{n}\nu\left(\beta_j\right).
      \end{align*}
      Without loss of generality, we can set $z_1,\dots,z_n = z$, as for each $z\in N$, $z \geq \nu\left(\alpha_0\right)$.\newline

      As in Case 1, we multiply \ref{set_function_conditionii} by $r$, add $t\delta_{1} + \cdots + t\delta_q$ to both sides, and substitute with \ref{set_function_id1}, yielding
      \begin{align*}
        r\alpha_1 + \cdots + r\alpha_n + rs\alpha_0 + t\delta_1 + \cdots + t\delta_q &\leq r\beta_1 + \cdots + r\beta_m + t\left(r\alpha_0\right) + t\delta_1 + \cdots + t\delta_q\\
        r\alpha_1 + \cdots + r\alpha_n + t\delta_1 + \cdots + t\delta_q + rs\alpha_0 &\leq r\beta_1 + \cdots + r\beta_m + t\gamma_1 + \cdots + t\gamma_p.
      \end{align*}
      Defining
      \begin{align*}
        z &= \frac{1}{r}\left(\sum_{j=1}^{p}\nu\left(\gamma_j\right) - \sum_{j=1}^{q}\nu\left(\delta_j\right)\right),
      \end{align*}
      we get
      \begin{align*}
        s\nu\left(\alpha_0\right) + \sum_{j=1}^{n}\nu\left(\alpha_j\right) &\leq \sum_{j=1}^{n}\nu\left(\alpha_j\right) + \frac{s}{sr}\left(r\sum_{j=1}^{m}\nu\left(\beta_j\right) - r\sum_{j=1}^{n}\nu\left(\alpha_j\right) + t\sum_{j=1}^{p}\nu\left(\gamma_j\right) - t\sum_{j=1}^{q}\nu\left(\delta_j\right)\right)\\
                                                                           &= tz + \sum_{j=1}^{m}\nu\left(\beta_j\right).
      \end{align*}
  \end{description}
  Thus, we have shown that $\nu$ extends in a manner that satisfies conditions (i) and (ii).
\end{proof}

We can ``upgrade'' our finitely additive set function to a semigroup homomorphism as follows.
\begin{theorem}[{\cite[Theorem 0.2.10]{lectures_on_amenability}}]\label{thm:homomorphism_existence}
  Let $\left(\mathcal{S},+\right)$ be a commutative semigroup with identity element $0$, and let $\epsilon\in \mathcal{S}$. Then, the following are equivalent:
  \begin{enumerate}[(i)]
    \item $\left(n+1\right)\epsilon \leq n\epsilon$ for all $n\in \N$;
    \item there is a semigroup homomorphism $\nu\colon \left(\mathcal{S},+\right)\rightarrow \left([0,\infty],+\right)$ such that $\nu(\epsilon) = 1$.
  \end{enumerate}
\end{theorem}
\begin{proof}
  To show that (ii) implies (i), we let $\nu\colon \left(\mathcal{S},+\right)\rightarrow \left([0,\infty],+\right)$ be a semigroup homomorphism with $\nu\left(\epsilon\right) = 1$. Then,
  \begin{align*}
    \nu\left(\left(n+1\right)\epsilon\right) &= \left(n+1\right)\nu\left(\epsilon\right)\\
                                             &= n+1\\
                                             &> n\\
                                             &= n\nu\left(\epsilon\right)\\
                                             &= \nu\left(n\epsilon\right),
  \end{align*}
  meaning that $\left(n+1\right)\epsilon \nleq n\epsilon$.\newline

  To show that (i) implies (ii), we suppose that for each $\alpha \in \mathcal{S}$, there is $n\in \N$ such that $\alpha \leq n\epsilon$ --- for any such $\alpha$ for which this is not the case, we define $\nu\left(\alpha\right) = \infty$.\newline

  For a finite subset $\mathcal{S}_0 \subseteq \mathcal{S}$ with $\epsilon\in \mathcal{S}_0$, we define $M_{\mathcal{S}_0}$ to be the set of all $\kappa\colon \mathcal{S}\rightarrow [0,\infty]$ such that
  \begin{itemize}
    \item $\kappa\left(\epsilon\right) = 1$;
    \item $\kappa\left(\alpha + \beta\right) = \kappa\left(\alpha\right) + \kappa\left(\beta\right)$ for $\alpha,\beta,\alpha + \beta\in \mathcal{S}_0$.
  \end{itemize}
  Since we assume condition (i), we know that such a $\kappa$ with $\kappa\left(\epsilon\right) = 1$ exists. Additionally, since
  \begin{align*}
    \alpha + \beta \leq \left(\alpha + \beta\right)
  \end{align*}
  and
  \begin{align*}
    \left(\alpha + \beta\right) \leq \alpha + \beta,
  \end{align*}
  it is the case that
  \begin{align*}
    \kappa\left(\alpha + \beta\right) \leq \kappa\left(\alpha\right) + \kappa\left(\beta\right) \leq \kappa\left(\alpha + \beta\right),
  \end{align*}
  meaning $\kappa\left(\alpha + \beta\right) = \kappa\left(\alpha\right) + \kappa\left(\beta\right)$. Thus, $M_{\mathcal{S}_0}$ is nonempty. It is also the case that $M_{\mathcal{S}_0}$ is closed, since any net of functions $\kappa_{p}\colon \mathcal{S}\rightarrow [0,\infty]$ with $\kappa_{p}\left(\epsilon\right) = 1$ and $\kappa_{p}\left(\alpha + \beta\right) = \kappa_{p}\left(\alpha\right) + \kappa_{p}\left(\beta\right)$ will necessarily satisfy these conditions in the limit.\newline

  We let $\left[0,\infty\right]^{\mathcal{S}} = \set{\kappa| \kappa:\mathcal{S}\rightarrow [0,\infty]}$ be equipped with the product topology. By Tychonoff's theorem, $\left[0,\infty\right]^{\mathcal{S}}$ is compact.\newline

  Furthermore, for any finite subcollection $\mathcal{S}_1,\dots,\mathcal{S}_n$, it is the case that
  \begin{align*}
    M_{\mathcal{S}_1\cup\cdots\cup \mathcal{S}_n} \subseteq M_{\mathcal{S}_1} \cap \cdots \cap M_{\mathcal{S}_n},
  \end{align*}
  as any such $\kappa\in M_{\mathcal{S}_1\cup\cdots\cup \mathcal{S}_n}$ must necessarily be in every $M_{\mathcal{S}_i}$.\newline

  Thus, the family
  \begin{align*}
    \mathcal{M} &= \set{M_{\mathcal{S}_0}| \mathcal{S}_0\subseteq \mathcal{S}\text{ finite}}
  \end{align*}
  has the finite intersection property. By compactness, there is some $\nu$ such that
  \begin{align*}
    \nu\in \bigcap \mathcal{M}
  \end{align*}
  with $\nu\left(\epsilon\right) = 1$ and, for all $\alpha,\beta\in \mathcal{S}$, since $\nu\in M_{\set{\alpha,\beta,\alpha + \beta}}$, $\nu\left(\alpha + \beta\right) = \nu\left(\alpha\right) + \nu\left(\beta\right)$.
\end{proof}
\section{Proof of Tarski's Theorem}%
Finally, we are able to prove the reverse direction of Tarski's Theorem. We restate the theorem before giving its proof.
\begin{tcolorbox}[blanker,breakable,left=3mm,before skip=10pt, after skip=10pt, borderline west={1pt}{0pt}{blue!50!white},sharp corners,]
\tarski*
\end{tcolorbox}
\begin{proof}[Proof of the Reverse Direction of Theorem \ref{thm:tarski}:]
  Let $\mathcal{S}$ be the type semigroup of the action of $G$ on $X$.\newline

  Suppose $E$ is not $G$-paradoxical. Then, $\left[E\right]\neq 2\left[E\right]$ by Fact \ref{fact:type_semigroup_paradoxicality}, meaning $\left(n+1\right)\left[E\right]\nleq n\left[E\right]$ for all $n\in \N$ by the contrapositive of Corollary \ref{cor:non_paradoxicality}.\newline

  Thus, by Theorem \ref{thm:homomorphism_existence}, there is a map $\nu\colon \mathcal{S}\rightarrow [0,\infty]$ with $\nu\left(\left[E\right]\right) = 1$. The map $\mu\colon P(X)\rightarrow [0,\infty]$ defined by
  \begin{align*}
    \mu\left(A\right) &= \nu\left(\left[A\right]\right)
  \end{align*}
  is the desired finitely additive measure.
\end{proof}
