While group amenability and the existence of an invariant state on the group are equivalent, as well as showing the group is non-paradoxical, we are interested in establishing a more combinatorial characterization for group amenability. This will eventually involve showing the equivalence between the existence of an invariant state, the Følner condition, and the existence of an approximate mean. We then take a tour of the essence of geometric group theory to apply Følner's condition to to the groups of subexponential growth.
\section{Følner's Condition}%
\begin{definition}\label{def:folner_condition}
  A group is said to satisfy the {Følner condition} if, for every $\ve > 0$ and $E\subseteq G$, there is a nonempty finite subset $F\subseteq G$ such that for all $t\in E$,
  \begin{align*}
    \frac{\left\vert tF\triangle F \right\vert}{\left\vert F \right\vert}\leq \ve.
  \end{align*}
  Equivalently, we can also say that the Følner condition is satisfied if and only if
  \begin{align*}
    \frac{\left\vert tF\cap F \right\vert}{\left\vert F \right\vert} \geq 1 - \ve
  \end{align*}
  for every $\ve > 0$.
\end{definition}
\begin{lemma}\label{lemma:folner_sequences}
  A countable group $G$ satisfies the Følner condition if and only if $G$ admits a sequence $\left(F_n\right)_n$ with $F_n\subseteq G$ finite such that
  \begin{align*}
    \left(\frac{\left\vert tF_n\triangle F_n \right\vert}{\left\vert F_n \right\vert}\right)_n \xrightarrow{n\rightarrow \infty}0
  \end{align*}
  for all $t\in G$. Such a sequence is known as a Følner sequence.
\end{lemma}
\begin{proof}
  Let $G$ admit a Følner sequence, $\left(F_n\right)_n$. Given $\ve > 0$ and $E\subseteq G$ finite, find $N$ such that for all $s\in E$ and $n\geq N$,
  \begin{align*}
    \frac{\left\vert sF_n\triangle F_n \right\vert}{\left\vert F_n \right\vert} &\leq \ve.
  \end{align*}
  We take $F = F_N$ in the definition of the Følner condition.\newline

  Let $G$ satisfy the Følner condition. We write $G = \bigcup_{n\geq 1}E_n$, with $E_1\subseteq E_2\subseteq \cdots$, and define $F_n$ such that for all $t\in E_n$,
  \begin{align*}
    \frac{\left\vert tF_n\triangle F_n \right\vert}{\left\vert F_n \right\vert} &\leq \frac{1}{n}.
  \end{align*}
  Given $t\in G$, then $t\in E_N$ for some $N$, so $t\in E_n$ For all $n\geq N$, so
  \begin{align*}
    \frac{\left\vert tF_n\triangle F_n \right\vert}{\left\vert F_n \right\vert} &\leq \frac{1}{n}
  \end{align*}
  for all $n\geq N$. Thus,
  \begin{align*}
    \left(\frac{\left\vert tF_n\triangle F_n \right\vert}{\left\vert F_n \right\vert}\right)\xrightarrow{n\rightarrow\infty}0.
  \end{align*}
\end{proof}
\begin{lemma}\label{lemma:folner_condition_generating_set}
  Let $G$ be a finitely generated group with generating set $S$ (see Definition \ref{def:generating_sets}). If $\left(F_n\right)_n$ is a sequence of finite subsets such that, for all $s\in S$,
  \begin{align*}
    \left(\frac{\left\vert sF_n\triangle F_n \right\vert}{\left\vert F_n \right\vert}\right)_n\rightarrow 0,
  \end{align*}
  then $\left(F_n\right)_n$ is a Følner sequence for $G$.
\end{lemma}
\begin{proof}
  Note that
  \begin{itemize}
    \item $s\left(A\triangle B\right) = sA\triangle sB$;
    \item $A\triangle C \subseteq \left(A\triangle B\right) \cup \left(B\triangle C\right)$.
  \end{itemize}
  We see that for any $s\in S$,
  \begin{align*}
    \frac{\left\vert s^{-1}F_n\triangle F_n \right\vert}{\left\vert F_n \right\vert} &= \frac{\left\vert s^{-1}\left(F_n\triangle sF_n\right) \right\vert}{\left\vert F_n \right\vert}\\
                                                                                     &= \frac{\left\vert F_n\triangle sF_n \right\vert}{\left\vert F_n \right\vert}\\
                                                                                     &\rightarrow 0.
  \end{align*}
  Thus, we may assume that $S$ is symmetric --- i.e., that $\set{s^{-1}| s\in S} = \set{s | s\in S}$.\newline

  For any $s,t\in S$, we have
  \begin{align*}
    \frac{\left\vert stF_n\triangle F_n \right\vert}{\left\vert F_n \right\vert} &\leq \frac{\left\vert stF_n\triangle F_n \right\vert}{\left\vert F_n \right\vert} + \frac{\left\vert sF_n\triangle F_n \right\vert}{\left\vert F_n \right\vert}\\
                                                                                 &= \frac{\left\vert s\left(tF_n\triangle F_n\right) \right\vert}{\left\vert F_n \right\vert} + \frac{\left\vert sF_n\triangle F_n \right\vert}{\left\vert F_n \right\vert}\\
                                                                                 &= \frac{\left\vert tF_n\triangle F_n \right\vert}{\left\vert F_n \right\vert} + \frac{\left\vert sF_n\triangle F_n \right\vert}{\left\vert F_n \right\vert}\\
                                                                                 &\rightarrow 0.
  \end{align*}
  We use induction to find the general case.
\end{proof}
\begin{example}
  Consider the group $\Z$. Since $\Z$ is generated by the element $\set{1}$, we see that for $F_n = [-n,n]$, that
  \begin{align*}
    \frac{\left\vert \left(F_n + 1\right)\triangle F_n \right\vert}{\left\vert F_n \right\vert} &= \frac{2}{2n+1}\\
                                                                                                &\rightarrow 0,
  \end{align*}
  meaning that $\Z$ satisfies the Følner condition.
\end{example}
\section{From Følner's Condition to Amenability: Enter Approximate Means}%
We have thus far proven that $G$ satisfies the Følner condition if and only if $G$ admits a Følner sequence, and that $G$ is amenable if and only if $G$ admits an invariant state.\newline

We will now begin harmonizing these two characterizations through the use of approximate means, eventually showing that $G$ satisfies the Følner condition if and only if $G$ admits an approximate mean, and that $G$ admits an approximate mean if and only if $G$ is amenable.
\begin{definition}\label{def:state_on_prob_g}
  For a group $G$, we define
  \begin{align*}
    \Prob\left(G\right) = \set{f\colon G\rightarrow [0,\infty) | \Card\left(\supp(f)\right)  < \infty,~\sum_{t\in G}f(t) = 1}.
  \end{align*}
  Note that $\Prob(G) \subseteq B_{\ell_1\left(G\right)}$. For $f\in \prob(G)$, we set $\varphi_f\colon \ell_{\infty}(G)\rightarrow \C$ defined by
  \begin{align*}
    \varphi_f\left(g\right) &= \sum_{t\in G}g(t)f(t).
  \end{align*}
\end{definition}
\begin{fact}\label{fact:prob_g_state}
  For $f\in \prob(G)$, $\varphi_f$ is a state on $\ell_{\infty}\left(G\right)$.
\end{fact}
\begin{proof}
We can see that, by definition, $\varphi_f$ is positive, linear, and has $\varphi_f\left(\1_{G}\right) = 1$.\newline

We only need to show that $\norm{\varphi_f} = 1$. We see that
\begin{align*}
  \left\vert \varphi_f\left(g\right) \right\vert &= \left\vert \sum_{t\in G}g(t)f(t) \right\vert\\
                                                 &\leq \sum_{t\in G}\left\vert g(t) \right\vert\left\vert f(t) \right\vert\\
                                                 &\leq \norm{g}_{\infty}\sum_{t\in G}\left\vert f(t) \right\vert\\
                                                 &= \norm{g}_{\infty},
\end{align*}
so $\norm{\varphi_f} \leq 1$. Since $\varphi_f\left(\1_G\right) = 1$, we must have $\norm{\varphi_f} = 1$.
\end{proof}
\begin{proposition}
  There is an action $\lambda\colon G\rightarrow \Isom\left(\ell_{1}\left(G\right)\right)$ such that $\prob(G)$ is invariant.
\end{proposition}
\begin{proof}
  Let $\lambda_s\left(f\right)\left(t\right) = f\left(s^{-1}t\right)$. Then,
  \begin{align*}
    \norm{\lambda_s\left(f\right)}_1 &= \sum_{t\in G}\left\vert \lambda_s\left(f\right)\left(t\right) \right\vert\\
                                     &= \sum_{t\in G}\left\vert f\left(s^{-1}t\right) \right\vert\\
                                     &= \sum_{r\in G}\left\vert f(r) \right\vert\\
                                     &= \norm{f}_{1}.
  \end{align*}
  Just as in Proposition \ref{prop:translation_action}, it is the case that $\lambda_s$ is linear. Additionally,
  \begin{align*}
    \lambda_r\circ \lambda_s\left(f\right)\left(t\right) &= \lambda_s\left(f\right)\left(r^{-1}t\right)\\
                                                         &= f\left(s^{-1}r^{-1}\left(t\right)\right)\\
                                                         &= f\left(\left(rs\right)^{-1}t\right)\\
                                                         &= \lambda_{rs}\left(f\right)\left(t\right).
  \end{align*}
  We see that if $f\in \prob(G)$, then for $f\geq 0$, we have $\lambda_s\left(f\right) \geq 0$, and
  \begin{align*}
    \sum_{t\in G}\lambda_s\left(f\right)\left(t\right) &= \sum_{t\in G}f\left(s^{-1}t\right)\\
                                                       &= \sum_{r\in G}f\left(r\right)\\
                                                       &= 1
  \end{align*}
  for any $f\in \prob(G)$.
\end{proof}
\begin{definition}\label{def:approximate_mean}
  For a countable group $G$, a sequence $\left(f_k\right)_k$ is called an approximate mean if, for all $s\in G$,
  \begin{align*}
    \norm{f_k - \lambda_s\left(f_k\right)}_{1} &\xrightarrow{k\rightarrow \infty}0.
  \end{align*}
\end{definition}
To begin the forward direction regarding the equivalence between the Følner condition, approximate means, and means, we begin by showing that the existence of a Følner sequence implies the existence of an approximate mean. Then, we will show that the existence of an approximate mean implies the existence of an invariant state (hence mean).
\begin{proposition}\label{prop:folner_implies_approx_mean}
  If $G$ admits a Følner sequence $\left(F_k\right)_k$, then $G$ admits an approximate mean.
\end{proposition}
\begin{proof}
  Set $f_k = \frac{1}{\left\vert F_k \right\vert}\1_{F_k}\in \prob(G)$. Then,
  \begin{align*}
    \norm{f_k - \lambda_s\left(f_k\right)}_{1} &= \frac{1}{\left\vert f_k \right\vert} \norm{\1_{F_k} - \lambda_s\left(\1_{F_k}\right)}\\
                                               &= \frac{1}{F_k}\norm{\1_{F_k} - \1_{sF_k}}\\
                                               &= \frac{\left\vert F_k\triangle sF_k \right\vert}{\left\vert F_k \right\vert}\\
                                               &\rightarrow 0.
  \end{align*}
\end{proof}
\begin{proposition}\label{prop:approx_mean_implies_amenable}
  If $G$ admits an approximate mean, then $G$ is amenable.
\end{proposition}
\begin{proof}
  Let $\left(f_k\right)_k$ be an approximate mean. We define $\varphi_k = \left(\varphi_{f_k}\right)_k$ (as in Definition \ref{def:state_on_prob_g}) to be a sequence of states on $\ell_{\infty}\left(G\right)$.\newline

  Since the state space on $\ell_{\infty}\left(G\right)$ is $w^{\ast}$-compact, there is a state $\mu$ and a subnet $\left(\varphi_{k_j}\right)_j \xrightarrow{w^{\ast}}\mu$. \newline

  We only need to show that $\mu$ is invariant. Note that
  \begin{align*}
    \left\vert \mu\left(g\right) - \mu\left(\lambda_s\left(g\right)\right) \right\vert &\leq \left\vert \mu\left(g\right) - \varphi_{k_j}\left(g\right) \right\vert + \left\vert \varphi_{k_j}\left(g\right) - \varphi_{k_j}\left(\lambda_s\left(g\right)\right) \right\vert + \left\vert \varphi_{k_j}\left(\lambda_s\left(g\right)\right) - \mu\left(\lambda_s\left(g\right)\right) \right\vert
  \end{align*}
  for all $g\in \ell_{\infty}\left(G\right)$, $s\in G$, and all $j$.\newline

  Given $\ve > 0$, we find $J$ such that for $j\geq J$,
  \begin{align*}
    \left\vert \mu\left(g\right) - \varphi_{k_j}\left(g\right) \right\vert &< \ve/3\\
    \left\vert \mu\left(\lambda_s\left(g\right)\right) - \varphi_{k_j}\left(\lambda_s\left(g\right)\right)\right\vert &< \ve/3.
  \end{align*}
  We also see that
  \begin{align*}
    \left\vert \varphi_{k_j}\left(g\right) - \varphi_{k_j}\left(\lambda_s\left(g\right)\right) \right\vert &= \left\vert \sum_{t\in G}g(t)f_{k_j}\left(t\right) - \sum_{t\in G}g\left(s^{-1}t\right)f_{k_j}\left(t\right) \right\vert\\
                                                                                                           &= \left\vert \sum_{t\in G}g(t)f_{k_j}\left(t\right) - \sum_{r\in G}g(r)f_{k_j}\left(sr\right) \right\vert \tag*{$r = s^{-1}t$}\\
                                                                                                           &= \left\vert \sum_{t\in G}g(t)\left(f_{k_j}\left(t\right)-\lambda_{s^{-1}}\left(f_{k_j}\right)\left(t\right)\right) \right\vert\\
                                                                                                           &\leq \norm{g}_{\infty}\sum_{t\in G}\left\vert f_{k_j}\left(t\right) - \lambda_{s^{-1}}\left(f_{k_j}\right)\left(t\right) \right\vert\\
                                                                                                           &= \norm{g}_{\infty}\norm{f_{k_j} - \lambda_{s^{-1}}\left(f_{k_j}\right)}_{1}\\
                                                                                                           &< \ve/3
  \end{align*}
  for large $j$. Thus, we have
  \begin{align*}
    \left\vert \mu\left(g\right) - \mu\left(\lambda_{s}\left(g\right)\right) \right\vert &< \ve,
  \end{align*}
  for all $\ve > 0$, so $\mu\left(g\right) = \mu\left(\lambda_{s}\left(g\right)\right)$.
\end{proof}

We will now commence with the reverse direction, starting by showing that amenability implies the existence of an approximate mean, and then showing that the existence of an approximate mean implies that the Følner condition is satisfied.
\begin{proposition}\label{prop:amenable_implies_approx_mean}
  If $G$ is amenable, then $G$ admits an approximate mean.
\end{proposition}
\begin{proof}
  Suppose $G$ does not admit an approximate mean. Then, there exists a finite subset $E_0\subseteq G$ and $\ve_0 > 0$ such that for all $s\in E_0$ and all $f\in \Prob(G)$, we have $\norm{f - \lambda_s\left(f\right)} \geq \ve_0$.\newline

  Consider the set
  \begin{align*}
    X &= \bigoplus_{\left\vert E_0 \right\vert} \ell_1\left(G\right),
  \end{align*}
  endowed with the norm
  \begin{align*}
    \norm{\left(f_s\right)_{s\in E_0}} &= \sum_{s\in E_0}\sum_{t\in G}\left\vert f_s(t) \right\vert\\
                                       &= \sum_{s\in E_0}\norm{f_s}_{1},
  \end{align*}
  and let
  \begin{align*}
    C &= \set{\left(f - \lambda_s\left(f\right)\right)_{s\in E_0} | f\in \Prob(G)}.
  \end{align*}
  Since $\Prob(G)$ is convex, it is the case that $C$ is convex, and since $\left\vert E_0 \right\vert$ is finite, $C$ is necessarily bounded. Note that $0\notin \overline{C}$.\newline

  By the Hahn--Banach separation for convex sets (Theorem \ref{thm:hb_separation_lctvs}), there is a real-valued $\varphi\in X^{\ast}$ such that $\varphi\left(C\right)\geq 1$. Here,
  \begin{align*}
    X^{\ast} &\cong \bigoplus_{\left\vert E_0 \right\vert}\ell_1\left(G\right)^{\ast}\\
             &\cong \sum_{\left\vert E_0 \right\vert}\ell_{\infty}\left(G\right),
  \end{align*}
  endowed with the norm
  \begin{align*}
    \norm{\left(g_s\right)_{s\in E_0}} &= \max_{s\in E_0}\left(\sup_{t\in G}\left\vert g_s(t) \right\vert\right)\\
                                       &= \max_{s\in E_0}\norm{g_s}_{\infty}.
  \end{align*}
  We let $\varphi = \left(\varphi_{g_s}\right)_{s\in E_0}$, where $g_s\in \ell_{\infty}\left(G\right)$ is defined by the duality
  \begin{align*}
    \varphi_{g_s}\left(f\right) &= \sum_{t\in G}f(t)g_s(t).
  \end{align*}
  Thus, for all $f\in \Prob(G)$, we have
  \begin{align*}
    1 &\leq \varphi\left(\left(f - \lambda_s\left(f\right)\right)_{s\in E_0}\right)\\
      &= \sum_{s\in E_0}\varphi_{g_s}\left(f - \lambda-s\left(f\right)\right)\\
      &= \sum_{s\in E_0}\sum_{t\in G}\left(f - \lambda_s\left(f\right)\right)(t)g_s(t)\\
      &= \sum_{s\in E_0}\left(\sum_{t\in G}f(t)g_s(t) - \sum_{t\in G}f\left(s^{-1}t\right)g_s(t)\right)\\
      &= \sum_{s\in E_0}\left(\sum_{t\in G}f(t)g_s(t) - \sum_{r\in G}f\left(r\right)g_s\left(sr\right)\right)\\
      &= \sum_{s\in E_0}\left(\sum_{r\in G}f(r)g_s(r) - \sum_{r\in G}f(r)\lambda_{s^{-1}}\left(g\right)(r)\right)\\
      &= \sum_{s\in E_0}\sum_{r\in G}f(r)\left(g_s - \lambda_{s^{-1}}\left(g_s\right)\right)(r).
      \intertext{Note that this holds for any $f\in \Prob(G)$, including the case of $f = \delta_t$ for a given $t\in G$. We must have}
      &= \sum_{s\in E_0}\sum_{r\in G}\delta_{t}\left(r\right)\left(g_s\left(r\right) - \lambda_{s^{-1}}\left(g_s\right)\right)\left(r\right)\\
      &= \sum_{s\in E_0}\left(g_s - \lambda_{s^{-1}}\left(g\right)\right)\left(t\right),
      \intertext{and in particular,}
      &\geq \1_{G}.
  \end{align*}
  Since $G$ is amenable, there is a mean $\mu\colon \ell_{\infty}\left(G\right)\rightarrow \C$ with $\mu\left(g_s\right) = \mu\left(\lambda_{s^{-1}}\left(g_s\right)\right)$, meaning
  \begin{align*}
    0 &= \mu\left(\sum_{s\in E_0}\left(g_s - \lambda_{s^{-1}}\left(g_s\right)\right)\left(t\right)\right)\\
      &\geq \mu\left(\1_{G}\right)\\
      &= 1,
  \end{align*}
  which is a contradiction.
\end{proof}
To show that the existence of an approximate mean implies the Følner condition, we require the following lemma.
\begin{lemma}\label{lemma:layer_cake_representation}
  Let $f\colon S\rightarrow \R$ be finitely supported with $\sum_{s\in S}f(s) = 1$. Then, there exist subsets $\set{F_i}_{i=1}^{n}$, where $F_1\supseteq F_2\supseteq \cdots \supseteq F_n$, and constants $\set{c_i}_{i=1}^{n}$, such that
  \begin{align*}
    f &= \sum_{i=1}^{n}c_i\1_{F_i},
  \end{align*}
  where
  \begin{align*}
    \sum_{i=1}^{n}c_i\left\vert F_i \right\vert &= 1.
  \end{align*}
  This is known as the layer cake representation for $f$.
\end{lemma}
\begin{proof}
  We define $F_1 = \supp\left(f\right)$, and take $c_1 = \min\left(\Ran\left(f\right)\right)$. Taking $E_1 = f^{-1}\left(c_1\right)$ (as a set-theoretic inverse), we define $F_2 = F_1\setminus E_1$.\newline

  Take $d_1 = \min\left(f\left(F_2\right)\right)$, and define $c_2 = d_1 - c_1$. Then, defining $E_2 = f^{-1}\left(d_1\right)$, $F_3 = F_2 \setminus E_2$, and $d_2 = \min\left(f\left(F_3\right)\right)$, we define $c_3 = d_2 - c_2 - c_1$.\newline

  Continuing in this pattern, we find $d_{i-1} = \min\left(f\left(F_i\right)\right)$, $E_i = f^{-1}\left(d_{i-1}\right)$, and $c_i = d_{i-1} - \sum_{j=1}^{i-1}c_i$.\newline

  This yields a decomposition $F_1\supseteq F_2\supseteq \cdots \supseteq F_n$, where $\sum_{i=1}^{n}c_i\1_{F_i} = f$ by construction.\newline

  We now verify that $\sum_{i=1}^n c_i\left\vert F_i \right\vert = 1$.
  \begin{align*}
    1 &= \sum_{s\in S}f(s)\\
      &= \sum_{s\in S}\sum_{i=1}^{n}c_i\1_{F_i}\left(s\right).
      \intertext{By definition, if $s\in F_j$ for some $j$, we see that $c_j$ is summed for $\left\vert F_j \right\vert$ many times. Thus, we obtain}
      &= \sum_{i=1}^{n}c_i\left\vert F_i \right\vert.
  \end{align*}
\end{proof}
\begin{remark}
  Instead of using this construction where we take set-theoretic inverses and remove ``residual'' sets, there is an alternative method of construction that involves ordering the range as $r_1 < r_2< \cdots < r_n$, and constructing the set $F_k = \set{s | f(s) \geq r_k}$.
\end{remark}


We will use the layer cake decomposition to prove that if $G$ admits an approximate mean, then $G$ satisfies the Følner condition.
\begin{proposition}\label{prop:approx_mean_implies_folner}
  Let $G$ admit an approximate mean. Then, $G$ satisfies the Følner condition.
\end{proposition}
\begin{proof}
  Let $\left(f_k\right)_k$ be an approximate mean, as in Definition \ref{def:approximate_mean}. Fix a finite nonempty set $S \subseteq G$. Then, by the definition of an approximate mean, there must exist some $N\in\N$ such that for all $k\geq N$ and all $s\in G$,
  \begin{align*}
    \norm{f_k - \lambda_s\left(f_k\right)}_{1} &\leq \frac{\ve}{|S|}.
  \end{align*}
  In particular, this holds for $f_N$ and for all $s\in S$.\newline

  Since $f_N\in \Prob(G)$ is finitely supported and $\sum_{s\in G}f_N(s) = 1$, we may use Lemma \ref{lemma:layer_cake_representation} to rewrite $f_N$ as
  \begin{align*}
    f_N &= \sum_{i=1}^{n}c_i\1_{F_i},
  \end{align*}
  where $F_1 \supseteq F_2\supseteq \cdots \supseteq F_n$, and $\sum_{i=1}^{n}c_i\left\vert F_i \right\vert = 1$.\newline

  For a given $1 \leq i \leq n$, for each $s\in S$ and $t\in sF_i\triangle F_i$, we have
  \begin{align*}
    f_N\left(t\right) - f_N\left(s^{-1}t\right) &= \begin{cases}
      c_i & t\in F_i\setminus sF_i\\
      -c_i & t\in sF_i \setminus F_i
    \end{cases}.
  \end{align*}
  Thus, we see that $\left\vert f_N\left(t\right)-\lambda_s\left(f_N\right)\left(t\right) \right\vert\geq c_i$ on $sF_i\triangle F_i$. Thus, for each $s\in S$,
  \begin{align*}
    \sum_{i=1}^{n}c_i\left\vert sF_i \triangle F_i \right\vert &\leq \sum_{t\in S}\left\vert f_N\left(t\right) -  \lambda_s\left(f\right)\left(t\right)\right\vert\\
                                                               &< \frac{\ve}{\left\vert S \right\vert}\\
                                                               &= \frac{\ve}{\left\vert S \right\vert} \sum_{i=1}^{n}c_i\left\vert F_i \right\vert.
  \end{align*}
  Therefore, we have
  \begin{align*}
    \sum_{s\in S}\sum_{i=1}^{n}c_i\left\vert sF_i\triangle F_i \right\vert &< \frac{\ve}{\left\vert S \right\vert}\sum_{s\in S}\sum_{i=1}^{n}c_i\left\vert F_i \right\vert\\
                                                                           &= \ve \sum_{i=1}^{n}c_i\left\vert F_i \right\vert.
  \end{align*}
  Thus, by the pigeonhole principle, there must exist some $1\leq i \leq n$ for which
  \begin{align*}
    \sum_{s\in S}c_i\left\vert sF_i\triangle F_i \right\vert < \ve c_i\left\vert F_i \right\vert.
  \end{align*}
  Setting $F = F_i$, we find that, for all $s\in S$,
  \begin{align*}
    \frac{\left\vert sF\triangle F \right\vert}{\left\vert F \right\vert} &\leq \sum_{s\in S}\frac{\left\vert sF\triangle F \right\vert}{\left\vert F \right\vert}\\
                                                                          &< \ve.
  \end{align*}
\end{proof}

Thus far, we have shown the following to be equivalent for a discrete group $G$:
\begin{enumerate}[(1)]
  \item $G$ is non-paradoxical;
  \item $G$ is amenable;
  \item $G$ admits an invariant state;
  \item $G$ admits an approximate mean;
  \item $G$ satisfies the Følner condition.
\end{enumerate}
The equivalence between (1) and (2) follows from Theorem \ref{thm:tarski}, the equivalence between (2) and (3) follows from Propositions \ref{prop:state_implies_mean} and \ref{prop:mean_implies_state}, and the equivalence between (3), (4), and (5) follows from Propositions \ref{prop:folner_implies_approx_mean}, \ref{prop:approx_mean_implies_amenable}, \ref{prop:amenable_implies_approx_mean}, and \ref{prop:approx_mean_implies_folner}.
\section{Applying Følner's Condition: Groups of Subexponential Growth}%
Before we move to Chapters 7 and 8 to discuss representations of groups inside the algebra of bounded operators on a Hilbert space, we will provide an application of Følner's condition by taking a tour into geometric group theory. In this section, we will establish the amenability of yet another wide class of groups (just as we established that all abelian groups are amenable in Chapter 5) --- the groups of subexponential growth.\newline

First, we construct a little bit of machinery to understand the growth rate of a group, then we prove that Følner's condition holds for these special classes of groups.
\begin{definition}\label{def:word_metric}
  Let $G$ be a group with finite symmetric generating set $S$ (see Definition \ref{def:generating_sets}). Then, we define the word length of $g\in G$ with respect to $S$ to be
  \begin{align*}
    \ell_{G,S}\left(g\right) &= \min\set{n | g = s_1\dots s_n,~s_i\in S},
  \end{align*}
  taking $\ell_{G,S}\left(e_G\right) = 0$. We define the word metric on $G$ with respect to $S$ by taking
  \begin{align*}
    d_{S}\left(g,h\right) &= \ell_{G,S}\left(g^{-1}h\right).
  \end{align*}
\end{definition}
\begin{fact}\label{fact:word_metric_equivalent_metrics}
  If $S$ and $T$ are finite symmetric generating sets for $G$, then the respective word metrics $d_{S}$ and $d_{T}$ are equivalent (as in the sense of Definition \ref{def:metrics_and_equivalent_metrics}).
\end{fact}
\begin{proof}
  We start by showing that $d_S$ is indeed a metric. Notice that the following facts necessarily hold by our definition of the word length:
  \begin{itemize}
    \item $\ell_{G,S}\left(g\right) = \ell_{G,S}\left(g^{-1}\right)$;
    \item $\ell_{G,S}\left(gh\right) \leq \ell_{G,S}\left(g\right) + \ell_{G,S}\left(h\right)$.
  \end{itemize}
  We thus have:
  \begin{align*}
    d_{S}\left(g,h\right) &= \ell_{G,S}\left(g^{-1}h\right)\\
                          &= \ell_{G,S}\left(h^{-1}g\right)\\
                          &= d_S\left(h,g\right)\\
                          \\
    d_{S}\left(g,h\right) &= \ell_{G,S}\left(g^{-1}h\right)\\
                          &= \ell_{G.S}\left(g^{-1}kk^{-1}h\right)\\
                          &\leq \ell_{G,S}\left(g^{-1}k\right) + \ell_{G,S}\left(k^{-1}h\right)\\
                          &= d_{S}\left(g,k\right) + d_{S}\left(k,h\right)\\
                          \\
    d_{S}\left(g,g\right) &= \ell_{G,S}\left(g^{-1}g\right)\\
                          &= \ell_{G,S}\left(e_G\right)\\
                          &= 0\\
    d_{S}\left(g,h\right) = 0 &\Leftrightarrow \ell_{G,S}\left(g^{-1}h\right) = 0\\
                              &\Leftrightarrow g^{-1}h = e_{G}\\
                              &\Leftrightarrow g = h.
  \end{align*}
  Thus, $d_S$ is indeed a metric.\newline

  Let $S$ and $T$ be finite symmetric generating sets for $G$. It is sufficient to show that there exists some $k\in \N$ such that, for all $g\in G$,
  \begin{align*}
    \frac{1}{k}\ell_{G,S}\left(g\right) \leq \ell_{G,T}\left(g\right) \leq k\ell_{G,S}\left(g\right).
  \end{align*}
  Set
  \begin{align*}
    M &= \max\set{\ell_{G,T}\left(s\right) | s\in S}\\
    N &= \max\set{\ell_{G,S}\left(t\right) | t\in T}.
  \end{align*}
  Now, let $n = \ell_{G,S}\left(g\right)$, such that $g = s_1\cdots s_n$, where $s_i\in S$. Then, we have
  \begin{align*}
    \ell_{G,T}\left(g\right) &= \ell_{G,T}\left(s_1\cdots s_n\right)\\
                             &\leq \ell_{G,T}\left(s_1\right) + \cdots + \ell_{G,T}\left(s_n\right)\\
                             &\leq M\ell_{G,S}\left(g\right),
  \end{align*}
  and similarly, $\ell_{G,S}\left(g\right) \leq N\ell_{G,T}\left(g\right)$. Setting $k = \max\left(M,N\right)$, we get
  \begin{align*}
    \frac{1}{k}\ell_{G,S}\left(g\right) \leq \ell_{G,T}\left(g\right) \leq k\ell_{G,S}\left(g\right).
  \end{align*}
\end{proof}
Now, we may begin defining the growth rate of a group. We will use the fact that all word metrics with respect to a generating set are symmetric in order to show that the growth rate is well-defined (i.e., independent of the generating set for $G$).
\begin{definition}
  Let $G$ be a group with finite symmetric generating set $S$. We define
  \begin{align*}
    B_{G,S}\left(n\right) &= \set{g\in G | \ell_{G,S}\left(g\right) \leq n};\\
    \gamma_{G,S}\left(n\right) &= \left\vert B_{G,S}\left(n\right) \right\vert.
  \end{align*}
\end{definition}
The following facts hold for $\gamma$.
\begin{fact}\label{fact:properties_of_gamma_generating_set}
  Let $G$ be a finitely generated group. The following facts hold:
  \begin{enumerate}[(1)]
    \item $\gamma_{G,S}\left(n\right)$ is an increasing function;
    \item $\gamma_{G,S}\left(n+m\right)\leq \gamma_{G,S}\left(n\right)\gamma_{G,S}\left(m\right)$;
    \item $\displaystyle \lim_{n\rightarrow\infty}\left(\gamma_{G,S}\left(n\right)\right)^{1/n} = \rho_{G,S}$ exists;
    \item if $S$ and $T$ are finite symmetric generating sets for $G$, then there exists $k\in \N$ such that $\gamma_{G,T}\left(n\right)\leq \gamma_{G,S}\left(kn\right)$ for all $n\in\N$, and $\rho_{G,S} = \rho_{G,T}$.
  \end{enumerate}
\end{fact}
\begin{proof}\hfill
  \begin{enumerate}[(1)]
    \item Since $B_{G,S}\left(n\right)\subseteq B_{G,S}\left(n+1\right)$, we have $\gamma_{G,S}\left(n\right) \leq \gamma_{G,S}\left(n+1\right)$, so $\gamma_{G,S}$ is increasing.
    \item We start by showing that $B_{G,S}\left(n\right)B_{G,S}\left(m\right) = B_{G,S}\left(n+m\right)$. First, if $g\in B_{G,S}\left(n\right)$ and $h\in B_{G,S}\left(m\right)$, we know that $\ell_{G,S}\left(gh\right) \leq \ell_{G,S}\left(g\right) + \ell_{G,S}\left(h\right)\leq m+n$, so $B_{G,S}\left(n\right)B_{G,S}\left(n\right) \subseteq B_{G,S}\left(n+m\right)$. Additionally, if $g\in B_{G,S}\left(n+m\right)$, we may write
      \begin{align*}
        g &= \underbrace{s_{1}\cdots s_{\ell}}_{g_1}\underbrace{s_{\ell+1}\cdots s_{k}}_{g_2},
      \end{align*}
      where $k\leq n+m$, $\ell \leq n$, and $k-\ell \leq m$, so $g_1\in B_{G,S}\left(n\right)$ and $g_2\in B_{G,S}\left(m\right)$. Thus, we have $B_{G,S}\left(n\right)B_{G,S}\left(m\right) = B_{G,S}\left(n+m\right)$.\newline

      Now, we have
      \begin{align*}
        \gamma_{G,S}\left(n+m\right) &= \left\vert B_{G,S}\left(n+m\right) \right\vert\\
                                     &= \left\vert B_{G,S}\left(n\right)B_{G,S}\left(m\right) \right\vert\\
                                     &\leq \left\vert B_{G,S}\left(n\right) \right\vert\left\vert B_{G,S}\left(m\right) \right\vert\\
                                     &= \gamma_{G,S}\left(n\right)\gamma_{G,S}\left(m\right).
      \end{align*}
    \item From (2), we know that $\gamma_{G,S}\left(n\right) \leq \gamma_{G,S}\left(1\right)^{n}$. Inductively, we have
      \begin{align*}
        \gamma_{G,S}\left(n+1\right) &\leq \gamma_{G,S}\left(1\right)^{n+1},
      \end{align*}
      and thus,
      \begin{align*}
        1 \leq \gamma_{G,S}\left(n\right)^{1/n}\leq \gamma_{G,S}\left(1\right).
      \end{align*}
    \item We know that there exists $k$ such that $\frac{1}{k}\ell_{G,S} \leq \ell_{G,T}\leq k\ell_{G,S}$ by the proof of Fact \ref{fact:word_metric_equivalent_metrics}. Thus, if $g\in B_{G,T}\left(n\right)$, then $\ell_{G,T}\left(g\right) \leq n$, so $\ell_{G,S}\left(g\right) \leq kn$, so $g\in B_{G,S}\left(kn\right)$ and $B_{G,T}\left(n\right)\subseteq B_{G,T}\left(kn\right)$. We have $\gamma_{G,T}\left(n\right)\leq \gamma_{G,S}\left(kn\right)$.\newline

      Similarly, if $g\in B_{G,S}\left(n\right)$, then $\ell_{G,S}\left(g\right)\leq n$, so $\ell_{G,T}\left(g\right) \leq kn$, and $g\in B_{G,T}\left(kn\right)$. Thus, we get $B_{G,S}\left(n\right)\subseteq B_{G,T}\left(kn\right)$, so $\gamma_{G,S}\left(n\right)\leq \gamma_{G,T}\left(kn\right)$.\newline

      It follows that
      \begin{align*}
        \gamma_{G,S}\left(\frac{n}{k}\right)^{1/n} \leq \gamma_{G,T}\left(n\right)^{1/n} \leq \left(\gamma_{G,S}\left(kn\right)^{k}\right)^{1/kn}.
      \end{align*}
      Sending $n\rightarrow\infty$, we get $\rho_{G,S}\leq \rho_{G,T}\leq \rho_{G,S}$, so $\rho_{G,S} = \rho_{G,T}$.
  \end{enumerate}
\end{proof}
\begin{definition}
  Let $G$ be a group with finite symmetric generating set $S$. The quantity
  \begin{align*}
    \rho_{G} &= \limsup_{n\rightarrow\infty}\gamma_{G,S}\left(n\right)^{1/n}
  \end{align*}
  is known as the growth rate of the group $G$. If we have $\rho = 1$, then we say $G$ is of subexponential growth.
\end{definition}
\begin{fact}\label{fact:finite_groups_subexponential_growth}
  All finite groups are of subexponential growth.
\end{fact}
\begin{proof}
Note that since $\rho$ is independent of the generating set (as we proved in Fact \ref{fact:properties_of_gamma_generating_set}), we can set $S = G$, and we have $\limsup_{n\rightarrow\infty} \left\vert G \right\vert^{1/n} = 1$.
\end{proof}
\begin{fact}\label{fact:finitely_generated_abelian_groups_subexponential_growth}
  Let $\Gamma$ be a finitely generated abelian group. Then, $\Gamma$ is of subexponential growth.
\end{fact}
\begin{proof}
  We start by showing that $G = \Z^d$ is of subexponential growth. Notice that every element of $\Z^d$ is some linear combination of the set
  \begin{align*}
    S &= \set{e_1,e_2,\dots,e_d},\label{eq:generating_set_free_abelian_group}\tag{\textasteriskcentered}
  \end{align*}
  where
  \begin{align*}
    e_{j} &= (0,0,\dots,\underbrace{1}_{\text{position $j$}},0,0,\dots).
  \end{align*}
  Additionally, we see that any element of $B_{G,S}(n)$ is of the form $e_1^{i_1}e_2^{i_2}\dots e_d^{i_d}$, where $\sum_{j=1}^{d} i_j \leq n$. Thus, we must have $\gamma_{G,S}(n) \leq n^{d}$, meaning that 
  \begin{align*}
    \rho &= \limsup_{n\rightarrow\infty} \gamma_{G,S}(n)^{1/n}\\
         &= \limsup_{n\rightarrow\infty}n^{d/n}\\
         &= 1,
  \end{align*}
  so $\Z^d$ is of subexponential growth.\newline

  Now, if $G' = \Z^d\times \Z/k_1\Z\times \cdots \times \Z/k_r\Z$, then since there is a torsion subgroup in $G'$, we must have $\gamma_{G',S'}(n) \leq \gamma_{\Z^{d+r},T}(n)$ for any $n$, where $T$ is a generating set for $\Z^{d+r}$ and $S'$ is a generating set for $G'$. Since
  \begin{align*}
    \rho_{\Z^{d+r}} &= \limsup_{n\rightarrow\infty}\gamma_{\Z^{d+r},T}(n)^{1/n}\\
                    &= 1,
  \end{align*}
  and $1 \leq \gamma_{G',S'}(n)$, we must have $\rho_{G'} = 1$.\newline

  Since, by the fundamental theorem of finitely generated abelian groups, it is the case that $\Gamma\cong \Z^{d}\times \Z/k_1\Z\times\cdots\times \Z/k_r\Z$ for some $d,k_1,\dots,k_r\in \N$, $\Gamma$ is of subexponential growth.
\end{proof}
To prove that the groups of subexponential growth are amenable, we use the following lemma from real analysis.
\begin{lemma}
  Let $\left(a_n\right)_n$ be a sequence such that $a_n > 0$ for each $n$. Then,
  \begin{align*}
    \lim_{n\rightarrow\infty}\frac{a_{n+1}}{a_n} &= \lim_{n\rightarrow\infty} \left(a_n\right)^{1/n}.
  \end{align*}
  Similarly,
  \begin{align*}
    \limsup_{n\rightarrow\infty}\frac{a_{n+1}}{a_n} &= \limsup_{n\rightarrow\infty}\left(a_n\right)^{1/n}.
  \end{align*}
\end{lemma}
\begin{theorem}\label{thm:subexponential_growth_implies_amenable}
  Let $\Gamma$ be a finitely generated group of subexponential growth. Then, $\Gamma$ is amenable.
\end{theorem}
\begin{proof}
  To prove that $\Gamma$ is amenable, we show that it satisfies the Følner condition. From the results in Chapter \ref{ch:folner_condition}, we know that this implies that $\Gamma$ is amenable. Let $S$ be a finite symmetric generating set for $\Gamma$.\newline

  For any $\ve > 0$, we see that there is some $k\in \N$ such that
  \begin{align*}
    \left\vert B_{\Gamma,S}\left(k\right) \right\vert^{1/k} &\leq 1 + \ve.
  \end{align*}
  Thus, by the lemma above, we must have
  \begin{align*}
    \frac{\left\vert B_{\Gamma,S}\left(k+1\right) \right\vert}{\left\vert B_{\Gamma,S} \left(k\right)\right\vert} \leq 1 + \ve.
  \end{align*}
  Note that, by Lemma \ref{lemma:folner_condition_generating_set}, we only need to verify that the Følner condition holds on $S$. For any $s\in S$, we have
  \begin{align*}
    \frac{\left\vert sB_{\Gamma,S}\left(k\right)\triangle B_{\Gamma,S}\left(k\right) \right\vert}{\left\vert B_{\Gamma,S}(k) \right\vert} &\leq \frac{2\left(\left\vert B_{G,S}\left(k+1\right) \right\vert - \left\vert B_{\Gamma,S}(k) \right\vert\right)}{\left\vert B_{\Gamma,S}(k) \right\vert}\\
                                                                                                                    &\leq 2\ve.
  \end{align*}
  Therefore, $\Gamma$ satisfies the Følner condition, hence is amenable.
\end{proof}
\begin{remark}
  The result in Theorem \ref{thm:subexponential_growth_implies_amenable} can be used along with Fact \ref{fact:finitely_generated_abelian_groups_subexponential_growth} and Corollary \ref{cor:direct_limit_amenable} to prove Corollary \ref{cor:abelian_groups_amenable}.
\end{remark}
