Having proven Tarski's theorem, we can turn our attention to a more definite understanding of amenability. We will use theorems and techniques from functional analysis to help understand the space $\ell_{\infty}\left(G\right)$, which will open a wide variety of characterizations for amenability, beyond that which was established in Tarski's theorem.
\section{Means and Invariant States}%
\begin{definition}
  Let $G$ be a group, with $P(G)$ denoting its power set.\newline

  An invariant {mean} on $G$ is a set function $m\colon P(G)\rightarrow [0,1]$ which satisfies, for all $t\in G$ and $E,F\subseteq G$,
  \begin{itemize}
    \item $m(G) = 1$;
    \item $m\left(E\sqcup F\right) = m(E) + m(F)$;
    \item $m\left(tE\right) = m\left(E\right)$.
  \end{itemize}
  We say $G$ is amenable if $G$ admits a mean.\newline

  The mean $m$ is, in other words, a translation-invariant probability measure on the measurable space $\left(G,P(G)\right)$.
\end{definition}
We have shown in the proof of Theorem \ref{thm:tarski} that an equivalent condition for amenability is that the group is not paradoxical.\newline

Using some essential results in group theory, we can establish some preliminary results on subgroups and quotient groups.
\begin{proposition}\label{prop:subgroups_quotientgroups_amenability}
  Let $G$ be an amenable group with $H\leq G$. Then, the following are true:
  \begin{enumerate}[(1)]
    \item $H$ is amenable;
    \item for $H\trianglelefteq G$, $G/H$ is amenable.
  \end{enumerate}
\end{proposition}
\begin{proof}\hfill
  \begin{enumerate}[(1)]
    \item Let $R$ be a right transversal for $H$, wherein we select one element of each right coset of $H$ to make up $R$.\newline

      If $m$ is a mean for $G$, we set $\lambda\colon P(H)\rightarrow [0,1]$ defined by
      \begin{align*}
        \lambda(E) = m\left(ER\right).
      \end{align*}
       We have
      \begin{align*}
        \lambda(H) &= m\left(HR\right)\\
                   &= m\left(G\right)\\
                   &= 1.
      \end{align*}
      We claim that if $E\cap F = \emptyset$, then $ER \cap FR = \emptyset$. Suppose toward contradiction this is not the case. Then, $xr_1 = yr_2$ for some $x\in E$, $y\in F$, and $r_1,r_2\in R$. Then, we must have $r_2r_1^{-1} = y^{-1}x \in H$, meaning $r_1 = r_2$ as, by definition, $R$ contains exactly one element of each right coset. Thus, $x=y$, so $E\cap F \neq \emptyset$.\newline

      We then have
      \begin{align*}
        \lambda\left(E\sqcup F\right) &= m\left(\left(E\sqcup F\right)R\right)\\
                                      &= m\left(ER\sqcup FR\right)\\
                                      &= m\left(ER\right) + m\left(FR\right)\\
                                      &= \lambda\left(E\right) + \lambda\left(F\right),
      \end{align*}
      and
      \begin{align*}
        \lambda\left(sE\right) &= m\left(sER\right)\\
                               &= m\left(ER\right)\\
                               &= \lambda\left(E\right).
      \end{align*}
    \item Let $\pi\colon G\rightarrow G/H$ be the canonical projection, defined by $\pi\left(t\right) = tH$. We define
      \begin{align*}
        \lambda\colon P\left(G/H\right) \rightarrow [0,1]
      \end{align*}
      by $\lambda(E) = m\left(\pi^{-1}\left(E\right)\right)$. We have
      \begin{align*}
        \lambda\left(G/H\right) &= m\left(\pi^{-1}\left(G/H\right)\right)\\
                                &= m\left(G\right)\\
                                &= 1,
      \end{align*}
      and
      \begin{align*}
        \lambda\left(E\sqcup F\right) &= m\left(\pi^{-1}\left(E\sqcup F\right)\right)\\
                                      &= m\left(\pi^{-1}\left(E\right)\sqcup \pi^{-1}\left(F\right)\right)\\
                                      &= m\left(\pi^{-1}\left(E\right)\right) + m\left(\pi^{-1}\left(F\right)\right)\\
                                      &= \lambda(E) + \lambda(F).
      \end{align*}
      To show translation-invariance, we let $sH = \pi(s)\in G/H$, and $E\subseteq G/H$. Note that
      \begin{align*}
        \pi^{-1}\left(\pi(s)E\right) &= s\pi^{-1}\left(E\right),
      \end{align*}
      since for $r\in s\pi^{-1}(E)$, we have $r = st$ for $t\in \pi(E)$, so $\pi\left(r\right) =\pi\left(st\right) = \pi\left(s\right)\pi\left(t\right)\in \pi\left(s\right)E$.\newline

      Additionally, if $r\in \pi^{-1}\left(\pi(s)E\right)$, we have $\pi(r)\in \pi(s)E$, so $\pi\left(s^{-1}r\right)\in E$, meaning $s^{-1}r\in \pi^{-1}E$.\newline

      Thus,
      \begin{align*}
        \lambda\left(\pi\left(s\right)E\right) &= m\left(\pi^{-1}\left(\pi\left(s\right)E\right)\right)\\
                                               &= m\left(s\pi^{-1}\left(E\right)\right)\\
                                               &= m\left(\pi^{-1}\left(E\right)\right)\\
                                               &= \lambda\left(E\right).
      \end{align*}
  \end{enumerate}
\end{proof}

Now that we understand some useful properties of means in relation to groups and subgroups, we turn our attention toward finding means on groups. In order to do this, we turn our attention towards the space $\ell_{\infty}\left(G\right)$, which allows us to use theories from functional analysis to better understand means on $G$.
\begin{definition}
  Let $G$ be a group.
  \begin{enumerate}[(1)]
    \item The space $\mathcal{F}\left(G,\R\right)$ is defined by
      \begin{align*}
        \mathcal{F}\left(G,\R\right) &= \set{f | f\colon G\rightarrow \R\text{ is a function}}.
      \end{align*}
    \item A function $f\in \mathcal{F}\left(G,\R\right)$ is called positive if $f(x) \geq 0$ for all $x\in G$.
    \item A function $f\in \mathcal{F}\left(G,\R\right)$ is called simple if $\ran(f)$ is finite. We let
      \begin{align*}
        \Sigma &= \set{f\in \mathcal{F}\left(G,\R\right) | f\text{ is simple}}.
      \end{align*}
  \end{enumerate}
\end{definition}
\begin{fact}
  It is the case that $\Sigma \subseteq \mathcal{F}\left(G,\R\right)$ is a linear subspace.
\end{fact}
\begin{definition}
  For $E\subseteq G$, we define
  \begin{align*}
    \1_{E}\colon G\rightarrow \R
  \end{align*}
  by
  \begin{align*}
    \1_{E}\left(x\right) &= \begin{cases}
      1 & x\in E\\
      0 & x\notin E
    \end{cases}.
  \end{align*}
  This is the characteristic function of $E$.
\end{definition}
\begin{fact}
  We have
  \begin{align*}
    \Span\set{\1_{E}| E\subseteq G} &= \Sigma.
  \end{align*}
\end{fact}
\begin{proof}
  We see that $\1_{E}\in \Sigma$ for any $E\subseteq G$, and that $\Sigma$ is a subspace.\newline

  If $\phi\in \Sigma$ with $\Ran\left(\phi\right) = \set{t_1,\dots,t_n}$, where $t_i$ are distinct, we set
  \begin{align*}
    E_i &= \phi^{-1}\left(\set{t_i}\right),
  \end{align*}
  yielding
  \begin{align*}
    \phi &= \sum_{i=1}^{n}t_i\1_{E_i}.
  \end{align*}
\end{proof}
\begin{definition}\hfill
  \begin{enumerate}[(1)]
    \item A function $f\in \mathcal{F}\left(G,\R\right)$ is bounded if there exists $M > 0$ such that $\Ran\left(f\right) \subseteq \left[-M,M\right]$.
    \item The space $\ell_{\infty}\left(G\right)$ is defined by
      \begin{align*}
        \ell_{\infty}\left(G\right) &= \set{f\in \mathcal{F}\left(G,\R\right)| f\text{ is bounded}}.
      \end{align*}
    \item The norm on $\ell_{\infty}\left(G\right)$ is defined by
      \begin{align*}
        \norm{f} &= \sup_{x\in G}\left\vert f(x) \right\vert.
      \end{align*}
  \end{enumerate}
\end{definition}
\begin{proposition}
  The space $\ell_{\infty}(G)$ is complete. Additionally, $\overline{\Sigma} = \ell_{\infty}\left(G\right)$.
\end{proposition}
\begin{proof}
  Let $\left(f_n\right)_n$ be $\norm{\cdot}$-Cauchy in $\ell_{\infty}\left(G\right)$. Then, for all $x\in G$, it is the case that
  \begin{align*}
    \left\vert f_n(x) - f_m(x) \right\vert &= \left\vert \left(f_n - f_m\right)\left(x\right) \right\vert\\
                                           &\leq \norm{f_n - f_m},
  \end{align*}
  meaning $\left(f_n\left(x\right)\right)_n$ is Cauchy in $\R$. We define $f(x) = \lim_{n\rightarrow\infty}f_n(x)$. We must show that $f\in \ell_{\infty}\left(G\right)$, and $\norm{f_n-f}\rightarrow 0$.\newline

  We have
  \begin{align*}
    \left\vert f(x) \right\vert &= \left\vert \lim_{n\rightarrow\infty}f_n\left(x\right) \right\vert\\
                                &= \lim_{n\rightarrow\infty}\left\vert f_n\left(x\right) \right\vert\\
                                &\leq \limsup_{n\rightarrow\infty}\norm{f_n}\\
                                &\leq C,
  \end{align*}
  as Cauchy sequences are always bounded. Thus, $\sup_{x\in G}\left\vert f(x) \right\vert\leq C$.\newline

  Given $\ve > 0$, we find $N$ such that for all $m,n\geq N$, $\norm{f_n - f_m} \leq \ve$. Thus, for $x\in G$, we have
  \begin{align*}
    \left\vert f_n(x) - f_m(x) \right\vert &\leq \norm{f_n - f_m}\\
                                           &\leq \ve.
  \end{align*}
  Taking $m\rightarrow\infty$, we get $\left\vert f_n(x) - f(x) \right\vert \leq \ve$, for all $n\geq N$, so $\norm{f_n - f}\leq \ve$ for all $n\geq N$.\newline

  For $f\in \ell_{\infty}\left(G\right)$, let $\Ran\left(f\right) \subseteq \left[-M,M\right]$ for some $M > 0$. Let $\ve > 0$. Since $\left[-M,M\right]$ is compact, it is totally bounded, so we can find intervals $I_{1},\dots,I_n$ with $\left[-M,M\right] = \bigsqcup_{k=1}^{n}I_k$, with the length of each $I_k$ less than $\ve$.\newline

  Set $E_k = f^{-1}\left(I_k\right)$. Pick some $t_k\in I_k$. We set
  \begin{align*}
    \phi &= \sum_{i=1}^{n}t_k\1_{E_k}.
  \end{align*}
  Then, it is the case that $\norm{\phi - f} < \ve$.
\end{proof}
\begin{corollary}
  For any $f\in \ell_{\infty}\left(G\right)$, there is a sequence $\left(\phi_n\right)_n$ with $\norm{\phi_n -f}\rightarrow 0$. If $f\geq 0$, then we can select $\phi_n\geq 0$.
\end{corollary}
Now that we understand how simple functions relate to $\ell_{\infty}(G)$, we start by defining a translation action on $\ell_{\infty}(G)$, from which we will be able to convert the idea of means into invariant elements of the state space of the dual of $\ell_{\infty}\left(G\right)$.
\begin{proposition}\label{prop:translation_action}
  Let $G$ be a group. There is an action
  \begin{align*}
    \lambda_s\colon G\rightarrow \Isom\left(\ell_{\infty}\left(G\right)\right)
  \end{align*}
  defined by
  \begin{align*}
    \lambda_{s}\left(f\right)\left(t\right) &= f\left(s^{-1}t\right)
  \end{align*}
\end{proposition}
\begin{proof}
  We have
  \begin{align*}
    \lambda_s\left(f + \alpha g\right)\left(t\right) &= \left(f + \alpha g\right) \left(s^{-1}t\right)\\
                                                     &= f\left(s^{-1}t\right) \alpha g\left(s^{-1}t\right)\\
                                                     &= \lambda_s\left(f\right)\left(t\right) + \alpha \lambda_s\left(g\right)\left(t\right)\\
                                                     &= \left(\lambda_s\left(f\right) + \alpha \lambda_s\left(g\right)\right)(t).
  \end{align*}
  Thus, $\lambda_s$ is linear. Additionally,
  \begin{align*}
    \norm{\lambda_s\left(f\right)} &= \sup_{t\in G}\left\vert \lambda_s\left(f\right)\left(t\right) \right\vert\\
                                   &= \sup_{t\in G}\left\vert f\left(s^{-1}t\right) \right\vert\\
                                   &= \norm{f},
  \end{align*}
  and
  \begin{align*}
    \norm{\lambda_s\left(f\right) - \lambda_s\left(g\right)} &= \norm{\lambda_s\left(f-g\right)}\\
                                                             &= \norm{f-g},
  \end{align*}
  meaning $\lambda_s$ is an isometry.\newline

  We have
  \begin{align*}
    \lambda_s\circ \lambda_r\left(f\right)\left(t\right) &= \lambda_r\left(f\right)\left(s^{-1}t\right)\\
                                                         &= \lambda_r\left(r^{-1}s^{-1}t\right)\\
                                                         &= f\left(\left(sr\right)^{-1}t\right)\\
                                                         &= \lambda_{sr}\left(f\right)\left(t\right),
  \end{align*}
  establishing that $\lambda_s\circ \lambda_r = \lambda_{sr}$.\newline

  By a similar process, we find that $\lambda_{s}\left(\1_{E}\right) = \1_{sE}$ for any $E\subseteq G$ and $s\in G$.
\end{proof}
\begin{definition}
  A {state} on $\ell_{\infty}\left(G\right)$ is a continuous linear functional $\mu\in \left(\ell_{\infty}\left(G\right)\right)^{\ast}$ such that the following are true:
  \begin{itemize}
    \item $\mu$ is positive;
    \item $\mu\left(\1_{G}\right) = 1$.
  \end{itemize}
  A state is called left-invariant if
  \begin{align*}
    \mu\left(\lambda_s\left(f\right)\right) = \mu\left(f\right).
  \end{align*}
\end{definition}
\begin{example}
  The evaluation functional, $\delta_x\colon \ell_{\infty}\rightarrow \R$, defined by
  \begin{align*}
    \delta_{x}\left(f\right) &= f(x),
  \end{align*}
  is a state. However, it is not necessarily invariant, as
  \begin{align*}
    \delta_x\left(\lambda_s\left(f\right)\right) &= \lambda_s\left(f\right)\left(x\right)\\
                                                 &= f\left(s^{-1}x\right)\\
                                                 &\neq f(x).
  \end{align*}
  However, we can use the evaluation functional to create an invariant state. If $G$ is finite, we define
  \begin{align*}
    \mu &= \frac{1}{\left\vert G \right\vert} \sum_{x\in G}\delta_x,
  \end{align*}
  which is indeed an invariant state.
\end{example}
We can characterize states slightly differently, which will enable us to show the equivalence between invariant states and means.
\begin{lemma}\label{lemma:characterizing_states}\hfill
  \begin{enumerate}[(1)]
    \item If $\mu$ is a state on $\ell_{\infty}\left(G\right)$, then
      \begin{align*}
        \norm{\mu}_{\op} = 1.
      \end{align*}
    \item If $\mu\in \left(\ell_{\infty}\left(G\right)\right)^{\ast}$ is such that
      \begin{align*}
        \norm{\mu}_{\op} &= \mu\left(\1_{G}\right)\\
                               &= 1,
      \end{align*}
      then $\mu$ is positive and a state.
  \end{enumerate}
\end{lemma}
\begin{proof}\hfill
  \begin{enumerate}[(1)]
    \item Let $\mu$ be a state. Given $f\in \ell_{\infty}\left(G\right)$, we have
      \begin{align*}
        \norm{f}\1_{G} - f &\geq 0\\
        \norm{f}\1_{G} + f &\geq 0,
      \end{align*}
      so
      \begin{align*}
        0 &\leq \mu\left(\norm{f}\1_{G} - f\right) \\
          &= \norm{f}\mu\left(\1_{G}\right) - \mu\left(f\right)
          \intertext{meaning}
        \mu\left(f\right) &\leq \norm{f}.
        \intertext{Additionally,}
        0 &\leq \mu\left(\norm{f}\1_{G} + f\right)\\
          &= \norm{f}\mu\left(\1_{G}\right) + \mu\left(f\right),
          \intertext{meaning}
        -\mu\left(f\right) &\leq \norm{f}.
      \end{align*}
      Thus, we have $\left\vert \mu\left(f\right) \right\vert \leq \norm{f}$, so $\norm{\mu}_{\op} \leq 1$. However, since $\mu\left(\1_{G}\right) = 1$, we must have $\norm{\mu}_{\op} = 1$.
  \item Suppose $\norm{\mu}_{\op} = \mu\left(\1_{G}\right) = 1$. Let $f\geq 0$. Set $g = \frac{1}{\norm{f}_u}f$.\newline

    Then, $\Ran(g) \subseteq [0,1]$, and $\Ran\left(g - \1_{G}\right) \subseteq \left[-1,1\right]$. Thus, $\norm{g - \1_{G}}_{u} \leq 1$.\newline

    Since $\norm{\mu}_{\op} = 1$, we must have
    \begin{align*}
      \left\vert \mu\left(g - \1_{G}\right) \right\vert &\leq 1\\
      \left\vert \mu\left(g\right) - 1 \right\vert &\leq 1,
    \end{align*}
    and since $\mu\left(\1_{G}\right) = 1$, we have $\mu\left(g\right) \in [0,2]$. Thus, $\mu\left(f\right) = \norm{f}\mu\left(g\right) \geq 0$.
  \end{enumerate}
\end{proof}

To show the equivalence between means and invariant states, we need to be able to characterize the state space on $\left(\ell_{\infty}\left(G\right)\right)^{\ast}$. To do this, we make use of some results from functional analysis.\newline

If $X$ is a normed vector space, then the topology on $X^{\ast}$ induced by $X^{\ast\ast}$ is known as the weak* topology. The weak* topology is the topology of pointwise convergence in $X^{\ast}$ --- a net $\left(\varphi_{\alpha}\right)_{\alpha}$ converges to $\varphi$ in the weak* topology if and only if, for all $\hat{x}\in X^{\ast\ast}$, we have
\begin{align*}
  \left(\hat{x}\left(\varphi_{\alpha}\right)\right)_{\alpha}\rightarrow \hat{x}\left(\varphi\right),
\end{align*}
or by the definition of $X^{\ast\ast}$, this is equivalent to
\begin{align*}
  \left(\varphi_{\alpha}\left(x\right)\right) \rightarrow \varphi\left(x\right)
\end{align*}
for all $x\in X$.\newline

We state some important results in functional analysis here without proof. The proofs of these results can be found in functional analysis textbooks such as \cite{rudin_functional_analysis}.
\begin{theorem}[Hahn--Banach Continuous Extension Theorem]
  Let $X$ be a normed vector space, $E\subseteq X$ a subspace, and $\varphi\in E^{\ast}$ a bounded linear functional. Then, there exists a continuous $\psi\in X^{\ast}$ such that $\norm{\varphi}_{\op} = \norm{\psi}_{\op}$, and $\psi|_{E} = \varphi$.
\end{theorem}
\begin{theorem}[Hahn--Banach Separation Theorems]
  Let $X$ be a normed vector space.
  \begin{enumerate}[(1)]
    \item Given a nonzero $x_0\in X$, there is a $\varphi\in X^{\ast}$ with $\norm{\varphi}_{\op} = 1$ and $\varphi\left(x_0\right) = \norm{x}$. We call $\varphi$ a norming functional.
    \item Given a proper closed subspace $E\subseteq X$ and $x_0\in X\setminus E$, there is a $\varphi\in X^{\ast}$ such that $\varphi|_{E} = 0$, $\norm{\varphi}_{\op} = 1$, and $\varphi\left(x\right) = \dist_{E}(x)$ for all $x\in X$.
  \end{enumerate}
\end{theorem}
\begin{theorem}[Banach--Alaoglu Theorem]
  Let $X$ be a normed vector space.
  \begin{enumerate}[(1)]
    \item The closed unit ball in the dual space, $B_{X^{\ast}}$, is compact in the $w^{\ast}$ topology.
    \item A subset $C\subseteq X$ is $w^{\ast}$-compact if and only if $C$ is $w^{\ast}$-closed and norm bounded.
  \end{enumerate}
\end{theorem}
\begin{corollary}
  The set of states in $\left(\ell_{\infty}\left(G\right)\right)^{\ast}$ forms a $w^{\ast}$-compact subset of $B_{\left(\ell_{\infty}\left(G\right)\right)^{\ast}}$.
\end{corollary}
\begin{proof}
  From the Banach--Alaoglu Theorem, we only need to show that the set of states, $S\left(\ell_{\infty}\left(G\right)\right)$, is $w^{\ast}$-closed, as every element of $S\left(\ell_{\infty}\left(G\right)\right)$ has norm $1$.\newline

  Let $f\in \ell_{\infty}\left(G\right)$ be positive, and let $\left(\varphi_{i}\right)_i$ be a net in $S\left(\ell_{\infty}\left(G\right)\right)$ with $\left(\varphi_{i}\right)_i\xrightarrow{w^{\ast}} \varphi\in \left(\ell_{\infty}\left(G\right)\right)^{\ast}$. From Lemma \ref{lemma:characterizing_states}, we must show that $\varphi$ is positive and $\varphi\left(\1_{G}\right) = 1$.\newline

  We start by seeing that, since each $\varphi_i$ is a state, we have $\varphi_{i}\left(f\right) \geq 0$ for each $i\in I$, so we must have $\varphi\left(f\right) \geq 0$.\newline

  Similarly, since $\varphi_{i}\left(\1_{G}\right) = 1$ for each $i\in I$, and $\left(\varphi_i\right)_i \xrightarrow{w^{\ast}} \varphi$, we have $\varphi\left(\1_{G}\right) = 1$. Thus, by Lemma \ref{lemma:characterizing_states}, we have that $S\left(\ell_{\infty}\left(G\right)\right)$ is $w^{\ast}$-closed.
\end{proof}

Now, we may show the correspondence between invariant states and means.
\begin{proposition}\label{prop:state_implies_mean}
  If $\mu\in \left(\ell_{\infty}\left(G\right)\right)^{\ast}$ is a state, then $m\colon P(G)\rightarrow [0,1]$ defined by $m(E) = \mu\left(\1_{E}\right)$ is a finitely additive probability measure on $G$.\newline

  Moreover, if $\mu$ is invariant, then $m$ is a mean.
\end{proposition}
\begin{proof}
  We have
  \begin{align*}
    m\left(G\right) &= \mu\left(\1_{G}\right)\\
                    &= 1\\
                    \\
    m\left(\emptyset\right) &= \mu\left(0\right)\\
                            &= 0\\
                            \\
    m\left(E\sqcup F\right) &= \mu\left(\1_{E\sqcup F}\right)\\
                            &= \mu\left(\1_{E} + \1_{F}\right)\\
                            &= \mu\left(\1_{E}\right) + \mu\left(\1_{F}\right)\\
                            &= m\left(E\right) + m\left(F\right).
  \end{align*}
  Additionally, since $0 \leq \1_{E}\leq \1_{G}$, we have $0 \leq \mu\left(\1_{E}\right) \leq 1$, so $0 \leq m(E) \leq 1$.\newline

  If $\mu$ is invariant, then
  \begin{align*}
    m\left(sE\right) &= \mu\left(\1_{sE}\right)\\
                     &= \mu\left(\lambda_s\left(\1_{E}\right)\right)\\
                     &= \mu\left(\1_{E}\right)\\
                     &= m\left(E\right).
  \end{align*}
\end{proof}
\begin{proposition}\label{prop:mean_implies_state}
  If $G$ admits a mean, then $\left(\ell_{\infty}\left(G\right)\right)^{\ast}$ admits an invariant state.
\end{proposition}
\begin{proof}
  Let $m$ be a mean. Define $\mu_0\colon \Sigma\rightarrow \R$ by
  \begin{align*}
    \mu_0\left(\sum_{k=1}^{n}t_k\1_{E_k}\right) &= \sum_{k=1}^{n}t_km\left(E_k\right).
  \end{align*}
  Since $m$ is finitely additive, it is the case that $\mu_0$ is well-defined, linear, and positive, with $\mu_0\left(\1_{G}\right) = m\left(G\right) = 1$.\newline

  Additionally, since $m$ is a mean, then for $f = \sum_{k=1}^{n}t_k\1_{E_k}$, we have
  \begin{align*}
    \mu_0\left(\lambda_s\left(f\right)\right) &= \mu_0\left(\lambda_s\left(\sum_{k=1}^{n}t_k\1_{E_k}\right)\right)\\
                                              &= \mu_0\left(\sum_{k=1}^{n}t_k\1_{sE_k}\right)\\
                                              &= \sum_{k=1}^{n}t_km\left(sE_k\right)\\
                                              &= \sum_{k=1}^{n}t_km\left(E_k\right)\\
                                              &= \mu_0\left(f\right).
  \end{align*}
  We see that
  \begin{align*}
    \left\vert \mu_0\left(f\right) \right\vert &= \left\vert \sum_{k=1}^{n}t_km\left(E_k\right) \right\vert\\
                                               &\leq \sum_{k=1}^{n}\left\vert t_k \right\vert m\left(E_k\right)\\
                                               &\leq \sum_{k=1}^{n}\norm{f}\sum_{k=1}^{n}m\left(E_k\right)\\
                                               &= \norm{f}\sum_{k=1}^{n}m\left(E_k\right)\\
                                               &\leq \norm{f},
  \end{align*}
  meaning $\mu_0$ is continuous, so $\mu_0$ is uniformly continuous.\newline

  Since $\overline{\Sigma} = \ell_{\infty}\left(G\right)$, uniform continuity provides that $\mu_0$ extends to a continuous linear functional $\mu\colon \ell_{\infty}\left(G\right)\rightarrow \R$ with $\mu\left(\1_{G}\right) = \mu_0\left(\1_{G}\right) = 1$.\newline

  For $f\geq 0$, we find a sequence $\left(\phi_n\right)_n$ in $\Sigma$ with $\phi_n\geq 0$ and $\norm{\phi_n - f} \xrightarrow{n\rightarrow\infty}0$. We set
  \begin{align*}
    \mu\left(f\right) &= \lim_{n\rightarrow\infty}\mu\left(\phi_n\right)\\
                      &= \lim_{n\rightarrow\infty}\mu_0\left(\phi_n\right)\\
                      &\geq 0,
  \end{align*}
  so $\mu$ is a state.\newline

  If $f\in \ell_{\infty}\left(G\right)$, $s\in G$, and $\left(\phi_n\right)_n$ a sequence in $\Sigma$ with $\left(\phi_n\right)_n\rightarrow f$, then
  \begin{align*}
    \norm{\lambda_s\left(\phi_n\right) - \lambda_s\left(f\right)} &= \norm{\lambda_s\left(\phi_n - f\right)}\\
                                                                  &= \norm{\phi_n - f}\\
                                                                  &\rightarrow 0.
  \end{align*}
  Thus, we have
  \begin{align*}
    \mu\left(\lambda_s\left(\phi_n\right)\right) &= \mu_0\left(\lambda_s\left(\phi_n\right)\right)\\
                                                 &= \mu_0\left(\phi_n\right)\\
                                                 &= \mu\left(\phi_n\right)\\
                                                 &\rightarrow \mu\left(f\right),
  \end{align*}
  so $\mu\left(f\right) = \mu\left(\lambda_s\left(f\right)\right)$. Thus, $\mu\in \left(\ell_{\infty}\left(G\right)\right)^{\ast}$ is an invariant state.
\end{proof}
\section{Establishing Amenability using Invariant States}%
Owing to the correspondence between invariant states and means, we are now able to establish the amenability of large classes of groups.
\begin{proposition}
  The group of integers, $\Z$, is amenable.
\end{proposition}
\begin{proof}
  We define the left shift, $\lambda_1\colon \ell_{\infty}\left(\Z\right) \rightarrow \ell_{\infty}\left(\Z\right)$, by
  \begin{align*}
    \lambda_1\left(f\right)\left(k\right) &= f\left(k-1\right).
  \end{align*}
  This is an action as in Proposition \ref{prop:translation_action}. \newline

  We set $Y = \Ran\left(\id - \lambda_1\right)\subseteq \ell_{\infty}\left(\Z\right)$. We claim that $\dist_{Y}\left(\1_{\Z}\right) \geq 1$.\newline

  Suppose toward contradiction that there is $y\in Y$ with $\norm{\1_{\Z} - y}_{u} = r < 1$. Then, $y = f - \lambda_1 f$ for some $f\in \ell_{\infty}(\Z)$, so
  \begin{align*}
    \norm{\1_{\Z} - \left(f - \lambda_1\left(f\right)\right)} &= r.
  \end{align*}
  Thus, for all $k\in\Z$, we have
  \begin{align*}
    \left\vert 1 - \left(f(k) - f(k-1)\right) \right\vert &\leq r,
  \end{align*}
  so $\left\vert f(k) - f\left(k-1\right) \right\vert \geq 1-r > 0$. However, such an $f$ cannot be bounded.\newline

  Since $\dist_{\overline{Y}}\left(\1_{\Z}\right) = \dist_{Y}\left(\1_{\Z}\right)$, the Hahn--Banach separation theorems provide $\mu\in \left(\ell_{\infty}\left(\Z\right)\right)^{\ast}$ with $\norm{\mu}_{\op} = 1$, $\mu|_{\overline{Y}} = 0$, and $\mu\left(\1_{\Z}\right) = \dist_{Y}\left(\1_{\Z}\right) \geq 1$.\newline

  Since $\norm{\mu}_{\op} = 1$ and $\mu\left(\1_{\Z}\right) \geq 1$, we must have $\mu\left(\1_{\Z}\right) = 1$.\newline

  Additionally, since $\norm{\mu}_{\op} = \mu\left(\1_{\Z}\right) = 1$, we have that $\mu$ is a state on $\ell_{\infty}\left(\Z\right)$, and since $\mu\left(y\right) = 0$ for all $y\in Y$, we have
  \begin{align*}
    \mu\left(f - \lambda_1\left(f\right)\right) &= 0\\
    \mu\left(f\right) &= \mu\left(\lambda_1\left(f\right)\right).
  \end{align*}
  Inductively, this means that $\mu\left(f\right) = \mu\left(\lambda_k\left(f\right)\right)$ for all $k\in \Z$, so $\mu$ is an invariant state on $\ell_{\infty}\left(\Z\right)$. Thus, $\Z$ is amenable.
\end{proof}
\begin{proposition}
  If $N\trianglelefteq G$ and $G/N$ are amenable, then $G$ is amenable.
\end{proposition}
\begin{proof}
  Let $\rho\in \left(\ell_{\infty}\left(G/N\right)\right)^{\ast}$ be an invariant state, and let $p\colon P(N)\rightarrow [0,1]$ be a mean. For $E\subseteq G$, we define $f_E\colon G/N\rightarrow \R$ by
  \begin{align*}
    f_E\left(tN\right) &= p\left(N\cap t^{-1}E\right).
  \end{align*}
  We start by verifying that $f_E$ is well-defined. For $tN = sN$, we have $s^{-1}t\in N$, so
  \begin{align*}
    p\left(N\cap t^{-1}E\right) &= p\left(s^{-1}t\left(N\cap t^{-1}E\right)\right)\\
                                &= p\left(s^{-1}tN \cap s^{-1}E\right)\\
                                &= p\left(N\cap s^{-1}E\right).
  \end{align*}
  Since $f_E$ is defined through $p$, we can see that $f_E$ is bounded. Additionally,
  \begin{align*}
    f_{E\sqcup F}\left(tN\right) &= p\left(N\cap t^{-1}\left(E\sqcup F\right)\right)\\
                                 &= p\left(N\cap \left(t^{-1}E\sqcup t^{-1}F\right)\right)\\
                                 &= p\left(\left(N\cap t^{-1}E\right) \sqcup \left(N\cap t^{-1}F\right)\right)\\
                                 &= p\left(N\cap t^{-1}E\right) + p\left(N\cap t^{-1}F\right)\\
                                 &= f_E\left(tN\right) + f_F\left(tN\right)\\
                                 &= \left(f_E + f_F\right)\left(tN\right),
  \end{align*}
  and
  \begin{align*}
    f\left(sE\right) \left(tN\right) &= p\left(N\cap t^{-1}sE\right)\\
                                     &= f_E\left(s^{-1}tN\right)\\
                                     &= \lambda_{sN}\left(f_E\right)\left(tN\right),
  \end{align*}
  so $f_{sE} = \lambda_{sN}\left(f_E\right)$. Finally,
  \begin{align*}
    f_G\left(tN\right) &= p\left(N\cap t^{-1}G\right)\\
                       &=p\left(N\right)\\
                       &= 1,
  \end{align*}
  meaning $f_G = \1_{G/N}$.\newline

  We define $m\colon P(G)\rightarrow [0,1]$ by
  \begin{align*}
    m(E) &= \rho\left(f_E\right).
  \end{align*}
  Then, we have
  \begin{align*}
    m\left(E\sqcup F\right) &= m(E) + m(F)\\
                            \\
    m\left(G\right) &= 1\\
    \\
    m\left(sE\right) &= \rho\left(f_{sE}\right)\\
                     &= \rho\left(\lambda_{sN}\left(f_{E}\right)\right)\\
                     &= \rho\left(f_E\right)\\
                     &= m(E),
  \end{align*}
  so $m$ is a mean.
\end{proof}
\begin{corollary}
  The finite direct product of amenable groups is amenable.
\end{corollary}
\begin{proof}
  For $H$ and $K$ amenable groups, we know that $K\cong \left(H\times K\right)/H$ and $H$ are amenable, so $H\times K$ is amenable. Induction provides the general case.
\end{proof}
\begin{corollary}
  Finitely generated abelian groups are amenable.
\end{corollary}
\begin{proof}
  By the fundamental theorem of finitely generated abelian groups, all finitely generated abelian groups are isomorphic to $\Z^{d}\times \Z/n_1\Z\times\cdots\times \Z/{n_k}\Z$.\newline

  Since $\Z^{d}$ is a finite direct product of $\Z$, and the torsion subgroup $\Z/n_1\Z\times\cdots\times \Z/n_k\Z$ is finite, we see that a finitely generated abelian group is a direct product of two amenable groups, hence amenable.
\end{proof}
\begin{corollary}
  If $\set{G_i}_{i\in I}$ is a directed family of amenable groups, then the direct limit,
  \begin{align*}
    G &= \bigcup_{i\in I}G_i,
  \end{align*}
  is also amenable.
\end{corollary}
\begin{proof}
  Let $\mu_i\in \left(\ell_{\infty}\left(G_i\right)\right)^{\ast}$ be invariant states.\newline

  Set
  \begin{align*}
    M_i &= \set{\mu\in S\left(\ell_{\infty}\left(G\right)\right)| \mu\left(\lambda_s\left(f\right)\right) = \mu\left(f\right)\text{ for all }s\in G_i}.
  \end{align*}
  We set $\mu\left(f\right) = \mu_i\left(f|_{G_i}\right)$. Since each $G_i$ is amenable, it is the case that each $M_i$ is nonempty. Similarly, seeing as we have established the state space as $w^{\ast}$-closed in $B_{\left(\ell_{\infty}\left(G\right)\right)^{\ast}}$, it is the case that each $M_i$ is $w^{\ast}$-closed in $B_{\left(\ell_{\infty}\left(G\right)\right)^{\ast}}$.\newline

  For $i_1,\dots,i_n$, we find $G_j \supseteq G_{i_1},\dots,G_{i_n}$, which exists since $\set{G_i}_{i\in I}$ is directed. We have that $M_j\subseteq \bigcap_{k=1}^{n}M_{i_k}$, so $\set{M_i}_{i\in I}$ has the finite intersection property.\newline

  Thus, there is $\mu\in \bigcap_{i\in I}M_i$, which is necessarily invariant on $G$.
\end{proof}
\begin{corollary}
  All abelian groups are amenable.
\end{corollary}
\begin{proof}
  Every abelian group is the direct limit of its finitely generated subgroups.
\end{proof}
\begin{corollary}
  All solvable groups are amenable.
\end{corollary}
\begin{proof}
  Let $e_G = G_0 \leq G_1\leq\cdots\leq G_n\leq G$ be such that $G_{j-1}\trianglelefteq G_j$ for $j=1,\dots,n$, and $G_i/G_j$ is abelian.\newline

  Since $G_0$ is abelian, it is amenable, as is $G_1/G_0$, so $G_1$ is amenable. We see then that $G_2$ is amenable as $G_1$ and $G_2/G_1$ are amenable.\newline

  Continuing in this fashion, we see that $G$ is amenable.
\end{proof}
\section{Følner's Condition and Approximate Means}%
While showing the existence of an invariant state is necessary and sufficient for showing a group is amenable, as well as showing the group is non-paradoxical, it is often difficult to establish either of these conditions.\newline

However, we can often more easily create a sequence (or net) of finitely supported functions whose limit is an invariant state. This will require the use of the Følner condition.
\begin{definition}\label{def:folner_condition}
  A group is said to satisfy the {Følner condition} if, for every $\ve > 0$ and $E\subseteq G$, there is a nonempty finite subset $F\subseteq G$ such that for all $t\in E$,
  \begin{align*}
    \frac{\left\vert tF\triangle F \right\vert}{\left\vert F \right\vert}\leq \ve.
  \end{align*}
  Equivalently, we can also say that the Følner condition is satisfied if and only if
  \begin{align*}
    \frac{\left\vert tF\cap F \right\vert}{\left\vert F \right\vert} \geq 1 - \ve
  \end{align*}
  for every $\ve > 0$.
\end{definition}
\begin{lemma}\label{lemma:folner_sequences}
  A countable group $G$ satisfies the Følner condition if and only if $G$ admits a sequence $\left(F_n\right)_n$ with $F_n\subseteq G$ finite such that
  \begin{align*}
    \left(\frac{\left\vert tF_n\triangle F_n \right\vert}{\left\vert F_n \right\vert}\right)_n \xrightarrow{n\rightarrow \infty}0
  \end{align*}
  for all $t\in G$. Such a sequence is known as a Følner sequence.
\end{lemma}
\begin{proof}
  Let $G$ admit a Følner sequence, $\left(F_n\right)_n$. Given $\ve > 0$ and $E\subseteq G$ finite, find $N$ such that for all $s\in E$ and $n\geq N$,
  \begin{align*}
    \frac{\left\vert sF_n\triangle F_n \right\vert}{\left\vert F_n \right\vert} &\leq \ve.
  \end{align*}
  We take $F = F_N$ in the definition of the Følner condition.\newline

  Let $G$ satisfy the Følner condition. We write $G = \bigcup_{n\geq 1}E_n$, with $E_1\subseteq E_2\subseteq \cdots$, and define $F_n$ such that for all $t\in E_n$,
  \begin{align*}
    \frac{\left\vert tF_n\triangle F_n \right\vert}{\left\vert F_n \right\vert} &\leq \frac{1}{n}.
  \end{align*}
  Given $t\in G$, then $t\in E_N$ for some $N$, so $t\in E_n$ For all $n\geq N$, so
  \begin{align*}
    \frac{\left\vert tF_n\triangle F_n \right\vert}{\left\vert F_n \right\vert} &\leq \frac{1}{n}
  \end{align*}
  for all $n\geq N$. Thus,
  \begin{align*}
    \left(\frac{\left\vert tF_n\triangle F_n \right\vert}{\left\vert F_n \right\vert}\right)\xrightarrow{n\rightarrow\infty}0.
  \end{align*}
\end{proof}
\begin{lemma}
  Let $G$ be a finitely generated group with generating set $S$. If $\left(F_n\right)_n$ is a sequence of finite subsets such that, for all $s\in S$,
  \begin{align*}
    \left(\frac{\left\vert sF_n\triangle F_n \right\vert}{\left\vert F_n \right\vert}\right)_n\rightarrow 0,
  \end{align*}
  then $\left(F_n\right)_n$ is a Følner sequence for $G$.
\end{lemma}
\begin{proof}
  Note that
  \begin{itemize}
    \item $s\left(A\triangle B\right) = sA\triangle sB$;
    \item $A\triangle C \subseteq \left(A\triangle B\right) \cup \left(B\triangle C\right)$.
  \end{itemize}
  We see that for any $s\in S$,
  \begin{align*}
    \frac{\left\vert s^{-1}F_n\triangle F_n \right\vert}{\left\vert F_n \right\vert} &= \frac{\left\vert s^{-1}\left(F_n\triangle sF_n\right) \right\vert}{\left\vert F_n \right\vert}\\
                                                                                     &= \frac{\left\vert F_n\triangle sF_n \right\vert}{\left\vert F_n \right\vert}\\
                                                                                     &\rightarrow 0.
  \end{align*}
  Thus, we may assume that $S$ is symmetric --- i.e., that $\set{s^{-1}| s\in S} = \set{s | s\in S}$.\newline

  For any $s,t\in S$, we have
  \begin{align*}
    \frac{\left\vert stF_n\triangle F_n \right\vert}{\left\vert F_n \right\vert} &\leq \frac{\left\vert stF_n\triangle F_n \right\vert}{\left\vert F_n \right\vert} + \frac{\left\vert sF_n\triangle F_n \right\vert}{\left\vert F_n \right\vert}\\
                                                                                 &= \frac{\left\vert s\left(tF_n\triangle F_n\right) \right\vert}{\left\vert F_n \right\vert} + \frac{\left\vert sF_n\triangle F_n \right\vert}{\left\vert F_n \right\vert}\\
                                                                                 &= \frac{\left\vert tF_n\triangle F_n \right\vert}{\left\vert F_n \right\vert} + \frac{\left\vert sF_n\triangle F_n \right\vert}{\left\vert F_n \right\vert}\\
                                                                                 &\rightarrow 0.
  \end{align*}
  We use induction to find the general case.
\end{proof}
\begin{example}
  Consider the group $\Z$. Since $\Z$ is generated by the element $\set{1}$, we see that for $F_n = [-n,n]$, that
  \begin{align*}
    \frac{\left\vert \left(F_n + 1\right)\triangle F_n \right\vert}{\left\vert F_n \right\vert} &= \frac{2}{2n+1}\\
                                                                                                &\rightarrow 0,
  \end{align*}
  meaning that $\Z$ satisfies the Følner condition.
\end{example}
We have thus far proven that $G$ satisfies the Følner condition if and only if $G$ admits a Følner sequence, and that $G$ is amenable if and only if $G$ admits an invariant state.\newline

We will now begin harmonizing these two characterizations through the use of approximate means, eventually showing that $G$ satisfies the Følner condition if and only if $G$ admits an approximate mean, and that $G$ admits an approximate mean if and only if $G$ is amenable.
\begin{definition}\label{def:state_on_prob_g}
  For a group $G$, we define
  \begin{align*}
    \Prob\left(G\right) = \set{f\colon G\rightarrow [0,\infty) | \left\vert \supp(f) \right\vert < \infty,~\sum_{t\in G}f(t) = 1}.
  \end{align*}
  Note that $\Prob(G) \subseteq B_{\ell_1\left(G\right)}$. For $f\in \prob(G)$, we set $\varphi_f\colon \ell_{\infty}(G)\rightarrow \C$ defined by
  \begin{align*}
    \varphi_f\left(g\right) &= \sum_{t\in G}g(t)f(t).
  \end{align*}
\end{definition}
\begin{fact}\label{fact:prob_g_state}
  For $f\in \prob(G)$, $\varphi_f$ is a state on $\ell_{\infty}\left(G\right)$.
\end{fact}
\begin{proof}
We can see that, by definition, $\varphi_f$ is positive, linear, and has $\varphi_f\left(\1_{G}\right) = 1$.\newline

We only need to show that $\norm{\varphi_f} = 1$. We see that
\begin{align*}
  \left\vert \varphi_f\left(g\right) \right\vert &= \left\vert \sum_{t\in G}g(t)f(t) \right\vert\\
                                                 &\leq \sum_{t\in G}\left\vert g(t) \right\vert\left\vert f(t) \right\vert\\
                                                 &\leq \norm{g}_{\infty}\sum_{t\in G}\left\vert f(t) \right\vert\\
                                                 &= \norm{g}_{\infty},
\end{align*}
so $\norm{\varphi_f} \leq 1$. Since $\varphi_f\left(\1_G\right) = 1$, we must have $\norm{\varphi_f} = 1$.
\end{proof}
\begin{proposition}
  There is an action $\lambda\colon G\xrightarrow \Isom\left(\ell_{1}\left(G\right)\right)$ such that $\prob(G)$ is invariant.
\end{proposition}
\begin{proof}
  Let $\lambda_s\left(f\right)\left(t\right) = f\left(s^{-1}t\right)$. Then,
  \begin{align*}
    \norm{\lambda_s\left(f\right)}_1 &= \sum_{t\in G}\left\vert \lambda_s\left(f\right)\left(t\right) \right\vert\\
                                     &= \sum_{t\in G}\left\vert f\left(s^{-1}t\right) \right\vert\\
                                     &= \sum_{r\in G}\left\vert f(r) \right\vert\\
                                     &= \norm{f}_{1}.
  \end{align*}
  Just as in Proposition \ref{prop:translation_action}, it is the case that $\lambda_s$ is linear. Additionally,
  \begin{align*}
    \lambda_r\circ \lambda_s\left(f\right)\left(t\right) &= \lambda_s\left(f\right)\left(r^{-1}t\right)\\
                                                         &= f\left(s^{-1}r^{-1}\left(t\right)\right)\\
                                                         &= f\left(\left(rs\right)^{-1}t\right)\\
                                                         &= \lambda_{rs}\left(f\right)\left(t\right).
  \end{align*}
  We see that if $f\in \prob(G)$, then for $f\geq 0$, we have $\lambda_s\left(f\right) \geq 0$, and
  \begin{align*}
    \sum_{t\in G}\lambda_s\left(f\right)\left(t\right) &= \sum_{t\in G}f\left(s^{-1}t\right)\\
                                                       &= \sum_{r\in G}f\left(r\right)\\
                                                       &= 1
  \end{align*}
  for any $f\in \prob(G)$.
\end{proof}
\begin{definition}\label{def:approximate_mean}
  For a countable group $G$, a sequence $\left(f_k\right)_k$ is called an approximate mean if, for all $s\in G$,
  \begin{align*}
    \norm{f_k - \lambda_s\left(f_k\right)}_{1} &\xrightarrow{k\rightarrow \infty}0.
  \end{align*}
\end{definition}
To begin the forward direction regarding the equivalence between the Følner condition, approximate means, and means, we begin by showing that the existence of a Følner sequence implies the existence of an approximate mean. Then, we will show that the existence of an approximate mean implies the existence of an invariant state (hence mean).
\begin{proposition}
  If $G$ admits a Følner sequence $\left(F_k\right)_k$, then $G$ admits an approximate mean.
\end{proposition}
\begin{proof}
  Set $f_k = \frac{1}{\left\vert F_k \right\vert}\1_{F_k}\in \prob(G)$. Then,
  \begin{align*}
    \norm{f_k - \lambda_s\left(f_k\right)}_{1} &= \frac{1}{\left\vert f_k \right\vert} \norm{\1_{F_k} - \lambda_s\left(\1_{F_k}\right)}\\
                                               &= \frac{1}{F_k}\norm{\1_{F_k} - \1_{sF_k}}\\
                                               &= \frac{\left\vert F_k\triangle sF_k \right\vert}{\left\vert F_k \right\vert}\\
                                               &\rightarrow 0.
  \end{align*}
\end{proof}
\begin{proposition}
  If $G$ admits an approximate mean, then $G$ is amenable.
\end{proposition}
\begin{proof}
  Let $\left(f_k\right)_k$ be an approximate mean. We define $\varphi_k = \left(\varphi_{f_k}\right)_k$ (as in Definition \ref{def:state_on_prob_g}) to be a sequence of states on $\ell_{\infty}\left(G\right)$.\newline

  Since the state space on $\ell_{\infty}\left(G\right)$ is $w^{\ast}$-compact, there is a state $\mu$ and a subnet $\left(\varphi_{k_j}\right)_j \xrightarrow{w^{\ast}}\mu$. \newline

  We only need to show that $\mu$ is invariant. Note that
  \begin{align*}
    \left\vert \mu\left(g\right) - \mu\left(\lambda_s\left(g\right)\right) \right\vert &\leq \left\vert \mu\left(g\right) - \varphi_{k_j}\left(g\right) \right\vert + \left\vert \varphi_{k_j}\left(g\right) - \varphi_{k_j}\left(\lambda_s\left(g\right)\right) \right\vert + \left\vert \varphi_{k_j}\left(\lambda_s\left(g\right)\right) - \mu\left(\lambda_s\left(g\right)\right) \right\vert
  \end{align*}
  for all $g\in \ell_{\infty}\left(G\right)$, $s\in G$, and all $j$.\newline

  Given $\ve > 0$, we find $J$ such that for $j\geq J$,
  \begin{align*}
    \left\vert \mu\left(g\right) - \varphi_{k_j}\left(g\right) \right\vert &< \ve/3\\
    \left\vert \mu\left(\lambda_s\left(g\right)\right) \varphi_{k_j}\left(\lambda_s\left(g\right)\right)\right\vert &< \ve/3.
  \end{align*}
  We also see that
  \begin{align*}
    \left\vert \varphi_{k_j}\left(g\right) - \varphi_{k_j}\left(\lambda_s\left(g\right)\right) \right\vert &= \left\vert \sum_{t\in G}g(t)f_{k_j}\left(t\right) - \sum_{t\in G}g\left(s^{-1}t\right)f_{k_j}\left(t\right) \right\vert\\
                                                                                                           &= \left\vert \sum_{t\in G}g(t)f_{k_j}\left(t\right) - \sum_{r\in G}g(r)f_{k_j}\left(sr\right) \right\vert \tag*{$r = s^{-1}t$}\\
                                                                                                           &= \left\vert \sum_{t\in G}g(t)\left(f_{k_j}\left(t\right)-\lambda_{s^{-1}}\left(f_{k_j}\right)\left(t\right)\right) \right\vert\\
                                                                                                           &\leq \norm{g}_{\infty}\sum_{t\in G}\left\vert f_{k_j}\left(t\right) - \lambda_{s^{-1}}\left(f_{k_j}\right)\left(t\right) \right\vert\\
                                                                                                           &= \norm{g}_{\infty}\norm{f_{k_j} - \lambda_{s^{-1}}\left(f_{k_j}\right)}_{1}\\
                                                                                                           &< \ve/3
  \end{align*}
  for large $j$. Thus, we have
  \begin{align*}
    \left\vert \mu\left(g\right) - \mu\left(\lambda_{s}\left(g\right)\right) \right\vert &< \ve,
  \end{align*}
  for all $\ve > 0$, so $\mu\left(g\right) = \mu\left(\lambda_{s}\left(g\right)\right)$.
\end{proof}

We will now commence with the reverse direction, starting by showing that amenability implies the existence of an approximate mean, and then showing that the existence of an approximate mean implies that the Følner condition is satisfied.
\begin{proposition}
  If $G$ is amenable, then $G$ admits an approximate mean.
\end{proposition}
\begin{proof}
  Suppose $G$ does not admit an approximate mean. Then, there exists a finite subset $E_0\subseteq G$ and $\ve_0 > 0$ such that for all $s\in E_0$ and all $f\in \Prob(G)$, we have $\norm{f - \lambda_s\left(f\right) \geq \ve_0}$.\newline

  Consider the set
  \begin{align*}
    X &= \bigoplus_{\left\vert E_0 \right\vert} \ell_1\left(G\right),
  \end{align*}
  endowed with the norm
  \begin{align*}
    \norm{\left(f_s\right)_{s\in E_0}} &= \sum_{s\in E_0}\sum_{t\in G}\left\vert f_s(t) \right\vert\\
                                       &= \sum_{s\in E_0}\norm{f_s}_{1},
  \end{align*}
  and let
  \begin{align*}
    C &= \set{\left(f - \lambda_s\left(f\right)\right)_{s\in E_0} | f\in \Prob(G)}.
  \end{align*}
  Since $\Prob(G)$ is convex, it is the case that $C$ is convex, and since $\left\vert E_0 \right\vert$ is finite, $C$ is necessarily bounded. Note that $0\notin \overline{C}$.\newline

  By the Hahn--Banach separation theorem for convex sets, there is a real-valued $\varphi\in X^{\ast}$ such that $\varphi\left(C\right)\geq 1$. Here,
  \begin{align*}
    X^{\ast} &\cong \bigoplus_{\left\vert E_0 \right\vert}\ell_1\left(G\right)^{\ast}\\
             &\cong \sum_{\left\vert E_0 \right\vert}\ell_{\infty}\left(G\right),
  \end{align*}
  endowed with the norm
  \begin{align*}
    \norm{\left(g_s\right)_{s\in E_0}} &= \max_{s\in E_0}\left(\sup_{t\in G}\left\vert g_s(t) \right\vert\right)\\
                                       &= \max_{s\in E_0}\norm{g_s}_{\infty}.
  \end{align*}
  We let $\varphi = \left(\varphi_{g_s}\right)_{s\in E_0}$, where $g_s\in \ell_{\infty}\left(G\right)$ is defined by the duality
  \begin{align*}
    \varphi_{g_s}\left(f\right) &= \sum_{t\in G}f(t)g_s(t).
  \end{align*}
  Thus, for all $f\in \Prob(G)$, we have
  \begin{align*}
    1 &\leq \varphi\left(\left(f - \lambda_s\left(f\right)\right)_{s\in E_0}\right)\\
      &= \sum_{s\in E_0}\varphi_{g_s}\left(f - \lambda-s\left(f\right)\right)\\
      &= \sum_{s\in E_0}\sum_{t\in G}\left(f - \lambda_s\left(f\right)\right)(t)g_s(t)\\
      &= \sum_{s\in E_0}\left(\sum_{t\in G}f(t)g_s(t) - \sum_{t\in G}f\left(s^{-1}t\right)g_s(t)\right)\\
      &= \sum_{s\in E_0}\left(\sum_{t\in G}f(t)g_s(t) - \sum_{r\in G}f\left(r\right)g_s\left(sr\right)\right)\\
      &= \sum_{s\in E_0}\left(\sum_{r\in G}f(r)g_s(r) - \sum_{r\in G}f(r)\lambda_{s^{-1}}\left(g\right)(r)\right)\\
      &= \sum_{s\in E_0}\sum_{r\in G}f(r)\left(g_s - \lambda_{s^{-1}}\left(g\right)\right)(r).
      \intertext{Note that this holds for any $f\in \Prob(G)$, including the case of $f = \delta_t$ for a given $t\in G$. We must have}
      &= \sum_{s\in E_0}\sum_{r\in G}\delta_{t}\left(r\right)\left(g_s\left(r\right) - \lambda_{s^{-1}}\left(g_s\right)\right)\left(r\right)\\
      &= \sum_{s\in E_0}\left(g_s - \lambda_{s^{-1}}\left(g\right)\right)\left(t\right).
      \intertext{In particular, we must have}
      &\geq \1_{G}.
  \end{align*}
  Since $G$ is amenable, there is a mean $\mu\colon \ell_{\infty}\left(G\right)\rightarrow \C$ with $\mu\left(g_s\right) = \mu\left(\lambda_{s^{-1}}\left(g_s\right)\right)$, meaning
  \begin{align*}
    0 &= \mu\left(\sum_{s\in E_0}\left(g_s - \lambda_{s^{-1}}\left(g_s\right)\right)\left(t\right)\right)\\
      &\geq \mu\left(\1_{G}\right)\\
      &= 1,
  \end{align*}
  which is a contradiction.
\end{proof}
To show that the existence of an approximate mean implies the Følner condition, we require the following lemma.
\begin{lemma}\label{lemma:layer_cake_representation}
  Let $f\colon S\rightarrow \R$ be finitely supported with $\sum_{s\in S}f(s) = 1$. Then, there exist subsets $\set{F_i}_{i=1}^{n}$, where $F_1\supseteq F_2\supseteq \cdots \supseteq F_n$, and constants $\set{c_i}_{i=1}^{n}$, such that
  \begin{align*}
    f &= \sum_{i=1}^{n}c_i\1_{F_i},
  \end{align*}
  where
  \begin{align*}
    \sum_{i=1}^{n}c_i\left\vert F_i \right\vert &= 1.
  \end{align*}
  This is known as the layer cake representation for $f$.
\end{lemma}
\begin{proof}[Proof of Lemma \ref{lemma:layer_cake_representation}:]
%  Let $\set{r_1,\dots,r_n}$ be the range of $f$, ordered with $r_1 < r_2 < \cdots < r_n$.\newline
%
%  We define $E_{j} = \set{s | f(s) \geq r_j}$ for each $j=1,\dots,n$, and set $c_1 = r_1$, $c_2 = r_2 - r_1$, and $c_j = r_j - r_{j-1}$ for each $j$.\newline
%
%  This gives a decomposition of $f$ as
%  \begin{align*}
%    f &= \sum_{j=1}^{n}c_j\1_{E_j},
%  \end{align*}
%  which is necessarily equal to $f$ at each point by construction.
%
  We define $F_1 = \supp\left(f\right)$, and take $c_1 = \min\left(\Ran\left(f\right)\right)$. Taking $E_1 = f^{-1}\left(c_1\right)$ (as a set-theoretic inverse), we define $F_2 = F_1\setminus E_1$.\newline

  Take $d_1 = \min\left(f\left(F_2\right)\right)$, and define $c_2 = d_1 - c_1$. Then, defining $E_2 = f^{-1}\left(d_1\right)$, $F_3 = F_2 \setminus E_2$, and $d_2 = \min\left(f\left(F_3\right)\right)$, we define $c_3 = d_2 - c_2 - c_1$.\newline

  Continuing in this pattern, we find $d_{i-1} = \min\left(f\left(F_i\right)\right)$, $E_i = f^{-1}\left(d_{i-1}\right)$, and $c_i = d_{i-1} - \sum_{j=1}^{i-1}c_i$.\newline

  This yields a decomposition $F_1\supseteq F_2\supseteq \cdots \supseteq F_n$, where $\sum_{i=1}^{n}c_i\1_{F_i} = f$ by construction.\newline

  We now verify that $\sum_{i=1}^n c_i\left\vert F_i \right\vert = 1$.
  \begin{align*}
    1 &= \sum_{s\in S}f(s)\\
      &= \sum_{s\in S}\sum_{i=1}^{n}c_i\1_{F_i}\left(s\right).
      \intertext{By definition, if $s\in F_j$ for some $j$, we see that $c_j$ is summed for $\left\vert F_j \right\vert$ many times. Thus, we obtain}
      &= \sum_{i=1}^{n}c_i\left\vert F_i \right\vert.
  \end{align*}
\end{proof}
\begin{remark}
  Instead of using this construction where we take set-theoretic inverses and remove ``residual'' sets, there is an alternative method of construction that involves ordering the range as $r_1 < r_2< \cdots < r_n$, and constructing the set $F_k = \set{s | f(s) \geq r_k}$.
\end{remark}


We will use the layer cake decomposition to prove that if $G$ admits an approximate mean, then $G$ satisfies the Følner condition.
\begin{proposition}
  Let $G$ admit an approximate mean. Then, $G$ satisfies the Følner condition.
\end{proposition}
\begin{proof}
  Let $\left(f_k\right)_k$ be an approximate mean, as in Definition \ref{def:approximate_mean}. Fix a finite nonempty set $S \subseteq G$. Then, by the definition of an approximate mean, there must exist some $N\in\N$ such that for all $k\geq N$ and all $s\in G$,
  \begin{align*}
    \norm{f_k - \lambda_s\left(f_k\right)}_{1} &\leq \frac{\ve}{|S|}.
  \end{align*}
  In particular, this holds for $f_N$ and for all $s\in S$.\newline

  Since $f_N\in \Prob(G)$ is finitely supported and $\sum_{s\in G}f_N(s) = 1$, we may use Lemma \ref{lemma:layer_cake_representation} to rewrite $f_N$ as
  \begin{align*}
    f_N &= \sum_{i=1}^{n}c_i\1_{F_i},
  \end{align*}
  where $F_1 \supseteq F_2\supseteq \cdots \supseteq F_n$, and $\sum_{i=1}^{n}c_i\left\vert F_i \right\vert = 1$.\newline

  For a given $1 \leq i \leq n$, for each $s\in S$ and $t\in sF_i\triangle F_i$, we have
  \begin{align*}
    f_N\left(t\right) - f_N\left(s^{-1}t\right) &= \begin{cases}
      c_i & t\in F_i\setminus sF_i\\
      -c_i & t\in sF_i \setminus F_i
    \end{cases}.
  \end{align*}
  Thus, we see that $\left\vert f_N\left(t\right)-\lambda_s\left(f_N\right)\left(t\right) \right\vert\geq c_i$ on $sF_i\triangle F_i$. Thus, for each $s\in S$,
  \begin{align*}
    \sum_{i=1}^{n}c_i\left\vert sF_i \triangle F_i \right\vert &\leq \sum_{t\in S}\left\vert f_N\left(t\right) -  \lambda_s\left(f\right)\left(t\right)\right\vert\\
                                                               &< \frac{\ve}{\left\vert S \right\vert}\\
                                                               &= \frac{\ve}{\left\vert S \right\vert} \sum_{i=1}^{n}c_i\left\vert F_i \right\vert.
  \end{align*}
  Therefore, we have
  \begin{align*}
    \sum_{s\in S}\sum_{i=1}^{n}c_i\left\vert sF_i\triangle F_i \right\vert &< \frac{\ve}{\left\vert S \right\vert}\sum_{s\in S}\sum_{i=1}^{n}c_i\left\vert F_i \right\vert\\
                                                                           &= \ve \sum_{i=1}^{n}c_i\left\vert F_i \right\vert.
  \end{align*}
  Thus, by the pigeonhole principle, there must exist some $1\leq i \leq n$ for which
  \begin{align*}
    \sum_{s\in S}c_i\left\vert sF_i\triangle F_i \right\vert < \ve c_i\left\vert F_i \right\vert.
  \end{align*}
  Setting $F = F_i$, we find that, for all $s\in S$,
  \begin{align*}
    \frac{\left\vert sF\triangle F \right\vert}{\left\vert F \right\vert} &\leq \sum_{s\in S}\frac{\left\vert sF\triangle F \right\vert}{\left\vert F \right\vert}\\
                                                                          &< \ve.
  \end{align*}
  
\end{proof}

Thus far, we have shown the following to be equivalent for a discrete group $G$:
\begin{enumerate}[(1)]
  \item $G$ is non-paradoxical;
  \item $G$ is amenable;
  \item $G$ admits an invariant state;
  \item $G$ admits an approximate mean;
  \item $G$ satisfies the Følner condition.
\end{enumerate}
The equivalence between (1) and (2) follows from Tarski's theorem (Theorem \ref{thm:tarski}), the equivalence between (2) and (3) follows from Propositions \ref{prop:state_implies_mean} and \ref{prop:mean_implies_state}, and the equivalence between (3), (4), and (5) follows from 
