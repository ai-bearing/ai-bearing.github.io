When we start defining and working with amenability, we will realize the idea of a mean as a special type of measure on a group. Thus, we will need a pretty strong foundation in measure theory to fully appreciate many of the results relating to amenability.
\section{Constructing Measurable Spaces}%
Fix a set $\Omega$. We let $\mathcal{A} =\set{A_{i}}_{i\in I}$ be a collection of subsets of $\Omega$.
\begin{definition}[Algebra of Subsets]
  The collection $\mathcal{A} = \set{A_i}_{i\in I}$ is known as an \textit{algebra of subsets} for $\Omega$ if
  \begin{itemize}
    \item $\emptyset,\Omega \in \mathcal{A}$;
    \item for any $A_i\in \mathcal{A}$, $A_i^{c}\in \mathcal{A}$;
    \item for any $A_i,A_j\in \mathcal{A}$, $A_i \cup A_j \in \mathcal{A}$.
  \end{itemize}
\end{definition}
We can refine the concept of an algebra of subsets to consider countable unions rather than finite unions. The $\sigma$-algebras of subsets are the fundamental items that we will apply measures to.
\begin{definition}[$\sigma$-Algebra of Subsets]
  The collection $\mathcal{A} = \set{A_i}_{i\in I}$ is known as a \textit{$\sigma$-algebra of subsets} for $\Omega$ if
  \begin{itemize}
    \item $\emptyset,\Omega \in \mathcal{A}$;
    \item for any $A_i\in \mathcal{A}$, $A_i^{c}\in \mathcal{A}$;
    \item for any countable collection $\set{A_n}_{n\geq 1}\subseteq \mathcal{A}$, $\bigcup_{n\geq 1}A_{n} \in \mathcal{A}$.
  \end{itemize}
\end{definition}
\begin{definition}[Measurable Space]
A pair $\left(\Omega,\mathcal{A}\right)$, where $\Omega$ is a set and $A\subseteq P(\Omega)$ is a $\sigma$-algebra, is called a \textit{measurable space}. Elements in the measurable space are called $\mathcal{A}$-measurable sets.
\end{definition}
\begin{definition}[Restriction of a $\sigma$-Algebra]
  For a measurable space $\left(\Omega,\mathcal{A}\right)$, with $E\in \mathcal{A}$, the family
  \begin{align*}
    \mathcal{A}_{E} &= \set{E\cap A\mid A\in \mathcal{A}}
  \end{align*}
  is a $\sigma$-algebra on $E$, known as the \textit{restriction} of $\mathcal{A}$ to $E$.
\end{definition}
\begin{definition}[Produced $\sigma$-Algebra]
Let $\left(\Omega,\mathcal{A}\right)$ be a measurable space, and $f\colon \Omega\rightarrow \Lambda$ is a map. The $\sigma$-algebra \textit{produced} by $f$ on $\Lambda$ is the collection
\begin{align*}
  \mathcal{N} &= \set{E\mid E\subseteq \Lambda,~f^{-1}(E) \in \mathcal{A}}.
\end{align*}
\end{definition}
\begin{definition}[Generated $\sigma$-Algebra]
  For a family $\mathcal{E}\subseteq P\left(\Omega\right)$, the $\sigma$-algebra \textit{generated} by $E$ is the smallest $\sigma$-algebra that contains $E$.
  \begin{align*}
    \sigma\left(\mathcal{E}\right) &= \bigcap\set{\mathcal{M} | \mathcal{E}\subseteq \mathcal{M},\mathcal{M}\text{ is a $\sigma$-algebra}}.
  \end{align*}
\end{definition}
\begin{definition}[Borel $\sigma$-Algebra]
  If $\Omega$ is a topological space with the topology $\tau\subseteq P(\Omega)$, we define
  \begin{align*}
    \mathcal{B}_{\Omega} &= \sigma\left(\tau\right)
  \end{align*}
  to be the \textit{Borel $\sigma$-algebra}.
\end{definition}
All open, closed, clopen, $F_{\sigma}$, and $G_{\delta}$ subsets of $\Omega$ are Borel.\break

We can now begin examining measurable functions.
\begin{definition}[Measurable Functions]
  Let $\left(\Omega,\mathcal{M}\right)$ and $\left(\Lambda,\mathcal{N}\right)$ be measurable spaces.
  \begin{enumerate}[(1)]
    \item We say a map $f\colon \Omega\rightarrow \Lambda$ is \textit{$\mathcal{M}$-$\mathcal{N}$-measurable} if $f^{-1}\left(E\right)\in \mathcal{M}$ for all $E\in \mathcal{N}$.
    \item We say a map $f\colon \Omega\rightarrow \R$ is measurable if it is $\mathcal{M}$-$\mathcal{B}_{\R}$-measurable.
    \item We say a map $f\colon \Omega\rightarrow \C$ is measurable if both $\re(f)$ and $\im(f)$ are measurable.
  \end{enumerate}
  The set of all measurable functions on $\left(\Omega,\mathcal{M}\right)$ is denoted $L_{0}\left(\Omega,\mathcal{M}\right)$.\newline

  The collection of all bounded measurable functions is the set
  \begin{align*}
    B_{\infty}\left(\Omega,\mathcal{M}\right) &= \set{f\in L_0\left(\Omega,\mathcal{M}\right)\mid \sup_{x\in\Omega}\left\vert f(x) \right\vert < \infty}.
  \end{align*}
\end{definition}
\begin{example}
  If $f\colon \Omega\rightarrow \Lambda$ is a continuous map between topological spaces, then $f$ is $\mathcal{B}_{\Omega}$-$\mathcal{B}_{\Lambda}$-measurable, since
  \begin{align*}
    \mathcal{F} &= \set{E\subseteq \Lambda\mid f^{-1}\left(E\right)\in \mathcal{B}_{\Omega}}
  \end{align*}
  is a $\sigma$-algebra containing every open set in $\Lambda$, so $\mathcal{F}$ contains $\mathcal{B}_{\Lambda}$.
\end{example}
\begin{example}
  If $\left(\Omega,\mathcal{M}\right)$ is a measurable space, and $f\colon \Omega\rightarrow \Lambda$ is a map, the measurable space $\left(\Lambda,\mathcal{N}\right)$ produced by $f$ is necessarily $\mathcal{M}$-$\mathcal{N}$-measurable.
\end{example}

\begin{fact}
  If $\left(\Omega,\mathcal{M}\right)$, $\left(\Lambda,\mathcal{N}\right)$, and $\left(\Sigma,\mathcal{L}\right)$ are measurable spaces, with $f\colon \Omega\rightarrow \Lambda$ and $g\colon \Lambda\rightarrow \Sigma$ measurable, then $g\circ f$ is measurable.\label{fact:composition}
\end{fact}
\begin{proposition}
  Let $\left(\Omega,\mathcal{M}\right)$ be a measurable space. Let $\F = \C$ or $\R$. Suppose $f,g,h_n\colon \Omega\rightarrow \F$ are measurable for $n\geq 1$.
  \begin{enumerate}[(1)]
    \item If $\alpha \in \F$, then $f + \alpha g$ is measurable.
    \item $\overline{f}$ is measurable.
    \item $fg$ is measurable.
    \item $\frac{f}{g}$ is measurable assuming it is well-defined.
    \item if $h_n$ are $\R$-valued, and $\left(h_n\left(x\right)\right)_n$ is bounded for each $x\in \Omega$, then $\sup h_n$ and $\inf h_n$ are measurable.
    \item If $f$ and $g$ are $\R$ valued, then $\max\left(f,g\right)$ and $\min\left(f,g\right)$ are measurable. In particular,
      \begin{align*}
        f_{+} &= \max\left(f,0\right)\\
        f_{-} &= \max\left(0,-f\right)
      \end{align*}
      are measurable.
    \item $\left\vert f \right\vert$ is measurable.
    \item The pointwise limit of measurable functions is measurable --- if $\lim_{n\rightarrow\infty}h_n\left(x\right)$ exists for all $x\in \Omega$, then $h = \lim_{n\rightarrow\infty}h_n$ is measurable.
  \end{enumerate}
\end{proposition}
\begin{definition}[Simple Functions]
  A \textit{simple function} $s\colon \Omega\rightarrow \F$ is a function with finite range. In other words, $s$ is of the form
  \begin{align*}
    s &= \sum_{k=1}^{n}c_k\1_{E_k}
  \end{align*}
  for $E_k\subseteq \Omega$ and $c_k\in \F$.
\end{definition}
\begin{fact}
A simple function is measurable if and only if $E_k\in \mathcal{M}$ for each $k$.
\end{fact}
\section{Constructing Measures}%
A measure assigns a nonnegative ``length'' or ``volume'' to measurable sets.
\begin{definition}[Measures on Measurable Spaces]\label{def:measure_basics}
  A \textit{measure} on a measurable space $\left(\Omega,\mathcal{M}\right)$ is a map $\mu\colon \mathcal{M}\rightarrow \left[0,\infty\right]$ that satisfies the following.
  \begin{enumerate}[(i)]
    \item $\mu\left(\emptyset\right) = 0$;
    \item $\displaystyle \mu\left(\bigsqcup_{j=1}^{\infty}E_j\right) = \sum_{j=1}^{\infty}\mu\left(E_j\right)$.
  \end{enumerate}
  The triple $\left(\Omega,\mathcal{M},\mu\right)$ is called a \textit{measure space}.\newline

  A measure $\mu$ is \textit{finite} if $\mu\left(\Omega\right) < \infty$\newline

  If $\mu\left(\Omega\right) = 1$, then $\mu$ is called a \textit{probability measure}.\newline

  A measure $\mu$ is called \textit{finitely additive} if $\mu\left(E\sqcup F\right) = \mu(E) + \mu(F)$.\newline

  A measure $\mu$ is called \textit{$\sigma$-finite} if there is a countable family $\set{E_n}_{n\geq 1}\subseteq \mathcal{M}$ such that
  \begin{align*}
    \Omega &= \bigcup_{n\geq 1}E_n
  \end{align*}
  and $\mu\left(E_n\right) < \infty$.\newline

  A measure $\mu$ on $\left(\Omega,\mathcal{M}\right)$ is called \textit{semi-finite} if, for every $E\in \mathcal{M}$ with $\mu(E) = \infty$, there exists $F\in \mathcal{M}$ with $F\subseteq E$ and $0 < \mu(F) < \infty$.
\end{definition}

\begin{lemma}
  Let $\left(\Omega,\mathcal{M},\mu\right)$ be a measure space.
  \begin{enumerate}[(1)]
    \item If $E,F\in \mathcal{M}$ with $F\subseteq E$, then $\mu\left(F\right) \subseteq \mu\left(E\right)$.
    \item If $\left(E_n\right)_{n}$ is a sequence of measurable sets, then
      \begin{align*}
        \mu\left(\bigcup_{n\geq 1}E_n\right) &\leq \sum_{n=1}^{\infty}\mu\left(E_n\right).
      \end{align*}
    \item If $\left(E_n\right)_{n\geq 1}$ is an increasing family of measurable sets, then
      \begin{align*}
        \mu\left(\bigcup_{n\geq 1}E_n\right) &= \lim_{n\rightarrow\infty}\mu\left(E_n\right).
      \end{align*}
  \end{enumerate}
\end{lemma}
Just as functions can produce $\sigma$-algebras that create measurable spaces in the codomain by ``pushing forward'' the $\sigma$-algebra of the domain, measurable functions can push forward the measure on their domain.
\begin{definition}[Pushforward Measure]
  Let $\left(\Omega,\mathcal{M},\mu\right)$ be a measure space, and let $\left(\Lambda,\mathcal{N}\right)$ be a measurable space. Let $f\colon \Omega\rightarrow \Lambda$ be measurable. The map
  \begin{align*}
    f_{\ast}\mu\colon \mathcal{N}\rightarrow [0,\infty]
  \end{align*}
  defined by
  \begin{align*}
    f_{\ast}\mu\left(E\right) &= \mu\left(f^{-1}\left(E\right)\right)
  \end{align*}
  defines a measure on $\left(\Lambda,\mathcal{N}\right)$. This is known as the \textit{pushforward measure} of $\mu$.\newline

  If $\mathcal{N}$ on $\Lambda$ is produced by $f$, then the pushforward measure is necessarily defined on $\mathcal{N}$, and that any function $g\colon \Lambda\rightarrow \F$ is measurable if and only if $g\circ f$ is measurable.
\end{definition}
\begin{definition}
  Let $\left(\Omega,\mathcal{M},\mu\right)$ be a measure space.\newline

  A \textit{null set} is a measurable set $N\in \mathcal{M}$ with $\mu\left(N\right) = 0$.\newline

  A property which holds for all $x\in \Omega\setminus N$ for some null set $N$ is said to hold \textit{$\mu$-almost everywhere,} or $\mu$-a.e.
\end{definition}
\begin{definition}
  If $\left(\Omega,\mathcal{M},\mu\right)$ is a measure space, we can define an equivalence relation on the set $L_{0}\left(\Omega,\mathcal{M},\mu\right)$, by
  \begin{align*}
    f\sim_{\mu}g \text{ if and only if } \mu\left(\set{x\mid f(x)\neq g(x)}\right) = 0.
  \end{align*}
  We define the set of all classes of measurable functions by
  \begin{align*}
    L\left(\Omega,\mu\right) &= L_{0}\left(\Omega,\mathcal{M}\right)/\sim_{\mu}\\
                             &= \set{\left[f\right]_{\mu}\mid f\in L_{0}\left(\Omega,\mathcal{M}\right)}.
  \end{align*}
\end{definition}
\begin{fact}
  The operations
  \begin{itemize}
    \item $\displaystyle \left[f\right]_{\mu} + \left[g\right]_{\mu} = \left[f + g\right]_{\mu}$;
    \item $\displaystyle \left[f\right]_{\mu}\left[g\right]_{\mu} = \left[fg\right]_{\mu}$;
    \item and $\displaystyle \alpha \left[f\right]_{\mu} = \left[\alpha f\right]_{\mu}$
  \end{itemize}
  are well-defined.
\end{fact}
\begin{definition}[Essentially Bounded Functions and Continuous Functions]
  Let $\left(\Omega,\mathcal{M},\mu\right)$ be a measure space, and $f\colon \Omega\rightarrow \C$ be measurable. We say $f$ is \textit{$\mu$-essentially bounded} if there is $C\geq 0$ such that
  \begin{align*}
    \mu\left(\set{x\in \Omega\mid \left\vert f(x) \right\vert\geq C}\right) = 0.
  \end{align*}
  We say $C$ is an essential bound for $f$. The infimum of all essential bounds is the \textit{essential supremum}, which gives the norm
  \begin{align*}
    \norm{f}_{L_\infty} &= \esssup(f)\\
                      &= \inf\set{C\geq 0 \mid \mu\left(\set{x\in \Omega\mid \left\vert f(x) \right\vert\geq C}\right) = 0}.
  \end{align*}
  The collection of all $\mu$-essentially bounded functions is denoted
  \begin{align*}
    L_{\infty}\left(\Omega,\mu\right) = \set{\left[f\right]_{\mu}\in L\left(\Omega,\mu\right)\mid \norm{f}_{L_\infty} < \infty}.
  \end{align*}
  Note that $B_{\infty}\left(\Omega,\mu\right) = L_{\infty}\left(\Omega,\mu\right)$ as sets.\newline

  For $\mu$ a measure on $\left(\Omega,\mathcal{B}_{\Omega}\right)$, the $\mu$-equivalence classes of continuous functions are
  \begin{align*}
    C\left(\Omega,\mu\right) = \set{\left[f\right]_{\mu}\mid f\in C\left(\Omega\right)}.
  \end{align*}
\end{definition}
\begin{fact}
  If $\Omega$ is a topological space, with $\mathcal{B}_{\Omega}$ the Borel $\sigma$-algebra, we have $C\left(\Omega\right)\subseteq L_{0}\left(\Omega,\mathcal{B}_{\Omega}\right)$.\newline
\end{fact}
\begin{remark}
  Members of $L\left(\Omega,\mu\right)$ and $L_{\infty}\left(\Omega,\mu\right)$ are equivalence classes of functions (rather than functions themselves), but we use the abuse of notation that $\left[f\right]_{\mu} = f$.
\end{remark}

\begin{fact}
  Let $\left(\Omega,\mathcal{M},\mu\right)$ be a measure space, and let $f,g\colon \Omega\rightarrow \C$ be measurable, and $\alpha \in \C$. Then, the following are true:
  \begin{itemize}
    \item $\displaystyle \norm{f+g}_{L_\infty}\leq \norm{f}_{L_\infty} + \norm{g}_{L_\infty}$;
    \item $\displaystyle \norm{\alpha f}_{L_\infty} = \left\vert \alpha \right\vert\norm{f}_{L_\infty}$;
    \item if $\displaystyle \norm{f}_{L_\infty} = 0$, then $f = 0$ $\mu$-a.e.;
    \item $\displaystyle \norm{f}_{L_\infty}\leq \norm{f}_{u}$;
    \item if $f$ is essentially bounded, then
      \begin{align*}
        \mu\left(\set{x\mid \left\vert f(x) \right\vert\geq \norm{f}_{L_\infty}}\right) &= 0.
      \end{align*}
  \end{itemize}
\end{fact}

\begin{definition}[Complete Measure Space]
A measure space $\left(\Omega,\mathcal{M},\mu\right)$ is said to be \textit{complete} if all subsets of null sets are measurable (and null).
\end{definition}
\section{Integration}%
Measure theory was developed to allow for integration in its most general form.
\begin{definition}
  If $\phi\colon \Omega\rightarrow [0,\infty)$ is a positive, simple, and measurable function, 
  \begin{align*}
    \phi &= \sum_{k=1}^{n}c_k\1_{E_k},
  \end{align*}
  then the \textit{integral} of $\phi$ is defined as
  \begin{align*}
    \int_{\Omega}\phi\:d\mu &= \sum_{k=1}^{n}c_k\mu\left(E_k\right),
  \end{align*}
  with the convention that $0\cdot \infty = 0$.
\end{definition}
\begin{fact}
The value of this integral is not dependent on the representation of $\phi$.
\end{fact}
\begin{definition}
  If $f\colon \Omega\rightarrow [0,\infty)$ is a positive measurable function, then
  \begin{align*}
    \int_{\Omega}f\:d\mu &= \sup\set{\int_{\Omega}\phi\:d\mu\mid \phi\text{ measurable and simple, $0\leq \phi \leq f$}}.
  \end{align*}
  If $E\subseteq \Omega$ is measurable, we define
  \begin{align*}
    \int_{E} f\:d\mu &= \int_{\Omega}f\1_{E}\:d\mu.
  \end{align*}
\end{definition}
\begin{proposition}
  Let $\left(\Omega,\mathcal{M}\right)$ be a measurable space, and let $f:\Omega\rightarrow \C$ be measurable. There is a sequence $\left(\phi_{n}\right)_n$ of simple, measurable functions with $\left(\phi_{n}\left(x\right)\right)_{n}\xrightarrow{n\rightarrow\infty}f(x)$.\newline

  If $f\geq 0$, we can take $\phi_n$ to be positive and pointwise increasing.\newline

  If $f$ is bounded, then this convergence is uniform, and $\left(\phi_{n}\right)_n$ can be chosen to be uniformly bounded.
\end{proposition}

\begin{theorem}[Monotone Convergence Theorem]
  Let $\left(f_n:\Omega\rightarrow [0,\infty)\right)_n$ be an increasing sequence of positive, measurable functions converging pointwise to $f\colon \Omega\rightarrow [0,\infty)$. Then, $f$ is measurable, and
  \begin{align*}
    \lim_{n\rightarrow\infty}\int_{\Omega}^{} f_n\:d\mu &= \int_{\Omega}^{} f\:d\mu.
  \end{align*}
\end{theorem}
\begin{definition}
  Let $\left(\Omega,\mathcal{M},\mu\right)$ be a measure space.
  \begin{enumerate}[(1)]
    \item A measurable function $f\colon \Omega\rightarrow [0,\infty)$ is \textit{integrable} if
      \begin{align*}
        \int_{\Omega}^{} f\:d\mu < \infty.
      \end{align*}
    \item A measurable function $f\colon \Omega\rightarrow \R$ is integrable if both $f_{+}$ and $f_{-}$ are integrable. We define
      \begin{align*}
        \int_{\Omega}^{} f\:d\mu &= \int_{\Omega}^{} f_{+}\:d\mu - \int_{\Omega}^{} f_{-}\:d\mu.
      \end{align*}
    \item A measurable function $f\colon \Omega\rightarrow \C$ is said to be integrable if both $\re(f)$ and $\im(f)$ are integrable. We define
      \begin{align*}
        \int_{\Omega}^{} f\:d\mu &= \int_{\Omega}^{} \re(f)\:d\mu + i\int_{\Omega}^{} \im(f)\:d\mu.
      \end{align*}
  \end{enumerate}
\end{definition}
\begin{fact}
  Let $f,g\colon \Omega\rightarrow \C$ be integrable functions, and $\alpha\in\C$. Then,
  \begin{itemize}
    \item $f + \alpha g$ is integrable, and $\displaystyle \int_{\Omega}^{} \left(f + \alpha g\right)\:d\mu = \int_{\Omega}f\:d\mu + \alpha\int_{\Omega}g\:d\mu$;
    \item if $f$ and $g$ are real-valued, and $f\leq g$, then $\displaystyle \int_{\Omega}^{} f\:d\mu \leq \int_{\Omega}^{} g\:d\mu$;
    \item $\displaystyle \left\vert \int_{\Omega}^{} f\:d\mu \right\vert\leq \int_{\Omega}^{} \left\vert f \right\vert\:d\mu$.
  \end{itemize}
\end{fact}
\begin{fact}
  If $f = g$ $\mu$-a.e., then
  \begin{align*}
    \int_{\Omega}^{} f\:d\mu &= \int_{\Omega}^{} g\:d\mu.
  \end{align*}
\end{fact}
\begin{fact}
  If $f\colon \Omega\rightarrow \C$ is measurable, then $\int_{\Omega}^{} \left\vert f \right\vert\:d\mu = 0$ if and only if $f = 0$ $\mu$-a.e.
\end{fact}
\begin{fact}
  A measurable function $f\colon \Omega\rightarrow \C$ is integrable if and only if $\left\vert f \right\vert$ is integrable.
\end{fact}
\begin{definition}[Integrable Functions]
  Let $\left(\Omega,\mathcal{M},\mu\right)$ be a measure space.
  \begin{enumerate}[(1)]
    \item We define the set of (equivalence classes of) integrable functions to be
      \begin{align*}
        L_{1}\left(\Omega,\mu\right) = \set{\left[f\right]_{\mu}\in L\left(\Omega,\mu\right)\mid \text{$f$ is integrable}}.
      \end{align*}
    \item We define the set of (equivalence classes of) square-integrable functions to be
      \begin{align*}
        L_{2}\left(\Omega,\mu\right) &= \set{\left[f\right]_{\mu}\in L\left(\Omega,\mu\right)\mid \left\vert f \right\vert^2\text{ is integrable}}.
      \end{align*}
  \end{enumerate}
\end{definition}
\begin{definition}
  Let $\left(\Omega,\mathcal{M},\mu\right)$ be a measure space. If $f$ and $\left(f_n\right)_n$ are integrable with $\norm{f-f_n}_{L_1}\xrightarrow{n\rightarrow\infty}0$, we say $\left(f_n\right)_n$ \textit{converges in mean} to $f$.
\end{definition}

\begin{fact}
  Let $\left(\Omega,\mathcal{M},\mu\right)$ be a measure space.
  \begin{enumerate}[(1)]
    \item For $f\in L_{1}\left(\Omega,\mu\right)$, the maps
      \begin{align*}
        \left[f\right]_{\mu} &\longmapsto \int_{\Omega}^{} f\:d\mu\\
        \left[f\right]_{\mu} &\longmapsto \int_{\Omega}^{} \left\vert f \right\vert\:d\mu
      \end{align*}
      are well-defined.
    \item For $f\in L_{1}\left(\Omega,\mu\right)$, we define
      \begin{align*}
        \norm{f}_{L_1} &= \int_{\Omega}^{} \left\vert f \right\vert\:d\mu.
      \end{align*}
      This is a well-defined norm.
      \begin{align*}
        \norm{f+g}_{L_1} &\leq \norm{f}_{L_1} + \norm{g}_{L_1}\\
        \norm{\alpha f}_{L_1} &= \left\vert \alpha \right\vert\norm{f}_{L_1},
      \end{align*}
      and $\norm{f}_{L_1} = 0$ if and only if $f = 0$ $\mu$-a.e.
    \item 
      \begin{align*}
        d\left(\left[f\right]_{\mu},\left[g\right]_{\mu}\right) &= \norm{f-g}_{L_1}
      \end{align*}
      is a metric on $L_{1}\left(\Omega,\mu\right)$.
  \end{enumerate}
\end{fact}
\begin{theorem}[Dominated Convergence Theorem]
  Let $\left(f_n:\Omega\rightarrow \C\right)_{n}$ be a sequence of measurable functions converging pointwise to a measurable function $f\colon \Omega\rightarrow \C$. If there is an integrable $g\colon \Omega \rightarrow [0,\infty)$ with $\left\vert f_n \right\vert\leq g$ for all $n$, then
  \begin{align*}
    \int_{\Omega}^{} f_n\:d\mu \xrightarrow{n\rightarrow\infty}\int_{\Omega}^{} f\:d\mu.
  \end{align*}
\end{theorem}
\begin{corollary}
  If $f\colon \Omega\rightarrow \C$ is integrable, then there is a sequence of simple integrable functions $\left(\phi_n\right)_n$ with $\norm{f - \phi_{n}}_{L_1}\xrightarrow{n\rightarrow\infty}0$.
\end{corollary}
\begin{corollary}
  If $f\colon \R\rightarrow\C$ is integrable, then there is a sequence $\left(f_n\right)_n$ of compactly supported integrable functions such that $\norm{f - f_n}_{L_1}\xrightarrow{n\rightarrow\infty} 0$.
\end{corollary}
\begin{theorem}
  If $f\colon \R\rightarrow\C$ is integrable, and $\ve > 0$, there is a continuous, compactly supported function $g$ with $\norm{f - g}_{L_1} < \ve$.
\end{theorem}
\begin{proposition}
  Let $\left(\Omega,\mathcal{M},\mu\right)$ be a measure space, and let $\left(\Lambda,\mathcal{N}\right)$ be a measurable space with $f\colon \Omega\rightarrow \Lambda$ a measurable map. Let $f_{\ast}\mu$ be the pushforward measure on $\left(\Lambda,\mathcal{N}\right)$. For a measurable function $g\colon \Lambda\rightarrow [0,\infty)$, then
  \begin{align*}
    \int_{\Lambda}^{} g\:d\left(f_{\ast}\mu\right) &= \int_{\Omega}^{} \left(g\circ f\right)\:d\mu.
  \end{align*}
  Moreover, if $g\colon \Lambda\rightarrow \F$ is integrable with respect to $f_{\ast}\mu$, then so too is $g\circ f$ with respect to $\mu$.
\end{proposition}
\section{Complex Measures}%
If $\mathcal{M}$ is a $\sigma$-algebra, it need not be the case that a measure $\mu\colon \mathcal{M}\rightarrow [0,\infty]$ have $[0,\infty]$ as a codomain. In the most general case in this section, we consider measures of the form $\mu\colon \mathcal{M}\rightarrow \C$.
\begin{example}
  If $\left(\Omega,\mathcal{M},\mu\right)$ is a measure space, then the map $\mu_f(E) = \int_{E}^{} f\:d\mu$ is a well-defined measure.
\end{example}
\begin{definition}
  Let $\left(\Omega,\mathcal{M},\mu\right)$ be a measurable space.
  \begin{enumerate}[(1)]
    \item A \textit{complex measure} on $\left(\Omega,\mathcal{M},\mu\right)$ is a map $\mu\colon \mathcal{M}\rightarrow \C$ satisfying the following conditions.
      \begin{itemize}
        \item $\mu\left(\emptyset\right) = 0$;
        \item $\displaystyle \mu\left(\bigsqcup_{k=1}^{\infty}E_k\right) = \sum_{k=1}^{\infty}\mu\left(E_k\right)$ for $\set{E_k}_{k\geq 1}\subseteq \mathcal{M}$.
      \end{itemize}
    \item We write $M\left(\Omega\right)$ to be the set of all complex measures on $\left(\Omega,\mathcal{M}\right)$.
    \item If $\mu\in M\left(\Omega\right)$, and $\mu\left(E\right)\in \R$ for all $E\in \mathcal{M}$, then we say $\mu$ is a \textit{real measure} on $\left(\Omega,\mathcal{M}\right)$.
    \item If $\mu\in M\left(\Omega\right)$ and $\mu(E) \geq 0$ for all $E\in \mathcal{M}$, then we say $\mu$ is a \textit{positive measure} on $\left(\Omega,\mathcal{M}\right)$.
    \item If $\mu$ is a positive measure on $\left(\Omega,\mathcal{M}\right)$ with $\mu\left(\Omega\right) = 1$, we say $\mu$ is a probability measure on $\left(\Omega,\mathcal{M}\right)$. We write $\mathcal{P}\left(\Omega,\mathcal{M}\right)$ to be the collection of all probability measures on $\left(\Omega,\mathcal{M}\right)$.
    \item If $\Omega$ is a LCH space, we always let $M\left(\Omega\right)$ be the set of all complex Borel measures on $\Omega$.
  \end{enumerate}
\end{definition}
\begin{definition}
  If $\left(\Omega,\mathcal{M}\right)$ is a measurable space, and $x\in \Omega$, the \textit{Dirac measure} at $x$ is defined by
  \begin{align*}
    \delta_{x}\colon \mathcal{M}&\rightarrow [0,1]\\
    \delta_x\left(E\right) &= \begin{cases}
      1 & x\in E\\
      0 & x\notin E
    \end{cases}.
  \end{align*}
    If $x_1,\dots,x_n$ are distinct points in $\Omega$, and $t_1,\dots,t_n\in [0,1]$ with $\sum_{j=1}^{n}t_j = 1$, then
    \begin{align*}
      \mu &= \sum_{j=1}^{n}t_j\delta_{x_j}
    \end{align*}
    is a probability measure on $\left(\Omega,\mathcal{M}\right)$.
\end{definition}
\begin{fact}
  If $\mu$ is a complex measure on $\left(\Omega,\mathcal{M}\right)$, then $\overline{\mu}$, defined by $\overline{\mu}\left(E\right) = \overline{\mu\left(E\right)}$ for $E\in \mathcal{M}$, is also a complex measure. Additionally, $\re\left(\mu\right)$ and $\im\left(\mu\right)$, defined by
  \begin{align*}
    \re\left(\mu\right)\left(E\right) &= \re\left(\mu\left(E\right)\right)\\
    \im\left(\mu\right)\left(E\right) &= \im\left(\mu\left(E\right)\right)
  \end{align*}
  are real measures.
\end{fact}
\begin{definition}
  If $\mu\in M\left(\Omega\right)$, then the \textit{total variation} of $\mu$ is the quantity
  \begin{align*}
    \left\vert \mu \right\vert\colon \mathcal{M}\rightarrow [0,\infty]
    \end{align*}
    with
    \begin{align*}
    \left\vert \mu \right\vert\left(E\right) = \sup\set{\left.\sum_{j=1}^{\infty}\left\vert \mu\left(E_j\right) \right\vert\right|E = \bigsqcup_{j=1}^{\infty}E_j,~E_j\in\mathcal{M}}.
  \end{align*}
\end{definition}
\begin{fact}
  If $\mu\in M\left(\Omega\right)$, then $\left\vert \mu \right\vert$ is a positive, finite measure. Additionally, if $\mu,\nu\in M\left(\Omega\right)$ with $\alpha\in \C$, then
  \begin{enumerate}[(a)]
    \item $\left\vert \mu\left(E\right) \right\vert\leq \left\vert \mu \right\vert\left(E\right)$
    \item $\left\vert \mu + \nu \right\vert\left(E\right) \leq \left\vert \mu \right\vert\left(E\right) + \left\vert \nu \right\vert\left(E\right)$
    \item $\left\vert \alpha\mu \right\vert\left(E\right) = \left\vert \alpha \right\vert\left\vert \mu \right\vert\left(E\right)$.
  \end{enumerate}
\end{fact}
\begin{definition}[Absolute Continuity of Measures]
  Let $\left(\Omega,\mathcal{M}\right)$ be a measurable space, and let $\mu$ and $\nu$ be positive measures on this space. If $\mu(A) > 0$ implies $\nu(A) > 0$ for a given $A\in \mathcal{M}$, we say $\mu$ is \textit{absolutely continuous} with respect to $\nu$. We write $\mu \ll \nu$.
\end{definition}
\begin{theorem}[Radon--Nikodym Theorem]
  If $\mu \ll \nu$ on $\left(\Omega,\mathcal{M}\right)$, then there exists a measurable function $f\colon \Omega\rightarrow [0,\infty]$ such that
  \begin{align*}
    \nu(A) &= \int_{A}^{} f \:d\nu
  \end{align*}
  for each $A\in \mathcal{M}$.
\end{theorem}
\begin{remark}
The Radon--Nikodym theorem extends to signed and complex measures.
\end{remark}
\begin{fact}
  Let $\left(\Omega,\mathcal{M},\lambda\right)$ be a measure space, and suppose $f\in L_{1}\left(\Omega,\lambda\right)$. Then, $\mu\left(E\right) = \int_{E}^{} f\:d\lambda$ defines a complex measure. We write $f = \diff{\mu}{\lambda}$, which is the \textit{Radon--Nikodym derivative} of $\mu$ with respect to $\lambda$.\newline

  It is also the case that
  \begin{align*}
    \left\vert \mu \right\vert\left(E\right) &= \int_{E}^{} \left\vert f \right\vert\:d\lambda.
  \end{align*}
\end{fact}
\begin{fact}
  If $\mu\in M\left(\Omega\right)$, there exists a measurable function $f\colon \Omega\rightarrow \C$ such that $\left\vert f \right\vert = 1$ and $\mu\left(E\right) = \int_{E}^{} f\:d\left\vert \mu \right\vert$ for all $E\in \mathcal{M}$.
\end{fact}
\begin{definition}
  Let $\Omega$ be a LCH space equipped with the Borel $\sigma$-algebra, $\mathcal{B}_{\Omega}$.
  \begin{enumerate}[(1)]
    \item A Borel measure $\mu\colon \mathcal{B}_{\Omega}\rightarrow [0,\infty]$ is called
      \begin{itemize}
        \item \textit{inner regular} on $E\in \mathcal{B}_{\Omega}$ if
          \begin{align*}
            \mu(E) &= \sup\set{\mu\left(K\right)\mid K\subseteq E, K\text{ compact}};
          \end{align*}
        \item \textit{outer regular} on $E\in \mathcal{B}_{\Omega}$ if
          \begin{align*}
            \mu(E) &= \inf\set{\mu\left(U\right)\mid U\supseteq E,U\text{ open}};
          \end{align*}
        \item \textit{regular} on $E$ if it is inner regular and outer regular on $E$;
        \item regular if it is regular on all $E\in \mathcal{B}_{\Omega}$;
        \item \textit{Radon} if
          \begin{itemize}
            \item $\mu(K) < \infty$ for all compact $K\subseteq \Omega$;
            \item $\mu$ is inner regular on all open sets and outer regular on all Borel sets.
          \end{itemize}
      \end{itemize}
    \item A complex Borel measure $\mu\colon \mathcal{B}_{\Omega}\rightarrow \C$ is regular if $\left\vert \mu \right\vert$ is regular; $\mu$ is Radon if $\left\vert \mu \right\vert$ is Radon.
    \item We write $M_{r}\left(\Omega\right)$ to denote the set of all complex regular measures on $\left(\Omega,\mathcal{B}_{\Omega}\right)$.
  \end{enumerate}
\end{definition}
\begin{fact}
  Every positive Radon measure is regular. Thus, every complex Borel measure is regular if and only if it is Radon.\newline

  Moreover, if $\Omega$ is a second countable LCH space, then every complex Borel measure is regular.
\end{fact}

\begin{definition}
  Let $\left(\Omega,\tau\right)$ be a topological space, and suppose $\mu\colon \mathcal{B}_{\Omega}\rightarrow [0,\infty]$ is a Borel measure.
  \begin{enumerate}[(1)]
    \item The \textit{kernel} of $\mu$ is the set
      \begin{align*}
        N_{\mu} &= \bigcup\set{U\subseteq \Omega\mid U\in\tau,~\mu(U) = 0}.
      \end{align*}
    \item The \textit{support} of $\mu$ is the complement of the kernel, $\supp(\mu) = N_{\mu}^{c}$.
  \end{enumerate}
\end{definition}
\begin{fact}
  If $\mu$ is a Radon measure on a LCH space $\Omega$, then $\mu\left(N_{\mu}\right) = 0$, meaning $\mu\left(\Omega\right) = \mu\left(\supp\left(\mu\right)\right)$.
\end{fact}

\begin{theorem}[Hahn and Jordan Decomposition]
  Let $\left(\Omega,\mathcal{M}\right)$ be a measurable space, and let $\mu\colon \mathcal{M}\rightarrow \R$ be a real measure. Then, there is a measurable partition $\Omega = P\sqcup N$ such that for all $E\subseteq P$, $\mu(E) \geq 0$, and for all $E\subseteq N$, $\mu(E) \leq 0$. This partition is unique up to a $\mu$-null symmetric difference --- that is, for any $P',N'$ satisfying the conditions, $\mu\left(P'\triangle P\right) = 0$ and $\mu\left(N'\triangle N\right) = 0$.\newline

  There is a unique decomposition $\mu = \mu_{+} - \mu_{-}$, with $\mu_{\pm}$ that are positive such that if $E\subseteq P$, then $\mu_{-}\left(E\right) = 0$, and if $E\subseteq N$, $\mu_{+}\left(E\right) = 0$.
\end{theorem}
\begin{definition}
  Let $\left(\Omega,\mathcal{M}\right)$ be a measurable space, and let $f\colon \Omega\rightarrow \C$ be measurable.
  \begin{enumerate}[(1)]
    \item If $\mu\colon \mathcal{M}\rightarrow \R$ is a real measure with $\mu = \mu_{+} - \mu_{-}$, we say that $f$ is $\mu$-integrable if it is both $\mu_{+}$ and $\mu_{-}$-integrable. We define
      \begin{align*}
        \int_{\Omega}^{} f\:d\mu &= \int_{\Omega}^{} f\:d\mu_{+} - \int_{\Omega}^{} f\:d\mu_{-}.
      \end{align*}
    \item If $\mu:\mathcal{M}\rightarrow \C$ is a complex measure with $\mu_{1} = \re\left(\mu\right)$ and $\mu_{2} = \im\left(\mu\right)$, we say $f$ is $\mu$-integrable if it is both $\mu_{1}$ and $\mu_{2}$-integrable. We define
      \begin{align*}
        \int_{\Omega}^{} f\:d\mu &= \int_{\Omega}^{} f\:d\mu_{1} + i\int_{\Omega}^{} f\:d\mu_{2}.
      \end{align*}
  \end{enumerate}
\end{definition}
%\begin{theorem}[Riesz Representation Theorem on $C_c\left(\Omega\right)$]
%  Let $\Omega$ be a LCH space. If $\varphi\colon C_{c}\left(\Omega\right)\rightarrow \C$ is a positive linear functional, then there is a unique Radon measure $\mu$ such that
%  \begin{align*}
%    \varphi\left(f\right) &= \int_{\Omega}^{} f\:d\mu
%  \end{align*}
%  for all $f\in C_c\left(\Omega\right)$. Additionally, for every open $U\subseteq \Omega$, we have
%  \begin{align*}
%    \mu\left(U\right) &= \sup\set{\varphi\left(f\right) | f\in C_c\left(\Omega,[0,1]\right),\supp(f) \subseteq U},
%  \end{align*}
%  and for every compact $K\subseteq \Omega$, we have
%  \begin{align*}
%    \mu\left(K\right) &= \inf\set{\varphi(f) | f\geq \1_{K}}.
%  \end{align*}
%\end{theorem}
\begin{theorem}[Riesz Representation Theorem on $C\left(X\right)$]
  Let $X$ be a compact metric space, and let $\varphi\in \left(C\left(X\right)\right)^{\ast}$ be a positive linear functional with $\varphi\left(\1_{X}\right) = \norm{\varphi} = 1$. Then, for $f\in C(X)$, there is a unique Borel probability measure such that
  \begin{align*}
    \varphi\left(f\right) &= \int_{X}^{} f\:d\mu.
  \end{align*}
\end{theorem}
\begin{remark}
There is a version of the Riesz Representation Theorem that holds for the space $C_c\left( \Omega \right)$, where $\Omega$ is a LCH space.
\end{remark}

%\begin{theorem}[Markov--Riesz Theorem]
%  Let $\Omega$ be a LCH space. Then, $M_r\left(\Omega\right)\cong C_0\left(\Omega\right)^{\ast}$.
%\end{theorem}
%\begin{definition}
%  Let $\Omega$ be a LCH space, and let $\tau\colon \Omega\rightarrow \Omega$ be a continuous transformation. A regular Borel probability measure $\mu\in \mathcal{P}_{r}\left(\Omega\right)$ is called $\tau$-invariant if $\tau_{\ast}\mu = \mu$.
%\end{definition}
