In order to properly discuss amenability, we need a strong foundation in measure theory.

\section{Constructing Measurable Spaces}%
Fix a set $\Omega$. We let $\mathcal{A} =\set{A_{i}}_{i\in I}$ be a collection of subsets of $\Omega$.
\begin{definition}[Algebra of Subsets]
  The collection $\mathcal{A} = \set{A_i}_{i\in I}$ is known as an algebra of subsets for $\Omega$ if
  \begin{itemize}
    \item $\emptyset,\Omega \in \mathcal{A}$;
    \item for any $A_i\in \mathcal{A}$, $A_i^{c}\in \mathcal{A}$;
    \item for any $A_i,A_j\in \mathcal{A}$, $A_i \cup A_j \in \mathcal{A}$.
  \end{itemize}
\end{definition}
We can refine the concept of an algebra of subsets to consider countable unions rather than finite unions. This is known as a $\sigma$-algebra.
\begin{definition}[$\sigma$-Algebra of Subsets]
  The collection $\mathcal{A} = \set{A_i}_{i\in I}$ is known as a $\sigma$-algebra of subsets for $\Omega$ if
  \begin{itemize}
    \item $\emptyset,\Omega \in \mathcal{A}$;
    \item for any $A_i\in \mathcal{A}$, $A_i^{c}\in \mathcal{A}$;
    \item for any countable collection $\set{A_n}_{n\geq 1}\subseteq \mathcal{A}$, $\bigcup_{n\geq 1}A_{n} \in \mathcal{A}$.
  \end{itemize}
\end{definition}
\begin{definition}[Measurable Space]
A pair $\left(\Omega,\mathcal{A}\right)$, where $\Omega$ is a set and $A\subseteq P(\Omega)$ is a $\sigma$-algebra, is called a measurable space. Elements in the measurable space are called $\mathcal{A}$-measurable sets.
\end{definition}
\begin{definition}[Restriction of a $\sigma$-Algebra]
  For a measurable space $\left(\Omega,\mathcal{A}\right)$, with $E\in \mathcal{A}$, the family
  \begin{align*}
    \mathcal{A}_{E} &= \set{E\cap A\mid A\in \mathcal{A}}
  \end{align*}
  is a $\sigma$-algebra on $E$, known as the restriction of $\mathcal{A}$ to $E$.
\end{definition}
\begin{definition}[Produced $\sigma$-Algebra]
Let $\left(\Omega,\mathcal{A}\right)$ be a measurable space, and $f: \Omega\rightarrow \Lambda$ is a map. The $\sigma$-algebra produced by $f$ on $\Lambda$ is the collection
\begin{align*}
  \mathcal{N} &= \set{E\mid E\subseteq \Lambda,~f^{-1}(E) \in \mathcal{A}}.
\end{align*}

\end{definition}

\begin{definition}[Generated $\sigma$-Algebra]
  For a family $\mathcal{E}\subseteq P\left(\Omega\right)$, the $\sigma$-algebra generated by $E$ is the smallest $\sigma$-algebra that contains $E$.
  \begin{align*}
    \sigma\left(\mathcal{E}\right) &= \bigcap_{\substack{\mathcal{E}\in \mathcal{M}_i \\ \text{$\mathcal{M}_i$ $\sigma$-Algebra}}} \mathcal{M}_i
  \end{align*}
\end{definition}
\begin{definition}[Borel $\sigma$-Algebra]
  If $\Omega$ is a topological space with topology $\tau\subseteq P(\Omega)$, we define
  \begin{align*}
    \mathcal{B}_{\Omega} &= \sigma\left(\tau\right)
  \end{align*}
  to be the Borel $\sigma$-algebra.
\end{definition}
All open, closed, clopen, $F_{\sigma}$, and $G_{\delta}$ subsets of $\Omega$ are Borel.\break

We can now begin examining measurable functions.
\begin{definition}[Measurable Functions]
  Let $\left(\Omega,\mathcal{M}\right)$ and $\left(\Lambda,\mathcal{N}\right)$ be measurable spaces.
  \begin{enumerate}[(1)]
    \item We say a map $f: \Omega\rightarrow \Lambda$ is $\mathcal{M}$-$\mathcal{N}$-measurable if $f^{-1}\left(E\right)\in \mathcal{M}$ for all $E\in \mathcal{N}$.
    \item We say a map $f: \Omega\rightarrow \R$ is measurable if it is $\mathcal{M}$-$\mathcal{B}_{\R}$-measurable.
    \item We say a map $f: \Omega\rightarrow \C$ is measurable if both $\re(f)$ and $\im(f)$ are measurable.
  \end{enumerate}
  The set of all measurable functions on $\left(\Omega,\mathcal{M}\right)$ is denoted $L_{0}\left(\Omega,\mathcal{M}\right)$.\newline

  The collection of all bounded measurable functions is the set
  \begin{align*}
    B_{\infty}\left(\Omega,\mathcal{M}\right) &= \set{f\in L_0\left(\Omega,\mathcal{M}\right)\mid \sup_{x\in\Omega}\left\vert f(x) \right\vert < \infty}.
  \end{align*}
\end{definition}

\begin{example}
  If $f: \Omega\rightarrow \Lambda$ is a continuous map between topological spaces, then $f$ is $\mathcal{B}_{\Omega}$-$\mathcal{B}_{\Lambda}$-measurable, since
  \begin{align*}
    \mathcal{F} &= \set{E\subseteq \Lambda\mid f^{-1}\left(E\right)\in \mathcal{B}_{\Omega}}
  \end{align*}
  is a $\sigma$-algebra containing every open set in $\Lambda$, so $\mathcal{F}$ contains $\mathcal{B}_{\Lambda}$.
\end{example}

\begin{example}
  If $\left(\Omega,\mathcal{M}\right)$ is a measurable space, and $f: \Omega\rightarrow \Lambda$ is a map, the measurable space $\left(\Lambda,\mathcal{N}\right)$ produced by $f$ is necessarily measurable.
\end{example}

\begin{fact}\label{fact:composition}
  If $\left(\Omega,\mathcal{M}\right)$, $\left(\Lambda,\mathcal{N}\right)$, and $\left(\Sigma,\mathcal{L}\right)$ are measurable spaces, with $f: \Omega\rightarrow \Lambda$ and $g: \Lambda\rightarrow \Sigma$ measurable, then $g\circ f$ is measurable.
\end{fact}
\begin{proof}[Proof of Fact \ref{fact:composition}]
  If $E\in \mathcal{L}$, then $g^{-1}\left(E\right) \in \mathcal{N}$, so $f^{-1}\left(g^{-1}\left(E\right)\right)\in \mathcal{M}$. Thus, $\left(g\circ f\right)^{-1}\left(E\right)\in \mathcal{M}$, so $g\circ f$ is measurable.
\end{proof}

\begin{proposition}
  Let $\left(\Omega,\mathcal{M}\right)$ be a measurable space. Let $\F = \C$ or $\R$. Suppose $f,g,h_n: \Omega\rightarrow \F$ are measurable for $n\geq 1$.
  \begin{enumerate}[(1)]
    \item If $\alpha \in \F$, then $f + \alpha g$ is measurable.
    \item $\overline{f}$ is measurable.
    \item $fg$ is measurable.
    \item $\frac{f}{g}$ is measurable assuming it is well-defined.
    \item if $h_n$ are $\R$-valued, and $\left(h_n\left(x\right)\right)_n$ is bounded for each $x\in \Omega$, then $\sup h_n$ and $\inf h_n$ are measurable.
    \item If $f$ and $g$ are $\R$ valued, then $\max\left(f,g\right)$ and $\min\left(f,g\right)$ are measurable. In particular,
      \begin{align*}
        f_{+} &= \max\left(f,0\right)\\
        f_{-} &= \max\left(0,-f\right)
      \end{align*}
      are measurable.
    \item $\left\vert f \right\vert$ is measurable.
    \item The pointwise limit of measurable functions is measurable --- if $\lim_{n\rightarrow\infty}h_n\left(x\right)$ exists for all $x\in \Omega$, then $h = \lim_{n\rightarrow\infty}h_n$ is measurable.
  \end{enumerate}
\end{proposition}
\begin{definition}[Simple Functions]
  A simple function $s: \Omega\rightarrow \F$ is a function with finite range. In other words, $s$ is of the form
  \begin{align*}
    s &= \sum_{k=1}^{n}c_k\1_{E_k}
  \end{align*}
  for $E_k\subseteq \Omega$ and $c_k\in \F$.\newline

  A simple function is measurable if and only if $E_k\in \mathcal{M}$ for each $k$.
\end{definition}

\section{Constructing Measures}%
A measure assigns a nonnegative ``length'' or ``volume'' to measurable sets.
\begin{definition}[Basics of Measures]
  A measure on a measurable space $\left(\Omega,\mathcal{M}\right)$ is a map $\mu: \mathcal{M}\rightarrow \left[0,\infty\right]$ that satisfies the following.
  \begin{enumerate}[(i)]
    \item $\mu\left(\emptyset\right) = 0$;
    \item $\displaystyle \mu\left(\bigsqcup_{j=1}^{\infty}E_j\right) = \sum_{j=1}^{\infty}\mu\left(E_j\right)$.
  \end{enumerate}
  The triple $\left(\Omega,\mathcal{M},\mu\right)$ is called a measure space.\newline

  A measure $\mu$ is finite if $\mu\left(\Omega\right) < \infty$\newline

  If $\mu\left(\Omega\right) = 1$, then $\mu$ is called a probability measure.\newline

  A measure $\mu$ is called finitely additive if $\mu\left(E\sqcup F\right) = \mu(E) + \mu(F)$.\newline

  A measure $\mu$ is called $\sigma$-finite if there is a countable family $\set{E_n}_{n\geq 1}\subseteq \mathcal{M}$ such that
  \begin{align*}
    \Omega &= \bigcup_{n\geq 1}E_n
  \end{align*}
  and $\mu\left(E_n\right) < \infty$.\newline

  A measure $\mu$ on $\left(\Omega,\mathcal{M}\right)$ is called semi-finite if, for every $E\in \mathcal{M}$ with $\mu(E) = \infty$, there exists $F\in \mathcal{M}$ with $F\subseteq E$ and $0 < \mu(F) < \infty$.
\end{definition}

\begin{lemma}
  Let $\left(\Omega,\mathcal{M},\mu\right)$ be a measure space.
  \begin{enumerate}[(1)]
    \item If $E,F\in \mathcal{M}$ with $F\subseteq E$, then $\mu\left(F\right) \subseteq \mu\left(E\right)$.
    \item If $\left(E_n\right)_{n}$ is a sequence of measurable sets, then
      \begin{align*}
        \mu\left(\bigcup_{n\geq 1}E_n\right) &\leq \sum_{n=1}^{\infty}\mu\left(E_n\right).
      \end{align*}
    \item If $\left(E_n\right)_{n\geq 1}$ is an increasing family of measurable sets, then
      \begin{align*}
        \mu\left(\bigcup_{n\geq 1}E_n\right) &= \lim_{n\rightarrow\infty}\mu\left(E_n\right).
      \end{align*}
  \end{enumerate}
\end{lemma}
\begin{proof}\hfill
  \begin{enumerate}[(1)]
    \item Since $F\subseteq E$, we can write $E = F\sqcup \left(E\setminus F\right)$. Thus,
      \begin{align*}
        \mu\left(E\right) &= \mu\left(F\right) + \mu\left(E\setminus F\right)\\
                          &\geq \mu\left(F\right).
      \end{align*}
    \item We write
      \begin{align*}
        F_1 &= E_1\\
        F_2 &= E_2\setminus E_1\\
            &\vdots\\
        F_n &= E_n\setminus \left(\bigcup_{k=1}^{n-1}E_k\right).
      \end{align*}
      Since each $F_n$ is measurable, and $F_n\subseteq E_n$, we have
      \begin{align*}
        \mu\left(\bigcup_{n\geq 1}E_n\right) &= \mu\left(\bigsqcup_{n\geq 1}F_n\right)\\
                                             &= \sum_{n=1}^{\infty}F_n\\
                                             &\leq \sum_{n=1}^{\infty}\mu\left(E_n\right).
      \end{align*}
    \item We write $F_n$ as the respective disjoint union for $\set{E_n}_{n\geq 1}$. We have $\bigsqcup_{k=1}^{n}F_k = E_n$. Then,
      \begin{align*}
        \mu\left(\bigcup_{n\geq 1}E_n\right) &= \sum_{n=1}^{\infty}\mu\left(F_n\right)\\
                                             &= \lim_{n\rightarrow\infty}\left(\sum_{k=1}^{n}\mu\left(F_k\right)\right)\\
                                             &= \lim_{n\rightarrow\infty}\mu\left(\bigsqcup_{k=1}^{n}F_k\right)\\
                                             &= \lim_{n\rightarrow\infty}\mu\left(E_n\right).
      \end{align*}
  \end{enumerate}
\end{proof}
\begin{definition}[Counting Measure]
  If $\Omega$ is any set, the \textbf{counting measure} on $\left(\Omega,P\left(\Omega\right)\right)$ assigns $\left\vert A \right\vert$ for each $A\in P\left(\Omega\right)$ finite, and $\infty$ for any infinite subset.
\end{definition}
\begin{definition}[Restricting Measures]
  If $\left(\Omega,\mathcal{M},\mu\right)$ is a measure space, $\mathcal{B}$ is a $\sigma$-algebra on $\Omega$ with $\mathcal{B}\subseteq \mathcal{M}$, the restriction $\mu|_{\mathcal{B}}$ is a measure on $\left(\Omega,\mathcal{B}\right)$.\newline

  If $E\in \mathcal{M}$, we can restrict $\mu$ to $\mathcal{M}_{E}$ (the restriction of $\mathcal{M}$ to $E$), yielding the measure space $\left(E,\mathcal{M}_{E},\mu|_{\mathcal{M}_E}\right)$. We denote this restricted measure $\mu_{E}$, such that $\mu_{E}\left(M\cap E\right) = \mu\left(M\cap E\right)$ for all $M\in \mathcal{M}_{E}$.
\end{definition}
\begin{definition}[Pushforward Measure]
  Let $\left(\Omega,\mathcal{M},\mu\right)$ be a measure space, and let $\left(\Lambda,\mathcal{N}\right)$ be a measurable space. Let $f: \Omega\rightarrow \Lambda$ be measurable. The map
  \begin{align*}
    f_{\ast}\mu: \mathcal{N}\rightarrow [0,\infty]
  \end{align*}
  defined by
  \begin{align*}
    f_{\ast}\mu\left(E\right) &= \mu\left(f^{-1}\left(E\right)\right)
  \end{align*}
  defines a measure on $\left(\Lambda,\mathcal{N}\right)$. This is known as the pushforward measure of $\mu$.\newline

  If $\mathcal{N}$ on $\Lambda$ is produced by $f$, then the pushforward measure is necessarily defined on $\mathcal{N}$, and that any function $g: \Lambda\rightarrow \F$ is measurable if and only if $g\circ f$ is measurable.
\end{definition}
\begin{definition}[Disjoint Union]
  Let $\set{\left(\Omega_n,\mathcal{M}_n,\mu_n\right)}$ be a countable family of measure spaces.\newline

  We define the co-product of this family by taking
  \begin{align*}
    \Sigma := \bigsqcup_{n=1}^{\infty}\Omega_n,
  \end{align*}
  to be our set equipped with the canonical inclusion map $\iota_{n}\left(x\right) = \left(x,n\right)$, such that for each $n$,
  \begin{align*}
    \mathcal{M} &:= \set{E\subseteq \Sigma\mid \iota_{n}^{-1}\left(E\right)\in \mathcal{M}_n}.
  \end{align*}
  The measure is defined by
  \begin{align*}
    \mu: \mathcal{M}\rightarrow [0,\infty]\\
    \mu(E) := \sum_{n=1}^{\infty}\mu_{n}\left(\iota_{n}^{-1}\left(E\right)\right).
  \end{align*}
  We can identify each $\Omega_n$ with the subset $\Omega_{n}^{\ast} = \set{\left(x,n\right)\mid x\in \Omega_n}\subseteq \Sigma$, with $\iota_{n}^{-1}\left(E\right)\subseteq \Omega_{n}$ identified with $E\cap \Omega_{n}^{\ast}$.\newline

  The family $\set{\Omega_{n}^{\ast}}_{n\geq 1}$ forms a measurable partition of $\Sigma$, and that $\mu|_{\Omega_{n}^{\ast}}$ are the pushforwards of $\mu_{n}$ by $\iota_{n}$.\newline

  Note that a map $f: \Sigma\rightarrow \C$ is measurable if and only if $f\circ \iota_{n}: \Omega_n\rightarrow \C$ is measurable for all $n$.\newline

  If $\left(f_n: \Omega_n\rightarrow \C\right)_{n}$ is a sequence of measurable maps, the disjoint union
  \begin{align*}
    f = \bigsqcup_{n\geq 1}f_n: \Sigma\rightarrow \C
  \end{align*}
  defined by $f\left(x,n\right) = f_n(x)$, is measurable.
\end{definition}
