We start by introducing some categorical constructions, such as free groups, free vector spaces, free algebras, and the group algebra, as well as understanding their universal properties. We will be using the various properties of these categorical constructions to further understand amenability and its various characterizations.
\section{Free Groups}%
Given a set $A$, we want to know how exactly we can create a group structure from the elements in $A$ such that they extend from $A$ to a group generated by $A$ in a particularly ``natural'' way. This will be the free group, whose properties we will discuss in Chapter \ref{ch:paradoxical_decompositions}.
\begin{definition}\label{def:generating_sets}
  Let $G$ be a group, and $S\subseteq G$ be a subset. We define the subgroup {generated by} $S$ to be
  \begin{align*}
    \left\langle S \right\rangle_{G} &= \bigcap \set{H | S\subseteq H,~H\text{ a subgroup}}.
  \end{align*}
  We say $S$ generates $G$ if $\left\langle S \right\rangle_{G} = G$.\newline

  Generated subgroups can be broadly characterized as follows:
  \begin{align*}
    \left\langle S \right\rangle_{G} &= \set{s_1^{a_1}s_2^{a_2}\cdots s_n^{a_n} | n\in\N,~s_1,\dots,s_n\in S,~a_1,\dots,a_n\in \set{-1,1}}.
  \end{align*}
  We say $\left\langle S \right\rangle_{G}$ is finitely generated if $\Card(S) < \infty$.\newline

  If $S$ is such that, for any $x\in S$, we have $x^{-1}\in S$, then we say $S$ is symmetric.
\end{definition}
To construct a free group, we begin by stating its universal property --- that is, its innate nature as an ``extension'' of a set-function into a group structure. Then, we will show that a more constructive definition of the free group satisfies this universal property. The following section draws heavily from \cite{loh_geometric_group_theory}, but we will mostly focus on the construction of the free group rather than the proof of uniqueness.
\begin{definition}\label{def:free_group}
  Let $S$ be a set. A group $F$ containing $S$ is said to be freely generated if, for every group $G$, and every map $\phi\colon S\rightarrow G$, there is a unique group homomorphism $\varphi\colon F\rightarrow G$ that extends $\varphi$. The following diagram, where $\iota$ denotes the inclusion of $S$ into $F$, commutes:
  \begin{center}
\begin{tikzcd}
S \arrow[d, "\iota"', hook] \arrow[r, "\phi"] & G \\
F \arrow[ru, "\varphi"']                      &  
\end{tikzcd}
  \end{center}
We say $F$ is the {free group} generated by $S$.
\end{definition}
Intuitively, to construct the free group, if we have $a\mapsto \phi(a)$ between $S$ and $G$, then we will define $\varphi\left(a^n\right) = \phi(a)^n$ inside $F(S)$. Uniqueness will follow from the fact that we can take two groups that satisfy the universal property, $F$ and $F'$, and apply the universal property on set-valued functions between $S$ and $F$ and $S$ and $F'$ respectively. The formal proof of the construction of the free group is as follows.
\begin{proof}
  We will construct a group consisting of ``words'' made up of elements of $S$ and their inverses. This starts by considering the alphabet $A = S\cup \widehat{S}$, where $\widehat{S}$ is a disjoint copy of $S$ --- every $\hat{s}\in \widehat{S}$ will play the role of an inverse to $s$ in our group.
  \begin{itemize}
    \item Define $A^{\ast}$ to be the set of all words over the alphabet $A$, including the empty word, $\epsilon$. We define the operation $A^{\ast}\times A^{\ast}\rightarrow A^{\ast}$ by concatenating words, which is an associative operation with neutral element $\epsilon$.
    \item Define the equivalence relation $\sim$ generated by the following two relations, where for all $x,y\in A^{\ast}$ and $s\in S$, we have
      \begin{align*}
        xs\hat{s}y &\sim xy\\
        x\hat{s}sy &\sim xy.
      \end{align*}
      The equivalence classes with respect to $\sim$ will be denoted $\left[\cdot\right]$.\newline

      We have a well-defined composition $\left[x\right]\left[y\right] = \left[xy\right]$ mapping $F(S) \times F(S) \rightarrow F(S)$ for all $x,y\in A^{\ast}$.
  \end{itemize}
  We show that $F(S)$ with the concatenation operation is a group. Here, we see that $\left[\epsilon\right]$ is the neutral element for the composition, and associativity is inherited from associativity of concatenation in $A^{\ast}$. To show the existence of inverses, we define the inverse map inductively by taking $I\left(\epsilon\right) = \epsilon$, and
  \begin{align*}
    I\left(sx\right) &= I(x)\hat{s}\\
    I\left(\hat{s}x\right) &= I(x)s
  \end{align*}
  for all $x\in A^{\ast}$  and $s\in S$. Inductively, we can see that $I\left(I\left(x\right)\right) = x$ and
  \begin{align*}
    \left[I(x)\right]\left[x\right] &= \left[I(x)x\right]\\
                                    &= \left[\epsilon\right]\\
    \left[x\right]\left[I(x)\right] &= \left[xI(x)\right]\\
                                    &= \left[\epsilon\right].
  \end{align*}
  Thus, $F(S)$ is a group.\newline

  Now, we show $F(S)$ is freely generated. Let $i\colon S\rightarrow F(S)$ be the map that sends $s\mapsto \left[s\right]$. By our construction, we know that $i(S)\subseteq F(S)$ is a generating set for $F(S)$. We will show the universal property holds for $F(S)$.\newline

  To start, let $\phi\colon S\rightarrow G$ be a set-valued map between $S$ and an arbitrary group $G$. We construct $\phi^{\ast}\colon A^{\ast}\rightarrow G$ by taking
  \begin{align*}
    \epsilon &\mapsto e\\
    sx &\mapsto \phi(s)\phi^{\ast}\left(x\right)\\
    \hat{s}x &\mapsto \left(\phi(s)\right)^{-1}\phi^{\ast}\left(x\right)
  \end{align*}
  for all $x\in A^{\ast}$ and $s\in S$. This definition of $\phi^{\ast}$ is compatible with the equivalence relation on $A^{\ast}$, and we see that $\phi^{\ast}\left(xy\right) = \phi^{\ast}\left(x\right)\phi^{\ast}\left(y\right)$. Thus, we get a well-defined map $\varphi\colon F(S)\rightarrow G$, taking $\left[x\right]\mapsto \left[\phi^{\ast}\left(x\right)\right]$.\newline

  It remains to be shown that the map $i\colon S\rightarrow F(S)$ is injective, which will show that $F(S)$ is freely generated by $S$. Let $s_1,s_2\in S$, and consider the set-function $\phi\colon S\rightarrow \Z$ given by $\phi\left(s_1\right) = 1$ and $\phi\left(s_2\right) = -1$. Then, we must have
  \begin{align*}
    \varphi\left(i\left(s_1\right)\right) &= \phi\left(s_1\right)\\
                                          &= 1\\
                                          &\neq -1\\
                                          &= \phi\left(s_2\right)\\
                                          &= \varphi\left(i\left(s_2\right)\right).
  \end{align*}
  Thus, we have $i\left(s_1\right)\neq i\left(s_2\right)$, so $i$ is injective.
\end{proof}
\section{Free Vector Spaces}%
Given a set $A$, just as we are able to construct a free group, we can take any set $A$ and construct a ``universal'' vector space out of the set. The free vector space (as it is known) is the universal object that extends any set-valued function into a linear map, treating elements of the set as its basis (see Definition \ref{def:basis}). We are interested in the case of the free vector space over the complex numbers, but note that the following definition of the free vector space applies over any field. 
\begin{theorem}
  Let $\Gamma$ be a nonempty set. There is a vector space, $\C\left[\Gamma\right]$, with $\Dim\left(\C\left[\Gamma\right]\right) = \Card\left(\Gamma\right)$, and an injective map $\delta\colon \Gamma\rightarrow \C\left[\Gamma\right]$ such that the following universal property holds: if $V$ is a $\C$-vector space, and $\phi\colon \Gamma\rightarrow V$ is a set-valued function, then there is a unique linear map $T_{\phi}\colon \C\left[\Gamma\right]\rightarrow V$ such that $T_{\phi}\circ \delta = \phi$.
  \begin{center}
    % https://tikzcd.yichuanshen.de/#N4Igdg9gJgpgziAXAbVABwnAlgFyxMJZABgBpiBdUkANwEMAbAVxiRAB12BxOgW17ogAvqXSZc+QigCM5KrUYs2nAMLJOPfnQrDRIDNjwEis6fPrNWiEADVh8mFADm8IqABmAJwi8kZEDgQSLIKlsrssAw4giIe3r6IIYFIAEzUDHQARjAMAAriRlIgDDDuOCDUFkrWACoA+sCcaAAWWEK6cT5+1MmIacVZOfmGkmyeWE7N5ZWKVhzsLVj2QkA
\begin{tikzcd}
\Gamma \arrow[r, "\delta"] \arrow[rd, "\phi"'] & {\C[\Gamma]} \arrow[d, "T_{\phi}"] \\
                                               & V                                 
\end{tikzcd}
  \end{center}
\end{theorem}
\begin{proof}
  Consider the linear subspace of finitely supported functions, $\C\left[\Gamma\right]\subseteq \mathcal{F}\left(\Gamma,\C\right)$. For each $t\in \Gamma$, we define
  \begin{align*}
    \delta_t\left(s\right) &= \begin{cases}
      1 & s=t\\
      0 & \text{else}
    \end{cases}.
  \end{align*}
    We see that $\delta_t\neq \delta_s$ whenever $s\neq t$, meaning that the map $\delta\colon \Gamma\rightarrow \C\left[\Gamma\right]$, defined by $s \mapsto \delta_s$, is injective.\newline

    We will show that $\set{\delta_s}_{s\in \Gamma}$ is a linear basis for $\C\left[\Gamma\right]$. If $f\in \F\left[\Gamma\right]$, with $\supp\left(f\right) = \set{s_1,\dots,s_n}\subseteq \Gamma$, we set $\alpha_j = f\left(t_j\right)$, and see that
    \begin{align*}
      f &= \sum_{j=1}^{n}\alpha_j\delta_{s_j},
    \end{align*}
    which shows that $\set{\delta_s}_{s\in\Gamma}$ is a spanning set.\newline

    To show that $\set{\delta_s}_{s\in\Gamma}$ is linearly independent, consider $g = \sum_{j=1}^{n}\alpha_j\delta_{s_j}\in \F\left[\Gamma\right]$ such that $g = 0$. Then, $g(t) = 0$ for all $t\in\Gamma$, and in particular, $g\left(s_i\right) = 0$ for every $1 \leq i \leq n$. Thus, we have
    \begin{align*}
      0 &= g\left(s_j\right)\\
        &= \sum_{j=1}^{n}\alpha_j\delta_{s_j}\left(s_i\right)\\
        &= \alpha_i,
    \end{align*}
    so $\alpha_j = 0$ for each $j$. Thus, $\set{\delta_s}_{s\in \Gamma}$ is linearly independent.\newline

    Turning to the universal property, we define $T_{\phi}\colon \F\left[\Gamma\right]\rightarrow V$ in terms of $\phi$ as follows:
    \begin{align*}
      T_{\phi}\left(\sum_{j=1}^{n}\alpha_j\delta_{s_j}\right) &= \sum_{j=1}^{n}\alpha_j\phi\left(s_j\right).
    \end{align*}
    This yields an expression of $T_{\phi}$ uniquely in terms of $\phi$ and $\delta$, thereby satisfying the universal property.
\end{proof}
\begin{example}
  Let $z$ be an abstract variable, and consider the set of ``formal powers'' of $z$, $\set{z^k}_{k\in\N}$. Then, the free vector space generated by this set, $\C\left[z\right]$, is the set of all polynomials with coefficients in $\C$. By the universal property, we know that every polynomial $p\in \C\left[z\right]$ has a unique expression $p = \sum_{j=0}^{n}a_jz^j$.
\end{example}
\section{Free Algebras}%
Chapter 8 of this thesis will be focused on understanding the properties of the (reduced) group $C^{\ast}$-algebra. This will require some background in the theory of algebras, so we will understand the purely algebraic properties here before diving into the analytic properties in Chapter 5.
\begin{definition}\label{def:star_algebra}
  Let $A$ be an algebra over $\C$ (see Definition \ref{def:vector_space_and_algebra}). An involution on $A$ is a unary operation $\ast\colon A\rightarrow A$ that satisfies the following, for all $a,b\in A$ and $\alpha\in\C$:
  \begin{itemize}
    \item $\left(a^{\ast}\right)^{\ast} = a$;
    \item $\left(a+b\right)^{\ast} = a^{\ast} + b^{\ast}$;
    \item $\left(\alpha a\right)^{\ast} = \overline{\alpha} a^{\ast}$;
    \item $\left(ab\right)^{\ast} = b^{\ast}a^{\ast}$.
  \end{itemize}
  If $A$ is equipped with an involution, we say $A$ is a $\ast$-algebra.
\end{definition}
\begin{example}
  Consider the algebra of $n\times n$ matrices over $\C$, $\Mat_n\left(\C\right)$, with element-wise addition and scalar multiplication, as well as traditional matrix multiplication. From linear algebra, we know that if $T\in \Mat_n\left(\C\right)$, then $T$ admits a unique adjoint map (or conjugate transpose), $T^{\ast}$, such that for any $x,y\in \C^n$, $ \iprod{T\left(x\right)}{y} =  \iprod{x}{T^{\ast}\left(y\right)}$, where $ \iprod{\cdot}{\cdot} $ denotes the complex inner product. The map $\ast\colon \Mat_n\left(\C\right)\rightarrow \Mat_n\left(\C\right)$, given by $T\mapsto T^{\ast}$, satisfies the definition of an involution.\newline

  If $x,y\in \C^n$, $\alpha\in\C$, and $T,S\in \Mat_n\left(\C\right)$,
  \begin{align*}
    \iprod{T\left(x\right)}{y} &= \iprod{x}{T^{\ast}\left(y\right)}\\
                               &= \overline{ \iprod{T^{\ast}\left(y\right)}{x} }\\
                               &= \overline{ \iprod{y}{T^{\ast\ast}\left(x\right)} }\\
                               &= \iprod{T^{\ast\ast}\left(x\right)}{y}\\
                               \\
    \iprod{\left(T+S\right)\left(x\right)}{y} &= \iprod{T\left(x\right)}{y} + \iprod{S\left(x\right)}{y}\\
                                              &= \iprod{x}{T^{\ast}\left(y\right)} + \iprod{x}{S^{\ast}\left(y\right)}\\
                                              &= \iprod{x}{\left(T^{\ast} + S^{\ast}\right)\left(y\right)}\\
                                              \\
    \iprod{\alpha T\left(x\right)}{y} &= \iprod{T\left(\alpha x\right)}{y}\\
                                      &= \iprod{\alpha x}{T^{\ast}\left(y\right)}\\
                                      &= \iprod{x}{\overline{\alpha}T^{\ast}\left(y\right)}\\
                                      \\
    \iprod{TS\left(x\right)}{y} &= \iprod{S\left(x\right)}{T^{\ast}\left(y\right)}\\
                                &= \iprod{x}{S^{\ast}T^{\ast}\left(y\right)}.
  \end{align*}
  Thus, we can see that $\Mat_n\left(\C\right)$ is a $\ast$-algebra.
\end{example}
\begin{definition}
  Let $A$ be a $\ast$-algebra, and let $B\subseteq A$.
  \begin{itemize}
    \item We say $B$ is self-adjoint (or $\ast$-closed) if for any $x\in B$, $x^{\ast}\in B$.
    \item We say $B$ is a subalgebra of $A$ if $B\subseteq A$ is a linear subspace and for any $b_1,b_2\in B$, $b_1b_2\in B$. If $B$ is $\ast$-closed, then we say $B$ is a $\ast$-subalgebra.
    \item If $B$ is a subalgebra such that, for any $b\in B$ and $a\in A$, $ab\in B$ and $ba\in B$, then $B$ is an ideal. If $B$ is $\ast$-closed, then we say $B$ is a $\ast$-ideal.
  \end{itemize}
  If $J\subseteq A$ is a $\ast$-ideal, the linear space $A/J$ admits multiplication and involution, defined by
  \begin{align*}
    \left(a+J\right)\left(b+J\right) &= ab + J\\
    \left(a+J\right)^{\ast} &= a^{\ast}+J.
  \end{align*}
  For any $a\in A$, the $\ast$-ideal generated by $a$ is denoted
  \begin{align*}
    \operatorname{ideal}\left(a\right) &= \bigcap\set{J\subseteq A | a\in J\text{ and }J\text{ is a $\ast$-ideal} }.
  \end{align*}
  We say an ideal $J\subseteq A$ is maximal if $J$ is a proper ideal in $A$ and, if $J\subseteq I$ for some ideal $I$, then $I = J$ or $I = A$.
\end{definition}
Just as there are free groups and free vector spaces, we can also talk about free algebras. In Chapter 8, we will construct special norms on free algebras to elucidate properties of the underlying group.\newline

Similar to a free group, the free algebra (or free $\ast$-algebra) is constructed by taking a certain collection of ``words'' over a set of symbols, and then, if desired, ``modding out'' by the ideal generated by a set of relations. We formalize this in steps.
\begin{definition}
  Let $E=\set{x_i}_{i\in I}$ be a collection of symbols that may not commute. The space of all polynomials over $E$ is the free vector space over the set of words formed by symbols in $E$,
  \begin{align*}
    \Gamma_E &= \set{x_{i_1}x_{i_2}\cdots x_{i_n} | n\in\N,i_1,\dots,i_n\in I}.
  \end{align*}
  We denote this space $\C \left\langle E \right\rangle$.\newline

  In the free vector space $\C \left\langle E \right\rangle$, we may define multiplication by concatenation:
  \begin{align*}
    \left(x_{i_1}x_{i_2}\cdots x_{i_n}\right)\left(x_{j_1}x_{j_2}\cdots x_{j_m}\right) &= x_{i_1}x_{i_2}\cdots x_{i_n}x_{j_1}x_{j_2}\cdots x_{j_m},
  \end{align*}
  where $i_1,\dots,i_n,j_1,\dots,j_m\in I$. The space $\C\left\langle E \right\rangle$, equipped with multiplication by concatenation, is known as the free algebra on $E$.\newline

  To turn $\C\left\langle E \right\rangle$ into a $\ast$-algebra, we define the formal set $E^{\ast} = \set{x_{i}^{\ast}}_{i\in I}$, and define the involution on $\C\left\langle E\cup E^{\ast} \right\rangle$ by taking
  \begin{align*}
    \left(\alpha x_{i_1}^{\ve_1}x_{i_2}^{\ve_2}\cdots x_{i_n}^{\ve_n}\right)^{\ast} &= \overline{\alpha}x_{i_n}^{\delta_n}x_{i_{n-1}}^{\delta_{n-1}}\cdots x_{i_2}^{\delta_2}x_{i_1}^{\delta_1},
  \end{align*}
  where
  \begin{align*}
    \delta_j &= \begin{cases}
      \ast & \ve_j = 1\\
      1 & \ve_j = \ast
    \end{cases}.
  \end{align*}
  The set $\C\left\langle E\cup E^{\ast} \right\rangle$ with the involution defined above is known as the free $\ast$-algebra on $E$, and is usually denoted $\mathbb{A}^{\ast}\left(E\right)$.\newline

  If $R\subseteq \mathbb{A}^{\ast}\left(E\right)$ is a collection of relations, we let $I(R) = \operatorname{ideal}\left(R\right)$. Then, the quotient algebra
  \begin{align*}
    \mathbb{A}^{\ast}\left(E|R\right) &= \mathbb{A}^{\ast}\left(E\right)/I(R)
  \end{align*}
  is known as the universal $\ast$-algebra on $E$ with relations $R$.
\end{definition}
One of the most important $\ast$-algebras we will study is generated from a group by taking the free vector space over the group.
\begin{definition}
  Let $\Gamma$ be a group with identity element $e$, and let $\C\left[\Gamma\right]$ be the free vector space generated by $\Gamma$. We define a multiplication $f \ast g$, where $f,g\in \C\left[\Gamma\right]$ are finitely supported functions, by convolution:
  \begin{align*}
    f\ast g(s) &= \sum_{t\in\Gamma}f(t)g\left(t^{-1}s\right)\\
               &= \sum_{r\in\Gamma}f\left(sr^{-1}\right)g\left(r\right).
  \end{align*}
  The involution on $\C\left[\Gamma\right]$ is defined by $f^{\ast}\left(t\right) = \overline{f\left(t^{-1}\right)}$. The multiplicative identity is $\delta_e$, and multiplication satisfies $\delta_s\ast \delta_t = \delta_{st}$.
\end{definition}
