\section{Normed Vector Spaces and Algebras}%
The fundamental unit of functional analysis is functions --- specifically, collections of functions equipped with particular operations and a norm that turn them into vector spaces and algebras. This section will focus on some of the basic facts and theory surrounding normed vector spaces and algebras.
\begin{definition}\label{def:vector_space_and_algebra}
  Let $\F = \R$ or $\C$ be a field. An $\F$-vector space $X$ is a nonempty set equipped with two operations:
  \begin{align*}
    a\colon X\times X &\rightarrow X\\
    m\colon \F\times X \rightarrow X,
  \end{align*}
  given by $\left(x,y\right) \xmapsto{a} x+y$ and $\left(\alpha,y\right) \xmapsto{m} \alpha y$, satisfying the following properties:
  \begin{enumerate}[(i)]
    \item for all $x,y,z\in X$, $\left(x+y\right)+z = x + \left(y+z\right)$;
    \item there exists $0_X\in X$ such that for all $x\in X$, $x + 0_X = 0_X + x = x$;
    \item for all $x\in X$, there exists $y\in X$ such that $x + y = y + x = 0_X$;
    \item for all $x,y\in X$, $x+ y = y + x$;
    \item for all $\alpha,\beta \in \F$ and all $x,y\in X$, $\alpha\left(x+y\right) = \alpha x + \alpha y$ and $\left(\alpha + \beta\right)x = \alpha x + \beta x$;
    \item for all $\alpha,\beta \in \F$ and for all $x\in X$, $\alpha\left(\beta x\right) = \left(\alpha\beta\right)x$;
    \item for all $x\in X$, $\left(1_{\F}\right)x = x $.
  \end{enumerate}
  We call $a$ vector addition and $m$ scalar multiplication.\newline

  If $X$ is also equipped with a structure to allow multiplication between vectors,
  \begin{align*}
    X\times X &\rightarrow X\\
    \left(x,y\right) &\mapsto xy,
  \end{align*}
  then we say $X$ is an $\F$-algebra. The multiplication structure satisfies the following properties:
  \begin{itemize}
    \item for all $a,b,c\in X$, $a\left(bc\right) = \left(ab\right)c$;
    \item for all $a,b,c\in X$, $a\left(b+c\right) = ab + ac$ and $\left(a+b\right)c = ac + bc$;
    \item for all $\alpha\in \F$ and $a,b\in X$, $\alpha \left(ab\right) = \left(\alpha a\right)b = a\left(\alpha b\right)$.
  \end{itemize}
  If $X$ admits a (unique) $1_X\in X$ such that $a1_X = 1_X a = a$, then we say $X$ is a unital algebra. If multiplication is commutative, then we say $X$ is a commutative algebra.
\end{definition}
Vector spaces can also be equipped with notions of length and distance --- thus yielding a normed vector space.
\begin{definition}[Seminorms and Norms]\label{def:norms}
  Let $X$ be a $\F$-vector space, and let $p\colon X\rightarrow [0,\infty)$ be a function. If
  \begin{itemize}
    \item $p\left(\lambda x\right) = \left\vert \lambda \right\vert p(x)$ for all $\lambda\in \F$ and $x\in X$ (homogeneity), and
    \item $p\left(x + y\right)\leq p\left(x\right) + p\left(y\right)$ for all $x,y\in X$ (triangle inequality),
  \end{itemize}
  we say $p$ is a seminorm. If $p$ also satisfies
  \begin{itemize}
    \item $p\left(x\right) = 0 $ if and only if $x = 0$ (positive definite),
  \end{itemize}
  then $p$ is a norm. Norms on vector spaces are usually denoted $\norm{\cdot}$. Additionally, if $X$ is an algebra, the (semi)norm also has to be sub-multiplicative --- i.e.,
  \begin{align*}
    \norm{xy} &\leq \norm{x}\norm{y}.
  \end{align*}
  Two norms, $\norm{\cdot}_a$ and $\norm{\cdot}_b$, are said to be equivalent if there exist constants $C_1$ and $C_2$ such that
  \begin{align*}
    \norm{x}_a &\leq C_1\norm{x}_b\\
    \norm{x}_b &\leq C_2 \norm{x}_a
  \end{align*}
  for all $x\in X$.\newline

  If $X$ is complete with respect to the metric $d\left(x,y\right) = \norm{x-y}$, we call $X$ a Banach space.
\end{definition}
\begin{theorem}
  Let $X$ be a normed vector space. Then, $X$ is complete if and only if, for every sequence of vectors $\left(x_k\right)_k$, if$\sum_{k=1}^{\infty}\norm{x_k}$ converges, then $\sum_{k=1}^{\infty}x_k$ converges.
\end{theorem}
\begin{definition}[Open Balls, Closed Balls, Spheres]\label{def:open_closed_balls}
  Let $X$ be a normed vector space.
  \begin{itemize}
    \item We write 
      \begin{align*}
        U\left(x,\delta\right) &= \set{y\in X | \norm{y-x} < \delta}
      \end{align*}
      to be the open ball of radius $\delta$ centered at $x$. We write $U_X = U\left(0,1\right)$.
    \item We write
      \begin{align*}
        B\left(x,\delta\right) &= \set{y\in X | \norm{y-x} \leq \delta}
      \end{align*}
      to be the closed ball of radius $\delta$ centered at $x$. We write $B_X = B\left(0,1\right)$.
    \item We write
      \begin{align*}
        S\left(x,\delta\right) &= \set{y\in X | \norm{y-x} = \delta}
      \end{align*}
      to be the sphere of radius $\delta$ centered at $x$. We write $S_X = S\left(0,1\right)$.
  \end{itemize}
\end{definition}
\begin{definition}\label{def:basis}
  Let $X$ be a normed vector space.
  \begin{itemize}
    \item A basis for $X$ is a set $\set{x_i}_{i\in I}$ such that for any $x\in X$, there is a unique finite sum
      \begin{align*}
        x &= \sum_{i\in I}\alpha_i x_i.
      \end{align*}
    \item If $A\subseteq X$ is a set of vectors, then the span of $A$, denoted $\Span\left(A\right)$, is the set of all finite linear combinations of elements in $A$.
    \item If $A\subseteq X$ is such that $\overline{\Span}\left(A\right) = X$, then we say $A$ is a total subset of $X$.
  \end{itemize}
\end{definition}
\begin{definition}
  Let $T\colon X\rightarrow Y$ be a linear map between normed vector spaces. We say $T$ is bounded if its operator norm, defined by
  \begin{align*}
    \norm{T}_{\op} &= \sup_{x\in B_X}\norm{T\left(x\right)}
  \end{align*}
  is finite. We write $\B\left(X,Y\right)$ for the set of all bounded linear maps between $X$ and $Y$.
\end{definition}
\begin{remark}
  Note that if $Y$ is complete, then $\B\left(X,Y\right)$ is a Banach space with pointwise addition and scalar multiplication.\newline

  A quick sketch of the proof is as follows: consider a $\norm{\cdot}_{\op}$-Cauchy sequence $\left(T_n\right)_n$ in $\B\left(X,Y\right)$, and define $T$ to be the pointwise limit of $\left(T_n\right)_n$, which exists as for any $y\in Y$, $\left(T_n\left(y\right)\right)_{y}$ is Cauchy in $Y$. Then, it can be shown that defining $T$ in this manner yields convergence in operator norm.\newline

  Furthermore, if we define $\B(X) = \B(X,X)$, then this space is a normed algebra with pointwise addition, scalar multiplication, and composition of operators. The algebra $\B(X)$ is complete if $X$ is complete.
\end{remark}
\begin{fact}
  The following are equivalent for a linear map $T$:
  \begin{itemize}
    \item $T$ is continuous at $0$;
    \item $T$ is continuous;
    \item $T$ is uniformly continuous;
    \item $T$ is Lipschitz continuous;
    \item $T$ is bounded.
  \end{itemize}
\end{fact}
\begin{definition}
  Let $T\colon X\rightarrow Y$ be a linear map between normed vector spaces.
  \begin{itemize}
    \item We say $T$ is bounded below if there exists $C > 0$ such that $\norm{T(x)}\geq C\norm{x}$ for all $x\in X$.
    \item If $T$ is bounded and bounded below, we say $T$ is bicontinuous.
    \item If $T$ is a linear isomorphism that is bicontinuous, we say $T$ is a bicontinuous isomorphism, and say $X\cong Y$ are bicontinuously isomorphic.
    \item If $T$ is a linear isomorphism and is such that $\norm{T(x)} = \norm{x}$ for all $x$, then we say $T$ is an isometric isomorphism.
  \end{itemize}
\end{definition}
\begin{definition}
  Let $X$ be a normed vector space. The algebraic dual of $X$, denoted $X'$, is the set of all linear functionals on $X$:
  \begin{align*}
    X' &= \mathcal{L}\left(X,\F\right).
  \end{align*}
  The subset $X^{\ast}\subseteq X'$ is the set of all \textit{continuous} linear functionals on $X$:
  \begin{align*}
    X^{\ast} &= \B\left(X,\F\right).
  \end{align*}
\end{definition}

\section{The Fundamental Theorems of Banach Spaces}%
\begin{definition}
  Let $X$ be a topological space.
  \begin{itemize}
    \item A subset $F\subseteq X$ is called nowhere dense if $\left(\overline{F}\right)^{\circ} = \emptyset$.
    \item If $X$ is a countable union of nowhere dense sets, we say $X$ is meager (or of the first category).
    \item We say $X$ is a Baire space if, for any countable collection of open, dense subsets $\set{A_n}_{n\geq 1}$, their intersection $\bigcap_{n\geq 1}A_n\subseteq X$ is also dense.
  \end{itemize}
\end{definition}
The Baire category theorem serves as one of the central bridges between functional analysis and topology. Note that the property of being a Baire space is a purely topological definition, while completeness is an analytic concept.
\begin{theorem}[Baire Category Theorem]\label{thm:baire}
  Let $X$ be a complete metric space. Then, $X$ is a Baire space.
\end{theorem}
The Baire category theorem is used to prove many important theorems in functional analysis, such as the ones that follow. They all fundamentally rely on the completeness of Banach spaces, which is expressed through the Baire category theorem. Proofs for these theorems can be found in functional analysis textbooks such as \cite{rudin_functional_analysis}.
\begin{theorem}[Open Mapping Theorem]\label{thm:open_mapping}
  Let $T\colon X\rightarrow Y$ be a surjective linear map between Banach spaces. Then, $T$ is an open map --- i.e., if $U\subseteq X$ is open, then $V$ is also open.
\end{theorem}
\begin{corollary}[Bounded Inverse]\label{cor:bounded_inverse}
  If $T\colon X\rightarrow Y$ is a bounded linear map that is bijective, then $T^{-1}$ is also bounded.
\end{corollary}
\begin{theorem}[Closed Graph Theorem]\label{thm:closed_graph}
  Let $T\colon X\rightarrow Y$ be a linear map between Banach spaces. Then, $T$ is bounded if and only if $\graph\left(T\right) = \set{\left(x,T(x)\right)| x\in X} \subseteq X\times Y$ is closed in the product topology.
\end{theorem}
\begin{theorem}[Uniform Boundedness Principle]\label{thm:uniform_boundedness}
  Let $\set{T_i}_{i\in I}$ be a family of maps in $\B\left(X,Y\right)$ such that, for all $x\in X$, $\sup_{i\in I}\norm{T_i(x)} < \infty$. Then, $\sup_{i\in I}\norm{T_i}_{\op} < \infty$.
\end{theorem}
Now, we turn our attention towards linear functionals. Consider the following problem in algebra: suppose $X$ is a finite-dimensional vector space, and $Y\subseteq X$ is a subspace. If we have a linear functional $\varphi\colon Y\rightarrow \F$, can this linear functional be extended to the full space?\newline

The answer is yes --- if $\mathcal{B} = \set{x_1,\dots,x_m}$ is a basis for $Y$, a fundamental result in linear algebra states that this basis can be extended to a basis for $X$, $\mathcal{C} = \set{x_1,\dots,x_m,x_{m+1},\dots,x_n}$; if we define $c_i = \varphi\left(x_i\right)$ for $1 \leq i \leq m$, we may define $\varphi\left(x_i\right) = 0$ for $m+1 \leq i \leq n$. In other words, we may always extend elements of the algebraic dual from a subspace to the full space.\newline

However, if $X$ is infinite-dimensional and equipped with a norm (or, more generally, a locally convex topology, see [write definition in LCTVS section]), we also care about continuity, norm, and whether these extensions preserve continuity and norm. This is the domain of the Hahn--Banach theorems, which establish extension and separation results in normed vector spaces (and, as detailed later, locally convex topological vector spaces, see [Hahn--Banach theorems for LCTVS]).
\begin{definition}[Minkowski Functional]
  We call a map $p\colon X\rightarrow [0,\infty)$ a Minkowski functional if
  \begin{itemize}
    \item $p\left(x + y\right)\leq p\left(x\right) + p\left(y\right)$ for all $x,y\in X$, and;
    \item $p\left(tx\right) = tp\left(x\right)$ for all $x\in X$ and $t > 0$.
  \end{itemize}
\end{definition}
\begin{theorem}[Hahn--Banach--Minkowski Extension]
  Let $Y\subseteq X$ be a linear subspace of a normed vector space $X$, and let $\phi\in Y^{\ast}$ and $p$ a Minkowski functional be such that for all $y\in Y$, $\phi\left(y\right) \leq p\left(y\right)$. Then, there is a map $\Phi\in X^{\ast}$ such that
  \begin{itemize}
    \item $\Phi|_{Y} = \phi$, and;
    \item $\Phi\left(x\right) \leq p\left(x\right)$ for all $x\in X$.
  \end{itemize}
\end{theorem}
\begin{theorem}[Hahn--Banach Continuous Extension]
  Let $X$ be a normed vector space, and $\phi\in Y^{\ast}$, where $Y\subseteq X$ is a linear subspace. Then, there exists a linear functional $\Phi\in X^{\ast}$ such that $\norm{\Phi} = \norm{\phi}$, and $\Phi|_{Y} = \phi$.\footnote{This extension is not necessarily unique.}
\end{theorem}
\begin{theorem}[Hahn--Banach Separation]
  Let $X$ be a normed vector space.
  \begin{itemize}
    \item For a fixed $x_0\in X$, there exists a linear functional $\phi\in X^{\ast}$ such that $\phi\left(x_0\right) = \norm{x_0}$.
    \item For a proper closed subspace $Y\subseteq X$ and some fixed $x_0\in X\setminus Y$, there is a $\phi\in X^{\ast}$ such that $\norm{\phi} \leq 1$, $\phi|_{Y} = 0$, and $\phi\left(x_0\right) = \dist_{Y}\left(x_0\right)$.
  \end{itemize}
\end{theorem}
\begin{corollary}
  Let $X$ be a normed space. For every $x\in X$, we have
  \begin{align*}
    \sup_{\phi\in B_{X^{\ast}}}\left\vert \varphi\left(x\right)  \right\vert = \norm{x}.
  \end{align*}
\end{corollary}
\section{Duality}%
Here, we discuss a little bit more of the theory of dual spaces.
\begin{definition}
  Let $X$ be a normed vector space. The double dual of $X$, $X^{\ast\ast}$, is the set of all linear functionals
  \begin{align*}
    \hat{x}\colon X^{\ast\ast}\rightarrow \C,
  \end{align*}
  where $\hat{x}$ is such that $\hat{x}\left(\varphi\right) = \varphi(x)$ for each $\varphi\in X^{\ast}$. We define the canonical isometric embedding $\iota_X\colon X\hookrightarrow X^{\ast\ast}$ by $x\mapsto \hat{x}$.
\end{definition}
\begin{definition}
  Let $X$ be a normed space. A norm completion of $X$ is a pair $\left(Z,j\right)$, where $Z$ is a Banach space, $j\colon X\hookrightarrow Z$ is a linear isometry, and $\overline{\Ran}\left(j\right) = Z$.
\end{definition}
\begin{proposition}
  Let $X$ be a normed space, and set $\widetilde{X} = \overline{\iota_X\left(X\right)}^{\norm{\cdot}_{\op}}\subseteq X^{\ast\ast}$. Then, $\left(\widetilde{X},\iota_X\right)$ is a norm completion of $X$. Additionally, if $\left(Z,j\right)$ is any other norm completion of $X$, then there is an isometric isomorphism $Z\rightarrow \widetilde{X}$.
\end{proposition}
\begin{proposition}
  Let $X$ and $Y$ be normed spaces, and let $T\in \B\left(X,Y\right)$. Then, there is a unique $\widetilde{T}\in B\left(\widetilde{X},\widetilde{Y}\right)$ such that $\widetilde{T}\circ \iota_X = \iota_Y\circ T$. The diagram below commutes.
  \begin{center}
    % https://tikzcd.yichuanshen.de/#N4Igdg9gJgpgziAXAbVABwnAlgFyxMJZABgBoBGAXVJADcBDAGwFcYkQANEAX1PU1z5CKchWp0mrdgE0efEBmx4CRMsXEMWbRCAA6ugO5ZYeRrGAduc-kqFFR6mpqk79Rk1jMxg0q93EwUADm8ESgAGYAThAAtkgATDQ4EEgAzE6S2nqGxjCm5gAqfvJRsUhkIMlIohJa7AUgNIz0AEYwjAAKAsrCIJFYQQAWONYgpXGIFVWIibUu2fg49AD6XLwR0RM10+kgzW2d3XY6-UMjGXWuuosrsv7cQA
\begin{tikzcd}
\widetilde{X} \arrow[r, "\widetilde{T}"] & \widetilde{Y}           \\
X \arrow[r, "T"'] \arrow[u, "\iota_X"]   & Y \arrow[u, "\iota_Y"']
\end{tikzcd}
  \end{center}
  Furthermore, we have $\norm{T}_{\op} = \norm{\widetilde{T}}_{\op}$. If $T$ is isometric, then so is $\widetilde{T}$, and if $T$ is an isometric isomorphism, then so is $\widetilde{T}$.
\end{proposition}
\begin{definition}
  A normed space is called a dual space if there is a normed space $Z$ such that $Z^{\ast} \cong X$ are isometrically isomorphic. We call $Z$ the predual of $X$.
\end{definition}
\begin{example}\hfill
  \begin{itemize}
    \item We have $c_0^{\ast}\cong \ell_1$, where $c_0$ is the space of all sequence vanishing at infinity, and $\ell_1$ is the space of all absolutely summable sequences.
    \item We have $\ell_1^{\ast}\cong \ell_{\infty}$, where $\ell_{\infty}$ is space of all bounded sequences.
    \item If $\mu$ is a $\sigma$-finite measure on the measurable space $\left(\Omega,\mathcal{M}\right)$, and $p,q\in (1,\infty)$ are such that $p^{-1} + q^{-1} = 1$, then $L_{p}\left(\Omega,\mu\right)^{\ast} \cong L_q\left(\Omega,\mu\right)$. Here,
      \begin{align*}
        L_p\left(\Omega,\mu\right) &= \set{f\colon \Omega\rightarrow \C | \int_{\Omega}^{} \left\vert f \right\vert^p\:d\mu < \infty}.
      \end{align*}
      Additionally, if $\mu$ is semi-finite, then $L_1\left(\Omega,\mu\right)^{\ast}\cong L_{\infty}\left(\Omega,\mu\right)$.
  \end{itemize}
\end{example}
\begin{theorem}[Riesz--Markov Theorem]
  Let $\Omega$ be a LCH space. Then, $M_r\left(\Omega\right) \cong C_0\left(\Omega\right)^{\ast}$ are isometrically isomorphic, where $M_r\left(\Omega\right)$ is equipped with the norm $\norm{\mu} = \left\vert \mu \right\vert\left(\Omega\right)$ for $\mu\in M_r\left(\Omega\right)$.
\end{theorem}
\section{Topological Vector Spaces}%
\section{Hilbert Spaces and Operators}%
\section{Banach Algebras and \texorpdfstring{$C^{\ast}$-Algebras}{C*-Algebras}}%
