\section{Normed Vector Spaces and Algebras}%
The fundamental unit of functional analysis is functions --- specifically, collections of functions equipped with particular operations and a norm that turn them into vector spaces and algebras. This section will focus on some of the basic facts and theory surrounding normed vector spaces and algebras.
\begin{definition}\label{def:vector_space_and_algebra}
  Let $\F = \R$ or $\C$ be a field. An $\F$-vector space $X$ is a nonempty set equipped with two operations:
  \begin{align*}
    a\colon X\times X &\rightarrow X\\
    m\colon \F\times X \rightarrow X,
  \end{align*}
  given by $\left(x,y\right) \xmapsto{a} x+y$ and $\left(\alpha,y\right) \xmapsto{m} \alpha y$, satisfying the following properties:
  \begin{enumerate}[(i)]
    \item for all $x,y,z\in X$, $\left(x+y\right)+z = x + \left(y+z\right)$;
    \item there exists $0_X\in X$ such that for all $x\in X$, $x + 0_X = 0_X + x = x$;
    \item for all $x\in X$, there exists $y\in X$ such that $x + y = y + x = 0_X$;
    \item for all $x,y\in X$, $x+ y = y + x$;
    \item for all $\alpha,\beta \in \F$ and all $x,y\in X$, $\alpha\left(x+y\right) = \alpha x + \alpha y$ and $\left(\alpha + \beta\right)x = \alpha x + \beta x$;
    \item for all $\alpha,\beta \in \F$ and for all $x\in X$, $\alpha\left(\beta x\right) = \left(\alpha\beta\right)x$;
    \item for all $x\in X$, $\left(1_{\F}\right)x = x $.
  \end{enumerate}
  We call $a$ vector addition and $m$ scalar multiplication.\newline

  If $X$ is also equipped with a structure to allow multiplication between vectors,
  \begin{align*}
    X\times X &\rightarrow X\\
    \left(x,y\right) &\mapsto xy,
  \end{align*}
  then we say $X$ is an $\F$-algebra. The multiplication structure satisfies the following properties:
  \begin{itemize}
    \item for all $a,b,c\in X$, $a\left(bc\right) = \left(ab\right)c$;
    \item for all $a,b,c\in X$, $a\left(b+c\right) = ab + ac$ and $\left(a+b\right)c = ac + bc$;
    \item for all $\alpha\in \F$ and $a,b\in X$, $\alpha \left(ab\right) = \left(\alpha a\right)b = a\left(\alpha b\right)$.
  \end{itemize}
  If $X$ admits a (unique) $1_X\in X$ such that $a1_X = 1_X a = a$, then we say $X$ is a unital algebra. If multiplication is commutative, then we say $X$ is a commutative algebra.
\end{definition}
Vector spaces can also be equipped with notions of length and distance --- thus yielding a normed vector space.
\begin{definition}[Seminorms and Norms]\label{def:norms}
  Let $X$ be a $\F$-vector space, and let $p\colon X\rightarrow [0,\infty)$ be a function. If
  \begin{itemize}
    \item $p\left(\lambda x\right) = \left\vert \lambda \right\vert p(x)$ for all $\lambda\in \F$ and $x\in X$ (homogeneity), and
    \item $p\left(x + y\right)\leq p\left(x\right) + p\left(y\right)$ for all $x,y\in X$ (triangle inequality),
  \end{itemize}
  we say $p$ is a seminorm. If $p$ also satisfies
  \begin{itemize}
    \item $p\left(x\right) = 0 $ if and only if $x = 0$ (positive definite),
  \end{itemize}
  then $p$ is a norm. Norms on vector spaces are usually denoted $\norm{\cdot}$. Additionally, if $X$ is an algebra, the (semi)norm also has to be sub-multiplicative --- i.e.,
  \begin{align*}
    \norm{xy} &\leq \norm{x}\norm{y}.
  \end{align*}
  Two norms, $\norm{\cdot}_a$ and $\norm{\cdot}_b$, are said to be equivalent if there exist constants $C_1$ and $C_2$ such that
  \begin{align*}
    \norm{x}_a &\leq C_1\norm{x}_b\\
    \norm{x}_b &\leq C_2 \norm{x}_a
  \end{align*}
  for all $x\in X$.\newline

  If $X$ is complete with respect to the metric $d\left(x,y\right) = \norm{x-y}$, we call $X$ a Banach space. If $X$ is a normed algebra that is complete with respect to its induced metric, then we say $X$ is a Banach algebra.
\end{definition}
\begin{theorem}
  Let $X$ be a normed vector space. Then, $X$ is complete if and only if, for every sequence of vectors $\left(x_k\right)_k$, if$\sum_{k=1}^{\infty}\norm{x_k}$ converges, then $\sum_{k=1}^{\infty}x_k$ converges.
\end{theorem}
\begin{definition}[Open Balls, Closed Balls, Spheres]\label{def:open_closed_balls}
  Let $X$ be a normed vector space.
  \begin{itemize}
    \item We write 
      \begin{align*}
        U\left(x,\delta\right) &= \set{y\in X | \norm{y-x} < \delta}
      \end{align*}
      to be the open ball of radius $\delta$ centered at $x$. We write $U_X = U\left(0,1\right)$.
    \item We write
      \begin{align*}
        B\left(x,\delta\right) &= \set{y\in X | \norm{y-x} \leq \delta}
      \end{align*}
      to be the closed ball of radius $\delta$ centered at $x$. We write $B_X = B\left(0,1\right)$.
    \item We write
      \begin{align*}
        S\left(x,\delta\right) &= \set{y\in X | \norm{y-x} = \delta}
      \end{align*}
      to be the sphere of radius $\delta$ centered at $x$. We write $S_X = S\left(0,1\right)$.
  \end{itemize}
\end{definition}
\begin{definition}\label{def:basis}
  Let $X$ be a normed vector space.
  \begin{itemize}
    \item A basis for $X$ is a set $\set{x_i}_{i\in I}$ such that for any $x\in X$, there is a unique finite sum
      \begin{align*}
        x &= \sum_{i\in I}\alpha_i x_i.
      \end{align*}
    \item If $A\subseteq X$ is a set of vectors, then the span of $A$, denoted $\Span\left(A\right)$, is the set of all finite linear combinations of elements in $A$.
    \item If $A\subseteq X$ is such that $\overline{\Span}\left(A\right) = X$, then we say $A$ is a total subset of $X$.
  \end{itemize}
\end{definition}
\begin{definition}
  Let $T\colon X\rightarrow Y$ be a linear map between normed vector spaces. We say $T$ is bounded if its operator norm, defined by
  \begin{align*}
    \norm{T}_{\op} &= \sup_{x\in B_X}\norm{T\left(x\right)}
  \end{align*}
  is finite. We write $\B\left(X,Y\right)$ for the set of all bounded linear maps between $X$ and $Y$.
\end{definition}
\begin{remark}
  Note that if $Y$ is complete, then $\B\left(X,Y\right)$ is a Banach space with pointwise addition and scalar multiplication.\newline

  A quick sketch of the proof is as follows: consider a $\norm{\cdot}_{\op}$-Cauchy sequence $\left(T_n\right)_n$ in $\B\left(X,Y\right)$, and define $T$ to be the pointwise limit of $\left(T_n\right)_n$, which exists as for any $y\in Y$, $\left(T_n\left(y\right)\right)_{y}$ is Cauchy in $Y$. Then, it can be shown that defining $T$ in this manner yields convergence in operator norm.\newline

  Furthermore, if we define $\B(X) = \B(X,X)$, then this space is a normed algebra with pointwise addition, scalar multiplication, and composition of operators. The algebra $\B(X)$ is complete if $X$ is complete.
\end{remark}
\begin{fact}
  The following are equivalent for a linear map $T$:
  \begin{itemize}
    \item $T$ is continuous at $0$;
    \item $T$ is continuous;
    \item $T$ is uniformly continuous;
    \item $T$ is Lipschitz continuous;
    \item $T$ is bounded.
  \end{itemize}
\end{fact}
\begin{definition}
  Let $T\colon X\rightarrow Y$ be a linear map between normed vector spaces.
  \begin{itemize}
    \item We say $T$ is bounded below if there exists $C > 0$ such that $\norm{T(x)}\geq C\norm{x}$ for all $x\in X$.
    \item If $T$ is bounded and bounded below, we say $T$ is bicontinuous.
    \item If $T$ is a linear isomorphism that is bicontinuous, we say $T$ is a bicontinuous isomorphism, and say $X\cong Y$ are bicontinuously isomorphic.
    \item If $T$ is a linear isomorphism and is such that $\norm{T(x)} = \norm{x}$ for all $x$, then we say $T$ is an isometric isomorphism.
  \end{itemize}
\end{definition}
\begin{definition}
  Let $X$ be a normed vector space. The algebraic dual of $X$, denoted $X'$, is the set of all linear functionals on $X$:
  \begin{align*}
    X' &= \mathcal{L}\left(X,\F\right).
  \end{align*}
  The subset $X^{\ast}\subseteq X'$ is the set of all \textit{continuous} linear functionals on $X$:
  \begin{align*}
    X^{\ast} &= \B\left(X,\F\right).
  \end{align*}
\end{definition}
\begin{definition}[Unconditional Summability]
  If $\Omega$ is a set and $f\colon \Omega\rightarrow X$ is any function between $\Omega$ and suitable vector space $X$ (see Definition \ref{def:tvs}), we say the unconditional series $\sum_{j\in\Omega}f(j)$ converges to some value $k\in X$ if the net $\left(s_{F}\right)_{F\in \mathcal{F}}$ converges to $k$, where
  \begin{align*}
    \mathcal{F} &= \set{F | F\subseteq J,\Card(F) < \infty}
  \end{align*}
  is the collection of finite subsets of $\Omega$ directed by inclusion.
\end{definition}
\begin{definition}[Three Fundamental Function Spaces]\label{def:three_function_spaces}
  Let $\Omega$ be any set.
  \begin{itemize}
    \item The space $\ell_{1}(\Omega)$ is the set of all functions $f\colon \Omega\rightarrow \C$ such that $\displaystyle \sum_{t\in\Omega}\left\vert f(t) \right\vert < \infty$.
    \item The space $\ell_{2}(\Omega)$ is the set of all functions $f\colon \Omega\rightarrow \C$ such that $\displaystyle \sum_{t\in\Omega}\left\vert f(t) \right\vert^2 < \infty$.
    \item The space $\ell_{\infty}(\Omega)$ is the set of all functions $f\colon \Omega\rightarrow \C$ such that $\sup_{t\in\Omega}\left\vert f(t) \right\vert < \infty$.
  \end{itemize}
\end{definition}

\section{The Fundamental Theorems of Banach Spaces}%
\begin{definition}
  Let $X$ be a topological space.
  \begin{itemize}
    \item A subset $F\subseteq X$ is called nowhere dense if $\left(\overline{F}\right)^{\circ} = \emptyset$.
    \item If $X$ is a countable union of nowhere dense sets, we say $X$ is meager (or of the first category).
    \item We say $X$ is a Baire space if, for any countable collection of open, dense subsets $\set{A_n}_{n\geq 1}$, their intersection $\bigcap_{n\geq 1}A_n\subseteq X$ is also dense.
  \end{itemize}
\end{definition}
The Baire category theorem serves as one of the central bridges between functional analysis and topology. Note that the property of being a Baire space is a purely topological definition, while completeness is an analytic concept.
\begin{theorem}[Baire Category Theorem]\label{thm:baire}
  Let $X$ be a complete metric space. Then, $X$ is a Baire space.
\end{theorem}
The Baire category theorem is used to prove many important theorems in functional analysis, such as the ones that follow. They all fundamentally rely on the completeness of Banach spaces, which is expressed through the Baire category theorem. Proofs for these theorems can be found in functional analysis textbooks such as \cite{rudin_functional_analysis}.
\begin{theorem}[Open Mapping Theorem]\label{thm:open_mapping}
  Let $T\colon X\rightarrow Y$ be a surjective linear map between Banach spaces. Then, $T$ is an open map --- i.e., if $U\subseteq X$ is open, then $V$ is also open.
\end{theorem}
\begin{corollary}[Bounded Inverse]\label{cor:bounded_inverse}
  If $T\colon X\rightarrow Y$ is a bounded linear map that is bijective, then $T^{-1}$ is also bounded.
\end{corollary}
\begin{theorem}[Closed Graph Theorem]\label{thm:closed_graph}
  Let $T\colon X\rightarrow Y$ be a linear map between Banach spaces. Then, $T$ is bounded if and only if $\graph\left(T\right) = \set{\left(x,T(x)\right)| x\in X} \subseteq X\times Y$ is closed in the product topology.
\end{theorem}
\begin{theorem}[Uniform Boundedness Principle]\label{thm:uniform_boundedness}
  Let $\set{T_i}_{i\in I}$ be a family of maps in $\B\left(X,Y\right)$ such that, for all $x\in X$, $\sup_{i\in I}\norm{T_i(x)} < \infty$. Then, $\sup_{i\in I}\norm{T_i}_{\op} < \infty$.
\end{theorem}
Now, we turn our attention towards linear functionals. Consider the following problem in algebra: suppose $X$ is a finite-dimensional vector space, and $Y\subseteq X$ is a subspace. If we have a linear functional $\varphi\colon Y\rightarrow \F$, can this linear functional be extended to the full space?\newline

The answer is yes --- if $\mathcal{B} = \set{x_1,\dots,x_m}$ is a basis for $Y$, a fundamental result in linear algebra states that this basis can be extended to a basis for $X$, $\mathcal{C} = \set{x_1,\dots,x_m,x_{m+1},\dots,x_n}$; if we define $c_i = \varphi\left(x_i\right)$ for $1 \leq i \leq m$, we may define $\varphi\left(x_i\right) = 0$ for $m+1 \leq i \leq n$. In other words, we may always extend elements of the algebraic dual from a subspace to the full space.\newline

However, if $X$ is infinite-dimensional and equipped with a norm (or, more generally, a locally convex topology, see Definition \ref{def:lctvs}), we also care about continuity, norm, and whether these extensions preserve continuity and norm. This is the domain of the Hahn--Banach theorems, which establish extension and separation results in normed vector spaces (and, as detailed later, locally convex topological vector spaces, see Theorem \ref{thm:hb_continuous_extension_lctvs}).
\begin{definition}[Minkowski Functional]
  We call a map $p\colon X\rightarrow [0,\infty)$ a Minkowski functional if
  \begin{itemize}
    \item $p\left(x + y\right)\leq p\left(x\right) + p\left(y\right)$ for all $x,y\in X$, and;
    \item $p\left(tx\right) = tp\left(x\right)$ for all $x\in X$ and $t > 0$.
  \end{itemize}
\end{definition}
\begin{theorem}[Hahn--Banach--Minkowski Extension]\label{thm:hbm_extension}
  Let $Y\subseteq X$ be a linear subspace of a normed vector space $X$, and let $\phi\in Y^{\ast}$ and $p$ a Minkowski functional be such that for all $y\in Y$, $\phi\left(y\right) \leq p\left(y\right)$. Then, there is a map $\Phi\in X^{\ast}$ such that
  \begin{itemize}
    \item $\Phi|_{Y} = \phi$, and;
    \item $\Phi\left(x\right) \leq p\left(x\right)$ for all $x\in X$.
  \end{itemize}
\end{theorem}
\begin{theorem}[Hahn--Banach Continuous Extension]\label{thm:hb_continuous_extension}
  Let $X$ be a normed vector space, and $\phi\in Y^{\ast}$, where $Y\subseteq X$ is a linear subspace. Then, there exists a linear functional $\Phi\in X^{\ast}$ such that $\norm{\Phi} = \norm{\phi}$, and $\Phi|_{Y} = \phi$.\footnote{This extension is not necessarily unique.}
\end{theorem}
The Hahn--Banach extension theorems lend themselves nicely to understanding the separation properties of linear functionals in the continuous dual space. These results allow us to know that there are ``enough'' linear functionals in the dual space of any normed vector space that allow us to distinguish points from closed subspaces and distinguish points from each other.
\begin{theorem}[Hahn--Banach Separation]\label{thm:hb_separation}
  Let $X$ be a normed vector space.
  \begin{itemize}
    \item For a fixed $x_0\in X$, there exists a linear functional $\phi\in X^{\ast}$ such that $\phi\left(x_0\right) = \norm{x_0}$.
    \item For a proper closed subspace $Y\subseteq X$ and some fixed $x_0\in X\setminus Y$, there is a $\phi\in X^{\ast}$ such that $\norm{\phi} \leq 1$, $\phi|_{Y} = 0$, and $\phi\left(x_0\right) = \dist_{Y}\left(x_0\right)$.
  \end{itemize}
\end{theorem}
\begin{corollary}
  Let $X$ be a normed space. For every $x\in X$, we have
  \begin{align*}
    \sup_{\phi\in B_{X^{\ast}}}\left\vert \varphi\left(x\right)  \right\vert = \norm{x}.
  \end{align*}
\end{corollary}
\section{Duality}%
Here, we discuss a little bit more of the theory of dual spaces.
\begin{definition}
  Let $X$ be a normed vector space. The linear functional $\hat{x}\colon X^{\ast}\rightarrow \C$, defined by
  \begin{align*}
    \hat{x}(\varphi) &= \varphi(x)
  \end{align*}
  is bounded with norm $\norm{\hat{x}}_{\op} = \norm{x}$. We define the embedding $\iota\colon X\hookrightarrow X^{\ast\ast}$ by
  \begin{align*}
    \iota(x) &= \hat{x}.
  \end{align*}
  We call $\iota$ the canonical embedding.
\end{definition}
\begin{definition}
  Let $X$ be a normed space. A norm completion of $X$ is a pair $\left(Z,j\right)$, where $Z$ is a Banach space, $j\colon X\hookrightarrow Z$ is a linear isometry, and $\overline{\Ran}\left(j\right) = Z$.
\end{definition}
\begin{proposition}
  Let $X$ be a normed space, and set $\widetilde{X} = \overline{\iota_X\left(X\right)}^{\norm{\cdot}_{\op}}\subseteq X^{\ast\ast}$. Then, $\left(\widetilde{X},\iota_X\right)$ is a norm completion of $X$. Additionally, if $\left(Z,j\right)$ is any other norm completion of $X$, then there is an isometric isomorphism $Z\rightarrow \widetilde{X}$.
\end{proposition}
\begin{proposition}
  Let $X$ and $Y$ be normed spaces, and let $T\in \B\left(X,Y\right)$. Then, there is a unique $\widetilde{T}\in B\left(\widetilde{X},\widetilde{Y}\right)$ such that $\widetilde{T}\circ \iota_X = \iota_Y\circ T$. The diagram below commutes.
  \begin{center}
    % https://tikzcd.yichuanshen.de/#N4Igdg9gJgpgziAXAbVABwnAlgFyxMJZABgBoBGAXVJADcBDAGwFcYkQANEAX1PU1z5CKchWp0mrdgE0efEBmx4CRMsXEMWbRCAA6ugO5ZYeRrGAduc-kqFFR6mpqk79Rk1jMxg0q93EwUADm8ESgAGYAThAAtkgATDQ4EEgAzE6S2nqGxjCm5gAqfvJRsUhkIMlIohJa7AUgNIz0AEYwjAAKAsrCIJFYQQAWONYgpXGIFVWIibUu2fg49AD6XLwR0RM10+kgzW2d3XY6-UMjGXWuuosrsv7cQA
\begin{tikzcd}
\widetilde{X} \arrow[r, "\widetilde{T}"] & \widetilde{Y}           \\
X \arrow[r, "T"'] \arrow[u, "\iota_X"]   & Y \arrow[u, "\iota_Y"']
\end{tikzcd}
  \end{center}
  Furthermore, we have $\norm{T}_{\op} = \norm{\widetilde{T}}_{\op}$. If $T$ is isometric, then so is $\widetilde{T}$, and if $T$ is an isometric isomorphism, then so is $\widetilde{T}$.
\end{proposition}
\begin{definition}
  A normed space is called a dual space if there is a normed space $Z$ such that $Z^{\ast} \cong X$ are isometrically isomorphic. We call $Z$ the predual of $X$.
\end{definition}
\begin{example}\hfill
  \begin{itemize}
    \item We have $c_0^{\ast}\cong \ell_1$, where $c_0$ is the space of all sequence vanishing at infinity, and $\ell_1$ is the space of all absolutely summable sequences.
    \item We have $\ell_1^{\ast}\cong \ell_{\infty}$, where $\ell_{\infty}$ is space of all bounded sequences.
    \item If $\mu$ is a $\sigma$-finite measure on the measurable space $\left(\Omega,\mathcal{M}\right)$, and $p,q\in (1,\infty)$ are such that $p^{-1} + q^{-1} = 1$, then $L_{p}\left(\Omega,\mu\right)^{\ast} \cong L_q\left(\Omega,\mu\right)$. Here,
      \begin{align*}
        L_p\left(\Omega,\mu\right) &= \set{f\colon \Omega\rightarrow \C | \int_{\Omega}^{} \left\vert f \right\vert^p\:d\mu < \infty}.
      \end{align*}
      Additionally, if $\mu$ is semi-finite, then $L_1\left(\Omega,\mu\right)^{\ast}\cong L_{\infty}\left(\Omega,\mu\right)$.
  \end{itemize}
\end{example}
\begin{theorem}[Riesz--Markov Theorem]
  Let $\Omega$ be a LCH space. Then, $M_r\left(\Omega\right) \cong C_0\left(\Omega\right)^{\ast}$ are isometrically isomorphic, where $M_r\left(\Omega\right)$ is equipped with the norm $\norm{\mu} = \left\vert \mu \right\vert\left(\Omega\right)$ for $\mu\in M_r\left(\Omega\right)$.
\end{theorem}
\section{Topological Vector Spaces}%
Earlier, we discussed some of the features of normed vector spaces and Banach spaces. Here, we expand our scope to to examine the analytic properties of vector spaces whose topology is not necessarily induced by a norm. 
\begin{definition}\label{def:tvs}
  Let $X$ be a $\C$-vector space, and let $\tau$ be a topology on $X$. We say $\tau$ is compatible with the vector space structure of $X$ if
  \begin{enumerate}[(1)]
    \item $X$ is T1 (see Definition \ref{def:separation_axioms});
    \item scalar multiplication, $(\lambda,x)\mapsto \lambda x$ is continuous, where $\C\times X$ is given the product topology;
    \item vector addition, $(x,y) \mapsto x + y$ is continuous, where $X\times X$ is given the product topology.
  \end{enumerate}
  If $X$ is equipped with a topology compatible with the vector space structure of $X$, then $(X,\tau)$ is called a topological vector space. We abbreviate it as TVS.
\end{definition}
\begin{remark}
  It can be shown that if $(X,\tau)$ is a TVS, the topology on $X$ is automatically Hausdorff.
\end{remark}
\begin{definition}
  Let $X$ be a $\C$-vector space.
  \begin{enumerate}[(1)]
    \item If $A,B\subseteq X$, then we define
      \begin{align*}
        A + B &= \set{x + y | x\in A,y\in B}.
      \end{align*}
      If $A = \set{x_0}$, we abbreviate $\set{x_0} + B$ as $x_0 + B$, which is called the translation of $B$.
    \item If $A\subseteq X$, and $\alpha\in \C$, then
      \begin{align*}
        \alpha A &= \set{\alpha x | x\in A}
      \end{align*}
      is the scaling of $A$ by $\alpha$. We write $(-1)A = -A$.
    \item A subset $A\subseteq X$ is called symmetric if $-A = A$.
    \item A subset $A\subseteq X$ is called balanced if $\alpha A\subseteq A$ for all $\left\vert \alpha \right\vert\leq 1$.
    \item A subset $C\subseteq X$ is called convex if for all $t\in [0,1]$ and $x_1,x_2\in C$, $\left(1-t\right)x_1 + tx_2 \in C$.
  \end{enumerate}
  We define
  \begin{align*}
    \operatorname{conv}\left(A\right) &= \bigcap\set{C | A\subseteq C\subseteq X,C\text{ is convex}}\\
                                      &= \set{\sum_{j=1}^{n}t_ja_j | n\in\N,t_j\geq 0,\sum_{j=1}^{n}t_j = 1,a_j\in A}.
  \end{align*}
\end{definition}
\begin{definition}\label{def:lctvs}
A TVS $\left(X,\tau\right)$ is called locally convex if $X$ admits a neighborhood base (see Definition \ref{def:neighborhoods_and_bases}) consisting of convex sets. We abbreviate as LCTVS.\newline

It can be shown that every LCTVS has a neighborhood base consisting of \textit{balanced} convex sets.
\end{definition}
An important structural result in the theory of topological vector spaces is the fact that every locally convex topology is generated by a separating family of seminorms.
\begin{proposition}\label{prop:structure_of_lctvs}
  Let $X$ be a $\C$-vector space, and let $\mathcal{P}$ be a family of seminorms on $X$. For each $z\in X$ and $p\in \mathcal{P}$, we define $f_{p,z}\colon X\rightarrow [0,\infty)$ by
  \begin{align*}
    f_{p,z}(x) &= p\left(x-z\right).
  \end{align*}
  The topology $\tau_{\mathcal{P}}$ is the initial topology on $X$ induced by the family
  \begin{align*}
    \mathcal{F}_{\mathcal{P}} &= \set{f_{p,z} | p\in \mathcal{P},z\in X}.
  \end{align*}
  If $\mathcal{P}$ is such that for each $x\neq 0$, there is some $p$ such that $p(x)\neq 0$ (i.e., $\mathcal{P}$ separates the points of $X$), then the family $\mathcal{F}_{\mathcal{P}}$ separates points in $X$. It is then the case that $\left(X,\tau_{\mathcal{P}}\right)$ is a LCTVS.\newline

  Convergence of nets in the topology $\tau_{\mathcal{P}}$ is defined by $\left(x_{\alpha}\right)_{\alpha}\xrightarrow{\tau_{\mathcal{P}}}x$ if and only if $p\left(x_{\alpha}-x\right)\rightarrow 0$ for all $p\in \mathcal{P}$.\newline

  Furthermore, if $\left(X,\tau\right)$ is any LCTVS, then there is a corresponding family of separating seminorms $\mathcal{P}$ such that $\id\colon\left(X,\tau\right)\rightarrow \left(X,\tau_{\mathcal{P}}\right)$ is a homeomorphism.
\end{proposition}
The Hahn--Banach theorems (such as the extension and separation results) that we established in Theorems \ref{thm:hbm_extension}, \ref{thm:hb_continuous_extension}, and \ref{thm:hb_separation} have corresponding results in topological vector spaces.
\begin{theorem}[Hahn--Banach Extension for LCTVS]\label{thm:hb_continuous_extension_lctvs}
  Let $X$ be a LCTVS, and suppose $E\subseteq X$ is a subspace. If $\varphi\in E^{\ast}$, then there is a $\psi\in X^{\ast}$ such that $\psi|_{E} = \varphi$.
\end{theorem}
\begin{corollary}
  Let $X$ be a LCTVS. Let $\set{x_1,\dots,x_n}\subseteq X$ be linearly independent, and $\set{\alpha_1,\dots,\alpha_n}\in \C$. Then, there exists $\varphi\in X^{\ast}$ such that $\varphi\left(x_j\right) = \alpha_j$ for all $j$.
\end{corollary}
To provide some context for the Hahn--Banach separation results, consider two open, disjoint, convex subsets $A,B\subseteq \R^n$. The hyperplane separation theorem from convex optimization (see \cite[Chapter 2.6]{convex_optimization}) states that there is a nonzero vector $m\in \R^n$ and some $b\in \R$ such that the map $\varphi\colon \R^n\rightarrow \R$, defined by $\varphi(x) = m^{T}x - b$, is strictly negative for all $x\in A$ and is strictly positive for all $x\in B$. The affine hyperplane defined by $\set{x | \varphi(x) = b}$ is known as a separating hyperplane for $A$ and $B$.\newline

What the Hahn--Banach theorems allow us to do is extend this result beyond $\R^n$ to the case of any TVS --- with a special case if the TVS is locally convex.
\begin{theorem}[Hahn--Banach Separation for TVS]\label{thm:hb_separation_tvs}
  Let $X$ be a TVS (that may or may not be locally convex) over $\C$. Let $A$ and $B$ be convex and disjoint subsets of $X$. If $A$ is open, then there exists $\varphi\in X^{\ast}$, with $\varphi = u + iv$, and $t\in \R$ such that
  \begin{align*}
    u(a) < t \leq u(b)
  \end{align*}
  for all $a\in A$ and $b\in B$.\newline

  If $A$ and $B$ are open, then the inequalities can be taken to be strict.
\end{theorem}
The requirement that $A$ and $B$ be open can be relaxed in the case of a LCTVS, where we can separate closed, disjoint, convex sets, so long as one of the sets is compact. Specifically, we are able to separate the sets by a double hyperplanes if the topology on $X$ is locally convex.
\begin{theorem}[Hahn--Banach Separation for LCTVS]\label{thm:hb_separation_lctvs}
  Let $X$ be a LCTVS, and suppose $C,K\subseteq X$ are closed, disjoint, convex subsets of $X$, with $K$ compact. Then, there exists $\varphi\in X^{\ast}$, with $\varphi = u + iv$, $t\in \R$, and $\delta > 0$ such that
  \begin{align*}
    u(x) \leq t \leq t + \delta \leq u(y)
  \end{align*}
  for all $x\in C$ and $y\in K$.
\end{theorem}
\begin{proposition}
  Let $W\subseteq X'$, where $X'$ is the algebraic dual of $X$. For each $\varphi\in W$, consider the seminorm
  \begin{align*}
    p_{\varphi}(x) &= \left\vert \varphi(x) \right\vert.
  \end{align*}
  We let $\mathcal{P}_{W} = \set{p_{\varphi} | \varphi\in W}$. If $\mathcal{P}_{W}$ separates points, then we may construct the topology $\tau_{\mathcal{P}_{W}}$ as in Proposition \label{prop:structure_of_lctvs}.\newline

  Alternatively, we may consider the initial topology on $X$ induced by the family $W$, written $\sigma\left(X,W\right)$.\newline

  It is the case that $\id\colon \left(X,\tau_{\mathcal{P}_{W}}\right)\rightarrow \left(X,\sigma\left(X,W\right)\right)$ is a homeomorphism.
\end{proposition}
\begin{definition}[Norm Topology]\label{def:norm_topology}
  Let $X$ be a normed vector space. If $\mathcal{P} = \set{\norm{\cdot}}$, then the topology $\tau_{\mathcal{P}}$ is known as the norm topology on $X$.\newline

  Convergence is defined by $\left(x_n\right)_n\xrightarrow{\norm{\cdot}}x$ if and only if $\norm{x_n - x}\rightarrow 0$.
\end{definition}
\begin{remark}
  Normed vector spaces are metric space, and hence first countable (so sequences are sufficient to define convergence).
\end{remark}
\begin{definition}[Weak Topology]\label{def:weak_topology}
  If $X$ is a normed vector space, we say $\sigma\left(X,X^{\ast}\right)$ is the weak topology on $X$.\newline

  Convergence is defined by $\left(x_\alpha\right)_\alpha\xrightarrow{w}x$ if and only if $\left(\varphi\left(x_\alpha\right)\right)_\alpha\rightarrow \varphi\left(x\right)$ for all $\varphi\in X^{\ast}$.
\end{definition}
\begin{definition}[Weak* Topology]\label{def:weak_star_topology}
  If $X$ is a normed vector space, we say $\sigma\left(X^{\ast},\iota(X)\right)$, where $\iota$ is the canonical embedding, is the weak* topology on $X^{\ast}$.\newline

  Convergence is defined by $\left(\varphi_{\alpha}\right)_\alpha\xrightarrow{w^{\ast}} \varphi$ if and only if $\left(\varphi_{\alpha}(x)\right)_{\alpha}\rightarrow \varphi(x)$ for all $x\in X$.
\end{definition}
\begin{theorem}[Banach--Alaoglu Theorem]\label{thm:banach_alaoglu}
  Let $X$ be a normed vector space. 
  \begin{enumerate}[(1)]
    \item The unit ball in the dual space, $B_{X^{\ast}}$, is $w^{\ast}$-compact.
    \item A subset $C\subseteq X^{\ast}$ is $w^{\ast}$-compact if and only if $C$ is $w^{\ast}$-closed and norm bounded.
  \end{enumerate}
\end{theorem}
\section{Hilbert Spaces and Operators}%
In Chapters \ref{ch:left_regular_representation} and \ref{ch:nuclearity}, we discuss the relationship between a group $\Gamma$ and the way the group is represented as an algebra of bounded operators on a Hilbert space. Here, we discuss more exactly what is meant by ``algebra of bounded operators on a Hilbert space.''
\begin{definition}
  Let $X$ be a vector space. A semi-inner product on $X$ is a map $ \iprod{\cdot}{\cdot}\colon X\times X \rightarrow \C $ such that
  \begin{itemize}
    \item $ \iprod{\alpha x + y}{z} = \alpha \iprod{x}{z} + \iprod{y}{z}$;
    \item $ \iprod{x}{\alpha y + z} = \overline{\alpha}\iprod{x}{y} + \iprod{x}{z}$;
    \item $ \iprod{x}{x}\geq 0 $.
  \end{itemize}
  If $ \iprod{x}{x} = 0 $ if and only if $ x = 0 $, then $ \iprod{\cdot}{\cdot} $ is an inner product with induced norm $\norm{x}^2 = \iprod{x}{x}$. We call $X$ an inner product space if it is equipped with an inner product.\newline

  If $X$ is an inner product space that is complete with respect to the induced norm, then we say $X$ is a Hilbert space. We usually denote Hilbert spaces by $ \mathcal{H} $.
\end{definition}
\begin{theorem}[Polarization Identity]
  Let $\mathcal{H}$ be a Hilbert space, and let $x,y\in \mathcal{H}$. Then,
  \begin{align*}
    \iprod{x}{y} &= \frac{1}{4}\sum_{j=0}^{3} \norm{x + i^jy}^2.
  \end{align*}
\end{theorem}
\begin{definition}
  Let $\mathcal{H}$ be a Hilbert space. A subset $\set{x_i}_{i\in I}$ is called orthonormal if 
  \begin{align*}
    \iprod{x_i}{x_j} &= \begin{cases}
      1 & i=j\\
      0 & i\neq j
    \end{cases}\\
                     &= \delta_{ij}.
  \end{align*}
  A maximal orthonormal set in $\mathcal{H}$ is called an orthonormal basis; equivalently, the set $\set{x_i}_{i\in I}$ is an orthonormal basis if and only if $\Span\left(\set{x_i}_{i\in I}\right)$ is dense in $\mathcal{H}$.
\end{definition}
\begin{theorem}[Bessel's Inequality and Parseval's Identity]
  Let $\set{e_i}_{i\in I}$ be an orthonormal set in a Hilbert space $\mathcal{H}$. Then,
  \begin{align*}
    \sum_{i\in I} \left\vert \iprod{x}{e_i} \right\vert^2 &\leq \norm{x}^2.
  \end{align*}
  If $\set{e_i}_{i\in I}$ is an orthonormal basis, then
  \begin{align*}
    \sum_{i\in I} \left\vert \iprod{x}{e_i} \right\vert^2 &= \norm{x}^2.
  \end{align*}
\end{theorem}
\begin{theorem}\label{thm:projection_theorem}
  If $M\subseteq \mathcal{H}$ is a closed subspace of a Hilbert space $\mathcal{H}$, then for any $x\in \mathcal{H}$, there is a unique $y_x\in M$ such that $\norm{x-y_x}$ is minimal.\newline

  The map $P_M\colon \mathcal{H}\rightarrow M$, $x\mapsto y_x$ is known as the orthogonal projection onto $M$. Additionally, the map $P_M$ has the following properties:
  \begin{itemize}
    \item $P_M$ is linear;
    \item $P_M^2 = P_M$;
    \item $\norm{P_M} = 1$ (if $M = \set{0}$, then $\norm{P_M} = 0$).
  \end{itemize}
  Setting $M^{\perp}$ to be the range of $I_{\mathcal{H}} - P_M$, it is also the case that $\mathcal{H}/M \cong M^{\perp}$, with $\mathcal{H} = M\oplus M^{\perp}$.
\end{theorem}
One of the most important structural results on Hilbert spaces relates the continuous dual of a Hilbert space to the inner product. 
\begin{theorem}[Riesz Representation Theorem for Hilbert Spaces]
  Let $\mathcal{H}$ be a Hilbert space, and let $\varphi\in \mathcal{H}^{\ast}$. Then, there is a unique $f_{\varphi}\in \mathcal{H}$ such that
  \begin{align*}
    \varphi\left(g\right) &= \iprod{g}{f_{\varphi}}
  \end{align*}
  for all $g\in \mathcal{H}$.
\end{theorem}
Now that we understand the structure of Hilbert spaces and their closed subspaces, we can now begin understanding bounded operators on Hilbert spaces.
\begin{definition}
  Let $T\colon \mathcal{H}\rightarrow \mathcal{H}$ be a bounded operator between Hilbert spaces. We define the adjoint of $T$, $T^{\ast}$, to be the unique operator $T^{\ast}\colon \mathcal{H}\rightarrow \mathcal{H}$ such that
  \begin{align*}
    \iprod{T\left(x\right)}{y} &= \iprod{x}{T^{\ast}\left(y\right)}
  \end{align*}
  for all $x,y\in \mathcal{H}$. The adjoint satisfies the following properties:
  \begin{itemize}
    \item $\left(T + \lambda S\right)^{\ast} = T^{\ast} + \overline{\lambda}S^{\ast}$;
    \item $T^{\ast\ast} = T$;
    \item $\left(R\circ T\right)^{\ast} = T^{\ast}\circ R^{\ast}$;
    \item if $T$ is invertible, then $\left(T^{-1}\right)^{\ast} = \left(T^{\ast}\right)^{-1}$;
    \item $\norm{T^{\ast}} = \norm{T}$;
    \item $\norm{T^{\ast}T} = \norm{T}^2$ (known as the $C^{\ast}$-property).
  \end{itemize}
\end{definition}
There are a variety of topologies one can place on the space $\B\left(\mathcal{H}\right)$. We detail three.
\begin{definition}
  Let $\left(T_{\alpha}\right)_{\alpha}$ be a net in $\B\left(\mathcal{H}\right)$.
  \begin{itemize}
    \item We say $\left(T_{\alpha}\right)_{\alpha}\xrightarrow{\norm{\cdot}_{\op}} T$ if $\norm{T_{\alpha} - T}_{\op}\rightarrow 0$. This is the norm topology on $\B\left(\mathcal{H}\right)$.
    \item We say $\left(T_{\alpha}\right)_{\alpha}\xrightarrow{ \text{SOT} } T$ if, for all $\xi\in \mathcal{H}$, $\norm{T_{\alpha}\left(x\right) - T(x)}\rightarrow 0$. This is the strong operator topology (or topology of pointwise convergence) on $\B\left(\mathcal{H}\right)$.
    \item We say $\left(T_{\alpha}\right)_{\alpha}\xrightarrow{\text{WOT}}T$ if, for all $\xi,\eta\in \mathcal{H}$, $ \iprod{T_{\alpha}\left(\xi\right)}{\eta}\rightarrow \iprod{T\left(\xi\right)}{\eta} $. This is the weak operator topology on $\B\left(\mathcal{H}\right)$.
  \end{itemize}
\end{definition}
\begin{definition}\hfill
  \begin{itemize}
    \item We say $T\in \B\left(\mathcal{H}\right)$ is normal if $T^{\ast}T = TT^{\ast}$.
    \item We say $T\in \B\left(\mathcal{H}\right)$ is self-adjoint if $T^{\ast}= T$. We write $\B\left(\mathcal{H}\right)_{\sa}$ to refer to the set of all self-adjoint operators in $\B\left(\mathcal{H}\right)$.
    \item We say $P\in \B\left(\mathcal{H}\right)$ is a projection if $P^2 = P^{\ast} = P$.
    \item We say $V\in \B\left(\mathcal{H}\right)$ is an isometry if $V^{\ast}V = I_{\mathcal{H}}$.
    \item We say $T\in \B\left(\mathcal{H}\right)$ is a \textit{partial} isometry if $TT^{\ast}T = T$.
    \item We say $U\in \B\left(\mathcal{H}\right)$ is a unitary if $U^{\ast} = U^{-1}$.
  \end{itemize}
  We write $\mathcal{U}\left(\mathcal{H}\right)$ to refer to the set of all unitary operators on $\mathcal{H}$. Two operators $T,S\in \B\left(\mathcal{H}\right)$ are called unitarily equivalent if there is $U\in \mathcal{U}\left(\mathcal{H}\right)$ such that $UTU^{\ast} = S$.\newline

  The space of unitary operators, $\mathcal{U}\left(\mathcal{H}\right)$, is a group with respect to operator composition.
\end{definition}
The set $\B\left(\mathcal{H}\right)_{\sa}$ admits an order structure.
\begin{definition}
  Let $T\in \B\left(\mathcal{H}\right)_{\sa}$. We say $T$ is positive if, for every $\xi\in \mathcal{H}$, we have
  \begin{align*}
    \iprod{T\left(\xi\right)}{\xi} &\geq 0.
  \end{align*}
  We write $\B\left(\mathcal{H}\right)_{+}$ to refer to the operator norm-closed cone of positive operators in $\B\left(\mathcal{H}\right)_{\sa}$.\newline

  If $T,S\in \B\left(\mathcal{H}\right)_{\sa}$, we say $T\geq S$ if $T-S \in \B\left(\mathcal{H}\right)_{\sa}$.
\end{definition}
\begin{remark}
  It can be shown that an operator $T\in \B\left(\mathcal{H}\right)_{+}$ if and only if there is some $S\in \B\left(\mathcal{H}\right)$ such that $T = S^{\ast}S$.
\end{remark}
\begin{definition}
  Let $x,y\in \mathcal{H}$. We define the rank-one bounded operator $\theta_{x,y}\colon \mathcal{H}\rightarrow \mathcal{H}$ by
  \begin{align*}
    \theta_{x,y}(z) &= \iprod{z}{y}x.
  \end{align*}
  If $T\in \B\left(\mathcal{H}\right)$ is such that
  \begin{align*}
    T &= \sum_{j=1}^{n}\theta_{x_j,y_j},
  \end{align*}
  where $x_j,y_j\in \mathcal{H}$, then $T$ is of finite rank --- i.e., $\Dim\left(\Ran\left(T\right)\right) < \infty$. We write $T\in \F\left(\mathcal{H}\right)$.\newline

  A map $T\in \B\left(\mathcal{H}\right)$ is called compact if $T$ maps bounded sets to sets with compact closure. The space of compact operators is written $\mathbb{K}\left(\mathcal{H}\right)$.
\end{definition}
\begin{theorem}
  The operator norm-closure of the finite rank operators is the compact operators. That is,
  \begin{align*}
    \overline{\F\left(\mathcal{H}\right)}^{\norm{\cdot}_{\op}} &= \mathbb{K}\left(\mathcal{H}\right).
  \end{align*}
\end{theorem}
\section{Banach Algebras and \texorpdfstring{$C^{\ast}$-Algebras}{C*-Algebras}}%
Earlier, we looked at the definition of an algebra
\begin{definition}
  A Banach $\ast$-algebra is a complete normed $\ast$-algebra (see Definition \ref{def:star_algebra}) satisfying $\norm{a^{\ast}} = a$.\newline

  If $A$ is a Banach algebra that satisfies the $C^{\ast}$-property, $\norm{a^{\ast}a} = \norm{a}^2$, then $A$ is called a $C^{\ast}$-algebra.
\end{definition}
There are a variety of distinguished elements in $C^{\ast}$-algebras, whose names borrow from their respective names in $\B\left(\mathcal{H}\right)$.
%\begin{definition}
%  Let $A$ be a $C^{\ast}$-algebra.
%  \begin{itemize}
%    \item An element $a\in A$ is called normal if $a^{\ast}a = aa^{\ast}$.
%    \item An element $a\in A$ is called self-adjoint if $a^{\ast} = a$. We write $A_{\sa}$ to refer to the set of all self-adjoint elements.
%    \item An element $a\in A_{\sa}$ is called positive if there is some $b\in A$ such that $a = b^{\ast}b$. We write $A_{+}$ to refer to the cone of positive elements inside $A_{\sa}$.
%    \item An element $p\in A$ is called a projection if $p^2 = p^{\ast}= p$.
%    \item An element $v\in A$ is called a partial isometry if $vv^{\ast}v = v$.
%  \end{itemize}
%  If $A$ is also unital, the following distinguished elements also exist.
%  \begin{itemize}
%    \item An element $v\in A$ is called an isometry if $v^{\ast}v = 1_A$. It is called a proper isometry if $vv^{\ast}\neq 1_A$.
%    \item An element $s\in A$ is invertible if there exists some (necessarily unique) $s^{-1}\in A$ such that $s^{-1}s = ss^{-1} = 1_A$. We write $\operatorname{GL}\left(A\right)$ to refer to the set of all invertible elements in $A$. The set $\operatorname{GL}\left(A\right)$ is a group when equipped with multiplication of elements.
%    \item An element $u\in A$ is called a unitary if $u^{\ast}u = uu^{\ast} = 1_A$.
%  \end{itemize}
%\end{definition}
%We are most interested in maps between $C^{\ast}$-algebras that preserve the underlying algebraic structure. These are the $\ast$-homomorphisms (and $\ast$-isomorphisms).
%\begin{definition}
%  Let $A$ and $B$ be $C^{\ast}$-algebras.
%  \begin{itemize}
%    \item If $B = \C$, and $\varphi$ is not the zero map, then the $\ast$-homomorphism $\varphi\colon A\rightarrow \C$ is called a character on the $C^{\ast}$-algebra $A$.
%    \item If $A$ and $B$ are $\ast$-preserving, a linear map $\phi\colon A\rightarrow B$ is called positive if $\phi\left(A_{+}\right)\subseteq B_{+}$
%  \end{itemize}
%\end{definition}

