\epigraph{There is always something left undone...}{Paul Halmos, ``How to Write Mathematics''}
In \cite[48]{brown_and_ozawa}, the authors remark that ``amenable groups admit approximately $10^{10^{10}}$ characterizations.'' This paper was not long enough to discuss all of them, even when restricted to the case of discrete groups.\newline

In Section \ref{sec:subexponential_growth}, we discussed an application of the Følner condition to establishing amenability for a crucial class of groups in geometric group theory (the groups of subexponential growth) --- yet another direction in amenability concerns further study into geometric group theory, including (but not limited to) discussion of how amenability of a group relates to properties of its Cayley graph (see \cite[Section 3.2]{loh_geometric_group_theory} for more discussion on Cayley graphs). There is a notion of graph amenability related to the growth of a graph's neighboring vertex set that, it can be shown, is equivalent to the Følner condition in the case of Cayley graphs (see \cite{monfared_cayley_graphs}).\newline

There are some other directions in amenability that we might be able to take this text. There is a rich theory of amenability in locally compact groups, as well as amenability in Banach algebras and von Neumann algebras. Some of the authoritative texts on this subject include \cite{kazhdan_property_t} and \cite{amenable_banach_algebras} --- we have only touched the surface of what these texts have to offer in the discussion of amenability. Amenability in locally compact groups requires a much stronger command of abstract measure theory, especially concerning the Haar measure (which is a type of translation-invariant measure on the group), as well as integration theory with respect to abstract measures.\newline

In general, most of the results we discussed in this text that pertain to amenability in discrete groups can be translated with relative ease to the case of locally compact groups, primarily by replacing sums (which are really integrals with respect to the counting measure) with integrals with respect to the Haar measure. One analogous result, for instance, is that compact groups are amenable. However, there are also other criteria --- one may define a version of amenability for Banach algebras, and then show that a locally compact group $G$ is amenable if and only if the space $L_1\left( G \right)$ is amenable as a Banach algebra, where $L_1\left( G \right)$ is the space of integrable functions with respect to the Haar measure.\newline

Another direction in amenability concerns deeper discussion of random walks on groups, which was the primary topic of \cite{kesten_means} and \cite{kesten_random_walks}. When we discussed Kesten's criterion in the text (Theorem \ref{thm:kesten_criterion}) we only looked at the basic case where $M(S)$ was defined with respect to the symmetric generating set itself, rather than the general case of a finitely supported probability measure $\mu$ with $S\subseteq \supp(\mu)$.\newline

However, despite all of this, the main regret I have is that this text was ultimately a bit too scattered --- because there are so many different definitions for amenability, this thesis touched on so many topics that to provide enough background development for all the prerequisite ideas (coming from the ambitious assumption that anyone with a background in the standard third/fourth year real analysis courses would be able to glean most of the information in the text) would have resulted in a thesis that was even longer than this one already was. There were many points while I drafted the thesis that I had to go back and add particular bits of information that I didn't expect would be necessary, and there are certainly many tidbits and background highlights that are missing, as well as certain points where notations got all mixed up (this was especially prevalent in the proof of Theorem \ref{thm:nuclearity_implies_amenability}).\newline

Ultimately, this project will never be fully complete, as there are tons of characterizations and nuances that appear as we go deeper into the topic. Nonetheless, I like to believe I gave the topics discussed herein a fair shake, and did so in a manner that was more than deep enough for most to find the text substantially rewarding.
