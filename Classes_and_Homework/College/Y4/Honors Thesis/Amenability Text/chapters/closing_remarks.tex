\epigraph{There is always something left undone, always either something more to say, a better way to say something, or, at the very least, a disturbing vague sense that the perfect addition or improvement is around the corner...}{Paul Halmos, ``How to Write Mathematics''}
In \cite[48]{brown_and_ozawa}, the authors remark that ``amenable groups admit approximately $10^{10^{10}}$ characterizations.'' This paper was not long enough to discuss all of them, even when restricted to the case of discrete groups.\newline

In Section \ref{sec:subexponential_growth}, we discussed an application of the Følner condition to establishing amenability for a crucial class of groups in geometric group theory (the groups of subexponential growth) --- yet another direction in amenability concerns further study into geometric group theory, including (but not limited to) discussion of how amenability of a group relates to properties of its Cayley graph (see \cite[Section 3.2]{loh_geometric_group_theory} for more discussion on Cayley graphs). There is a notion of graph amenability related to the growth of a graph's neighboring vertex set that, it can be shown, is equivalent to the Følner condition in the case of Cayley graphs (see \cite{monfared_cayley_graphs}).\newline

There are some other directions in amenability that we might be able to take this text. There is a rich theory of amenability in locally compact groups, as well as amenability in Banach algebras and von Neumann algebras. Some of the authoritative texts on this subject include \cite{kazhdan_property_t} and \cite{amenable_banach_algebras} --- we have only touched the surface of what these texts have to offer in the discussion of amenability. Amenability in locally compact groups requires a much stronger command of abstract measure theory, especially concerning the Haar measure (which is a type of translation-invariant measure on the group), as well as integration theory with respect to abstract measures.\newline

Another direction in amenability concerns deeper discussion of random walks on groups, which was the primary topic of \cite{kesten_means} and \cite{kesten_random_walks}. When we discussed Kesten's criterion in the text (Theorem \ref{thm:kesten_criterion}) we only looked at the basic case where $M(S)$ was defined with respect to the symmetric generating set itself, rather than the general case of a finitely supported probability measure $\mu$ with $S\subseteq \supp(\mu)$.\newline

Ultimately, this project will probably never be complete --- there are yet more characterizations and discussions of amenability that deserve their fair shake, many more than could be learned in two semesters. Yet, I like to believe I tried my best, and gave the topics discussed herein their due share.
