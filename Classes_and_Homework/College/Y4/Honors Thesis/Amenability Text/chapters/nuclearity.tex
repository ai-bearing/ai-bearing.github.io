Here, we will establish the equivalence between group amenability and certain properties of the group $C^{\ast}$-algebra(s). The results in here will draw from a lot of theory that we discuss a bit more in depth in Chapter \ref{ch:operator_algebras}. Some excellent books on this topic include \cite{brown_and_ozawa} and \cite{completely_bounded_maps_and_operator_algebras}, both of which go deeper into the ramifications of the results we will present herein.
\section{Norms on the Group \texorpdfstring{$\ast$-Algebras}{*-Algebras}}%
From Definition \ref{def:group_star_algebra}, we know that for any group $\Gamma$, there is a free vector space, $\C\left[ \Gamma \right]$, consisting of finitely supported functions on $\Gamma$. Elements of $\C\left[ \Gamma \right]$ are finite sums of the form
\begin{align*}
  a &= \sum_{s\in\Gamma}a(s)\delta_s,
\end{align*}
where $\delta_s$ is the point mass function
\begin{align*}
  \delta_s\left( t \right) &= \begin{cases}
    1 & s = t\\
    0 & s\neq t
  \end{cases}.
\end{align*}
This admits a multiplication by convolution:
\begin{align*}
  f\ast g(s) &= \sum_{t\in\Gamma}f(t)g\left( t^{-1}s \right)\\
              &= \sum_{r\in\Gamma}f\left( sr^{-1} \right)g\left( r \right)
\end{align*}
and an involution
\begin{align*}
  f^{\ast}\left( t \right) &= \overline{f\left( t^{-1} \right)},
\end{align*}
which turn $\C\left[ \Gamma \right]$ into a $\ast$-algebra.\newline

Now, we are interested in applying norms on the group $\ast$-algebra, turning them into group $C^{\ast}$-algebras. We will do this through the use of unitary representations --- there is an intimate relationship between unitary representations of groups and unital representations (see Definition \ref{def:unital_representation}) of the group $\ast$-algebra generated by the group.
\begin{proposition}[{\cite[Proposition 7.2.46]{rainone_analysis}}]\label{prop:unital_unitary_representation}
  Let $\Gamma$ be a group and let $\mathcal{H}$ be a Hilbert space.
  \begin{enumerate}[(1)]
    \item If $u\colon \Gamma\rightarrow \mathcal{U}\left( \mathcal{H} \right)$ is a unitary representation of $\Gamma$, then $\pi_u\colon \C\left[ \Gamma \right]\rightarrow \B\left( \mathcal{H} \right)$, given by
      \begin{align*}
        \pi_u(a) &= \sum_{s\in\Gamma}a(s)u_s
      \end{align*}
      is a unital representation of $\Gamma$.
    \item If $\pi\colon \C\left[ \Gamma \right]\rightarrow \B\left( \mathcal{H} \right)$ is a unital representation, then $u\colon \Gamma\rightarrow \mathcal{U}\left( \mathcal{H} \right)$, given by
      \begin{align*}
        u(s) &\coloneq \pi\left( \delta_s \right),
      \end{align*}
      is a unitary representation of $\Gamma$.
  \end{enumerate}
\end{proposition}
\begin{proof}\hfill
  \begin{enumerate}[(1)]
    \item Via the universal property of the free vector space, we know that the map $s\mapsto u_s\in \B\left( \mathcal{H} \right)$ extends to a linear map $\pi_u\colon \C\left[ \Gamma \right]\rightarrow \B\left( \mathcal{H} \right)$. Now, we must ensure that this map is faithful to the underlying multiplication structure. Letting $s,t\in \Gamma$ be arbitrary, via the properties of unitary representations, we have
      \begin{align*}
        \pi_u\left( \delta_s\delta_t \right) &= \pi_u\left( \delta_{st} \right)\\
                                             &= u_{st}\\
                                             &= u_su_t\\
                                             &= \pi_u\left( \delta_s \right)\pi_u\left( \delta_t \right)\\
        \pi_u\left( \delta_{s}^{\ast} \right) &= \pi_u\left( \delta_{s^{-1}} \right)\\
                                              &= u_{s^{-1}}\\
                                              &= u_s^{\ast}\\
                                              &= \pi_u\left( \delta_s \right)^{\ast}.
      \end{align*}
      Therefore, via linearity, we obtain that $\pi_u$ is multiplicative and $\ast$-preserving.
    \item Every $\delta_s\in \C\left[ \Gamma \right]$ is a unitary element, and since unital $\ast$-homomorphisms preserve unitary elements (Fact \ref{fact:unitary_preservation}), we know that each $u(s)$ is unitary. Furthermore, for any $s,t\in\Gamma$, we have
      \begin{align*}
        u\left( st \right) &= \pi\left( \delta_{st} \right)\\
                           &= \pi\left( \delta_s\delta_t \right)\\
                           &= \pi\left( \delta_s \right)\pi\left( \delta_t \right)\\
                           &= u(s)u(t),
      \end{align*}
      meaning $u$ is a unitary representation.
  \end{enumerate}
\end{proof}
Now, using the interplay between unitary and unital representations of $\Gamma$ and $\C\left[ \Gamma \right]$ respectively, we may define two special $C^{\ast}$-norms on $\C\left[ \Gamma \right]$. We will investigate the properties of their respective group $C^{\ast}$-algebras.
\begin{proposition}
  Let $\Gamma$ be a group. If $\lambda\colon\Gamma\rightarrow \mathcal{U}\left( \ell_2\left( \Gamma \right) \right)$ is the left-regular representation (Theorem \ref{thm:left_regular_representation}), then $\lambda$ extends to an injective representation $\pi_{\lambda}\colon \C\left[ \Gamma \right]\rightarrow \B\left( \ell_2\left( \Gamma \right) \right)$, given by
  \begin{align*}
    \pi_{\lambda}(a) &= \sum_{s\in\Gamma}a(s)\lambda_s.
  \end{align*}
\end{proposition}
\begin{proof}
  Suppose $\pi_{\lambda}(a) = 0$ for some $a = \sum_{s\in\Gamma}a(s)\delta_s$ in $\C\left[ \Gamma \right]$. Taking the evaluation at $\delta_{e}$, we get
  \begin{align*}
    0 &= \pi_{\lambda}\left( a \right)\left( \delta_e \right)\\
      &= \left( \sum_{s\in\Gamma}a(s)\lambda_s \right)\left( \delta_e \right)\\
      &= \sum_{s\in\Gamma}a(s)\lambda_s\left( \delta_e \right)\\
      &= \sum_{s\in\Gamma}a(s)\delta_s.
  \end{align*}
  Since the $\set{\delta_t}_{t\in\Gamma}$ are linearly independent, we must have that $a(s) = 0$ for all $s\in\Gamma$, meaning $a = 0$.
\end{proof}
\begin{definition}\label{def:reduced_group_cstar_algebra}
  Define the $C^{\ast}$-norm
  \begin{align*}
    \norm{a}_{\lambda} &\coloneq \norm{\pi_{\lambda}(a)}_{\op}
  \end{align*}
  on $\C\left[ \Gamma \right]$.\newline

  The completion of $\C\left[ \Gamma \right]$ with respect to $\norm{\cdot}_{\lambda}$ is known as the \textit{reduced group $C^{\ast}$-algebra}, denoted $C^{\ast}_{\lambda}\left( \Gamma \right)$.
\end{definition}
\begin{proposition}\label{prop:universal_group_cstar_algebra}
  Let $\Gamma$ be a group, and let $\C\left[ \Gamma \right]$ be the group $\ast$-algebra.\newline

  Define the \textit{universal norm} (or maximum norm) on $\C\left[ \Gamma \right]$ by
  \begin{align*}
    \norm{a}_{u} &\coloneq \sup\set{\norm{\pi(a)}_{\op} | \pi\colon \C\left[ \Gamma \right]\rightarrow \B\left( \mathcal{H}_{\pi} \right)\text{ is a unital representation}}.
  \end{align*}
  Then, this is a $C^{\ast}$-norm on $\C\left[ \Gamma \right]$. The completion with respect to this norm yields the \textit{universal group $C^{\ast}$-algebra}, and is denoted $C^{\ast}\left( \Gamma \right)$.
\end{proposition}
\begin{proof}
  First, we show that the quantity
  \begin{align*}
    \norm{a}_{u}\coloneq \sup\set{\norm{\pi(a)}_{\op} | \pi\colon \C\left[ \Gamma \right]\rightarrow \B\left( \mathcal{H}_{\pi} \right)\text{ is a unital representation}}
  \end{align*}
  is finite (i.e., that the universal norm exists).\newline

  Note that for any representation $\pi\colon \C\left[ \Gamma \right]\rightarrow \B\left( \mathcal{H}_{\pi} \right)$, the elements $\pi\left( \delta_s \right)$ are unitary in $\B\left( \mathcal{H}_{\pi} \right)$, meaning they have norm $1$. Therefore, for a finitely supported function $a = \sum_{s\in\Gamma}a(s)\delta_s$, we have
  \begin{align*}
    \norm{\pi(a)}_{\op} &= \norm{\pi\left( \sum_{s\in\Gamma}a(s)\delta_s \right)}_{\op}\\
                        &= \norm{\sum_{s\in\Gamma}a(s)\pi\left( \delta_s \right)}_{\op}\\
                        &\leq \sum_{s\in\Gamma}\norm{a(s)\pi\left( \delta_s \right)}_{\op}\\
                        &= \sum_{s\in\Gamma}\left\vert a(s) \right\vert\norm{\pi\left( \delta_s \right)}_{\op}\\
                        &= \sum_{s\in\Gamma}\left\vert a(s) \right\vert,
  \end{align*}
  so that $\norm{a}_u\leq \sum_{s\in\Gamma}\left\vert a(s) \right\vert < \infty$.\newline

  That $\norm{\cdot}_{u}$ is a $C^{\ast}$-seminorm follows from the fact that for any representation $\pi$ and any $a,b\in \Gamma$, we have
  \begin{align*}
    \norm{\pi\left( ab \right)}_{\op} &= \norm{\pi(a)\pi(b)}_{\op}\\
                                      &\leq \norm{\pi(a)}_{\op}\norm{\pi(b)}_{\op},
  \end{align*}
  so by taking the supremum over all representations, we obtain $\norm{ab}_{u}\leq \norm{a}_{u}\norm{b}_{u}$. Similarly, for any $a\in\Gamma$, we have
  \begin{align*}
    \norm{\pi(a)^{\ast}\pi(a)}_{\op} &= \norm{\pi(a)}_{\op}^2,
  \end{align*}
  so by taking the supremum over all representations, we obtain $\norm{a^{\ast}a} = \norm{a}^2$, showing that it is indeed a $C^{\ast}$-seminorm.\newline

  To verify that $\norm{\cdot}_u$ is a norm, we note that if $\norm{a}_{u} = 0$, then since $\pi_{\lambda}$ is a representation, we must have $\norm{a}_{\lambda} = 0$, and since $\pi_{\lambda}$ is injective, it follows that $a = 0$. Thus, $\norm{\cdot}_u$ is a norm.
\end{proof}
The moniker ``universal'' is apt for the universal $C^{\ast}$-algebra, as it admits a universal property.
\begin{theorem}[{\cite[Proposition 7.2.47]{rainone_analysis}}]
  Let $\Gamma$ be a discrete group. If $u\colon \Gamma\rightarrow \mathcal{U}\left( \mathcal{H} \right)$ is a unitary representation, then there is a contractive $\ast$-homomorphism $\pi_u\colon C^{\ast}\left( \Gamma \right)\rightarrow \B\left( \mathcal{H} \right)$ such that $\pi_u\left( \delta_s \right) = u(s)$ for all $s\in\Gamma$.
  \begin{center}
    % https://tikzcd.yichuanshen.de/#N4Igdg9gJgpgziAXAbVABwnAlgFyxMJZABgBpiBdUkANwEMAbAVxiRAGEA9YAHR7rg4AvnwYwAZjgAUfAOJ0Atgrp8ATlgDmACxwBKEENLpMufIRQBGclVqMWbPgCFRE6X2U4tAY0bAAEiI86tp6BkYgGNh4BERWFjb0zKyIIO50nj4MwACqgWKSMjwe3r4Bapo6+obGUWZEZPHUifYpcorKBjYwUBrwRKDiqhAKSADM1DgQSABM1Ax0AEYwDAAKJtHmIME6IE12ySBMYQNDI4hkIJNIVrZJDjxoWAD6R9Ugg8NjE1PnE3RYDDYWggEAA1sd3qcZt9rn8AUCQeChBQhEA
    \begin{tikzcd}
      C^{\ast}\left(\Gamma\right) \arrow[r, "\pi_u"] & \B\left(\mathcal{H}\right)                          \\
      \Gamma \arrow[r, "u"'] \arrow[u, hook]         & \mathcal{U}\left(\mathcal{H}\right) \arrow[u, hook]
    \end{tikzcd}
  \end{center}
\end{theorem}
\begin{proof}
  From Proposition \ref{prop:unital_unitary_representation}, we know that there is a unital representation $\pi_u\colon \C\left[ \Gamma \right]\rightarrow \B\left( \mathcal{H} \right)$ that extends $u\colon \Gamma\rightarrow \mathcal{U}\left( \mathcal{H} \right)$.\newline

  By the definition of the universal norm, it follows that $\norm{\pi_u(a)}_{\op}\leq \norm{a}_{u}$. The continuous extension $\pi_u\colon C^{\ast}\left( \Gamma \right)\rightarrow \B\left( \mathcal{H} \right)$ is thus a contractive $\ast$-homomorphism.
\end{proof}
\section{Ordering Properties of \texorpdfstring{$C^{\ast}$-Algebras}{C*-Algebras}}%
Recall from Definition \ref{def:positive_operators} that the space $\B\left( \mathcal{H} \right)_{\sa}$ admits an order structure --- we say that an operator is \textit{positive} if, for any $\xi\in \mathcal{H}$, we have $ \iprod{T\left( \xi \right)}{\xi} \geq 0 $. It can be shown that any positive operator is of the form $T = S^{\ast}S$, where $S$ is any operator on $\mathcal{H}$.\newline

Similarly, when we discussed algebras, we discussed a definition of positivity very similar to the case of bounded operators on Hilbert spaces (Definition \ref{def:distinguished_elements_of_algebras}).\newline

In this section, we investigate the ordering aspects of $C^{\ast}$-algebras in depth, and how to apply the properties of their spectra towards understanding ordering and positivity. This will lead naturally to discussions of positive and completely positive (linear) maps between $C^{\ast}$-algebras in the following section, paving the way to the cornucopia of characterizations of amenability that $C^{\ast}$-algebras admit.
\begin{definition}
  If $A$ is a $\ast$-algebra, and $a\in A$, then $a$ can be written as $h + ik$, where $h,k\in A_{\sa}$ are defined by
  \begin{align*}
    h &= \frac{1}{2}\left( a + a^{\ast} \right)\\
    k &= \frac{i}{2}\left( a^{\ast}-a \right).
  \end{align*}
  This is known as the \textit{Cartesian decomposition} of $a$.
\end{definition}
\begin{proposition}[{\cite[Proposition 7.3.61]{rainone_analysis}}]\label{prop:self_adjoint_positive_decomposition}
  Let $A$ be a $C^{\ast}$-algebra, and let $h\in A_{\sa}$. Then, there exist unique positive elements $p,q\in A_{+}$ such that
  \begin{enumerate}[(a)]
    \item $h = p-q$;
    \item $pq = 0$;
    \item $\sigma\left( p \right),\sigma\left( q \right)\subseteq [0,\infty)$.
  \end{enumerate}
\end{proposition}
\begin{proof}
  Assume $A$ is unital. Let $\phi_h\colon C\left( \sigma\left( h \right) \right)\rightarrow C^{\ast}\left( h,1_A \right)$ be the continuous functional calculus at $h$. Since $\sigma\left( h \right)\subseteq \R$ (Proposition \ref{prop:spectra_cstar_algebras}), we consider the continuous functions
  \begin{align*}
    f(t) &= \max\set{t,0};\\
    g(t) &= \max\set{-t,0}.
  \end{align*}
  Note that $fg = 0$ and $\id_{\sigma\left( h \right)} = f-g$. Setting $p = \phi_h(f)$ and $q = \phi_h(g)$. Since $f$ and $g$ are positive, and the continuous functional calculus is a $\ast$-homomorphism, both $p$ and $q$ are positive.\newline

  Moreover, by spectral mapping (Theorem \ref{thm:continuous_functional_calculus}), we have
  \begin{align*}
    \sigma\left( p \right) &= \sigma\left( f\left( h \right) \right)\\
                           &= f\left( \sigma\left( h \right) \right)\\
                           &\subseteq [0,\infty),
  \end{align*}
  and similarly for $\sigma\left( q \right)$. Furthermore,
  \begin{align*}
    pq &= \phi_h\left( f \right)\phi_h\left( g \right)\\
       &= \phi_h\left( fg \right)\\
       &= \phi_h\left( 0 \right)\\
       &= 0
  \end{align*}
  and
  \begin{align*}
    h &= \phi_h\left( \id_{\sigma\left( h \right)} \right)\\
      &= \phi_h\left( f-g \right)\\
      &= \phi_h\left( f \right)-\phi_h\left( g \right)\\
      &= p-q.
  \end{align*}
  Since $\phi_h$ is also isometric, we have
  \begin{align*}
    \norm{h} &= \norm{\phi_h\left( \id_{\sigma\left( h \right)} \right)}\\
             &= \norm{\id_{\sigma\left( h \right)}}_{u}\\
             &= \max\set{\norm{f}_{u},\norm{g}_{u}}\\
             &= \max\set{\norm{\phi_h(f)},\norm{\phi_h(g)}}\\
             &= \max\set{\norm{p},\norm{q}}.
  \end{align*}
  Now we show uniqueness. Let $x,y\in A_{+}$ be such that $h = x-y$, $xy = 0$, and $\sigma\left( x \right),\sigma\left( y \right)\subseteq [0,\infty)$. By induction, we have $h^n = x^n + \left( -y \right)^n$, so for any polynomials $p\in \C\left[ x \right]$ with nonconstant term, we have
  \begin{align*}
    p\left( h \right) = p\left( x \right) + p\left( -y \right).
  \end{align*}
  Since $f(0) = 0$, there is a sequence of polynomials with nonconstant terms, $\left( p_n \right)_n$, such that $\left( p_n \right)_n\rightarrow f$ uniformly on the compact set (Theorem \ref{thm:spectrum_of_cstar_algebras})
  \begin{align*}
    K &= \sigma\left( h \right)\cup \sigma\left( x \right)\cup \sigma\left( -y \right).
  \end{align*}
  Applying the functional calculus at $h$, $x$, and $y$, we have
  \begin{align*}
    p &= f(h)\\
      &= \lim_{n\rightarrow\infty}p_n(h)\\
      &= \lim_{n\rightarrow\infty} \left( p_n\left( x \right)-p_n\left( y \right) \right)\\
      &= f(x) + f(-y).
  \end{align*}
  Now, since $\sigma\left( x \right)\subseteq [0,\infty)$, and $f(t) = t$ on $[0,\infty)$, we have $f(x) = x$. Additionally, since $\sigma\left( -y \right) = -\sigma\left( y \right)$, by Theorem \ref{thm:continuous_functional_calculus} part (3), we have $\sigma\left( -y \right)\subseteq (-\infty,0]$, so $f\left( -y \right) = 0$. Thus, $p = x$, and then $q = p-h = y$.\newline

  If $A$ is nonunital, then we use the functional calculus on $\phi_h\colon C\left( \sigma\left( h \right) \right)\rightarrow C^{\ast}\left( h,1_{\widetilde{A}} \right)$ in the unitization. Since $0\in \sigma\left( h \right)$ and $f(0) = g(0) = 0$, we have that $p$ and $q$ belong to $A$, by Theorem \ref{thm:continuous_functional_calculus} part (6).
\end{proof}
We know that the self-adjoint elements of a $C^{\ast}$-algebra have real spectrum --- it is possible to show that for elements that have positive spectrum (which we will soon show are the positive elements), they are preserved under the traditional operations that define a cone (Example \ref{ex:ordered_vector_space}).
\begin{lemma}[{\cite[Lemma 7.3.63]{rainone_analysis}}]\label{lemma:spectrum_conic_operations}
  Let $A$ be a $C^{\ast}$-algebra.
  \begin{enumerate}[(1)]
    \item If $\sigma\left( a \right)\subseteq [0,\infty)$, and $t\geq 0$, then $\sigma\left( ta \right)\subseteq [0,\infty)$.
    \item Assume $A$ is unital, and let $a\in A_{\sa}$, $t \geq 0$, such that $\norm{a}\leq t$. Then, $\sigma\left( a \right)\subseteq [0,\infty)$ if and only if $\norm{t1_A - a} \leq t$.
    \item Let $a,b\in A_{\sa}$ with $\sigma\left( a \right),\sigma\left( b \right)\subseteq [0,\infty)$. Then, $\sigma\left( a+b \right)\subseteq [0,\infty)$.
  \end{enumerate}
\end{lemma}
\begin{proof}\hfill
  \begin{enumerate}[(1)]
    \item By Theorem \ref{thm:continuous_functional_calculus} (3), we have $\sigma\left( ta \right) = t\sigma\left( a \right)\subseteq [0,\infty)$.
    \item Since $a = a^{\ast}$ and $\norm{a}\leq t$, we have $\sigma\left( a \right)\subseteq [-t,t]$. Let $\phi_a\colon C\left( \sigma\left( a \right) \right)\rightarrow C^{\ast}\left( a,1_A \right)\subseteq A$ be the continuous functional calculus at $a$. We have
      \begin{align*}
        \norm{t1_A - a} &= \norm{\phi_{a}\left( t\1_{\sigma\left( a \right)} - \iota \right)}\\
                        &= \norm{t\1_{\sigma\left( a \right)}-\iota}_{u}\\
                        &\leq 1,
      \end{align*}
      which holds if and only if $\left\vert t - \lambda \right\vert \leq t$ for all $\lambda\in \sigma\left( a \right)$, which holds if and only if $\sigma\left( a \right)\subseteq [0,2t]$.
    \item We may assume that $A$ is unital. From (2), we have that $\norm{\left( \norm{a}1_A \right) - a}\leq \norm{a}$, and $\norm{\left( \norm{b}1_B \right) - b}\leq \norm{b}$. So,
      \begin{align*}
        \norm{\left( \norm{a} + \norm{b} \right)1_A - \left( a+b \right)} &\leq \norm{\left( \norm{a}1_A \right)-a} + \norm{\left( \norm{b}1_B \right) - b}\\
                                                                          &\leq \norm{a} + \norm{b}.
      \end{align*}
      Since $\norm{a +b}\leq \norm{a} + \norm{b}$, setting $t \coloneq \norm{a} + \norm{b}$, we have $\sigma\left( a + b \right)\subseteq [0,\infty)$.
  \end{enumerate}
\end{proof}
\begin{theorem}[{\cite[Theorem 7.3.64]{rainone_analysis}}]
  Let $A$ be a $C^{\ast}$-algebra. Then, $a\in A_{+}$ if and only if $a\in A_{\sa}$ and $\sigma\left( a \right)\subseteq [0,\infty)$.
\end{theorem}
\begin{proof}
  Let $a$ be self-adjoint with $\sigma\left( a \right)\subseteq [0,\infty)$. Note that $a$ is normal. Let $A$ be unital, and let $\phi_a\colon C\left( \sigma\left( a \right) \right)\rightarrow C^{\ast}\left( a,1_A \right)$ be the continuous functional calculus at $a$. Since $\sigma\left( a \right)\subseteq [0,\infty)$, the function $f\left( t \right) = \sqrt{t}$ is well-defined, self-adjoint, and continuous on $\sigma\left( a \right)$. Set $b = \phi_a\left( f \right) = f(a)$. Then,
  \begin{align*}
    a &= \phi_a\left( \iota \right)\\
      &= \phi_a\left( f^2 \right)\\
      &= \phi_a\left( f \right)\phi_a\left( f \right)\\
      &= b^2.
  \end{align*}
  Since $b$ is self-adjoint, as it is the $\ast$-homomorphic image of a self-adjoint element, we have $a = b^{\ast}b\in A_{+}$. If $A$ is nonunital, then we look at the continuous functional calculus on the unitization, $\phi_a\colon C\left( \sigma\left( a \right) \right)\rightarrow C^{\ast}\left( a,1_{\widetilde{A}} \right)$, and note that $f(0) = 0$, so $b\in C^{\ast}\left( a \right)\subseteq A$ by Theorem \ref{thm:continuous_functional_calculus} (6).\newline

  Now, assume $a = b^{\ast}b$ for some $b\in A$. Since $a$ is self-adjoint, we must show that $\sigma\left( a \right)\subseteq [0,\infty)$. We may assume $A$ is unital, and write $a = p-q$, where $p$ and $q$ are in Proposition \ref{prop:self_adjoint_positive_decomposition}. Set $c = bq$.\newline

  Then, we have
  \begin{align*}
    c^{\ast}c &= \left( bq \right)^{\ast}bq\\
              &= q^{\ast}b^{\ast}bq\\
              &= qaq\\
              &= q\left( p-q \right)q\\
              &= -q^3,
  \end{align*}
  since $pq = 0$. Thus, from Theorem \ref{thm:continuous_functional_calculus} (3), and the result from Proposition \ref{prop:self_adjoint_positive_decomposition}, we have
  \begin{align*}
    \sigma\left( c^{\ast}c \right)&=\sigma\left( -q^3 \right)\\
                                  &= -\sigma\left( q \right)^3\\
                                  &\subseteq (-\infty,0].
  \end{align*}
  Thus, from Fact \ref{fact:spectrum_commutative} and Theorem \ref{thm:continuous_functional_calculus} (3), we have $\sigma\left( -cc^{\ast} \right)\subseteq [0,\infty)$.\newline

  We may write $c = h + ik$ via the Cartesian decomposition, and compute
  \begin{align*}
    c^{\ast}c + cc^{\ast} &= \left( h + ik \right)^{\ast}\left( h + ik \right) + \left( h + ik \right)\left( h + ik \right)^{\ast}\\
                          &= 2\left( h^2 + k^2 \right).
  \end{align*}
  Thus, we have that $c^{\ast}c = 2h^2 + 2k^2 + \left( -cc^{\ast} \right)$. Therefore, by Lemma \ref{lemma:spectrum_conic_operations}, we also have that $\sigma\left( c^{\ast}c \right)\subseteq [0,\infty)$, so $\sigma\left( c^{\ast}c \right) = \set{0}$. Thus, by Proposition \ref{prop:normal_spectral_radius}, we have
  \begin{align*}
    \norm{c}^2 &= \norm{c^{\ast}c}\\
               &= r\left( c^{\ast}c \right)\\
               &= 0,
  \end{align*}
  so $-q^3 = c^{\ast}c = 0$. Thus, $q = 0$, so $a = p$ and $\sigma\left( a \right) = \sigma\left( p \right)\subseteq [0,\infty)$.
\end{proof}
We may now show that $A_{+}$ is a cone in $A_{\sa}$, which allows us to define an ordering on $A$.
\begin{corollary}[{\cite[Corollary 7.3.66]{rainone_analysis}}]
  Let $A$ be a $C^{\ast}$-algebra. The collection of positive elements, $A_{+}$, is a generating norm-closed cone in $A_{\sa}$.
\end{corollary}
\begin{proof}
  If $a,-a\in A_{+}$, then $\sigma\left( a \right)\subseteq [0,\infty)$ and $\sigma\left( -a \right) \subseteq [0,\infty)$. By Theorem \ref{thm:continuous_functional_calculus} (3), we also have $\sigma\left( a \right) = -\sigma\left( -a \right)\subseteq (-\infty,0]$, meaning that $\sigma\left( a \right) = \set{0}$. Thus, by Proposition \ref{prop:normal_spectral_radius}, we have $r(a) = \norm{a} = 0$, so $a = 0$.\newline

  From Proposition \ref{prop:self_adjoint_positive_decomposition}, we know that $A_{+}$ generates $A_{\sa}$.\newline

  Now, we show that $A_{+}$ is closed. Assume $A$ is unital, and let $\left( a_n \right)_n$ be a sequence in $A_{+}$ converging to $a\in A$. Then,
  \begin{align*}
    a &= \lim_{n\rightarrow\infty}\left( a_n \right)_n\\
      &= \lim_{n\rightarrow\infty}\left( a_n^{\ast} \right)_n\\
      &= a^{\ast},
  \end{align*}
  so $a$ is self-adjoint. Let $C > 0$ be such that $\norm{a_n}\leq C$ for all $n\geq 1$. Then, $\norm{a}\leq C$, and $\norm{C1_A - a} = \lim_{n\rightarrow\infty}\norm{C1_A - a_n}\leq C$, so $a\in A_{+}$.
\end{proof}
\begin{definition}\label{def:ordering_of_cstar_algebras}
  Let $A$ be a $C^{\ast}$-algebra. We define an ordering on $A_{\sa}$ by $x\leq y$ if and only if $y-x\in A_{+}$.
\end{definition}
\begin{proposition}[{\cite[Proposition 7.3.67]{rainone_analysis}}]
  The ordering on $A_{\sa}$ satisfies the following.
  \begin{enumerate}[(1)]
    \item If $x\leq y$ in $A_{\sa}$, then for any $z\in A$, $z^{\ast}xz \leq z^{\ast}yz$.
    \item If $A$ is unital, $x\in A_{\sa}$, and $s,t\in \R$, then $s1_A \leq x \leq t1_A$ if and only if $\sigma\left( x \right)\subseteq [s,t]$.
    \item If $A$ is unital, and $x\in A_{\sa}$, then $-\norm{x}1_A \leq x \leq \norm{x}1_A$.
    \item For all $a\geq 0$, $a\leq 1_A$ if and only if $\norm{a}\leq 1$.
    \item If $0\leq a \leq b$, then $\norm{a}\leq \norm{b}$.
    \item If $a,b\in A$, then $0\leq b^{\ast}a^{\ast}ab \leq \norm{a}^2b^{\ast}b$.
  \end{enumerate}
\end{proposition}
\begin{proof}\hfill
  \begin{enumerate}[(1)]
    \item If $x\leq y$, then $y-x = a^{\ast}a$ for some $a\in A$. Thus,
      \begin{align*}
        z^{\ast}yz - z^{\ast}xz &= z^{\ast}\left( y-x \right)z\\
                                &= z^{\ast}a^{\ast}az\\
                                &= \left( az \right)^{\ast}az\\
                                &\in A_{+},
      \end{align*}
      so $z^{\ast}xz \leq z^{\ast}yz$.
    \item We have $s1_A - x$ is self-adjoint, so it has a real spectrum. Using Theorem \ref{thm:continuous_functional_calculus} (3), we have
      \begin{align*}
        s1_A \leq x &\leftrightarrow x-s1_A \in A_{+}\\
                    &\Leftrightarrow \sigma\left( x-s1_A \right)\subseteq [0,\infty)\\
                    &\Leftrightarrow \sigma\left( x \right)-s \subseteq [0,\infty)\\
                    &\Leftrightarrow \sigma\left( x \right)\subseteq [s,\infty).
      \end{align*}
      Similarly, if $x\leq t1_A$, then $\sigma\left( x \right)\subseteq (-\infty,t]$. Thus, $s1_A \leq x \leq t1_A$ if and only if $\sigma\left( x \right)\subseteq [s,t]$.
    \item Since $\sigma\left( x \right)\subseteq \left[-\norm{x},\norm{x}\right]$, this follows from (2).
    \item Taking $s = 0$ and $t = 1$ in (2), we have
      \begin{align*}
        0\leq a \leq 1 &\Leftrightarrow \sigma\left( a \right)\subseteq [0,1]\\
                       &\Leftrightarrow r(a)\leq 1\\
                       &\Leftrightarrow \norm{a}\leq 1,
      \end{align*}
      where we use Proposition \ref{prop:normal_spectral_radius}.
    \item From (3), we have that $0\leq a \leq b \leq \norm{b}1_A$, so $\sigma\left( a \right)\subseteq [0,\norm{b}]$. Since $a$ is normal, $\norm{a}\in \sigma\left( a \right)$, so $\norm{a}\leq \norm{b}$.
    \item If $A$ is unital, then by the $C^{\ast}$-identity, we have $a^{\ast}a\leq \norm{a}^21_A$. We then apply (1) with $z = b$ to get $b^{\ast}a^{\ast}ab \leq \norm{a}^2b^{\ast}b$. If $A$ is nonunital, then we operate in the unitization of $A$, $\widetilde{A}$.
  \end{enumerate}
\end{proof}
\begin{corollary}\label{cor:positive_map_order_preserving}
  Let $\phi\colon A\rightarrow B$ be a linear map between $C^{\ast}$-algebras. If $\phi\left( A_{+} \right)\subseteq B_{+}$, then $\phi$ is order-preserving.
\end{corollary}
\begin{proof}
  If $x\leq y$ in $A$, then $y-x\in A_{+}$. Thus, $\phi\left( y-x \right)\in B_{+}$, meaning $\phi\left( y \right)-\phi\left( x \right)\in B_{+}$, whence $\phi\left( x \right)\leq \phi\left( y \right)$.
\end{proof}

\section{Positive Maps in \texorpdfstring{$C^{\ast}$-Algebras}{C*-Algebras}}%
Now that we have discussed positivity of elements in $C^{\ast}$-algebras, we may discuss linear maps that preserve positivity. Note that $\ast$-homomorphisms preserve positivity and order (Fact \ref{fact:positivity_homomorphisms}), so we are interested in linear maps with a more general domain than just $C^{\ast}$-algebras, and specifically a particular class of map that preserves positivity after amplification to matrix algebras.\newline

Recall that a linear map $\phi\colon A\rightarrow B$ between algebras is called positive if $\phi\left( A_{+} \right)\subseteq \phi\left( B \right)_{+}$. In this section, we will prove a result on a class of maps known as \textit{completely positive} maps --- specifically, that all such maps are compressions of $\ast$-homomorphisms to special subspaces of $C^{\ast}$-algebras. This will lend itself to a discussion of nuclearity in $C^{\ast}$-algebras.\newline

A lot of this section will follow the exposition in \cite{completely_bounded_maps_and_operator_algebras}.
\begin{definition}\label{def:operator_systems_operator_spaces}
  Let $A$ be a unital $C^{\ast}$-algebra.
  \begin{itemize}
    \item A linear subspace $M\subseteq A$ is known as an \textit{operator space}.
    \item A linear subspace $S\subseteq A$ is known as an \textit{operator system} if $S$ is an operator space such that, for all $a\in S$, $a^{\ast}\in S$ and $1_A \in S$.
  \end{itemize}
  Note that all unital $C^{\ast}$-algebras are operator spaces and operator systems.
\end{definition}
\begin{remark}\label{rem:positive_elements_operator_system}
  If $S$ is an operator system, we are always able to write $h\in S$ as a difference of two positive elements via the following decomposition:
  \begin{align*}
    h &= \frac{1}{2}\left( \norm{h}1_A + h \right) - \frac{1}{2}\left( \norm{h}1_A - h \right).
  \end{align*}
\end{remark}
The location of what we will study in this section is the amplification of a Hilbert space $\mathcal{H}$.
\begin{definition}
  Let $\mathcal{H}$ be a Hilbert space. The \textit{$n$-fold amplification} of $\mathcal{H}$, denoted $\mathcal{H}^{(n)}$, is the space of all vectors
  \begin{align*}
    \begin{pmatrix}\xi_1\\\vdots\\\xi_n\end{pmatrix}
  \end{align*}
  subject to the inner product
  \begin{align*}
    \iprod{ \begin{pmatrix}\xi_1\\\vdots\\\xi_n\end{pmatrix} }{ \begin{pmatrix}\eta_1\\\vdots\\\eta_n\end{pmatrix} } &= \sum_{j=1}^{n} \iprod{\xi_j}{\eta_j}.
  \end{align*}
\end{definition}
\begin{theorem}
  There is a $\ast$-isomorphism between $\Mat_n\left( \B\left( \mathcal{H} \right) \right)$ (as in Theorem \ref{thm:matrix_algebras_tensor_product}) and the space $\B\left( \mathcal{H}^{(n)} \right)$.
\end{theorem}
\begin{proof}
  For any $\left( T_{ij} \right)_{ij}\in \Mat_n\left( \B\left( \mathcal{H} \right) \right)$, we define the linear map $T\in \B\left( \mathcal{H}^{(n)} \right)$ by
  \begin{align*}
    T\left( \begin{pmatrix}h_1\\\vdots\\h_n\end{pmatrix} \right) &= \begin{pmatrix}\sum_{j=1}^{n}T_{1j}\left( h_j \right) \\ \vdots \\ \sum_{j=1}^{n}T_{nj}\left( h_j \right)\end{pmatrix}.
  \end{align*}
  The map $\left( T_{ij} \right)_{ij} \mapsto T$ is the desired $\ast$-homomorphism.
\end{proof}
\begin{remark}
  The positive cone on $\Mat_n\left( \B\left( \mathcal{H} \right) \right)$ is defined by $\left( T_{ij} \right)_{ij}\in \Mat_n\left( \B\left( \mathcal{H} \right) \right)_{+}$ if and only if, for all $x_1,\dots,x_n\in \mathcal{H}$ and all $v\in \C^n$,
  \begin{align*}
    \iprod{\left( \iprod{T_{ij}\left( x_j \right)}{x_i} \right)_{ij} \left( v \right)}{v} &\geq 0.
  \end{align*}
  In other words, a matrix of operators in $\Mat_n\left( \B\left( \mathcal{H} \right) \right)$ is positive if and only if its matrix representation, $\left( \iprod{T_{ij}\left( x_j \right)}{x_i} \right)_{ij}$, is positive in $\Mat_n\left( \C \right)$, for all such matrix representations.
\end{remark}
As it turns out, any $C^{\ast}$-algebra can be faithfully represented as some $C^{\ast}$-subalgebra of $\B\left( \mathcal{H} \right)$. This follows from the GNS construction, which we state without proof. An outline of a proof can be found in \cite[Section II.6.4]{blackadar_operator_algebras}.
\begin{theorem}[GNS Construction]\label{thm:gns_construction}
  Let $A$ be a $C^{\ast}$-algebra, and let $\phi\colon A\rightarrow \C$ be a positive linear functional.\newline

  Then, there exists a Hilbert space $\mathcal{H}$, a vector $\xi_{\phi}\in \mathcal{H}$, and an isometric $\ast$-homomorphism $\pi\colon A\rightarrow \B\left( \mathcal{H} \right)$ such that $\phi(a) = \iprod{\pi\left( a \right)\left( \xi_{\phi} \right)}{\xi_{\phi}}$ for all $a\in A$, the subspace
  \begin{align*}
    \left[ \pi\left( A \right)\xi_{\phi} \right] \coloneq \set{\pi(a)\left( \xi_{\phi} \right) | a\in A}
  \end{align*}
  is dense in $\mathcal{H}$, and $\norm{\phi}_{\op} = \norm{\xi_{\phi}}^2$. We call the triplet $\left( \pi,\xi_{\phi},\mathcal{H} \right)$ a \textit{GNS representation} of the $C^{\ast}$-algebra $A$.
\end{theorem}
Due to the GNS construction, we are able to turn $\Mat_n\left( A \right)$ into a $C^{\ast}$-algebra by identifying $A$ as a $\ast$-subalgebra of $\Mat_n\left( \B\left( \mathcal{H} \right) \right)$, where $\mathcal{H}$ is the Hilbert space in the GNS construction of $A$. Letting $\pi_{n}\colon \Mat_n\left( A \right)\rightarrow \Mat_n\left( \B\left( \mathcal{H} \right) \right)$ be the isometric (hence injective, from Theorem \ref{thm:cstar_homomorphism_isometric_injective}), we are then able to define a $C^{\ast}$-norm on $\Mat_n\left( A \right)$ by
\begin{align*}
  \norm{\left( a_{ij} \right)_{ij}} &= \norm{\pi_n\left( \left( a_{ij} \right)_{ij} \right)}_{\op}.
\end{align*}
Furthermore, since all $C^{\ast}$-norms are equivalent on any $C^{\ast}$-algebra (Proposition \ref{prop:cstar_norm_equivalent}), this uniquely defines the $C^{\ast}$-norm on $\Mat_n\left( A \right)$. Thus, any $C^{\ast}$-algebra carries with it a family of canonically defined norms (and order properties) via the amplification to matrix algebras.
\subsection{Amplifying Positive Maps}%
\begin{definition}
  Given two $C^{\ast}$-algebras, $A$ and $B$, and a (linear) map $\phi\colon A\rightarrow B$, we define the \textit{$n$-fold amplification} of $\phi$, $\phi_n\colon \Mat_n\left( A \right)\rightarrow \Mat_n\left( B \right)$, by
  \begin{align*}
    \phi_n\left( \left( a_{ij} \right)_{ij} \right) &= \left( \phi\left( a_{ij} \right) \right)_{ij}.
  \end{align*}
  If $\phi$ is a positive map, then we say $\phi$ is \textit{$n$-positive} if $\phi_n$ is positive. We say $\phi$ is \textit{completely positive} if, for all $n$, $\phi_n$ is positive, in that $\phi_n$ maps positive elements of $\Mat_n\left( A \right)$ to the positive elements of $\Mat_n\left( B \right)$.\newline

  If $\phi$ is contractive, then we say $\phi$ is \textit{$n$-contractive} if $\phi_n$ is contractive, and if $\phi_n$ is contractive for all $n$, we say $\phi$ is completely contractive.\newline

  We say $\phi$ is \textit{completely bounded} if there exists some $C > 0$ such that for all $n$, the canonical map $\phi_n\colon \Mat_n\left( S \right)\rightarrow \Mat_n\left( B \right)$ is such that $\norm{\phi_n}_{\op}\leq C$. We write
  \begin{align*}
    \norm{\phi}_{\cb}\coloneq \sup_{n} \norm{\phi_n}_{\op}
  \end{align*}
\end{definition}
One of the useful facts about positive maps is that they are always bounded linear.
\begin{proposition}
  Let $S\subseteq A$ be an operator system, and let $B$ be a $C^{\ast}$-algebra. If $\phi\colon S\rightarrow B$ is a positive map, then $\phi$ is bounded, with
  \begin{align*}
    \norm{\phi}_{\op} &\leq 2\norm{\phi\left( 1_A \right)}.
  \end{align*}
\end{proposition}
\begin{proof}
  If $p$ is positive, then $0\leq p \leq \norm{p}1_A$, so since $\phi$ is positive, Corollary \ref{cor:positive_map_order_preserving} gives $0\leq \phi\left( p \right)\leq \norm{p}\phi\left( 1_A \right)$. Therefore, we have
  \begin{align*}
    \norm{\phi\left( p \right)} &\leq \norm{p}\norm{\phi\left( 1_A \right)}.
  \end{align*}
  When $p_1,p_2$ are positive, we have $\norm{p_1 - p_2}\leq \max\set{\norm{p_1},\norm{p_2}}$. Now, if $h$ is self-adjoint, we use Remark \ref{rem:positive_elements_operator_system} and the fact that $\phi$ is order-preserving to obtain
  \begin{align*}
    \phi(h) &= \frac{1}{2}\phi\left( \norm{h}1_A + h \right) + \frac{1}{2}\phi\left( \norm{h}1_A - h \right),
  \end{align*}
  which is the difference of two positive elements in $B$, meaning
  \begin{align*}
    \norm{\phi(h)} &\leq \frac{1}{2}\max\set{\norm{\phi\left( \norm{h}1_A + h \right)},\norm{\phi\left( \norm{h}1_A - h \right)}}\\
                   &\leq \norm{h}\norm{\phi\left( 1_A \right)}.
  \end{align*}
  Now, if $a\in S$ is arbitrary, we use the Cartesian decomposition to write $a = h + ik$, where
  \begin{align*}
    h &= \frac{1}{2}\left( a + a^{\ast} \right)\\
    k &= \frac{i}{2}\left( a^{\ast}-a \right),
  \end{align*}
  and obtain
  \begin{align*}
    \norm{\phi(a)} &\leq \norm{\phi(h)} + \norm{\phi(k)}\\
                   &\leq 2\norm{a}\norm{\phi\left( 1_A \right)}.
  \end{align*}
  Thus,
  \begin{align*}
    \norm{\phi}_{\op} &\leq 2\norm{\phi\left( 1_A \right)}.
  \end{align*}
\end{proof}
\begin{remark}
  If $S\subseteq A$ is an operator system, $B$ a unital $C^{\ast}$-algebra, and $\phi\colon S\rightarrow B$ is a positive map, then $\phi$ is self-adjoint --- i.e., $\phi\left( x^{\ast} \right) = \phi\left( x \right)^{\ast}$.\newline

  Furthermore, if $\phi\colon S\rightarrow B$ a unital contraction, then $\phi$ is positive --- the proof requires the fact that if $f\colon S\rightarrow \C$ is a linear functional with $f\left(1_A\right) = \norm{f} = 1$ (i.e., that $f$ is a state, see Definition \ref{def:state_linear_functional}), then $f(a)\in \overline{\conv}\left( \sigma\left( a \right) \right)$ for any normal element $a\in A$.
\end{remark}
We begin by establishing our first, ``one-dimensional'' relationship(s) between contractive maps and positive extensions.
\begin{proposition}\label{prop:extension_of_contractions}
  Let $A$ be a unital $C^{\ast}$-algebra and $M\subseteq A$ a unital operator space. If $B$ is a unital $C^{\ast}$-algebra, and $\phi\colon M\rightarrow B$ is a unital contraction, then the map $\widetilde{\phi}\colon M + M^{\ast}\rightarrow B$, given by $\widetilde{\phi}\left( a + b^{\ast} \right) = \phi\left( a \right) + \phi\left( b \right)^{\ast}$, is a well-defined, unique, positive extension of $\phi$ to $M + M^{\ast}$.
\end{proposition}
\begin{proof}
  We only need to show that this formula yields a well-defined, positive map, as its uniqueness of $\widetilde{\phi}$ follows from the fact that $\phi$ is self-adjoint.\newline

  To show that $\widetilde{\phi}$ is well-defined, it is enough to prove that if $a,a^{\ast}\in M$, then $\phi\left( a^{\ast} \right) = \phi\left( a \right)^{\ast}$. Set
  \begin{align*}
    S_1 &= \set{a | a\in M,a^{\ast}\in M}.
  \end{align*}
  Then, $S_1$ is an operator system, and $\phi$ is a unital, contractive map on $S_1$, hence positive. Since positive maps are self-adjoint, $\widetilde{\phi}$ is well-defined.\newline

  Now, we show that $\widetilde{\phi}$ is positive. It is sufficient to assume that $B = \B\left( \mathcal{H} \right)$ (else, represent faithfully represent $B$ using the GNS construction), and fix $x\in \mathcal{H}$ with $\norm{x} = 1$. We set $\widetilde{\rho}\left( a \right) = \iprod{\left( \widetilde{\phi}\left( a \right) \right)(x)}{x}$. We will show that $\widetilde{\rho}$ is positive.\newline

  Let $\rho\colon M\rightarrow \C$ be defined by $\rho(a) = \iprod{\left( \phi(a) \right)(x)}{x}$. Then, $\norm{\rho}_{\op} = 1$ (take $a = 1_A$), so by Theorem \ref{thm:hb_continuous_extension}, there is a norm-preserving extension $\rho_1\colon M + M^{\ast}\rightarrow \C$. However, since $\rho_1$ is positive, as it is a unital contraction on an operator space, we have
  \begin{align*}
    \rho_1\left( a + b^{\ast} \right) &= \rho\left( a \right) + \overline{\rho\left( b \right)}\\
                                      &= \widetilde{\rho}\left( a + b^{\ast} \right),
  \end{align*}
  so $\widetilde{\rho}$ is positive.
\end{proof}
Now, we may begin investigating amplifications of positive maps to the matrix algebras. We start a useful lemma characterizing positive matrices in $\Mat_n\left( A \right)$.
\begin{lemma}\label{lemma:positive_elements_from_matrix_algebras}
  Let $A$ be a unital $C^{\ast}$-algebra. Then, the following hold.
  \begin{enumerate}[(i)]
    \item If $a\in A$, then $\norm{a}\leq 1$ if and only if
      \begin{align*}
        M &= \begin{pmatrix}1_A & a\\a^{\ast} & 1_A\end{pmatrix}.
      \end{align*}
      is positive in $\Mat_2\left( A \right)$.
    \item If $a,b\in A$, then 
      \begin{align*}
        M &= \begin{pmatrix}1_A & a \\ a^{\ast} & b\end{pmatrix}
      \end{align*}
      is positive in $\Mat_2\left( A \right)$ if and only if $a^{\ast}a \leq b$.
  \end{enumerate}
\end{lemma}
\begin{proof}\hfill
  \begin{enumerate}[(i)]
    \item Let $A$ be represented by $\pi\colon A\rightarrow \B\left( \mathcal{H} \right)$ for some Hilbert space $\mathcal{H}$, and set $T = \pi\left( a \right)$. Now, if $\norm{T}_{\op}\leq 1$, then for any $x,y\in \mathcal{H}$, we have
      \begin{align*}
        \iprod{ \begin{pmatrix}I & T \\ T^{\ast} & I\end{pmatrix} \begin{pmatrix}x\\y\end{pmatrix} }{ \begin{pmatrix}x\\y\end{pmatrix} } &= \iprod{x}{x} + \iprod{T\left( y \right)}{x} + \iprod{x}{T\left( y \right)} + \iprod{y}{y}\\
                                 &\geq \norm{x}^2 - 2\norm{T}_{\op}\norm{x}\norm{y} + \norm{y}^2\\
                                 &\geq 0.
      \end{align*}
      Conversely, if $\norm{T}_{\op} > 1$, then there exist unit vectors $x$ and $y$ such that $ \iprod{T\left( y \right)}{x} < -1 $, meaning that the inner product is negative.
    \item Assume $b\geq a^{\ast}a$, where we represent $a,b$ as elements $A,B\in \B\left( \mathcal{H} \right)$. Then, for all $y\in \mathcal{H}$, we have
      \begin{align*}
        \iprod{\left( B-A^{\ast}a \right)\left( y \right)}{y} &\geq 0,\\
        \iprod{B\left( y \right)}{y} &\geq \norm{A\left( y \right)}^2.
      \end{align*}
      Now, for $ \begin{pmatrix}x\\y\end{pmatrix}\in \mathcal{H}^{(2)} $, we have
      \begin{align*}
        \iprod{ \begin{pmatrix}I & A \\ A^{\ast} & B\end{pmatrix} \begin{pmatrix}x\\y\end{pmatrix}}{ \begin{pmatrix}x\\y\end{pmatrix} } &= \iprod{x}{x} + \iprod{A\left( y \right)}{x} + \iprod{A^{\ast}\left( x \right)}{y} + \iprod{B\left( y \right)}{y}\\
                                   &\geq \iprod{x}{x} + \iprod{A\left( y \right)}{x} + \iprod{A^{\ast}\left( x \right)}{y} + \norm{A\left( y \right)}^2\\
                                   &= \norm{x}^2 + \norm{A\left( y \right)}^2 + 2\re\left( \iprod{A\left( y \right)}{x} \right)\\
                                   &\geq \norm{x}^2 + \norm{A\left( y \right)}^2 - 2\norm{A\left( y \right)}\norm{x}\\
                                   &\geq 0,
      \end{align*}
      meaning the matrix is positive.\newline

      Now, suppose $B\ngeq A^{\ast}A$. Then, there is some $y\in \mathcal{H}$ such that $ \iprod{\left( B-A^{\ast}A \right)\left( y \right)}{y}  < 0$, giving $ \iprod{B\left( y \right)}{y} < \norm{A(y)}^2$. We may scale $y$ such that $\norm{A\left( y \right)}^2 = 1$, and setting $x = -A\left( y \right)$, we get
      \begin{align*}
        \iprod{ \begin{pmatrix}I & A \\ A^{\ast} & B\end{pmatrix} \begin{pmatrix}x\\y\end{pmatrix}}{ \begin{pmatrix}x\\y\end{pmatrix} } &= \iprod{x}{x} + \iprod{A\left( y \right)}{x} + \iprod{A^{\ast}\left( x \right)}{y} + \iprod{B\left( y \right)}{y}\\
                                 &= \iprod{x}{x} + \iprod{A\left( y \right)}{x} + \iprod{x}{A^{\ast}y} + \iprod{B\left( y \right)}{y}\\
                                 &= \iprod{-A\left( y \right)}{-A\left( y \right)} + \iprod{A\left( y \right)}{-A\left( y \right)} + \iprod{-A\left( y \right)}{A\left( y \right)} + \iprod{B\left( y \right)}{y}\\
                                 &= \norm{A\left( y \right)}^2 - 2\norm{A\left( y \right)}^2 + \iprod{B\left( y \right)}{y}\\
                                 &= -1 + \iprod{B\left( y \right)}{y}\\
                                 &< -1 + \norm{A\left( y \right)}^2\\
                                 &= 0.
      \end{align*}
      Thus, the matrix is not positive.
  \end{enumerate}
\end{proof}
\subsection{Completely Positive and Completely Contractive Maps}%
Now, we may discuss the important interplay between positive maps and contractive maps through the amplification to matrix algebras.
\begin{proposition}\label{prop:two_positive_contractive}
  Let $S$ be an operator system, $B$ a unital $C^{\ast}$-algebra, and $\phi\colon S\rightarrow B$ unital $2$-positive map. Then, $\phi$ is contractive.
\end{proposition}
\begin{proof}
  Let $a\in S$ be such that $\norm{a}\leq 1$. Then,
  \begin{align*}
    \phi_2 \begin{pmatrix}1_A & a \\ a^{\ast} & 1_A\end{pmatrix} &= \begin{pmatrix} 1 & \phi\left( a \right) \\ \phi\left( a \right)^{\ast} & 1\end{pmatrix},
  \end{align*}
  is a positive matrix by the $2$-positivity of $\phi$, so by Lemma \ref{lemma:positive_elements_from_matrix_algebras} (i), we have that $\norm{\phi(a)}\leq 1$, meaning $\phi$ is contractive.
\end{proof}
\begin{proposition}[Cauchy--Schwarz for $2$-Positive Maps]
  Let $A$ and $B$ be unital $C^{\ast}$-algebras, and let $\phi\colon A\rightarrow B$ be a unital $2$-positive map. Then,
  \begin{align*}
    \phi\left( a \right)^{\ast}\phi\left( a \right)\leq \phi\left( a^{\ast}a \right).
  \end{align*}
\end{proposition}
\begin{proof}
  Note that
  \begin{align*}
    \begin{pmatrix}1 & a \\ 0 & 0\end{pmatrix}^{\ast} \begin{pmatrix}1 & a \\ 0 & 0\end{pmatrix} &= \begin{pmatrix}1 & a \\ a^{\ast} & a^{\ast}a\end{pmatrix}\\
                     &\geq 0,
  \end{align*}
  so by the $2$-positivity of $\phi$, we have
  \begin{align*}
    \begin{pmatrix}1 & \phi\left( a \right) \\ \phi\left( a \right)^{\ast} & \phi\left( a^{\ast}a \right)\end{pmatrix} &\geq 0,
  \end{align*}
  so $\phi\left( a^{\ast}a \right) \geq \phi\left( a \right)^{\ast}\phi\left( a \right)$ by Lemma \ref{lemma:positive_elements_from_matrix_algebras} (ii).
\end{proof}
\begin{proposition}
  Let $A$ and $B$ be unital $C^{\ast}$-algebras, and let $M\subseteq A$ be a unital operator space. Let $S = M + M^{\ast}$, where $M^{\ast}$ denotes pointwise involution of elements of $M$. Then, if $\phi\colon M\rightarrow B$ is unital and $2$-contractive, then $\widetilde{\phi}\colon S\rightarrow B$, given by $\widetilde{\phi}\left( a + b^{\ast} \right) = \phi\left( a \right) + \phi\left( b \right)^{\ast}$, is $2$-positive and contractive.
\end{proposition}
\begin{proof}
  Since $\phi$ is contractive, the map $\widetilde{\phi}$ is positive and well-defined by Proposition \ref{prop:extension_of_contractions}. Note that
  \begin{align*}
    \Mat_2\left( S \right) &= \Mat_2\left( M \right) + \Mat_2\left( M \right)^{\ast}.
  \end{align*}
  The map $\widetilde{\phi}\colon M + M^{\ast}\rightarrow B$ extends via a similar scheme to $\widetilde{\phi}_2\colon \Mat_2\left( M \right) + \Mat_2\left( M \right)^{\ast}$. Since $\phi_2$ is contractive, $\widetilde{\phi}_2$ is positive, so $\widetilde{\phi}$ is contractive by Proposition \ref{prop:two_positive_contractive}.
\end{proof}
\begin{proposition}
  Let $A$ and $B$ be unital $C^{\ast}$-algebras, let $M\subseteq A$ be a unital subspace, and let $S = M + M^{\ast}$. If $\phi\colon M\rightarrow B$ is unital and completely contractive, then $\widetilde{\phi}\colon S\rightarrow B$ is completely positive and completely contractive.
\end{proposition}
\begin{proof}
  Since $\phi_n$ is unital and contractive, $\widetilde{\phi}_n$ is positive by Proposition \ref{prop:extension_of_contractions}. Now, since $\left( \widetilde{\phi}_n \right)_{2}$ is positive, $\widetilde{\phi}_n$ is contractive by Proposition \ref{prop:two_positive_contractive}. 
\end{proof}
\begin{remark}
  Since $\Mat_{2}\left( \Mat_n\left( A \right) \right)\cong \Mat_{2n}\left( A \right)$ are $\ast $-isomorphic, since $C^{\ast}$-algebras have a unique norm, the norm on $\Mat_{2}\left( \Mat_n\left( A \right) \right)$ and the norm on $\Mat_{2n}\left( A \right)$ are equal.
\end{remark}
\begin{example}
  If $A$ and $B$ are $C^{\ast}$-algebras, and $\pi\colon A\rightarrow B$ is a $\ast$-homomorphism, then $\pi$ is both completely positive and completely contractive. This follows from the fact that each $\pi_n\colon \Mat_n\left( A \right)\rightarrow \Mat_n\left( B \right)$ is a $\ast$-homomorphism that is defined canonically, and $\ast$-homomorphisms are positive and contractive maps.
\end{example}
\begin{remark}
  As it turns out, if $\phi\colon A\rightarrow \B\left( \mathcal{H} \right)$ is a $\ast$-homomorphism, and $v_i\colon \mathcal{H}\rightarrow \mathcal{K}$ are bounded linear operators, then all completely bounded maps are of the form $\phi\left( a \right) = v_2^{\ast}\pi(a)v_1$ for some $v_1,v_2$.\newline

  Furthermore, Stinespring's Dilation Theorem (Theorem \ref{thm:stinespring_dilation}) says that all completely positive maps are of the form $\phi\left( a \right) = V^{\ast}\pi(a)V$ for some $\ast$-homomorphism $\pi\colon A\rightarrow \B\left( \mathcal{H} \right)$ and bounded linear map $V\colon \mathcal{H}\rightarrow \mathcal{K}$. Additionally, all ``minimal'' such representations are unitarily equivalent.
\end{remark}
One of the most useful facts about completely positive maps is that they are always completely bounded.
\begin{proposition}\label{prop:norm_of_completely_positive_map}
  Let $S\subseteq A$ be an operator system, $B$ a $C^{\ast}$-algebra, and $\phi\colon S\rightarrow B$ a completely positive map. Then, $\phi$ is completely bounded, and $\norm{\phi}_{\op} = \norm{\phi(1)} = \norm{\phi}_{\cb}$.
\end{proposition}
\begin{proof}
  We can see by definition that $\norm{\phi(1)}\leq \norm{\phi}_{\op}\leq \norm{\phi}_{\cb}$. It is thus sufficient to show that $\norm{\phi}_{\cb}\leq \norm{\phi}_{\op}$.\newline

  Let $T = \left( t_{ij} \right)_{ij}\in \Mat_n\left( S \right)$ be such that $\norm{T}\leq 1$. Let $I_n$ be the identity matrix in $\Mat_n\left( A \right)$. Since the matrix
  \begin{align*}
    M &= \begin{pmatrix}I_n & A \\ A^{\ast} & I_n\end{pmatrix}
  \end{align*}
  is positive by Lemma \ref{lemma:positive_elements_from_matrix_algebras} (i), we have that the application of $\phi_{2n}$ on the $2\times 2$ repeated copy of $M$, and 
  \begin{align*}
    \phi_{2n}\left( \begin{pmatrix}I_n & A \\ A^{\ast} & I_n\end{pmatrix}_{1,2} \right) &= \begin{pmatrix} \begin{pmatrix}\phi_n\left( I_n \right) & \phi_n\left( A \right) \\ \phi_n\left( A \right)^{\ast} & \phi_n\left( I_n \right)\end{pmatrix}\end{pmatrix}
  \end{align*}
  is positive. Therefore, we have $\norm{\phi_n\left( A \right)}\leq \norm{\phi_n\left( I_n \right)} = \norm{\phi(1)}$.
\end{proof}
A useful fact is that bounded linear functionals are completely bounded, and positive linear functionals are completely positive.
\begin{proposition}
  Let $S\subseteq A$ be an operator system, and let $f\colon S\rightarrow \C$ be a bounded linear functional. Then, $\norm{f}_{\cb} = \norm{f}_{\op}$, and if $f$ is positive, then $f$ is completely positive.
\end{proposition}
\begin{proof}
  Let $\left( a_{ij} \right)_{ij}\in \Mat_n\left( S \right)$, and let $x,y$ be unit vectors in $\C^n$. Then,
  \begin{align*}
    \left\vert \iprod{f_n\left( \left( a_{ij} \right)_{ij} \right)\left( x \right)}{y} \right\vert &= \left\vert \sum_{i,j=1}^{n}f\left( a_{ij} \right)x_j\overline{y_i} \right\vert\\
                                                                                                   &= \left\vert f\left( \sum_{i,j=1}^{n} a_{ij}x_j \overline{y_i} \right) \right\vert\\
                                                                                                   &\leq \norm{f}_{\op}\norm{\sum_{i,j=1}^{n}a_{ij}x_j\overline{y_i}}.
  \end{align*}
  Now, we must show that the latter element has a norm less than $\norm{\left( a_{ij} \right)_{ij}}$. Note that the sum is equal to the $(1,1)$ entry of
  \begin{align*}
    M &= \begin{pmatrix}\overline{y_1}1_A & \cdots & \overline{y_n}1_A \\ 0 & \cdots & 0 \\ \vdots & \ddots & \vdots \\ 0 & \cdots & 0\end{pmatrix} \begin{pmatrix}a_{11} & \cdots & a_{1n} \\ \vdots & \ddots & \vdots \\ a_{n1} & \cdots & a_{nn}\end{pmatrix} \begin{pmatrix}x_1 1_A & 0 & \cdots & 0 \\ \vdots & \vdots & \ddots & \vdots \\ x_n1_A  & 0 & \cdots & 0\end{pmatrix},\label{eq:matrix_positive_linear_functional}\tag{\textasteriskcentered}
  \end{align*}
  whose outer factors each have norm $1$. This shows that $f$ is completely bounded.\newline

  To see that $f$ is completely positive, we must show that
  \begin{align*}
    \iprod{f_n\left( \left( a_{ij} \right)_{ij} \right)\left( x \right)}{x} &= f\left( \sum_{i,j=1}^{n}a_{ij}x_j\overline{x_i} \right)
  \end{align*}
  is positive whenever $\left( a_{ij} \right)_{ij}$ is positive. However, using $x = y$ in \eqref{eq:matrix_positive_linear_functional}, we find that $f$ is evaluated at the $(1,1)$ entry of a positive matrix, hence is positive.
\end{proof}
Now, we discuss a little bit about the case where $\phi\colon S\rightarrow C\left( X \right)$ is a map between an operator system and a commutative $C^{\ast}$-algebra (Theorem \ref{thm:gelfand_naimark}), where $X$ is a compact Hausdorff space. Every element $F = \left( f_{ij} \right)_{ij}\in \Mat_n\left( C\left( X \right) \right)$ is a continuous matrix-valued function with pointwise multiplication and $\ast$-operations. In order to convert the space $\Mat_n\left( C\left( X \right) \right)$ into a $C^{\ast}$-algebra, we define the norm $\norm{F} = \sup_{x\in X}\norm{F(x)}$, and since $C^{\ast}$-norms are unique, this is the only way to create a $C^{\ast}$-algebra.\newline

This allows us to establish that if the range of such a $\phi$ is commutative, then $\phi$ is completely bounded, and if $\phi$ is positive, then $\phi$ is completely positive.
\begin{theorem}\label{thm:commutative_range_completely_positive}
  Let $S$ be an operator system, $\phi\colon S\rightarrow C(X)$ a bounded linear map. Then, $\norm{\phi}_{\cb} = \norm{\phi}_{\op}$, and if $\phi$ is positive, then $\phi$ is completely positive.
\end{theorem}
\begin{proof}
  For any $x\in X$, define $\phi^{x}\colon S\rightarrow \C$ by $\phi^x(a) = \phi(a)(x)$. Then, we have
  \begin{align*}
    \norm{\phi_n}_{\op} &= \sup_{x\in X}\norm{\phi_n^{x}}\\
                        &= \sup_{x\in X}\norm{\phi^{x}}\\
                        &= \norm{\phi}_{\op}.
  \end{align*}
  Similarly, $\phi_n\left( \left( a_{ij} \right)_{ij} \right)$ is positive if and only if $\phi_n^{x}\left( \left( a_{ij} \right)_{ij} \right)$ is positive for all $x\in X$.
\end{proof}
  There is a converse --- i.e., if the domain is a commutative $C^{\ast}$-algebra, then any positive map is completely positive. However, this direction is a little bit more involved. We will state it without proof.
\begin{theorem}\label{thm:commutative_domain_completely_positive}
  Let $\phi\colon C\left( X \right)\rightarrow B$ be a positive map. Then, $\phi$ is completely positive.
\end{theorem}
\subsection{Extending and Approximating Completely Positive Maps}%
Now that we've established some important properties that underly completely positive, completely contractive, and completely bounded maps, we discuss two major theorems regarding completely positive maps, though we do not state their proofs.
\begin{theorem}[Stinespring's Dilation Theorem]\label{thm:stinespring_dilation}
  Let $A$ be a unital $C^{\ast}$-algebra, and let $\phi\colon A\rightarrow \B\left( \mathcal{H} \right)$ be a completely positive map. Then, there exists a Hilbert space, $\mathcal{K}$, a unital $\ast$-homomorphism $\pi\colon A\rightarrow \B\left( \mathcal{K} \right)$, and a bounded operator $V\colon \mathcal{H}\rightarrow \mathcal{K}$, with $\norm{\phi(1)} = \norm{V}_{\op}^2$, such that
  \begin{align*}
    \phi(a) &= V^{\ast}\pi(a)V.
  \end{align*}
\end{theorem}
The proof can be found in \cite[Chapter 4]{completely_bounded_maps_and_operator_algebras}, and follows a similar line of argument to the proof of the GNS construction, where a sesquilinear form is defined using the completely positive map $\phi$.\newline

Arveson's Extension Theorem is one of the other major foundational results in the theory of completely positive maps.
\begin{theorem}[Arveson's Extension Theorem]\label{thm:arveson}
  Let $A$ be a $C^{\ast}$-algebra, $S\subseteq A$ an operator system, and $\phi\colon S\rightarrow \B\left( \mathcal{H} \right)$ a completely positive map. Then, there exists a completely positive map $\psi\colon A\rightarrow \B\left( \mathcal{H} \right)$ that extends $\phi$. In other words, $\B\left( \mathcal{H} \right)$ is injective in the category of operator systems whose morphisms are completely positive maps:
  \begin{center}
    % https://tikzcd.yichuanshen.de/#N4Igdg9gJgpgziAXAbVABwnAlgFyxMJZABgBpiBdUkANwEMAbAVxiRGJAF9T1Nd9CKAIzkqtRizYBlLjxAZseAkQBMo6vWatEIAIKzeigURFCxmyToA6VgEI2GMAGY4AFDYC2dHAAsAxozAABKcNgBOWADmPjgAlFxiMFCR8ESgTmEQHkhkIDgQSELc6ZnZiCJ5BYgAzNQMdABGMAwACnxKgiAR0TggGhLaIDZoPlgGIBlZhdT5SCrFE6VzM1W1IPVNre3GOo4ufeJabMPYCZxAA
    \begin{tikzcd}
    0 \arrow[r] & S \arrow[d, "\phi"'] \arrow[r] & A \arrow[ld, "\psi"] \\
                & \B\left(\mathcal{H}\right)     &                     
    \end{tikzcd}
  \end{center}
\end{theorem}
Arveson's Extension Theorem is proven in \cite[Chapter 7]{completely_bounded_maps_and_operator_algebras} by proving for the special case of $\mathcal{H} = \C^n$, and then using a compactness argument.\newline

Next, we establish basic properties of maps between $C^{\ast}$-algebras that can be approximated by ``factoring'' through finite-rank completely positive maps. These are known as nuclear maps, and they play an integral role in establishing amenability. The exposition from here on out will follow (loosely) the results from \cite{brown_and_ozawa}.
\begin{definition}
  Let $\theta\colon A\rightarrow B$ be a map between $C^{\ast}$-algebras. We say $\theta$ is nuclear if there exist completely positive contractions $\varphi_n\colon A\rightarrow \Mat_{k(n)}\left( \C \right)$ and $\psi_n\colon \Mat_{k(n)}\left( \C \right)\rightarrow B$, such that
  \begin{align*}
    \norm{\theta(a) - \psi_n\circ \varphi_n\left( a \right)}\xrightarrow{n\rightarrow\infty} 0
  \end{align*}
  for all $a\in A$.
\end{definition}
\begin{remark}
Every nuclear map is completely positive. Furthermore, we assume that all of $A,B,\theta,\varphi_n,\psi_n$ are unital maps, as we will focus on establishing nuclearity with the group $C^{\ast}$-algebra(s), which are unital $C^{\ast}$-algebras.
\end{remark}
One of the most important facts about nuclearity is the fact that it is preserved under composition.
\begin{proposition}\label{prop:nuclearity_composition}
  Let $\theta\colon A\rightarrow B$ and $\sigma\colon B\rightarrow C$ be completely positive maps between $C^{\ast}$-algebras. If one of $\sigma$ or $\theta$ is nuclear, then the composition $\sigma\circ\theta$ is nuclear.
\end{proposition}
\begin{proof}
  Assume that $\theta$ is nuclear. Then, there exist $\varphi_n\colon A\rightarrow \Mat_{k(n)}\left( \C \right)$ and $\psi_n\colon \Mat_{k(n)}\left( \C \right)\rightarrow B$ such that $\norm{\theta(a) - \psi_n\circ\varphi_n(a)} \rightarrow 0$. This means that for any $\ve > 0$, there exists a finite set $F\subseteq A$ such that for all $a\in F$,
  \begin{align*}
    \norm{\theta(a) - \psi_n\circ\varphi_n(a)} < \ve.
  \end{align*}
  Now, note that
  \begin{align*}
    \sigma\circ\theta(a) - \sigma\circ\psi_n\circ\varphi_n(a) &= \sigma\left( \theta(a) - \psi_n\circ \varphi_n(a) \right).
  \end{align*}
  We define $\psi_n' \coloneq \sigma\circ \psi_n\colon \Mat_{k(n)}\left( \C \right)\rightarrow C$, and see that, for all $a\in F$,
  \begin{align*}
    \norm{\sigma\circ\theta\left( a \right) - \psi_n'\circ\varphi_n\left( a \right)} &= \norm{\sigma\left( \theta(a) - \psi_n\circ\varphi_n\left( a \right) \right)}\\
                                                                                     &= \norm{\sigma\left( \left( 1_B \right)\left( \theta\left( a \right) - \psi_n\circ\varphi_n\left( a \right) \right) \right)}\\
                                                                                     &\leq \norm{\sigma\left(1_B\right)}\norm{\theta(a) - \psi_n\circ\varphi_n\left( a \right)}\\
                                                                                     &< \norm{\sigma\left(1_B\right)} \ve.
  \end{align*}
  Since $\norm{\sigma\left( 1_B \right)} = \norm{\sigma}_{\op} < \infty$ (Proposition \ref{prop:norm_of_completely_positive_map}), we see that $\sigma\circ\theta$ is nuclear for all $a$.\newline

  Now, if $\sigma$ is nuclear, then there exist $\varphi_n\colon B\rightarrow \Mat_{k(n)}\left( \C \right)$ and $\psi_n\colon \Mat_{k(n)}\left( \C \right)\rightarrow C$ such that for all $b\in B$, $\norm{\sigma(b) - \psi_n\circ\varphi_n(b)} \rightarrow 0$. In particular, this applies to $\theta(A)\subseteq B$, so by defining $\varphi_n' \coloneq \varphi_n\circ\theta\colon A \rightarrow \Mat_{k(n)}\left( \C \right)$, we have that
  \begin{align*}
    \norm{\sigma\circ\theta\left( a \right) - \psi_n\circ\varphi_n'\left( a \right)}\rightarrow 0,
  \end{align*}
  so $\sigma\circ\theta$ is nuclear.
\end{proof}
\begin{corollary}
  If $\id\colon A\rightarrow A$ is nuclear, then for any completely positive map $\theta\colon A\rightarrow B$, $\theta$ is nuclear.
\end{corollary}
\begin{proof}
  Write $\theta = \theta\circ \id$, and apply Proposition \ref{prop:nuclearity_composition}.
\end{proof}
\begin{definition}\label{def:nuclear_cstar_algebra}
  We say a $C^{\ast}$-algebra $A$ is \textit{nuclear} if the identity map $\id\colon A\rightarrow A$ is nuclear --- i.e., that all completely positive maps with domain $A$ are nuclear.
\end{definition}
One broad example of nuclear $C^{\ast}$-algebras is commutative ones.
\begin{proposition}
  If $A$ is a commutative $C^{\ast}$-algebra, then $A$ is nuclear.
\end{proposition}
\begin{proof}
  If $A$ is commutative then $A = C\left( X \right)$ for some compact Hausdorff space $X$ (Theorem \ref{thm:gelfand_naimark}). Now, for any finite $F\subseteq A$, there is an open cover $\set{U_1,\dots,U_n}$ of $X$ such that for each $f\in F$ and $1 \leq i \leq n$, we have $\left\vert f(x) - f(y) \right\vert < \ve$ for any $x,y\in U_i$, which follows from compactness.\newline

  Let $y_i\in U_i$ be arbitrary, and let $\set{\sigma_1,\dots,\sigma_n}$ be a partition of unity subordinate to the cover $\set{U_1,\dots,U_n}$ (\cite[Proposition 4.41]{folland_real_analysis}), where $\sum_{i=1}^{n}\sigma_i = 1$ and $\supp\left( \sigma_i \right) \subseteq U_i$ for each $i$.\newline

  Define $\varphi\colon A\rightarrow \C^n$ by $\varphi\left( f \right) = \left( f\left(y_1\right),\dots,f\left( y_n \right) \right)$. Since $\varphi$ is a unital $\ast$-homomorphism, it is a completely positive contraction.\newline

  Now, define $\psi\colon \C^n\rightarrow A$ by
  \begin{align*}
    \psi\left( d_1,\dots,d_n \right) &= \sum_{i=1}^{n}d_i\sigma_i.
  \end{align*}
  Now, $\psi$ is a positive map, and since $\C^n$ is a commutative $C^{\ast}$-algebra, $\psi$ is nuclear. Thus, we have
  \begin{align*}
    \norm{f - \psi\circ\phi\left( f \right)} &= \norm{\left( \sum_{i=1}^{n}\sigma_i \right)f - \sum_{i=1}^{n}f\left( y_i \right)\sigma_i}\\
                                             &= \norm{\sum_{i=1}^{n}\left( f-f\left( y_i \right) \right)\sigma_i}\\
                                             &\leq \ve,
  \end{align*}
  which holds for all $f\in F$ and any finite $F\subseteq A$, meaning that $A$ is nuclear.
\end{proof}
\section{Characterizing Amenability using \texorpdfstring{$C^{\ast}$-Algebras}{C*-Algebras}}%
The ultimate goal of this section is to prove the following theorem. In the process, we will conduct a tour of various concepts in the theory of $C^{\ast}$-algebras and von Neumann algebras.
\begin{theorem}[{\cite[Theorem 2.6.8]{brown_and_ozawa}}]\label{thm:amenability_cstar_algebras_formulations}
  Let $\Gamma$ be a discrete group. The following are equivalent:
  \begin{enumerate}[(i)]
    \item $\Gamma$ is amenable;
    \item $C^{\ast}_{\lambda}\left( \Gamma \right)$ is nuclear;
    \item $C^{\ast}_{\lambda}\left( \Gamma \right) = C^{\ast}\left( \Gamma \right)$;
    \item $C^{\ast}_{\lambda}\left( \Gamma \right)$ admits a character (Definition \ref{def:algebra_homomorphisms_and_characters}).
  \end{enumerate}
\end{theorem}
We will prove Theorem \ref{thm:amenability_cstar_algebras_formulations} using a variety of subtheorems.
\subsection{Amenability and Nuclearity}%
\begin{theorem}
  If $\Gamma$ is amenable, then $C^{\ast}_{\lambda}\left( \Gamma \right)$ is nuclear.
\end{theorem}
\begin{proof}
  Let $\left( F_n \right)_n\subseteq \Gamma$ be a Følner sequence, and let $P_n\in \B\left( \ell_2\left( \Gamma \right) \right)$ be the orthogonal projection onto the subspace of $\ell_2\left( \Gamma \right)$ spanned by $\set{\delta_g | g\in F_n}$. Here, $\set{\delta_t}_{t\in\Gamma}$ refers to the canonical orthonormal basis for $\C\left[ \Gamma \right]$.\newline

  Note that each $P_n$ is a finite-rank projection, so each $P_n$ can be written as
  \begin{align*}
    P_n\coloneq \sum_{g\in F_n}\theta_{\delta_{g},\delta_{g}},
  \end{align*}
  where $\theta_{\delta_g,\delta_g}$ denotes the rank-one bounded operator (Definition \ref{def:rank_one_bounded_operator}). Now, since each $F_n$ is finite-dimensional, we can rewrite the orthonormal basis as $\set{e_{1},\dots,e_{r}}$ for some $r$. Note that we have
  \begin{align*}
    T \theta_{e_j,e_j}\left( e_k \right) &= \iprod{e_k}{e_j}T\left( e_j \right)\\
                                         &= \theta_{T\left( e_j \right),e_j}\left( e_k \right),\label{eq:operator_applied_to_rank_one}\tag{\textdagger}
                                         \intertext{so}
    \theta_{e_i,e_i} T \theta_{e_j,e_j}\left( e_k \right) &= \iprod{T\left( e_j \right)}{e_i}\theta_{e_i,e_j}.
  \end{align*}
  Thus, by taking the summation over the representation of $P_n$, we find that there is an isomorphism
  \begin{align*}
    P_n\B\left( \ell_2\left( \Gamma \right) \right)P_n &\cong \Mat_{\left\vert F_n \right\vert}\left( \C \right).
  \end{align*}
  Let $\set{e_{p,q}}_{p,q\in F_n}$ be the matrix units of $P_n\B\left( \ell_2\left( \Gamma \right) \right)P_n$, which are equal to the rank-one bounded operators $\theta_{\delta_p,\delta_q}$. Note that if $\lambda_s\colon \ell_2\left( \Gamma \right)\rightarrow \ell_2\left( \Gamma \right)$ denotes the left-regular representation map $\delta_t \mapsto \delta_{st}$, then
  \begin{align*}
    e_{p,p}\lambda_se_{q,q} &= \theta_{\delta_p,\delta_p}\lambda_s\theta_{\delta_q,\delta_q}\\
                            &= \theta_{\delta_p,\delta_p}\theta_{\delta_{sq},\delta_{q}}\\
                            &= \begin{cases}
                              0 & sq\neq p\\
                              \theta_{p,q} & sq = p
                            \end{cases},
  \end{align*}
  where the latter quantity is found by evaluating \eqref{eq:operator_applied_to_rank_one} with $T = \theta_{p,p}$. Now, since $P_n = \sum_{p\in F_n}e_{p,p}$, we have
  \begin{align*}
    P_n \lambda_{s} P_n &= \sum_{p,q\in F_k}e_{p,p}\lambda_s e_{q,q}\\
                        &= \sum_{p\in F_k\cap sF_k} e_{p,s^{-1}p}.
  \end{align*}
  We start by defining $\varphi_{n}\colon C^{\ast}_{\lambda}\left( \Gamma \right)\rightarrow \Mat_{F_n}\left( \C \right)$ by $x\mapsto P_n x P_n$, which is a completely positive map, as projections preserve positivity.\newline

  Next, we define $\psi_n\colon \Mat_{F_n}\left( \C \right)\rightarrow C^{\ast}_{\lambda}\left( \Gamma \right)$ by taking $e_{p,q} \mapsto \frac{1}{\left\vert F_n \right\vert}\lambda_{p}\lambda_{q}^{\ast}$.\newline

  Note that if $A$ is a $C^{\ast}$-algebra, then any positive element of $\Mat_n\left( A \right)$ is a sum of elements of the form $\left( a_i^{\ast}a_j \right)_{ij}$ (\cite[Lemma 3.13]{completely_bounded_maps_and_operator_algebras}), so each of the $\psi_n$ maps is a completely positive contraction.\newline

  We start by evaluating on $\lambda_s$, where $s\in\Gamma$. This gives
  \begin{align*}
    \psi_n\circ\varphi_n\left( \lambda_s \right) &= \psi_n\left( \sum_{p\in F_n\cap sF_n}e_{p,s^{-1}p} \right)\\
                                                 &= \sum_{p\in F_n\cap sF_k}\frac{1}{\left\vert F_n \right\vert}\lambda_s\\
                                                 &= \frac{\left\vert F_n \cap sF_n \right\vert}{\left\vert F_n \right\vert}\lambda_s.
  \end{align*}
  Thus, by the definition of the Følner condition (Definition \ref{def:folner_condition}), we have that
  \begin{align*}
    \norm{\lambda_s - \psi_n\circ\varphi_n\left( \lambda_s \right)} &\rightarrow 0,
  \end{align*}
  Since $\Span\left( \set{\lambda_s}_{s\in\Gamma} \right)$ is norm-dense in $C^{\ast}_{\lambda}\left( \Gamma \right)$, linearity and continuity show that $C^{\ast}_{\lambda}\left( \Gamma \right)$ is nuclear.
\end{proof}
For the reverse direction, we need a little bit of background on the theory of von Neumann algebras, which are a special type of $\ast$-subalgebra of $\B\left( \mathcal{H} \right)$.
\begin{definition}
  A \textit{von Neumann algebra} is a $\ast$-subalgebra $A\subseteq \B\left( \mathcal{H} \right)$ that is closed in the weak operator topology (Definition \ref{def:operator_topologies}).
\end{definition}
\begin{definition}\label{def:state_linear_functional}
  Let $A$ be a $C^{\ast}$-algebra, and let $\varphi\colon A\rightarrow \C$ be a linear functional. 
  \begin{itemize}
    \item We say $\varphi$ is a \textit{state} if $\varphi\left( 1_A \right) = \norm{\varphi}_{\op} = 1$.
    \item We say $\varphi$ is \textit{tracial} if $\varphi\left( ab \right) = \varphi\left( ba \right)$ for all $a,b\in A$.
    \item If $A$ is a von Neumann algebra, then we say $\varphi$ is \textit{normal} if, for all norm-bounded, monotonically increasing nets $\left( x_{\alpha} \right)_{\alpha}\subseteq A_{\sa}$, we have $\varphi\left( \sup_{\alpha}x_{\alpha} \right) = \sup_{\alpha}\varphi\left( x_{\alpha} \right)$.
  \end{itemize}
\end{definition}
\begin{example}
  If $A\subseteq \B\left( \mathcal{H} \right)$ is a $C^{\ast}$-algebra, and $x\in S_{\mathcal{H}}$ is a unit vector, then the map $a\mapsto \iprod{ax}{x}$ defines a state on $A$, known as the \textit{vector state} on $A$.
\end{example}
The following is a useful structural result on von Neumann algebras, proven in \cite[Section III.2.4]{blackadar_operator_algebras}.
\begin{theorem}
  Let $M$ be a von Neumann algebra. Then, there is a unique Banach space $X$ such that $X^{\ast}\cong M$. Specifically, $X$ can be identified with the space of normal linear functionals on $M$.\newline

  Furthermore, if $A$ is a $C^{\ast}$-algebra such that there exists a Banach space $X$ with $X^{\ast}\cong A$, then $A$ is a von Neumann algebra.
\end{theorem}
Since every von Neumann algebra is a dual space, we may consider the weak* topology induced by the predual of a von Neumann algebra.
\begin{definition}
  Let $M$ be a von Neumann algebra, and let $M_{\ast}$ denote the predual of $M$. The \textit{ultraweak} topology on $M$ is the weak* topology on $M$ induced by $M_{\ast}$.\newline

  If $X$ is any Banach space, and $\B\left( X,M \right)$ is the space of bounded linear maps $T\colon X\rightarrow M$, the predual $\B\left( X,M \right)_{\ast}$ is canonically identified (Definition \ref{def:double_dual_and_canonical_embedding}) with the closed linear span of $x\otimes \xi\in \B\left( X,M \right)^{\ast}$, where $x\in X$, $\xi\in M_{\ast}$, and $x\otimes \xi\left( T \right) \coloneq \xi\left( T(x) \right)$.\newline

  On bounded sets, the weak* topology on $\B\left( X,M \right)$ induced by $\B\left( X,M \right)_{\ast}$ is known as the \textit{point-ultraweak} topology, where net convergence is defined by $\left( T_{\alpha} \right)_{\alpha}\rightarrow T$ if and only if $\xi\left( T_{\alpha}\left( x \right) \right) \rightarrow \xi\left( T\left( x \right) \right)$ for all $\xi\in M_{\ast}$ and $x\in X$.
\end{definition}
Since every space $\B\left( X,M \right)$ is a dual space, the Banach--Alaoglu Theorem (Theorem \ref{thm:banach_alaoglu}) applies, meaning that every bounded net of linear maps $T_{\alpha}\colon X\rightarrow M$ admits a convergent subnet.\newline

Now, similar to the case of a $C^{\ast}$-algebra generated by some set (Definition \ref{def:generated_operator_algebras}), there is such a thing as a von Neumann algebra generated by some set, which is defined similarly --- i.e., the smallest von Neumann algebra containing a given set. In particular, we are interested in the enveloping von Neumann algebra of the group $C^{\ast}$-algebras.
\begin{definition}
  Let $A\subseteq \B\left( \mathcal{H} \right)$ be a von Neumann algebra. The \textit{commutant} of $A$, denoted $A'$, is the set of all operators $T\in \B\left( \mathcal{H} \right)$ such that for any $S\in A$, $ST = TS$.\newline

  We define $A'' \coloneq \left( A' \right)'$, known as the \textit{double commutant} of $A$.
\end{definition}
\begin{theorem}[Double Commutant Theorem]
  For any $\ast$-subalgebra $A\subseteq \B\left( \mathcal{H} \right)$, we have
  \begin{align*}
    \overline{A}^{\text{SOT}} = \overline{A}^{\text{WOT}} = A''.
  \end{align*}
  Furthermore, $A$ is a von Neumann algebra if and only if $A = A''$.
\end{theorem}
\begin{definition}
  If $\Gamma$ is a group with reduced group $C^{\ast}$-algebra $C^{\ast}_{\lambda}\left( \Gamma \right)$, then the \textit{group von Neumann algebra} is the space $L\left( \Gamma \right) \coloneq C^{\ast}_{\lambda}\left( \Gamma \right)'' \subseteq \B\left( \ell_2\left( \Gamma \right) \right)$.
\end{definition}
\begin{proposition}
  The vector state $x\mapsto \iprod{x\delta_e}{\delta_e}$ defines a faithful tracial state on $C^{\ast}_{\lambda}\left( \Gamma \right)$.
\end{proposition}
\begin{proof}
  Let $\lambda_s,\lambda_t\in C^{\ast}_{\lambda}\left( \Gamma \right)$. Then,
  \begin{align*}
    \iprod{\lambda_s\lambda_t\left( \delta_e \right)}{\delta_e} &= \iprod{\lambda_{st}\left( \delta_e \right)}{\delta_e}\\
                                                                &= \iprod{\delta_{st}}{\delta_e}\\
                                                                &= \sum_{r\in\Gamma}\delta_e\left( r \right)\delta_{st}\left( r \right)\\
                                                                &= \delta_{st}\left( e \right)\\
                                                                &= \delta_{ts}\left( e \right)\\
                                                                &= \iprod{\delta_{ts}}{\delta_e}\\
                                                                &= \iprod{\lambda_t\lambda_s\left( \delta_e \right)}{\delta_e}.
  \end{align*}
  Thus, the vector state is tracial.
\end{proof}
One more fact we need is related to treating $C^{\ast}$-algebras as particular types of modules and multiplicative domains. More information on the subject can be found in \cite[Chapter 3]{completely_bounded_maps_and_operator_algebras}.
\begin{definition}
  Let $R$ be a ring, and let $M$ be an abelian group. A \textit{left-action} of $R$ on $M$ is a map $\rho\colon R\times M \rightarrow M$ such that
  \begin{itemize}
    \item $\rho\left( r,m+n \right) = \rho\left( r,m \right) + \rho\left( r,n \right)$
    \item $\rho\left( r + s,m \right) = \rho\left( r,m \right) + \rho\left( s,m \right)$
    \item $\rho\left( rs,m \right) = \rho\left( r,\rho\left( s,m \right) \right)$
    \item $\rho\left( 1,m \right) = m$.
  \end{itemize}
  We write $r\cdot m \coloneq \rho\left( r,m \right)$. The left-action of $R$ on $M$ defines an \textit{left $R$-module} structure on $M$.\newline

  Analogously, we may define \textit{right $R$-modules} and \textit{$R$-bimodules} on $M$.
\end{definition}
Note that if $R\subseteq S$ is an inclusion of rings, then we can consider $S$ to be an $R$-module.\newline

Similarly, if $A$ is a unital $C^{\ast}$-subalgebra with unital $C^{\ast}$-subalgebra $C\subseteq A$, we may consider $A$ as a left $C$-module, or a right $C$-module, or a $C$-bimodule. 
\begin{definition}
  Let $\phi\colon A\rightarrow B$ be map between unital $C^{\ast}$-algebras such that $C\subseteq A$ and $C\subseteq B$. Then, we say $\phi$ is a \textit{left $C$-module map} if $\phi\left( ca \right) = c\phi\left( a \right)$ for all $a\in A$ and $c\in C$. Similar definitions hold for right $C$-module maps and $C$-bimodule maps.
\end{definition}
\begin{theorem}\label{thm:preview_multiplicative_domain}
  Let $A$ and $B$ be unital $C^{\ast}$-algebras, and let $\phi\colon A\rightarrow B$ be a completely positive map with $\phi\left(1_A\right) = 1_B$. Then, if
  \begin{align*}
    S &\coloneq\set{a\in A | \phi\left( a^{\ast}a \right) = \phi\left( a \right)^{\ast}\phi\left( a \right)\text{ and }\phi\left( aa^{\ast} \right) = \phi\left( a \right)\phi\left( a^{\ast} \right)}\\
      T&\coloneq \set{a\in A | \phi\left( ab \right) = \phi\left( a \right)\phi\left( b \right)\text{ and }\phi\left( ba \right) = \phi\left( b \right)\phi\left( a \right)\text{ for all }b\in A},
  \end{align*}
  we have $S = T$.\newline

  Furthermore, the set $S$ is a $C^{\ast}$-subalgebra of $A$ such that $\phi$ is a $\ast$-homomorphism when restricted to this set.
\end{theorem}
A proof of this theorem can be found in \cite{completely_bounded_maps_and_operator_algebras} with modifications. 
\begin{definition}
  If $\phi\colon A\rightarrow B$ is a completely positive map, the set
  \begin{align*}
    \set{a | \phi\left( a^{\ast}a \right) = \phi\left( a \right)^{\ast}\phi\left( a \right)\text{ and }\phi\left( aa^{\ast} \right) = \phi\left( a \right)\phi\left( a \right)^{\ast}}
  \end{align*}
  is known as the \textit{multiplicative domain} for $\phi$.
\end{definition}
The multiplicative domain is essentially the $C^{\ast}$-subalgebra of a domain of a completely positive map that, when restricted, makes the map a $\ast$-homomorphism. Specifically, if a completely positive map restricts to the identity on some $C^{\ast}$-subalgebra, then that $C^{\ast}$-subalgebra is automatically in the multiplicative domain of the map.
\begin{corollary}\label{cor:multiplicative_domain}
  Let $A$, $B$, and $C$ be unital $C^{\ast}$-algebras, and suppose $C$ is a unital $C^{\ast}$-subalgebra of both $A$ and $B$. If $\phi\colon A\rightarrow B$ is a completely positive map such that $\phi(c) = c$ for all $c\in C$, then $\phi$ is a $C$-bimodule map.
\end{corollary}
\begin{theorem}\label{thm:nuclearity_implies_amenability}
  Let $C^{\ast}_{\lambda}\left( \Gamma \right)$ be nuclear. Then, $\Gamma$ is amenable.
\end{theorem}
\begin{proof}
  We will construct an invariant state on $\ell_{\infty}\left( \Gamma \right)$, which we consider as multiplication operators in $\B\left( \ell_2\left( \Gamma \right) \right)$ --- i.e., $M_{f}(x) = \sum_{t\in\Gamma}f(t)x(t)$ for $f\in \ell_{\infty}\left( \Gamma \right)$ and $x\in \ell_2\left( \Gamma \right)$.\newline

  Let $\varphi_n\colon C^{\ast}_{\lambda}\left( \Gamma \right)\rightarrow \Mat_{k(n)}\left( \C \right)$ and $\psi_n\colon \Mat_{k(n)}\left( \C \right)\rightarrow C^{\ast}_{\lambda}\left( \Gamma \right)$ be completely positive contractions such that
  \begin{align*}
    \norm{a - \psi_n\circ\varphi_n\left( a \right)}\xrightarrow{n\rightarrow\infty} 0
  \end{align*}
  for all $a\in A$.\newline

  Since $\Mat_{k(n)}\left( \C \right) \cong \B\left( \C^{k(n)} \right)$, we may use Arveson's extension theorem (Theorem \ref{thm:arveson}) to consider $\varphi_n\colon \B\left( \ell_2\left( \Gamma \right) \right)\rightarrow \Mat_{k(n)}\left( \C \right)$. Letting $\Phi_{n}\coloneq \psi_n\circ\varphi_n$, we note that $\norm{\Phi_n}\leq \norm{\psi_n}\norm{\varphi_n}\leq 1$, and that $\Phi_n\colon \B\left( \ell_2\left( \Gamma \right) \right)\rightarrow \C$ is such that $\Phi_n\left( x \right)\rightarrow x$ for all $x\in C^{\ast}_{\lambda}\left( \Gamma \right)\subseteq L\left( \Gamma \right)$.\newline

  Now, since the $\Phi_n$ are all bounded, by the Banach--Alaoglu Theorem (Theorem \ref{thm:banach_alaoglu}), we may find a subnet (or subsequence) converging to $\Phi\colon \B\left( \ell_2\left( \Gamma \right) \right)\rightarrow L\left( \Gamma \right)$ in the point-ultraweak topology, where $\Phi(x) = x$ for all $x\in C^{\ast}_{\lambda}\left( \Gamma \right)$.\newline

  Letting $\tau\colon L\left( \Gamma \right)\rightarrow \C$ be the vector trace $x\mapsto \iprod{x\delta_e}{\delta_e}$, we define the state $\eta\coloneq \tau\circ \Phi$ on $\B\left( \ell_2\left( \Gamma \right) \right)$, which we restrict to $\ell_{\infty}$.\newline

  We note that left translation of the form $f(t) \mapsto f\left( s^{-1}t \right)$ (which we refer to by $\lambda_s$ in Proposition \ref{prop:translation_action}) is implemented via $M_{f}(\delta_t) \mapsto \lambda_{s}M_f\lambda_{s}^{\ast}\left( \delta_t \right)$. This can be seen by direct calculation:
  \begin{align*}
    \lambda_sM_f\lambda_s^{\ast}\left( \delta_t \right) &= \lambda_s\left( \sum_{r\in\Gamma}f(r)\lambda_{s^{-1}}\left( \delta_t \right) \left( r \right) \right)\\
                                                        &= \lambda_s\left( \sum_{r\in\Gamma}f(r)\delta_{s^{-1}t}(r) \right)\\
                                                        &= \sum_{r\in\Gamma}f\left( s^{-1}r \right)\delta_t(r)\\
                                                        &= f\left( s^{-1}t \right).
  \end{align*}
  Define $\mu\colon \ell_{\infty}\left( \Gamma \right)\rightarrow \C$ by $\mu\left( f \right) = \eta\left( M_f \right)$.\newline

  From Corollary \ref{cor:multiplicative_domain}, and the fact that for every $s\in\Gamma$, $\lambda_s\in C^{\ast}_{\lambda}\left( \Gamma \right)$, we have, for any $T\in \B\left( \ell_2\left( \Gamma \right) \right)$,
  \begin{align*}
    \eta\left( \lambda_s T \lambda_s^{\ast} \right) &= \tau\left( \Phi\left( \lambda_s T \lambda_s^{\ast} \right) \right)\\
                                                    &= \tau\left( \lambda_s \Phi(T) \lambda_s^{\ast}\right)\\
                                                    &= \tau\left( \lambda_s^{\ast}\lambda_s\Phi(T) \right)\label{eq:using_traciality}\tag{\textasteriskcentered}\\
                                                    &= \tau\left( \Phi\left( T \right) \right)\\
                                                    &= \eta(T),
  \end{align*}
  where in \eqref{eq:using_traciality} we use the fact that $\tau$ is tracial. In particular, this identity holds for $T= M_f$ where $f\in \ell_{\infty}\left( \Gamma \right)$. Thus,
  \begin{align*}
    \mu\left( \lambda_s(f) \right) &= \eta\left( \lambda_sM_f\lambda_s^{\ast} \right)\\
                                   &= \eta\left( M_f \right)\\
                                   &= \mu\left( f \right),
  \end{align*}
  so $\mu\colon \ell_{\infty}\left( \Gamma \right)$ is an invariant state. Thus, $\Gamma$ is amenable.
\end{proof}

\subsection{Amenability and Further Structural Properties of Group $C^{\ast}$-Algebras}%
In order to prove the cycle (i) $\Rightarrow$ (iii) $\Rightarrow$ (iv) $\Rightarrow$ (i), we will need to know a little bit more about tensor products of Hilbert spaces and operators.\newline

If $\mathcal{H}$ and $\mathcal{K}$ are Hilbert spaces (Definition \ref{def:hilbert_spaces}), we define an inner product on the tensor product $\mathcal{H}\otimes \mathcal{K}$ by
\begin{align*}
  \iprod{x\otimes y}{x'\otimes y'} &= \iprod{x}{x'} \iprod{y}{y'}.
\end{align*}
Completing $\mathcal{H}\otimes \mathcal{K}$ with respect to the norm induced by this inner product yields the Hilbert space tensor product, which we will yet again denote $\mathcal{H}\otimes \mathcal{K}$.\newline

The Hilbert space $\ell_2\left( \Gamma,\mathcal{H} \right)$ consists of all functions $f\colon \Gamma\rightarrow \mathcal{H}$ such that
\begin{align*}
  \norm{f}_{\ell_2\left( \Gamma,\mathcal{H} \right)}^2 &\coloneq \sum_{t\in\Gamma}\norm{f(t)}^2\\
                                                       &< \infty,
\end{align*}
with inner product
\begin{align*}
  \iprod{f}{g} &= \sum_{t\in\Gamma} \iprod{f(t)}{g(t)}.
\end{align*}
It can be shown that $\ell_2\left( \Gamma,\mathcal{H} \right)$ is isometrically isomorphic to $\ell_2\left( \Gamma \right)\otimes \mathcal{H}$.\newline

Now, if $T\in \B\left( \mathcal{H} \right)$ and $S\in \B\left( \mathcal{K} \right)$, there is a linear map $T\otimes S\in \B\left( \mathcal{H}\otimes \mathcal{K} \right)$ such that $T\otimes S \left( x\otimes y \right) = T(x)\otimes S(y)$, and $\norm{T\otimes S}_{\op} = \norm{T}_{\op}\norm{S}_{\op}$.\newline

This is the object of our study, specifically for the purpose of proving a property of the left-regular representation, known as Fell's Absorption Principle.
\begin{theorem}[Fell's Absorption Principle]
  Let $\Gamma$ be a discrete group, let $\lambda\colon \Gamma\rightarrow \mathcal{U}\left( \ell_2\left( \Gamma \right) \right)$ be the left-regular representation (Theorem \ref{thm:left_regular_representation}), and let $\pi\colon \Gamma\rightarrow \mathcal{U}\left( \mathcal{H} \right)$ be any unitary representation.\newline

  Then, there is a unitary operator $U\colon \ell_2\left( \Gamma \right)\otimes \mathcal{H} \rightarrow \ell_2\left( \Gamma \right)\otimes \mathcal{H}$ such that $U\left( \lambda\otimes 1_{\mathcal{H}} \right) U^{\ast}= \lambda\otimes \pi$, where $1_{\mathcal{H}}\colon \Gamma\rightarrow \mathcal{H}$ denotes the representation that maps every $s\in\Gamma$ to $I_{\mathcal{H}}$.
\end{theorem}
\begin{proof}
  Elements of $\ell_2\left( \Gamma \right)\otimes \mathcal{H}$ are of the form $\sum_{t\in\Gamma}\delta_t\otimes \xi_{t}$, where $\delta_t$ is the point mass at $t\in\Gamma$ and $\xi_{t}\in \mathcal{H}$ is some vector.\newline

  By linearity and continuity, it is sufficient to verify that $\lambda\otimes \pi$ and $\lambda\otimes 1_{\mathcal{H}}$ are unitarily equivalent on elementary tensors. For $\delta_t\otimes \xi_t\in \ell_2\left( \Gamma \right)\otimes \mathcal{H}$, we define
  \begin{align*}
    U\left( \delta_t\otimes \xi_t \right) &= \delta_t\otimes \pi(t)\left(\xi_t\right).
  \end{align*}
  This map is unitary since, for any $t$, $\pi(t)\in \mathcal{U}\left( \mathcal{H} \right)$ by definition. Thus, we have
  \begin{align*}
    U\left( \left( \lambda\otimes 1_{\mathcal{H}} \right)(s) \right)\left( \delta_t\otimes \xi_t \right) &= U\left( \lambda_s\left(\delta_t\right)\otimes \xi_t \right)\\
                                                                                                         &= U\left( \delta_{st}\otimes \xi_t \right)\\
                                                                                                         &= U\left( \delta_{r}\otimes \xi_{s^{-1}r} \right)\\
                                                                                                         &= \delta_{r}\otimes \pi(r)\left( \xi_{s^{-1}r} \right).
  \end{align*}
  Meanwhile,
  \begin{align*}
    \left( \lambda\otimes \pi \right)(s)\left( U\left( \delta_t\otimes \xi_t \right) \right) &= \left( \left( \lambda\otimes \pi \right)(s) \right)\left( \delta_t\otimes \pi(t)\left( \xi_t \right) \right)\\
                                                                                             &= \lambda_s\left( \delta_t \right)\otimes \pi(s)\pi(t)\left( \xi_t \right)\\
                                                                                             &- \delta_{st}\otimes \pi\left( st \right)\left( \xi_{t} \right)\\
                                                                                             &= \delta_{r}\otimes \pi\left( r \right)\left( \xi_{s^{-1}r} \right).
  \end{align*}
  Therefore, $U\left( \lambda\otimes 1_{\mathcal{H}} \right)U^{\ast} = \lambda\otimes \pi $.
\end{proof}
Now, we may prove the cycle (i) $\Rightarrow $ (iii) $\Rightarrow$ (iv) $\Rightarrow$ (i) from Theorem \ref{thm:nuclearity_implies_amenability}.
\begin{theorem}
  Let $\Gamma$ be a discrete group. The following are equivalent:
  \begin{description}[font=\normalfont]
    \item[(i)] $\Gamma$ is amenable;
    \item[(iii)] $C^{\ast}_{\lambda}\left( \Gamma \right) = C^{\ast}\left( \Gamma \right)$;
    \item[(iv)] $C^{\ast}_{\lambda}\left( \Gamma \right)$ admits a character (Definition \ref{def:algebra_homomorphisms_and_characters}).
  \end{description}
\end{theorem}
\begin{proof}
  Suppose $\Gamma$ is amenable. Then, the left-regular representation, $\lambda\colon \Gamma\rightarrow \mathcal{U}\left( \ell_2\left( \Gamma \right) \right)$, admits an almost-invariant vector, $\left( \xi_n \right)_n\subseteq S_{\ell_2\left( \Gamma \right)}$.\newline

  For any unitary representation $\widetilde{\pi}\colon \Gamma\rightarrow \mathcal{U}\left( \mathcal{H} \right)$, we let $\pi\colon \C\left[ \Gamma \right]\rightarrow \B\left( \mathcal{H} \right)$ be the corresponding unital representation (Proposition \ref{prop:unital_unitary_representation}). Then, for any $x\in \C\left[ \Gamma \right]$ and any $\eta_1,\eta_2\in S_{\mathcal{H}}$, we use the almost-invariance of $\left( \xi_n \right)_n$ and the Cauchy--Schwarz inequality to find
  \begin{align*}
    \left\vert \iprod{\pi(x)\left( \eta_1 \right)}{\eta_2} \right\vert &= \left\vert \iprod{\left( \left( \lambda\otimes \pi \right)\left( x \right) \right)\left( \xi_n\otimes \eta_1 \right)}{\xi_n\otimes \eta_2} \right\vert\\
                                                                       &\leq \norm{\left( \lambda\otimes \pi \right)(x)}_{\op}\\
                                                                       &= \norm{\left( \lambda\otimes 1_{\mathcal{H}} \right)\left( x \right)}_{\op}\\
                                                                       &= \norm{\lambda(x)}_{\op},
  \end{align*}
  where the second-to-last equality used the fact that $\lambda\otimes \pi$ and $\lambda\otimes 1_{\mathcal{H}}$ are unitarily equivalent. Now, taking the supremum over all $\eta_1,\eta_2\in S_{\mathcal{H}}$, we get
  \begin{align*}
    \norm{\pi(x)}_{\op} &\leq \norm{\lambda(x)}_{\op},
  \end{align*}
  and taking the supremum over all representations $\pi\colon \C\left[ \Gamma \right]\rightarrow \mathcal{H}$, we get
  \begin{align*}
    \norm{x}_{u} &\leq \norm{\lambda(x)}_{\op},
  \end{align*}
  so $\norm{x}_u = \norm{\lambda(x)}_{\op}$. Thus, the identity map $\id\colon C^{\ast}\left( \Gamma \right)\rightarrow C^{\ast}_{\lambda}\left( \Gamma \right)$ is isometric, so $C^{\ast}\left( \Gamma \right) = C^{\ast}_{\lambda}\left( \Gamma \right)$.\newline

  Let $C^{\ast}_{\lambda}\left( \Gamma \right) = C^{\ast}\left( \Gamma \right)$. Now, considering the trivial representation from Example \ref{ex:some_representations}, $1_{\Gamma}\colon \Gamma\rightarrow \C$, we know from the universal property of the universal group $C^{\ast}$-algebra that $1_{\Gamma}$ extends to a unital $\ast$-homomorphism $\tau\colon \C\left[ \Gamma \right]\rightarrow \B\left( \C \right) = \C$. Since $C^{\ast}_{\lambda}\left( \Gamma \right) = C^{\ast}\left( \Gamma \right)$, we have a nonzero $\ast$-homomorphism (or a character) $\tau\colon C^{\ast}_{\lambda}\left( \Gamma \right)\rightarrow \C$.\newline

  Let $\tau\colon C^{\ast}_{\lambda}\left( \Gamma \right)\rightarrow \C$ be a character. By Arveson's extension theorem, we may extend $\tau$ to $\sigma\colon \B\left( \ell_2\left( \Gamma \right) \right)\rightarrow \C$; note that $\sigma\bigr\vert_{C^{\ast}_{\lambda}\left( \Gamma \right)} = \tau$, so $C^{\ast}_{\lambda}\left( \Gamma \right)$ is part of the multiplicative domain of $\sigma$.\newline

  Considering $\ell_{\infty}\left( \Gamma \right)\subseteq \B\left( \ell_2\left( \Gamma \right) \right)$ as multipliers, we use the fact that $\lambda_s(f) $ (as in Proposition \ref{prop:translation_action}) and $\lambda_sM_f\lambda_s^{\ast}$ are interchangeable (where $M_f$ denotes the multiplication operator). Define $\mu\colon \ell_{\infty}\left( \Gamma \right)\rightarrow \C$ by $\mu(f) = \sigma\left( M_f \right)$. Then, since $\lambda_s,\lambda_s^{\ast}\in C^{\ast}_{\lambda}\left( \Gamma \right)$, we have
  \begin{align*}
    \mu\left( \lambda_s(f) \right) &= \sigma\left( \lambda_sM_f\lambda_s^{\ast} \right)\\
                                   &= \sigma\left( \lambda_s \right)\sigma\left( M_f \right)\sigma\left( \lambda_s^{\ast} \right)\\
                                   &= \sigma\left( \lambda_s\lambda_s^{\ast} \right)\sigma\left( M_f \right)\\
                                   &= \sigma\left( M_f \right)\\
                                   &= \mu\left( f \right),
  \end{align*}
  so $\mu$ is an invariant state, meaning $\Gamma$ is amenable.
\end{proof}

\section{Remarks and Notes}%
The original definition of nuclearity for $C^{\ast}$-algebras, discussed in \cite{cross_norm_takesaki}, concerns norms on the tensor product $A\otimes B$ of a $C^{\ast}$-algebra $A$ with any other $C^{\ast}$-algebra $B$. There are two distinguished norms that can be from a tensor product of $C^{\ast}$-algebras. The ``maximal norm'' is defined akin to the universal norm for the group $C^{\ast}$-algebra, where one takes the supremum over all representations, and the ``minimal norm'' is defined with respect to faithful representations of the particular $C^{\ast}$-algebras, similar to how the reduced group $C^{\ast}$-algebra was defined with respect to the left-regular representation. Takesaki says that a $C^{\ast}$-algebra is nuclear if these two norms coincide whenever the $C^{\ast}$-algebra $A$ has its tensor product taken with any other $C^{\ast}$-algebra $B$.\newline

By this definition, we already know that $\Mat_n\left( \C \right)$ is a nuclear $C^{\ast}$-algebra, as we mentioned that for any other $C^{\ast}$-algebra $A$, there is a canonically defined norm on $\Mat_n\left( A \right)\cong \Mat_n\left( \C \right)\otimes A$. Furthermore, in the case where $C^{\ast}_{\lambda}\left( \Gamma \right)$ is nuclear, as in Definition \ref{def:nuclear_cstar_algebra}, it can be shown using Fell's absorption principle that this implies the fact that $C^{\ast}\left( \Gamma \right) = C^{\ast}_{\lambda}\left( \Gamma \right)$, which is akin to Takesaki's definition of nuclearity. However, showing the equivalence between the completely positive approximation property of Definition \ref{def:nuclear_cstar_algebra} and Takesaki's definition was shown in \cite{choi_nuclearity}.
