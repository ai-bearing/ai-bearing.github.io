Here, we will establish the equivalence between group amenability and certain properties of the group $C^{\ast}$-algebra(s). The results in here will draw from a lot of theory that we discuss a bit more in depth in Chapter \ref{ch:operator_algebras}. Some excellent books on this topic include \cite{brown_and_ozawa} and \cite{completely_bounded_maps_and_operator_algebras}, both of which go deeper into the ramifications of the results we will present herein.
\section{Norms on the Group \texorpdfstring{$\ast$-Algebras}{*-Algebras}}%
From Definition \ref{def:group_star_algebra}, we know that for any group $\Gamma$, there is a free vector space, $\C\left[ \Gamma \right]$, consisting of finitely supported functions on $\Gamma$. Elements of $\C\left[ \Gamma \right]$ are finite sums of the form
\begin{align*}
  a &= \sum_{s\in\Gamma}a(s)\delta_s,
\end{align*}
where $\delta_s$ is the point mass function
\begin{align*}
  \delta_s\left( t \right) &= \begin{cases}
    1 & s = t\\
    0 & s\neq t
  \end{cases}.
\end{align*}
This admits a multiplication by convolution:
\begin{align*}
  f\ast g(s) &= \sum_{t\in\Gamma}f(t)g\left( t^{-1}s \right)\\
              &= \sum_{r\in\Gamma}f\left( sr^{-1} \right)g\left( r \right)
\end{align*}
and an involution
\begin{align*}
  f^{\ast}\left( t \right) &= \overline{f\left( t^{-1} \right)},
\end{align*}
which turn $\C\left[ \Gamma \right]$ into a $\ast$-algebra.\newline

Now, we are interested in applying norms on the group $\ast$-algebra, turning them into group $C^{\ast}$-algebras. We will do this through the use of unitary representations --- there is an intimate relationship between unitary representations of groups and unital representations (see Definition \ref{def:unital_representation}) of the group $\ast$-algebra generated by the group.
\begin{proposition}[{\cite[Proposition 7.2.46]{rainone_analysis}}]\label{prop:unital_unitary_representation}
  Let $\Gamma$ be a group and let $\mathcal{H}$ be a Hilbert space.
  \begin{enumerate}[(1)]
    \item If $u\colon \Gamma\rightarrow \mathcal{U}\left( \mathcal{H} \right)$ is a unitary representation of $\Gamma$, then $\pi_u\colon \C\left[ \Gamma \right]\rightarrow \B\left( \mathcal{H} \right)$, given by
      \begin{align*}
        \pi_u(a) &= \sum_{s\in\Gamma}a(s)u_s
      \end{align*}
      is a unital representation of $\Gamma$.
    \item If $\pi\colon \C\left[ \Gamma \right]\rightarrow \B\left( \mathcal{H} \right)$ is a unital representation, then $u\colon \Gamma\rightarrow \mathcal{U}\left( \mathcal{H} \right)$, given by
      \begin{align*}
        u(s) &\coloneq \pi\left( \delta_s \right),
      \end{align*}
      is a unitary representation of $\Gamma$.
  \end{enumerate}
\end{proposition}
\begin{proof}\hfill
  \begin{enumerate}[(1)]
    \item Via the universal property of the free vector space, we know that the map $s\mapsto u_s\in \B\left( \mathcal{H} \right)$ extends to a linear map $\pi_u\colon \C\left[ \Gamma \right]\rightarrow \B\left( \mathcal{H} \right)$. Now, we must ensure that this map is faithful to the underlying multiplication structure. Letting $s,t\in \Gamma$ be arbitrary, via the properties of unitary representations, we have
      \begin{align*}
        \pi_u\left( \delta_s\delta_t \right) &= \pi_u\left( \delta_{st} \right)\\
                                             &= u_{st}\\
                                             &= u_su_t\\
                                             &= \pi_u\left( \delta_s \right)\pi_u\left( \delta_t \right)\\
        \pi_u\left( \delta_{s}^{\ast} \right) &= \pi_u\left( \delta_{s^{-1}} \right)\\
                                              &= u_{s^{-1}}\\
                                              &= u_s^{\ast}\\
                                              &= \pi_u\left( \delta_s \right)^{\ast}.
      \end{align*}
      Therefore, via linearity, we obtain that $\pi_u$ is multiplicative and $\ast$-preserving.
    \item Every $\delta_s\in \C\left[ \Gamma \right]$ is a unitary element, and since unital $\ast$-homomorphisms preserve unitary elements (Fact \ref{fact:unitary_preservation}), we know that each $u(s)$ is unitary. Furthermore, for any $s,t\in\Gamma$, we have
      \begin{align*}
        u\left( st \right) &= \pi\left( \delta_{st} \right)\\
                           &= \pi\left( \delta_s\delta_t \right)\\
                           &= \pi\left( \delta_s \right)\pi\left( \delta_t \right)\\
                           &= u(s)u(t),
      \end{align*}
      meaning $u$ is a unitary representation.
  \end{enumerate}
\end{proof}
Now, using the interplay between unitary and unital representations of $\Gamma$ and $\C\left[ \Gamma \right]$ respectively, we may define two special $C^{\ast}$-norms on $\C\left[ \Gamma \right]$. We will investigate the properties of their respective group $C^{\ast}$-algebras.
\begin{proposition}
  Let $\Gamma$ be a group. If $\lambda\colon\Gamma\rightarrow \mathcal{U}\left( \ell_2\left( \Gamma \right) \right)$ is the left-regular representation (Theorem \ref{thm:left_regular_representation}), then $\lambda$ extends to an injective representation $\pi_{\lambda}\colon \C\left[ \Gamma \right]\rightarrow \B\left( \ell_2\left( \Gamma \right) \right)$, given by
  \begin{align*}
    \pi_{\lambda}(a) &= \sum_{s\in\Gamma}a(s)\lambda_s.
  \end{align*}
\end{proposition}
\begin{proof}
  Suppose $\pi_{\lambda}(a) = 0$ for some $a = \sum_{s\in\Gamma}a(s)\delta_s$ in $\C\left[ \Gamma \right]$. Taking the evaluation at $\delta_{e}$, we get
  \begin{align*}
    0 &= \pi_{\lambda}\left( a \right)\left( \delta_e \right)\\
      &= \left( \sum_{s\in\Gamma}a(s)\lambda_s \right)\left( \delta_e \right)\\
      &= \sum_{s\in\Gamma}a(s)\lambda_s\left( \delta_e \right)\\
      &= \sum_{s\in\Gamma}a(s)\delta_s.
  \end{align*}
  Since the $\set{\delta_t}_{t\in\Gamma}$ are linearly independent, we must have that $a(s) = 0$ for all $s\in\Gamma$, meaning $a = 0$.
\end{proof}
\begin{definition}\label{def:reduced_group_cstar_algebra}
  Define the $C^{\ast}$-norm
  \begin{align*}
    \norm{a}_{\lambda} &\coloneq \norm{\pi_{\lambda}(a)}_{\op}
  \end{align*}
  on $\C\left[ \Gamma \right]$.\newline

  The completion of $\C\left[ \Gamma \right]$ with respect to $\norm{\cdot}_{\lambda}$ is known as the \textit{reduced group $C^{\ast}$-algebra}, denoted $C^{\ast}_{\lambda}\left( \Gamma \right)$.
\end{definition}
\begin{proposition}\label{prop:universal_group_cstar_algebra}
  Let $\Gamma$ be a group, and let $\C\left[ \Gamma \right]$ be the group $\ast$-algebra.\newline

  Define the \textit{universal norm} (or maximum norm) on $\C\left[ \Gamma \right]$ by
  \begin{align*}
    \norm{a}_{u} &\coloneq \sup\set{\norm{\pi(a)}_{\op} | \pi\colon \C\left[ \Gamma \right]\rightarrow \B\left( \mathcal{H}_{\pi} \right)\text{ is a unital representation}}.
  \end{align*}
  Then, this is a $C^{\ast}$-norm on $\C\left[ \Gamma \right]$. The completion with respect to this norm yields the \textit{universal group $C^{\ast}$-algebra}, and is denoted $C^{\ast}\left( \Gamma \right)$.
\end{proposition}
\begin{proof}
  First, we show that the quantity
  \begin{align*}
    \norm{a}_{u}\coloneq \sup\set{\norm{\pi(a)}_{\op} | \pi\colon \C\left[ \Gamma \right]\rightarrow \B\left( \mathcal{H}_{\pi} \right)\text{ is a unital representation}}
  \end{align*}
  is finite (i.e., that the universal norm exists).\newline

  Note that for any representation $\pi\colon \C\left[ \Gamma \right]\rightarrow \B\left( \mathcal{H}_{\pi} \right)$, the elements $\pi\left( \delta_s \right)$ are unitary in $\B\left( \mathcal{H}_{\pi} \right)$, meaning they have norm $1$. Therefore, for a finitely supported function $a = \sum_{s\in\Gamma}a(s)\delta_s$, we have
  \begin{align*}
    \norm{\pi(a)}_{\op} &= \norm{\pi\left( \sum_{s\in\Gamma}a(s)\delta_s \right)}_{\op}\\
                        &= \norm{\sum_{s\in\Gamma}a(s)\pi\left( \delta_s \right)}_{\op}\\
                        &\leq \sum_{s\in\Gamma}\norm{a(s)\pi\left( \delta_s \right)}_{\op}\\
                        &= \sum_{s\in\Gamma}\left\vert a(s) \right\vert\norm{\pi\left( \delta_s \right)}_{\op}\\
                        &= \sum_{s\in\Gamma}\left\vert a(s) \right\vert,
  \end{align*}
  so that $\norm{a}_u\leq \sum_{s\in\Gamma}\left\vert a(s) \right\vert < \infty$.\newline

  That $\norm{\cdot}_{u}$ is a $C^{\ast}$-seminorm follows from the fact that for any representation $\pi$ and any $a,b\in \Gamma$, we have
  \begin{align*}
    \norm{\pi\left( ab \right)}_{\op} &= \norm{\pi(a)\pi(b)}_{\op}\\
                                      &\leq \norm{\pi(a)}_{\op}\norm{\pi(b)}_{\op},
  \end{align*}
  so by taking the supremum over all representations, we obtain $\norm{ab}_{u}\leq \norm{a}_{u}\norm{b}_{u}$. Similarly, for any $a\in\Gamma$, we have
  \begin{align*}
    \norm{\pi(a)^{\ast}\pi(a)}_{\op} &- \norm{\pi(a)}_{\op}^2,
  \end{align*}
  so by taking the supremum over all representations, we obtain $\norm{a^{\ast}a} = \norm{a}^2$, showing that it is indeed a $C^{\ast}$-seminorm.\newline

  To verify that $\norm{\cdot}_u$ is a norm, we note that if $\norm{a}_{u} = 0$, then since $\pi_{\lambda}$ is a representation, we must have $\norm{a}_{\lambda} = 0$, and since $\pi_{\lambda}$ is injective, it follows that $a = 0$. Thus, $\norm{\cdot}_u$ is a norm.
\end{proof}
The moniker ``universal'' is apt for the universal $C^{\ast}$-algebra, as it admits a universal property.
\begin{theorem}[{\cite[Proposition 7.2.47]{rainone_analysis}}]
  Let $\Gamma$ be a discrete group. If $u\colon \Gamma\rightarrow \mathcal{U}\left( \mathcal{H} \right)$ is a unitary representation, then there is a contractive $\ast$-homomorphism $\pi_u\colon C^{\ast}\left( \Gamma \right)\rightarrow \B\left( \mathcal{H} \right)$ such that $\pi_u\left( \delta_s \right) = u(s)$ for all $s\in\Gamma$.
  \begin{center}
    % https://tikzcd.yichuanshen.de/#N4Igdg9gJgpgziAXAbVABwnAlgFyxMJZABgBpiBdUkANwEMAbAVxiRAGEA9YAHR7rg4AvnwYwAZjgAUfAOJ0Atgrp8ATlgDmACxwBKEENLpMufIRQBGclVqMWbPgCFRE6X2U4tAY0bAAEiI86tp6BkYgGNh4BERWFjb0zKyIIO50nj4MwACqgWKSMjwe3r4Bapo6+obGUWZEZPHUifYpcorKBjYwUBrwRKDiqhAKSADM1DgQSABM1Ax0AEYwDAAKJtHmIME6IE12ySBMYQNDI4hkIJNIVrZJDjxoWAD6R9Ugg8NjE1PnE3RYDDYWggEAA1sd3qcZt9rn8AUCQeChBQhEA
    \begin{tikzcd}
      C^{\ast}\left(\Gamma\right) \arrow[r, "\pi_u"] & \B\left(\mathcal{H}\right)                          \\
      \Gamma \arrow[r, "u"'] \arrow[u, hook]         & \mathcal{U}\left(\mathcal{H}\right) \arrow[u, hook]
    \end{tikzcd}
  \end{center}
\end{theorem}
\begin{proof}
  From Proposition \ref{prop:unital_unitary_representation}, we know that there is a unital representation $\pi_u\colon \C\left[ \gamma \right]\rightarrow \B\left( \mathcal{H} \right)$ that extends $u\colon \Gamma\rightarrow \mathcal{U}\left( \mathcal{H} \right)$.\newline

  By the definition of the universal norm, it follows that $\norm{\pi_u(a)}_{\op}\leq \norm{a}_{u}$. The continuous extension $\pi_u\colon C^{\ast}\left( \Gamma \right)\rightarrow \B\left( \mathcal{H} \right)$ is thus a contractive $\ast$-homomorphism.
\end{proof}
\section{Ordering Properties of \texorpdfstring{$C^{\ast}$-Algebras}{C*-Algebras}}%
Recall from Definition \ref{def:positive_operators} that the space $\B\left( \mathcal{H} \right)_{\sa}$ admits an order structure --- we say that an operator is \textit{positive} if, for any $\xi\in \mathcal{H}$, we have $ \iprod{T\left( \xi \right)}{\xi} \geq 0 $. It can be shown that any positive operator is of the form $T = S^{\ast}S$, where $S$ is any operator on $\mathcal{H}$.\newline

Similarly, when we discussed algebras, we discussed a definition of positivity very similar to the case of bounded operators on Hilbert spaces (Definition \ref{def:distinguished_elements_of_algebras}).\newline

In this section, we investigate the ordering aspects of $C^{\ast}$-algebras in depth, and how to apply the properties of their spectra towards understanding ordering and positivity. This will lead naturally to discussions of positive and completely positive (linear) maps between $C^{\ast}$-algebras in the following section, paving the way to the cornucopia of characterizations of amenability that $C^{\ast}$-algebras admit.
\begin{definition}
  If $A$ is a $\ast$-algebra, and $a\in A$, then $a$ can be written as $h + ik$, where $h,k\in A_{\sa}$ are defined by
  \begin{align*}
    h &= \frac{1}{2}\left( a + a^{\ast} \right)\\
    k &= \frac{i}{2}\left( a^{\ast}-a \right).
  \end{align*}
  This is known as the \textit{Cartesian decomposition} of $a$.
\end{definition}
\begin{proposition}[{\cite[Proposition 7.3.61]{rainone_analysis}}]
  Let $A$ be a $C^{\ast}$-algebra, and let $h\in A_{\sa}$. Then, there exist unique positive elements $p,q\in A_{+}$ such that
  \begin{enumerate}[(a)]
    \item $h = p-q$;
    \item $pq = 0$;
    \item $\sigma\left( p \right),\sigma\left( q \right)\subseteq [0,\infty)$.
  \end{enumerate}
\end{proposition}
\begin{proof}
  Assume $A$ is unital. Let $\phi_h\colon C\left( \sigma\left( h \right) \right)\rightarrow C^{\ast}\left( h,1_A \right)$ be the continuous functional calculus at $h$. Since $\sigma\left( h \right)\subseteq \R$ (Proposition \ref{prop:spectra_cstar_algebras}), we consider the continuous functions
  \begin{align*}
    f(t) &= \max\set{t,0};
    g(t) &= \max\set{-t,0}.
  \end{align*}
  Note that $fg = 0$ and $\id_{\sigma\left( h \right)} = f-g$. Setting $p = \phi_h(f)$ and $q = \phi_h(g)$. Since $f$ and $g$ are positive, and the continuous functional calculus is a $\ast$-homomorphism, both $p$ and $q$ are positive.\newline

  Moreover, by spectral mapping (Theorem \ref{thm:continuous_functional_calculus}), we have
  \begin{align*}
    \sigma\left( p \right) &= \sigma\left( f\left( h \right) \right)\\
                           &= f\left( \sigma\left( h \right) \right)\\
                           &\subseteq [0,\infty),
  \end{align*}
  and similarly for $\sigma\left( q \right)$. Furthermore,
  \begin{align*}
    pq &= \phi_h\left( f \right)\phi_h\left( g \right)\\
       &= \phi_h\left( fg \right)\\
       &= \phi_h\left( 0 \right)\\
       &= 0
  \end{align*}
  and
  \begin{align*}
    h &= \phi_h\left( \id_{\sigma\left( h \right)} \right)\\
      &= \phi_h\left( f-g \right)\\
      &= \phi_h\left( f \right)-\phi_h\left( g \right)\\
      &= p-q.
  \end{align*}
\end{proof}

\section{Positive Maps in \texorpdfstring{$C^{\ast}$-Algebras}{C*-Algebras}}%

\section{Characterizing Amenability using \texorpdfstring{$C^{\ast}$-Algebras}{C*-Algebras}}%
