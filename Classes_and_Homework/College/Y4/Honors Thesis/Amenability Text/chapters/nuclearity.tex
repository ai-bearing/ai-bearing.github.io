Finally, we get to the most ``advanced'' chapter of this text --- here, we will establish the equivalence between group amenability and certain properties of the group $C^{\ast}$-algebra(s). The results in here will draw from a lot of theory that we discuss a bit more in depth in Chapter \ref{ch:operator_algebras}. Some excellent books that provide significantly more elaboration on this topic than this chapter include \cite{brown_and_ozawa} and \cite{completely_bounded_maps_and_operator_algebras}.
\section{Ordering Properties of $C^{\ast}$-Algebras}%
Recall from Definition \ref{def:positive_operators} that the space $\B\left( \mathcal{H} \right)_{\sa}$ admits an order structure --- we say that an operator is \textit{positive} if, for any $\xi\in \mathcal{H}$, we have $ \iprod{T\left( \xi \right)}{\xi} \geq 0 $. It can be shown that any positive operator is of the form $T = S^{\ast}S$, where $S$ is any operator on $\mathcal{H}$.\newline

Similarly, when we discussed algebras, we discussed a definition of positivity very similar to the case of bounded operators on Hilbert spaces (Definition \ref{def:distinguished_elements_of_algebras}). We have also mentioned that the spectrum (Definition \ref{def:spectrum}) is a fundamental construct inside unital algebras.\newline

In this section, we investigate the properties of $C^{\ast}$-algebras in depth, specifically related to how to apply the properties of their spectra towards understanding ordering and positivity. This will enable us in the next section to discuss positive maps, completely positive maps, and dilations, paving the way towards the cornucopia of characterizations of amenability that $C^{\ast}$-algebras enable.
\section{Positive Maps in \texorpdfstring{$C^{\ast}$-Algebras}{C*-Algebras}}%

\section{Characterizing Amenability using \texorpdfstring{$C^{\ast}$-Algebras}{C*-Algebras}}%
