\documentclass[10pt]{mypackage}

% sans serif font:
%\usepackage{cmbright,sfmath,bbold}
%\renewcommand{\mathcal}{\mathtt}

%Euler:
\usepackage{newpxtext,eulerpx,eucal,eufrak}
\renewcommand*{\mathbb}[1]{\varmathbb{#1}}
\renewcommand*{\hbar}{\hslash}

%kp fonts:
%\usepackage{kpfonts}
%\renewcommand{\mathbb}{\mathds}
%\usepackage{homework}

\pagestyle{fancy} %better headers
\fancyhf{}
\rhead{Avinash Iyer}
\lhead{More on $C^{\ast}$-Algebras}

\setcounter{secnumdepth}{0}

\begin{document}
\RaggedRight
\tableofcontents
\section{Introduction}%
Finally, the last part of my notes on $C^{\ast}$-algebras and amenability as part of my Honors Thesis independent study. Specifically, I am going to focus more on the theory of $C^{\ast}$-algebras, discussing ideas such as amenability and nuclearity in $C^{\ast}$-algebras. There are a few central results I'm going to be working on understanding and proving: almost-invariant vectors, Kesten's criterion, Hulanicki's criterion, nuclearity, and the equivalence of $C^{\ast}_{\lambda}\left(G\right)$ and $C^{\ast}\left(G\right)$.\newline

I will be using a variety of sources more focused on amenability, including but not limited to Volker Runde's \textit{Amenable Banach Algebras}, Kate Juschenko's \textit{Amenability of Discrete Groups by Examples}, and Brown and Ozawa's \textit{$C^{\ast}$-Algebras and Finite-Dimensional Approximations}.
\section{Review: Representations, the Reduced Group $C^{\ast}$-Algebra, and the Universal Group $C^{\ast}$-Algebra}%
\subsection{Left-Regular Representation}%
Let $\Gamma$ be a group. Consider the space $\ell_2\left(\Gamma\right)$. For every $s\in\Gamma$, we define the operator
\begin{align*}
  \lambda_s\left(\xi\right)\left(t\right) &= \xi\left(s^{-1}t\right).
\end{align*}
The map is linear, well-defined, and an isometry, as
\begin{align*}
  \norm{\lambda_s\left(\xi\right)}^2 &= \sum_{t\in\Gamma}\left\vert \lambda_s\left(\xi\right)\left(t\right) \right\vert^2\\
                                     &= \sum_{t\in\Gamma}\left\vert \xi\left(s^{-1}t\right) \right\vert^2\\
                                     &= \sum_{r\in\Gamma}\left\vert \xi\left(r\right) \right\vert^2\\
                                     &= \norm{\xi}^2.
\end{align*}
Additionally, each $\lambda_{s}$ admits an inverse, $\lambda_{s^{-1}} = \lambda_s^{\ast}$. Applying to the orthonormal basis $\set{\delta_t}_{t\in\Gamma}$, we get
\begin{align*}
  \lambda_s\left(\delta_t\right) &= \delta_{st}.
\end{align*}
Thus, $\lambda_{s}\circ \lambda_r = \lambda_{sr}$, and we have the unitary representation of $\Gamma$, $\lambda\colon \Gamma\rightarrow \mathcal{U}\left(\ell_2\left(\Gamma\right)\right)$, where $\lambda(s) = \lambda_s$, for $s\in \Gamma$. This is the left-regular representation of $\Gamma$.\newline
 
Note that the left regular representation is a faithful representation, hence injective.\newline

Because the $\lambda$ operator is linear, we may extend it to the case of any positive finitely supported function,
\begin{align*}
  \lambda_{f}\left(\xi\right)(t) &= \left(\sum_{s\in\Gamma}f(t)\lambda_{s}\left(\xi\right)\right)\left(t\right)\\
                                 &= \sum_{s\in\Gamma}f(s)\xi\left(s^{-1}t\right)
\end{align*}
Note that the space of finitely supported functions on $\Gamma$, $\C\left[\Gamma\right]$,\footnote{Also known as the free vector space over $\C$ with basis $\Gamma$.} is a $\ast$-algebra, where multiplication is given by convolution:
\begin{align*}
  f\ast g(t) &= \sum_{s\in\Gamma}f\left(s\right)g\left(s^{-1}t\right)\\
             &= \sum_{r\in\Gamma}f\left(tr^{-1}\right)g(r).
\end{align*}
Note that we are using $\ast$ both to refer to the involution (when as a superscript) as well as the group operation (when not a superscript). This is to maintain coherence with the traditional way that convolution is written. The involution on $\C\left[\Gamma\right]$ is given by
\begin{align*}
  f^{\ast}\left(t\right) &= \overline{f\left(t^{-1}\right)}.
\end{align*}
\subsection{A Bit on Representations and $C^{\ast}$-(Semi)norms}%
A $C^{\ast}$-seminorm on a $\ast$-algebra is a seminorm such that defined by
\begin{itemize}
  \item $\norm{ab}\leq \norm{a}\norm{b}$;
  \item $\norm{a^{\ast}} = \norm{a}$;
  \item $\norm{a^{\ast}a} = \norm{a}^2$.
\end{itemize}
If $A_0$ is a $\ast$-algebra, then a representation of $A_0$ is a pair $\left(\pi_0,\mathcal{H}\right)$, where $\mathcal{H}$ is a Hilbert space and $\pi\colon A_0\rightarrow \B\left(\mathcal{H}\right)$ is a $\ast$-homomorphism.\newline

Additionally, if $A_0$ is a $\ast$-algebra with representation $\pi_0$, then we have $C^{\ast}$-seminorm
\begin{align*}
  \norm{a}_{\pi_0} &= \norm{\pi_0\left(a\right)}_{\op}.
\end{align*}
If $\pi_0$ is injective, then $\norm{\cdot}_{\pi_0}$ is a $C^{\ast}$-norm. If $\pi_0$ is a $C^{\ast}$-norm, then the completion of $A_0$ with respect to $\norm{\cdot}_{\pi_0}$ is a $C^{\ast}$-algebra.\newline

The universal norm on $A_0$ is defined as
\begin{align*}
  \norm{a}_{u} &= \sup_{p\in \mathcal{P}}p(a),
\end{align*}
where $\mathcal{P}$ is the collection of all $C^{\ast}$-seminorms on $A_0$. If $\norm{a}_u < \infty$ for all $a\in A_0$, then $\norm{\cdot}_u$ is a $C^{\ast}$-seminorm on $A_0$. Note that if one of $p\in \mathcal{P}$ is a norm, then $\norm{\cdot}_{u}$ defines a $C^{\ast}$-norm on $A_0$.\newline

If we have the unitary representation $u\colon \C\left[\Gamma\right]\rightarrow \B\left(\mathcal{H}\right)$, then
\begin{align*}
  \pi_u(a) &= \sum_{s\in\Gamma}u_s
\end{align*}
is a representation of $\C\left[\Gamma\right]$. If $\lambda\colon \Gamma\rightarrow \mathcal{U}\left(\ell_2\left(\Gamma\right)\right)$ is the left-regular representation, then the left-regular group $C^{\ast}$-algebra is the group $\ast$-algebra with $C^{\ast}$-norm defined by $\norm{a} = \norm{\pi_{\lambda}(a)}$.\newline

The universal group $C^{\ast}$-algebra is defined as the norm completion of 
\begin{align*}
  \norm{a}_{\max} &= \sup\set{\norm{\pi\left(a\right)}_{\op} | \pi\colon \C\left[\Gamma\right]\rightarrow \B\left(\mathcal{H}_{\pi}\right) \text{ is a representation}}.
\end{align*}
Note that
\begin{align*}
  \norm{\pi\left(a\right)} &= \norm{\pi\left(\sum_{s\in\Gamma}a_s\delta_s\right)}\\
                           &= \norm{\sum_{s\in\Gamma}a_s\pi\left(\delta_s\right)}\\
                           &\leq \sum_{s\in\Gamma}\norm{a_s\pi\left(\delta_s\right)}\\
                           &= \sum_{s\in\Gamma}\left\vert a_s \right\vert.
\end{align*}
Note that since $\norm{\cdot}_{\lambda}$ is a norm, we must have $a=0$ if and only if $\norm{a}_{\max} = 0$. The full group $C^{\ast}$-algebra admits a universal property.
\begin{proposition}
  Let $\Gamma$ be a discrete group. If $u\colon \Gamma\rightarrow \B\left(\mathcal{H}\right)$, then there is a contractive $\ast$-homomorphism $\pi_u\colon C^{\ast}\left(\Gamma\right)\rightarrow \B\left(\mathcal{H}\right)$ that satisfies $\pi_u\left(\delta_s\right) = u(s)$.
\end{proposition}
\section{Using the Left-Regular Representation to Establish Amenability}%
If $\pi\colon \Gamma\rightarrow \mathcal{U}\left(\mathcal{H}\right)$ is a unitary representation of $\mathcal{H}$, then a vector $\xi\in \mathcal{H}$ is called invariant for $\pi$ if $\pi(g)\left(\xi\right) = \xi$ for all $g\in \Gamma$.
\begin{proposition}
  The left-regular representation for $\Gamma$ admits an invariant vector if and only if $\Gamma$ is finite.
\end{proposition}
\begin{proof}
  Let $\Gamma$ be finite. Since $\Gamma$ is finite, all functions $a\colon \Gamma\rightarrow \C$ are square-summable. Thus, $\xi = \1_{\Gamma}$ is square-summable, and since $s\Gamma = \Gamma$ for all $s\in\Gamma$, we have $\1_{\Gamma}$ is invariant for $\lambda$.\newline

  Now, let $\lambda\colon \Gamma\rightarrow \mathcal{U}\left(\ell_2\left(\Gamma\right)\right)$ be the left-regular representation, and suppose there is $\xi\in \ell_2\left(\Gamma\right)$ such that for all $s\in \Gamma$, we have
  \begin{align*}
    \lambda_s\left(\xi\right) &= \xi.
  \end{align*}
  In particular, this means that for any $t\in \Gamma$, we have
  \begin{align*}
    \lambda_s\left(\xi\right)\left(t\right) &= \xi\left(s^{-1}t\right)\\
                                            &= \xi\left(t\right).
  \end{align*}
  Since this holds for all $s\in \Gamma$, we have that $\xi = c\1_{\Gamma}$ for some $c\in \C$. However, since $\xi\in \ell_2\left(\Gamma\right)$, we must have that $\sum_{t\in\Gamma} \left\vert c \right\vert^2 < \infty$, which only holds if $\Gamma$ is finite.
\end{proof}
An almost-invariant vector for a representation $\pi\colon \Gamma\rightarrow \mathcal{U}\left(\ell_2\left(\Gamma\right)\right)$, as the name suggests,\footnote{I'm only mostly being facetious here.} a sequence (or net) of unit vectors $\left(\xi_i\right)_{i\in I}$ such that
\begin{align*}
  \lim_{i\in I}\norm{\pi(g)\left(\xi_i\right) - \xi_i} &= 0.
\end{align*}
\begin{theorem}
  A group $\Gamma$ is amenable if and only if the left-regular representation has an almost-invariant vector.
\end{theorem}
\begin{proof}
  Let $\Gamma$ be amenable, and let $F_i$ be a Følner sequence, where $\frac{\left\vert sF_i\triangle F_i \right\vert}{\left\vert F_i \right\vert}\rightarrow 0$ for all $s\in\Gamma$.\newline

  Define $\xi_i = \frac{1}{\sqrt{\left\vert F_i \right\vert}}\1_{F_i}$. Then,
  \begin{align*}
    \norm{\lambda_{s}\left(\xi_i\right) - \xi_i}^2 &= \sum_{t\in\Gamma} \left\vert \lambda_{s}\left(\xi_i\right)\left(t\right) - \xi_i\left(t\right) \right\vert^2\\
                                                   &= \sum_{t\in\Gamma} \left\vert \lambda_s\left(\frac{1}{\sqrt{\left\vert F_i \right\vert}}\1_{F_i}\right)\left(t\right) - \frac{1}{\sqrt{\left\vert F_i \right\vert}}\1_{F_i} \right\vert^2\\
                                                   &= \sum_{t\in\Gamma}\left\vert \frac{1}{\sqrt{\left\vert F_i \right\vert}}\1_{sF_i}(t) - \frac{1}{\sqrt{\left\vert F_i \right\vert}}\1_{sF_i}(t) \right\vert^2\\
                                                   &= \frac{\left\vert sF_i\triangle F_i \right\vert}{\left\vert F_i \right\vert}.
  \end{align*}
  Thus, $\lambda$ has an almost-invariant vector.\newline

  Suppose there exists an almost-invariant vector $\left(\xi_i\right)_i\in \ell_2\left(\Gamma\right)$. It is sufficient to construct an approximate mean. Since $\xi_i\in \ell_2\left(\Gamma\right)$, we have that $\xi_i^2\in \ell_1\left(\Gamma\right)$. Setting $\mu_i = \xi_i^2$, we plug this into the expression for an approximate mean, and obtain
  \begin{align*}
    \norm{\lambda_s\left(u_i\right) - u_i}_{\ell_1} &= \sum_{t\in\Gamma}\left\vert \lambda_s\left(\xi_i^2\right)\left(t\right) - \xi_i^2\left(t\right) \right\vert\\
                                                    &= \sum_{t\in\Gamma}\left\vert \left(\lambda_s\left(\xi_i\right)\left(t\right) - \xi_i\left(t\right)\right)\left(\lambda_s\left(\xi_i\right)\left(t\right) + \xi_i\left(t\right)\right) \right\vert\\
                                                    &= \norm{\left(\lambda_s\left(\xi_i\right) - \xi_i\right)\left(\lambda_s\left(\xi_i\right) + \xi_i\right)}_{\ell_1}\\
                                                    &\leq \norm{\lambda_s\left(\xi_i\right) - \xi_i}_{\ell_2}\norm{\lambda_s\left(\xi_i\right) + \xi_{i}}\\
                                                    &\leq 2\norm{\lambda_s\left(\xi_i\right) - \xi_i}\\
                                                    &\rightarrow 0.
  \end{align*}
  Thus, $\mu_i$ is an approximate mean.
\end{proof}
Using the criterion of almost invariant vectors, we may show that a group is amenable if and only if the trivial representation --- defined by $1_{\Gamma}\colon \Gamma\rightarrow \C$, $1_{\Gamma}(g) = 1$ is what is known as weakly contained in the left-regular representation.\newline

A representation $\pi\colon \Gamma\rightarrow \mathcal{U}\left(\mathcal{H}\right)$ is weakly contained in another representation $\rho\colon \Gamma\rightarrow \mathcal{U}\left(\mathcal{H}\right)$, denoted $\pi\prec \rho$, if for every $\xi\in \mathcal{H}$, finite $E\subseteq \Gamma$, and $\ve > 0$, then there are $\eta_1,\dots,\eta_n\in \mathcal{K}$ such that
\begin{align*}
  \left\vert \iprod{\pi(g)\left(\xi\right)}{\xi} - \sum_{i=1}^{n} \iprod{\rho(g)\left(\eta_i\right)}{\eta_i} \right\vert < \ve.
\end{align*}
\begin{theorem}
  A discrete group $\Gamma$ is amenable if and only if $1_{\Gamma}\prec \lambda$, where $\lambda$ is the left-regular representation.
\end{theorem}
\begin{proof}
  We show that $1_{\Gamma}\prec \lambda$ is equivalent to the existence of an almost invariant vector for $\lambda$. We assume $\lambda$ admits an almost-invariant vector. It is sufficient to show that for every $\ve > 0$ and every finite set $E\subseteq \Gamma$, there are $\eta_1,\dots,\eta_n\in \ell_2\left(\Gamma\right)$ such that
  \begin{align*}
    \left\vert 1-\sum_{i=1}^{n} \iprod{\lambda_t\left(\eta_i\right)}{\eta_i} \right\vert < \ve
  \end{align*}
  for every $t\in E$. If we take $n = 1$ and $\eta_1 = \xi$, where $\xi$ is almost-invariant for all $g\in E$ --- i.e., $\norm{\lambda_g\left(\xi\right) - \xi}_{\ell_2} < \ve$ for all $g\in E$. Note that we have
  \begin{align*}
    \norm{\lambda_g\left(\xi\right) - \xi}^2 &= \iprod{\lambda_g\left(\xi\right) - \xi}{\lambda_g\left(\xi\right) - \xi}\\
                                             &= \iprod{\lambda_g\left(\xi\right)}{\lambda_g\left(\xi\right)} + \iprod{\xi}{\xi} - 2\re\left( \iprod{\lambda_g\left(\xi\right)}{\xi}\right)\\
                                             &= 2 - 2\re\left( \iprod{\lambda_g\left(\xi\right)}{\xi}\right)\\
                                             &= 2\re\left(1 -  \iprod{\lambda_g\left(\xi\right)}{\xi}\right)\\
                                             &\leq 2\left\vert 1 - \iprod{\lambda_g\left(\xi\right)}{\xi} \right\vert.
  \end{align*}
  Additionally,
  \begin{align*}
    \left\vert 1- \iprod{\lambda_g\left(\xi\right)}{\xi} \right\vert^2 &= \left(1 - \iprod{\lambda_g\left(\xi\right)}{\xi}\right) \left( 1 - \overline{ \iprod{\lambda_g\left(\xi\right)}{\xi} } \right)\\
                                                                       &= 1 - \overline{ \iprod{\lambda_g\left(\xi\right)}{\xi} } - \iprod{\lambda_g\left(\xi\right)}{\xi} + \left\vert \iprod{\lambda_g\left(\xi\right)}{\xi} \right\vert^2\\
                                                                       &\leq 2 - 2\re\left( \iprod{\lambda_g\left(\xi\right)}{\xi}\right)\\
                                                                       &= \norm{\lambda_g\left(\xi\right) - \xi}^2.
  \end{align*}
  Thus, we have that
  \begin{align*}
    \left\vert 1- \iprod{\lambda_g\left(\xi\right)}{\xi} \right\vert &\leq \norm{ \lambda_g\left(\xi\right) - \xi }\\
                                                                     &< \ve.
  \end{align*}
%  Now, we suppose that $1_{\Gamma}\prec \lambda$. For every $\ve > 0$ and finite subset $E\subseteq \Gamma$, there are $\eta_1,\dots,\eta_n\in \ell_2\left(\Gamma\right)$ such that 
%  \begin{align*}
%    \left\vert 1 - \sum_{i=1}^{n} \iprod{\lambda_t\left(\eta_i\right)}{\eta_i} \right\vert &< \ve.
%  \end{align*}
%  for all $t\in E$. We may assume that $e_G\in E$, yielding
%  \begin{align*}
%    \left\vert 1-\sum_{i=1}^{n} \norm{\eta_i}^2 \right\vert < \ve.
%  \end{align*}
%  Furthermore, we may assume that $\sum_{i=1}^{n} \norm{\eta_i}^2 = 1$.\newline
%
%  Suppose toward contradiction that $\lambda$ does not have an almost-invariant vector. Then, there exists $C > 0$ and $S\subseteq \Gamma$ such that
%  \begin{align*}
%    \norm{\xi}^2\left\vert S \right\vert - \sum_{\gamma\in S} \iprod{\lambda_{\Gamma}\left(\xi\right)}{\xi} &> C\norm{\xi}^2.
%  \end{align*}
  We start by showing that $1_{\Gamma}\prec \lambda$ if and only if for every finite $S\subseteq \Gamma$ and every $\ve > 0$, there exists a unit vector $\xi\in \mathcal{H}$ such that
  \begin{align*}
    \norm{\lambda_s\left(\xi\right) - \xi}_{\ell_2} < \ve.
  \end{align*}
  In the forward direction, we see that there exists a unit vector $\xi$ such that $\left\vert 1 - \iprod{\lambda_s\left(\xi\right)}{\xi} \right\vert < \ve^2/2$, meaning $\norm{\lambda_s\left(\xi\right) - \xi} < \ve$ by above. Similarly, if $\norm{\lambda_s\left(\xi\right)-\xi} < \ve$, then $1_{\Gamma}\prec \lambda$.\newline

  Now, we assume $1_{\Gamma} \prec \lambda$. Thus, for a finite $E\subseteq \Gamma$ and $\ve > 0$, then there exists $f\in \ell_2\left(\Gamma\right)$ with $\norm{f}_{\ell_2} = 1$ such that $\norm{\lambda_s\left(f\right) - f} < \ve$ for all $s\in E$.\newline

  Setting $g = \left\vert f \right\vert^2$, we have $g\in \ell_1\left(\Gamma\right)$. From Hölder's inequality, we have
  \begin{align*}
    \norm{\lambda_s\left(g\right) - g}_{\ell_1} &\leq \norm{\lambda_s\left(\overline{f}\right) + \overline{f}}_{\ell_2} \norm{\lambda_s\left(f\right) - f}\\
                                                &\leq 2\norm{\lambda_s\left(f\right) - f}_{\ell_2}\\
                                                &< 2\ve.
  \end{align*}
  Thus, $\Gamma$ admits an approximate mean, hence is amenable.
\end{proof}
Having obtained some more resources on Kesten's criterion, we can now prove that.
\begin{definition}
  Let $\lambda\colon \Gamma\rightarrow \B\left(\ell_2\left(\Gamma\right)\right)$ be the left-regular representation. Then, for a finite set $E\subseteq \Gamma$, we define the Markov operator $M\left(E\right)$ by
  \begin{align*}
    M\left(E\right) &= \sum_{t\in E}\lambda_t.
  \end{align*}
\end{definition}
Note that since $\lambda_t$ is an isometry for each $t$, we have
\begin{align*}
  \norm{M\left(E\right)}_{\op} &= \norm{\frac{1}{\left\vert E \right\vert}\sum_{t\in E}\lambda_t}_{\op}\\
                               &= \frac{1}{\left\vert E \right\vert} \norm{\sum_{t\in E}\lambda_t}_{\op}\\
                               &\leq \frac{1}{\left\vert E \right\vert}\sum_{t\in E}\norm{\lambda_t}_{\op}\\
                               &= 1,
\end{align*}
so the Markov operator is a bounded operator (indeed, a contraction).
\begin{theorem}[Kesten's Criterion]
  Let $\Gamma$ contain a finite symmetric generating set $S$. Then, $\Gamma$ is amenable if and only if
  \begin{align*}
    \norm{M(S)}_{\op} &= 1.
  \end{align*}
\end{theorem}
\begin{proof}
  Let $\Gamma$ be amenable. Then, $\lambda$ admits an almost-invariant vector, $\left(\xi_n\right)_n\subseteq S_{\ell_2\left(\Gamma\right)}$, such that
  \begin{align*}
    \norm{\lambda_s\left(\xi_n\right) -\xi_n}_{\ell_2} \rightarrow 0
  \end{align*}
  for all $s\in \Gamma$. In particular, we have
  \begin{align*}
    \left\vert \left(\norm{\left(\frac{1}{\left\vert S \right\vert}\sum_{t\in S}\lambda_t\right)\left(\xi_n\right)}_{\ell_2}\right)  - \norm{\xi_n}_{\ell_2}\right\vert &\leq \norm{\left(\frac{1}{\left\vert S \right\vert}\sum_{t\in S}\lambda_t\right)\left(\xi_n\right) - \xi_n}_{\ell_2}\\
                                                                                                                                                                                  &= \frac{1}{\left\vert S \right\vert}\norm{\left(\sum_{t\in S}\lambda_t\right)\left(\xi_n\right) - \left\vert S \right\vert\xi_n}_{\ell_2}\\
                                                                                                                        &\leq \frac{1}{\left\vert S \right\vert} \sum_{t\in S}\norm{\lambda_t\left(\xi_n\right) - \xi_n}_{\ell_2}\\
                                                                                                                        &\rightarrow 0,
  \end{align*}
  meaning that
  \begin{align*}
    \sup_{\xi\in S_{\ell_2\left(\Gamma\right)}} \norm{\left(\frac{1}{\left\vert S \right\vert}\sum_{t\in S}\lambda_t\right)\left(\xi\right)} &= \norm{\xi},
  \end{align*}
  and so the norm of the Markov operator is $1$.\newline

  Suppose
  \begin{align*}
    \norm{\frac{1}{\left\vert S \right\vert}\sum_{t\in S}\lambda_t}_{\op} &= 1,
  \end{align*}
  or
  \begin{align*}
    \norm{\sum_{t\in S}\lambda_t}_{\op} &= \left\vert S \right\vert.
  \end{align*}
  \begin{proposition}
    If $T\in \B\left(\mathcal{H}\right)$ is a self-adjoint operator, then
    \begin{align*}
      \norm{T}_{\op} &= \sup_{x\in S_{\mathcal{H}}} \left\vert \iprod{T\left(x\right)}{x} \right\vert.
    \end{align*}
  \end{proposition}
  \begin{proof}
    We have that
    \begin{align*}
      \left\vert \iprod{T\left(x\right)}{x} \right\vert &\leq \norm{T\left(x\right)}\norm{x}\\
                                                        &\leq \norm{T}_{\op}\norm{x}^2\\
                                                        &= \norm{T}_{\op}.
    \end{align*}
    Now, we seek to establish the opposite direction. Note that since $T$ is self-adjoint, we know that $ \iprod{T\left(x\right)}{x}\in \R $ for any $x\in \mathcal{H}$, so by the polarization identity, we have that
    \begin{align*}
      \iprod{T\left(x\right)}{y} &= \frac{1}{4}\left( \iprod{T\left(x+y\right)}{x+y} - \iprod{T\left(x-y\right)}{x-y}\right).
    \end{align*}
    Note that we know that
    \begin{align*}
      \norm{T}_{\op} &= \sup_{x,y\in S_{\mathcal{H}}}\left\vert \iprod{T\left(x\right)}{y} \right\vert.
    \end{align*}
    Now, we set $\alpha = \sup_{x\in S_{\mathcal{H}}} \left\vert \iprod{T(x)}{x} \right\vert$. Note that for any nonzero $x\in \mathcal{H}$, we have
    \begin{align*}
      \left\vert \iprod{T\left(\frac{x}{\norm{x}}\right)}{\frac{x}{\norm{x}}} \right\vert &\leq \alpha\\
      \left\vert \iprod{T\left(x\right)}{x} \right\vert &\leq \alpha \norm{x}^2.
    \end{align*}
    Now, for any $x,y\in \mathcal{H}$, we may assume that $ \iprod{T\left(x\right)}{y} \in \R $, as we may multiply $ \iprod{T\left(x\right)}{y} $ by $\sgn \left( \iprod{T\left(x\right)}{y}\right)$. Thus, by the polarization identity and the fact that $T$ is self-adjoint, we have
    \begin{align*}
      \iprod{T\left(x\right)}{y} &= \frac{1}{4}\left( \iprod{T\left(x+y\right)}{x+y} - \iprod{T\left(x-y\right)}{x-y}\right)\\
      \left\vert \iprod{T\left(x\right)}{y} \right\vert &= \left\vert \frac{1}{4}\left( \iprod{T\left(x+y\right)}{x+y} - \iprod{T\left(x-y\right)}{x-y}\right) \right\vert\\
                                                        &\leq \frac{1}{4}\left( \left\vert \iprod{T\left(x+y\right)}{x+y} \right\vert + \left\vert \iprod{T\left(x-y\right)}{x-y} \right\vert\right)\\
                                                        &\leq \frac{\alpha}{4} \left(\norm{x+y}^2 + \norm{x-y}^2\right)\\
                                                        &= \frac{\alpha}{4}\left(2\norm{x}^2 + 2\norm{y}^2\right)\\
                                                        &= \alpha.
    \end{align*}
    Thus, we have $\norm{T}_{\op}\leq \sup_{x\in S_{\mathcal{H}}} \left\vert \iprod{T\left(x\right)}{x} \right\vert$.
  \end{proof}
  Now, since $S$ is symmetric, we have that $M(S)$ is self-adjoint. Therefore, we know that there is some $\xi_n\in S_{\mathcal{H}}$ such that
  \begin{align*}
    1-\frac{1}{n} &< \iprod{\left(\frac{1}{\left\vert S \right\vert}\sum_{t\in S}\lambda_t\right)\left(\xi_n\right)}{\xi_n}\\
                  &\leq \iprod{\left(\frac{1}{\left\vert S \right\vert}\sum_{t\in S}\lambda_t\right)\left(\left\vert \xi_n \right\vert\right)}{\left\vert \xi_n \right\vert}.
  \end{align*}
  Thus, rearranging, we have
  \begin{align*}
    1 - \iprod{\left(\frac{1}{\left\vert S \right\vert}\sum_{t\in S}\lambda_t\right)\left(\left\vert \xi_n \right\vert\right)}{\left\vert \xi_n \right\vert} &< \frac{1}{n}.
  \end{align*}
  Since $M(S)$ is a self-adjoint operator, we have that $\re\left( \iprod{\left( \frac{1}{\left\vert S \right\vert}\sum_{t\in S}\lambda_t \right)\left( \xi_n \right)}{\xi_n} \right) = \iprod{\left( \frac{1}{\left\vert S \right\vert}\sum_{t\in S}\lambda_t \right)\left( \xi_n \right)}{\xi_n}$. This gives
  \begin{align*}
    \norm{\left( \frac{1}{S}\sum_{t\in S}\lambda_t \right)\left( \xi \right) - \xi} &\leq \frac{1}{\left\vert S \right\vert}\sum_{t\in S}\norm{\lambda_t\left( \xi \right) - \xi}\\
                                                                                    &\leq \frac{1}{\left\vert S \right\vert}\sum_{t\in S}\sqrt{2}\left\vert 1 - \iprod{\lambda_t\left( \xi \right)}{\xi} \right\vert\\
                                                                                    &= \sqrt{2}\left\vert 1 - \frac{1}{\left\vert S \right\vert} \sum_{t\in S} \iprod{\lambda_t\left( \xi \right)}{\xi}\right\vert\\
                                                                                    &\rightarrow 0.
  \end{align*}
  Thus, $\lambda$ admits an almost-invariant vector.
\end{proof}
Next, we turn to Hulanicki's Criterion.
\begin{definition}
  Let $f\in \ell_1\left( \Gamma \right)$. Then, we define the bounded operator
  \begin{align*}
    \lambda_{f(t)} &= \sum_{t\in\Gamma}f(t)\lambda_t.
  \end{align*}
\end{definition}
\begin{theorem}
  If $\Gamma$ is a discrete group, then $\Gamma$ is amenable if and only if for every positive finitely-supported $f\colon \Gamma\rightarrow \C$, we have
  \begin{align*}
    \sum f(t) &\leq \norm{\lambda_{f(t)}}_{\op}.
  \end{align*}
\end{theorem}
\begin{proof}
  Suppose $\Gamma$ is amenable. Let $f \geq 0$ be a finitely supported function, and let $\left( F_n \right)_n$ be a Følner sequence such that for every $g\in \supp\left( f \right)$, we have
  \begin{align*}
    \frac{\left\vert gF_n\triangle F_n \right\vert}{\left\vert F_n \right\vert} &\leq \frac{1}{n}.
  \end{align*}
  Let $\xi_n = \frac{1}{\sqrt{\left\vert F_n \right\vert}}\1_{F_n}$. Note that $\norm{\xi_n}_{\ell_2} = 1$.\newline

  We will use the fact that
  \begin{align*}
    \sup_{x\in S_{\mathcal{H}}} \left\vert \iprod{T\left( x \right)}{x} \right\vert &\leq \norm{T}_{\op}.
  \end{align*}
  We see that
  \begin{align*}
    \left\vert \iprod{\left( \sum_{t\in\Gamma}f(t)\lambda_t \right)\left( \xi_n \right)}{\xi_n} \right\vert &= \left\vert \sum_{t\in\Gamma}f(t) \iprod{\lambda_t\left( \xi_n \right)}{\xi_n} \right\vert\\
                                                                                                            &= \left\vert \sum_{t,s\in\Gamma}f(t) \xi_n\left( t^{-1}s \right)\xi_n\left( s \right) \right\vert\\
                                                                                                            &\leq \norm{\lambda_{f(t)}},
                                                                                                            \intertext{meaning}
    \lim_{n\rightarrow\infty} \left\vert \iprod{\left( \sum_{t\in\Gamma}f(t)\lambda_t \right)\left( \xi_n \right)}{\xi_n} \right\vert &\leq \norm{\lambda_{f(t)}}.
  \end{align*}
  Notice that $\xi_n$ is an almost-invariant vector for $\lambda$, meaning that $\xi_n\left( t^{-1}s \right)\rightarrow \xi_n\left( s \right)$. Therefore, this means
  \begin{align*}
    \lim_{n\rightarrow\infty}\left\vert \sum_{t,s\in\Gamma}f(t)\xi_n\left( t^{-1}s \right)\xi_n\left( s \right) \right\vert &= \lim_{n\rightarrow\infty} \left\vert \sum_{t,s\in\Gamma}f(t)\left\vert \xi_n\left( s \right) \right\vert^2 \right\vert\\
                                                                                                                            &= \sum_{t\in\Gamma}f(t)\left\vert \sum_{s\in\Gamma}\left\vert \xi_n\left( s \right) \right\vert^2 \right\vert\\
                                                                                                                            &= \sum_{t\in\Gamma} f(t)\\
                                                                                                                            &\leq \norm{\lambda_{f(t)}}_{\op}.
  \end{align*}
  This establishes that there is some $C > 0$ such that
  \begin{align*}
    \sum_{t\in\Gamma}f(t) &\leq C\norm{\lambda_{f(t)}}_{\op}.
  \end{align*}
  To show that $C = 1$, we note that, by the definition of convolution, we must have
  \begin{align*}
    \left( \sum_{t\in\Gamma}f(t) \right)^{n} &= \sum_{t\in\Gamma} \left( f\ast \cdots \ast f \right)(t),
  \end{align*}
  and
  \begin{align*}
    \left( \lambda_{f(t)} \right)^{n} &= \left( \sum_{t\in\Gamma}f(t)\lambda)_t \right)^{n}\\
                                      &= \sum_{t\in\Gamma}\left( f\ast\cdots\ast f \right)\left( t \right)\lambda_t\\
                                      &= \lambda_{\left( f\ast\cdots\ast f \right)\left( t \right)}.
  \end{align*}
  Thus, we have
  \begin{align*}
    \left( \sum_{t\in\Gamma}f(t) \right)^{n}  &= \sum_{t\in\Gamma}\left( f\ast\cdots\ast f \right)\left( t \right)\\
                                              &\leq C\norm{\lambda_{\left( f\ast\cdots\ast f \right)\left( t \right)}}\\
                                              &= C\left( \norm{\lambda_{f(t)}}_{\op} \right)^n.
  \end{align*}
  This means we have
  \begin{align*}
    \sum_{t\in\Gamma}f(t) &\leq C^{1/n}\norm{\lambda_{f(t)}}_{\op}.
  \end{align*}
  Since $n$ is arbitrary, this means $C = 1$.\newline

  Now, if for all finitely supported $f$, we have
  \begin{align*}
    \sum_{t\in\Gamma} f(t) &\leq \norm{\lambda_{f(t)}}_{\op}.
  \end{align*}
  If $f = \1_{E}$ for some finite $E\subseteq \Gamma$, we see that
  \begin{align*}
    \norm{\sum_{t\in E}\lambda_{t}}_{\op} &= \left\vert E \right\vert,
  \end{align*}
  so by Kesten's criterion, we have that $\Gamma$ is amenable.
\end{proof}
\section{More theory of $C^{\ast}$-Algebras.}%
We begin this section with an overview of positive maps, completely positive maps, and extensions. These will be useful for understanding the theorem that a group is amenable if and only if the reduced group $C^{\ast}$-algebra is nuclear. The ultimate goal here is to prove Arveson's extension theorem (i.e., that $\B\left( \mathcal{H} \right)$ is injective with respect to completely positive maps).

\end{document}
