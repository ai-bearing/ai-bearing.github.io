\documentclass[10pt]{mypackage}

% sans serif font:
%\usepackage{cmbright,sfmath,bbold}
%\renewcommand{\mathcal}{\mathtt}

%Euler:
\usepackage{newpxtext,eulerpx,eucal,eufrak}
\renewcommand*{\mathbb}[1]{\varmathbb{#1}}
\renewcommand*{\hbar}{\hslash}

%kp fonts:
%\usepackage{kpfonts}
%\renewcommand{\mathbb}{\mathds}
%\usepackage{homework}

\pagestyle{fancy} %better headers
\fancyhf{}
\rhead{Avinash Iyer}
\lhead{More on $C^{\ast}$-Algebras}

\setcounter{secnumdepth}{0}

\begin{document}
\RaggedRight
\tableofcontents
\section{Introduction}%
Finally, the last part of my notes on $C^{\ast}$-algebras and amenability as part of my Honors Thesis independent study. Specifically, I am going to focus more on the theory of $C^{\ast}$-algebras, discussing ideas such as amenability and nuclearity in $C^{\ast}$-algebras. There are a few central results I'm going to be working on understanding and proving: almost-invariant vectors, Kesten's criterion, Hulanicki's criterion, nuclearity, and the equivalence of $C^{\ast}_{\lambda}\left(G\right)$ and $C^{\ast}\left(G\right)$.\newline

I will be using a variety of sources more focused on amenability, including but not limited to Volker Runde's \textit{Amenable Banach Algebras}, Kate Juschenko's \textit{Amenability of Discrete Groups by Examples}, and Brown and Ozawa's \textit{$C^{\ast}$-Algebras and Finite-Dimensional Approximations}.
\section{Review: Representations, the Reduced Group $C^{\ast}$-Algebra, and the Universal Group $C^{\ast}$-Algebra}%
\subsection{Left-Regular Representation}%
Let $\Gamma$ be a group. Consider the space $\ell_2\left(\Gamma\right)$. For every $s\in\Gamma$, we define the operator
\begin{align*}
  \lambda_s\left(\xi\right)\left(t\right) &= \xi\left(s^{-1}t\right).
\end{align*}
The map is linear, well-defined, and an isometry, as
\begin{align*}
  \norm{\lambda_s\left(\xi\right)}^2 &= \sum_{t\in\Gamma}\left\vert \lambda_s\left(\xi\right)\left(t\right) \right\vert^2\\
                                     &= \sum_{t\in\Gamma}\left\vert \xi\left(s^{-1}t\right) \right\vert^2\\
                                     &= \sum_{r\in\Gamma}\left\vert \xi\left(r\right) \right\vert^2\\
                                     &= \norm{\xi}^2.
\end{align*}
Additionally, each $\lambda_{s}$ admits an inverse, $\lambda_{s^{-1}} = \lambda_s^{\ast}$. Applying to the orthonormal basis $\set{\delta_t}_{t\in\Gamma}$, we get
\begin{align*}
  \lambda_s\left(\delta_t\right) &= \delta_{st}.
\end{align*}
Thus, $\lambda_{s}\circ \lambda_r = \lambda_{sr}$, and we have the unitary representation of $\Gamma$, $\lambda\colon \Gamma\rightarrow \mathcal{U}\left(\ell_2\left(\Gamma\right)\right)$, where $\lambda(s) = \lambda_s$, for $s\in \Gamma$. This is the left-regular representation of $\Gamma$.\newline
 
Note that the left regular representation is a faithful representation, hence injective.\newline

Because the $\lambda$ operator is linear, we may extend it to the case of any positive finitely supported function,
\begin{align*}
  \lambda_{f}\left(\xi\right)(t) &= \left(\sum_{s\in\Gamma}f(t)\lambda_{s}\left(\xi\right)\right)\left(t\right)\\
                                 &= \sum_{s\in\Gamma}f(s)\xi\left(s^{-1}t\right)
\end{align*}
Note that the space of finitely supported functions on $\Gamma$, $\C\left[\Gamma\right]$,\footnote{Also known as the free vector space over $\C$ with basis $\Gamma$.} is a $\ast$-algebra, where multiplication is given by convolution:
\begin{align*}
  f\ast g(t) &= \sum_{s\in\Gamma}f\left(s\right)g\left(s^{-1}t\right)\\
             &= \sum_{r\in\Gamma}f\left(tr^{-1}\right)g(r).
\end{align*}
Note that we are using $\ast$ both to refer to the involution (when as a superscript) as well as the group operation (when not a superscript). This is to maintain coherence with the traditional way that convolution is written. The involution on $\C\left[\Gamma\right]$ is given by
\begin{align*}
  f^{\ast}\left(t\right) &= \overline{f\left(t^{-1}\right)}.
\end{align*}
\subsection{A Bit on Representations and $C^{\ast}$-(Semi)norms}%
A $C^{\ast}$-seminorm on a $\ast$-algebra is a seminorm such that defined by
\begin{itemize}
  \item $\norm{ab}\leq \norm{a}\norm{b}$;
  \item $\norm{a^{\ast}} = \norm{a}$;
  \item $\norm{a^{\ast}a} = \norm{a}^2$.
\end{itemize}
If $A_0$ is a $\ast$-algebra, then a representation of $A_0$ is a pair $\left(\pi_0,\mathcal{H}\right)$, where $\mathcal{H}$ is a Hilbert space and $\pi\colon A_0\rightarrow \B\left(\mathcal{H}\right)$ is a $\ast$-homomorphism.\newline

Additionally, if $A_0$ is a $\ast$-algebra with representation $\pi_0$, then we have $C^{\ast}$-seminorm
\begin{align*}
  \norm{a}_{\pi_0} &= \norm{\pi_0\left(a\right)}_{\op}.
\end{align*}
If $\pi_0$ is injective, then $\norm{\cdot}_{\pi_0}$ is a $C^{\ast}$-norm. If $\pi_0$ is a $C^{\ast}$-norm, then the completion of $A_0$ with respect to $\norm{\cdot}_{\pi_0}$ is a $C^{\ast}$-algebra.\newline

The universal norm on $A_0$ is defined as
\begin{align*}
  \norm{a}_{u} &= \sup_{p\in \mathcal{P}}p(a),
\end{align*}
where $\mathcal{P}$ is the collection of all $C^{\ast}$-seminorms on $A_0$. If $\norm{a}_u < \infty$ for all $a\in A_0$, then $\norm{\cdot}_u$ is a $C^{\ast}$-seminorm on $A_0$. Note that if one of $p\in \mathcal{P}$ is a norm, then $\norm{\cdot}_{u}$ defines a $C^{\ast}$-norm on $A_0$.\newline

If we have the unitary representation $u\colon \C\left[\Gamma\right]\rightarrow \B\left(\mathcal{H}\right)$, then
\begin{align*}
  \pi_u(a) &= \sum_{s\in\Gamma}u_s
\end{align*}
is a representation of $\C\left[\Gamma\right]$. If $\lambda\colon \Gamma\rightarrow \mathcal{U}\left(\ell_2\left(\Gamma\right)\right)$ is the left-regular representation, then the left-regular group $C^{\ast}$-algebra is the group $\ast$-algebra with $C^{\ast}$-norm defined by $\norm{a} = \norm{\pi_{\lambda}(a)}$.\newline

The universal group $C^{\ast}$-algebra is defined as the norm completion of 
\begin{align*}
  \norm{a}_{u} &= \sup\set{\norm{\pi\left(a\right)}_{\op} | \pi\colon \C\left[\Gamma\right]\rightarrow \B\left(\mathcal{H}_{\pi}\right)}.
\end{align*}
Note that
\begin{align*}
  \norm{\pi\left(a\right)} &= \norm{\pi\left(\sum_{s\in\Gamma}a_s\delta_s\right)}\\
                           &= \norm{\sum_{s\in\Gamma}a_s\pi\left(\delta_s\right)}\\
                           &\leq \sum_{s\in\Gamma}\norm{a_s\pi\left(\delta_s\right)}\\
                           &= \sum_{s\in\Gamma}\left\vert a_s \right\vert.
\end{align*}
Note that since $\norm{\cdot}_{\lambda}$ is a norm, we must have $a=0$ if and only if $\norm{a}_{u} = 0$. The full group $C^{\ast}$-algebra admits a universal property.
\begin{proposition}
  Let $\Gamma$ be a discrete group. If $u\colon \Gamma\rightarrow \mathcal{U}\left(\mathcal{H}\right)$, then there is a contractive $\ast$-homomorphism $\pi_u\colon C^{\ast}\left(\Gamma\right)\rightarrow \B\left(\mathcal{H}\right)$ that satisfies $\pi_u\left(\delta_s\right) = u(s)$.
\end{proposition}
\section{Using the Left-Regular Representation to Establish Amenability}%
If $\pi\colon \Gamma\rightarrow \mathcal{U}\left(\mathcal{H}\right)$ is a unitary representation of $\mathcal{H}$, then a vector $\xi\in \mathcal{H}$ is called invariant for $\pi$ if $\pi(g)\left(\xi\right) = \xi$ for all $g\in \Gamma$.
\begin{proposition}
  The left-regular representation for $\Gamma$ admits an invariant vector if and only if $\Gamma$ is finite.
\end{proposition}
\begin{proof}
  Let $\Gamma$ be finite. Since $\Gamma$ is finite, all functions $a\colon \Gamma\rightarrow \C$ are square-summable. Thus, $\xi = \1_{\Gamma}$ is square-summable, and since $s\Gamma = \Gamma$ for all $s\in\Gamma$, we have $\1_{\Gamma}$ is invariant for $\lambda$.\newline

  Now, let $\lambda\colon \Gamma\rightarrow \mathcal{U}\left(\ell_2\left(\Gamma\right)\right)$ be the left-regular representation, and suppose there is $\xi\in \ell_2\left(\Gamma\right)$ such that for all $s\in \Gamma$, we have
  \begin{align*}
    \lambda_s\left(\xi\right) &= \xi.
  \end{align*}
  In particular, this means that for any $t\in \Gamma$, we have
  \begin{align*}
    \lambda_s\left(\xi\right)\left(t\right) &= \xi\left(s^{-1}t\right)\\
                                            &= \xi\left(t\right).
  \end{align*}
  Since this holds for all $s\in \Gamma$, we have that $\xi = c\1_{\Gamma}$ for some $c\in \C$. However, since $\xi\in \ell_2\left(\Gamma\right)$, we must have that $\sum_{t\in\Gamma} \left\vert c \right\vert^2 < \infty$, which only holds if $\Gamma$ is finite.
\end{proof}
An almost-invariant vector for a representation $\pi\colon \Gamma\rightarrow \mathcal{U}\left(\ell_2\left(\Gamma\right)\right)$, as the name suggests,\footnote{I'm only mostly being facetious here.} a sequence (or net) of unit vectors $\left(\xi_i\right)_{i\in I}$ such that
\begin{align*}
  \lim_{i\in I}\norm{\pi(g)\left(\xi_i\right) - \xi_i} &= 0.
\end{align*}
\begin{theorem}
  A group $\Gamma$ is amenable if and only if the left-regular representation has an almost-invariant vector.
\end{theorem}
\begin{proof}
  Let $\Gamma$ be amenable, and let $F_i$ be a Følner sequence --- $\frac{\left\vert sF_i\triangle F_i \right\vert}{\left\vert F_i \right\vert}\rightarrow 0$ for all $s\in\Gamma$. Define $\xi_i = \frac{1}{\sqrt{\left\vert F_i \right\vert}}\1_{F_i}$. Thus,
  \begin{align*}
    \norm{\lambda_{s}\left(\xi_i\right) - \xi_i}^2 &= \sum_{t\in\Gamma} \left\vert \lambda_{s}\left(\xi_i\right)\left(t\right) - \xi_i\left(t\right) \right\vert^2\\
                                                   &= \sum_{t\in\Gamma} \left\vert \lambda_s\left(\frac{1}{\sqrt{\left\vert F_i \right\vert}}\1_{F_i}\right)\left(t\right) - \frac{1}{\sqrt{\left\vert F_i \right\vert}}\1_{F_i} \right\vert^2\\
                                                   &= \sum_{t\in\Gamma}\left\vert \frac{1}{\sqrt{\left\vert F_i \right\vert}}\1_{sF_i}(t) - \frac{1}{\sqrt{\left\vert F_i \right\vert}}\1_{sF_i}(t) \right\vert^2\\
                                                   &= \frac{\left\vert sF_i\triangle F_i \right\vert}{\left\vert F_i \right\vert}.
  \end{align*}
  Thus, $\lambda$ has an almost-invariant vector.\newline

  Suppose there exists an almost-invariant vector $\left(\xi_i\right)_i\in \ell_2\left(\Gamma\right)$. It is sufficient to construct an approximate mean. Since $\xi_i\in \ell_2\left(\Gamma\right)$, we have that $\xi_i^2\in \ell_1\left(\Gamma\right)$. Setting $\mu_i = \xi_i^2$, we plug this into the expression for an approximate mean, and obtain
  \begin{align*}
    \norm{\lambda_s\left(u_i\right) - u_i}_{\ell_1} &= \sum_{t\in\Gamma}\left\vert \lambda_s\left(\xi_i^2\right)\left(t\right) - \xi_i^2\left(t\right) \right\vert\\
                                                    &= \sum_{t\in\Gamma}\left\vert \left(\lambda_s\left(\xi_i\right)\left(t\right) - \xi_i\left(t\right)\right)\left(\lambda_s\left(\xi_i\right)\left(t\right) + \xi_i\left(t\right)\right) \right\vert\\
                                                    &= \norm{\left(\lambda_s\left(\xi_i\right) - \xi_i\right)\left(\lambda_s\left(\xi_i\right) + \xi_i\right)}_{\ell_1}\\
                                                    &\leq \norm{\lambda_s\left(\xi_i\right) - \xi_i}_{\ell_2}\norm{\lambda_s\left(\xi_i\right) + \xi_{i}}\\
                                                    &\leq 2\norm{\lambda_s\left(\xi_i\right) - \xi_i}\\
                                                    &\rightarrow 0.
  \end{align*}
  Thus, $\mu_i$ is an approximate mean.
\end{proof}
Using the criterion of almost invariant vectors, we may show that a group is amenable if and only if the trivial representation --- defined by $1_{\Gamma}\colon \Gamma\rightarrow \C$, $1_{\Gamma}(g) = 1$ is what is known as weakly contained in the left-regular representation.\newline

A representation $\pi\colon \Gamma\rightarrow \mathcal{U}\left(\mathcal{H}\right)$ is weakly contained in another representation $\rho\colon \Gamma\rightarrow \mathcal{U}\left(\mathcal{H}\right)$, denoted $\pi\prec \rho$, if for every $\xi\in \mathcal{H}$, finite $E\subseteq \Gamma$, and $\ve > 0$, then there are $\eta_1,\dots,\eta_n\in \mathcal{K}$ such that
\begin{align*}
  \left\vert \iprod{\pi(g)\left(\xi\right)}{\xi} - \sum_{i=1}^{n} \iprod{\rho(g)\left(\eta_i\right)}{\eta_i} \right\vert < \ve.
\end{align*}
\begin{theorem}
  A discrete group $\Gamma$ is amenable if and only if $1_{\Gamma}\prec \lambda$, where $\lambda$ is the left-regular representation.
\end{theorem}
\begin{proof}
  We show that $1_{\Gamma}\prec \lambda$ is equivalent to the existence of an almost invariant vector for $\lambda$. We assume $\lambda$ admits an almost-invariant vector. It is sufficient to show that for every $\ve > 0$ and every finite set $E\subseteq \Gamma$, there are $\eta_1,\dots,\eta_n\in \ell_2\left(\Gamma\right)$ such that
  \begin{align*}
    \left\vert 1-\sum_{i=1}^{n} \iprod{\lambda_t\left(\eta_i\right)}{\eta_i} \right\vert < \ve
  \end{align*}
  for every $t\in E$. If we take $n = 1$ and $\eta_1 = \xi$, where $\xi$ is almost-invariant for all $g\in E$ --- i.e., $\norm{\lambda_g\left(\xi\right) - \xi}_{\ell_2} < \ve$ for all $g\in E$. Note that we have
  \begin{align*}
    \norm{\lambda_g\left(\xi\right) - \xi}^2 &= \iprod{\lambda_g\left(\xi\right) - \xi}{\lambda_g\left(\xi\right) - \xi}\\
                                             &= \iprod{\lambda_g\left(\xi\right)}{\lambda_g\left(\xi\right)} + \iprod{\xi}{\xi} - 2\re\left( \iprod{\lambda_g\left(\xi\right)}{\xi}\right)\\
                                             &= 2 - 2\re\left( \iprod{\lambda_g\left(\xi\right)}{\xi}\right)\\
                                             &= 2\re\left(1 -  \iprod{\lambda_g\left(\xi\right)}{\xi}\right)\\
                                             &\leq 2\left\vert 1 - \iprod{\lambda_g\left(\xi\right)}{\xi} \right\vert.
  \end{align*}
  Additionally,
  \begin{align*}
    \left\vert 1- \iprod{\lambda_g\left(\xi\right)}{\xi} \right\vert^2 &= \left(1 - \iprod{\lambda_g\left(\xi\right)}{\xi}\right) \left\vert 1 - \overline{ \iprod{\lambda_g\left(\xi\right)}{\xi} } \right\vert\\
                                                                       &= 1 - \overline{ \iprod{\lambda_g\left(\xi\right)}{\xi} } - \iprod{\lambda_g\left(\xi\right)}{\xi} + \left\vert \iprod{\lambda_g\left(\xi\right)}{\xi} \right\vert^2\\
                                                                       &\leq 2 - 2\re\left( \iprod{\lambda_g\left(\xi\right)}{\xi}\right)\\
                                                                       &= \norm{\lambda_g\left(\xi\right) - \xi}^2.
  \end{align*}
  Thus, we have that
  \begin{align*}
    \left\vert 1- \iprod{\lambda_g\left(\xi\right)}{\xi} \right\vert &\leq \norm{ \lambda_g\left(\xi\right) - \xi }\\
                                                                     &< \ve.
  \end{align*}
  Now, we suppose that $1_{\Gamma}\prec \lambda$. For every $\ve > 0$ and finite subset $E\subseteq \Gamma$, there are $\eta_1,\dots,\eta_n\in \ell_2\left(\Gamma\right)$ such that 
  \begin{align*}
    \left\vert 1 - \sum_{i=1}^{n} \iprod{\lambda_t\left(\eta_i\right)}{\eta_i} \right\vert &< \ve.
  \end{align*}
  for all $t\in E$. We may assume that $e_G\in E$, yielding
  \begin{align*}
    \left\vert 1-\sum_{i=1}^{n} \norm{\eta_i}^2 \right\vert < \ve.
  \end{align*}
  Furthermore, we may assume that $\sum_{i=1}^{n} \norm{\eta_i}^2 = 1$.\newline

  Suppose toward contradiction that $\lamda$ does not have an almost-invariant vector. Then, there exists $C > 0$ and $S\subseteq \Gamma$ such that
  \begin{align*}
    \norm{\xi}^2\left\vert S \right\vert - \sum_{\gamma\in S} \iprod{\lambda_{\Gamma}\left(\xi\right)}{\xi} &> C\norm{\xi}^2.
  \end{align*}
  
\end{proof}

\end{document}
