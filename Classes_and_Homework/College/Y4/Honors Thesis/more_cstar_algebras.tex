\documentclass[10pt]{mypackage}

% sans serif font:
%\usepackage{cmbright,sfmath,bbold}
%\renewcommand{\mathcal}{\mathtt}

%Euler:
\usepackage{newpxtext,eulerpx,eucal,eufrak}
\renewcommand*{\mathbb}[1]{\varmathbb{#1}}
\renewcommand*{\hbar}{\hslash}

%kp fonts:
%\usepackage{kpfonts}
%\renewcommand{\mathbb}{\mathds}
%\usepackage{homework}

\pagestyle{fancy} %better headers
\fancyhf{}
\rhead{Avinash Iyer}
\lhead{More on $C^{\ast}$-Algebras}

\setcounter{secnumdepth}{0}

\begin{document}
\RaggedRight
\tableofcontents
\section{Introduction}%
Finally, the last part of my notes on $C^{\ast}$-algebras and amenability as part of my Honors Thesis independent study. Specifically, I am going to focus more on the theory of $C^{\ast}$-algebras, discussing ideas such as amenability and nuclearity in $C^{\ast}$-algebras. There are a few central results I'm going to be working on understanding and proving: almost-invariant vectors, Kesten's criterion, Hulanicki's criterion, nuclearity, and the equivalence of $C^{\ast}_{\lambda}\left(G\right)$ and $C^{\ast}\left(G\right)$.\newline

I will be using a variety of sources more focused on amenability, including but not limited to Volker Runde's \textit{Amenable Banach Algebras}, Kate Juschenko's \textit{Amenability of Discrete Groups by Examples}, and Brown and Ozawa's \textit{$C^{\ast}$-Algebras and Finite-Dimensional Approximations}.
\section{Review: Representations, the Reduced Group $C^{\ast}$-Algebra, and the Universal Group $C^{\ast}$-Algebra}%
Let $\Gamma$ be a group. Consider the space $\ell_2\left(\Gamma\right)$. For every $s\in\Gamma$, we define the operator
\begin{align*}
  \lambda_s\left(\xi\right)\left(t\right) &= \xi\left(s^{-1}t\right).
\end{align*}
The map is linear, well-defined, and an isometry, as
\begin{align*}
  \norm{\lambda_s\left(\xi\right)}^2 &= \sum_{t\in\Gamma}\left\vert \lambda_s\left(\xi\right)\left(t\right) \right\vert^2\\
                                     &= \sum_{t\in\Gamma}\left\vert \xi\left(s^{-1}t\right) \right\vert^2\\
                                     &= \sum_{r\in\Gamma}\left\vert \xi\left(r\right) \right\vert^2\\
                                     &= \norm{\xi}^2.
\end{align*}
Additionally, each $\lambda_{s}$ admits an inverse, $\lambda_{s^{-1}} = \lambda_s^{\ast}$. Applying to the orthonormal basis $\set{\delta_t}_{t\in\Gamma}$, we get
\begin{align*}
  \lambda_s\left(\delta_t\right) &= \delta_{st}.
\end{align*}
Thus, $\lambda_{s}\circ \lambda_r = \lambda_{sr}$, and we have the unitary representation of $\Gamma$, $\lambda\colon \Gamma\rightarrow \mathcal{U}\left(\ell_2\left(\Gamma\right)\right)$, where $\lambda(s) = \lambda_s$, for $s\in \Gamma$. This is the left-regular representation of $\Gamma$.\newline
 
Note that the left regular representation is a faithful representation, hence injective.\newline

Because the $\lambda$ operator is linear, we may extend it to the case of any positive finitely supported function,
\begin{align*}
  \lambda_{f}\left(\xi\right)(t) &= \left(\sum_{s\in\Gamma}f(t)\lambda_{s}\left(\xi\right)\right)\left(t\right)\\
                                 &= \sum_{s\in\Gamma}f(s)\xi\left(s^{-1}t\right)
\end{align*}
Note that the space of finitely supported functions on $\Gamma$, $\C\left[\Gamma\right]$,\footnote{Also known as the free vector space over $\C$ with basis $\Gamma$.} is a $\ast$-algebra, where multiplication is given by convolution:
\begin{align*}
  f\ast g(t) &= \sum_{s\in\Gamma}f\left(s\right)g\left(s^{-1}t\right)\\
             &= \sum_{r\in\Gamma}f\left(tr^{-1}\right)g(r).
\end{align*}
Note that we are using $\ast$ both to refer to the involution (when as a superscript) as well as the group operation (when not a superscript). This is to maintain coherence with the traditional way that convolution is written. The involution on $\C\left[\Gamma\right]$ is given by
\begin{align*}
  f^{\ast}\left(t\right) &= \overline{f\left(t^{-1}\right)}.
\end{align*}
\end{document}
