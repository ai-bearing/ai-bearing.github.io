\documentclass[10pt]{mypackage}

% sans serif font:
%\usepackage{cmbright}
%\usepackage{sfmath}
%\usepackage{bbold} %better blackboard bold

%serif font + different blackboard bold for serif font
\usepackage{newpxtext,eulerpx}
\renewcommand*{\mathbb}[1]{\varmathbb{#1}}

\pagestyle{fancy}
\fancyhf{}
\rhead{Avinash Iyer}
\lhead{Amenability}

\setcounter{secnumdepth}{0}

\begin{document}
\RaggedRight
\section{Introduction}%
This is going to be the notes from my Honors Thesis project on amenability. We will be covering different results that are used to show that a topological group has a translation-invariant finitely additive probability measure (i.e., a mean).\newline

The primary source texts to inform this independent study will be Volker Runde's \textit{Lectures on Amenability} and Timothy Rainone's \textit{Functional Analysis-en route to Operator Algebras}, as well as various notes compiled by my professor.
\section{Group Actions, Paradoxical Decompositions, and the Banach--Tarski Paradox}%
In order to introduce Tarski's theorem, which is where our first condition about the amenability of groups appears, we begin by discussing paradoxical decompositions, with the goal of this section being a proof of the Banach--Tarski Paradox. The Banach--Tarski paradox says the following:
\begin{quote}
  If $A$ and $B$ are any bounded subsets of $\R^{3}$ with nonempty interior, there is a partition of $A$ into finitely many disjoint subsets such that a sequence of isometries applied to these subsets yields $B$.
\end{quote}
\subsection{Basics of Group Actions}%
The information for these essentials about group actions will be drawn from Dummit and Foote's \textit{Abstract Algebra}.
\begin{definition}[Group Action]
  A (left) group action of $G$ onto a set $A$ is a map from $G\times A$ to $A$ that satisfies:
  \begin{itemize}
    \item $g_1\cdot \left(g_2\cdot a\right) = \left(g_1g_2\right)\cdot a$ for all $g_1,g_2\in G$ and $a\in A$;
    \item $e\cdot a = a$\footnote{The identity element is usually written as $1$, but I will write it as $e$ out of familiarity.} for all $a\in A$.
  \end{itemize}
\end{definition}
\begin{definition}[Permutation Representation]
  For each $g$, the map $\sigma_g: A\rightarrow A$ defined by $\sigma_g(a) = g\cdot a$ (the group element $g$ acts on $a$) is a permutation of $A$. There is a homomorphism associated to these actions:
  \begin{align*}
    \varphi: G\rightarrow S_A,
  \end{align*}
  where $\varphi(g) = \sigma_g$. Recall that $S_A$ is the symmetric group (group of permutations) on the elements of $A$. \newline

  This is the permutation representation for the action.\newline

  In particular, given any nonempty set $A$ and a homomorphism $G$ into $S_A$, we can define an action of $G$ on $A$ by taking $g\cdot a = \varphi(g)(a)$.
\end{definition}
\begin{definition}[Kernel]
  The kernel of the action of $G$ is the set of elements in $g$ that act trivially on $A$:
  \begin{align*}
    \set{g\in G\mid \forall a\in A,~g\cdot a = a}
  \end{align*}
\end{definition}
\begin{note}
  The kernel of the action is the kernel of the permutation representation $\varphi: G\rightarrow S_{A}$.
\end{note}
\begin{definition}[Stabilizer]
  For each $a\in A$, the stabilizer of $a$ under $G$ is the set of elements in $G$ that fix $a$:
  \begin{align*}
    G_a &= \set{g\in G\mid g\cdot a = a}.
  \end{align*}
\end{definition}
\begin{note}
  The kernel of the group action is the intersection of the stabilizers of every element of $G$:
  \begin{align*}
    \text{kernel} &= \bigcap_{a\in A}G_a.
  \end{align*}
\end{note}
\begin{note}
  For each $a\in A$, $G_a$ is a subgroup of $G$.
\end{note}
\begin{definition}[Faithful Action]
  An action is faithful if the kernel of the action is $e$.
\end{definition}
\begin{definition}[Free Action]
  For a set $X$ with $G$ acting on $X$, the action of $G$ on $X$ is free if, for every $x$, $g\cdot x = x$ if and only if $g = e_G$.\newline

  If the action of $G$ on $X$ is a free action, we say $G$ acts freely on $X$.
\end{definition}
\begin{proposition}[Equivalence Relation on $A$]
  Let $G$ be a group that acts on a nonempty set $A$. We define a relation $a\sim b$ if and only if $a = g\cdot b$ for some $g\in G$. This is an equivalence relation, with the number of elements in $\left[a\right]_{\sim}$ found by taking $\left\vert G:G_a \right\vert$, which is the index of the stabilizer of $a$.
\end{proposition}
\begin{proof}
  We can see that $a\sim a$, since $e\cdot a = a$. Similarly, we can see that if $a\sim b$, then $b = g^{-1}\cdot a$, meaning $b\sim a$. Finally, let $a\sim b$ and $b\sim c$. Then, we have $a = g\cdot b$ for some $g\in G$, and $b = h\cdot c$ for some $h\in G$. Thus, we have
  \begin{align*}
    a &= g\cdot \left(h\cdot c\right)\\
      &= \left(gh\right)\cdot c,
  \end{align*}
  meaning $a\sim c$.\newline

  We say there is a bijection between the left cosets of $G_a$ and the elements of the equivalence class of $a$.\newline

  Define $\mathcal{C}_a$ to be the set $\set{g\cdot a\mid g\in G}$, and let $b = g\cdot a$. Define a map $g\cdot a \mapsto gG_a$. This map is surjective since $g\cdot a$ is always an element of $\mathcal{C}_a$. Additionally, since $g\cdot a = h\cdot a$ if and only if $\left(h^{-1}g\right)\cdot a = a$, meaning $h^{-1}g \in G_a$, and $h^{-1}g\in G_a$ if and only if $gG_a = hG_a$, this map is injective.\newline

  Since there is a one-to-one map between the equivalence classes of $a$ under the action of $G$, and the number of left cosets of $G_a$, we now know that the number of equivalence classes of $a$ under the action of $G$ is $\left\vert G:G_a \right\vert$.
\end{proof}
\begin{definition}[Orbit]
  For any $a\in A$, we define the orbit under $G$ of $a$ by
  \begin{align*}
    G\cdot a &= \set{b\in A\mid \forall g\in G,~b = g\cdot a}
  \end{align*}
  In particular, if $c\in G\cdot a$ for some $a\in A$, then $G\cdot c = G\cdot a$.
\end{definition}
\subsection{Paradoxical Decompositions}%
Most of the information from this section will be drawn from Volker Runde's \textit{Lectures on Amenability}.
\begin{definition}[Paradoxical Sets and Decompositions]
  Let $G$ be a group that acts on a set $X$. Let $E\subseteq X$.\newline

  If there exist pairwise disjoint $A_1,\dots,A_n,B_1,\dots,B_m\subseteq E$ and $g_1,\dots,g_n,h_1,\dots,h_m\in G$ such that
  \begin{align*}
    E &= \bigcup_{j=1}^{n}g_j\cdot A_j
    \intertext{and}
    E &= \bigcup_{j=1}^{m}h_j\cdot B_j,
  \end{align*}
  then we say that $E$ is $G$-paradoxical.\newline

  In particular, a paradoxical group is one where $G$ acts on itself by left-multiplication.
\end{definition}
\begin{example}[Our First Paradoxical Group]
  The free group on two generators, $\mathbb{F}\left(a,b\right)$,\footnote{The set of all reduced words over $\set{a,b,a^{-1},b^{-1},e_{\F(a,b)}}$. In particular, a word is reduced when the pairs $aa^{-1}$ and $bb^{-1}$ are replaced with the identity $e_{\F(a,b)}$.} is paradoxical. To see this, we let
  \begin{align*}
    W(x) &= \set{w\in \F(a,b)\mid w\text{ starts with }x}.
  \end{align*}
  Here, ``starts with'' refers to the left-most element. For instance, $ba^2ba^{-1}\in W\left(b\right)$.\newline

  In particular, we can see that
  \begin{align*}
    \F(a,b) &= \set{e_{\F(a,b)}} \sqcup W(a) \sqcup W(b) \sqcup W\left(a^{-1}\right)\sqcup W\left(b^{-1}\right).
  \end{align*}
  For any $w\in \F(a,b)\setminus W(a)$, we can see that $a^{-1}w\in W\left(a^{-1}\right)$, meaning $w\in aW\left(a^{-1}\right)$. Therefore, $\F(a,b) = W(a)\sqcup aW\left(a^{-1}\right)$.\newline

  Similarly, for any $v\in \F(a,b)\setminus W(b)$, $b^{-1}v \in W\left(b^{-1}\right)$, so $v \in bW\left(b^{-1}\right)$. Therefore, $\F(a,b) = W(b) \sqcup bW\left(b^{-1}\right)$.
\end{example}
\begin{proposition}[Free Action of a Paradoxical Group]
  Let $G$ be a paradoxical group that acts freely on $X$. Then, $X$ is $G$-paradoxical.
\end{proposition}
\begin{proof}
  Let $A_1,\dots,A_n,B_1,\dots,B_m\subseteq G$ be pairwise disjoint, with $g_1,\dots,g_n,h_1,\dots,h_m\in G$ such that
  \begin{align*}
    G &= \bigcup_{j=1}^{n}g_j A_j\\
      &= \bigcup_{j=1}^{m}h_jB_j.
  \end{align*}
  We let $M\subseteq X$ contain exactly one element from every orbit of $G$.\newline

  The set $\set{g\cdot M\mid g\in G}$ is a partition of $X$. Since $M$ contains exactly one element from every orbit of $G$, it is then the case that $\bigcup_{g\in G}g\cdot M = X$, since $G\cdot M = X$.\newline

  Additionally, if $x,y\in M$ with $g\cdot x = h\cdot y$, then $\left(h^{-1}g\right)\cdot x = y$, meaning $y$ is in the orbit of $x$ and vice versa, implying $x = y$. Thus, we must have $h^{-1}g = e_G$, as we assume $G$ acts freely.\newline

  Thus, we can see that $g_1\cdot M \neq g_2\cdot M$ if $g_1\neq g_2$, meaning $\set{g\cdot M\mid g\in G}$ is a partition.\newline

  Define $A_{j}^{\ast}$ to be the subset of $X$ that is the result of the elements of $A_j$ acting on $M$. In other words,
  \begin{align*}
    A_j^{\ast} &= \bigcup_{g\in A_j}g\cdot M.
  \end{align*}
  As a useful shorthand, we can say $A_j^{\ast} = A_j\cdot M$.\footnote{Yes, I know that $A_j$ is not technically a group acting on $M$, but this will help illuminate the final conclusion.} Similarly, we define
  \begin{align*}
    B_j^{\ast} &= \bigcup_{h\in B_j}h\cdot M\\
               &= B_j\cdot M.
  \end{align*}
  We can see that $A_1^{\ast},A_2^{\ast},\dots,A_n^{\ast},B_1^{\ast},B_2^{\ast},\dots,B_m^{\ast}\subseteq X$ are disjoint, since $\set{g\cdot M\mid g\in G}$ is a partition, and $A_1,\dots,A_n,B_1,\dots,B_m$ are disjoint in $G$.\newline

  Thus, we have
  \begin{align*}
    \bigcup_{j=1}^{n}g_j\cdot A_j^{\ast} &= \bigcup_{j=1}^{n} \left(g_jA_j\right)\cdot M\\
                                         &= G\cdot M\\
                                         &= X.
  \end{align*}
  Similarly,
  \begin{align*}
    \bigcup_{j=1}^{m}h_j\cdot B_j^{\ast} &= \bigcup_{j=1}^{m}\left(h_jB_j\right)\cdot M\\
                                         &= G\cdot M\\
                                         &= X.
  \end{align*}
  Thus, we see that $X$ has a paradoxical decomposition, meaning $X$ is $G$-paradoxical.
\end{proof}
\begin{note}
  We invoked the axiom of choice when we defined $M$ to contain exactly one element from each orbit in $X$.
\end{note}
\subsection{Paradoxical Decompositions of the Unit Sphere}%
We are aware of $\F(a,b)$ being a paradoxical group --- in particular, we hope to use the properties of $\F(a,b)$ to yield paradoxical decompositions of the unit sphere in $\R^{3}$, denoted $S^{2}$.
\begin{definition}[Special Orthogonal Group]
  For $n\in \N$, we define the special orthogonal group to consist of all real $n\times n$ matrices $A$ such that
  \begin{align*}
    A^{T}A = AA^{T} = I,
  \end{align*}
  with $\det(A) = 1$.
\end{definition}
In particular, $SO(3)$ denotes the set of all rotations about some line that runs through the origin. An important fact about $SO(3)$ is that it contains a paradoxical subgroup.
\begin{theorem}
  There are rotations $A$ and $B$ about lines through the origin in $\R^3$ which generate a subgroup of $SO(3)$ isomorphic to $\F(a,b)$.
\end{theorem}
\begin{proof}
  We set
  \begin{align*}
    A^{\pm} &= \begin{bmatrix}1/3 & \mp\frac{2\sqrt{2}}{3} & 0\\ \pm \frac{2\sqrt{2}}{3} & 1/3 & 0 \\ 0 & 0 & 1 \end{bmatrix}\\
    B^{\pm} &= \begin{bmatrix}1 & 0 & 0 \\ 0 & 1/3 & \mp \frac{2\sqrt{2}}{3}\\ 0 & \pm \frac{2\sqrt{2}}{3} & 1/3\end{bmatrix}
  \end{align*}
  Here, $A^{+}$ denotes $A$, and $A^{-}$ denotes $A^{-1}$, and similarly with $B$.\newline

  Let $w$ be a reduced word in $A$, $B$, $A^{-1}$, and $B^{-1}$ which is not the empty word. We claim that $w$ cannot be the identity. Without loss of generality, we assume $w$ ends in $A$ or $A^{-1}$ (otherwise, we can conjugate $w$ with $A$ and $A^{-1}$)
\end{proof}
\end{document}
