\documentclass[10pt]{mypackage}

% sans serif font:
%\usepackage{cmbright}
%\usepackage{sfmath}
%\usepackage{bbold} %better blackboard bold

%serif font + different blackboard bold for serif font
\usepackage{newpxtext,eulerpx}
\renewcommand*{\mathbb}[1]{\varmathbb{#1}}

\fancyhf{}
\rhead{Avinash Iyer}
\lhead{Amenability and C$^{\ast}$-Algebras}

\setcounter{secnumdepth}{0}

\begin{document}
\RaggedRight
\section{Introduction}%
This is going to be the notes document for my honors thesis. Hopefully, we will cover locally compact topological groups, (Haar) measures, Tarski's theorem, group algebras, the theory of Banach algebras and C$^{\ast}$-algebras, and the proof that a C$^{\ast}$-algebra is nuclear if and only if it's amenable as a Banach algebra.\newline

The primary source texts to inform this independent study will be Volker Runde's \textit{Lectures on Amenability} and Timothy Rainone's \textit{Functional Analysis-en route to Operator Algebras}.
\section{Group Actions and Paradoxical Decompositions}%
This section will draw information from Dummit and Foote's \textit{Abstract Algebra}, as well as Runde's \textit{Lectures on Amenability}.

\begin{definition}[Group Action]
  A (left) group action of $G$ onto a set $A$ is a map from $G\times A$ to $A$ that satisfies:
  \begin{itemize}
    \item $g_1\cdot \left(g_2\cdot a\right) = \left(g_1g_2\right)\cdot a$ for all $g_1,g_2\in G$ and $a\in A$;
    \item $e\cdot a = a$\footnote{The identity element is usually written as $1$, but I will write it as $e$ out of familiarity.} for all $a\in A$.
  \end{itemize}
\end{definition}
\begin{definition}[Permutation Representation]
  For each $g$, the map $\sigma_g: A\rightarrow A$ defined by $\sigma_g(a) = g\cdot a$ (the group element $g$ acts on $a$) is a permutation of $A$. There is a homomorphism associated to these actions:
  \begin{align*}
    \varphi: G\rightarrow S_A,
  \end{align*}
  where $\varphi(g) = \sigma_g$. Recall that $S_A$ is the symmetric group (group of permutations) on the elements of $A$. \newline

  This is the permutation representation for the action.\newline

  In particular, given any nonempty set $A$ and a homomorphism $G$ into $S_A$, we can define an action of $G$ on $A$ by taking $g\cdot a = \varphi(g)(a)$.
\end{definition}
\begin{definition}[Kernel]
  The kernel of the action of $G$ is the set of elements in $g$ that act trivially on $A$:
  \begin{align*}
    \set{g\in G\mid \forall a\in A,~g\cdot a = a}
  \end{align*}
\end{definition}
\begin{note}
  The kernel of the action is the kernel of the permutation representation $\varphi: G\rightarrow S_{A}$.
\end{note}
\begin{definition}[Stabilizer]
  For each $a\in A$, the stabilizer of $a$ under $G$ is the set of elements in $G$ that fix $a$:
  \begin{align*}
    G_a &= \set{g\in G\mid g\cdot a = a}.
  \end{align*}
\end{definition}
\begin{note}
  The kernel of the group action is the intersection of the stabilizers of every element of $G$:
  \begin{align*}
    \text{kernel} &= \bigcap_{a\in A}G_a.
  \end{align*}
\end{note}
\begin{note}
  For each $a\in A$, $G_a$ is a subgroup of $G$.
\end{note}
\begin{definition}[Faithful Action]
  An action is faithful if the kernel of the action is $e$.
\end{definition}
\begin{proposition}[Equivalence Relation on $A$]
  Let $G$ be a group that acts on a nonempty set $A$. We define a relation $a\sim b$ if and only if $a = g\cdot b$ for some $g\in G$. This is an equivalence relation, with the number of elements in $\left[a\right]_{\sim}$ found by taking $\left\vert G:G_a \right\vert$, which is the index of the stabilizer of $a$.
\end{proposition}
\begin{proof}
  We can see that $a\sim a$, since $e\cdot a = a$. Similarly, we can see that if $a\sim b$, then $b = g^{-1}\cdot a$, meaning $b\sim a$. Finally, let $a\sim b$ and $b\sim c$. Then, we have $a = g\cdot b$ for some $g\in G$, and $b = h\cdot c$ for some $h\in G$. Thus, we have
  \begin{align*}
    a &= g\cdot \left(h\cdot c\right)\\
      &= \left(gh\right)\cdot c,
  \end{align*}
  meaning $a\sim c$.\newline

  We say there is a bijection between the left cosets of $G_a$ and the elements of the equivalence class of $a$.\newline

  Define $\mathcal{C}_a$ to be the set $\set{g\cdot a\mid g\in G}$, and let $b = g\cdot a$. Define a map $g\cdot a \mapsto gG_a$. This map is surjective since $g\cdot a$ is always an element of $\mathcal{C}_a$. Additionally, since $g\cdot a = h\cdot a$ if and only if $\left(h^{-1}g\right)\cdot a = a$, meaning $h^{-1}g \in G_a$, and $h^{-1}g\in G_a$ if and only if $gG_a = hG_a$, this map is injective.\newline

  Since there is a one-to-one map between the equivalence classes of $a$ under the action of $G$, and the number of left cosets of $G_a$, we now know that the number of equivalence classes of $a$ under the action of $G$ is $\left\vert G:G_a \right\vert$.
\end{proof}
\begin{definition}[Orbit]
  For any $a\in A$, we define the orbit under $G$ of $a$ by
  \begin{align*}
    G\cdot a &= \set{b\in A\mid \forall g\in G,~b = g\cdot a}
  \end{align*}
\end{definition}
\end{document}
