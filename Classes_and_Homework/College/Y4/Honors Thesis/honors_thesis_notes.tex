\documentclass[12pt]{mypackage}

% sans serif font:
%\usepackage{cmbright}
%\usepackage{sfmath}
%\usepackage{bbold} %better blackboard bold

%serif font + different blackboard bold for serif font
\usepackage{newpxtext,eulerpx}
\renewcommand*{\mathbb}[1]{\varmathbb{#1}}
\usepackage{nicematrix}

\pagestyle{fancy}
\fancyhf{}
\rhead{Avinash Iyer}
\lhead{Amenability}

\setcounter{secnumdepth}{0}

\begin{document}
\RaggedRight
\tableofcontents
\section{Introduction}%
This is going to be the notes from my Honors Thesis project on amenability. We will be covering different results that are used to show that a topological group has a translation-invariant finitely additive probability measure (i.e., a mean).\newline

The primary source texts to inform this independent study will be Volker Runde's \textit{Lectures on Amenability} and Timothy Rainone's \textit{Functional Analysis-en route to Operator Algebras}, as well as various notes compiled by my professor.
\section{Group Actions, Paradoxical Decompositions, and the Banach--Tarski Paradox}%
In order to introduce Tarski's theorem, which is where our first condition about the amenability of groups appears, we begin by discussing paradoxical decompositions, with the goal of this section being a proof of the Banach--Tarski Paradox. The Banach--Tarski paradox says the following:
\begin{quote}
  If $A$ and $B$ are any bounded subsets of $\R^{3}$ with nonempty interior, there is a partition of $A$ into finitely many disjoint subsets such that a sequence of isometries applied to these subsets yields $B$.
\end{quote}
\subsection{Basics of Group Actions}%
The information for these essentials about group actions will be drawn from Dummit and Foote's \textit{Abstract Algebra}.
\begin{definition}[Group Action]
  A (left) group action of $G$ onto a set $A$ is a map from $G\times A$ to $A$ that satisfies:
  \begin{itemize}
    \item $g_1\cdot \left(g_2\cdot a\right) = \left(g_1g_2\right)\cdot a$ for all $g_1,g_2\in G$ and $a\in A$;
    \item $e\cdot a = a$\footnote{The identity element is usually written as $1$, but I will write it as $e$ out of familiarity.} for all $a\in A$.
  \end{itemize}
\end{definition}
\begin{definition}[Permutation Representation]
  For each $g$, the map $\sigma_g: A\rightarrow A$ defined by $\sigma_g(a) = g\cdot a$ (the group element $g$ acts on $a$) is a permutation of $A$. There is a homomorphism associated to these actions:
  \begin{align*}
    \varphi: G\rightarrow S_A,
  \end{align*}
  where $\varphi(g) = \sigma_g$. Recall that $S_A$ is the symmetric group (group of permutations) on the elements of $A$. \newline

  This is the permutation representation for the action.\newline

  In particular, given any nonempty set $A$ and a homomorphism $G$ into $S_A$, we can define an action of $G$ on $A$ by taking $g\cdot a = \varphi(g)(a)$.
\end{definition}
\begin{definition}[Kernel]
  The kernel of the action of $G$ is the set of elements in $g$ that act trivially on $A$:
  \begin{align*}
    \set{g\in G\mid \forall a\in A,~g\cdot a = a}
  \end{align*}
\end{definition}
\begin{note}
  The kernel of the action is the kernel of the permutation representation $\varphi: G\rightarrow S_{A}$.
\end{note}
\begin{definition}[Stabilizer]
  For each $a\in A$, the stabilizer of $a$ under $G$ is the set of elements in $G$ that fix $a$:
  \begin{align*}
    G_a &= \set{g\in G\mid g\cdot a = a}.
  \end{align*}
\end{definition}
\begin{note}
  The kernel of the group action is the intersection of the stabilizers of every element of $G$:
  \begin{align*}
    \text{kernel} &= \bigcap_{a\in A}G_a.
  \end{align*}
\end{note}
\begin{note}
  For each $a\in A$, $G_a$ is a subgroup of $G$.
\end{note}
\begin{definition}[Faithful Action]
  An action is faithful if the kernel of the action is $e$.
\end{definition}
\begin{definition}[Free Action]
  For a set $X$ with $G$ acting on $X$, the action of $G$ on $X$ is free if, for every $x$, $g\cdot x = x$ if and only if $g = e_G$.\newline

  If the action of $G$ on $X$ is a free action, we say $G$ acts freely on $X$.
\end{definition}
\begin{proposition}[Equivalence Relation on $A$]
  Let $G$ be a group that acts on a nonempty set $A$. We define a relation $a\sim b$ if and only if $a = g\cdot b$ for some $g\in G$. This is an equivalence relation, with the number of elements in $\left[a\right]_{\sim}$ found by taking $\left\vert G:G_a \right\vert$, which is the index of the stabilizer of $a$.
\end{proposition}
\begin{proof}
  We can see that $a\sim a$, since $e\cdot a = a$. Similarly, we can see that if $a\sim b$, then $b = g^{-1}\cdot a$, meaning $b\sim a$. Finally, let $a\sim b$ and $b\sim c$. Then, we have $a = g\cdot b$ for some $g\in G$, and $b = h\cdot c$ for some $h\in G$. Thus, we have
  \begin{align*}
    a &= g\cdot \left(h\cdot c\right)\\
      &= \left(gh\right)\cdot c,
  \end{align*}
  meaning $a\sim c$.\newline

  We say there is a bijection between the left cosets of $G_a$ and the elements of the equivalence class of $a$.\newline

  Define $\mathcal{C}_a$ to be the set $\set{g\cdot a\mid g\in G}$, and let $b = g\cdot a$. Define a map $g\cdot a \mapsto gG_a$. This map is surjective since $g\cdot a$ is always an element of $\mathcal{C}_a$. Additionally, since $g\cdot a = h\cdot a$ if and only if $\left(h^{-1}g\right)\cdot a = a$, meaning $h^{-1}g \in G_a$, and $h^{-1}g\in G_a$ if and only if $gG_a = hG_a$, this map is injective.\newline

  Since there is a one-to-one map between the equivalence classes of $a$ under the action of $G$, and the number of left cosets of $G_a$, we now know that the number of equivalence classes of $a$ under the action of $G$ is $\left\vert G:G_a \right\vert$.
\end{proof}
\begin{definition}[Orbit]
  For any $a\in A$, we define the orbit under $G$ of $a$ by
  \begin{align*}
    G\cdot a &= \set{b\in A\mid \forall g\in G,~b = g\cdot a}
  \end{align*}
  In particular, if $c\in G\cdot a$ for some $a\in A$, then $G\cdot c = G\cdot a$.
\end{definition}
\subsection{Paradoxical Decompositions}%
Most of the information from this section will be drawn from Volker Runde's \textit{Lectures on Amenability}, as well as \textit{Amenable Banach Algebras --- A Panorama}.
\begin{definition}[Paradoxical Sets and Decompositions]
  Let $G$ be a group that acts on a set $X$. Let $E\subseteq X$.\newline

  If there exist pairwise disjoint $A_1,\dots,A_n,B_1,\dots,B_m\subseteq E$ and $g_1,\dots,g_n,h_1,\dots,h_m\in G$ such that
  \begin{align*}
    E &= \bigcup_{j=1}^{n}g_j\cdot A_j
    \intertext{and}
    E &= \bigcup_{j=1}^{m}h_j\cdot B_j,
  \end{align*}
  then we say that $E$ is $G$-paradoxical.\newline

  In particular, a paradoxical group is one where $G$ acts on itself by left-multiplication.
\end{definition}
\begin{example}[Our First Paradoxical Group]
  The free group on two generators, $\mathbb{F}\left(a,b\right)$,\footnote{The set of all reduced words over $\set{a,b,a^{-1},b^{-1},e_{\F(a,b)}}$. In particular, a word is reduced when the pairs $aa^{-1}$ and $bb^{-1}$ are replaced with the identity $e_{\F(a,b)}$.} is paradoxical. To see this, we let
  \begin{align*}
    W(x) &= \set{w\in \F(a,b)\mid w\text{ starts with }x}.
  \end{align*}
  Here, ``starts with'' refers to the left-most element. For instance, $ba^2ba^{-1}\in W\left(b\right)$.\newline

  In particular, we can see that
  \begin{align*}
    \F(a,b) &= \set{e_{\F(a,b)}} \sqcup W(a) \sqcup W(b) \sqcup W\left(a^{-1}\right)\sqcup W\left(b^{-1}\right).
  \end{align*}
  For any $w\in \F(a,b)\setminus W(a)$, we can see that $a^{-1}w\in W\left(a^{-1}\right)$, meaning $w\in aW\left(a^{-1}\right)$. Therefore, $\F(a,b) = W(a)\sqcup aW\left(a^{-1}\right)$.\newline

  Similarly, for any $v\in \F(a,b)\setminus W(b)$, $b^{-1}v \in W\left(b^{-1}\right)$, so $v \in bW\left(b^{-1}\right)$. Therefore, $\F(a,b) = W(b) \sqcup bW\left(b^{-1}\right)$.
\end{example}
\begin{proposition}[Free Action of a Paradoxical Group]
  Let $G$ be a paradoxical group that acts freely on $X$. Then, $X$ is $G$-paradoxical.
\end{proposition}
\begin{proof}
  Let $A_1,\dots,A_n,B_1,\dots,B_m\subseteq G$ be pairwise disjoint, with $g_1,\dots,g_n,h_1,\dots,h_m\in G$ such that
  \begin{align*}
    G &= \bigcup_{j=1}^{n}g_j A_j\\
      &= \bigcup_{j=1}^{m}h_jB_j.
  \end{align*}
  We let $M\subseteq X$ contain exactly one element from every orbit of $G$.\newline

  The set $\set{g\cdot M\mid g\in G}$ is a partition of $X$. Since $M$ contains exactly one element from every orbit of $G$, it is then the case that $\bigcup_{g\in G}g\cdot M = X$, since $G\cdot M = X$.\newline

  Additionally, if $x,y\in M$ with $g\cdot x = h\cdot y$, then $\left(h^{-1}g\right)\cdot x = y$, meaning $y$ is in the orbit of $x$ and vice versa, implying $x = y$. Thus, we must have $h^{-1}g = e_G$, as we assume $G$ acts freely.\newline

  Thus, we can see that $g_1\cdot M \neq g_2\cdot M$ if $g_1\neq g_2$, meaning $\set{g\cdot M\mid g\in G}$ is a partition.\newline

  Define $A_{j}^{\ast}$ to be the subset of $X$ that is the result of the elements of $A_j$ acting on $M$. In other words,
  \begin{align*}
    A_j^{\ast} &= \bigcup_{g\in A_j}g\cdot M.
  \end{align*}
  As a useful shorthand, we can say $A_j^{\ast} = A_j\cdot M$.\footnote{Yes, I know that $A_j$ is not technically a group acting on $M$, but this will help illuminate the final conclusion.} Similarly, we define
  \begin{align*}
    B_j^{\ast} &= \bigcup_{h\in B_j}h\cdot M\\
               &= B_j\cdot M.
  \end{align*}
  We can see that $A_1^{\ast},A_2^{\ast},\dots,A_n^{\ast},B_1^{\ast},B_2^{\ast},\dots,B_m^{\ast}\subseteq X$ are disjoint, since $\set{g\cdot M\mid g\in G}$ is a partition, and $A_1,\dots,A_n,B_1,\dots,B_m$ are disjoint in $G$.\newline

  Thus, we have
  \begin{align*}
    \bigcup_{j=1}^{n}g_j\cdot A_j^{\ast} &= \bigcup_{j=1}^{n} \left(g_jA_j\right)\cdot M\\
                                         &= G\cdot M\\
                                         &= X.
  \end{align*}
  Similarly,
  \begin{align*}
    \bigcup_{j=1}^{m}h_j\cdot B_j^{\ast} &= \bigcup_{j=1}^{m}\left(h_jB_j\right)\cdot M\\
                                         &= G\cdot M\\
                                         &= X.
  \end{align*}
  Thus, we see that $X$ has a paradoxical decomposition, meaning $X$ is $G$-paradoxical.
\end{proof}
\begin{note}
  We invoked the axiom of choice when we defined $M$ to contain exactly one element from each orbit in $X$.
\end{note}
\subsection{Paradoxical Decompositions of the Unit Sphere and Unit Ball}%
We are aware of $\F(a,b)$ being a paradoxical group --- in particular, we hope to use the properties of $\F(a,b)$ to yield paradoxical decompositions of the unit sphere in $\R^{3}$, denoted $S^{2}$.
\begin{definition}[Special Orthogonal Group]
  For $n\in \N$, we define the special orthogonal group to consist of all real $n\times n$ matrices $A$ such that
  \begin{align*}
    A^{T}A = AA^{T} = I,
  \end{align*}
  with $\det(A) = 1$.
\end{definition}
In particular, $SO(3)$ denotes the set of all rotations about some line that runs through the origin. An important fact about $SO(3)$ is that it contains a paradoxical subgroup.
\begin{theorem}
  There are rotations $A$ and $B$ about lines through the origin in $\R^3$ which generate a subgroup of $SO(3)$ isomorphic to $\F(a,b)$.
\end{theorem}
\begin{proof}
  We set
  \begin{align*}
    A^{\pm} &= \begin{bmatrix}1/3 & \mp\frac{2\sqrt{2}}{3} & 0\\ \pm \frac{2\sqrt{2}}{3} & 1/3 & 0 \\ 0 & 0 & 1 \end{bmatrix}\\
    B^{\pm} &= \begin{bmatrix}1 & 0 & 0 \\ 0 & 1/3 & \mp \frac{2\sqrt{2}}{3}\\ 0 & \pm \frac{2\sqrt{2}}{3} & 1/3\end{bmatrix}
  \end{align*}
  Here, $A^{+}$ denotes $A$, and $A^{-}$ denotes $A^{-1}$, and similarly with $B$.\newline

  Let $w$ be a reduced word in $A$, $B$, $A^{-1}$, and $B^{-1}$ which is not the empty word. We claim that $w$ cannot be the identity. Without loss of generality, we assume $w$ ends in $A$ or $A^{-1}$ --- this is because $w$ acts as the identity if and only if $AwA^{-1}$ or $A^{-1}wA$ act as the identity.\newline

  In particular, we will show that there exist $a,b,c\in \Z$ with $b\nequiv 0$ modulo $3$ such that
  \begin{align*}
    w \cdot \begin{pmatrix}1\\0\\0\end{pmatrix} &= \frac{1}{3^{k}} \begin{pmatrix}a\\b\sqrt{2}\\c\end{pmatrix},
  \end{align*}
  where $k$ is the length of $w$. The main reason we wish to show this is that, if we have $b\nequiv 0$ modulo $3$, it is the case that $w$ necessarily cannot map $ \begin{pmatrix}1\\0\\0\end{pmatrix} $ to itself.\newline

  We start with induction on the length of $w$. In particular, for $w = A^{\pm}$, we have
  \begin{align*}
    w \cdot \begin{pmatrix}1\\0\\0\end{pmatrix} &= \frac{1}{3}\begin{pmatrix}1\\\pm2\sqrt{2}\\0\end{pmatrix},
  \end{align*}
  proving the base case.\newline

  Suppose $k > 0$, meaning $w = A^{\pm}w'$ or $w = B^{\pm}w'$, with $w'$ not equal to the empty word. The inductive hypothesis says that
  \begin{align*}
    w' \cdot \begin{pmatrix}1\\0\\0\end{pmatrix} &= \frac{1}{3^{k-1}} \begin{pmatrix}a'\\b'\sqrt{2}\\c'\end{pmatrix},
  \end{align*}
  for some  $a',b',c'\in \Z$ with $b\nequiv 0$ modulo 3. In particular,
  \begin{align*}
    A^{\pm}w' \cdot \begin{pmatrix}1\\0\\0\end{pmatrix} &= \frac{1}{3^k} \begin{pmatrix}a' \mp 4b'\\ \left(b' \pm 2a'\right)\sqrt{2}\\ 3c'\end{pmatrix}\\
  B^{\pm}w' \cdot \begin{pmatrix}1\\0\\0\end{pmatrix} &= \frac{1}{3^k}\begin{pmatrix}3a'\\\left(b' \mp 2c'\right)\sqrt{2} \\ c' \pm 4b'\end{pmatrix},
  \end{align*}
  where we say
  \begin{align*}
    w \cdot \begin{pmatrix}1\\0\\0\end{pmatrix} &= \frac{1}{3^k}\begin{pmatrix}a\\b\\c\end{pmatrix},
  \end{align*}
  i.e., we set the coordinates of $w\cdot \begin{pmatrix}1\\0\\0\end{pmatrix}$ through their definition in $A^{\pm}w'$ or $B^{\pm}w'$.\newline

  In order to show that $b\nequiv 0$ modulo $3$, we must examine the following four cases.\newline

  Let $w^{\ast}$ denote the word such that
  \begin{align*}
    w^{\ast} \cdot \begin{pmatrix}1\\0\\0\end{pmatrix} &= \frac{1}{3^{k-2}} \begin{pmatrix}a''\\b''\sqrt{2}\\c''\end{pmatrix},
  \end{align*}
  with $a'',b'',c''\in \Z$ and $b''\nequiv 0$ modulo $3$. It is important to note here that $w^{\ast}$ may be the empty word.
  \begin{description}[font=\normalfont]
    \item[Case 1:] Suppose $w = A^{\pm}B^{\pm}w^{\ast}$. Then, we have $b = b'\mp 2a'$, where $a' = 3a''$. Since $b'\nequiv 0$ modulo $3$ by the inductive hypothesis assumption, it is also the case that $b\nequiv 0$ modulo $3$.
    \item[Case 2:] Suppose $w = B^{\pm}A^{\pm}w^{\ast}$. Then, we have $b = b' \mp 2c'$, where $c' = 3c''$. Similarly, since $b'\nequiv 0$ modulo $3$ by the inductive hypothesis assumption, it is also the case that $b\nequiv 0$ modulo $3$.
    \item[Case 3:] Suppose $w = A^{\pm}A^{\pm}w^{\ast}$. Then, we have
      \begin{align*}
        b &= b' \pm 2a'\\
          &= b' \pm 2\left(a'' \mp 4b''\right)\\
          &= b' + \left(b'' \pm 2a''\right) - 9b''\\
          &= 2b' - 9b''.
      \end{align*}
      Since $b',b''\nequiv 0$  modulo $3$ by the inductive hypothesis, it is also the case that $b\nequiv 0$ modulo $3$.
    \item[Case 4:] Suppose $w = B^{\pm}B^{\pm}w^{\ast}$. Then, we have
      \begin{align*}
        b &= b' \mp 2c'\\
          &= b' \mp 2\left(c'' \pm 4b''\right)\\
          &= b' + \left(b''\mp 2c''\right) - 9b''\\
          &= 2b' - 9b''.
      \end{align*}
      Since $b',b''\nequiv 0$ modulo $3$ by the inductive hypothesis, it is also the case that $b\nequiv 0$ modulo $3$.
  \end{description}
  Thus, we have shown that any non-empty reduced word over $A$, $A^{-1}$, $B$, $B^{-1}$ does not act as the identity. The subgroup of $SO(3)$ generated by $A$, $B$, $A^{-1}$, and $B^{-1}$ is thus isomorphic to $\F(a,b)$.
\end{proof}
\begin{remark}
  For any element of $SO(n)$ with $n \geq 3$, we can write $A_n$ (denoting the $n\times n$ matrix corresponding to $A$) as
  \begin{align*}
    A_n &= 
    \begin{pmatrix}
      A_3 & \mathbf{0}\\
      \mathbf{0} & \mathbf{1}
    \end{pmatrix}\\
    B_n &= \begin{pmatrix}B_3 & \mathbf{0}\\ \mathbf{0} & \mathbf{1}\end{pmatrix},
  \end{align*}
  where $\mathbf{0}$ denotes a block matrix consisting of $0$ and $\mathbf{1}$ denotes a block matrix equal to the identity.\newline

  This means that our subgroup of $SO(3)$ isomorphic to $\mathbf{F}\left(a,b\right)$ embeds into $SO(n)$ via the above block matrices.
\end{remark}
\begin{theorem}[Hausdorff Paradox]
  There is a countable subset $D$ of $S^2$ such that $S^2\setminus D$ is paradoxical under the action of $SO(3)$.
\end{theorem}
\begin{proof}
  Let $A$ and $B$ be the rotations in $SO(3)$ that serve as the generators of the subgroup isomorphic to $\mathbb{F}\left(a,b\right)$.\newline

  Since $A$ and $B$ are rotations, any word in the subgroup generated by $A$ and $B$ will also be a rotation --- as a result, all such (non-empty) words contain two fixed points.\newline

  Let
  \begin{align*}
    F &= \set{x\in S^2\mid x\text{ is a fixed point for some word $w$}}.
  \end{align*}
  Since the set of all words in $A$ and $B$ is countably infinite, so too is $F$. Therefore, the union of all these fixed points under the action of all such words $w$ is also countable:
  \begin{align*}
    D &= \bigcup_{w\in G} w\cdot F.
  \end{align*}
  Since the set of words in $A$ and $B$ act freely on $S^2\setminus D$, it must be the case that $S^{2}\setminus D$ is paradoxical under the action of the group of all such words.
\end{proof}
\begin{definition}[Equidecomposable Sets]
  Let $G$ act on $X$, $A,B\subseteq X$. We say $A$ and $B$ are equidecomposable under $G$ if there are $A_1,\dots,A_n\subseteq A$, $B_1,\dots,B_n\subseteq B$, and $g_1,\dots,g_n\in G$ such that
  \begin{enumerate}[(i)]
    \item $A = \bigcup_{j=1}^{n}A_j$ and $B = \bigcup_{j=1}^{n}B_j$;
    \item the collection $\set{A_j}_{j=1}^{n}$ are pairwise disjoint and the collection $\set{B_j}_{j=1}^{n}$ are pairwise disjoint;
    \item for each $j$, $g_j\cdot A_j = B_j$.
  \end{enumerate}
  We write $A\sim_{G} B$ if $A$ and $B$ are equidecomposable under $G$.
\end{definition}
\begin{remark}
  The relation $\sim_{G}$ is an equivalence relation.\newline

  In particular, to see transitivity, we have the partitions $A_1,\dots,A_n\subseteq B$ and $B_1,\dots,B_n\subseteq B$ with $g_i\cdot A_i = B_i$, and the partitions $B_1,\dots,B_m\subseteq B$, $C_1\cdots C_m\subseteq C$ with $h_j\cdot B_j = C_j$.\newline

  We find the partition of $A$ by taking $A_{ij} = B_i\cap B_j$, where $i\in \set{1,2,\dots,n}$ and $j\in \set{1,2,\dots,m}$. We then have $h_jg_i\cdot A_{ij}$ maps to a refined partition of $C$, yielding equidecomposability between $A$ and $C$.
\end{remark}
\begin{remark}
  For equidecomposable sets $A$ and $B$, there is a bijection $\phi: A\rightarrow B$ by, for each $C\subseteq A$, taking $C_{i} = C\cap A_i$, where $A_1,\dots,A_n$ is the partition of $A$, and mapping $\varphi\left(C_i\right) = g_i\cdot C_i$.
\end{remark}
\begin{proposition}
  Let $D\subseteq S^2$ be countable. Then, $S^2$ and $S^{2}\setminus D$ are equidecomposable under the action of $SO(3)$.
\end{proposition}
\begin{proof}
  Let $L$ be a line in $\R^3$ with the property that $L \cap D = \emptyset$. Such a $L$ must necessarily exist as the set of all antipodes in $S^{2}$ is uncountable.\newline

  Define $\rho_{\theta}\in SO(3)$ to be a rotation about $L$ by an angle of $\theta$. For fixed $n\in \N$ and fixed $\theta\in [0,2\pi)$, define $R_{n,\theta} = \set{x\in D\mid \rho_{\theta}^{n}\cdot x\in D}$. Since $D$ is countable, $R_{n,\theta}$ is necessarily countable.\newline

  Define $W_{n} = \set{\theta\mid R_{n,\theta}\neq \emptyset}$. The injection $\theta \mapsto \rho^{n}_{\theta}\cdot x$ into $D$ shows that for each $n$, $W_n$ is countable. Thus, defining
  \begin{align*}
    W &= \bigcup_{n\in N}W_n
  \end{align*}
  it is evident that $W$ is countable.\newline

  Thus, there must exist $\omega \in [0,2\pi)\setminus W$. Define $\rho$ to be a rotation about $L$ by $\omega$. Then, for every $n,m\in \N$,
  \begin{align*}
    \rho^{n}\cdot D \cap \rho^{m}\cdot D = \emptyset.
  \end{align*}
  We let $\tilde{D} = \bigsqcup_{n=0}^{\infty}\rho^{n}\cdot D$. Notice that, in particular, $\rho\cdot\tilde{D} = \bigsqcup_{n=1}^{\infty}\rho^{n}\cdot D$, meaning $\tilde{D}$ and $\tilde{D}\setminus D$ are equidecomposable under $SO(3)$.\newline

  Thus, we have
  \begin{align*}
    S^{2} &= \tilde{D}\sqcup \left(S^2\setminus \tilde{D}\right)\\
          &\sim_{SO(3)} \rho\cdot \tilde{D}\sqcup\left(S^2\setminus \tilde{D}\right)\\
          &= \left(\tilde{D}\setminus D\right)\sqcup\left(S^2\setminus \tilde{D}\right)\\
          &= S^2\setminus D,
  \end{align*}
  establishing the equidecomposability of $S^2$ and $S^2\setminus D$.
\end{proof}
\begin{proposition}
  Let $G$ act on $X$, with $E$ and $E'$ subsets of $X$ such that $E \sim_{G}E'$. Then, if $E$ is paradoxical under the action of $G$, so too is $E'$.
\end{proposition}
\begin{proof}
  Let $A_1,\dots,A_n$, $B_1,\dots,B_m$ be pairwise disjoint subsets of $E$ and $g_1,\dots,g_n,h_1,\dots,h_m\in G$ such that
  \begin{align*}
    E &= \bigsqcup_{i=1}^{n} g_i\cdot A_i\\
      &= \bigsqcup_{j=1}^{m}h_j\cdot B_j,
  \end{align*}
  which follows from the paradoxicality of $E$. We let
  \begin{align*}
    A &= \bigsqcup_{i=1}^{n}A_i\\
    B &= \bigsqcup_{j=1}^{m} B_j.
  \end{align*}
  It follows that $A\sim_{G} E$ and $B\sim_{G} E$; to see this, set the partition of $A$ to be $A_1,\dots,A_n$, and set the partition of $E$ to be $g_i\cdot A_i$ for $i\in \set{1,\dots,n}$, and similarly for $G$.\newline

  Since $E\sim_{G} E'$, and $\sim_{G}$ is an equivalence relation, it follows that $A\sim_{G}E'$ and $B\sim_{G}E'$, implying that there exists a paradoxical decomposition of $E'$ in $A_1,\dots,A_n$ and $B_1,\dots,B_m$.
\end{proof}
Since $S^{2}\setminus D$ and $S^{2}$ are equidecomposable under the action of $SO(3)$, and $S^{2}\setminus D$ is paradoxical under the action of $SO(3)$, the above proposition implies the following corollary.
\begin{corollary}
  $S^{2}$ is paradoxical under $SO(3)$.
\end{corollary}
\begin{definition}[Euclidean Group]
  The Euclidean group $E(n)$ consists of all isometries of a Euclidean space. An isometry of a Euclidean space consists of translations, flips about the origin, and rotation.\newline

  In particular, $E(n) = T(n)\rtimes O(n)$, where $T(n)$ denotes all translations and $O(n)$ is the orthogonal group, which denotes all rotations or flips.\newline

  We define $E_{+}(n)$ to be all orientation-preserving isometries of Euclidean space. In particular, $E_{+}(n) = T(n)\rtimes SO(n)$, where $SO(n)$ is the special orthogonal group, which denotes all orientation-preserving rotations.
\end{definition}
\begin{corollary}[Weak Banach--Tarski Paradox]
  Every closed ball in $\R^3$ is paradoxical under the Euclidean group $E(3)$.
\end{corollary}
\begin{proof}
  We only need to show that the closed unit ball, $B(0,1)$, is paradoxical under the action of $E(3)$.\newline

  To start, we can show that $B(0,1)\setminus \set{0}$ is paradoxical. Since $SO(3)$ is paradoxical, there exist pairwise disjoint $A_1,\dots,A_n,B_1,\dots,B_m\subseteq S^2$ and $g_1,\dots,g_n,h_1,\dots,h_m\in SO(3)$ such that
  \begin{align*}
    S^2 &= \bigsqcup_{i=1}^{n}g_i\cdot A_i\\
        &= \bigsqcup_{j=1}^{m}h_j\cdot B_j.
  \end{align*}
  Define
  \begin{align*}
    A_i^{\ast} &= \set{tx\mid t\in (0,1],x\in A_i}\\
    B_j^{\ast} &= \set{ty\mid t\in (0,1],y\in B_j}.
  \end{align*}
  Then, $A_1^{\ast},\dots,A_n^{\ast},B_1^{\ast},\dots,B_m^{\ast}\subseteq B(0,1)\setminus \set{0}$ are pairwise disjoint, and
  \begin{align*}
    B(0,1)\setminus {0} &= \bigcup_{i=1}^{n}g_i\cdot A_i^{\ast}\\
           &= \bigcup_{j=1}^{m}h_j\cdot B_j^{\ast}.
  \end{align*}
  Thus, we have established that $B(0,1)\setminus \set{0}$ is paradoxical under $SO(3)\leq E(3)$.\footnote{Essentially, we take the paradoxical decomposition of $S^2$ under $SO(3)$, and scale by $t$ to cover all of $B(0,1)\setminus \set{0}$.}\newline

  Now, we want to show that $B(0,1)\setminus \set{0}$ and $B(0,1)$ are equidecomposable under $E(3)$. To do this, let $x \in B(0,1)\setminus \set{0}$, and let $\rho$ be a rotation about $x$ by a line that misses the origin such that $\rho^{n}\cdot 0 \neq \rho^{m}\cdot 0$ for all $n,m\in \N$ with $n\neq m$.\footnote{This is why we need our underlying group acting on $\R^3$ to be the Euclidean group rather than $SO(3)$. It is still the case that $SO(3)\leq E(3)$, meaning that $E(3)$ is necessarily paradoxical when acting on $\R^3$.} Let $D = \set{\rho^{n}\cdot 0\mid n\in\N}$. We can see that $\rho\cdot D = D\setminus \set{0}$, and that $D$ and $\rho\cdot D$ are equidecomposable under $E(3)$.\newline

  Thus, we have
  \begin{align*}
    B(0,1) &= D\sqcup \left(B(0,1)\setminus D\right)\\
           &\sim_{E(3)} \left(\rho\cdot D\right)\sqcup \left(B(0,1)\setminus D\right)\\
           &= \left(D\setminus \set{0}\right) \sqcup \left(B(0,1)\setminus D\right)\\
           &= B(0,1)\setminus \set{0},
  \end{align*}
  establishing the equidecomposability of $B(0,1)$ and $B(0,1)\setminus \set{0}$.\newline

  Thus, $B(0,1)$ is paradoxical under the action of $E(3)$.
\end{proof}
\begin{definition}
  For $G$ acting on a set $X$, we write $A\preceq_{G}B$ if $A$ is equidecomposable with a subset of $B$.
\end{definition}
\begin{remark}
  We can see that $\preceq_{G}$ is reflexive since $A$ is equidecomposable with $A$, and $A\subseteq A$.\newline

  To show transitivity, let $A\preceq_{G} B$ and $B\preceq_{G} C$. We let $g_1,\dots,g_n\in G$ such that $A\sim_{G}B_{\alpha}$, where $B_{\alpha}\subseteq B$. In particular, we have $A_1,\dots,A_n\subseteq A$ and $B_{1,\alpha},\dots,B_{n,\alpha}\subseteq B_{\alpha}$ such that $g_i\cdot A_i = B_{i,\alpha}$. We let $h_1,\dots,h_m\in G$ and $C_{\beta}\subseteq C$ such that $h_j\cdot B_j = C_{j,\beta}$ for each $j\in \set{1,\dots,m}$.\newline

  We take a refinement on $B$ taking intersections $B_{i,j,\alpha} = B_i \cap B_{j,\alpha}$ for each $i\in \set{1,\dots,n}$ and $j\in \set{1,\dots,m}$. Thus, taking $h_jg_i\cdot A_i$, we see that $A$ is equidecomposable with a subset of $C$ (namely, the subset of $C$ ``generated'' by the disjoint subsets of $B_{\alpha}$ refined by $B_i$).
\end{remark}
\begin{remark}
  Since $A\sim_{G} B$ implies the existence of a bijection $\phi: A\rightarrow G$, the $\preceq_{G}$ relation is akin to the $\leq$ relation for cardinalities; in particular, $A\preceq_{G}B$ implies the existence of an injection $\phi: A\hookrightarrow B$.\newline

  This analogy between cardinality and the $\preceq_{G}$ relation naturally lends itself to the following theorem.
\end{remark}
\begin{theorem}
  Let $G$ be a group that acts on $X$. Let $A,B$ be subsets of $X$ with $A\preceq_{G} B$ and $B\preceq_{G} A$. Then, $A\sim_{G} B$.
\end{theorem}
\begin{proof}
  Let $B_1\subseteq B$ with $A\sim_{G} B_1$, and let $A_1\subseteq A$ with $B\sim_{G} A_1$.\newline

  We know that there exist bijections $\phi: A\rightarrow B_1$ and $\psi: B\rightarrow A_1$. Define $C_0 = A\setminus A_1$, $C_{n+1} = \psi\left(\phi\left(C_n\right)\right)$. We set
  \begin{align*}
    C &= \bigcup_{n\geq 1} C_n.
  \end{align*}
  Since $\psi^{-1}\left(\psi\left(\phi\left(C_n\right)\right)\right) = \phi\left(C_n\right)$, we have
  \begin{align*}
    \psi^{-1}\left(A\setminus C\right) &= B\setminus \phi(C),
  \end{align*}
  meaning $A\setminus C \sim B\setminus \phi(C)$. Additionally, $C\sim \phi(C)$. Thus,
  \begin{align*}
    A &= \left(A\setminus C\right)\cup C\\
      &\sim \left(B\setminus \phi(C)\right) \cup \phi(C)\\
      &= B.
  \end{align*}
\end{proof}
\begin{theorem}[Banach--Tarski Paradox]
  Let $A$ and $B$ be bounded subsets of $\R^3$ with nonempty interior. Then, $A \sim_{E(3)} B$.
\end{theorem}
\begin{proof}
  It is sufficient to show that $A\preceq_{E(3)} B$.\newline

  Since $A$ is bounded, there is $r > 0$ such that $A\subseteq B(0,r)$. Let $x\in B^{\circ}$. Then, there is $\epsilon > 0$ such that $B(x,\epsilon)\subseteq B$.\newline

  Since $B(0,r)$ is compact (and thus totally bounded), there are translations $g_1,\dots,g_n$ such that
  \begin{align*}
    B(0,r)\subseteq g_1\cdot B(x,\epsilon) \cup \cdots \cup g_n\cdot B(x,\epsilon).
  \end{align*}
  Choose translations $h_1,\dots,h_n$ such that $h_j\cdot B(x,\epsilon)\cap h_k\cdot B(x,\epsilon) = \emptyset$ for $j\neq k$. Set
  \begin{align*}
    S &= \bigcup_{j=1}^{n}h_j\cdot B(x,\epsilon).
  \end{align*}
  Since each of the subsets $h_j\cdot B(x,\epsilon)$ is equidecomposable with any arbitrary closed ball subset of $B(x,\epsilon)$, it is the case that $S\subseteq B(x,\epsilon)$.\newline

  Thus, we have
  \begin{align*}
    A &\subseteq B(0,r)\\
      &\subseteq g_1\cdot B(x,\epsilon)\cup g_n\cdot B(x,\epsilon)\\
      &\preceq S\\
      &\preceq B(x,\epsilon)\\
      &\subseteq B.
  \end{align*}
\end{proof}
\begin{remark}
  The axiom of choice was invoked when we stated that $h_j\cdot B(x,\epsilon)$ is equidecomposable with an arbitrary closed ball subset of $B(x,\epsilon)$.
\end{remark}
\section{Tarski's Theorem}%
One of the central facts that allowed for the Banach--Tarski paradox to be true is that $\F(a,b)$ does not have a property known as amenability. We had also proved that $\F(a,b)$ is paradoxical.\newline

In this section, we will prove paradoxicality and non-amenability are equivalent. This is formulated in Tarski's Theorem
\begin{theorem}[Tarski's Theorem]
  Let $G$ be a group that acts on a set $X$, and let $E$ be a subset of $X$. There is a finitely additive set function invariant under $G$, $\mu: P(X)\rightarrow \left[0,\infty\right]$ with $\mu(E)\in (0,\infty)$ if and only if $E$ is not $G$-paradoxical.
\end{theorem}
\begin{remark}
  It is possible to see that if $G$ is paradoxical, with $X = G$ and $G$ acting on itself via left-multiplication, that this finitely-additive set function eventually ``blows up.''\newline

  Let $G$ be paradoxical. Suppose toward contradiction that there existed such a $\nu: P(G) \rightarrow [0,\infty]$. For $E_1,\dots,E_n\subseteq G$ and $t_1,\dots,t_n\in G$, $F_1,\dots,F_m\subseteq G$ and $s_1,\dots,s_m\in G$ with $E_1,\dots,E_n,F_1,\dots,F_m$ pairwise disjoint, we have
      \begin{align*}
        \nu(G) &= \nu\left(\bigsqcup_{j=1}^{n}t_jE_j\right)\\
               &= \sum_{j=1}^{n}\nu(t_jE_j)\\
               &= \sum_{j=1}^{n}\nu(E_j).
               \intertext{We know that $G \cup s_1F_1 = G$, meaning $\nu(G) = \nu(G \cup s_1F_1)$. However,}
        \nu(G \cup s_1F_1) &= \nu\left(\bigsqcup_{j=1}^{n}t_jE_j \sqcup s_1F_1\right)\\
                           &= \sum_{j=1}^{n}\nu(t_jE_j) + \nu(s_1F_1)\\
                           &= \nu(G) + \nu(s_1F_1)\\
                           &= \nu(G) + \nu(F_1)\\
                           &> \nu(G).
      \end{align*}
\end{remark}
\subsection{König's Theorem and Perfect Matchings}%
In order to prove that non-paradoxicality implies amenability, we must use some essential results from graph theory.
\begin{definition}[Graphs and Paths]
  A graph is a triple $\left(V,E,\phi\right)$, where $V$ and $E$ are nonempty sets, and $\phi: E\rightarrow P_{2}(V)$ is a map from $e$ to the set of all unordered subset pairs of $V$.\newline

  We say that for $\phi(e) = \set{v,w}$, we say $v$ and $w$ are endpoints of $e$, with $e$ incident on $v$ and $w$.\newline

  A path in $(V,E,\phi)$ is a finite sequence $\left(e_1,\dots,e_n\right)$ of edges along with a finite sequence of vertices $v_0,\dots,v_n$, with $\phi\left(e_k\right) = \set{v_{k-1},v_k}$.\newline

  The degree of a vertex, $\deg(v)$, is the number of edges incident on the vertex.
\end{definition}
\begin{definition}[Bipartite Graphs and $k$-Regularity]
  Let $(V,E,\phi)$ be a graph, $k\in \N$.
  \begin{enumerate}[(i)]
    \item If $\deg(v) = k$ for each $v\in V$, we say $(V,E,\phi)$ is $k$-regular.
    \item If $V = X\sqcup Y$, with each edge having an endpoint in $X$ and an endpoint in $Y$, we say $(V,E,\phi)$ is bipartite.
  \end{enumerate}
\end{definition}
\begin{definition}[Perfect Matching]
Let $(X,Y,E,\phi)$ be a bipartite graph. Let $A\subseteq X$ and $B\subseteq Y$. A perfect matching of $A$ and $B$ is a subset $F\subseteq E$ with
\begin{enumerate}[(i)]
  \item each element of $A\cup B$ is an endpoint of exactly one $f\in F$
  \item all endpoints of edges in $F$ are in $A\cup B$.
\end{enumerate}
\end{definition}
  \begin{exercise}[Hall's Theorem]
  Let $(X,Y,E,\phi)$ be a bipartite graph which is $k$-regular for some $k\in\N$. Suppose $|V| < \infty$.
  \begin{enumerate}[(i)]
    \item Show that $|E| < \infty$ and that $|X| = |Y|$.
    \item For any $M\subseteq V$, let $N(M)$ be the set of those vertices which are joined by an edge with a point in $M$. Show that $|N(M)| \geq |M|$.
    \item Let $A\subseteq X$ and $B\subseteq Y$ be such that there is a perfect matching $F$ of $A$ and $B$ with $|F|$ maximal. Show that $A = X$.
    \item Conclude that there is a perfect matching of $X$ and $Y$.
  \end{enumerate}
  \end{exercise}
  \begin{solution}\hfill
    \begin{enumerate}[(i)]
      \item $|E| = k|X| = k|Y|$, meaning $|X| = |Y|$.
      \item Let $M_X = M\cap X$ and $M_Y = M\cap Y$. Notice that $M = M_X\sqcup M_Y$.\newline

        Let $\left[M_X,N\left(M_X\right)\right]$ denote the set of edges with endpoints in $M_X$ and $N\left(M_X\right)$, and similarly for $\left[M_Y,N\left(M_Y\right)\right]$. We also let $\left[X,N\left(M_X\right)\right]$ and $\left[Y,N\left(M_Y\right)\right]$ denote the set of edges with endpoints in $X$ and $N\left(M_X\right)$ and the set of edges with endpoints in $Y$ and $N\left(M_Y\right)$ respectively.\newline

        We can see that $\left[M_X,N\left(M_X\right)\right]\subseteq \left[X,N\left(M_X\right)\right]$, and similarly with $\left[M_Y,N\left(M_Y\right)\right]\subseteq \left[Y,N\left(M_Y\right)\right]$. Additionally, $\left\vert \left[M_X,N\left(M_X\right)\right] \right\vert = k\left\vert M_X \right\vert$, with $\left\vert \left[X,N\left(M_X\right)\right] \right\vert = k\left\vert N\left(M_X\right) \right\vert$, meaning $|M_X| \leq |N\left(M_X\right)|$, and similarly for $|M_Y|$ and $|N\left(M_Y\right)|$.\newline

        Thus, $|M| \leq |N(M)|$.
    \end{enumerate}
  \end{solution}
\end{document}
