\documentclass[10pt]{mypackage}

% sans serif font:
%\usepackage{cmbright}
%\usepackage{sfmath}
%\usepackage{bbold} %better blackboard bold

%serif font + different blackboard bold for serif font
\usepackage{newpxtext,eulerpx}
\renewcommand*{\mathbb}[1]{\varmathbb{#1}}
\renewcommand*{\hbar}{\hslash}
\DeclareMathOperator{\Graph}{graph}

\pagestyle{fancy} %better headers
\fancyhf{}
\rhead{Avinash Iyer}
\lhead{A Foray into Functional Analysis}

\setcounter{secnumdepth}{0}

\begin{document}
\RaggedRight
\tableofcontents
\section{Introduction}%
This is going to be part of the notes for my Honors thesis independent study, which will be focused on amenability and $C^{\ast}$-algebras. This section of notes will be focused on the essential results in functional analysis, starting from normed vector spaces, working our way up through $C^{\ast}$-algebras.\newline

The primary source for this section is going to be Timothy Rainone's \textit{Functional Analysis-En Route to Operator Algebras}, which has not been published yet.\newline

I do not claim any of this work to be original.
\section{Normed Vector Spaces}%
\subsection{Vector Spaces, Norms, and Basic Properties}%
All vector spaces are defined over $\C$. Most of the information here is in my Real Analysis II notes, so I'm going to skip to some of the more important content.
\begin{definition}[Vector Space]
  A vector space $V$ is a set closed under two operations
  \begin{align*}
    a: V\times V \rightarrow V,~\left(v_1,v_2\right)\mapsto v_1 + v_2\\
    m: \C\times V\rightarrow V,~\left(\lambda,v\right) \mapsto \lambda v.
  \end{align*}
  We refer to $a$ as addition, and $m$ as scalar multiplication; $(V,+)$ is an abelian ring.
\end{definition}
\begin{definition}[Norm]
  A norm is a function
  \begin{align*}
    \norm{\cdot}: V \rightarrow \R^+,~x\mapsto \norm{x}
  \end{align*}
  that satisfies the following properties:
  \begin{itemize}
    \item Positive definiteness: $\norm{v} = 0$ if and only if $v = 0_V$.
    \item Triangle inequality: $\norm{v+w} \leq \norm{v} + \norm{w}$.
    \item Absolute Homogeneity: $\norm{\lambda v} = \left\vert \lambda \right\vert\norm{v}$, for $\lambda \in \C$.
  \end{itemize}
  If a function $p: V\rightarrow \R^+$ satisfies the triangle inequality and absolute homogeneity, we say $p$ is a seminorm.
\end{definition}
We say the pair $\left(V,\norm{\cdot}\right)$ is a normed vector space.
\begin{definition}[Balls and Spheres]
  Let $X$ be a normed vector space, $x\in X$, and $\delta > 0$. Then,
  \begin{align*}
    U(x,\delta) &= \set{y\in X\mid d(x,y) < \delta}\\
    B(x,\delta) &= \set{y\in X\mid d(x,y) \leq \delta}\\
    S(x,\delta) &= \set{y\in X\mid d(x,y) = \delta}.
  \end{align*}
  For a normed vector space, we will use the following conventions for common sets:
  \begin{align*}
    U_X &= U(0,1)\\
    B_X &= B(0,1)\\
    S_X &= S(0,1)\\
    \mathbb{D} &= U_{\C}\\
    \mathbb{T} &= S_{\C}.
  \end{align*}
\end{definition}
\begin{definition}[Equivalent Norms]
  Two norms on $V$, $\norm{\cdot}_{a}$ and $\norm{\cdot}_{b}$ are said to be equivalent if there are two constants $C_1$ and $C_2$ such that
  \begin{align*}
    \norm{v}_{a} &\leq C_1\norm{v}_b\\
    \norm{v}_{b} &\leq C_2\norm{v}_a
  \end{align*}
  for all $v\in V$. We say $\norm{\cdot}_{a}\sim \norm{\cdot}_{b}$.
\end{definition}
\subsection{Examples}%
\begin{example}[Finite-Dimensional Vector Spaces]
  The vector space $\C^n$ is with the $p$-norm is denoted $\ell_{p}^{n}$, where for $p \in [1,\infty]$, the $p$-norm is defined by
  \begin{align*}
    \norm{x}_{p} &= \left(\sum_{i=1}^{n}\left\vert x_i \right\vert^p\right)^{1/p}.
  \end{align*}
  In the case with $p=2$, this gives the traditional Euclidean norm, and with $p = \infty$, this gives the $\sup$ norm:
  \begin{align*}
    \norm{x}_{\infty} &= \max_{1\leq i \leq n}\left\vert x_i \right\vert.
  \end{align*}
\end{example}
\begin{example}[A Sequence Space]
  We let $\ell_{p} = \set{\left(x_{n}\right)_n\mid x_n\in \C,\norm{x}_p < \infty}$ be the collection of sequences in $\C$ with finite $p$-norm. Here,
  \begin{align*}
    \norm{x}_p &= \left(\sum_{n=1}^{\infty}\left\vert x_n \right\vert^p\right)^{1/p}.
  \end{align*}
  In the case with $p = \infty$, this gives the sequence space $\ell_{\infty}$, which has norm
  \begin{align*}
    \norm{x}_{\infty} &= \sup_{n\in \N}\left\vert x_n \right\vert.
  \end{align*}
\end{example}
\begin{example}[A Function Space]
  We let $\ell^{\infty}\left(\Omega\right)$ denote the set of all bounded functions $f: \Omega \rightarrow \C$, equipped with the norm
  \begin{align*}
    \norm{f}_{\infty} &= \sup_{x\in \Omega}\left\vert f(x) \right\vert.
  \end{align*}
  If $\Omega = \left(\Omega,\mathcal{M},\mu\right)$ is a measure space, then we let $L^{\infty}\left(\Omega\right)$ be the space of $\mu$-a.e. equal essentially bounded measurable functions, under the norm
  \begin{align*}
    \norm{f}_{\infty} &= \esssup_{x\in \Omega}\left\vert f(x) \right\vert.
  \end{align*}
\end{example}
\subsection{Series Convergence and Completeness}%
\begin{proposition}[Criteria for Banach Spaces]
Let $X$ be a normed vector space. The following are equivalent:
\begin{enumerate}[(i)]
  \item $X$ is a Banach space.\footnote{Complete normed vector space.}
  \item If $\left(x_k\right)_k$ is a sequence of vectors such that $\sum_{k=1}^{\infty}\norm{x_k}$ converges, then $\sum_{k=1}^{\infty}x_k$ converges.
  \item If $\left(x_k\right)_k$ is a sequence in $X$ such that $\norm{x_k} < 2^{-k}$, then $\sum_{k=1}^{\infty}x_k$ converges.
\end{enumerate}
\end{proposition}
\begin{proof}
  To show (i) implies (ii), for $n > m > N$, we have
  \begin{align*}
    \norm{s_n - s_m} &= \norm{\sum_{k=m+1}^{n}x_k}\\
                     &\leq \sum_{k=m+1}^{n}\norm{x_k}\\
                     &< \epsilon,
  \end{align*}
  implying that $s_n$ is Cauchy, and thus converges since $X$ is complete.\newline

  Since $\sum_{k=1}^{\infty}2^{-k}$ converges, it is clear that (ii) implies (iii).\newline

  To show (iii) implies (i), we let $\left(x_n\right)_n$ be a Cauchy sequence in $X$. We only need construct a convergent subsequence in order to show that $\left(x_n\right)_n$ converges.\newline

  Chose $n_1\in \N$ such that for $n,m\geq n_1$, $ \norm{x_m - x_n} < \frac{1}{2^2}$, and inductively define $n_j > n_{j-1}$ such that $n,m\geq n_j$ implies $\norm{x_m - x_n} < \frac{1}{2^{j+1}}$.\newline

  Let $y_1 = x_{n_1}$, $y_{j} = x_{n_j} - x_{n_{j-1}}$. Then,
  \begin{align*}
    \norm{y_j} &= \norm{x_{n_j} - x_{n_{j-1}}}\\
               &< \frac{1}{2^{j}},
  \end{align*}
  so $\sum_{j=1}^{\infty}y_j$ converges by our assumption. By telescoping, we see that $\sum_{j=1}^{k}y_j = x_{n_k}$, so $\left(x_{n_{k}}\right)_k$ converges.
\end{proof}
\subsection{Quotient Spaces}%
Let $X$ be a normed vector space. Then, for $E\subseteq X$ a subspace, there is a quotient space $X/E$ with the projection map $\pi: X\rightarrow X/E$, $x\mapsto x + E$. We want to make $X/E$ into a normed space --- in order to do this, we use the distance function:
\begin{align*}
  \dist_{E}(x) &= \inf_{y\in E}d(x,y),
\end{align*}
which is uniformly continuous. For $E$ closed, then $\dist_{E}(x) = 0$ if and only if $x\in E$.
\begin{proposition}[Quotient Space Norm]
  Let $X$ be a normed vector space, and $E\subseteq X$ a subspace. Set
  \begin{align*}
    \norm{x + E}_{X/E} &= \dist_{E}(x).
  \end{align*}
  Then,
  \begin{enumerate}[(1)]
    \item $\norm{\cdot}_{X/E}$ is a well-defined seminorm on $X/E$.
    \item If $E$ is closed, then $\norm{\cdot}_{X/E}$ is a norm on $X/E$.
    \item $\norm{x+E}_{X/E} \leq \norm{x}$ for all $x\in X$.
    \item If $E$ is closed, then $\pi: X\rightarrow X/E$ is Lipschitz.
    \item If $X$ is a Banach space and $E$ is closed, then $X/E$ is also a Banach space.
  \end{enumerate}
\end{proposition}
\begin{proof}\hfill
  \begin{enumerate}[(1)]
    \item We will show that $\norm{\cdot}_{X/E}$ is well-defined. If $x + E = x' + E$, $x'-x\in E$, so for every $y\in E$, $x'-x + y\in E$. Thus,
    \begin{align*}
      \norm{x-y} &= \norm{x'-\left(x'-x+y\right)}\\
                 &\geq \inf_{z\in E}\norm{x' - z}\\
                 &= \norm{x' + E}_{X/E}.
    \end{align*}
    Thus, $\norm{x + E}_{X/E} \geq \norm{x' + E}_{X/E}$, and vice versa.\newline

    Let $\lambda \in \C\setminus \set{0}$, and $x\in X$. Then,
    \begin{align*}
      \norm{\lambda\left(x + E\right)}_{X/E} &= \norm{\lambda x + E}_{X/E}\\
                                             &= \inf_{y\in E}\norm{\lambda x - y}\\
                                             &= |\lambda|\inf_{y\in E}\norm{x - \lambda^{-1}y}\\
                                             &= |\lambda|\inf_{y'\in E}\norm{x-y}\\
                                             &= |\lambda|\norm{x + E}_{X/E}
    \end{align*}
    Given $x,x'\in X$ and a fixed $\ve > 0$, we have
    \begin{align*}
      \norm{x + E} + \frac{\ve}{2} &> \norm{x-y}
    \end{align*}
    for some $y\in E$, and
    \begin{align*}
      \norm{x' + E} + \frac{\ve}{2} &> \norm{x'-y'}
    \end{align*}
    for some $y'\in E$. Thus,
    \begin{align*}
      \norm{\left(x+x'\right)-\left(y+y'\right)} &\leq \norm{x-y} + \norm{x' - y'}\\
                                                 &< \ve + \norm{x + E} + \norm{x' + E}.
    \end{align*}
    Since $y + y'\in E$, we have
    \begin{align*}
      \norm{\left(x+E\right) + \left(x' + E\right)}_{X/E} &= \norm{x + x' + E}_{X/E}\\
                                                    &\leq \norm{\left(x+x'\right) - \left(y+y'\right)}\\
                                                    &< \ve + \norm{x + E}_{X/E} + \norm{x' + E}_{X/E},
    \end{align*}
    meaning
    \begin{align*}
      \norm{\left(x+E\right) + \left(x' + E\right)} \leq \norm{x + E} + \norm{x' + E}.
    \end{align*}
  \item If $E$ is closed, and $\norm{x + E} = 0$, then $x\in E$ so $x + E = 0_{X/E}$.
  \item For $x\in X$,
    \begin{align*}
      \norm{x + E}_{X/E} &= \inf_{y\in E}\norm{x-y}\\
                         &\leq \norm{x}.
    \end{align*}
  \item We have
    \begin{align*}
      \norm{\left(x+E\right) - \left(x' + E\right)}_{X/E} &= \norm{x-x' + E}_{X/E}\\
                                                          &\leq \norm{x-x'}.
    \end{align*}
  \item Let $X$ be complete and $E\subseteq X$ be closed. Let $\left(x_k + E\right)_k$ be a sequence in $X/E$ with $\norm{x_k + E} < 2^{-k}$. We want to show that $\sum_{k=1}^{\infty}\left(x_k + E\right)$ converges.\newline

    For each $k$, since $\norm{x_k + E} < 2^{-k}$, there exists $y_k\in E$ such that $\norm{x_k - y_k} < 2^{-k}$. Since $X$ is complete, $\sum_{k=1}^{\infty}x_k - y_k$ converges.\newline

    Let $\left(\sum_{k=1}^{n}x_k - y_k\right)_n \rightarrow x$ in $X$. Applying the canonical projection map, $\pi$, to both sides, we get
    \begin{align*}
      \sum_{k=1}^{n}\left(x_k + E\right) &= \sum_{k=1}^{n}\pi\left(x_k\right)\\
                                         &= \pi\left(\sum_{k=1}^{n}\left(x_k - y_k\right)\right)\\
                                         &\rightarrow \pi(x),
    \end{align*}
    implying that $\sum_{k=1}^{\infty}\left(x_k + E\right)$ converges.
  \end{enumerate}
\end{proof}
\begin{exercise}
  Consider $\ell_{\infty}$ and its closed subspace $c_0$. If $\pi: \ell_{\infty}\rightarrow \ell_{\infty}/c_0$ denotes the canonical quotient map, with $\left(z_k\right)_k\in \ell_{\infty}$, show that
  \begin{align*}
    \norm{\left(z_k\right)_k + c_0} &= \limsup_{k\rightarrow\infty}\left\vert z_k \right\vert
  \end{align*}
\end{exercise}
\begin{solution}
  Let $z = \left(z_k\right)_k\in \ell_{\infty}$. We define the distance
  \begin{align*}
    \dist_{c_0}(z) &= \inf_{t\in c_0}\left\vert z_k - t_k \right\vert.
  \end{align*}
  Let $w\in c_c$ be defined by
  \begin{align*}
    w &= \left(z_1,z_2,\dots,z_{n-1},0,0,\dots\right).
  \end{align*}
  Then,
  \begin{align*}
    \norm{z-w}_{\infty} &= \sup_{k\in \N}\left\vert z_k - w_k \right\vert\\
                        &= \sup_{k\geq n}\left\vert z_k - w_k \right\vert,
  \end{align*}
  meaning that
  \begin{align*}
    \dist_{c_c}(z) &\leq \sup_{k\geq n}\left\vert z_k \right\vert.
  \end{align*}
  Since $c_0 \supseteq c_c$, we have
  \begin{align*}
    \dist_{c_0}(z) &\leq \dist_{c_c}(z)\\
                   &\leq \inf_{n\geq 1}\left(\sup_{k\geq n}\left\vert z_k \right\vert\right)\\
                   &= \limsup_{k\rightarrow\infty}\left\vert z_k \right\vert.
  \end{align*}
  Now, we show that $\limsup_{k\rightarrow\infty}\left\vert z_k \right\vert \leq \dist_{c_c}\left(z\right)$. Given $\ve > 0$, there exists $w\in c_c$ such that
  \begin{align*}
    \norm{z-w} < \dist_{c_c}(z) + \ve.
  \end{align*}
  Additionally, for $w$ that terminates at $n-1$ (i.e., is equal to $0$ for all $k\geq n$), we have
  \begin{align*}
    \sup_{k\geq n}\left\vert z_k-w_k \right\vert &\leq \sup_{k\in \N}\left\vert z_k - w_k \right\vert,
  \end{align*}
  meaning
  \begin{align*}
    \limsup_{k\rightarrow\infty}\left\vert z_k \right\vert &= \inf_{n\geq 1}\left(\sup_{k\geq n}\left\vert z_k \right\vert\right)\\
                                                         &\leq \sup_{k\geq n}\left\vert z_k - w_k \right\vert\\
                                                         &\leq \sup_{k\in \N}\left\vert z_k - w_k \right\vert\\
                                                         &= \norm{z-w}\\
                                                         &< \dist_{c_c}(z) + \ve,
  \end{align*}
  implying that
  \begin{align*}
    \limsup_{k\rightarrow\infty}\left\vert z_k \right\vert &= \dist_{c_c}(z).
  \end{align*}
  For $\ve > 0$, let $w\in c_0$ be such that
  \begin{align*}
    \norm{z-w} &< \dist_{c_0}\left(z\right) + \ve/2.
  \end{align*}
  Additionally, let $\lambda \in c_c$ such that $\norm{\lambda - w} < \ve/2$. Then, we have
  \begin{align*}
    \dist_{c_0}\left(z\right) + \ve &> \norm{z-\lambda} + \norm{\lambda - w}\\
                                    &\geq \dist_{c_c}\left(z\right) + \ve/2\\
                                    &\geq \limsup_{k\rightarrow\infty}\left\vert z_k \right\vert.
  \end{align*}
  Thus, $\limsup_{k\rightarrow\infty}\left\vert z_k \right\vert \leq \dist_{c_0}\left(z\right)$, meaning $\limsup_{k\rightarrow\infty}\left\vert z_k \right\vert = \dist_{c_0}\left(z\right)$.
\end{solution}
\subsection{Bounded Linear Operators}%
\begin{definition}[Continuous Functions]
  A function $f: \left(X, d_X\right)\rightarrow \left(Y,d_Y\right)$ is called Lipschitz if there is a constant $C>0$ such that
  \begin{align*}
    d_Y\left(f(x),f(x')\right) \leq Cd_x\left(x,x'\right)
  \end{align*}
  for all $x,x'\in X$.\newline

  If $C \leq 1$, a Lipschitz map is known as a contraction.\newline

  If
  \begin{align*}
    d_Y\left(f(x),f\left(x'\right)\right) = d_X\left(x,x'\right)
  \end{align*}
  for all $x,x'\in X$, then $f$ is known as an isometry.
\end{definition}
\begin{proposition}[Categorization of Continuous Linear Maps]
  Let $X$ and $Y$ be normed vector spaces, and let $T: X\rightarrow Y$ be a linear map. The following are equivalent:
  \begin{enumerate}[(i)]
    \item $T$ is continuous at $0$.
    \item $T$ is continuous.
    \item $T$ is uniformly continuous.
    \item $T$ is Lipschitz.
    \item There exists a constant $C > 0$ such that $\norm{T(x)}\leq C\norm{x}$ for all $x\in X$.
  \end{enumerate}
\end{proposition}
\begin{definition}[Bounded Linear Operator]
  Let $X$ and $Y$ be normed vector spaces, and let $T: X\rightarrow Y$ be a linear map.
  \begin{enumerate}[(1)]
    \item $T$ is bounded if $T\left(B_X\right)$ is bounded in $Y$. Equivalently, $T$ is bounded if and only if
      \begin{align*}
        \sup_{x\in B_X}\norm{T(x)} < \infty,
      \end{align*}
      or that $\exists r > 0$ such that $T\left(B_X\right) \subseteq B_Y\left(0,r\right)$.
    \item The operator norm of $T$ is the value
      \begin{align*}
        \norm{T}_{\text{op}} &= \sup_{x\in B_X}\norm{T(x)}.
      \end{align*}
  \end{enumerate}
\end{definition}
\begin{lemma}
  Let $T: X\rightarrow Y$ be a linear map between normed vector spaces. Then,
  \begin{align*}
    \norm{T}_{\text{op}} &= \sup_{x\in S_X}\norm{T(x)}
    \intertext{and for all $x\in X$,}
    \norm{T(x)} \leq \norm{T}_{\text{op}}\norm{x}.
  \end{align*}
\end{lemma}
\begin{lemma}
  Let $T: X\rightarrow Y$ be a bounded linear map between normed vector spaces. Then, for any $x\in X$ and $r > 0$,
  \begin{align*}
    r\norm{T}_{\text{op}}\leq \sup_{y\in B\left(x,r\right)}\norm{T(y)}
  \end{align*}
\end{lemma}
\begin{proof}
  Let $C = \sup_{y\in B\left(x,r\right)}\norm{T(y)}$. If $z\in B\left(0,r\right)$, then $z+x,z-x\in B(x,r)$, meaning
  \begin{align*}
    2T\left(z\right) &= T\left(z+x\right) + T\left(z-x\right),
  \end{align*}
  so by the triangle inequality, we get
  \begin{align*}
    2\norm{T(z)} &\leq \norm{T(z+x)} + \norm{T(z-x)}\\
                 &\leq 2\max\set{\norm{T(z+x)},\norm{T\left(z-x\right)}}\\
                 &\leq 2C.
  \end{align*}
  Thus,
  \begin{align*}
    \norm{T(z)} \leq \sup_{y\in B\left(x,r\right)}\norm{T(y)},
  \end{align*}
  meaning
  \begin{align*}
    r\norm{T}_{\text{op}} \leq \sup_{y\in B\left(x,r\right)}\norm{T(y)}.
  \end{align*}
\end{proof}
\begin{remark}
For a linear map $T: X\rightarrow Y$, the following are equivalent:
\begin{enumerate}[(1)]
  \item $T$ is continuous.
  \item $T$ is bounded.
  \item $\norm{T}_{\text{op}} < \infty$.
\end{enumerate}
\end{remark}
\begin{definition}
  Let $X$ and $Y$ be normed spaces, $T: X\rightarrow Y$ a linear map.
  \begin{enumerate}[(1)]
    \item $T$ is bounded below if there exists $C_2$ such that $\norm{T(x)}\geq C_2\norm{x}$ for all $x\in X$.
    \item $T$ is bicontinuous if $T$ is bounded and bounded below.
      \begin{align*}
        C_2\norm{x} \leq \norm{T(x)}\leq C_1\norm{x}
      \end{align*}
    \item $T$ is a bicontinuous isomorphism if $T$ is bijective, linear, and bicontinuous. We say $X$ and $Y$ are bicontinuously isomorphic.
    \item We say $T$ is an isometric isomorphism if $T$ is bijective, linear, and an isometry.
  \end{enumerate}
\end{definition}
\begin{example}
  Let $\rho$ be the continuous surjective wrapping function $\rho: [0,2\pi]\rightarrow \mathbb{T}$, $\rho(t) = e^{it}$. There is an induced isometry
  \begin{align*}
    T_{\rho}: C\left(\mathbb{T}\right) \rightarrow C\left([0,2\pi]\right),
  \end{align*}
  defined by $T_{\rho}\left(f\right)(t) = f\circ \rho\left(t\right) = f\left(e^{it}\right)$.\newline

  The range of $T_{\rho}$ is $C= \set{G\in C\left([0,2\pi]\right)\mid g(0) = g(2\pi)}$, which means that $C\left(\mathbb{T}\right) $ and $ C$ are isometrically isomorphic Banach spaces.
\end{example}
\begin{proposition}
  Let $X$ and $Y$ be normed spaces, and $T:X\rightarrow Y$ be a linear map. The following are equivalent.
  \begin{enumerate}[(i)]
    \item $T$ is bicontinuous.
    \item $T: X\rightarrow \ran(T)$ is a linear isomorphism and homeomorphism.
  \end{enumerate}
\end{proposition}
\begin{proof}
  Let $T$ be bicontinuous. Then, $T$ is linear, injective, and surjective onto $\ran(T)$. Since $T$ is continuous, $T$ is bounded. Let $S: \ran(T) \rightarrow X$ be defined by $S\left(T(x)\right) = x$. We can see that $S$ is well-defined, since $T: X\rightarrow \ran(T)$ is surjective, and so has a left inverse. Similarly, since $\norm{S(T(x))} = \norm{x} \leq \frac{1}{C_2}\norm{T(x)}$, $S$ is continuous.\newline

  Let $S: \ran(T) \rightarrow X$ be defined by $S(T(x)) = x$. Since $T$ is continuous, it is bounded, so
  \begin{align*}
    \norm{T(x)}\leq \norm{T}_{\text{op}}\norm{x}.
  \end{align*}
  Since $S$ is bounded,
  \begin{align*}
    \norm{x} &= \norm{S(T(x))}\\
             &= \norm{S}_{\text{op}}\norm{T(x)},
  \end{align*}
  so $\frac{1}{\norm{S}_{\text{op}}}\norm{x} \leq \norm{T(x)}$.
\end{proof}
\begin{corollary}
  Let $X$ be a vector space with $\norm{\cdot}$ and $\norm{\cdot}'$ two norms. The following are equivalent:
  \begin{enumerate}[(i)]
    \item The norms $\norm{\cdot}$ and $\norm{\cdot}'$ are equivalent.
    \item The map $\id_{X}:\left(X,\norm{\cdot}\right)\rightarrow \left(X,\norm{\cdot}'\right)$.
  \end{enumerate}
\end{corollary}
\begin{proposition}[Properties of Bounded Linear Operators]
  Let $X,Y,Z$ be normed spaces, $T: X\rightarrow Y$, $S: X\rightarrow Y$, and $R:Y\rightarrow Z$ be linear maps.
  \begin{enumerate}[(1)]
    \item $\norm{\alpha T}_{\text{op}}= \left\vert \alpha \right\vert\norm{T}_{\text{op}}$
    \item $\norm{T + S}_{\text{op}}\leq \norm{T}_{\text{op}} + \norm{S}_{\text{op}}$
    \item $\norm{T}_{\text{op}}  = 0$ if and only if $T = 0$
    \item $\norm{R\circ T}_{\text{op}}\leq \norm{R}_{\text{op}}\norm{T}_{\text{op}}$
    \item $\norm{\id_{X}}_{\text{op}} = 1$
    \item If $E\subseteq X$ is a subspace, then $\norm{T|_{E}}_{\text{op}}\leq \norm{T}_{\text{op}}$
  \end{enumerate}
\end{proposition}
\begin{proof}
  We will prove (4) here. For $x\in B_{X}$, we have
  \begin{align*}
    \norm{R\circ T(x)} &= \norm{R\left(T(x)\right)}\\
                       &\leq \norm{R}_{\text{op}}\norm{T(x)}\\
                       &\leq \norm{R}_{\text{op}}\norm{T}_{\text{op}}.
  \end{align*}
  Taking the supremum, we obtain $\norm{R\circ T}_{\text{op}}\leq \norm{R}_{\text{op}}\norm{T}_{\text{op}}$.
\end{proof}
\begin{recall}
  $\mathcal{L}(X,Y)$ is the set of all linear operators with domain $X$ and codomain $Y$.
\end{recall}
\begin{proposition}
  Let $X$ and $Y$ be normed spaces.
  \begin{enumerate}[(1)]
    \item The collection $\mathcal{B}(X,Y) = \set{T\in \mathcal{L}\left(X,Y\right)\mid \norm{T}_{\text{op}} < \infty}$ equipped with the operator norm is a normed space known as the space of bounded linear operators between $X$ and $Y$.
    \item If $Y$ is a Banach space, then $\mathcal{B}\left(X,Y\right)$ is a Banach space.
    \item The continuous dual space, $X^{\ast} = \mathcal{B}\left(X,\C\right)$ is a Banach space.
  \end{enumerate}
\end{proposition}
\begin{proof}
  We will prove (2). Let $\left(T_n\right)_n$ be Cauchy under $\norm{\cdot}_{\text{op}}$. Since Cauchy sequences are bounded, there is some $C > 0$ such that $\norm{T_n}_{\text{op}}\leq C$ for all $n\geq 1$. For $x\in X$,
  \begin{align*}
    \norm{T_n(x) - T_m(x)} \leq \norm{T_n - T_m}_{\text{op}}\norm{x},
  \end{align*}
  meaning $\left(T_n(x)\right)_{n}$ is Cauchy in $Y$. Since $Y$ is complete, we define
  \begin{align*}
    T(x) &= \lim_{n\rightarrow\infty}T_n(x)
  \end{align*}
  in $Y$. If $x\in B_X$, we have
  \begin{align*}
    \norm{T(x)} &= \norm{\lim_{n\rightarrow\infty}T_n(x)}\\
                &= \lim_{n\rightarrow\infty}\norm{T_n(x)}\\
                &\leq \limsup_{n\rightarrow\infty}\norm{T_n(x)}\\
                &\leq C\norm{x},
  \end{align*}
  meaning $\norm{T}_{\text{op}} \leq C$.\newline

  Let $\ve > 0$, and $N\in \N$ large such that $n,m\geq N$, $\norm{T_n - T_m}_{\text{op}} \leq \ve$. For $x\in B_X$,
  \begin{align*}
    \norm{T_n(x)-T(x)} &= \lim_{m\rightarrow\infty}\norm{T_n(x)-T_m(x)}\\
                      &\leq \limsup_{m\rightarrow\infty}\norm{T_n - T_m}_{\text{op}}\norm{x}\\
                      &< \ve.
  \end{align*}
  Thus, $\norm{T - T_n}_{\text{op}} < \ve$ for all $n\geq N$.
\end{proof}
\begin{definition}[Algebras]
  Let $A$ be an algebra over $\C$.
  \begin{enumerate}[(1)]
    \item If $A$ admits a norm $\norm{\cdot}$ satisfying $\norm{ab} \leq \norm{a}\norm{b}$, then $A$ is a normed algebra. If $A$ is unital, then $\norm{1_A} = 1$.
    \item If $A$ is complete with respect to its norm, then $A$ is called a Banach algebra, and if $A$ is unital, then $A$ is a unital Banach algebra.
  \end{enumerate}
\end{definition}
\begin{lemma}
  In a normed algebra $A$, the map $\cdot: A\times A \rightarrow A,(a,b)\mapsto ab$ is continuous.
\end{lemma}
\begin{proposition}
  Let $X$ be a normed space. The set of bounded operators $\mathcal{B}\left(X,X\right) = \mathcal{B}(X)$ is a unital normed algebra. Moreover, if $X$ is a Banach space, then $\mathcal{B}\left(X\right)$ is a Banach algebra.
\end{proposition}
\begin{proposition}
  Let $A$ be a unital Banach algebra, $a\in A$. The series
  \begin{align*}
    \exp(a) &= \sum_{n=0}^{\infty}\frac{a^n}{n!}
  \end{align*}
  converges absolutely in $A$. We call $\exp(a) $ the exponential of $a$.
  \begin{enumerate}[(1)]
    \item $\exp(0) = 1_A$
    \item If $A$ is commutative, then $\exp(a+b) = \exp(a)\exp(b)$.
    \item We have $\exp(a)\in \text{GL}(A)$ with $\exp(a)^{-1} = \exp(-a)$.
    \item $\norm{\exp(a)}\leq \exp\left(\norm{a}\right)$.
  \end{enumerate}
\end{proposition}
\subsection{Quotient Maps}%
\begin{definition}
  A map $f: X\rightarrow Y$ is called open if $U\subseteq X$ is open implies $f(U)\subseteq Y$ is open.
\end{definition}
\begin{proposition}
  Let $X$ and $Y$ be normed spaces, $T: X\rightarrow Y$ a linear map. The following are equivalent:
  \begin{enumerate}[(i)]
    \item $T$ is surjective and open.
    \item $T\left(U_X\right)\subseteq Y$ is open.
    \item There exists $\delta > 0$ such that $\delta U_Y \subseteq T\left(U_X\right)$.
    \item There exists $\delta$ such that $\delta B_Y \subseteq T\left(B_X\right)$.
    \item There exists $M > 0$ such that for all $y\in Y$, there exists $x\in X$ with $T(x) = y$ and $\norm{x} \leq M\norm{y}$.
  \end{enumerate}
\end{proposition}
\begin{proof}\hfill
  To see (i) implies (ii), if $T$ is surjective and open, then it is clear that $T\left(U_X\right)$, which is the image of an open set, is open.\newline

  To see (ii) implies (iii), if $T\left(U_X\right)$ is open, we have $0_Y\in T\left(U_X\right)$, so there is some $\delta$ such that $U\left(0,\delta\right) \subseteq T\left(U_X\right)$, meaning $\delta U_{Y} \subseteq T\left(U_X\right)$.\newline

  Assuming (iii), we see that $\frac{\delta}{2}B_Y \subseteq \delta U_Y \subseteq T\left(U_X\right)\subseteq T\left(B_X\right)$.\newline

  To see (iv) implies (v), let $\delta$ be such that $\delta B_Y\subseteq T\left(B_X\right)$, and set $M = \frac{1}{\delta}$. Note that for $y\in Y,y\neq 0$, $\frac{\delta}{\norm{y}}y\in \delta B_Y$, meaning $\frac{\delta}{\norm{y}} y = T(x)$ for some $x\in B_X$, implying that $T\left(\frac{\norm{y}}{\delta}x\right) = y$. Finally, since $x\in B_X$, $\frac{\norm{y}}{\delta}\norm{x} \leq \frac{1}{\delta}\norm{y} = M\norm{y}$.\newline

  To see (v) implies (i), we can see that $T$ is surjective by the assumption. Let $U\subseteq X$ be open, $y_0\in T(U)$. Then, there exists $x_0$ such that $T\left(x_0\right) = y_0$, and $\delta > 0$ such that $U\left(x_0,\delta\right)\subseteq U$. Note that $U\left(x_0,\delta\right) = x_0 + \delta U_X$, so $x_0 + \delta U_X \subseteq U$. Applying $T$, we get $T\left(x_0 + \delta U_X\right)\subseteq T(U)$, or $y_0 + \delta T\left(U_X\right)\subseteq T(U)$. By assumption, since given $y\in U_Y$, there exists $x\in X$ such that $\norm{x} \leq M\norm{y}$, meaning $\norm{x}\leq M$, we have $U_Y\subseteq T\left(MU_X\right)$. Thus, $\frac{1}{M}U_Y\subseteq T\left(U_X\right)$, meaning $y_0 + \frac{\delta}{M}U_Y\subseteq y_0\delta T\left(U_X\right)\subseteq T(U)$, so $U_Y\left(y_0,\frac{\delta}{M}\right)\subseteq T(U)$.
\end{proof}
\begin{definition}
  Let $X$ and $Y$ be normed vector spaces.
  \begin{enumerate}[(1)]
    \item A bounded linear map $T: X\rightarrow Y$ that is surjective and open is known as a quotient map.
    \item If $T\left(U_X\right) = U_Y$, then $T$ is called a $1$-quotient map.
  \end{enumerate}
\end{definition}
\begin{exercise}
  If $T\left(B_X\right) = B_Y$, show that $T\left(U_X\right) = U_Y$.
\end{exercise}
\begin{solution}
  Since $T\left(B_X\right) = B_Y$, it is the case that $\left(T\left(B_X\right)\right)^{\circ} = B_Y^{\circ}$. Since $T$ is an open map, $T$ is continuous, meaning $\left(T\left(B_X\right)\right)^{\circ} = T\left(B_X^{\circ}\right)$. Thus, $T\left(U_X\right) = U_Y$.
\end{solution}
\begin{proposition}
  Let $X$ and $Y$ be normed vector spaces with $T: X\rightarrow Y$ a quotient map. If $X$ is a Banach space, then $Y$ is a Banach space.
\end{proposition}
\begin{proof}
  We will show that $Y$ is complete by showing that an absolutely convergent series converges.\newline

  Let $\left(y_k\right)_k$ be a sequence in $Y$ with $\sum_{k=1}^{\infty}\norm{y_k} < \infty$. Since $T$ is a quotient map, there is a universal $M > 0$ such that for all $k$, there is $x_k\in X$ such that $T\left(x_k\right) = y_k$ and $\norm{x_k} \leq M\norm{y_k}$. Thus,
  \begin{align*}
    \sum_{k=1}^{\infty} &\leq M\sum_{k=1}^{\infty}\norm{y_k}\\
    &< \infty.
  \end{align*}
  Since $X$ is complete, $\sum_{k=1}^{\infty}x_k$ converges. Let $\sum_{k=1}^{\infty}x_k = x$. Then, $\left(T\left(\sum_{k=1}^{n}x_k\right)\right)_n\xrightarrow{n\rightarrow\infty}T(x) $, meaning $\sum_{k=1}^{\infty}y_k = T(x)$. Thus, $\sum_{k=1}^{\infty}y_k$ converges in $Y$, so $Y$ is a Banach space.
\end{proof}
\begin{proposition}
  Let $X$ be a normed vector space, $E\subseteq X$ a closed subspace. The canonical quotient map, $\pi: X\rightarrow X/E$ is a $1$-quotient map.
\end{proposition}
\begin{proof}
  We know that $\norm{\pi\left(x\right)} \leq \norm{x}$, meaning $\pi\left(U_X\right)\subseteq U_{X/E}$.\newline

  Let $\pi(x) = x+E \subseteq U_{X/E}$. Then, $\inf_{y\in E}\norm{x-y}\leq 1$, meaning there exists some $y$ such that $\norm{x-y} < 1$, meaning $\pi\left(x-y\right) = \pi(x)$.
\end{proof}
\begin{corollary}
  If $X$ is a Banach space, $E\subseteq X$ a closed subspace, then $X/E$ is a Banach space.
\end{corollary}
\begin{corollary}
  Let $X$ be a normed vector space and $E\subseteq X$ be closed. If two of $X,E,X/E$ are complete, the third is also complete.
\end{corollary}
\begin{proof}
  We have shown that if $X$ is complete, then $E$ is necessarily complete (since $E$ is closed) and $X/E$ is complete as shown above.\newline

  Let $E$ and $X/E$ be complete. We now want to show that $X$ is complete. Let $\left(x_k\right)_k$ be Cauchy in $X$.\newline

  For each $k$, let $x_k = s_k + y_k$, where $y_k\in E$ and $s_k + E = \pi\left(x_k\right)$. Notice that, since $x_k$ is Cauchy, so too is $s_k$, as $\norm{s_k} \leq \norm{x_k}$ for all $k$. Additionally, for $m,n \geq N$, we have
  \begin{align*}
    \norm{x_{m} - x_{n}} &= \norm{s_m + y_m - \left(s_n + y_n\right)}\\
                         &\leq \norm{s_m - s_n} + \norm{y_m - y_n}\\
                         &< \ve,
  \end{align*}
  implying that $\left(y_k\right)_k$ is Cauchy in $E$. Since $X/E$ and $E$ are complete, we define $x = \lim_{k\rightarrow\infty}s_k + \lim_{k\rightarrow\infty}y_k$. Finally, for $m,n\geq N$, we have
  \begin{align*}
    \norm{x - x_n} &= \lim_{m\rightarrow\infty}\norm{x_m - x_n}\\
                   &\leq \ve,
  \end{align*}
  meaning $\left(x_k\right)_k \xrightarrow{k\rightarrow\infty} x$, so $X$ is complete.
\end{proof}
\begin{proposition}
  Let $X$ and $Y$ be normed spaces, $E\subseteq X$ a closed subspace, and $T: X\rightarrow Y$ bounded linear with $E\subseteq \ker(T)$. Then, there exists a unique bounded linear map $\overline{T}: X/E\rightarrow Y$ such that $\overline{T}\circ \pi = T$. Moreover, $\overline{T}$ is injective if and only if $E = \ker(T)$ and $\norm{\overline{T}}=\norm{T}$.
\end{proposition}
\begin{proof}
  The existence and uniqueness of $\overline{T}: X/E\rightarrow Y$ such that $\overline{T}\circ \pi = T$ follows from the First Isomorphism Theorem for vector spaces, as does the fact that $\overline{T}$ is injective and only if $\ker(T) = E$.\newline

  Let $x+E\in X/E$. For $y\in E$, we have
  \begin{align*}
    \norm{\overline{T}\left(x+E\right)} &= \norm{\overline{T}\left(x-y+E\right)}\\
                                        &= \norm{T\left(x-y\right)}\\
                                        &\leq \norm{T}\norm{x-y}.
  \end{align*}
  Taking infimum over all $y\in E$, we get $\norm{\overline{T}\left(x+E\right)} \leq \norm{T}\norm{x+E}$, meaning $\norm{\overline{T}}\leq \norm{T}$. Additionally,
  \begin{align*}
    \norm{T} &= \norm{\overline{T}\circ \pi}\\
             &\leq \norm{\overline{T}}\norm{\pi}\\
             &= \norm{\overline{T}}.
  \end{align*}
\end{proof}
\begin{theorem}[First Isomorphism Theorem for Normed Vector Spaces]
  Let $X$ and $Y$ be normed vector spaces, $T\in \mathcal{B}\left(X,Y\right)$.
  \begin{enumerate}[(1)]
    \item $T$ is a quotient map if and only if $\overline{T}: X/\ker(T) \rightarrow Y$ is a bicontinuous isomorphism.
    \item $T$ is a $1$-quotient map if and only if $\overline{T}: X/\ker(T) \rightarrow Y$ is an isometric isomorphism.
  \end{enumerate}
\end{theorem}
\begin{proof}\hfill
  \begin{enumerate}[(1)]
    \item Let $\overline{T}: X/\ker(T)\rightarrow Y$ be a bicontinuous isomorphism. Since $\overline{T}$ is bicontinuous, it is a homeomorphism, meaning it is open and surjective. Since $\pi$ is a quotient map, so too is $T: \overline{T}\circ \pi$.\newline

      Suppose $T$ is a quotient map. Then, $T$ is surjective, meaning $\overline{T}$ is an isomorphism. Since $T$ is bounded below, $\overline{T}$ is also bounded. Let $\pi(x) = x + \ker(T)\in X/\ker(T)$, with $T(x) = y$. Let $M$ be such that $\norm{x}\leq M\norm{y}$. There is an $x'\in X$ with $T\left(x'\right) = y$, and $\norm{x'}\leq M\norm{y}$. Thus, $x-x'\in \ker(T)$, so $\pi(x) = \pi\left(x'\right)$, meaning
      \begin{align*}
        \norm{\overline{T}\circ \pi (x)} &= \norm{T\circ \pi\left(x'\right)}\\
                                         &= \norm{y}\\
                                         &\geq M^{-1}\norm{x'}\\
                                         &\geq M^{-1}\norm{\pi\left(x'\right)}\\
                                         &= M^{-1}\norm{\pi\left(x\right)},
      \end{align*}
      meaning $T$ is bounded below.
    \item Suppose $\overline{T}:X/\ker(T) \rightarrow Y$ is an isometric isomorphism. Then, $\overline{T}$ is a $1$-quotient map, and since $\pi$ is a $1$-quotient map, so too is $T = \overline{T}\circ \pi$.\newline

      Suppose $T$ is a $1$-quotient map. Since $T$ is surjective, $\overline{T}$ is an isomorphism. Since $T$ is a $1$-quotient map, $\norm{T} = \sup_{x\in U_X}\norm{T(x)}\leq 1$, meaning $\norm{\overline{T}}\leq \norm{T} \leq 1$. Consider $S = \left(\overline{T}\right)^{-1}: Y\rightarrow X/\ker(T)$; $S$ is also an isomorphism, so $S\circ \overline{T} = = \id_{X/\ker(T)}$. We will now show $S$ is a contraction, meaning $\overline{T}$ is an isometry.\newline

      Let $y\in U_Y$. Since $T$ is a $1$-quotient map, there exists $x\in U_X$ such that $T(x) = y$. Then, $\overline{T}\left(x + \ker(T)\right) = T(x) = y$, meaning $S(y) = x + \ker(T)$, and
      \begin{align*}
        \norm{S(y)} &= \norm{x + \ker(T)}\\
                    &\leq \norm{x}\\
                    &\leq 1,
      \end{align*}
      meaning $\norm{S} \leq 1$.
  \end{enumerate}
\end{proof}
\begin{proposition}
  Every separable Banach space is isometrically isomorphic to a quotient of $\ell_1$.
\end{proposition}
\begin{proof}
  Let $X$ be a separable Banach space. Since $X$ is separable, so too is $S_{X}$. Let $\left(z_n\right)_n$ be norm-dense in $S_X$, and define
  \begin{align*}
    T: \ell_1\rightarrow X\\
    \left(\lambda_n\right)_n \rightarrow \sum_{n=1}^{\infty}\lambda_nz_n.
  \end{align*}
  This series converges absolutely:
  \begin{align*}
    \sum_{n=1}^{\infty}\norm{\lambda_nz_n} &= \sum_{n=1}^{\infty}\left\vert \lambda_n \right\vert\\
                                           &<\infty,
  \end{align*}
  so this series converges in $X$. We can also see that $T$ is linear; additionally, $T$ is a contraction:
  \begin{align*}
    \norm{T\left(\left(\lambda_n\right)_n\right)} &= \norm{\sum_{n=1}^{\infty}\lambda_nz_n}\\
                                                  &= \lim_{N\rightarrow\infty}\norm{\sum_{n=1}^{N}\lambda_nz_n}\\
                                                  &\leq \lim_{N\rightarrow\infty}\sum_{n=1}^{N}\norm{\lambda_nz_n}\\
                                                  &= \lim_{N\rightarrow\infty}\sum_{n=1}^{N}\left\vert \lambda_n \right\vert\\
                                                  &= \norm{\left(\lambda_n\right)_n}.
  \end{align*}
  Thus, $T\left(U_{\ell_1}\right) \subseteq U_X$. To show that $T\left(U_{\ell}\right) = U_X$, we will use the following fact (which follows from the density of $z_n$).
  \begin{fact}
    For $\delta > 0$ and $x\neq 0$ in $X$, and $k\in \N$, there exists $n > k$ such that
    \begin{align*}
      \norm{\frac{x}{\norm{x}} - z_n} &< \frac{\delta}{\norm{x}}\\
      \norm{x - \left(\norm{x}\right)z_n} &< \delta
    \end{align*}
  \end{fact}
  Let $x\in U_X$ with $x \neq 0$, and let $\ve > 0$. Find $n_1$ such that
  \begin{align*}
    \norm{x - \left(\norm{x}\right)z_{n_1}} < \frac{\ve}{2},
  \end{align*}
  and set $\lambda_{n_1} = \norm{x}$.\newline

  We find $n_{2}$ with $n_2 > n_1$ and
  \begin{align*}
    \norm{\left(x - \lambda_{n_1}z_{n_1}\right) - \left(\norm{x-\lambda_{n_1}z_{n_1}}\right)z_{n_2}} < \frac{\ve}{2^2},
  \end{align*}
  and set $\lambda_{n_2} = \norm{x - \lambda_{n_1}z_{n_1}}$. We have
  \begin{align*}
    \norm{x - \left(\lambda_{n_1}z_{n_1} + \lambda_{n_2}z_{n_2}\right)} < \frac{\ve}{2^2},
  \end{align*}
  and $\lambda_{n_2} < \frac{\ve}{2}$.\newline

  Inductively, we obtain the subsequence $\left(z_{n_k}\right)_k$ in $z_n$ and a sequence of scalars $\left(\lambda_{n_k}\right)_{k}$ such that
  \begin{align*}
    \norm{x - \sum_{j=1}^{k}\lambda_{n_j}z_{n_j}} < \frac{\ve}{2^k}
  \end{align*}
  and
  \begin{align*}
    \norm{\lambda_{n_k}} < \frac{\ve}{2^{k-1}}.
  \end{align*}
  Let $\lambda = \left(\lambda_{1},\lambda_2,\dots\right)$ with $\lambda_{i} = 0$ for $i\notin \set{n_1,n_2,\dots}$. We can see that
  \begin{align*}
    \norm{\lambda_{n_1}} &= \norm{\lambda_{n_1} + \sum_{k=2}^{\infty}\lambda_{n_k}}\\
                         &\leq \norm{x} + \sum_{k=2}^{\infty}\frac{\ve}{2^{k-1}}\\
                         &= \norm{x} + \ve.
  \end{align*}
  We choose $\ve$ such that $\norm{x} + \ve < 1$, meaning $\lambda \in U_{\ell_1}$.\newline

  We can also see that $\sum_{j=1}^{\infty}\lambda_{n_j}z_{n_j} = x$, meaning $T$ is a $1$-quotient map.
\end{proof}
\section{Pillars of Functional Analysis}%
The five main theorems of functional analysis are:
\begin{itemize}
  \item Baire Category Theorem;
  \item Open Mapping Theorem (and Bounded Inverse Theorem);
  \item Closed Graph Theorem;
  \item Uniform Boundedness Principle;
  \item and the Hahn Banach Theorems:
    \begin{itemize}
      \item Hahn--Banach--Minkowski Theorem;
      \item Hahn--Banach Extension Theorem;
      \item Hahn--Banach Separation Theorem.
    \end{itemize}
\end{itemize}
These theorems will appear time and again as we work through the fundamentals of functional analysis.
\subsection{Baire Category Theorem}%
\begin{definition}[Baire Space]
  Let $\set{A_n}_{n\geq 1}$ be a countable collection of open, dense subsets of a topological space $X$. We say $X$ is a Baire space if
  \begin{align*}
    \bigcap_{n\geq 1}A_n
  \end{align*}
  is dense for every such collection.
\end{definition}
\begin{definition}[Meager Set]
  If $X = \bigcup_{n\geq 1}F_n$, where $\left(\overline{F_n}\right)^{\circ} = \emptyset$ for each $n$, then we say $X$ is meager.\footnote{In other words, $X$ is meager if $X$ is a countable union of nowhere dense subsets.}
\end{definition}
\begin{proposition}[Meager Spaces]
  If $X$ is a Baire space, then $X$ is nonmeager.
\end{proposition}
\begin{proof}
  Suppose toward contradiction that $X = \bigcup_{n\geq 1}F_n$, with $F_n$ all nowhere dense. Then,
  \begin{align*}
    X &= \bigcup_{n\geq 1}C_n,
  \end{align*}
  where $C_n = \overline{F_n}$ are closed with $C_n^{\circ} = \emptyset$.\newline

  Let $A_n = C_n^{c}$. Then, $A_n$ is open for all $n$, and $\overline{A_n} = \overline{C_n^{c}} = \left(C_n^{c}\right)^{\circ} = X$, meaning $A_n$ are all open and dense.\newline

  Since $X$ is a Baire space, we know that $\bigcap_{n\geq 1}A_n $ is dense. However, we also have
  \begin{align*}
    \emptyset &= X^{c}\\
              &= \left(\bigcup_{n\geq 1}C_n\right)^{c}\\
              &= \bigcap_{n\geq 1}C_n^{c}\\
              &= \bigcap_{n\geq 1}A_n.
  \end{align*}
\end{proof}
\begin{theorem}[Baire Category Theorem]
  If $(X,d)$ is a complete metric space, then $X$ is a Baire space.
\end{theorem}
\begin{proof}
  Let $\set{A_n}_{n\geq 1}$ be a collection of open dense subsets of $X$. Let $U_0$ be any ball of radius $r > 0$, and set $B_0 = \overline{U_0}$. Since $A_1\cap U_0$ is open and nonempty, it contains a closed ball $B_1$ with radius less than $r/2$.\newline

  Set $U_1 = B_1^{\circ}$. Similarly, we find a closed ball $B_2$ with radius less than $r/4$ such that $B_2\subseteq A_2\cap U_1$, and set $U_2 = B_2^{\circ}$.\newline

  Continuing in this manner, we find a closed ball $B_n$ with radius less than $r/2^n$ with $B_n \subseteq A_n\cap U_{n-1}$, and the chain
  \begin{align*}
    B_0\supseteq U_0\supseteq B_1\supseteq U_1\supseteq B_2\supseteq U_2\supseteq \cdots.
  \end{align*}
  Letting $\left(x_n\right)_n$ be the center of $B_n$, we can see that $x_n$ forms a Cauchy sequence in $X$, as the distance between $x_m$ and $x_n$ with $n > m$ is no more than $\frac{r}{2^{m-1}}$.\newline

  Since $X$ is complete, $\left(x_n\right)_n\rightarrow x\in X$. We claim that $x$ belongs to $\bigcap_{n\geq 1}B_n$.\newline

  Suppose toward contradiction that $x\notin B_{N}$ for some $N\in \N$. For $n\geq N$, we have $x\notin B_n$, so $d\left(x_n,x\right) \geq \dist_{B_n}(x) > 0$, which contradicts the fact that $\left(x_n\right)_n\rightarrow x$.\newline
  
  Thus, $x\in \bigcap_{n\geq 1}B_n\subseteq \bigcap_{n\geq 1}A_n$. Since $\bigcap_{n\geq 1}B_n\subseteq U_0$, we have $\left(\bigcap_{n\geq 1}A_n\right)\cap U_0 \neq \emptyset$, meaning $\bigcap_{n\geq 1}A_n$ is dense in $X$.
\end{proof}
\begin{corollary}
  Let $X$ be an infinite-dimensional Banach space. The cardinality of the Hamel basis of $X$ is uncountable.
\end{corollary}
\begin{proof}
  Suppose toward contradiction that $\set{b_k}_{k\in \N}$ is a Hamel basis for $X$. For each $n$, set $E_n = \Span\set{b_1,\dots,b_n}$. Each $E_n$ is closed, meaning $\overline{E_n} = E_n \neq X$ since $X$ is infinite-dimensional.\newline

  Additionally, $E_n^{\circ} = \emptyset$ for each $n$, meaning the $E_n$ are nowhere dense.\newline

  Since $\set{b_k}_{k\in \N}$ is a spanning set,
  \begin{align*}
    X &= \bigcup_{n\geq 1}E_n,
  \end{align*}
  implying that $X$ is meager.
\end{proof}
\begin{exercise}
  Let $X$ be a Banach space, and $Z\subseteq X$ a subspace. Is it true that $\Dim\left(Z\right) = \Dim\left(\overline{Z}\right)$?
\end{exercise}
\begin{solution}
  It is not the case that $\Dim\left(Z\right) = \Dim\left(\overline{Z}\right)$. For example, consider the subspace $c_c\subseteq \ell_{\infty}$. Then, the Hamel basis of $c_c$ consists of $e_n$, which consists of $1$ at index $n$ and zero elsewhere, implying that $\dim\left(c_c\right) = \aleph_{0}$. However, $\overline{c_c} = c_0$, and $c_0$ is an infinite-dimensional Banach space, meaning that $\dim\left(\overline{c_c}\right) = 2^{\aleph_0}\neq \aleph_{0}$.
\end{solution}
\subsection{Open Mapping Theorem}%
A surjective continuous map between topological spaces is not necessarily an open map --- however, if $X$ and $Y$ are Banach spaces, and $f:X\rightarrow Y$ is a surjective linear map. This is the Open Mapping theorem, which yields the result that a continuous linear bijection between Banach spaces always admits a bounded inverse.
  \begin{lemma}
    Let $X$ and $Y$ be Banach spaces, and suppose $T\in \mathcal{B}\left(X,Y\right)$. 
    \begin{enumerate}[(1)]
      \item If $U_Y \subseteq \overline{T\left(\delta U_X\right)}$ for some $\delta > 0$, then $U_Y\subseteq T\left(2\delta U_X\right)$.
      \item If $\delta U_Y\subseteq \overline{T\left(U_X\right)}$ for some $\delta > 0$, then $\frac{\delta}{2}U_Y\subseteq T\left(U_X\right)$.
    \end{enumerate}
  \end{lemma}
  \begin{proof}\hfill
    \begin{enumerate}[(1)]
      \item Let $y\in U_Y$. By our assumption, there exists $x_1\in \delta U_X$ such that $\norm{y - T\left(x_1\right)} < 1/2$. Additionally,
        \begin{align*}
          y - T\left(x_1\right) &\in \frac{1}{2}U_Y\\
                                &\subseteq \frac{1}{2}\overline{T\left(\delta U_X\right)}\\
                                &= \overline{T\left(\frac{\delta}{2}U_X\right)}.
        \end{align*}
        Thus, there exists $x_2\in \frac{\delta}{2}U_X$ such that $\norm{\left(y-T\left(x_1\right)\right)-T\left(x_2\right)} < \frac{1}{4}$, implying that
        \begin{align*}
          y - T\left(x_1\right) -T\left(x_2\right) &\in \frac{1}{4}U_Y\\
                                                   &\subseteq \overline{T\left(\frac{\delta}{4}U_X\right)}.
        \end{align*}
        Inductively, we have a sequence $\left(x_k\right)_k\in \frac{\delta}{2^{k-1}}U_X$ for each $k$, and
        \begin{align*}
          \norm{y - \sum_{j=1}^{k}T\left(x_j\right)} < 2^{-k}.
        \end{align*}
        We consider $\sum_{j=1}^{\infty}x_j$. Since
        \begin{align*}
          \sum_{j=1}^{\infty}\norm{x_j} &\leq \sum_{j=1}^{\infty}\frac{\delta}{2^{j-1}}\\
                                        &= 2\delta\\
                                        &< \infty,
        \end{align*}
        the series converges to $x\in X$ since $X$ is complete.\newline

        Additionally, since $\norm{x}\leq \sum_{j=1}^{\infty}\norm{x_j} \leq 2\delta$, we have $x\in 2\delta U_X$, and $T\left(x\right) = y$ by the continuity of $T$.
      \item If $\delta U_y\subseteq \overline{T\left(U_X\right)}$, then $U_Y\subseteq \frac{1}{\delta}\overline{T\left(U_X\right)}$, so $U_Y\subseteq \overline{T\left(\frac{1}{\delta}U_X\right)}$, meaning $U_Y\subseteq T\left(\frac{2}{\delta}U_X\right)$, or $\frac{\delta}{2}U_Y\subseteq T\left(U_X\right)$.
    \end{enumerate}
  \end{proof}
  \begin{theorem}[Open Mapping Theorem]
    Let $X$ and $Y$ be Banach spaces, $T\in \mathcal{B}\left(X,Y\right)$ surjective. Then, $T$ is open and thus a quotient mapping.
  \end{theorem}
  \begin{proof}
    We will show that $\delta U_Y\subseteq T\left(U_X\right)$ for some $\delta > 0$. This is enough to show that $T$ is a quotient mapping.\newline

    We can write
    \begin{align*}
      X &= n\bigcup_{n\geq 1}U_X\\
      Y &= T\left(X\right)\\
        &= \bigcup_{n\geq 1}T\left(nU_X\right)
    \end{align*}
    since $T$ is onto. Since $Y$ is nonmeager, there is an $m \geq 1$ such that $\overline{T\left(mU_X\right)}^{\circ} \neq \emptyset$. There exists $y_0\in Y$ and $\ve > 0$ such that $U_Y\left(y_0,\ve\right) \subseteq \overline{T\left(mU_X\right)}$. We claim that
    \begin{align*}
      \ve U_Y &= U_Y\left(0,\ve\right)\\
              &\subseteq T\left(mU_X\right).
    \end{align*}
    Let $z\in \ve U_Y$. Note that $y_0 + z$ and $y_0 - z$ are in $U_Y\left(y_0,\ve\right)$, and
    \begin{align*}
      2z &= \left(y_0 + z\right) - \left(y_0 - z\right)\\
         &\in \overline{T\left(mU_X\right)} - \overline{T\left(mU_X\right)}.
    \end{align*}
    We write $2z = z_1 - z_2$, with $z_1,z_2\in \overline{T\left(mU_X\right)}$. We can find sequences $\left(T\left(x_k\right)\right)_k$ and $\left(T\left(x'_k\right)\right)_k$ with $\left(T\left(x_k\right)\right)_k\rightarrow z_1$ and $\left(T\left(x_k'\right)\right)_k\rightarrow z_2$. Thus, we have
    \begin{align*}
      2z &= \lim_{k\rightarrow\infty}\left(T\left(x_k\right) - T\left(x'_k\right)\right)\\
         &= \lim_{k\rightarrow\infty}T\left(x_k -x_k'\right),
    \end{align*}
    where $\norm{x_k - x_k'} \leq 2m$. Thus, $2x\in \overline{T\left(mU_X\right)} = 2\overline{T\left(mU_X\right)}$, so $z\in \overline{T\left(mU_X\right)}$.\newline

    We now have
    \begin{align*}
      \frac{\ve}{m}U_Y\subseteq \overline{T\left(U_X\right)},
    \end{align*}
    so
    \begin{align*}
      \frac{\ve}{2m}U_Y\subseteq T\left(U_X\right).
    \end{align*}
    Setting $\delta = \frac{\ve}{2m}$, we finish the proof.
  \end{proof}
  If $T: X\rightarrow Y$ is bijective linear, then $T^{-1}:Y\rightarrow X$ is linear. If $X = Y$, we say $T$ is invertible in the unital algebra $\mathcal{L}\left(X\right)$. However, if $X$ and $Y$ are normed vector spaces, we also have to be concerned with the continuity of $T^{-1}$.
  \begin{corollary}[Bounded Inverse Theorem]
    Let $X$ and $Y$ be Banach spaces, $T: X\rightarrow Y$ is linear, bounded, and bijective. Then, $T^{-1}: Y\rightarrow X$ is also bounded.
  \end{corollary}
  \begin{proof}
    Since $T$ is surjective, $T$ is open, so $T^{-1}$ is continuous.
  \end{proof}
  \begin{example}
    Consider the normed space $Y = \left(C\left([0,1]\right),\norm{\cdot}_1\right)$. To show that $Y$ is not complete, we let $X = \left(C\left([0,1]\right),\norm{\cdot}_{u}\right)$, which we know is complete.\newline

    The identity function from $X$ to $Y$ is bijective and bounded linear since $\norm{\cdot}_{1}\leq \norm{\cdot}_u$. If $Y$ were to be complete, then it would imply that the inverse map is bounded. However, since there is no $C$ such that $\norm{\cdot}_u\leq C\norm{\cdot}_1$, it is not the case that $Y$ is complete.
  \end{example}
  \begin{definition}
    Let $X$ and $Y$ be normed spaces. A bounded linear map $T\in \mathcal{B}\left(X,Y\right)$ is called invertible if there is a bounded linear map $S\in \mathcal{B}\left(Y,X\right)$ with $T\circ S = \id_Y$ and $S\circ T = \id_X$. We write $T^{-1} = S$.
  \end{definition}
  \begin{corollary}
    Let $T\in \mathcal{B}\left(X,Y\right)$ with $X$ and $Y$ Banach spaces. The following are equivalent.
    \begin{enumerate}[(i)]
      \item $T$ is bounded below.
      \item $T$ is injective and $\Ran(T)\subseteq Y$ is closed.
      \item $T: X\rightarrow \Ran(T)$ is a bicontinuous isomorphism.
    \end{enumerate}
  \end{corollary}
  \begin{proof}
    For (i) to (ii), if $T$ is bounded below, then $\ker T = \set{0}$, so $T$ is injective. Additionally, since $T$ is bounded below, if $\left(T\left(x_n\right)\right)_n$ is a Cauchy sequence in $\Ran(T)$, then
    \begin{align*}
      C\norm{x_n - x_m} &\leq \norm{T\left(x_n - x_m\right)}\\
                        &= \norm{T\left(x_n\right) - T\left(x_m\right)},
    \end{align*}
    meaning $\left(x_n\right)_n$ is a Cauchy sequence in $X$. Since $T$ is continuous, $\left(T\left(x_n\right)\right)_n \rightarrow T(x)\in \Ran(T)$.\newline

    For (ii) to (i), since $Y$ is complete and $\Ran(T)\subseteq Y$ is closed, $\Ran(T)$ is a Banach space, so $T^{-1}:\Ran(T) \rightarrow X$ is bounded. Thus,
    \begin{align*}
      \norm{x} &= \norm{T^{-1}\left(T(x)\right)}\\
               &\leq \norm{T^{-1}}_{\text{op}}\norm{T(x)},
    \end{align*}
    meaning $\norm{T(x)}\geq \norm{T^{-1}}_{\text{op}}^{-1}\norm{x}$ for all $x\in X$.\newline

    To show that (ii) is true if and only if (iii) is true, we can see that since $T$ is bounded and $T$ is bounded below, it is the case that $T$ is a bicontinuous isomorphism.
  \end{proof}
  \begin{corollary}
    Let $X$ and $Y$ be Banach spaces, $T\in \mathcal{B}\left(X,Y\right)$. Then, $T$ is invertible if and only if $T$ is bounded below and surjective.
  \end{corollary}
  \subsubsection{Complemented Subspaces and Direct Sums}%
  For any normed vector spaces $X$ and $Y$, we can form the product $X\oplus_{p}Y$ by defining $\norm{\left(x,y\right)} = \left(\norm{x}^p + \norm{y}^p\right)^{1/p}$ for all $p\in [1,\infty)$.\newline

  A vector space $Z$ with subspaces $X$ and $Y$ is called the direct sum of $X$ and $Y$ if
  \begin{enumerate}[(a)]
    \item for all $z\in Z$, there exist $x\in X$ and $y\in Y$ such that $z = x+y$;
    \item $X\cap Y = \set{0}$.
  \end{enumerate}
  We write $Z = X\oplus Y$ for the internal direct sum.
  \begin{proposition}
    Let $\left(Z,\norm{\cdot}_Z\right)$ be a Banach space, and suppose $X$ and $Y$ are closed subspaces of $Z$ with $Z = X\oplus Y$. Then, $Z\cong X\oplus_{p}Y$ for all $p\in [1,\infty]$.
  \end{proposition}
  \begin{proof}
    Let $p = 1$. Set $\phi: X\oplus_{1}Y \rightarrow Z$ by taking $\phi\left((x,y)\right) = x + y$. Since $Z = X\oplus Y$, this is a bijection, hence an isomorphism. Additionally,
    \begin{align*}
      \norm{\phi\left(\left(x,y\right)\right)}_{Z} &= \norm{x + y}_Z\\
                                                   &\leq \norm{x}_Z + \norm{y}_Z\\
                                                   &= \norm{\left(x,y\right)}_1,
    \end{align*}
    meaning $\phi$ is bounded. Thus, $\phi^{-1}$ is also bounded, meaning $\phi$ is bicontinuous. The proof is similar for all other $p\in (1,\infty]$.
  \end{proof}
  \begin{definition}
    If $Z$ is a normed space, $X$ and $Y$ are closed subspaces of $Z$ such that $Z = X\oplus Y$, we say $Z$ is the topological internal direct sum of $X$ and $Y$.
  \end{definition}
  \begin{definition}
    Let $Z$ be a normed space, and suppose $X$ is a closed subspace of $Z$. We say $X$ is complemented in $Z$ if there is a closed $Y\subseteq Z$ with $X\oplus Y = Z$.
  \end{definition}
  Not all closed subspaces are complemented.
  \begin{proposition}
    Let $T: X\rightarrow Y$ be a bounded linear map between Banach spaces. If $Z\subseteq Y$ is a closed subspace such that $Y = \Ran(T) \oplus Z$, then $\Ran(T)$ is closed (meaning the internal direct sum is topological).
  \end{proposition}
  \begin{proof}
    Passing to the quotient
    \begin{align*}
      X/\ker(T) \rightarrow Y,~x + \ker(T) \mapsto T(x),
    \end{align*}
    we may assume that $T$ is injective. The map $S: X\oplus_{\infty}Z \rightarrow Y$, $S(x,z) = T(x) + z$ is bounded and bijective. Thus, $S$ is bounded below, so for some $C > 0$, we have
    \begin{align*}
      \norm{T(x)} &= \norm{S\left(x,0\right)}\\
                  &\geq C\norm{\left(x,0\right)}_{\infty}\\
                  &= C\norm{x},
    \end{align*}
    meaning $T$ is bounded below, and thus has closed range.
  \end{proof}
  \begin{corollary}
    If $X$ and $Y$ are Banach spaces, and $T: X\rightarrow Y$ is bounded Fredholm,\footnote{A linear map is Fredholm if both $\ker(T)$ and $\operatorname{coker}(T) $ are finite. Here, $\operatorname{coker}(T) = Y/\Ran(T)$.} then $T$ has closed range.
  \end{corollary}
  \begin{proof}
    There is a subspace $C\subseteq Y$ with $C$ linearly isomorphic to $\operatorname{coker}(T)$, and $Y = \Ran(T)\oplus C$. Since $T$ is Fredholm, $\Dim(C)$ is finite, meaning $C$ is closed. Thus, $\Ran(T)$ is closed.
  \end{proof}
  \subsection{Closed Graph Theorem}%
  \begin{definition}
    If $f: A\rightarrow B$ is a map between arbitrary sets, then the graph of $f$ is
    \begin{align*}
      \Graph(f) &= \set{\left(a,f(a)\right)\mid a\in A}\\
                &\subseteq A\times B.
    \end{align*}
  \end{definition}
  \begin{proposition}
    If $\left(X,d\right)$ and $\left(Y,\rho\right)$ are metric spaces, and $f: \left(X,d\right)\rightarrow \left(Y,\rho\right)$ is continuous, then $\Graph(f) \subseteq X\times Y$ is closed under the product topology.\footnote{The product topology is the coarsest topology on $X\times Y$ such that the projection maps $\pi_X$ and $\pi_Y$ are continuous.}
  \end{proposition}
  \begin{proof}
    Let $\left(x_n,f\left(x_n\right)\right)_n$ be a sequence in $\Graph(f)$ such that $\left(x_n,f\left(x_n\right)\right)_n\rightarrow \left(x,y\right)$ in $X\times Y$. Then, $\left(x_n\right)_n \rightarrow x$ in $X$ and $\left(f\left(x_n\right)\right)_n\rightarrow y$ in $Y$.\newline

    By the continuity of $f$, we have $\left(f\left(x_n\right)\right)_n\rightarrow f(x)$, and since limits are unique, we have $f(x) = y$. Thus,
    \begin{align*}
      \left(x,y\right) &= \left(x,f(x)\right)\\
                       &\in \Graph(f).
    \end{align*}
  \end{proof}
  Thus, we can see that the graph of any continuous function is closed in the product topology. However, the converse fails in the general case. For instance,
  \begin{align*}
    f: \R\rightarrow \R\\
    f(x) &= \begin{cases}
      \frac{1}{x} & x\neq 0\\
      0 & x = 0
    \end{cases}
  \end{align*}
  has a closed graph, but $f$ is not continuous.\newline

  However, with linear maps between Banach spaces, the converse is actually true.
  \begin{theorem}[Closed Graph Theorem]
    Let $X$ and $Y$ be Banach spaces, and let $T: X\rightarrow Y$ be a linear map. Then, $T$ is continuous if and only if $\Graph(T) \subseteq X\times Y$ is closed with respect to the product topology on $X\times Y$.
  \end{theorem}
  \begin{proof}
    The forward direction follows from the previous proposition.\newline

    Suppose $\Graph(T)\subseteq X\times Y$ is closed in the product topology. Note that the product topology coincides with the $\norm{\cdot}_1$ topology, with $\norm{\left(x,y\right)}_{1} = \norm{x} + \norm{y}$. Thus, $\left(\Graph(T),\norm{\cdot}_1\right)$ is a Banach space.\newline

    Consider the projection map $P: \Graph(T) \rightarrow X$ defined by $P\left(\left(x,T(x)\right)\right) = x$, which is bijective. We also have
    \begin{align*}
      \norm{P\left(\left(x,T(x)\right)\right)} &= \norm{x}\\
                                               &\leq \norm{x} + \norm{T(x)}\\
                                               &= \norm{\left(x,T(x)\right)}_{1},
    \end{align*}
    meaning $P$ is bounded. Thus, $P$ is bicontinuous, meaning it is bounded below, so for some constant $C$, we have
    \begin{align*}
      \norm{x} &= \norm{P\left(\left(x,T(x)\right)\right)}\\
               &\geq C\norm{\left(x,T(x)\right)}_{1}\\
               &\geq C\norm{T(x)},
    \end{align*}
    meaning $\norm{T(x)}\leq \frac{1}{C}\norm{x}$, so $T$ is bounded.
  \end{proof}
\end{document}
