\documentclass[10pt]{mypackage}

% sans serif font:
%\usepackage{cmbright}
%\usepackage{sfmath}
%\usepackage{bbold} %better blackboard bold

%serif font + different blackboard bold for serif font
\usepackage{newpxtext,eulerpx}
\renewcommand*{\mathbb}[1]{\varmathbb{#1}}
\renewcommand*{\hbar}{\hslash}
\DeclareMathOperator{\Graph}{graph}
\DeclareMathOperator{\supp}{supp}

\pagestyle{fancy} %better headers
\fancyhf{}
\rhead{Avinash Iyer}
\lhead{A Foray into Functional Analysis}

\setcounter{secnumdepth}{0}

\begin{document}
\RaggedRight
\tableofcontents
\section{Introduction}%
This is going to be part of the notes for my Honors thesis independent study, which will be focused on amenability and $C^{\ast}$-algebras. This section of notes will be focused on the essential results in functional analysis, starting from normed vector spaces, working our way up through locally convex topological vector spaces and the Krein--Milman theorem.\newline

The primary source for this section is going to be Timothy Rainone's \textit{Functional Analysis-En Route to Operator Algebras}, which has not been published yet.\newline

I do not claim any of this work to be original.
\section{Normed Vector Spaces}%
\subsection{Vector Spaces, Norms, and Basic Properties}%
All vector spaces are defined over $\C$. Most of the information here is in my Real Analysis II notes, so I'm going to skip to some of the more important content.
\begin{definition}[Vector Space]
  A vector space $V$ is a set closed under two operations
  \begin{align*}
    a: V\times V \rightarrow V,~\left(v_1,v_2\right)\mapsto v_1 + v_2\\
    m: \C\times V\rightarrow V,~\left(\lambda,v\right) \mapsto \lambda v.
  \end{align*}
  We refer to $a$ as addition, and $m$ as scalar multiplication; $(V,+)$ is an abelian ring.
\end{definition}
\begin{definition}[Norm]
  A norm is a function
  \begin{align*}
    \norm{\cdot}: V \rightarrow \R^+,~x\mapsto \norm{x}
  \end{align*}
  that satisfies the following properties:
  \begin{itemize}
    \item Positive definiteness: $\norm{v} = 0$ if and only if $v = 0_V$.
    \item Triangle inequality: $\norm{v+w} \leq \norm{v} + \norm{w}$.
    \item Absolute Homogeneity: $\norm{\lambda v} = \left\vert \lambda \right\vert\norm{v}$, for $\lambda \in \C$.
  \end{itemize}
  If a function $p: V\rightarrow \R^+$ satisfies the triangle inequality and absolute homogeneity, we say $p$ is a seminorm.
\end{definition}
We say the pair $\left(V,\norm{\cdot}\right)$ is a normed vector space.
\begin{definition}[Balls and Spheres]
  Let $X$ be a normed vector space, $x\in X$, and $\delta > 0$. Then,
  \begin{align*}
    U(x,\delta) &= \set{y\in X\mid d(x,y) < \delta}\\
    B(x,\delta) &= \set{y\in X\mid d(x,y) \leq \delta}\\
    S(x,\delta) &= \set{y\in X\mid d(x,y) = \delta}.
  \end{align*}
  For a normed vector space, we will use the following conventions for common sets:
  \begin{align*}
    U_X &= U(0,1)\\
    B_X &= B(0,1)\\
    S_X &= S(0,1)\\
    \mathbb{D} &= U_{\C}\\
    \mathbb{T} &= S_{\C}.
  \end{align*}
\end{definition}
\begin{definition}[Equivalent Norms]
  Two norms on $V$, $\norm{\cdot}_{a}$ and $\norm{\cdot}_{b}$ are said to be equivalent if there are two constants $C_1$ and $C_2$ such that
  \begin{align*}
    \norm{v}_{a} &\leq C_1\norm{v}_b\\
    \norm{v}_{b} &\leq C_2\norm{v}_a
  \end{align*}
  for all $v\in V$. We say $\norm{\cdot}_{a}\sim \norm{\cdot}_{b}$.
\end{definition}
\subsection{Examples}%
\begin{example}[Finite-Dimensional Vector Spaces]
  The vector space $\C^n$ is with the $p$-norm is denoted $\ell_{p}^{n}$, where for $p \in [1,\infty]$, the $p$-norm is defined by
  \begin{align*}
    \norm{x}_{p} &= \left(\sum_{i=1}^{n}\left\vert x_i \right\vert^p\right)^{1/p}.
  \end{align*}
  In the case with $p=2$, this gives the traditional Euclidean norm, and with $p = \infty$, this gives the $\sup$ norm:
  \begin{align*}
    \norm{x}_{\infty} &= \max_{1\leq i \leq n}\left\vert x_i \right\vert.
  \end{align*}
\end{example}
\begin{example}[A Sequence Space]
  We let $\ell_{p} = \set{\left(x_{n}\right)_n\mid x_n\in \C,\norm{x}_p < \infty}$ be the collection of sequences in $\C$ with finite $p$-norm. Here,
  \begin{align*}
    \norm{x}_p &= \left(\sum_{n=1}^{\infty}\left\vert x_n \right\vert^p\right)^{1/p}.
  \end{align*}
  In the case with $p = \infty$, this gives the sequence space $\ell_{\infty}$, which has norm
  \begin{align*}
    \norm{x}_{\infty} &= \sup_{n\in \N}\left\vert x_n \right\vert.
  \end{align*}
\end{example}
\begin{example}[A Function Space]
  We let $\ell^{\infty}\left(\Omega\right)$ denote the set of all bounded functions $f: \Omega \rightarrow \C$, equipped with the norm
  \begin{align*}
    \norm{f}_{\infty} &= \sup_{x\in \Omega}\left\vert f(x) \right\vert.
  \end{align*}
  If $\Omega = \left(\Omega,\mathcal{M},\mu\right)$ is a measure space, then we let $L^{\infty}\left(\Omega\right)$ be the space of $\mu$-a.e. equal essentially bounded measurable functions, under the norm
  \begin{align*}
    \norm{f}_{\infty} &= \esssup_{x\in \Omega}\left\vert f(x) \right\vert.
  \end{align*}
\end{example}
\subsection{Series Convergence and Completeness}%
\begin{proposition}[Criteria for Banach Spaces]
Let $X$ be a normed vector space. The following are equivalent:
\begin{enumerate}[(i)]
  \item $X$ is a Banach space.\footnote{Complete normed vector space.}
  \item If $\left(x_k\right)_k$ is a sequence of vectors such that $\sum_{k=1}^{\infty}\norm{x_k}$ converges, then $\sum_{k=1}^{\infty}x_k$ converges.
  \item If $\left(x_k\right)_k$ is a sequence in $X$ such that $\norm{x_k} < 2^{-k}$, then $\sum_{k=1}^{\infty}x_k$ converges.
\end{enumerate}
\end{proposition}
\begin{proof}
  To show (i) implies (ii), for $n > m > N$, we have
  \begin{align*}
    \norm{s_n - s_m} &= \norm{\sum_{k=m+1}^{n}x_k}\\
                     &\leq \sum_{k=m+1}^{n}\norm{x_k}\\
                     &< \epsilon,
  \end{align*}
  implying that $s_n$ is Cauchy, and thus converges since $X$ is complete.\newline

  Since $\sum_{k=1}^{\infty}2^{-k}$ converges, it is clear that (ii) implies (iii).\newline

  To show (iii) implies (i), we let $\left(x_n\right)_n$ be a Cauchy sequence in $X$. We only need construct a convergent subsequence in order to show that $\left(x_n\right)_n$ converges.\newline

  Chose $n_1\in \N$ such that for $n,m\geq n_1$, $ \norm{x_m - x_n} < \frac{1}{2^2}$, and inductively define $n_j > n_{j-1}$ such that $n,m\geq n_j$ implies $\norm{x_m - x_n} < \frac{1}{2^{j+1}}$.\newline

  Let $y_1 = x_{n_1}$, $y_{j} = x_{n_j} - x_{n_{j-1}}$. Then,
  \begin{align*}
    \norm{y_j} &= \norm{x_{n_j} - x_{n_{j-1}}}\\
               &< \frac{1}{2^{j}},
  \end{align*}
  so $\sum_{j=1}^{\infty}y_j$ converges by our assumption. By telescoping, we see that $\sum_{j=1}^{k}y_j = x_{n_k}$, so $\left(x_{n_{k}}\right)_k$ converges.
\end{proof}
\subsection{Quotient Spaces}%
Let $X$ be a normed vector space. Then, for $E\subseteq X$ a subspace, there is a quotient space $X/E$ with the projection map $\pi: X\rightarrow X/E$, $x\mapsto x + E$. We want to make $X/E$ into a normed space --- in order to do this, we use the distance function:
\begin{align*}
  \dist_{E}(x) &= \inf_{y\in E}d(x,y),
\end{align*}
which is uniformly continuous. For $E$ closed, then $\dist_{E}(x) = 0$ if and only if $x\in E$.
\begin{proposition}[Quotient Space Norm]
  Let $X$ be a normed vector space, and $E\subseteq X$ a subspace. Set
  \begin{align*}
    \norm{x + E}_{X/E} &= \dist_{E}(x).
  \end{align*}
  Then,
  \begin{enumerate}[(1)]
    \item $\norm{\cdot}_{X/E}$ is a well-defined seminorm on $X/E$.
    \item If $E$ is closed, then $\norm{\cdot}_{X/E}$ is a norm on $X/E$.
    \item $\norm{x+E}_{X/E} \leq \norm{x}$ for all $x\in X$.
    \item If $E$ is closed, then $\pi: X\rightarrow X/E$ is Lipschitz.
    \item If $X$ is a Banach space and $E$ is closed, then $X/E$ is also a Banach space.
  \end{enumerate}
\end{proposition}
\begin{proof}\hfill
  \begin{enumerate}[(1)]
    \item We will show that $\norm{\cdot}_{X/E}$ is well-defined. If $x + E = x' + E$, $x'-x\in E$, so for every $y\in E$, $x'-x + y\in E$. Thus,
    \begin{align*}
      \norm{x-y} &= \norm{x'-\left(x'-x+y\right)}\\
                 &\geq \inf_{z\in E}\norm{x' - z}\\
                 &= \norm{x' + E}_{X/E}.
    \end{align*}
    Thus, $\norm{x + E}_{X/E} \geq \norm{x' + E}_{X/E}$, and vice versa.\newline

    Let $\lambda \in \C\setminus \set{0}$, and $x\in X$. Then,
    \begin{align*}
      \norm{\lambda\left(x + E\right)}_{X/E} &= \norm{\lambda x + E}_{X/E}\\
                                             &= \inf_{y\in E}\norm{\lambda x - y}\\
                                             &= |\lambda|\inf_{y\in E}\norm{x - \lambda^{-1}y}\\
                                             &= |\lambda|\inf_{y'\in E}\norm{x-y}\\
                                             &= |\lambda|\norm{x + E}_{X/E}
    \end{align*}
    Given $x,x'\in X$ and a fixed $\ve > 0$, we have
    \begin{align*}
      \norm{x + E} + \frac{\ve}{2} &> \norm{x-y}
    \end{align*}
    for some $y\in E$, and
    \begin{align*}
      \norm{x' + E} + \frac{\ve}{2} &> \norm{x'-y'}
    \end{align*}
    for some $y'\in E$. Thus,
    \begin{align*}
      \norm{\left(x+x'\right)-\left(y+y'\right)} &\leq \norm{x-y} + \norm{x' - y'}\\
                                                 &< \ve + \norm{x + E} + \norm{x' + E}.
    \end{align*}
    Since $y + y'\in E$, we have
    \begin{align*}
      \norm{\left(x+E\right) + \left(x' + E\right)}_{X/E} &= \norm{x + x' + E}_{X/E}\\
                                                    &\leq \norm{\left(x+x'\right) - \left(y+y'\right)}\\
                                                    &< \ve + \norm{x + E}_{X/E} + \norm{x' + E}_{X/E},
    \end{align*}
    meaning
    \begin{align*}
      \norm{\left(x+E\right) + \left(x' + E\right)} \leq \norm{x + E} + \norm{x' + E}.
    \end{align*}
  \item If $E$ is closed, and $\norm{x + E} = 0$, then $x\in E$ so $x + E = 0_{X/E}$.
  \item For $x\in X$,
    \begin{align*}
      \norm{x + E}_{X/E} &= \inf_{y\in E}\norm{x-y}\\
                         &\leq \norm{x}.
    \end{align*}
  \item We have
    \begin{align*}
      \norm{\left(x+E\right) - \left(x' + E\right)}_{X/E} &= \norm{x-x' + E}_{X/E}\\
                                                          &\leq \norm{x-x'}.
    \end{align*}
  \item Let $X$ be complete and $E\subseteq X$ be closed. Let $\left(x_k + E\right)_k$ be a sequence in $X/E$ with $\norm{x_k + E} < 2^{-k}$. We want to show that $\sum_{k=1}^{\infty}\left(x_k + E\right)$ converges.\newline

    For each $k$, since $\norm{x_k + E} < 2^{-k}$, there exists $y_k\in E$ such that $\norm{x_k - y_k} < 2^{-k}$. Since $X$ is complete, $\sum_{k=1}^{\infty}x_k - y_k$ converges.\newline

    Let $\left(\sum_{k=1}^{n}x_k - y_k\right)_n \rightarrow x$ in $X$. Applying the canonical projection map, $\pi$, to both sides, we get
    \begin{align*}
      \sum_{k=1}^{n}\left(x_k + E\right) &= \sum_{k=1}^{n}\pi\left(x_k\right)\\
                                         &= \pi\left(\sum_{k=1}^{n}\left(x_k - y_k\right)\right)\\
                                         &\rightarrow \pi(x),
    \end{align*}
    implying that $\sum_{k=1}^{\infty}\left(x_k + E\right)$ converges.
  \end{enumerate}
\end{proof}
\begin{exercise}
  Consider $\ell_{\infty}$ and its closed subspace $c_0$. If $\pi: \ell_{\infty}\rightarrow \ell_{\infty}/c_0$ denotes the canonical quotient map, with $\left(z_k\right)_k\in \ell_{\infty}$, show that
  \begin{align*}
    \norm{\left(z_k\right)_k + c_0} &= \limsup_{k\rightarrow\infty}\left\vert z_k \right\vert
  \end{align*}
\end{exercise}
\begin{solution}
  Let $z = \left(z_k\right)_k\in \ell_{\infty}$. We define the distance
  \begin{align*}
    \dist_{c_0}(z) &= \inf_{t\in c_0}\left\vert z_k - t_k \right\vert.
  \end{align*}
  Let $w\in c_c$ be defined by
  \begin{align*}
    w &= \left(z_1,z_2,\dots,z_{n-1},0,0,\dots\right).
  \end{align*}
  Then,
  \begin{align*}
    \norm{z-w}_{\infty} &= \sup_{k\in \N}\left\vert z_k - w_k \right\vert\\
                        &= \sup_{k\geq n}\left\vert z_k - w_k \right\vert,
  \end{align*}
  meaning that
  \begin{align*}
    \dist_{c_c}(z) &\leq \sup_{k\geq n}\left\vert z_k \right\vert.
  \end{align*}
  Since $c_0 \supseteq c_c$, we have
  \begin{align*}
    \dist_{c_0}(z) &\leq \dist_{c_c}(z)\\
                   &\leq \inf_{n\geq 1}\left(\sup_{k\geq n}\left\vert z_k \right\vert\right)\\
                   &= \limsup_{k\rightarrow\infty}\left\vert z_k \right\vert.
  \end{align*}
  Now, we show that $\limsup_{k\rightarrow\infty}\left\vert z_k \right\vert \leq \dist_{c_c}\left(z\right)$. Given $\ve > 0$, there exists $w\in c_c$ such that
  \begin{align*}
    \norm{z-w} < \dist_{c_c}(z) + \ve.
  \end{align*}
  Additionally, for $w$ that terminates at $n-1$ (i.e., is equal to $0$ for all $k\geq n$), we have
  \begin{align*}
    \sup_{k\geq n}\left\vert z_k-w_k \right\vert &\leq \sup_{k\in \N}\left\vert z_k - w_k \right\vert,
  \end{align*}
  meaning
  \begin{align*}
    \limsup_{k\rightarrow\infty}\left\vert z_k \right\vert &= \inf_{n\geq 1}\left(\sup_{k\geq n}\left\vert z_k \right\vert\right)\\
                                                         &\leq \sup_{k\geq n}\left\vert z_k - w_k \right\vert\\
                                                         &\leq \sup_{k\in \N}\left\vert z_k - w_k \right\vert\\
                                                         &= \norm{z-w}\\
                                                         &< \dist_{c_c}(z) + \ve,
  \end{align*}
  implying that
  \begin{align*}
    \limsup_{k\rightarrow\infty}\left\vert z_k \right\vert &= \dist_{c_c}(z).
  \end{align*}
  For $\ve > 0$, let $w\in c_0$ be such that
  \begin{align*}
    \norm{z-w} &< \dist_{c_0}\left(z\right) + \ve/2.
  \end{align*}
  Additionally, let $\lambda \in c_c$ such that $\norm{\lambda - w} < \ve/2$. Then, we have
  \begin{align*}
    \dist_{c_0}\left(z\right) + \ve &> \norm{z-\lambda} + \norm{\lambda - w}\\
                                    &\geq \dist_{c_c}\left(z\right) + \ve/2\\
                                    &\geq \limsup_{k\rightarrow\infty}\left\vert z_k \right\vert.
  \end{align*}
  Thus, $\limsup_{k\rightarrow\infty}\left\vert z_k \right\vert \leq \dist_{c_0}\left(z\right)$, meaning $\limsup_{k\rightarrow\infty}\left\vert z_k \right\vert = \dist_{c_0}\left(z\right)$.
\end{solution}
\subsection{Bounded Linear Operators}%
\begin{definition}[Continuous Functions]
  A function $f: \left(X, d_X\right)\rightarrow \left(Y,d_Y\right)$ is called Lipschitz if there is a constant $C>0$ such that
  \begin{align*}
    d_Y\left(f(x),f(x')\right) \leq Cd_x\left(x,x'\right)
  \end{align*}
  for all $x,x'\in X$.\newline

  If $C \leq 1$, a Lipschitz map is known as a contraction.\newline

  If
  \begin{align*}
    d_Y\left(f(x),f\left(x'\right)\right) = d_X\left(x,x'\right)
  \end{align*}
  for all $x,x'\in X$, then $f$ is known as an isometry.
\end{definition}
\begin{proposition}[Categorization of Continuous Linear Maps]
  Let $X$ and $Y$ be normed vector spaces, and let $T: X\rightarrow Y$ be a linear map. The following are equivalent:
  \begin{enumerate}[(i)]
    \item $T$ is continuous at $0$.
    \item $T$ is continuous.
    \item $T$ is uniformly continuous.
    \item $T$ is Lipschitz.
    \item There exists a constant $C > 0$ such that $\norm{T(x)}\leq C\norm{x}$ for all $x\in X$.
  \end{enumerate}
\end{proposition}
\begin{definition}[Bounded Linear Operator]
  Let $X$ and $Y$ be normed vector spaces, and let $T: X\rightarrow Y$ be a linear map.
  \begin{enumerate}[(1)]
    \item $T$ is bounded if $T\left(B_X\right)$ is bounded in $Y$. Equivalently, $T$ is bounded if and only if
      \begin{align*}
        \sup_{x\in B_X}\norm{T(x)} < \infty,
      \end{align*}
      or that $\exists r > 0$ such that $T\left(B_X\right) \subseteq B_Y\left(0,r\right)$.
    \item The operator norm of $T$ is the value
      \begin{align*}
        \norm{T}_{\text{op}} &= \sup_{x\in B_X}\norm{T(x)}.
      \end{align*}
  \end{enumerate}
\end{definition}
\begin{lemma}
  Let $T: X\rightarrow Y$ be a linear map between normed vector spaces. Then,
  \begin{align*}
    \norm{T}_{\text{op}} &= \sup_{x\in S_X}\norm{T(x)}
    \intertext{and for all $x\in X$,}
    \norm{T(x)} \leq \norm{T}_{\text{op}}\norm{x}.
  \end{align*}
\end{lemma}
\begin{lemma}
  Let $T: X\rightarrow Y$ be a bounded linear map between normed vector spaces. Then, for any $x\in X$ and $r > 0$,
  \begin{align*}
    r\norm{T}_{\text{op}}\leq \sup_{y\in B\left(x,r\right)}\norm{T(y)}
  \end{align*}
\end{lemma}
\begin{proof}
  Let $C = \sup_{y\in B\left(x,r\right)}\norm{T(y)}$. If $z\in B\left(0,r\right)$, then $z+x,z-x\in B(x,r)$, meaning
  \begin{align*}
    2T\left(z\right) &= T\left(z+x\right) + T\left(z-x\right),
  \end{align*}
  so by the triangle inequality, we get
  \begin{align*}
    2\norm{T(z)} &\leq \norm{T(z+x)} + \norm{T(z-x)}\\
                 &\leq 2\max\set{\norm{T(z+x)},\norm{T\left(z-x\right)}}\\
                 &\leq 2C.
  \end{align*}
  Thus,
  \begin{align*}
    \norm{T(z)} \leq \sup_{y\in B\left(x,r\right)}\norm{T(y)},
  \end{align*}
  meaning
  \begin{align*}
    r\norm{T}_{\text{op}} \leq \sup_{y\in B\left(x,r\right)}\norm{T(y)}.
  \end{align*}
\end{proof}
\begin{remark}
For a linear map $T: X\rightarrow Y$, the following are equivalent:
\begin{enumerate}[(1)]
  \item $T$ is continuous.
  \item $T$ is bounded.
  \item $\norm{T}_{\text{op}} < \infty$.
\end{enumerate}
\end{remark}
\begin{definition}
  Let $X$ and $Y$ be normed spaces, $T: X\rightarrow Y$ a linear map.
  \begin{enumerate}[(1)]
    \item $T$ is bounded below if there exists $C_2$ such that $\norm{T(x)}\geq C_2\norm{x}$ for all $x\in X$.
    \item $T$ is bicontinuous if $T$ is bounded and bounded below.
      \begin{align*}
        C_2\norm{x} \leq \norm{T(x)}\leq C_1\norm{x}
      \end{align*}
    \item $T$ is a bicontinuous isomorphism if $T$ is bijective, linear, and bicontinuous. We say $X$ and $Y$ are bicontinuously isomorphic.
    \item We say $T$ is an isometric isomorphism if $T$ is bijective, linear, and an isometry.
  \end{enumerate}
\end{definition}
\begin{example}
  Let $\rho$ be the continuous surjective wrapping function $\rho: [0,2\pi]\rightarrow \mathbb{T}$, $\rho(t) = e^{it}$. There is an induced isometry
  \begin{align*}
    T_{\rho}: C\left(\mathbb{T}\right) \rightarrow C\left([0,2\pi]\right),
  \end{align*}
  defined by $T_{\rho}\left(f\right)(t) = f\circ \rho\left(t\right) = f\left(e^{it}\right)$.\newline

  The range of $T_{\rho}$ is $C= \set{G\in C\left([0,2\pi]\right)\mid g(0) = g(2\pi)}$, which means that $C\left(\mathbb{T}\right) $ and $ C$ are isometrically isomorphic Banach spaces.
\end{example}
\begin{proposition}
  Let $X$ and $Y$ be normed spaces, and $T:X\rightarrow Y$ be a linear map. The following are equivalent.
  \begin{enumerate}[(i)]
    \item $T$ is bicontinuous.
    \item $T: X\rightarrow \ran(T)$ is a linear isomorphism and homeomorphism.
  \end{enumerate}
\end{proposition}
\begin{proof}
  Let $T$ be bicontinuous. Then, $T$ is linear, injective, and surjective onto $\ran(T)$. Since $T$ is continuous, $T$ is bounded. Let $S: \ran(T) \rightarrow X$ be defined by $S\left(T(x)\right) = x$. We can see that $S$ is well-defined, since $T: X\rightarrow \ran(T)$ is surjective, and so has a left inverse. Similarly, since $\norm{S(T(x))} = \norm{x} \leq \frac{1}{C_2}\norm{T(x)}$, $S$ is continuous.\newline

  Let $S: \ran(T) \rightarrow X$ be defined by $S(T(x)) = x$. Since $T$ is continuous, it is bounded, so
  \begin{align*}
    \norm{T(x)}\leq \norm{T}_{\text{op}}\norm{x}.
  \end{align*}
  Since $S$ is bounded,
  \begin{align*}
    \norm{x} &= \norm{S(T(x))}\\
             &= \norm{S}_{\text{op}}\norm{T(x)},
  \end{align*}
  so $\frac{1}{\norm{S}_{\text{op}}}\norm{x} \leq \norm{T(x)}$.
\end{proof}
\begin{corollary}
  Let $X$ be a vector space with $\norm{\cdot}$ and $\norm{\cdot}'$ two norms. The following are equivalent:
  \begin{enumerate}[(i)]
    \item The norms $\norm{\cdot}$ and $\norm{\cdot}'$ are equivalent.
    \item The map $\id_{X}:\left(X,\norm{\cdot}\right)\rightarrow \left(X,\norm{\cdot}'\right)$.
  \end{enumerate}
\end{corollary}
\begin{proposition}[Properties of Bounded Linear Operators]
  Let $X,Y,Z$ be normed spaces, $T: X\rightarrow Y$, $S: X\rightarrow Y$, and $R:Y\rightarrow Z$ be linear maps.
  \begin{enumerate}[(1)]
    \item $\norm{\alpha T}_{\text{op}}= \left\vert \alpha \right\vert\norm{T}_{\text{op}}$
    \item $\norm{T + S}_{\text{op}}\leq \norm{T}_{\text{op}} + \norm{S}_{\text{op}}$
    \item $\norm{T}_{\text{op}}  = 0$ if and only if $T = 0$
    \item $\norm{R\circ T}_{\text{op}}\leq \norm{R}_{\text{op}}\norm{T}_{\text{op}}$
    \item $\norm{\id_{X}}_{\text{op}} = 1$
    \item If $E\subseteq X$ is a subspace, then $\norm{T|_{E}}_{\text{op}}\leq \norm{T}_{\text{op}}$
  \end{enumerate}
\end{proposition}
\begin{proof}
  We will prove (4) here. For $x\in B_{X}$, we have
  \begin{align*}
    \norm{R\circ T(x)} &= \norm{R\left(T(x)\right)}\\
                       &\leq \norm{R}_{\text{op}}\norm{T(x)}\\
                       &\leq \norm{R}_{\text{op}}\norm{T}_{\text{op}}.
  \end{align*}
  Taking the supremum, we obtain $\norm{R\circ T}_{\text{op}}\leq \norm{R}_{\text{op}}\norm{T}_{\text{op}}$.
\end{proof}
\begin{recall}
  $\mathcal{L}(X,Y)$ is the set of all linear operators with domain $X$ and codomain $Y$.
\end{recall}
\begin{proposition}
  Let $X$ and $Y$ be normed spaces.
  \begin{enumerate}[(1)]
    \item The collection $\mathcal{B}(X,Y) = \set{T\in \mathcal{L}\left(X,Y\right)\mid \norm{T}_{\text{op}} < \infty}$ equipped with the operator norm is a normed space known as the space of bounded linear operators between $X$ and $Y$.
    \item If $Y$ is a Banach space, then $\mathcal{B}\left(X,Y\right)$ is a Banach space.
    \item The continuous dual space, $X^{\ast} = \mathcal{B}\left(X,\C\right)$ is a Banach space.
  \end{enumerate}
\end{proposition}
\begin{proof}
  We will prove (2). Let $\left(T_n\right)_n$ be Cauchy under $\norm{\cdot}_{\text{op}}$. Since Cauchy sequences are bounded, there is some $C > 0$ such that $\norm{T_n}_{\text{op}}\leq C$ for all $n\geq 1$. For $x\in X$,
  \begin{align*}
    \norm{T_n(x) - T_m(x)} \leq \norm{T_n - T_m}_{\text{op}}\norm{x},
  \end{align*}
  meaning $\left(T_n(x)\right)_{n}$ is Cauchy in $Y$. Since $Y$ is complete, we define
  \begin{align*}
    T(x) &= \lim_{n\rightarrow\infty}T_n(x)
  \end{align*}
  in $Y$. If $x\in B_X$, we have
  \begin{align*}
    \norm{T(x)} &= \norm{\lim_{n\rightarrow\infty}T_n(x)}\\
                &= \lim_{n\rightarrow\infty}\norm{T_n(x)}\\
                &\leq \limsup_{n\rightarrow\infty}\norm{T_n(x)}\\
                &\leq C\norm{x},
  \end{align*}
  meaning $\norm{T}_{\text{op}} \leq C$.\newline

  Let $\ve > 0$, and $N\in \N$ large such that $n,m\geq N$, $\norm{T_n - T_m}_{\text{op}} \leq \ve$. For $x\in B_X$,
  \begin{align*}
    \norm{T_n(x)-T(x)} &= \lim_{m\rightarrow\infty}\norm{T_n(x)-T_m(x)}\\
                      &\leq \limsup_{m\rightarrow\infty}\norm{T_n - T_m}_{\text{op}}\norm{x}\\
                      &< \ve.
  \end{align*}
  Thus, $\norm{T - T_n}_{\text{op}} < \ve$ for all $n\geq N$.
\end{proof}
\begin{definition}[Algebras]
  Let $A$ be an algebra over $\C$.
  \begin{enumerate}[(1)]
    \item If $A$ admits a norm $\norm{\cdot}$ satisfying $\norm{ab} \leq \norm{a}\norm{b}$, then $A$ is a normed algebra. If $A$ is unital, then $\norm{1_A} = 1$.
    \item If $A$ is complete with respect to its norm, then $A$ is called a Banach algebra, and if $A$ is unital, then $A$ is a unital Banach algebra.
  \end{enumerate}
\end{definition}
\begin{lemma}
  In a normed algebra $A$, the map $\cdot: A\times A \rightarrow A,(a,b)\mapsto ab$ is continuous.
\end{lemma}
\begin{proposition}
  Let $X$ be a normed space. The set of bounded operators $\mathcal{B}\left(X,X\right) = \mathcal{B}(X)$ is a unital normed algebra. Moreover, if $X$ is a Banach space, then $\mathcal{B}\left(X\right)$ is a Banach algebra.
\end{proposition}
\begin{proposition}
  Let $A$ be a unital Banach algebra, $a\in A$. The series
  \begin{align*}
    \exp(a) &= \sum_{n=0}^{\infty}\frac{a^n}{n!}
  \end{align*}
  converges absolutely in $A$. We call $\exp(a) $ the exponential of $a$.
  \begin{enumerate}[(1)]
    \item $\exp(0) = 1_A$
    \item If $A$ is commutative, then $\exp(a+b) = \exp(a)\exp(b)$.
    \item We have $\exp(a)\in \text{GL}(A)$ with $\exp(a)^{-1} = \exp(-a)$.
    \item $\norm{\exp(a)}\leq \exp\left(\norm{a}\right)$.
  \end{enumerate}
\end{proposition}
\subsection{Quotient Maps}%
\begin{definition}
  A map $f: X\rightarrow Y$ is called open if $U\subseteq X$ is open implies $f(U)\subseteq Y$ is open.
\end{definition}
\begin{proposition}
  Let $X$ and $Y$ be normed spaces, $T: X\rightarrow Y$ a linear map. The following are equivalent:
  \begin{enumerate}[(i)]
    \item $T$ is surjective and open.
    \item $T\left(U_X\right)\subseteq Y$ is open.
    \item There exists $\delta > 0$ such that $\delta U_Y \subseteq T\left(U_X\right)$.
    \item There exists $\delta$ such that $\delta B_Y \subseteq T\left(B_X\right)$.
    \item There exists $M > 0$ such that for all $y\in Y$, there exists $x\in X$ with $T(x) = y$ and $\norm{x} \leq M\norm{y}$.
  \end{enumerate}
\end{proposition}
\begin{proof}\hfill
  To see (i) implies (ii), if $T$ is surjective and open, then it is clear that $T\left(U_X\right)$, which is the image of an open set, is open.\newline

  To see (ii) implies (iii), if $T\left(U_X\right)$ is open, we have $0_Y\in T\left(U_X\right)$, so there is some $\delta$ such that $U\left(0,\delta\right) \subseteq T\left(U_X\right)$, meaning $\delta U_{Y} \subseteq T\left(U_X\right)$.\newline

  Assuming (iii), we see that $\frac{\delta}{2}B_Y \subseteq \delta U_Y \subseteq T\left(U_X\right)\subseteq T\left(B_X\right)$.\newline

  To see (iv) implies (v), let $\delta$ be such that $\delta B_Y\subseteq T\left(B_X\right)$, and set $M = \frac{1}{\delta}$. Note that for $y\in Y,y\neq 0$, $\frac{\delta}{\norm{y}}y\in \delta B_Y$, meaning $\frac{\delta}{\norm{y}} y = T(x)$ for some $x\in B_X$, implying that $T\left(\frac{\norm{y}}{\delta}x\right) = y$. Finally, since $x\in B_X$, $\frac{\norm{y}}{\delta}\norm{x} \leq \frac{1}{\delta}\norm{y} = M\norm{y}$.\newline

  To see (v) implies (i), we can see that $T$ is surjective by the assumption. Let $U\subseteq X$ be open, $y_0\in T(U)$. Then, there exists $x_0$ such that $T\left(x_0\right) = y_0$, and $\delta > 0$ such that $U\left(x_0,\delta\right)\subseteq U$. Note that $U\left(x_0,\delta\right) = x_0 + \delta U_X$, so $x_0 + \delta U_X \subseteq U$. Applying $T$, we get $T\left(x_0 + \delta U_X\right)\subseteq T(U)$, or $y_0 + \delta T\left(U_X\right)\subseteq T(U)$. By assumption, since given $y\in U_Y$, there exists $x\in X$ such that $\norm{x} \leq M\norm{y}$, meaning $\norm{x}\leq M$, we have $U_Y\subseteq T\left(MU_X\right)$. Thus, $\frac{1}{M}U_Y\subseteq T\left(U_X\right)$, meaning $y_0 + \frac{\delta}{M}U_Y\subseteq y_0\delta T\left(U_X\right)\subseteq T(U)$, so $U_Y\left(y_0,\frac{\delta}{M}\right)\subseteq T(U)$.
\end{proof}
\begin{definition}
  Let $X$ and $Y$ be normed vector spaces.
  \begin{enumerate}[(1)]
    \item A bounded linear map $T: X\rightarrow Y$ that is surjective and open is known as a quotient map.
    \item If $T\left(U_X\right) = U_Y$, then $T$ is called a $1$-quotient map.
  \end{enumerate}
\end{definition}
\begin{exercise}
  If $T\left(B_X\right) = B_Y$, show that $T\left(U_X\right) = U_Y$.
\end{exercise}
\begin{solution}
  Since $T\left(B_X\right) = B_Y$, it is the case that $\left(T\left(B_X\right)\right)^{\circ} = B_Y^{\circ}$. Since $T$ is an open map, $T$ is continuous, meaning $\left(T\left(B_X\right)\right)^{\circ} = T\left(B_X^{\circ}\right)$. Thus, $T\left(U_X\right) = U_Y$.
\end{solution}
\begin{proposition}
  Let $X$ and $Y$ be normed vector spaces with $T: X\rightarrow Y$ a quotient map. If $X$ is a Banach space, then $Y$ is a Banach space.
\end{proposition}
\begin{proof}
  We will show that $Y$ is complete by showing that an absolutely convergent series converges.\newline

  Let $\left(y_k\right)_k$ be a sequence in $Y$ with $\sum_{k=1}^{\infty}\norm{y_k} < \infty$. Since $T$ is a quotient map, there is a universal $M > 0$ such that for all $k$, there is $x_k\in X$ such that $T\left(x_k\right) = y_k$ and $\norm{x_k} \leq M\norm{y_k}$. Thus,
  \begin{align*}
    \sum_{k=1}^{\infty} &\leq M\sum_{k=1}^{\infty}\norm{y_k}\\
    &< \infty.
  \end{align*}
  Since $X$ is complete, $\sum_{k=1}^{\infty}x_k$ converges. Let $\sum_{k=1}^{\infty}x_k = x$. Then, $\left(T\left(\sum_{k=1}^{n}x_k\right)\right)_n\xrightarrow{n\rightarrow\infty}T(x) $, meaning $\sum_{k=1}^{\infty}y_k = T(x)$. Thus, $\sum_{k=1}^{\infty}y_k$ converges in $Y$, so $Y$ is a Banach space.
\end{proof}
\begin{proposition}
  Let $X$ be a normed vector space, $E\subseteq X$ a closed subspace. The canonical quotient map, $\pi: X\rightarrow X/E$ is a $1$-quotient map.
\end{proposition}
\begin{proof}
  We know that $\norm{\pi\left(x\right)} \leq \norm{x}$, meaning $\pi\left(U_X\right)\subseteq U_{X/E}$.\newline

  Let $\pi(x) = x+E \subseteq U_{X/E}$. Then, $\inf_{y\in E}\norm{x-y}\leq 1$, meaning there exists some $y$ such that $\norm{x-y} < 1$, meaning $\pi\left(x-y\right) = \pi(x)$.
\end{proof}
\begin{corollary}
  If $X$ is a Banach space, $E\subseteq X$ a closed subspace, then $X/E$ is a Banach space.
\end{corollary}
\begin{corollary}
  Let $X$ be a normed vector space and $E\subseteq X$ be closed. If two of $X,E,X/E$ are complete, the third is also complete.
\end{corollary}
\begin{proof}
  We have shown that if $X$ is complete, then $E$ is necessarily complete (since $E$ is closed) and $X/E$ is complete as shown above.\newline

  Let $E$ and $X/E$ be complete. We now want to show that $X$ is complete. Let $\left(x_k\right)_k$ be Cauchy in $X$.\newline

  For each $k$, let $x_k = s_k + y_k$, where $y_k\in E$ and $s_k + E = \pi\left(x_k\right)$. Notice that, since $x_k$ is Cauchy, so too is $s_k$, as $\norm{s_k} \leq \norm{x_k}$ for all $k$. Additionally, for $m,n \geq N$, we have
  \begin{align*}
    \norm{x_{m} - x_{n}} &= \norm{s_m + y_m - \left(s_n + y_n\right)}\\
                         &\leq \norm{s_m - s_n} + \norm{y_m - y_n}\\
                         &< \ve,
  \end{align*}
  implying that $\left(y_k\right)_k$ is Cauchy in $E$. Since $X/E$ and $E$ are complete, we define $x = \lim_{k\rightarrow\infty}s_k + \lim_{k\rightarrow\infty}y_k$. Finally, for $m,n\geq N$, we have
  \begin{align*}
    \norm{x - x_n} &= \lim_{m\rightarrow\infty}\norm{x_m - x_n}\\
                   &\leq \ve,
  \end{align*}
  meaning $\left(x_k\right)_k \xrightarrow{k\rightarrow\infty} x$, so $X$ is complete.
\end{proof}
\begin{proposition}
  Let $X$ and $Y$ be normed spaces, $E\subseteq X$ a closed subspace, and $T: X\rightarrow Y$ bounded linear with $E\subseteq \ker(T)$. Then, there exists a unique bounded linear map $\overline{T}: X/E\rightarrow Y$ such that $\overline{T}\circ \pi = T$. Moreover, $\overline{T}$ is injective if and only if $E = \ker(T)$ and $\norm{\overline{T}}=\norm{T}$.
\end{proposition}
\begin{proof}
  The existence and uniqueness of $\overline{T}: X/E\rightarrow Y$ such that $\overline{T}\circ \pi = T$ follows from the First Isomorphism Theorem for vector spaces, as does the fact that $\overline{T}$ is injective and only if $\ker(T) = E$.\newline

  Let $x+E\in X/E$. For $y\in E$, we have
  \begin{align*}
    \norm{\overline{T}\left(x+E\right)} &= \norm{\overline{T}\left(x-y+E\right)}\\
                                        &= \norm{T\left(x-y\right)}\\
                                        &\leq \norm{T}\norm{x-y}.
  \end{align*}
  Taking infimum over all $y\in E$, we get $\norm{\overline{T}\left(x+E\right)} \leq \norm{T}\norm{x+E}$, meaning $\norm{\overline{T}}\leq \norm{T}$. Additionally,
  \begin{align*}
    \norm{T} &= \norm{\overline{T}\circ \pi}\\
             &\leq \norm{\overline{T}}\norm{\pi}\\
             &= \norm{\overline{T}}.
  \end{align*}
\end{proof}
\begin{theorem}[First Isomorphism Theorem for Normed Vector Spaces]
  Let $X$ and $Y$ be normed vector spaces, $T\in \mathcal{B}\left(X,Y\right)$.
  \begin{enumerate}[(1)]
    \item $T$ is a quotient map if and only if $\overline{T}: X/\ker(T) \rightarrow Y$ is a bicontinuous isomorphism.
    \item $T$ is a $1$-quotient map if and only if $\overline{T}: X/\ker(T) \rightarrow Y$ is an isometric isomorphism.
  \end{enumerate}
\end{theorem}
\begin{proof}\hfill
  \begin{enumerate}[(1)]
    \item Let $\overline{T}: X/\ker(T)\rightarrow Y$ be a bicontinuous isomorphism. Since $\overline{T}$ is bicontinuous, it is a homeomorphism, meaning it is open and surjective. Since $\pi$ is a quotient map, so too is $T: \overline{T}\circ \pi$.\newline

      Suppose $T$ is a quotient map. Then, $T$ is surjective, meaning $\overline{T}$ is an isomorphism. Since $T$ is bounded below, $\overline{T}$ is also bounded. Let $\pi(x) = x + \ker(T)\in X/\ker(T)$, with $T(x) = y$. Let $M$ be such that $\norm{x}\leq M\norm{y}$. There is an $x'\in X$ with $T\left(x'\right) = y$, and $\norm{x'}\leq M\norm{y}$. Thus, $x-x'\in \ker(T)$, so $\pi(x) = \pi\left(x'\right)$, meaning
      \begin{align*}
        \norm{\overline{T}\circ \pi (x)} &= \norm{T\circ \pi\left(x'\right)}\\
                                         &= \norm{y}\\
                                         &\geq M^{-1}\norm{x'}\\
                                         &\geq M^{-1}\norm{\pi\left(x'\right)}\\
                                         &= M^{-1}\norm{\pi\left(x\right)},
      \end{align*}
      meaning $T$ is bounded below.
    \item Suppose $\overline{T}:X/\ker(T) \rightarrow Y$ is an isometric isomorphism. Then, $\overline{T}$ is a $1$-quotient map, and since $\pi$ is a $1$-quotient map, so too is $T = \overline{T}\circ \pi$.\newline

      Suppose $T$ is a $1$-quotient map. Since $T$ is surjective, $\overline{T}$ is an isomorphism. Since $T$ is a $1$-quotient map, $\norm{T} = \sup_{x\in U_X}\norm{T(x)}\leq 1$, meaning $\norm{\overline{T}}\leq \norm{T} \leq 1$. Consider $S = \left(\overline{T}\right)^{-1}: Y\rightarrow X/\ker(T)$; $S$ is also an isomorphism, so $S\circ \overline{T} = = \id_{X/\ker(T)}$. We will now show $S$ is a contraction, meaning $\overline{T}$ is an isometry.\newline

      Let $y\in U_Y$. Since $T$ is a $1$-quotient map, there exists $x\in U_X$ such that $T(x) = y$. Then, $\overline{T}\left(x + \ker(T)\right) = T(x) = y$, meaning $S(y) = x + \ker(T)$, and
      \begin{align*}
        \norm{S(y)} &= \norm{x + \ker(T)}\\
                    &\leq \norm{x}\\
                    &\leq 1,
      \end{align*}
      meaning $\norm{S} \leq 1$.
  \end{enumerate}
\end{proof}
\begin{proposition}
  Every separable Banach space is isometrically isomorphic to a quotient of $\ell_1$.
\end{proposition}
\begin{proof}
  Let $X$ be a separable Banach space. Since $X$ is separable, so too is $S_{X}$. Let $\left(z_n\right)_n$ be norm-dense in $S_X$, and define
  \begin{align*}
    T: \ell_1\rightarrow X\\
    \left(\lambda_n\right)_n \rightarrow \sum_{n=1}^{\infty}\lambda_nz_n.
  \end{align*}
  This series converges absolutely:
  \begin{align*}
    \sum_{n=1}^{\infty}\norm{\lambda_nz_n} &= \sum_{n=1}^{\infty}\left\vert \lambda_n \right\vert\\
                                           &<\infty,
  \end{align*}
  so this series converges in $X$. We can also see that $T$ is linear; additionally, $T$ is a contraction:
  \begin{align*}
    \norm{T\left(\left(\lambda_n\right)_n\right)} &= \norm{\sum_{n=1}^{\infty}\lambda_nz_n}\\
                                                  &= \lim_{N\rightarrow\infty}\norm{\sum_{n=1}^{N}\lambda_nz_n}\\
                                                  &\leq \lim_{N\rightarrow\infty}\sum_{n=1}^{N}\norm{\lambda_nz_n}\\
                                                  &= \lim_{N\rightarrow\infty}\sum_{n=1}^{N}\left\vert \lambda_n \right\vert\\
                                                  &= \norm{\left(\lambda_n\right)_n}.
  \end{align*}
  Thus, $T\left(U_{\ell_1}\right) \subseteq U_X$. To show that $T\left(U_{\ell}\right) = U_X$, we will use the following fact (which follows from the density of $z_n$).
  \begin{fact}
    For $\delta > 0$ and $x\neq 0$ in $X$, and $k\in \N$, there exists $n > k$ such that
    \begin{align*}
      \norm{\frac{x}{\norm{x}} - z_n} &< \frac{\delta}{\norm{x}}\\
      \norm{x - \left(\norm{x}\right)z_n} &< \delta
    \end{align*}
  \end{fact}
  Let $x\in U_X$ with $x \neq 0$, and let $\ve > 0$. Find $n_1$ such that
  \begin{align*}
    \norm{x - \left(\norm{x}\right)z_{n_1}} < \frac{\ve}{2},
  \end{align*}
  and set $\lambda_{n_1} = \norm{x}$.\newline

  We find $n_{2}$ with $n_2 > n_1$ and
  \begin{align*}
    \norm{\left(x - \lambda_{n_1}z_{n_1}\right) - \left(\norm{x-\lambda_{n_1}z_{n_1}}\right)z_{n_2}} < \frac{\ve}{2^2},
  \end{align*}
  and set $\lambda_{n_2} = \norm{x - \lambda_{n_1}z_{n_1}}$. We have
  \begin{align*}
    \norm{x - \left(\lambda_{n_1}z_{n_1} + \lambda_{n_2}z_{n_2}\right)} < \frac{\ve}{2^2},
  \end{align*}
  and $\lambda_{n_2} < \frac{\ve}{2}$.\newline

  Inductively, we obtain the subsequence $\left(z_{n_k}\right)_k$ in $z_n$ and a sequence of scalars $\left(\lambda_{n_k}\right)_{k}$ such that
  \begin{align*}
    \norm{x - \sum_{j=1}^{k}\lambda_{n_j}z_{n_j}} < \frac{\ve}{2^k}
  \end{align*}
  and
  \begin{align*}
    \norm{\lambda_{n_k}} < \frac{\ve}{2^{k-1}}.
  \end{align*}
  Let $\lambda = \left(\lambda_{1},\lambda_2,\dots\right)$ with $\lambda_{i} = 0$ for $i\notin \set{n_1,n_2,\dots}$. We can see that
  \begin{align*}
    \norm{\lambda_{n_1}} &= \norm{\lambda_{n_1} + \sum_{k=2}^{\infty}\lambda_{n_k}}\\
                         &\leq \norm{x} + \sum_{k=2}^{\infty}\frac{\ve}{2^{k-1}}\\
                         &= \norm{x} + \ve.
  \end{align*}
  We choose $\ve$ such that $\norm{x} + \ve < 1$, meaning $\lambda \in U_{\ell_1}$.\newline

  We can also see that $\sum_{j=1}^{\infty}\lambda_{n_j}z_{n_j} = x$, meaning $T$ is a $1$-quotient map.
\end{proof}
\section{Pillars of Functional Analysis}%
The five main theorems of functional analysis are:
\begin{itemize}
  \item Baire Category Theorem;
  \item Open Mapping Theorem (and Bounded Inverse Theorem);
  \item Closed Graph Theorem;
  \item Uniform Boundedness Principle;
  \item and the Hahn Banach Theorems:
    \begin{itemize}
      \item Hahn--Banach--Minkowski Theorem;
      \item Hahn--Banach Extension Theorem;
      \item Hahn--Banach Separation Theorem.
    \end{itemize}
\end{itemize}
These theorems will appear time and again as we work through the fundamentals of functional analysis.
\subsection{Baire Category Theorem}%
\begin{definition}[Baire Space]
  Let $\set{A_n}_{n\geq 1}$ be a countable collection of open, dense subsets of a topological space $X$. We say $X$ is a Baire space if
  \begin{align*}
    \bigcap_{n\geq 1}A_n
  \end{align*}
  is dense for every such collection.
\end{definition}
\begin{definition}[Meager Set]
  If $X = \bigcup_{n\geq 1}F_n$, where $\left(\overline{F_n}\right)^{\circ} = \emptyset$ for each $n$, then we say $X$ is meager.\footnote{In other words, $X$ is meager if $X$ is a countable union of nowhere dense subsets.}
\end{definition}
\begin{proposition}[Meager Spaces]
  If $X$ is a Baire space, then $X$ is nonmeager.
\end{proposition}
\begin{proof}
  Suppose toward contradiction that $X = \bigcup_{n\geq 1}F_n$, with $F_n$ all nowhere dense. Then,
  \begin{align*}
    X &= \bigcup_{n\geq 1}C_n,
  \end{align*}
  where $C_n = \overline{F_n}$ are closed with $C_n^{\circ} = \emptyset$.\newline

  Let $A_n = C_n^{c}$. Then, $A_n$ is open for all $n$, and $\overline{A_n} = \overline{C_n^{c}} = \left(C_n^{c}\right)^{\circ} = X$, meaning $A_n$ are all open and dense.\newline

  Since $X$ is a Baire space, we know that $\bigcap_{n\geq 1}A_n $ is dense. However, we also have
  \begin{align*}
    \emptyset &= X^{c}\\
              &= \left(\bigcup_{n\geq 1}C_n\right)^{c}\\
              &= \bigcap_{n\geq 1}C_n^{c}\\
              &= \bigcap_{n\geq 1}A_n.
  \end{align*}
\end{proof}
\begin{theorem}[Baire Category Theorem]
  If $(X,d)$ is a complete metric space, then $X$ is a Baire space.
\end{theorem}
\begin{proof}
  Let $\set{A_n}_{n\geq 1}$ be a collection of open dense subsets of $X$. Let $U_0$ be any ball of radius $r > 0$, and set $B_0 = \overline{U_0}$. Since $A_1\cap U_0$ is open and nonempty, it contains a closed ball $B_1$ with radius less than $r/2$.\newline

  Set $U_1 = B_1^{\circ}$. Similarly, we find a closed ball $B_2$ with radius less than $r/4$ such that $B_2\subseteq A_2\cap U_1$, and set $U_2 = B_2^{\circ}$.\newline

  Continuing in this manner, we find a closed ball $B_n$ with radius less than $r/2^n$ with $B_n \subseteq A_n\cap U_{n-1}$, and the chain
  \begin{align*}
    B_0\supseteq U_0\supseteq B_1\supseteq U_1\supseteq B_2\supseteq U_2\supseteq \cdots.
  \end{align*}
  Letting $\left(x_n\right)_n$ be the center of $B_n$, we can see that $x_n$ forms a Cauchy sequence in $X$, as the distance between $x_m$ and $x_n$ with $n > m$ is no more than $\frac{r}{2^{m-1}}$.\newline

  Since $X$ is complete, $\left(x_n\right)_n\rightarrow x\in X$. We claim that $x$ belongs to $\bigcap_{n\geq 1}B_n$.\newline

  Suppose toward contradiction that $x\notin B_{N}$ for some $N\in \N$. For $n\geq N$, we have $x\notin B_n$, so $d\left(x_n,x\right) \geq \dist_{B_n}(x) > 0$, which contradicts the fact that $\left(x_n\right)_n\rightarrow x$.\newline
  
  Thus, $x\in \bigcap_{n\geq 1}B_n\subseteq \bigcap_{n\geq 1}A_n$. Since $\bigcap_{n\geq 1}B_n\subseteq U_0$, we have $\left(\bigcap_{n\geq 1}A_n\right)\cap U_0 \neq \emptyset$, meaning $\bigcap_{n\geq 1}A_n$ is dense in $X$.
\end{proof}
\begin{corollary}
  Let $X$ be an infinite-dimensional Banach space. The cardinality of the Hamel basis of $X$ is uncountable.
\end{corollary}
\begin{proof}
  Suppose toward contradiction that $\set{b_k}_{k\in \N}$ is a Hamel basis for $X$. For each $n$, set $E_n = \Span\set{b_1,\dots,b_n}$. Each $E_n$ is closed, meaning $\overline{E_n} = E_n \neq X$ since $X$ is infinite-dimensional.\newline

  Additionally, $E_n^{\circ} = \emptyset$ for each $n$, meaning the $E_n$ are nowhere dense.\newline

  Since $\set{b_k}_{k\in \N}$ is a spanning set,
  \begin{align*}
    X &= \bigcup_{n\geq 1}E_n,
  \end{align*}
  implying that $X$ is meager.
\end{proof}
\begin{exercise}
  Let $X$ be a Banach space, and $Z\subseteq X$ a subspace. Is it true that $\Dim\left(Z\right) = \Dim\left(\overline{Z}\right)$?
\end{exercise}
\begin{solution}
  It is not the case that $\Dim\left(Z\right) = \Dim\left(\overline{Z}\right)$. For example, consider the subspace $c_c\subseteq \ell_{\infty}$. Then, the Hamel basis of $c_c$ consists of $e_n$, which consists of $1$ at index $n$ and zero elsewhere, implying that $\dim\left(c_c\right) = \aleph_{0}$. However, $\overline{c_c} = c_0$, and $c_0$ is an infinite-dimensional Banach space, meaning that $\dim\left(\overline{c_c}\right) = 2^{\aleph_0}\neq \aleph_{0}$.
\end{solution}
\subsection{Open Mapping Theorem}%
A surjective continuous map between topological spaces is not necessarily an open map --- however, if $X$ and $Y$ are Banach spaces, and $f:X\rightarrow Y$ is a surjective linear map. This is the Open Mapping theorem, which yields the result that a continuous linear bijection between Banach spaces always admits a bounded inverse.
  \begin{lemma}
    Let $X$ and $Y$ be Banach spaces, and suppose $T\in \mathcal{B}\left(X,Y\right)$. 
    \begin{enumerate}[(1)]
      \item If $U_Y \subseteq \overline{T\left(\delta U_X\right)}$ for some $\delta > 0$, then $U_Y\subseteq T\left(2\delta U_X\right)$.
      \item If $\delta U_Y\subseteq \overline{T\left(U_X\right)}$ for some $\delta > 0$, then $\frac{\delta}{2}U_Y\subseteq T\left(U_X\right)$.
    \end{enumerate}
  \end{lemma}
  \begin{proof}\hfill
    \begin{enumerate}[(1)]
      \item Let $y\in U_Y$. By our assumption, there exists $x_1\in \delta U_X$ such that $\norm{y - T\left(x_1\right)} < 1/2$. Additionally,
        \begin{align*}
          y - T\left(x_1\right) &\in \frac{1}{2}U_Y\\
                                &\subseteq \frac{1}{2}\overline{T\left(\delta U_X\right)}\\
                                &= \overline{T\left(\frac{\delta}{2}U_X\right)}.
        \end{align*}
        Thus, there exists $x_2\in \frac{\delta}{2}U_X$ such that $\norm{\left(y-T\left(x_1\right)\right)-T\left(x_2\right)} < \frac{1}{4}$, implying that
        \begin{align*}
          y - T\left(x_1\right) -T\left(x_2\right) &\in \frac{1}{4}U_Y\\
                                                   &\subseteq \overline{T\left(\frac{\delta}{4}U_X\right)}.
        \end{align*}
        Inductively, we have a sequence $\left(x_k\right)_k\in \frac{\delta}{2^{k-1}}U_X$ for each $k$, and
        \begin{align*}
          \norm{y - \sum_{j=1}^{k}T\left(x_j\right)} < 2^{-k}.
        \end{align*}
        We consider $\sum_{j=1}^{\infty}x_j$. Since
        \begin{align*}
          \sum_{j=1}^{\infty}\norm{x_j} &\leq \sum_{j=1}^{\infty}\frac{\delta}{2^{j-1}}\\
                                        &= 2\delta\\
                                        &< \infty,
        \end{align*}
        the series converges to $x\in X$ since $X$ is complete.\newline

        Additionally, since $\norm{x}\leq \sum_{j=1}^{\infty}\norm{x_j} \leq 2\delta$, we have $x\in 2\delta U_X$, and $T\left(x\right) = y$ by the continuity of $T$.
      \item If $\delta U_y\subseteq \overline{T\left(U_X\right)}$, then $U_Y\subseteq \frac{1}{\delta}\overline{T\left(U_X\right)}$, so $U_Y\subseteq \overline{T\left(\frac{1}{\delta}U_X\right)}$, meaning $U_Y\subseteq T\left(\frac{2}{\delta}U_X\right)$, or $\frac{\delta}{2}U_Y\subseteq T\left(U_X\right)$.
    \end{enumerate}
  \end{proof}
  \begin{theorem}[Open Mapping Theorem]
    Let $X$ and $Y$ be Banach spaces, $T\in \mathcal{B}\left(X,Y\right)$ surjective. Then, $T$ is open and thus a quotient mapping.
  \end{theorem}
  \begin{proof}
    We will show that $\delta U_Y\subseteq T\left(U_X\right)$ for some $\delta > 0$. This is enough to show that $T$ is a quotient mapping.\newline

    We can write
    \begin{align*}
      X &= n\bigcup_{n\geq 1}U_X\\
      Y &= T\left(X\right)\\
        &= \bigcup_{n\geq 1}T\left(nU_X\right)
    \end{align*}
    since $T$ is onto. Since $Y$ is nonmeager, there is an $m \geq 1$ such that $\overline{T\left(mU_X\right)}^{\circ} \neq \emptyset$. There exists $y_0\in Y$ and $\ve > 0$ such that $U_Y\left(y_0,\ve\right) \subseteq \overline{T\left(mU_X\right)}$. We claim that
    \begin{align*}
      \ve U_Y &= U_Y\left(0,\ve\right)\\
              &\subseteq T\left(mU_X\right).
    \end{align*}
    Let $z\in \ve U_Y$. Note that $y_0 + z$ and $y_0 - z$ are in $U_Y\left(y_0,\ve\right)$, and
    \begin{align*}
      2z &= \left(y_0 + z\right) - \left(y_0 - z\right)\\
         &\in \overline{T\left(mU_X\right)} - \overline{T\left(mU_X\right)}.
    \end{align*}
    We write $2z = z_1 - z_2$, with $z_1,z_2\in \overline{T\left(mU_X\right)}$. We can find sequences $\left(T\left(x_k\right)\right)_k$ and $\left(T\left(x'_k\right)\right)_k$ with $\left(T\left(x_k\right)\right)_k\rightarrow z_1$ and $\left(T\left(x_k'\right)\right)_k\rightarrow z_2$. Thus, we have
    \begin{align*}
      2z &= \lim_{k\rightarrow\infty}\left(T\left(x_k\right) - T\left(x'_k\right)\right)\\
         &= \lim_{k\rightarrow\infty}T\left(x_k -x_k'\right),
    \end{align*}
    where $\norm{x_k - x_k'} \leq 2m$. Thus, $2x\in \overline{T\left(mU_X\right)} = 2\overline{T\left(mU_X\right)}$, so $z\in \overline{T\left(mU_X\right)}$.\newline

    We now have
    \begin{align*}
      \frac{\ve}{m}U_Y\subseteq \overline{T\left(U_X\right)},
    \end{align*}
    so
    \begin{align*}
      \frac{\ve}{2m}U_Y\subseteq T\left(U_X\right).
    \end{align*}
    Setting $\delta = \frac{\ve}{2m}$, we finish the proof.
  \end{proof}
  If $T: X\rightarrow Y$ is bijective linear, then $T^{-1}:Y\rightarrow X$ is linear. If $X = Y$, we say $T$ is invertible in the unital algebra $\mathcal{L}\left(X\right)$. However, if $X$ and $Y$ are normed vector spaces, we also have to be concerned with the continuity of $T^{-1}$.
  \begin{corollary}[Bounded Inverse Theorem]
    Let $X$ and $Y$ be Banach spaces, $T: X\rightarrow Y$ is linear, bounded, and bijective. Then, $T^{-1}: Y\rightarrow X$ is also bounded.
  \end{corollary}
  \begin{proof}
    Since $T$ is surjective, $T$ is open, so $T^{-1}$ is continuous.
  \end{proof}
  \begin{example}
    Consider the normed space $Y = \left(C\left([0,1]\right),\norm{\cdot}_1\right)$. To show that $Y$ is not complete, we let $X = \left(C\left([0,1]\right),\norm{\cdot}_{u}\right)$, which we know is complete.\newline

    The identity function from $X$ to $Y$ is bijective and bounded linear since $\norm{\cdot}_{1}\leq \norm{\cdot}_u$. If $Y$ were to be complete, then it would imply that the inverse map is bounded. However, since there is no $C$ such that $\norm{\cdot}_u\leq C\norm{\cdot}_1$, it is not the case that $Y$ is complete.
  \end{example}
  \begin{definition}
    Let $X$ and $Y$ be normed spaces. A bounded linear map $T\in \mathcal{B}\left(X,Y\right)$ is called invertible if there is a bounded linear map $S\in \mathcal{B}\left(Y,X\right)$ with $T\circ S = \id_Y$ and $S\circ T = \id_X$. We write $T^{-1} = S$.
  \end{definition}
  \begin{corollary}
    Let $T\in \mathcal{B}\left(X,Y\right)$ with $X$ and $Y$ Banach spaces. The following are equivalent.
    \begin{enumerate}[(i)]
      \item $T$ is bounded below.
      \item $T$ is injective and $\Ran(T)\subseteq Y$ is closed.
      \item $T: X\rightarrow \Ran(T)$ is a bicontinuous isomorphism.
    \end{enumerate}
  \end{corollary}
  \begin{proof}
    For (i) to (ii), if $T$ is bounded below, then $\ker T = \set{0}$, so $T$ is injective. Additionally, since $T$ is bounded below, if $\left(T\left(x_n\right)\right)_n$ is a Cauchy sequence in $\Ran(T)$, then
    \begin{align*}
      C\norm{x_n - x_m} &\leq \norm{T\left(x_n - x_m\right)}\\
                        &= \norm{T\left(x_n\right) - T\left(x_m\right)},
    \end{align*}
    meaning $\left(x_n\right)_n$ is a Cauchy sequence in $X$. Since $T$ is continuous, $\left(T\left(x_n\right)\right)_n \rightarrow T(x)\in \Ran(T)$.\newline

    For (ii) to (i), since $Y$ is complete and $\Ran(T)\subseteq Y$ is closed, $\Ran(T)$ is a Banach space, so $T^{-1}:\Ran(T) \rightarrow X$ is bounded. Thus,
    \begin{align*}
      \norm{x} &= \norm{T^{-1}\left(T(x)\right)}\\
               &\leq \norm{T^{-1}}_{\text{op}}\norm{T(x)},
    \end{align*}
    meaning $\norm{T(x)}\geq \norm{T^{-1}}_{\text{op}}^{-1}\norm{x}$ for all $x\in X$.\newline

    To show that (ii) is true if and only if (iii) is true, we can see that since $T$ is bounded and $T$ is bounded below, it is the case that $T$ is a bicontinuous isomorphism.
  \end{proof}
  \begin{corollary}
    Let $X$ and $Y$ be Banach spaces, $T\in \mathcal{B}\left(X,Y\right)$. Then, $T$ is invertible if and only if $T$ is bounded below and surjective.
  \end{corollary}
  \subsubsection{Complemented Subspaces and Direct Sums}%
  For any normed vector spaces $X$ and $Y$, we can form the product $X\oplus_{p}Y$ by defining $\norm{\left(x,y\right)} = \left(\norm{x}^p + \norm{y}^p\right)^{1/p}$ for all $p\in [1,\infty)$.\newline

  A vector space $Z$ with subspaces $X$ and $Y$ is called the direct sum of $X$ and $Y$ if
  \begin{enumerate}[(a)]
    \item for all $z\in Z$, there exist $x\in X$ and $y\in Y$ such that $z = x+y$;
    \item $X\cap Y = \set{0}$.
  \end{enumerate}
  We write $Z = X\oplus Y$ for the internal direct sum.
  \begin{proposition}
    Let $\left(Z,\norm{\cdot}_Z\right)$ be a Banach space, and suppose $X$ and $Y$ are closed subspaces of $Z$ with $Z = X\oplus Y$. Then, $Z\cong X\oplus_{p}Y$ for all $p\in [1,\infty]$.
  \end{proposition}
  \begin{proof}
    Let $p = 1$. Set $\phi: X\oplus_{1}Y \rightarrow Z$ by taking $\phi\left((x,y)\right) = x + y$. Since $Z = X\oplus Y$, this is a bijection, hence an isomorphism. Additionally,
    \begin{align*}
      \norm{\phi\left(\left(x,y\right)\right)}_{Z} &= \norm{x + y}_Z\\
                                                   &\leq \norm{x}_Z + \norm{y}_Z\\
                                                   &= \norm{\left(x,y\right)}_1,
    \end{align*}
    meaning $\phi$ is bounded. Thus, $\phi^{-1}$ is also bounded, meaning $\phi$ is bicontinuous. The proof is similar for all other $p\in (1,\infty]$.
  \end{proof}
  \begin{definition}
    If $Z$ is a normed space, $X$ and $Y$ are closed subspaces of $Z$ such that $Z = X\oplus Y$, we say $Z$ is the topological internal direct sum of $X$ and $Y$.
  \end{definition}
  \begin{definition}
    Let $Z$ be a normed space, and suppose $X$ is a closed subspace of $Z$. We say $X$ is complemented in $Z$ if there is a closed $Y\subseteq Z$ with $X\oplus Y = Z$.
  \end{definition}
  Not all closed subspaces are complemented.
  \begin{proposition}
    Let $T: X\rightarrow Y$ be a bounded linear map between Banach spaces. If $Z\subseteq Y$ is a closed subspace such that $Y = \Ran(T) \oplus Z$, then $\Ran(T)$ is closed (meaning the internal direct sum is topological).
  \end{proposition}
  \begin{proof}
    Passing to the quotient
    \begin{align*}
      X/\ker(T) \rightarrow Y,~x + \ker(T) \mapsto T(x),
    \end{align*}
    we may assume that $T$ is injective. The map $S: X\oplus_{\infty}Z \rightarrow Y$, $S(x,z) = T(x) + z$ is bounded and bijective. Thus, $S$ is bounded below, so for some $C > 0$, we have
    \begin{align*}
      \norm{T(x)} &= \norm{S\left(x,0\right)}\\
                  &\geq C\norm{\left(x,0\right)}_{\infty}\\
                  &= C\norm{x},
    \end{align*}
    meaning $T$ is bounded below, and thus has closed range.
  \end{proof}
  \begin{corollary}
    If $X$ and $Y$ are Banach spaces, and $T: X\rightarrow Y$ is bounded Fredholm,\footnote{A linear map is Fredholm if both $\ker(T)$ and $\operatorname{coker}(T) $ are finite. Here, $\operatorname{coker}(T) = Y/\Ran(T)$.} then $T$ has closed range.
  \end{corollary}
  \begin{proof}
    There is a subspace $C\subseteq Y$ with $C$ linearly isomorphic to $\operatorname{coker}(T)$, and $Y = \Ran(T)\oplus C$. Since $T$ is Fredholm, $\Dim(C)$ is finite, meaning $C$ is closed. Thus, $\Ran(T)$ is closed.
  \end{proof}
  \subsection{Closed Graph Theorem}%
  \begin{definition}
    If $f: A\rightarrow B$ is a map between arbitrary sets, then the graph of $f$ is
    \begin{align*}
      \Graph(f) &= \set{\left(a,f(a)\right)\mid a\in A}\\
                &\subseteq A\times B.
    \end{align*}
  \end{definition}
  \begin{proposition}
    If $\left(X,d\right)$ and $\left(Y,\rho\right)$ are metric spaces, and $f: \left(X,d\right)\rightarrow \left(Y,\rho\right)$ is continuous, then $\Graph(f) \subseteq X\times Y$ is closed under the product topology.\footnote{The product topology is the coarsest topology on $X\times Y$ such that the projection maps $\pi_X$ and $\pi_Y$ are continuous.}
  \end{proposition}
  \begin{proof}
    Let $\left(x_n,f\left(x_n\right)\right)_n$ be a sequence in $\Graph(f)$ such that $\left(x_n,f\left(x_n\right)\right)_n\rightarrow \left(x,y\right)$ in $X\times Y$. Then, $\left(x_n\right)_n \rightarrow x$ in $X$ and $\left(f\left(x_n\right)\right)_n\rightarrow y$ in $Y$.\newline

    By the continuity of $f$, we have $\left(f\left(x_n\right)\right)_n\rightarrow f(x)$, and since limits are unique, we have $f(x) = y$. Thus,
    \begin{align*}
      \left(x,y\right) &= \left(x,f(x)\right)\\
                       &\in \Graph(f).
    \end{align*}
  \end{proof}
  Thus, we can see that the graph of any continuous function is closed in the product topology. However, the converse fails in the general case. For instance,
  \begin{align*}
    f: \R\rightarrow \R\\
    f(x) &= \begin{cases}
      \frac{1}{x} & x\neq 0\\
      0 & x = 0
    \end{cases}
  \end{align*}
  has a closed graph, but $f$ is not continuous.\newline

  However, with linear maps between Banach spaces, the converse is actually true.
  \begin{theorem}[Closed Graph Theorem]
    Let $X$ and $Y$ be Banach spaces, and let $T: X\rightarrow Y$ be a linear map. Then, $T$ is continuous if and only if $\Graph(T) \subseteq X\times Y$ is closed with respect to the product topology on $X\times Y$.
  \end{theorem}
  \begin{proof}
    The forward direction follows from the previous proposition.\newline

    Suppose $\Graph(T)\subseteq X\times Y$ is closed in the product topology. Note that the product topology coincides with the $\norm{\cdot}_1$ topology, with $\norm{\left(x,y\right)}_{1} = \norm{x} + \norm{y}$. Thus, $\left(\Graph(T),\norm{\cdot}_1\right)$ is a Banach space.\newline

    Consider the projection map $P: \Graph(T) \rightarrow X$ defined by $P\left(\left(x,T(x)\right)\right) = x$, which is bijective. We also have
    \begin{align*}
      \norm{P\left(\left(x,T(x)\right)\right)} &= \norm{x}\\
                                               &\leq \norm{x} + \norm{T(x)}\\
                                               &= \norm{\left(x,T(x)\right)}_{1},
    \end{align*}
    meaning $P$ is bounded. Thus, $P$ is bicontinuous, meaning it is bounded below, so for some constant $C$, we have
    \begin{align*}
      \norm{x} &= \norm{P\left(\left(x,T(x)\right)\right)}\\
               &\geq C\norm{\left(x,T(x)\right)}_{1}\\
               &\geq C\norm{T(x)},
    \end{align*}
    meaning $\norm{T(x)}\leq \frac{1}{C}\norm{x}$, so $T$ is bounded.
  \end{proof}
  \begin{example}
    Consider the collection of sequences
    \begin{align*}
      X &= \set{f: \N\rightarrow\C\mid \sum_{k\geq 1}k\left\vert f(k) \right\vert < \infty}.
    \end{align*}
    Note that $X\subseteq \ell_1$, and $X$ is a linear subspaces. Let
    \begin{align*}
      T: \left(X,\norm{\cdot}_1\right) \rightarrow \ell_1\\
      T\left(f\right)\left(k\right) &= \left(kf(k)\right)_1,
    \end{align*}
    which is well-defined and linear. We will show that $T$ is unbounded.\newline

    Let $f_n$ denote the element of $X$ defined by
    \begin{align*}
      f_n &= (\underbrace{1,1,\dots,1}_{\text{$n$ times}},0,0,\dots).
    \end{align*}
    For each $f_n$, the norm is $\norm{f_n}_1 = n$. We also see that
    \begin{align*}
      T\left(f_n\right) &= \left(1,2,\dots,n,0,0,\dots\right),
    \end{align*}
    with norm
    \begin{align*}
      \norm{T\left(f_n\right)}_1 &= \frac{n\left(n+1\right)}{2}.
    \end{align*}
    If $T$ were bounded, we would have a constant $C$ such that
    \begin{align*}
      \frac{n\left(n+1\right)}{2}\leq Cn
    \end{align*}
    for all $n$. This is not possible, meaning $T$ is unbounded.\newline

    However, at the same time, $\Graph(T)$ is closed, as for $\left(f_n\right)_n\rightarrow f$ in $X$ and $\left(T\left(f_n\right)\right)_n\rightarrow g$ in $\ell_1$, we have
    \begin{align*}
      \left(f_n\left(k\right)\right)_n\rightarrow f(k)
    \end{align*}
    and
    \begin{align*}
      \left(kf(k)\right)_n\rightarrow g(k)
    \end{align*}
    for all $k\in \N$, implying that $g(k) = kf(k)$, or that $T\left(f\right) = g$.\newline

    The closed graph theorem thus implies that $X\subseteq \ell_1$ must not be closed.
  \end{example}
  We can use the closed graph theorem to obtain insights about the orthogonal projection operators.\footnote{Idempotent operator, where $P^2 = P$, are also known as projections. I will refer to them as projections.}
  \begin{proposition}
    Let $Z = X\oplus Y$ be a topological internal direct sum of a Banach space $Z$. Then, the projection operators $P_{X}$ and $P_{Y}$ onto the closed subspaces of $X$ and $Y$ respectively are bounded.
  \end{proposition}
  \begin{proof}
    Let $\left(z_n\right)_n$ and $\left(P_X\left(z_n\right)\right)_n$ be sequences in $z$ with $\left(z_n\right)_n \rightarrow z$ and $\left(P_{X}\left(z_n\right)\right)_n \rightarrow u$ for some $z,u\in Z$.\newline

     For each $n\in \N$, we may write $z_n = x_n + y_n$, with $x_n\in X$ and $y_n \in Y$. Since $X$ is closed, and $x_n = P_X\left(z_n\right)$, it is the case that $u\in X$. Then, $y_n = z_n - x_n$ converges to $z - u$; since $Y$ is closed, $z-u\in Y$. Setting $y = z-u$, we have
     \begin{align*}
       P_X\left(z\right) &= P_X\left(u + y\right)\\
                         &= P_X\left(u\right) + P_X\left(y\right)\\
                        &= u.
     \end{align*}
     Thus, the graph of $P_X$ is closed.
  \end{proof}
  \begin{proposition}
    Let $Z$ be a Banach space, and let $P\in \mathcal{B}\left(Z\right)$ be a projection. Then, $\ker\left(P\right)$ and $\Ran\left(P\right)$ are closed subspaces, and
    \begin{align*}
      Z &= \Ran\left(P\right) \oplus \ker\left(P\right).
    \end{align*}
  \end{proposition}
  \begin{proof}
    For every $z\in Z$, we may express $z = P(z) + \left(z - P(z)\right)$. Since $P\left(z-P(z)\right) = P(z) - P^2\left(z\right) = P(z)-P(z) = 0$, we can see that $Z = \Ran\left(P\right) \oplus \ker\left(P\right)$.\newline

    We now need to show that $\Ran\left(P\right)$ is closed. Let $\left(x_n\right)_n$ be a sequence in $\Ran\left(p\right)$ with $\left(x_n\right)_n\rightarrow z$. Then, for each $n\in \N$, we may write $x_n = P\left(z_n\right)$. Since $P$ is continuous, we can see that $\left(P\left(x_n\right)\right)\rightarrow P\left(z\right)$. However, at the same time, we may write $x_n = P\left(z_n\right)$, so
    \begin{align*}
      P\left(x_n\right) &= P\left(P\left(z_n\right)\right)\\
                        &= P\left(z_n\right)\\
                        &= x_n\\
                        &\rightarrow z,
    \end{align*}
    Which implies that $z = P(z)\in \Ran\left(P\right)$ by the uniqueness of limits.
  \end{proof}
  \begin{corollary}
    Let $Z$ be a Banach space, and let $X\subseteq Z$ be a closed subspace. Then, $X$ is complemented in $Z$ if and only if there is a projection $P: Z\rightarrow Z$ with $\Ran\left(P\right) = X$.
  \end{corollary}
  \subsection{Uniform Boundedness Principle}%
  The uniform boundedness principle is often useful when when dealing with a family of bounded operators. It says that a pointwise bounded family of operators on a Banach space is bounded.
  \begin{theorem}[Uniform Boundedness Principle]
    Let $X$ and $Y$ be normed vector spaces, and consider a family of bounded operators $\set{T_i}_{i\in I}\subseteq \mathcal{B}\left(X,Y\right)$.
    \begin{enumerate}[(1)]
      \item If $\sup_{T_i(x)} < \infty$ for all $x\in A\subseteq X$ with $A$ nonmeager, then
        \begin{align*}
          \sup_{i\in I}\norm{T_i}_{\text{op}} < \infty.
        \end{align*}
      \item If $X$ is a Banach space with $\sup_{i\in I}\norm{T_i(x)} < \infty$ for all $x\in X$, then
        \begin{align*}
          \sup_{i\in I}\norm{T_i}_{\text{op}} < \infty.
        \end{align*}
    \end{enumerate}
  \end{theorem}
  \begin{proof}
    We only need to prove (1), as (2) follows.\newline

    For each $i\in I$ and $n\in \N$, define
    \begin{align*}
      E_{n,i} &= \set{x\in X\mid \norm{T_i(x)} \leq n}.
    \end{align*}
    Notice that 
    \begin{align*}
      E_{n,i} &= T_i^{-1}\circ \norm{\cdot}^{-1}\left([0,n]\right).
    \end{align*}
    Since both the norm and $T_i$ are continuous, each $E_{n,i}$ is closed. Therefore,
    \begin{align*}
      E_{n} &= \bigcap_{i\in I}E_{n,i}\\
            &= \set{x\in X\mid \sup_{i\in I}\norm{T_i(x)} \leq n}
    \end{align*}
    is also closed, meaning $\overline{E_n} = E_n$.\newline

    By our assumption,
    \begin{align*}
      A &= \bigcup_{n\geq 1}E_n.
    \end{align*}
    Since $A$ is nonmeager, there is $m\in \N,$  $x_0\in X$, and $r > 0$ such that $B\left(x_0,r\right)\subseteq E_m$.\footnote{Select $\delta$ such that $U\left(x_0,\delta\right)\subseteq E_m$, and then select $r = \delta/2$.}\newline

    We will now prove that $B\left(0,r\right) = rB_{X}\subseteq E_{2m}$. Let $x$ be such that $\norm{x} < r$. Then, $x + x_0\in B\left(x_0,r\right) \subseteq E_m$. For each $i$, we also have
    \begin{align*}
      \norm{T_i(x)} &= \norm{T_i(x) - T_i\left(x_0\right) + T_i\left(x_0\right)}\\
                    &\leq \norm{T_i(x) + T_i\left(x_0\right)} + \norm{T_i\left(x_0\right)}\\
                    &= \norm{T_i\left(x + x_0\right)} + \norm{T_i\left(x_0\right)}\\
                    &\leq 2m,
    \end{align*}
    meaning $\sup_{i\in I}\norm{T_i(x)} \leq 2m$, meaning $x\in E_{2m}$.\newline

    Given $x\in B_X \setminus \set{0}$, we have $rx\in E_{2m}$, so $T_i\left(rx\right) \leq 2m$ for all $i\in I$, meaning
    \begin{align*}
      \norm{T_i(x)} \leq \frac{2m}{r}
    \end{align*}
    for all $i\in I$, so $\sup_{x\in B_X}\norm{T_i(x)}\leq \frac{2m}{r}$. Thus, $\sup_{i\in I}\norm{T_i}_{\text{op}} \leq \frac{2m}{r}$.
  \end{proof}
  \begin{proof}[Alternative Proof of (2)]
    Suppose toward contradiction that $\sup_{i\in I}\norm{T_i}_{\text{op}} = \infty$. We can choose a sequence of operators $\left(T_{i_n}\right)_n$ such that $\norm{T_{i_n}}_{\text{op}}\geq 4^n$. We write $T_{i_n}:=T_n$. We will construct some $x\in X$ such that $\left(\norm{T_n(x)}\right)_n$ is unbounded, yielding a contradiction.\newline

    For any $x\in X$ and $r > 0$, we have
    \begin{align*}
      \sup_{y\in B\left(x,r\right)} &\geq r\norm{T}_{\text{op}}\\
                                    &> \frac{2}{3}r\norm{T}_{\text{op}}.
    \end{align*}
    Setting $x_0 = 0$, $T = T_1$, and $r_0 = T^0$, we can find $x_1$ with
    \begin{align*}
      \norm{x_1 - x_0} \leq r_0\\
      \norm{T_1\left(x_1\right)} &\geq \frac{2r}{3}\norm{T_1}_{\text{op}}.
    \end{align*}
    We find $x_2$ given $T_2$ such that for $r_1 = 3^{-1}$
    \begin{align*}
      \norm{x_2 - x_1} & \leq r_1
      \norm{T_2\left(x_2\right)} \geq \frac{2r_1}{3}\norm{T_2}_{\text{op}}.
    \end{align*}
    Inductively, we find $x_{n+1}$ such that for $r_n = 3^{-n}$,
    \begin{align*}
      \norm{x_{n+1}-x_n} &\leq r_n\\
      \norm{T_{n+1}\left(x_{n+1}\right)} \geq \frac{2r_{n}}{3}\norm{T_{n+1}}_{\text{op}}.
    \end{align*}
    Thus, we have a sequence $\left(x_n\right)_n$ with
    \begin{align*}
      \norm{x_{n+1}-x_n} &\leq 3^{-n}\\
      \norm{T_n\left(x_n\right)}\geq \frac{2}{3^n}\norm{T_n}_{\text{op}}.
    \end{align*}
    This means $\left(x_n\right)_n$ is Cauchy, and thus converges to $x\in X$ with $\norm{x-x_n}\leq \frac{1}{(2)\left(3^{n-1}\right)}$. Thus yields
    \begin{align*}
      \frac{1}{\left(2\right)\left(3^{n-1}\right)}\norm{T_n}_{\text{op}} &\geq \norm{T_n}_{\text{op}}\norm{x_n - x}\\
                                                                     &\geq \norm{T_n\left(x_n -x\right)}\\
                                                                     &= \norm{T_n\left(x_n\right) - T_n\left(x\right)}\\
                                                                     &\geq \norm{T_n\left(x_n\right)} - \norm{T_n\left(x\right)}\\
                                                                     &\geq \frac{2}{3^n}\norm{T_n}_{\text{op}} - \norm{T_n\left(x\right)}.
    \end{align*}
    Thus, we have
    \begin{align*}
      \norm{T_n\left(x\right)} &\geq \frac{2}{3^n}\norm{T_n}_{\text{op}} - \frac{1}{\left(2\right)\left(3^{n-1}\right)}\norm{T_n}_{\text{op}}\\
                               &= \frac{1}{\left(2\right)3^{n}}\norm{T_n}_{\text{op}}\\
                               &\geq \frac{1}{2}\left(\frac{4}{3}\right)^{n}
    \end{align*}
  \end{proof}
  \begin{corollary}[Banach--Steinhaus Theorem]
    Let $X,Y$ be Banach spaces, and let $\left(T_n\right)_n$ be a sequence of bounded linear operators in $\mathcal{B}\left(X,Y\right)$ with $\lim_{n\rightarrow\infty}\left(T_n\left(x\right)\right)$ exists for every $x\in X$. Then,
    \begin{align*}
      T\left(x\right) &= \lim_{n\rightarrow\infty}\left(T_n\left(x\right)\right)
    \end{align*}
    defines a bounded linear map from $X$ to $Y$.
  \end{corollary}
  \begin{proof}
    We can see that $T$ is linear by its definition. Since $\left(T_n\left(x\right)\right)_n$ converges in $Y$ for all $x\in X$, we have $\left(T_n\left(x\right)\right)$ is bounded in $Y$ for all $x\in X$. Thus, $\sup_{n\geq 1}\norm{T_n\left(x\right)} < \infty$ for all $x\in X$, so $\sup_{n\geq 1}\norm{T_n}_{\text{op}} = C$ for some $C > 0$.\newline

    Given $x\in B_X$, we can see that
    \begin{align*}
      \norm{T\left(x\right)} &= \norm{\lim_{n\rightarrow\infty}\left(T_{n}\left(x\right)\right)}\\
                             &= \lim_{n\rightarrow\infty}\norm{T_n\left(x\right)}\\
                             &\leq \limsup_{n\rightarrow\infty}\norm{T_n}_{\text{op}}\norm{x}\\
                             &\leq \limsup_{n\rightarrow\infty}\norm{T_n}_{\text{op}}\\
                             &\leq C,
    \end{align*}
    meaning $\norm{T}_{\text{op}} \leq C$.
  \end{proof}
  \begin{example}
    Let $\left(z_k\right)_k$ be a sequence such that $\sum_{k=1}^{\infty}z_ky_k$ converges for every $\left(y_k\right)_k\in c_0$. We claim that $\left(z_k\right)_k\in \ell_1$.\newline

    Consider the map
    \begin{align*}
      T_n: c_0\rightarrow \C\\
      T_n\left(y\right) &= \sum_{k=1}^{n}z_ky_k.
    \end{align*}
    We can see that
    \begin{align*}
      \left\vert T_n\left(y\right) \right\vert &= \left\vert \sum_{k=1}^{n}z_ky_k \right\vert\\
                                               &\leq \sum_{k=1}^{n}\left\vert z_ky_k \right\vert\\
                                               &\leq \norm{y}_{\infty}\left(\sum_{k=1}^{n}\left\vert z_k \right\vert\right).
    \end{align*}
    Thus, each $T_n$ is bounded linear with $\left\vert T_n \right\vert_{\text{op}}\leq \sum_{k=1}^{n}\left\vert z_k \right\vert$.\footnote{It is an exercise to show that this is an inequality. I'm not sure how to do it though.}\newline

    By assumption, we have that $\left(T_n\left(y\right)\right)_n$ converges, meaning $\sup_{n\geq 1}\left\vert T_n\left(y\right) \right\vert < \infty$ for all $y\in c_0$. The Uniform Boundedness Principle gives $\sup_{n\geq 1}\norm{T_n}_{\text{op}} < \infty$. Thus,
    \begin{align*}
      \norm{z}_1 &= \sum_{k=1}^{\infty}\\
                 &= \sup_{n\geq 1}\left(\sum_{k=1}^{n}\left\vert z_k \right\vert\right)\\
                 &= \sup_{n\geq 1}\norm{T_n}_{\text{op}}\\
                 &<\infty.
    \end{align*}
  \end{example}
  \begin{example}[Product Maps]
    Let $\set{X_i}_{i\in I}$ be a family of Banach spaces, and consider the product
    \begin{align*}
      \prod_{i\in I}X_i &= \set{\left(x_i\right)_i\mid x_i\in X_i, \sup_{i\in I}\norm{x_i} < \infty}.
    \end{align*}
    The product space is a Banach space with pointwise operations and the norm $\norm{\left(x_i\right)_{i\in I}}_{\infty} = \sup_{i\in I}\norm{x_i}$.\newline

    Consider a Banach space $Z$ and a linear map $Z\rightarrow \prod_{i\in I}X_i$.\newline

    We can see that if $T$ is continuous, then $\pi_j\circ T: Z\rightarrow X_j$, where $\pi_j$ is the canonical projection map on $\prod_{i\in I}X_i$, is also continuous for each $j$. However, it is also the case that this is a sufficient condition --- that is, if $\pi_j\circ T$ is continuous for each $j$, then $T$ is continuous.
    \begin{proof}
      Let $\iota_j: X_j\hookrightarrow \prod_{i\in I}X_i$ be the inclusion map, and $T_j: Z\rightarrow \prod_{i\in I}X_i$ be given by
      \begin{align*}
        T_j &= \iota_j\circ \pi_j\circ T.
      \end{align*}
      Then,
      \begin{align*}
        \sup_{j\in I}\norm{T_j\left(z\right)} &= \sup_{j\in I}\norm{\pi_j\circ T(z)}\\
                                              &= \norm{T(z)}\\
                                              &< \infty
      \end{align*}
      for all $z\in Z$. Thus, $\sup_{j\in I}\norm{T_j}_{\text{op}} < \infty$, so $\norm{T}_{\text{op}}$ is bounded.
    \end{proof}
  \end{example}
  \subsection{Hahn--Banach Theorems}%
  The Hahn--Banach extension theorems allow us to extend continuous linear functionals from subspaces to the full vector space, while the separation theorems allow us to separate points from closed subspaces.
  \begin{definition}[Algebraic Dual]
    Let $X$ be a vector space over $\F$. Then, $X' = \mathcal{L}\left(X,\F\right)$, which consists of \textit{all} linear functionals from $X$ into $\F$, is known as the algebraic dual.
  \end{definition}
  \begin{definition}[Continuous Dual]
    Let $X$ be a normed vector space over $\F$. Then, $X^{\ast} = \mathcal{B}\left(X,\F\right)$, which consists of all \textit{continuous} linear functionals from $X$ into $\F$, is known as the continuous dual.\footnote{Since this is functional analysis, any mention of the ``dual space'' will refer to the continuous dual.}
  \end{definition}
  \begin{definition}[Hyperplane]
    A hyperplane is a subspace $E\subseteq X$ such that $\Dim\left(X/E\right) = 1$. An affine hyperplane is of the form $x_0 + E$ for some fixed $x_0\in X$.
  \end{definition}
  \begin{proposition}
    Let $X$ be a normed vector space over $\F$.
    \begin{enumerate}[(1)]
      \item If $\varphi\in X'$ is not zero, then $\varphi$ is bounded if and only if $\ker\left(\varphi\right) \subseteq X$ is closed. In this case, $X/\ker\left(\varphi\right) \cong \F$ are bicontinuously isomorphic.
      \item Given a closed hyperplane $M\subseteq X$, there is a $\varphi\in X^{\ast}$ with $\ker\left(\varphi\right) = M$.
      \item If $\varphi\in X'$ is unbounded, then $\ker\left(\varphi\right)\subseteq X$ is norm-dense.
    \end{enumerate}
  \end{proposition}
  \begin{proof}\hfill
    \begin{enumerate}[(1)]
      \item If $\varphi$ is continuous, then $\ker\left(\varphi\right) = \varphi^{-1}\left(0\right)$ is closed. Thus, from the definition of a quotient map, $\tilde{\varphi}: X/\ker\left(\varphi\right) \rightarrow \F$ defined by $x + \ker\left(\varphi\right) = \varphi(x)$, is a bicontinuous isomorphism from $X/\ker\left(\varphi\right)$ to $\F$.\newline

        Conversely, if $\ker\left(\varphi\right)$ is closed in $X$, then $X/\ker\left(\varphi\right)$ is a normed space with dimension $1$, meaning $\tilde{\varphi}$ is continuous (as it is a linear map of dimension $1$). Thus, $\varphi = \tilde{\varphi}\circ \pi$ is continuous.
      \item Since $X/M$ is a normed space with $\Dim\left(X/M\right) = 1$, there is a bicontinuous isomorphism $\psi: X/M\rightarrow \F$. Set $\varphi = \psi\circ\pi$, where $\pi$ denotes the canonical projection map. Then, $\varphi$ is continuous with $\ker\left(\varphi\right) = M$.
      \item Let $M = \ker\left(\varphi\right)$. Since $\varphi$ is unbounded, then $M\subset \overline{M}$, meaning $p: X/M\rightarrow X/\overline{M}$ defined by $x+M \mapsto x + \overline{M}$ is well-defined with nontrivial kernel.\newline

        However, since $\Dim\left(X/M\right) = 1$, we must have $\ker\left(p\right) = X/M$, implying that $x+\overline{M} = 0_{X/\overline{M}}$ for all $x\in X$, meaning $x\in \overline{M}$ for all $x\in X$, so $X\subseteq \overline{M}$.
    \end{enumerate}
  \end{proof}
  For a vector space $X$ over $\C$, we can consider $X$ as a $\R$-vector space by simply ``forgetting'' the imaginary scalars. We can also consider the respective dual spaces over $\R$ and $\C$, which we write as follows.
  \begin{align*}
    \mathcal{L}_{\C}\left(X,\C\right) &:= \set{\varphi: X\rightarrow \C\mid \varphi\text{ is $\C$-linear}}\\
    \mathcal{L}_{\R}\left(X,\R\right) &:= \set{\varphi: X\rightarrow \R\mid\varphi\text{ is $\R$-linear}}.
  \end{align*}
  For $\varphi\in \mathcal{L}_{\mathcal{C}}\left(X,\C\right)$, we define $\re\left(\varphi\right): X\rightarrow \R$ and $\im\left(\varphi\right): X\rightarrow \R$ to denote the real and imaginary parts of $\varphi$.
  \begin{proposition}
    Let $X$ be a $\F$-vector space.
    \begin{enumerate}[(1)]
      \item For $\varphi\in \mathcal{L}_{\C}\left(X,\C\right)$, then $u := \re\left(\varphi\right)$ and $v := \im\left(\varphi\right)$, and $v(x) = -u\left(ix\right)$, implying $\varphi(x) = u(x) - iu\left(ix\right)$.
      \item For $u\in \mathcal{L}_{\R}\left(X,\R\right)$, the map $v: X\rightarrow \R$ defined by $v(x) := -u\left(ix\right)$ belongs to $\mathcal{L}_{\R}\left(X,\R\right)$, and the map $\varphi: X\rightarrow \C$ defined by $\varphi(x) = u(x) + iv(x)$ belongs to $\mathcal{L}_{\C}\left(X,\C\right)$.
      \item If $\left(X,\norm{\cdot}\right)$ is a normed space, with $\varphi$ and $u$ as in (1) or (2), then $\norm{\varphi} = \norm{u}$.
    \end{enumerate}
  \end{proposition}
  \begin{proof}\hfill
    \begin{enumerate}[(1)]
      \item For $x,y\in X$ and $t\in \R$, we find
        \begin{align*}
          u\left(x + ty\right) &= \re\left(\varphi\left(x + ty\right)\right)\\
                               &= \re\left(\varphi(x) + t\varphi\left(y\right)\right)\\
                               &= \re\left(\varphi(x)\right) + \re\left(t\varphi(y)\right)\\
                               &= \re\left(\varphi(x)\right) + t\re\left(\varphi(y)\right)\\
                               &= u(x) + tu(y).
        \end{align*}
        For any $z\in \C$, we have $\re\left(iz\right) = -\im\left(z\right)$, meaning
        \begin{align*}
          v(x) &= \im\left(\varphi(x)\right)\\
               &= -\re\left(i\varphi(x)\right)\\
               &= -\re\left(\varphi\left(ix\right)\right)\\
               &= -u\left(ix\right).
        \end{align*}
      \item Since $u$ is $\R$-linear, so too are $v$ and $\varphi$. Thus,
        \begin{align*}
          \varphi\left(ix\right) &= u\left(ix\right) - iu\left(-x\right)\\
                                 &= u\left(ix\right) + iu\left(x\right)\\
                                 &= i\varphi(x).
        \end{align*}
        Thus, $\varphi$ is $\C$-linear.
      \item Note that $\left\vert u(x) \right\vert = \left\vert \re\left(\varphi(x)\right) \right\vert \leq \left\vert \varphi(x) \right\vert$ for all $x\in X$, meaning $\norm{u}\leq \norm{\phi}$.\newline

        For $\varphi(x) \neq 0$, we set $\alpha = \frac{\left\vert \varphi(x) \right\vert}{\varphi(x)}$, meaning $\left\vert \alpha \right\vert = 1$ and $\alpha\varphi(x) = \left\vert \varphi(x) \right\vert$. Thus, we have
        \begin{align*}
          \left\vert \varphi(x) \right\vert &= \alpha\varphi(x)\\
                                            &= \varphi\left(\alpha x\right)\\
                                            &= \re\left(\varphi\left(\alpha x\right)\right)\\
                                            &= u\left(\alpha x\right)\\
                                            &\leq \norm{u}\norm{\alpha x}\\
                                            &= \norm{u}\norm{x},
        \end{align*}
        meaning $\norm{\varphi}\leq \norm{u}$.
    \end{enumerate}
  \end{proof}
  \begin{definition}[Minkowski Functional]
  Let $X$ be an $\R$-vector space. A Minkowski functional on $X$ is a map $m: X\rightarrow \R$ such that the following are satisfied for every $x,y\in X$ and $t\geq 0$:
  \begin{enumerate}[(i)]
    \item $m\left(x+y\right)\leq m(x) + m(y)$;
    \item $m\left(tx\right) = tm(x)$.
  \end{enumerate}
  \end{definition}
  Before we go into the Hahn--Banach theorems, we want to consider the problem of whether, for an $\F$-vector space $X$ and a subspace $E\subseteq X$, if there is a linear functional $\varphi: E\rightarrow \F$, is there an extension to $\psi: X\rightarrow \F$ such that $\psi|_{E} = \varphi$?\newline

  The answer is yes. If we select a basis for $E$, $\mathcal{B}_0$, we extend it to a basis for $X$, $\mathcal{B}$, and define $\psi_0: X\rightarrow \F$ by sending $b\mapsto \varphi(b)$ for $b\in \mathcal{B}_{0}$, and $0$ for elements in $\mathcal{B}\setminus \mathcal{B}_0$.\newline

  We may be interested in extending \textit{bounded} linear functionals as well, and if we can control the norms of these various extensions. This is where the Hahn--Banach theorems start to play a major role.
  \begin{theorem}[Hahn--Banach--Minkowski]
    Let $X$ be a real vector space with a Minkowski functional $m: X\rightarrow \R$, and let $E\subseteq X$ be a subspace. Suppose $\varphi: E\rightarrow \R$ is a linear functional with $\varphi(y) \leq m(y)$ for all $y\in E$.\newline

    Then, there exists a linear functional $\psi: X\rightarrow \R$ such that $\psi|_{E} = \varphi$ and $\psi(x)\leq m(x)$ for all $x\in X$.
  \end{theorem}
  \begin{proof}
    Consider the collection $\mathcal{Z} = \set{\left(Z,\phi\right)}$ of all linear functionals that extend $\varphi$ and are dominated by $m$. That is, $E\subseteq Z\subseteq X$ are subspaces, $\phi\in Z'$ with $\phi|_{E} = \varphi$, and $\phi(z) \leq m(z)$ for all $z\in Z$.\newline

    Note that $\left(E,\varphi\right)$ is in $\mathcal{Z}$, meaning $\mathcal{Z}$ is nonempty. We define an ordering on $\mathcal{Z}$ by taking
    \begin{align*}
      \left(Z_1,\phi_1\right)\leq \left(Z_2,\phi_2\right)
    \end{align*}
    if and only if $Z_1\subseteq Z_2$ and $\phi_2|_{Z_1} = \phi_1$.\newline

    Consider a chain in $\mathcal{Z}$, $\mathcal{C} = \left(Z_i,\phi_i\right)_{i\in I}$. Let $Y = \bigcup_{i\in I}Z_i$. Since $Z_i$ are totally ordered by inclusion, it is the case that $Y$ is a subspace of $X$. We define $\eta: Y\rightarrow \R$ by $\eta(x) = \phi_i(x)$ for $x\in Z_i$. Since $Z_i$ are totally ordered by inclusion, for $Z_j\subseteq Z_i$, it is the case that $\phi_{i}|_{Z_j} = \phi_i$, meaning $\left(Y,\eta\right)$ is an upper bound for $\mathcal{C}$.\newline

    By Zorn's lemma,\footnote{If $P$ is a partially ordered set, and every totally ordered $C\subseteq P$ admits an upper bound, then $P$ has a maximal element.} there is a maximal element $\left(Z_0,\phi_0\right)$. We will now show that $Z_0 = X$.\newline

    Suppose there exists $x_1\in X\setminus Z_0$. Consider the subspace
    \begin{align*}
      Z_1 &= \set{x + \lambda x_1\mid \lambda\in \R,~x\in Z_0}.
    \end{align*}
    Note that a vector in $Z_1$ has a unique expression $x + \lambda x_1$, since $x_1\notin Z_0$. Thus, the linear functional $\phi_1: Z_1\rightarrow \R$ defined by $\phi_1\left(x + \lambda x_1\right) = \phi_0\left(x\right) + \lambda \alpha$ is well-defined. Note that $E\subseteq Z_1$ and $\phi_1\in Z_1'$ extends $\varphi$, since $\phi_0$ extends $\varphi$. We need to find some $\alpha$ such that $\phi_1\left(z\right) \leq m(z)$ for all $z\in Z_1$, which will contradict the maximality of $\left(Z_0,\phi_0\right)$.\newline

    For any $u,v\in Z_0$, we have
    \begin{align*}
      \phi_0\left(u\right) + \phi_0\left(v\right) &= \phi_0(u+v)\\
                                                  &\leq m\left(u+v\right)\\
                                                  &= m\left(u-x_1 + v+x_1\right)\\
                                                  &\leq m\left(u-x_1\right) + m\left(v + x_1\right),
    \end{align*}
    meaning that for all $u,v\in Z_0$,
    \begin{align*}
      \phi_0\left(u\right) - m\left(u-x_1\right) \leq m\left(v + x_1\right) - \phi_0(v).
    \end{align*}
    We define
    \begin{align*}
      \alpha &= \sup_{u\in Z_0}\left(\phi_0\left(u\right) - m\left(u-x_1\right)\right).
    \end{align*}
    Then, for all $u\in Z_0$,
    \begin{align*}
      m\left(u-x_1\right) &\geq \phi_0\left(u\right) - \alpha,
    \end{align*}
    meaning $\alpha \leq m\left(v+x_1 \right) - \phi_0(v)$ for all $v\in Z_0$, so
    \begin{align*}
      \phi_0(v) + \alpha \leq m\left(v + x_1\right).
    \end{align*}
    Thus, for $t = \pm 1$, and $z\in Z_0$, we have
    \begin{align*}
      \phi_0(z) + t\alpha \leq m\left(z + t\alpha\right)
    \end{align*}
    Scaling by $c > 0$, we find
    \begin{align*}
      \phi_0\left(cz\right) + ct\alpha \leq m\left(cz + ct\alpha\right).
    \end{align*}
    Setting $c = |\lambda|$, $z = \frac{1}{c}x$, and $t = \operatorname{sgn}\left(\lambda\right)$, we find 
    \begin{align*}
      \phi_1\left(x\right) + \lambda\alpha \leq m\left(x + \lambda\alpha\right),
    \end{align*}
    thus contradicting maximality.
  \end{proof}
  \begin{theorem}[Hahn--Banach Extension]
    Let $X$ be a normed vector space, $E\subseteq X$ a subspace with a bounded linear functional $\varphi\in E^{\ast}$. Then, there exists a continuous $\psi\in X^{\ast}$ such that $\norm{\varphi} = \norm{\psi}$ and $\psi|_{E} = \varphi$.
  \end{theorem}
  \begin{proof}
    Let $m(x) = \norm{\varphi}\norm{x}$ be our desired Minkowski functional. Restricting our view to $\R$, we find
    \begin{align*}
      \varphi(x) &\leq \left\vert \varphi(x) \right\vert\\
                 &\leq \norm{\varphi}\norm{x}\\
                 &= m(x).
    \end{align*}
    Thus, there exists $\psi\in X'$ such that $\psi|_{E} = \varphi$ and $\psi$ is dominated by $m$ on $X$.\newline

    We must now show that $\norm{\psi}=\norm{\varphi}  $. We have
    \begin{align*}
      \psi(x) &\leq m(x)\\
              &\leq \norm{\varphi}
    \end{align*}
    for all $x\in B_X$, and we also have
    \begin{align*}
      -\psi\left(x\right) &= \psi\left(-x\right)\\
                          &\leq \norm{\varphi}\norm{x}\\
                          &\leq \norm{\varphi}.
    \end{align*}
    Thus, $\left\vert \psi(x) \right\vert\leq \norm{\varphi}$ for all $x\in B_X$. Additionally, since $\norm{\varphi}\leq \norm{\psi}$ necessarily (as $\psi$ is defined on a larger space than $\varphi$), it is the case that $\norm{\varphi} = \norm{\psi}$.\newline

    Now, turning our attention to $\C$, we decompose $\varphi = u + iv$, where $u = \re\left(\varphi\right)$ and $v(y) = -u\left(iy\right)$ for all $y\in E$. Then, $u: E\rightarrow \R$ is an $\R$-linear functional with $\norm{u} = \norm{\varphi}$. We extend $u$ to $U\in X^{\ast}$ such that $\norm{U} = \norm{u} = \norm{\varphi}$ with $U|_{E} = u$. Defining $V(x) = -U(ix)$ for all $x\in X$, we find $\psi = U + iV$ is a $\C$-linear functional with $\norm{\psi} = \norm{U} = \norm{\varphi}$, and $\psi|_{E} = \varphi$.
  \end{proof}
  Thanks to the Hahn--Banach extension result, we can see that there are ``enough'' linear functionals in $X^{\ast}$ for any normed vector space $X$.
  \begin{theorem}[Hahn--Banach Separation]
    Let $X$ be a normed vector space.
    \begin{enumerate}[(1)]
      \item Given a nonzero $x_0\in X$, there is a $\varphi\in X^{\ast}$ with $\norm{\varphi} = 1$ and $\varphi(x_0) = \norm{x_0}$. We call $\varphi$ a norming functional.
      \item Given a proper closed subspace $E\subseteq X$ and $x_0\in X\setminus E$, there is a $\varphi\in X^{\ast}$ such that $\varphi|_{E} = 0$, $\norm{\varphi} \leq 1$, and $\varphi(x_0) = \dist_{E}\left(x_0\right)$.
      \item The unit sphere in $X^{\ast}$, $S_{X^{\ast}}$, separates the points of $X$.
    \end{enumerate}
  \end{theorem}
  \begin{proof}\hfill
    \begin{enumerate}[(1)]
      \item Let $E = \Span\left(x_0\right)$, and $\varphi: E\rightarrow \F$ defined by $\varphi(\lambda x) = \lambda\norm{x}$. This is a linear functional, with
        \begin{align*}
          \left\vert \varphi(\lambda x) \right\vert &= \left\vert \lambda \right\vert\norm{x}\\
                                                    &= \norm{\lambda x},
        \end{align*}
        meaning $\norm{\varphi} = 1$. We can extend $\varphi$ to $\psi$ on $X$ such that $\psi|_{E} = \varphi$, meaning $\psi\left(x_0\right) = \varphi\left(x_0\right) = \norm{x_0}$.
      \item There is a $\psi\in \left(X/E\right)^{\ast}$ with $\norm{\psi} = 1$ and $\psi\left(x_0 + E\right) = \norm{x_0 + E} = \dist_{E}\left(x_0\right)$. For the canonical quotient map $\pi: X\rightarrow X/E$, we set $\varphi = \psi\circ\pi$. Since $\pi$ sends $E$ to $0 + E$, $\varphi$ must send any element of $E$ to $0$.\newline

        Additionally,
        \begin{align*}
          \norm{\varphi} &= \norm{\psi\circ\pi}\\
                         &\leq \norm{\psi}\norm{\pi}\\
                         &= 1.
        \end{align*}
        Thus, $\varphi$ is such that $\varphi\left(x_0\right) = \dist_{E}\left(x_0\right)$.
      \item If $x\neq y$ with $x,y\in X$, then $x-y\neq 0$, meaning there is $\varphi\in S_{X^{\ast}}$ with $\varphi\left(x-y\right) = \norm{x-y}\neq 0$, meaning $\varphi(x)\neq \varphi(y)$.
    \end{enumerate}
  \end{proof}
  \begin{corollary}
    Let $X$ be a normed space. For every $x\in X$, we have
    \begin{align*}
      \sup_{\varphi\in B_{X^{\ast}}} \left\vert \varphi(x) \right\vert &= \sup_{\varphi\in S_{X^{\ast}}}\left\vert \varphi(x) \right\vert\\
                                                                       &= \norm{x}.
    \end{align*}
  \end{corollary}
  \begin{proposition}
    Let $X$ be a normed vector space with separable dual $X^{\ast}$. Then, there is an isometric embedding $X\hookrightarrow \ell_{\infty}$.
  \end{proposition}
  \begin{proof}
    Let $\left(\varphi_{n}\right)_{n\geq 1}$ be a norm-dense subset of $B_{X^{\ast}}$. Consider the map
    \begin{align*}
      \mu: X\rightarrow \ell_{\infty}
    \end{align*}
    with $\mu(x) = \left(\varphi_{n}\left(x\right)\right)_{n\geq 1}$ for $x\in X$. It is pretty clear that $\mu$ is linear. Since $\norm{\varphi_n}\leq 1$ for each $n\geq 1$, we have
    \begin{align*}
      \norm{\mu(x)}_{\infty} &= \sup_{n\geq 1}\left\vert \varphi_n(x) \right\vert\\
                             &\leq \sup_{n\geq 1}\norm{\varphi_n}\norm{x}\\
                             &\leq \norm{x}.
    \end{align*}
    Suppose $\norm{\mu(x)}_{\infty} < \norm{x}$ for some $x\in B_X$. Then, since $x\neq 0$, we set $\delta = \norm{x} -\norm{\mu(x)}_{\infty}$.\newline

    We can find $\psi\in B_{X^{\ast}}$ with $\psi(x) = \norm{x}$. Since $\left(\varphi_n\right)_{n\geq 1}$ is norm-dense, there is $m\in \N$ with
    \begin{align*}
      \norm{\psi - \varphi_{m}} < \delta,
    \end{align*}
    meaning
    \begin{align*}
      \norm{x} &= \psi(x)\\
               &\leq \left\vert \psi(x) - \varphi_{m}(x) \right\vert + \left\vert \varphi_m(x) \right\vert\\
               &\leq \norm{\psi - \varphi_m}\norm{x} + \norm{\mu(x)}\\
               &< \delta + \norm{\mu(x)}_{\infty}\\
               &= \norm{x},
    \end{align*}
    which is a contradiction. Thus, $\mu$ is isometric on the unit ball.
  \end{proof}
  \begin{definition}
    Let $X$ be a normed vector space with $S\subseteq X$ and $T\subseteq X^{\ast}$ subsets. We define the annihilator of $S$ by
    \begin{align*}
      S^{\perp} &= \set{\varphi\in X^{\ast}\mid \varphi(x)= 0~\forall x\in S},
    \end{align*}
    and the pre-annihilator of $T$ by
    \begin{align*}
      ^{\perp}T &= \set{x\in X\mid \varphi(x) = 0~\forall \varphi \in T}.
    \end{align*}
  \end{definition}
  \begin{exercise}
    If $X$ is a normed space and $E\subset X$ is a proper subspace, show that $E^{\perp}$ is nonzero. Also, show that $S^{\perp}\subseteq X^{\ast}$ and $^{\perp}T\subseteq X$ are norm-closed subspaces.
  \end{exercise}
  \begin{solution}\hfill
    \begin{enumerate}[(a)]
      \item For $E\subset X$ a proper subspace, we know there exists some $\phi\in X^{\ast}$ such that $\norm{\phi} \leq 1$ and $\norm{x_0} = \dist_{E}\left(x_0\right)$ for some $x_0\in X$, meaning $E^{\perp}\neq 0$.
      \item For $\varphi,\psi\in S^{\perp}$, we have $\varphi(x) = 0$ and $\psi(x) = 0$ for all $x\in \S$, meaning
        \begin{align*}
          \left(\varphi + \alpha \psi\right)\left(x\right) &= \varphi(x) + \alpha \psi(x)\\
                                                           &= 0,
        \end{align*}
        so $\varphi + \alpha \psi\in S^{\perp}$, so $S$ is a subspace.\newline

        Additionally, for a sequence $\left(\varphi_n\right)_{n\geq 1} $ in $S^{\perp}$ such that $\left(\varphi_{n}\right)_n\rightarrow \varphi\in X^{\ast}$, we have $\varphi_n(x) = 0$ for all $x\in S$, so $\varphi(x) = 0$ for all $x\in S$.
      \item Similarly, for $x,y\in ^{\perp}T$, we have
        \begin{align*}
          \varphi\left(x + \alpha y\right) &= \varphi(x) + \alpha\varphi(y)\\
                                           &= 0,
        \end{align*}
        meaning $^{\perp}T$ is a subspace. Similarly, for any sequence $\left(x_n\right)_{n\geq 1}$ in $^{\perp}T$ such that $\left(x_n\right)_n\rightarrow x\in X$, we have $\varphi\left(x_n\right) = 0$ for all $\varphi \in T$, so $\varphi(x) = 0$.
    \end{enumerate}
  \end{solution}
  \begin{exercise}
    Let $T: X\rightarrow Y$ be a bounded linear map between normed vector spaces. Show
    \begin{enumerate}[(1)]
      \item $\Ran(T)^{\perp} = \ker\left(T^{\ast}\right)$
      \item $^{\perp}\Ran\left(T\right) = \ker\left(T\right)$.
    \end{enumerate}
  \end{exercise}
  \begin{solution}
    Note that $T^{\ast}: Y^{\ast}\rightarrow X^{\ast}$ is defined by $T^{\ast}\left(\varphi\right)(x) = \varphi\left(T(x)\right)$.
    \begin{enumerate}[(1)]
      \item 
        \begin{align*}
          \Ran\left(T\right)^{\perp} &= \set{\varphi\in Y^{\ast}\mid \varphi\left(y\right) = 0,~\forall y\in \Ran(T)}\\
                                     &= \set{\varphi\in Y^{\ast}\mid \varphi\left(T\left(x\right)\right) = 0,~\forall x\in X}\\
                                     &= \set{\varphi\in Y^{\ast}\mid T^{\ast}\left(\varphi\right)(x) = 0,~\forall x\in X}\\
                                     &= \set{\varphi\in Y^{\ast}\mid T^{\ast}\left(\varphi\right) = 0}\\
                                     &= \ker\left(T^{\ast}\right).
        \end{align*}
      \item 
        \begin{align*}
          ^{\perp}\ran\left(T^{\ast}\right) &= \set{x\in X\mid \psi(x) = 0,~\forall \psi\in \Ran\left(T^{\ast}\right)}\\
                                            &= \set{x\in X\mid \varphi\left(T(x)\right) = 0,~\forall \varphi\in Y^{\ast}}\\
                                            &= \set{x\in X\mid T(x) = 0}\\
                                            &= \ker\left(T\right).
        \end{align*}
    \end{enumerate}
  \end{solution}
  \begin{corollary}
    Let $X$ be a normed vector space, and suppose $S\subseteq X$ is a subset. Then,
    \begin{align*}
      ^{\perp}\left(S^{\perp}\right) &= \overline{\Span}(S).
    \end{align*}
  \end{corollary}
  \begin{proof}
    Since $S\subseteq ^{\perp}\left(S^{\perp}\right)$, and the latter is a norm-closed subspace, it is the case that $Z = \overline{\Span}(S)\subseteq ^{\perp}\left(S^{\perp}\right)$.\newline

    Suppose there exists $x_0\in ^{\perp}\left(S^{\perp}\right)\setminus Z$. Then, we must have a $\varphi\in X^{\ast}$ with $\varphi|_{Z} = 0$ and $\varphi(x_0)\neq 0$, meaning $\varphi\in S^{\perp}$, so $\varphi\left(x_0\right) = 0$, which is a contradiction.
  \end{proof}
  Having used the Hahn--Banach theorems to understand some structure results on normed vector spaces, we now turn to an application to complex analysis.
  \begin{definition}
    Let $X$ be a normed vector space and consider a function $f: \Omega\rightarrow X$, where $\Omega\subseteq \C$ is open. We say $f$ is differentiable at $z\in \Omega$ if
    \begin{align*}
      f'(z) &= \lim_{w\rightarrow z}\frac{f(w)-f(z)}{w-z}\\
            &= \lim_{\zeta \rightarrow 0}\frac{f(z + \zeta) - f(z)}{\zeta}
    \end{align*}
    exists in $X$. Here, the limit is defined through the norm of $X$.\newline

    If $f'(z)$ exists for all $z\in \Omega$, we say $f$ is holomorphic on $\Omega$. If $\Omega = \C$, then we say $f$ is entire.
  \end{definition}
  \begin{example}
    Let $A$ be a Banach algebra with $a\in A$ fixed. The exponential map
    \begin{align*}
      f: \C\rightarrow A\\
      z\mapsto \exp(az)
    \end{align*}
    is entire.\newline

    For any $z\in \C$,
    \begin{align*}
      f'(z) &= \lim_{w \rightarrow 0}\frac{\exp(az + aw) - \exp(az)}{w}\\
            &= \lim_{w\rightarrow 0}\frac{\exp(az)\exp(aw) - \exp(az)}{w}\\
            &= \exp(az)\left(\lim_{w\rightarrow 0}\frac{\exp(aw)-1}{w}\right)\\
            &= a\exp(az).
    \end{align*}
  \end{example}
  \begin{lemma}
    Let $X$ be a normed space, and suppose $f: \Omega\rightarrow X$ is differentiable at $z\in \Omega$. If $\varphi\in X^{\ast}$, then $\varphi\circ f$ is also differentiable at $z$. Similarly, if $f$ is holomorphic on $\Omega$, then $\varphi\circ f$ is holomorphic on $\Omega$.
  \end{lemma}
  \begin{proof}
    By continuity and linearity of $\varphi$, we have
    \begin{align*}
      \left(\varphi\circ f\right)'(z) &= \lim_{w\rightarrow z}\frac{\varphi\left(f(w)\right) - \varphi\left(f(z)\right)}{w-z}\\
                                      &= \lim_{w\rightarrow z}\varphi\left(\frac{f(w)-f(z)}{w-z}\right)\\
                                      &= \varphi\left(\lim_{w\rightarrow z}\frac{f(w)-f(z)}{w-z}\right)\\
                                      &= \varphi\left(f'(z)\right).
    \end{align*}
  \end{proof}
  Liouville's theorem is a fundamental result in complex function theory which provides insight into the behavior of holomorphic functions.
  \begin{theorem}[Liouville's Theorem]
    If $f: \C\rightarrow \C$ is entire, and $\sup_{z\in \C}\left\vert f(z) \right\vert < \infty$, then there is some $z_0\in \C$ such that $f(z) = z_0$ for all $z\in \C$.
  \end{theorem}
  A proof of Liouville's theorem is presented below.
  \begin{proof}
    Let $f$ be holomorphic and bounded on $\C$. Then, $f$ is analytic, meaning
    \begin{align*}
      f(z) &= \sum_{k=0}^{\infty}a_kz^k,
    \end{align*}
    and Cauchy's integral formula gives
    \begin{align*}
      a_k &= \frac{f^{(k)}(0)}{k!}\\
          &= \frac{1}{2\pi i}\oint_{C_r}\frac{f(w)}{w^{k+1}}\:dw,
    \end{align*}
    where $C_r$ denotes a circle of radius $r$ about the origin. Since $f$ is bounded on $\C$, $f$ is bounded on $C_{r}$. Thus, we have
    \begin{align*}
      \left\vert a_k \right\vert &= \frac{1}{2\pi} \left\vert \oint_{C_r}\frac{f(w)}{w^{k+1}}\:dw \right\vert\\
                                 &\leq \frac{1}{2\pi}\oint_{C_r}\left\vert \frac{f(w)}{w^{k+1}} \right\vert\left\vert dw \right\vert\\
                                 &\leq \frac{1}{2\pi}\oint_{C_r}\frac{M}{r^{k+1}}\left\vert dw \right\vert\\
                                 &= \frac{M}{2\pi r^{k+1}}\oint_{C_r}\left\vert dw \right\vert\\
                                 &= \frac{M}{r^{k}}.
    \end{align*}
    Since $r$ was arbitrary, this means $a_k = 0$ for all $k\geq 1$, meaning $f(z) = a_0$ for all $z$.
  \end{proof}
  \begin{corollary}
    Let $X$ be a normed space. If $f: \C\rightarrow X$ is entire and bounded, then there is $x_0\in X$ such that $f(z) = x_0$ for all $z\in \C$.
  \end{corollary}
  \begin{proof}
    Suppose this is not the case. Then, there are $z_1,z_2\in \C$ such that $f\left(z_1\right)\neq f\left(z_2\right)$. Thus, there is $\varphi\in X^{\ast}$ such that $\varphi\left(f\left(z_1\right)\right) \neq \varphi\left(f\left(z_2\right)\right)$. Additionally, $\varphi\circ f: \C\rightarrow \C$ is entire, and
    \begin{align*}
      \sup_{z\in\C}\left\vert \varphi\circ f \right\vert &= \sup_{z\in\C}\left\vert \varphi\left(f(z)\right) \right\vert\\
                                                         &\leq \sup_{z\in \C}\norm{\varphi}\left\vert f(z) \right\vert\\
                                                         &< \infty,
    \end{align*}
    so $\varphi\circ f$ is constant. However, this contradicts $\varphi\left(f\left(z_1\right)\right)\neq \varphi\left(f\left(z_2\right)\right)$.
  \end{proof}
\section{Duality}%
If $X$ is a normed vector spaces, we know that the continuous dual space, $X^{\ast} = \mathcal{B}\left(X,\C\right)$, is a Banach space as $\C$ is complete. This section will be focused on understanding dual spaces and their operators.
\subsection{Completions and the Double Dual}%
\begin{definition}[Double Dual]
We define $X^{\ast\ast} = \mathcal{B}\left(X^{\ast},\C\right)$ to be the double dual of $X$.
\end{definition}
\begin{remark}
$X^{\ast\ast}$ is complete.
\end{remark}
\begin{proposition}
  Let $X$ be a normed vector space. If $x\in X$, the map
  \begin{align*}
    \hat{x}: X^{\ast}\rightarrow \F\\
    \hat{x}(\varphi) = \varphi(x)
  \end{align*}
  is linear and bounded with $\hat{x}_{\text{op}} = \norm{x}$. The map $\iota_X: X\hookrightarrow X^{\ast\ast}$ with $x\mapsto \hat{x}$, is a linear isometry known as the canonical embedding.
\end{proposition}
\begin{proof}
  First, we show that $\hat{x}$ is linear. Let $\varphi,\psi\in X^{\ast}$ and $\alpha \in \F$. Then,
  \begin{align*}
    \hat{x}\left(\varphi + \alpha \psi\right) &= \left(\varphi + \alpha \psi\right)\left(x\right)\\
                                              &= \varphi(x) + \alpha\psi(x)\\
                                              &= \hat{x}\left(\varphi\right) + \alpha\hat{x}(\psi).
  \end{align*}
  We can also see, by the Hahn--Banach Separation theorem,
  \begin{align*}
    \norm{\hat{x}}_{\text{op}} &= \sup_{\varphi\in B_{X^{\ast}}}\left\vert \hat{x}\left(\varphi\right) \right\vert\\
                               &= \sup_{\varphi\in B_{X^{\ast}}} \left\vert \varphi(x) \right\vert\\
                               &= \norm{x}.
  \end{align*}
  Now, we will show that $\iota_X$ is linear. Let $x,y\in X$, $\lambda \in \F$. Then,
  \begin{align*}
    \iota_X\left(x + \lambda y\right) &= \widehat{x + \lambda y}\\
                                    &= \hat{x} + \widehat{\lambda y}\\
                                    &= \hat{x} + \lambda \hat{y}\\
                                    &= \iota_X\left(x\right) + \lambda\iota_X\left(y\right).
  \end{align*}
\end{proof}
\begin{definition}[Completion of a Metric Space]
  For a metric space $X$, the completion $\widetilde{X}$ is a complete metric space where $X\subseteq \widetilde{X}$ and $\overline{X} = \widetilde{X}$.
\end{definition}
Since $X^{\ast\ast}$ is complete, we can make a completion of $X$ using the canonical embedding.
\begin{definition}[Norm Completion]
  Let $X$ be a normed vector space. A norm completion of $X$ is a pair $\left(Z,j\right)$, with $Z$ is a Banach space and $j: X\hookrightarrow Z$ is a linear isometry such that $\overline{\Ran}\left(j\right)= Z$.
\end{definition}
First, we show the existence of a norm completion. Then, we will show that norm completions are unique up to isometric isomorphism.
\begin{proposition}
  Let $X$ be a normed vector space. Set $\widetilde{X} = \overline{\iota_X(X)}^{\norm{\cdot}_{\text{op}}}\subseteq X^{\ast\ast}$, where $\iota_X$ is the canonical embedding. The pair, $\left(\widetilde{X},\iota_X\right)$, is a completion of $X$.
\end{proposition}
\begin{proof}
  Since $X^{\ast\ast}$ is a Banach space, the closed subspace $\overline{\iota_{X}(X)}^{\norm{\cdot}_{\text{op}}}$ is complete. Since $\iota_X$ is an isometric isomorphism, and $\overline{\Ran}\left(\iota_X\right) = \overline{\iota_X(X)}^{\norm{\cdot}_{\text{op}}}$.
\end{proof}
\begin{note}
  If $X$ is a Banach space, $\Ran\left(\iota_X\right)$ is already norm-closed, since $X$ is complete, so $\widetilde{X} = \iota_X(X) \cong X$.
\end{note}
To show that completions are unique up to isometric isomorphism, we need to be able to extend bounded linear operators from normed spaces to their completions that preserve the operator norm.
\begin{proposition}
  Let $X$ be a normed space with $E\subseteq X$ a linear subspace.
  \begin{enumerate}[(1)]
    \item The closure $\overline{E}$ is a closed linear subspace.
    \item If $T: E\rightarrow Z$ is a bounded linear map into a Banach space $Z$, then there is a unique bounded linear map $\widetilde{T}: \overline{E}\rightarrow Z$ such that $\widetilde{T}|_{E} = T$ and $\norm{\widetilde{T}}_{\text{op}} = \norm{T}_{\text{op}}$. If $T$ is an isometry, then so is $\widetilde{T}$, and $\Ran\left(\widetilde{T}\right) = \overline{\Ran}(T)$.
  \end{enumerate}
\end{proposition}
\begin{proof}\hfill
  \begin{enumerate}[(1)]
    \item Let $\left(x_n\right)_n\rightarrow x$ and $\left(y_n\right)_n\rightarrow y$ in $\overline{E}$. Then, $x + \alpha y = \lim_{n\rightarrow\infty}\left(x_n\right)_n + \alpha \lim_{n\rightarrow\infty}\left(y_n\right)_n$, or $x + \alpha y = \lim_{n\rightarrow\infty}\left(x_n + \alpha y_n\right)_n$, meaning $x + \alpha y \in \overline{E}$.
    \item Let $x\in \overline{E}$. There is a sequence $\left(x_n\right)_{n}$ in $E$ with $\left(x_n\right)_n\rightarrow x$. Then,
      \begin{align*}
        \norm{T\left(x_n\right) - T\left(x_m\right)} &= \norm{T\left(x_n - x_m\right)}\\
                                                     &\leq \norm{T}\norm{x_n - x_m}.
      \end{align*}
      Thus, $\left(T\left(x_n\right)\right)_n$ is a Cauchy sequence in $Z$, meaning it converges to $z\in Z$.\newline

      If $\left(x_n'\right)_n$ is a different sequence in $E$ with $\left(x_n'\right)_n\rightarrow x$, then similarly, $\left(T\left(x_n'\right)\right)_n$ is Cauchy in $Z$ that converges to $z'$. We have
      \begin{align*}
        z &= \lim_{n\rightarrow \infty}T\left(x_n\right)\\
          &= \lim_{n\rightarrow\infty}\left(T\left(x_n\right) - T\left(x'_n\right) + T\left(x'_n\right)\right)\\
          &= \lim_{n\rightarrow\infty}\left(T\left(x_n\right) - T\left(x'_n\right)\right) + \lim_{n\rightarrow\infty}T\left(x_n'\right)\\
          &= z'.
      \end{align*}
      We define $\widetilde{T}: \overline{E}\rightarrow Z$ with $\widetilde{T}(x) := \lim_{n\rightarrow\infty}T\left(x_n\right)$ for any sequence $\left(x_n\right)_n$ in $E$ that converges to $x$.\newline

      Linearity of $\widetilde{T}$ follows from the linearity of limits.\newline

      We now verify that $\norm{\widetilde{T}}_{\text{op}} = \norm{T}_{\text{op}}$.
      \begin{align*}
        \norm{\widetilde{T}(x)} &= \norm{\lim_{n\rightarrow\infty}T\left(x_n\right)}\\
                                &= \lim_{n\rightarrow\infty}\norm{T\left(x_n\right)}\\
                                &\leq \lim_{n\rightarrow\infty}\norm{T}\norm{x_n}\\
                                &= \norm{T}\norm{x},
      \end{align*}
      implying that $\norm{\widetilde{T}}_{\text{op}}\leq \norm{T}_{\text{op}}$. Additionally, since $\widetilde{T}$ is defined on a larger space than $T$, $\norm{T}_{\text{op}}\leq \norm{\widetilde{T}}_{\text{op}}$. Uniqueness follows from the fact that two continuous functions defined on a dense subset agree on the whole set.\newline

      Additionally, if $T$ is an isometry, then
      \begin{align*}
        \norm{\widetilde{T}\left(x\right)} &= \norm{\lim_{n\rightarrow\infty}T\left(x_n\right)}\\
                                           &= \lim_{n\rightarrow\infty}\norm{T\left(x_n\right)}\\
                                           &= \lim_{n\rightarrow\infty}\norm{x_n}\\
                                           &= \norm{x}.
      \end{align*}
      By definition, we can see that $\Ran(T) \subseteq \Ran\left(\widetilde{T}\right) \subseteq \overline{\Ran}(T)$. Since $\widetilde{T}$ is an isometry, its range is closed, so we have $\Ran\left(\widetilde{T}\right) = \overline{\Ran}(T)$.
  \end{enumerate}
\end{proof}
\begin{theorem}
  Let $X$ be a normed vector space. There is a Banach space $\widetilde{X}$ and a linear isometry $j: X\hookrightarrow \widetilde{X}$ such that $\overline{j(X)} = \widetilde{X}$. If $\left(Z,k\right)$ is any other completion of $X$, then $Z\cong X$ are isometrically isomorphic.
\end{theorem}
\begin{proof}
  We have shown existence with the earlier proposition.\newline

  Let $\left(Z_1,j_1\right)$ and $\left(Z_2,j_2\right)$ be norm completions of $X$. Consider the mappings $T: \Ran\left(j_1\right) \rightarrow Z_2$ and $S: \Ran\left(j_2\right) \rightarrow Z_1$ given by
  \begin{align*}
    T\left(j_1(x)\right) &= j_2(x)\\
    S\left(j_2(x)\right) &= j_1(x).
  \end{align*}
  Since $j_1$ and $j_2$ are isometries, then $T$ and $S$ are well-defined linear isometries. Thus, $T$ and $S$ extend to isometries $T: Z_1\rightarrow Z_2$ and $S: Z_2\rightarrow Z_1$.\newline

  We have $S\circ T(z) = z$ for all $z\in \Ran\left(j_1\right)$, meaning $S\circ T = \id_{Z_1}$, and similarly, $T\circ S = \id_{Z_2}$. Thus, $Z_1\cong Z_2$ are isometrically isomorphic.
\end{proof}
\begin{corollary}
  Let $Z$ be a Banach space with $X\subseteq Z$ a subspace. Then, $\widetilde{X} \cong \overline{X}$.
\end{corollary}
\begin{remark}
  For a normed space $\left(X,\norm{\cdot}\right)$, we let $\overline{X}^{\norm{\cdot}}$ denote the norm completion $\widetilde{X}$, and $X\subseteq \overline{X}^{\norm{\cdot}}$ is considered as a dense subspace.
\end{remark}
We are now able to extend bounded linear operators to the completion.
\begin{proposition}
  Let $X$ and $Y$ be normed spaces. If $T\in \mathcal{B}\left(X,Y\right)$, there is a unique $\widetilde{T}\in \mathcal{B}\left(\widetilde{X},\widetilde{Y}\right)$ such that $\widetilde{T}\circ \iota_X = \iota_Y\circ T$. The following diagram commutes.
  \begin{center}
    % https://tikzcd.yichuanshen.de/#N4Igdg9gJgpgziAXAbVABwnAlgFyxMJZABgBpiBdUkANwEMAbAVxiRAB12B3LWPB2MAAaAXxAjS6TLnyEUARnJVajFm048+WATGABNMRKnY8BImXnL6zVohBDxkkBhOyiiy9Wtq7e8cpgoAHN4IlAAMwAnCABbJAAmahwIJABmL1VbEAAVRwjouMQyEGSkRRUbdW5eGH5BbMMnKNiEpJSijMq7TnwcOgB9ByMQZsL0kvbyhjoAIxgGAAVpUzkQSKwggAscEE6fDnZegb8RChEgA
\begin{tikzcd}
\widetilde{X} \arrow[r, "\widetilde{T}"] & \widetilde{Y}           \\
X \arrow[r, "T"] \arrow[u, "\iota_X",hookrightarrow]    & Y \arrow[u, "\iota_Y"',hookrightarrow]
\end{tikzcd}
Additionally, $\norm{T}_{\text{op}} = \norm{\widetilde{T}}_{\text{op}}$. If $T$ is an isometry, then so is $\widetilde{T}$, and if $T$ is an isometric isomorphism, so is $\widetilde{T}$.
  \end{center}
\end{proposition}
\begin{proof}
  Define $T_0: \Ran\left(\iota_X\right)\rightarrow \tilde{Y}$ by $T_0\left(\iota_X\left(x\right)\right) = \iota_Y\left(T(x)\right)$. Since $\iota_X$ is injective, $T_0$ is well-defined and linear. Additionally, since $\iota_X$ and $\iota_Y$ are isometries, for every $x\in X$, we have
  \begin{align*}
    \norm{T_0\left(\iota_X\left(x\right)\right)} &= \norm{\iota_Y\left(T(x)\right)}\\
                                                 &= \norm{T(x)}\\
                                                 &\leq \norm{T}_{\text{op}}\norm{x}\\
                                                 &= \norm{T}_{\text{op}}\norm{\iota_X(x)},
  \end{align*}
  meaning $T_0$ is bounded. Thus, $T_0$ extends to $\widetilde{T}$ defined on $\overline{\Ran}\left(\iota_X\right) = \tilde{X}$, such that $\norm{\widetilde{T}}_{\text{op}} = \norm{T_0}_{\text{op}} \leq \norm{T}_{\text{op}}$. It is also the case that $\norm{T}_{\text{op}}\leq \norm{\widetilde{T}}_{\text{op}}$ since $\widetilde{T}$ is an extension.\newline

  Thus, the diagram commutes. Uniqueness follows from the fact that $\Ran\left(\iota_X\right)\subseteq \widetilde{X}$ is dense.\newline

  If $T$ is isometric, then
  \begin{align*}
    \norm{T_0\left(\iota_X\left(x\right)\right)} &= \norm{\iota_Y\left(T\left(x\right)\right)}\\
                                                 &= \norm{T(x)}\\
                                                 &= \norm{x}\\
                                                 &= \norm{\iota_X\left(x\right)},
  \end{align*}
  meaning $T_0$ is an isometry. Thus, the extension to $\widetilde{T}: \widetilde{X}\rightarrow \widetilde{Y}$ is also an isometry.\newline

  If $T$ is an isometric isomorphism, then
  \begin{align*}
    \Ran\left(\widetilde{T}\right) &= \overline{\Ran}\left(T_0\right)\\
                                   &= \overline{\iota_Y\left(\Ran\left(T\right)\right)}\\
                                   &= \overline{\iota_Y\left(Y\right)}\\
                                   &= \widetilde{Y},
  \end{align*}
  meaning $\widetilde{T}$ is an isomorphism as well.
\end{proof}
\begin{corollary}
  Let $X$ be a vector space and $Y$ a Banach space. Suppose there is an injective linear map $T: X\rightarrow Y$, and we write $\norm{\cdot}_X$ to be the norm on $X$ induced by $T$, defined by $\norm{x}_X = \norm{T(x)}_Y$. If $Z:=\overline{T(X)}$ is the closure of the range of $T$ in $Y$, then $\widetilde{X} \cong Z$.
\end{corollary}
\begin{proof}
  By the definition of the norm, $T: X\rightarrow Z$ is an isometry into the Banach space $Z$, and so extends to $\widetilde{T}: \widetilde{X}\rightarrow Z$, where $\widetilde{T}|_{X} = T$. Since $T$ is an isometry and $\widetilde{X}$ is complete, the range $\widetilde{T}\left(\widetilde{X}\right)$ is closed in $Y$, meaning
  \begin{align*}
    T(X) &\subseteq \widetilde{T}\left(\widetilde{X}\right)\\
    \overline{T(X)} &\subseteq \widetilde{T}\left(\widetilde{X}\right).
  \end{align*}
  Thus, $\widetilde{T}$ is surjective.
\end{proof}
\subsection{Dual Spaces}%
\begin{definition}
  A normed vector space $X$ is called a dual space if there is a normed vector space $Z$ such that $Z^{\ast}\cong X$ are isometrically isomorphic. The space $Z$ is known as the pre-dual of $X$.
\end{definition}
It is generally not easy to find whether or not $X$ is a dual space. However, we can attempt to find some dual spaces that for various Banach spaces.\newline

A linear map is uniquely defined on the Hamel basis of its domain. However, when dealing with infinite-dimensional Banach spaces, we know that they cannot have a countable Hamel basis, which makes working with them very difficult. However, we can use a different notion of basis to assist us in our study.
\begin{definition}
  Let $X$ be a Banach space over $\F$. A sequence $\left(e_n\right)_n$ in $X$ is called a Schauder basi for $X$ if every $x\in X$ admits a unique sequence $\left(\lambda_n\right)_n$ in $\F$ such that
  \begin{align*}
    x &= \sum_{n=1}^{\infty}\lambda_ne_n
  \end{align*}
  is a norm convergent sum in $X$.
\end{definition}
\begin{proposition}
  The canonical coordinate vectors $\left(e_n\right)_n$\footnote{The sequence with $1$ at position $n$ and $0$ everywhere else.} is a Schauder basis for $c_0$ and $\ell_p$ for $1 \leq p < \infty$.
\end{proposition}
\begin{proof}
  We will show this for $c_0$. Let $x\in c_0$, $x = \left(\lambda_n\right)_n$. For each $N\geq 1$, define
  \begin{align*}
    x_n &= \sum_{n=1}^{N}\lambda_ne_n.
  \end{align*}
  Whenever $N\geq M$, we have
  \begin{align*}
    \norm{x_N - x_M}_{\infty} &= \norm{\sum_{n=M+1}^{N}\lambda_ne_n}_{\infty}\\
                     &=\sup_{M+1 \leq n \leq N}\left\vert \lambda_n \right\vert.
  \end{align*}
  Since $\left(\lambda_n\right)_n\rightarrow 0$ as $x\in c_0$, we have $\norm{x_N - x_M}_{\infty}$ is small for large $N,M$. Thus, $\left(x_N\right)_N$ is Cauchy.\newline

  By the completeness of $c_0$, we know that $\sum_{n=1}^{\infty}\lambda_ne_n$ thus converges. Since
  \begin{align*}
    \norm{x-x_N}_{\infty} &= \sup_{n\geq N+1} \left\vert \lambda_n \right\vert\\
                          &\rightarrow 0,
  \end{align*}
  we have $x = \sum_{n=1}^{\infty}\lambda_ne_n$. Thus, we have shown existence.\newline

  To show uniqueness, let $\varphi_k\in c_0^{\ast}$ be given by $\varphi_k\left(\left(z_n\right)_n\right) = z_k$ for each $k\in \N$. Suppose that there exist two sequences of coefficients $\left(\lambda_n\right)_n$ and $\left(\mu_n\right)_n$ such that
  \begin{align*}
    x &= \sum_{n=1}^{\infty}\lambda_ne_n\\
      &= \sum_{n=1}^{\infty}\mu_ne_n.
  \end{align*}
  For each $k$, since $\varphi_k$ is continuous, we have
  \begin{align*}
  \lambda_k &= \sum_{n=1}^{\infty}\lambda_n\varphi_k\left(e_n\right)\\
            &= \varphi_k\left(\sum_{n=1}^{\infty}\lambda_n e_n\right)\\
            &= \varphi_k\left(x\right)\\
            &= \varphi_k\left(\sum_{n=1}^{\infty}\mu_ne_n\right)\\
            &= \sum_{n=1}^{\infty}\mu_n\varphi_k\left(e_n\right)\\
            &= \mu_k.
  \end{align*}
  Thus, $\left(\lambda_n\right)_n = \left(\mu_n\right)_n$.
\end{proof}
\begin{exercise}
  Every Banach space that admits a Schauder basis is separable. Conclude that the canonical sequence of coordinate vectors $\left(e_n\right)_n$ is not a Schauder basis for $\ell_{\infty}$.
\end{exercise}
\begin{solution}
  Let $X$ admit a Schauder basis $\left(e_n\right)_n$, and let $\left(\lambda_n\right)_n$ be a sequence of coefficients in $\C$ such that
  \begin{align*}
    x &= \sum_{n=1}^{\infty}\lambda_ne_n.
  \end{align*}
  Define
  \begin{align*}
    E &= \set{\sum_{k=1}^{\infty}\alpha_ke_k\mid \alpha_k\in \C_{\Q}}.
  \end{align*}
  Since the coefficients in $E$ are defined over $\C_{\Q}$, $E$ is countable. We claim that $\overline{E} = X$.\newline

  Let $\ve > 0$. Since $\overline{\C_{\Q}} = \C$, we find $\alpha_1\in \C_{\Q}$ such that $\norm{\left(\alpha_1 - \lambda_1\right)e_1} < \frac{\ve}{2}$, and inductively find $\alpha_k\in \C_{\Q}$ such that
  \begin{align*}
    \norm{\left(\alpha_k - \lambda_k\right)e_k} < \frac{\ve}{2^k}.
  \end{align*}
  Define $\mu = \sum_{k=1}^{\infty}\alpha_ke_k$. Then,
  \begin{align*}
    \norm{\mu - x} &= \norm{\sum_{k=1}^{\infty}\left(\alpha_k - \lambda_k\right)e_k}\\
                   &\leq \sum_{k=1}^{\infty}\norm{\left(\alpha_k - \lambda_k\right)e_k}\\
                   &< \ve.
  \end{align*}
  Since $\ell_{\infty}$ is not separable,\footnote{The set $E = \set{\left(a_k\right)_k\mid a_k\in \set{0,1}}$ under $\norm{\cdot}_{\infty}$ has the discrete metric but is uncountable (hence, not separable), but any subset of a separable set is necessarily separable.} it is the case that $\ell_{\infty}$ does not admit a Schauder basis.
\end{solution}
We can now find some dual spaces.
\begin{proposition}
  The dual space of $c_0$ is isometrically isomorphic to $\ell_{1}$. That is, $c_0^{\ast} \cong \ell_1$.
\end{proposition}
\begin{proof}
  Let $\theta: \ell_1\rightarrow c_0^{\ast}$ be defined by $\theta\left(\lambda\right)\left(z\right) = \sum_{n=1}^{\infty}\lambda_nz_n$, where $\lambda = \left(\lambda_n\right)_n\in \ell_1$ and $z = \left(z_n\right)_n \in c_0$. We will show that $\theta$ is an isometric isomorphism.\newline

  We start by showing that $\theta$ is well-defined. To see this, note that
  \begin{align*}
    \sum_{n=1}^{\infty}\left\vert \lambda_nz_n \right\vert &\leq \sum_{n=1}^{\infty}\left\vert \lambda_n \right\vert\norm{z}_{\infty}\\
                                                           &= \norm{\lambda}_1\norm{z}_{\infty},
  \end{align*}
  which is finite since $\lambda\in \ell_1$ and $z\in \ell_{\infty}$. Thus, $\sum_{n=1}^{\infty}\lambda_nz_n$ converges absolutely, and thus is convergent.\newline

  For each $\lambda\in \ell_1$, the map $\theta\left(\lambda\right): c_0\rightarrow \F$ is linear. We now need to show that $\theta$ is bounded.\newline

  Let $z\in c_0$. Then,
  \begin{align*}
    \left\vert \theta\left(\lambda\right)\left(z\right) \right\vert &= \left\vert \sum_{n=1}^{\infty}\lambda_nz_n \right\vert\\
                                                                    &\leq \sum_{n=1}^{\infty}\left\vert \lambda_n \right\vert\left\vert z_n \right\vert\\
                                                                    &\leq \sum_{n=1}^{\infty}\left\vert \lambda_n \right\vert\norm{z}_{\infty}\\
                                                                    &= \norm{\lambda}_1\norm{z}_{\infty}.
  \end{align*}
  Thus, $\norm{\theta\left(\lambda\right)}_{\text{op}}\leq \norm{\lambda}_1$, so $\theta\left(\lambda\right) \in c_0^{\ast}$.\newline

  The map $\lambda \mapsto \theta\left(\lambda\right)$ is linear. Now, we show that $\theta$ is an isometry. We know that $\norm{\theta\left(\lambda\right)}_{\text{op}}\leq \norm{\lambda}_{1}$. For $\left(\lambda_n\right)_n\in \ell_1$, we know know that
  \begin{align*}
    z_{\lambda} &= \left(\sgn\left(\lambda_1\right),\sgn\left(\lambda_1\right),\dots,\sgn\left(\lambda_n\right)\right)
  \end{align*}
  is a member of $B_{c_0}$, as its nonzero terms have modulus at most $1$. Here, $\sgn\left(z\right)$ is the complex number of modulus $1$ defined by $\sgn(z) = \frac{|z|}{z}$. Thus,
  \begin{align*}
    \norm{\theta\left(\lambda\right)}_{\text{op}} &\geq \left\vert \theta\left(\lambda\right)\left(z_1\right) \right\vert\\
                                                  &= \left\vert \sum_{i=1}^{n}\lambda_i\sgn\left(\lambda_i\right) \right\vert\\
                                                  &= \sum_{i=1}^{n}\left\vert \lambda_i \right\vert.
  \end{align*}
  Sending $n\rightarrow\infty$, we have $\norm{\theta\left(\lambda\right)}_{\text{op}} \geq \norm{\lambda}_1$. Thus, $\theta$ is an isometry.\newline

  Finally, we must show that $\theta$ is onto. Let $\varphi\in c_0^{\ast}$, and let $\lambda_n = \varphi\left(e_n\right)$. We set $\omega_n = \sgn\left(\varphi\left(e_n\right)\right)$ Then,
  \begin{align*}
    \sum_{n=1}^{N}\left\vert \lambda_n \right\vert &= \sum_{n=1}^{N}\left\vert \varphi\left(e_n\right) \right\vert\\
                                                   &= \sum_{n=1}^{N}\varphi_n\sgn\left(\varphi\left(e_n\right)\right)\\
                                                   &= \sum_{n=1}^{N}\varphi_n\left(e_n\omega_n\right)\\
                                                   &= \varphi\left(\sum_{n=1}^{N}\omega_ne_n\right)\\
                                                   &\leq \norm{\varphi}_{\text{op}}\norm{\sum_{n=1}^{N}\omega_ne_n}_{\infty}\\ %Note that here, e_n are the canonical coordinate vectors on c_0, and \omega_n are all complex numbers of modulus 1.
                                                   &\leq \norm{\varphi}_{\text{op}}.
  \end{align*}
  Sending $N\rightarrow\infty$, we find that $\left(\lambda_n\right)_n\in \ell_1$. For an arbitrary $\left(z_n\right)_n\in c_0$, we have
  \begin{align*}
    \varphi\left(z\right) &= \varphi\left(\sum_{z=1}^{\infty}z_ne_n\right)\\
                          &= \sum_{n=1}^{\infty}z_n\varphi\left(e_n\right)\\
                          &= \sum_{n=1}^{\infty}z_n\lambda_n\\
                          &= \theta\left(\lambda\right)\left(z\right),
  \end{align*}
  thus meaning $\varphi$ and $\theta\left(\lambda\right)$ agree on an arbitrary member of $c_0$.
\end{proof}
\begin{remark}
  What made this problem so much easier than it might have been is the fact that since $c_0$ has a Schauder basis, we could define how $\varphi\in c_0^{\ast}$ behaved entirely through its action on the basis elements. This is similar to how we can find properties of linear maps in linear algebra by finding their actions on the Hamel bases of the vector space.
\end{remark}
\begin{exercise}
  Consider the map $\phi: \ell_1\rightarrow \ell_{\infty}^{\ast}$ defined by $\phi\left(\lambda\right)\left(z\right) = \sum_{n=1}^{\infty}\lambda_nz_n$ for $\lambda = \left(\lambda_n\right)_n\in \ell_1$ and $z = \left(z_n\right)_n\in \ell_{\infty}$. Prove that $\phi$ is a linear isometry, but $\phi$ is \textit{not} onto.
\end{exercise}
\begin{solution}
  A similar method to the proof that $c_0^{\ast}\cong \ell_1$ gives $\phi$ is an isometry. We will replicate it here.\newline

  First, we show that $\phi$ is bounded and well-defined.
  \begin{align*}
    \left\vert \phi\left(\lambda\right)\left(z\right) \right\vert &= \left\vert \sum_{n=1}^{\infty}\lambda_nz_n \right\vert\\
                                                                  &\leq \sum_{n=1}^{\infty}\left\vert \lambda_nz_n \right\vert\\
                                                                  &\leq \sum_{n=1}^{\infty}\left\vert \lambda_n \right\vert\norm{z}_{\infty}\\
                                                                  &= \norm{\lambda}_1\norm{z}_{\infty}\\
                                                                  &< \infty.
  \end{align*}
  Thus, $\norm{\phi\left(\lambda\right)}_{\text{op}}\leq \norm{\lambda}_1$, so $\phi$ is bounded. It is also clear that $\phi$ is linear in $\lambda$ and in $z$. Similarly, for $\lambda = \left(\lambda_n\right)_n\in \ell_1$, we define
  \begin{align*}
    z_{\lambda} &= \left(\sgn\left(\lambda_1\right),\dots,\sgn\left(\lambda_n\right),0,0,\dots\right)\\
                &\in B_{c_0}.
  \end{align*}
  Thus, 
  \begin{align*}
    \norm{\phi\left(\lambda\right)}_{\text{op}} &\geq \left\vert \phi\left(\lambda\right)\left(z_{\lambda}\right) \right\vert\\
                                                &= \sum_{k=1}^{n}\left\vert \lambda_k \right\vert,
  \end{align*}
  meaning $\norm{\phi\left(\lambda\right)}_{\text{op}} \geq \norm{\lambda}_1$, so $\phi$ is an isometry.\newline

  In particular, $\phi$ is an isometric isomorphism from $\ell_1$ to $c_0^{\ast}\subset \ell_{\infty}^{\ast}$, so $\phi$ cannot be onto $\ell_{\infty}^{\ast}$.
\end{solution}
\begin{proposition}
  The dual of $\ell_1$ is isometrically isomorphic to $\ell_{\infty}$.
\end{proposition}
\begin{proof}
  Let $\theta: \ell_{\infty}\rightarrow \ell_1^{\ast}$ be defined by $\theta\left(z\right)\left(\lambda\right) = \sum_{n=1}^{\infty}z_n\lambda_n$ for $z = \left(z_n\right)_n\in \ell_{\infty}$ and $\left(\lambda_n\right)_n\in \ell_1$.\newline

  We can see that $\theta$ is well-defined with $\norm{\theta(z)}_{\text{op}}\leq \norm{z}_{\infty}$ by a similar reasoning to the case of $\theta: \ell_1\rightarrow c_0^{\ast}$. Additionally, since
  \begin{align*}
    \norm{\theta(z)}_{\text{op}} &\geq \left\vert \theta(z)\left(e_n\right) \right\vert\\
                                 &= \left\vert z_n \right\vert
  \end{align*}
  for every $n\geq 1$, we have $\norm{z}_{\infty}\leq \norm{\theta(z)}_{\text{op}}$, meaning $\theta$ is an isometry.\newline

  Suppose $\varphi\in \ell_1^{\ast}$. Set $z_n = \varphi\left(e_n\right)$. Note that $z\in \ell_{\infty}$ since $\left\vert z_n \right\vert = \left\vert \varphi\left(e_n\right) \right\vert \leq \norm{\varphi}_{\text{op}}$ for all $n\geq 1$. For $\lambda = \left(\lambda_n\right)_n\in \ell_1$, then $\lambda = \sum_{n=1}^{\infty}\lambda_ne_n$ is convergent in the $1$-norm, as
  \begin{align*}
    \norm{\sum_{n=1}^{\infty}\lambda_ne_n} &\leq \sum_{n=1}^{\infty}\norm{\lambda_ne_n}\\
                                           &= \sum_{n=1}^{\infty}\left\vert \lambda_n \right\vert\norm{e_n}\\
                                           &= \sum_{n=1}^{\infty}\left\vert \lambda_n \right\vert\\
                                           &< \infty.
  \end{align*}
  Thus, 
  \begin{align*}
    \varphi\left(\lambda\right) &= \varphi\left(\sum_{n=1}^{\infty}\lambda_ne_n\right)\\
                                &= \sum_{n=1}^{\infty}\lambda_n\varphi\left(e_n\right)\\
                                &= \sum_{n=1}^{\infty}\lambda_nz_n\\
                                &= \theta\left(z\right)\left(\lambda\right).
  \end{align*}
\end{proof}
This result can be generalized to $L_{p}$ spaces, which are spaces of measurable functions satisfying some integration property.
\begin{theorem}
  Let $\left(\Omega,\mathcal{M},\mu\right)$ be a measure space.
  \begin{enumerate}[(1)]
    \item If $\mu$ is a $\sigma$-finite measure, and $p,q\in \left(1,\infty\right)$ satisfying $p^{-1} + q^{-1} = 1$, then the map $L_q\left(\Omega,\mu\right)\rightarrow L_p\left(\Omega,\mu\right)^{\ast}$ given by $f \mapsto \varphi_f$, with
      \begin{align*}
        \varphi_f\left(g\right) &= \int_{\Omega}gf\:d\mu,
      \end{align*}
      is an isometric isomorphism.
    \item If $\mu$ is semi-finite, then $L_{\infty}\rightarrow L_{1}\left(\Omega,\mu\right)^{\ast}$ given by $f\mapsto \varphi_f$ is an isometric isomorphism.
  \end{enumerate}
\end{theorem}
%\subsubsection{Duals of Continuous Function Spaces}%
We will now turn our attention to spaces of continuous functions. Let $\Omega$ be a locally compact Hausdorff space. We want to understand $C_0\left(\Omega\right)^{\ast}$, the space of bounded linear functionals on $C_0\left(\Omega\right)$.\newline

Note that $C_0\left(\Omega\right)$ is a $\ast$-algebra, and admits a cone of positive elements, $C_0\left(\Omega\right)_{+}$.\footnote{$C_0\left(\Omega\right)_{+} = \set{f\in C_0\left(\Omega\right)\mid f(x)\geq 0~\forall x\in \Omega}$.} We start by understanding the properties of positive linear functionals.
\begin{definition}
  Let $\Omega$ be a locally compact Hausdorff space.
  \begin{enumerate}[(1)]
    \item A linear functional $\varphi: C_0\left(\Omega\right)\rightarrow \C$ is called positive if $\varphi(f) \geq 0$ whenever $f\in C_0\left(\Omega\right)_{+}$.
    \item A positive linear functional $\varphi: C_0\left(\Omega\right)\rightarrow \C$ is called faithful if
      \begin{align*}
        \ker\left(\varphi\right)\cap C_0\left(\Omega\right)_{+} &= \set{0}.
      \end{align*}
    \item A state on $C_0\left(\Omega\right)$ is a positive linear functional $\varphi\in C_0\left(\Omega\right)'$ with $\norm{\varphi}_{\text{op}} = 1$. The collection of states
      \begin{align*}
        S\left(C_0\left(\Omega\right)\right) &= \set{\varphi\in C_0\left(\Omega\right)^{\ast}\mid \text{ $\varphi$ is a state}}\\
                                             &\subseteq B_{C_{0}\left(\Omega\right)^{\ast}}.
      \end{align*}
      is called the state space of $\Omega$, which we abbreviate $S\left(\Omega\right)$.
  \end{enumerate}
\end{definition}
\begin{lemma}
  Let $\Omega$ be a locally compact Hausdorff space, and suppose $\varphi: C_0\left(\varphi\right)\rightarrow \C$ is a positive linear functional.
  \begin{enumerate}[(1)]
    \item If $h\in C_0\left(\Omega,\R\right)$, then $\varphi\left(h\right)\in \R$.
    \item If $h\leq k$ in $C_0\left(\Omega,\R\right)$, then $\varphi\left(h\right) \leq \varphi\left(k\right)$.
    \item If $f\in C_0\left(\Omega\right)$, then $\varphi\left(f^{\ast}\right) = \overline{\varphi\left(f\right)}$.
  \end{enumerate}
\end{lemma}
\begin{proof}\hfill
  \begin{enumerate}[(1)]
    \item We write $h = h_{+} - h_{-}$, and see that $\varphi(h) = \varphi\left(h_{+}\right) - \varphi\left(h_{-}\right) \in \R$.
    \item If $h \leq k$ in $C_0\left(\Omega,\R\right)$, then $k-h\geq 0$, so $\varphi\left(k-h\right) \geq 0$, so $\varphi\left(h\right)\leq \varphi\left(k\right)$.
      \item We write $f$ as the Cartesian decomposition, $f = h + ik$, where $h,k\in C_0\left(\Omega,\R\right)$. Thus,
        \begin{align*}
          \varphi\left(f^{\ast}\right) &= \varphi\left(h-ik\right)\\
                                       &= \varphi\left(h\right) - i\varphi\left(k\right)\\
                                       &= \overline{\varphi(h) + i\varphi{k}}\\
                                       &= \overline{\varphi(f)}.
        \end{align*}
  \end{enumerate}
\end{proof}
\begin{proposition}
  If $\varphi: C_0\left(\Omega\right)\rightarrow \C$ is positive and linear, then $\varphi$ is bounded.
\end{proposition}
\begin{proof}
  Let
  \begin{align*}
    M &= \sup\set{\varphi\left(f\right)\mid f\in C_0\left(\Omega\right)_{+},\norm{f}_{u}\leq 1}.
  \end{align*}
  We claim that $M$ is finite. If $M = \infty$< then for each $n\in \N$, there exists $f_n\in C_0\left(\Omega\right)_{+}$ with $\norm{f_n}_{u}\leq 1$ and $\varphi\left(f_n\right) \geq 2^n$. Note that
  \begin{align*}
    \sum_{n=1}^{\infty}\norm{2^{-n}f_n}_{u} &\leq \sum_{n=1}^{\infty}2^{-n}\\
                                            &= 1,
  \end{align*}
  meaning the series $f = \sum_{n=1}^{\infty}2^{-n}f_n$ converges uniformly in $C_0\left(\Omega\right)$.\newline

  For $N\in \N$, we can see that $f\geq \sum_{n=1}^{N}2^{-n}f_n \geq 0$, as $f_n$ are all positive. Since $\varphi$ preserves the order on $C_0\left(\Omega\right)$, we have
  \begin{align*}
    \varphi(f) &\geq \varphi\left(\sum_{n=1}^{N}2^{-n}f_n\right)\\
               &= \sum_{n=1}^{n}2^{-n}\varphi\left(f_n\right)\\
               &\geq \sum_{n=1}^{N}2^{-n}2^{n}\\
               &= N.
  \end{align*}
  This is absurd since $N$ is arbitrary. Thus, $M < \infty$.\newline

  Let $f\in C_0\left(\Omega\right)$ with $\norm{f}_u \leq 1$. Let $f = h + ik$ be the Cartesian decomposition. Then, we can write $f = \left(h_{+} - h_{-}\right) + i\left(k_{+} - k_{-}\right)$. Since $\varphi$ is linear, $\left\vert \varphi(f) \right\vert \leq 4M$, meaning $\norm{\varphi}_{\text{op}} \leq 4M$.
\end{proof}
\begin{proposition}
  If $\Omega$ is a compact Hausdorff space, and $\varphi: C\left(\Omega\right)\rightarrow \F$ is linear, then $\varphi$ is positive if and only if $\norm{\varphi} = \varphi\left(\1_{\Omega}\right)$.
\end{proposition}
\begin{proof}
  Let $h\in C\left(\Omega,\R\right)$. Then,
  \begin{align*}
    -\norm{h}\1_{\Omega}\leq h \leq \norm{h}\1_{\Omega}.
  \end{align*}
  Thus, since $\varphi$ is positive,
  \begin{align*}
    -\norm{h}\varphi\left(\1_{\Omega}\right) \leq \varphi(h) \leq \norm{h}\varphi\left(\1_{\Omega}\right).
  \end{align*}
  Thus, $\left\vert \varphi(h) \right\vert\leq \norm{h}\varphi\left(\1_{\Omega}\right)$.\newline

  Let $f\in C\left(\Omega\right)$, and find $\omega\in \C$ such that $\left\vert \omega \right\vert = 1$ and $\omega\varphi(f) = \left\vert \varphi(f) \right\vert$. Set $g = \omega f$. Note that $\norm{g}_{u} = \norm{f}_u$. Then,
  \begin{align*}
    \left\vert \varphi(f) \right\vert &= \re\left(\left\vert \varphi(f) \right\vert\right)\\
                                      &= \re\left(\omega\varphi(f)\right)\\
                                      &= \re\left(\varphi\left(\omega f\right)\right)\\
                                      &= \re\left(\varphi(g)\right)\\
                                      &= \frac{1}{2}\left(\varphi(g) + \overline{\varphi(g)}\right)\\
                                      &= \varphi\left(\frac{g + \overline{g}}{2}\right)\\
                                      &= \varphi\left(\re(g)\right)\\
                                      &\leq \varphi\left(\1_\Omega\right)\norm{\re(g)}_{u}\\
                                      &\leq \varphi\left(\1_\Omega\right)\norm{g}\\
                                      &= \varphi\left(\1_{\Omega}\right)\norm{f}_u,
  \end{align*}
  meaning $\norm{\varphi}_{\text{op}}\leq \varphi\left(\1_\Omega\right)$.\newline

  Suppose $\norm{\varphi} = \varphi\left(\1_{\Omega}\right)$. We start by claiming that if $h\in C\left(\Omega,\R\right)$, then $\varphi(h)\in \R$. Let $\varphi(h) = a + bi$; we will show that $b = 0$. Let $t\in \R$ be arbitrary.
  \begin{align*}
    \norm{h + it\1_{\Omega}}_{u}^2 &= \left(\sup_{x\in \Omega}\left\vert h(x) + it \right\vert\right)^2\\
                                   &= \sup_{x\in \Omega}\left\vert h(x) + it \right\vert^2\\
                                   &= \sup_{x\in \Omega}\left(\left(h(x)\right)^2 + t^2\right)\\
                                   &= \norm{h}_{u}^2 + t^2.
  \end{align*}
  Thus,
  \begin{align*}
    \left\vert \varphi\left(h + it\1_\Omega\right) \right\vert^2 &\leq \left(\varphi\left(\1_\Omega\right)\right)^2\norm{h + it\1_{\Omega}}^2\\
                                                                 &= \left(\varphi\left(\1_\Omega\right)\right)^2\left(\norm{h}^2 + t^2\right).
  \end{align*}
  Alternatively,
  \begin{align*}
    \left\vert \varphi\left(h + it\1_\Omega\right) \right\vert^2 &= \left\vert \varphi(h) + it\varphi\left(\1_\Omega\right) \right\vert^2\\
                                                                 &= \left\vert a + i\left(b + t\varphi\left(\1_\Omega\right)\right) \right\vert^2\\
                                                                 &= a^2 + \left(b + t\varphi\left(\1_\Omega\right)\right)^2\\
                                                                 &= a^2 + b^2 + t^2\left(\varphi\left(\1_\Omega\right)\right)^2 + 2bt\varphi\left(\1_\Omega\right).
  \end{align*}
  Thus, we get $a^2 + b^2 + 2bt\varphi\left(\1_\Omega\right) \leq \left(\varphi\left(\1_\Omega\right)\right)\norm{h}^2$. If $b \neq 0$, sending $t\rightarrow\infty$ yields a contradiction. Thus, $b = 0$.\newline

  Assume $f\in C\left(\Omega\right)_+$. Observe that $\Ran\left(f - \norm{f}\1_\Omega\right) \subseteq \left[-\norm{f},\norm{f}\right]$, meaning $\norm{f - \norm{f}\1_\Omega}\leq \norm{f}$. Thus,
  \begin{align*}
    \left\vert \varphi(f) - \norm{f}\varphi\left(\1_\Omega\right) \right\vert &= \left\vert \varphi\left(f - \norm{f}\1_\Omega\right) \right\vert\\
                                                                              &\leq \varphi\left(\1_\Omega\right)\norm{f - \norm{f}\1_\Omega}\\
                                                                              &\leq \varphi\left(\1_\Omega\right)\norm{f}.
  \end{align*}
  Since $\varphi(f)\in \R$, we have $\varphi(f) \geq 0$, so $\varphi$ is positive.
\end{proof}
\begin{corollary}
  Let $\Omega$ be a compact Hausdorff space. The state space
  \begin{align*}
    S\left(\Omega\right) &= \set{\varphi\in C\left(\Omega\right)^{\ast}\mid \varphi\left(\1_\Omega\right) = 1 = \norm{\varphi}}
  \end{align*}
  is convex.
\end{corollary}
\begin{proof}
  Suppose $\varphi_1 ,\varphi_2\in S(\Omega)$, $t\in [0,1]$. Then, $\left(1-t\right)\varphi_1 + t\varphi_2$ is a positive functional, meaning
  \begin{align*}
    \norm{\left(1-t\right)\varphi_1 + t\varphi_2} &= \left(1-t\right)\varphi\left(\1_\Omega\right) + t\varphi\left(\1_\Omega\right)\\
    &= \left(1-t\right) + t\\
    &= 1,
  \end{align*}
  so $S(\Omega)$ is convex.
\end{proof}
Now, we can associate bounded linear functionals with regular measures.
\begin{theorem}[Riesz Representation Theorem]
  Let $\Omega$ be a locally compact Hausdorff space. If $\varphi: C_c\left(\Omega\right) \rightarrow \C$ is a positive linear functional, then there exists a unique Radon measure $\mu$ such that
  \begin{align*}
    \varphi(f) &= \int_{\Omega}f\:d\mu,
  \end{align*}
  with $\left\vert \varphi(f) \right\vert \leq \norm{f}\mu\!\left(\supp(f)\right)$. The Radon measure $\mu$ also satisfies the following.
  \begin{enumerate}[(1)]
    \item For every open $U\subseteq \Omega$, we have
      \begin{align*}
        \mu\!\left(U\right) &= \sup\set{\varphi(f)\mid f\in C_c\left(\Omega,[0,1]\right),\supp(f)\subseteq U}.
      \end{align*}
    \item For every compact $K\subseteq \Omega$,
      \begin{align*}
        \mu\!\left(K\right) &= \inf\set{\varphi(f)\mid f\geq \1_{K}}.
      \end{align*}
  \end{enumerate}
\end{theorem}
The following proof is adapted from Folland's \textit{Real Analysis}.
\begin{notation}
  For an open set $U$, we let $f\prec U$ denote $f\in C_c\left(\Omega,[0,1]\right)$ and $\supp(f) \subset U$.
\end{notation}
\begin{proof}
  We start by establishing uniqueness.\newline

  Let $\mu$ be a Radon measure such that $\varphi(f) = \int_{\Omega}f\:d\mu$ for all $f\in C_c\left(\Omega\right)$. For $U\subseteq X$ open, it is the case that $\varphi(f) \leq \mu\!\left(U\right)$ for $f\prec U$.\newline

  If $K\subseteq U$ is compact, Urysohn's lemma gives $f\in C_c\left(\Omega\right)$ such that $f\prec U$ and $f = 1$ on $K$, meaning $\mu\!\left(K\right) \leq \int_{\Omega}f\:d\mu = \varphi(f)$. Since $\mu$ is inner regular on $U$, we satisfy condition (1). Thus, $\mu$ is determined on all open sets by $\varphi$, and so is determined on all Borel sets since $\mu$ is outer regular.\newline

  To show existence, we start by defining
  \begin{align*}
    \mu\!\left(U\right) &=\sup\set{\varphi\left(f\right)\mid f\in C_c\left(\Omega\right),f\prec U}.
  \end{align*}
  For any $E\subseteq X$, we define
  \begin{align*}
    \mu^{\ast}\!\left(E\right) &= \inf\set{\mu\left(U\right)\mid E\subseteq U,~\text{$U$ open}}.
  \end{align*}
  Note that by the way we define $\mu$, if $U\subseteq V$, then $\mu\!\left(U\right) \leq \mu\!\left(V\right)$, meaning $\mu^{\ast}\!\left(U\right) = \mu\!\left(U\right)$ if $U$ is open.\newline

  We begin by showing that $\mu^{\ast}$ is an outer measure. It suffices to show that for a sequence of open sets $\set{U_j}_{j=1}^{n}$ such that $U = \bigcup_{j=1}^{\infty}U_j$, $\mu\!\left(U\right) \leq \sum_{j=1}^{\infty}\mu\!\left(U_j\right)$. From this, we can see that, for any $E\subseteq X$,
  \begin{align*}
    \mu^{\ast}\!\left(E\right) &= \inf\set{\sum_{j=1}^{\infty}\mu\!\left(U_j\right)\mid U_j\text{ open, }E\subseteq \bigcup_{j=1}^{\infty}U_j}\\
                               &= \inf\set{\mu\!\left(U\right)\mid E\subseteq U,\text{ $U$ open}}
  \end{align*}
  defines an outer measure.\newline

  Let $U = \bigcup_{j=1}^{\infty}U_j$ and $f\in C_c\left(\Omega\right)$ with $f\prec U$. Let $K = \supp\left(f\right)$. Since $K$ is compact, $K\subseteq \bigcup_{i=1}^{n}U_i$ for some $n$.
  \begin{proposition}
    If $K$ is a compact set in a locally compact Hausdorff space $\Omega$ with an open cover $\set{U_j}_{j=1}^{n}$, then there is a partition of unity\footnote{A collection of functions in $C\left(X,[0,1]\right)$ such that their sum is $1$ on $E$ and every neighborhood has only finitely many nonzero functions.} on $K$ consisting of compactly supported functions $g_j$ such that $\supp\left(g_j\right)\subseteq U_j$.
  \end{proposition}
  \begin{proof}
    For each $x\in K$, there is a compact neighborhood $N_x$ such that $N_x\subseteq U_j$ for some $j$. The set $\set{N_x^{\circ}\mid x\in K}$ is an open cover of $K$, meaning there exist $x_1,\dots,x_m$ such that $K\subseteq \bigcup_{k=1}^{m}N_{x_k}$.\newline

    Let $F_j$ be the union of the set of $N_{x_k}$ that are a subset of $U_j$. Then, $F_j\subseteq U_j$ is compact, so by Urysohn's lemma, there are $g_1,\dots,g_n\in C_c\left(\Omega\right)$ such that $g_j = 1$ on $F_j$ and $\supp\left(g_j\right) \subseteq U_j$.\newline

    Since $F_j$ covers $K$, it is the case that $\sum_{k=1}^{n}g_k \geq 1$ on $K$. By Urysohn's Lemma, there $f\in C_c\left(\Omega,[0,1]\right)$ such that $f = 1$ on $K$ and $\supp(f) \subseteq \set{x\mid \sum_{k=1}^{n}g_k(x) > 0}$.\newline

    Define $g_{n+1} = 1 - f$, meaning $\sum_{k=1}^{n+1}g_{k} > 0$ everywhere. For $j = 1,\dots,n$, let $h_j = \frac{g_j}{\sum_{k=1}^{n+1}g_k}$. Then, $\supp\left(h_j\right) = \supp\left(g_j\right)$, $\supp\left(h_j\right) \subseteq U_j$, and $\sum_{j=1}^{n}h_j = 1$ on $K$.
  \end{proof}
  With this proposition, we can construct $g_i \prec U_i$ with $\sum_{i=1}^{n}g_j = 1$ on $K$. Then, $f = \sum_{i=1}^{n}fg_i$, with $fg_i\prec U_i$. Thus,
  \begin{align*}
    \varphi(f) &= \sum_{i=1}^{n}\varphi\left(fg_i\right)\\
               &\leq \sum_{i=1}^{n}\mu\!\left(U_i\right)\\
               &\leq \sum_{j=1}^{\infty}\mu\!\left(U_j\right).
  \end{align*}
  Thus, $\sup_{f\in C_c\left(\Omega\right)}\left(\varphi(f)\right) = \mu\!\left(U\right) \leq \sum_{j=1}^{\infty}\mu\!\left(U_j\right)$. Since this holds for any $f\prec U$, we conclude that $\mu\!\left(U\right) \leq \sum_{j=1}^{\infty}\mu\!\left(U_j\right)$.\newline

  We now show that for any open set $U\subseteq \Omega$, $U$ is $\mu^{\ast}$-measurable.\newline

  Let $U$ be open, $E\subseteq \Omega$ with $\mu^{\ast}\!\left(E\right) < \infty$. Suppose $E$ is open. Then, $E\cap U$ is open, so for $\ve > 0$, we can find $f\in C_c\left(\Omega\right)$ such that $f\prec E\cap U$ and $\varphi(f) > \mu\!\left(E\cap U\right) - \ve$. Additionally, $E\setminus \supp\left(f\right)$ is open, so we can find $g\in C_c\left(\Omega\right)$ such that $g\prec E\setminus \supp(f)$, and $\varphi(g) > \mu\!\left(E\setminus \supp(f)\right) - \ve$. However, $f + g\prec E$, so
  \begin{align*}
    \mu(E) &\geq \varphi(f) + \varphi(g)\\
           &> \mu\!\left(E\cap U\right) + \mu\!\left(E\setminus \supp(f)\right) - 2\ve\\
           &\geq \mu^{\ast}\!\left(E\cap U\right) + \mu^{\ast}\!\left(E\setminus U\right) - 2\ve.
  \end{align*}
  We obtain the desired inequality as $\ve\rightarrow 0$.\newline

  For any $E\subseteq \Omega$, we find $V$ open with $E\subseteq V$ and $\mu(V) < \mu^{\ast}\!(E) + \ve$ (by our definition of $\mu^{\ast}$). Thus,
  \begin{align*}
    \mu^{\ast}\!\left(E\right) + \ve &> \mu\!\left(V\right)\\
                                     &\geq \mu^{\ast}\!\left(V\cap U\right) + \mu^{\ast}\!\left(V\setminus U\right)\\
                                     &\geq \mu^{\ast}\!\left(E\cap U\right) + \mu^{\ast}\!\left(E\setminus U\right).
  \end{align*}
  We let $\ve \rightarrow 0$ to obtain our desired inequality.\newline

  Now, we show that for any compact $K\subseteq \Omega$, $\mu\!\left(K\right) = \inf\set{\varphi(f)\mid f\in C_c\left(\Omega\right),f\geq \1_{K}}$.\newline

  Let $K$ be compact, $f\in C_c\left(\Omega\right)$, and $f\geq \1_K$. Let $U_{\ve} = \set{x\mid f(x) > 1-\ve}$. Then, $U_{\ve}$ is open, and for $g\prec U_{\ve}$, we have $\left(1-\ve\right)^{-1}f - g \geq 0$.\newline

  Thus, $\varphi(g) \leq \left(1-\ve\right)^{-1}\varphi(f)$, meaning $\mu\!\left(K\right) \leq \mu\!\left(U_{\ve}\right) \leq \left(1-\ve\right)^{-1}\varphi(f)$. Letting $\ve \rightarrow 0$, we find $\mu\!\left(K\right) \leq \varphi(f)$.\newline

  For any $K\subseteq U$, Urysohn's lemma gives $f\in C_c\left(\Omega\right)$ such that $f\geq \1_K$ and $f\prec U$, meaning $\varphi(f) \leq \mu(U)$. Since $\mu$ is outer regular, we obtain condition (2).\newline

  With our measure $\mu$ defined, we only need to show that $\varphi(f) = \int_{\Omega}f\:d\mu$ for $f\in C_c\left(\Omega,[0,1]\right)$, as $C_c\left(\Omega\right)$ is the linear span of $C_c\left(\Omega,[0,1]\right)$.\newline

  Set $f\in C_c\left(\Omega,[0,1]\right)$. Given $N\in \N$, for $1\leq j\leq N$, let $K_j = \set{x\mid f(x) \geq \frac{j}{N}}$. Let $K_0 = \supp(f)$. Define $f_1,\dots,f_N\in C_c\left(\Omega\right)$ with $f_j(x) = 0$ for $x\notin K_{j-1}$, $f_j(x) = f(x) - \frac{j-1}{N}$ for $x\in K_{j-1}\setminus K_j$, and $f_j = \frac{1}{N}$ for $x\in K_j$.\newline

  We, thus have $\frac{\1_{K_j}}{N} \leq f_j \leq \frac{\1_{K_{j-1}}}{N}$, meaning
  \begin{align*}
    \frac{1}{N}\mu\!\left(K_j\right) \leq \int_{\Omega}f_j\:d\mu \leq \frac{1}{N}\mu\!\left(K_{j-1}\right).
  \end{align*}
  If $U$ is an open set containing $K_{j-1}$, then $Nf_{j} \prec U$, meaning $\varphi\left(f_j\right) \leq \frac{\mu\!\left(U\right)}{N}$. Thus, by condition (2), and outer regularity,
  \begin{align*}
    \frac{1}{N}\mu\!\left(K_{j}\right) \leq \varphi\left(f_j\right) \leq \frac{1}{N}\mu\!\left(K_{j-1}\right).
  \end{align*}
  Since $f = \sum_{j=1}^{N} f_j$, we have
  \begin{align*}
    \frac{1}{N}\sum_{j=1}^{N}\mu\!\left(K_j\right) \leq \int_{\Omega}f\:d\mu \leq \frac{1}{N}\sum_{j=1}^{N}\mu\!\left(K_{j-1}\right),
  \end{align*}
  and
  \begin{align*}
    \frac{1}{N}\sum_{j=1}^{N}\mu\!\left(K_j\right) \leq \varphi(f)\leq \frac{1}{N}\sum_{j=1}^{N-1}\mu\!\left(K_j\right).
  \end{align*}
  Thus,
  \begin{align*}
    \left\vert \varphi(f) - \int_{\Omega}f\:d\mu \right\vert &\leq \frac{\mu\!\left(K_0\right) - \mu\!\left(K_{N}\right)}{N}\\
    &\leq \frac{\mu\!\left(\supp(f)\right)}{N}.
  \end{align*}
  Since $\mu\!\left(\supp(f)\right) < \infty$, and $N$ is arbitrary, we have $I(f) = \int_{\Omega}f\:d\mu$.
\end{proof}
Having proven the Riesz Representation Theorem, we can begin to describe all bounded linear functionals on continuous function spaces, starting with the positive ones.
\begin{proposition}
  Let $\Omega$ be a locally compact Hausdorff space.
  \begin{enumerate}[(1)]
    \item For $\mu$ a positive and finite regular measure on $\Omega$,
      \begin{align*}
        \varphi_{\mu}: C_0\left(\Omega\right)\rightarrow \C\\
        \varphi_{\mu}(f) &= \int_{\Omega}f\:d\mu
      \end{align*}
      defines a positive linear functional with $\norm{\varphi_{\mu}}_{\text{op}} = \mu\!\left(\Omega\right)$. Moreover, $\varphi_{\mu}$ is faithful if $\mu$ has full support.
    \item If $\varphi: C_0\left(\Omega\right)\rightarrow \C$ is a positive linear functional, then there exists a unique positive and finite regular measure $\mu_{\varphi}$ such that $\varphi - \varphi_{\mu_{\varphi}}$. That is,
      \begin{align*}
        \varphi(f) &= \int_{\Omega}f\:d\mu_{\varphi}.
      \end{align*}
      Moreover, $\norm{\varphi}_{\text{op}} = \mu\left(\Omega\right)$, and $\mu$ has full support if $\varphi$ is faithful.
  \end{enumerate}
\end{proposition}
\begin{proof}
  The map
  \begin{align*}
    \varphi_{\mu}\left(f\right) &= \int_{\Omega}f\:d\mu
  \end{align*}
  is positive, linear, and well=defined, since
  \begin{align*}
    \left\vert \varphi_{\mu}(f) \right\vert &\leq \left\vert \int_{\Omega}f\:d\mu \right\vert\\
                                            &\leq \int_{\Omega}\left\vert f \right\vert\:d\mu\\
                                            &\leq \norm{f}_{u} \mu\!\left(\Omega\right).
  \end{align*}
  This also shows that $\norm{\varphi_{\mu}}_{\text{op}}\leq \mu\!\left(\Omega\right)$. Given $\ve > 0$, the regularity of $\mu$ allows us to find $K\subseteq \Omega$ with $\mu\!\left(\Omega\right)-\ve < \mu\left(K\right)$. Urysohn's lemma gives a continuous $f\in C_c\left(\Omega,[0,1]\right)$ with $f = 1$ on $K$. Thus,
  \begin{align*}
    \norm{\varphi_{\mu}}_{\text{op}} &\geq \varphi_{\mu}\left(f\right)\\
                                     &= \int_{\Omega}f\:d\mu\\
                                     &\geq \int_{\Omega}\1_{K}\:d\mu\\
                                     &= \mu\!\left(K\right)\\
                                     &> \mu\!\left(\Omega\right) - \ve.
  \end{align*}
  Since $\ve$ was arbitrary, $\mu\!\left(\Omega\right)\leq \norm{\varphi_{\mu}}_{\text{op}}$.\newline

  Suppose $g\in C_0\left(\Omega\right)$ with $g\geq 0$ and $g\neq 0$. Then, there is a nonempty open subset $U\subseteq \Omega$ and $\delta > 0$ such that $g(x) \geq \delta$ for all $x\in U$. If $\mu$ has full support, $\mu\!\left(U\right) > 0$, meaning
  \begin{align*}
    \varphi(g) &= \int_{\Omega}g\:d\mu\\
               &\geq \int_{\Omega}\delta\1_{U}\:d\mu\\
               &= \delta\mu\!\left(U\right)\\
               &> 0.
  \end{align*}
  Thus, we have proven (1).\newline

  If $\varphi: C_0\left(\Omega\right)\rightarrow \C$ is a positive linear functional, then $\varphi$ is bounded. The restriction $\varphi_0 = \varphi|_{C_c\left(\Omega\right)}: C_c\left(\Omega\right)\rightarrow \C$ is positive and linear, meaning $\varphi_0 = I_{\mu}$ for some Radon measure $\mu$ by the Riesz Representation Theorem. We set $\mu_{\varphi} = \mu$.\newline

  By the Riesz Representation theorem, we have
  \begin{align*}
    \mu\!\left(\Omega\right) &= \sup\set{I_{\mu}\left(f\right)\mid f\in C_c\left(\Omega,[0,1]\right),~\supp\left(f\right) \subseteq \Omega}\\
                             &\leq \sup\set{\varphi(f)\mid f\in C_0\left(\Omega\right),\norm{f}\leq 1}\\
                             &= \norm{\varphi}_{\text{op}}\\
                             &< \infty.
  \end{align*}
  Additionally, if $f\in C_c\left(\Omega\right)$, then
  \begin{align*}
    \left\vert \varphi(f) \right\vert &= \left\vert \varphi_0\left(f\right) \right\vert\\
                                      &\leq \left\vert \int_{\Omega}f\:d\mu \right\vert\\
                                      &\leq \int_{\Omega}\left\vert f \right\vert\:d\mu\\
                                      &\leq \norm{f}_{u}\mu\!\left(\Omega\right).
  \end{align*}
  Since $C_c\left(\Omega\right)\subseteq C_0\left(\Omega\right)$ is dense, we have $\left\vert \varphi(f) \right\vert\leq \norm{f}_u\mu\!\left(\Omega\right)$ for all $f\in C_0\left(\Omega\right)$, meaning $\norm{\varphi}_{\text{op}}\leq \mu\!\left(\Omega\right)$.\newline

  Since $\mu$ is finite and Radon, $\mu$ is regular. We can see that $\varphi$ and $\varphi_u$ are bounded functionals that agree on $C_c\left(\Omega\right)$, so $\varphi = \varphi_{\mu}$.\newline

  Suppose $\varphi$ is faithful. Let $U\subseteq \Omega$ be open and nonempty. By Urysohn's lemma, there is $f: \Omega\rightarrow [0,1]$ that is nonzero and compactly supported with $\supp(f) \subseteq U$. Thus,
  \begin{align*}
    \mu\!\left(U\right) &= \int_{\Omega}\1_{U}\:d\mu\\
                        &\geq \int_{\Omega}f\:d\mu\\
                        &= \varphi_{\mu}\left(f\right)\\
                        &= \varphi(f)\\
                        &> 0,
  \end{align*}
  meaning $\mu$ has full support.
\end{proof}
We let $\mathcal{P}_r\left(\Omega\right)$ be the convex set of all regular probability measures on $\left(\Omega,\mathcal{B}_\Omega\right)$. The following corollary follows from the previous proposition.
\begin{corollary}
  Let $\Omega$ be a compact Hausdorff space. There exists a bijective affine map $T: S\left(\Omega\right)\rightarrow \mathcal{P}_{r}\left(\Omega\right)$ given by $T\left(\varphi\right) = \mu_{\varphi}$.
\end{corollary}
We can use the positive linear functionals to generate all bounded linear functionals on a function space.
\begin{theorem}[Jordan Decomposition]
  Let $\Omega$ be a locally compact Hausdorff space, and suppose $\varphi: C_0\left(\Omega\right)\rightarrow \C$ be bounded linear. Then, there exists positive linear functionals $\varphi_1,\varphi_2,\varphi_3,\varphi_4$ such that
  \begin{align*}
    \varphi &= \left(\varphi_1-\varphi_2\right) + i\left(\varphi_3 - \varphi_4\right),
  \end{align*}
  and $\norm{\varphi_j}_{\text{op}}\leq \norm{\varphi}_{\text{op}}$.
\end{theorem}
\begin{remark}
  The Jordan decomposition for positive linear functionals follows from the Hahn decomposition for complex measures.
\end{remark}
If $\varphi\in C_0\left(\Omega\right)^{\ast}$, then the Jordan decomposition provides $\varphi = \left(\varphi_1-\varphi_2\right) + i\left(\varphi_3 - \varphi_4\right)$, which gives the positive regular measures $\mu_1,\dots,\mu_4$ such that $\varphi_j\left(f\right) = \int_{\Omega}f\:d\mu_j$ for $f\in C_0\left(\Omega\right)$.\newline

Then, $\mu = \left(\mu_1 - \mu_2\right) + i\left(\mu_3 - \mu_4\right)$ has total variation $\left\vert \mu \right\vert$, meaning it is regular itself. Thus,
\begin{align*}
  \varphi(f) &= \varphi_1\left(f\right) - \varphi_2\left(f\right) + i\left(\varphi_3\left(f\right) - \varphi_4\left(f\right)\right)\\
             &= \int_{\Omega}f\:d\mu_1 - \int_{\Omega}f\:d\mu_2 + i \left(\int_{\Omega}f\:d\mu_4 - \int_{\Omega}f\:d\mu_4\right)\\
             &= \int_{\Omega}f\:d\mu\\
             &= \varphi_{\mu}\left(f\right).
\end{align*}
Thus, we get the following characterization of $C_0\left(\Omega\right)^{\ast}$.
\begin{theorem}[Riesz--Markov Theorem]
  Let $\Omega$ be a locally compact Hausdorff space. Then, $M_r\left(\Omega\right) \cong C_0\left(\Omega\right)^{\ast}$. The map $\Phi: M_r\left(\Omega\right)\rightarrow C_0\left(\Omega\right)^{\ast}$ defined by $\Phi\left(\mu\right) = \varphi_{\mu}$, where
  \begin{align*}
    \varphi_\mu &= \int_{\Omega}f\:d\mu,
  \end{align*}
  is an isometric isomorphism.
\end{theorem}
\subsection{The Banach Space Adjoint}%
\begin{definition}
  Let $X,Y$ be normed vector spaces, and let $T\in \mathcal{B}\left(X,Y\right)$. Then, the map $T^{\ast}:Y^{\ast}\rightarrow X^{\ast}$, defined by $T^{\ast}\left(\varphi\right) = \varphi\circ T$ is called the Banach space adjoint (or adjoint) of $T$.
\end{definition}
We can see that $T^{\ast}$ is well-defined since it is the composition of two bounded linear operators. Additionally,
\begin{align*}
  \norm{T^{\ast}\left(\varphi\right)}_{\text{op}} &= \norm{\varphi\circ T}_{\text{op}}\\
                                                  &\leq \norm{\varphi}_{\text{op}}\norm{T}_{\text{op}}.
\end{align*}
\begin{proposition}
  If $X$ and $Y$ are normed vector spaces spaces, $T\in \mathcal{B}\left(X,Y\right)$, and $T^{\ast}$ be the adjoint of $T$. Then, $T^{\ast}\in \mathcal{B}\left(Y^{\ast},X^{\ast}\right)$, and $\norm{T^{\ast}}_{\text{op}} = \norm{T}_{\text{op}}$.
\end{proposition}
\begin{proof}
  Let $\varphi_1,\varphi_2\in Y^{\ast}$, $\alpha \in \C$. Then,
  \begin{align*}
    T^{\ast}\left(\varphi_1 + \alpha\varphi_2\right) &= \left(\varphi_1 + \alpha\varphi_2\right)\circ T\\
                                                     &= \varphi_1\circ T + \alpha\varphi_2\circ T\\
                                                     &= T^{\ast}\left(\varphi_1\right) + \alpha T^{\ast}\left(\varphi_2\right),
  \end{align*}
  meaning $T^{\ast}$ is linear. Recall that for any normed space $Z$ and $z\in Z$, we have $\norm{z} = \sup_{\varphi\in B_{Z^{\ast}}}\left\vert \varphi(z) \right\vert$. Thus, for $x\in X$, we have
  \begin{align*}
    \norm{T(x)} &= \sup_{\varphi\in B_{Y^{\ast}}}\left\vert \varphi\left(T(x)\right) \right\vert\\
                &= \sup_{\varphi\in B_{Y^{\ast}}}\left\vert T^{\ast}\left(\varphi(x)\right) \right\vert\\
                &= \sup_{\varphi\in B_{Y^{\ast}}}\left\vert T^{\ast}\left(\varphi\right) \right\vert\norm{x}\\
                &\leq \norm{T^{\ast}}_{\text{op}}.
  \end{align*}
  Thus, we have $\norm{T}_{\text{op}}\leq \norm{T^{\ast}}_{\text{op}}$.
\end{proof}
We can now understand some of the analytic and algebraic properties of the adjoint.
\begin{theorem}
  Let $X,Y,Z$ be normed spaces, and let $T,S: X\rightarrow Y$, $R: Y\rightarrow Z$ be bounded linear operators. The following are true.
  \begin{enumerate}[(1)]
    \item $\left(T+S\right)^{\ast} = T^{\ast} + S^{\ast}$
    \item $\left(\alpha T\right)^{\ast} = \alpha T^{\ast}$
    \item $\left(R\circ T\right)^{\ast} = T^{\ast} \circ R^{\ast}$
    \item $\left(\id_{X}\right)^{\ast} = \id_{X^{\ast}}$
    \item The following diagram commutes.
      \begin{center}
        % https://tikzcd.yichuanshen.de/#N4Igdg9gJgpgziAXAbVABwnAlgFyxMJZABgBpiBdUkANwEMAbAVxiRAA0A9YAHR7rg4+AnAF8Qo0uky58hFAEZyVWoxZsAmt2GCdYiVJAZseAkTIKV9Zq0QcD0k3KJLL1a+rsaJKmFADm8ESgAGYAThAAtkhkIDgQSEqqNmwAKtr8upn6kqER0YgATNTxSADM1Ax0AEYwDAAKMqbyIGFY-gAWOCDuarYgqQ4g4VFIxXEJiLEe-Xz4OHQA+uxDIwUVE4mVNXWNTmZ2bZ3dvSl2cxALi96iFKJAA
\begin{tikzcd}
X^{\ast\ast} \arrow[r, "T^{\ast\ast}"] & Y^{\ast\ast}            \\
X \arrow[r, "T"'] \arrow[u, "\iota_X",hookrightarrow] & Y \arrow[u, "\iota_Y"',hookrightarrow]
\end{tikzcd}
      \end{center}
  \end{enumerate}
\end{theorem}
\begin{proof}[Proof of (5)]
  \begin{align*}
    T^{\ast\ast}\left(\iota_X\left(x\right)\right)\left(\varphi\right) &= T^{\ast\ast}\left(\hat{x}\right)\left(\varphi\right)\\
                                                                       &= \hat{x}\circ T^{\ast}\left(\varphi\right)\\
                                                                       &= \hat{x}\left(\varphi\circ T\right)\\
                                                                       &= \varphi\circ T\left(x\right)\\
                                                                       &= \widehat{T(x)}\left(\varphi\right)\\
                                                                       &= \iota_Y\left(T\left(x\right)\right)\left(\varphi\right).
  \end{align*}
  Thus, $T^{\ast\ast}\left(\iota_X(x)\right) = \iota_Y\left(T\left(x\right)\right)$, meaning $T^{\ast\ast}\circ \iota_X = \iota_Y\circ T$.
\end{proof}
\begin{theorem}
  Let $X$ and $Y$ be normed spaces.
  \begin{enumerate}[(1)]
    \item If $T: X\rightarrow Y$ is an isometric isomorphism, then $T^{\ast}: Y^{\ast}\rightarrow X^{\ast}$ is an isometric isomorphism.
    \item If $T: X\rightarrow Y$ is a bicontinuous isomorphism, then $\left(T^{-1}\right)^{\ast} = \left(T^{\ast}\right)^{-1}$, so $T^{\ast}$ is a bicontinuous isomorphism.
    \item If $T: X\rightarrow Y$ is an isometry, then $T^{\ast}: Y^{\ast}\rightarrow X^{\ast}$ is a $1$-quotient map. We have $T^{\ast}\left(B_{Y^{\ast}}\right) = B_{X^{\ast}}$.
    \item If $T: X\rightarrow Y$ is a $1$-quotient map, then $T^{\ast}:Y^{\ast}\rightarrow X^{\ast}$.
    \item $T: X\rightarrow Y$ is an isometry if and only if $T^{\ast\ast}: X^{\ast\ast}\rightarrow Y^{\ast\ast}$ is an isometry.
    \item If $T\in \mathcal{B}\left(X,Y\right)$, and $T^{\ast}$ is a $1$-quotient map, then $T$ is an isometry.
    \item If $X,Y$ are Banach spaces, and $T\in \mathcal{B}\left(X,Y\right)$, and $T^{\ast}$ is an isometry, then $T$ is a $1$-quotient map.
  \end{enumerate}
\end{theorem}
\begin{proof}
  Starting by proving (2), we have
  \begin{align*}
    \id_{X^{\ast}} &= \left(\id_{X}\right)^{\ast}\\
                   &= \left(T^{-1}\circ T\right)^{\ast}\\
                   &= T^{\ast}\circ \left(T^{-1}\right)^{\ast}
                   \intertext{and}
    \id_{X^{\ast}} &= T^{\ast}\circ \left(T^{-1}\right)^{\ast},
  \end{align*}
  meaning $\left(T^{\ast}\right)^{-1} = \left(T^{-1}\right)^{\ast}$.\newline

  Now, turning our attention to (4), if $T$ is a $1$-quotient map, then $T\left(U_X\right) = U_Y$. For $\psi\in Y^{\ast}$, we have
  \begin{align*}
    \norm{T^{\ast}\left(\psi\right)}_{\text{op}} &= \norm{\psi\circ T}_{\text{op}}\\
                                                 &= \sup_{x\in U_X}\norm{\psi\circ T\left(x\right)}\\
                                                 &= \sup_{y\in Y}\norm{\psi(y)}\\
                                                 &= \norm{\psi}_{\text{op}}.
  \end{align*}
  Thus, $T^{\ast}$ is an isometry.\newline

  Focusing on (3), we have $\norm{T^{\ast}}_{\text{op}} = \norm{T}_{\text{op}} = 1$, so $T^{\ast}\left(B_{Y^{\ast}}\right)\subseteq B_{X^{\ast}}$. Let $\varphi\in B_{X^{\ast}}$. We want to find $\psi\in B_{Y^{\ast}}$ such that $T^{\ast}\left(\psi\right) = \varphi$. Let $\psi_0: \Ran(T) \rightarrow \F$ be defined by $\psi_0\left(T\left(x\right)\right) = \varphi(x)$. THis is well-defined since $T$ is injective. For $x\in B_X$, we have
  \begin{align*}
    \left\vert \psi_0\left(T\left(x\right)\right) \right\vert &= \left\vert \varphi(x) \right\vert\\
                                                              &\leq \norm{\varphi}_{\text{op}}\norm{x}\\
                                                              &\leq \norm{\varphi}_{\text{op}}.
  \end{align*}
  Thus, $\norm{\psi_0}_{\text{op}} \leq \norm{\varphi}_{\text{op}} \leq 1$. There is an extension $\psi\in Y^{\ast}$ of $\psi_0$ such that $\norm{\psi}_{\text{op}} = \norm{\psi_0}_{\text{op}}$, and $\psi\circ T = \varphi$. Thus, $T^{\ast}\left(B_{Y^{\ast}}\right) = B_{X^{\ast}}$.\newline

  The veracity of (1) follows from (2), (3), and (4).\newline

  To show (5), we let $T$ be isometric, which implies that $T^{\ast}$ is a $1$-quotient map. Thus, we have $T^{\ast\ast}$ is isometric. Conversely, if $T^{\ast\ast}$ is isometric, then
  \begin{align*}
    \norm{T(x)} &= \norm{\iota_Y\left(T(x)\right)}\\
                &= \norm{T^{\ast\ast}\left(\iota_X(x)\right)}\\
                &= \norm{\iota_X(x)}\\
                &= \norm{x}.
  \end{align*}
  To see (6), if $T^{\ast}$ is a $1$-quotient map, then $T^{\ast\ast}$ is isometric, so $T$ is isometric.\newline

  We will show (7) later.
\end{proof}
\begin{example}
  Let $\tau: \Omega \rightarrow \Lambda$ be a proper map\footnote{Inverse image of a compact subset of $\Lambda$ is compact in $\Omega$} between locally compact Hausdorff spaces. The induced contractive map $T_{\tau}: C_0\left(\Lambda\right)\rightarrow C_0\left(\Omega\right)$ has an adjoint map
  \begin{align*}
    T_{\tau}^{\ast}: C_0\left(\Omega\right)\rightarrow C_0\left(\Lambda\right)^{\ast},
  \end{align*}
  where $T_{\tau}^{\ast}\left(\varphi\right) = \varphi\circ T_{\tau}$. By the Riesz--Markov theorem, we can identify $C_0\left(\Omega\right)^{\ast}\cong M_{r}\left(\Omega\right)$.\newline

  The push forward measure is an induced map
  \begin{align*}
    \tau_{\ast}: M(\Omega) \rightarrow M(\Lambda),
  \end{align*}
  defined by
  \begin{align*}
    \tau_{\ast}\mu\!\left(E\right) &:= \mu\!\left(\tau^{-1}\left(E\right)\right)
  \end{align*}
  for any $E\in \mathcal{B}_{\Lambda}$. We can verify that $\tau_{\ast}\mu$ is a complex measure on $\left(\Lambda,\mathcal{B}_{\Lambda}\right)$, and that $\tau_{\ast}\left(\mathcal{P}\left(\Omega\right)\right)\subseteq \mathcal{P}\left(\Lambda\right)$.\newline

  We can also see that for all $f\in C_0\left(\Lambda\right)$,
  \begin{align*}
    \int_{\Omega}f\circ \tau\:d\mu &= \int_{\Lambda}f\:d\left(\tau_{\ast}\mu\right).
  \end{align*}
  This is because, for some $E\in \mathcal{B}_{\Lambda}$, we have $\1_{E}\circ \tau = \1_{\tau^{-1}\left(E\right)}$, so
  \begin{align*}
    \int_{\Omega}\1_{E}\circ \tau\:d\mu &= \int_{\Omega}\1_{\tau^{-1}\left(E\right)}\:d\mu\\
                                        &= \mu\!\left(\tau^{-1}\left(E\right)\right)\\
                                        &= \tau_{\ast}\mu\!\left(E\right)\\
                                        &= \int_{\Lambda}\1_{E}\:d\left(\tau_{\ast}\mu\right).
  \end{align*}
  Thus, since this identity holds for all simple functions, and every bounded function is the uniform limit of a simiple function, we get that
  \begin{align*}
    \int_{\Omega}f\circ \tau\:d\mu &= \int_{\Lambda}f\:d\left(\tau_{\ast}\mu\right)
  \end{align*}
  for all $f\in C_0\left(\Lambda\right)$ by Dominated Convergence.\newline

  Thus, we can see that for any $\tau: \Omega\rightarrow \Lambda$ that is a proper map between locally compact Hausdorff spaces, then $T_{\tau}^{\ast}\left(\varphi_{\mu}\right) = \varphi_{\tau_{\ast}\mu}$ for all $\mu\in M_{r}\left(\Omega\right)$.
  \begin{align*}
    T_{\tau}^{\ast}\left(\varphi_{\mu}\right)\left(f\right) &= \varphi_{\mu}\circ T_{\tau}\left(f\right)\\
                                                            &= \varphi_{\mu}\left(f\circ \tau\right)\\
                                                            &= \int_{\Omega}f\circ \tau\:d\mu\\
                                                            &= \int_{\Lambda}f\:d\left(\tau_{\ast}\mu\right)\\
                                                            &= \varphi_{\tau_{\ast}\mu}(f).
  \end{align*}
  This also shows that the push forward of a regular complex Borel measure is regular, or $\tau_{\ast}\left(M_{r}\left(\Omega\right)\right) \subseteq M_{r}\left(\Lambda\right)$.
\end{example}
Suppose we are dealing with the special case of $\tau: \Omega\rightarrow \Lambda$ is a homeomorphism. Then, $T_{\tau}$ is an isometric isomorphism of Banach spaces, meaning
\begin{align*}
  T_{\tau}^{\ast}: C_0\left(\Omega\right)^{\ast} \rightarrow C_0\left(\Lambda\right)^{\ast},
  \intertext{or}
  \tau_{\ast}: M_r\left(\Omega\right)\rightarrow M_r\left(\Lambda\right),
\end{align*}
is an isometric isomorphism of Banach spaces. The restriction to the state space
\begin{align*}
  T_{\tau}^{\ast}|_{S(\Omega)}: S(\Omega)\rightarrow S(\Lambda),
  \intertext{or}
  \tau_{\ast}|_{\mathcal{P}_r(\Omega)}: \mathcal{P}_r\left(\Omega\right)\rightarrow \mathcal{P}_r\left(\Lambda\right)
\end{align*}
is an affine bijection.\newline

This allows us to consider the idea of an \textit{invariant} measure (or, equivalently, invariant state).
\begin{definition}
  Let $\Omega$ be a locally compact state, and let $\tau: \Omega\rightarrow\Omega$ be a continuous transformation. A regular Borel probability measure, $\mu\in \mathcal{P}_r\left(\Omega\right)$ is called $\tau$-invariant if $\tau_{\ast}\mu = \mu$, or $T_{\tau}^{\ast}\left(\varphi_{\mu}\right) = \varphi_{\mu}$. We write
  \begin{align*}
    \mathcal{P}_{r}\left(\Omega,\tau\right) := \set{\mu\in \mathcal{P}_{r}\left(\Omega\right)\mid \tau_{\ast}\mu = \mu}
  \end{align*}
  to denote the set of $\tau$-invariant measures.
\end{definition}
We can now use the analytic and algebraic properties of adjoints to identify duals of subspaces and quotient spaces. Recall that for a normed vector space $X$ and $E\subseteq X$ a subset, then
\begin{align*}
  E^{\perp} &= \set{\varphi\in X^{\ast}\mid \varphi|_{E} = 0}
\end{align*}
is the annihilator of $E$.
\begin{proposition}
  Let $X$ be a normed vector space, $E\subseteq X$ a subspace. The dual space of $E$ is isometrically isomorphic to $X^{\ast}/E^{\perp}$.
\end{proposition}
\begin{proof}
  The embedding $\iota: E\hookrightarrow X$ is an isometry, so $\iota^{\ast}: X^{\ast}\rightarrow E^{\ast}$ is a $1$-quotient map. The first isomorphism theorem gives $X^{\ast}/\ker\left(\iota^{\ast}\right) \cong E^{\ast}$. Thus,
  \begin{align*}
    \ker\left(\iota^{\ast}\right) &= \set{\varphi\in X^{\ast}\mid \iota^{\ast}\left(\varphi\right) = 0}\\
                                  &= \set{\varphi\in X^{\ast}\mid \varphi\circ\iota = 0}\\
                                  &= \set{\varphi\in X^{\ast}\mid \varphi|_{E} = 0}\\
                                  &= E^{\perp}.
  \end{align*}
\end{proof}
Now, we can look at the dual of a quotient space.
\begin{proposition}
  Let $X$ be a normed vector space and $E\subseteq X$ a closed subspace. Then, $\left(X/E\right)^{\ast} \cong E^{\perp}$.
\end{proposition}
\begin{proof}
  Let $\pi: X\rightarrow X/E$ be the canonical projection map. Since $\pi$ is a $1$-quotient map, we have $\pi^{\ast}:\left(X/E\right)^{\ast}\rightarrow X^{\ast}$ is an isometry, so $\left(X/E\right)^{\ast} \cong \Ran\left(\pi^{\ast}\right)$. We have
  \begin{align*}
    \Ran\left(\pi^{\ast}\right) &= \set{\pi^{\ast}\left(\psi\right)\mid \psi\in \left(X/E\right)^{\ast}}\\
                                &= \set{\psi\circ \pi\mid \psi\in \left(X/E\right)^{\ast}}.
  \end{align*}
  We claim that $\Ran\left(\pi^{\ast}\right) = E^{\perp}$. For any $\psi\in \left(X/E\right)^{\ast}$, we have $\psi\circ \pi\in X^{\ast}$, and $\psi\circ \pi|_{E} = 0$, meaning $\Ran\left(\pi^{\ast}\right) \subseteq E^{\perp}$.\newline

  For the reverse direction, if $\varphi\in E^{\perp}$, and $\varphi: X\rightarrow \C$ sends $E$ to $0$, then $\varphi$ factors through $X/E$, meaning $\varphi(x) = \widetilde{\varphi}\left(x+E\right)$ for some bounded linear operator $\widetilde{\varphi}$ such that $\norm{\widetilde{\varphi}}_{\text{op}} = \norm{\varphi}_{\text{op}}$. Thus, $\widetilde{\varphi}\circ \pi = \varphi$, so $\varphi\in \Ran\left(\pi^{\ast}\right)$, so $\Ran\left(\pi^{\ast}\right) = E^{\perp}$.\newline

  Thus, we have $\left(X/E\right)^{\ast} \cong \Ran\left(\pi^{\ast}\right) \cong E^{\perp}$.
\end{proof}
\begin{proposition}[Existence of Auerbach Basis]
  Let $X$ be a finite dimensional normed vector space. Then, there exists a collection $\set{u_1,\dots,u_n}\subseteq S_X$ that forms a basis for $X$, and a collection $\set{\varphi_1,\dots,\varphi_n}\subseteq S_{X^{\ast}}$ that forms a basis for $X^{\ast}$, such that $\varphi_u\left(u_j\right) = \delta_{ij}$. We call the collection $\set{\left(u_i,\varphi_i\right)}_{j=1}^{n}$ and Auerbach basis for $X$.
\end{proposition}
\begin{proof}
  Let $\theta: X\rightarrow \ell_{\infty}^{n}$ be a bicontinuous isomorphism. Consider the map
  \begin{align*}
    D: \underbrace{B_{X}\times\cdots\times B_X}_{\text{$n$ times}}\rightarrow \C
  \end{align*}
  defined by
  \begin{align*}
    D\left(x_1,\dots,x_n\right) &= \det \begin{pmatrix}\theta\left(x_1\right) & \theta\left(x_2\right) & \cdots & \theta\left(x_n\right)\end{pmatrix},
  \end{align*}
  where $\theta\left(x_n\right)$ are column vectors. It is the case that $D$ is continuous, as $D$ is a polynomial over its entries, and since $B_X\times\cdots\times B_X$ is compact (as $B_X$ is compact, and the finite product of compact sets is compact), we must have
  \begin{align*}
    \max_{x_j\in B_X}\left\vert D\left(x_1,\dots,x_n\right) \right\vert &= \left\vert D\left(u_1,\dots,u_n\right) \right\vert\\
                                                                        &:= M
  \end{align*}
  for some $u_1,\dots,u_n\in B_X\times\cdots\times B_X$. Define $\varphi: X\rightarrow \C$ by
  \begin{align*}
    \varphi_k\left(x\right) &= \frac{1}{M}D\left(u_1,\dots,x,\dots,u_n\right),
  \end{align*}
  where $x$ is in the $k$th index. Since $D$ is multilinear, $\varphi$ is linear. Additionally, $\left\vert \varphi_k(x) \right\vert\leq 1$ for each $B_x$, so $\norm{\varphi_k}_{\text{op}} \leq 1$, while $\left\vert \varphi_k\left(u_k\right) \right\vert = 1$, so $\norm{\varphi_k}_{\text{op}} =1 $.\newline

  Let $\omega_k = \sgn\left(\varphi_k\left(u_k\right)\right)$, and set $v_k = \omega_ku_k$. Then, $\varphi_k\left(v_j\right) = \delta_{jk}$.\newline

  If $D\left(x_1,\dots,x_n\right) = 0$, then $\set{\theta\left(x_1\right),\dots,\theta\left(x_n\right)}$ is linearly dependent in $\F^n$, and since $\theta$ is a linear isomorphism, then $\set{x_1,\dots,x_n}$ is linearly dependent. Thus, $\set{v_1,\dots,v_n}$ is a basis for $X$.\newline

  We will show that $\set{\varphi_1,\dots,\varphi_n}$ is a basis for $X^{\ast}$ later.
\end{proof}

\end{document}
