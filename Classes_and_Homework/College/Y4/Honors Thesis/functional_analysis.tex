\documentclass[10pt]{mypackage}

% sans serif font:
%\usepackage{cmbright}
%\usepackage{sfmath}
%\usepackage{bbold} %better blackboard bold

%serif font + different blackboard bold for serif font
\usepackage{newpxtext,eulerpx}
\renewcommand*{\mathbb}[1]{\varmathbb{#1}}
\renewcommand*{\hbar}{\hslash}

\pagestyle{fancy} %better headers
\fancyhf{}
\rhead{Avinash Iyer}
\lhead{A Foray into Functional Analysis}

\setcounter{secnumdepth}{0}

\begin{document}
\RaggedRight
\tableofcontents
\section{Introduction}%
This is going to be part of the notes for my Honors thesis independent study, which will be focused on amenability and $C^{\ast}$-algebras. This section of notes will be focused on the essential results in functional analysis, starting from normed vector spaces, working our way up through $C^{\ast}$-algebras.\newline

The primary source for this section is going to be Timothy Rainone's \textit{Functional Analysis-En Route to Operator Algebras}, which has not been published yet.\newline

I do not claim any of this work to be original.
\section{Normed Vector Spaces}%
\subsection{Vector Spaces, Norms, and Basic Properties}%
All vector spaces are defined over $\C$. Most of the information here is in my Real Analysis II notes, so I'm going to skip to some of the more important content.
\begin{definition}[Vector Space]
  A vector space $V$ is a set closed under two operations
  \begin{align*}
    a: V\times V \rightarrow V,~\left(v_1,v_2\right)\mapsto v_1 + v_2\\
    m: \C\times V\rightarrow V,~\left(\lambda,v\right) \mapsto \lambda v.
  \end{align*}
  We refer to $a$ as addition, and $m$ as scalar multiplication; $(V,+)$ is an abelian ring.
\end{definition}
\begin{definition}[Norm]
  A norm is a function
  \begin{align*}
    \norm{\cdot}: V \rightarrow \R^+,~x\mapsto \norm{x}
  \end{align*}
  that satisfies the following properties:
  \begin{itemize}
    \item Positive definiteness: $\norm{v} = 0$ if and only if $v = 0_V$.
    \item Triangle inequality: $\norm{v+w} \leq \norm{v} + \norm{w}$.
    \item Absolute Homogeneity: $\norm{\lambda v} = \left\vert \lambda \right\vert\norm{v}$, for $\lambda \in \C$.
  \end{itemize}
  If a function $p: V\rightarrow \R^+$ satisfies the triangle inequality and absolute homogeneity, we say $p$ is a seminorm.
\end{definition}
We say the pair $\left(V,\norm{\cdot}\right)$ is a normed vector space.
\begin{definition}[Balls and Spheres]
  Let $X$ be a normed vector space, $x\in X$, and $\delta > 0$. Then,
  \begin{align*}
    U(x,\delta) &= \set{y\in X\mid d(x,y) < \delta}\\
    B(x,\delta) &= \set{y\in X\mid d(x,y) \leq \delta}\\
    S(x,\delta) &= \set{y\in X\mid d(x,y) = \delta}.
  \end{align*}
  For a normed vector space, we will use the following conventions for common sets:
  \begin{align*}
    U_X &= U(0,1)\\
    B_X &= B(0,1)\\
    S_X &= S(0,1)\\
    \mathbb{D} &= U_{\C}\\
    \mathbb{T} &= S_{\C}.
  \end{align*}
\end{definition}
\begin{definition}[Equivalent Norms]
  Two norms on $V$, $\norm{\cdot}_{a}$ and $\norm{\cdot}_{b}$ are said to be equivalent if there are two constants $C_1$ and $C_2$ such that
  \begin{align*}
    \norm{v}_{a} &\leq C_1\norm{v}_b\\
    \norm{v}_{b} &\leq C_2\norm{v}_a
  \end{align*}
  for all $v\in V$. We say $\norm{\cdot}_{a}\sim \norm{\cdot}_{b}$.
\end{definition}
\subsection{Examples}%
\begin{example}[Finite-Dimensional Vector Spaces]
  The vector space $\C^n$ is with the $p$-norm is denoted $\ell_{p}^{n}$, where for $p \in [1,\infty]$, the $p$-norm is defined by
  \begin{align*}
    \norm{x}_{p} &= \left(\sum_{i=1}^{n}\left\vert x_i \right\vert^p\right)^{1/p}.
  \end{align*}
  In the case with $p=2$, this gives the traditional Euclidean norm, and with $p = \infty$, this gives the $\sup$ norm:
  \begin{align*}
    \norm{x}_{\infty} &= \max_{1\leq i \leq n}\left\vert x_i \right\vert.
  \end{align*}
\end{example}
\begin{example}[A Sequence Space]
  We let $\ell_{p} = \set{\left(x_{n}\right)_n\mid x_n\in \C,\norm{x}_p < \infty}$ be the collection of sequences in $\C$ with finite $p$-norm. Here,
  \begin{align*}
    \norm{x}_p &= \left(\sum_{n=1}^{\infty}\left\vert x_n \right\vert^p\right)^{1/p}.
  \end{align*}
  In the case with $p = \infty$, this gives the sequence space $\ell_{\infty}$, which has norm
  \begin{align*}
    \norm{x}_{\infty} &= \sup_{n\in \N}\left\vert x_n \right\vert.
  \end{align*}
\end{example}
\begin{example}[A Function Space]
  We let $\ell^{\infty}\left(\Omega\right)$ denote the set of all bounded functions $f: \Omega \rightarrow \C$, equipped with the norm
  \begin{align*}
    \norm{f}_{\infty} &= \sup_{x\in \Omega}\left\vert f(x) \right\vert.
  \end{align*}
  If $\Omega = \left(\Omega,\mathcal{M},\mu\right)$ is a measure space, then we let $L^{\infty}\left(\Omega\right)$ be the space of $\mu$-a.e. equal essentially bounded measurable functions, under the norm
  \begin{align*}
    \norm{f}_{\infty} &= \esssup_{x\in \Omega}\left\vert f(x) \right\vert.
  \end{align*}
\end{example}
\subsection{Series Convergence and Completeness}%
\begin{proposition}[Criteria for Banach Spaces]
Let $X$ be a normed vector space. The following are equivalent:
\begin{enumerate}[(i)]
  \item $X$ is a Banach space.\footnote{Complete normed vector space.}
  \item If $\left(x_k\right)_k$ is a sequence of vectors such that $\sum_{k=1}^{\infty}\norm{x_k}$ converges, then $\sum_{k=1}^{\infty}x_k$ converges.
  \item If $\left(x_k\right)_k$ is a sequence in $X$ such that $\norm{x_k} < 2^{-k}$, then $\sum_{k=1}^{\infty}x_k$ converges.
\end{enumerate}
\end{proposition}
\begin{proof}
  To show (i) implies (ii), for $n > m > N$, we have
  \begin{align*}
    \norm{s_n - s_m} &= \norm{\sum_{k=m+1}^{n}x_k}\\
                     &\leq \sum_{k=m+1}^{n}\norm{x_k}\\
                     &< \epsilon,
  \end{align*}
  implying that $s_n$ is Cauchy, and thus converges since $X$ is complete.\newline

  Since $\sum_{k=1}^{\infty}2^{-k}$ converges, it is clear that (ii) implies (iii).\newline

  To show (iii) implies (i), we let $\left(x_n\right)_n$ be a Cauchy sequence in $X$. We only need construct a convergent subsequence in order to show that $\left(x_n\right)_n$ converges.\newline

  Chose $n_1\in \N$ such that for $n,m\geq n_1$, $ \norm{x_m - x_n} < \frac{1}{2^2}$, and inductively define $n_j > n_{j-1}$ such that $n,m\geq n_j$ implies $\norm{x_m - x_n} < \frac{1}{2^{j+1}}$.\newline

  Let $y_1 = x_{n_1}$, $y_{j} = x_{n_j} - x_{n_{j-1}}$. Then,
  \begin{align*}
    \norm{y_j} &= \norm{x_{n_j} - x_{n_{j-1}}}\\
               &< \frac{1}{2^{j}},
  \end{align*}
  so $\sum_{j=1}^{\infty}y_j$ converges by our assumption. By telescoping, we see that $\sum_{j=1}^{k}y_j = x_{n_k}$, so $\left(x_{n_{k}}\right)_k$ converges.
\end{proof}
\subsection{Quotient Spaces}%
Let $X$ be a normed vector space. Then, for $E\subseteq X$ a subspace, there is a quotient space $X/E$ with the projection map $\pi: X\rightarrow X/E$, $x\mapsto x + E$. We want to make $X/E$ into a normed space --- in order to do this, we use the distance function:
\begin{align*}
  \dist_{E}(x) &= \inf_{y\in E}d(x,y),
\end{align*}
which is uniformly continuous. For $E$ closed, then $\dist_{E}(x) = 0$ if and only if $x\in E$.
\begin{proposition}[Quotient Space Norm]
  Let $X$ be a normed vector space, and $E\subseteq X$ a subspace. Set
  \begin{align*}
    \norm{x + E}_{X/E} &= \dist_{E}(x).
  \end{align*}
  Then,
  \begin{enumerate}[(1)]
    \item $\norm{\cdot}_{X/E}$ is a well-defined seminorm on $X/E$.
    \item If $E$ is closed, then $\norm{\cdot}_{X/E}$ is a norm on $X/E$.
    \item $\norm{x+E}_{X/E} \leq \norm{x}$ for all $x\in X$.
    \item If $E$ is closed, then $\pi: X\rightarrow X/E$ is Lipschitz.
    \item If $X$ is a Banach space and $E$ is closed, then $X/E$ is also a Banach space.
  \end{enumerate}
\end{proposition}
\begin{proof}\hfill
  \begin{enumerate}[(1)]
    \item We will show that $\norm{\cdot}_{X/E}$ is well-defined. If $x + E = x' + E$, $x'-x\in E$, so for every $y\in E$, $x'-x + y\in E$. Thus,
    \begin{align*}
      \norm{x-y} &= \norm{x'-\left(x'-x+y\right)}\\
                 &\geq \inf_{z\in E}\norm{x' - z}\\
                 &= \norm{x' + E}_{X/E}.
    \end{align*}
    Thus, $\norm{x + E}_{X/E} \geq \norm{x' + E}_{X/E}$, and vice versa.\newline

    Let $\lambda \in \C\setminus \set{0}$, and $x\in X$. Then,
    \begin{align*}
      \norm{\lambda\left(x + E\right)}_{X/E} &= \norm{\lambda x + E}_{X/E}\\
                                             &= \inf_{y\in E}\norm{\lambda x - y}\\
                                             &= |\lambda|\inf_{y\in E}\norm{x - \lambda^{-1}y}\\
                                             &= |\lambda|\inf_{y'\in E}\norm{x-y}\\
                                             &= |\lambda|\norm{x + E}_{X/E}
    \end{align*}
    Given $x,x'\in X$ and a fixed $\ve > 0$, we have
    \begin{align*}
      \norm{x + E} + \frac{\ve}{2} &> \norm{x-y}
    \end{align*}
    for some $y\in E$, and
    \begin{align*}
      \norm{x' + E} + \frac{\ve}{2} &> \norm{x'-y'}
    \end{align*}
    for some $y'\in E$. Thus,
    \begin{align*}
      \norm{\left(x+x'\right)-\left(y+y'\right)} &\leq \norm{x-y} + \norm{x' - y'}\\
                                                 &< \ve + \norm{x + E} + \norm{x' + E}.
    \end{align*}
    Since $y + y'\in E$, we have
    \begin{align*}
      \norm{\left(x+E\right) + \left(x' + E\right)}_{X/E} &= \norm{x + x' + E}_{X/E}\\
                                                    &\leq \norm{\left(x+x'\right) - \left(y+y'\right)}\\
                                                    &< \ve + \norm{x + E}_{X/E} + \norm{x' + E}_{X/E},
    \end{align*}
    meaning
    \begin{align*}
      \norm{\left(x+E\right) + \left(x' + E\right)} \leq \norm{x + E} + \norm{x' + E}.
    \end{align*}
  \item If $E$ is closed, and $\norm{x + E} = 0$, then $x\in E$ so $x + E = 0_{X/E}$.
  \item For $x\in X$,
    \begin{align*}
      \norm{x + E}_{X/E} &= \inf_{y\in E}\norm{x-y}\\
                         &\leq \norm{x}.
    \end{align*}
  \item We have
    \begin{align*}
      \norm{\left(x+E\right) - \left(x' + E\right)}_{X/E} &= \norm{x-x' + E}_{X/E}\\
                                                          &\leq \norm{x-x'}.
    \end{align*}
  \item Let $X$ be complete and $E\subseteq X$ be closed. Let $\left(x_k + E\right)_k$ be a sequence in $X/E$ with $\norm{x_k + E} < 2^{-k}$. We want to show that $\sum_{k=1}^{\infty}\left(x_k + E\right)$ converges.\newline

    For each $k$, since $\norm{x_k + E} < 2^{-k}$, there exists $y_k\in E$ such that $\norm{x_k - y_k} < 2^{-k}$. Since $X$ is complete, $\sum_{k=1}^{\infty}x_k - y_k$ converges.\newline

    Let $\left(\sum_{k=1}^{n}x_k - y_k\right)_n \rightarrow x$ in $X$. Applying the canonical projection map, $\pi$, to both sides, we get
    \begin{align*}
      \sum_{k=1}^{n}\left(x_k + E\right) &= \sum_{k=1}^{n}\pi\left(x_k\right)\\
                                         &= \pi\left(\sum_{k=1}^{n}\left(x_k - y_k\right)\right)\\
                                         &\rightarrow \pi(x),
    \end{align*}
    implying that $\sum_{k=1}^{\infty}\left(x_k + E\right)$ converges.
  \end{enumerate}
\end{proof}
\begin{exercise}
  Consider $\ell_{\infty}$ and its closed subspace $c_0$. If $\pi: \ell_{\infty}\rightarrow \ell_{\infty}/c_0$ denotes the canonical quotient map, with $\left(z_k\right)_k\in \ell_{\infty}$, show that
  \begin{align*}
    \norm{\left(z_k\right)_k + c_0} &= \limsup_{k\rightarrow\infty}\left\vert z_k \right\vert
  \end{align*}
\end{exercise}
\begin{solution}
  By the definition of the quotient norm, we have
  \begin{align*}
    \norm{\left(z_k\right)_k + c_0}_{\ell_{\infty}/c_0} &= \inf_{\left(a_k\right)_k\in c_0}\norm{\left(z_k\right)_k - \left(a_k\right)_k}\\
                                                        &= \inf_{\left(a_k\right)_k\in c_0}\sup_{k\in \N}\left\vert z_k - a_k \right\vert\\
                                                        &= \limsup_{k\rightarrow\infty}\left\vert z_k \right\vert.
  \end{align*}
\end{solution}
\subsection{Bounded Linear Operators}%
\begin{definition}[Continuous Functions]
  A function $f: \left(X, d_X\right)\rightarrow \left(Y,d_Y\right)$ is called Lipschitz if there is a constant $C>0$ such that
  \begin{align*}
    d_Y\left(f(x),f(x')\right) \leq Cd_x\left(x,x'\right)
  \end{align*}
  for all $x,x'\in X$.\newline

  If $C \leq 1$, a Lipschitz map is known as a contraction.\newline

  If
  \begin{align*}
    d_Y\left(f(x),f\left(x'\right)\right) = d_X\left(x,x'\right)
  \end{align*}
  for all $x,x'\in X$, then $f$ is known as an isometry.
\end{definition}
\begin{proposition}[Categorization of Continuous Linear Maps]
  Let $X$ and $Y$ be normed vector spaces, and let $T: X\rightarrow Y$ be a linear map. The following are equivalent:
  \begin{enumerate}[(i)]
    \item $T$ is continuous at $0$.
    \item $T$ is continuous.
    \item $T$ is uniformly continuous.
    \item $T$ is Lipschitz.
    \item There exists a constant $C > 0$ such that $\norm{T(x)}\leq C\norm{x}$ for all $x\in X$.
  \end{enumerate}
\end{proposition}
\begin{definition}[Bounded Linear Operator]
  Let $X$ and $Y$ be normed vector spaces, and let $T: X\rightarrow Y$ be a linear map.
  \begin{enumerate}[(1)]
    \item $T$ is bounded if $T\left(B_X\right)$ is bounded in $Y$. Equivalently, $T$ is bounded if and only if
      \begin{align*}
        \sup_{x\in B_X}\norm{T(x)} < \infty,
      \end{align*}
      or that $\exists r > 0$ such that $T\left(B_X\right) \subseteq B_Y\left(0,r\right)$.
    \item The operator norm of $T$ is the value
      \begin{align*}
        \norm{T}_{\text{op}} &= \sup_{x\in B_X}\norm{T(x)}.
      \end{align*}
  \end{enumerate}
\end{definition}
\begin{lemma}
  Let $T: X\rightarrow Y$ be a linear map between normed vector spaces. Then,
  \begin{align*}
    \norm{T}_{\text{op}} &= \sup_{x\in S_X}\norm{T(x)}
    \intertext{and for all $x\in X$,}
    \norm{T(x)} \leq \norm{T}_{\text{op}}\norm{x}.
  \end{align*}
\end{lemma}
\begin{lemma}
  Let $T: X\rightarrow Y$ be a bounded linear map between normed vector spaces. Then, for any $x\in X$ and $r > 0$,
  \begin{align*}
    r\norm{T}_{\text{op}}\leq \sup_{y\in B\left(x,r\right)}\norm{T(y)}
  \end{align*}
\end{lemma}
\begin{proof}
  Let $C = \sup_{y\in B\left(x,r\right)}\norm{T(y)}$. If $z\in B\left(0,r\right)$, then $z+x,z-x\in B(x,r)$, meaning
  \begin{align*}
    2T\left(z\right) &= T\left(z+x\right) + T\left(z-x\right),
  \end{align*}
  so by the triangle inequality, we get
  \begin{align*}
    2\norm{T(z)} &\leq \norm{T(z+x)} + \norm{T(z-x)}\\
                 &\leq 2\max\set{\norm{T(z+x)},\norm{T\left(z-x\right)}}\\
                 &\leq 2C.
  \end{align*}
  Thus,
  \begin{align*}
    \norm{T(z)} \leq \sup_{y\in B\left(x,r\right)}\norm{T(y)},
  \end{align*}
  meaning
  \begin{align*}
    r\norm{T}_{\text{op}} \leq \sup_{y\in B\left(x,r\right)}\norm{T(y)}.
  \end{align*}
\end{proof}

\end{document}
