\documentclass[10pt]{mypackage}

% sans serif font:
%\usepackage{cmbright,sfmath,bbold}
%\renewcommand{\mathcal}{\mathtt}

%Euler:
%\usepackage{newpxtext,eulerpx,eucal,eufrak}
%\renewcommand*{\mathbb}[1]{\varmathbb{#1}}
%\renewcommand*{\hbar}{\hslash}

\usepackage[light]{kpfonts}
%\renewcommand{\mathbb}{\mathds}
%\usepackage{homework}

\pagestyle{fancy} %better headers
\fancyhf{}
\rhead{Avinash Iyer}
\lhead{Amenability: A (Somewhat) Brief Introduction}

\setcounter{secnumdepth}{0}

\begin{document}
\RaggedRight
These are some definitions and ideas I will be using regularly throughout this presentation.
\begin{definition}[Groups]
  Let $A$ be a set, and let $\star\colon A\times A \rightarrow A$ be such that
  \begin{itemize}
    \item $a\star\left( b\star c \right) = \left( a\star b \right)\star c$;
    \item there exists $e_A\in A$ such that $e_A\star a = a\star e_A$ for all $a\in A$;
    \item for each $a\in A$, there is $a^{-1}$ such that $a\star a^{-1} = a^{-1}\star a = e_{A}$.
  \end{itemize}
  Then, we say the pair $\left( A,\star \right)$ is a \textit{group}. We abbreviate $a\star b = ab$.
\end{definition}
\begin{definition}[Subgroups and Quotient Groups]
  Let $G$ be a group.
  \begin{itemize}
    \item If $H\subseteq G$ is a subset such that for all $a,b\in H$, $ab^{-1}\in H$, then we say $H$ is a \textit{subgroup}.
    \item If $N\subseteq G$ is a subgroup such that for all $g\in G$ and $h\in N$, $ghg^{-1}\in N$, then we say $N$ is a \textit{normal subgroup}.
    \item If $N$ is a normal subgroup, we may define the equivalence relation $g\sim_{N}g'$  if $g^{-1}g'\in N$; the equivalence classes $gN\coloneq \left[ g \right]_{\sim_N}$ form the \textit{quotient group}, $G/N$.
    \item If $H\subseteq G$ is a subgroup, then the \textit{index} of $H$ is the number of cosets, $gH\coloneq \set{gh | h\in H}$, written $\left[ G:H \right]$.
  \end{itemize}
\end{definition}
\begin{definition}[Group Actions]
  If $G$ is a group, and $X$ is a set, then $\rho\colon G\times X\rightarrow X$ is called an \textit{action} of $G$ onto $X$ if $\rho$ satisfies
  \begin{itemize}
    \item $\rho\left( e_G,x \right) = x$;
    \item $\rho\left( g,\rho\left( h,x \right) \right) = \rho\left( gh,x \right)$.
  \end{itemize}
  We write $\rho\left( g,x \right) = g\cdot x$.\newline

  Every group is equipped with a family of canonical actions, $\sigma\colon G\times G \rightarrow G$, given by $\left( a,x \right)\mapsto ax$, known as \textit{left-multiplication}.
\end{definition}
\begin{definition}[Algebras, $\sigma$-Algebras of Subsets]
  If $X$ is a set, then a collection $\mathcal{A} = \set{A_i}_{i\in I}\subseteq P(X)$ is known as an \textit{algebra} of subsets of $X$ if
  \begin{enumerate}[(1)]
    \item $\emptyset,X\in \mathcal{A}$;
    \item for all $A_i\in \mathcal{A}$, $A_i^{c}\in \mathcal{A}$;
    \item for all $A_i,A_j\in \mathcal{A}$, $A_i\cup A_j\in \mathcal{A}$.
  \end{enumerate}
  If condition (3) holds for any countable subcollection $\set{A_n}_{n\geq 1}\subseteq \mathcal{A}$, then we say $\mathcal{A}$ is a \textit{$\sigma$-algebra} of subsets.
\end{definition}
\begin{definition}[Measures]
  If $X$ is a set and $\mathcal{A}$ is a $\sigma$-algebra, then a map $\mu\colon \mathcal{A}\rightarrow [0,\infty]$ that satisfies
  \begin{itemize}
    \item $\mu\left( \emptyset \right) = 0$;
    \item for disjoint $A,B\in \mathcal{A}$, $\mu\left( A\sqcup B \right) = \mu\left( A \right) + \mu\left( B \right)$,
  \end{itemize}
  then we say $\mu$ is a \textit{finitely additive measure}. If, for any countable collection of disjoint sets, $\set{A_n}_{n\geq 1}$, we have
  \begin{align*}
    \mu\left( \bigsqcup_{n\geq 1}A_n \right) &= \sum_{n\geq 1}\mu\left( A_n \right),
  \end{align*}
  then we say $\mu$ is a \textit{measure}. If $\mu\left( X \right) = 1$, then we say $\mu$ is a probability measure.
\end{definition}

\end{document}
