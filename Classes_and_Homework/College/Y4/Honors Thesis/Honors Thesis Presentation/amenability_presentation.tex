\documentclass{beamer-custom}
\usepackage{beamer-custom}

\usefonttheme{serif}
% Theme
\usetheme{Montpellier}
\usecolortheme{beaver}
\useinnertheme{circles}

% Navigation
\setbeamertemplate{navigation symbols}{}
\setbeamertemplate{footline}[frame number]

% Frame title formatting
\setbeamertemplate{frametitle}{
    \vspace{0.5em}
    \insertframetitle
    \par
}
%\usepackage{cmbright,bbold}
%\renewcommand{\mathcal}{\mathtt}
\usepackage{kpfonts}
\AtBeginSection[]
{
  \begin{frame}
    \frametitle{Contents}
    \tableofcontents[currentsection]
  \end{frame}
}
% Title page
\title{Amenability: A (Somewhat) Brief Introduction}
\author{Avinash Iyer}
\institute{Occidental College}
\date{March 20, 2025}

\begin{document}
\begin{frame}
    \titlepage
\end{frame}

\begin{frame}{Outline}
    \tableofcontents
\end{frame}
\section{Definitions}%
\begin{frame}
  \frametitle{Groups}
  If $A$ is a set, and $\star\colon A\times A \rightarrow A$ is an operation such that
  \begin{itemize}
    \item $a\star\left(b\star c\right) = \left(a\star b\right)\star c$;
    \item there exists $e_A$ such that $a\star e_A = e_A\star a = a$;
    \item for each $a$ there exists $a^{-1}$ such that $a\star a^{-1} = a^{-1}\star a = e_A$,
  \end{itemize}
  then we call the pair $\left(A,\star\right)$ a \textit{group}.\pause\hfill\break

  We abbreviate $a\star b$ as $ab$.
\end{frame}
\begin{frame}
  \frametitle{Subgroups, Quotient Groups}
  Let $G$ be a group.
  \begin{itemize}
    \item If $H\subseteq G$ is a subset that satisfies, for all $a,b\in H$, $ab^{-1}\in H$, then we say $H$ is a \textit{subgroup}.\pause
    \item If $N\subseteq G$ is a subgroup that satisfies, for all $g\in G$ and $h\in N$, $ghg^{-1}\in N$, then we say $N$ is a \textit{normal subgroup}.\pause
    \item The equivalence classes under the relation $g\sim_{N} g'$ if $g^{-1}g' \in N$ form a group $gN\coloneq \left[ g \right]_{\sim}$ known as the \textit{quotient group} $G/N$.
  \end{itemize}
\end{frame}
\begin{frame}
  \frametitle{Some Groups}
  \begin{itemize}
    \item The integers $\Z$ are a group under addition.
    \item The group of invertible $n\times n$ matrices over $\C$, $\text{GL}_{n}\left(\C\right)$, is a group under matrix multiplication.
    \item The subgroup $\text{SO}(n)\subseteq \text{GL}_{n}\left( \R \right)$ consisting of $n\times n$ orthogonal matrices with determinant $1$ is a group under multiplication.
  \end{itemize}
\end{frame}
\begin{frame}
  \frametitle{Group Actions}
  Let $G$ be a group, and $X$ a set. Let $\rho\colon G\times X \rightarrow X$ be a function that satisfies, for all $g,h\in G$ and $x\in X$,
  \begin{itemize}
    \item $\rho\left( e_G, x\right) = x$;
    \item $\rho\left( g,\rho\left( h,x \right) \right) = \rho\left( gh,x \right)$.
  \end{itemize}
  Then, we say $\rho$ is an \textit{action} of $G$ on $X$. We write $\rho\left( g,x \right) = g\cdot x$.
\end{frame}
\begin{frame}
  \frametitle{$\sigma$-Algebras and Measures}
  If $X$ is a set, then a collection of subsets $\set{A_{i}}_{i\in I} = \mathcal{A}\subseteq P(X)$ is known as an \textit{algebra} of subsets if
  \begin{enumerate}
    \item $\emptyset,X\in \mathcal{A}$;
    \item for any $A_i\in \mathcal{A}$, $A_i^{c}\in \mathcal{A}$;
    \item for any $A_i,A_j\in \mathcal{A}$, $A_i\cup A_j\in \mathcal{A}$.
  \end{enumerate}\pause
  If, for any countable collection, $\set{A_n}_{n\geq 1}\subseteq \mathcal{A}$, condition (3) holds, then we say $\mathcal{A}$ is a \textit{$\sigma$-algebra} of subsets.
\end{frame}
\begin{frame}
  \frametitle{$\sigma$-Algebras and Measures, Cont'd}
  If $X$ is a set and $\mathcal{A}$ is a $\sigma$-algebra, then a map $\mu\colon \mathcal{A}\rightarrow [0,\infty]$ that satisfies:
  \begin{itemize}
    \item $\mu\left( \emptyset \right) = 0$;
    \item for disjoint sets $A,B\in \mathcal{A}$, $\mu\left( A\sqcup B \right) = \mu\left( A \right) + \mu\left( B \right)$,
  \end{itemize}
  then we say $\mu$ is a \textit{finitely additive} measure.\pause If $\set{A_n}_{n\geq 1}$ is a countable collection of disjoint sets, then if $\mu$ satisfies
  \begin{itemize}
    \item $\mu\left( \bigcup_{n\geq 1}A_n \right) = \sum_{n\geq 1}\mu\left( A_n \right)$,
  \end{itemize}
  we say $\mu$ is a measure.
\end{frame}
\section{Paradoxical Decompositions}%
\begin{frame}
  \frametitle{Questions?}
  \begin{itemize}
    \item If $G$ is a group, is it possible to reconstruct $G$ by using some subset of $G$?
    \item When may we find a finitely additive probability measure $\mu\colon P(G)\rightarrow [0,1]$ such that $\mu\left( E \right) = \mu\left( tE \right)$ for all $E\subseteq G$?
    \item Are these questions even related?
  \end{itemize}
\end{frame}
\begin{frame}
  \frametitle{Free Groups}
  \begin{itemize}
    \item We begin by considering a special group, known as $F(a,b)$ or the \textit{free group on two generators}.\pause
    \item We define $F(a,b)$ to be the set of all ``words'' in the alphabet $\set{a,b,a^{-1},b^{-1}}$, subject to the condition that, for $w,w'\in F(a,b)$,
      \begin{align*}
        waa^{-1}w' \sim wa^{-1}aw' \sim ww'\\
        wbb^{-1}w' \sim wb^{-1}bw' \sim ww'.
      \end{align*}
    \item Examples: $a^2bab^{-1},b^{-1}a^2b^2ab\in F(a,b)$.
  \end{itemize}
\end{frame}
\section{From Paradoxical Decompositions to Amenability}%
\section{Equivalent Definitions}%
\end{document}
