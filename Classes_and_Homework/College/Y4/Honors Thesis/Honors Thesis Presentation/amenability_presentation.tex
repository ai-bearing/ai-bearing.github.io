\documentclass{beamer-custom}
\usepackage{beamer-custom}

% Theme
\usetheme{Montpellier}
\usecolortheme{beaver}
\useinnertheme{circles}

% Navigation
\setbeamertemplate{navigation symbols}{}
\setbeamertemplate{footline}[frame number]

% Frame title formatting
\setbeamertemplate{frametitle}{
    \vspace{0.5em}
    \insertframetitle
    \par
}

% Title page
\title{Amenability: A Not-Particularly-Brief Introduction}
\author{Avinash Iyer}
\institute{Occidental College}
\date{\today}

\begin{document}

\begin{frame}
    \titlepage
\end{frame}

\begin{frame}{Outline}
    \tableofcontents
\end{frame}
\section{Definitions and Motivation}%
\begin{frame}
  \frametitle{Groups}
  If $A$ is a set, and $\star\colon A\times A \rightarrow A$ is an operation such that
  \begin{itemize}
    \item $a\star\left(b\star c\right) = \left(a\star b\right)\star c$;
    \item there exists $e_A$ such that $a\star e_A = e_A\star a = a$;
    \item for each $a$ there exists $a^{-1}$ such that $a\star a^{-1} = a^{-1}\star a = e_A$,
  \end{itemize}
  then we call the pair $\left(A,\star\right)$ a group.\pause\hfill\break

  We abbreviate $a\star b$ as $ab$.
\end{frame}
\begin{frame}
  \frametitle{Some Groups}
  \begin{itemize}
    \item The integers $\Z$ are a group under addition.
    \item The group of invertible $n\times n$ matrices over $\C$, $\text{GL}_{n}\left(\C\right)$, is a group under addition.
    \item The 
  \end{itemize}
\end{frame}
\begin{frame}
  \frametitle{$\sigma$-Algebras and Measures}

\end{frame}
\section{From Paradoxical Decompositions to Amenability}%
\section{Equivalent Definitions}%
\end{document}
