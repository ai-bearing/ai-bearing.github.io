\documentclass{beamer-custom}
\usepackage{beamer-custom}

\usefonttheme{serif}
% Theme
\usetheme{Montpellier}
\usecolortheme{beaver}
\useinnertheme{circles}

% Navigation
\setbeamertemplate{navigation symbols}{}
\setbeamertemplate{footline}[frame number]

% Frame title formatting
\setbeamertemplate{frametitle}{
    \vspace{0.5em}
    \insertframetitle
    \par
}
%\usepackage{cmbright,bbold}
%\renewcommand{\mathcal}{\mathtt}
\usepackage{kpfonts}

\AtBeginSection[]
{
  \begin{frame}
    \frametitle{Contents}
    \tableofcontents[currentsection]
  \end{frame}
}
%\AtBeginSection[] {
%    \ifnum\insertsubsectionstartpage=\insertsectionstartpage
%        \relax%
%    \else%
%      \begin{frame}<beamer>
%      \frametitle{Contents} %
%      \tableofcontents[currentsection]  
%      \end{frame}
%  \fi%
%}
%
%\AtBeginSubsection[] {
%    \begin{frame}<beamer>
%    \frametitle{Contents} %
%    \tableofcontents[currentsubsection]  
%    \end{frame}
%}
\usepackage[backend=biber, style=alphabetic,sorting=nyt]{biblatex}
\addbibresource{references.bib}
% Title page
\title{Amenability: A (Somewhat) Brief Introduction}
\author{Avinash Iyer}
\institute{Occidental College}
\date{March 20, 2025}
\DeclareMathOperator{\op}{op}

\begin{document}
\begin{frame}
    \titlepage
\end{frame}

\begin{frame}{Outline}
    \tableofcontents
\end{frame}
\section{Definitions}%
\begin{frame}
  \frametitle{Groups}
  If $A$ is a set, and $\star\colon A\times A \rightarrow A$ is an operation such that
  \begin{itemize}
    \item $a\star\left(b\star c\right) = \left(a\star b\right)\star c$;\pause
    \item there exists $e_A$ such that $a\star e_A = e_A\star a = a$;\pause
    \item for each $a$ there exists $a^{-1}$ such that $a\star a^{-1} = a^{-1}\star a = e_A$,\pause
  \end{itemize}
  then we call the pair $\left(A,\star\right)$ a \textit{group}.\pause\hfill\break

  We (usually) abbreviate $a\star b$ as $ab$.\pause\:If $ab = ba$, then we say the group is \textit{abelian}.
\end{frame}
\begin{frame}
  \frametitle{Subgroups, Quotient Groups}
  Let $G$ be a group.
  \begin{itemize}
    \item If $H\subseteq G$ is a subset that satisfies, for all $a,b\in H$, $ab^{-1}\in H$, then we say $H$ is a \textit{subgroup}.\pause
    \item If $N\subseteq G$ is a subgroup that satisfies, for all $g\in G$ and $h\in N$, $ghg^{-1}\in N$, then we say $N$ is a \textit{normal subgroup}.\pause
    \item The equivalence classes under the relation $g\sim_{N} g'$ if $g^{-1}g' \in N$ form a group $gN\coloneqq \left[ g \right]_{\sim}$ known as the \textit{quotient group} $G/N$.\pause
    \item The \textit{index} of a subgroup $H\leq G$ is the number of cosets, $gH\coloneqq \set{gh | h\in H}$, written $\left[ G:H \right]$.
  \end{itemize}
\end{frame}
\begin{frame}
  \frametitle{Some Groups}
  \begin{itemize}
    \item The integers $\Z$ are a group under addition.\pause
    \item The group $\text{SO}(n)$ consisting of $n\times n$ orthogonal matrices with determinant $1$ is a group under matrix multiplication.\pause
    \item The group $\text{E}\left( 3 \right)$ consists of all translations, rotations, and flips in $\R^3$, and is also known as the \textit{isometry group} of $\R^3$.
  \end{itemize}
\end{frame}
\begin{frame}
  \frametitle{Group Actions}
  Let $G$ be a group, and $X$ a set. Let $\rho\colon G\times X \rightarrow X$ be a function that satisfies, for all $g,h\in G$ and $x\in X$,\pause
  \begin{itemize}
    \item $\rho\left( e_G, x\right) = x$;\pause
    \item $\rho\left( g,\rho\left( h,x \right) \right) = \rho\left( gh,x \right)$.\pause
  \end{itemize}
  Then, we say $\rho$ is an \textit{action} of $G$ on $X$. We write $\rho\left( g,x \right) = g\cdot x$.\pause\:The above lines become $e_G\cdot x = x$ and $g\cdot \left( h\cdot x \right) = gh\cdot x$.\pause\:If $g\cdot x =x$ only when $g = e_G$, then we say the action is \textit{free}.\newline

  Every group is equipped with a family of canonical actions, $\sigma_{a}\colon G\rightarrow G$ for each $a\in G$, given by $x\mapsto ax$, known as \textit{left-multiplication}.
\end{frame}
\begin{frame}
  \frametitle{$\sigma$-Algebras and Measures}
  If $X$ is a set, then a collection of subsets $\set{A_{i}}_{i\in I} = \mathcal{A}\subseteq P(X)$ is known as an \textit{algebra} of subsets if
  \begin{enumerate}
    \item $\emptyset,X\in \mathcal{A}$;
    \item for any $A_i\in \mathcal{A}$, $A_i^{c}\in \mathcal{A}$;
    \item for any $A_i,A_j\in \mathcal{A}$, $A_i\cup A_j\in \mathcal{A}$.
  \end{enumerate}\pause
  If, for any countable collection, $\set{A_n}_{n\geq 1}\subseteq \mathcal{A}$, condition (3) holds, then we say $\mathcal{A}$ is a \textit{$\sigma$-algebra} of subsets.\pause\newline

  The most important $\sigma$-algebra, and the one we will be dealing with throughout this talk, is $P(G)$, where $G$ is a group.
\end{frame}
\begin{frame}
  \frametitle{$\sigma$-Algebras and Measures, Cont'd}
  If $X$ is a set and $\mathcal{A}$ is a $\sigma$-algebra, then a map $\mu\colon \mathcal{A}\rightarrow [0,\infty]$ that satisfies:
  \begin{itemize}
    \item $\mu\left( \emptyset \right) = 0$;
    \item for disjoint sets $A,B\in \mathcal{A}$, $\mu\left( A\sqcup B \right) = \mu\left( A \right) + \mu\left( B \right)$,
  \end{itemize}
  then we say $\mu$ is a \textit{finitely additive} measure.\pause \newline

  If $\set{A_n}_{n\geq 1}$ is a countable collection of disjoint sets, then if $\mu$ satisfies
  \begin{itemize}
    \item $\displaystyle \mu\left( \bigcup_{n\geq 1}A_n \right) = \sum_{n\geq 1}\mu\left( A_n \right)$,
  \end{itemize}
  we say $\mu$ is a measure.\pause\:If $\mu\left( X \right) = 1$, then we say $\mu$ is a probability measure.
\end{frame}
\section{Paradoxical Decompositions}%
\begin{frame}
  \frametitle{Motivating Questions}
  \begin{itemize}
    \item If $G$ is a group, is it possible to reconstruct $G$ by using some subset of $G$?\pause
    \item When may we find a finitely additive probability measure $\mu\colon P(G)\rightarrow [0,1]$ such that $\mu\left( E \right) = \mu\left( tE \right)$ for all $E\subseteq G$?\pause
    \item Are these questions even related?
  \end{itemize}
\end{frame}
\begin{frame}
  \frametitle{Free Groups}
  \begin{itemize}
    \item We begin by considering a special group, known as $F(a,b)$ or the \textit{free group on two generators}.\pause
    \item We define $F(a,b)$ to be the set of all ``words'' in the alphabet $\set{a,b,a^{-1},b^{-1}}$, subject to the condition that, for $w,w'\in F(a,b)$,
      \begin{align*}
        waa^{-1}w' \sim wa^{-1}aw' \sim ww'\\
        wbb^{-1}w' \sim wb^{-1}bw' \sim ww'.
      \end{align*}
    \item Examples: $a^2bab^{-1},b^{-1}a^2b^2ab\in F(a,b)$.
  \end{itemize}
\end{frame}
\begin{frame}
  \frametitle{A Curiosity}
  Let $W(b)\subseteq F(a,b)$ be all the words that start with $b$. Then, $b^{-1}W(b)$ consists of \pause
  \begin{itemize}
    \item all words that start with $a$;\pause
    \item all words that start with $a^{-1}$;\pause
    \item all words that start with $b$ --- think words that start with $b^2$ before you multiply $b^{-1}$.\pause
  \end{itemize}
  Thus, all we need to do is add back $W\left( b^{-1} \right)$ to get $F(a,b)$ back.
  \begin{align*}
    F(a,b) &= W\left( b^{-1} \right)\cup b^{-1}W(b).
  \end{align*}
\end{frame}
\begin{frame}
  \frametitle{A Curiosity, Cont'd}
  Similarly, we can do this for $a$, giving a decomposition of $F(a,b)$ in two separate ways:
  \begin{align*}
    F(a,b) &= b^{-1}W(b)\cup W\left( b^{-1} \right)\\
           &= a^{-1}W(a)\cup W\left( a^{-1} \right).
  \end{align*}\pause
  Furthermore, note that $W\left( a \right),W\left( b \right),W\left( a^{-1} \right),W\left( b^{-1} \right)$ are disjoint.\pause\newline

  We're able to take part of the group $F(a,b)$, take some translations, and, miraculously, obtain the entire group back.
\end{frame}
\begin{frame}
  \frametitle{Paradoxical Decompositions of Groups}
  Let $G$ be a group. A \textit{paradoxical decomposition} of $G$ consists of
  \begin{itemize}
    \item pairwise disjoint subsets $A_1,\dots,A_n,B_1,\dots,B_m\subseteq G$; and
    \item elements $g_1,\dots,g_n,h_1,\dots,h_m\in G$;
  \end{itemize}
  such that
  \begin{align*}
    G &= \bigcup_{i=1}^{n}g_iA_i\\
      &= \bigcup_{j=1}^{m}h_jB_j.
  \end{align*}\pause
  If $G$ admits a paradoxical decomposition, we say $G$ is \textit{paradoxical}.
\end{frame}
\begin{frame}
  \frametitle{Paradoxical Decompositions of Sets}
  If $G$ acts on a set $X$, then a subset $A\subseteq X$ is \textit{$G$-paradoxical} if there exist
  \begin{itemize}
    \item pairwise disjoint subsets $A_1,\dots,A_n,B_1,\dots,B_m\subseteq A$; and
    \item elements $g_1,\dots,g_n,h_1,\dots,h_m\in G$
  \end{itemize}
  such that
  \begin{align*}
    A &= \bigcup_{i=1}^{n}g_i\cdot A_i\\
      &= \bigcup_{j=1}^{m}h_j\cdot B_j.
  \end{align*}\pause
  A paradoxical group is a paradoxical set under the action of left-multiplication.\pause\:If $G$ is paradoxical and acts freely on $A\subseteq X$, then $A$ is $G$-paradoxical.
\end{frame}
\begin{frame}
  \frametitle{Depiction}
%  \begin{Center}
%    \includegraphics[scale=0.17]{images/paradoxical_decomposition_3.png}
%  \end{center}
  \begin{center}
\tikzset{every picture/.style={line width=0.75pt}} %set default line width to 0.75pt        

\begin{tikzpicture}[x=0.75pt,y=0.75pt,yscale=-1,xscale=1,scale=0.5]
%uncomment if require: \path (0,463); %set diagram left start at 0, and has height of 463

%Shape: Circle [id:dp7888194146123797] 
\draw   (181,100.5) .. controls (181,57.15) and (216.15,22) .. (259.5,22) .. controls (302.85,22) and (338,57.15) .. (338,100.5) .. controls (338,143.85) and (302.85,179) .. (259.5,179) .. controls (216.15,179) and (181,143.85) .. (181,100.5) -- cycle ;
%Shape: Circle [id:dp617800520698538] 
\draw   (442,105.5) .. controls (442,62.15) and (477.15,27) .. (520.5,27) .. controls (563.85,27) and (599,62.15) .. (599,105.5) .. controls (599,148.85) and (563.85,184) .. (520.5,184) .. controls (477.15,184) and (442,148.85) .. (442,105.5) -- cycle ;
%Shape: Circle [id:dp7464117940471128] 
\draw   (323,316.5) .. controls (323,273.15) and (358.15,238) .. (401.5,238) .. controls (444.85,238) and (480,273.15) .. (480,316.5) .. controls (480,359.85) and (444.85,395) .. (401.5,395) .. controls (358.15,395) and (323,359.85) .. (323,316.5) -- cycle ;
%Curve Lines [id:da5554211938824235] 
\draw [line width=0.75] [line join = round][line cap = round]   (291,173) .. controls (291,159.08) and (283.63,146.13) .. (279,133) .. controls (271.16,110.79) and (278.88,78.12) .. (295,62) .. controls (300.41,56.59) and (306.47,45) .. (313,45) ;
%Curve Lines [id:da08915852048267425] 
\draw [line width=0.75] [line join = round][line cap = round]   (209.39,161.59) .. controls (209.39,130.53) and (249.15,131.7) .. (270.89,132.41) .. controls (274.41,132.52) and (277.64,135.65) .. (280.61,135.65) ;
%Curve Lines [id:da035853299183386755] 
\draw [line width=0.75] [line join = round][line cap = round]   (260,22) .. controls (260,26.67) and (259.34,31.38) .. (260,36) .. controls (261.29,45.01) and (280.44,69) .. (289,69) ;
%Curve Lines [id:da3831843525173042] 
\draw [line width=0.75] [line join = round][line cap = round]   (220.28,32.86) .. controls (220.28,52.33) and (233.93,70.2) .. (223.48,89.3) .. controls (220.12,95.42) and (216.8,104.42) .. (213.89,109.73) .. controls (211.59,113.94) and (203.25,119.45) .. (200.04,122.38) .. controls (197.37,124.82) and (188.33,128.76) .. (188.33,131.14) ;
%Curve Lines [id:da08149494112624311] 
\draw    (188,105) .. controls (89.49,217.44) and (273.15,339.77) .. (338.03,365.62) ;
\draw [shift={(339,366)}, rotate = 201.34] [color={rgb, 255:red, 0; green, 0; blue, 0 }  ][line width=0.75]    (10.93,-3.29) .. controls (6.95,-1.4) and (3.31,-0.3) .. (0,0) .. controls (3.31,0.3) and (6.95,1.4) .. (10.93,3.29)   ;
%Curve Lines [id:da8168391395626808] 
\draw    (236,163) .. controls (201.35,212.5) and (301.95,255.14) .. (347.63,256.96) ;
\draw [shift={(349,257)}, rotate = 181.27] [color={rgb, 255:red, 0; green, 0; blue, 0 }  ][line width=0.75]    (10.93,-3.29) .. controls (6.95,-1.4) and (3.31,-0.3) .. (0,0) .. controls (3.31,0.3) and (6.95,1.4) .. (10.93,3.29)   ;
%Curve Lines [id:da9056524070522074] 
\draw    (209,132) .. controls (165.44,212.19) and (275.76,298.26) .. (321.63,309.68) ;
\draw [shift={(323,310)}, rotate = 192.53] [color={rgb, 255:red, 0; green, 0; blue, 0 }  ][line width=0.75]    (10.93,-3.29) .. controls (6.95,-1.4) and (3.31,-0.3) .. (0,0) .. controls (3.31,0.3) and (6.95,1.4) .. (10.93,3.29)   ;
%Curve Lines [id:da22269351080305821] 
\draw    (320,135) .. controls (333.86,203.31) and (423.19,198.11) .. (463.78,168.89) ;
\draw [shift={(465,168)}, rotate = 143.13] [color={rgb, 255:red, 0; green, 0; blue, 0 }  ][line width=0.75]    (10.93,-3.29) .. controls (6.95,-1.4) and (3.31,-0.3) .. (0,0) .. controls (3.31,0.3) and (6.95,1.4) .. (10.93,3.29)   ;
%Curve Lines [id:da5779701148660346] 
\draw    (285,33) .. controls (324.4,3.45) and (448.21,20.47) .. (491.1,27.68) ;
\draw [shift={(493,28)}, rotate = 189.69] [color={rgb, 255:red, 0; green, 0; blue, 0 }  ][line width=0.75]    (10.93,-3.29) .. controls (6.95,-1.4) and (3.31,-0.3) .. (0,0) .. controls (3.31,0.3) and (6.95,1.4) .. (10.93,3.29)   ;

% Text Node
\draw (302,94) node [anchor=north west][inner sep=0.75pt]   [align=left] {\tiny$ A_{1}$};
% Text Node
\draw (273,33) node [anchor=north west][inner sep=0.75pt]   [align=left] {\tiny$ A_{2}$};
% Text Node
\draw (241,145) node [anchor=north west][inner sep=0.75pt]   [align=left] {\tiny $ B_{1}$};
% Text Node
\draw (239,75) node [anchor=north west][inner sep=0.75pt]   [align=left] {\tiny $ B_{2}$};
% Text Node
\draw (191,73) node [anchor=north west][inner sep=0.75pt]   [align=left] {\tiny$ B_{3}$};
% Text Node
\draw (188,287) node [anchor=north west][inner sep=0.75pt]   [align=left] {\small $ h_{3}$};
% Text Node
\draw (241,238) node [anchor=north west][inner sep=0.75pt]   [align=left] {\small$ h_{2}$};
% Text Node
\draw (260,204) node [anchor=north west][inner sep=0.75pt]   [align=left] {\small$ h_{1}$};
% Text Node
\draw (381,161) node [anchor=north west][inner sep=0.75pt]   [align=left] {\small$ g_{1}$};
% Text Node
\draw (400,19) node [anchor=north west][inner sep=0.75pt]   [align=left] {\small $ g_{2}$};


\end{tikzpicture}
  \end{center}
\end{frame}
\begin{frame}
  \frametitle{Some Paradoxical Groups}
  \begin{itemize}
    \item The free group $F(a,b)$ is paradoxical.\pause
    \item Any group that contains a paradoxical subgroup is paradoxical.\pause
    \item $F(S)$, where $S$ is any nonempty set with more than two elements, is paradoxical.
  \end{itemize}
\end{frame}
\begin{frame}
  \frametitle{A Paradoxical Subgroup of $\text{SO}(3)$}
  The following two matrices (and their inverses) generate a subgroup of $\text{SO}(3)$ that is isomorphic to $F(a,b)$.
  \begin{align*}
    A &= \begin{pmatrix}3/5 & 4/5 & 0 \\  -4/5 & 3/5 & 0 \\ 0 & 0 & 1\end{pmatrix}\\
    B &= \begin{pmatrix}1 & 0 & 0 \\ 0 & 3/5 & -4/5 \\ 0 & 4/5 & 3/5\end{pmatrix}.
  \end{align*}\pause
  Thus, $\text{SO}(3)$ is paradoxical\pause\:--- can we use it to find a paradoxical decomposition?
\end{frame}
\begin{frame}
  \frametitle{Introducing the Banach--Tarski Paradox}
  \begin{theorem}[The Banach--Tarski Paradox]
    Let $A$ and $B$ be bounded subsets of $\R^3$ with nonempty interior. There is a partition of $A$ into finitely many disjoint subsets such that a sequence of isometries applied to these subsets yields $B$.
  \end{theorem}\pause
  \begin{itemize}
    \item In other words, not all subsets of $\R^3$ have a definite ``volume'' invariant under isometry.\pause
  \end{itemize}
  We need to introduce a few more concepts before we can show the proof of the paradox.
\end{frame}
\begin{frame}
  \frametitle{Equidecomposability}
  Let $G$ be a group that acts on a set $X$, and let $A,B\subseteq X$.\pause\:If there exist
  \begin{itemize}
    \item finite partitions, $A_1,\dots,A_n\subseteq A$, $B_1,\dots,B_n\subseteq B$
    \item group elements $g_1,\dots,g_n\in G$
  \end{itemize}
  such that $g_i\cdot A_i = B_i$, then we say $A$ and $B$ are \textit{$G$-equidecomposable}.\pause\newline

  Effectively, $A$ and $B$ are ``equal'' to each other up to the group action.\pause\newline

  If $A$ is $G$-paradoxical, then so too is $B$.
\end{frame}
\begin{frame}
  \frametitle{The Banach--Tarski Paradox: Proof Outline}
  \small
  \begin{itemize}
    \item Use the two matrices
      \begin{align*}
        A = \begin{pmatrix}3/5 & 4/5 & 0 \\  -4/5 & 3/5 & 0 \\ 0 & 0 & 1\end{pmatrix}
                               && B = \begin{pmatrix}1 & 0 & 0 \\ 0 & 3/5 & -4/5 \\ 0 & 4/5 & 3/5\end{pmatrix}.
      \end{align*}
      to generate a subgroup of $\text{SO}\left( 3 \right)$ isomorphic to $F(a,b)$.\pause
    \item Copy $F(a,b)$ over by using the decomposition
      \begin{align*}
        F(a,b) &= a^{-1}W\left( a \right)\cup W\left( a^{-1} \right)\\
                &= b^{-1}W\left( b \right) \cup W\left( b^{-1} \right).
      \end{align*}\pause
      Removing fixed points yields a decomposition of $S^{2}\setminus D$.\:This is known as the \textit{Hausdorff paradox}.
  \end{itemize}
\end{frame}
\begin{frame}
  \frametitle{The Banach--Tarski Paradox: Proof Outline, cont'd}
  \begin{itemize}
    \item We show that $S^{2}$ and $S^{2}\setminus D$ are $\text{SO}(3)$-equidecomposable.\pause\:There is thus a paradoxical decomposition of $S^{2}$.\pause
    \item Use scaling and an off-center rotation to show that $B(0,1)\subseteq \R^3$ is paradoxical under the isometry group $\text{E}(3)$.\pause
  \end{itemize}
  Thus, we have shown the \textit{weak} Banach--Tarski paradox.\pause\:For the full paradox, we need one more thing.
\end{frame}
\begin{frame}
  \frametitle{The Banach--Tarski Paradox: Proof Outline, cont'd}
  Consider the relation $A\preceq B$ if $A$ is $G$-equidecomposable with \textit{a subset of} $B$.\pause\newline

  This relation is reflexive (since all sets are subsets of themselves, take the identity action),\pause\:transitive (compose the group action),\pause\:and antisymmetric (this one takes a bit more work).\pause\newline

  Thus, to show that $A$ and $B$ are $G$-equidecomposable, it suffices to show that $A\preceq B$.\pause\:We do this, and we are done.
\end{frame}
%\begin{frame}[allowframebreaks]
%  \frametitle{The Banach--Tarski Paradox: Proof Outline}
%  \begin{enumerate}[(1)]
%    \item We use the two matrices
%      \begin{align*}
%        A &= \begin{pmatrix}3/5 & 4/5 & 0 \\  -4/5 & 3/5 & 0 \\ 0 & 0 & 1\end{pmatrix}\\
%        B &= \begin{pmatrix}1 & 0 & 0 \\ 0 & 3/5 & -4/5 \\ 0 & 4/5 & 3/5\end{pmatrix}.
%      \end{align*}
%      to generate a subgroup of $\text{SO}\left( 3 \right)$ isomorphic to $F(a,b)$.
%    \item We use the decomposition 
%      \begin{align*}
%        F\left( a,b \right) &= a^{-1}W\left( a \right)\cup W\left( a^{-1} \right)\\
%                            &= b^{-1}W\left( b \right) \cup W\left( b^{-1} \right)
%      \end{align*}
%      to duplicate the unit sphere in $\R^3$, $S^2$, except for a countable subset $D$. (The \textit{Hausdorff Paradox}.)
%    \item We show that $S^2$ and $S^2\setminus D$ are $\text{SO}\left( 3 \right)$-equidecomposable --- there is thus a paradoxical decomposition of $S^2$.
%    \item We show that the unit ball, $B\left( 0,1 \right)\subseteq \R^3$, is paradoxical under the isometry group $\text{E}\left( 3 \right)$.
%    \item Define a relation $A\preceq B$ if $A$ is $G$-equidecomposable with a subset of $B$, and show that if $A\preceq B$ and $B\preceq A$, then $A$ and $B$ are $G$-equidecomposable.
%    \item Show that $A\subseteq \R^3$ is equidecomposable with a subset of $B\subseteq \R^3$.\pause
%  \end{enumerate}
%  We're done...\pause or are we?
%\end{frame}
\section{From Paradoxical Decompositions to Amenability}%
\begin{frame}
  \frametitle{Ill-Behaved Groups}
  \begin{itemize}
    \item The way that our copy of $F(a,b)$ helped ``create'' the Banach--Tarski paradox suggests that $F(a,b)$ is a particularly ill-behaved group.\pause
    \item Let $\nu\colon F(a,b)\rightarrow [0,1]$ be a probability measure\pause\:--- we will show that $\nu$ \textit{cannot} be translation-invariant (i.e., $\nu\left( tE \right) = \nu\left( E \right)$ for all $t\in F(a,b),E\subseteq F(a,b)$).
  \end{itemize}
\end{frame}
\begin{frame}
  \frametitle{Ill-Behaved Groups, Cont'd}
  Suppose such a translation-invariant $\nu$ exists. Taking
  \begin{align*}
    F(a,b) &= W(a)\sqcup W\left( a^{-1} \right) \sqcup W\left( b \right) \sqcup W\left( b^{-1} \right),
  \end{align*}
  we have
  \begin{align*}
    1 &= \nu\left( W\left( a \right) \right) + \nu\left( W\left( a^{-1} \right) \right) + \nu\left( W\left( b \right) \right) + \nu\left( W\left( b^{-1} \right) \right)\\
    \uncover<+(1)->{&= \color{blue}{\nu\left( a^{-1}W\left( a \right) \right)} \color{black}{+ \nu\left( W\left( a^{-1} \right) \right)} + \color{blue}{\nu\left( b^{-1}W\left( b \right) \right)}\color{black} {+ \nu\left( W\left( b^{-1} \right) \right)}}\\
      \uncover<+(2)->{&= \nu\left( a^{-1}W\left( a \right)\sqcup W\left( a^{-1} \right) \right) + \nu\left( b^{-1}W\left( b \right) \sqcup W\left( b^{-1} \right) \right)}\\
      \uncover<+(3)->{&= \nu\left( F(a,b) \right) + \nu\left( F(a,b) \right)}\\
      \uncover<+(3)->{&= 2.}
  \end{align*}
\end{frame}
\begin{frame}
  \frametitle{Amenability}
  Let $G$ be a group. A \textit{mean} is a finitely additive probability measure $\nu\colon P(G)\rightarrow [0,1]$ such that
  \begin{align*}
    \nu\left( tE \right) &= \nu\left( E \right)
  \end{align*}
  for all $t\in G$ and $E\subseteq G$.\pause\newline

  If $G$ admits a mean, we say $G$ is \textit{amenable}.\pause
  \begin{itemize}
    \item In other words, $G$ is sufficiently ``well-behaved.''
  \end{itemize}
\end{frame}
\begin{frame}
  \frametitle{Inheritance Properties of Amenability}
  \begin{itemize}
    \item If $G$ is amenable, then any subgroup of $G$ is amenable.\pause
    \item If $G$ is amenable, then quotient groups, $G/N$, are amenable.\pause
    \item If $H\leq G$ is an amenable subgroup such that $\left[ G:H \right] < \infty$, then $G$ is amenable.\pause
    \item If $N\trianglelefteq G$ and $G/N$ are amenable, then $G$ is amenable.\pause
    \item If $G$ and $H$ are amenable, then so is $G\times H$.\pause
    \item If $\left( G_i,\varphi_i \right)_{i\in I}$ is a directed system of amenable groups, then the union $G = \bigcup_{i\in I}G_i$ is amenable.
  \end{itemize}
\end{frame}
\begin{frame}
  \frametitle{Examples}
  \begin{itemize}
    \item Finite groups are amenable: let $\delta_t$ be the point mass at $t\in G$,
      \begin{align*}
        \delta_t(s) &= \begin{cases}
          1 & t = s\\
          0 & t\neq s
        \end{cases}.
      \end{align*}
      Then,
      \begin{align*}
        \nu &= \frac{1}{\left\vert G \right\vert} \sum_{t\in G}\delta_t
      \end{align*}
      is a mean.
    \item Abelian groups are amenable.
    \item The free group, $F(a,b)$, is \textit{not} amenable.
  \end{itemize}
\end{frame}
\begin{frame}
  \frametitle{Paradoxical Groups and Amenability}
  Every paradoxical group is \textit{not} amenable --- the argument is similar to the case for $F(a,b)$.\pause\newline

  More surprisingly, though, every \textit{non}-paradoxical group is amenable.\pause
  \begin{theorem}[Tarski's Theorem]
    Let $G$ be a group. Then, $G$ is non-paradoxical if and only if $G$ is amenable.
  \end{theorem}\pause
  Unfortunately, the proof that every non-paradoxical group is amenable is significantly harder.
\end{frame}
\section{Equivalent Definitions and Other Criteria}%
% Følner condition, Cayley graph, talk about intuition behind Følner condition
% Growth, Functional Analysis, Representations, 
\subsection{A Taste of Functional Analysis}%
\begin{frame}
  \frametitle{Why Find Alternative Characterizations?}
  On first glance, it may seem like we're finished, but we're really not.\pause\newline

  Our methods so far --- the existence of a mean, or showing non-paradoxicality --- are quite difficult to establish.\pause\newline

  As it turns out, amenability touches a variety of fields:
  \begin{itemize}
    \item functional analysis;
    \item geometric group theory;
    \item representation theory;
    \item operator algebras.
  \end{itemize}
\end{frame}
\begin{frame}
  \frametitle{Normed Vector Spaces}
  Functional analysis is, of course, the study of normed vector spaces.\pause\newline

  If $V$ is a vector space, then a \textit{norm} on $V$ is a map $\norm{\cdot}\colon V\rightarrow [0,\infty)$ satisfying:\pause
  \begin{itemize}
    \item definiteness: $\norm{v} \geq 0$, with equality if and only if $v = 0$;\pause
    \item homogeneity: $\norm{\alpha v} = \left\vert \alpha \right\vert\norm{v}$ for all $\alpha\in \C$;\pause
    \item triangle inequality: $\norm{v + w}\leq \norm{v} + \norm{w}$.
  \end{itemize}
\end{frame}
\begin{frame}
  \frametitle{A Normed Vector Space}
  The best example is that of $\R^n$ or $\C^n$ with the Euclidean norm,
  \begin{align*}
    \norm{x} &= \left( \sum_{i=1}^{n}\left\vert x_i \right\vert^2 \right)^{1/2}
  \end{align*}\pause
  However, we need a few more dimensions in order to get to where we're going.
\end{frame}
\begin{frame}
  \frametitle{Function Spaces}
  There are three main function spaces that we're concerned with for our studies:
  { \footnotesize \begin{align*}
    \ell_{\infty}\left( \Gamma \right) & =\set{f\colon \Gamma\rightarrow \C | \sup_{t\in\Gamma}\left\vert f(t) \right\vert < \infty};\\
    \ell_{1}\left( \Gamma \right) &= \set{f\colon \Gamma\rightarrow \C | \sum_{t\in\Gamma}\left\vert f(t) \right\vert < \infty};\\
    \ell_2\left( \Gamma \right) &= \set{f\colon \Gamma\rightarrow \C | \sum_{t\in\Gamma}\left\vert f(t) \right\vert^2 < \infty}.
  \end{align*}}\pause
  They are equipped with the respective norms of
  \begin{itemize}
    \item $\norm{f}_{\ell_{\infty}} \coloneqq \sup_{t\in\Gamma}\left\vert f(t) \right\vert$;
    \item $\norm{f}_{\ell_1}\coloneqq \sum_{t\in\Gamma}\left\vert f(t) \right\vert$;
    \item $\norm{f}_{\ell_2}\coloneqq \left( \sum_{t\in\Gamma}\left\vert f(t) \right\vert^2 \right)^{1/2}$.
  \end{itemize}
\end{frame}
\begin{frame}
  \frametitle{Linear Maps and Linear Functionals}
  A linear transformation $T\colon V\rightarrow W$ is called \textit{bounded} if
  \begin{align*}
    \sup_{\norm{v}=1}\norm{T(v)} &< \infty.
  \end{align*}\pause
  We call the quantity on the left the \textit{operator norm}, denoted $\norm{T}_{\op}$.\pause\:A linear map is bounded if and only if it is continuous.\pause\newline

  If $W = \C$, then we call $T$ a \textit{linear functional}.
\end{frame}
\begin{frame}
  \frametitle{Operator Norm Pictorial Depiction}
  Courtesy of Tai-Danae Bradley.
  \begin{center}
    \includegraphics{images/operator_norm_depiction.png}
  \end{center}
\end{frame}
\begin{frame}
  \frametitle{Positive Linear Functionals on $\ell_{\infty}\left( \Gamma \right)$}
  If $\varphi\colon \ell_{\infty}\left( \Gamma \right)\rightarrow \C$ is a linear functional, we say $\varphi $ is \textit{positive} if, for any $f\in \ell_{\infty}\left( \Gamma \right)$ with $f \geq 0$, $\varphi(f)\geq 0$.
  \begin{itemize}
    \item It can be shown that $\varphi$ is positive if and only if $\varphi\left( \1_{\Gamma} \right) = \norm{\varphi}_{\op}$.\pause\:All positive linear functionals are automatically continuous.
    \item If $\varphi\left( \1_{\Gamma} \right) = \norm{\varphi}_{\op} = 1$, then we say $\varphi$ is a \textit{state}.
  \end{itemize}
\end{frame}
\begin{frame}
  \frametitle{Translations of $\ell_{\infty}\left( \Gamma \right)$}
  If $f\in \ell_{\infty}\left( \Gamma \right)$, we define the translation $\lambda_s\colon \ell_{\infty}\left( \Gamma \right)\rightarrow \ell_{\infty}\left( \Gamma \right)$ by
  \begin{align*}
    \lambda_s(f)(t) &= f\left( s^{-1}t \right)
  \end{align*}
  for all $t\in \Gamma$ and fixed $s\in\Gamma$.\pause\newline

  If $\varphi\colon \ell_{\infty}\left( \Gamma \right)\rightarrow \C$ is a state such that $\varphi\left( \lambda_s(f) \right) = \varphi(f)$ for all $f\in \ell_{\infty}\left( \Gamma \right)$, then we say $\varphi$ is an \textit{invariant state}.
\end{frame}
\begin{frame}
  \frametitle{Invariant States and Means}
  Invariant states and means are interchangeable.\pause\newline

  If $\varphi$ is an invariant state on $\ell_{\infty}\left( \Gamma \right)$, define
  \begin{align*}
    \mu\left( E \right) &= \varphi\left( \1_{E} \right)
  \end{align*}
  for all $E\subseteq \Gamma$.
\end{frame}
\subsection{Introducing Approximations}%
\begin{frame}
  \frametitle{Approximations and Amenability}
  There is actually one way that working with sets makes life easier.\pause\newline

  Remember when we decomposed
  \begin{align*}
    F(a,b) &= W(a) \sqcup W\left( a^{-1} \right) \sqcup W\left( b \right) \sqcup W\left( b^{-1} \right).
  \end{align*}\pause
  Translating $W\left( a \right) \mapsto a^{-1}W\left( a \right)$ gave us a set that was ``significantly'' ``bigger'' than $W\left( a^{-1} \right)$;\pause\:specifically, it gave us $F\left( a,b \right) \setminus W\left( a^{-1} \right)$.\pause\newline

  But what does ``bigger'' actually mean?
\end{frame}
\begin{frame}
  \frametitle{Følner's Condition}
  \begin{theorem}[Følner's Theorem]
    Let $\Gamma$ be a countable, discrete group. Then, $\Gamma$ is amenable if and only if there exists a sequence of finite subsets $\left( F_n \right)_n$ such that
    \begin{align*}
      \lim_{n\rightarrow\infty} \frac{\left\vert sF_n \cap F_n \right\vert}{\left\vert F_n \right\vert} &= 1
    \end{align*}
    for all $s\in\Gamma$.
  \end{theorem}\pause
  If $\Gamma = \left\langle S \right\rangle$ for some finite generating set $S$, we only need to check for all $s\in S$.
\end{frame}
\begin{frame}
  \frametitle{$\Z$ is Amenable}
  Consider $\Z = \set{\dots,-2,-1,0,1,2,\dots}$ under addition.\pause\newline

  We can find a Følner sequence by defining
  \begin{align*}
    F_n &= \set{-n,-n+1,\dots,-1,0,1,\dots,n-1,n}.
  \end{align*}
  Then, since $\Z$ is generated by $1$, we verify
  \begin{align*}
    \frac{\left\vert \left( F_n + 1 \right)\cap F_n \right\vert}{\left\vert F_n \right\vert} &= \frac{2n-1}{2n+1}\\
                                                                                             &\rightarrow 1.
  \end{align*}\pause
  Thus, $\Z$ is amenable.
\end{frame}
\begin{frame}
  \frametitle{Approximate Means}
  The Følner condition allows us to find an ``approximate'' version of a mean.\pause\newline

  Keeping $\lambda_s(f)(t) = f\left( s^{-1}t \right)$, if $\left( f_k \right)_k\subseteq \ell_1\left( \Gamma \right)$ is such that
  \begin{align*}
    \lim_{k\rightarrow\infty}\norm{f_k - \lambda_s\left( f_k \right)}_{\ell_1} &= 0,
  \end{align*}
  then we say $\left( f_k \right)_k$ is an \textit{approximate mean}.
  \end{frame}
  \begin{frame}
  \frametitle{Approximate Means, Cont'd}
  This is equal to Følner's condition.\newline

  In one direction, we take
  \begin{align*}
    f_k &= \frac{1}{\left\vert F_k \right\vert}\1_{F_k},
  \end{align*}
\end{frame}
\begin{frame}
  \frametitle{Approximate Means, Cont'd}
  In the other direction, we arbitrarily approximate $f\in\ell_1\left( \Gamma \right)$ with a ``sufficient'' finitely supported function $g$,
  \begin{align*}
    \norm{g - f}_{\ell_1} &< \ve/2,
  \end{align*}
  then use a ``layer cake'' decomposition to find our Følner sets:
  \begin{align*}
    g &= \sum_{i=1}^{n}c_i\1_{F_i},
  \end{align*}
  where $F_1\supseteq F_2\supseteq \cdots \supseteq F_n$.
\end{frame}
%\begin{frame}
%  \frametitle{Using Følner's Condition}
%  If $S$ is a (finite) generating set for $G$, then letting
%  \begin{align*}
%    S^n &\coloneqq \set{g_1\cdots g_n | g_i\in S},
%  \end{align*}
%  we call groups such that
%  \begin{align*}
%    \limsup_{n\rightarrow\infty} \left\vert S^n \right\vert^{1/n} &= 1
%  \end{align*}
%  \textit{groups of subexponential growth}. Følner's condition is used to show that they are amenable.
%\end{frame}
\begin{frame}
  \frametitle{Graphs and Amenability}
  Given a group $\Gamma$ with generating set $S$, we may define a graph --- known as the Cayley graph --- with vertices consisting of group elements and edges defined by ``walking'' along the generators.\pause
  \begin{center}
    \includegraphics[scale=0.06]{images/cayley_graph.png}
  \end{center}
\end{frame}
\begin{frame}
  \frametitle{Graphs and Amenability, cont'd}
  If $S\subseteq V(G)$ is a subset of vertices of a graph $G$, the \textit{neighbor vertex set}, $N(S)$, is the set of vertices in $G$ that are adjacent to $S$ (not including elements of $S$).\pause\newline

  If $G$ is the Cayley graph of $\Gamma$, then $\Gamma$ is amenable if and only if
  \begin{align*}
    \inf\set{\frac{\left\vert N(S) \right\vert}{\left\vert S \right\vert} | S\subseteq V(G),\left\vert S \right\vert\text{ finite}} &= 0.
  \end{align*}\pause
  \begin{itemize}
    \item Essentially, the Cayley graph doesn't ``get too big'' ``too fast.''\pause
    \item This is proven with the Følner condition.
  \end{itemize}
\end{frame}
\subsection{Approximations with Representations and Operators}%
\begin{frame}
  \frametitle{Hilbert Spaces}
  If $\mathcal{H}$ is a vector space, an \textit{inner product} on $\mathcal{H}$ is a map $ \iprod{\cdot}{\cdot}\colon \mathcal{H}\times \mathcal{H}\rightarrow \C $ that satisfies\pause
  \begin{itemize}
    \item $ \iprod{x}{x} \geq 0 $, with equality only when $x = 0$;\pause
    \item $ \iprod{x_1 + \alpha x_2}{y} = \iprod{x_1}{y} + \alpha \iprod{x_2}{y} $;\pause
    \item $ \iprod{x}{y_1 + \alpha y_2} = \iprod{x}{y_1} + \overline{\alpha} \iprod{x}{y_2} $.\pause
  \end{itemize}
  The inner product induces a norm $\norm{x}^2 = \iprod{x}{x}$.\pause\newline

  If $\mathcal{H}$ is complete with respect to this norm, we call $\mathcal{H}$ a Hilbert space.
\end{frame}
\begin{frame}
  \frametitle{Operators on Hilbert Spaces}
  Bounded linear maps on Hilbert spaces, $T\colon \mathcal{H}\rightarrow \mathcal{H}$, include a special structure called an adjoint that ``plays nicely'' with the inner product:
  \begin{align*}
    \iprod{T(x)}{y} &= \iprod{x}{T^{\ast}\left( y \right)}.
  \end{align*}\pause
  If $U\colon \mathcal{H}\rightarrow \mathcal{H}$ is such that
  \begin{align*}
    U^{\ast}U &= I\\
    UU^{\ast} &= I,
  \end{align*}
  then we call $U$ a \textit{unitary operator}.\pause\:The space of unitary operators, $\mathcal{U}\left( \mathcal{H} \right)$, is a group under composition.
\end{frame}
\begin{frame}
  \frametitle{Representations}
  A map $\lambda\colon \Gamma\rightarrow \mathcal{U}\left( \mathcal{H} \right)$ that satisfies
  \begin{align*}
    \lambda\left( st \right) &= \lambda(s)\lambda(t)\\
    \lambda\left( s^{-1} \right) &= \lambda\left( s \right)^{\ast}
  \end{align*}
  is called a \textit{unitary representation} of $\Gamma$.\newline

  All discrete groups are able to be unitarily represented\pause\:by the trivial representation $1_{\Gamma}\colon \Gamma\rightarrow \C$, given by $1_{\Gamma}(s) = 1$.
\end{frame}
\begin{frame}
  \frametitle{The Left-Regular Representation}
  As it turns out, the map $\lambda_s(f)(t) = f\left( s^{-1}t \right)$ is a unitary operator on $\ell_2\left( \Gamma \right)$, where $\lambda_s^{\ast} = \lambda_{s^{-1}}$.\pause\newline

  It can also be shown that $\lambda_s\lambda_t = \lambda_{st}$, meaning that the map $s\mapsto \lambda_s$ is a unitary representation.\pause\newline

  The map $\lambda\colon \Gamma\rightarrow \mathcal{U}\left( \ell_2\left( \Gamma \right) \right)$, given by $s\mapsto \lambda_s$ is a very special representation, known as the \textit{left-regular representation}.\pause\newline

  This is because it ``encodes'' the group's left-multiplication action, in the sense that $\lambda_s\left(\delta_t\right) = \delta_{st}$, where $\delta_t$ is the point mass at $t\in\Gamma$.
\end{frame}
\begin{frame}
  \frametitle{The Left-Regular Representation and Amenability}
  A sequence $\left( f_k \right)_k\subseteq \ell_2\left( \Gamma \right)$ is known as an \textit{almost-invariant vector} for $\lambda\colon \Gamma\rightarrow \mathcal{U}\left( \ell_2\left( \Gamma \right) \right)$ if
  \begin{align*}
    \lim_{k\rightarrow\infty}\norm{f_k - \lambda_s\left( f_k \right)}_{\ell_2} &= 0.
  \end{align*}\pause
  If $\lambda\colon \Gamma\rightarrow \mathcal{U}\left( \ell_2\left( \Gamma \right) \right)$ admits an almost-invariant vector, then $\Gamma$ is amenable.
\end{frame}
\begin{frame}
  \frametitle{Introduction to $C^{\ast}$-Algebras}
  \small
  The space of \textit{all} bounded linear operators, $T\colon \mathcal{H}\rightarrow \mathcal{H}$, written $\B\left( \mathcal{H} \right)$, along with the norm $\norm{\cdot}_{\op}$, is a very special vector space.\pause\:The adjoint map satisfies:\pause
  \begin{itemize}
    \item $\left( T + \alpha S \right)^{\ast} = T^{\ast} + \overline{\alpha} S^{\ast}$;\pause
    \item $T^{\ast\ast} = T$;\pause
    \item $\left( TS \right)^{\ast} = S^{\ast}T^{\ast}$.\pause
  \end{itemize}
  Furthermore, the operator norm ``plays well'' with operator composition and the adjoint, in the sense that:\pause
  \begin{itemize}
    \item $\norm{TS}_{\op}\leq \norm{T}_{\op}\norm{S}_{\op}$;\pause
    \item $\norm{T^{\ast}}_{\op}= \norm{T}_{\op}$;\pause
    \item $\norm{T^{\ast}T}_{\op} = \norm{T}^2_{\op}$.\pause
  \end{itemize}
  These make $\B\left( \mathcal{H} \right)$ a \textit{$C^{\ast}$-algebra}.\pause\:However, there are other $C^{\ast}$-algebras.
\end{frame}
\begin{frame}
  \frametitle{A Group $C^{\ast}$-Algebra}
  If $\Gamma$ is a group, we may define a vector space, $\C\left[ \Gamma \right]$, by finite sums
  \begin{align*}
    x &= \sum_{t\in\Gamma}x(t)\delta_t,
  \end{align*}
  where $\delta_t$ is the point mass at $t\in \Gamma$.\pause\newline

  This becomes a $\ast$-algebra when endowed with multiplication (by convolution) and involution:
  \begin{align*}
    f\ast g(s) &= \sum_{t\in\Gamma}f(t)g\left( s^{-1}t \right)\\
    f^{\ast}(t) &= \overline{f\left( t^{-1} \right)}.
  \end{align*}
\end{frame}
\begin{frame}
  \frametitle{A Group $C^{\ast}$-Algebra, cont'd}
  If we represent $\pi_{\lambda}\colon \C\left[ \Gamma \right]\rightarrow \B\left( \ell_2\left( \Gamma \right) \right)$ by mapping $\delta_t \mapsto \lambda_{t}\in \mathcal{U}\left( \ell_2\left( \Gamma \right) \right)$, extending linearly, and taking
  \begin{align*}
    \norm{x}_{\lambda} &= \norm{\pi_{\lambda}(x)}_{\op},
  \end{align*}
  we get the \textit{reduced group $C^{\ast}$-algebra} on $\Gamma$ (upon norm completion).
\end{frame}
\begin{frame}
  \frametitle{Finite-Dimensional Approximations}
  The $n\times n$ matrices, $\Mat_n\left( \C \right)$, are also $C^{\ast}$-algebras.\pause\:In fact, they're a very special kind of $C^{\ast}$-algebra --- we care a lot about whether other $C^{\ast}$-algebras can be ``sufficiently'' approximated by matrices.\pause\newline

  We can use these sufficient approximations to establish amenability.
\end{frame}
\begin{frame}
  \frametitle{Nuclearity}
  A $C^{\ast}$-algebra, $A$, is called \textit{nuclear} if there exist two sequences of maps, $\varphi_n\colon A\rightarrow \Mat_{k(n)}\left( \C \right)$ and $\psi_n\colon \Mat_{k(n)}\left( \C \right)\rightarrow A$, such that
  \begin{align*}
    \norm{a - \psi_n\circ\varphi_n\left( a \right)} \xrightarrow{n\rightarrow\infty} 0.
  \end{align*}\pause
  \begin{itemize}
    \item Essentially, any $a\in A$ is ``close enough'' to a certain family of finite-dimensional analogues.
  \end{itemize}
\end{frame}
\begin{frame}
  \frametitle{Nuclearity and Amenability}
  A group $\Gamma$ is amenable if and only if the reduced group $C^{\ast}$-algebra, $C^{\ast}_{\lambda}\left( \Gamma \right)$, is nuclear.\pause\newline

  This is also proven using the Følner condition.\pause\newline

  Specifically, by showing that the approximation of $\frac{\left\vert sF_n\cap F_n \right\vert}{\left\vert F_n \right\vert} \rightarrow 1$ corresponds to the existence of maps $\varphi_n\colon C^{\ast}_{\lambda}\left( \Gamma \right)\rightarrow \Mat_{\left\vert F_n \right\vert}\left( \C \right)$ and $\psi_n\colon \Mat_{\left\vert F_n \right\vert}\left( \C \right)\rightarrow C^{\ast}_{\lambda}\left( \Gamma \right)$ that satisfy
  \begin{align*}
    \norm{x - \psi_n\circ\varphi_n(x)}\xrightarrow{n\rightarrow\infty} 0.
  \end{align*}
\end{frame}
\subsection{Review}%
\begin{frame}
  \frametitle{What We've Learned}
  \small
  If $\Gamma$ is a discrete group, then $\Gamma$ is amenable if and only if\pause
  \begin{itemize}
    \item $\Gamma$ is non-paradoxical (Tarski's theorem);\pause
    \item $\ell_{\infty}\left( \Gamma \right)$ admits a state, $\varphi\colon \ell_{\infty}\left( \Gamma \right)\rightarrow \C$, such that $\varphi\left( \lambda_s(f) \right) = \varphi\left( f \right)$ (invariant states);\pause
    \item there is a sequence of finite subsets, $\left( F_n \right)_n$, such that for all $s\in\Gamma$, $\frac{\left\vert sF_n\cap F_n \right\vert}{\left\vert F_n \right\vert} \rightarrow 1$ (Følner's Theorem);\pause
    \item there is a sequence $\left( f_k \right)_k\subseteq \ell_1\left( \Gamma \right)$ such that $\norm{f_k - \lambda_s\left(f_k\right)}_{\ell_1}\rightarrow 0$ (Approximate Means);\pause
    \item the Cayley graph of $\Gamma$ satisfies $\inf\set{\frac{\left\vert N(S) \right\vert}{\left\vert S \right\vert} | S\subseteq V(G),S\text{ finite}} = 0$ (graph amenability);\pause
    \item there is a sequence $\left( f_k \right)_k\subseteq \ell_2\left( \Gamma \right)$ such that $\norm{f_k - \lambda_s\left(f_k\right)}_{\ell_2} \rightarrow 0$ (almost-invariant vectors);\pause
    \item the reduced group $C^{\ast}$-algebra, $C^{\ast}_{\lambda}\left( \Gamma \right)$, is nuclear (nuclearity).
  \end{itemize}
\end{frame}
\section{Remarks and Acknowledgments}%
\begin{frame}
  \frametitle{Present-Day Programs of Research}
  Amenability is still a very active field of study, especially when it comes to $C^{\ast}$-algebras.\pause\newline

  If a $C^{\ast}$-algebra is nuclear, then by calculating a certain invariant, the $C^{\ast}$-algebra can be classified; if the $C^{\ast}$-algebra is not nuclear, then the process of classification is significantly more difficult.\pause\newline

  Additionally, other questions related to finite-dimensional approximations of $C^{\ast}$-algebras, such as quasidiagonality, were only resolved recently.\pause\:A paper in 2015 proved that if $\Gamma$ is amenable, then $C^{\ast}_{\lambda}\left( \Gamma \right)$ is quasidiagonal (the reverse direction was shown in 1987).
\end{frame}
\begin{frame}
  \frametitle{Acknowledgments}
  A large thank you goes to
  \begin{itemize}
    \item the professors of the math department;
    \item friends, family, and acquaintances both in the math major and outside;
    \item everyone in attendance.
  \end{itemize}
\end{frame}
\begin{frame}[allowframebreaks]
  \frametitle{References}
  \nocite{*}
  {\tiny \printbibliography}
\end{frame}
\end{document}
