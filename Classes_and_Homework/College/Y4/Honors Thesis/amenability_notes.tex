\documentclass[10pt]{mypackage}

% sans serif font:
%\usepackage{cmbright}
%\usepackage{sfmath}
%\usepackage{bbold} %better blackboard bold

%serif font + different blackboard bold for serif font
\usepackage{newpxtext,eulerpx,eucal}
\renewcommand*{\mathbb}[1]{\varmathbb{#1}}
\usepackage{nicematrix}

\pagestyle{fancy}
\fancyhf{}
\rhead{Avinash Iyer}
\lhead{Paradoxical Decompositions and Tarski's Theorem}

\setcounter{secnumdepth}{0}

\begin{document}
\RaggedRight
\tableofcontents
\section{Introduction}%
This is going to be the notes for my Honors thesis independent study, which will be focused on amenability and $C^{\ast}$-algebras. This section of notes will be focused on amenability of groups, starting from group actions and inevitably proving Tarski's theorem, which states that a group is amenable if and only if it is non-paradoxical.\newline

The primary source texts to inform this section of my independent study will be Volker Runde's \textit{Lectures on Amenability} and \textit{Amenable Banach Algebras} --- specifically, chapter $0$ of both texts, which is focused on paradoxical groups and Tarski's theorem.\newline

I do not claim any of this work to be original.
\section{Group Actions, Paradoxical Decompositions, and the Banach--Tarski Paradox}%
In order to introduce Tarski's theorem, which is where our first condition about the amenability of groups appears, we begin by discussing paradoxical decompositions, with the goal of this section being a proof of the Banach--Tarski Paradox. The Banach--Tarski paradox says the following:
\begin{quote}
  If $A$ and $B$ are any bounded subsets of $\R^{3}$ with nonempty interior, there is a partition of $A$ into finitely many disjoint subsets such that a sequence of isometries applied to these subsets yields $B$.
\end{quote}
\subsection{Basics of Group Actions}%
The information for these essentials about group actions will be drawn from Dummit and Foote's \textit{Abstract Algebra}.
\begin{definition}[Group Action]
  A (left) group action of $G$ onto a set $A$ is a map from $G\times A$ to $A$ that satisfies:
  \begin{itemize}
    \item $g_1\cdot \left(g_2\cdot a\right) = \left(g_1g_2\right)\cdot a$ for all $g_1,g_2\in G$ and $a\in A$;
    \item $e\cdot a = a$\footnote{The identity element is usually written as $1$, but I will write it as $e$ out of familiarity.} for all $a\in A$.
  \end{itemize}
\end{definition}
\begin{definition}[Permutation Representation]
  For each $g$, the map $\sigma_g: A\rightarrow A$ defined by $\sigma_g(a) = g\cdot a$ (the group element $g$ acts on $a$) is a permutation of $A$. There is a homomorphism associated to these actions:
  \begin{align*}
    \varphi: G\rightarrow S_A,
  \end{align*}
  where $\varphi(g) = \sigma_g$. Recall that $S_A$ is the symmetric group (group of permutations) on the elements of $A$. \newline

  This is the permutation representation for the action.\newline

  In particular, given any nonempty set $A$ and a homomorphism $G$ into $S_A$, we can define an action of $G$ on $A$ by taking $g\cdot a = \varphi(g)(a)$.
\end{definition}
\begin{definition}[Kernel]
  The kernel of the action of $G$ is the set of elements in $g$ that act trivially on $A$:
  \begin{align*}
    \set{g\in G\mid \forall a\in A,~g\cdot a = a}
  \end{align*}
\end{definition}
\begin{note}
  The kernel of the action is the kernel of the permutation representation $\varphi: G\rightarrow S_{A}$.
\end{note}
\begin{definition}[Stabilizer]
  For each $a\in A$, the stabilizer of $a$ under $G$ is the set of elements in $G$ that fix $a$:
  \begin{align*}
    G_a &= \set{g\in G\mid g\cdot a = a}.
  \end{align*}
\end{definition}
\begin{note}
  The kernel of the group action is the intersection of the stabilizers of every element of $G$:
  \begin{align*}
    \text{kernel} &= \bigcap_{a\in A}G_a.
  \end{align*}
\end{note}
\begin{note}
  For each $a\in A$, $G_a$ is a subgroup of $G$.
\end{note}
\begin{definition}[Faithful Action]
  An action is faithful if the kernel of the action is $e$.
\end{definition}
\begin{definition}[Free Action]
  For a set $X$ with $G$ acting on $X$, the action of $G$ on $X$ is free if, for every $x$, $g\cdot x = x$ if and only if $g = e_G$.\newline

  If the action of $G$ on $X$ is a free action, we say $G$ acts freely on $X$.
\end{definition}
\begin{proposition}[Stabilizers and Orbits]
  Let $G$ be a group that acts on a nonempty set $A$. We define a relation $a\sim b$ if and only if $a = g\cdot b$ for some $g\in G$. This is an equivalence relation, with the number of elements in $\left[a\right]_{\sim}$ found by taking $\left\vert G:G_a \right\vert$, which is the index of the stabilizer of $a$.
\end{proposition}
\begin{proof}
  We can see that $a\sim a$, since $e\cdot a = a$. Similarly, we can see that if $a\sim b$, then $b = g^{-1}\cdot a$, meaning $b\sim a$. Finally, let $a\sim b$ and $b\sim c$. Then, we have $a = g\cdot b$ for some $g\in G$, and $b = h\cdot c$ for some $h\in G$. Thus, we have
  \begin{align*}
    a &= g\cdot \left(h\cdot c\right)\\
      &= \left(gh\right)\cdot c,
  \end{align*}
  meaning $a\sim c$.\newline

  We say there is a bijection between the left cosets of $G_a$ and the elements of the equivalence class of $a$.\newline

  Define $\mathcal{C}_a$ to be the set $\set{g\cdot a\mid g\in G}$, and let $b = g\cdot a$. Define a map $g\cdot a \mapsto gG_a$. This map is surjective since $g\cdot a$ is always an element of $\mathcal{C}_a$. Additionally, since $g\cdot a = h\cdot a$ if and only if $\left(h^{-1}g\right)\cdot a = a$, meaning $h^{-1}g \in G_a$, and $h^{-1}g\in G_a$ if and only if $gG_a = hG_a$, this map is injective.\newline

  Since there is a one-to-one map between the equivalence classes of $a$ under the action of $G$, and the number of left cosets of $G_a$, we now know that the number of equivalence classes of $a$ under the action of $G$ is $\left\vert G:G_a \right\vert$.
\end{proof}
\begin{definition}[Orbit]
  For any $a\in A$, we define the orbit under $G$ of $a$ by
  \begin{align*}
    G\cdot a &= \set{b\in A\mid \forall g\in G,~b = g\cdot a}
  \end{align*}
  In particular, if $c\in G\cdot a$ for some $a\in A$, then $G\cdot c = G\cdot a$.
\end{definition}
\subsection{Paradoxical Decompositions}%
Most of the information from this section will be drawn from Volker Runde's \textit{Lectures on Amenability}, as well as \textit{Amenable Banach Algebras: A Panorama}.
\begin{definition}[Paradoxical Sets and Decompositions]
  Let $G$ be a group that acts on a set $X$. Let $E\subseteq X$.\newline

  If there exist pairwise disjoint $A_1,\dots,A_n,B_1,\dots,B_m\subseteq E$ and $g_1,\dots,g_n,h_1,\dots,h_m\in G$ such that
  \begin{align*}
    E &= \bigcup_{j=1}^{n}g_j\cdot A_j
    \intertext{and}
    E &= \bigcup_{j=1}^{m}h_j\cdot B_j,
  \end{align*}
  then we say that $E$ is $G$-paradoxical.\newline

  In particular, a paradoxical group is one where $G$ acts on itself by left-multiplication.
\end{definition}
\begin{example}[Our First Paradoxical Group]
  The free group on two generators, $\mathbb{F}\left(a,b\right)$,\footnote{The set of all reduced words over $\set{a,b,a^{-1},b^{-1},e_{\F(a,b)}}$. In particular, a word is reduced when the pairs $aa^{-1}$ and $bb^{-1}$ are replaced with the identity $e_{\F(a,b)}$.} is paradoxical. To see this, we let
  \begin{align*}
    W(x) &= \set{w\in \F(a,b)\mid w\text{ starts with }x}.
  \end{align*}
  Here, ``starts with'' refers to the left-most element. For instance, $ba^2ba^{-1}\in W\left(b\right)$.\newline

  In particular, we can see that
  \begin{align*}
    \F(a,b) &= \set{e_{\F(a,b)}} \sqcup W(a) \sqcup W(b) \sqcup W\left(a^{-1}\right)\sqcup W\left(b^{-1}\right).
  \end{align*}
  For any $w\in \F(a,b)\setminus W(a)$, we can see that $a^{-1}w\in W\left(a^{-1}\right)$, meaning $w\in aW\left(a^{-1}\right)$. Therefore, $\F(a,b) = W(a)\sqcup aW\left(a^{-1}\right)$.\newline

  Similarly, for any $v\in \F(a,b)\setminus W(b)$, $b^{-1}v \in W\left(b^{-1}\right)$, so $v \in bW\left(b^{-1}\right)$. Therefore, $\F(a,b) = W(b) \sqcup bW\left(b^{-1}\right)$.
\end{example}
\begin{proposition}[Free Action of a Paradoxical Group]
  Let $G$ be a paradoxical group that acts freely on $X$. Then, $X$ is $G$-paradoxical.
\end{proposition}
\begin{proof}
  Let $A_1,\dots,A_n,B_1,\dots,B_m\subseteq G$ be pairwise disjoint, with $g_1,\dots,g_n,h_1,\dots,h_m\in G$ such that
  \begin{align*}
    G &= \bigcup_{j=1}^{n}g_j A_j\\
      &= \bigcup_{j=1}^{m}h_jB_j.
  \end{align*}
  We let $M\subseteq X$ contain exactly one element from every orbit of $G$.\newline

  The set $\set{g\cdot M\mid g\in G}$ is a partition of $X$. Since $M$ contains exactly one element from every orbit of $G$, it is then the case that $\bigcup_{g\in G}g\cdot M = X$, since $G\cdot M = X$.\newline

  Additionally, if $x,y\in M$ with $g\cdot x = h\cdot y$, then $\left(h^{-1}g\right)\cdot x = y$, meaning $y$ is in the orbit of $x$ and vice versa, implying $x = y$. Thus, we must have $h^{-1}g = e_G$, as we assume $G$ acts freely.\newline

  Thus, we can see that $g_1\cdot M \neq g_2\cdot M$ if $g_1\neq g_2$, meaning $\set{g\cdot M\mid g\in G}$ is a partition.\newline

  Define $A_{j}^{\ast}$ to be the subset of $X$ that is the result of the elements of $A_j$ acting on $M$. In other words,
  \begin{align*}
    A_j^{\ast} &= \bigcup_{g\in A_j}g\cdot M.
  \end{align*}
  As a useful shorthand, we can say $A_j^{\ast} = A_j\cdot M$.\footnote{Yes, I know that $A_j$ is not technically a group acting on $M$, but this will help illuminate the final conclusion.} Similarly, we define
  \begin{align*}
    B_j^{\ast} &= \bigcup_{h\in B_j}h\cdot M\\
               &= B_j\cdot M.
  \end{align*}
  We can see that $A_1^{\ast},A_2^{\ast},\dots,A_n^{\ast},B_1^{\ast},B_2^{\ast},\dots,B_m^{\ast}\subseteq X$ are disjoint, since $\set{g\cdot M\mid g\in G}$ is a partition, and $A_1,\dots,A_n,B_1,\dots,B_m$ are disjoint in $G$.\newline

  Thus, we have
  \begin{align*}
    \bigcup_{j=1}^{n}g_j\cdot A_j^{\ast} &= \bigcup_{j=1}^{n} \left(g_jA_j\right)\cdot M\\
                                         &= G\cdot M\\
                                         &= X.
  \end{align*}
  Similarly,
  \begin{align*}
    \bigcup_{j=1}^{m}h_j\cdot B_j^{\ast} &= \bigcup_{j=1}^{m}\left(h_jB_j\right)\cdot M\\
                                         &= G\cdot M\\
                                         &= X.
  \end{align*}
  Thus, we see that $X$ has a paradoxical decomposition, meaning $X$ is $G$-paradoxical.
\end{proof}
\begin{note}
  We invoked the axiom of choice when we defined $M$ to contain exactly one element from each orbit in $X$.
\end{note}
\subsection{Paradoxical Decompositions of the Unit Sphere and Unit Ball}%
We are aware of $\F(a,b)$ being a paradoxical group --- in particular, we hope to use the properties of $\F(a,b)$ to yield paradoxical decompositions of the unit sphere in $\R^{3}$, denoted $S^{2}$.
\begin{definition}[Special Orthogonal Group]
  For $n\in \N$, we define the special orthogonal group to consist of all real $n\times n$ matrices $A$ such that
  \begin{align*}
    A^{T}A = AA^{T} = I,
  \end{align*}
  with $\det(A) = 1$.
\end{definition}
In particular, $SO(3)$ denotes the set of all rotations about some line that runs through the origin. An important fact about $SO(3)$ is that it contains a paradoxical subgroup.
\begin{theorem}
  There are rotations $A$ and $B$ about lines through the origin in $\R^3$ which generate a subgroup of $SO(3)$ isomorphic to $\F(a,b)$.
\end{theorem}
\begin{proof}
  We set
  \begin{align*}
    A^{\pm} &= \begin{bmatrix}1/3 & \mp\frac{2\sqrt{2}}{3} & 0\\ \pm \frac{2\sqrt{2}}{3} & 1/3 & 0 \\ 0 & 0 & 1 \end{bmatrix}\\
    B^{\pm} &= \begin{bmatrix}1 & 0 & 0 \\ 0 & 1/3 & \mp \frac{2\sqrt{2}}{3}\\ 0 & \pm \frac{2\sqrt{2}}{3} & 1/3\end{bmatrix}
  \end{align*}
  Here, $A^{+}$ denotes $A$, and $A^{-}$ denotes $A^{-1}$, and similarly with $B$.\newline

  Let $w$ be a reduced word in $A$, $B$, $A^{-1}$, and $B^{-1}$ which is not the empty word. We claim that $w$ cannot be the identity. Without loss of generality, we assume $w$ ends in $A$ or $A^{-1}$ --- this is because $w$ acts as the identity if and only if $AwA^{-1}$ or $A^{-1}wA$ act as the identity.\newline

  In particular, we will show that there exist $a,b,c\in \Z$ with $b\not\equiv 0$ modulo $3$ such that
  \begin{align*}
    w \cdot \begin{pmatrix}1\\0\\0\end{pmatrix} &= \frac{1}{3^{k}} \begin{pmatrix}a\\b\sqrt{2}\\c\end{pmatrix},
  \end{align*}
  where $k$ is the length of $w$. The main reason we wish to show this is that, if we have $b\not\equiv 0$ modulo $3$, it is the case that $w$ necessarily cannot map $ \begin{pmatrix}1\\0\\0\end{pmatrix} $ to itself.\newline

  We start with induction on the length of $w$. In particular, for $w = A^{\pm}$, we have
  \begin{align*}
    w \cdot \begin{pmatrix}1\\0\\0\end{pmatrix} &= \frac{1}{3}\begin{pmatrix}1\\\pm2\sqrt{2}\\0\end{pmatrix},
  \end{align*}
  proving the base case.\newline

  Suppose $k > 0$, meaning $w = A^{\pm}w'$ or $w = B^{\pm}w'$, with $w'$ not equal to the empty word. The inductive hypothesis says that
  \begin{align*}
    w' \cdot \begin{pmatrix}1\\0\\0\end{pmatrix} &= \frac{1}{3^{k-1}} \begin{pmatrix}a'\\b'\sqrt{2}\\c'\end{pmatrix},
  \end{align*}
  for some  $a',b',c'\in \Z$ with $b\not\equiv 0$ modulo 3. In particular,
  \begin{align*}
    A^{\pm}w' \cdot \begin{pmatrix}1\\0\\0\end{pmatrix} &= \frac{1}{3^k} \begin{pmatrix}a' \mp 4b'\\ \left(b' \pm 2a'\right)\sqrt{2}\\ 3c'\end{pmatrix}\\
  B^{\pm}w' \cdot \begin{pmatrix}1\\0\\0\end{pmatrix} &= \frac{1}{3^k}\begin{pmatrix}3a'\\\left(b' \mp 2c'\right)\sqrt{2} \\ c' \pm 4b'\end{pmatrix},
  \end{align*}
  where we say
  \begin{align*}
    w \cdot \begin{pmatrix}1\\0\\0\end{pmatrix} &= \frac{1}{3^k}\begin{pmatrix}a\\b\\c\end{pmatrix},
  \end{align*}
  i.e., we set the coordinates of $w\cdot \begin{pmatrix}1\\0\\0\end{pmatrix}$ through their definition in $A^{\pm}w'$ or $B^{\pm}w'$.\newline

  In order to show that $b\not\equiv 0$ modulo $3$, we must examine the following four cases.\newline

  Let $w^{\ast}$ denote the word such that
  \begin{align*}
    w^{\ast} \cdot \begin{pmatrix}1\\0\\0\end{pmatrix} &= \frac{1}{3^{k-2}} \begin{pmatrix}a''\\b''\sqrt{2}\\c''\end{pmatrix},
  \end{align*}
  with $a'',b'',c''\in \Z$ and $b''\not\equiv 0$ modulo $3$. It is important to note here that $w^{\ast}$ may be the empty word.
  \begin{description}[font=\normalfont]
    \item[Case 1:] Suppose $w = A^{\pm}B^{\pm}w^{\ast}$. Then, we have $b = b'\mp 2a'$, where $a' = 3a''$. Since $b'\not\equiv 0$ modulo $3$ by the inductive hypothesis assumption, it is also the case that $b\not\equiv 0$ modulo $3$.
    \item[Case 2:] Suppose $w = B^{\pm}A^{\pm}w^{\ast}$. Then, we have $b = b' \mp 2c'$, where $c' = 3c''$. Similarly, since $b'\not\equiv 0$ modulo $3$ by the inductive hypothesis assumption, it is also the case that $b\not\equiv 0$ modulo $3$.
    \item[Case 3:] Suppose $w = A^{\pm}A^{\pm}w^{\ast}$. Then, we have
      \begin{align*}
        b &= b' \pm 2a'\\
          &= b' \pm 2\left(a'' \mp 4b''\right)\\
          &= b' + \left(b'' \pm 2a''\right) - 9b''\\
          &= 2b' - 9b''.
      \end{align*}
      Since $b',b''\not\equiv 0$  modulo $3$ by the inductive hypothesis, it is also the case that $b\not\equiv 0$ modulo $3$.
    \item[Case 4:] Suppose $w = B^{\pm}B^{\pm}w^{\ast}$. Then, we have
      \begin{align*}
        b &= b' \mp 2c'\\
          &= b' \mp 2\left(c'' \pm 4b''\right)\\
          &= b' + \left(b''\mp 2c''\right) - 9b''\\
          &= 2b' - 9b''.
      \end{align*}
      Since $b',b''\not\equiv 0$ modulo $3$ by the inductive hypothesis, it is also the case that $b\not\equiv 0$ modulo $3$.
  \end{description}
  Thus, we have shown that any non-empty reduced word over $A$, $A^{-1}$, $B$, $B^{-1}$ does not act as the identity. The subgroup of $SO(3)$ generated by $A$, $B$, $A^{-1}$, and $B^{-1}$ is thus isomorphic to $\F(a,b)$.
\end{proof}
\begin{remark}
  For any element of $SO(n)$ with $n \geq 3$, we can write $A_n$ (denoting the $n\times n$ matrix corresponding to $A$) as
  \begin{align*}
    A_n &= 
    \begin{pmatrix}
      A_3 & \mathbf{0}\\
      \mathbf{0} & \mathbf{1}
    \end{pmatrix}\\
    B_n &= \begin{pmatrix}B_3 & \mathbf{0}\\ \mathbf{0} & \mathbf{1}\end{pmatrix},
  \end{align*}
  where $\mathbf{0}$ denotes a block matrix consisting of $0$ and $\mathbf{1}$ denotes a block matrix equal to the identity.\newline

  This means that our subgroup of $SO(3)$ isomorphic to $\mathbb{F}\left(a,b\right)$ embeds into $SO(n)$ via the above block matrices.
\end{remark}
\begin{theorem}[Hausdorff Paradox]
  There is a countable subset $D$ of $S^2$ such that $S^2\setminus D$ is paradoxical under the action of $SO(3)$.
\end{theorem}
\begin{proof}
  Let $A$ and $B$ be the rotations in $SO(3)$ that serve as the generators of the subgroup isomorphic to $\mathbb{F}\left(a,b\right)$.\newline

  Since $A$ and $B$ are rotations, any word in the subgroup generated by $A$ and $B$ will also be a rotation --- as a result, all such (non-empty) words contain two fixed points.\newline

  Let
  \begin{align*}
    F &= \set{x\in S^2\mid x\text{ is a fixed point for some word $w$}}.
  \end{align*}
  Since the set of all words in $A$ and $B$ is countably infinite, so too is $F$. Therefore, the union of all these fixed points under the action of all such words $w$ is also countable:
  \begin{align*}
    D &= \bigcup_{w\in G} w\cdot F.
  \end{align*}
  Since the set of words in $A$ and $B$ act freely on $S^2\setminus D$, it must be the case that $S^{2}\setminus D$ is paradoxical under the action of the group of all such words.
\end{proof}
\begin{definition}[Equidecomposable Sets]
  Let $G$ act on $X$, $A,B\subseteq X$. We say $A$ and $B$ are equidecomposable under $G$ if there are $A_1,\dots,A_n\subseteq A$, $B_1,\dots,B_n\subseteq B$, and $g_1,\dots,g_n\in G$ such that
  \begin{enumerate}[(i)]
    \item $A = \bigcup_{j=1}^{n}A_j$ and $B = \bigcup_{j=1}^{n}B_j$;
    \item the collection $\set{A_j}_{j=1}^{n}$ are pairwise disjoint and the collection $\set{B_j}_{j=1}^{n}$ are pairwise disjoint;
    \item for each $j$, $g_j\cdot A_j = B_j$.
  \end{enumerate}
  We write $A\sim_{G} B$ if $A$ and $B$ are equidecomposable under $G$.
\end{definition}
\begin{remark}
  The relation $\sim_{G}$ is an equivalence relation.\newline

  In particular, to see transitivity, we have the partitions $A_1,\dots,A_n\subseteq B$ and $B_1,\dots,B_n\subseteq B$ with $g_i\cdot A_i = B_i$, and the partitions $B_1,\dots,B_m\subseteq B$, $C_1\cdots C_m\subseteq C$ with $h_j\cdot B_j = C_j$.\newline

  We find the partition of $A$ by taking $A_{ij} = B_i\cap B_j$, where $i\in \set{1,2,\dots,n}$ and $j\in \set{1,2,\dots,m}$. We then have $h_jg_i\cdot A_{ij}$ maps to a refined partition of $C$, yielding equidecomposability between $A$ and $C$.
\end{remark}
\begin{remark}
  For equidecomposable sets $A$ and $B$, there is a bijection $\phi: A\rightarrow B$ by, for each $C\subseteq A$, taking $C_{i} = C\cap A_i$, where $A_1,\dots,A_n$ is the partition of $A$, and mapping $\varphi\left(C_i\right) = g_i\cdot C_i$.
\end{remark}
\begin{proposition}
  Let $D\subseteq S^2$ be countable. Then, $S^2$ and $S^{2}\setminus D$ are equidecomposable under the action of $SO(3)$.
\end{proposition}
\begin{proof}
  Let $L$ be a line in $\R^3$ with the property that $L \cap D = \emptyset$. Such a $L$ must necessarily exist as the set of all antipodes in $S^{2}$ is uncountable.\newline

  Define $\rho_{\theta}\in SO(3)$ to be a rotation about $L$ by an angle of $\theta$. For fixed $n\in \N$ and fixed $\theta\in [0,2\pi)$, define $R_{n,\theta} = \set{x\in D\mid \rho_{\theta}^{n}\cdot x\in D}$. Since $D$ is countable, $R_{n,\theta}$ is necessarily countable.\newline

  Define $W_{n} = \set{\theta\mid R_{n,\theta}\neq \emptyset}$. The injection $\theta \mapsto \rho^{n}_{\theta}\cdot x$ into $D$ shows that for each $n$, $W_n$ is countable. Thus, defining
  \begin{align*}
    W &= \bigcup_{n\in N}W_n
  \end{align*}
  it is evident that $W$ is countable.\newline

  Thus, there must exist $\omega \in [0,2\pi)\setminus W$. Define $\rho$ to be a rotation about $L$ by $\omega$. Then, for every $n,m\in \N$,
  \begin{align*}
    \rho^{n}\cdot D \cap \rho^{m}\cdot D = \emptyset.
  \end{align*}
  We let $\tilde{D} = \bigsqcup_{n=0}^{\infty}\rho^{n}\cdot D$. Notice that, in particular, $\rho\cdot\tilde{D} = \bigsqcup_{n=1}^{\infty}\rho^{n}\cdot D$, meaning $\tilde{D}$ and $\tilde{D}\setminus D$ are equidecomposable under $SO(3)$.\newline

  Thus, we have
  \begin{align*}
    S^{2} &= \tilde{D}\sqcup \left(S^2\setminus \tilde{D}\right)\\
          &\sim_{SO(3)} \rho\cdot \tilde{D}\sqcup\left(S^2\setminus \tilde{D}\right)\\
          &= \left(\tilde{D}\setminus D\right)\sqcup\left(S^2\setminus \tilde{D}\right)\\
          &= S^2\setminus D,
  \end{align*}
  establishing the equidecomposability of $S^2$ and $S^2\setminus D$.
\end{proof}
\begin{proposition}
  Let $G$ act on $X$, with $E$ and $E'$ subsets of $X$ such that $E \sim_{G}E'$. Then, if $E$ is paradoxical under the action of $G$, so too is $E'$.
\end{proposition}
\begin{proof}
  Let $A_1,\dots,A_n$, $B_1,\dots,B_m$ be pairwise disjoint subsets of $E$ and $g_1,\dots,g_n,h_1,\dots,h_m\in G$ such that
  \begin{align*}
    E &= \bigsqcup_{i=1}^{n} g_i\cdot A_i\\
      &= \bigsqcup_{j=1}^{m}h_j\cdot B_j,
  \end{align*}
  which follows from the paradoxicality of $E$. We let
  \begin{align*}
    A &= \bigsqcup_{i=1}^{n}A_i\\
    B &= \bigsqcup_{j=1}^{m} B_j.
  \end{align*}
  It follows that $A\sim_{G} E$ and $B\sim_{G} E$; to see this, set the partition of $A$ to be $A_1,\dots,A_n$, and set the partition of $E$ to be $g_i\cdot A_i$ for $i\in \set{1,\dots,n}$, and similarly for $G$.\newline

  Since $E\sim_{G} E'$, and $\sim_{G}$ is an equivalence relation, it follows that $A\sim_{G}E'$ and $B\sim_{G}E'$, implying that there exists a paradoxical decomposition of $E'$ in $A_1,\dots,A_n$ and $B_1,\dots,B_m$.
\end{proof}
Since $S^{2}\setminus D$ and $S^{2}$ are equidecomposable under the action of $SO(3)$, and $S^{2}\setminus D$ is paradoxical under the action of $SO(3)$, the above proposition implies the following corollary.
\begin{corollary}
  $S^{2}$ is paradoxical under $SO(3)$.
\end{corollary}
\begin{definition}[Euclidean Group]
  The Euclidean group $E(n)$ consists of all isometries of a Euclidean space. An isometry of a Euclidean space consists of translations, flips about the origin, and rotation.\newline

  In particular, $E(n) = T(n)\rtimes O(n)$, where $T(n)$ denotes all translations and $O(n)$ is the orthogonal group, which denotes all rotations or flips.\newline

  We define $E_{+}(n)$ to be all orientation-preserving isometries of Euclidean space. In particular, $E_{+}(n) = T(n)\rtimes SO(n)$, where $SO(n)$ is the special orthogonal group, which denotes all orientation-preserving rotations.
\end{definition}
\begin{corollary}[Weak Banach--Tarski Paradox]
  Every closed ball in $\R^3$ is paradoxical under the Euclidean group $E(3)$.
\end{corollary}
\begin{proof}
  We only need to show that the closed unit ball, $B(0,1)$, is paradoxical under the action of $E(3)$.\newline

  To start, we can show that $B(0,1)\setminus \set{0}$ is paradoxical. Since $SO(3)$ is paradoxical, there exist pairwise disjoint $A_1,\dots,A_n,B_1,\dots,B_m\subseteq S^2$ and $g_1,\dots,g_n,h_1,\dots,h_m\in SO(3)$ such that
  \begin{align*}
    S^2 &= \bigsqcup_{i=1}^{n}g_i\cdot A_i\\
        &= \bigsqcup_{j=1}^{m}h_j\cdot B_j.
  \end{align*}
  Define
  \begin{align*}
    A_i^{\ast} &= \set{tx\mid t\in (0,1],x\in A_i}\\
    B_j^{\ast} &= \set{ty\mid t\in (0,1],y\in B_j}.
  \end{align*}
  Then, $A_1^{\ast},\dots,A_n^{\ast},B_1^{\ast},\dots,B_m^{\ast}\subseteq B(0,1)\setminus \set{0}$ are pairwise disjoint, and
  \begin{align*}
    B(0,1)\setminus {0} &= \bigcup_{i=1}^{n}g_i\cdot A_i^{\ast}\\
           &= \bigcup_{j=1}^{m}h_j\cdot B_j^{\ast}.
  \end{align*}
  Thus, we have established that $B(0,1)\setminus \set{0}$ is paradoxical under $SO(3)\leq E(3)$.\footnote{Essentially, we take the paradoxical decomposition of $S^2$ under $SO(3)$, and scale by $t$ to cover all of $B(0,1)\setminus \set{0}$.}\newline

  Now, we want to show that $B(0,1)\setminus \set{0}$ and $B(0,1)$ are equidecomposable under $E(3)$. To do this, let $x \in B(0,1)\setminus \set{0}$, and let $\rho$ be a rotation about $x$ by a line that misses the origin such that $\rho^{n}\cdot 0 \neq \rho^{m}\cdot 0$ for all $n,m\in \N$ with $n\neq m$.\footnote{This is why we need our underlying group acting on $\R^3$ to be the Euclidean group rather than $SO(3)$. It is still the case that $SO(3)\leq E(3)$, meaning that $E(3)$ is necessarily paradoxical when acting on $\R^3$.} Let $D = \set{\rho^{n}\cdot 0\mid n\in\N}$. We can see that $\rho\cdot D = D\setminus \set{0}$, and that $D$ and $\rho\cdot D$ are equidecomposable under $E(3)$.\newline

  Thus, we have
  \begin{align*}
    B(0,1) &= D\sqcup \left(B(0,1)\setminus D\right)\\
           &\sim_{E(3)} \left(\rho\cdot D\right)\sqcup \left(B(0,1)\setminus D\right)\\
           &= \left(D\setminus \set{0}\right) \sqcup \left(B(0,1)\setminus D\right)\\
           &= B(0,1)\setminus \set{0},
  \end{align*}
  establishing the equidecomposability of $B(0,1)$ and $B(0,1)\setminus \set{0}$.\newline

  Thus, $B(0,1)$ is paradoxical under the action of $E(3)$.
\end{proof}
\begin{definition}
  For $G$ acting on a set $X$, we write $A\preceq_{G}B$ if $A$ is equidecomposable with a subset of $B$.
\end{definition}
\begin{remark}
  We can see that $\preceq_{G}$ is reflexive since $A$ is equidecomposable with $A$, and $A\subseteq A$.\newline

  To show transitivity, let $A\preceq_{G} B$ and $B\preceq_{G} C$. We let $g_1,\dots,g_n\in G$ such that $A\sim_{G}B_{\alpha}$, where $B_{\alpha}\subseteq B$. In particular, we have $A_1,\dots,A_n\subseteq A$ and $B_{1,\alpha},\dots,B_{n,\alpha}\subseteq B_{\alpha}$ such that $g_i\cdot A_i = B_{i,\alpha}$. We let $h_1,\dots,h_m\in G$ and $C_{\beta}\subseteq C$ such that $h_j\cdot B_j = C_{j,\beta}$ for each $j\in \set{1,\dots,m}$.\newline

  We take a refinement on $B$ taking intersections $B_{i,j,\alpha} = B_i \cap B_{j,\alpha}$ for each $i\in \set{1,\dots,n}$ and $j\in \set{1,\dots,m}$. Thus, taking $h_jg_i\cdot A_i$, we see that $A$ is equidecomposable with a subset of $C$ (namely, the subset of $C$ ``generated'' by the disjoint subsets of $B_{\alpha}$ refined by $B_i$).
\end{remark}
\begin{remark}
  Since $A\sim_{G} B$ implies the existence of a bijection $\phi: A\rightarrow G$, the $\preceq_{G}$ relation is akin to the $\leq$ relation for cardinalities; in particular, $A\preceq_{G}B$ implies the existence of an injection $\phi: A\hookrightarrow B$.\newline

  This analogy between cardinality and the $\preceq_{G}$ relation naturally lends itself to the following theorem.
\end{remark}
\begin{theorem}
  Let $G$ be a group that acts on $X$. Let $A,B$ be subsets of $X$ with $A\preceq_{G} B$ and $B\preceq_{G} A$. Then, $A\sim_{G} B$.
\end{theorem}
\begin{proof}
  Let $B_1\subseteq B$ with $A\sim_{G} B_1$, and let $A_1\subseteq A$ with $B\sim_{G} A_1$.\newline

  We know that there exist bijections $\phi: A\rightarrow B_1$ and $\psi: B\rightarrow A_1$. Define $C_0 = A\setminus A_1$, $C_{n+1} = \psi\left(\phi\left(C_n\right)\right)$. We set
  \begin{align*}
    C &= \bigcup_{n\geq 1} C_n.
  \end{align*}
  Since $\psi^{-1}\left(\psi\left(\phi\left(C_n\right)\right)\right) = \phi\left(C_n\right)$, we have
  \begin{align*}
    \psi^{-1}\left(A\setminus C\right) &= B\setminus \phi(C),
  \end{align*}
  meaning $A\setminus C \sim B\setminus \phi(C)$. Additionally, $C\sim \phi(C)$. Thus,
  \begin{align*}
    A &= \left(A\setminus C\right)\cup C\\
      &\sim \left(B\setminus \phi(C)\right) \cup \phi(C)\\
      &= B.
  \end{align*}
\end{proof}
\begin{theorem}[Banach--Tarski Paradox]
  Let $A$ and $B$ be bounded subsets of $\R^3$ with nonempty interior. Then, $A \sim_{E(3)} B$.
\end{theorem}
\begin{proof}
  It is sufficient to show that $A\preceq_{E(3)} B$.\newline

  Since $A$ is bounded, there is $r > 0$ such that $A\subseteq B(0,r)$. Let $x\in B^{\circ}$. Then, there is $\epsilon > 0$ such that $B(x,\epsilon)\subseteq B$.\newline

  Since $B(0,r)$ is compact (and thus totally bounded), there are translations $g_1,\dots,g_n$ such that
  \begin{align*}
    B(0,r)\subseteq g_1\cdot B(x,\epsilon) \cup \cdots \cup g_n\cdot B(x,\epsilon).
  \end{align*}
  Choose translations $h_1,\dots,h_n$ such that $h_j\cdot B(x,\epsilon)\cap h_k\cdot B(x,\epsilon) = \emptyset$ for $j\neq k$. Set
  \begin{align*}
    S &= \bigcup_{j=1}^{n}h_j\cdot B(x,\epsilon).
  \end{align*}
  Since each of the subsets $h_j\cdot B(x,\epsilon)$ is equidecomposable with any arbitrary closed ball subset of $B(x,\epsilon)$, it is the case that $S\subseteq B(x,\epsilon)$.\newline

  Thus, we have
  \begin{align*}
    A &\subseteq B(0,r)\\
      &\subseteq g_1\cdot B(x,\epsilon)\cup g_n\cdot B(x,\epsilon)\\
      &\preceq S\\
      &\preceq B(x,\epsilon)\\
      &\subseteq B.
  \end{align*}
\end{proof}
\begin{remark}
  The axiom of choice was invoked when we stated that $h_j\cdot B(x,\epsilon)$ is equidecomposable with an arbitrary closed ball subset of $B(x,\epsilon)$.
\end{remark}
\section{Tarski's Theorem}%
One of the central facts that allowed for the Banach--Tarski paradox to be true is that $\F(a,b)$ does not have a property known as amenability. We had also proved that $\F(a,b)$ is paradoxical.\newline

In this section, we will prove paradoxicality and non-amenability are equivalent. This is formulated in Tarski's Theorem
\begin{theorem}[Tarski's Theorem]
  Let $G$ be a group that acts on a set $X$, and let $E$ be a subset of $X$. There is a finitely additive set function invariant under $G$, $\mu: P(X)\rightarrow \left[0,\infty\right]$ with $\mu(E)\in (0,\infty)$ if and only if $E$ is not $G$-paradoxical.
\end{theorem}
\begin{remark}
  It is possible to see that if $G$ is paradoxical, with $X = G$ and $G$ acting on itself via left-multiplication, that this finitely-additive set function eventually ``blows up.''\newline

  Let $G$ be paradoxical. Suppose toward contradiction that there existed such a $\nu: P(G) \rightarrow [0,\infty]$. For $E_1,\dots,E_n\subseteq G$ and $t_1,\dots,t_n\in G$, $F_1,\dots,F_m\subseteq G$ and $s_1,\dots,s_m\in G$ with $E_1,\dots,E_n,F_1,\dots,F_m$ pairwise disjoint, we have
      \begin{align*}
        \nu(G) &= \nu\left(\bigsqcup_{j=1}^{n}t_jE_j\right)\\
               &= \sum_{j=1}^{n}\nu(t_jE_j)\\
               &= \sum_{j=1}^{n}\nu(E_j).
               \intertext{We know that $G \cup s_1F_1 = G$, meaning $\nu(G) = \nu(G \cup s_1F_1)$. However,}
        \nu(G \cup s_1F_1) &= \nu\left(\bigsqcup_{j=1}^{n}t_jE_j \sqcup s_1F_1\right)\\
                           &= \sum_{j=1}^{n}\nu(t_jE_j) + \nu(s_1F_1)\\
                           &= \nu(G) + \nu(s_1F_1)\\
                           &= \nu(G) + \nu(F_1)\\
                           &> \nu(G).
      \end{align*}
\end{remark}
\subsection{Perfect Matchings in Infinite Bipartite Graphs}%
In order to prove that non-paradoxicality implies amenability, we must use some essential results from graph theory.
\begin{definition}[Graphs and Paths]
  A graph is a triple $\left(V,E,\phi\right)$, where $V$ and $E$ are nonempty sets, and $\phi: E\rightarrow P_{2}(V)$ is a map from $e$ to the set of all unordered subset pairs of $V$.\newline

  We say that for $\phi(e) = \set{v,w}$, we say $v$ and $w$ are endpoints of $e$, with $e$ incident on $v$ and $w$.\newline

  A path in $(V,E,\phi)$ is a finite sequence $\left(e_1,\dots,e_n\right)$ of edges along with a finite sequence of vertices $v_0,\dots,v_n$, with $\phi\left(e_k\right) = \set{v_{k-1},v_k}$.\newline

  The degree of a vertex, $\deg(v)$, is the number of edges incident on the vertex.
\end{definition}
\begin{definition}[Bipartite Graphs and $k$-Regularity]
  Let $(V,E,\phi)$ be a graph, $k\in \N$.
  \begin{enumerate}[(i)]
    \item If $\deg(v) = k$ for each $v\in V$, we say $(V,E,\phi)$ is $k$-regular.
    \item If $V = X\sqcup Y$, with each edge having an endpoint in $X$ and an endpoint in $Y$, we say $(V,E,\phi)$ is bipartite.
  \end{enumerate}
\end{definition}
\begin{definition}[Perfect Matching]
Let $(X,Y,E,\phi)$ be a bipartite graph. Let $A\subseteq X$ and $B\subseteq Y$. A perfect matching of $A$ and $B$ is a subset $F\subseteq E$ with
\begin{enumerate}[(i)]
  \item each element of $A\cup B$ is an endpoint of exactly one $f\in F$
  \item all endpoints of edges in $F$ are in $A\cup B$.
\end{enumerate}
\end{definition}
  \begin{exercise}[Hall's Theorem]
  Let $(X,Y,E,\phi)$ be a bipartite graph which is $k$-regular for some $k\in\N$. Suppose $|V| < \infty$.
  \begin{enumerate}[(i)]
    \item Show that $|E| < \infty$ and that $|X| = |Y|$.
    \item For any $M\subseteq V$, let $N(M)$ be the set of those vertices which are joined by an edge with a point in $M$. Show that $|N(M)| \geq |M|$.
    \item Let $A\subseteq X$ and $B\subseteq Y$ be such that there is a perfect matching $F$ of $A$ and $B$ with $|F|$ maximal. Show that $A = X$.
    \item Conclude that there is a perfect matching of $X$ and $Y$.
  \end{enumerate}
  \end{exercise}
  \begin{solution}\hfill
    \begin{enumerate}[(i)]
      \item $|E| = k|X| = k|Y|$, meaning $|X| = |Y|$.
      \item Let $M_X = M\cap X$ and $M_Y = M\cap Y$. Notice that $M = M_X\sqcup M_Y$.\newline

        Let $\left[M_X,N\left(M_X\right)\right]$ denote the set of edges with endpoints in $M_X$ and $N\left(M_X\right)$, and similarly for $\left[M_Y,N\left(M_Y\right)\right]$. We also let $\left[X,N\left(M_X\right)\right]$ and $\left[Y,N\left(M_Y\right)\right]$ denote the set of edges with endpoints in $X$ and $N\left(M_X\right)$ and the set of edges with endpoints in $Y$ and $N\left(M_Y\right)$ respectively.\newline

        We can see that $\left[M_X,N\left(M_X\right)\right]\subseteq \left[X,N\left(M_X\right)\right]$, and similarly with $\left[M_Y,N\left(M_Y\right)\right]\subseteq \left[Y,N\left(M_Y\right)\right]$. Additionally, $\left\vert \left[M_X,N\left(M_X\right)\right] \right\vert = k\left\vert M_X \right\vert$, with $\left\vert \left[X,N\left(M_X\right)\right] \right\vert = k\left\vert N\left(M_X\right) \right\vert$, meaning $|M_X| \leq |N\left(M_X\right)|$, and similarly for $|M_Y|$ and $|N\left(M_Y\right)|$.\newline

        Thus, $|M| \leq |N(M)|$. For future reference, the condition $|M| \leq |N(M)|$ is known as Hall's marriage condition.
      \item Let $A\subseteq X$ and $B\subseteq Y$. Let $F$ be a perfect matching between $A$ and $B$. Suppose toward contradiction that $A\neq X$.\newline

        Thus, there exists $x\in X\setminus A$. Consider the set $Z\subseteq V$ consisting of all vertices such that there exists a $F$-alternating path $\left(e_1,\dots,e_n\right)$ between $z\in Z$ and $x$.\newline

        It cannot be the case that $Z\cap Y$ is empty, since the number of neighbors of $x$ is greater than or equal to $1$ by Hall's marriage condition, and if $Z\cap Y$ were empty, then we could add an element to $F$ consisting of $x$ and one element of $N\left(\set{x}\right)$, which would contradict the maximality of $F$.\newline

        Consider a path that traverses along $Z$, $\left(e_1,\dots,e_n\right)$. It must be the case that $e_n\in F$, as otherwise we would be able to ``flip'' the matching $F$ by exchanging $e_i$ with $e_{i+1}$ if $e_i\in F$. Thus, we must have that every element of $Z\cap Y$ is satisfied by $F$, so $Z\cap Y\subseteq B$.\newline

        Additionally, since each element in $Z\cap Y$ is paired with exactly one element of $Z\cap X$, with one left over, it is the case that $\left\vert Z\cap X \right\vert = \left\vert Z\cap Y \right\vert + 1$.\newline

        Suppose toward contradiction that there exists $y\in N\left(Z\cap X\right)$ with $y\notin Z\cap Y$. Then, there exists $v\in Z\cap X$ and $e\in E$ such that $\phi(e) = \set{v,y}$. However, this means $y$ is connected via a path to $x$, implying that $y\in Z$, so $y\in Z\cap Y$, which is a contradiction. Thus, we must have $N\left(Z\cap X\right) = Z\cap Y$.\newline

        Thus, we have
        \begin{align*}
          \left\vert Z\cap X \right\vert &= \left\vert Z\cap Y \right\vert + 1\\
                                         &= \left\vert N\left(Z\cap X\right)  \right\vert + 1,
        \end{align*}
        which contradicts Hall's marriage condition we established in part (ii). Therefore, $A = X$.
      \item By symmetry, we have $A = X$ and $B = Y$, implying that there is a perfect matching in $\left(X,Y,E,\phi\right)$.
    \end{enumerate}
  \end{solution}
  \begin{definition}[Hall's Marriage Condition]
    For a bipartite graph $\left(X,Y,E,\phi\right)$, there is an $X$-perfect matching\footnote{A matching on $\left(X,Y,E,\phi\right)$ that satisfies every vertex of $X$.} if and only if for all $S\subseteq X$, $\left\vert N(S) \right\vert \geq \left\vert S \right\vert$.\newline

    Equivalently, for a finite collection of (not necessarily distinct) finite sets $\mathcal{G} = \set{X_i}_{i=1}^{n}$, there is a system of distinct representatives\footnote{A set $\set{x_i}_{i=1}^{n}$ such that $x_i\in X_i$ and $x_{i}\neq x_j$ for $i\neq j$.} for $\mathcal{G}$ if and only if for all subcollections $\mathcal{Y}_k = \set{X_{i_k}}_{k=1}^{m}$, $\left\vert \mathcal{Y}_k \right\vert\leq \left\vert \bigcup_{k=1}^{m}X_{i_k} \right\vert$.
  \end{definition}
  \begin{remark}
    These two formulations of Hall's marriage condition are equivalent regarding a $X$-perfect matching.\newline

    In the case of a graph, this yields an injection $f: X\hookrightarrow Y$, and in the case of a collection of sets, this yields an injection $f: \mathcal{G}\hookrightarrow \bigcup_{i=1}^{n}X_i$.\newline
  \end{remark}
  \begin{definition}[Choice Function]
    Let $\mathcal{Y} = \set{X_i}_{i\in I}$ be a collection of sets. A function $f: \mathcal{Y}\rightarrow \bigcup_{i\in I}X_i$ is known as a choice function if, for each $i\in I$, $f\left(X_i\right)\in X_i$.\newline

    We also say that $f: \mathcal{Y}\rightarrow \bigcup_{i\in I}X_i$ is a choice function if $f\in \prod_{i\in I}X_i$.
  \end{definition}
  \begin{remark}
    A choice function on an infinite collection $\mathcal{Y}$ is analogous to a system of distinct representatives on a finite collection $\mathcal{G}$.
  \end{remark}
  \begin{theorem}[Tychonoff]
    Let $\set{X_i}_{i\in I}$ be a family of compact topological spaces. Then, $\prod_{i\in I}X_i$ is compact.
  \end{theorem}
  There is a way to extend Hall's Marriage Condition to a particular infinite case, which we will use to show König's Theorem.
  \begin{theorem}[Infinite Marriage Theorem]
    Let $\mathcal{G} = \set{X_i}_{i\in I}$ be a collection of (not necessarily distinct) finite sets. If, for every finite subcollection $\mathcal{Y} = \set{X_{i_k}}_{k=1}^{n}$,
    \begin{align*}
      n \leq \left\vert \bigcup_{k=1}^{n}X_{i_k} \right\vert,
    \end{align*}
    then there is a choice function on $\mathcal{G}$.
  \end{theorem}
  \begin{proof}
    We endow each $X_i\in \set{X_i}_{i\in I}$ with the discrete topology. Since each $X_i$ is finite, it is the case that each $X_i$ is compact.\newline

    By Tychonoff's theorem, it is the case that $\prod_{i\in I}X_i$ is compact.\newline

    For every finite subset $Y$ of $\mathcal{G}$, define
    \begin{align*}
      S_Y &= \set{\left. f\in \prod_{i\in I}X_i \right|~f\vert_{Y}\text{ is injective}}
    \end{align*}
    In particular, the injectivity of $f\vert_{Y}$ is equivalent the existence of a system of distinct representatives on $Y$. Since $Y$ satisfies Hall's marriage condition, it is the case that each $S_Y$ is nonempty. Additionally, since $\prod_{i\in I}X_i$ is endowed with the discrete topology, it is also the case that $S_Y$ is closed.\newline

    We define $F = \set{S_Y\mid Y\subseteq \mathcal{G}\text{ finite}}$. Notice that, for finite $Y_1,Y_2\subseteq \mathcal{G}$, since every system of distinct representatives on $Y_1\cup Y_2$ is a system of distinct representatives on $Y_1$ and a system of distinct representatives on $Y_2$, it is the case that $S_{Y_1\cup Y_2}\subseteq S_{Y_1}\cap S_{Y_2}$, meaning our collection $\set{S_{Y_j}}$ has the finite intersection property. \newline

    Since $F$ is compact, $\bigcap_{Y\subseteq \mathcal{G}} F$ is nonempty. Thus, for $f\in \bigcap F$, $f$ is necessarily a choice function. 
  \end{proof}
  \begin{remark}
    This is equivalent to the existence of an injection $f: \mathcal{G}\hookrightarrow \bigcup_{i\in I}X_i$.
  \end{remark}
  %\begin{theorem}[König's Theorem]
  %  Let $(X,Y,E,\phi)$ be a bipartite graph which is $k$-regular for some $k\in \N$. Then, there is a perfect matching of $X$ and $Y$.
  %\end{theorem}
  \begin{theorem}[König's Theorem]
    Let $(X,Y,E,\phi)$ be a bipartite graph which is $k$-regular for some $k\in\N$. Then, there is a perfect matching of $X$ and $Y$.
  \end{theorem}
  \begin{proof}
    If $k = 1$, it is clear that there is a perfect matching in $(X,Y,E,\phi)$, consisting exclusively of the edges in $(X,Y,E,\phi)$.\newline

    Let $k\geq 2$. Since any finite subset of $X$ satisfies Hall's marriage condition (as displayed in the proof that a $k$-regular bipartite graph has a perfect matching), it is the case that there is some $X$-perfect matching. We call this $X$-perfect matching $F$.\newline

    Considering $f: X\hookrightarrow Y$ by taking $x \mapsto y$, where $\set{x,y}\in F$, we see that $f$ is an injection.\newline

    Similarly, since any finite subset of $Y$ satisfies Hall's marriage condition (by symmetry), there is some $Y$-perfect matching. We call this $Y$-perfect matching $G$. Considering $g: Y\hookrightarrow X$ by taking $y\mapsto x$, where $\set{x,y}\in G$, we see that $g$ is an injection.\newline

    Consider, now, $\left(X,Y,F\cup G,\phi\vert_{F\cup G}\right)$. We can see that, when restricted to this subgraph of $(X,Y,E,\phi)$, the injections $f$ and $g$ between $X$ and $Y$ still hold. Thus, by the Cantor--Schröder-Bernstein theorem, there is a bijection $h: X\rightarrow Y$ in $\left(X,Y,F\cup G,\phi\vert_{F\cup G}\right)$. This is our desired perfect matching.
  \end{proof}
  \subsection{Type Semigroups}%
  \begin{definition}
    Let $G$ be a group that acts on a set $X$.
    \begin{enumerate}[(i)]
      \item Define $X^{\ast} = X\times \N_{0}$, and
        \begin{align*}
          G^{\ast} &= \set{\left(g,\pi\right)\mid g\in G,\pi\in \text{Sym}\left(\N_0\right)}.
        \end{align*}
        Let $G^{\ast}$ act on $X^{\ast}$ by
        \begin{align*}
          \left(g,\pi\right)\cdot \left(x,n\right) &= \left(g\cdot x,\pi(n)\right).
        \end{align*}
      \item If $A\subseteq X^{\ast}$, then the values of $n\in \N_0$ such that there is an element of $A$ whose second coordinate is $n$ are called the levels of $A$.
    \end{enumerate}
  \end{definition}
  \begin{exercise}
    Show that for $E_1,E_2\subseteq X$, $E_1\sim_{G} E_2$ if and only if $E_1\times\set{n}\sim_{G^{\ast}}E_2\set{m}$ for all $n,m\in \N_0$.
  \end{exercise}
  \begin{solution}
    Let $E_1\sim_{G}E_2$, with $g_1,\dots,g_n\in G$ that satisfy the definition of equidecomposability. Then, composing the permutation $\pi_1: \N_0\rightarrow \N_0$ where $\pi_1(n) = m$, $\pi_1(m) = n$, and $\pi_1(k) = k$ for all $k\neq n$, with $\pi_2,\dots,\pi_n = \id$, we obtain elements $\left(g_1,\pi_1\right),\dots,\left(g_n,\pi_n\right)\in G^{\ast}$ that yield $E_1\times\set{n}\sim_{G^{\ast}}E_2\times \set{m}$.\newline

    Similarly, if $E_1\times\set{n} \sim_{G^{\ast}}E_2\times \set{m}$, we select $g_1,\dots,g_n$ from the elements $\left(g_1,\pi_1\right),\dots,\left(g_n,\pi_n\right)$ that satisfy the definition of equidecomposability. This yields $E_1\sim_{G}E_2$.
  \end{solution}
  \begin{definition}
    Let $G$ be a group that acts on $X$, with $G^{\ast}$ and $X^{\ast}$ defined as previous.
    \begin{enumerate}[(i)]
      \item A set $A\subseteq X^{\ast}$ is called \textit{bounded} if it has finitely many levels.
      \item If $A\subseteq X^{\ast}$ is bounded, the equivalence class of $A$ with respect to equidecomposability under $G^{\ast}$ is called the type of $A$, denoted $[A]$.
      \item If $E\subseteq X$, we write $[E] = \left[E\times \set{0}\right]$.
      \item Let $A,B\subseteq X^{\ast}$ be bounded, with $k\in \N_0$ such that for
        \begin{align*}
          B' &:= \set{\left(b,n+k\right)\mid \left(b,n\right) \in B},
        \end{align*}
        $B'\cap A = \emptyset$. We define $[A] + [B] = \left[A \sqcup B'\right]$.
      \item We define
        \begin{align*}
          \mathcal{S} &= \set{[A]\mid A\subseteq X^{\ast}\text{ bounded}}
        \end{align*}
        under the addition defined in part (iv) to be the type semigroup of the action of $G$ on $X$.
    \end{enumerate}
  \end{definition}
  \begin{exercise}
    Let $G$ be a group acting on a set $X$, and let $\left(\mathcal{S},+\right)$ be the corresponding type semigroup.
    \begin{enumerate}[(i)]
      \item Show that $+$ is well-defined.
      \item Show that $\left(\mathcal{S},+\right)$ is a commutative semigroup with identity $\left[\emptyset\right]$.
    \end{enumerate}
  \end{exercise}
  \begin{solution}\hfill
    \begin{enumerate}[(i)]
      \item Let $\left[A_1\right] = \left[A_2\right]$ and $\left[B_1\right] = \left[B_2\right]$. Without loss of generality, we have $A_1\cap B_1 = \emptyset$ and $A_2\cap B_2 = \emptyset$.\newline

        By the definition of the type, $A_1\sim_{G^{\ast}}A_2$ and $B_1\sim_{G^{\ast}}B_2$, meaning
        \begin{align*}
          A_1\sqcup B_1 \sim_{G^{\ast}}A_2\sqcup B_2,
        \end{align*}
        so
        \begin{align*}
          \left[A_1\right] + \left[B_1\right] &= \left[A_1\sqcup B_1\right]\\
                                              &= \left[A_2\sqcup B_2\right]\\
                                              &= \left[A_2\right] + \left[B_2\right],
        \end{align*}
        proving that addition is well-defined.
      \item We are aware that $+$ is a well-defined operation on $\mathcal{S}$. Since $A\sqcup B = B\sqcup A$, we can also see that $+$ is commutative.\newline

        Finally,
        \begin{align*}
          \left[A\right] + \left[\emptyset\right] &= \left[A\sqcup \emptyset\right]\\
                                                  &= \left[A\right],
        \end{align*}
        meaning $\left[\emptyset\right]$ is the identity on $\mathcal{S}$.\newline

        Finally, since $A$ has finitely many levels, and $B$ has finitely many levels, so too must $A\cup B$ for any $A$ and $B$, meaning $\left[A\right] + \left[B\right]$ is contained in $\mathcal{S}$.
    \end{enumerate}
  \end{solution}
  \begin{definition}
    For any commutative semigroup $\mathcal{S}$ with $\alpha \in \mathcal{S}$ and $n\in \N$,
    \begin{align*}
      n\alpha &:= \underbrace{\alpha + \cdots + \alpha}_{n\text{ times}}
    \end{align*}
  \end{definition}
  \begin{definition}
    For $\alpha,\beta \in \mathcal{S}$, if there exists $\gamma \in \mathcal{S}$ such that $\alpha + \gamma = \beta$, we write $\alpha \leq \beta$.
  \end{definition}
  \begin{exercise}
    Let $G$ be a group acting on a set $X$, and let $\mathcal{S}$ be the corresponding type semigroup.
    \begin{enumerate}[(i)]
      \item If $\alpha,\beta \in S$ with $\alpha \leq \beta$ and $\beta \leq \alpha$, then $\alpha = \beta$.
      \item A set $E\subseteq X$ is paradoxical under the action of $G$ if and only if $[E] = 2[E]$.
    \end{enumerate}
  \end{exercise}
  \begin{solution}
    Let $G$ act on $X$, and let $\mathcal{S}$ denote the corresponding type semigroup.
    \begin{enumerate}[(i)]
      \item If $\left[A\right] \leq \left[B\right]$ and $\left[B\right] \leq \left[A\right]$, then there exist $C_1\subseteq X^{\ast}$ and $C_2\subseteq X^{\ast}$ such that $\left[A\right] + \left[C_1\right] = \left[B\right]$ and $\left[B\right] + \left[C_1\right] = \left[A\right]$. In particular, without loss of generality, $C_1 \cap A = \emptyset$ and $C_2\cap B = \emptyset$, meaning $\left[A\sqcup C_1\right] = \left[B\right]$, and $\left[B\sqcup C_2\right] = \left[A\right]$. Thus, we can see that $A\preceq_{G^{\ast}} B$ and $B\preceq_{G^{\ast}}A$, meaning $A\sim_{G^{\ast}} B$.
      \item Let $E$ be paradoxical under the action of $G$. Then, $E\sim_{G} \bigcup_{i=1}^{n}g_i\cdot A_i$ and $E\sim_{G}\bigcup_{j=1}^{m}h_j\cdot B_j$ for some disjoint $A_1,\dots,A_n,B_1,\dots,B_m\subseteq E$ and $g_1,\dots,g_n,h_1,\dots,h_m\in G$.\newline

        This yields $\left[E\right] = \left[\left(\bigcup_{i=1}^{n}g_i\cdot A_i\right) \sqcup \left(\bigcup_{j=1}^{m}h_j\cdot B_j\right)\right]$, so $\left[E\right] = 2\left[E\right]$.\newline

        Similarly, if $\left[E\right] = 2\left[E\right]$, there necessarily exist $A,B$ such that $\left[E\right] = \left[A\right] + \left[B\right]$, so $\left[E\right] = \left[A\sqcup B\right]$, so $A$ and $B$ are equidecomposable with $E$ under $G$, meaning $E$ is paradoxical.
    \end{enumerate}
  \end{solution}
  \begin{theorem}
    Let $\mathcal{S}$ be the type semigroup for some group action, and let $\alpha,\beta \in S$, $n\in \N$ with $n\alpha = n\beta$. Then, $\alpha = \beta$.
  \end{theorem}
  \begin{proof}
    If $n\alpha = n\beta$, there are two disjoint bounded sets $E,E'\subseteq X^{\ast}$ with $E\sim_{G^{\ast}}E'$, and pairwise disjoint subsets $A_1,\dots,A_n\subseteq E$, $B_1,\dots,B_n\subseteq E'$, with
    \begin{itemize}
      \item $E = A_1\cup \cdots \cup A_n$, $E' = B_1\cup \cdots B_n$,
      \item $\left[A_{j}\right] = \alpha$ and $\left[B_j\right] = \beta$ for $j=1,\dots,n$.
    \end{itemize}
    Let $\chi: E\rightarrow E'$ be a bijection defined from equidecomposability, and $\phi_j: A_1\rightarrow A_j$, $\psi_j: B_1\rightarrow B_j$ also be bijections defined from equidecomposability, with $\psi_1,\phi_1 = \id$.\newline

    For each $a\in A_1$, $b\in B_1$, define
    \begin{align*}
      \overline{a} &:= \set{a,\phi_2(a),\dots,\phi_n(a)}\\
      \overline{b} &:= \set{b,\psi_2(b),\dots,\psi_n(b)}.
    \end{align*}
    Define a bipartite graph with $X = \set{\overline{a}\mid a\in A_1}$, $Y = \set{\overline{b}\mid b\in B_1}$. For each $j$ from $1,\dots,n$ define an edge $\set{\overline{a},\overline{b}}$ if $\chi\left(\phi_j\left(a\right)\right)\in \overline{b}$.\newline

    This graph is $n$-regular, as $\chi$ is a bijection, so for $j = 1,\dots,n$, $\chi\left(\phi_j(a)\right)$ must land in some $B_k$, and only one such $B_k$ (as the $B_k$ are disjoint).\newline

    By König's theorem, this graph has a perfect matching $F$.\newline

    Thus, for each $\overline{a}\in X$, there is a unique $\overline{b}\in Y$ and a unique edge $\left(\overline{a},\overline{b}\right)\in F$ such that $\chi\left(\phi_j\left(a\right)\right) = \psi_k\left(b\right)$ for some $j,k \in \set{1,\dots,n}$.\newline

    Define
    \begin{align*}
      C_{j,k} &:= \set{a\in A_1\mid \left(\overline{a},\overline{b}\right)\in F, \chi\left(\phi_j(a)\right) = \psi_k(b)}
      D_{j,k} &:= \set{b\in A_1\mid \left(\overline{a},\overline{b}\right)\in F, \chi\left(\phi_j(a)\right) = \psi_k(b)}
    \end{align*}
    Since $\psi^{-1}_{k}\circ \chi \circ \phi_j$ is a bijection from $C_{j,k}$ to $D_{j,k}$, this means $C_{j,k}\sim_{G^{\ast}}D_{j,k}$ (as each of $\chi$, $\psi$, and $\phi$ are defined by equidecomposability under $G^{\ast}$).\newline

    Since $C_{j,k}$ and $D_{j,k}$ are partitions of $A_1$ and $B_1$ respectively, it follows that $A_1\sim_{G^{\ast}}B_1$, meaning $\alpha = \beta$.
  \end{proof}
  \begin{corollary}
    Let $\mathcal{S}$ be the type semigroup of some group action, and let $\alpha \in \mathcal{S}$, $n\in \N$ such that $\left(n+1\right)\alpha \leq n\alpha$. Then, $\alpha = 2\alpha$.
  \end{corollary}
  \begin{proof}
    \begin{align*}
      2\alpha + n\alpha &= \left(n+1\right)\alpha + \alpha\\
                        &\leq n\alpha + \alpha\\
                        &= \left(n+1\right)\alpha\\
                        &\leq n\alpha.
    \end{align*}
    Inductively repeating this argument, we get $n\alpha \geq 2n\alpha$. Since it is the case that $n\alpha \leq 2n\alpha$, we have $n\alpha = 2n\alpha$, so $\alpha = 2\alpha$.
  \end{proof}
  We require the following lemma on semigroups in order to be able to create a homomorphism that is necessary to prove Tarski's theorem.
  \begin{lemma}
    Let $\mathcal{S}$ be a commutative semigroup, $\mathcal{S}_0 \subseteq S$ finite, and $\epsilon \in \mathcal{S}_0$ satisfy the following assumptions:
    \begin{enumerate}[(a)]
      \item $\left(n+1\right)\epsilon \not\leq n\epsilon$ for all  $n\in \N$;
      \item for each $\alpha \in \mathcal{S}$, there exists $n\in N$ such that $a \leq n\epsilon$.
    \end{enumerate}
    Then, there is a set function $\nu: \mathcal{S}_0 \rightarrow \left[0,\infty\right]$ that satisfies the following conditions:
    \begin{enumerate}[(i)]
      \item $\nu(\epsilon) = 1$;
      \item for $\alpha_1,\dots,\alpha_n,\beta_1,\dots\beta_m\in \mathcal{S}_0$ with $\alpha_1 + \cdots + \alpha_n \leq \beta_1 + \cdots + \beta_m$, it is the case that
        \begin{align*}
          \sum_{j=1}^{n}\nu\left(\alpha_j\right) \leq \sum_{j=1}^{m}\nu\left(\beta_j\right).
        \end{align*}
    \end{enumerate}
  \end{lemma}
  \begin{proof}
    Inducting on the cardinality of $\mathcal{S}_0$, we start with $\left\vert \mathcal{S}_0 \right\vert = 1$. If that is the case, we can define $\nu\left(\epsilon\right) = 1$, satisfying condition (i). To satisfy condition (ii), for $n,m\in \N$ with $n\epsilon \leq m\epsilon$, then if $n \geq m + 1$, it is the case that $\left(m+1\right)\epsilon \leq n\epsilon \leq m\epsilon$, which is a contradiction of assumption (a).\newline

    Let $\alpha_0\in \mathcal{S}_0 \setminus \set{\epsilon}$. The induction hypothesis states that there is a function $\nu: \mathcal{S}_0\setminus \set{\alpha_0}\rightarrow \left[0,\infty\right]$.\newline

    For $r\in \N$ and $\gamma_1,\dots,\gamma_p,\delta_1,\dots,\delta_q\in \mathcal{S}\setminus \set{\alpha_0}$ such that
    \begin{align*}
      \delta_1 + \cdots + \delta_q + r\alpha_0 \leq \gamma_1 + \dots + \gamma_p,
    \end{align*}
    we let $N$ be the set
    \begin{align*}
      N &= \set{\frac{1}{r}\left(\sum_{j=1}^{p} \nu\left(\gamma_j\right) - \sum_{j=1}^{q}\nu\left(\delta_j\right)\right)},
    \end{align*}
    and define
    \begin{align*}
      \nu\left(\alpha_0\right) &= \inf N.
    \end{align*}
    This infimum is well-defined since, by assumption (b), there is some $n\epsilon$ such that $\alpha_0 + \delta_1 \leq n\epsilon$, and $\nu\left(\epsilon\right)$ is defined.\newline

    We now want to show that this extension of $\nu$ satisfies condition (ii).\newline

    Let $\alpha_1,\dots,\alpha_n,\beta_1,\dots,\beta_m\in \mathcal{S}_0 \setminus \set{\alpha_0}$, with $s,t\in \N_0$ such that
    \begin{align*}
      \alpha_1 + \cdots + \alpha_n + s\alpha_0 \leq \beta_1 + \cdots + \beta_m + t\alpha_0 \label{eq:eq1}\tag{\textasteriskcentered}
    \end{align*}
    For case 0, if $s = t = 0$, the induction hypothesis says that $\nu$ satisfies condition (ii).\newline

    For case 1, let $s = 0$ and $t > 0$. We want to show that
    \begin{align*}
      \sum_{j=1}^{n}\nu\left(\alpha_j\right) \leq t\nu\left(\alpha_0\right) + \sum_{j=1}^{m}\nu\left(\beta_j\right),
    \end{align*}
    which implies
    \begin{align*}
      \nu\left(\alpha_0\right) \geq \frac{1}{t}\left(\sum_{j=1}^{n}\nu\left(\alpha_j\right) - \sum_{j=1}^{m}\nu\left(\beta_j\right)\right).
    \end{align*}
    By the definition of infimum, it suffices to show that for $r\in\N$ and $\delta_1,\dots,\delta_q,\gamma_1,\dots,\gamma_p\in \mathcal{S}\setminus \set{\alpha_0}$ satisfying
    \begin{align*}
      \delta_1 + \cdots + \delta_q + r\alpha_0 \leq \gamma_1 + \cdots + \gamma_p,
    \end{align*}
    it is the case that
    \begin{align*}
      \frac{1}{r}\left(\sum_{j=1}^{p}\nu\left(\gamma_j\right) - \sum_{j=1}^{q}\nu\left(\delta_j\right)\right) \geq \frac{1}{t}\left(\sum_{j=1}^{n}\nu\left(\alpha_j\right) - \sum_{j=1}^{m}\nu\left(\beta_j\right)\right).
    \end{align*}
    Multiplying \ref{eq:eq1} by $r$ on both sides, and adding $t\delta_1 + \cdots + t\delta_q$ to both sides, we get
    \begin{align*}
      r\alpha_1 + \cdots + r\alpha_n + t\delta_1 + \cdots + t\delta_q \leq r\beta_1 + \cdots + r\beta_m + t\left(r\alpha_0\right) + t\delta_1 + \cdots + t\delta_q,
    \end{align*}
    and substituting the relevant inequality for $r\alpha_0$, we find
    \begin{align*}
      r\alpha_1 + \cdots + r\alpha_n + t\delta_1 + \cdots + t\delta_q \leq r\beta_1 + \cdots + r\beta_m + t\gamma_1 + \cdots + t\gamma_p.
    \end{align*}
    Now, we can apply the induction hypothesis, yielding
    \begin{align*}
      r\sum_{j=1}^{n}\nu\left(\alpha_j\right) + t\sum_{j=1}^{q}\nu\left(\delta_j\right) \leq r\sum_{j=1}^{m}\nu\left(\beta_j\right) + t\sum_{j=1}^{p}\nu\left(\gamma_j\right),
    \end{align*}
    yielding
    \begin{align*}
      \frac{1}{r}\left(\sum_{j=1}^{p}\nu\left(\gamma_j\right) - \sum_{j=1}^{q}\nu\left(\delta_j\right)\right) \geq \frac{1}{t}\left(\sum_{j=1}^{n}\nu\left(\alpha_j\right) - \sum_{j=1}^{m}\nu\left(\beta_j\right)\right).
    \end{align*}
    For case 2, let $s > 0$. For $z_1,\dots,z_t\in N$, we need to show that
    \begin{align*}
      s\nu\left(\alpha_0\right) + \sum_{j=1}^{n}\nu\left(\alpha_j\right) &\leq z_1 + \cdots + z_t + \sum_{j=1}^{m}\nu\left(\beta_j\right),
    \end{align*}
    Without loss of generality, we can set $z_1,\dots,z_t = z$ as, for each $z\in N$, $z \geq \nu(\alpha_0)$.\newline

    Multiplying \ref{eq:eq1} by $r\in \N$ and adding $t\delta_1 + \cdots + t\delta_q$ to both sides, we get
    \begin{align*}
      r\alpha_1 + \cdots + r\alpha_n + rs\alpha_0 + t\delta_1 + \cdots + t\delta_q \leq r\beta_1 + \cdots + r\beta_m + t\left(r\alpha_0\right) + t\delta_1 + \cdots + t\delta_q\label{eq:eq2}\tag{\textasteriskcentered\textasteriskcentered}
    \end{align*}
    Let $\delta_1,\dots,\delta_q,\gamma_1,\dots,\gamma_p\in \mathcal{S}_0\setminus \set{\alpha_0}$ satisfy
    \begin{align*}
      \delta_1 +\cdots + \delta_q + r\alpha_0 \leq \gamma_1 + \cdots + \gamma_p, \label{eq:eq3}\tag{\textasteriskcentered\textasteriskcentered\textasteriskcentered}
    \end{align*}
    where $\delta_1,\dots,\delta_q$ and $r$ are as in \ref{eq:eq2}, and define
    \begin{align*}
      z &= \frac{1}{r} \left(\sum_{j=1}^{p}\nu\left(\gamma_j\right) - \sum_{j=1}^{p}\nu\left(\delta_j\right)\right).
    \end{align*}
    Substituting \ref{eq:eq3} into \ref{eq:eq2}, we get
    \begin{align*}
      r\alpha_1 + \cdots + r\alpha_n + t\delta_1 + \cdots + t\delta_q + rs\alpha_0 \leq r\beta_1 + \cdots + r\beta_m + t\gamma_1 + \cdots + t\gamma_p,
    \end{align*}
    yielding
    \begin{align*}
      s\nu\left(\alpha_0\right) + \sum_{j=1}^{n}\nu\left(\alpha_j\right) &\leq \sum_{j=1}^{n}\nu\left(\alpha_j\right) + \frac{s}{sr}\left(r\sum_{j=1}^{m}\nu\left(\beta_j\right) - r\sum_{j=1}^{n}\nu\left(\alpha_j\right) + t\sum_{j=1}^{p}\nu\left(\gamma_j\right) -t\sum_{j=1}^{q}\nu\left(\delta_j\right)\right)\\
                                                                         &= tz + \sum_{j=1}^{m}\nu\left(\beta_j\right).
    \end{align*}
  \end{proof}
  \begin{theorem}
    Let $\left(\mathcal{S},+\right)$ be a commutative semigroup with identity element $0$, and let $\epsilon \in \mathcal{S}$. Then, the following are equivalent:
    \begin{enumerate}[(i)]
      \item $\left(n+1\right)\epsilon \nleq n\epsilon$ for all $n\in \N$.
      \item There is a semigroup homomorphism $\nu: \left(\mathcal{S},+\right)\rightarrow \left([0,\infty],+\right)$ such that $\nu\left(\epsilon\right) = 1$.
    \end{enumerate}
  \end{theorem}
  \begin{proof}
    To show (ii) implies (i), we let $\nu: \left(\mathcal{S},+\right)\rightarrow \left([0,\infty],+\right)$ be a semigroup homomorphism with $\nu\left(\epsilon\right) = 1$. Then, by the definition of a semigroup homomorphism, it is the case that, for any $n\in \N$,
    \begin{align*}
      \nu\left(\left(n+1\right)\epsilon\right) &= \left(n+1\right)\nu\left(\epsilon\right)\\
                                               &= n + 1\\
                                               &> n\\
                                               &= n\nu\left(\epsilon\right)\\
                                               &= \nu\left(n\epsilon\right)
    \end{align*}
    implying that $\left(n+1\right)\epsilon \nleq n\epsilon$.\newline

    To show that (i) implies (ii), we suppose that for each $\alpha \in \mathcal{S}$, there is $n\in \N$ such that $\alpha \leq n\epsilon$ --- we define any such $\alpha$ for which this is not the case by $\nu\left(\alpha\right) = \infty$.\newline

    For a finite subset $\mathcal{S}_0 \subseteq \mathcal{S}$ with $\epsilon \in \mathcal{S}_0$, define $M_{\mathcal{S}_0}$ to be the set of all $\kappa: \mathcal{S}\rightarrow [0,\infty]$ such that
    \begin{itemize}
      \item $\kappa\left(\epsilon\right) = 1$;
      \item $\kappa \left(\alpha + \beta\right) = \kappa\left(\alpha\right) + \kappa\left(\beta\right)$, with $\alpha,\beta,\alpha + \beta \in \mathcal{S}_0$.
    \end{itemize}
    Since condition (i) is assumed, we know that such a $\kappa$ with $\kappa\left(\epsilon\right) = 1$ must exist, and since
    \begin{align*}
      \alpha + \beta \leq \left(\alpha + \beta\right)
      \intertext{and}
      \left(\alpha + \beta\right) \leq \alpha + \beta,
    \end{align*}
    it is the case that
    \begin{align*}
      \kappa\left(\alpha + \beta\right) \leq \kappa\left(\alpha\right) + \kappa\left(\beta\right) \leq \kappa\left(\alpha + \beta\right),
    \end{align*}
    meaning $\kappa\left(\alpha + \beta\right) = \kappa\left(\alpha\right) + \kappa\left(\beta\right)$. Thus, $M_{\mathcal{S}_0}$ is nonempty.\newline

    We let $\set{\kappa\mid \kappa: \mathcal{S}\rightarrow [0,\infty]} = \left[0,\infty\right]^{\mathcal{S}}$ be equipped with the product topology. By Tychonoff's theorem, $\left[0,\infty\right]^{\mathcal{S}}$ is compact.\newline

    It is also the case that $M_{\mathcal{S}_0}$ is closed, since any net of functions $\kappa_{p}: \mathcal{S}\rightarrow [0,\infty]$ with $\kappa_{p}\left(\epsilon\right) = 1$ and $\kappa_{p}\left(\alpha + \beta\right) = \kappa_{p}\left(\alpha\right) + \kappa_{p}\left(\beta\right)$ will necessarily satisfy these conditions in the limit.\newline

    Additionally, the family
    \begin{align*}
      \set{M_{\mathcal{S}_0}\mid \mathcal{S}_0\subseteq \mathcal{S}\text{ finite}}
    \end{align*}
    has the finite intersection property since, for any $\mathcal{S}_1,\dots,\mathcal{S}_n$, it is the case that
    \begin{align*}
      M_{\mathcal{S}_1\cup\cdots\cup \mathcal{S}_n} &\subseteq M_{\mathcal{S}_1}\cap \cdots \cap M_{\mathcal{S}_n},
    \end{align*}
    since any such $\kappa\in M_{\mathcal{S}_1\cup\cdots\cup \mathcal{S}_n}$ must necessarily be in every $M_{\mathcal{S}_i}$.\newline

    Thus, by Cantor's intersection theorem, there is some $\nu\in \bigcap\set{M_{\mathcal{S}_0}\mid \mathcal{S}_0\subseteq \mathcal{S}\text{ finite}}$, meaning $\nu\left(\epsilon\right) = 1$ and for all $\alpha,\beta \in \mathcal{S}$, $\nu \in M_{\set{\alpha,\beta,\alpha + \beta}}$, meaning $\nu\left(\alpha + \beta\right) = \nu\left(\alpha\right) + \nu\left(\beta\right)$.
  \end{proof}
  \subsection{Proof of Tarski's Theorem}%
  \begin{proof}
    Let $\mathcal{S}$ be the type semigroup of the action of $G$ on $X$.\newline

    Suppose $E$ is not paradoxical under the action of $G$. Then, $\left[E\right] \neq 2\left[E\right]$, implying $\left(n+1\right)\left[E\right]\nleq n\left[E\right]$ for all $n\in \N$.\newline

    Thus, there is a map $\nu: \mathcal{S} \rightarrow [0,\infty]$ with $\nu\left(\left[E\right]\right) = 1$. Thus, the map
    \begin{align*}
      \mu: P\left(X\right)\rightarrow [0,\infty]
    \end{align*}
    defined by $\mu\left(A\right) = \nu\left(\left[A\right]\right)$ is the desired finitely additive measure.
  \end{proof}
  For any set $X$, with a finitely additive measure $\mu: P\left(X\right)\rightarrow [0,\infty]$, and $\mu(X) < \infty$, we can define a mean $m\in \ell^{\infty}\left(X\right)^{\ast}$ by taking
  \begin{align*}
    m\left(\phi\right) &= \int_{X}^{} \phi(x)\:d\mu(x),
  \end{align*}
  where $\phi \in \ell^{\infty}\left(X\right)$.\newline

  When $G$ is non-paradoxical, the function $m$ also has the property that, for all $g\in G$ and all $\phi \in \ell^{\infty}\left(G\right)$,
  \begin{align*}
    m\left(\delta_{g}\ast \phi\right) &= \int_{G}^{} \phi\left(gs\right)\:d\mu(s)\\
                                      &= \int_{G}^{} \phi\left(s\right)\:d\mu(s)\\
                                      &= m\left(\phi\right).
  \end{align*}
  \begin{corollary}[Equivalent Conditions for Group Amenability]
    For a group $G$, the following are equivalent.
    \begin{enumerate}[(i)]
      \item $G$ is non-paradoxical.
      \item there is a translation-invariant finitely additive probability measure on $G$ --- i.e., $\mu: P(G)\rightarrow [0,1]$ with $\mu(G) = 1$ and $\mu(gE) = \mu(E)$ for all $g\in G$ and $E\subseteq G$;
      \item there is $m\in \ell^{\infty}\left(G\right)^{\ast}$ with $m\left(\mathbb{1}_{G}\right) = 1$, $\norm{m}_{\text{op}} = 1$, and for all $g\in G$ and $\phi \in \ell^{\infty}\left(G\right)$,
        \begin{align*}
          m\left(\delta_{g}\ast \phi\right) &= m\left(\phi\right).
        \end{align*}
    \end{enumerate}
  \end{corollary}
\end{document}
