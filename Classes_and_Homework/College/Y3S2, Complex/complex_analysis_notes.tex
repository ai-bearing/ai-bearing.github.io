\documentclass[10pt]{extarticle}
\title{}
\author{}
\date{}
\usepackage[shortlabels]{enumitem}


%paper setup
\usepackage{geometry}
\geometry{letterpaper, portrait, margin=1in}
\usepackage{fancyhdr}
% sans serif font:
\usepackage{cmbright}
%symbols
\usepackage{amsmath}
\usepackage{bigints}
\usepackage{amssymb}
\usepackage{amsthm}
\usepackage{mathtools}
\usepackage{bbm}
\usepackage[hidelinks]{hyperref}
\usepackage{gensymb}
\usepackage{multirow,array}
\usepackage{multicol}

\newtheorem*{remark}{Remark}
\usepackage[T1]{fontenc}
\usepackage[utf8]{inputenc}

%chemistry stuff
%\usepackage[version=4]{mhchem}
%\usepackage{chemfig}

%plotting
\usepackage{pgfplots}
\usepackage{tikz}
\tikzset{middleweight/.style={pos = 0.5}}
%\tikzset{weight/.style={pos = 0.5, fill = white}}
%\tikzset{lateweight/.style={pos = 0.75, fill = white}}
%\tikzset{earlyweight/.style={pos = 0.25, fill=white}}

%\usepackage{natbib}

%graphics stuff
\usepackage{graphicx}
\graphicspath{ {./images/} }
\usepackage[style=numeric, backend=biber]{biblatex} % Use the numeric style for Vancouver
\addbibresource{the_bibliography.bib}
%code stuff
%when using minted, make sure to add the -shell-escape flag
%you can use lstlisting if you don't want to use minted
%\usepackage{minted}
%\usemintedstyle{pastie}
%\newminted[javacode]{java}{frame=lines,framesep=2mm,linenos=true,fontsize=\footnotesize,tabsize=3,autogobble,}
%\newminted[cppcode]{cpp}{frame=lines,framesep=2mm,linenos=true,fontsize=\footnotesize,tabsize=3,autogobble,}

%\usepackage{listings}
%\usepackage{color}
%\definecolor{dkgreen}{rgb}{0,0.6,0}
%\definecolor{gray}{rgb}{0.5,0.5,0.5}
%\definecolor{mauve}{rgb}{0.58,0,0.82}
%
%\lstset{frame=tb,
%	language=Java,
%	aboveskip=3mm,
%	belowskip=3mm,
%	showstringspaces=false,
%	columns=flexible,
%	basicstyle={\small\ttfamily},
%	numbers=none,
%	numberstyle=\tiny\color{gray},
%	keywordstyle=\color{blue},
%	commentstyle=\color{dkgreen},
%	stringstyle=\color{mauve},
%	breaklines=true,
%	breakatwhitespace=true,
%	tabsize=3
%}
% text + color boxes
\renewcommand{\mathbf}[1]{\mathbbm{#1}}
\usepackage[most]{tcolorbox}
\tcbuselibrary{breakable}
\tcbuselibrary{skins}
\newtcolorbox{problem}[1]{colback=white,enhanced,title={\small #1},
          attach boxed title to top center=
{yshift=-\tcboxedtitleheight/2},
boxed title style={size=small,colback=black!60!white}, sharp corners, breakable}
%including PDFs
%\usepackage{pdfpages}
\setlength{\parindent}{0pt}
\usepackage{cancel}
\pagestyle{fancy}
\fancyhf{}
\rhead{Avinash Iyer}
\lhead{Complex Analysis: Class Notes}
\newcommand{\card}{\text{card}}
\newcommand{\ran}{\text{ran}}
\newcommand{\N}{\mathbbm{N}}
\newcommand{\Q}{\mathbbm{Q}}
\newcommand{\Z}{\mathbbm{Z}}
\newcommand{\R}{\mathbbm{R}}
\newcommand{\C}{\mathbbm{C}}
\setcounter{secnumdepth}{0}
\begin{document}
\section{Complex Numbers}%
  A complex number is an ordered pair of real numbers, $(a,b)=a+bi$. A vector in $\R^2$ is also an ordered pair, $(a,b)$ of real numbers.\\

  Indeed, vector addition and scalar multiplication on complex numbers are defined just as with $\R^2$. However, unlike vectors in $\R^2$, there is also an operation $\cdot$. We desire for $(0,1)\cdot (0,1) = (-1,0)$; essentially, $i^2 = -1$. We say that $i$ is a square foot of $-1$; every complex number except $0$ has two square roots.
  \begin{align*}
    (a,b)\cdot (c,d) &= (a+bi) + (c+di)\\
                     &:= a(c) + adi + bci + bd(i^2)\\
                     &:= (ac-bd) + (ad+bc)i\\
                     &= (ac-bd,ad+bc)
  \end{align*}
  Thus, $\R^2$ with the operations $+$ and the above defined complex multiplication is known as $\C$. We write as $a+bi$ instead of $(a,b)$.\\

  Given $z=(a+bi)\in \C$, we write $\text{Re}(z) = a$ and $\text{Im}(z) = b$. If $\text{Im}(z) = 0$, then $z\in \R\times\{0\} \subset \C$. However, many people say that $\R\subseteq \C$, even if $\C$ isn't defined as such.
  \subsection{Reciprocals of Complex Numbers}%
  Let $z\in \C$, where $z\neq 0$. Then, $\exists w\in C$ such that $zw = 1$.\\

  Let $w = c+di$. We want to show that $zw = 1$.
  \begin{align*}
    (a+bi) + (c+di) &= (ac-bd) + (ad+bc)i\\
    \shortintertext{with the condition that}
    ac-bd &= 1\\
    ad + bc &= 0.\\
    \shortintertext{Thus, let $w = c+di$, with $a,b\neq 0$}
    c &= \frac{a}{a^2 + b^2}\\
    d &= \frac{-b}{a^2 + b^2}
  \end{align*}
  For every $z\neq 0$, with $z = a+bi$, the \textit{reciprocal} of $z$ is defined as $\frac{1}{z} = \frac{a}{a^2 + b^2} + \frac{-b}{a^2 + b^2}i$. Then, for $w\in \C$, we define
  \begin{align*}
    \frac{w}{z} &:= w \left(\frac{1}{z}\right).
  \end{align*}
  \section{Properties of Complex Numbers}%
  Let $z = a+bi\in C$. Then, the (Euclidean) norm (or absolute value) of $z$ is defined as
  \begin{align*}
    |z| &= \sqrt{a^2 + b^2}.
  \end{align*}
  The conjugate of $z = a+bi$ is $\overline{z} = a-bi$.
  \begin{enumerate}[(i)]
    \item $z\overline{z} = |z|^2$
    \item $\overline{(\overline{z})}=z$
    \item $\overline{(z+w)} = \overline{z} + \overline{w}$
    \item $\overline{zw} = \overline{z}\cdot\overline{w}$
    \item $z + \overline{z} = 2\text{Re}(z)$, so $\text{Re}(z) = \frac{z + \overline{z}}{2}$
    \item $z - \overline{z} = 2\text{Im}(z)i$, so $\text{Im}(z) = \frac{z-\overline{z}}{2i}$
  \end{enumerate}
  \subsection{Polar Representation}%
  Let $z = a+bi$ (or $z = (a,b)$). Then, $|z| = \sqrt{a^2 + b^2}$ is the \textit{radius}, and the \textit{argument} is found by $\theta = \arctan(b/a)$ for $a\neq 0$. Therefore, the full polar representation is as follows:
  \begin{align*}
    z = |z|\left(\cos\theta + i\sin\theta\right) \tag*{$\theta \in [0,2\pi)$}.
  \end{align*}
  If $z = 0$, then $|z| = 0$, and $\arg z$ is undefined.\\

  For example, we can find $\arg i$ in $[\pi,3\pi)$ as $\frac{5\pi}{2}$.\\

  For $z_1$ and $z_2$ in polar form, we have:
  \begin{align*}
    |z_1z_2| &= |z_1||z_2| \tag*{(1)}\\
    \arg(z_1z_2) &= \arg z_1 + \arg z_2 \mod 2\pi \tag*{(2)}
  \end{align*}
  Proof of (1):
  \begin{align*}
    |z_1z_2|^2 &= (z_1z_2)\overline{(z_1z_2)}\\
               &= z_1z_2\overline{z_1}\overline{z_2}\\
               &= z_1\overline{z_1}z_2\overline{z_2}\\
               &= |z_1|^2|z_2|^2
  \end{align*}
  Since $|z|\geq 0$, we get $|z_1z_2| = |z_1||z_2|$.\\

  Let $z = 2(\cos \pi/6 + i\sin\pi/6)$, and let $f: \C \rightarrow \C$ defined as $f(w) = zw$. Then, $f$ rotates $w$ by $\pi/6$ and scales $w$ by $2$.\\

  \begin{description}
    \item[Theorem:] For $n\in \N$, if $z = r(\cos\theta + i\sin\theta)$, then $z^n = r^n(\cos(n\theta) + i\sin(n\theta))$.
    \item[Proof:] Induct on $n$. For the base case, we know that $n=1$ satisfies this property. For $n > 1$, we have:
      \begin{align*}
        z^{n+1} &= (z^n)(z)\\
                &= \left(r^n(\cos(n\theta) + i\sin(n\theta))\right)r(\cos\theta + i\sin\theta)\\
                &= (r^n)(r)\left(\cos(n\theta + \theta) + i\sin(n\theta + \theta)\right) \tag*{Polar Representation Definition}\\
                &= r^{n+1}(\cos\left((n+1)\theta\right) + i\sin((n+1)\theta))
      \end{align*}
  \end{description}
  We can use this technique to find the ``roots of unity.'' For example, to find all $z$ such that $z^3 = 1$, we use our technique:
  \begin{align*}
    z^3 &= 1\\
    |z| &= 1\\
    \arg z^3 &= 0\\
    3\arg z &= 0 \mod 2\pi\\
    \arg z &= \frac{k2\pi}{3}\\
           &= 0,\frac{2\pi}{3},\frac{4\pi}{3}\\
    z_1 &= 1\\
    z_2 &= (\cos 2\pi/3 + i\sin 2\pi/3)\\
    z_3 &= (\cos 4\pi/3 + i\sin 4\pi/3)
  \end{align*}
  We can see that $z_2^2 = z_3$.\\

  For the $n$ case, we find $z_2 = \cos(2\pi/n) + i\sin(2\pi/n)$, and $z_{k} = z_2^{k-1}$.
\end{document}
