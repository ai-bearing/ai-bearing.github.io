\documentclass[10pt]{extarticle}
\title{}
\author{}
\date{}
\usepackage[shortlabels]{enumitem}


%paper setup
\usepackage{geometry}
\geometry{letterpaper, portrait, margin=1in}
\usepackage{fancyhdr}
% sans serif font:
\usepackage{cmbright}
%symbols
\usepackage{amsmath}
\usepackage{bigints}
\usepackage{amssymb}
\usepackage{amsthm}
\usepackage{mathtools}
\usepackage{bbm}
\usepackage[hidelinks]{hyperref}
\usepackage{gensymb}
\usepackage{multirow,array}
\usepackage{multicol}

\newtheorem*{remark}{Remark}
\usepackage[T1]{fontenc}
\usepackage[utf8]{inputenc}

%chemistry stuff
%\usepackage[version=4]{mhchem}
%\usepackage{chemfig}

%plotting
\usepackage{pgfplots}
\usepackage{tikz}
\tikzset{middleweight/.style={pos = 0.5}}
%\tikzset{weight/.style={pos = 0.5, fill = white}}
%\tikzset{lateweight/.style={pos = 0.75, fill = white}}
%\tikzset{earlyweight/.style={pos = 0.25, fill=white}}

%\usepackage{natbib}

%graphics stuff
\usepackage{graphicx}
\graphicspath{ {./images/} }
\usepackage[style=numeric, backend=biber]{biblatex} % Use the numeric style for Vancouver
\addbibresource{the_bibliography.bib}
%code stuff
%when using minted, make sure to add the -shell-escape flag
%you can use lstlisting if you don't want to use minted
%\usepackage{minted}
%\usemintedstyle{pastie}
%\newminted[javacode]{java}{frame=lines,framesep=2mm,linenos=true,fontsize=\footnotesize,tabsize=3,autogobble,}
%\newminted[cppcode]{cpp}{frame=lines,framesep=2mm,linenos=true,fontsize=\footnotesize,tabsize=3,autogobble,}

%\usepackage{listings}
%\usepackage{color}
%\definecolor{dkgreen}{rgb}{0,0.6,0}
%\definecolor{gray}{rgb}{0.5,0.5,0.5}
%\definecolor{mauve}{rgb}{0.58,0,0.82}
%
%\lstset{frame=tb,
%	language=Java,
%	aboveskip=3mm,
%	belowskip=3mm,
%	showstringspaces=false,
%	columns=flexible,
%	basicstyle={\small\ttfamily},
%	numbers=none,
%	numberstyle=\tiny\color{gray},
%	keywordstyle=\color{blue},
%	commentstyle=\color{dkgreen},
%	stringstyle=\color{mauve},
%	breaklines=true,
%	breakatwhitespace=true,
%	tabsize=3
%}
% text + color boxes
\renewcommand{\mathbf}[1]{\mathbbm{#1}}
\usepackage[most]{tcolorbox}
\tcbuselibrary{breakable}
\tcbuselibrary{skins}
\newtcolorbox{problem}[1]{colback=white,enhanced,title={\small #1},
          attach boxed title to top center=
{yshift=-\tcboxedtitleheight/2},
boxed title style={size=small,colback=black!60!white}, sharp corners, breakable}
%including PDFs
%\usepackage{pdfpages}
\setlength{\parindent}{0pt}
\usepackage{cancel}
\pagestyle{fancy}
\fancyhf{}
\rhead{Avinash Iyer}
\lhead{Complex Analysis: Class Notes}
\newcommand{\card}{\text{card}}
\newcommand{\ran}{\text{ran}}
\newcommand{\N}{\mathbbm{N}}
\newcommand{\Q}{\mathbbm{Q}}
\newcommand{\Z}{\mathbbm{Z}}
\newcommand{\R}{\mathbbm{R}}
\newcommand{\C}{\mathbbm{C}}
\setcounter{secnumdepth}{0}
\begin{document}
\section{Complex Numbers}%
  A complex number is an ordered pair of real numbers, $(a,b)=a+bi$. A vector in $\R^2$ is also an ordered pair, $(a,b)$ of real numbers.\\

  Indeed, vector addition and scalar multiplication on complex numbers are defined just as with $\R^2$. However, unlike vectors in $\R^2$, there is also an operation $\cdot$. We desire for $(0,1)\cdot (0,1) = (-1,0)$; essentially, $i^2 = -1$. We say that $i$ is a square foot of $-1$; every complex number except $0$ has two square roots.
  \begin{align*}
    (a,b)\cdot (c,d) &= (a+bi) + (c+di)\\
                     &:= a(c) + adi + bci + bd(i^2)\\
                     &:= (ac-bd) + (ad+bc)i\\
                     &= (ac-bd,ad+bc)
  \end{align*}
  Thus, $\R^2$ with the operations $+$ and the above defined complex multiplication is known as $\C$. We write as $a+bi$ instead of $(a,b)$.\\

  Given $z=(a+bi)\in \C$, we write $\text{Re}(z) = a$ and $\text{Im}(z) = b$. If $\text{Im}(z) = 0$, then $z\in \R\times\{0\} \subset \C$. However, many people say that $\R\subseteq \C$, even if $\C$ isn't defined as such.
  \subsection{Reciprocals of Complex Numbers}%
  Let $z\in \C$, where $z\neq 0$. Then, $\exists w\in \C$ such that $zw = 1$.\\

  Let $w = c+di$. We want to show that $zw = 1$.
  \begin{align*}
    (a+bi) + (c+di) &= (ac-bd) + (ad+bc)i\\
    \shortintertext{with the condition that}
    ac-bd &= 1\\
    ad + bc &= 0.\\
    \shortintertext{Thus, let $w = c+di$, with $a,b\neq 0$}
    c &= \frac{a}{a^2 + b^2}\\
    d &= \frac{-b}{a^2 + b^2}
  \end{align*}
  For every $z\neq 0$, with $z = a+bi$, the \textit{reciprocal} of $z$ is defined as $\frac{1}{z} = \frac{a}{a^2 + b^2} + \frac{-b}{a^2 + b^2}i$. Then, for $w\in \C$, we define
  \begin{align*}
    \frac{w}{z} &:= w \left(\frac{1}{z}\right).
  \end{align*}
  \section{Properties of Complex Numbers}%
  Let $z = a+bi\in C$. Then, the (Euclidean) norm (or absolute value) of $z$ is defined as
  \begin{align*}
    |z| &= \sqrt{a^2 + b^2}.
  \end{align*}
  The conjugate of $z = a+bi$ is $\overline{z} = a-bi$.
  \begin{enumerate}[(i)]
    \item $z\overline{z} = |z|^2$
    \item $\overline{(\overline{z})}=z$
    \item $\overline{(z+w)} = \overline{z} + \overline{w}$
    \item $\overline{zw} = \overline{z}\cdot\overline{w}$
    \item $z + \overline{z} = 2\text{Re}(z)$, so $\text{Re}(z) = \frac{z + \overline{z}}{2}$
    \item $z - \overline{z} = 2\text{Im}(z)i$, so $\text{Im}(z) = \frac{z-\overline{z}}{2i}$
  \end{enumerate}
  \subsection{Polar Representation}%
  Let $z = a+bi$ (or $z = (a,b)$). Then, $|z| = \sqrt{a^2 + b^2}$ is the \textit{radius}, and the \textit{argument} is found by $\theta = \arctan(b/a)$ for $a\neq 0$. Therefore, the full polar representation is as follows:
  \begin{align*}
    z = |z|\left(\cos\theta + i\sin\theta\right) \tag*{$\theta \in [0,2\pi)$}.
  \end{align*}
  If $z = 0$, then $|z| = 0$, and $\arg z$ is undefined.\\

  For example, we can find $\arg i$ in $[\pi,3\pi)$ as $\frac{5\pi}{2}$.\\

  For $z_1$ and $z_2$ in polar form, we have:
  \begin{align*}
    |z_1z_2| &= |z_1||z_2| \tag*{(1)}\\
    \arg(z_1z_2) &= \arg z_1 + \arg z_2 \mod 2\pi \tag*{(2)}
  \end{align*}
  Proof of (1):
  \begin{align*}
    |z_1z_2|^2 &= (z_1z_2)\overline{(z_1z_2)}\\
               &= z_1z_2\overline{z_1}\overline{z_2}\\
               &= z_1\overline{z_1}z_2\overline{z_2}\\
               &= |z_1|^2|z_2|^2
  \end{align*}
  Since $|z|\geq 0$, we get $|z_1z_2| = |z_1||z_2|$.\\

  Let $z = 2(\cos \pi/6 + i\sin\pi/6)$, and let $f: \C \rightarrow \C$ defined as $f(w) = zw$. Then, $f$ rotates $w$ by $\pi/6$ and scales $w$ by $2$.\\

  \begin{description}
    \item[Theorem:] For $n\in \N$, if $z = r(\cos\theta + i\sin\theta)$, then $z^n = r^n(\cos(n\theta) + i\sin(n\theta))$.
    \item[Proof:] Induct on $n$. For the base case, we know that $n=1$ satisfies this property. For $n > 1$, we have:
      \begin{align*}
        z^{n+1} &= (z^n)(z)\\
                &= \left(r^n(\cos(n\theta) + i\sin(n\theta))\right)r(\cos\theta + i\sin\theta)\\
                &= (r^n)(r)\left(\cos(n\theta + \theta) + i\sin(n\theta + \theta)\right) \tag*{Polar Representation Definition}\\
                &= r^{n+1}(\cos\left((n+1)\theta\right) + i\sin((n+1)\theta))
      \end{align*}
  \end{description}
  We can use this technique to find the ``roots of unity.'' For example, to find all $z$ such that $z^3 = 1$, we use our technique:
  \begin{align*}
    z^3 &= 1\\
    |z| &= 1\\
    \arg z^3 &= 0\\
    3\arg z &= 0 \mod 2\pi\\
    \arg z &= \frac{k2\pi}{3}\\
           &= 0,\frac{2\pi}{3},\frac{4\pi}{3}\\
    z_1 &= 1\\
    z_2 &= (\cos 2\pi/3 + i\sin 2\pi/3)\\
    z_3 &= (\cos 4\pi/3 + i\sin 4\pi/3)
  \end{align*}
  We can see that $z_2^2 = z_3$.\\

  For the $n$ case, we find $z_2 = \cos(2\pi/n) + i\sin(2\pi/n)$, and $z_{k} = z_2^{k-1}$.
  \section{Exponential, Logarithm, and Trigonometric Functions in $\C$}%
  \subsection{Exponential}%
  Let $z = a+bi$. We define $e^{a+bi}$ as follows:
  \begin{align*}
    e^{a+bi} &= e^a \left(\cos b + i\sin b\right)
  \end{align*}
  Recall that for every nonzero complex number, $z = |z|\left(\cos \theta + i\sin\theta\right)$, where $\theta = \arg z$. Thus,
  \begin{align*}
    z &= |z|e^{i\theta}\\
      &= |z|e^{i\arg z}.
  \end{align*}
  The function $e^z$ has some properties similar to the function $e^x$ in real numbers, and some properties varying with the real numbers.
  \begin{align*}
    e^ze^w &= e^{z+w}\\
    e^z &\neq 0
  \end{align*}
  However, there are some differences:
  \begin{align*}
    |e^{i\theta}| &= 1 \tag*{$\forall \theta$}\\
    e^{a+bi} &= e^a
  \end{align*}
  From these properties, we find Euler's equation:
  \begin{align*}
    e^{i\pi} + 1 &= 0
  \end{align*}
  Additionally, $e^z$ is periodic, while $f(x) = e^x$ is injective:
  \begin{align*}
    e^{z + 2n\pi} &= e^{z}\left(\cos(2n\pi) + i\sin{2n\pi}\right)\\
                  &= e^z
  \end{align*}
  When examining the function $f: \C\rightarrow\C\setminus\{0\}$, $z \mapsto e^z$, we find that the following happen:
  \begin{itemize}
    \item $f(\R) = (0,\infty)$ --- we apply $f(x) = e^x$.
    \item $f(a+bi) = e^ae^{bi}$ --- $e^a$ is rotated by $b$.
    \item $f(\R + bi)$ is expressed as the line along $b$ radians through the origin.
    \item Therefore, $f(A_0) = \C\setminus\{0\}$, where $A_0 = \{a+bi\mid a\in \R, b\in [0,2\pi)\}$.
  \end{itemize}
  \subsection{Logarithm}%
  
  Recall that for a function $f: A\rightarrow B$, $f^{-1}$ is a function if $f$ is injective. However, for any $f$, it is the case that $f^{-1}(b)$ does exist, defined as follows:
  \begin{align*}
    f^{-1}(b) &= \{a\mid f(a) = b\}.
  \end{align*}

  For the function $f(z) = e^z$, $f$ is not one to one, so for $w = e^z$, $f^{-1}(w) = \{z'\in\C \mid e^{z'} = w\}$. We can find this as $f^{-1}(w) = \{z + 2n\pi i\mid n\in\Z\}$.\\

  We define $\log(w) := \{z\in\C\mid e^z = w\}$. For a fixed $\theta\in\R$, we define $\log_{A_{\theta}}(w) := \{z\mid e^{z}=w,z\in A_{\theta}\}$.\\

  Let $z = 1 + \frac{5\pi}{2}i$. Then,
  \begin{align*}
    \log_{A_{-\pi}}e^z &= 1 + \frac{\pi}{2}i
  \end{align*}
  Let $w\in \C\setminus\{0\}$. To find $\log w$ (all values), then
  \begin{align*}
    z&\in \log w\\
    e^z &= w\\
        &= |w|e^{i\arg w}\\
    e^{a+bi} &= |w|e^{i\arg w}\\
    e^ae^{ib} &= |w|e^{i\arg w}.
  \end{align*}
  Therefore, $a = \ln|w|$ and $b = \arg w$. Additionally, the following hold, for $z_1,z_2\in \C$:
  \begin{align*}
    \log_{A_{\theta}}(z_1z_2) &= \log_{A_{\theta}}(z_1) + \log_{A_{\theta}}(z_2) + 2n\pi i
  \end{align*}
  \subsection{Cosine and Sine}%
  \begin{align*}
    e^{ib} &= \cos b + i\sin b\\
    e^{-ib} &= \cos b - i\sin b\\
    \cos z &:= \frac{e^{iz} + e^{-iz}}{2}\\
    \sin z &:= \frac{e^{iz}-e^{-iz}}{2i}
  \end{align*}
  \subsection{Complex Powers}%
  Recall that for $s,t\in\R$, $s^t = e^{t\ln s}$, where $s > 0$. For $z,w\in \C$, $z^w = e^{w\log z}$., where $z\neq 0$.
  \begin{align*}
    (-2)^{i} &= e^{i\log(-2)}\\
             &= e^{i\left(\ln(2) + i\pi\right)}\\
             &= e^{i\ln 2 - (\pi + 2\pi n)}\\
             &= e^{-\pi + 2\pi n + i\ln 2}
  \end{align*}
  This has \textit{infinitely} many values.\\

  Let $\alpha = u + vi$. Then,
  \begin{align*}
    z^{\alpha} &= e^{\alpha \log z}\\
               &= e^{(u+vi)(\ln|z| + i\arg z)}\\
               &= e^{(u\ln|z| - v\arg z)}e^{i(v\ln|z| + u\arg z)}\\
   \intertext{Since $\arg z = \theta + 2\pi n$ for some real $\theta\in[0,2\pi)$,}\\
               &= e^{u\ln z} e^{-v(\theta + 2\pi n)} e^{iv\ln|z|} e^{iu(\theta + 2\pi n)}
  \end{align*}
  Therefore, complex exponentiation is single-valued if $\alpha\in \R$. If $\alpha\in \Z$, then $z^{\alpha}$ has only one value; if $\alpha\in \Q$, where $\alpha = \frac{p}{q}$ and $\gcd(p,q) = 1$, then $z^{\alpha}$ takes $q$ distinct values, which are the $q$th-roots.
  \section{Continuous Functions with Complex Domains}%
  Let $z\in \C$, let $r > 0$.
  \begin{itemize}
    \item The set $D(z;r) := \{w\mid w\in\C,|z-w| < r\}$ is the $r$-neighborhood of $z$.
    \item A subset $A\subseteq \C$ is open if $\left(\forall z\in A\right)\left(\exists r > 0\right) \ni D(z;r)\subseteq A$.
  \end{itemize}
  For example, if $A = \{z\mid \text{Re}(z) > 0\}$, we can find $r$ equal to half the magnitude of the real component of $z$ for any $z\in A$, meaning $A$ is open.\\

  Meanwhile, if $A = \{z\mid \text{Re}(z) \geq 0\}$, this is not the case. If $z = 0$, then $\nexists r > 0$ such that $D(z;r)\subseteq A$, as any open ball of radius $r$ will have some element in $\overline{A}$.
  \begin{itemize}
    \item A subset $B\subseteq \C$ is closed if $\overline{B}\subseteq \C$ is open.
  \end{itemize}
  For example, $A = \emptyset$ is open, by vacuous truth, so $\overline{A} = \C$ is closed. Similarly, since $\C$ is open, $\emptyset$ is closed.\\

  Meanwhile, $A = \{x + iy\mid -1\leq x < 1\}$ is neither open nor closed.
\end{document}
