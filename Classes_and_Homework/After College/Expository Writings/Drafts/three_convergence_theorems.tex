\documentclass[10pt]{mypackage}

% sans serif font:
%\usepackage{cmbright,sfmath,bbold}
%\renewcommand{\mathcal}{\mathtt}

%Euler:
%\usepackage{newpxtext,eulerpx,eucal,eufrak}
%\renewcommand*{\mathbb}[1]{\varmathbb{#1}}
%\renewcommand*{\hbar}{\hslash}

%\renewcommand{\mathbb}{\mathds}
\usepackage{homework}
\usepackage{exposition}

\pagestyle{fancy} %better headers
\fancyhf{}
\rhead{Avinash Iyer}
\lhead{Three Convergence Theorems}

\setcounter{secnumdepth}{0}

\begin{document}
\RaggedRight
\begin{abstract}
  \noindent We discuss and prove the three big theorems of real analysis --- the Monotone Convergence Theorem, Fatou's Lemma, and the Dominated Convergence Theorem.
\end{abstract}
\section{Integration: An Introduction}%
In order to discuss integration, we need to start with the building blocks of all functions --- simple functions.
\begin{definition}
  Let $X$ be a measure space, and let $\phi\colon X\rightarrow [0,\infty]$ be a function. We say $\phi$ is a \textit{simple function} if it has finite range (and does not take the value $+\infty$).\newline

  The \textit{standard form} of a simple function $\phi$ is
  \begin{align*}
    \phi &= \sum_{k=1}^{n} c_k\1_{E_k},
  \end{align*}
  where $\set{c_1,\dots,c_n} = \Ran\left( \phi \right)$, and $E_k = \phi^{-1}\left( \set{c_k} \right)$.
\end{definition}
Recall that a function $f\colon X\rightarrow \R$, where $\left( X,\mathcal{M},\mu \right)$ is a measure space, is called Borel-measurable (or just measurable) if, for every $E\in \mathcal{B}_{\R}$, $f^{-1}\left( E \right)\in \mathcal{M}$.
\begin{definition}
  If $\phi\colon X\rightarrow [0,\infty]$ is a simple, measurable function defined on a measure space $\left( X,\mathcal{M},\mu \right)$, then the \textit{integral} of $\phi$ is defined to be
  \begin{align*}
    \int_{X}^{} \phi\:d\mu &= \sum_{k=1}^{n}c_k\mu\left( E_k \right).
  \end{align*}
\end{definition}
\begin{proposition}
  Let $\phi,\psi\colon X\rightarrow [0,\infty]$ be simple functions with standard forms
  \begin{align*}
    \phi &= \sum_{j=1}^{n}a_j\1_{E_j}\\
    \psi &= \sum_{k=1}^{m} b_k\1_{F_k}.
  \end{align*}
  Then, the following hold
  \begin{enumerate}[(a)]
    \item for all $c> 0$, $\displaystyle \int_{X}^{} c\phi\:d\mu = c \int_{X}^{} \phi\:d\mu$;
    \item $\displaystyle \int_{X}^{} \phi + \psi\:d\mu = \int_{X}^{} \phi\:d\mu + \int_{X}^{} \psi\:d\mu$;
    \item if $\phi\leq \psi$ pointwise, then $\displaystyle \int_{X}^{} \phi\:d\mu \leq \int_{X}^{} \psi\:d\mu$.
  \end{enumerate}
\end{proposition}

\end{document}
