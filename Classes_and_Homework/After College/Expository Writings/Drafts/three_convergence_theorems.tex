\documentclass[12pt]{mypackage}

% sans serif font:
%\usepackage{cmbright,sfmath,bbold}
%\renewcommand{\mathcal}{\mathtt}

%Euler:
%\usepackage{newpxtext,eulerpx,eucal,eufrak}
%\renewcommand*{\mathbb}[1]{\varmathbb{#1}}
%\renewcommand*{\hbar}{\hslash}

%\renewcommand{\mathbb}{\mathds}
\usepackage{homework}
\usepackage{exposition}

\pagestyle{fancy} %better headers
\fancyhf{}
\rhead{Avinash Iyer}
\lhead{Three Convergence Theorems}

\setcounter{secnumdepth}{0}

\begin{document}
\RaggedRight
\begin{abstract}
  \noindent We discuss and prove the three big theorems of real analysis --- the Monotone Convergence Theorem, Fatou's Lemma, and the Dominated Convergence Theorem.
\end{abstract}
\section{Integration: An Introduction}%
In order to discuss integration, we need to start with the building blocks of all functions --- simple functions.
\begin{definition}
  Let $X$ be a measure space, and let $\phi\colon X\rightarrow [0,\infty]$ be a function. We say $\phi$ is a \textit{simple function} if it has finite range (and does not take the value $+\infty$).\newline

  The \textit{standard form} of a simple function $\phi$ is
  \begin{align*}
    \phi &= \sum_{k=1}^{n} c_k\1_{E_k},
  \end{align*}
  where $\set{c_1,\dots,c_n} = \Ran\left( \phi \right)$, and $E_k = \phi^{-1}\left( \set{c_k} \right)$.
\end{definition}
Recall that a function $f\colon X\rightarrow \R$, where $\left( X,\mathcal{M},\mu \right)$ is a measure space, is called Borel-measurable (or just measurable) if, for every $E\in \mathcal{B}_{\R}$, $f^{-1}\left( E \right)\in \mathcal{M}$.
\begin{definition}
  If $\phi\colon X\rightarrow [0,\infty]$ is a simple, measurable function defined on a measure space $\left( X,\mathcal{M},\mu \right)$, then the \textit{integral} of $\phi$ is defined to be
  \begin{align*}
    \int_{X}^{} \phi\:d\mu &= \sum_{k=1}^{n}c_k\mu\left( E_k \right).\label{eq:integral_simple_function}\tag{$\dag$}
  \end{align*}
\end{definition}
\begin{proposition}
  Let $\phi,\psi\colon X\rightarrow [0,\infty]$ be simple functions with standard forms
  \begin{align*}
    \phi &= \sum_{j=1}^{n}a_j\1_{E_j}\\
    \psi &= \sum_{k=1}^{m} b_k\1_{F_k}.
  \end{align*}
  Then, the following hold
  \begin{enumerate}[(a)]
    \item for all $c> 0$, $\displaystyle \int_{X}^{} c\phi\:d\mu = c \int_{X}^{} \phi\:d\mu$;
    \item $\displaystyle \int_{X}^{} \phi + \psi\:d\mu = \int_{X}^{} \phi\:d\mu + \int_{X}^{} \psi\:d\mu$;
    \item if $\phi\leq \psi$ pointwise, then $\displaystyle \int_{X}^{} \phi\:d\mu \leq \int_{X}^{} \psi\:d\mu$.
  \end{enumerate}
\end{proposition}
\begin{proof}\hfill
  \begin{enumerate}[(a)]
    \item We see that
      \begin{align*}
        \int_{X}^{} c\phi\:d\mu &= \sum_{j=1}^{n}(c)\left( a_j \right)\mu\left( E_k \right)\\
                                &= c\sum_{k=1}^{n}a_j\mu\left( E_k \right)\\
                                &= c \int_{X}^{} \phi\:d\mu.
      \end{align*}
    \item Note that since
      \begin{align*}
        X &= \bigsqcup_{j=1}^{n}E_j\\
          &= \bigsqcup_{k=1}^{m}F_k,
      \end{align*}
      we must have
      \begin{align*}
        E_j &= \bigsqcup_{k=1}^{m}E_j\cap F_k\\
        F_k &= \bigsqcup_{j=1}^{m}F_k\cap E_j
      \end{align*}
      as a disjoint union. Therefore,
      \begin{align*}
        \int_{X}^{} \phi\:d\mu + \int_{X}^{} \psi\:d\mu &= \sum_{j=1}^{n}\sum_{k=1}^{m}\left( a_j + b_k \right)\mu\left( E_j\cap F_k \right)\\
                                                        &= \int_{X}^{} \phi + \psi\:d\mu.
      \end{align*}
    \item If $\phi\leq \psi$, $a_j\leq b_k$ whenever $E_j\cap F_k\neq \emptyset$. Therefore,
      \begin{align*}
        \int_{X}^{} \phi\:d\mu &= \sum_{k=1}^{m}\sum_{j=1}^{n}a_j\mu\left( E_j\cap F_k \right)\\
                               &\leq \sum_{k=1}^{m}\sum_{j=1}^{n}b_k\mu\left( E_j\cap F_k \right)\\
                               &= \int_{X}^{} \psi\:d\mu.
      \end{align*}
  \end{enumerate}
\end{proof}
Having established integrals for simple functions, we need to establish a convergence property for simple functions for all measurable functions.
\begin{theorem}
  Let $\left( X,\mathcal{M},\mu \right)$ be a measure space, and let $f\colon X\rightarrow [0,\infty]$ be a measurable function. Then, there is an increasing sequence $\left( \phi_n \right)_n$ of simple functions that converges pointwise to $f$. This sequence converges uniformly to $f$ on any bounded sets.
\end{theorem}
\begin{proof}
  For each $n$, partition the interval $\left[ 0,2^n \right]$ into subintervals of length $2^{-n}$. There are $2^{2n}$ subintervals, with
  \begin{align*}
    I_{n,0} &= \left[ 0,\frac{1}{2^{n}} \right]\\
    I_{n,k} &= \left(\frac{k}{2^{n}},\frac{k + 1}{2^{n}}\right],
  \end{align*}
  where $0 \leq k \leq 2^{2n}-1$. We define $J_n = \left( 2^{n},\infty \right]$. Define
  \begin{align*}
    E_{n,k} &= f^{-1}\left( I_{n,k} \right)\\
    F_n &= f^{-1}\left( J_n \right).
  \end{align*}
  Then, we may take
  \begin{align*}
    \phi_n &= \sum_{k=0}^{2^{2n}-1}\frac{k}{2^{n}}\1_{E_{n,k}} + 2^{n}\1_{F_n}.
  \end{align*}
  The family $\phi_n$ are simple, measurable, positive, and increasing.\newline

  Fix $x\in X$ such that $f(x) < \infty$, and find $N$ such that $f(x) \leq 2^{N}$. Then, for a fixed $n\geq N$, there is $0 \leq k \leq 2^{2n}-1$ such that $x\in E_{n,k}$. Thus,
  \begin{align*}
    \left\vert \phi_n(x) - f(x) \right\vert &= \left\vert f(x) - \frac{k}{2^{n}} \right\vert\label{eq:convergence_of_simple_functions}\tag{$\ast$}\\
                                            &\leq \frac{1}{2^{n}}.
  \end{align*}
  Thus, this family is pointwise convergent.\newline

  If $f(x) = +\infty$, then $\phi_n(x) = 2^{n}$ for all $n$, meaning $\phi_n(x)$ also converges to $f(x)$.\newline

  If $f(x)$ is bounded, then for a sufficiently large $n$, $F_n = \emptyset$, and the construction in \eqref{eq:convergence_of_simple_functions} is valid for all $x\in X$, meaning $\norm{\phi_n - f}_{u} \leq \frac{1}{2^{n}}$, and $\sup_{n}\norm{\phi_n}_{u} \leq \norm{f}_{u}$.
\end{proof}
\begin{remark}
  By decomposing any complex-valued function $f$ using the Cartesian decomposition to yield $f = \left( f_{+} - f_{-} \right) + i\left( g_{+} - g_{-} \right)$, the above theorem can be extended to all complex-valued functions. There, the modulus of the simple functions, $\left( \left\vert \phi_n \right\vert \right)_n$ can be taken to be pointwise increasing and bounded above by $\left\vert f \right\vert$, with uniform convergence on sets where $f$ is bounded in modulus.
\end{remark}
\section{The Monotone Convergence Theorem}%
Since any measurable function $f\colon X\rightarrow [0,\infty]$ is a pointwise limit of simple functions, we may define the integral of a function as follows.
\begin{definition}
  Let $\left( X,\mathcal{M},\mu \right)$ be a measure space, and let $f\colon X\rightarrow [0,\infty]$ be a measurable function. The \textit{integral} of $f$ is defined to be
  \begin{align*}
    \int_{X}^{} f\:d\mu &= \sup\set{ \int_{X}^{} \phi_n\:d\mu | \phi\text{ simple},0\leq \phi \leq f}.
  \end{align*}
\end{definition}
This definition of the integral agrees with the definition in \eqref{eq:integral_simple_function} whenever $f$ is simple. Furthermore, it follows that, for all $c\in [0,\infty)$,
\begin{align*}
  \int_{X}^{} cf\:d\mu &= c \int_{X}^{} f\:d\mu,
\end{align*}
and whenever $f\leq g$,
\begin{align*}
  \int_{X}^{} f\:d\mu &\leq \int_{X}^{} g\:d\mu.
\end{align*}
Yet, the issue is that our family of simple functions is uncountable. In order to (more easily) establish this integral, we need to be able to extract a sequence.
\begin{theorem}[Monotone Convergence Theorem]
  Let $\left( f_n \right)_n$ be a family of $[0,\infty]$-valued measurable functions on $X$ such that $f_j\leq f_{j+1}$ for all $j$. Define 
  \begin{align*}
    f &= \lim_{n\rightarrow\infty}f_n\\
      &= \sup_{n\in \N} f_n.
  \end{align*}
  Then,
  \begin{align*}
    \int_{X}^{} f\:d\mu &= \lim_{n\rightarrow\infty} \int_{X}^{} f_n\:d\mu.
  \end{align*}
\end{theorem}
\begin{proof}
  The sequence $\left( \int_{X}f_n\:d\mu \right)$ is an increasing sequence of real numbers, so it has a limit (which may be equal to $\infty$). Furthermore, $\int_{X}f_n\:d\mu \leq \int_{X}f\:d\mu$ for all $n$, meaning $\sup \left( \int_{X}^{} f_n\:d\mu \right) \leq \int_{X}^{} f\:d\mu$.\newline

  To establish the reverse inequality, let $\alpha\in (0,1)$, $0\leq \phi \leq f$ a simple function, and let
  \begin{align*}
    E_n &= \set{x | f_n(x) \geq \alpha \phi(x)}.
  \end{align*}
  The family $\set{E_n}_{n\in\N}$ is an increasing sequence of measurable sets whose union is $X$.\footnote{To see that their union is equal to $X$, recall that $f$ is the pointwise limit of $f_n$.} We have
  \begin{align*}
    \int_{X}^{} f_n\:d\mu &\geq \int_{E_n}^{} f_n\:d\mu\\
                          &\geq \alpha \int_{E_n}^{} \phi\:d\mu.
  \end{align*}
  Since
  \begin{align*}
    \lim_{n\rightarrow\infty} \int_{E_n}^{} \phi\:d\mu &= \int_{X}^{} \phi\:d\mu,
  \end{align*}
  we have
  \begin{align*}
    \lim_{n\rightarrow\infty} \int_{X}^{} f_n\:d\mu &\geq \alpha \int_{X}^{} \phi\:d\mu.
  \end{align*}
  We may take the supremum over all $\alpha\in (0,1)$, meaning
  \begin{align*}
    \lim_{n\rightarrow\infty} \int_{X}^{} f_n\:d\mu &\geq \int_{X}^{} \phi\:d\mu.
  \end{align*}
  Taking the supremum over all simple $0\leq \phi \leq f$, we obtain
  \begin{align*}
    \lim_{n\rightarrow\infty} \int_{X}^{} f_n\:d\mu &\geq \int_{X}^{} f\:d\mu.
  \end{align*}
\end{proof}
\section{Fatou's Lemma}%
Going deeper into our quest to find out when (pointwise) convergence of functions implies convergence of their integrals, we establish the ``next best'' option. 
\end{document}
