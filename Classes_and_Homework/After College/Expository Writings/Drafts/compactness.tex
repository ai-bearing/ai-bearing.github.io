\documentclass[10pt]{mypackage}

% sans serif font:
%\usepackage{cmbright,sfmath,bbold}
%\renewcommand{\mathcal}{\mathtt}

%Euler:
%\usepackage{newpxtext,eulerpx,eucal,eufrak}
%\renewcommand*{\mathbb}[1]{\varmathbb{#1}}
%\renewcommand*{\hbar}{\hslash}

%\renewcommand{\mathbb}{\mathds}
\usepackage{homework}
\usepackage{exposition}

\pagestyle{fancy} %better headers
\fancyhf{}
\rhead{Avinash Iyer}
\lhead{Compactness}

\setcounter{secnumdepth}{0}

\begin{document}
\RaggedRight
\begin{abstract}
  \noindent We discuss compactness in topological spaces, normed spaces, and weak compactness, covering results such as Tychonoff's Theorem, relations between norm-compactness and dimension, sequential compactness, the Banach--Alaoglu Theorem, and the Eberlein--\v{S}mulian theorem.
\end{abstract}
\section{Compactness in Topological Spaces}%
Traditionally, one is introduced to compactness in their first class on topology. There, the definition of compactness appears a bit strange --- but we'll see soon enough that there are a variety of simpler, equivalent ways to use compactness that are just as powerful as the original definition. However, as is customary, we start with the standard definition.
\begin{definition}
  Let $X$ be a topological space. An \textit{open cover} of $X$ is a family of open sets $\set{U_i}_{i\in I}$ such that
  \begin{align*}
    X &\subseteq \bigcup_{i\in I}U_i.
  \end{align*}
\end{definition}
\begin{definition}
  Let $X$ be a topological space. We say $X$ is \textit{compact} if, for any open cover of $X$, $\set{U_i}_{i\in I}$, there is a finite $F\subseteq I$ such that
  \begin{align*}
    X &\subseteq \bigcup_{i\in F}U_i.
  \end{align*}
  In other words, $X$ is compact if every open cover admits a finite subcover.
\end{definition}
\subsection{Nets, Filters, and Ultrafilters}%
\subsection{Tychonoff's Theorem}%
\section{Compactness in Normed Spaces and Metric Spaces}%
\subsection{Compactness and Dimension}%
\subsection{Compactness and Sequential Compactness}%
\subsection{Compactness in Continuous Function Spaces}%
\section{Weak Compactness}%
\subsection{The Banach--Alaoglu Theorem}%
\subsection{Goldstine's Theorem}%
\subsection{The Eberlein--\v{S}mulian Theorem}%

\end{document}
