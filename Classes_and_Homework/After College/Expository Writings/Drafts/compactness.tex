\documentclass[10pt]{mypackage}

% sans serif font:
%\usepackage{cmbright,sfmath,bbold}
%\renewcommand{\mathcal}{\mathtt}

%Euler:
%\usepackage{newpxtext,eulerpx,eucal,eufrak}
%\renewcommand*{\mathbb}[1]{\varmathbb{#1}}
%\renewcommand*{\hbar}{\hslash}

%\renewcommand{\mathbb}{\mathds}
\usepackage{homework}
\usepackage{exposition}

\pagestyle{fancy} %better headers
\fancyhf{}
\rhead{Avinash Iyer}
\lhead{Compactness}

\setcounter{secnumdepth}{0}

\begin{document}
\RaggedRight
\begin{abstract}
  \noindent We discuss compactness in topological spaces, normed spaces, and weak compactness, covering results such as Tychonoff's Theorem, relations between norm-compactness and dimension, sequential compactness, the Banach--Alaoglu Theorem, and the Eberlein--\v{S}mulian theorem.
\end{abstract}
\section{Compactness in Topological Spaces}%
Traditionally, one is introduced to compactness in their first class on topology. There, the definition of compactness appears a bit strange --- but we'll see soon enough that there are a variety of simpler, equivalent ways to use compactness that are just as powerful as the original definition. However, as is customary, we start with the standard definition.
\begin{definition}
  Let $X$ be a topological space. An \textit{open cover} of $X$ is a family of open sets $\set{U_i}_{i\in I}$ such that
  \begin{align*}
    X &\subseteq \bigcup_{i\in I}U_i.
  \end{align*}
\end{definition}
\begin{definition}
  Let $X$ be a topological space. We say $X$ is \textit{compact} if, for any open cover of $X$, $\set{U_i}_{i\in I}$, there is a finite $F\subseteq I$ such that
  \begin{align*}
    X &\subseteq \bigcup_{i\in F}U_i.
  \end{align*}
  In other words, $X$ is compact if every open cover admits a finite subcover.
\end{definition}
One of the early equivalent characterizations we encounter with compactness relates to what happens when we take complements of the definition.
\begin{theorem}
  Let $X$ be a topological space. The following are equivalent:
  \begin{itemize}
    \item $X$ is compact;
    \item for every family of closed sets $\set{C_i}_{i\in I}$ in $X$ with the finite intersection property --- that is, the intersection of finitely many such $C_i$ is nonempty --- then 
      \begin{align*}
        \bigcap_{i\in I}C_i\neq \emptyset.
      \end{align*}
  \end{itemize}
\end{theorem}
\begin{proof}
  Let $X$ be a compact topological space, and let $\set{C_i}_{i\in I}$ be a family with the finite intersection property. Defining $U_i = C_i^{c}$, it is then the case that any finite union of $U_i$ does not fully cover $X$. Since any open cover of $X$ must admit a finite subcover, and the union of no finite subcollection of $\set{U_i}_{i\in I}$ covers $X$, it is then the case that
  \begin{align*}
    \emptyset &\neq \left( \bigcup_{i\in I}U_i \right)^{c}\\
              &= \bigcap_{i\in I}C_i.
  \end{align*}
  Now, if $X$ is not compact, there is an open cover $\set{U_i}_{i\in I}$ of $X$ with no finite subcover, or that for any finite subcollection $\set{U_i}_{i\in F}$, their union ``misses'' at least one element of $X$. Defining $C_i = U_i^{c}$, this means that the family $\set{C_i}_{i\in I}$ has the finite intersection property, but since $\bigcup_{i\in I}U_i = X$, the intersection $\bigcap_{i\in I}C_i = \emptyset$.
\end{proof}
The finite intersection property characterization is immensely useful in an analytical context --- for instance, it allows us to establish the existence of choice functions.
\begin{exercise}
  Show that closed subsets of compact sets are compact.
\end{exercise}
\subsection{Compactness through Nets}%
Here, we give a crash course in nets before discussing the first 
\begin{definition}
  Let $X$ be a topological space. An \textit{open neighborhood} of $x$ is a subset $U\in \tau$ such that $x\in U$. The family of open neighborhoods of $x$ is denoted $\mathcal{O}_{x}$.\newline

  A \textit{neighborhood} of $x\in X$ is a subset $N\subseteq X$ such such that there exists $U\in \mathcal{O}_{x}$ such that $U\subseteq N$. The family of open neighborhoods of $x$ is denoted $\mathcal{N}_{x}$, and is known as the \textit{neighborhood system} at $x$.
\end{definition}
\begin{definition}
  Let $A$ be a set, and let $\preceq$ be a relation on $A$. We say $\left( A,\preceq \right)$ is \textit{directed} if, for any $a,b\in A$, there exists $c\in A$ such that $a\preceq c$ and $b\preceq c$.
\end{definition}
\begin{definition}
  A \textit{net} is a function from a directed set to a topological space. We write $\left( x_{\alpha} \right)_{\alpha}\subseteq X$ to denote a net.
\end{definition}
\begin{definition}
  A net $\left( x_{\alpha} \right)_{\alpha}$ \textit{converges to} $x\in X$ if, for every $U\in \mathcal{N}_{x}$, there exists $\beta\in A$ such that for all $\alpha \geq \beta$, $x_{\alpha}\in U$. We write $\left( x_{\alpha} \right)_{\alpha}\rightarrow x$.\newline

  An element $x\in X$ is known as a \textit{cluster point} (or \textit{accumulation point}) of $\left( x_{\alpha} \right)_{\alpha}$ if, for all $U\in \mathcal{O}_{x}$ and all $\beta\in A$, there exists $\alpha \in A$ such that $x_{\alpha}\in U$.
\end{definition}
\begin{definition}
  If $A$ and $B$ are directed sets, a map $\phi\colon B\rightarrow A$ is called \textit{cofinal} if, for every $\alpha \in A$, there is $\beta_{\alpha}\in B$ such that $\alpha\preceq \phi\left( \beta \right)$ for all $\beta\in B$ where $\beta_{\alpha}\preceq \beta$.\newline

  If $\left( x_{\alpha} \right)_{\alpha}$ and $\left( y_{\beta} \right)_{\beta}$ are nets in $X$, then we say $\left( y_{\beta} \right)_{\beta}$ is a subnet of $\left( x_{\alpha} \right)_{\alpha}$ if $y_{\beta} = x_{\phi\left( \beta \right)}$ for some cofinal map $\phi\colon B\rightarrow A$. We write $\left( x_{\alpha_{\beta}} \right)_{\beta}$.
\end{definition}
There are a variety of characterizations of topological qualities that can be rephrased in terms of nets. We state them here and leave their proofs as exercises.
\begin{theorem}
  Let $X$ and $Y$ be topological spaces, let $f\colon X\rightarrow Y$ be a map, and let $A\subseteq X$ be some subset.
  \begin{itemize}
    \item We have $x\in \overline{A}$ if and only if there is a net $\left( x_{\alpha} \right)_{\alpha}\subseteq A$ such that $\left( x_{\alpha} \right)_{\alpha}\rightarrow x$.
    \item A point $x\in X$ is a cluster point for $\left( x_{\alpha} \right)_{\alpha}$ if and only if $\left( x_{\alpha} \right)_{\alpha}$ admits a subnet, $\left( x_{\alpha_{\beta}} \right)_{\beta}\rightarrow x$.
    \item The map $f\colon X\rightarrow Y$ is continuous if and only if, for all nets $\left( x_{\alpha} \right)_{\alpha}\rightarrow x$ in $X$, we have $\left( f\left( x_{\alpha} \right) \right)_{\alpha}\rightarrow f\left( x \right)$.
    \item A topological space $X$ is Hausdorff if and only if any net converges to at most one point.
  \end{itemize}
\end{theorem}
The one we will focus on today is, of course, compactness.
\begin{theorem}
  Let $X$ be a topological space. Then, $X$ is compact if and only if every net $\left( x_{\alpha} \right)_{\alpha}$ admits a cluster point.
\end{theorem}
\begin{proof}
  Let $X$ be compact, and set $V_{\alpha} = \set{x_{\beta} | \beta\geq \alpha}$. Then, the family $ \set{\overline{V_{\alpha}}}_{\alpha \in A} $ satisfies the finite intersection property --- for any $\alpha_1,\dots,\alpha_n\in A$, there is some $\alpha\in A$ such that $\alpha_i\preceq \alpha$ for all $i$, meaning that $x_{\alpha}\in \bigcap_{i=1}^{n} \overline{V_{\alpha_i}}$. Therefore, there is some $x\in X$ such that $x\in \bigcap_{\alpha\in A} \overline{V_{\alpha}}$. Let $U\in \mathcal{O}_{x}$, and let $\alpha \in A$. Then, since $x\in \overline{V_{\alpha}}$, there is some $\beta \geq \alpha$ such that $x_{\beta}\in U$. Therefore, $x$ is a cluster point of $\left( x_{\alpha} \right)_{\alpha}$.\newline

  Now, let $X$ be noncompact, and let $\set{U_{i}}_{i\in I}$ be an open cover of $X$ that does not admit any finite subcover. Let $\mathcal{F}$ be a family of finite subsets of $I$ directed by inclusion. For each $F\in \mathcal{F}$, there is some $x_{F}\in X\setminus \bigcup_{i\in F}U_i$ by assumption. The net $\left( x_F \right)_{F\in \mathcal{F}}$ does not admit any accumulation point, since for any $i_0\in I$, and for all $S$ such that $\set{i_0}\subseteq S$, we have defined $x_{S}$ such that $x_S\notin U_{i_0}$.
\end{proof}
\subsection{Compactness through Filters and Ultrafilters}%
Nets are nice and all, but there is another way to discuss convergence (that is actually in one-to-one correspondence with nets): filters.
\begin{definition}
  Let $X$ be a set, and let $\mathcal{F}$ be a family of subsets. We say $\mathcal{F}$ is a \textit{filter} if
  \begin{enumerate}[(a)]
    \item $\emptyset\notin \mathcal{F}$ and $X\in \mathcal{F}$;
    \item if $A,B\in \mathcal{F}$, $A\cap B\in \mathcal{F}$;
    \item if $A\subseteq B$ and $A\in \mathcal{F}$, then $B\in \mathcal{F}$.
  \end{enumerate}
  We say the filter $\mathcal{F}$ is \textit{free} if $\bigcap_{A\in \mathcal{F}}A = \emptyset$. If $\mathcal{F}$ is not free, we say $\mathcal{F}$ is \textit{principal}.\newline

  If $\mathcal{F}$ and $\mathcal{G}$ are two filters, we say $\mathcal{G}$ is a \textit{subfilter} of $\mathcal{F}$ if $\mathcal{G}\supseteq \mathcal{F}$ --- i.e., that $\mathcal{G}$ is a ``finer'' filter than $\mathcal{F}$.
\end{definition}
\begin{example}\hfill
  \begin{itemize}
    \item If $x$ is a set, then the neighborhood system at $x$, $\mathcal{N}_x$, is a filter.
    \item If $S\subseteq X$ is some nonempty set, the family $\mathcal{F} = \set{A\subseteq X | S\subseteq A}$ is a filter.
    \item If $X$ is an infinite set, the family of cofinite sets --- i.e., $A\in \mathcal{F}$ if $A^{c}$ is finite --- is a free filter.
  \end{itemize}
\end{example}
\begin{definition}
  An \textit{ultrafilter} is a maximal proper filter. Equivalently, a filter $\mathcal{U}$ is an ultrafilter if $\mathcal{U}$ is a filter and, for all $A\subseteq X$, either $A\in \mathcal{U}$ or $A^{c}\in \mathcal{U}$.
\end{definition}
  If there is a finite union $A_1\cup\cdots\cup A_n\in \mathcal{U}$ for an ultrafilter $\mathcal{U}$, then at least one of the $A_i$ is an element of $\mathcal{U}$. This can be proven by induction and using the maximality of the ultrafilter.
\begin{theorem}[Ultrafilter Lemma]
  If $\mathcal{F}$ is a filter, then there is an ultrafilter $\mathcal{U}$ that contains $\mathcal{F}$.
\end{theorem}
\begin{proof}
  Consider the partially ordered set of all (proper) filters on $X$ that contain $\mathcal{F}$. This family is partially ordered by inclusion, and for any chain $\mathcal{C}$ of this family, the family $\set{A | A\in \mathcal{G}\text{ for some }\mathcal{G}\in \mathcal{C}}$ is a filter that serves as an upper bound for $\mathcal{C}$. By Zorn's Lemma, this family admits a maximal element.
\end{proof}
\begin{definition}
  If $\mathcal{B}$ is a nonempty collection of subsets of $X$, we say $\mathcal{B}$ is a \textit{filter base} if
  \begin{enumerate}[(a)]
    \item $\emptyset\notin \mathcal{B}$;
    \item if $A,B\in \mathcal{B}$, there is some $C\in \mathcal{B}$ such that $C\subseteq A\cap B$ --- i.e., $\mathcal{B}$ is directed by containment.
  \end{enumerate}
  The collection of sets
  \begin{align*}
    \mathcal{F}_{\mathcal{B}} &= \set{A\subseteq X | B\subseteq A\text{ for some }B\in \mathcal{B}}
  \end{align*}
  is a filter known as the filter \textit{generated by} $\mathcal{B}$.
\end{definition}
\begin{example}
  If $X$ is a topological space, and $x\in X$, the family $\mathcal{O}_{x}$ is a filter base that generates the neighborhood system, $\mathcal{N}_{x}$.
\end{example}
\begin{proposition}
  Any collection $\set{S_i}_{i\in I}$ with the finite intersection property is a filter base.
\end{proposition}
Similar to the case of nets, there are definitions of convergence and cluster points.
\begin{definition}
  Let $\mathcal{F}$ be a filter on a topological space $X$, and let $x\in X$.
  \begin{itemize}
    \item We say $\mathcal{F}$ \textit{converges to} $x$ if $\mathcal{N}_{x}\subseteq \mathcal{F}$. In other words, if $\mathcal{F}$ is a subfilter of the neighborhood system at $x$, then we say the filter converges to $x$.
    \item We say $x$ is a \textit{cluster point} of $\mathcal{F}$ if, for all $F\in \mathcal{F}$, $x\in \overline{F}$. Equivalently, $x$ is a cluster point for $\mathcal{F}$ if, for all $U\in \mathcal{O}_{x}$ and for all $F\in \mathcal{F}$, $U\cap F \neq \emptyset$.
  \end{itemize}
\end{definition}
\begin{theorem}
  If $\mathcal{F}$ is a filter in a topological space $X$, an element $x\in X$ is a cluster point of $\mathcal{F}$ if and only if $\mathcal{F}$ admits a subfilter, $\mathcal{G}$, such that $\mathcal{G}$ converges to $x$.
\end{theorem}
One of the most important facts about filters is that every filter has a corresponding net, and vice versa.
\begin{definition}
  Let $\left( x_{\alpha} \right)_{\alpha}$ be a net in the topological space $X$.
  \begin{itemize}
    \item For each $\alpha$, the \textit{tail} of $\left( x_{\alpha} \right)_{\alpha}$ is $F_{\alpha}\coloneq \set{x_{\beta} | \alpha\preceq \beta}$.
    \item The family $\mathcal{B} = \set{F_{\alpha} | \alpha\in A}$ serves as a filter base for the \textit{section filter} of $\left( x_{\alpha} \right)_{\alpha}$
  \end{itemize}
\end{definition}
\begin{theorem}
  The net $\left( x_{\alpha} \right)_{\alpha}$ and its corresponding section filter $\mathcal{F}$ have the same cluster points.
\end{theorem}
\begin{proof}
  If $x$ is a cluster point of $\left( x_{\alpha} \right)_{\alpha}$, then there is a subnet $\left( x_{\alpha_{\beta}} \right)_{\beta}\rightarrow x$. By cofinality, the section filter $\mathcal{G}$ generated by $\left( x_{\alpha_{\beta}} \right)_{\beta}$ is a subfilter of $\mathcal{F}$ with $\mathcal{G}$ converging to $x$.\newline

  Conversely, if $x$ is a cluster point of $\mathcal{F}$, then for each $\alpha$ and for each $V\in \mathcal{N}_x$, we have $V\cap F_{\alpha}\neq \emptyset$. We may choose $y_{\alpha,V}\in V\cap F_{\alpha}$; the net $\left( y_{\alpha,V} \right)_{\alpha\in A,V\in \mathcal{N}_x}$ is a subnet of $\left( x_{\alpha} \right)_{\alpha}$ that converges to $x$, meaning $x$ is a cluster point of $\left( x_{\alpha} \right)_{\alpha}$
\end{proof}
\begin{definition}
  If $\mathcal{F}$ is a filter in a topological space $X$, then we define the set $A = \set{\left( \alpha_S ,S \right) | S\in \mathcal{F},\alpha_S\in S}$, with direction $\left( \alpha_S,S \right)\succeq \left( \alpha_T,T \right)$ if $S\subseteq T$. The family $\left( x_{\left( \alpha_S,S \right)} \right)_{\left( \alpha_S,S \right)\in A}$, where $x_{\left( \alpha_S,S \right)} = \alpha_S$, is a net in $X$ known as the \textit{net generated by the filter} $\mathcal{F}$.
\end{definition}
  Note that the net generated by the section filter of a net $\left( x_{\alpha} \right)_{\alpha}$ is $\left( x_{\alpha} \right)_{\alpha}$.\newline

  Now, we turn towards characterizations of compactness using filters and ultrafilters.
  \begin{theorem}
    Let $X$ be a topological space. Then, $X$ is compact if and only if every filter in $X$ has a cluster point.
  \end{theorem}
  \begin{proof}
    Let $X$ be compact, and let $\mathcal{F}$ be a filter on $X$. Define $\mathcal{S} = \set{ \overline{A} | A\in \mathcal{F} }$. Since all filters are filter bases, the family $ \mathcal{S} $ has the finite intersection property. Since $X$ is compact, $\bigcap_{ \overline{A}\in \mathcal{S} } \overline{A} = \bigcap_{A\in \mathcal{F}} \overline{A} \neq \emptyset$, meaning $\mathcal{F}$ admits a cluster point.\newline

    Suppose every filter on $X$ admits a cluster point. If $\mathcal{C}$ is a family of closed subsets of $X$ with the finite intersection property, then $\mathcal{C}$ is a filter base for some filter $\mathcal{F}$, which has a cluster point. Thus, $\bigcap_{C\in \mathcal{C}} C = \bigcap_{A\in \mathcal{F}} \overline{A} \neq \emptyset$, so $X$ is compact.
  \end{proof}
  \begin{theorem}
    Let $X$ be a topological space. Then, $X$ is compact if and only if every ultrafilter in $X$ converges to at least one point.
  \end{theorem}
  \begin{proof}
    We know from the previous theorem that $X$ is compact if and only if every filter on $X$ admits a cluster point, meaning that every filter $\mathcal{F}$ on $X$ has a convergent subfilter $\mathcal{G}\rightarrow x$. Using the Ultrafilter Lemma, we may expand $\mathcal{G}$ to an ultrafilter $\mathcal{U}\rightarrow x$.\newline

    Now, suppose toward contradiction that $X$ is compact and there is an ultrafilter $\mathcal{U}$ on $X$ that does not converge to any point. Then, by definition, for each $x$, there is some $U_x\in \mathcal{O}_x$ such that $U_x\notin \mathcal{U}$. The family $\set{U_x}_{x\in X}$ is an open cover of $X$, which admits the finite subcover $X = \bigcup_{i=1}^{n}U_{x_i}$. However, since $\mathcal{U}$ is an ultrafilter, and $X\in \mathcal{U}$, it must be the case that one of the $U_{x_i}$ is an element of $\mathcal{U}$, which is a contradiction.
  \end{proof}
\subsection{Tychonoff's Theorem}%
In the realm of compactness, there is probably no more powerful a theorem than Tychonoff's theorem. There are many proofs of Tychonoff's theorem (including one using the Stone--\v{C}ech compactification, apparently), but we will provide the proof using filters.\newline

First, we need to review some notions from general topology.
\begin{definition}
  If $\set{Y_i}_{i\in I}$, $X$ is a topological space, and $f_i\colon X \rightarrow Y_i$ is a family of maps, the \textit{initial topology} on $X$ is the coarsest topology on $X$ such that all the $f_i$ are continuous.\newline

  A topological base for the initial topology on $X$ induced by $\set{f_i}_{i\in I}$ is given by
  \begin{align*}
    \mathcal{B} &= \set{f_{i_1}^{-1}\left( U_1 \right)\cap\cdots\cap f_{i_n}^{-1}\left( U_n \right) | U_j\subseteq Y_j\text{ is open}}.
  \end{align*}
\end{definition}
\begin{proposition}
  If $\tau$ is the initial topology on $X_i$ induced by $\set{f_i}_{i\in I}$, then a net $\left( x_{\alpha} \right)_{\alpha}$ converges to $x$ if and only if $\left( f_i\left( x_{\alpha} \right) \right)_{\alpha}\rightarrow f_i\left( x \right)$ for all $i$.
\end{proposition}
\begin{proof}
  Let $\left( x_{\alpha} \right)_{\alpha}\rightarrow x$. Since each $f_i\colon X\rightarrow Y$ is continuous, $\left( f_i\left( x_{\alpha} \right) \right)_{\alpha}\rightarrow f(x)$ for each $i$.\newline

  Let $\left( f_i\left( x_{\alpha} \right) \right)_{\alpha}\rightarrow x$ for all $i$. Then, if $U\in \mathcal{O}_{x}$, the base of the initial topology gives open $U_j\subseteq Y_j$ such that
  \begin{align*}
    x &\in f_{i_1}^{-1}\left( U_1 \right)\cap\cdots\cap f_{i_n}^{-1}\left( U_n \right)\\
      &\subseteq U.
  \end{align*}
  Since $\left( f_{i_j}\left( x_{\alpha} \right) \right)_{\alpha}\rightarrow f_{i_j}\left( x \right)$ for each $j=1,\dots,n$, there are $\alpha_1,\dots,\alpha_n$ such that, for each $j$ and for all $\alpha\succeq \alpha_j$, $f_{i_j}\left( x_{\alpha} \right)\in U_j$.

  Since $A$ is directed, there is some $\alpha_N\succeq \alpha_j$ for each $j$, so for all $\alpha\succeq \alpha_N$,
  \begin{align*}
    x_{\alpha} &\in f_{i_1}^{-1}\left( U_1 \right)\cap\cdots\cap f_{i_n}^{-1}\left( U_n \right)\\
               &\subseteq U,
  \end{align*}
  so $\left( x_{\alpha} \right)_{\alpha}\rightarrow x$.
\end{proof}
\begin{definition}
  If $\set{X_i}_{i\in I}$ is a family of topological spaces, the \textit{product topology} is the initial topology on $\prod_{i\in I}X_i$ such that the family of projection maps, $\pi_j\colon \prod_{i\in I}X_i\rightarrow X_j$, defined by $\left( x_i \right)_{i\in I} \mapsto x_j$, is continuous.\newline

  A base for this topology consists of all sets of the form
  \begin{align*}
    \mathcal{B} &= \set{\prod_{i\in I}U_{i} | U_i\subseteq X_i\text{ is open, }U_i = X_i\text{ for all but finitely many }U_i}
  \end{align*}
\end{definition}
This definition of the product topology, along with the idea of the pushforward of a filter/ultrafilter, will allow us to prove Tychonoff's Theorem.
\begin{definition}
  Let $X,Y$ be sets, and let $f\colon X\rightarrow Y$ be a map. Let $\mathcal{F}$ be a filter on $X$. Then, the collection
  \begin{align*}
    \mathcal{G} &\coloneq \set{V\subseteq Y | f^{-1}\left( V \right) \in \mathcal{F}}
  \end{align*}
  defines a filter on $Y$ known as the \textit{pushforward} of $\mathcal{F}$ with respect to $f$. We write $f_{\ast}\mathcal{F}$.
\end{definition}
\begin{proposition}
  If $\mathcal{F}$ is an ultrafilter on $X$, $f\colon X\rightarrow Y$ is a map, and $f_{\ast}\mathcal{F}$ is the pushforward of $\mathcal{F}$, then $f_{\ast}\mathcal{F}$ is an ultrafilter on $Y$.
\end{proposition}
\begin{proof}
  Let $B\subseteq Y$. Since $\mathcal{F}$ is an ultrafilter, we have either $f^{-1}\left( B \right)\in \mathcal{F}$ or $f^{-1}\left( B \right)^{c}\in \mathcal{F}$, so that either $f^{-1}\left( B \right)\in \mathcal{F}$ or $f^{-1}\left( B^{c} \right)\in \mathcal{F}$. By definition, this means either $B\in f_{\ast}\mathcal{F}$ or $B^{c}\in f_{\ast}\mathcal{F}$, so $f_{\ast}\mathcal{F}$ is an ultrafilter on $Y$.
\end{proof}
\begin{theorem}[Tychonoff's Theorem]
  Let $\set{X_i}_{i\in I}$ be a family of nonempty compact topological spaces. Then, $\prod_{i\in I}X_i$, endowed with the product topology, is compact.
\end{theorem}
\begin{proof}
  Let $\mathcal{U}$ be an ultrafilter on $\prod_{i\in I}X_i$. Pushing forward this ultrafilter to each $X_i$, we have that $\left( \pi_i \right)_{\ast}\mathcal{U}$ converges to $ x_i$ for some $x_i\in X$.\newline

  By the definition of the product topology, $\mathcal{U}\rightarrow \left( x_i \right)_{i}$, so $\mathcal{U}$ converges, meaning $\prod_{i\in I}X_i$ is compact.
\end{proof}
\section{Compactness in Normed Spaces and Metric Spaces}%
Narrowing our focus from topological spaces to metric spaces and normed vector spaces, we are then able to use sequences (rather than nets) to determine convergence, as now we are dealing with first-countable spaces. We assume that the reader is familiar with the basic notions of normed vector spaces and metric spaces, such as completeness, equivalence of metrics, etc.
\subsection{Compactness and Sequential Compactness}%
We will prove that the idea of compactness as expressed via nets (in the case of topological spaces) restricts down to the case of sequences in metric spaces, as metric spaces are first countable.
\begin{definition}
  A topological space $X$ is \textit{sequentially compact} if every sequence $\left( x_n \right)_n\subseteq X$ admits a convergent subsequence.
\end{definition}
\begin{theorem}
  Let $X$ be a metric space. Then, $X$ is compact if and only if $X$ is sequentially compact.
\end{theorem}
\begin{proof}
  Suppose $X$ is compact. Let $\left( x_k \right)_{k}\subseteq X$, and let
  \begin{align*}
    C_0 &= \overline{\set{x_1,x_2,\dots}}\\
    C_1 &= \overline{\set{x_2,x_3,\dots}}\\
        &\vdots\\
    C_n &= \overline{\set{x_{n+1},x_{n+1},\dots}}.
  \end{align*}
  We note that $C_0\supseteq C_1\supseteq \cdots$, and that the family $\set{C_k}_{k\geq 0}$ has the finite intersection property. Since $X$ is compact, there is some $x\in \bigcap_{k\geq 0}C_k$.\newline

  Now, since $x\in C_1$, there is some $k_1 \geq 1$ such that $d\left( x,x_{k_1} \right) < 1$; similarly, since $x\in C_{k_1}$, there is some $k_2 > k_1$ such that $d\left( x,x_{k_2} \right) < 1/2$, and so on constructing $\left( x_{k_n} \right)_{n}$, such that $\left( x_{k_n} \right)_n\rightarrow x$.\newline

  Let $X$ be sequentially compact, and let $\set{C_n}_{n\geq 1}$ be a countable family of closed sets\footnote{We are allowed to consider countable families since $X$ is a first countable space.} with the finite intersection property. Then, defining
  \begin{align*}
    x_1 &\in C_1\\
    x_2 &\in C_1\cap C_2\\
    x_3 &\in C_1\cap C_2\cap C_3\\
        &\vdots\\
    x_n &\in \bigcap_{i=1}^{n}C_i,
  \end{align*}
  we have a sequence in $\left( x_n \right)_n\in X$. Since $X$ is sequentially compact, there is a subsequence $\left( x_{n_k} \right)_{k}\rightarrow x$, so that $x \in \bigcap_{n=1}^{\infty}C_n$, meaning $X$ is compact.
\end{proof}
Sequential compactness allows us to establish a multi-dimensional version of the Bolzano--Weierstrass theorem.
\begin{theorem}
  Let $S = \prod_{i=1}^{d}\left[ a_i,b_i \right]$ be a $d$-dimensional box in $\R^d$. Then, $S$ is compact.
\end{theorem}
\begin{proof}
  Let $\left( x_k \right)_k$ be a sequence in $S$.\newline

  Projecting $\left( \pi_1\left( x_k \right) \right)_k$, there is a convergent subsequence $\left( \pi_1\left( x_{k_j} \right) \right)_j\rightarrow t_1$.\newline

  Since $\left[ a_1,b_1 \right]$ is closed, $t_1\in \left[ a_1,b_1 \right]$.\newline

  Similarly, we may find $t_2,\dots,t_d$ by mapping the subsequences in order $\left( \pi_2\left( x_{k_j} \right) \right)_j\rightarrow t_2$, etc., meaning that $\left( x_k \right)_k$ admits a convergent subsequence in $S$. Thus, $S$ is sequentially compact, hence compact.
\end{proof}
\begin{theorem}[Heine--Borel Theorem]
  Let $S\subseteq \R^d$ be any nonempty subset. Then, $S$ is compact if and only if $S$ is closed and bounded.
\end{theorem}
\begin{proof}
  If $S$ is compact, then for some $x\in S$, the family $\set{U\left( x,n \right)}_{n\geq 1}$ is an open cover of $S$, which admits a finite subcover, meaning that $S$ is bounded. Similarly, since $S$ is compact, $S$ is sequentially compact, meaning that all convergent sequences in $S$ converge in $S$, so $S$ is closed.\newline

  If $S$ is closed and bounded, then there is some box $\prod_{i=1}^{n}\left[ a_i,b_i \right]$ such that $S\subseteq \prod_{i=1}^{n}\left[ a_i,b_i \right]$. We have established that $\prod_{i=1}^{n}\left[ a_i,b_i \right]$ is compact; since $S$ is a closed subset of a compact set, $S$ is compact.
\end{proof}
\subsection{Compactness and Dimension}%
Finite-dimensional normed spaces have a special interaction with compactness.
\begin{definition}
  Let $X$ be a vector space, and let $\norm{\cdot}$ and $\norm{\cdot}'$ be norms on $X$. We say the norms $\norm{\cdot}$ and $\norm{\cdot}'$ are \textit{equivalent} if there exist constants $C_1$ and $C_2$ such that
  \begin{align*}
    \norm{x}&\leq C_1\norm{x}'\\
    \norm{x}' &\leq C_2\norm{x}.
  \end{align*}
\end{definition}
\begin{definition}
  Let $T\colon X\rightarrow Y$ be a linear map between normed spaces. We say $T$ is \textit{bicontinuous} if there exist nonzero constants $C_1$ and $C_2$ such that
  \begin{align*}
    C_1\norm{x}\leq \norm{T(x)}\leq C_2\norm{x}.
  \end{align*}
\end{definition}
\begin{definition}
  If $1 \leq p < \infty$, the space $\ell_{p}^{n}$ consists of $\C^n$ with norm
  \begin{align*}
    \norm{x} &= \left( \sum_{i=1}^{n}\left\vert x_i \right\vert^{p} \right)^{1/p}.
  \end{align*}
  If $ p = \infty $, then
  \begin{align*}
    \norm{x} &= \max_{i=1}^{n}\left\vert x_i \right\vert.
  \end{align*}
\end{definition}
\begin{exercise}
  Show that the $p$-norms on $\ell_{p}^{n}$ for $1 \leq p \leq \infty$ are equivalent.
\end{exercise}
\begin{proposition}
  Let $X$ and $Y$ be normed vector spaces. Then, $T$ is bicontinuous if and only if $T$ is injective and a homeomorphism onto its range.
\end{proposition}
\begin{proof}
  Since $T$ is bounded below, $\ker\left( T \right) = \set{0}$, so that $T$ is injective. Furthermore, since $T$ is bounded, $T$ is continuous.\newline

  Let $S\colon \Ran\left( T \right)\rightarrow X$ be defined by $S\left( T(x) \right) = x$. Then, $S$ is well-defined and linear, with
  \begin{align*}
    \norm{S(T(x))} &= \norm{x}\\
                   &\leq \frac{1}{C_2}\norm{T(x)},
  \end{align*}
  so $S$ is continuous.\newline

  Now, if $S\colon \Ran\left( T \right)\rightarrow X$ is as above, we have
  \begin{align*}
    \norm{T(x)} &\leq \norm{T}_{\op}\norm{x}
  \end{align*}
  Since $S$ is continuous, we have
  \begin{align*}
    \norm{x} &= \norm{S(T(x))}\\
             &\leq \norm{S}_{\op}\norm{T(x)},
  \end{align*}
  so
  \begin{align*}
    \frac{1}{\norm{S}_{\op}} \norm{x} &\leq \norm{T(x)}.
  \end{align*}
\end{proof}
\begin{corollary}
  Two norms on $X$ are equivalent if and only if the identity map is a bicontinuous isomorphism.
\end{corollary}
\begin{theorem}
  Let $X$ be a normed vector space with $\Dim\left( X \right) = n < \infty$. Then, $X\cong \ell_{\infty}^{n} \cong \ell_{2}^{n}$ are bicontinuously isomorphic.
\end{theorem}
\begin{proof}
\end{proof}

\subsection{Compactness in Continuous Function Spaces}%
\section{Weak Compactness}%
\subsection{The Banach--Alaoglu Theorem}%
\subsection{Goldstine's Theorem}%
\subsection{The Eberlein--\v{S}mulian Theorem}%
\end{document}
