\documentclass[10pt]{mypackage}

% sans serif font:
%\usepackage{cmbright}
%\usepackage{sfmath}
%\usepackage{bbold} %better blackboard bold

%\usepackage{homework}
\usepackage{notes}
%\usepackage{newpxtext,eulerpx,eucal}
%\renewcommand*{\mathbb}[1]{\varmathbb{#1}}

\fancyhf{}
\fancyhead[R]{Avinash Iyer}
\fancyhead[L]{Regular Values}
\fancyfoot[C]{\thepage}

\setcounter{secnumdepth}{0}

\begin{document}
\RaggedRight
\begin{abstract}
  \noindent We discuss the much celebrated Regular Value Theorem and Sard's Theorem, and discuss some of the consequences of these results.
\end{abstract}
A smooth map between manifolds $f\colon M\rightarrow N$ includes a certain family of local information; for instance, the derivative $D_pf\colon T_pM\rightarrow T_{f(p)}N$, which is a linear map between tangent spaces at $p$ and $q$, is defined on a coordinate chart $U\subseteq M$ for $p$ and a corresponding coordinate chart $V\subseteq N$ for $f(p)$. Yet, the properties of this linear map can give us information about the underlying map $f$.\newline

To understand this, we need to dive into the world of regular and critical values.
\section{Sard's Theorem}%
\begin{definition}
  Let $f\colon M\rightarrow N$ be a smooth map, and let $p\in M$. We say $p$ is a \textit{critical point} for $f$ if $D_pf$ does not have the same rank as the dimension of $T_{f(p)} N$. If $D_pf$ has the same rank as the dimension of $T_{f(p)}N$, then we say that $p$ is a \textit{regular point} of $f$.\newline

  We say $q\in N$ is a \textit{critical value} for $f$ if $f^{-1}\left( \set{q} \right)$ contains a critical point for $f$. Else,we say that $q$ is a \textit{regular value}.
\end{definition}
\end{document}
