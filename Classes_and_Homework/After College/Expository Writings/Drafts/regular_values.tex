\documentclass[10pt]{mypackage}

% sans serif font:
%\usepackage{cmbright}
%\usepackage{sfmath}
%\usepackage{bbold} %better blackboard bold

%\usepackage{homework}
\usepackage{notes}
%\usepackage{newpxtext,eulerpx,eucal}
%\renewcommand*{\mathbb}[1]{\varmathbb{#1}}

\fancyhf{}
\fancyhead[R]{Avinash Iyer}
\fancyhead[L]{Regular Values}
\fancyfoot[C]{\thepage}

\setcounter{secnumdepth}{0}

\begin{document}
\RaggedRight
\begin{abstract}
  \noindent We discuss the much celebrated Regular Value Theorem and Sard's Theorem, and discuss some of the consequences and applications of these results.
\end{abstract}
A smooth map between manifolds $f\colon M\rightarrow N$ includes a certain family of local information; for instance, the derivative $D_pf\colon T_pM\rightarrow T_{f(p)}N$, which is a linear map between tangent spaces at $p$ and $q$, is defined on a coordinate chart $U\subseteq M$ for $p$ and a corresponding coordinate chart $V\subseteq N$ for $f(p)$. Yet, the properties of this linear map can give us information about the underlying map $f$.\newline

To understand this, we need to dive into the world of regular and critical values.
\section{Sard's Theorem}%
\begin{definition}
  Let $f\colon M\rightarrow N$ be a smooth map, and let $p\in M$. We say $p$ is a \textit{critical point} for $f$ if $D_pf$ does not have the same rank as the dimension of $T_{f(p)} N$. If $D_pf$ has the same rank as the dimension of $T_{f(p)}N$, then we say that $p$ is a \textit{regular point} of $f$.\newline

  We say $q\in N$ is a \textit{critical value} for $f$ if $f^{-1}\left( \set{q} \right)$ contains a critical point for $f$. Else,we say that $q$ is a \textit{regular value}.
\end{definition}
We start with the case of Sard's Theorem on $\R^{n}$. Then, we will expand this to the case of any arbitrary manifold by means of a technical lemma.
\begin{theorem}[Sard's Theorem]
  Let $f\colon \R^{n}\supseteq U\rightarrow \R^{m}$ be a smooth map. Then, if $C$ is the set of critical points for $f$, we have $f(C) \subseteq \R^{m}$ has measure zero.
\end{theorem}
\begin{proof}
  We use induction on $n$. The statement only makes sense for $n\geq 0$ and $p\geq 1$. Clearly, the theorem is true for $n = 0$.\newline

  Let $C_1\subseteq C$ be the set of all $x\in U$ such that $D_x f$ is zero, and similarly, let $C_i$ be the set of all $x$ such that $\left( D_x \right)^{j} f$ is zero for all $j\leq i$. We obtain a descending sequence of closed sets $C\supseteq C_1\supseteq C_2\supseteq \cdots$.\newline

  We start by showing that $f\left( C\setminus C_1 \right)$ has measure zero. For each $x\in C\setminus C_1$, we find an open neighborhood $V\subseteq \R^{n}$ such that $f\left( V\cap C \right)$ has measure zero. Since $\R^{n}$ is second countable, $C\setminus C_1$ is covered by countably many such open neighborhoods, it follows that $f\left( C\setminus C_1 \right)$ has measure zero.\newline

  Since $x\notin C_1$, there is some partial derivative, which we use change of coordinates to write as $ \pd{f}{x_1} $, that is not zero at $x$. Let
  \begin{align*}
    h(x) &= \left( f_1(x),x_2,\dots,x_n \right).
  \end{align*}
  Then, since $D_xh$ is nonsingular, by the \href{https://en.wikipedia.org/wiki/Inverse_function_theorem}{inverse function theorem}, $h$ maps some neighborhood $V$ of $x$ diffeomorphically onto an open set $V'\subseteq \R^{n}$. The composition $f\circ h^{-1}$ then maps $V'$ to $R^{m}$ then maps $V'$ to $\R^{m}$.\newline

  Observe that the set of critical points of $g$ is precisely $h\left( V\cap C \right)$, so the set $g\left( C' \right)$ is equal to $f\left( V\cap C \right)$.
\end{proof}
\end{document}
