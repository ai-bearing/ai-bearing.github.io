\documentclass[10pt]{mypackage}

% sans serif font:
%\usepackage{cmbright}
%\usepackage{sfmath}
%\usepackage{bbold} %better blackboard bold

%serif font + different blackboard bold for serif font
%\usepackage{homework}
\usepackage{exposition}
\usepackage{newpxtext,eulerpx}
\renewcommand*{\mathbb}[1]{\varmathbb{#1}}

\fancyhf{}
\rhead{Avinash Iyer}
\lhead{Differentiation and the Fundamental Theorem of Calculus}

\setcounter{secnumdepth}{0}

\begin{document}
\RaggedRight
\begin{abstract}
  \noindent We discuss and prove some fundamental results about differentiation, after which prove the fundamental theorem of calculus for Lebesgue integrals.
\end{abstract}
\section{Preliminary}%
In our discussion of the \href{https://ai-bearing.github.io/Classes_and_Homework/After\%20College/Expository\%20Writings/radon_nikodym.pdf}{Radon--Nikodym Theorem}, we were able to define an abstract derivative of a ($\sigma$-finite) complex measure with respect to a different ($\sigma$-finite) measure. In Euclidean space, $\R^n$, we may consider trying to define a ``pointwise'' derivative by taking
\begin{align*}
  F(x) &= \lim_{r\rightarrow 0} \frac{\nu\left( U\left( x,r \right) \right)}{m\left( U\left( x,r \right) \right)},
\end{align*}
where $m$ is the Lebesgue measure, and $\nu$ is our given complex measure. If we take the Lebesgue--Radon--Nikodym decomposition
\begin{align*}
  d\nu &= \lambda + f\:dm,
\end{align*}
we would hope that $F = f$ almost everywhere. Indeed, we will show this to be the case, after which we may prove a stronger version of the fundamental theorem of calculus, this time for Lebesgue integrals.\newline

Note that from now on, every measure-theoretic term (i.e., integrable, almost everywhere, etc.) is taken with respect to the Lebesgue measure on $\R^n$.\newline

We start with a fundamental lemma in measure theory for Euclidean spaces.
\begin{theorem}[Vitali Covering Lemma]
  Let $\mathcal{C}$ be a collection of open balls in $\R^n$, and let $U = \bigcup_{B\in \mathcal{C}} B$.\newline

  If $c < m\left( U \right)$, then there exist disjoint $B_1,\dots,B_k$ such that
  \begin{align*}
    3^{-n} c &\leq \sum_{j=1}^{k} m\left( B_j \right).
  \end{align*}
\end{theorem}
\begin{proof}
  By inner regularity, there is a compact $K\subseteq U$ such that $m(K) > c$; finitely many balls in $\mathcal{C}$, which we call $A_1,\dots,A_m$, cover $K$.\newline

  We proceed via exhaustion; select $B_1$ to be the largest of the $A_j$, $B_2$ to be the largest of the $A_j$ disjoint from $B_1$, $B_3$ the largest of the $A_j$ disjoint from $B_2$ and $B_1$, etc. According to this construction, if $A_i$ is not among the $B_j$, then there is $j$ such that $A_i\cap B_j \neq \emptyset$, and if $j$ is the smallest such index, then the radius of $A_i$ is at most that of $B_j$. Via some triangle inequality magic, we see that $A_i\subseteq B_j^{\ast}$, where $B_j^{\ast}$ is defined to the ball with the same center as $B_j$ and three times the radius.\newline

  Then, $K\subseteq \bigcup_{j=1}^{k}B_j^{\ast}$, so that
  \begin{align*}
    c &< m(K)\\
      &\leq \sum_{j=1}^{k} m\left( B_j^{\ast} \right)\\
      &= 3^{n} \sum_{j=1}^{k} m\left( B_j \right).
  \end{align*}
\end{proof}
\section{The Lebesgue Differentiation Theorem}%

\section{The Fundamental Theorem of Calculus for Lebesgue Integration}%

\end{document}
