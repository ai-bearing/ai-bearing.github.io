\documentclass[10pt]{mypackage}

% sans serif font:
%\usepackage{cmbright}
%\usepackage{sfmath}
%\usepackage{bbold} %better blackboard bold

%serif font + different blackboard bold for serif font
%\usepackage{homework}
\usepackage{exposition}
%\usepackage{newpxtext,eulerpx}
%\renewcommand*{\mathbb}[1]{\varmathbb{#1}}

\fancyhf{}
\rhead{Avinash Iyer}
\lhead{Differentiation and the Fundamental Theorem of Calculus}

\setcounter{secnumdepth}{0}

\begin{document}
\RaggedRight
\begin{abstract}
  \noindent We discuss and prove some fundamental results about differentiation, after which prove the fundamental theorem of calculus for Lebesgue integrals.
\end{abstract}
\section{Preliminary}%
In our discussion of the \href{https://ai-bearing.github.io/Classes_and_Homework/After\%20College/Expository\%20Writings/radon_nikodym.pdf}{Radon--Nikodym Theorem}, we were able to define an abstract derivative of a ($\sigma$-finite) complex measure with respect to a different ($\sigma$-finite) measure. In Euclidean space, $\R^n$, we may consider trying to define a ``pointwise'' derivative by taking
\begin{align*}
  F(x) &= \lim_{r\rightarrow 0} \frac{\nu\left( U\left( x,r \right) \right)}{m\left( U\left( x,r \right) \right)},
\end{align*}
where $m$ is the Lebesgue measure, and $\nu$ is our given complex measure. If we take the Lebesgue--Radon--Nikodym decomposition
\begin{align*}
  d\nu &= d\lambda + f\:dm,
\end{align*}
we would hope that $F = f$ almost everywhere. Indeed, we will show this to be the case, after which we may prove a stronger version of the fundamental theorem of calculus, this time for Lebesgue integrals.\newline

Note that from now on, every measure-theoretic term (i.e., integrable, almost everywhere, etc.) is taken with respect to the Lebesgue measure on $\R^n$.\newline

We start with a fundamental lemma in measure theory for Euclidean spaces.
\begin{theorem}[Vitali Covering Lemma]
  Let $\mathcal{C}$ be a collection of open balls in $\R^n$, and let $U = \bigcup_{B\in \mathcal{C}} B$.\newline

  If $c < m\left( U \right)$, then there exist disjoint $B_1,\dots,B_k$ such that
  \begin{align*}
    3^{-n} c &\leq \sum_{j=1}^{k} m\left( B_j \right).
  \end{align*}
\end{theorem}
\begin{proof}
  By inner regularity, there is a compact $K\subseteq U$ such that $m(K) > c$; finitely many balls in $\mathcal{C}$, which we call $A_1,\dots,A_m$, cover $K$.\newline

  We proceed via exhaustion; select $B_1$ to be the largest of the $A_j$, $B_2$ to be the largest of the $A_j$ disjoint from $B_1$, $B_3$ the largest of the $A_j$ disjoint from $B_2$ and $B_1$, etc. According to this construction, if $A_i$ is not among the $B_j$, then there is $j$ such that $A_i\cap B_j \neq \emptyset$, and if $j$ is the smallest such index, then the radius of $A_i$ is at most that of $B_j$. Via some triangle inequality magic, we see that $A_i\subseteq B_j^{\ast}$, where $B_j^{\ast}$ is defined to the ball with the same center as $B_j$ and three times the radius.\newline

  Then, $K\subseteq \bigcup_{j=1}^{k}B_j^{\ast}$, so that
  \begin{align*}
    c &< m(K)\\
      &\leq \sum_{j=1}^{k} m\left( B_j^{\ast} \right)\\
      &= 3^{n} \sum_{j=1}^{k} m\left( B_j \right).
  \end{align*}
\end{proof}
\section{The Lebesgue Differentiation Theorem}%
\begin{definition}
  A function $f\colon \R^n\rightarrow \C$ is called \textit{locally integrable} if $ \int_{K}^{} \left\vert f \right\vert\:dm < \infty $ for every bounded measurable $K\subseteq \R^n$.\footnote{Note that we still use the convention $0\cdot \infty = 0$.}\newline

  The space of locally integrable functions is denoted $L_{1, \text{loc}}$.
\end{definition}
\begin{definition}
  If $f\in L_{1,\text{loc}}$, and $x\in \R^n$, and $r > 0$, define
  \begin{align*}
    A_rf(x) &= \frac{1}{m\left( B\left( x,r \right) \right)} \int_{B\left( x,r \right)}^{} f(y)\:dy
  \end{align*}
  to be the \textit{average} of $f$ on $B\left( x,r \right)$.
\end{definition}
\begin{lemma}
  If $f\in L_{1,\text{loc}}$, then $A_rf$ is jointly continuous in $r$ and $x$.
\end{lemma}
\begin{proof}
  We know that $m\left( B\left( x,r \right) \right) = cr^{n}$, where $c = m\left( B\left( 0,1 \right) \right)$, and $m\left( S\left( x,r \right) \right) = 0$, where $S\left( x,r \right) = \set{y | \left\vert y-x \right\vert = r}$.\newline

  Moreover, as $r\rightarrow r_0$ and $x\rightarrow x_0$, $\1_{B\left( x,r \right)}\rightarrow \1_{B\left( x_0,r_0 \right)}$ pointwise on $\R^n\setminus S\left( x_0,r_0 \right)$, so the convergence is pointwise almost everywhere. Furthermore, note that $\left\vert \1_{B\left( x,r \right)} \right\vert \leq \1_{B\left( x_0,r_0 + 1 \right)}$ for $r < r_0 + 1/2$ and $\left\vert x-x_0 \right\vert < 1/2$. Thus, by dominated convergence, it follows that $\int_{B\left( x,r \right)}^{} f(y)\:dy$ is continuous in $r$ and $x$, and so is $A_rf(x)$.
\end{proof}
\begin{definition}
  If $f\in L_{1,\text{loc}}$, we define the \textit{Hardy--Littlewood Maximal Function}, $Hf$, by
  \begin{align*}
    Hf(x) &= \sup_{r > 0} A_r\left\vert f \right\vert(x)\\
          &= \sup_{r > 0} \frac{1}{m\left( B\left( x,r \right) \right)} \int_{B\left( x,r \right)}^{} \left\vert f(y) \right\vert\:dy.
  \end{align*}
\end{definition}
\begin{theorem}[The Maximal Theorem]
  There is a constant $C > 0$ such that for all $f\in L_1$ and all $\alpha > 0$,
  \begin{align*}
    m\left( \set{x | Hf(x) > \alpha} \right) &\leq \frac{C}{\alpha} \int_{\R^n}^{} \left\vert f(x) \right\vert\:dx.
  \end{align*}
\end{theorem}
\begin{proof}
  Let $E_{\alpha} = \set{x | Hf(x) > \alpha}$. For each $x\in E_{\alpha}$, we may find $r_x > 0$ such that $A_{r_x}\left\vert f \right\vert(x) > \alpha$. The balls $U\left( x,r_x \right)$ cover $E_{\alpha}$, so by the Vitali Covering Lemma, if $c < m\left( E_{\alpha} \right)$, then there are $x_1,\dots,x_k$ such that $B_j = B\left( x_j,r_{x_j} \right)$ are disjoint and $\sum_{j=1}^{k}m\left( B_j \right) > 3^{-n}c$.\newline

  Then, we see that
  \begin{align*}
    c &< 3^{n}\sum_{j=1}^{k} m\left( B_j \right)\\
      &\leq \frac{3^{n}}{\alpha} \sum_{j=1}^{k} \int_{B_j}^{} \left\vert f(y) \right\vert\:dy\\
      &\leq \frac{3^{n}}{\alpha} \int_{\R^{n}}^{} \left\vert f(y) \right\vert\:dy.
  \end{align*}
  Thus, letting $c \rightarrow m\left( E_{\alpha} \right)$, we obtain our desired result.
\end{proof}
\begin{exercise}
  A variant of the Hardy--Littlewood Maximal Function is defined by
  \begin{align*}
    H^{\ast}f(x) &= \sup\set{ \frac{1}{m\left( B \right)}\int_{B}^{} \left\vert f(y) \right\vert\:dy | B\text{ is a ball, }x\in B}.
  \end{align*}
  Show that $Hf \leq H^{\ast}f \leq 2^{n}Hf$.
\end{exercise}
\begin{solution}
  We see that, necessarily,
  \begin{align*}
    \frac{1}{m\left( B\left( x,r \right) \right)} \int_{B\left( x,r \right)}^{} \left\vert f(y) \right\vert\:dy &\leq H^{\ast}f(x),
  \end{align*}
  so that $Hf(x) \leq H^{\ast}f(x)$.\newline

  Now, if $r > 0$ is such that $x\in B\left( z,r \right)$, then clearly $B\left( z,r \right)\subseteq B\left( y,2r \right)$, so
  \begin{align*}
    \frac{1}{m\left( B\left( z,r \right) \right)} \int_{B\left( z,r \right)}^{} \left\vert f(y) \right\vert\:dy &\leq \frac{1}{m\left( B\left( z,r \right) \right)} \int_{B\left( x, 2r\right)}^{} \left\vert f(y) \right\vert\:dy\\
                                                                                                                &\leq \frac{2^{n}}{m\left( B\left( x,2r \right) \right)} \int_{B\left( x,2r \right)}^{} \left\vert f(y) \right\vert\:dy\\
                                                                                                                &\leq 2^{n}Hf(x).
  \end{align*}
  Thus, by taking suprema, we see that $H^{\ast}f(x) \leq 2^{n}Hf(x)$.
\end{solution}
\begin{definition}
  If $f\colon \R\rightarrow \C$ is a function, then the limit superior as $r$ approaches $R$ is defined to be
  \begin{align*}
    \limsup_{r\rightarrow R} \phi\left( r \right) &\coloneq \lim_{\ve\rightarrow 0} \left( \sup_{0 < \left\vert r-R \right\vert < \ve} \phi(r) \right).
  \end{align*}
\end{definition}
\begin{remark}
  Note that
  \begin{align*}
    \lim_{r\rightarrow R} \phi(r) &= C
  \end{align*}
  if and only if
  \begin{align*}
    \limsup_{r\rightarrow R} \left\vert \phi(r) - c \right\vert = 0.
  \end{align*}
\end{remark}
We will prove progressively stronger versions of the Lebesgue Differentiation Theorem.
\begin{theorem}
  If $f\in L_{1,\text{loc}}$, then $\lim_{r\rightarrow 0}A_rf(x) = f(x)$ for a.e. $x\in \R^n$.
\end{theorem}
\begin{proof}
  It suffices to show that for any $N\in \N$, $A_rf(x) \rightarrow f(x)$ for almost every $x$ with $\left\vert x \right\vert \leq N$. Furthermore, we may replace $f$ by $f\1_{B\left( 0,N+1 \right)}$ in this scenario, as $A_rf(x)$ only depends on the value $f(y)$ for $\left\vert y \right\vert \leq N+1$. Thus, we may assume $f\in L_1$.\newline

  Given $\ve > 0$, there is a compactly supported continuous function $g$ such that $\norm{g-f}_{L_1} < \ve$. Since $g$ is continuous, for any $x\in \R^n$ and $\delta > 0$, there is $r > 0$ such that $\left\vert g(y)-g(x) \right\vert < \delta$ whenever $\left\vert y-x \right\vert < r$. Thus,
  \begin{align*}
    \left\vert A_rg(x) - g(x) \right\vert &= \frac{1}{m\left( B\left( x,r \right) \right)} \left\vert \int_{B\left( x,r \right)}^{} g(y)-g(x)\:dy \right\vert\\
                                          &< \delta,
  \end{align*}
  meaning $A_rg(x) \rightarrow g(x)$ as $r\rightarrow 0$ for every $x$. Thus,
  \begin{align*}
    \limsup_{r\rightarrow 0} \left\vert A_rf(x) - f(x) \right\vert &= \limsup_{r\rightarrow 0} \left\vert A_r\left( f-g \right)(x) + \left( A_rg-g \right)(x) + \left( g-f \right)(x) \right\vert\\
                                                                   &\leq H\left( f-g \right)(x) + \left\vert f-g \right\vert(x).
  \end{align*}
  Now, if we set
  \begin{align*}
    E_{\alpha} &= \set{x | \limsup_{r\rightarrow 0}\left\vert A_rf(x) - f(x) \right\vert > \alpha}\\
    F_{\alpha} &= \set{x | \left\vert f-g \right\vert(x) > \alpha},
  \end{align*}
  then
  \begin{align*}
    E_{\alpha} &\subseteq F_{\alpha/2} \cup \set{x | H\left( f-g \right)(x) > \alpha/2}
  \end{align*}
  Now, we see that
  \begin{align*}
    m\left( F_{\alpha/2} \right) &\leq \frac{2}{\alpha} \int_{F_{\alpha/2}}^{} \left\vert f(x)-g(x) \right\vert\:dx\\
                                 &< \ve,
  \end{align*}
  so by the Maximal Theorem,
  \begin{align*}
    m\left( E_{\alpha} \right) &\leq \frac{2}{\alpha} \ve + \frac{2C}{\alpha} \ve,
  \end{align*}
  and since $\ve$ is arbitrary, $m\left( E_{\alpha} \right) = 0$. Thus, $\lim_{r\rightarrow 0}A_rf(x) = f(x)$ for all $x\notin \bigcup_{n=1}^{\infty}E_{1/n}$.\newline

  Thus, we find that if $f\in L_{1,\text{loc}}$, then
  \begin{align*}
    \lim_{r\rightarrow 0} \frac{1}{m\left( B\left( x,r \right) \right)} \int_{B\left( x,r \right)}^{} \left( f(y)-f(x) \right)\:dy &= 0
  \end{align*}
  for almost every $x$.
\end{proof}
 In fact, we can prove something stronger.
 \begin{definition}
   Let $f\in L_{1,\text{loc}}$, then we define
   \begin{align*}
     L_{f} &= \set{x | \lim_{r\rightarrow 0}\frac{1}{m\left( B\left( x,r \right) \right)} \int_{B\left( x,r \right)}^{} \left\vert f(y)-f(x) \right\vert\:dy = 0}
   \end{align*}
   to be the \textit{Lebesgue set} of $f$.
 \end{definition}
 \begin{theorem}
   If $f\in L_{1,\text{loc}}$, then $m\left( \left( L_f \right)^{c} \right) = 0$.
 \end{theorem}
 \begin{proof}
   For each $c\in \C$, we may apply the previous theorem to the function $g_c(x) = \left\vert f(x)-c \right\vert$ to get that, except for a certain null set $E_c$,
   \begin{align*}
     \lim_{r\rightarrow 0} \frac{1}{m\left( B\left( x,r \right) \right)} \int_{B\left( x,r \right)}^{} \left\vert f(y)-c \right\vert\:dy &= \left\vert f(x)-c \right\vert.
   \end{align*}
   Now, if $D$ is a countable dense subset of $\C$, and $E = \bigcup_{c\in D}E_c$, then $m(E) = 0$, and if $x\notin E$, there is $c\in D$ with $\left\vert f(x)-c \right\vert < \ve$, so that $\left\vert f(y)-f(x) \right\vert < \left\vert f(y)-c \right\vert + \ve$, and
   \begin{align*}
     \limsup_{r\rightarrow 0} \frac{1}{m\left( B\left( x,r \right) \right)} \int_{B\left( x,r \right)}^{} \left\vert f(y)-f(x) \right\vert\:dy &\leq \left\vert f(x)-c \right\vert + \ve\\
                                                                                                                                                       &< 2\ve.
   \end{align*}
   Since $\ve$ is arbitrary, our desired result follows.
 \end{proof}
 \begin{definition}
   We say a family of subsets $\set{E_r}_{r > 0}$ of Borel subsets of $\R^n$ \textit{shrinks nicely} to $x\in \R^n$ if
   \begin{itemize}
     \item $E_r\subseteq B\left( x,r \right)$ for each $r$;
      \item there is $\alpha > 0$ independent of $r$ such that $m\left( E_r \right) > \alpha m\left( B\left( x,r \right) \right)$.
   \end{itemize}
 \end{definition}
 \begin{remark}
   The sets $E_r$ need not contain $x$.
 \end{remark}
 \begin{theorem}[Lebesgue Differentiation Theorem]
   If $f\in L_{1,\text{loc}}$, then for every $x\in L_f$,
   \begin{align*}
     \lim_{r\rightarrow 0} \frac{1}{m\left( E_r \right)} \int_{E_r}^{} \left\vert f(y)-f(x) \right\vert\:dy &= 0\\
     \lim_{r\rightarrow 0} \frac{1}{m\left( E_r \right)} \int_{E_r}^{} f(y)\:dy &= f(x)
   \end{align*}
   for every family $\set{E_r}_{r > 0}$ that shrinks nicely to zero.
 \end{theorem}
 \begin{proof}
   For some $\alpha > 0$, the definition of $\set{E_r}_{r > 0}$ allows us to take
   \begin{align*}
     \frac{1}{m\left( E_r \right)} \int_{E_r}^{} \left\vert f(y)-f(x) \right\vert\:dy &\leq \frac{1}{m\left( E_r \right)} \int_{B\left( x,r \right)}^{} \left\vert f(y)-f(x) \right\vert\:dy\\
                                                                                      &\leq \frac{1}{\alpha m\left( B\left( x,r \right) \right)} \int_{B\left( x,r \right)}^{} \left\vert f(y)-f(x) \right\vert\:dy.
   \end{align*}
 \end{proof}
 \begin{definition}
   A Borel measure $\nu$ on $\R^n$ is called \textit{regular} if
   \begin{enumerate}[(i)]
     \item $\nu\left( K \right) < \infty$ for all compact sets $K$;
     \item for all $E\in \mathcal{B}_{\R^n}$,
       \begin{align*}
         \nu\left( E \right) &= \inf\set{\nu\left( U \right) | U\text{ open, }E\subseteq U}.
       \end{align*}
   \end{enumerate}
   If $\nu$ is a signed measure, then we say $\nu$ is regular if $\left\vert \nu \right\vert$ is regular.
 \end{definition}
 \begin{proposition}
   If $\lambda$ and $\mu$ are positive, mutually singular, and $\lambda + \mu$ is regular, then $\lambda$ and $\mu$ are regular.
 \end{proposition}
 \begin{proof}
   Let $A\subseteq \R^n$ be such that $A$ is $\mu$-null and $A^{c}$ is $\lambda$-null.\newline
   
   We see that condition (i) in the definition of regularity holds necessarily, so we show condition (ii). Now, let $\ve > 0$ and let $E \subseteq A$ be Borel. Since $\lambda + \mu$ is regular, and $\lambda$ is concentrated on $A$, there is an open $U\in \mathcal{B}_{\R^n}$ such that
   \begin{align*}
     \left( \lambda + \mu \right)\left( U \right) &< \left( \lambda + \mu \right)\left( E \right) + \ve\\
                                                  &= \lambda\left( E \right) + \ve,
   \end{align*}
   meaning that
   \begin{align*}
     \lambda\left( U \right) &< \lambda\left( E \right) + \ve,
   \end{align*}
   so condition (ii) for $\lambda$, and similarly for $\mu$ (by taking $E\subseteq A^{c}$).
 \end{proof}
 \begin{proposition}
   The measure $f\:dm$ is regular if and only if $f\in L_{1,\text{loc}}$.
 \end{proposition}
 \begin{proof}
   The condition $f\in L_{1,\text{loc}}$ is equivalent to $f\:dm$ being finite on compact sets, so condition (i) holds.\newline

   Now, if $E$ is a bounded Borel set, then given $\delta > 0$, there is a bounded open $U\supseteq E$ such that $m\left( U \right) < m\left( E \right) + \delta$, meaning $m\left( U\setminus E \right) < \delta$. At the same time, given $\ve > 0$, there is an open $U\supseteq E$ such that $ \int_{U\setminus E}^{} f\:dm < \ve $, meaning $ \int_{U}^{} f\:dm < \int_{E}^{} f\:dm + \ve $ with $m\left( U\setminus E \right) < \delta$.\newline

   If $E$ is unbounded, then we write $E = \bigcup_{j=1}^{\infty}E_j$ as a union of bounded Borel sets, and finding $U_j\supseteq E_j$ with $\int_{U_j\setminus E_j}^{} f\:dm < \ve 2^{-j}$.
 \end{proof}
 \begin{theorem}
   Let $\nu$ be a regular signed or complex Borel measure on $\R^n$, and let $d\nu = d\lambda + f\:dm$ be the Lebesgue--Radon--Nikodym representation. Then, $m$-a.e. $x\in \R^n$,
   \begin{align*}
     \lim_{r\rightarrow 0} \frac{\nu\left( E_r \right)}{m\left( E_r \right)} &= f(x)
   \end{align*}
   for every family $\set{E_r}_{r > 0}$ that shrinks nicely to $x$.
 \end{theorem}
 \begin{proof}
   Note that $d\left\vert \nu \right\vert = d\left\vert \lambda \right\vert + \left\vert f \right\vert\:dm$, so regularity of $\nu$ means that both $\lambda$ and $f\:dm$ are regular.\newline

   Since $f\in L_{1,\text{loc}}$, it suffices to show that if $\lambda$ is regular and $\lambda \perp m$, then for $m$-a.e. $x$,
   \begin{align*}
     \lim_{r\rightarrow 0} \frac{\lambda\left( E_r \right)}{m\left( E_r \right)} &= 0
   \end{align*}
   whenever $E_r$ shrinks nicely to $x$.\newline

   It also suffices to take $E_r = B\left( x,r \right)$, and assume $\lambda$ is positive, since for some $\alpha > 0$,
   \begin{align*}
     \left\vert \frac{\lambda\left( E_r \right)}{m\left( E_r \right)}  \right\vert &\leq \frac{\left\vert \lambda \right\vert\left( E_r \right)}{m\left( E_r \right)}\\
                                                                                   &\leq \frac{\left\vert \lambda \right\vert\left( B\left( x,r \right) \right)}{m\left( E_r \right)}\\
                                                                                   &\leq \frac{\left\vert \lambda \right\vert\left( B\left( x,r \right) \right)}{\alpha m\left( B\left( x,r \right) \right)}.
   \end{align*}
   Thus, let $A$ be a Borel set such that $\lambda\left( A \right) = m\left( A^{c} \right) = 0$. Let
   \begin{align*}
     F_k &= \set{x\in A | \limsup_{r\rightarrow 0} \frac{\lambda\left( B\left( x,r \right) \right)}{m\left( B\left( x,r \right) \right)} > \frac{1}{k}}.
   \end{align*}
   We will show that $m\left( F_k \right) = 0$ for all $k$.\newline

   Now, by regularity of $\lambda$, given $\ve > 0$, there is $U_{\ve} \supseteq A$ such that $\lambda\left( U_{\ve} \right) < \ve$. Each $x\in F_k$ is the center of an open ball $U_{x}\subseteq U_{\ve}$ such that $\lambda\left( U_x \right) > \frac{1}{k}m\left( U_{x} \right)$ (by the properties of the limit superior). Now, if $V_{\ve} = \bigcup_{x\in F_k} U_x$, and $c < m\left( V_{\ve} \right)$, then by the Vitali Covering Lemma, there are $U_{x_1},\dots,U_{x_J}$ that are disjoint such that
   \begin{align*}
     c &< 3^{n}\sum_{j=1}^{J} m\left( U_{x_j} \right)\\
       &< 3^{n}k\sum_{j=1}^{J} \lambda\left( U_{x_j} \right)\\
       &< 3^{n}k\lambda\left( V_{\ve} \right)\\
       &\leq 3^{n}k\lambda\left( U_{\ve} \right)\\
       &\leq 3^{n}k\ve,
   \end{align*}
   meaning that $m\left( V_{\ve} \right) \leq 3^{n}k\ve$, and since $F_k\subseteq V_{\ve}$ and $\ve$ is arbitrary, $m\left( F_k \right) = 0$.
 \end{proof}
\section{The Fundamental Theorem of Calculus for Lebesgue Integration}%
Recall from the construction of the \href{https://ai.avinash-iyer.com/Classes_and_Homework/After%20College/Expository%20Writings/lebesgue_measure.pdf}{Lebesgue measure} that there is a one-to-one correspondence between increasing, right-continuous function on $\R$ and Borel measures $\mu_F$ determined by $\mu_F\left( (a,b] \right) = F(b)-F(a)$. We will use this to help prove the almost-everywhere differentiability of increasing functions.
  \begin{theorem}
    Let $F\colon \R\rightarrow \R$ be increasing, and let $G(x) = F(x+)$.
    \begin{enumerate}[(a)]
      \item The set of points at which $F$ is discontinuous is countable.
      \item The functions $F$ and $G$ are differentiable almost everywhere, and $F' = G'$ almost everywhere.
    \end{enumerate}
  \end{theorem}
  \begin{proof}\hfill
    \begin{enumerate}[(a)]
      \item Since $F$ is increasing, the intervals $\left( F\left( x- \right),F\left( x+ \right) \right)$ for each $x$ are disjoint, and for $\left\vert x \right\vert < N$, they lie in the interval $\left( F\left( -N, \right),F\left( N \right) \right)$. Thus,
        \begin{align*}
          \sum_{\left\vert x \right\vert < N} \left( F\left( x+ \right)-F\left( x- \right) \right) &\leq F\left( N \right) - F\left( -N \right)\\
                                                                                                   &< \infty,
        \end{align*}
        meaning that the set of all $x$ in $(-N,N)$ such that $F(x+) \neq F(x-)$ is countable.
      \item Observe that $G$ is increasing and right-continuous, and $G = F$ almost everywhere. Moreover, we see that
        \begin{align*}
          G\left( x+h \right) - G\left( x \right) &= \begin{cases}
            \mu_G\left( \left( x,x+h \right] \right) & h > 0\\
            -\mu_G\left( \left( x+h,x \right] \right) & h < 0,
          \end{cases}
        \end{align*}
        and the families $\set{\left( x-\left\vert h \right\vert,x \right]}$ and $\set{\left( x,x+\left\vert h \right\vert \right]}$ shrink nicely to $x$ as $\left\vert h \right\vert \rightarrow 0$. Applying the previous theorem, since $\mu_G$ is regular, we see that $G'(x)$ exists almost everywhere.\newline

        Finally, we show that if $H = G-F$, then $H'$ is zero almost everywhere. Letting $\set{x_j}_{j=1}^{\infty}$ be an enumeration of points where $H\neq 0$, we see that $H\left( x_j \right) > 0$, and $\sum_{\left\vert x_j \right\vert < N}H\left( x_j \right) < \infty$ for any $N$.\newline

        Let $\delta_j$ be the point mass at $x_j$, and set $\mu = \sum_{j=1}^{\infty}H\left( x_j \right)\delta_j$. Note that $\mu$ is finite on compact sets, and $\mu$ is regular, and $\mu\perp m$ since $m\left( \set{x_j}_{j=1}^{\infty} \right) = \mu\left( \left( \set{x_j}_{j=1}^{\infty} \right)^{c} \right) = 0$.\newline

        Then,
        \begin{align*}
          \left\vert \frac{H\left( x+h \right) - H\left( x \right)}{h} \right\vert &\leq \frac{H\left( x+h \right) + H\left( x \right)}{h}\\
                                                                                   &\leq \frac{4\mu\left( \left( x-2\left\vert h \right\vert , x + 2\left\vert h \right\vert\right) \right)}{4\left\vert h \right\vert},
        \end{align*}
        which tends to zero as $h\rightarrow 0$ for almost every $x$, meaning $H' = 0$ almost everywhere.
    \end{enumerate}
  \end{proof}
\end{document}
