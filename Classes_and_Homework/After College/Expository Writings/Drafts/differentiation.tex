\documentclass[10pt]{mypackage}

% sans serif font:
%\usepackage{cmbright}
%\usepackage{sfmath}
%\usepackage{bbold} %better blackboard bold

%serif font + different blackboard bold for serif font
%\usepackage{homework}
\usepackage{exposition}
%\usepackage{newpxtext,eulerpx}
%\renewcommand*{\mathbb}[1]{\varmathbb{#1}}

\fancyhf{}
\rhead{Avinash Iyer}
\lhead{Differentiation and the Fundamental Theorem of Calculus}

\setcounter{secnumdepth}{0}

\begin{document}
\RaggedRight
\begin{abstract}
  \noindent We discuss and prove some fundamental results about differentiation, after which prove the fundamental theorem of calculus for Lebesgue integrals.
\end{abstract}
\section{Preliminary}%
In our discussion of the \href{https://ai-bearing.github.io/Classes_and_Homework/After\%20College/Expository\%20Writings/radon_nikodym.pdf}{Radon--Nikodym Theorem}, we were able to define an abstract derivative of a ($\sigma$-finite) complex measure with respect to a different ($\sigma$-finite) measure. In Euclidean space, $\R^n$, we may consider trying to define a ``pointwise'' derivative by taking
\begin{align*}
  F(x) &= \lim_{r\rightarrow 0} \frac{\nu\left( U\left( x,r \right) \right)}{m\left( U\left( x,r \right) \right)},
\end{align*}
where $m$ is the Lebesgue measure, and $\nu$ is our given complex measure. If we take the Lebesgue--Radon--Nikodym decomposition
\begin{align*}
  d\nu &= \lambda + f\:dm,
\end{align*}
we would hope that $F = f$ almost everywhere. Indeed, we will show this to be the case, after which we may prove a stronger version of the fundamental theorem of calculus, this time for Lebesgue integrals.\newline

Note that from now on, every measure-theoretic term (i.e., integrable, almost everywhere, etc.) is taken with respect to the Lebesgue measure on $\R^n$.\newline

We start with a fundamental lemma in measure theory for Euclidean spaces.
\begin{theorem}[Vitali Covering Lemma]
  Let $\mathcal{C}$ be a collection of open balls in $\R^n$, and let $U = \bigcup_{B\in \mathcal{C}} B$.\newline

  If $c < m\left( U \right)$, then there exist disjoint $B_1,\dots,B_k$ such that
  \begin{align*}
    3^{-n} c &\leq \sum_{j=1}^{k} m\left( B_j \right).
  \end{align*}
\end{theorem}
\begin{proof}
  By inner regularity, there is a compact $K\subseteq U$ such that $m(K) > c$; finitely many balls in $\mathcal{C}$, which we call $A_1,\dots,A_m$, cover $K$.\newline

  We proceed via exhaustion; select $B_1$ to be the largest of the $A_j$, $B_2$ to be the largest of the $A_j$ disjoint from $B_1$, $B_3$ the largest of the $A_j$ disjoint from $B_2$ and $B_1$, etc. According to this construction, if $A_i$ is not among the $B_j$, then there is $j$ such that $A_i\cap B_j \neq \emptyset$, and if $j$ is the smallest such index, then the radius of $A_i$ is at most that of $B_j$. Via some triangle inequality magic, we see that $A_i\subseteq B_j^{\ast}$, where $B_j^{\ast}$ is defined to the ball with the same center as $B_j$ and three times the radius.\newline

  Then, $K\subseteq \bigcup_{j=1}^{k}B_j^{\ast}$, so that
  \begin{align*}
    c &< m(K)\\
      &\leq \sum_{j=1}^{k} m\left( B_j^{\ast} \right)\\
      &= 3^{n} \sum_{j=1}^{k} m\left( B_j \right).
  \end{align*}
\end{proof}
\section{The Lebesgue Differentiation Theorem}%
\begin{definition}
  A function $f\colon \R^n\rightarrow \C$ is called \textit{locally integrable} if $ \int_{K}^{} \left\vert f \right\vert\:dm < \infty $ for every bounded measurable $K\subseteq \R^n$.\footnote{Note that we still use the convention $0\cdot \infty = 0$.}\newline

  The space of locally integrable functions is denoted $L_{1, \text{loc}}$.
\end{definition}
\begin{definition}
  If $f\in L_{1,\text{loc}}$, and $x\in \R^n$, and $r > 0$, define
  \begin{align*}
    A_rf(x) &= \frac{1}{m\left( B\left( x,r \right) \right)} \int_{B\left( x,r \right)}^{} f(y)\:dy
  \end{align*}
  to be the \textit{average} of $f$ on $B\left( x,r \right)$.
\end{definition}
\begin{lemma}
  If $f\in L_{1,\text{loc}}$, then $A_rf$ is jointly continuous in $r$ and $x$.
\end{lemma}
\begin{proof}
  We know that $m\left( B\left( x,r \right) \right) = cr^{n}$, where $c = m\left( B\left( 0,1 \right) \right)$, and $m\left( S\left( x,r \right) \right) = 0$, where $S\left( x,r \right) = \set{y | \left\vert y-x \right\vert = r}$.\newline

  Moreover, as $r\rightarrow r_0$ and $x\rightarrow x_0$, $\1_{B\left( x,r \right)}\rightarrow \1_{B\left( x_0,r_0 \right)}$ pointwise on $\R^n\setminus S\left( x_0,r_0 \right)$, so the convergence is pointwise almost everywhere. Furthermore, note that $\left\vert \1_{B\left( x,r \right)} \right\vert \leq \1_{B\left( x_0,r_0 + 1 \right)}$ for $r < r_0 + 1/2$ and $\left\vert x-x_0 \right\vert < 1/2$. Thus, by dominated convergence, it follows that $\int_{B\left( x,r \right)}^{} f(y)\:dy$ is continuous in $r$ and $x$, and so is $A_rf(x)$.
\end{proof}
\begin{definition}
  If $f\in L_{1,\text{loc}}$, we define the \textit{Hardy--Littlewood Maximal Function}, $Hf$, by
  \begin{align*}
    Hf(x) &= \sup_{r > 0} A_r\left\vert f \right\vert(x)\\
          &= \sup_{r > 0} \frac{1}{m\left( B\left( x,r \right) \right)} \int_{B\left( x,r \right)}^{} \left\vert f(y) \right\vert\:dy.
  \end{align*}
\end{definition}
\begin{theorem}[The Maximal Theorem]
  There is a constant $C > 0$ such that for all $f\in L_1$ and all $\alpha > 0$,
  \begin{align*}
    m\left( \set{x | Hf(x) > \alpha} \right) &\leq \frac{C}{\alpha} \int_{\R^n}^{} \left\vert f(x) \right\vert\:dx.
  \end{align*}
\end{theorem}
\begin{proof}
  Let $E_{\alpha} = \set{x | Hf(x) > \alpha}$. For each $x\in E_{\alpha}$, we may find $r_x > 0$ such that $A_{r_x}\left\vert f \right\vert(x) > \alpha$. The balls $U\left( x,r_x \right)$ cover $E_{\alpha}$, so by the Vitali Covering Lemma, if $c < m\left( E_{\alpha} \right)$, then there are $x_1,\dots,x_k$ such that $B_j = B\left( x_j,r_{x_j} \right)$ are disjoint and $\sum_{j=1}^{k}m\left( B_j \right) > 3^{-n}c$.\newline

  Then, we see that
  \begin{align*}
    c &< 3^{n}\sum_{j=1}^{k} m\left( B_j \right)\\
      &\leq \frac{3^{n}}{\alpha} \sum_{j=1}^{k} \int_{B_j}^{} \left\vert f(y) \right\vert\:dy\\
      &\leq \frac{3^{n}}{\alpha} \int_{\R^{n}}^{} \left\vert f(y) \right\vert\:dy.
  \end{align*}
  Thus, letting $c \rightarrow m\left( E_{\alpha} \right)$, we obtain our desired result.
\end{proof}
\section{The Fundamental Theorem of Calculus for Lebesgue Integration}%

\end{document}
