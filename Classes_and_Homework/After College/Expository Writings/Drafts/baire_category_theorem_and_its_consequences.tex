\documentclass[10pt]{mypackage}

% sans serif font:
%\usepackage{cmbright,sfmath,bbold}
%\renewcommand{\mathcal}{\mathtt}

%Euler:
%\usepackage{newpxtext,eulerpx,eucal,eufrak}
%\renewcommand*{\mathbb}[1]{\varmathbb{#1}}
%\renewcommand*{\hbar}{\hslash}

%\renewcommand{\mathbb}{\mathds}
%\usepackage{homework}
\usepackage{exposition}

\pagestyle{fancy} %better headers
\fancyhf{}
\rhead{Avinash Iyer}
\lhead{Understanding and Using the Baire Category Theorem}

\setcounter{secnumdepth}{0}

\begin{document}
\RaggedRight
\begin{abstract}
  \noindent We discuss the much-celebrated Baire Category Theorem and some of its consequences.
\end{abstract}
\section{The Theorem}%
The Baire Category Theorem is a very celebrated theorem in analysis and topology that bridges the gap between the two fields. In words, it allows us to understand properties of complete metric spaces via their topological properties (i.e., density of open sets).
\begin{definition}[Baire Space]
Let $\set{A_n}_{n\geq 1}$ be a countable collection of open, dense subsets of a topological space $X$. We say $X$ is a \textit{Baire Space} if
\begin{align*}
A &= \bigcap_{n\geq 1}A_n
\end{align*}
is dense for every such collection.
\end{definition}
\begin{definition}
We say a topological space $X$ is \textit{meager} if it is a countable union of nowhere dense subsets.\footnote{Recall that a set $A\subseteq X$ is nowhere dense if $\overline{A}^{\circ} = \emptyset$.}
\end{definition}
\begin{theorem}[Baire Category Theorem]
  If $X$ is a complete metric space, then $X$ is a Baire space.
\end{theorem}
\begin{proof}
  Let $\set{A_n}_{n\geq 1}$ be a countable family of open, dense subsets of $X$. Let $U_0$ be any ball of radius $r > 0$, and set $B_0 = \overline{U_0}$.\newline

  Since $A_1\cap U_0$ is nonempty, there is a closed subset $B_1\subseteq A_1\cap U_0$ with radius less than $r/2$. Set $U_1 = B_1^{\circ}$. Similarly, find $B_2\subseteq A_2\cap U_1$ with radius less than $r/4$, and set $U_2 = B_2^{\circ}$, and continue finding $B_m\subseteq A_m\cap U_{m-1}$ with radius less than $r/2^{m}$. We get a chain
  \begin{align*}
    B_0 \supseteq U_0\supseteq B_1\supseteq U_1\supseteq \cdots,
  \end{align*}
  and let $\left( x_n \right)_n$ be the sequence of centers of $B_n$. The sequence $\left( x_n \right)_n$ is Cauchy in $X$, as the distance between any two $x_m$ and $x_n$ is at most $r/2^{m-1}$ for $n > m$. Since $X$ is complete, we get $\left( x_n \right)_n\rightarrow x$ for some $x\in X$. We claim $x\in \bigcap_{n\geq 1}B_n$.\newline

  If not, then $x\notin B_{N}$ for some $N\in \N$, and for $n \geq N$, $x\notin B_n$, so $d\left( x_n,x \right) \geq \dist_{B_n}(x) > 0$, which contradicts $x_n\rightarrow x$.\newline

  Therefore, $x\in \bigcap_{n\geq 1}B_n\subseteq \bigcap_{n\geq 1}A_n$, and since $\bigcap_{n\geq 1}B_n\subseteq U_0$, we have $\left( \bigcap_{n\geq 1}A_n \right)\cap U_0 \neq \emptyset$, so $\bigcap_{n\geq 1}A_n$ is dense in $X$.
\end{proof}
\begin{remark}
  By taking complements, we see that if $X$ is a complete metric space, then $X$ is not the countable union of nowhere dense subsets.
\end{remark}

\section{The Consequences}%
There are a variety of great results that can be found from applying the Baire Category Theorem. First, we use the theorem to establish a fundamental result in the theory of complete metric spaces.
\begin{proposition}
  Let $X$ be a complete metric space with no isolated points, and let $S\subseteq X$ be a countable and dense subset. Then, $S$ is not a $G_{\delta}$ set.
\end{proposition}
\begin{proof}
  Since $S$ is countable, we may write $S = \set{s_k}_{k\in\N}$. Suppose $S$ is $G_{\delta}$, so that $S = \bigcap_{n\geq 1} V_n$, where the $V_n$ are open. We note that $S^{c} = \bigcup_{n\geq 1}V_n^{c}$, and write
  \begin{align*}
    X &= S^{c} \cup \set{s_k}_{k\in\N}\\
      &= \bigcup_{n\geq 1}V_n^{c} \cup \set{s_k}_{k\in\N}.
  \end{align*}
  We see that $V_n^{c}$ and $\set{s_k}$ are closed subsets of $X$ for each $n,k$, and $\set{s_k}_{k\in\N}^{\circ} = \emptyset$ as $X$ does not contain any isolated points.\newline

  By the Baire Category Theorem, there must be some $n\in\N$ such that $\left( V_n^{c} \right)^{\circ}\neq \emptyset$ , so there is some $\delta > 0$ and $x\in X$ such that $U\left( x,\delta \right)\subseteq V_n^{c}\subseteq S^{c}$ for some $\delta > 0$. However, since $S$ is dense, $U\left( x,\delta \right)\cap S \neq \emptyset$, which is a contradiction.
\end{proof}
\subsection{Bases of Banach Spaces}%
One of the simplest, yet surprisingly interesting results that uses the Baire Category Theorem regards the dimension of Banach spaces.
\begin{proposition}
  If $X$ is an infinite-dimensional Banach space, then $X$ has an uncountable basis.
\end{proposition}
\begin{proof}
  Suppose that $X$ had a countable basis $\set{e_k}_{k=1}^{\infty}$. Then, we may consider the sequence of subspaces
  \begin{align*}
    E_1 &= \Span\set{e_1}\\
    E_2 &= \Span\set{e_1,e_2}\\
        &\vdots\\
    E_{n} &= \Span\set{e_1,\dots,e_n}.
  \end{align*}
  We see that these finite-dimensional subspaces are closed, that $\bigcup_{n\geq 1}E_n = X$, and since $X$ has infinitely many dimensions, each of the interiors of the $E_n$ are empty. However, since $X$ is a Banach space, this cannot hold, so $X$ does not have a countable basis.
\end{proof}
\begin{example}
  The sequence space $c_c$ consists of all finitely supported sequences in $\C$, equipped with the supremum norm. The canonical basis for the sequence space is $\set{e_n}_{n\geq 1}$, where $e_n$ denotes the sequence with $1$ at position $n$ and zero elsewhere. So, $c_c$ has a countable basis.\newline

  Yet, by taking the completion of $c_c$ with respect to the supremum norm, we obtain $c_0$, which is the sequences in $\C$ that converge to zero, equipped with the supremum norm.\footnote{This is a pretty simple exercise.} Since $c_0$ \textit{is} complete, it is thus equipped with an uncountable basis.
\end{example}
\subsection{Theorems from Functional Analysis}%
A lot of functional analysis is related to continuous linear maps between (likely infinite-dimensional) Banach spaces; the Baire Category Theorem allows us to elucidate many useful properties of these continuous linear maps via some very basic conditions.
\begin{lemma}[Open Mapping Lemma]
  Let $X$ and $Y$ be Banach spaces, and suppose $T\colon X\rightarrow Y$ is a continuous linear operator. 
  \begin{enumerate}[(i)]
    \item If $U_Y\subseteq \overline{T\left( \delta U_X \right)}$ for some $\delta > 0$, then $U_Y\subseteq T\left( 2\delta U_X \right)$.
    \item If $\delta U_Y\subseteq \overline{T\left( U_X \right)}$ for some $\delta > 0$, then $\frac{\delta}{2}U_Y\subseteq T\left( U_X \right)$. In particular, this means that $T$ is a quotient map.\footnote{That is, a surjective open map.}
  \end{enumerate}
\end{lemma}
\begin{proof}\hfill
  \begin{enumerate}[(i)]
    \item Let $y\in U_Y$. By assumption, there is $x_1\in \delta U_X$ such that $\norm{y - T\left( x_1 \right)} < \frac{1}{2}$. Then, we have the chain of inclusions
      \begin{align*}
        y-T\left( x_1 \right) &\in \frac{1}{2}U_Y\\
                              &\subseteq \frac{1}{2} \overline{T\left( \delta U_X \right)}\\
                              &= \overline{T\left( \frac{\delta}{2}U_X \right)}.
      \end{align*}
      Then, there exists $x_2\in \frac{\delta}{2}U_X$ such that $\norm{\left( y-T\left( x_1 \right) \right)-T\left( x_2 \right)} < \frac{1}{4}$, whence $y - T\left( x_1 + x_2 \right)\subseteq \overline{T\left( \frac{\delta}{4}U_X \right)}$. Inductively, we have the sequence of $\left( x_k \right)_k\in \frac{\delta}{2^{k-1}}U_X$ with
      \begin{align*}
        \norm{Y - \sum_{j=1}^{k}T\left( x_j \right)} &< 2^{-k}.
      \end{align*}
      Letting $x = \sum_{j=1}^{\infty}x_j$, we observe that the series is complete and converges absolutely, since
      \begin{align*}
        \sum_{j=1}^{\infty}\norm{x_j} &\leq \sum_{j=1}^{\infty} \frac{\delta}{2^{j-1}}\\
                                      &= 2\delta\\
                                      &< \infty.
      \end{align*}
      In particular, this means that $\norm{x} \leq 2\delta$, meaning $x\in 2\delta U_X$, and that $T\left( x \right) = y$ by continuity.
    \item If $\delta U_Y\subseteq \overline{T\left( U_X \right)}$, for some $\delta > 0$, then $U_Y\subseteq \overline{T\left( \frac{1}{\delta}U_X \right)}$, whence $U_Y\subseteq T\left( \frac{2}{\delta}U_X \right)$, meaning $\frac{\delta}{2}U_Y\subseteq T\left( U_X \right)$.
  \end{enumerate}
\end{proof}
\subsection{Partitioning the Reals}%
\end{document}
