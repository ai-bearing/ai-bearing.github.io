\documentclass[10pt]{mypackage}

% sans serif font:
%\usepackage{cmbright,sfmath,bbold}
%\renewcommand{\mathcal}{\mathtt}

%Euler:
%\usepackage{newpxtext,eulerpx,eucal,eufrak}
%\renewcommand*{\mathbb}[1]{\varmathbb{#1}}
%\renewcommand*{\hbar}{\hslash}

%\renewcommand{\mathbb}{\mathds}
%\usepackage{homework}
\usepackage{exposition}

\pagestyle{fancy} %better headers
\fancyhf{}
\rhead{Avinash Iyer}
\lhead{The Baire Category Theorem and its Consequences}

\setcounter{secnumdepth}{0}

\begin{document}
\RaggedRight
\begin{abstract}
  \noindent We discuss the much-celebrated Baire Category Theorem and some of its consequences.
\end{abstract}
\section{The Theorem}%
The Baire Category Theorem is a very celebrated theorem in analysis and topology that bridges the gap between the two fields. In words, it allows us to understand properties of complete metric spaces via their topological properties (i.e., density of open sets).
\begin{definition}[Baire Space]
Let $\set{A_n}_{n\geq 1}$ be a countable collection of open, dense subsets of a topological space $X$. We say $X$ is a \textit{Baire Space} if
\begin{align*}
A &= \bigcap_{n\geq 1}A_n
\end{align*}
is dense for every such collection.
\end{definition}
\begin{definition}
We say a topological space $X$ is \textit{meager} if it is a countable union of nowhere dense subsets.\footnote{Recall that a nowhere dense set $A\subseteq X$ is nowhere dense if $\overline{A}^{\circ} = \emptyset$.}
\end{definition}
\begin{theorem}[Baire Category Theorem]
  If $X$ is a complete metric space, then $X$ is a Baire space.
\end{theorem}
\begin{proof}
  Let $\set{A_n}_{n\geq 1}$ be a countable family of open, dense subsets of $X$. Let $U_0$ be any ball of radius $r > 0$, and set $B_0 = \overline{U_0}$.\newline

  Since $A_1\cap U_0$ is nonempty, there is a closed subset $B_1\subseteq A_1\cap U_0$ with radius less than $r/2$. Set $U_1 = B_1^{\circ}$. Similarly, find $B_2\subseteq A_2\cap U_1$ with radius less than $r/4$, and set $U_2 = B_2^{\circ}$, and continue finding $B_m\subseteq A_m\cap U_{m-1}$ with radius less than $r/2^{m}$. We get a chain
  \begin{align*}
    B_0 \supseteq U_0\supseteq B_1\supseteq U_1\supseteq \cdots,
  \end{align*}
  and let $\left( x_n \right)_n$ be the sequence of centers of $B_n$. The sequence $\left( x_n \right)_n$ is Cauchy in $X$, as the distance between any two $x_m$ and $x_n$ is at most $r/2^{m-1}$ for $n > m$. Since $X$ is complete, we get $\left( x_n \right)_n\rightarrow x$ for some $x\in X$. We claim $x\in \bigcap_{n\geq 1}B_n$.\newline

  If not, then $x\notin B_{N}$ for some $N\in \N$, and for $n \geq N$, $x\notin B_n$, so $d\left( x_n,x \right) \geq \dist_{B_n}(x) > 0$, which contradicts $x_n\rightarrow x$.\newline

  Therefore, $x\in \bigcap_{n\geq 1}B_n\subseteq \bigcap_{n\geq 1}A_n$, and since $\bigcap_{n\geq 1}B_n\subseteq U_0$, we have $\left( \bigcap_{n\geq 1}A_n \right)\cap U_0 \neq \emptyset$, so $\bigcap_{n\geq 1}A_n$ is dense in $X$.
\end{proof}
\section{The Consequences}%
\subsection{Bases of Banach Spaces}%
\subsection{Theorems from Functional Analysis}%
\subsection{Continuity Sets}%
\subsection{Partitioning the Reals}%
\end{document}
