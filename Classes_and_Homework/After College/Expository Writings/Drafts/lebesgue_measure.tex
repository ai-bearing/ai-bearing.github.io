\documentclass[10pt]{mypackage}

% sans serif font:
%\usepackage{cmbright,sfmath,bbold}
%\renewcommand{\mathcal}{\mathtt}

%Euler:
%\usepackage{newpxtext,eulerpx,eucal,eufrak}
%\renewcommand*{\mathbb}[1]{\varmathbb{#1}}
%\renewcommand*{\hbar}{\hslash}

%\renewcommand{\mathbb}{\mathds}
%\usepackage{homework}
\usepackage{exposition}

\pagestyle{fancy} %better headers
\fancyhf{}
\rhead{Avinash Iyer}
\lhead{The Lebesgue Measure}

\setcounter{secnumdepth}{0}

\begin{document}
\RaggedRight
\begin{abstract}
  \noindent We detail the construction and some of the properties of the Lebesgue measure.
\end{abstract}
Consider a set-function $\lambda\colon P\left( \R \right)\rightarrow [0,\infty]$ that satisfies
\begin{itemize}
  \item $\lambda\left( \emptyset \right) = 0$;
  \item for any finite or infinite sequence of disjoint sets, $\set{E_j}_{j=1}^{\infty}$, we have
    \begin{align*}
      \lambda\left( \bigsqcup_{j=1}^{\infty} \right) &= \sum_{j=1}^{\infty}\lambda\left( E_j \right);
    \end{align*}
  \item $\lambda\left( I \right) = b - a$, where $I$ is an interval (either open, closed, or a half-interval);
  \item $\lambda\left( s + E \right) = \lambda\left( E \right)$.
\end{itemize}
Unfortunately, such a set-function doesn't exist.\newline

In order to construct a set function on a restricted domain $\lambda\colon \mathcal{L}\rightarrow [0,\infty]$, we need to define a particular class of measurable subsets of $\R$. This is where the concept of an \textit{outer measure} comes in.
\begin{definition}
  Let $X$ be a set, and let $\mu^{\ast}\colon P\left( X \right)\rightarrow [0,\infty]$ be a set function. We say $\mu^{\ast}$ is an \textit{outer measure} if
  \begin{itemize}
    \item $\mu^{\ast}\left( \emptyset \right)= 0$;
    \item $\mu^{\ast}\left( A \right)\leq \mu^{\ast}\left( B \right)$ if $A\subseteq B$;
    \item $\displaystyle \mu^{\ast}\left( \bigcup_{j=1}^{\infty}A_j \right)\leq \sum_{j=1}^{\infty}\mu^{\ast}\left( A_j \right)$.
  \end{itemize}
\end{definition}
We will obtain an outer measure on the entirety of $P(X)$ by defining a notion of measure on some ``satisfactory'' subfamily $\mathcal{E}\subseteq P(X)$, then by approximating other subsets using this family.
\end{document}
