\documentclass[10pt]{mypackage}

% sans serif font:
%\usepackage{cmbright,sfmath,bbold}
%\renewcommand{\mathcal}{\mathtt}

%Euler:
%\usepackage{newpxtext,eulerpx,eucal,eufrak}
%\renewcommand*{\mathbb}[1]{\varmathbb{#1}}
%\renewcommand*{\hbar}{\hslash}

%\renewcommand{\mathbb}{\mathds}
%\usepackage{homework}
\usepackage{exposition}

\pagestyle{fancy} %better headers
\fancyhf{}
\rhead{Avinash Iyer}
\lhead{Signed Measures and the Lebesgue--Radon--Nikodym Theorem}

\setcounter{secnumdepth}{0}

\begin{document}
\RaggedRight
\begin{abstract}
  \noindent Measures are just set functions that follow some particular basic properties, but we can expand them beyond the positive real numbers towards complex numbers; to conceptualize these signed and complex measures, we need to make use of results like the Lebesgue--Radon--Nikodym Theorem and the Hahn Decomposition Theorem that allow us to understand their structural properties.
\end{abstract}
\section{Signed Measures and the Hahn Decomposition}%
We know that a measure is a set function $\mu\colon \mathcal{M}\rightarrow [0,\infty]$ on a $\sigma$-algebra such that
\begin{itemize}
  \item $\mu\left( \emptyset \right) = 0$;
  \item for a family of disjoint sets $\set{E_j}_{j=1}^{\infty}\subseteq \mathcal{M}$,
    \begin{align*}
      \mu\left( \bigsqcup_{j=1}^{\infty}E_j \right) &= \sum_{j=1}^{\infty}\mu\left( E_j \right).
    \end{align*}
\end{itemize}
We may ask what happens if we change the codomain from $[0,\infty]$ to $\R$ or $\C$. This is where \textit{signed measures} come in.
\begin{definition}
  A \textit{signed measure} $\mu$ is a real-valued countably additive set function such that $\mu\left( \emptyset \right) = 0$ and $\mu$ takes on at most one of $-\infty$ or $\infty$.
\end{definition}
We begin by establishing some basic properties of signed measures (akin to the basic properties of measures).
\begin{theorem}
  Let $\mu$ be a signed measure.
  \begin{enumerate}[(a)]
    \item If $E$ and $F$ are measurable sets with $E\subseteq F$ and $\left\vert \mu\left( F \right) \right\vert < \infty$, then $\left\vert \mu\left( E \right) \right\vert < \infty$.
    \item If $\set{E_j}_{j=1}^{\infty}\subseteq \mathcal{M}$ is a disjoint sequence of measurable subsets such that $\left\vert \mu\left( \bigsqcup_{j=1}^{\infty}E_j \right) \right\vert < \infty$, then the series $\sum_{j=1}^{\infty}\mu\left( E_j \right)$ is absolutely convergent.
    \item If $\set{E_j}_{j=1}^{\infty}$ is a monotone sequence of measurable sets --- and if decreasing, $\left\vert \mu\left( E_n \right) \right\vert < \infty$ for at least one such $n$ --- then 
      \begin{align*}
        \mu\left( \lim_{j\rightarrow\infty} E_j \right) &= \lim_{j\rightarrow\infty}\mu\left( E_j \right).
      \end{align*}
  \end{enumerate}
\end{theorem}
\begin{proof}\hfill
  \begin{enumerate}[(a)]
    \item We see that $\mu\left( F \right) = \mu\left( F\setminus E \right) + \mu\left( E \right)$. If exactly one of the summands is infinite, then so is $\mu\left( F \right)$. If both are infinite, then since $\mu$ takes on at most one of $-\infty$ or $\infty$, they are equal and then $\mu\left( F \right)$ is infinite. Therefore, both summands must be finite.
    \item We set
      \begin{align*}
        E_j^{+} &= \begin{cases}
          E_j & \mu\left( E_j \right) \geq 0\\
          \emptyset & \mu\left( E_j \right) < 0
        \end{cases}\\
          E_j^{-} &= \begin{cases}
            E_j & \mu\left( E_j \right) \leq 0\\
            \emptyset & \mu\left( E_j \right) > 0
          \end{cases}.
      \end{align*}
      Then,
      \begin{align*}
        \mu\left( \bigsqcup_{j=1}^{\infty}E_j^{+} \right) &= \sum_{j=1}^{\infty}\mu\left( E_j^{+} \right)\\
        \mu\left( \bigsqcup_{j=1}^{\infty}E_j^{-} \right) &= \sum_{j=1}^{\infty}\mu\left( E_j^{-} \right).
      \end{align*}
      Since the terms of both series have constant sign, and $\mu$ takes on at most one of $\pm\infty$, it follows that at least one of these series is convergent, and since $\sum_{j=1}^{\infty}\mu\left( E_j \right)$ is convergent, both series converge; therefore, the series is absolutely convergent.
    \item If $\set{E_n}_{n=1}^{\infty}$ is increasing, then we take
      \begin{align*}
        \mu\left( \bigsqcup_{j=1}^{\infty}E_j \right) &= \mu\left( \bigsqcup_{j=2}^{\infty}\left( E_{j}\setminus E_{j-1} \right) \right)\\
                                                      &= \sum_{j=2}^{\infty}\mu\left( E_{j}\setminus E_{j-1} \right)\\
                                                      &= \lim_{n\rightarrow\infty}\sum_{j=2}^{n}\mu\left( E_j\setminus E_{j-1} \right)\\
                                                      &= \lim_{n\rightarrow\infty}\mu\left( \bigsqcup_{j=2}^{n}\left( E_j\setminus E_{j-1} \right) \right)\\
                                                      &= \lim_{j\rightarrow\infty}\mu\left( E_j \right),
      \end{align*}
      and similarly for a decreasing sequence, using part (a) to ensure finiteness.
  \end{enumerate}
\end{proof}
Now, we discuss the structure of positive-valued and negative-valued measurable sets.
\begin{definition}
  Let $\mu$ be a signed measure on $\left( X,\mathcal{M} \right)$. We call a set $E\in \mathcal{M}$ \textit{positive} if, for every measurable $F\subseteq E$, $\mu\left( F \right) \geq 0$; similarly, we call $E\in \mathcal{M}$ \textit{negative} if, for every measurable $F\subseteq E$, $\mu\left( F \right) \leq 0$.
\end{definition}
\begin{theorem}[Hahn Decomposition Theorem]
  If $\mu$ is a signed measure, then there exist two disjoint sets $A$ and $B$ such that $A\sqcup B = X$, $A$ is positive with respect to $\mu$, and $B$ is negative with respect to $\mu$. This decomposition unique up to $\mu$-null symmetric difference.
\end{theorem}
\end{document}
