\documentclass[10pt]{mypackage}

% sans serif font:
%\usepackage{cmbright,sfmath,bbold}
%\renewcommand{\mathcal}{\mathtt}

%Euler:
%\usepackage{newpxtext,eulerpx,eucal,eufrak}
%\renewcommand*{\mathbb}[1]{\varmathbb{#1}}
%\renewcommand*{\hbar}{\hslash}

%\renewcommand{\mathbb}{\mathds}
\usepackage{homework}
\usepackage{exposition}
\usepackage{microtype}

\pagestyle{fancy} %better headers
\fancyhf{}
\rhead{Avinash Iyer}
\lhead{Signed Measures and the Lebesgue--Radon--Nikodym Theorem}

\setcounter{secnumdepth}{0}

\begin{document}
\RaggedRight
\begin{abstract}
  \noindent Measures are just set functions that follow some particular basic properties, but we can expand them beyond the positive real numbers towards complex numbers; to conceptualize these signed and complex measures, we need to make use of results like the Lebesgue--Radon--Nikodym Theorem and the Hahn Decomposition Theorem that allow us to understand their structural properties.
\end{abstract}
\section{Signed Measures and the Hahn Decomposition}%
We know that a measure is a set function $\mu\colon \mathcal{M}\rightarrow [0,\infty]$ on a $\sigma$-algebra such that
\begin{itemize}
  \item $\mu\left( \emptyset \right) = 0$;
  \item for a family of disjoint sets $\set{E_j}_{j=1}^{\infty}\subseteq \mathcal{M}$,
    \begin{align*}
      \mu\left( \bigsqcup_{j=1}^{\infty}E_j \right) &= \sum_{j=1}^{\infty}\mu\left( E_j \right).
    \end{align*}
\end{itemize}
We may ask what happens if we change the codomain from $[0,\infty]$ to $\R$ or $\C$. This is where \textit{signed measures} come in.
\begin{definition}
  A \textit{signed measure} $\mu$ is a real-valued countably additive set function such that $\mu\left( \emptyset \right) = 0$ and $\mu$ takes on at most one of $-\infty$ or $\infty$.
\end{definition}
We begin by establishing some basic properties of signed measures (akin to the basic properties of measures).
\begin{theorem}
  Let $\mu$ be a signed measure.
  \begin{enumerate}[(a)]
    \item If $E$ and $F$ are measurable sets with $E\subseteq F$ and $\left\vert \mu\left( F \right) \right\vert < \infty$, then $\left\vert \mu\left( E \right) \right\vert < \infty$.
    \item If $\set{E_j}_{j=1}^{\infty}\subseteq \mathcal{M}$ is a disjoint sequence of measurable subsets such that $\left\vert \mu\left( \bigsqcup_{j=1}^{\infty}E_j \right) \right\vert < \infty$, then the series $\sum_{j=1}^{\infty}\mu\left( E_j \right)$ is absolutely convergent.
    \item If $\set{E_j}_{j=1}^{\infty}$ is a monotone sequence of measurable sets --- and if decreasing, $\left\vert \mu\left( E_n \right) \right\vert < \infty$ for at least one such $n$ --- then 
      \begin{align*}
        \mu\left( \lim_{j\rightarrow\infty} E_j \right) &= \lim_{j\rightarrow\infty}\mu\left( E_j \right).
      \end{align*}
  \end{enumerate}
\end{theorem}
\begin{proof}\hfill
  \begin{enumerate}[(a)]
    \item We see that $\mu\left( F \right) = \mu\left( F\setminus E \right) + \mu\left( E \right)$. If exactly one of the summands is infinite, then so is $\mu\left( F \right)$. If both are infinite, then since $\mu$ takes on at most one of $-\infty$ or $\infty$, they are equal and then $\mu\left( F \right)$ is infinite. Therefore, both summands must be finite.
    \item We set
      \begin{align*}
        E_j^{+} &= \begin{cases}
          E_j & \mu\left( E_j \right) \geq 0\\
          \emptyset & \mu\left( E_j \right) < 0
        \end{cases}\\
          E_j^{-} &= \begin{cases}
            E_j & \mu\left( E_j \right) \leq 0\\
            \emptyset & \mu\left( E_j \right) > 0
          \end{cases}.
      \end{align*}
      Then,
      \begin{align*}
        \mu\left( \bigsqcup_{j=1}^{\infty}E_j^{+} \right) &= \sum_{j=1}^{\infty}\mu\left( E_j^{+} \right)\\
        \mu\left( \bigsqcup_{j=1}^{\infty}E_j^{-} \right) &= \sum_{j=1}^{\infty}\mu\left( E_j^{-} \right).
      \end{align*}
      Since the terms of both series have constant sign, and $\mu$ takes on at most one of $\pm\infty$, it follows that at least one of these series is convergent, and since $\sum_{j=1}^{\infty}\mu\left( E_j \right)$ is convergent, both series converge; therefore, the series is absolutely convergent.
    \item If $\set{E_n}_{n=1}^{\infty}$ is increasing, then we take
      \begin{align*}
        \mu\left( \bigsqcup_{j=1}^{\infty}E_j \right) &= \mu\left( \bigsqcup_{j=2}^{\infty}\left( E_{j}\setminus E_{j-1} \right) \right)\\
                                                      &= \sum_{j=2}^{\infty}\mu\left( E_{j}\setminus E_{j-1} \right)\\
                                                      &= \lim_{n\rightarrow\infty}\sum_{j=2}^{n}\mu\left( E_j\setminus E_{j-1} \right)\\
                                                      &= \lim_{n\rightarrow\infty}\mu\left( \bigsqcup_{j=2}^{n}\left( E_j\setminus E_{j-1} \right) \right)\\
                                                      &= \lim_{j\rightarrow\infty}\mu\left( E_j \right),
      \end{align*}
      and similarly for a decreasing sequence, using part (a) to ensure finiteness.
  \end{enumerate}
\end{proof}
Now, we discuss the structure of positive-valued and negative-valued measurable sets.
\begin{definition}
  Let $\mu$ be a signed measure on $\left( X,\mathcal{M} \right)$. We call a set $E\in \mathcal{M}$ \textit{positive} if, for every measurable $F\subseteq E$, $\mu\left( F \right) \geq 0$; similarly, we call $E\in \mathcal{M}$ \textit{negative} if, for every measurable $F\subseteq E$, $\mu\left( F \right) \leq 0$.
\end{definition}
\begin{theorem}[Hahn Decomposition Theorem]
  If $\mu$ is a signed measure, then there exist two disjoint sets $A$ and $B$ such that $A\sqcup B = X$, $A$ is positive with respect to $\mu$, and $B$ is negative with respect to $\mu$. This decomposition unique up to $\mu$-null symmetric difference.
\end{theorem}
\begin{proof}
  Without loss of generality, we may assume that for all $E\in \mathcal{M}$, $-\infty < \mu\left( E \right) \leq \infty$.\newline

  We note that the difference of two negative sets is negative, and the disjoint, countable union of negative sets is negative, so every countable union of negative sets is negative. We let $\beta = \inf\left( \mu\left( B \right) \right)$ for all negative $B\in \mathcal{M}$. We let $\left( B_j \right)_j\subseteq \mathcal{M}$ be a sequence of measurable negative sets such that $\lim_{j\rightarrow\infty}\mu\left( B_j \right) = \beta$, and set $B = \bigcup_{j=1}^{\infty}B_j$. We see then that $B$ is a negative set for which $\mu\left( B \right)$ is minimal.\newline

  We now prove that $A = X\setminus B$ is a positive set. Suppose toward contradiction that there is $E_0\subseteq A$ such that $\mu\left( E_0 \right) < 0$. The set $E_0$ cannot be a negative set, or else $B \cup E_0$ would be a negative set with a smaller measure than $\mu\left( B \right)$, which is not possible.\newline

  Let $k_1$ be the smallest natural number such that there is $E_1\subseteq E_0$ with $\mu\left( E_1 \right) \geq \frac{1}{k_1}$. Since
  \begin{align*}
    \mu\left( E_0\setminus E_1 \right) &= \mu\left( E_0 \right) - \mu\left( E_1 \right)\\
                                       &\leq \mu\left( E_0 \right) - \frac{1}{k_1}\\
                                       &< 0.
  \end{align*}
  The argument applied to $E_0$ is now applicable to $E_0\setminus E_1$. Letting $k_2$ be the smallest natural number such that $E_0\setminus E_1$ contains $E_2\subseteq E_0\setminus E_1$ with $\mu\left( E_2 \right) \geq \frac{1}{k_2}$, and proceeding ad infinitum, we see that since $\mu$ is finitely valued for measurable subsets of $E_0$, we have $\lim_{n\rightarrow\infty}\frac{1}{k_n} = 0$.\newline

  It follows that for every measurable subset $F$ of
  \begin{align*}
    F_0 &= E_0\setminus \left( \bigcup_{j=1}^{\infty}E_j \right),
  \end{align*}
  we have $\mu\left( F \right) \leq 0$ --- i.e., $F_0$ is a measurable negative set. Since $F_0$ is disjoint from $B$, and
  \begin{align*}
    \mu\left( F_0 \right) &= \mu\left( E_0 \right) - \sum_{j=1}^{\infty}\mu\left( E_j \right)\\
                          &\leq \mu\left( E_0 \right)\\
                          &< 0,
  \end{align*}
  this contradicts the minimality of $B$. Therefore, the hypothesis $\mu\left( E_0 \right) < 0$ is untenable.\newline

  If $A_1\sqcup B_1$ and $A_2\sqcup B_2$ are two Hahn decompositions for $X$, then $A_1\setminus A_2 \subseteq B_2$ and $A_1\setminus A_2 \subseteq A_1$, meaning that $A_1\setminus A_2$ is both positive and negative, hence null; similarly for $A_2\setminus A_1$, so that $A_1\triangle A_2$ is $\mu$-null, and similarly for $B_1\triangle B_2$.
\end{proof}
\begin{definition}
  We say two measures, $\mu$ and $\nu$, are \textit{mutually singular} if there exist $A,B\in \mathcal{M}$ with $A\cap B = \emptyset$, $A\sqcup B = X$, $A$ is $\mu$-null, and $B$ is $\nu$-null. In other words, $\mu$ and $\nu$ ``live on disjoint sets.'' We write $\mu\perp \nu$.
\end{definition}
\begin{definition}
  Let $X = A\sqcup B$ be a Hahn decomposition for the signed measure $\mu$. We define
  \begin{align*}
    \mu^{+}\left( E \right) &= \mu\left( E\cap A \right)\\
    \mu^{-}\left( E \right) &= -\mu\left( E\cap B \right)
  \end{align*}
  to be the \textit{positive and negative variation} of $\mu$. The \textit{total variation} of $\mu$ is defined to be
  \begin{align*}
    \left\vert \mu \right\vert &= \mu^{+} + \mu^{-}.
  \end{align*}
\end{definition}
\begin{theorem}[Jordan Decomposition]
  Every signed measure is the difference of two mutually singular measures, at least one of which is finite.
\end{theorem}
\begin{proof}
  Set $\mu = \mu^{+}-\mu^{-}$. Then, by definition, $\mu\left( E \right) = \mu\left( A\cap E \right) + \mu\left( B\cap E \right)$ where $X = A\sqcup B$ is a Hahn decomposition.
\end{proof}
\begin{exercise}
  If $\mu$ and $\nu$ are signed measures, then $\nu\perp \mu$ if and only if $\left\vert \nu \right\vert\perp \mu$ if and only if $\nu^{+}\perp \mu$ and $\nu^{-}\perp \mu$.
\end{exercise}
\begin{solution}
  We use the decomposition $X = E\sqcup F$, where $\nu$ is concentrated on $E$ and $\mu$ is concentrated on $F$. Then, we apply the Hahn decomposition to each of $E$ and $F$.
\end{solution}
\section{Absolute Continuity and the Lebesgue--Radon--Nikodym Theorem}%
\begin{definition}
  Let $\nu$ be a signed measure, $\mu$ a positive measure on $\left( X,\mathcal{M} \right)$. We say $\nu$ is absolutely continuous with respect to $\mu$, written
  \begin{align*}
    \nu\ll\mu,
  \end{align*}
  if $\nu\left( E \right) = 0$ for all $E\in \mathcal{M}$ where $\mu\left( E \right) = 0$.
\end{definition}
\begin{exercise}
  If $\nu$ is a signed measure and $\mu$ a positive measure on $\left( X,\mathcal{M} \right)$, then $\nu\ll\mu$ if and only if $\left\vert \nu \right\vert\ll\mu$ if and only if $\nu^{+} \ll\mu$ and $\nu^{-}\ll\mu$.
\end{exercise}
\begin{solution}
  Let $\mu\left( E \right) = 0$. Then, $\nu\left( E \right) = \nu^{+}\left( E \right) - \nu^{-}\left( E \right) = 0$ for each $E$. Now, in particular, if $P\sqcup N$ is a Hahn decomposition for $\nu$, then $\mu\left( E\cap P \right) = 0$, meaning $\nu\left( E\cap P \right) = 0 = \nu^{+}\left( E \right)$, and similarly for $\nu^{-}$. Thus, $\nu^{+}\ll\mu$ and $\nu^{-}\ll\mu$, and similarly, $\left\vert \nu \right\vert\ll\mu$.
\end{solution}

The terminology ``continuity'' makes sense for absolute continuity of measures via the following theorem.
\begin{theorem}
  Let $\nu$ be a finite signed measure, $\mu$ a positive measure on $\left( X,\mathcal{M} \right)$. Then, $\nu\ll\mu$ if and only if for every $\ve > 0$, there exists $\delta > 0$ such that $\left\vert \nu\left( E \right) \right\vert < \ve$ whenever $\mu\left( E \right) < \delta$.
\end{theorem}
\begin{proof}
  Since $\nu\ll\mu$ if and only if $\left\vert \nu \right\vert\ll\mu$, and $\left\vert \nu\left( E \right) \right\vert \leq \left\vert \nu \right\vert\left( E \right)$, it suffices to assume that $\nu$ is positive. The $\ve$-$\delta$ condition clearly implies that $\nu\ll\mu$.\newline

  In the other direction, if the $\ve$-$\delta$ condition is not satisfied, then there exists $\ve_0 > 0$ such that for all $n\in \N$, we may find $E_{n}\in \mathcal{M}$ with $\mu\left( E_n \right) < 2^{-n}$ and $\nu\left( E_n \right) \geq \ve_0$. Letting $F_k = \bigcup_{n=k}^{\infty} E_n$, and $F = \bigcap_{k=1}^{\infty}F_k$ (i.e., $F = \limsup_{n\rightarrow\infty} E_n$), then
  \begin{align*}
    \mu\left( F_k \right) &< \sum_{n=k}^{\infty} 2^{-n}\\
                          &= 2^{1-k},
  \end{align*}
  so $\mu\left( F \right) = 0$.\footnote{\scriptsize We may also show this using the Borel--Cantelli Lemma. To see this, note that $\mu\left( \limsup_{n\rightarrow\infty} E_n \right) = \lim_{k\rightarrow\infty}\mu\left( \bigcup_{n=k}^{\infty} E_n \right)$, by continuity from above, meaning that $\mu\left( \limsup_{n\rightarrow\infty} E_n \right) \leq \lim_{k\rightarrow\infty}\sum_{n=k}^{\infty} \mu\left( E_n \right)\rightarrow 0$.} Yet, since $\nu\left( F_k \right) \geq \ve$ for all $k$, and $\nu$ is finite, $\nu\left( F \right) \geq \ve$, meaning $\nu\not\ll\mu$.
\end{proof}

\end{document}
