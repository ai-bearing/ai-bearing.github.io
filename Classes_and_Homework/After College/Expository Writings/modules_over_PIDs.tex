\documentclass[10pt]{mypackage}

% sans serif font:
%\usepackage{cmbright,sfmath,bbold}
%\renewcommand{\mathcal}{\mathtt}

%Euler:
%\usepackage{newpxtext,eulerpx,eucal,eufrak}
%\renewcommand*{\mathbb}[1]{\varmathbb{#1}}
%\renewcommand*{\hbar}{\hslash}

\renewcommand{\mathbb}{\mathds}
%\usepackage{homework}
\usepackage{exposition}
\DeclareMathOperator{\tor}{tor}

\pagestyle{fancy} %better headers
\fancyhf{}
\rhead{Avinash Iyer}
\lhead{Modules over Principal Ideal Domains}

\setcounter{secnumdepth}{0}

\begin{document}
\RaggedRight
\begin{abstract}
  \noindent We show that if $E$ is a module defined over a principal ideal domain $R$, then $E$ is uniquely decomposable as $E \cong R^{r}\oplus R/\left\langle q_1 \right\rangle\oplus \cdots \oplus R/\left\langle q_n \right\rangle$, where $R^{r}$ is a free module of rank $r$, and $q_1 | q_2 | \cdots | q_n$, a result known as the structure theorem for modules over principal ideal domains. To do this, we provide an overview of results from the theory of modules before stating and proving the result.
\end{abstract}
Most of the exposition follows Lang's \textit{Algebra}, though there are occasionally some details added or terminology changed.
\section{Module Basics}%
\begin{definition}
  Let $A$ be a ring. A \textit{left $A$-module} $M$ is an abelian group with an operation of $A$ on $M$ such that
  \begin{align*}
    \left( a + b \right)x &= ax + bx\\
    a\left( x + y \right) &= ax + ay
  \end{align*}
  for all $a,b\in A$ and $x,y\in M$.\newline

  If $M$ is an $A$-module, then $N\subseteq M$ is known as a \textit{submodule} of $N$ is a subgroup such that $AN \subseteq N$.
\end{definition}
One of the most important submodules is the torsion submodule.
\begin{definition}
  Let $A$ be an integral domain, and let $M$ be an $A$-module. The \textit{torsion submodule} of $M$, denoted $M_{\tor}$, is the subset of elements $x\in M$ such that there exists a nonzero $a\in A$ with $ax = 0$.\newline

  If $M_{\tor} = \set{0}$, then we say $M$ is \textit{torsion-free}.
\end{definition}
We assume that all our modules are over integral domains.\newline

Just as there are isomorphism theorems for groups and rings, there are isomorphism theorems for modules. There is also a rich theory of morphisms between modules that we will discuss elsewhere.
\begin{definition}
  Let $\mathfrak{a}\subseteq A$ be a left ideal, and let $M$ be a module. We define $\mathfrak{a}M$ to be the set of all elements
  \begin{align*}
    a_1x_1 + \cdots + a_nx_n,
  \end{align*}
  where $a_i\in \mathfrak{a}$ and $x_i\in M$.
\end{definition}

We will now discuss modules generated by some subset of the module $M$. These will become important as we go deeper into establishing the structure theorem.
\begin{definition}
  Let $M$ be an $A$-module, and let $S\subseteq M$. A linear combination of elements of $S$ is a finite sum of the form
  \begin{align*}
    \sum_{x\in S}a_x x,
  \end{align*}
  where $a_x\in A$.\newline

  If $N$ is the set of all linear combinations of $S$, then $N$ is a submodule of $M$, known as the submodule generated by $S$, written $N = A\left\langle S \right\rangle$.\newline

  If $S$ consists of one element $x$, the module generated by $x$ is written $Ax$, or $\left\langle x \right\rangle$, which we call a principal module.
\end{definition}
\begin{definition}
  A module $M$ is said to be finitely generated if it has a finite number of generators.
\end{definition}
If $M$ is an $A$-module, and $\set{M_i}_{i\in I}$ is a family of submodules, we have a family of inclusion homomorphisms $\lambda_i\colon M_i\rightarrow M$, which induces a module homomorphism $\lambda\colon \bigoplus_{i\in I}M_i\rightarrow M$, where
\begin{align*}
  \lambda\left( \left( x_i \right)_{i\in I} \right) &= \sum_{i\in I}x_i,
\end{align*}
are finite sums.\newline

Now, if $\lambda\colon \bigoplus_{i\in I}M_i\rightarrow M$ is an isomorphism, then the family $\set{M_i}_{i\in I}$ is a direct sum decomposition of $M$.\newline

If $M$ is a module, and $N,N'$ are submodules such that $N + N' = M$ and $N\cap N' = \set{0}$, then we have a module isomorphism $M\cong N\oplus N'$.
\section{Establishing the Structure Theorem}%
From here on out, we assume all modules are over principal ideal domains.
\begin{definition}
  If $M$ is an $A$-module, then a subset $S\subseteq M$ is called a \textit{basis} if $S$ is nonempty, linearly independent,\footnote{Linear independence is defined for modules similar to how it is defined for vector spaces.} and generates $M$.\newline

  A \textit{free module} is a module that admits a basis.
\end{definition}
\begin{theorem}
  Let $M$ be a free $A$-module with basis $\set{x_i}_{i\in I}$. Then, if $N$ is an $A$-module with $\set{y_i}_{i\in I}\subseteq N$ indexed by the same indexing set as $\set{x_i}_{i\in I}$, then there is a unique module homomorphism $f\colon M\rightarrow N$ such that $f\left( x_i \right) = y_i$ for all $i$.
\end{theorem}
\begin{proof}
  Let $x = \sum_{i\in I}a_ix_i$, where the $\set{a_i}_{i\in I}$ are unique. Define
  \begin{align*}
    f(x) &= \sum_{i\in I}a_iy_i,
  \end{align*}
  which yields a unique homomorphism between $M$ and $N$.
\end{proof}
Note that we may take this homomorphism to be surjective, meaning that any module $N$ is a quotient $M/\ker\left( \varphi \right)$, where $M$ is a free module, and $\varphi\colon M\rightarrow N$ is a surjective module homomorphism.
\begin{corollary}
  Let $E$ be a finitely generated module, and let $E'$ be a submodule. Then, $E'$ is finitely generated.
\end{corollary}
\begin{proof}
  Let $\set{v_1,\dots,v_n}$ be generators for $E$. Since $E$ is a module, there is a free module $F$ with basis $\set{x_1,\dots,x_n}$ and a surjective module homomorphism $\varphi\colon F\rightarrow E$ such that $x_i\mapsto v_i$.\newline

  Then, $\varphi^{-1}\left( E' \right)\subseteq F$ is a submodule that is free and finitely generated, so $E'$ is finitely generated.
\end{proof}
\begin{theorem}
  Let $E$ be a finitely generated module with torsion submodule $E_{\tor}$. Then, $E/E_{\tor}$ is free, and there exists a free submodule $F$ of $E$ such that
  \begin{align*}
    E &= E_{\tor}\oplus F.
  \end{align*}
  Furthermore, the dimension of $F$ is uniquely determined.
\end{theorem}
\begin{proof}
  We start by proving that $E/E_{\tor}$ is torsion-free. For $x\in E$, write $\overline{x}$ to be the residue class modulo $E_{\tor}$. If $-\neq b\in R$ is such that $b\overline{x} = 0$, then $bx \in E_{\tor}$, so there exists $0\neq c\in R$ such that $cbx = 0$. Thus, $x\in E_{\tor}$, and $\overline{x} = 0$, meaning that $E/E_{\tor}$ is torsion-free (and finitely-generated, as it is a quotient of a finitely generated module).\newline

  Let $M$ be a finitely generated torsion-free module. Let $\set{y_1,\dots,y_m}$ be the set of generators of $M$, and let $\set{v_1,\dots,v_n}$ be a maximally linearly independent set. If $y\in \set{y_1,\dots,y_m}$, then there exist elements $a,b_1,\dots,b_n\in R$ not all zero such that
  \begin{align*}
    0 &= ay + b_1v_1 + \cdots + b_nv_n,
  \end{align*}
  and $a\neq 0$. Thus, $ay\in \left\langle v_1,\dots,v_n \right\rangle$, meaning that for any $j = 1,\dots,m$, we may find $a_j\in R$ with $a_j\neq 0$ and $a_jy_j\in \left\langle v_1,\dots,v_n \right\rangle$.\newline

  Letting $a = a_1\cdots a_m$ be the product, we have $aM \subseteq \left\langle v_1,\dots,v_m \right\rangle$ with $a\neq 0$, so the map $x\mapsto ax$ is an injective homomorphism whose image is contained in a free module isomorphic to $M$, meaning $M$ is a free module.\newline

  To obtain the free submodule $F$, we need a lemma.
  \begin{lemma}
    Let $E$ and $E'$ be modules, with $E'$ free. If $f\colon E\rightarrow E'$ is a surjective homomorphism, there is a free submodule $F$ of $E$ such that restricting $f$ to $F$ induces an isomorphism between $F$ and $E'$, with $E = F \oplus \ker\left( f \right)$.
  \end{lemma}
  \begin{proof}
    Let $\set{x_i'}_{i\in I}$ be a basis of $E'$. For each $i$, let $x_i$ be an element of $E$ such that $f\left( x_i \right) = x_i'$, and let $F = \left\langle \set{x_i}_{i\in I} \right\rangle$. The family $\set{x_i}_{i\in I}$ as selected is linearly independent, meaning $F$ is free.\newline

    Given $x\in E$, there exist $a_i\in R$ such that
    \begin{align*}
      f(x) &= \sum_{i\in I}a_ix_i',
    \end{align*}
    and $x - \sum_{i\in I}a_ix_i\in \ker\left( f \right)$. Thus, $E = \ker\left( f \right) + F$. Furthermore, by linear independence, $\ker\left( f \right)\cap F = \set{0}$, so the sum is direct.
  \end{proof}
  We apply this lemma to the homomorphism $E\rightarrow E/E_{\tor}$ to get the decomposition $E = E_{\tor}\oplus F$; since $F$ is isomorphic to $E/E_{\tor}$, the dimension is unique.
\end{proof}
\begin{definition}
  The dimension of $F$ in the decomposition $E = E_{\tor}\oplus F$ is known as the \textit{rank} of $E$.
\end{definition}
\begin{definition}
  Let $E$ be an $R$-module, and let $x\in E$. The map $a\mapsto ax$ is a homomorphism of $R$ onto $\left\langle x \right\rangle$, whose kernel is a principal ideal $\left\langle m \right\rangle\subseteq R$. We say $m$ is a \textit{period} of $x$. Note that $m$ is detemrined up to multiplication by a unit (assuming $m\neq 0$)\newline

  An element $0\neq c\in R$ is called an exponent for $E$ (for $x$) if $cE = 0$ (if $cx = 0$).\newline

  If $p$ is a prime element (i.e., an element of a prime ideal), we call $E(p)$ the submodule of $E$ that consists of all elements $x\in E$ such that $x$ has an exponent of $p^{r}$ for some $r\geq 1$.
\end{definition}
\begin{remark}
Modulo units, we select exactly one element of the prime ideal $\left\langle p \right\rangle$ for each of our prime elements.
\end{remark}
\begin{definition}
  If $0\neq m\in R$, we call $E_m$ the kernel of the map $x\mapsto mx$, which consists of all elements with exponent $m$.\newline

  The module $E$ is called \textit{cyclic} if it is isomorphic to $R/\left\langle a \right\rangle$ for some $a\in R$; if $a\neq 0$, then $a$ is a product of primes in our system of representatives, which we call the order of our module.\newline

  A $p$-module $E$ is said to be of type $\left( p^{r_1},\dots,p^{r_s} \right)$ if it is isomorphic to the product of modules $R/\left\langle p^{r_i} \right\rangle$, where $i = 1,\dots,s$.
\end{definition}
\begin{theorem}
  Let $E$ be a finitely generated nonzero torsion module. Then, $E$ is the direct sum
  \begin{align*}
    E &= \bigoplus_{p} E(p),
  \end{align*}
  where $p$ is prime and $E(p)\neq 0$. Each $E(p)$ is a direct sum of the form
  \begin{align*}
    E(p) &= R/\left\langle p^{v_1} \right\rangle\oplus\cdots\oplus R/\left\langle p^{v_s} \right\rangle,
  \end{align*}
  with $1\leq v_1 \leq \cdots \leq v_s$. The sequence $v_1,\dots,v_s$ is uniquely determined.
\end{theorem}
\begin{proof}
  Let $a$ be an exponent for $E$, and suppose $a = bc$ with $\left\langle b,c \right\rangle = R$ (i.e., that $b$ and $c$ have greatest common divisor $1$). Let $x,y\in R$ be such that
  \begin{align*}
    1 &= xb + yc.
  \end{align*}
  We claim that $E = E_b \oplus E_c$, where $E_b,E_c$ are the kernels of $x\mapsto bx$ and $x \mapsto cx$ respectively. The assertion as above follows from induction by expressing $a$ as a product of prime powers.\newline

  Let $v\in E$. Then,
  \begin{align*}
    v &= xbv + ycv.
  \end{align*}
  Note that $xbv\in E_{c}$, as $cxbv = xav = 0$. Similarly, $ycv\in E_{b}$. Finally, $E_b\cap E_c = \set{0}$ by the fact that $\left\langle b \right\rangle\cap \left\langle c \right\rangle = \set{0}$. Thus, $E = E_b\oplus E_c$.\newline

  Now, we prove that $E(p)$ is the above direct sum. We say the set $\set{y_1,\dots,y_m}\subseteq E$ are \textit{independent} if, whenever
  \begin{align*}
    a_1y_1 + \cdots + a_my_m &= 0
  \end{align*}
  for $a_i\in R$, we must have $a_iy_i = 0$ For all $i$ (note that our module has torsion). We see that $y_1,\dots,y_m$ are independent if and only if the module $\left\langle y_1,\dots,y_m \right\rangle$ has direct sum decomposition
  \begin{align*}
    \left\langle y_1,\dots,y_m \right\rangle &= \left\langle y_1 \right\rangle \oplus\cdots \oplus \left\langle y_m \right\rangle.
  \end{align*}
  
\end{proof}

\end{document}
