\documentclass[10pt]{mypackage}

% sans serif font:
%\usepackage{cmbright,sfmath,bbold}
%\renewcommand{\mathcal}{\mathtt}

%Euler:
%\usepackage{newpxtext,eulerpx,eucal,eufrak}
%\renewcommand*{\mathbb}[1]{\varmathbb{#1}}
%\renewcommand*{\hbar}{\hslash}

\renewcommand{\mathbb}{\mathds}
%\usepackage{homework}

\pagestyle{fancy} %better headers
\fancyhf{}
\rhead{Avinash Iyer}
\lhead{Modules over Principal Ideal Domains}

\setcounter{secnumdepth}{0}

\begin{document}
\RaggedRight
\begin{abstract}
  \noindent We show that if $E$ is a module defined over a principal ideal domain $R$, then $E$ is uniquely decomposable as $E \cong R^{r}\oplus R/\left\langle q_1 \right\rangle\oplus \cdots \oplus R/\left\langle q_n \right\rangle$, where $R^{r}$ is a free module of rank $r$, and $q_1 | q_2 | \cdots | q_n$.
\end{abstract}
\begin{definition}
  Let $A$ be a ring. A \textit{left $A$-module} $M$ is an abelian group with an operation of $A$ on $M$ such that
  \begin{align*}
    \left( a + b \right)x &= ax + bx\\
    a\left( x + y \right) &= ax + ay
  \end{align*}
  for all $a,b\in A$ and $x,y\in M$.\newline

  If $M$ is an $A$-module, then $N\subseteq M$ is known as a \textit{submodule} of $N$ is a subgroup such that $AN \subseteq N$.
\end{definition}

\end{document}
