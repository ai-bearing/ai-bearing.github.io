\documentclass[10pt]{mypackage}

% sans serif font:
%\usepackage{cmbright,sfmath,bbold}
%\renewcommand{\mathcal}{\mathtt}

%Euler:
%\usepackage{newpxtext,eulerpx,eucal,eufrak}
%\renewcommand*{\mathbb}[1]{\varmathbb{#1}}
%\renewcommand*{\hbar}{\hslash}

%\renewcommand{\mathbb}{\mathds}
%\usepackage{homework}

\pagestyle{fancy} %better headers
\fancyhf{}
\rhead{Avinash Iyer}
\lhead{Egorov's Theorem and Lusin's Theorem}

\setcounter{secnumdepth}{0}

\begin{document}
\RaggedRight
\begin{abstract}
  \noindent The mathematician J.E. Littlewood introduced three principles of real analysis:
  \begin{enumerate}[(1)]
    \item every measurable set is nearly a finite union of intervals;
    \item every measurable function is nearly continuous;
    \item every pointwise convergent sequence is nearly uniformly convergent.
  \end{enumerate}
  Here, we prove (ii) and (iii), which are the substance of Lusin's Theorem and Egorov's Theorem respectively.
\end{abstract}
\end{document}
