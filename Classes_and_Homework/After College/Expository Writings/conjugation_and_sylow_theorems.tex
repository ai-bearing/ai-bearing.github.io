\documentclass[10pt]{mypackage}

% sans serif font:
%\usepackage{cmbright,sfmath,bbold}
%\renewcommand{\mathcal}{\mathtt}

%Euler:
%\usepackage{newpxtext,eulerpx,eucal,eufrak}
%\renewcommand*{\mathbb}[1]{\varmathbb{#1}}
%\renewcommand*{\hbar}{\hslash}

%\renewcommand{\mathbb}{\mathds}
\usepackage{homework}
\usepackage{exposition}
\usepackage{microtype}

\pagestyle{fancy} %better headers
\fancyhf{}
\rhead{Avinash Iyer}
\lhead{Conjugation and the Sylow Theorems}

\setcounter{secnumdepth}{0}

\begin{document}
\RaggedRight
\begin{abstract}
  \noindent We discuss the nuances of the conjugation action in groups, and use it to prove the Sylow theorems. We then use the Sylow theorems to classify the nature of groups of a particular order.
\end{abstract}
\section{Introduction to Conjugation}%
Every transitive left-action of a group on a set $S$ is, up to isomorphism, left-multiplication on the set of left-cosets of $G/\stab_G(a)$, where $\stab_G(a)$ denotes the stabilizer subgroup of $a\in S$. Furthermore, the number of elements of a finite orbit of $a\in O_a$ is the index of $\Stab_G(a)$ --- this is the much celebrated \textit{orbit-stabilizer theorem}.\newline

Note that from the orbit-stabilizer theorem, we can partition $S$ into a formula involving the conjugacy classes. Since every element of $s$ is either in an orbit or is in the set
\begin{align*}
  Z\coloneq \set{a\in S | g\cdot a = a\text{ for all $g\in G$}},
\end{align*}
we calculate
\begin{align*}
  \left\vert S \right\vert &= \left\vert Z \right\vert + \sum_{a\in A} \left\vert O_a \right\vert\\
                           &= \left\vert Z \right\vert + \sum_{a\in A} \left[ G:\Stab_G(a) \right],
\end{align*}
where $A$ is a system of representatives for the orbits. This is \textsl{a} class formula for the action of $G$ on $S$.\newline

The power of this class formula is that when $G$ is finite, $\left[ G:\Stab_G(a) \right]$ always divides $G$, which is a very strong constraint when we know something about $\left\vert G \right\vert$.
\begin{proposition}
  Let $\left\vert G \right\vert = p^{n}$ be a group that acts on a finite set $S$, and let $Z$ be the set of fixed points for the action. Then, $\left\vert Z \right\vert\equiv \left\vert S \right\vert$ modulo $p$.
\end{proposition}
\begin{proof}
  Since each summand of the form $\left[ G:\Stab_G(a) \right]$ is a number larger than $1$ and a power of $p$, each $\left[ G:\Stab_G(a) \right]$ is congruent to $0$ mod $p$.
\end{proof}
\begin{definition}[Conjugation Action]
  Let $G$ be a group. The \textit{conjugation action} of $G$ on itself is defined by $\rho\colon G\times G \rightarrow G$, where
  \begin{align*}
    \rho\left( g,a \right) &= gag^{-1}.
  \end{align*}
  This map is equal to a particular group homomorphism $\sigma\colon G\rightarrow \sym(G)$.
\end{definition}
\begin{definition}[Center]
  The \textit{center} of $G$, denoted $Z(G)$, is the subgroup $\ker\left( \sigma \right)\subseteq G$. Concretely, it is
  \begin{align*}
    Z(G)&= \set{g\in G | ga = ag\text{ for all $a\in G$}}.
  \end{align*}
\end{definition}
Note that $Z(G)$ is always a normal subgroup, and all elements of $Z(G)$ commute with each other. Furthermore, a group $G$ is abelian if and only if $Z(G) = G$.
\begin{lemma}
  Let $G$ be a finite group, and suppose $G/Z(G)$ is cyclic. Then, $G$ is commutative.
\end{lemma}
\begin{proof}
  Write $Z\coloneq Z(G)$, and suppose $G/Z$ is cyclic. Then, there is some $g\in G$ such that $\left\langle gZ \right\rangle = G/Z$. For all $a\in G$, there is some integer $r$ such that
  \begin{align*}
    aZ &= g^{r}Z,
  \end{align*}
  meaning there exists some $z\in Z$ such that $a = g^{r}z$. Similarly, we write $b = g^{s}w$ for some $w\in Z$ and integer $s$. However, this means
  \begin{align*}
    ab &= \left( g^{r}z \right)\left( g^{s}w \right)\\
       &= g^{r+s}zw\\
       &= \left( g^{s}w \right)\left( g^{r}z \right)\\
       &= ba,
  \end{align*}
  where we use the fact that $z$ and $w$ commute with every element of $G$.
\end{proof}
\begin{definition}
  Let $a\in G$. The \textit{centralizer} of $a$, denoted $Z_G(a)$, is the stabilizer of $a$ under conjugation. Concretely,
  \begin{align*}
    Z_G(a) &= \set{g\in G | ga = ag},
  \end{align*}
  or the set of elements of $G$ that commute with $a$.\newline

  Note that $Z(G)\subseteq Z_G(a)$ for all $a\in G$, and that
  \begin{align*}
    Z(G) &= \bigcap_{a\in G} Z_G(a).
  \end{align*}
\end{definition}
\begin{definition}
  The \textit{conjugacy class} of $a\in G$ is the orbit $\left[ a \right]$ of $a$ under conjugation.
\end{definition}
\section{The Class Equation}%
What we call \textsl{the} class equation is generally the class formula for conjugation.
\begin{definition}
  Let $G$ be a finite group. Then,
  \begin{align*}
    \left\vert G \right\vert &= \left\vert Z(G) \right\vert + \sum_{a\in A} \left[ G:Z_G(a) \right],
  \end{align*}
  where $A$ is a family of representatives of conjugacy classes in $G$. This is known as the \textit{class equation} for the group $G$.
\end{definition}
We are very easily able to apply the class equation to prove certain properties about $p$-groups.
\begin{proposition}
  Let $G$ be a nontrivial $p$-group. Then, $G$ has a nontrivial center.
\end{proposition}
\begin{proof}
  Since $\left\vert Z(G) \right\vert \equiv \left\vert G \right\vert$ modulo $p$, and $\left\vert G \right\vert > 1$ is a power of $p$, we have $\left\vert Z(G) \right\vert$ is a multiple of $p$. Since $e_G\in Z(G)$, we know that $\left\vert Z(G) \right\vert \geq p$.
\end{proof}
\begin{exercise}
  Let $p,q$ be prime numbers, and let $G$ be a group of order $pq$. Prove that either $G$ is commutative or the center of $G$ is trivial.\newline

  Conclude that every group of order $p^2$ is commutative.
\end{exercise}
\begin{solution}
  From the class equation, we know that
  \begin{align*}
    pq &= \left\vert Z(G) \right\vert + \sum_{a\in A} \left[ G:Z_G(a) \right].
  \end{align*}
  If $\left\vert Z(G) \right\vert  = pq$, then $G$ is abelian.\newline

  Suppose toward contradiction that, without loss of generality, $\left\vert Z(G) \right\vert = p$. Then, $\left\vert G/Z(G) \right\vert = q$, meaning $G/Z(G)$ is cyclic, so $G$ is abelian, so $Z(G) = G$. $\bot$\newline

  Since, in any $p$-group, $\left\vert Z(G) \right\vert \geq p$, we must have that $Z(G) = G$, so $G$ is abelian.
\end{solution}
\begin{example}
  Let $G$ be a group of order $6$. What are the possibilities for its class formula?\newline

  In general, if $G$ is commutative, the class formula doesn't say much --- in this case, it says $6 = 6$.\newline

  If $G$ is not commutative then its center is trivial ($ 6 = 2\times 3 $), so we have $6 = 1 + \cdots$. Specifically, the $\cdots$ refers to the sizes of the conjugacy classes, which must be smaller than $6$, greater than $1$, and divide $6$. Thus, we have
  \begin{align*}
    6 &= 1 + 2 + 3
  \end{align*}
  as the only possible class equation for a noncommutative group with six elements.
\end{example}
Note that normal subgroups are unions of conjugacy classes.\footnote{This is used to show that $A_5$ is simple, for those of us studying for qualifiers.} If $H$ is a normal subgroup, and $a\in H$ with $b = gag^{-1}$ conjugate to $a$, we must have $b\in gHg^{-1} = H$.\newline

Furthermore, every subgroup must have the identity and its size must divide the order of the group; therefore, a noncommutative group of order $6$ cannot have any subgroup of order $2$, since $2$ cannot be written as a sum of orders of conjugacy classes including the center.
\begin{definition}
  Let $A\subseteq G$ be a subset, $g\in G$ an element. The \textit{conjugate} of $A$ is the subset $gAg^{-1}$; the map $a\mapsto gag^{-1}$ is a bijection between $A$ and $gAg^{-1}$.
\end{definition}
\begin{definition}
  Let $A\subseteq G$ be a subset. The \textit{normalizer} $N_G(A)$ is its stabilizer under conjugation. The \textit{centralizer} of $A$ is the subgroup $Z_G(A)\subseteq N_G(A)$ that fixes each element of $A$.
\end{definition}
Therefore, $g\in N_G(A)$ if and only if $gAg^{-1} = A$, and $g\in Z_G(A)$ if and only if $gag^{-1} = a$ for all $a\in A$. If $A = \set{a}$, then $N_G(A) = Z_G(A) = Z_G(a)$. However, in general, $Z_G(A)\subsetneq N_G(A)$.\newline

If $H$ is a subgroup of $G$, then every conjugate $gHg^{-1}$ of $H$ is also a subgroup of $G$; all conjugate subgroups have the same order.
\begin{lemma}
  Let $H\subseteq G$ be a group. Then, the number of subgroups conjugate to $H$ equals the index $\left[ G:N_G(H) \right]$ of the normalizer of $H$ in $G$.
\end{lemma}
\begin{proof}
  Considering the group's self-action of conjugation, this follows from the orbit-stabilizer theorem.
\end{proof}
\begin{corollary}
  If $\left[ G:H \right]$ is finite, then the number of subgroups conjugate to $H$ is finite and divides $\left[ G:H \right]$.
\end{corollary}
\begin{proof}
  \begin{align*}
    \left[ G:H \right] &= \left[ G:N_G(H) \right]\left[ N_G(H):H \right].
  \end{align*}
\end{proof}
A useful fact is that if $H$ and $K$ are subgroups of $G$ such that $H\subseteq N_G(K)$ --- i.e., that $gKg^{-1} = K$ for all $g\in H$ --- then conjugation by $g\in H$ gives an automorphism of $K$. Thus, there is a set function
\begin{align*}
  \gamma\colon H\rightarrow \aut\left( K \right).
\end{align*}
\begin{exercise}
  Let $H$ and $K$ be subgroups of $G$ with $H\subseteq N_G(K)$. Verify that the function $\gamma\colon H\rightarrow \aut(K)$ defined by conjugation is a homomorphism of groups, and $\ker\left( \gamma \right) = H\cap Z_G(K)$, where $Z_G(K)$ is the centralizer of $K$.
\end{exercise}
\begin{solution}
  Let $a,b\in H$. Then,
  \begin{align*}
    \gamma\left( ab^{-1} \right) &= \left( ab^{-1} \right) K \left( ab^{-1} \right)^{-1}\\
                                 &= ab^{-1} K ba^{-1}\\
                                 &= \left( aKa^{-1} \right) \left( b^{-1}Kb \right)\\
                                 &= \left( aKa^{-1} \right)\left( bKb^{-1} \right)^{-1}\\
                                 &= \gamma\left( a \right)\gamma\left( b \right)^{-1}.
  \end{align*}
  The kernel of $\gamma$ consists of all those elements of $H$ that map to $\id\colon K\rightarrow K$ --- i.e., those that map to the centralizer of $K$. Therefore, $\ker\left( \gamma \right)$ consists of $H\cap Z_G(K)$.
\end{solution}
\section{The Sylow Theorems}%
The Sylow Theorems are a collection of theorems that concern $p$-subgroups of a certain finite group $G$. The first of the theorems says that $G$ contains $p$-groups of all sizes allowed by Lagrange's theorem.
\begin{theorem}[Cauchy's Theorem]
  Let $G$ be a finite group, and let $p$ be a prime divisor of $\left\vert G \right\vert$. Then, $G$ contains an element of order $p$.
\end{theorem}
We can show the abelian case as an exercise.
\begin{exercise}
  Suppose $G$ is a finite abelian group, and let $p$ be a prime divisor of $\left\vert G \right\vert$. Prove there exists an element of order $p$.
\end{exercise}
\begin{solution}
  Let $g\in G$. Then, $\left\langle g \right\rangle\subseteq G$ is a cyclic subgroup, meaning that it is of order $k$, where $k$ divides $\left\vert G \right\vert$. Assuming $k\geq 2$, we may write $k = p_1^{e_1}\cdots p_n^{e_n}$, where $p_i$ are prime. Then, for some prime $q $ such that $q$ divides $ k$, we may take the prime subgroup $\left\langle h \right\rangle \coloneq \left\langle g^{k/q} \right\rangle$.\newline

  Now, if $q = p$, we are done; else, we may take $G/\left\langle h \right\rangle$, which has order $\left\vert G \right\vert/q$, and since $p\neq q$, we may commence with the same process on $G/\left\langle h \right\rangle$.
\end{solution}
However, we can also show the general case.
\begin{proof}
  Let $S$ be the set of ordered $p$-tuples of elements of $G$, $\left( a_1,\dots,a_p \right)$, such that $a_1\cdots a_p = e$. Then, $\left\vert S \right\vert = \left\vert G \right\vert^{p-1}$, as we may choose $a_1,\dots,a_{p-1}$ arbitrarily, and select $a_p$ to be the inverse of $a_1\cdots a_{p-1}$.\newline

  Since $p$ divides the order of $S$, it divides the order of $G$. Note that if $a_1,\dots,a_p = e$, then
  \begin{align*}
    a_2\cdots a_pa_1 = e,
  \end{align*}
  as $a_1$ is a left-inverse to $a_2\cdots a_p$, so it is a right inverse. Therefore, we may act $\Z/p\Z$ on $S$ by taking $\left[ m \right]$ to act on $\left( a_1,\dots,a_p \right)$ yielding $\left( a_{m+1},\dots, a_p,a_1,\dots,a_m \right)$; this yields an element of $S$.\newline

  Thus, by the general class equation, we have $\left\vert Z \right\vert \equiv \left\vert S \right\vert \equiv 0$ modulo $p$, where $Z$ is the set of fixed points of the action.\newline

  In particular, fixed points are $p$-tuples of the form $\left( a,\dots,a \right)$; note that $Z\neq \emptyset$, since $\left( e,\dots,e \right)\in Z$. Thus, there is some element in $Z$ of the form $\left( a,\dots,a \right)$ with $a\neq e$.\newline

  In particular, this means there is $a\in G$ with $a\neq e$ and $a^p = e$.
\end{proof}
\begin{corollary}
  If $p$ is a prime divisor of $\left\vert G \right\vert$, and $N$ is the number of cyclic subgroups of order $p$, then $N\equiv 1$ mod $p$.
\end{corollary}
\begin{proof}
  Since $\left\vert Z \right\vert$ in the proof of Cauchy's theorem is congruent to $0$ modulo $p$, we must have $\left\vert Z \right\vert = mp$ for some $m \geq 1$. Now, since $e_G\in Z$ but $e_G$ is not of order $p$, we must have $mp - 1$ elements of order $p$ in $G$.\newline

  Since it takes $p-1$ elements of order $p$ to yield a cyclic subgroup of order $p$, we thus have $N = \frac{mp-1}{p-1}\equiv \frac{-1}{-1} = 1$ mod $p$.
\end{proof}
\begin{exercise}
  Let $G$ be a group. A subgroup $H$ of $G$ is called \textit{characteristic} if $\varphi\left( H \right) \subseteq H$ for all automorphisms $\varphi$ of $G$.
  \begin{enumerate}[(a)]
    \item Prove that characteristic subgroups are normal.
    \item Let $H\subseteq K \subseteq G$, with $H$ characteristic in $K$ and $K$ normal in $G$. Prove that $H$ is normal in $G$.
    \item Let $G,K$ be groups, and suppose $G$ contains a single subgroup $H$ isomorphic to $K$. Prove that $H$ is normal in $G$.
    \item Let $K$ be a normal subgroup of a finite  group $G$, and assume $\left\vert K \right\vert$ and $\left\vert G/K \right\vert$ are relatively prime. Prove that $K$ is characteristic in $G$.
  \end{enumerate}
\end{exercise}
\begin{solution}\hfill
  \begin{enumerate}[(a)]
    \item Since conjugation is an automorphism, we have $gHg^{-1}\subseteq H$ for each $g\in G$, meaning $H$ is normal.
    \item Since $K$ is normal in $G$, $K$ is preserved under conjugation by elements of $G$, so conjugation by elements of $G$ is an automorphism of $K$. Thus $H$ is preserved by conjugation of elements in $G$, so $H$ is normal in $G$.
    \item Let $\varphi\colon G\rightarrow K$ be a surjective homomorphism such that $\varphi(H)\cong K$. Then, 
      \begin{align*}
        \varphi\left( gHg^{-1} \right) &= \varphi\left( g \right)\varphi\left( H \right)\varphi\left( g \right)^{-1}\\
                                       &= K,
      \end{align*}
      so $gHg^{-1} = H$, meaning $H$ is normal.
    \item Let $\left\vert K \right\vert$ and $\left\vert G/K \right\vert$ be relatively prime. Let $\varphi\in \Aut\left( G \right)$. Then, by the second isomorphism theorem, we have that
      \begin{align*}
        \frac{K\varphi(K)}{K} &\cong \frac{\varphi(K)}{\varphi(K)\cap K}\\
                              &\coloneq H.
      \end{align*}
      Therefore, we have that $\left\vert H \right\vert$ divides $\left\vert \varphi(K) \right\vert = \left\vert K \right\vert$. However, at the same time, we also have
      \begin{align*}
        \left\vert \frac{K\varphi(K)}{K} \right\vert &= \left[ K\varphi(K):K \right]\\
                                                     &= \frac{\left[ G:K \right]}{\left[ G:K\varphi(K) \right]},
      \end{align*}
      meaning that $\left\vert H \right\vert$ divides $\left[ G:K \right]$, so $\left\vert H \right\vert = 1$, and $f(K)\cap K \subseteq K$, so $f(K)\subseteq K$.
  \end{enumerate}
\end{solution}
Using part (c) of the exercise, we can see that if there is only one cyclic subgroup $H$ of order $p$, then that subgroup must be normal.
\begin{definition}
  A group $G$ is called \textit{simple} if the only normal subgroups of $G$ are $\set{e}$ and $G$.
\end{definition}
\begin{example}
  Let $p$ be a positive prime integer. If $\left\vert G \right\vert = mp$ with $1 < m < p$, then $G$ is not simple.\newline

  Consider the subgroups of $G$ with $p$ elements. Then, the number of these subgroups is congruent to $1$ modulo $p$, so if there is more than one such subgroup, there must be at least $p + 1$. Any two distinct subgroups of prime order can only have trivial intersection (else, since both subgroups of prime order are cyclic, they would coincide on a generator), so this accounts for at least
  \begin{align*}
    p^2 &= 1 + \left( p+1 \right)\left( p-1 \right)
  \end{align*}
  elements in $G$. However, since $\left\vert G \right\vert = mp < p^2$, this is not possible. Therefore, there is only one cyclic subgroup of order $p$ in $G$, which must be normal, so $G$ is not simple.
\end{example}
Now, we can start on the Sylow theorems, the first of which generalizes the result from Cauchy's theorem.
\begin{definition}
  Let $p$ be a prime number. A $p$-Sylow subgroup of a finite group $G$ is a subgroup of order $p^{r}$, where $\left\vert G \right\vert = p^{r}m$ and $\gcd\left( p,m \right) = 1$.\newline

  In other words, $P\subseteq G$ is a ``maximal'' $p$-group, in the sense that $p$ does not divide $\left[ G:P \right]$.
\end{definition}
\begin{theorem}[First Sylow Theorem]
  Every finite group contains a $p$-Sylow subgroup for all primes $p$.
\end{theorem}
We may prove this from a more general result --- that if $p^{k}$ divides the order of $G$, then $G$ has a subgroup of $p^{k}$. But first, an exercise.
\begin{exercise}
  Let $p$ be a prime number, and let $G$ be a $p$-group, with $\left\vert G \right\vert = p^{r}$. Prove that $G$ contains a normal subgroup of order $p^{k}$ for all $0\leq k\leq r$.
\end{exercise}
\begin{proposition}
  If $p^{k}$ divides the order of $G$, then $G$ has a subgroup of order $p^{k}$
\end{proposition}
\begin{proof}
  If $k = 0$, there is nothing to prove, so we may assume that $k\geq 1$, and that $\left\vert G \right\vert$ is a multiple of $p$. We argue by induction on $\left\vert G \right\vert$. If $\left\vert G \right\vert = p$, there is nothing to prove, and if $\left\vert G \right\vert > p$ with subgroup $H\subseteq G$ where $\left[ G:H \right]$ is relatively prime to $p$, then $p^{k}$ divides the order of $H$, and $H$ has a subgroup of order $p^{k}$ by the inductive hypothesis.\newline

  Thus, we assume that all proper subgroups of $G$ have index divisible by $p$. By the class equation, $p$ divides the order of $Z(G)$, so by Cauchy's theorem, there exists $a\in Z(G)$ such that $a$ has order $p$. The cyclic subgroup $N\coloneq \left\langle a \right\rangle$ is contained in $Z(G)$, so it is normal on $G$.\newline

  Since $\left\vert G/N \right\vert = \left\vert G \right\vert/p$, and $p^{k}$ divides $\left\vert G \right\vert$, we have that $p^{k-1}$ divides $\left\vert G/N \right\vert$, so by the induction hypothesis, $G/N$ has a subgroup of order $p^{k-1}$. This subgroup must be of the form $P/N$ for some subgroup $P$ of $G$.\newline

  Therefore, we have $\left\vert P \right\vert = \left\vert P/N \right\vert\left\vert N \right\vert = p^{k}$.
\end{proof}
The second and third Sylow theorems are arguably more powerful than the first Sylow theorem. Specifically, the second Sylow theorem shows that all the $p$-Sylow subgroups are conjugates, and that every $p$-group is contained in a conjugate of some fixed $p$-Sylow subgroup. But first, an exercise.
\begin{exercise}
  Let $p$ be prime, let $G$ be a $p$-group, and let $S$ be such that $p \nmid \left\vert S \right\vert$. If $G$ acts on $S$, show that the action must have a fixed point.
\end{exercise}
\begin{solution}
  From the generalized class formula, we have
  \begin{align*}
    \left\vert S \right\vert &= \left\vert Z \right\vert + \sum_{a\in A} \left[ G:G_a \right].
  \end{align*}
  Taking the modulus on both sides, the left side is some nonzero value, while $\sum_{a\in A}\left[ G:G_a \right]$ is zero mod $p$ (as $G$ is a $p$-group and stabilizers are subgroups). Thus, $\left\vert Z \right\vert\neq 0$.
\end{solution}
\begin{theorem}[Second Sylow Theorem]
  Let $G$ be a finite group, and let $P$ be a $p$-Sylow subgroup. Let $H\subseteq G$ be a $p$-group. Then, $H$ is contained in a conjugate of $P$ --- i.e., there exists $g\in G$ such that $H\subseteq gPg^{-1}$.
\end{theorem}
\begin{proof}
  Use $H$ to act on the left-cosets of $P$ by left-multiplication. Since there are $\left[ G:P \right]$ cosets, and $p$ does not divide $\left[ G:P \right]$, the action must have a fixed point.\newline

  Let $gP$ be a fixed point. Then, for all $h\in H$, 
  \begin{align*}
    hgP &= gP,
  \end{align*}
  or that $g^{-1}hgP = P$ for all $h\in H$. Thus, $g^{-1}Hg\subseteq P$, and $H\subseteq gPg^{-1}$.
\end{proof}
Consider a chain
\begin{align*}
  \set{e} = H_0\subseteq H_2\subseteq \cdots \subseteq H_k
\end{align*}
of $p$-subgroups of $G$, with $\left\vert H_i \right\vert = p^{i}$. Then, $H_k$ is contained in some $p$-Sylow subgroup of order $p^{r}$, where $p^{r}$ is the maximum power of $p$ dividing the order of $G$. However, we claim that the chain can be continued all the way to the form
\begin{align*}
  \set{e} = H_0 \subseteq H_1\subseteq \cdots \subseteq H_k\subseteq H_{k+1}\subseteq \cdots \subseteq H_{r},
\end{align*}
and that $H_{k}$ is normal in $H_{k+1}$.
\begin{lemma}
  Let $H$ be a $p$-group contained in a finite group $G$. Then,
  \begin{align*}
    \left[ N_G(H) : H \right] \equiv \left[ G:H \right]
  \end{align*}
  modulo $p$.
\end{lemma}
\begin{proof}
  If $H$ is trivial, then $N_G(H) = G$.\newline

  Let $H$ be nontrivial, and act $H$ on the set of left cosets of $H$ by left-multiplication. The fixed points of this action are the cosets $gH$ such that
  \begin{align*}
    hgH &= gH,
  \end{align*}
  for all $h\in H$, or that $g^{-1}hg\in H$ for all $h\in H$; specifically, this means the set of all $g\in G$ such that $g H g^{-1} = H$, or that $g\in N_G(H)$. The statement then follows from the fact that $\left\vert Z \right\vert \equiv \left\vert S \right\vert$ mod $p$ if $S$ is a set and $Z$ is the set of fixed points.
\end{proof}
Thus, if $H_k$ is not a $p$-Sylow subgroup --- i.e., $p$ divides $\left[ G:H_k \right]$ --- then $p$ must also divide $\left[ N_G\left(H_k\right) : H_k \right]$.
\begin{proposition}
  Let $H$ be a $p$-subgroup of a finite group $G$, and assume $H$ is not a $p$-Sylow subgroup. Then, there exists a $p$-subgroup $H'$ of $G$ containing $H$ such that $\left[ H':H \right] = p$ and $H$ is normal in $H'$.
\end{proposition}
\begin{proof}
  Since $H$ is not a $p$-Sylow subgroup, $p$ divides $\left[ N_G(H):H \right]$. Since $H$ is normal in $N_G(H)$, we consider the quotient group $N_G(H)/H$; then, $p$ divides the order of this group. Thus, by Cauchy's theorem, $N_G(H)/H$ has an element of order $p$, which generates a subgroup of order $p$ of $N_G(H)/H$. This subgroup has the form $H'/H$ for some subgroup $H'$ of $N_G(H)$.\newline

  By construction, $\left[ H':H \right] = p$, and $H$ is normal in $H'$ since $H$ is normal in $N_G(H)$.
\end{proof}
The third Sylow theorem is probably the most commonly used of the bunch --- it allows us to control the number of $p$-Sylow subgroups of $G$, and also allows us to establish the existence of normal subgroups in $G$.
\begin{theorem}
  Let $p$ be a prime number, and let $G$ be a finite group such that $\left\vert G \right\vert = p^{r} m$ with $p$ not dividing $m$. Then, the number of $p$-Sylow subgroups divides $m$ and is congruent to $1$ modulo $p$.
\end{theorem}
\begin{proof}
  The $p$-Sylow subgroups are conjugates of any given $p$-Sylow subgroup $P$. If we let $N_p$ be the number of $p$-Sylow subgroups in $G$, then we know that $N_p = \left[ G:N_G(P) \right]$. This number must divide the index of $P$,
  \begin{align*}
    m &= \left[ G:P \right]\\
      &= \left[ G:N_G(P) \right]\left[ N_G(P):P \right]\\
      &= N_p\left[ N_G(P):P \right].
  \end{align*}
  Now, we have $\left[ G:P \right]\equiv \left[ N_G(P):P \right]$ mod $p$, so multiplying by $N_p$, we have $mN_p \equiv m$ mod $p$. Thus, $N_p\equiv 1$ mod $p$.
\end{proof}
\begin{exercise}
  Let $P$ be a $p$-Sylow subgroup of a finite group $G$, and let $H\subseteq G$ be a $p$-subgroup. Assume $H\subseteq N_{G}\left( P \right)$. Show that, $H\subseteq P$.
\end{exercise}
\begin{solution}
  Let $H\subseteq N_G(P)$. Since $P$ is normal on $N_G(P)$, we have that $PH$ is a subgroup of $N_G(P)$. Then, $\left\vert PH/P \right\vert = \left\vert H/\left( P\cap H \right) \right\vert$ by the second isomorphism theorem. Since $H$ is a $p$-group, $H/\left( P\cap H \right)$ is also a $p$-group, so $\left\vert PH/P \right\vert$ is a power of $p$. However, since $P$ is a $p$-Sylow subgroup of $G$, we must have $PH = P$, as otherwise, $PH$ would be a $p$-group of order greater than $P$. Thus, we have $H\subseteq P$.
\end{solution}
\begin{exercise}
  Let $P$ be a $p$-Sylow subgroup of a finite group $G$, and act with $P$ by conjugation on the set of $p$-Sylow subgroups of $G$. Show that $P$ is the unique fixed point of this action.
\end{exercise}
\begin{solution}
  Let $\mathcal{S}$ be the set of $p$-Sylow subgroups of $G$, and let $P$ act on $\mathcal{S}$ by conjugation. Suppose there exists $Q$ such that for all $p\in P$, $pQp^{-1} = Q$. Then, $P\subseteq N_{G}\left( Q \right)$, so $P\subseteq Q$; however, since all the $p$-Sylow subgroups are the same order, $P = Q$.
\end{solution}
\begin{exercise}
  Use the second Sylow Theorem, the previous exercise, and the fact that the number of subgroups conjugate to a subgroup divides $\left[ G:H \right]$ to prove the third Sylow theorem.
\end{exercise}
\begin{solution}
  Note that $\left[ G:P \right] = m$ for any $p$-Sylow subgroup, so
  \begin{align*}
    m &= \left[ G:N_G(P) \right]\left[ N_G(P):P \right].
  \end{align*}
  Since $\left[ N_G(P):P \right]$ is the number of $p$-Sylow subgroups conjugate to $P$, and all $p$-Sylow subgroups are conjugate to each other, we have that the number of $p$-Sylow subgroups divides $m$.\newline

  If $\mathcal{S}$ denotes the set of $p$-Sylow subgroups, then $\left\vert \mathcal{S} \right\vert = \left\vert Z \right\vert $ mod $p$ when $P$ acts on $\mathcal{S}$ with conjugation; since there is only one fixed point in this action, we must have the number of $p$-Sylow subgroups is congruent to $1$ mod $p$.
\end{solution}
\section{Applications of the Sylow Theorems}%
The main application of the Sylow theorems is to use the fact that the number of $p$-Sylow subgroups divides the $m$ where $\left\vert G \right\vert = p^{r}m$ and is congruent to $1$ mod $p$ to show that certain groups are simple or abelian.
\begin{claim}
  Let $G$ be a group of order $mp^{r}$, where $p$ is a prime number, and $1 < m < p$. Then, $G$ is not simple.
\end{claim}
\begin{proof}
  By the third Sylow theorem, the number of $p$-Sylow subgroups divides $m$ and is of the form $1 + kp$. Since $m < p$, we must have $k = 0$, $N_p = 1$, and $G$ has a normal subgroup of order $p^{r}$.
\end{proof}
\begin{example}
  There are no simple groups of order $2002$. Note that
  \begin{align*}
    2002 &= 2\cdot 7 \cdot 11 \cdot 13,
  \end{align*}
  and that the divisors of $2\cdot 7 \cdot 13$ are $1,2,7,13,14,26,91,182$, of which only $1$ is congruent to $1$ mod $11$. Thus, there is a normal subgroup of order $11$ in every group of order $2002$>
\end{example}
\begin{example}
  There are no simple groups of order $12$. This requires a bit more work; since $3\equiv 1$ mod $2$ and $4\equiv 1$ mod $3$, we can't use the above argument so easily. Suppose there is more than one $3$-Sylow subgroup --- then, there must be $4$, by the third Sylow theorem. These subgroups intersect at the identity, meaning there are exactly $8$ elements of order $3$. Excluding these elements leaves us with the identity and $3$ elements of order $2$ or $4$, which can only accommodate exactly one $2$-Sylow subgroup, which is necessarily normal.\newline

  Thus, there is either a normal $3$-Sylow subgroup, or a normal $2$-Sylow subgroup. Either way, the group is not simple.
\end{example}
\begin{example}
  There are no simple groups of order $24$. If $G$ is a group of order $24$, then there are either $1$ or $3$ $2$-Sylow subgroups. If there is $1$, then the $2$-Sylow subgroup is normal, and $G$ is not simple. Else, $G$ acts nontrivially by conjugation by conjugation on the set of three $2$-Sylow subgroups, which gives a nontrivial homomorphism $G\rightarrow S_3$ whose kernel is a proper nontrivial normal subgroup, meaning $G$ is still not simple.
\end{example}
Similarly, we can use this to solve some other exercises.
\begin{exercise}
  Let $G$ be a group of order $30$.
  \begin{enumerate}[(a)]
    \item Show that there is a normal subgroup of order $3$ or a normal subgroup of order $5$.
    \item Show that there is a normal subgroup of order $15$.
  \end{enumerate}
\end{exercise}
\begin{solution}\hfill
  \begin{enumerate}[(a)]
    \item Using the prime factorization $30 = 2\cdot 3 \cdot 5$, we note that there are either $1$ or $10$ $3$-Sylow subgroups, and there are either $1$ or $6$ $5$-Sylow subgroups. If there are $1$ of either, then we are done. If there are $10$ $3$-Sylow subgroups, these subgroups intersect at the identity, giving $20$ order $3$ elements; since there must be a $2$-Sylow subgroup and a $5$-Sylow subgroup, there can only be one of each of these subgroups, meaning that there is a normal subgroup of order $5$. Else, if there are $6$ $5$-Sylow subgroups, there are $24$ order $5$ elements, meaning that there is exactly one $3$-Sylow subgroup, which is necessarily normal.
    \item We note that any subgroup of order $15$ is normal, as it has index $2$. Therefore, we must show that there exists some subgroup of order $15$. Since there is one $3$-Sylow subgroup and one $5$-Sylow subgroup, their product has order $15$, which is normal.
  \end{enumerate}
\end{solution}
%\begin{exercise}
%  Assume that $G$ is a simple group of order $60$.
%  \begin{enumerate}[(a)]
%    \item Use the Sylow theorems and numerology to prove that $G$ has either five or fifteen $2$-Sylow subgroups, accounting for fifteen elements of order $2$ or $4$.
%    \item If there are fifteen $2$-Sylow subgroups, prove that there exists an element $g\in G$ of order $2$ contained in at least two of them. Prove that the centralizer of $g$ has index $5$.
%  \end{enumerate}
%  Conclude that every simple group of order $60$ contains a subgroup of index $5$.
%\end{exercise}
%\begin{solution}\hfill
%  \begin{enumerate}[(a)]
%    \item We note that $60 = 2^2\cdot 3 \cdot 5$. Then, we must have the number of $2$-Sylow subgroups equal to either $1$, $3$, $5$, and $15$. We will show that if there are $3$ $2$-Sylow subgroups, then there is a normal subgroup.
%  \end{enumerate}
%\end{solution}

\end{document}
