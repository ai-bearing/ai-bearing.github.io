\documentclass[10pt]{mypackage}

% sans serif font:
%\usepackage{cmbright,sfmath,bbold}
%\renewcommand{\mathcal}{\mathtt}

%Euler:
%\usepackage{newpxtext,eulerpx,eucal,eufrak}
%\renewcommand*{\mathbb}[1]{\varmathbb{#1}}
%\renewcommand*{\hbar}{\hslash}

%\renewcommand{\mathbb}{\mathds}
%\usepackage{homework}
\usepackage{exposition}
\pgfplotsset{compat=1.18}

\pagestyle{fancy} %better headers
\fancyhf{}
\rhead{Avinash Iyer}
\lhead{The Lebesgue Measure}

\setcounter{secnumdepth}{0}

\begin{document}
\RaggedRight
\begin{abstract}
  \noindent We detail the construction and some of the properties of the Lebesgue measure via the Lebesgue--Stieltjes Measure.
\end{abstract}
\section{Premeasures, Outer Measures, and Measures}%
Consider a set-function $\lambda\colon P\left( \R \right)\rightarrow [0,\infty]$ that satisfies
\begin{itemize}
  \item $\lambda\left( \emptyset \right) = 0$;
  \item for any finite or infinite sequence of disjoint sets, $\set{E_j}_{j=1}^{\infty}$, we have
    \begin{align*}
      \lambda\left( \bigsqcup_{j=1}^{\infty} \right) &= \sum_{j=1}^{\infty}\lambda\left( E_j \right);
    \end{align*}
  \item $\lambda\left( I \right) = b - a$, where $I$ is an interval (either open, closed, or a half-interval);
  \item $\lambda\left( s + E \right) = \lambda\left( E \right)$.
\end{itemize}
Unfortunately, such a set-function doesn't exist.\newline

In order to construct a set function on a restricted domain $\lambda\colon \mathcal{L}\rightarrow [0,\infty]$, we need to define a particular class of measurable subsets of $\R$. This is where the concept of an \textit{outer measure} comes in.
\begin{definition}
  Let $X$ be a set, and let $\mu^{\ast}\colon P\left( X \right)\rightarrow [0,\infty]$ be a set function. We say $\mu^{\ast}$ is an \textit{outer measure} if
  \begin{itemize}
    \item $\mu^{\ast}\left( \emptyset \right)= 0$;
    \item $\mu^{\ast}\left( A \right)\leq \mu^{\ast}\left( B \right)$ if $A\subseteq B$;
    \item $\displaystyle \mu^{\ast}\left( \bigcup_{j=1}^{\infty}A_j \right)\leq \sum_{j=1}^{\infty}\mu^{\ast}\left( A_j \right)$.
  \end{itemize}
\end{definition}
We will obtain an outer measure on the entirety of $P(X)$ by defining a notion of measure on some ``satisfactory'' subfamily $\mathcal{E}\subseteq P(X)$, then by approximating other subsets using this family.
\begin{proposition}
  Let $\mathcal{E}\subseteq P(X)$ be a family of subsets such that $\emptyset\in \mathcal{E}$ and $X\in \mathcal{E}$, and let $\rho\colon \mathcal{E}\rightarrow [0,\infty]$ be a set function such that $\rho\left( \emptyset \right) = 0$. For any $A\subseteq X$, define
  \begin{align*}
    \mu^{\ast}\left( E \right) &= \inf\set{\sum_{j\geq 1}\rho\left( E_j \right) | E_j\in \mathcal{E}, A\subseteq \bigcup_{j\geq 1}E_j}.
  \end{align*}
  Then, $\mu^{\ast}$ is an outer measure.
\end{proposition}
\begin{proof}
  We start by showing well-definedness, which stems from the fact that we may select $E_j = X$ for all $j$.\newline

  Since we may take $E_j = \emptyset$ for all $j$, we must have $\mu^{\ast}\left( \emptyset \right) = 0$. Furthermore, if $A\subseteq B$, since the set over which the infimum is taken for the definition of $\mu^{\ast}\left( A \right)$ includes the corresponding set for $B$, we must have $\mu^{\ast}\left( A \right)\leq \mu^{\ast}\left( B \right)$.\newline

  Finally, let $\set{A_j}_{j\geq 1}\subseteq P(X)$, and let $\ve > 0$. For each $j$, there exists $\set{E_{j,k}}_{k\geq 1}\subseteq \mathcal{E}$ such that $A_j\subseteq \bigcup_{k\geq 1}E_{j,k}$ and $\sum_{k\geq 1}\rho\left( E_{j,k} \right) \leq \mu^{\ast}\left( A_j \right) + \ve 2^{-j}$.\newline

  Then, if $A = \bigcup_{j\geq 1}A_j$, we have $A\subseteq \bigcup_{j,k\geq 1}E_{j,k}$, and $\sum_{j,k\geq 1}\rho\left( E_{j,k} \right) \leq \sum_{j\geq 1}\mu^{\ast}\left( A_j \right) + \ve$, so that $\mu^{\ast}\left( A \right)\leq \sum_{j\geq 1}\mu^{\ast}\left( A_j \right) + \ve$. Since $\ve$ is arbitrary, we are done.
\end{proof}
\begin{definition}
  A subset $A\subseteq X$ is said to be $\mu^{\ast}$-\textit{measurable} if for any $E\subseteq X$, $A$ serves as a good ``cookie cutter'' for $E$, in that
  \begin{align*}
    \mu^{\ast}\left( E \right) &= \mu^{\ast}\left( E\cap A \right) + \mu^{\ast}\left( E\cap A^{c} \right).
  \end{align*}
  Equivalently, due to subadditivity, we have $A$ is measurable if and only if for all $E\subseteq X$,
  \begin{align*}
    \mu^{\ast}\left( E \right)\geq \mu^{\ast}\left( E\cap A \right) + \mu^{\ast}\left( E\cap A^{c} \right).
  \end{align*}
\end{definition}
\begin{definition}
  Let $\mathcal{A}$ be an algebra of subsets of $X$. We call a set function $\mu_0\colon \mathcal{A}\rightarrow [0,\infty]$ a \textit{premeasure} if
  \begin{itemize}
    \item $\mu_0\left( \emptyset \right) = 0$;
    \item for a collection of disjoint elements of $\mathcal{A}$, $\set{A_j}_{j=1}^{\infty}$ where $\bigcup_{j\geq 1}A_j\in \mathcal{A}$, we have
      \begin{align*}
        \mu_0\left( \bigsqcup_{j\geq 1}A_j \right) &= \sum_{j\geq 1}\mu_0\left( A_j \right).
      \end{align*}
  \end{itemize}
\end{definition}
Every premeasure gives rise to an outer measure by taking
\begin{align*}
  \mu^{\ast}\left( E \right) &= \inf\set{\sum_{j\geq 1}\mu_0\left( A_j \right) | A_j\in \mathcal{A},E\subseteq \bigcup_{j\geq 1}A_j}.\label{eq:premeasure_to_outer_measure}\tag{$\ast$}
\end{align*}
A remarkable result by Caratheodory allows us to extend premeasures from algebras to measures on $\sigma$-algebras. To start, there is a little bit of build-up.
\begin{proposition}
  Let $\mu_0$ be a premeasure on $\mathcal{A}$, with $\mu^{\ast}$ defined by \eqref{eq:premeasure_to_outer_measure}. Then,
  \begin{enumerate}[(a)]
    \item $\mu^{\ast}|_{\mathcal{A}} = \mu_0$;
    \item every set in $\mathcal{A}$ is $\mu^{\ast}$-measurable.
  \end{enumerate}
\end{proposition}
\begin{proof}
  Suppose $E\in \mathcal{A}$. If $E\subseteq \bigcup_{j\geq 1}A_j$ with $A_j\in \mathcal{A}$, we let $B_n = E\cap \left( A_n\setminus \bigcup_{j=1}^{n-1}A_j \right)$. The $B_n$ are disjoint members of $\mathcal{A}$ whose union is $E$, so
  \begin{align*}
    \mu_0\left( E \right) &= \sum_{j=1}^{\infty}\mu_0\left( B_j \right)\\
                          &\leq \sum_{j=1}^{\infty}\mu_0\left( A_j \right).
  \end{align*}
  It follows that $\mu_0\left( E \right)\leq \mu^{\ast}\left( E \right)$. The reverse inequality is clear from the fact that we may specify $A_1 = E$ and $A_{j >1} = \emptyset$.\newline

  Meanwhile, if $A\in \mathcal{A}$, $E\subseteq X$, and $\ve > 0$, then there is a collection $\set{B_j}_{j\geq 1}\subseteq \mathcal{A}$ with $E\subseteq \bigcup_{j\geq 1}B_j$ and $\sum_{j\geq 1}\mu_0\left( B_j \right)\leq \mu^{\ast}\left( E \right) + \ve$. By additivity on $\mathcal{A}$, we get
  \begin{align*}
    \mu^{\ast}\left( E \right) + \ve &\geq \sum_{j=1}^{\infty}\mu_0\left( B_j\cap A \right) + \mu_0\left( B_j\cap A^{c} \right)\\
                                     &\geq \mu^{\ast}\left( E\cap A \right) + \mu^{\ast}\left( E\cap A^{c} \right),
  \end{align*}
  so $A$ is measurable.
\end{proof}
\begin{theorem}[Caratheodory's Theorem]
  Let $\mathcal{A}\subseteq P(X)$ be an algebra, let $\mu_0$ be a premeasure on $\mathcal{A}$, and let $\mathcal{M}$ be the $\sigma$-algebra generated by $\mathcal{A}$. There exists a measure $\mu$ on $\mathcal{M}$ whose restriction to $\mathcal{A}$ is $\mu_0$ --- namely, $\mu - \mu^{\ast}|_{\mathcal{M}}$, where $\mu^{\ast}$ is given by \eqref{eq:premeasure_to_outer_measure}.\newline

  If $\nu$ is another measure on $\mathcal{M}$ that extends $\mu_0$, then $\nu\left( E \right)\leq \mu\left( E \right)$, with equality for all $\mu\left( E \right) < \infty$. Furthermore, if $\mu_0$ is $\sigma$-finite, then $\mu$ is unique.
\end{theorem}
\begin{proof}
  We start by showing that if $\mu^{\ast}$ is an outer measure, then if $\mathcal{M}^{\ast}$ is the collection of $\mu^{\ast}$-measurable sets, $\mathcal{M}^{\ast}$ is a $\sigma$-algebra and $\mu^{\ast}|_{\mathcal{M}^{\ast}}$ is a complete measure.\footnote{This is Theorem 1.11 in Folland's \textit{Real Analysis}.}\newline

  By definition, $\mathcal{M}^{\ast}$ is closed under complements, as the definition of $\mu^{\ast}$-measurability is symmetric in $A$ and $A^{c}$. To show finite additivity, if $A,B\in \mathcal{M}^{\ast}$ and $E\subseteq X$, we have
  \begin{align*}
    \mu^{\ast}\left( E \right) &= \mu^{\ast}\left( E\cap A \right) + \mu^{\ast}\left( E\cap A^{c} \right)\\
                               &= \mu^{\ast}\left( E\cap A\cap B \right) + \mu^{\ast}\left( E\cap A \cap B^{c} \right)\\
                               &+ \mu^{\ast}\left( E\cap A^{c}\cap B \right) + \mu^{\ast}\left( E\cap A^{c}\cap B^{c} \right).
  \end{align*}
  We note that $A\cup B = \left( A\cap B \right) \cup \left( A\cap B^{c} \right)\cup \left( A^{c}\cap B \right)$, so subadditivity gives
  \begin{align*}
    \mu^{\ast}\left( E\cap A\cap B \right) + \mu^{\ast}\left( E\cap A \cap B^{c} \right) + \mu^{\ast}\left( E\cap A^{c}\cap B \right) &\geq \mu^{\ast}\left( E\cap \left( A\cup B \right) \right).
  \end{align*}
  Therefore,
  \begin{align*}
    \mu^{\ast}\left( E \right) &\geq \mu^{\ast}\left( E\cap \left( A\cup B \right) \right) + \mu^{\ast}\left( E\cap \left( A\cup B \right)^{c} \right).
  \end{align*}
  Therefore, $A\cup B\in \mathcal{M}^{\ast}$, so $\mathcal{M}^{\ast}$ is an algebra. Moreover, if $A,B\in \mathcal{M}^{\ast}$ are disjoint, then
  \begin{align*}
    \mu^{\ast}\left( A\cup B \right) &= \mu^{\ast}\left( \left( A\cup B \right)\cap A \right) + \mu^{\ast}\left( \left( A\cup B \right)\cap A^{c} \right)\\
                                     &= \mu^{\ast}\left( A \right) + \mu^{\ast}\left( B \right).
  \end{align*}
  To show that $\mathcal{M}^{\ast}$ is a $\sigma$-algebra, we show that $\mathcal{M}^{\ast}$ is closed under countable \textit{disjoint} unions. Let $\set{A_j}_{j\geq 1}$ be a sequence of disjoint sets in $\mathcal{M}^{\ast}$, and let $B_n = \bigsqcup_{j = 1}^{n} A_j$, with $B = \bigsqcup_{j\geq 1}A_j$. Then, for any $E\subseteq X$, we have
  \begin{align*}
    \mu^{\ast}\left( E\cap B_n \right) &= \mu^{\ast}\left( E\cap B_n\cap A_n \right) + \mu^{\ast}\left( E\cap B_n\cap A_n^{c} \right)\\
                                       &= \mu^{\ast}\left( E\cap A_n \right) + \mu^{\ast}\left( E\cap B_{n-1} \right),
  \end{align*}
  so by induction, we have
  \begin{align*}
    \mu^{\ast}\left( E\cap B_n \right) &= \sum_{j=1}^{n} \mu^{\ast}\left( E\cap A_j \right).
  \end{align*}
  This gives
  \begin{align*}
    \mu^{\ast}\left( E \right) &= \mu^{\ast}\left( E\cap B_n \right) + \mu^{\ast}\left( E\cap B_n^{c} \right)\\
                               &\geq \sum_{j=1}^{n}\mu^{\ast}\left( E\cap A_j \right) + \mu^{\ast}\left( E\cap B^c \right),
  \end{align*}
  and taking $n\rightarrow\infty$, we have
  \begin{align*}
    \mu^{\ast}\left( E \right) &\geq \sum_{j=1}^{\infty} \mu^{\ast}\left( E\cap A_j \right) + \mu^{\ast}\left( E\cap B^c \right)\\
                               &\geq \mu^{\ast}\left( \bigsqcup_{j\geq 1}E\cap A_j \right) + \mu^{\ast}\left( E\cap B^c \right)\\
                               &= \mu^{\ast}\left( E\cap B \right) + \mu^{\ast}\left( E\cap B^{c} \right)\\
                               &\geq \mu^{\ast}\left( E \right).
  \end{align*}
  Therefore, $B\in \mathcal{M}^{\ast}$, and if we take $E = B$,
  \begin{align*}
    \mu^{\ast}\left( B \right) &= \sum_{j=1}^{\infty}\mu^{\ast}\left( A_j \right),
  \end{align*}
  and $\mu^{\ast}$ is countably additive on $\mathcal{M}^{\ast}$. Finally, if $\mu^{\ast}\left( A \right) = 0$, we have
  \begin{align*}
    \mu^{\ast}\left( E \right) &\leq \mu^{\ast}\left( E\cap A \right) + \mu^{\ast}\left( E\cap A^{c} \right)\\
                               &= \mu^{\ast}\left( E\cap A^{c} \right)\\
                               &\leq \mu^{\ast}\left( E \right),
  \end{align*}
  so $A\in \mathcal{M}^{\ast}$, and $\mu^{\ast}|_{\mathcal{M}^{\ast}}$ is a complete measure.\newline

  Returning to our premeasure, $\mu_0$ and the corresponding outer measure $\mu^{\ast}$, we note that since every element of $\mathcal{A}$ is $\mu^{\ast}$-measurable, the $\sigma$-algebra of $\mu^{\ast}$-measurable sets includes $\mathcal{A}$, so it includes $\mathcal{M} = \sigma\left( \mathcal{A} \right)$.\newline

  Let $\nu$ be any other measure on $\mathcal{M}$ that extends $\mu_0$. If $E\in \mathcal{M}$, and $E\subseteq \bigcup_{j\geq 1}A_j$ with $A_j\in \mathcal{A}$, then $\nu\left( E \right) \leq \sum_{j\geq 1}\nu\left( A_j \right) = \sum_{j\geq 1}\mu_0\left( A_j \right)$, so $\nu\left( E \right)\leq \mu\left( E \right)$.\newline

  If we set $A = \bigcup_{j\geq 1}A_j$, the properties of the premeasure give us
  \begin{align*}
    \nu\left( A  \right) &= \lim_{n\rightarrow\infty} \nu\left( \bigcup_{j=1}^{n} A_j \right)\\
                         &= \lim_{n\rightarrow\infty} \mu\left( \bigcup_{j=1}^{n}A_j \right)\\
                         &= \mu\left( A \right).
  \end{align*}
  If $\mu\left( E \right) < \infty$, we may select the $A_j$ such that $\mu\left( A \right) < \mu\left( E \right) + \ve$, so $\mu\left( A\setminus E \right) < \ve$, and
  \begin{align*}
    \mu\left( E \right) &\leq \mu\left( A \right)\\
                        &= \nu\left( A \right)\\
                        &= \nu\left( E \right)  + \nu\left( A\setminus E \right)\\
                        &\leq \nu\left( E \right) +\mu\left( A\setminus E \right)\\
                        &\leq \nu\left( E \right) + \ve.
  \end{align*}
  Since $\ve$ is arbitrary, $\mu\left( E \right) = \nu\left( E \right)$.\newline

  Now, if $\mu_0$ is $\sigma$-finite, we write $X = \bigsqcup_{j\geq 1}A_j$, with $\mu_0\left( A_j \right) < \infty$ and the $A_j$ are disjoint. For any $E\in \mathcal{M}$, we have
  \begin{align*}
    \mu\left( E \right) &= \sum_{j\geq 1}\mu\left( E\cap A_j \right)\\
                        &= \sum_{j\geq 1}\nu\left( E\cap A_j \right)\\
                        &= \nu\left( E \right).
  \end{align*}
\end{proof}
\section{Construction of the Lebesgue Measure}%
With Caratheodory's theorem, we now know that it is possible to construct a unique measure from a suitable premeasure on a particular family of subsets. Here, we will use the family of half-open intervals, or h-intervals, of the form $\left( a,b \right]$, where $-\infty \leq a < b < \infty$, or $\left(a,\infty\right)$.\newline

The algebra of h-intervals, $\mathcal{A}$, generates the Borel $\sigma$-algebra, $\mathcal{B}_{\R}$.\newline

Consider a finite Borel measure on $\R$, and let $F(x) = \mu\left( \left( -\infty,x \right] \right)$. We often call $F(x)$ the \textit{distribution function} of $\mu$. Then, we see that $F$ is increasing and right-continuous, as
\begin{align*}
  \left( -\infty,x \right] &= \bigcap_{n\geq 1} \left( -\infty,x_n \right],
\end{align*}
where $x_n$ is a decreasing sequence convergence to $x$.\newline

As it turns out, we are able to reverse this process. Given an increasing, right-continuous function $F\colon \R\rightarrow \R$, there is a corresponding Borel measure.
\begin{proposition}
  Let $F\colon \R\rightarrow \R$ be increasing and right-continuous. If $\set{\left( a_j,b_j \right]}_{j=1}^{n}$ are disjoint $h$-intervals, we define
  \begin{align*}
    \mu_0\left( \bigcup_{j=1}^{n} \left( a_j,b_j \right] \right) &= \sum_{j=1}^{n} \left( F\left( b_j \right)-F\left( a_j \right) \right),
  \end{align*}
  and set $\mu_0\left( \emptyset \right) = 0$. Then, $\mu_0$ is a premeasure on $\mathcal{A}$.
\end{proposition}
\begin{proof}
  We start by verifying that $\mu_0$ is well-defined, seeing as elements of $\mathcal{A}$ can be written in more than one way as disjoint unions of h-intervals. If $\set{\left( a_j,b_j \right]}_{j=1}^{n}$ are disjoint, and $\bigcup_{j=1}^{n}\left( a_j,b_j \right] = \left( a,b \right]$, then after relabeling indices, we must have $a = a_1 < b_1 = a_2 < \cdots < b_n = b$, so $\sum_{j=1}^{n}\left( F\left( b_j \right)-F\left( a_j \right) \right) = F\left( b \right)-F\left( a \right)$.\newline

  Generally speaking, if $\set{I_i}_{i=1}^{n}$ and $\set{J_j}_{j=1}^{m}$ are disjoint finite sets of intervals such that $\bigcup_{i=1}^{n}I_i = \bigcup_{j=1}^{m}J_j$, then
  \begin{align*}
    \sum_{i=1}^{n}\mu_0\left( I_i \right) &= \sum_{i=1}^{n}\sum_{j=1}^{m} \mu_0\left( I_i\cap J_j \right)\\
                                          &= \sum_{j=1}^{m} \mu_0\left( J_j \right).
  \end{align*}
  Now, we must show that if $\set{I_j}_{j=1}^{\infty}$ is a sequence of disjoint h-intervals with $\bigcup_{j\geq 1}I_j \in \mathcal{A}$, then $\mu_0\left( \bigcup_{j\geq 1}I_j \right) = \sum_{j\geq 1}\mu_0\left( I_j \right)$ .\newline

  Since $\bigcup_{j\geq 1}I_j$ is a finite union of h-intervals, we may partition $\set{I_j}_{j\geq 1}$ into finitely many subfamilies such that the union in each subfamily is a single h-interval. Using the finite additivity of $\mu_0$, we may assume that $\bigcup_{j=1}^{\infty} I_j $ is an interval $I = \left( a,b \right]$. We thus have
  \begin{align*}
    \mu_0\left( I \right) &= \mu_0\left( \bigcup_{j=1}^{n}I_j \right) + \mu_0\left( I\setminus \bigcup_{j=1}^{n}I_j \right)\\
                          &\geq \mu_0\left( \bigcup_{j=1}^{n}I_j \right)\\
                          &= \sum_{j=1}^{n}\mu_0\left( I_j \right).
  \end{align*}
  Taking limits, we get $\mu_0\left( I \right) \geq \sum_{j\geq1} \mu_0\left( I_j \right)$.\newline

  To prove the reverse inequality, we suppose $a$ and $b$ are finite, and fix $\ve > 0$. Since $F$ is right-continuous, there exists $\delta > 0$ such that $F\left( a+\delta \right)-F\left( a \right) < \ve$, and if $I_j = \left( a_j,b_j \right]$, then for each $j$ there is $\delta_j > 0$ such that $F\left( b_j + \delta_j \right) - F\left( b_j \right) < \ve 2^{-j}$.\newline

  The open intervals $\left( a_j,b_j + \delta_j \right)$ cover the compact set $\left[ a+\delta,b \right]$, so there is a finite subcover. By discarding $\left( a_j,b_j + \delta_j \right)$ contained in larger ones, and relabeling indices, we may assume that
  \begin{itemize}
    \item the intervals $\left( a_1,b_1+\delta_1 \right),\dots,\left( a_N,b_N + \delta_N \right)$ cover $\left[ a+\delta,b \right]$;
    \item $b_j + \delta_j\in \left( a_{j+1},b_{j+1} + \delta_{j+1} \right)$ for each $j$.
  \end{itemize}
  Then,
  \begin{align*}
    \mu_0\left( I \right) &< F\left( b \right) - F\left( a + \delta \right) + \ve\\
                          &\leq F\left( b_N + \delta_N \right) - F\left( a_1 \right) + \ve\\
                          &= F\left( b_N + \delta_N \right) - F\left( a_N \right) + \sum_{j=1}^{N-1}\left( F\left( a_{j+1} \right)-F\left( a_j \right) \right) + \ve\\
                          &\leq F\left( b_N + \delta_N \right) - F\left( a_N \right) + \sum_{j=1}^{N-1}\left( F\left( b_j + \delta_j \right) - F\left( a_j \right) \right) + \ve\\
                          &< \sum_{j=1}^{N} \left( F\left( b_j \right) + \ve 2^{-j} - F\left( a_j \right) \right) + \ve\\
                          &< \sum_{j=1}^{\infty}\mu_0\left( I_j \right) + 2\ve.
  \end{align*}
  Since $\ve$ is arbitrary, we are done for the case that $a$ and $b$ are finite.\newline

  If $a = -\infty$, then for any $M < \infty$, the intervals $\left( a_j,b_j + \delta_j \right)$ cover $\left[ -M,b \right]$, so the same reasoning gives $F\left( b \right) - F\left( -M \right) \leq \sum_{j=1}^{\infty}\mu_0\left( I_j \right) + 2\ve$, whereas if $b = \infty$, we obtain $F\left( M \right) - F\left( a \right) \leq \sum_{j=1}^{\infty}\mu_0\left( I_j \right) + 2\ve$. Our desired result follows from taking $\ve\rightarrow 0$ and $M\rightarrow \infty$.
\end{proof}
This allows us to establish the correspondence between increasing and right-continuous functions and Borel measures.
\begin{theorem}
If $F\colon \R\rightarrow \R$ is an increasing, right-continuous function, then there is a unique Borel measure $\mu_F$ on $\R$ such that $\mu_F\left( \left( a,b \right] \right) = F(b) - F(a)$ for all $a,b$. If $G$ is another such function, then $\mu_F = \mu_G$ if and only if $F-G$ is constant.\newline

Conversely, if $\mu$ is a Borel measure on $\R$ that is finite on bounded sets, and we define
\begin{align*}
  F(x) &= \begin{cases}
    \mu\left( \left( 0,x \right] \right) & x > 0\\
    0 & x = 0\\
    -\mu\left( \left( x,0 \right] \right) & x < 0
  \end{cases},
\end{align*}
then $F$ is increasing and right-continuous, with $\mu = \mu_F$.
\end{theorem}
\begin{proof}
  We know that each $F$ induces a premeasure on $\mathcal{A}$ by the previous proposition, and by definition, $G$ induces the same premeasure if and only if $F-G$ is constant. These premeasures are $\sigma$-finite, since
  \begin{align*}
    \R &= \bigcup_{j=-\infty}^{\infty}\left( j,j+1 \right].
  \end{align*}
  Therefore, the induced measure $\mu_F$ on $\mathcal{B}_{\R}$ is unique by the Caratheodory extension theorem.\newline

  The last assertion follows from the fact $\mu$ is monotonic, and continuous from both above and below, so that $F$ is continuous for both $x \geq 0$ and $x < 0$. Since $\mu = \mu_F$ on $\mathcal{A}$, we have $\mu = \mu_F$ on $\mathcal{B}_{\R}$ by the uniqueness condition in the Caratheodory extension theorem.
\end{proof}
\begin{definition}
  If $F$ is an increasing and right-continuous function, then the completion of the measure $\mu_F$, which we write $\lambda_F$, is known as the \textit{Lebesgue--Stieltjes measure} associated to $F$.\newline

  We denote the $\sigma$-algebra associated to $\lambda_F$ as $\mathcal{M}_{\lambda}$. For any $E\in \mathcal{M}_{\lambda}$, we have
  \begin{align*}
    \lambda_F(E) &= \inf\set{\sum_{j=1}^{\infty}\left( F\left( b_j \right)-F\left( a_j \right) \right) | E\subseteq \bigcup_{j=1}^{\infty}\left( a_j,b_j \right]}\\
                 &= \inf\set{\sum_{j=1}^{\infty}\lambda_F\left( \left( a_j,b_j \right] \right) | E\subseteq \bigcup_{j=1}^{\infty}\left( a_j,b_j \right]}.
  \end{align*}
\end{definition}
As it turns out, we are allowed to replace the h-intervals in the formula for for $\lambda_F(E)$ with open intervals. Note that in \href{https://ai.avinash-iyer.com/Classes_and_Homework/College/Y3/Y3S2,\%20Real\%20II/real_2_notes.pdf}{Real Analysis II}, we defined the Lebesgue measure through this method.
\begin{lemma}
  For any $E\in \mathcal{M}_{\lambda}$, we have
  \begin{align*}
    \lambda_F\left( E \right) &= \inf\set{\sum_{j=1}^{\infty}\lambda_F\left( \left( a_j,b_j \right) \right) | E\subseteq \bigcup_{j=1}^{\infty}\left( a_j,b_j \right)}.
  \end{align*}
\end{lemma}
\begin{proof}
  We call the quantity on the right $\nu\left( E \right)$. Let $E\subseteq \bigcup_{j=1}^{\infty}\left( a_j,b_j \right)$. Each $\left( a_j,b_j \right)$ is a countable disjoint union of h-intervals of the form $I_{j,k} = \left( c_{j,k},c_{j,k+1} \right]$, where $\left( c_{j,k} \right)_k$ is a sequence with $c_{j,1} = a_j$ and $c_{j,k}\rightarrow b_j$. Thus, $E\subseteq \bigcup_{j,k=1}^{\infty}I_{j,k}$, so
  \begin{align*}
    \sum_{j=1}^{\infty}\lambda_F\left( \left( a_j,b_j \right) \right) &= \sum_{j,k=1}^{\infty}\lambda_F\left( I_{j,k} \right)\\
                                                                &\geq \lambda_F\left( E \right),
  \end{align*}
  so $\nu\left( E \right)\geq \lambda_F\left( E \right)$.\newline

  On the other hand, given $\ve > 0$, there exists $\set{\left( a_j,b_j \right]}_{j\geq 1}$ such that $E\subseteq \bigcup_{j\geq 1}\left( a_j,b_j \right]$ and $\sum_{j\geq 1}\lambda_F\left( \left( a_j,b_j \right] \right) \leq \lambda_F\left( E \right) + \ve$. For each $j$, right-continuity gives $\delta_j > 0$ such that $F\left( b_j + \delta_j \right) - F\left( b_j \right) < \ve 2^{-j}$.\newline

  Then, $E\subseteq \bigcup_{j\geq 1}\left( a_j,b_j + \delta_j \right)$, and
  \begin{align*}
    \sum_{j\geq 1} \lambda_F\left( \left( a_j,b_j + \delta_j \right) \right) &\leq \sum_{j\geq 1}\lambda_F\left( \left( a_j,b_j \right] \right) + \ve\\
                                                                       &\leq \lambda_F\left( E \right) + 2\ve,
  \end{align*}
  so $\nu\left( E \right) \leq \mu\left( E \right)$
\end{proof}
Now we may understand the regularity of the Lebesgue--Stieltjes measure.
\begin{theorem}
  Let $\lambda_F$ be a Lebesgue--Stieltjes measure on $\R$, and let $E\in \mathcal{M}_{\lambda}$. Then, the following hold:
  \begin{enumerate}[(a)]
    \item For all $\ve > 0$, there exists $U\subseteq \R$ open with $E\subseteq U$ and $\lambda_F\left( U\setminus E \right) < \ve$.
    \item There exists $V\subseteq \R$ $G_{\delta}$ with $E\subseteq V$ and $\lambda_F\left( V\setminus E \right) < \ve$.
    \item For all $\ve > 0$, there exists $C\subseteq \R$ closed with $C\subseteq E$ and $\lambda_F\left( E\setminus C \right) < \ve$.
    \item There exists $F\subseteq \R$ $F_{\sigma}$ with $E\subseteq F$ and $\lambda_F\left( F\setminus E \right) < \ve$.
  \end{enumerate}
\end{theorem}
\begin{proof}\hfill
  \begin{enumerate}[(a)]
    \item Let $\ve > 0$. By the previous theorem, and the definition of the outer measure, we have a set $\set{\left( a_j,b_j \right)}_{j=1}^{\infty}$ such that $E\subseteq \bigcup_{j\geq 1}\left( a_j,b_j \right)$, and
      \begin{align*}
        \lambda_F\left( E \right) + \ve &> \sum_{j=1}^{\infty}\lambda_F\left( \left( a_j,b_j \right) \right)\\
                                        &\geq \lambda_F\left( \bigcup_{j=1}^{\infty}\left( a_j,b_j \right) \right),
      \end{align*}
      so we set $U = \bigcup_{j\geq 1}\left( a_j,b_j \right)$.\newline

      Now, if $\lambda_F\left( E \right) < \infty$, then $\lambda_F\left( U\setminus E \right) = \lambda_F\left( U \right) - \lambda_F\left( E \right) < \ve$. Meanwhile, if $\lambda_F\left( E \right) =\infty$, we partition to get $E = \bigsqcup_{k\geq 1}E_k$ with $\lambda_F\left( E_k \right) < \infty$, and find $U_k$ such that $\lambda_F\left( U_k \setminus E_k \right) < \ve 2^{-k}$. Setting $U = \bigcup_{k\geq 1}U_k$, we get
      \begin{align*}
        \lambda_F\left( U\setminus E \right) &= \lambda_F\left( \bigcup_{k\geq 1}\left( U_k\setminus E_k \right) \right)\\
                                             &\leq \sum_{k\geq 1}\lambda_F\left( U_k\setminus E_k \right)\\
                                             &< \sum_{k\geq 1}\ve 2^{-k}\\
                                             &= \ve.
      \end{align*}
    \item For each $n$, we find $U_n\subseteq \R$ such that $E\subseteq U_n$ and $\lambda_F\left( U_n\setminus E \right) < 1/n$. Setting $V = \bigcap_{n\geq 1}U_n$, we find
      \begin{align*}
        \lambda_F\left( V\setminus E \right) &= \lambda_F\left( \bigcap_{n\geq 1}\left( U_n\setminus E \right) \right)\\
                                             &\leq \lambda_F\left( U_k\setminus E \right) \tag*{for all $k$}\\
                                             &< 1/k,
      \end{align*}
      so $\lambda_F\left( V\setminus E \right) = 0$.
    \item We may use the same methodology on $E^{c}$, and take complements.
    \item We may use the same methodology on $E^{c}$, and take complements, using the fact that the complement of a $G_{\delta}$ set is a $F_{\sigma}$ set.
  \end{enumerate}
\end{proof}
\begin{theorem}
  Let $E\in \mathcal{M}_{\lambda}$. Then,
  \begin{align*}
    \lambda_F\left( E \right) &= \inf\set{\lambda_F\left( U \right) | E\subseteq U,U\text{ open}}\\
                              &= \sup\set{\lambda_F\left( K \right) | K\subseteq E,K\text{ compact}}.
  \end{align*}
  The former equality is known as \textit{outer regularity}, and the latter equality is known as \textit{inner regularity}.
\end{theorem}
\begin{proof}
  We know that for all $\ve > 0$, there is $E\subseteq \bigcup_{j\geq 1}\left( a_j,b_j \right)$, and $\sum_{j\geq 1}\lambda_F\left( \left( a_j,b_j \right) \right) \leq \lambda_F\left( E \right) + \ve$. Setting $U = \bigcup_{j\geq 1}\left( a_j,b_j \right)$, we have $\lambda_F\left( U \right) \leq \lambda_F\left( E \right) + \ve$. Since $E\subseteq U$, we also have $\lambda_F\left( E \right) \leq \lambda_F\left( U \right)$, so the definition of outer regularity is satisfied.\newline

  We now show inner regularity. If $E$ is bounded, given $\ve > 0$, there is $C\subseteq E$ closed with $\lambda_F\left( E\setminus C \right) < \ve$. Since $C$ is bounded, $C$ is compact, so $\lambda_F\left( E \right) - \ve < \lambda_F\left( C \right)$, and so we have inner regularity whenever $E$ is bounded.\newline

  If $E$ is not bounded, we set $E_n = E\cap \left[ -n,n \right]$. We have $E_1\subseteq E_2\subseteq \cdots$, and $E = \bigcup_{n\geq 1}E_n$. Therefore, $\lambda_F\left( E \right) = \sup\left( \lambda_F\left( E_n \right) \right)$. There are two cases.\newline

  If $\lambda_F\left( E \right) = \infty$, then we may find compact $K_n\subseteq E_n$ such that $\lambda_F\left( E_n \right) - 1 < \lambda_F\left( K-n \right)$, so that $\lambda_F\left( K_n \right)\rightarrow \infty$.\newline

  If $\lambda_F\left( E \right) < \infty$, then given $\ve > 0$, we find $N$ such that $\lambda_F\left( E \right) - \ve / 2 < \lambda_F\left( E_n \right)$. We find compact $K$ with $K\subseteq E_n$ and $\lambda_F\left( E_n \right) - \ve/2 < \lambda_F\left( K \right)$. Thus, $K\subseteq E$ is compact, with $\lambda_F\left( E \right) - \ve < \lambda_F\left( K \right)$.
\end{proof}
\begin{proposition}
  If $E\in \mathcal{M}_{\lambda}$, and $\lambda_F\left( E  \right) < \infty$, then for every $\ve > 0$, there is a set $A$ that is a finite union of open intervals such that $\lambda_F\left( E\triangle A \right) < \ve$.
\end{proposition}
\begin{proof}
  By outer regularity, there is $U\subseteq \R$ open such that $E\subseteq U$, and $\lambda_F\left( U\setminus E \right) < \ve/2$. Every open subset of $\R$ is a countable disjoint union of open intervals, so that $\lambda_F \left( \bigsqcup_{j\geq 1}\left( \left( a_j,b_j \right)\setminus E \right) \right) < \ve$.\newline

  We find $N$ such that $\sum_{j=N+1}^{\infty}\lambda_F\left( \left( a_j,b_j \right) \right) < \ve/2$, and set $A = \bigsqcup_{j=1}^{N}\left( a_j,b_j \right)$.
\end{proof}
\begin{definition}
  \textit{The} Lebesgue measure is defined to be the Lebesgue--Stieltjes measure associated to the function $F(x) = x$. We denote it by $m$.\newline

  The domain of $m$ is known as the class of \textit{Lebesgue-measurable} sets, denoted $\mathcal{L}$.
\end{definition}
\begin{theorem}
  If $E\in \mathcal{L}$, then $E + s\in \mathcal{L}$ and $rE\in \mathcal{L}$ for all $r,s\in \R$. Moreover, $m\left( E + s \right)= m\left( E \right)$, and $m\left( rE \right) = \left\vert r \right\vert m(E)$.
\end{theorem}
\begin{proof}
  Since open intervals are invariant under translations and dilations, so is $\mathcal{B}_{\R}$. For $E\in \mathcal{B}_{\R}$, we let $m_s\left( E \right) = m\left( E + s \right)$, and $m^{r}\left( E \right) = m\left( rE \right)$.\newline

  Since $m_{s}$ and $m^{r}$ agree with $m$ and $\left\vert r \right\vert m$ on finite unions of intervals, they agree on $\mathcal{B}_{\R}$ by the Caratheodory extension theorem. In particular, whenever $E\in \mathcal{B}_{\R}$, and $m\left( E \right) = 0$, then $m\left( E + s \right) = m\left( rE \right) = 0$, so it follows that the class of Lebesgue-null sets is preserved under translations and dilations. Since all members of $\mathcal{L}$ are unions of a null set and a Borel set, it follows that $\mathcal{L}$ is preserved under translations and dilations. Therefore, $m\left( E + s \right) = m\left( E \right)$ and $m\left( rE \right) = \left\vert r \right\vert m\left( E \right)$ for all $E\in \mathcal{L}$.
\end{proof}
%Next, we turn our attention to the question of non-measurable subsets. Specifically, what is the criterion for the existence of a non-measurable subset of the real numbers.\newline
%
%We will establish Vitali's Theorem, which gives a sufficient condition for non-measurable subsets of any subset of $\R$, and then use this to construct\footnote{Well, ``construct'' is a strong word.} a set that is Lebesgue-measurable but not Borel-measurable.\newline
%
There are indeed elements of $\mathcal{L}$ that are not elements of $\mathcal{B}_{\R}$.\newline

First, recall that the Cantor set, $\Delta$ consists of all $x\in [0,1]$ such that the base 3 expansion $x = \sum_{j\geq 1}a_j3^{-j}$ is such that $a_j \in \set{0,2}$.\newline

Since we may map $\Delta$ onto $[0,1]$ by taking $a_j \mapsto a_j/2$ for each $j\geq 0$, we see that $\Delta$ is uncountable, and that $m(\Delta) = 0$. Therefore, every subset of $\Delta$ is of measure zero (since Lebesgue measure is complete), meaning that the cardinality of $\mathcal{L}$ is $2^{2^{\aleph_0}}$. Meanwhile, a result from descriptive set theory shows that $\mathcal{B}_{\R}$ has cardinality $2^{\aleph_0}$,\footnote{It turns out that the $\sigma$-algebra generated by a particular family $\mathcal{E}$ has cardinality $\aleph_1\cdot \left\vert \mathcal{E} \right\vert$. Since $\tau_{\R}$ has cardinality $2^{\aleph_0}$, and $2^{\aleph_0}\geq \aleph_1$ (depending on whether or not you accept the Continuum Hypothesis), $\aleph_1\cdot 2^{\aleph_0} = 2^{\aleph_0}$.} so there exists some Lebesgue-measurable set that isn't Borel-measurable.
%
%However, this is kind of an unsatisfactory solution to the question of a subset of $\R$ that is Lebesgue-measurable but not Borel-measurable. We will construct this non-measurable subset using Vitali's theorem.
%\begin{theorem}[Vitali]
%  Given $E\subseteq \R$ with $m^{\ast}(E) > 0$, there exists $N\subseteq E$ with $N\notin \mathcal{L}$.
%\end{theorem}
%\begin{proof}
%  First, we assume $E\subseteq [-a,a]$, and we create an equivalence relation on $E$ by defining $x\sim y$ if $x-y\in \Q$. We have
%  \begin{align*}
%    E &= \bigsqcup_{i\in I} \left[ x_i \right]_{\sim}.
%  \end{align*}
%  Set $N = \set{x_i}_{i\in I}$ --- i.e., select a family of representatives of equivalence classes for $\sim$. We claim that $N$ is not measurable.\newline
%
%  Let $\set{r_k}_{k=1}^{\infty}$ be an enumeration of the rationals inside $\Q\cap [-2a,2a]$. The $\set{r_k + N}_{k=1}^{\infty}$ are pairwise disjoint, and
%  \begin{align*}
%    E &\subseteq \bigsqcup_{k=1}^{\infty}r_k + N,
%  \end{align*}
%  since for any $x\in E$, there is some $x_i\in N$ such that $x\sim x_i$, meaning $x-x_i\in \Q\cap [-2a,2a]$, and $x-x_i= r_k$ for some $k$, meaning $x\in r_k + N$.\newline
%
%  If $N$ were measurable, then
%  \begin{align*}
%    0 &< m^{\ast}\left( E \right)\\
%      &\leq m^{\ast}\left( \bigsqcup_{k=1}^{\infty}r_k + N \right)\\
%      &= \sum_{k=1}^{\infty}m^{\ast}\left( r_k + N \right)\\
%      &= \sum_{k=1}^{\infty}m^{\ast}\left( N \right)\\
%      &= \sum_{k=1}^{\infty}m\left( N \right).
%  \end{align*}
%  Furthermore, we must have $m\left( N \right) \leq m^{\ast}\left( E \right) \leq 2a$. This yields a contradiction, as either $m\left( N \right) = 0$, giving $m^{\ast}\left( E \right) = 0$, which is a contradiction, or $m(N) \neq 0$, giving $m^{\ast}\left( E \right) = \infty$.
%\end{proof}
%With Vitali's theorem in hand, we now construct the Cantor--Lebesgue function. We start by remembering one of the primary constructions of the Cantor set. We start by taking
%\begin{align*}
%  C_0 &= [0,1]\\
%  C_1 &= [0,1/3]\cup [2/3,1]\\
%  C_2 &= [0,1/9]\cup [2/9,1/3]\cup[2/3,7/9]\cup[8/9,1]\\
%  C_n &= \frac{1}{3}\left( C_{n-1} \cup \left( 2 + C_{n-1} \right) \right)\\
%  \Delta &= \bigcap_{n=0}^{\infty}C_n.
%\end{align*}
%Then, we define
%\begin{align*}
%  G_1 &= C_0 \setminus C_1\\
%      &= \left( 1/3,2/3 \right)\\
%      &= I_{1,1}\\
%  G_2 &= C_1\setminus C_2\\
%      &= \left( 1/9,2/9 \right)\cup \left( 7/9,8/9 \right)\\
%      &= I_{2,1}\cup I_{2,2}\\
%      &\vdots,
%\end{align*}
%and so on. Then, we take
%\begin{align*}
%  g_{k} &= \sum_{j=1}^{2^{k}-1} \frac{2j-1}{2^{k}} \1_{I_{k,j}},
%\end{align*}
%and let $f_n = \sum_{k=1}^{n}g_k$. We let $\varphi_n\colon [0,1]\rightarrow [0,1]$ be the continuous extension of $f_n$ defined by taking $\varphi(0) = 0$, $\varphi(1) = 1$, and $\varphi_n$ linear on $C_n$.\newline
%
%Then, the $\varphi_n$ are uniformly Cauchy, as
%\begin{align*}
%  \norm{\varphi_{k+1}-\varphi_{k}}_{u} &\leq \frac{1}{2^{k}},
%\end{align*}
%and for $m > n$,
%\begin{align*}
%  \norm{\varphi_{m} - \varphi_{n}}_{u} &\leq \norm{\varphi_{m} - \varphi_{m-1}}_{u} + \cdots + \norm{\varphi_{n+1}-\varphi_n}_{u}\\
%                                       &\leq 2^{1-m} + \cdots + 2^{-n}\\
%                                       &\leq 2^{1-n}.
%\end{align*}
%Therefore, since $C\left( [0,1] \right)$ is complete, there is a continuous function $\varphi$ such that $\left( \varphi_{n} \right)_{n}\rightarrow \varphi$, which we call the Cantor--Lebesgue function. We note the following:
%\begin{itemize}
%  \item $\varphi$ is increasing, as $\varphi_n(x)\leq \varphi_n(y)$ whenever $x \leq y$, meaning $\varphi(x)\leq \varphi(y)$;
%  \item $\varphi$ is constant on $[0,1]\setminus \Delta$;
%  \item $\varphi\left( [0,1] \right) = [0,1]$ (intermediate value theorem);
%  \item $\varphi\left( \Delta \right) = [0,1]$, since we may define $[0,1] = \Delta\sqcup U$, where we define $U = [0,1]\setminus \Delta$, and take
%    \begin{align*}
%      [0,1] &= \varphi\left( [0,1] \right)\\
%            &= \varphi\left( \Delta \right)\sqcup \varphi\left( U \right).
%    \end{align*}
%    Note that $m\left( [0,1] \right) = 1$, and $m\left( \varphi\left( U \right) \right) = 0$, as $\varphi\left( U \right)\subseteq \Q\cap[0,1]$, so $m\left( \varphi\left( \Delta \right) \right) = 1$. Furthermore, $\varphi\left( \Delta \right) \subseteq [0,1]$ is compact, thus closed. Therefore, if there were $t\in [0,1]\setminus \varphi\left( \Delta \right)$, then there is $\delta > 0$ such that $\left( t-\delta,t+\delta \right)\subseteq [0,1]\setminus \varphi\left( \Delta \right)$, implying $m\left( \varphi\left( \Delta \right) \right) < 1$, which would be a contradiction.
%\end{itemize}

\end{document}
