\documentclass[12pt]{mypackage}

%\usepackage{mlmodern}
%\usepackage{newpxtext,eulerpx,eucal}
%\renewcommand*{\mathbb}[1]{\varmathbb{#1}}

%\usepackage{homework}
\usepackage{notes}

\usepackage[ backend=bibtex, style = alphabetic, sorting=ynt ]{biblatex}
\addbibresource{all_references.bib}

\usepackage{parskip}

\fancyhf{}
\fancyhead[R]{Avinash Iyer}
\fancyhead[L]{Projections in von Neumann Algebras}
\fancyfoot[C]{\thepage}

\setcounter{secnumdepth}{0}

\begin{document}
\RaggedRight
Here, we overview and discuss some of the most important results related to projections in von Neumann algebras.
\section{Comparison of Projections}%
Recall that if $H$ is a Hilbert space, an element $w\in B(H)$ is called a partial isometry if, for any $h\in  \ker\left( w \right) ^{\perp}$, we have $\norm{Wh} = \norm{h}$. We call $\ker\left( w \right)^{\perp}$ the initial space of $W$ and $\img(w)$ the final space of $W$.

There are a variety of equivalent definitions for partial isometries.
\begin{proposition}
  If $w\in B(H)$, then the following are equivalent:
  \begin{enumerate}[(i)]
    \item $w$ is a partial isometry;
    \item $w^{\ast}$ is a partial isometry;
    \item $w^{\ast}w$ is a projection onto the initial space of $w$;
    \item $ww^{\ast}$ is a projection onto the final space of $w$;
    \item $ww^{\ast}w = w$;
    \item $w^{\ast}ww^{\ast} = w^{\ast}$.
  \end{enumerate}
\end{proposition}
\begin{theorem}[Polar Decomposition]
  Let $a\in B(H)$. Then, there is a partial isometry $w\in B(H)$ with initial space $ \ker\left( a \right)^{\perp} $ and final space $ \overline{\img(a)} $ such that $a = w\left\vert a \right\vert$.

  If $a\in M\subseteq B(H)$, where $M$ is a von Neumann algebra, then both $\left\vert a \right\vert$ and $w$ are in $M$.
\end{theorem}
\subsection{Equivalence of Projections}%
If $M\subseteq B(H)$ is a von Neumann algebra, then we say two projections $p,q\in P(M)$, where $P(M)$ denotes the space of projections of $M$, are (Murray--von Neumann) \textit{equivalent} in $M$ if there is a partial isometry $v\in P(M)$ such that $v^{\ast}v = p$ and $vv^{\ast} = q$. We will write $p\sim q$.

Note that projections have an ordering by saying that $p\leq q$ if $pq = qp = p$, or $\img(p)\subseteq \img(q) $. This allows us to say that $p$ is \textit{sub-equivalent} to $q$ (in $M$), written $p\preceq q$, if there is a partial isometry $v\in M$ such that $v^{\ast}v = p$ and $vv^{\ast}\leq q$.\footnote{We will say that the projection $q$ majorizes $p$ if $p\preceq q$, and we will say that $q$ dominates $p$ if $p\leq q$.}

The sub-equivalence relation in fact forms a partial order, and equivalence as projections forms an equivalence relation. We will first show that it is a preorder.
\begin{proposition}
  In a von Neumann algebra, the relation $\sim$ is an equivalence relation on $P(M)$, and the relation $\preceq$ is a preorder.
\end{proposition}
\begin{proof}
  Reflexivity follows from the fact that projections are partial isometries, and symmetry follows from the fact that if $v$ is a partial isometry, then so is $v^{\ast}$.

  Now, we will show transitivity for $\preceq$, from which we will see that $\sim$ is transitive. Letting $p,q,r\in P(M)$ be such that $p\preceq q$ and $q\preceq r$, we have partial isometries $u,v\in M$ with $u^{\ast}u = p$, $uu^{\ast}\leq q$, $v^{\ast}v = q$, and $vv^{\ast}\leq r$. Then, we have
  \begin{align*}
    qu &= q uu^{\ast}u\\
       &= \left( quu^{\ast} \right)u\\
       &= uu^{\ast}u\\
       &= u,
  \end{align*}
  so that
  \begin{align*}
    \left( vu \right)^{\ast}\left( vu \right) &= u^{\ast}v^{\ast}vu\\
                                              &= u^{\ast}qu\\
                                              &= u^{\ast}u\\
                                              &= p\\
    \left( vu \right)\left( vu \right)^{\ast} &= vuu^{\ast}v^{\ast}\\
                                              &\leq vqv^{\ast}\\
                                              &= vv^{\ast}vv^{\ast}\\
                                              &= vv^{\ast}\\
                                              &\leq r.
  \end{align*}
  Therefore, $p\preceq r$, so $\preceq$ is a transitive relation.
\end{proof}
To see that $\preceq$ is a partial order, we need an analogue of the Cantor--Schröder--Bernstein theorem for projections. In fact, it can be proven in a similar manner. First, we discuss a simple lemma.
\begin{lemma}
  Let $M\subseteq B(H)$ be a von Neumann algebra. If $\set{p_i}_{i\in I}$ and $\set{q_i}_{i\in I}$ are pairwise orthogonal families of projections with $p_i\preceq q_i$, then $\sum_{i\in I}p_i\preceq \sum_{i\in I}q_i$.
\end{lemma}
\begin{proof}
  Let $u_i$ be the partial isometries with $u_i^{\ast}u_i = p_i$ and $r_i\coloneq u_iu_i^{\ast}\leq q_i$. Then, the $r_i$ are pairwise orthogonal since the $q_i$ are pairwise orthogonal, and for any $i\neq j$,
  \begin{align*}
    u_i^{\ast}u_j &= u_i^{\ast}u_iu_i^{\ast}u_ju_j^{\ast}u_j\\
                  &= u_ir_ir_ju_j\\
                  &= 0\\
    u_iu_j^{\ast} &= u_iu_i^{\ast}u_iu_j^{\ast}u_ju_j^{\ast}\\
                  &= u_ip_ip_ju_j^{\ast}\\
                  &= 0.
  \end{align*}
  Consequently, we get
  \begin{align*}
    \left( \sum_{i\in I}u_i^{\ast} \right)\left( \sum_{j\in I}u_j \right) &= \sum_{i\in I}u_i^{\ast}u_i\\
                                                                          &= \sum_{i\in I}p_i\\
    \left( \sum_{i\in I}u_i \right)\left( \sum_{j\in I}u_j^{\ast} \right) &= \sum_{i\in I}u_iu_i^{\ast}\\
                                                                          &\leq \sum_{i\in I}q_i.
  \end{align*}
  This gives $\sum_{i\in I}p_i\preceq \sum_{i\in I}q_i$.
\end{proof}
\begin{theorem}
  If $e\preceq f$ and $f\preceq e$, then $e\sim f$.
\end{theorem}
\begin{proof}
  We will let $e_0 \coloneq e$ and $f_0 \coloneq f$. Let $v$ and $w$ be partial isometries with $v^{\ast}v = e$, $vv^{\ast} = f_1\leq f$, $w^{\ast}w = f$, $ww^{\ast} = e_1\leq e$. Inductively define a sequence of projections as follows.

  Since $v$ maps the range of $e_1$ isometrically onto the range of some projection dominated by $f_1$, it follows that we may write $f_2\coloneq ve_1 \left( ve_1 \right)^{\ast}$ with $f_2\leq f_1$. Since $w$ maps the range of $f_1$ onto the range of some projection dominated by $e_1$, it follows that we may write $wf_1 \left( wf_1 \right)^{\ast}\eqcolon e_2$. Observe also that $v\left( e-e_1 \right)$ is a partial isometry with initial projection $e-e_1$ and final projection $f_1-f_2$.

  Inductively, we obtain decreasing sequences of projections $\left( e_n \right)_n$ and $\left( f_n \right)_n$ where $v$ maps the range of $e_n$ isometrically onto that of $f_{n+1}$, and $w$ maps the range of $f_n$ isometrically onto that of $e_{n+1}$. Defining $e_{\infty}\coloneq \inf_{n} e_n$ and $f_{\infty} = \inf_n f_n$, we have that $v$ maps the range of $e_{\infty}$ onto that of $f_{\infty}$, and $w$ that of $f_{\infty}$ onto the range of $e_{\infty}$. Note that we have $e_{\infty}\sim f_{\infty}$.

  As discussed earlier, we have that $e_n-e_{n+1}\sim f_{n+1}-f_{n+2}$, so since sums of pairwise orthogonal families of projections respects equivalence, we have
  \begin{align*}
    \sum_{n=0}^{\infty} \left( e_{2n}-e_{2n+1} \right) &\sim \sum_{n=0}^{\infty}\left( f_{2n+1}-f_{2n+2} \right)\\
    \sum_{n=0}^{\infty} \left( e_{2n+1}-e_{2n+2} \right) &\sim \sum_{n=0}^{\infty}\left( f_{2n}-f_{2n+1} \right).
  \end{align*}
  Therefore, we get
  \begin{align*}
    e &= e_{\infty} + \sum_{n=0}^{\infty} \left( e_{2n}-e_{2n+1} \right) + \sum_{n=0}^{\infty} \left( e_{2n+1}-e_{2n+2} \right)\\
      &\sim f_{\infty} + \sum_{n=0}^{\infty} \left( f_{2n+1}-f_{2n+2} \right) + \sum_{n=0}^{\infty} \left( f_{2n}-f_{2n+1} \right)\\
      &= f.
  \end{align*}
\end{proof}
\subsection{Central Projections and the Comparison Theorem}%
The projections in a von Neumann algebra form a complete lattice, as the collection of closed subspaces of $H$ form a complete lattice under the operations
\begin{align*}
  \bigvee_{i\in I}X_i &\coloneq \overline{\sum_{i\in I}X_i}\\
  \bigwedge_{i\in I}X_i &\coloneq \bigcap_{i\in I} X_i.
\end{align*}
If $S\subseteq H$ is any subset, then we will define the range projection of $S$ by
\begin{align*}
  \left[ S \right] &\coloneq P_{ \overline{\Span}(S) }.
\end{align*}
\begin{proposition}
  If $M\subseteq B(H)$ is a von Neumann algebra, and $x\in M$, then $\left[ xH \right]$ and $\left[ x^{\ast}H \right]$ are in $M$, with $\left[ xH \right] \sim \left[ x^{\ast}H \right]$ in $M$.
\end{proposition}
\begin{proof}
  Let $x = v\left\vert x \right\vert$ be the polar decomposition. Note that $v\in M$. Now, $vv^{\ast}$ is the projection onto $ \overline{xH} $ and $ v^{\ast}v  $ is the projection onto $\ker\left( x \right)^{\perp} = \overline{x^{\ast}H}$. Thus, these projections are equivalent in $M$.
\end{proof}
\begin{definition}
  Let $x\in M$. We define the \textit{central support} to be the projection
  \begin{align*}
    z(x) &= \inf\set{w\in P\left(Z(M)\right) | xw = wx = x}.
  \end{align*}
  We say $p$ and $q$ are centrally orthogonal if $z(p)z(q) = 0$.
\end{definition}
\begin{lemma}
  If $M\subseteq B(H)$ is a von Neumann algebra, then the central support of any $p\in P(M)$ is given by
  \begin{align*}
    z(p) &= \left[ MpH \right].
  \end{align*}
  Let $w = \left[ MpH \right]$. Since $M$ is unital, it follows that $p\leq w$, and since $ \overline{MpH} $ is a reducing subspace for both $M$ and $M'$, we have $w\in M\cap M'$, so $z(p)\leq w$.

  Conversely, if $x\in M$, then
  \begin{align*}
    xpH &= x z(p) p H\\
        &= z(p) xpH,
  \end{align*}
  meaning that $\left[ xpH \right]\leq z(p)$, so $w\leq z(p)$ as $x$ was arbitrary.
\end{lemma}
\begin{proposition}
  Let $M$ be a von Neumann algebra, and let $p,q\in P(M)$ be projections. The following are equivalent:
  \begin{enumerate}[(i)]
    \item $p$ and $q$ are centrally orthogonal;
    \item $pMq = \set{0}$;
    \item there do not exist projections $0 < p_0 \leq p$ and $0 < q_0\leq q$ with $p_0\sim q_0$.
  \end{enumerate}
\end{proposition}
\begin{proof}
  Let $p$ and $q$ be centrally orthogonal. Then, for any $x\in M$, we have
  \begin{align*}
    pxq &= pz(p)xz(q)q\\
        &= pxz(p)z(q)q\\
        &= 0.
  \end{align*}
  Therefore, $pMq = \set{0}$. Now, if $pMq = \set{0}$, then $pz(q) = \left[ MqH \right] = 0$, so $p \leq 1-z(q)$. Since $1-z(q)\in Z(M)$, we have $z(p) \leq 1-z(q)$, meaning that $z(p)z(q) = 0$. Therefore, (i) and (ii) are equivalent.

  Suppose (ii) is not the case. Let $x\in M$ be such that $pxq\neq 0$. Then, $qx^{\ast}p \neq 0$. Defining
  \begin{align*}
    p_0 = \left[ pxqH \right]\\
    q_0 &= \left[ qx^{\ast}pH \right],
  \end{align*}
  we have that $p_0\leq p$, $q_0\leq q$, and since $\left( pxq \right)^{\ast} = qx^{\ast}p$, we have $p_0\sim q_0$.

  Now, if there are nonzero projections $p_0\leq p$ and $q_0 \leq q$ such that $p_0\sim q_0$, then if $v$ is a partial isometry with $v^{\ast}v = p_0$, $vv^{\ast} = q_0$, then $v^{\ast} = p_0 v^{\ast}q_0$, meaning
  \begin{align*}
    pv^{\ast}q &= pp_0v^{\ast}q_0q\\
               &= p_0v^{\ast}q_0\\
               &= v^{\ast}\\
               &\neq 0,
  \end{align*}
  meaning $pMq \neq \set{0}$.
\end{proof}
\begin{theorem}[Comparison Theorem]
  Let $M\subseteq B(H)$ be a von Neumann algebra. For any $p,q\in P(M)$, there is a central projection $z\in P(Z(M))$ such that $pz\preceq qz$ and $q\left( 1-z \right)\preceq p\left( 1-z \right)$.
\end{theorem}
\begin{proof}
  By Zorn's Lemma, there exist maximal families $\set{p_i}_{i\in I}$ and $\set{q_i}_{i\in I}$ of pairwise orthogonal projections with $p_i\sim q_i$ and, setting
  \begin{align*}
    p_0 &= \sum_{i\in I}p_i\\
    q_0 &= \sum_{i\in I}q_i,
  \end{align*}
  we have $p_0\preceq q_0$. From above, we have that $p_0\sim q_0$.

  Let $w \coloneq z\left( q-q_0 \right)$. Since $\set{p_i}_{i\in I}$ and $\set{q_i}_{i\in I}$ are maximal, it follows that $z\left( q-q_0 \right)$ and $z\left( p-p_0 \right)$ are centrally orthogonal, yielding $\left( p-p_0 \right)w = 0$, meaning $pw = p_0 w$.

  If we let $v$ be a partial isometry implementing the equivalence $p_0\sim q_0$, then we have that $vw$ is a partial isometry implementing the equivalence $p_0w\sim q_0 w$. Therefore, we have
  \begin{align*}
    pw &= p_0 w\\
       &\sim q_0 w\\
       &\leq  q.
  \end{align*}
  Similarly, $p_0\left( 1-w \right)\sim q_0\left( 1-w \right)$, so since $q-q_0 \leq w$, we have $q\left( 1-w \right)\preceq p\left( 1-w \right)$.
\end{proof}
Recall that a factor is a von Neumann algebra $M$ such that $Z(M) = \C 1$.
\begin{corollary}
  If $M$ is a factor, then any two projections in $M$ can be compared.
\end{corollary}
\section{The Type Decomposition}%
\begin{definition}
  Let $M$ be a von Neumann algebra, and $p\in B(H)$ a projection not necessarily in $M$. The algebra $pMp$ is known as a corner (or compression) of $M$.
\end{definition}
\begin{theorem}
  Let $M\subseteq B(H)$ be a von Neumann algebra, and let $p\in P(M)$. Then, $pMp$ and $M'p$ are von Neumann algebras in $B(pH)$, and $\left( pMp \right)' = M'p$, $\left( M'p \right)' = pMp$.
\end{theorem}
\nocite{davidson_functional_analysis,conway_operator_theory}
\printbibliography
\end{document}
