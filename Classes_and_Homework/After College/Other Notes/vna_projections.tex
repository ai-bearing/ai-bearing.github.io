\documentclass[12pt]{mypackage}

%\usepackage{mlmodern}
%\usepackage{newpxtext,eulerpx,eucal}
%\renewcommand*{\mathbb}[1]{\varmathbb{#1}}

%\usepackage{homework}
\usepackage{notes}

\usepackage[ backend=bibtex, style = alphabetic, sorting=ynt ]{biblatex}
\addbibresource{all_references.bib}

\usepackage{parskip}

\fancyhf{}
\fancyhead[R]{Avinash Iyer}
\fancyhead[L]{Projections in von Neumann Algebras}
\fancyfoot[C]{\thepage}

\setcounter{secnumdepth}{0}

\begin{document}
\RaggedRight
Here, we overview and discuss some of the most important results related to projections in von Neumann algebras.
\section{Comparison of Projections: An Introduction}%
Recall that if $H$ is a Hilbert space, an element $w\in B(H)$ is called a partial isometry if, for any $h\in  \ker\left( w \right) ^{\perp}$, we have $\norm{Wh} = \norm{h}$. We call $\ker\left( w \right)^{\perp}$ the initial space of $W$ and $\img(w)$ the final space of $W$.

There are a variety of equivalent definitions for partial isometries.
\begin{proposition}
  If $w\in B(H)$, then the following are equivalent:
  \begin{enumerate}[(i)]
    \item $w$ is a partial isometry;
    \item $w^{\ast}$ is a partial isometry;
    \item $w^{\ast}w$ is a projection onto the initial space of $w$;
    \item $ww^{\ast}$ is a projection onto the final space of $w$;
    \item $ww^{\ast}w = w$;
    \item $w^{\ast}ww^{\ast} = w^{\ast}$.
  \end{enumerate}
\end{proposition}
If $M\subseteq B(H)$ is a von Neumann algebra, then we say two projections $p,q\in P(M)$, where $P(M)$ denotes the space of projections of $M$, are (Murray--von Neumann) \textit{equivalent} in $M$ if there is a partial isometry $v\in P(M)$ such that $v^{\ast}v = p$ and $vv^{\ast} = q$. We will write $p\sim q$.

Note that projections have an ordering by saying that $p\leq q$ if $pq = qp = p$, or $\img(p)\subseteq \img(q) $. This allows us to say that $p$ is \textit{sub-equivalent} to $q$ (in $M$), written $p\preceq q$, if there is a partial isometry $v\in M$ such that $v^{\ast}v = p$ and $vv^{\ast}\leq q$. In this scenario we will say that $q$ majorizes $p$; the choice for this vocabulary will be seen below.

The sub-equivalence relation in fact forms a partial order, and equivalence as projections forms an equivalence relation. We will first show that it is a preorder.
\begin{proposition}
  In a von Neumann algebra, the relation $\sim$ is an equivalence relation on $P(M)$, and the relation $\preceq$ is a preorder.
\end{proposition}
\begin{proof}
  Reflexivity follows from the fact that projections are partial isometries, and symmetry follows from the fact that if $v$ is a partial isometry, then so is $v^{\ast}$.

  Now, we will show transitivity for $\preceq$, from which we will see that $\sim$ is transitive. Letting $p,q,r\in P(M)$ be such that $p\preceq q$ and $q\preceq r$, we have partial isometries $u,v\in M$ with $u^{\ast}u = p$, $uu^{\ast}\leq q$, $v^{\ast}v = q$, and $vv^{\ast}\leq r$. Then, we have
  \begin{align*}
    qu &= q uu^{\ast}u\\
       &= \left( quu^{\ast} \right)u\\
       &= uu^{\ast}u\\
       &= u,
  \end{align*}
  so that
  \begin{align*}
    \left( vu \right)^{\ast}\left( vu \right) &= u^{\ast}v^{\ast}vu\\
                                              &= u^{\ast}qu\\
                                              &= u^{\ast}u\\
                                              &= p\\
    \left( vu \right)\left( vu \right)^{\ast} &= vuu^{\ast}v^{\ast}\\
                                              &\leq vqv^{\ast}\\
                                              &= vv^{\ast}vv^{\ast}\\
                                              &= vv^{\ast}\\
                                              &\leq r.
  \end{align*}
  Therefore, $p\preceq r$, so $\preceq$ is a transitive relation.
\end{proof}
To see that $\preceq$ is a partial order, we need an analogue of the Cantor--Schröder--Bernstein theorem for projections. In fact, it is proven in a similar manner.
\begin{theorem}
  If $e\preceq f$ and $f\preceq e$, then $e\sim f$.
\end{theorem}
The projections in a von Neumann algebra form a complete lattice, as the collection of closed subspaces of $H$ form a complete lattice under the operations
\begin{align*}
  \bigvee_{i\in I}X_i &\coloneq \overline{\sum_{i\in I}X_i}\\
  \bigwedge_{i\in I}X_i &\coloneq \bigcap_{i\in I} X_i.
\end{align*}
\nocite{davidson_functional_analysis,conway_operator_theory}
\printbibliography
\end{document}
