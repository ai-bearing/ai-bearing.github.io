\documentclass[12pt]{mypackage}

%\usepackage{mlmodern}
%\usepackage{newpxtext,eulerpx,eucal}
%\renewcommand*{\mathbb}[1]{\varmathbb{#1}}

%\usepackage{homework}
\usepackage{notes}

\usepackage[ backend=bibtex, style = alphabetic, sorting=ynt ]{biblatex}
\addbibresource{all_references.bib}

\usepackage{parskip}

\fancyhf{}
\fancyhead[R]{Avinash Iyer}
\fancyhead[L]{Projections in von Neumann Algebras}
\fancyfoot[C]{\thepage}

\setcounter{secnumdepth}{0}

\begin{document}
\RaggedRight
Here, we overview and discuss some of the most important results related to projections in von Neumann algebras.
\section{Comparison of Projections}%
Recall that if $H$ is a Hilbert space, an element $w\in B(H)$ is called a partial isometry if, for any $h\in  \ker\left( w \right) ^{\perp}$, we have $\norm{Wh} = \norm{h}$. We call $\ker\left( w \right)^{\perp}$ the initial space of $W$ and $\img(w)$ the final space of $W$.

There are a variety of equivalent definitions for partial isometries.
\begin{proposition}
  If $w\in B(H)$, then the following are equivalent:
  \begin{enumerate}[(i)]
    \item $w$ is a partial isometry;
    \item $w^{\ast}$ is a partial isometry;
    \item $w^{\ast}w$ is a projection onto the initial space of $w$;
    \item $ww^{\ast}$ is a projection onto the final space of $w$;
    \item $ww^{\ast}w = w$;
    \item $w^{\ast}ww^{\ast} = w^{\ast}$.
  \end{enumerate}
\end{proposition}
\begin{theorem}[Polar Decomposition]
  Let $a\in B(H)$. Then, there is a partial isometry $w\in B(H)$ with initial space $ \ker\left( a \right)^{\perp} $ and final space $ \overline{\img(a)} $ such that $a = w\left\vert a \right\vert$.

  If $a\in M\subseteq B(H)$, where $M$ is a von Neumann algebra, then both $\left\vert a \right\vert$ and $w$ are in $M$.
\end{theorem}
\subsection{Equivalence of Projections}%
If $M\subseteq B(H)$ is a von Neumann algebra, then we say two projections $p,q\in P(M)$, where $P(M)$ denotes the space of projections of $M$, are (Murray--von Neumann) \textit{equivalent} in $M$ if there is a partial isometry $v\in P(M)$ such that $v^{\ast}v = p$ and $vv^{\ast} = q$. We will write $p\sim q$.

Note that projections have an ordering by saying that $p\leq q$ if $pq = qp = p$, or $\img(p)\subseteq \img(q) $. This allows us to say that $p$ is \textit{sub-equivalent} to $q$ (in $M$), written $p\preceq q$, if there is a partial isometry $v\in M$ such that $v^{\ast}v = p$ and $vv^{\ast}\leq q$.\footnote{We will say that the projection $q$ majorizes $p$ if $p\preceq q$, and we will say that $q$ dominates $p$ if $p\leq q$.}

The sub-equivalence relation in fact forms a partial order, and equivalence as projections forms an equivalence relation. We will first show that it is a preorder.
\begin{proposition}
  In a von Neumann algebra, the relation $\sim$ is an equivalence relation on $P(M)$, and the relation $\preceq$ is a preorder.
\end{proposition}
\begin{proof}
  Reflexivity follows from the fact that projections are partial isometries, and symmetry follows from the fact that if $v$ is a partial isometry, then so is $v^{\ast}$.

  Now, we will show transitivity for $\preceq$, from which we will see that $\sim$ is transitive. Letting $p,q,r\in P(M)$ be such that $p\preceq q$ and $q\preceq r$, we have partial isometries $u,v\in M$ with $u^{\ast}u = p$, $uu^{\ast}\leq q$, $v^{\ast}v = q$, and $vv^{\ast}\leq r$. Then, we have
  \begin{align*}
    qu &= q uu^{\ast}u\\
       &= \left( quu^{\ast} \right)u\\
       &= uu^{\ast}u\\
       &= u,
  \end{align*}
  so that
  \begin{align*}
    \left( vu \right)^{\ast}\left( vu \right) &= u^{\ast}v^{\ast}vu\\
                                              &= u^{\ast}qu\\
                                              &= u^{\ast}u\\
                                              &= p\\
    \left( vu \right)\left( vu \right)^{\ast} &= vuu^{\ast}v^{\ast}\\
                                              &\leq vqv^{\ast}\\
                                              &= vv^{\ast}vv^{\ast}\\
                                              &= vv^{\ast}\\
                                              &\leq r.
  \end{align*}
  Therefore, $p\preceq r$, so $\preceq$ is a transitive relation.
\end{proof}
To see that $\preceq$ is a partial order, we need an analogue of the Cantor--Schröder--Bernstein theorem for projections. In fact, it can be proven in a similar manner. First, we discuss a simple lemma.
\begin{lemma}
  Let $M\subseteq B(H)$ be a von Neumann algebra. If $\set{p_i}_{i\in I}$ and $\set{q_i}_{i\in I}$ are pairwise orthogonal families of projections with $p_i\preceq q_i$, then $\sum_{i\in I}p_i\preceq \sum_{i\in I}q_i$.
\end{lemma}
\begin{proof}
  Let $u_i$ be the partial isometries with $u_i^{\ast}u_i = p_i$ and $r_i\coloneq u_iu_i^{\ast}\leq q_i$. Then, the $r_i$ are pairwise orthogonal since the $q_i$ are pairwise orthogonal, and for any $i\neq j$,
  \begin{align*}
    u_i^{\ast}u_j &= u_i^{\ast}u_iu_i^{\ast}u_ju_j^{\ast}u_j\\
                  &= u_ir_ir_ju_j\\
                  &= 0\\
    u_iu_j^{\ast} &= u_iu_i^{\ast}u_iu_j^{\ast}u_ju_j^{\ast}\\
                  &= u_ip_ip_ju_j^{\ast}\\
                  &= 0.
  \end{align*}
  Consequently, we get
  \begin{align*}
    \left( \sum_{i\in I}u_i^{\ast} \right)\left( \sum_{j\in I}u_j \right) &= \sum_{i\in I}u_i^{\ast}u_i\\
                                                                          &= \sum_{i\in I}p_i\\
    \left( \sum_{i\in I}u_i \right)\left( \sum_{j\in I}u_j^{\ast} \right) &= \sum_{i\in I}u_iu_i^{\ast}\\
                                                                          &\leq \sum_{i\in I}q_i.
  \end{align*}
  This gives $\sum_{i\in I}p_i\preceq \sum_{i\in I}q_i$.
\end{proof}
\begin{theorem}
  If $e\preceq f$ and $f\preceq e$, then $e\sim f$.
\end{theorem}
\begin{proof}
  We will let $e_0 \coloneq e$ and $f_0 \coloneq f$. Let $v$ and $w$ be partial isometries with $v^{\ast}v = e$, $vv^{\ast} = f_1\leq f$, $w^{\ast}w = f$, $ww^{\ast} = e_1\leq e$. Inductively define a sequence of projections as follows.

  Since $v$ maps the range of $e_1$ isometrically onto the range of some projection dominated by $f_1$, it follows that we may write $f_2\coloneq ve_1 \left( ve_1 \right)^{\ast}$ with $f_2\leq f_1$. Since $w$ maps the range of $f_1$ onto the range of some projection dominated by $e_1$, it follows that we may write $wf_1 \left( wf_1 \right)^{\ast}\eqcolon e_2$. Observe also that $v\left( e-e_1 \right)$ is a partial isometry with initial projection $e-e_1$ and final projection $f_1-f_2$.

  Inductively, we obtain decreasing sequences of projections $\left( e_n \right)_n$ and $\left( f_n \right)_n$ where $v$ maps the range of $e_n$ isometrically onto that of $f_{n+1}$, and $w$ maps the range of $f_n$ isometrically onto that of $e_{n+1}$. Defining $e_{\infty}\coloneq \inf_{n} e_n$ and $f_{\infty} = \inf_n f_n$, we have that $v$ maps the range of $e_{\infty}$ onto that of $f_{\infty}$, and $w$ that of $f_{\infty}$ onto the range of $e_{\infty}$. Note that we have $e_{\infty}\sim f_{\infty}$.

  As discussed earlier, we have that $e_n-e_{n+1}\sim f_{n+1}-f_{n+2}$, so since sums of pairwise orthogonal families of projections respects equivalence, we have
  \begin{align*}
    \sum_{n=0}^{\infty} \left( e_{2n}-e_{2n+1} \right) &\sim \sum_{n=0}^{\infty}\left( f_{2n+1}-f_{2n+2} \right)\\
    \sum_{n=0}^{\infty} \left( e_{2n+1}-e_{2n+2} \right) &\sim \sum_{n=0}^{\infty}\left( f_{2n}-f_{2n+1} \right).
  \end{align*}
  Therefore, we get
  \begin{align*}
    e &= e_{\infty} + \sum_{n=0}^{\infty} \left( e_{2n}-e_{2n+1} \right) + \sum_{n=0}^{\infty} \left( e_{2n+1}-e_{2n+2} \right)\\
      &\sim f_{\infty} + \sum_{n=0}^{\infty} \left( f_{2n+1}-f_{2n+2} \right) + \sum_{n=0}^{\infty} \left( f_{2n}-f_{2n+1} \right)\\
      &= f.
  \end{align*}
\end{proof}
\subsection{Central Projections and the Comparison Theorem}%
The projections in a von Neumann algebra form a complete lattice, as the collection of closed subspaces of $H$ form a complete lattice under the operations
\begin{align*}
  \bigvee_{i\in I}X_i &\coloneq \overline{\sum_{i\in I}X_i}\\
  \bigwedge_{i\in I}X_i &\coloneq \bigcap_{i\in I} X_i.
\end{align*}
If $S\subseteq H$ is any subset, then we will define the range projection of $S$ by
\begin{align*}
  \left[ S \right] &\coloneq P_{ \overline{\Span}(S) }.
\end{align*}
\begin{proposition}
  If $M\subseteq B(H)$ is a von Neumann algebra, and $x\in M$, then $\left[ xH \right]$ and $\left[ x^{\ast}H \right]$ are in $M$, with $\left[ xH \right] \sim \left[ x^{\ast}H \right]$ in $M$.
\end{proposition}
\begin{proof}
  Let $x = v\left\vert x \right\vert$ be the polar decomposition. Note that $v\in M$. Now, $vv^{\ast}$ is the projection onto $ \overline{xH} $ and $ v^{\ast}v  $ is the projection onto $\ker\left( x \right)^{\perp} = \overline{x^{\ast}H}$. Thus, these projections are equivalent in $M$.
\end{proof}
\begin{definition}
  Let $x\in M$. We define the \textit{central support} to be the projection
  \begin{align*}
    z(x) &= \inf\set{w\in P\left(Z(M)\right) | xw = wx = x}.
  \end{align*}
  We say $p$ and $q$ are centrally orthogonal if $z(p)z(q) = 0$.
\end{definition}
\begin{lemma}
  If $M\subseteq B(H)$ is a von Neumann algebra, then the central support of any $p\in P(M)$ is given by
  \begin{align*}
    z(p) &= \left[ MpH \right].
  \end{align*}
  Let $w = \left[ MpH \right]$. Since $M$ is unital, it follows that $p\leq w$, and since $ \overline{MpH} $ is a reducing subspace for both $M$ and $M'$, we have $w\in M\cap M'$, so $z(p)\leq w$.

  Conversely, if $x\in M$, then
  \begin{align*}
    xpH &= x z(p) p H\\
        &= z(p) xpH,
  \end{align*}
  meaning that $\left[ xpH \right]\leq z(p)$, so $w\leq z(p)$ as $x$ was arbitrary.
\end{lemma}
\begin{proposition}
  Let $M$ be a von Neumann algebra, and let $p,q\in P(M)$ be projections. The following are equivalent:
  \begin{enumerate}[(i)]
    \item $p$ and $q$ are centrally orthogonal;
    \item $pMq = \set{0}$;
    \item there do not exist projections $0 < p_0 \leq p$ and $0 < q_0\leq q$ with $p_0\sim q_0$.
  \end{enumerate}
\end{proposition}
\begin{proof}
  Let $p$ and $q$ be centrally orthogonal. Then, for any $x\in M$, we have
  \begin{align*}
    pxq &= pz(p)xz(q)q\\
        &= pxz(p)z(q)q\\
        &= 0.
  \end{align*}
  Therefore, $pMq = \set{0}$. Now, if $pMq = \set{0}$, then $pz(q) = \left[ MqH \right] = 0$, so $p \leq 1-z(q)$. Since $1-z(q)\in Z(M)$, we have $z(p) \leq 1-z(q)$, meaning that $z(p)z(q) = 0$. Therefore, (i) and (ii) are equivalent.

  Suppose (ii) is not the case. Let $x\in M$ be such that $pxq\neq 0$. Then, $qx^{\ast}p \neq 0$. Defining
  \begin{align*}
    p_0 = \left[ pxqH \right]\\
    q_0 &= \left[ qx^{\ast}pH \right],
  \end{align*}
  we have that $p_0\leq p$, $q_0\leq q$, and since $\left( pxq \right)^{\ast} = qx^{\ast}p$, we have $p_0\sim q_0$.

  Now, if there are nonzero projections $p_0\leq p$ and $q_0 \leq q$ such that $p_0\sim q_0$, then if $v$ is a partial isometry with $v^{\ast}v = p_0$, $vv^{\ast} = q_0$, then $v^{\ast} = p_0 v^{\ast}q_0$, meaning
  \begin{align*}
    pv^{\ast}q &= pp_0v^{\ast}q_0q\\
               &= p_0v^{\ast}q_0\\
               &= v^{\ast}\\
               &\neq 0,
  \end{align*}
  meaning $pMq \neq \set{0}$.
\end{proof}
\begin{theorem}[Comparison Theorem]
  Let $M\subseteq B(H)$ be a von Neumann algebra. For any $p,q\in P(M)$, there is a central projection $z\in P(Z(M))$ such that $pz\preceq qz$ and $q\left( 1-z \right)\preceq p\left( 1-z \right)$.
\end{theorem}
\begin{proof}
  By Zorn's Lemma, there exist maximal families $\set{p_i}_{i\in I}$ and $\set{q_i}_{i\in I}$ of pairwise orthogonal projections with $p_i\sim q_i$ and, setting
  \begin{align*}
    p_0 &= \sum_{i\in I}p_i\\
    q_0 &= \sum_{i\in I}q_i,
  \end{align*}
  we have $p_0\preceq q_0$. From above, we have that $p_0\sim q_0$.

  Let $w \coloneq z\left( q-q_0 \right)$. Since $\set{p_i}_{i\in I}$ and $\set{q_i}_{i\in I}$ are maximal, it follows that $z\left( q-q_0 \right)$ and $z\left( p-p_0 \right)$ are centrally orthogonal, yielding $\left( p-p_0 \right)w = 0$, meaning $pw = p_0 w$.

  If we let $v$ be a partial isometry implementing the equivalence $p_0\sim q_0$, then we have that $vw$ is a partial isometry implementing the equivalence $p_0w\sim q_0 w$. Therefore, we have
  \begin{align*}
    pw &= p_0 w\\
       &\sim q_0 w\\
       &\leq  q.
  \end{align*}
  Similarly, $p_0\left( 1-w \right)\sim q_0\left( 1-w \right)$, so since $q-q_0 \leq w$, we have $q\left( 1-w \right)\preceq p\left( 1-w \right)$.
\end{proof}
Recall that a factor is a von Neumann algebra $M$ such that $Z(M) = \C 1$.
\begin{corollary}
  If $M$ is a factor, then any two projections in $M$ can be compared.
\end{corollary}
\section{The Type Decomposition}%
\begin{definition}
  Let $M$ be a von Neumann algebra, and $p\in B(H)$ a projection not necessarily in $M$. The algebra $pMp$ is known as a corner (or compression) of $M$.
\end{definition}
\begin{theorem}
  Let $M\subseteq B(H)$ be a von Neumann algebra, and let $p\in P(M)$. Then, $pMp$ and $M'p$ are von Neumann algebras in $B(pH)$, and $\left( pMp \right)' = M'p$, $\left( M'p \right)' = pMp$.
\end{theorem}
\begin{corollary}
  If $M$ is a factor and $p\in P(M)$, then $pMp$ and $M'p$ are both factors.
\end{corollary}
\begin{definition}
  Let $M\subseteq B(H)$ be a von Neumann algebra. We say a projection $p\in P(M)$ is
  \begin{itemize}
    \item minimal if $p\neq 0$ and the only subprojections of $p$ are $0$ and $p$;
    \item abelian if $pMp$ is abelian;
    \item finite if $q\leq p$ and $q\sim p$ implies $q = p$;
    \item semifinite if there are pairwise orthogonal finite projections $p_i\in P(M)$ such that $p = \sum_{i\in I}p_i$;
    \item purely infinite if $p\neq 0$ and there is no nonzero finite projection $q\leq p$;
    \item properly infinite if $p\neq 0$ and $zp$ is not finite for any nonzero central projection $z\in Z(P(M))$.
  \end{itemize}
  We say that the von Neumann algebra $M$ is finite/semifinite/purely infinite/properly infinite if the projection $1$ satisfies its respective condition. Additionally, if $M$ has no minimal projections, we say it is diffuse.
\end{definition}
What we will be working towards is known as the type decomposition, which forms the basis for the dimension theory of von Neumann algebras. Eventually, we will show that every von Neumann algebra $M$ can be decomposed as
\begin{align*}
  M &= M_{\text{sf}}\oplus M_{\text{III}},
\end{align*}
where $M_{\text{sf}}$ is a semifinite von Neumann algebra and $M_{\text{III}}$ is a type III von Neumann algebra, a definition we will discuss shortly. First, we must expand on some of the ways that comparison of projections interacts with these properties of those projections.
\begin{lemma}
  Let $\set{p_i}_{i\in I}$ be a family of centrally orthogonal projections in a von Neumann algebra $M\subseteq B(H)$. If each $p_i$ is abelian (finite), then the sum $\sum_{i\in I}p_i$ is also abelian (finite).
\end{lemma}
\begin{proof}
  If each $p_i$ is abelian, then since they are centrally orthogonal for any $i\neq j$, then for any $x,y\in M$, we have $p_ixpyp_j = 0$. Therefore, we have
  \begin{align*}
    \left( pxp \right)\left( pyp \right) &= \sum_{i\in I}p_ixp_iyp_i\\
                                         &= \left( pyp \right)\left( pxp \right),
  \end{align*}
  so $p$ is abelian.

  Now, if each $p_i$ is finite, and $u\in M$ is such that $uu^{\ast}\leq u^{\ast}u = p$, then for all $i$ we have $z\left( p_i \right)u^{\ast}uz\left( p_i \right) = p$, and $uz\left( p_i \right)u^{\ast} = z\left( p_i \right)uu^{\ast} \leq p_i$, meaning $uz\left( p_i \right)u^{\ast} = p_i$, and
  \begin{align*}
    uu^{\ast} &= uz(p)u^{\ast}\\
              &= \sum_{i\in I} uz\left( p_i \right)u^{\ast}\\
              &= p.
  \end{align*}
\end{proof}
\begin{proposition}
  Let $p,q$ be nonzero projections in a von Neumann algebra with $p\preceq q$. If $q$ is finite (purely infinite), then $p$ is also finite (purely infinite).
\end{proposition}
\begin{proof}
  Suppose $q$ is finite, and $p\sim q$, with $v\in M$ implementing the equivalence. Let $u\in M$ be such that $u^{\ast}u = p$ and $uu^{\ast}\leq p$. Then,
  \begin{align*}
    \left( vuv^{\ast} \right)^{\ast}\left( vuv^{\ast} \right) &= q\\
    \left( vuv^{\ast} \right)\left( vuv^{\ast} \right)^{\ast} &\leq q,
  \end{align*}
  so since $q$ is finite, $\left( vuv^{\ast}\right)\left( vuv^{\ast} \right)^{\ast} = q$, so $uu^{\ast} = p$.

  Now, if $p\leq q$, then if $u^{\ast}u = p$ with $uu^{\ast}\leq p$, then by setting $w = u + \left( q-p \right)$, we have that $w^{\ast}w = q$ and $ww^{\ast}\leq q$, so $u^{\ast}u + \left( q-p \right) = ww^{\ast} = q$, meaning $uu^{\ast} = p$.

  In the general case, we have some $q_0\leq q$ such that $p\sim q_0 \leq q$.

  Since projections are purely infinite when they have no nonzero finite subprojections, the purely infinite case follows from the finite case.
\end{proof}
\begin{proposition}
  A projection $p\in P(M)$ is semifinite if and only if $p$ is the supremum of finite projections. In particular, a supremum of semifinite projections is also semifinite.
\end{proposition}
\begin{proof}
  If $p$ is semifinite, then $p$ is the sum (and hence the supremum) of a family of pairwise orthogonal finite projections.

  Now, suppose $p = \bigvee_{\alpha}p_{\alpha}$, where each $p_{\alpha}$ is finite. Let $\set{q_{\beta}}_{\beta}$ be a maximal family of pairwise orthogonal finite subprojections of $p$. If we set
  \begin{align*}
    q_0 &= p - \sum_{\beta}q_{\beta},
  \end{align*}
  and suppose that $q_0\neq 0$, then there exists some $p_{\alpha}$ such that $p_{\alpha}$ and $q_0$ are not orthogonal (else it would contradict maximality), so they are not centrally orthogonal. Therefore, we have a nonzero subprojection $q_1 \leq q_0$ such that $q_1\preceq p_{\alpha}$, so it is finite by what we showed previously; this contradicts the maximality of the set $\set{q_{\beta}}_{\beta}$.
\end{proof}
\begin{corollary}
  Let $p$ be a projection in a von Neumann algebra $M$. If $p$ is semifinite (purely infinite), then the central support $z(p)$ is also semifinite (purely infinite).
\end{corollary}
\begin{proof}
  The central support is the supremum over all equivalent projections, so since the supremum of semifinite projections is again semifinite, it follows from the previous proposition.

  Furthermore, a nonzero projection is purely infinite if and only if it is centrally orthogonal to every semifinite projection, so we obtain the corollary in this case.
\end{proof}
\begin{corollary}
  Let $p,q$ be nonzero projections in a von Neumann algebra such that $p\preceq q$. If $q$ is semifinite, then so is $p$.
\end{corollary}
\begin{proof}
  It is enough to consider the case when $q$ is central, in which case we may take $p\leq q$. Let $p_0$ be the maximal semifinite subprojection of $p$. Since $q$ is semifinite, it is the supremum of its finite subprojections. Since $z\left( p-p_0 \right)\leq q = z(q)$, it follows that if $p-p_0\neq 0$, then there would exist a nonzero finite subprojection that would be equivalent to a subprojection of $p-p_0$, which contradicts the definition of $p_0$. Thus, $p$ is the supremum of its finite subprojections, so it is semifinite.
\end{proof}
\begin{lemma}
  Let $M$ be a properly infinite von Neumann algebra. Then, there exists a projection $p\in P(M)$ such that $p\sim 1-p \sim 1$.
\end{lemma}
\begin{proof}
  Since $M$ is properly infinite, there exists $u\in M$ with $uu^{\ast}\leq u^{\ast}u = 1$. Set $p_0 = 1-uu^{\ast}$. Then, $p_n = u^{n}p_0\left( u^n \right)^{\ast}$ is a pairwise orthogonal family of equivalent projections. Let $\set{q_i}_{i\in I}$ be a maximal family of pairwise orthogonal equivalent projections in $M$ extending $\set{p_n}_{n\in \N}$, and set $q_0 = 1-\sum_{i\in I}q_i$.

  By the comparison theorem, there is $z\in Z(P(M))$ such that $q_0z\leq q_{i_0}z$ and $q_{i_0}\left( 1-z \right)\leq q_0\left( 1-z \right)$. If it were the case that $z = 0$, then we would have $q_{i_0}\leq q_0$, contradicting maximality of $\set{q_i}_{i\in I}$,  so $z\neq 0$, and we have
  \begin{align*}
    z &= q_0 z + \sum_{i\in I}q_i z\\
      &\preceq q_{i_0}z + \sum_{i\neq i_0} q_{i} z\\
      &= \sum_{i\in I}q_i z \leq z,
  \end{align*}
  so $z\sim \sum_{i\in i}q_i z$ by Cantor--Schröder--Bernstein for projections. Decomposing $\set{q_i}_{i\in I}$ into infinite subsets, we may construct two projections $p$ and $z-p$ such that $p \sim z-p \sim z$.

  Now, let $\set{r_j}_{j\in J}$ be a maximal family of centrally orthogonal projections with $r_j\sim z\left( r_j \right)-r_j\sim z\left( r_j \right)$. Then, the argument above shows that $\sum_{j\in J}z\left( r_j \right) = 1$, so by setting $p = \sum_{j\in J}r_j$, we obtain our desired result.
\end{proof}
\begin{proposition}
  Let $p$ and $q$ be finite projections in a von Neumann algebra $M$. Then, $p\vee q$ is finite.
\end{proposition}
\begin{proof}
  We use Kaplansky's formula for this, which gives $p\vee q - p \sim q-p\wedge q$. This follows from observing that $x = \left( 1-p \right)q$ has $\ker\left( x \right) = \ker\left( q \right)\oplus \left( qH \cap pH \right)$, meaning that $\left[ x^{\ast}H \right] = 1 - \left( \left( 1 - q \right) + q\wedge p \right) = q-q\wedge p$. Symmetrically, this gives
  \begin{align*}
    \left[ xH \right] &= \left( 1-p \right)-\left( 1-p \right)\wedge \left( 1-q \right)\\
                      &= p\vee q - p.
  \end{align*}
  In particular, since $q-p\wedge q \leq q$, we may assume that $p$ and $q$ are orthogonal, replacing $q$ with $p\vee q - p$. We may also assume that $p + q = 1$ by passing to $\left( p+q \right)M\left( p+q \right)$.

  Let $z_0$ be the supremum of all finite central projections. It follows that $z_0$ is finite. If $z_0 = 1$, then we are done. Else, we may use $\left( 1-z_0 \right)p$ and $\left( 1-z_0 q \right)$, wherein $z_0 = 0$ and thus we assume that $M$ is properly infinite.

  Therefore, we have a projection $r\in P(M)$ such that $r\sim 1-r \sim 1$. By comparison, there is $z\in P(Z(M))$ such that
  \begin{align*}
    z\left( p\wedge r \right) \preceq z\left( q\wedge \left( 1-r \right) \right)\\
    \left( 1-z \right)\left( q\wedge \left( 1-r \right) \right) &\preceq \left( 1-z \right)\left( p\wedge r \right).
  \end{align*}
  Additionally, $zr \sim z\left( 1-r \right)\sim z$, and
  \begin{align*}
    z\left( p\wedge r \right) &= zp\wedge zr\\
                              &\preceq z\left( 1-r \right)\wedge zq,
  \end{align*}
  so by using Kaplansky's formula, we have
  \begin{align*}
    zr &= z\left( r-r\wedge p \right) + z\left( r\wedge p \right)\\
       &\preceq z\left( r\vee p - p \right) + z\left( q\wedge \left( 1-r \right) \right)\\
       &= zq,
  \end{align*}
  Therefore, $zr = 0$ since $zq$ is finite and $M$ is properly infinite, so $z \sim 0$. In particular, this gives $q\wedge \left( 1-r \right) \preceq p\wedge r$. Replacing $p$ with $q$ and $r$ with $1-r$, we get
  \begin{align*}
    1-r &= \left( 1-r-\left( 1-r \right)\wedge q \right) + \left( \left( 1-r \right)\wedge q \right)\\
        &\preceq \left( \left( 1-r \right)\vee q - q \right) + \left( p\wedge r \right)\\
        &= p,
  \end{align*}
  which gives a contradiction since $p$ is finite.
\end{proof}
\begin{proposition}
  Let $p$ and $q$ be finite projections with $p\sim q$. Then, $1-p$ and $1-q$ are equivalent.
\end{proposition}
\begin{proof}
  We have that $p\vee q$ is finite, so we may assume that $M$ is finite. By the comparison theorem, there are projections $p_1$ and $q_1$ and a central projection $z\in P(Z(M))$ such that
  \begin{align*}
    \left( 1-p \right)z \sim q_1 &\leq \left( 1-q \right)z\\
    \left( 1-q \right)\left( 1-z \right)\sim p_1 &\leq \left( 1-p \right)\left( 1-z \right).
  \end{align*}
  Then, we have
  \begin{align*}
    z &= \left( 1-p \right)z + pz\\
      &\sim q_1 + qz\\
      &\leq \left( 1-q \right)z + qz\\
      &= z
  \end{align*}
  and
  \begin{align*}
    \left( 1-z \right) &= \left( 1-q \right)\left( 1-z \right) + q\left( 1-z \right)\\
                       &\sim p_1 + p\left( 1-z \right)\\
                       &\leq \left( 1-z \right),
  \end{align*}
  so since both $z$ and $\left( 1-z \right)$ are finite, we have $q_1 = \left( 1-q \right)z$ and $p_1 = \left( 1-p \right)\left( 1-z \right)$. Thus, $1-q\sim 1-p$.
\end{proof}

\begin{definition}
  Let $M$ be a von Neumann algebra.
  \begin{itemize}
    \item We say $M$ is type I if every nonzero central projection in $M$ majorizes a nonzero abelian projection in $M$.
    \item We say $M$ is type II\textsubscript{1} if it is finite, has no nonzero abelian projections, and every nonzero central projection in $M$ majorizes a nonzero finite projection.
    \item We say $M$ is type II\textsubscript{$\infty$} if every nonzero central projection majorizes a nonzero finite projection, and has no nonzero finite central projections.
    \item We say $M$ is type III if it is purely infinite.
  \end{itemize}
\end{definition}
\begin{theorem}
  Every von Neumann algebra $M$ uniquely decomposes into a direct sum of those of type I, II\textsubscript{1}, II\textsubscript{$\infty$}, and III. Moreover, very projection $e$ in $M$ can be uniquely written as the sum of centrally orthogonal projections $e_1$ and $e_2$ in $M$ such that $e_1$ is finite and $e_2$ is properly infinite.
\end{theorem}
\begin{proof}
  Let $\set{e_i}_{i\in I}$ be a maximal family of centrally orthogonal abelian projections in $M$, and let $e = \sum_{i\in I}e_i$. Then, $e$ is abelian; define $z_{\text{I}} = z(e)$.

  If $z$ is a nonzero central projection majorized by $z_{\text{I}}$, then $ze$ is a nonzero abelian projection, so $Mz_{\text{I}}$ is of type I.

  By construction, there is then no nonzero abelian projection in $M\left( 1-z_{\text{I}} \right)$, so it has no nontrivial direct summand of type I.

  Let $\set{f_j}_{j\in J}$ be a maximal family of centrally orthogonal finite projections in $M\left( 1-z_{\text{I}} \right)$, and let $f = \sum_{j\in J}f_j$. Then, $f$ is finite. Set $z_{\text{II}} = z(f)$. By construction, we have that $Mz_{\text{II}}$ has no nonzero abelian projections, and every nonzero projection $z$ in $M z_{\text{II}}$ majorizes a nonzero finite projection $zf$. Thus, $Mz_{\text{II}}$ is of type II.

  By the maximality of $\set{f_j}_{j\in J}$, it follows that $1-z_{\text{II}} = z_{\text{III}}$ does not majorize any finite projection, so $Mz_{\text{III}}$ is of type III. We have that $z_{\text{I}} + z_{\text{II}} + z_{\text{III}} = 1$.

  Now, let $\set{z_k}_{k\in K}$ be a maximal orthogonal family of finite central projections in $Mz_{\text{II}}$, and set $z_{\text{II\textsubscript{1}}} = \sum_{k\in K}z_k$, and $z_{\text{II\textsubscript{$\infty$}}} = z_{\text{II}} - z_{\text{II\textsubscript{1}}}$. It follows that $Mz_{\text{II\textsubscript{1}}}$ is of type II\textsubscript{1} and $Mz_{\text{II\textsubscript{$\infty$}}}$ is of type II\textsubscript{$\infty$}. Therefore, we get the direct sum decomposition
  \begin{align*}
    M &= Mz_{\text{I}} \oplus Mz_{\text{II\textsubscript{1}}}\oplus Mz_{\text{II\textsubscript{$\infty$}}} \oplus Mz_{\text{III}}.
  \end{align*}
  As for uniqueness, we suppose that there is another orthogonal decomposition $1 = w_{\text{I}} + w_{\text{II\textsubscript{1}}} + w_{\text{II\textsubscript{$\infty$}}} + w_{\text{III}}$. Then, we must have that $w_{\text{I}}\left( 1-z_{\text{I}} \right) = 0$ since $1-z_{\text{I}}$ does not majorize any nonzero abelian projection, while $w_{\text{I}}\left( 1-z_{\text{I}} \right)$ is a central projection in $Mw_{\text{I}}$. Therefore, $w_{\text{I}}\leq z_{\text{I}}$. Similarly, $z_{\text{I}}\left( 1-w_{\text{I}} \right) = 0$ for the same reason, meaning $z_{\text{I}}\leq w_{\text{I}}$, so they are equal. By similar arguments, all the other summands are equal to each other.

  Finally, we let $e$ be a nonzero projection in $M$. By considering $M_e\coloneq eMe$, we may assume that $e = 1$. We let $e_1$ be the sum of a maximal orthogonal family of finite central projections. Then, $e_1$ is finite and central, with $1-e_1$ properly infinite. The uniqueness of this decomposition has the same flavor as the arguments for $\set{z_{\text{I}},\dots,z_{\text{III}}}$.
\end{proof}
\begin{definition}
  We say a factor is atomic if it contains a minimal projection, and otherwise we say it is diffuse.
\end{definition}
\section{Structure of Type I and II von Neumann Algebras}%
Now, we will discuss some structural results related to type I and II von Neumann algebras. Before we can do this, we must discuss tensor products.
\subsection{Tensor Products of Hilbert Spaces and Operators}%
\begin{definition}
  Let $H$ and $K$ be Hilbert spaces. There is an inner product on the algebraic tensor product $H\odot K$ given by
  \begin{align*}
    \iprod{\xi_1\otimes \xi_2}{\eta_1\otimes \eta_2} &= \iprod{\xi_1}{\eta_1} \iprod{\xi_2}{\eta_2},
  \end{align*}
  whenever $\xi_1,\eta_1\in H$ and $\xi_2,\eta_2\in K$. The Hilbert space tensor product of $H$ and $K$, denoted $H\otimes K$, is the completion of $H\odot K$ with respect to the norm induced by this inner product.
\end{definition}
\begin{proposition}
  Let $H$ and $K$ be Hilbert spaces with orthonormal bases $\set{e_i}_{i\in I}$ and $\set{f_j}_{j\in J}$. Then,
  \begin{enumerate}[(i)]
    \item $\set{e_i\otimes f_j}_{i\in I,j\in J}$ is an orthonormal basis for $H\otimes K$;
    \item if $\left\vert J \right\vert = \alpha$ for some cardinal $\alpha$, then $H\otimes K\cong H^{(\alpha)}\cong \bigoplus_{j\in J}H$;
    \item if $H = L_2\left( X,\mu \right)$ for some $\sigma$-finite regular Borel measure space $\left( X,\mu \right)$, and $K$ is separable, then $H\otimes K \cong L_2\left( X,\mu,K \right)$, where the latter denotes the space of square-integrable Borel functions with respect to the norm on $K$.
  \end{enumerate}
\end{proposition}
\begin{proof}\hfill
  \begin{enumerate}[(i)]
    \item We observe that the set $\set{e_i\otimes f_j}_{i\in I,j\in J}$ is an orthonormal set. The spans of these elementary tensors are all the vectors of the form $x\otimes y$, provided that $x$ is a finite linear combination of the $e_i$ and $y$ is a finite linear combination of the $f_j$. Therefore, the completion is equal to the completion of $H\otimes K$, so the set is an orthonormal basis.
    \item We find that
      \begin{align*}
        H\otimes K &\cong \bigoplus_{j\in J} H\otimes \C f_j
      \end{align*}
      is an $\ell_2$ direct sum of $\left\vert J \right\vert$ copies of $H$.
    \item Define maps $Y_j\colon L_2\left( X,\mu,K \right)\rightarrow L_2\left( X,\mu \right)$ by taking
      \begin{align*}
        \left( Y_jf \right)(x) &= \iprod{f(x)}{f_j},
      \end{align*}
      a representative of an equivalence class modulo $\mu$. Then, we have that
      \begin{align*}
        f &\mapsto \sum_{j\in J}Y_jf
      \end{align*}
      defines a map from $L_2\left( X,\mu,K \right)\rightarrow \bigoplus_{j\in J}H\otimes \C f_j$. By Tonelli's theorem and Parseval's identity, we get
      \begin{align*}
        \norm{Yf}^2 &= \sum_{j\in J}\norm{Y_jf}^2\\
                    &= \sum_{j\in J} \int_{}^{} \left\vert \iprod{f(x)}{f_j} \right\vert^2\:d\mu\\
                    &= \int_{}^{} \sum_{j\in J} \left\vert \iprod{f(x)}{f_j} \right\vert^2\:d\mu\\
                    &= \int_{}^{} \norm{f(x)}^2\:d\mu\\
                    &= \norm{f}^2.
      \end{align*}
      Therefore, $Y$ is an isometry. The range is dense since, if $h\in L_2\left( \mu \right)$, then $f\coloneq hf_j$ is mapped to the vector with $h$ in position $j$ and $0$ elsewhere. Thus, $Y$ is a unitary map from $L_2\left( X,\mu,K \right)$ onto $H\otimes K$.
  \end{enumerate}
\end{proof}
\begin{proposition}
  Let $A\in B(H)$ and $B\in B(K)$. Then, there is a unique $A\otimes B\in B\left( H\otimes K \right)$ such that $\left( A\otimes B \right)\left( \xi\otimes \eta \right) = A\xi\otimes B\eta$. Furthermore, $\norm{A\otimes B} = \norm{A}\norm{B}$.
\end{proposition}
\begin{proof}
  Let $A\otimes I_K$ be the amplification to $H\otimes K$. Fixing an orthonormal basis $\set{\eta_j}_{j\in J}$ for $K$, then any vector $\omega\in H\otimes K$ (from part (ii) above) can be expressed as
  \begin{align*}
    \omega &= \left( \zeta_j\otimes \eta_j \right)_j
  \end{align*}
  for a collection of $\zeta_j\in H$, and we may define
  \begin{align*}
    \norm{\omega}^2 &= \sum_{j\in J}\norm{\zeta_j}^2\\
                    &< \infty
  \end{align*}
  Therefore, we may rewrite
  \begin{align*}
    \left( A\otimes I \right)\omega &= \left( A\zeta_j\otimes \eta_j \right)_j,
  \end{align*}
  giving
  \begin{align*}
    \norm{\left( A\otimes I \right)\omega}^2 &= \norm{\left( A\zeta_j\otimes \eta_j \right)_j}^2\\
                                             &= \sum_{j\in J} \norm{A\zeta_j}^2\\
                                             &\leq \norm{A}^2 + \sum_{j\in J}\norm{\zeta_j}^2\\
                                             &= \left( \norm{A}\norm{\omega} \right)^2.
  \end{align*}
  It follows that $\norm{A\otimes I} = \norm{A}$. Similarly, we may define $I\otimes B$, and set $A\otimes B = \left( A\otimes I \right)\left( I\otimes B \right)$. Thus, we have
  \begin{align*}
    \norm{A\otimes B}\left( \xi\otimes \eta \right) &= \left( A\otimes I \right)\left( I\otimes B \right)\left( \xi\otimes\eta \right)\\
                                                    &= \left( A\otimes I \right)\left( \xi\otimes B\eta \right)\\
                                                    &= A\xi\otimes B\eta.
  \end{align*}
  We have that $\norm{A\otimes B}\leq \norm{A\otimes I}\norm{I\otimes B} = \norm{A}\norm{B}$. Since $\norm{\left( A\otimes B \right)\left( \xi\otimes\eta \right)} = \norm{A\xi}\norm{B\eta}$, we may choose unit vectors $\xi$ and $\eta$ appropriately to approximate $\norm{A}\norm{B}$. Finally, since $H\odot K$ is dense in $H\otimes K$, we have that $A\otimes B$ is well-defined.
\end{proof}
\begin{definition}
  Let $M$ and $N$ be von Neumann algebras. Then, the von Neumann algebra tensor product $M\bar\otimes N$ is the WOT-closure of $M\odot N$ in $B\left( H\otimes K \right)$. If $N = \C 1$, then $M\bar\otimes \C 1 = M^{(\alpha)}$ is an amplification of $M$.
\end{definition}
\begin{definition}
  A system $\set{w_{ij}|i,j\in I}$ of elements in a von Neumann algebra is called a system of \textit{matrix units} if
  \begin{enumerate}[(i)]
    \item $w_{ij}^{\ast} = w_{ji}$;
    \item $w_{ij}w_{k\ell} = \delta_{jk}w_{i\ell}$;
    \item $\sum_{i\in I}w_{ii} = 1$.
  \end{enumerate}
\end{definition}
\begin{proposition}
  Let $M\subseteq B(H)$ be a von Neumann algebra. Let $\set{w_{ij} | i,j\in I}$ be a system of matrix units. Fix $i_0\in I$, and let $e = w_{i_0i_0}$. Then,
  \begin{align*}
    M &\cong \left( eMe \right)\otimes B\left( \ell_2(I) \right).
  \end{align*}
\end{proposition}
\begin{proof}
  Let $\set{\ve_i}_{i\in I}$ be an orthonormal basis in $\ell_2(I)$, and let $v_i = w_{i,i_0}$ for each $i\in I$. Then, $w_{ij} = v_iv_j^{\ast}$ and $v_i^{\ast}v_i = e$. Define a map
  \begin{align*}
    U\left( \sum_{i\in I}\xi_i\otimes \ve_i \right) &= \sum_{i\in I}v_i\xi_i
  \end{align*}
  for each $\sum_{i\in I}\xi_i\otimes \ve_i\in K\otimes \ell_2(I)$. Then, we have
  \begin{align*}
    \iprod{U\left( \sum_{i\in I}\xi_i\otimes \ve_i \right)}{U\left( \sum_{j\in i}\eta_j\otimes \ve_j \right)} &= \iprod{\sum_{i\in I}v_i\xi_i}{\sum_{j\in I}v_j\eta_j}\\
                                                                                                              &= \sum_{i,j\in I} \iprod{v_j^{\ast}v_i\xi_i}{\eta_j}\\
                                                                                                              &= \sum_{i\in I} \iprod{\xi_i}{\eta_i}\\
                                                                                                              &= \iprod{\sum_{i\in I}\xi_i\otimes \ve_i}{\sum_{j\in I}\eta_j\otimes \ve_j},
  \end{align*}
  so that $U$ is a well-defined isometry of $K\otimes \ell_2(I)$ onto $H$. Since the range of $U$ contains all $v_iK = w_{ii}H$, it follows that $U$ is surjective as $\sum_{i\in I}w_{ii} = 1$.

  Now, letting $x = \left( x_{ij} \right)\in eMe\otimes B\left( \ell_2(I) \right)$, we have
  \begin{align*}
    UxU^{\ast}\xi &= Ux\left( \sum_{j\in I}v_j^{\ast}\xi\otimes \ve_j \right)\\
                  &= U\left( \sum_{i,j\in I}x_{ij}v_j\xi\otimes \ve_i \right)\\
                  &= \sum_{i,j\in I}v_ix_{ij}v_j^{\ast}\xi,
  \end{align*}
  meaning that $UxU^{\ast} = \sum_{i,j\in I}v_ix_{ij}v_j^{\ast}$ is an element of $M$. Meanwhile, if $x\in M$, then
  \begin{align*}
    U^{\ast}xu\left( \sum_{j\in I}\xi_j\otimes \ve_j \right) &= U^{\ast}x\sum_{j\in I}v_j\xi_j\\
                                                             &= U^{\ast}\sum_{j\in I}xv_j\xi_j\\
                                                             &= \sum_{i,j\in I} \left( v_i^{\ast}xv_j\xi_j\otimes \ve_i \right).
  \end{align*}
  Therefore, $\left( U^{\ast}xU \right)_{i,j} = v_i^{\ast}xv_j$, so that $U^{\ast}xU\in eMe\bar\otimes B\left( \ell_2(I) \right)$.
\end{proof}
\subsection{Type I von Neumann Algebras}%
We start by considering the structure and classification of type I von Neumann algebras. For starters, we know that every projection dominates a rank one projection (which is minimal), so since any minimal projection is abelian, it follows that $B(H)$ is a factor of type I. Furthermore, we can show that any atomic von Neumann algebra is type I.
\begin{proposition}\hfill
  \begin{enumerate}[(i)]
    \item If $A$ is an abelian von Neumann algebra, then $A\bar\otimes B(H)$ is type I for any Hilbert space $H$.
    \item If $A$ is an abelian von Neumann algebra, then the matrix algebra $\M_n(A)$ is a finite von Neumann algebra of type I, and the $A$-valued map
      \begin{align*}
        \Phi(x) &= \frac{1}{n}\sum_{i=1}^{n}x_{ii}
      \end{align*}
      is
      \begin{itemize}
        \item center-valued: $\Phi\left( \left( \delta_{ij}a \right)_{i,j} \right) = a$ whenever $a\in A$;
        \item $A$-modular: $\Phi\left( axb \right) = a\Phi(x)b$ whenever $a,b\in A$;
        \item positive definiteness: $\Phi\left( x^{\ast}x \right)\geq 0$, with equality only when $x = 0$;
        \item traciality: $\Phi\left( x^{\ast}x \right) = \Phi\left( xx^{\ast} \right)$.
      \end{itemize}
  \end{enumerate}
\end{proposition}
\begin{proof}\hfill
  \begin{enumerate}[(i)]
    \item Let $p$ be a minimal projection in $B(H)$, and let $e = 1\otimes p$. Then,
      \begin{align*}
        e\left( A\bar\otimes B(H) \right)e &= A\bar\otimes pB(H)p\\
                                           &= A\otimes \C p,
      \end{align*}
      so $e$ is an abelian projection. Since the center of $A\bar\otimes B(H)$ is $A\otimes \C$, it follows that the central support $z(e)$ of $e$ is the identity. Thus, $A\bar\otimes B(H)$ is type I.
    \item The first two properties of $\Phi$ follow from the definition, while the latter two properties follow from taking
      \begin{align*}
        \Phi\left( x^{\ast}x \right) &= \frac{1}{n}\sum_{i,j=1}^{n}x_{ij}^{\ast}x_{ij}\\
                                     &= \frac{1}{n}\sum_{i,j=1}^{n} x_{ij}x_{ij}^{\ast}\\
                                     &= \Phi\left( xx^{\ast} \right).
      \end{align*}
  \end{enumerate}
\end{proof}
\begin{definition}
  An $A$-valued map satisfying the properties in (ii), except for possibly the faithfulness condition, is known as an $A$-valued trace.
\end{definition}
We will discuss traces later. First, we will discuss further characterizations and structures of type I von Neumann algebras.
\begin{proposition}
  A von Neumann algebra is type I if and only if it has an abelian projection whose central support is $1$. We call such projections faithful.
\end{proposition}
\begin{proof}
  Let $e$ be a faithful abelian projection in $M$, and let $g$ be any nonzero central projection. Then, $eg\neq 0$. Yet, $\left( eg \right)M\left( eg \right) = \left( eMe \right)g$, which is abelian; since $eg\leq g$, it follows that $M$ is type I.

  Now, if $M$ is type I, then by Zorn's lemma, there is a maximal family of centrally orthogonal abelian projections $\set{e_i}_{i\in I}$. Set $z = \sum_{i}z\left( e_i \right)$. Then, $\left( 1-z \right)M\left( 1-z \right)$ is type I, so if $1-z\neq 0$ we get a contradiction by the maximality of $\set{e_i}_{i\in I}$. Thus, $\sum_{i\in I}z\left( e_i \right) = 1$.

  If we let $e = \sum_{i\in I}e_i$, then $e$ is abelian and $z\left( e \right) = 1$.
\end{proof}
\begin{definition}
  Let $n$ be a cardinal number. We say a type I von Neumann algebra is type $\text{I}_n$ if $1$ is a sum of $n$ equivalent nonzero abelian projections. We say it is type $\text{I}_{\infty}$ if it is type $\text{I}_n$ for an infinite cardinal $n$.

  Furthermore, if $z$ is any projection that is the sum of $n$ equivalent abelian projections with central support equal to $z$, then we say $z$ is $n$-homogeneous.
\end{definition}
\begin{lemma}
  Let $M$ be a von Neumann algebra, and let $p\in P(M)$ be a projection with central support $z(p)$. Then,
  \begin{align*}
    M'z(p) \rightarrow M'p
  \end{align*}
  given by $yz(p)\mapsto yp$ is an isomorphism of $\ast$-algebras. In particular, $Z\left( pMp \right) = Z\left( Mz(p) \right)p$.
\end{lemma}
\begin{proof}
  Recalling that $z(p) = \left[ MpH \right]$, we have that if $yp = 0$, then for all $x\in M$ and $\xi\in H$,
  \begin{align*}
    yxp\xi &= xyp\xi\\
           &= 0.
  \end{align*}
  Therefore, $yz(p) = 0$, meaning that $yz(p)\mapsto yp$ is injective. Since this map is surjective, it follows that we get this as an isomorphism.

  In particular, we have that $Z\left( pMp \right) = Z\left( M'p \right)$, and since $yp\in Z\left( M'p \right)$ if and only if $yz(p)\in Z\left( M'z(p) \right)$, we find our desired equality.
\end{proof}
An important fact about abelian projections is that their central supports fully encode their order relation (in the following sense).
\begin{lemma}
  Let $M\subseteq B(H)$ be a von Neumann algebra. If $p,q\in P(M)$ are abelian projections with $z(p) = z(q)$, then $p\sim q$.
\end{lemma}
Note that we already have that if $p\sim q$, then $z(p) = z(q)$, but this provides a type of converse.
\begin{proof}
  Suppose $p\preceq q$, and let $v\in M$ implement this subequivalence, where $vv^{\ast} = q_0\leq q$. Since $q$ is abelian, we have
  \begin{align*}
    qMq &= Z\left( qMq \right)\\
        &= Z\left( Mz(q) \right)q,
  \end{align*}
  so since $q_0\in qMq$, we have $q_0 = zq$ for some central projection $z\in Z\left( Mz(q) \right)$. Therefore, we get
  \begin{align*}
    z(p) &= z\left( q_0 \right)\\
         &= zz\left( q \right)\\
         &= z(q)\\
         &= z(p).
  \end{align*}
  In particular, this gives $zz\left( q \right) = z\left( q \right)$, so $z(q)\leq z$, and thus $z = z(q)$, $q_0 = q$, and $p\sim q$.

  In the case that $p$ is not subequivalent to $q$, then by the comparison theorem, there is $z\in Z(M)$ such that $pz\preceq qz$ and $q\left( 1-z \right)\preceq p\left( 1-z \right)$. Now, we have $z\left( pz \right) = z\left( p \right)z = z\left( qz \right)$, so $pz\sim qz$, and $q\left( 1-z \right)\sim p\left( 1-z \right)$, so $p\sim q$.
\end{proof}
\begin{lemma}
  Let $M\subseteq B(H)$ be a von Neumann algebra. If $n$ and $M$ are cardinals such that $M$ is type $\text{I}_n$ and $\text{I}_m$, then $n = m$.
\end{lemma}
\begin{proof}
  Let $\set{p_i}_{i\in I}$ and $\set{q_j}_{j\in J}$ be pairwise orthogonal equivalent abelian projections with $\left\vert I \right\vert = n$ and $\left\vert J \right\vert = m$ such that
  \begin{align*}
    1 &= \sum_{i\in I} p_i \\
      &= \sum_{j\in J}q_j.
  \end{align*}
  Note that $z\left( p_i \right)$ is constant across $i$, so that
  \begin{align*}
    1-z\left( p_i \right) &= \left( 1-z\left( p_i \right) \right)\sum_{k\in I}p_k\\
                          &= 0,
  \end{align*}
  so that $z\left( p_i \right) = 1$ for all $i\in I$. Similarly, $z\left( q_j \right) = 1$ for all $j\in J$, meaning $p_i\sim q_j$ for all $i\in I$ and $j\in J$.

  Now, we start by assuming that $n < \infty$. For each $i\in I$ and each $j\in J$, let $v_i^{\ast}v_i = u_j^{\ast}u_j = p_{i_0}$, $v_iv_i^{\ast} = p_i$, and $u_ju_j^{\ast} = q_j$. Now, consider the map
  \begin{align*}
    \Phi(x) &= \frac{1}{n}\sum_{i\in I}v_i^{\ast}xv_i.
  \end{align*}
  Then,
  \begin{align*}
    \Phi\left( xy \right) &= \frac{1}{n}\sum_{i\in I}v_i^{\ast}xyv_i\\
                          &= \frac{1}{n}\sum_{i,k\in I}v_i^{\ast}xv_kv_k^{\ast}yv_i\\
                          &= \frac{1}{n}\sum_{i,k\in I} \left( p_{i_i}v_i^{\ast}xv_kp_{i_0} \right)\left( p_{i_0}v_k^{\ast}yv_ip_{i_0} \right)\\
                          &= \frac{1}{n}\sum_{i,k\in I} \left( p_{i_0}v_i^{\ast}yv_kp_{i_0} \right)\left( p_{i_0}v_k^{\ast}xv_ip_{i_0} \right)\\
                          &= \Phi\left( yx \right).
  \end{align*}
  Consequently, $\Phi\left( p_i \right) = \Phi\left( q_j \right) = \Phi\left( p_{i_0} \right)$ for all $i\in I$ and all $j\in J$, with common image $p_{i_0}$. Therefore, we get
  \begin{align*}
    np_{i_0} &= \sum_{i\in I}\Phi\left( p_i \right)\\
             &= \Phi\left( 1 \right)\\
             &= \sum_{j\in J}\Phi\left( q_j \right)\\
             &= mp_{i_0},
  \end{align*}
  so $n = m$.

  Now, suppose $n$ is infinite. By a symmetric argument, we have $m$ is infinite. Let $\varphi$ be a normal state with support $z\in Z(M)$; that is, $w\in \ker\left( \varphi \right)$ if and only if $zwz = 0$. For each $i\in I$, define $\sigma_i\colon Z\left( M \right)\rightarrow p_iMp_i$ by $\sigma_i(z) = zp_i$. This is an isomorphism from the earlier lemma, so we may define $\varphi_i$ by
  \begin{align*}
    \varphi_i\left( x \right) &= \varphi\left( \sigma_i^{-1}\left( p_ixp_i \right) \right).
  \end{align*}
  We have that $\varphi_i$ is a state with support $zp_i$. Since $\sigma_i$ is $\sigma$-WOT continuous, we have
  \begin{align*}
    1 &= \varphi_i\left( 1 \right)\\
      &= \varphi_i\left( \sum_{j\in J}q_j \right)\\
      &= \sum_{j\in J}\varphi_i\left( q_j \right).
  \end{align*}
  Since $\varphi_i\left( q_j \right) \geq 0$ for each $j\in J$, the set of all $j$ for which $\varphi_i\left( q_j \right) > 0$ is countable. Yet, if $j\notin j_i$, we have $\varphi_i\left( q_j \right) = 0$, meaning $\left( zp_i \right)q_j\left( zp_i \right) = 0$, and by centrality, we have $zp_iq_jp_iz = 0$, so that $q_jp_iz = 0$.

  Yet, for any $j\in J$, we have
  \begin{align*}
    \sum_{i\in I}q_jp_iz &= q_jz,
  \end{align*}
  which is nonzero as $z\left( q_j \right) = 1$. In particular, this means $q_jp_iz \neq 0$ for some $i\in I$, meaning $j\in J_i$ for some $i\in I$. That is, $J = \bigcup_{i\in I}J_i$, so that $m\leq n\aleph_0 = n$, and by symmetry, $m = n$.
\end{proof}
\begin{theorem}
  Let $M$ be a von Neumann algebra of type I. Then, there exists a unique family of orthogonal central projections $\set{z_{\alpha}}_{\alpha}$ indexed by cardinal numbers such that $\sum_{\alpha} z_{\alpha} = 1$, and $z_{\alpha}Mz_{\alpha}$ is isomorphic to the tensor product of an abelian von Neumann algebra $A_{\alpha}$ and $B\left( H_{\alpha} \right)$, where $\Dim\left( H_{\alpha} \right) = \alpha$. In particular, we have
  \begin{align*}
    M &\cong \bigoplus_{\alpha} A_{\alpha}\bar\otimes B\left( H_{\alpha} \right).
  \end{align*}
\end{theorem}
\begin{proof}
  We note that if $\set{z_i}_{i\in I}$ is a family of $\alpha$-homogeneous projections, then their sum is also $\alpha$-homogeneous. Therefore, upon taking suprema, we may find a greatest $z_{\alpha}$ that is $\alpha$-homogeneous.

  If $z_{\alpha}\neq 0$, then there exists an orthogonal family $\set{e_i}_{i\in I_{\alpha}}$ of abelian projections with $z\left( e_i \right) = z_{\alpha}$, and $I_{\alpha}$ has cardinality $\alpha$. Since these projections are mutually equivalent, we have
  \begin{align*}
    z_{\alpha}Mz_{\alpha} &\cong e_{1}Me_1\bar\otimes B\left( \ell_2\left( I_{\alpha} \right) \right).
  \end{align*}
  Since $e_1Me_1$ is abelian, we only have to show that $z_{\alpha}$ and $z_{\beta}$ are orthogonal for distinct $\alpha$ and $\beta$.

  If $\alpha$ is infinite, then both $I_{\alpha}$ and $I_{\alpha}\setminus \set{i_0}$ have the same cardinality, meaning that $z_{\alpha} = \sum_{i\in I_{\alpha}}e_i\sim \sum_{i\in I_{\alpha}\setminus \set{i_0}}e_i < z_{\alpha}$, meaning $z_{\alpha}$ is not a finite projection. In particular, $\alpha$ must be finite if $M$ is finite.
\end{proof}
\begin{proposition}
  A von Neumann algebra $M$ is type I if and only if $M$ admits a faithful normal representation $\pi\colon M\rightarrow B(H)$ such that $\pi\left( M \right)'$ is abelian.
\end{proposition}
\begin{proof}
  Let $M\subseteq B(H)$ be a von Neumann algebra with abelian commutant $M'$. Then, $M'$ coincides with $Z(M)$. For each $\xi\in H$, let $e_{\xi}$ denote the projection of $H$ onto $\left[ Z(M)\xi \right]$. Since $e_{\xi}$ commutes with $Z(M)$, it follows that $e_{\xi}$ is contained in $M$. Since $e_{\xi}Z(M)e_{\xi}\subseteq B\left( \left[ Z(M)\xi \right]H \right)$ admits a cyclic vector, namely $\xi$, it follows that $e_{\xi}Z(M)e_{\xi}$ is abelian, so $\left( e_{\xi}Z(M)e_{\xi} \right)' = e_{\xi}Me_{\xi}$ is abelian. Thus, $e_{\xi}$ is abelian for any $\xi\in H$. Furthermore, for any projection $p\in M$, we have that $e_{\xi}\leq p$ whenever $\xi\in pH$, so every nonzero projection majorizes an abelian projection, so $M$ is type I.

  Now, if $M$ is type I, then we may represent $M$ as
  \begin{align*}
    M &\cong \bigoplus_{\alpha}A_{\alpha}\bar\otimes B\left( H_{\alpha} \right)
  \end{align*}
  and take $A_{\alpha}$ as a masa of $B\left( H_{\alpha} \right)$, meaning that on the Hilbert space
  \begin{align*}
    H &\coloneq \bigoplus_{\alpha}H_{\alpha}\otimes H_{\alpha},
  \end{align*}
  we have
  \begin{align*}
    M' &= \left( \bigoplus_{\alpha}A'_{\alpha}\bar\otimes B\left( H_{\alpha} \right) \right)'\\
       &= \bigoplus_{\alpha} \left( A_{\alpha}\otimes \C \right)\\
       &= \bigoplus_{\alpha}A_{\alpha},
  \end{align*}
  so $M'$ is abelian.
\end{proof}
\begin{corollary}
  If $M$ is a von Neumann algebra of type I, then so is $M'$.
\end{corollary}
\subsection{Traces on Finite von Neumann Algebras}%
\begin{lemma}
  Let $M$ be a finite von Neumann algebra, and $p\in P(M)$ a nonzero projection. If we have $\set{p_i}_{i\in I}$ a family of pairwise orthogonal projections with $p_i\sim p$ for each $i$, then $I$ is finite.
\end{lemma}
\begin{proof}
  If this were not the case, then we would have a proper subset $J\subsetneq I$ with $\sum_{i\in I}p_i \sim \sum_{j\in J}p_j < \sum_{i\in I}p_i$, implying $\sum_{i\in I}p_i$ is not finite, but the sum of pairwise orthogonal finite projections is finite.
\end{proof}
\begin{lemma}
  Let $M\subseteq B(H)$ be a type $\text{II}_{1}$ von Neumann algebra. Then, there exists a projection $p_{1/2}\in P(M)$ such that $p_{1/2}\sim 1-p_{1/2}$.

  Furthermore, there is a family of projections $\set{p_r}_{r}$ indexed by dyadic rationals in $[0,1]$ such that $p_{r}\leq p_s$ if $r\leq s$, $p_s-p_r\sim p_{s'}-p_{r'}$ whenever $0\leq r \leq s \leq 1$, $0\leq r'\leq s'\leq 1$, and $s-r = s'-r'$, and each of the $p_r$ has central support equal to $1$.
\end{lemma}
\begin{proof}
  Let $\set{p_i,q_i}_{i\in I}$ be a maximal family of pairwise orthogonal projections with $p_i\sim q_i$ for all $i\in I$. Define
  \begin{align*}
    p_{1/2} &= \sum_{i\in I}p_i\\
    q &= \sum_{i\in I}q_i.
  \end{align*}
  We have that $p_{1/2}\sim q$, and we claim that $q\sim 1-p_{1/2}$. If not, then $1-\left( p_{1/2} + q \right) \neq 0$, so since $M$ is type II, it follows that $1-\left( p_{1/2} + q \right)$ is not abelian, so there exists some projection in the corner of $M$ with respect to $1-\left( p_{1/2} + q \right)$ that is strictly less than its central support in the corner. We denote this central support by $z$. Then, if we set $q_0 = z-p_0$, we have that $p_0$ and $q_0$ are not centrally orthogonal, so they have equivalent subprojections. Yet, this contradicts maximality of $\set{p_i,q_i}_{i\in I}$, so it must be the case that $q = 1-p_{1/2}$.

  Now, we construct the family of projections indexed by the dyadic rationals inductively. Let $p_{1/2}$ be defined as above, $p_0 = 0$, and $p_1 = 1$. If $v\in M$ is such that $v^{\ast}v = p_{1/2}$ and $vv^{\ast} = 1-p_{1/2}$. Then, $p_{1/2}Mp_{1/2}$ is type II, and specifically type $\text{II}_1$, since $p_{1/2}$ is a finite projection. If we let $q\sim p_{1/2}$ with $q < p_{1/2}$, then $q+ \left( 1-p_{1/2} \right) \sim 1$ with $q + \left( 1-p_{1/2} \right) < 1$, which contradicts the finitude of $1$. Therefore, we may obtain $p_{1/4}\leq p_{1/2}$ with $p_{1/4}\sim p_{1/2}-p_{1/4}$, and set $p_{3/4} = p_{1/2} + vp_{1/2}v^{\ast}$, with $p_0\leq p_{1/4}\leq p_{1/2}\leq p_{3/4}\leq p_1$, with $p_{1/4}\sim p_{\left( k+1 \right)/4}-p_{k/4}$ for each $k=0,1,2,3$. Thus, inductively, we obtain a family indexed by dyadic rationals satisfying $p_s-p_r\sim p_{s'}-p_{r'}$ whenever $s-r = s'-r'$.

  Now, we verify central supports. Fix a dyadic rational $r$, and set $z = 1-z\left( p_r \right)$. Let $n\in \N$ be large enough that $s = \frac{1}{2^{n}} \leq r$. Then, we have that $zp_s\leq zp_r = 0$, and $zp_s\sim z\left( p_{ks}-p_{\left( k-1 \right)s} \right)$ for every $1\leq k \leq 2^{n}$, meaning that $z\left( p_{ks}-p_{\left( k-1 \right)s} \right) = 0$. Thus,
  \begin{align*}
    z &= z\sum_{k=1}^{n} \left( p_{ks}-p_{\left( k-1 \right)s} \right)\\
      &= 0,
  \end{align*}
  so $z\left( p_r \right) = 1$.
\end{proof}
\begin{lemma}
  Let $M$ be a type $\text{II}_1$ von Neumann algebra, and let $\set{p_r}_{r}$ be the above collection of projections. If $p\in P(M)$, with $p\neq 0$, then there exists $z\in P(Z(M))$ such that $pz \neq 0$ and $p_rz \preceq pz$ for some positive dyadic rational $r$.
\end{lemma}
\begin{proof}
  Passing to $Mz(p)$ allows us to assume that $z(p) = 1$. If this does not hold, then we would have $p\preceq p_r$ for every $r > 0$, meaning that $p$ would be equivalent to a subprojection of $p_{2^{-k}}-p_{2^{-\left( k+1 \right)}}$, which contradicts the fact that any family of pairwise orthogonal projections equivalent to a fixed projection is finite.
\end{proof}
We say a projection $p\in P(M)$ is \textit{monic} if there exists a finite collection of pairwise orthogonal projections $\set{p_1,\dots,p_n}$ such that $p_i\sim p$ and $\sum_{i=1}^{n}p_i\in Z(M)$. The projections $p_{1/2^{k}}$ defined above in the $\text{II}_1$ case are monic.
\begin{proposition}
  If $M$ is a finite von Neumann algebra, then every projection is the sum of pairwise orthogonal monic projections.
\end{proposition}
\begin{proof}
  By passing to a maximal sub-family, it suffices to show that every nonzero projection has a nonzero monic subprojection. Furthermore, by passing to the type decomposition, it suffices to consider the cases where $M$ is type $\text{II}_1$ or $M$ is type $\text{I}_n$.

  In the case of type $\text{II}_1$, we get this from the above discussion, and in the case of type $\text{I}_n$, this follows from considering a nonzero abelian subprojection.
\end{proof}
\begin{definition}
  Let $M$ be a von Neumann algebra. We say a linear map $\varphi\colon M\rightarrow Z(M)$ is a \textit{center-valued state} if $\varphi\left( x^{\ast}x \right) \geq 0$, $\varphi|_{Z(M)} = \id$, and $\varphi\left( zx \right) = z\varphi(x)$ whenever $z\in Z(M)$ and $x\in M$.
\end{definition}
\begin{lemma}
  If $M$ is a von Neumann algebra and $\varphi\colon M\rightarrow Z(M)$ is a center-valued state, then $\varphi$ is bounded with $\norm{\varphi} = 1$.
\end{lemma}
\begin{lemma}
  Let $M\subseteq B(H)$ be a von Neumann algebra. Then, $M$ has a normal center-valued state.
\end{lemma}
\begin{proof}
  Since $Z(M)$ is an abelian von Neumann algebra, $Z(M)'$ is type I, so it has an abelian projection $q$ with full central support. Then, $qMq\subseteq qZ(M)'q = Z(M)q$, so $\theta(z) = zq$ defines a normal isomorphism from $Z(M)$ onto $Z(M)q$. Setting $\varphi(x) = \theta^{-1}\left( qxq \right)$, then $\varphi$ is a normal center-valued state.
\end{proof}
\begin{lemma}
  Let $M$ be a von Neumann algebra, and let $\tau\colon M\rightarrow Z(M)$ be a center-valued state. The following are equivalent:
  \begin{enumerate}[(i)]
    \item $\tau\left( xy \right) = \tau\left( yx \right)$ for all $x,y\in M$;
    \item $\tau\left( xx^{\ast} \right) = \tau\left( x^{\ast}x \right)$ for all $x\in M$;
    \item $\tau\left( p \right) = \tau\left( q \right)$ for all equivalent projections $p ,q\in P(M)$.
  \end{enumerate}
\end{lemma}
\begin{proof}
  We only need to show (iii) implies (i). First, for all $p\in P(M)$ and all unitaries $u$, we have $\tau\left( upu^{\ast} \right) = \tau\left( p \right)$. Since $\tau$ is bounded, then by applying the functional calculus, we have that $\tau\left( uxu^{\ast} \right) = \tau(x)$ for all self-adjoint $x\in M$. Separating real and imaginary parts, and replacing $x$ with $xu$, it follows that $\tau\left( ux \right) = \tau\left( xu \right)$ for all $x\in M$ and all unitaries $u$. Since every operator is a linear combination of four unitaries, this shows (i).
\end{proof}
As discussed above, such a center-valued state is known as a center-valued trace.
\begin{lemma}
  Let $\varphi\colon M\rightarrow Z(M)$ be a normal center-valued state. Then, for every $\ve > 0$, there exists $p\in P(M)$ with $\varphi(p)\neq 0$ and for all $x\in pMp$,
  \begin{align*}
    \varphi\left( xx^{\ast} \right)\leq \left( 1 + \ve \right) \varphi\left( x^{\ast}x \right).
  \end{align*}
\end{lemma}
\begin{proof}
  Let $q_0 = 1-\sum_{i\in I}q_i$, where $\set{q_i}_{i\in I}$ is a maximal family of pairwise orthogonal projections with $\varphi\left( q_0 \right) = 0$. By normality, we must have that $\varphi\left( q_0 \right) = 1$, and $\varphi$ is faithful when restricted to $q_0Mq_0$.

  Let $\set{e_i,f_i}_{i\in I}$ be a maximal family of projections with $\set{e_i}_{i\in I}$ and $\set{f_i}_{i\in I}$ pairwise orthogonal, $e_i\sim f_i$ for each $i$, and $\varphi\left( e_i \right) > \varphi\left( f_i \right)$ for each $i$. Setting $e = q_0 - \sum_{i\in I}e_i$ and $f = q_0 - \sum_{i\in I}f_i$, then unless $\varphi$ is already a trace, we have $\varphi(f) > \varphi(e) \geq 0$. Hence, $f\neq 0$, so $e\sim f$, meaning $e\neq 0$.

  Let $\mu$ be the smallest number such that $\varphi\left( \bar{e} \right)\leq \mu\varphi\left( \bar{f} \right)$ whenever $\bar{e}\leq e$, $\bar{f}\leq f$, and $\bar{e}\sim\bar{f}$. Then, $\mu\neq 0$ since $\varphi(e)\neq 0$, and there exists $\bar{e}\leq e$, $\bar{f}\leq f$ such that $\bar{e}\sim \bar{f}$, $\left( 1+\ve \right)\varphi\left( \bar{e} \right)\not\leq \mu\left( \varphi(f) \right)$. By cutting by a suitable central projection, we may assume that $\left( 1+\ve \right)\varphi\left( \bar{e} \right) > \mu\varphi\left( \bar{f} \right)$.
\end{proof}
\subsection{Traces on Semifinite von Neumann Algebras}%
\nocite{davidson_functional_analysis,conway_operator_theory,takesaki_1}
\printbibliography
\end{document}
