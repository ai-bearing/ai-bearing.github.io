\documentclass[10pt]{mypackage}

% sans serif font:
%\usepackage{cmbright,sfmath,bbold}
%\renewcommand{\mathcal}{\mathtt}

%Euler:
\usepackage{newpxtext,eulerpx,eucal,eufrak}
\renewcommand*{\mathbb}[1]{\varmathbb{#1}}
\renewcommand*{\hbar}{\hslash}

%\renewcommand{\mathbb}{\mathds}
%\usepackage{homework}
%\usepackage{exposition}

\pagestyle{fancy} %better headers
\fancyhf{}
\rhead{Avinash Iyer}
\lhead{Real Analysis Qualifier Preparation}

\setcounter{secnumdepth}{0}

\begin{document}
\RaggedRight
This is a collection of old real analysis qualifier exam solutions.
\section{\href{https://math.virginia.edu/graduate/exams/analysis/2019Aug_real.pdf}{August 2019}}%
\subsection{Problem 1}%
\begin{problem}
  Let $\mathcal{C}$ be the Cantor set on $[0,1]$.
  \begin{enumerate}[(a)]
    \item Show that $\mathcal{C} + \mathcal{C} = [0,2]$.
    \item Find two sets $A,B\subseteq \R$ that are closed and have Lebesgue measure zero such that $A+B = \R$.
  \end{enumerate}
\end{problem}
\begin{enumerate}[(a)]
  \item Recall that the Cantor set $\mathcal{C}$ is defined to consist of all $x\in [0,1]$ such that $x$ only contains $0$ and $2$ in the ternary expansion of $x$. Writing $a\in [0,2]$ as
    \begin{align*}
      a &= \sum_{k=0}^{\infty}\frac{a_k}{3^{k}},
    \end{align*}
    where $a_k\in\set{0,1,2}$, we may then find $a_k$ at each ternary expansion slot for $k$ as follows:
    \begin{itemize}
      \item if $a_k = 0$, we may find $b_k,c_k\in \mathcal{C}$ such that $b_k = c_k = 0$
      \item if $a_k = 2$, we may find $b_k,c_k\in \mathcal{C}$ such that $b_k = 2$ and $c_k = 0$ or vice versa.
      \item if $a_k = 1$, we may find $b_k,c_k\in \mathcal{C}$ such that $b_{k+1} = c_{k+1} = 2$.
    \end{itemize}
    Therefore, since every digit of every ternary expansion in $[0,2]$ can be obtained from $\mathcal{C}$, we see that $\mathcal{C} + \mathcal{C} = [0,2]$.
  \item We may set $B$ to be the union of all integer translates of $\mathcal{C}$, and set $A = \mathcal{C}$. This yields closed subsets of $\R$ with Lebesgue measure zero that sum to $\R$.
\end{enumerate}
\subsection{Problem 2}%
\begin{problem}
  Does there exist a finite measure space $\left( X,\mathcal{F},\mu \right)$ and a sequence $\left( f_n \right)_n$ of $\mu$-measurable functions such that
  \begin{itemize}
    \item $f_n(x) \geq 0$;
    \item $f_n(x)\rightarrow 0$ for all $x$;
    \item $ \int_{X}^{} f_n(x)\:d\mu(x) \rightarrow 0 $ as $n\rightarrow \infty$;
    \item $\Phi(x) = \sup_{n}f_n(x)$ has infinite integral?
  \end{itemize}
\end{problem}
Consider the sequence of functions
\begin{align*}
  f_n(x) &= n\1_{\left[ \frac{1}{n+1},\frac{1}{n} \right]},
\end{align*}
defined on $\left[ 0,1 \right]$. This sequence is pointwise convergent everywhere to zero, as $f_n(0) = 0$ and the Archimedean property give that for any $x\in (0,1]$, there is some $n$ large enough that gives $\frac{1}{n} < x$. Furthermore, we see that
\begin{align*}
  \int_{}^{} f_n\:d\mu &= n\left( \frac{1}{n}- \frac{1}{n+1} \right)\\
                       &= \frac{1}{n+1}\\
                       &\rightarrow 0.
\end{align*}
Finally, we see that by taking suprema, we have the integral
\begin{align*}
  \int_{}^{} \Phi\:d\mu &= \sum_{n=1}^{\infty}\frac{1}{n+1}\\
                        &\rightarrow \infty.
\end{align*}
\subsection{Problem 4}%
\begin{problem}
  Let $L_1(\R)$ be the space of Lebesgue integrable functions on $\R$. Suppose $f\in L_1\left( \R \right)$ is positive. Show that $\frac{1}{f(x)}\notin L_1\left( \R \right)$.
\end{problem}
Suppose toward contradiction that both $f$ and $1/f$ are in $L_1\left( \R \right)$. Then, from Hölder's Inequality, we have
\begin{align*}
  \infty &= \int_{}^{} 1\:d\mu\\
         &\leq \left( \int_{}^{} f\:d\mu \right)^{1/2} \left( \int_{}^{} \frac{1}{f}\:d\mu \right)^{1/2}\\
         &< \infty,
\end{align*}
which is a contradiction.
\subsection{Problem 5}%
\begin{problem}
  Applying the Gram--Schmidt orthogonalization to $\set{1,x,x^2,\dots}$ in the Hilbert space $L_2\left( [-1,1] \right)$ with Lebesgue measure, one gets the Legendre polynomials $L_n(x)$.
  \begin{enumerate}[(a)]
    \item Show that the Legendre polynomials form a basis (complete orthogonal system) in the Hilbert space $L_2\left( [-1,1] \right)$.
    \item Show that the Legendre polynomials are given by the formula $L_n(x) = c_n \diff{^{n}}{x^{n}} \left( x^2-1 \right)^{n}$.
  \end{enumerate}
\end{problem}
\begin{enumerate}[(a)]
  \item Let $f\in L_2\left( [-1,1] \right)$. We may find $g\in C\left( [-1,1] \right)$ such that $\norm{f-g}_{L_2} < \ve/2$. Similarly, we may find a polynomial $p$ such that $\norm{g-p}_{u} < \ve/4$, meaning that $\left\vert p(x)-g(x) \right\vert < \ve/4$ for all $x\in [-1,1]$. This yields
    \begin{align*}
      \norm{p-g}_{L_2} &= \left( \int_{-1}^{1} \left\vert p(x)-g(x) \right\vert^2\:dx \right)^{1/2}\\
                       &< \left( \int_{-1}^{1} \left( \frac{\ve}{4} \right)^2\:dx \right)^{1/2}\\
                       &= \left( \frac{\ve^2}{8} \right)^{1/2}\\
                       &< \frac{\ve}{2},
    \end{align*}
    so $\norm{f-p}_{L_2} < \ve$, meaning that the closed linear span of the monomials is dense in $L_2$, and the Legendre polynomials form an orthonormal system.
  \item We see that at every step in evaluating the expression
    \begin{align*}
      L_n(x) &= c_n \diff{^{n}}{x^{n}}\left( x^2-1 \right)^{n},\label{eq:legendre_expression}\tag{$\ast$}
    \end{align*}
    the degree of the polynomial increases by $1$, so each $L_n(x)$ has degree $n$. To verify that the polynomials generated from \eqref{eq:legendre_expression} are orthogonal to each other, we let $n > m$ without loss of generality, and use integration by parts to obtain
    \begin{align*}
      \iprod{L_n}{L_m} &= \int_{-1}^{1} \left( \diff{^{n}}{x^{n}}\left( x^2-1 \right)^{n} \right) \left( \diff{^{m}}{x^{m}}\left( x^2-1 \right)^{m} \right)\:dx\\
                       &= \diff{^{n-1}}{x^{n-1}}\left( x^2-1 \right)^{n}\diff{^{m}}{x^{m}}\left( x^2-1 \right)^{m}\biggr\vert_{-1}^{1} - \int_{-1}^{1} \diff{^{n-1}}{x^{n-1}}\left( x^2-1 \right)^{n}\diff{^{m+1}}{x^{m+1}}\left( x^2-1 \right)^{m} \:dx\\
                       &\vdots\\
                       &= \left( -1 \right)^{n} \int_{-1}^{1} \diff{^{m+n}}{x^{m+n}}\left( x^2-1 \right)^{m}\:dx\\
                       &= \left( -1 \right)^{n} \int_{-1}^{1} \diff{^{n}}{x^{n}}\left( \diff{^{m}}{x^{m}}\left( x^2-1 \right)^{m} \right)\:dx\\
                       &= \left( -1 \right)^{n} \int_{}^{} \diff{^{n}}{x^{n}}L_m(x)\:dx\\
                       &= 0,
    \end{align*}
    seeing as we are taking $n$ derivatives of a degree $m < n$ polynomial.
\end{enumerate}
\section{\href{https://math.virginia.edu/graduate/exams/analysis/2020Jan_real.pdf}{January 2020}}%
\subsection{Problem 1}%
\begin{problem}
  Let $\mu$ be the Lebesgue measure on $\R$, and let $A\subseteq [0,1]$ be Lebesgue-measurable.
  \begin{enumerate}[(a)]
    \item Prove or show a counterexample to the assertion that
      \begin{align*}
        \mu(A) &= \sup_{\substack{U\subseteq A\\U\text{ open}}} \mu\left( U \right).
      \end{align*}
    \item Prove or show a counterexample to the assertion that
      \begin{align*}
        \mu(A) &= \inf_{\substack{A\subseteq U\\U\text{ open}}} \mu\left( U \right).
      \end{align*}
  \end{enumerate}
\end{problem}
\begin{enumerate}[(a)]
  \item This is false. If $A\subseteq [0,1]$ is the ``fat Cantor set'' constructed similar to the traditional Cantor, but obtained by deleting the middle fourth of each subinterval rather than the middle third, then $\mu(A) = \frac{1}{2}$, but $A$ is nowhere dense, meaning that if $U\subseteq A$ is open, then $U = \emptyset$.\newline

    To see that $A$ is nowhere dense, we see that $A$ is closed, so if $x\in A\subseteq [0,1]$, and $\ve > 0$, we may show that the interval $\left( x-\ve,x+\ve \right)$ is not contained in $A$. In the recursive construction of $A$, we may see that there is some step $n_1$ such that $\frac{1}{4^{n_1}} < 2\ve$, implying that $\left( x-\ve,x+\ve \right)$ is not contained in the recursive construction at $n_1$. Therefore $A^{\circ} = \emptyset$.
  \item This is true. By the definition of the Lebesgue outer measure, for any $\ve > 0$, there are $\set{\left( a_k,b_k \right)}_{k=1}^{\infty}$ such that
    \begin{align*}
      \mu(A) + \ve &< \mu\left( \bigcup_{k=1}^{\infty}\left( a_k,b_k \right) \right),
    \end{align*}
    so by setting
    \begin{align*}
      U &= \bigcup_{k=1}^{\infty}\left( a_k,b_k \right),
    \end{align*}
    we have that $U$ is open, meaning that by the definition of infimum, we get 
    \begin{align*}
      \mu(A) &= \inf\set{U | A\subseteq U,U\text{ open}}.
    \end{align*}
\end{enumerate}
\begin{remark}
  Part (a) can be solved by selecting $A = \R\setminus \Q\cap [0,1]$.
\end{remark}

\subsection{Problem 3}%
\begin{problem}
  Let $X$ be a compact metric space, $C(X)$ the space of real-valued continuous functions on $X$ with the supremum norm. Assume that $\mathcal{A}\subseteq C(X)$ satisfies
  \begin{itemize}
    \item (algebra) for all $f,g\in \mathcal{A}$, $\alpha,\beta\in \R$, we have $\alpha f + \beta g \in \mathcal{A}$ and $fg\in \mathcal{A}$;
    \item (separates points) for any $x\neq y$ in $X$, there exists $f\in \mathcal{A}$ such that $f(x)\neq f(y)$.
  \end{itemize}
  \begin{enumerate}[(a)]
    \item Show by example that $\mathcal{A}$ need not be dense in $C(X)$.
    \item In order to conclude that $\mathcal{A}$ is dense by the Stone--Weierstrass Theorem, what additional condition(s) should be added.
  \end{enumerate}
\end{problem}
\begin{enumerate}[(a)]
  \item Consider the algebra of polynomials on $[0,1]$ without a constant term. Then, since linear combinations and multiplications still yield polynomials without constant term, and $f(x) = x$ separates points in $[0,1]$, this algebra satisfies the requirements of the question. Yet, since all elements of this algebra are equal to zero at $x= 0$, the uniform closure of the algebra yields all the continuous functions on $[0,1]$ with $f(0) = 0$.
  \item In order to satisfy the requirements of the Stone--Weierstrass theorem, we need the algebra $\mathcal{A}$ to include the constant functions.
\end{enumerate}
\subsection{Problem 4}%
\begin{problem}
  Let $\mu$ be a measure on $\left( \R,\mathcal{B} \right)$, where $\mathcal{B}$ is the Borel $\sigma$-algebra. Let $\mu\left( \R \right) = 1$. Next, let $\mathcal{F}\subseteq \mathcal{B}$ be the sub-$\sigma$-algebra generated by symmetric intervals.\newline

  Let $f\in L_1\left( \R,\mathcal{B},\mu \right)$. Find a function $g$ such that:
  \begin{itemize}
    \item $g\in L_1\left( \R,\mathcal{F},\mu \right)$ (in particular, $g$ is $\mathcal{F}$-measurable);
    \item for all $E\in \mathcal{F}$, $ \int_{E}^{} g\:d\mu = \int_{E}^{} f\:d\mu $.
  \end{itemize}
\end{problem}
We consider the signed measure on $ \mathcal{F} $ defined by
\begin{align*}
  \nu(E) &= \int_{E}^{} f\:d\mu,
\end{align*}
meaning that $\nu\ll \mu$, so the function $g \coloneq \diff{\nu}{\mu}$, where $\diff{\nu}{\mu}$ denotes the Radon--Nikodym derivative of $\nu$ with respect to $\mu$ (where we restrict $\mu$ to $\mathcal{F}$), is $\mathcal{F}$-measurable (by Radon--Nikodym) and in $L_1\left( \R,\mathcal{F},\mu \right)$. This gives, for all $E\in \mathcal{F}$,
\begin{align*}
  \int_{E}^{} g\:d\mu &= \int_{E}^{} \diff{\nu}{\mu}\:d\mu\\
                      &= \int_{E}^{} \:d\nu\\
                      &= \nu\left( E \right)\\
                      &= \int_{E}^{} f\:d\mu.
\end{align*}
\subsection{Problem 5}%
\begin{problem}
  Let $\mu$ be a finite measure on $\left( X,\mathcal{F} \right)$. Show that a sequence of $\mathcal{F}$-measurable functions $\left( f_n \right)_n$ converges to $f$ in measure if and only if
  \begin{align*}
    \int_{X}^{} \min\set{1,\left\vert f_n-f \right\vert}\:d\mu(x) &\rightarrow 0.
  \end{align*}
\end{problem}

Let $M = \mu(X)$.\newline

Let $\left( f_n \right)_n\rightarrow f$ in measure, and let $\ve > 0$. If we let
\begin{align*}
  A &= \set{x | \left\vert f_n(x)-f(x) \right\vert > \ve/2M}\\
  B &= \set{x | \left\vert f_n(x)-f(x) \right\vert \leq \ve/2M},
\end{align*}
we have
\begin{align*}
  \int_{X}^{} \min\left( 1,\left\vert f_n-f \right\vert \right)\:d\mu &= \int_{A}^{} \min\left( 1,\left\vert f_n-f \right\vert \right) \:d\mu + \int_{B}^{} \min\left( 1,\left\vert f_n-f \right\vert \right) \:d\mu\\
                                                                      &\leq \mu\left( A \right) + \ve/2\\
                                                                      &< \ve/2 + \ve/2\\
                                                                      &= \ve.
\end{align*}
Meanwhile, if
\begin{align*}
  \int_{X}^{} \min\left( 1,\left\vert f_n-f \right\vert \right)\:d\mu &\rightarrow 0,
\end{align*}
then by Chebyshev's Inequality, we have, for a fixed $0 < \ve \leq 1$,
\begin{align*}
  \mu\left( \set{x | \left\vert f_n-f \right\vert \geq \ve} \right) &= \mu \left( \set{x | \min\left( 1,\left\vert f_n-f \right\vert \right) \geq \ve} \right)\\
                                                                    &\leq \frac{1}{\ve} \int_{X}^{} \min\left( 1,\left\vert f_n-f \right\vert \right)\:d\mu\\
                                                                    &\rightarrow 0,
\end{align*}
so $\left( f_n \right)_n\rightarrow f$ in measure.
\section{\href{https://math.virginia.edu/graduate/exams/analysis/2020Aug_real.pdf}{August 2020}}%
\subsection{Problem 1}%
\begin{problem}
  Let $f\colon \R\rightarrow \R$ be continuous and almost everywhere differentiable such that $f'(x) = 1$ almost everywhere. Does this imply that $f(2)-f(1) = 1$?
\end{problem}
This is false. To see this, let $ \mathfrak{C}(x) $ denote the Cantor--Lebesgue function, and let
\begin{align*}
  h(x) &= \sum_{n=-\infty}^{\infty} \mathfrak{C}\left( x - n \right) + n.
\end{align*}
Then, since $\mathfrak{C}(x)$ has derivative zero almost everywhere, the sum of a number of translates of $\mathfrak{C}(x)$ still has derivative zero almost everywhere. Then, setting
\begin{align*}
  f(x) &= h(x) + x,
\end{align*}
we get that $f(x)$ has derivative equal to $1$ almost everywhere. However, at the same time, $f(2) - f(1) = 2$.
\subsection{Problem 2}%
\begin{problem}
  Prove or provide a counterexample to the assertion that every open set in $\R^2$ is a countable union of closed sets.
\end{problem}
We show the inverse problem, which is that every closed set in $\R^2$ is $G_{\delta}$. To do this, we let $A\subseteq \R^2$ be closed, nonempty, and proper (if $A = \emptyset$ or $A = \R^2$ the answer is trivial).\newline

Then, there is some $x\in A^{c}$, and specifically there is $x\in A^{c}$ with rational coordinates (else, select $y\in \Q^2$ within the ball of radius $\ve$ that allows $A^{c}$ to be open). Furthermore, since $\R^2$ is a metric space, $\R^2$ is regular, so there are open $U_{x}$ and $V_x$ such that $A\subseteq U_x$, $x\in V_x$, and $U_x\cap V_x = \emptyset$.\newline

Therefore, we get
\begin{align*}
  A &= \bigcap \set{U_x | x\in \Q^2\setminus A},
\end{align*}
meaning that $A$ is $G_{\delta}$. Taking complements, we thus get that every open set is $F_{\sigma}$.
\subsection{Problem 3}%
\begin{problem}
  Let $\mathcal{H}$ be a separable complex Hilbert space with basis $\left( f_n \right)_n$. Define $P\left( f_n \right) = f_{n+1}$.
  \begin{enumerate}[(a)]
    \item Find $P^{\ast}$, the adjoint to $P$.
    \item Find $PP^{\ast}$ and $P^{\ast}P$.
  \end{enumerate}
\end{problem}
\begin{enumerate}[(a)]
  \item We see that
    \begin{align*}
      \iprod{Pf_i}{f_j} &= \delta_{i+1,j}\\
                        &= \delta_{i,j-1}\\
                        &= \iprod{f_i}{f_{j-1}}\\
                        &= \iprod{f_i}{P^{\ast}f_j},
    \end{align*}
    so that $Pf_n = f_{n-1}$ if $n > 1$. Else, if $n = 1$, then $P^{\ast}f_n = 0$.
  \item We see that, acting on the orthonormal basis $\left( f_n \right)_n$, $P^{\ast}P\left( f_n \right) = f_n$, and
    \begin{align*}
      PP^{\ast}\left( f_n \right) &= \begin{cases}
        0 & n = 1\\
        1 & \text{else},
      \end{cases}
    \end{align*}
    so that $P^{\ast}P = I$ and $PP^{\ast}$ is as above.
\end{enumerate}
\subsection{Problem 4}%
\begin{problem}
  Let $\left( X,\mathcal{F},\mu \right)$ be a measure space with $\mu\left( X \right) = 1$. Let $f_n\colon X\rightarrow \R$ be measurable functions such that
  \begin{align*}
    \lim_{n\rightarrow\infty} \mu\left( \set{x | f_n(x) \leq t} \right) &= \begin{cases}
      0 & t < 0\\
      1 & t\geq 0
    \end{cases}.
  \end{align*}
  Show that $f_n\rightarrow 0$ in measure.
\end{problem}

We see that
\begin{align*}
  \mu\left( \set{x | f_n(x) > t} \right) &= \mu\left( X \right) - \mu\left( \set{x | f_n(x) \leq t} \right),
\end{align*}
so by taking limits, we find that
\begin{align*}
  \lim_{n\rightarrow\infty} \mu\left( \set{x | f_n(x) > t} \right) &= \begin{cases}
    1 & t < 0\\
    0 & t\geq 0
  \end{cases}.
\end{align*}
So, if $\ve > 0$, then
\begin{align*}
  \mu\left( \set{x | \left\vert f_n(x) \right\vert > \ve} \right) &= \mu\left( \set{x | f_n(x) < -\ve} \right) + \mu\left( \set{x | f_n(x) > \ve} \right)\\
                                                                  &\leq \mu\left( \set{x | f_n(x)\leq -\ve} \right) + \mu\left( \set{x | f_n(x) > \ve} \right)\\
                                                                  &\rightarrow 0.
\end{align*}
\section{\href{https://math.virginia.edu/graduate/exams/analysis/2021Jan_real.pdf}{January 2021}}%
\subsection{Problem 1}%
\begin{problem}
  Let $\left( f_n \right)_n$, $f$ be measurable functions on $\left( \Omega,\mathcal{F},\mu \right)$ such that $f_n\rightarrow f$ in measure. Does this imply that there exists a measurable set $A\subseteq \Omega$ with $\mu\left( \Omega\setminus A \right) = 0$ such that $f_n(x)\rightarrow f(x)$ for all $x\in A$.
\end{problem}

This is not true. To see this, consider the family of functions defined by
\begin{align*}
  f_1 &= \1_{[0,1]}\\
  f_2 &= \1_{[0,1/2]}\\
  f_3 &= \1_{[1/2,1]}\\
      &\vdots
\end{align*}
where $f_n$ is of width $\frac{1}{2^{k}}$ when $2^{k} \leq n < 2^{k+1}$, moving along $[0,1]$. Then, since $\mu\left( \set{x | \left\vert f_n(x) \right\vert > 0} \right) = \frac{1}{2^{k}}$, we have that $\left( f_n \right)_n\rightarrow 0$ in measure. Yet, since for any $x\in [0,1]$ there are infinitely many such $n$ such that $f_n(x) = 1$, the family $\left( f_n \right)_n$ does not converge to $0$ pointwise anywhere on $[0,1]$.
\subsection{Problem 2}%
\begin{problem}
  Let $B$ be a measurable subset of the two-dimensional plane such that the intersection of $B$ with every vertical line is either finite or countable. Find $\mu\left( B \right)$, where $\mu$ is the two-dimensional Lebesgue measure.
\end{problem}
Note that the two-dimensional Lebesgue measure is the completion of $m\times m$, where $m\times m$ is the product measure on the product $\sigma$-algebra $\mathcal{L}\left( \R \right)\otimes \mathcal{L}\left( \R \right)$. If $B\in \mathcal{L}\left( \R^2 \right)$, then $B = C\cup N$, where $N$ is a $\mu$-null set and $C\in \mathcal{L}\left( \R \right)\otimes \mathcal{L}\left( \R \right)$. Therefore, if we show that $\left( m\times m \right)\left( C \right) = 0$, we then show that $\mu\left( B \right) = 0$.\newline

To see that $\left( m\times m \right)\left( \C \right) = 0$, note that by our assumption, $B^{x} = \set{y \in \R | \left( x,y \right)\in B}$ is either finite or countable, so since $C^{x}\subseteq B^{x}$, we must have that $C^{x}$ is either finite or countable. By Tonelli's Theorem, since $\1_{C}$ is positive, we have
\begin{align*}
  \int_{\R^{2}}^{} \1_{C}\:d\left( m\times m \right) &= \int_{\R}^{} \int_{\R}^{} \1_{C^{x}}\:dy\:dx\\
                                                     &= \int_{\R}^{} m\left( C^{x} \right)\:dx\\
                                                     &= 0,
\end{align*}
so $\left( m\times m \right)\left( C^{x} \right) = 0$, meaning
\begin{align*}
  \mu\left( B \right) &= \mu\left( C \right) + \mu\left( N \right)\\
                      &= \left( m\times m \right)\left( C \right) + \mu\left( N \right)\\
                      &= 0.
\end{align*}
\subsection{Problem 3}%
\begin{problem}
  Let $\left( \Omega,\mathcal{F} \right)$ be a measurable space, $\mu,\nu,\rho$ finite positive measures with $\mu\ll\nu$. Show that there exists a measurable function $f$ on $\Omega$ such that for all $E\in \mathcal{F}$,
  \begin{align*}
    \mu\left( E \right) &= \int_{E}^{} f\:d\nu + \int_{E}^{} \left( f-1 \right)\:d\rho.
  \end{align*}
\end{problem}

Since $\mu\ll\nu$, and $\rho\ll\rho$, we have $\mu + \rho \ll \nu + \rho$, so by Radon--Nikodym, there is some measurable $f$ such that
\begin{align*}
  \mu\left( E \right) + \rho\left( E \right) &= \int_{E}^{} f\:d\left( \nu + \rho \right),
\end{align*}
so
\begin{align*}
  \mu\left( E \right) &= \int_{E}^{} f\:d\nu + \int_{E}^{} \left( f-1 \right)\:d\rho.
\end{align*}
\subsection{Problem 4}%
\begin{problem}
  Let $f,g$ be nonnegative measurable functions on $[0,1]$, and let $a,b,c,d\geq 0$ be arbitrary nonnegative numbers. Show that
  \begin{align*}
    \left( ac + bd + \int_{0}^{1} f(x)g(x)\:dx \right)^3 &\leq \left( a^3 + b^3 + \int_{0}^{1} \left( f(x) \right)^3\:dx \right)\left( c^{3/2} + d^{3/2} + \int_{0}^{1} \left( g(x) \right)^{3/2}\:dx \right)^2.
  \end{align*}
\end{problem}
Since all of $f,g,a,b,c,d$ are positive, we may show
\begin{align*}
  ac + bd + \int_{0}^{1} f(x)g(x)\:dx &\leq \left( a^3 + b^3 + \int_{0}^{1} \left( f(x) \right)^3\:dx \right)^{1/3}\left( c^{3/2} + d^{3/2} + \int_{0}^{1} \left( g(x) \right)^{3/2}\:dx \right)^{2/3}.
\end{align*}
To do this, we use Hölder's Inequality three times:
\begin{align*}
  ac + bd + \int_{0}^{1} f(x)g(x)\:dx &\leq \left( a^3 + b^3 \right)^{1/3}\left( c^{3/2} + d^{3/2} \right)^{2/3} + \int_{0}^{1} f(x)g(x)\:dx\\
                                      &\leq \left( a^3 + b^3 \right)^{1/3}\left( c^{3/2} + d^{3/2} \right)^{2/3} + \left( \int_{0}^{1} \left( f(x) \right)^{3}\:dx \right)^{1/3}\left( \int_{0}^{1} \left( g(x) \right)^{3/2}\:dx \right)^{2/3}\\
                                      &\leq \left( a^3 + b^3 + \int_{0}^{1} \left( f(x) \right)^{3}\:dx \right)^{1/3}\left( c^{3/2} + d^{3/2} + \int_{0}^{1} \left( g(x) \right)^{3/2}\:dx \right)^{2/3}.
\end{align*}
\subsection{Problem 5}%
\begin{problem}
  Let $f(x)$ be a continuous function on $[0,1]$. Show that for every $\ve > 0$ there exists $n\in \Z_{\geq 0}$ and $a_0,a_1,\dots,a_n\in \R$ such that for
  \begin{align*}
    D &\coloneq \sum_{k=0}^{n} a_k\left( \diff{}{x} \right)^{k},
  \end{align*}
  we have
  \begin{align*}
    \left\vert f(x)-e^{x^2}\left( De^{-x^2} \right) \right\vert &< \ve
  \end{align*}
  for all $x\in [0,1]$.
\end{problem}
We note that for each $n$,
\begin{align*}
  \left( \diff{}{x} \right)^{n}\left( e^{-x^2} \right) &= P_n(x)e^{-x^2}
\end{align*}
where $P_n(x)$ is a degree $n$ polynomial. To see this,
\begin{align*}
  \diff{}{x}\left( e^{-x^2} \right) &= -2xe^{-x^2}\\
  \diff{}{x}\left( P_n(x)e^{-x^2} \right) &= P_n'(x)e^{-x^2} - 2xP_n(x)e^{-x^2}\\
                                          &\coloneq P_{n+1}(x)e^{-x^2}.
\end{align*}
Therefore,
\begin{align*}
  e^{x^2}\left( \diff{}{x} \right)^{n}\left( e^{-x^2} \right) &= P_n(x).
\end{align*}
Since each $P_n(x)$ is linearly independent (as they have different degrees of polynomials), the linear combinations of $P_n$ form polynomials. By Stone--Weierstrass, there is some linear combination $\sum_{k=0}^{n}a_kP_{k}(x)$ such that $\left\vert f(x)-\sum_{k=0}^{n}a_kP_k(x) \right\vert < \ve$ for all $x\in [0,1]$. Thus, 
\begin{align*}
  \left\vert f(x)-e^{x^2}\left( De^{-x^2} \right) \right\vert < \ve
\end{align*}
for the particular expression
\begin{align*}
  D &= \sum_{k=0}^{n}a_k\left( \diff{}{x} \right)^{k}.
\end{align*}

\section{\href{https://math.virginia.edu/graduate/exams/analysis/2022Aug_real.pdf}{August 2022}}%
\subsection{Problem 1}%
\begin{problem}
  Compute
  \begin{align*}
    \lim_{n\rightarrow\infty} \int_{0}^{\infty} \frac{n\sin\left( x/n \right)}{x\left( 1+x^2 \right)}\:dx.
  \end{align*}
\end{problem}
We note that
\begin{align*}
  \left\vert \frac{n\sin\left( x/n \right)}{x\left( 1+x^2 \right)}  \right\vert &\leq \left\vert \frac{n\left( x/n \right)}{x\left( 1+x^2 \right)} \right\vert\\
                                                                                &= \frac{1}{1+x^2},
\end{align*}
and since $\frac{1}{1+x^2}$ is integrable, we may use Dominated Convergence to switch limit and integral, giving
\begin{align*}
  \lim_{n\rightarrow\infty} \int_{0}^{\infty} \frac{n\sin\left( x/n \right)}{x\left( 1+x^2 \right)}\:dx &= \int_{0}^{\infty} \lim_{n\rightarrow\infty}\frac{n\sin\left( x/n \right)}{x\left( 1+x^2 \right)}\:dx\\
                                                                                                        &= \int_{0}^{\infty} \lim_{h\rightarrow 0} \frac{\frac{1}{h}\sin\left( hx \right)}{x\left( 1+x^2 \right)}\:dx\\
                                                                                                        &= \int_{0}^{\infty} \frac{x}{x\left( 1+x^2 \right)}\:dx\\
                                                                                                        &= \frac{\pi}{2}.
\end{align*}
\subsection{Problem 2}%
\begin{problem}
  Fix $a < b$ in $\R$. For a Lipschitz function $g\colon [a,b]\rightarrow \C$, set
  \begin{align*}
    \norm{g}_{\text{Lip}} &= \sup_{x\neq y\in[a,b]} \frac{\left\vert g(x)-g(y) \right\vert}{\left\vert x-y \right\vert}.
  \end{align*}
  \begin{enumerate}[(a)]
    \item Show that $f\colon [a,b]\rightarrow \C$ is Lipschitz if and only if $f$ is absolutely continuous and $f'\in L_{\infty}\left( [a,b] \right)$.
    \item If $f\colon [a,b]\rightarrow \C$ is Lipschitz, show that $\norm{f}_{\text{Lip}} = \norm{f'}_{L_{\infty}}$.
  \end{enumerate}
\end{problem}
\begin{enumerate}[(a)]
  \item Let $f$ be Lipschitz, and let $M$ denote the Lipschitz constant --- i.e., $\left\vert f(x)-f(y) \right\vert\leq \left\vert x-y \right\vert$ for all $x,y\in [a,b]$. Set $\delta = \frac{\ve}{M}$. Then, if $\set{\left( a_j,b_j \right)}_{j=1}^{k}$ is a partition such that $\sum_{j=1}^{k}\left\vert b_j-a_j \right\vert < \delta$, we have
    \begin{align*}
      \sum_{j=1}^{k} \left\vert f\left(b_j\right)-f\left( a_j \right) \right\vert &\leq M\sum_{j=1}^{k} \left\vert b_j-a_j \right\vert\\
                                                                                  &< \ve.
    \end{align*}
    Thus, $f$ is absolutely continuous. Now, if $x,x+h\in [a,b]$, we have that
    \begin{align*}
      \left\vert \frac{f\left( x+h \right)-f\left( x \right)}{h} \right\vert &\leq M,
    \end{align*}
    meaning that
    \begin{align*}
      \left\vert f'(x) \right\vert &= \lim_{h\rightarrow 0} \left\vert \frac{f\left( x+h \right)-f(x)}{h} \right\vert\\
                                   &\leq M,
    \end{align*}
    and since $f'(x)$ exists for a.e. $x\in [a,b]$, we have that $\esssup_{x\in [a,b]} \left\vert f'(x) \right\vert \leq M$, so $f'\in L_{\infty}\left( [a,b] \right)$.\newline

    Let $f$ be absolutely continuous with bounded derivative. Then, if $M$ is the essential supremum of the $f'$, the fundamental theorem of calculus gives
    \begin{align*}
      \left\vert f(y)-f(x) \right\vert &= \left\vert \int_{x}^{y} f'(t)\:dt \right\vert\\
                                       &\leq \int_{x}^{y} \left\vert f'(t) \right\vert\:dt\\
                                       &\leq \int_{x}^{y} M\:dx\\
                                       &= M\left\vert y-x \right\vert,
    \end{align*}
    so $f$ is Lipschitz.
  \item If $f$ is such that $f'(x)$ exists, then for $x,x+h\in [a,b]$, we have
    \begin{align*}
      \left\vert \frac{f\left( x+h \right)-f(x)}{h} \right\vert &\leq \norm{f}_{\text{Lip}},
    \end{align*}
    so by taking limits, we have
    \begin{align*}
      \left\vert f'(x) \right\vert &\leq \norm{f}_{\text{Lip}}.
    \end{align*}
    Thus, this ordering must respect essential suprema, meaning
    \begin{align*}
      \norm{f'}_{L_{\infty}} &\leq \norm{f}_{\text{Lip}}.
    \end{align*}
    Furthermore, if $\ve > 0$, there are $x,y\in [a,b]$ with $x < y$  such that
    \begin{align*}
      \norm{f}_{\text{Lip}} - \ve &< \left\vert \frac{f(y)-f(x)}{y-x} \right\vert\\
                                  &= \frac{1}{\left\vert y-x \right\vert} \left\vert \int_{x}^{y} f'(t)\:dt \right\vert\\
                                  &\leq \frac{1}{\left\vert y-x \right\vert} \int_{x}^{y} \left\vert f'(t) \right\vert\:dt\\
                                  &\leq \frac{1}{\left\vert y-x \right\vert} \int_{x}^{y} \norm{f'}_{L_{\infty}}\:dt\\
                                  &= \norm{f'}_{L_{\infty}},
    \end{align*}
    and since $\ve$ is arbitrary, we have $\norm{f}_{\text{Lip}}\leq \norm{f'}_{L_{\infty}}$.
\end{enumerate}
\subsection{Problem 3}%
\begin{problem}
  Let $\left( X,\mu \right)$ be a $\sigma$-finite measure space. Show that if $f,g\in L_1\left( X,\mu \right)$ with $0\leq f,g$ almost everywhere, then
  \begin{align*}
    \norm{f-g}_{L_1} &= \int_{0}^{\infty} \mu\left( \set{x | f(x) > t} \triangle \set{x | g(x) > t} \right)\:dt.
  \end{align*}
\end{problem}
We start by showing that
\begin{align*}
  \left\vert a-b \right\vert &= \int_{0}^{\infty} \left\vert \1_{\left( t,\infty \right)}(a) - \1_{\left( t,\infty \right)}(b) \right\vert\:dt
\end{align*}
for all $a,b\in [0,\infty)$. Without loss of generality, $a \leq b$. To see this, note that there are three cases:
\begin{align*}
  \left\vert 1_{\left( t,\infty \right)}(a) - \1_{\left( t,\infty \right)}(b) \right\vert &= \begin{cases}
    0 & t<a,b\\
    1 & a\leq t < b\\
    0 & a,b\leq t
  \end{cases},
\end{align*}
giving
\begin{align*}
  \int_{0}^{\infty} \1_{[a,b)}\:dt &= \mu\left( [a,b) \right)\\
                                   &= b-a\\
                                   &= \left\vert a-b \right\vert.
\end{align*}
Now, we have
\begin{align*}
  \norm{f-g}_{L_1} &= \int_{X}^{} \left\vert f(x)-g(x) \right\vert\:d\mu(x)\\
                   &= \int_{X}^{} \int_{0}^{\infty} \left\vert \1_{(t,\infty)}\left(f(x)\right) - \1_{\left( t,\infty \right)}\left( g(x) \right) \right\vert\:dt\:d\mu(x),
                   \intertext{and by Tonelli's Theorem, we have}
                   &= \int_{0}^{\infty} \int_{X}^{} \left\vert \1_{f^{-1}\left( \left( t,\infty \right) \right)} - \1_{g^{-1}\left( \left( t,\infty \right) \right)} \right\vert\:d\mu(x)\:dt\\
                   &= \int_{0}^{\infty} \int_{X}^{} \1_{f^{-1}\left( \left( t,\infty \right) \right)\triangle g^{-1}\left( \left( t,\infty \right) \right)}\:d\mu(x)\:dt\\
                   &= \int_{0}^{\infty} \mu\left( f^{-1}\left( \left( t,\infty \right) \right)\triangle g^{-1}\left( \left( t,\infty \right) \right) \right)\:dt.
\end{align*}
\subsection{Problem 4}%
\begin{problem}
  Let $\left( X,\Sigma \right)$ be a measurable space. Suppose that $\mu,\nu$ are signed measures on $\Sigma$ such that $\norm{\mu}_{TV},\norm{\nu}_{TV} < \infty$, and $\left\vert \mu \right\vert\perp \left\vert \mu \right\vert$.
  \begin{enumerate}[(a)]
    \item If $\mu = \mu_1-\mu_2$ and $\nu = \nu_1 - \nu_2$ with $\mu_1\perp \mu_2$ and $\nu_1\perp \nu_2$, show that $\mu_i \perp \nu_j$ for all $i,j\in \set{1,2}$.
    \item Show that
      \begin{align*}
        \norm{\mu + \nu}_{\text{TV}} &= \norm{\mu}_{\text{TV}} + \norm{\nu}_{\text{TV}}.
      \end{align*}
  \end{enumerate}
\end{problem}

\begin{enumerate}[(a)]
  \item Since $\left\vert \mu \right\vert\perp \left\vert \nu \right\vert$, there are $U,V\subseteq X$ such that $\left\vert \mu \right\vert$ is concentrated on $U$ and $\left\vert \nu \right\vert$ is concentrated on $V$, with $U\cap V = \emptyset$.\newline

    Note that by the Jordan decompositions, we have $\left\vert \mu \right\vert = \mu_1 + \mu_2 \geq \mu_{1,2}$ so $\mu_{1,2}$ are concentrated on $U$, and similarly $\nu_{1,2}$ are concentrated on $V$, so $\mu_{i}\perp \nu_{j}$.
  \item We show that the measures $\mu_1 + \nu_1$ and $\mu_2 + \nu_2$ are mutually singular. To see this, note the following:
    \begin{itemize}
      \item $\mu_1 = 0$ on $N_{\mu}\cup V$;
      \item $\nu_1 = 0$ on $N_{\nu}\cup U$;
      \item $\mu_2 = 0$ on $P_{\mu}\cup V$;
      \item $\nu_2 = 0$ on $P_{\nu}\cup U$,
    \end{itemize}
    so $\mu_1 + \nu_1 = 0$ on $A = \left( N_{\mu}\cup V \right) \cap \left( N_{\nu}\cup U \right)$, and $\mu_2 + \nu_2 = 0$ on $B = \left( P_{\mu}\cup V \right)\cap \left( P_{\nu}\cup U \right)$. Therefore, since
    \begin{align*}
      A\cup B &= \left( N_{\mu}\cap N_{\nu} \right) \cup \left( N_{\mu}\cap U \right) \cup \left( N_{\nu}\cap V \right)\\
              &\cup \left( P_{\mu}\cap P_{\mu} \right) \cup \left( P_{\mu}\cap U \right) \cup \left( P_{\nu}\cap V \right)\\
              &= X\\
              \\
      A\cap B &= \left( N_{\mu}\cup V \right)\cap \left( N_{\nu}\cup U \right)\\
              &\cap \left( P_{\mu}\cup V \right) \cap \left( P_{\nu}\cup U \right)\\
              &= \emptyset,
    \end{align*}
    the measures $\mu_1 + \nu_1$ and $\mu_2 + \nu_2$ are mutually singular, so $A\sqcup B$ forms a Hahn decomposition for $\mu + \nu$ with corresponding Jordan decomposition of $\left( \mu_1 + \nu_1 \right) - \left( \mu_2 + \nu_2 \right)$. Thus,
    \begin{align*}
      \norm{\mu+\nu}_{\text{TV}} &= \left\vert \mu + \nu \right\vert(X)\\
                                                      &= \left( \mu_1 + \nu_1 \right)(X) + \left( \mu_2 + \nu_2 \right)(X)\\
                                          &= \left( \mu_1 + \mu_2 \right)(X) + \left( \nu_1 + \nu_2 \right)(X)\\
                                          &= \left\vert \mu \right\vert(X) + \left\vert \nu \right\vert(X)\\
                                          &= \norm{\mu}_{\text{TV}} + \norm{\nu}_{\text{TV}}.
    \end{align*}
    
\end{enumerate}
\subsection{Problem 5}%
\begin{problem}\hfill
  \begin{enumerate}[(a)]
    \item For $f\in L_1\left( [0,1] \right)$, let $L_f$ be the set of all $x\in [0,1]$ such that
      \begin{align*}
        \lim_{r\rightarrow 0} \frac{1}{2r} \int_{x-r}^{x+r} \left\vert f(y)-f(x) \right\vert\:dy &= 0.
      \end{align*}
      State the conclusion of the Lebesgue differentiation theorem regarding $L_f$.
    \item For $n\in \N$, $0 \leq j \leq 2^{n}-1$, set $I_{n,j} = \left[ j2^{-n},\left( j+1 \right)2^{-n} \right)$. For $f\in L_1\left( [0,1] \right)$, define
      \begin{align*}
        E_{n}f &= \sum_{j=0}^{2^{n}-1} \left( \frac{1}{m\left( I_{n,j} \right)} \int_{I_{n_j}}^{} f(t)\:dt \right)\1_{I_{n_j}}.
      \end{align*}
      Show that $\lim_{n\rightarrow\infty} \left( E_nf \right)(x) = f(x)$ for a.e. $x\in [0,1]$.
  \end{enumerate}
\end{problem}
\begin{enumerate}[(a)]
  \item The conclusion of the Lebesgue differentiation theorem states that $\mu\left( [0,1]\setminus L_f \right) = 0$.
  \item Let $x\in [0,1]$. We note that $x$ must be in exactly one such interval $\left( j2^{-n},\left( j+1 \right)2^{-n} \right]$ since these intervals are disjoint. If we select $r > 0$ such that $\frac{1}{2^{n}} < r \leq \frac{1}{2^{n-1}}$, then we note the following:
    \begin{itemize}
      \item $I_{n,j}\subseteq U\left( x,r \right)$ for exactly one such $j$;
      \item $m\left( U\left( x,r \right) \right) \leq 4\mu\left( I_{n,j} \right)$.
    \end{itemize}
    If $x\in L_f$, then for any $\ve > 0$, there is some $\delta > 0$ such that when $r < \delta$, then
    \begin{align*}
      \frac{1}{\mu\left( U\left( x,r \right) \right)} \int_{U\left( x,r \right)}^{} \left\vert f(t)-f(x) \right\vert\:dt &< \ve,
    \end{align*}
    by the Lebesgue Differentiation Theorem. If $n$ is such that $\frac{1}{2^{n-1}} < \delta$, then when $\frac{1}{2^{n}} < r \leq \frac{1}{2^{n-1}}$, then for any $x\in L_f$, we have
    \begin{align*}
      \left\vert E_nf\left( x \right) - f(x) \right\vert &= \left\vert \frac{1}{m\left( I_{n,j} \right)}\int_{I_{n,j}}^{} f(t)\:dt - f(x) \right\vert\\
                                                         &\leq \frac{1}{m\left( I_{n,j} \right)} \int_{I_{n,j}}^{} \left\vert f(t)-f(x) \right\vert\:dt\\
                                                         &\leq \frac{1}{m\left( I_{n,j} \right)} \int_{U\left( x,r \right)}^{} \left\vert f(t)-f(x) \right\vert\:dt\\
                                                         &\leq \frac{4}{U\left( x,r \right)} \int_{U\left( x,r \right)}^{} \left\vert f(t)-f(x) \right\vert\:dt\\
                                                         &< 4\ve,
    \end{align*}
    so $\lim_{n\rightarrow\infty}E_nf(x) = f(x)$ for all $x\in L_f$, meaning that it holds for a.e. $x\in [0,1]$.
\end{enumerate}
\section{\href{https://math.virginia.edu/graduate/exams/analysis/2023Jan_real.pdf}{January 2023}}%
\subsection{Problem 1}%
\begin{problem}
  Let $\left( X,\mu \right)$ be a $\sigma$-finite measure space, $p\in [1,\infty)$. Let $\left( f_n \right)_n$ be a sequence in $L_p\left( X,\mu \right)$, and suppose $\norm{f_n}_{L_p}\leq 1$, $\left( f_n \right)_n\rightarrow f$ almost everywhere. Show that $\norm{f}_{p}\leq 1$.
\end{problem}
By using Fatou's Lemma, and assuming WLOG that $\left( f_n \right)_n\rightarrow f$ pointwise everywhere, we get
\begin{align*}
  \int_{X}^{} \left\vert f \right\vert^{p}\:d\mu &= \int_{X}^{} \liminf_{n\rightarrow\infty} \left\vert f_n \right\vert^{p}\:d\mu\\
                                                 &\leq \liminf_{n\rightarrow\infty} \int_{X}^{} \left\vert f_n \right\vert^{p}\:d\mu\\
                                                 &\leq 1,
\end{align*}
so $\norm{f}_{L_p}\leq 1$.
\subsection{Problem 2}%
\begin{problem}
  Let $\mu$ be an atomless Borel probability measure on $\R$. Suppose $E\subseteq \R$ is a Borel set with $\mu\left( E \right) > 0$. Show that there is $t\in \R$ with $\mu\left( E\cap \left( -\infty,t \right) \right) = \frac{1}{2}\mu\left( E \right)$.
\end{problem}
Let
\begin{align*}
  f(t) &= \mu\left( E\cap \left( -\infty,t \right) \right),
\end{align*}
and for any sequence $\left( t_n \right)_n$, define
\begin{align*}
  E_n &= E\cap \left( -\infty,t_n \right).
\end{align*}
We will show that $f$ is left- and right-continuous, hence continuous. To start, if $\left( t_n \right)_n\searrow t$, then
\begin{align*}
  \bigcap_{n\in \N} E_n &= E\cap \left( -\infty,t \right],
\end{align*}
so
\begin{align*}
  f(t) &= \mu\left( \bigcap_{n\in \N} E_n\setminus \set{t} \right)\\
       &= \mu\left( \bigcap_{n\in \N} E_n \right) -\mu\left( \set{t} \right).
\end{align*}
Since $\mu$ is atomless, we see that $\mu\left( \set{t} \right) = 0$, so since $\mu\left( E \right) < \infty$,
\begin{align*}
  f(t) &= \mu\left( \bigcap_{n\in\N} E_n \right)\\
       &= \lim_{n\rightarrow\infty} \mu\left( E_n \right)\\
       &= \lim_{n\rightarrow\infty} f\left( t_n \right).
\end{align*}
Thus, $f$ is right-continuous. Similarly, if $f$ is left-continuous, and $\left( t_n \right)_n \nearrow t$, then
\begin{align*}
  \bigcup_{n\in \N} E_n &= E\cap \left( -\infty,t \right),
\end{align*}
so by continuity from below,
\begin{align*}
  f(t) &= \mu\left( \bigcup_{n\in\N} E_n \right)\\
       &= \lim_{n\rightarrow\infty} \mu\left( E_n \right)\\
       &= \lim_{n\rightarrow\infty} f\left( t_n \right).
\end{align*}
Therefore, $f$ is continuous. Since
\begin{align*}
  \lim_{t\rightarrow -\infty} f(t) &= 0\\
  \lim_{t\rightarrow\infty} f(t) &= \mu\left( E \right)\\
                                 &> 0,
\end{align*}
the intermediate value theorem gives some $t_0\in \R$ such that
\begin{align*}
  f\left(t_0\right) &= \mu\left( E\cap \left( -\infty,t_0 \right) \right)\\
                    &= \frac{1}{2}\mu\left( E \right).
\end{align*}
\subsection{Problem 4}%
\begin{problem}
  Fix $p\in [1,\infty)$. Let $W_{p}\left( [0,1] \right)$ be the space of absolutely continuous functions on $[0,1]$ such that $f'\in L_p\left( [0,1] \right)$. For all $f\in W_p\left( [0,1] \right)$, define
  \begin{align*}
    \norm{f}_{W_p} &= \left\vert f(0) \right\vert + \norm{f'}_{L_p}.
  \end{align*}
  Show that $\norm{\cdot}_{W_p}$ is a norm that makes $W_p\left( [0,1] \right)$ into a Banach space. You are allowed to use the fact that $L_p\left( [0,1] \right)$ is a Banach space.
\end{problem}
We start by showing that $\norm{\cdot}_{W_p}$ is indeed a norm. To see that $\norm{\cdot}_{W_p}$ is positive definite, if
\begin{align*}
  \norm{f}_{W_p} &= 0,
\end{align*}
then $\left\vert f(0) \right\vert = 0$ and $\norm{f'}_{L_p} = 0$. Since $\norm{f'}_{L_p} = 0$, $f' = 0$ a.e. as $L_p$ is a Banach space. Note that, by the fundamental theorem of calculus,
\begin{align*}
  f(x) &= f(0) + \int_{0}^{x} f'(t)\:dt,
\end{align*}
so $f(x) = 0$ almost everywhere, hence $f(x) = 0$ in $L_p$.\newline

Next, to see homogeneity, we have for all $\alpha\in\C$,
\begin{align*}
  \norm{\alpha f}_{W_p} &= \left\vert \alpha f(0) \right\vert + \norm{\left( \alpha f \right)'}_{L_p}\\
                        &= \left\vert \alpha \right\vert\left( \left\vert \alpha \right\vert + \norm{f'}_{L_p} \right)\\
                        &= \left\vert \alpha \right\vert\norm{f}_{W_p},
\end{align*}
as $\norm{\cdot}_{L_p}$ is a norm. Finally, we have
\begin{align*}
  \norm{f + g}_{W_p} &= \left\vert \left( f+g \right)(0) \right\vert + \norm{\left( f+g \right)'}_{L_p}\\
                     &\leq \left\vert f(0) \right\vert + \left\vert g(0) \right\vert + \norm{f'}_{L_p} + \norm{g'}_{L_p}\\
                     &= \norm{f}_{W_p} + \norm{g}_{W_p},
\end{align*}
as $\norm{\cdot}_{L_p}$ is a norm, so the triangle inequality holds. Thus, $\norm{\cdot}_{W_p}$ is a norm.\newline

Let $\left( f_n \right)_n$ be Cauchy in $W_p\left( [0,1] \right)$. Then, for all $\ve > 0$, there is $N\in \N$ such that for all $m,n\geq N$,
\begin{align*}
  \norm{f_n-f_m}_{W_p} &= \left\vert f_n(0)-f_m(0) \right\vert + \norm{f_n' - f_m'}_{L_p}\\
                       &< \ve,
\end{align*}
meaning that both
\begin{align*}
  \left\vert f_n(0)-f_m(0) \right\vert &< \ve\\
  \norm{f_n'-f_m'}_{L_p} &<\ve.
\end{align*}
Since $\C$ and $L_p\left( [0,1] \right)$ are complete, there is $c\in \C$ and $g\in L_p\left( [0,1] \right)$ such that
\begin{align*}
  f_n(0) &\rightarrow c\\
  f_n' &\rightarrow g.
\end{align*}
Define
\begin{align*}
  f(x) &= c + \int_{0}^{x} g(t)\:dt.
\end{align*}
Then, we note that by the Fundamental Theorem of Calculus,
\begin{align*}
  f'(x) &= g(x)\\
        &\in L_p\left( [0,1] \right),
\end{align*}
so $f\in W_p\left( [0,1] \right)$. Finally, we see that
\begin{align*}
  \norm{f_n-f}_{W_p\left( [0,1] \right)} &= \left\vert f_n(0) - f(0) \right\vert + \norm{f_n' - f'}_{L_p}\\
                                         &= \left\vert f_n(0) - c \right\vert + \norm{f_n'-g}_{L_p}\\
                                         &\rightarrow 0,
\end{align*}
so $\left( f_n \right)_n\rightarrow f$ in $W_p$, meaning $W_p$ is complete.
\subsection{Problem 5}%
\begin{problem}
  Let $m$ be Lebesgue measure on $\R$, $\Omega = \set{\1_{E} | E\subseteq \R\text{ Borel, }m(E) < \infty}$ be regarded as a subset of $L_1\left( \R \right)$. We regard $\Omega$ as a metric space with the $L_1$ distance.
  \begin{enumerate}[(i)]
    \item If $a < b$ are real numbers, show that the function $\Omega\rightarrow \R$ given by
      \begin{align*}
        \1_{E} &\mapsto m\left( E\cap [a,b] \right)
      \end{align*}
      is a continuous function.
    \item If $a < b$ are real numbers, let $U_{a,b}$ be the subset of $\Omega$ consisting of all $\1_{E}$ where $E\subseteq \R$ is Borel, and
      \begin{align*}
        0 < m\left( E\cap [a,b] \right) < b-a.
      \end{align*}
      Show that $U_{a,b}$ is open and dense in $\Omega$.
    \item Let $D$ be the set of all $\1_{E}$ where $E\subseteq \R$ is Borel, and for every interval $I$ of positive measure, we have
      \begin{align*}
        0 < m\left( E\cap I \right) < m\left( I \right).
      \end{align*}
      Show that there is a countable collection $\set{U_j}_{j\in J}$ of open and dense subsets of $\Omega$ with $\bigcap_{j\in J}U_j \subseteq D$.
  \end{enumerate}
\end{problem}
\begin{enumerate}[(i)]
  \item Letting $f\colon \Omega\rightarrow \R$ be defined by $f\left( \1_E \right) = m\left( E\cap [a,b] \right)$, we have
    \begin{align*}
      \left\vert m\left( E\cap [a,b] \right) - m\left( F\cap [a,b] \right) \right\vert &= \left\vert \int_{a}^{b} \1_{E} - \1_{F}\:dm \right\vert\\
                                                                                       &\leq \int_{a}^{b} \left\vert \1_{E} - \1_{F} \right\vert\:dm\\
                                                                                       &\leq \int_{\R}^{} \left\vert \1_{E} - \1_{F} \right\vert\:dm\\
                                                                                       &= \norm{\1_{E} - \1_{F}}_{L_1},
    \end{align*}
    meaning that $f$ is Lipschitz, hence continuous.
  \item Let $\1_{F}\in \Omega$. Then, $0 \leq \mu\left( F\cap [a,b] \right)\leq b-a$. If these inequalities are strict, then $F\in U_{a,b}$. Else, we let $\ve > 0$, and see two cases:
    \begin{itemize}
      \item if $\mu\left( F\cap [a,b] \right) = b-a$, then we may set $E = F\setminus \left( \left[a,a+\ve/\right)\cup \left( b-\ve/2,b \right] \right)$, so that $0 < \mu\left( E\cap [a,b] \right) < b-a$, and $\norm{\1_{E} - \1_{F}}_{L_1} = \mu\left( E\triangle F \right) \leq \ve$;
      \item if $\mu\left( F\cap [a,b] \right) = 0$, then we may set $E = F\cup \left( \left[ a,a+\ve/2 \right)\cup \left[ b-\ve/2,b \right) \right)$, meaning that $0 < \mu\left( E\cap [a,b] \right) < b-a$, and $\mu\left( E\triangle F \right) \leq \ve$.
    \end{itemize}
    Therefore, $U_{a,b}$ is dense in $\Omega$. To see that $U_{a,b}$ is open, notice that for any $\1_{E}\in U_{a,b}$, we may find $\ve > 0$ such that $0 < \mu\left( E\cap [a,b] \right) - \ve < \mu\left( E\cap [a,b] \right) < \mu\left( E\cap [a,b] \right) + \ve < b-a$, and for all $F$ with $\norm{\1_{F} - \1_{E}}_{L_1} < \ve$, we have
    \begin{align*}
      \left\vert \mu\left( F\cap [a,b] \right) - \mu\left( E\cap [a,b] \right) \right\vert &\leq \norm{\1_{F}-\1_{E}}_{L_1}\\
                                                                                           &< \ve,
    \end{align*}
    so $0 < \mu\left( F\cap [a,b] \right) < b-a$. Thus, $U_{a,b}$ is also open.
  \item If $\set{\left[ a_k,b_k \right]}$ is an enumeration of rational-endpoint intervals in $\R$, then for any interval $I$, there is some rational-endpoint interval $\left[a_k,b_k\right]\subseteq I$ by density and the characterization of an interval, meaning that $U_{a_k,b_k}$ is such that for the given interval $I$, $0 < \mu\left( E\cap I \right) < \mu\left( I \right)$ for all $E\in U_{a_k,b_k}$. Thus, taking the intersection of all such $U_{a_k,b_k}$, we have
    \begin{align*}
      \bigcap_{k=1}^{\infty}U_{a_k,b_k} &\subseteq D.
    \end{align*}
\end{enumerate}
\section{August 2023}%
\subsection{Problem 1}%
\begin{problem}
Let $\left( X,\mu \right)$ be a $\sigma$-finite Borel measure space. Let $\left( f_n \right)_n$ be a sequence in $L_2\left( X,\mu \right)$, and $f\in L_2\left( X,\mu \right)$ such that for every $g\in L_2\left( X,\mu \right)$, we have
\begin{align*}
  \lim_{n\rightarrow\infty} \int_{X}^{} f_n(x)g(x)\:d\mu(x) &= \int_{X}^{} f(x)g(x)\:d\mu(x).
\end{align*}
Furthermore, suppose that
\begin{align*}
  \lim_{n\rightarrow\infty}\norm{f_n}_{L_2} &= \norm{f}_{L_2}.
\end{align*}
Prove that there is a subsequence $\left( f_{n_j} \right)_j$ and a subset $E\subseteq X$ with $\mu\left( E \right) = 0$ such that for all $x\in X\setminus E$,
\begin{align*}
  \lim_{j\rightarrow\infty} \left\vert f_{n_j}(x)-f(x) \right\vert = 0.
\end{align*}
\end{problem}
In order to show that $\left( f_{n_j} \right)_j\rightarrow f$ pointwise a.e., we show that $\left( f_{n} \right)_n\rightarrow f$ in measure; it has been well-established that if $\left( f_n \right)_n\rightarrow f$ in measure, then $\left( f_n \right)_n$ admits a subsequence that converges to $f$ pointwise almost everywhere.\newline

By Chebyshev's Inequality, we have that
\begin{align*}
  \mu\left( \set{x | \left\vert f_n(x)-f(x) \right\vert \geq \ve} \right) &\leq \frac{1}{\ve^2}\norm{f_n-f}_{L_2}^2\\
                                                                          &= \frac{1}{\ve^2} \int_{X}^{} \left\vert f_n-f \right\vert^2\:d\mu.
\end{align*}
Focusing on the integral,
\begin{align*}
  \int_{X}^{} \left\vert f_n-f \right\vert^2\:d\mu &= \int_{X}^{} \left( f_n-f \right) \overline{\left( f_n-f \right)}\:d\mu\\
                                                   &= \int_{X}^{} \left\vert f_n \right\vert^2 - f_n \overline{f} - \overline{f_n}f + \left\vert f \right\vert^2\:d\mu\\
                                                   &= \int_{X}^{} \left\vert f_n \right\vert^2\:d\mu - \int_{X}^{} f_n \overline{f}\:d\mu + \int_{X}^{} \left\vert f \right\vert^2\:d\mu - \overline{\int_{X}^{} f_n \overline{f}\:d\mu}.
\end{align*}
Now, we note the following:
\begin{itemize}
  \item $\lim_{n\rightarrow\infty} \int_{X}^{} \left\vert f_n \right\vert^2\:d\mu = \int_{X}^{} \left\vert f \right\vert^2\:d\mu$; and
  \item if $f\in L_2\left( X,\mu \right)$, then so too is $ \overline{f} $.
\end{itemize}
Thus, by taking limits, we have
\begin{align*}
  \lim_{n\rightarrow\infty} \int_{X}^{} \left\vert f_n-f \right\vert^2\:dx &= \lim_{n\rightarrow\infty} \left( \int_{X}^{} \left\vert f_n \right\vert^2\:d\mu - \int_{X}^{} f_n \overline{f}\:d\mu + \int_{X}^{} \left\vert f \right\vert^2\:d\mu - \overline{\int_{X}^{} f_n \overline{f}\:d\mu} \right)\\
                                                                           &= \int_{X}^{} \left\vert f \right\vert^2\:d\mu - \int_{X}^{} \left\vert f \right\vert^2\:d\mu + \int_{X}^{} \left\vert f \right\vert^2\:d\mu - \overline{ \int_{X}^{} \left\vert f \right\vert^2\:d\mu }\\
                                                                           &= 0,
\end{align*}
so $\norm{f_n-f}_{L_2}^2 \rightarrow 0$. Thus, $\left( f_n \right)_n\rightarrow f$ in measure, and thus there is a subsequence $\left( f_{n_j} \right)_j\rightarrow f$ pointwise almost everywhere.
\subsection{Problem 3}%
\begin{problem}
  Let $X$ be a LCH space. Recall that $g\colon X\rightarrow \C$ vanishes at infinity if for every $\ve > 0$, there is a compact $K_{\ve}\subseteq X$ such that for all $x\in X\setminus K_{\ve}$, $\left\vert g(x) \right\vert < \ve$. Show that $C_0\left( X \right)$ is complete with respect to the sup norm.
\end{problem}
Let $\left( f_n \right)_n$ be Cauchy in the sup norm. Then, for all $\ve > 0$, there is $N$ such that for all $m,n\geq N$, $\norm{f_m - f_n} < \ve$. Therefore, for all $x\in X$, we have $\left\vert f_n(x)-f_m(x) \right\vert < \ve$, meaning that the sequence $\left( f_n(x) \right)_n$ is Cauchy in $\C$. Define $f(x) = \lim_{n\rightarrow\infty}f_n(x)$ for each $x$.\newline

We must now show that
\begin{itemize}
  \item $\left( f_n \right)_n\rightarrow f$ in the supremum norm;
  \item $f\in C_0\left( X \right)$.
\end{itemize}
For the first point, if $x\in X$, we see that for $\ve > 0$, there is $N$ such that for all $n \geq N$,
\begin{align*}
  \left\vert f_n(x)-f(x) \right\vert < \ve.
\end{align*}
Taking suprema, this gives
\begin{align*}
  \sup_{x\in X} \left\vert f_n(x)-f(x) \right\vert \leq \ve,
\end{align*}
so $\norm{f_n - f} \leq \ve$, meaning that $\left( f_n \right)_n\rightarrow f$ in the sup norm.\newline

Finally, we let $N_1$ be such that for all $n\geq N_1$, $\norm{f_n - f} < \ve/2$. Note that since $f_{N_1}\in C_0\left( X \right)$, we have a $K_{\ve/2}$ such that for all $x\in X\setminus K_{\ve/2}$, $\left\vert f_N(x) \right\vert < \ve/2$. Therefore, for all $x\in X\setminus K_{\ve/2}$, we have
\begin{align*}
  \left\vert f(x) \right\vert &\leq \left\vert f_{N_1}(x)-f(x) \right\vert + \left\vert f_{N_1}(x) \right\vert\\
                              &\leq \norm{f_{N_1} - f} + \left\vert f_{N_1}(x) \right\vert\\
                              &< \ve/2 + \ve/2\\
                              &= \ve,
\end{align*}
so $f\in C_0\left( X \right)$. Thus, $C_0\left( X \right)$ is complete.
\section{January 2024}%
\subsection{Problem 1}%
\begin{problem}
  Let $\left( X,\mu \right)$ be a $\sigma$-finite measure space, and suppose $\left( f_n \right)_n$ is a sequence in $L_2\left( X,\mu \right)$ such that $\sup_{n\geq 1}\norm{f_n}_{L_2} < \infty$ and $\left( f_n \right)_n\rightarrow f$ $\mu$-almost everywhere. Prove that $f\in L_2\left( X,\mu \right)$.
\end{problem}
Applying Fatou's Lemma, we find that
\begin{align*}
  \int_{X}^{} \left\vert f \right\vert^2\:d\mu &= \int_{X}^{} \liminf_{n\rightarrow\infty}\left\vert f_n \right\vert^2\:d\mu\\
                                               &\leq \liminf_{n\rightarrow\infty} \int_{X}^{} \left\vert f_n \right\vert^2\:d\mu\\
                                               &\leq \limsup_{n\rightarrow\infty} \int_{X}^{} \left\vert f_n \right\vert^2\:d\mu\\
                                               &\leq \sup_{n\geq 1} \int_{X}^{} \left\vert f_n \right\vert^2\:d\mu\\
                                               &< \infty.
\end{align*}
\subsection{Problem 2}%
\begin{problem}
  Let $\left( X,\mu \right)$ be a measure space, and let $p\in [1,\infty)$. Let $\left( f_n \right)_n\rightarrow f$ in $L_{p}$.
  \begin{enumerate}[(i)]
    \item Prove that there exists a subsequence $\left( f_{n_k} \right)$ such that $\norm{f_{n_{k+1}} - f_{n_{k}}}_{L_p} < 2^{-k}$.
    \item Show that for $\mu$-almost every $x$, we have $\lim_{k\rightarrow\infty} f_{n_{k}}(x) = f(x)$.
  \end{enumerate}
\end{problem}
\begin{enumerate}[(i)]
  \item Since $\left( f_{n} \right)_{n}\rightarrow f$ in $L_{p}$, we see that $\left( f_{n} \right)_{n}$ is $L_{p}$-Cauchy, so we may extract a subsequence as follows. Let $f_{n_{1}} = f_{1}$, and find $f_{n_{2}}$ with $n_2 > 1$ such that
    \begin{align*}
      \norm{f_{n_2} - f_{n_1}} &< \frac{1}{2}.
    \end{align*}
    Inductively, we may use the fact that $\left( f_{n} \right)_n$ is Cauchy to find $n_{k+1} > n_{k}$ such that
    \begin{align*}
      \norm{f_{n_{k+1}} - f_{n_{k}}} &< \frac{1}{2^{k}}.
    \end{align*}
  \item Consider the sequence $\left( s_n \right)_n$ given by
    \begin{align*}
      s_n &= \sum_{k=1}^{n} \left\vert f_{n_{k+1}} - f_{n_{k}} \right\vert.
    \end{align*}
    Then, by Minkowski's Inequality, we find that
    \begin{align*}
      \norm{s_{n}}_{L_p} &\leq \sum_{k=1}^{n}\norm{f_{n_{k+1}} - f_{n_{k}}}_{L_p}.
    \end{align*}
    In particular, since $s\coloneq \sum_{k=1}^{\infty}\left\vert f_{n_{k+1}} - f_{n_{k}} \right\vert$ is the limit of $\left( s_n \right)_n$, we may apply Fatou's Lemma to find that
    \begin{align*}
      \norm{s}_{L_p} &= \norm{\liminf_{n\rightarrow\infty}s_n}_{L_p}\\
                     &\leq \liminf_{n\rightarrow\infty} \norm{s_n}_{L_p}\\
                     &\leq \liminf_{n\rightarrow\infty} \sum_{k=1}^{n} \norm{f_{n_{k+1}} -f_{n_{k}}}_{L_p}\\
                     &\leq \liminf_{n\rightarrow\infty} \sum_{k=1}^{n} 2^{-k}\\
                     &= 1,
    \end{align*}
    meaning in particular that $s(x)$ is finite almost everywhere, so $\sum_{k=1}^{\infty} \left\vert f_{n_{k+1}} -f_{n_{k}} \right\vert$ converges almost everywhere. Therefore, the sum
    \begin{align*}
      h(x) &= f_{n_{1}}(x) + \sum_{k=1}^{\infty} \left( f_{n_{k+1}}(x) - f_{n_{k}}(x) \right)
    \end{align*}
    converges for almost every $x$. Defining $h(x) = 0$ for all $x$ where this sum does not converge absolutely, we notice that
    \begin{align*}
      f_{n_{1}}(x) + \sum_{k=1}^{m}\left( f_{n_{k+1}}(x) - f_{n_{k}}(x) \right) &= f_{n_{m+1}}(x),
    \end{align*}
    meaning that $h$ is the pointwise (almost everywhere) limit of the sequence $\left( f_{n_{k}} \right)_{k}$; by Minkowski's Inequality, and applying Fatou's Lemma, as earlier, we also find that
    \begin{align*}
      \norm{h}_{L_p} &\leq \norm{f_{n_1}}_{L_p} + \sum_{k=1}^{\infty}\norm{f_{n_{k+1}}-f_{n_{k}}}_{L_p}\\
                     &\leq \norm{f_{n_{1}}}_{L_p} + 1\\
                     &< \infty,
    \end{align*}
    meaning $h\in L_{p}\left( X,\mu \right)$. All we need to do now is show that $\norm{f-h}_{L_p} = 0$, meaning that $\left[ f \right] = \left[ h \right]$ under the pointwise almost everywhere equivalence relation. To see this, 
    \begin{align*}
      \int_{X}^{} \left\vert h-f \right\vert^{p}\:d\mu &= \int_{X}^{} \liminf_{k\rightarrow\infty} \left\vert f_{n_{k}} - f \right\vert^{p}\:d\mu\\
                                                       &\leq \liminf_{n\rightarrow\infty} \int_{X}^{} \left\vert f_{n_{k}} - f \right\vert^{p}\:d\mu\\
                                                       &= \liminf_{n\rightarrow\infty} \norm{f_{n_{k}} - f}_{L_p}^{p}\\
                                                       &= 0,
    \end{align*}
    where the last equality is derived from the fact that $\left( f_{n} \right)_n\rightarrow f$ in $L_p$, so every subsequence of $\left( f_{n} \right)_n$ converges to $f$ in $L_p$.
\end{enumerate}
\subsection{Problem 3}%
\begin{problem}
  Let $f$ be Lebesgue-integrable on $\R$, and let $g$ be a bounded continuous function on $\R$. Prove that the convolution 
  \begin{align*}
    \left( f\ast g \right)(x) &= \int_{\R}^{} f(y)g\left( x-y \right)\:dy
  \end{align*}
  is a continuous function on $\R$.
\end{problem}
Let $M = \sup_{x\in \R}\left\vert g(x) \right\vert$. We have
\begin{align*}
  \left\vert \left( f\ast g \right)\left( x \right) - \left( f\ast g \right)\left( y \right) \right\vert &= \left\vert \int_{\R}^{} f(t)g\left( x-t \right)-g\left( y-t \right)\:dt \right\vert\\
                                                                                                         &\leq \int_{\R}^{} \left\vert f(t) \right\vert\left\vert g\left( x-t \right)-g\left( y-t \right) \right\vert\:dt. \tag{$\ast$}
\end{align*}
Now, since $f\in L_1\left( \R \right)$, this means that the sequence
\begin{align*}
  \left( \int_{B\left( 0,n \right)}^{} \left\vert f(t) \right\vert\:dt \right)_{n} &\rightarrow \int_{\R}^{} \left\vert f(t) \right\vert\:dt,
\end{align*}
so that there is some value of $N$ such that, for any $\ve > 0$ and for all $n\geq N$,
\begin{align*}
  \int_{\R\setminus B\left( 0,n \right)}^{} \left\vert f(t) \right\vert\:dt &< \ve/2M.
\end{align*}
Splitting the right side of ($\ast$), we have
\begin{align*}
  \int_{\R}^{} \left\vert f(t) \right\vert\left\vert g\left( x-t \right)-g\left( y-t \right) \right\vert\:dt &= \int_{\R\setminus B\left( 0,N \right)}^{} \left\vert f(t) \right\vert\left\vert g\left( x-t \right)-g\left( y-t \right) \right\vert\:dt + \int_{B\left( 0,N \right)}^{} \left\vert f(t) \right\vert\left\vert g\left( x-t \right)-g\left( y-t \right) \right\vert\:dt.\\
                                                                                                             &\leq 2M \int_{\R\setminus B\left( 0,N \right)}^{} \left\vert f(t) \right\vert\:dt + \int_{B\left( 0,N \right)}^{} \left\vert f(t) \right\vert\left\vert g\left( x-t \right)-g\left( y-t \right) \right\vert\:dt\\
                                                                                                             &= \ve + \int_{B\left( 0,N \right)}^{} \left\vert f(t) \right\vert\left\vert g\left( x-t \right)-g\left( y-t \right) \right\vert\:dt. \tag{$\ast\ast$}
\end{align*}
Now, for the integral in ($\ast\ast$), we use the fact that $B\left( 0,N \right)$ is compact, so there is some $\delta$ such that whenever $\left\vert x-y \right\vert = \left\vert \left( x-t \right)-\left( y-t \right) \right\vert < \delta$ with $x,y\in B\left( 0,N \right)$, we have
\begin{align*}
  \left\vert g\left( x-t \right)-g\left( y-t \right) \right\vert &< \frac{\ve}{ \int_{B\left( 0,N \right)}^{} \left\vert f(t) \right\vert\:dt }.
\end{align*}
Thus, rewriting ($\ast\ast$), we have, whenever $x,y\in B\left( 0,N \right)$ with $\left\vert x-y \right\vert < \delta$,
\begin{align*}
  \ve + \int_{B\left( 0,N \right)}^{} \left\vert f(t) \right\vert\left\vert g\left( x-t \right)-g\left( y-t \right) \right\vert\:dt &< \ve + \int_{B\left( 0,N \right)}^{} \left\vert f(t) \right\vert \frac{\ve}{ \int_{B\left( 0,N \right)}^{} \left\vert f(t) \right\vert\:dt } \:dt\\
                                                                                                                                    &= 2\ve.
\end{align*}
Now, we may prove the desired result. Let $\ve > 0$. Let $N\in \N$ be such that for all $n\geq N$
\begin{align*}
  \int_{\R\setminus B\left( 0,n \right)}^{} \left\vert f(t) \right\vert\:dt < \ve/4M,
\end{align*}
Now, for a given $n\geq N$, since $B\left( 0,n \right)$ is compact, we may find $\delta$ such that whenever $x,y\in B\left( 0,n \right)$ with $\left\vert x-y \right\vert < \delta$, we have $\left\vert g(x)-g(y) \right\vert < \frac{\ve}{ 2\int_{B\left( 0,n \right)}^{} \left\vert f(t) \right\vert\:dt }$. Thus, whenever $\left\vert x-y \right\vert < \delta$ in $B\left( 0,n \right)$, it holds for all $t$ where $x-t,y-t\in B\left( 0,n \right)$ that $\left\vert x-t -\left( y-t \right) \right\vert < \delta$. This gives
\begin{align*}
  \left\vert \left( f\ast g \right)\left( x \right) - \left( f\ast g \right)\left( y \right) \right\vert &= \left\vert \int_{\R}^{} f(t)\left( g\left( x-t \right)-g\left( y-t \right) \right)\:dt \right\vert\\
                                                                                                         &\leq \int_{\R\setminus B\left( 0,n \right)}^{} \left\vert f(t) \right\vert \left\vert g\left( x-t \right)-g\left( y-t \right) \right\vert\:dt + \int_{B\left( 0,n \right)}^{} \left\vert f(t) \right\vert\left\vert g\left( x-t \right)-g\left( y-t \right) \right\vert\:dt\\
                                                                                                         &\leq 2M \left( \frac{\ve}{4M} \right) + \frac{\ve}{ 2 \int_{B\left( 0,n \right)}^{} \left\vert f(t) \right\vert\:dt }  \int_{B\left( 0,n \right)}^{} \left\vert f(t) \right\vert\:dt\\
                                                                                                         &= \ve,
\end{align*}
meaning that $f\ast g$ is uniformly continuous on $B\left( 0,n \right)$ for all $n\geq N$. Since for any $x\in \R$, there is some $n$ such that $x\in B\left( 0,n \right)$, where $n\geq N$, $f\ast g$ is continuous at $x$ for all $x\in \R$, and we are done.
\subsection{Problem 4}%
\begin{problem}
  Let $\left( a_n \right)_n$ be a sequence of complex numbers such that $\left\vert a_n \right\vert < 1$ for all $n$ and $\lim_{n\rightarrow\infty} a_n = 0$.
  \begin{enumerate}[(i)]
    \item Show that if $\sum_{n\geq 1}\left\vert a_n \right\vert < \infty$, then the sequence $\left( p_n \right)_n$ defined by $p_n = \prod_{i=1}^{n}\left( 1 + a_i \right)$ is convergent.
    \item Does the converse hold? In other words, is it true that if $\left( p_n \right)_n$ is convergent, we must have $\sum_{n\geq 1}\left\vert a_n \right\vert < \infty$? Recall the conditions that $\left\vert a_n \right\vert < 1$ for all $n$ and $\lim_{n\rightarrow \infty}a_n = 0$.
  \end{enumerate}
\end{problem}

\end{document}
