\documentclass[10pt]{mypackage}

% sans serif font:
%\usepackage{cmbright,sfmath,bbold}
%\renewcommand{\mathcal}{\mathtt}

%Euler:
\usepackage{newpxtext,eulerpx,eucal,eufrak}
\renewcommand*{\mathbb}[1]{\varmathbb{#1}}
\renewcommand*{\hbar}{\hslash}

%\renewcommand{\mathbb}{\mathds}
%\usepackage{homework}
%\usepackage{exposition}

\pagestyle{fancy} %better headers
\fancyhf{}
\rhead{Avinash Iyer}
\lhead{Real Analysis Qualifier Preparation}

\setcounter{secnumdepth}{0}

\begin{document}
\RaggedRight
This is a collection of old real analysis qualifier exam solutions.
\section{\href{https://math.virginia.edu/graduate/exams/analysis/2019Aug_real.pdf}{August 2019}}%
\subsection{Problem 1}%
\begin{enumerate}[(a)]
  \item Recall that the Cantor set $\mathcal{C}$ is defined to consist of all $x\in [0,1]$ such that $x$ only contains $0$ and $2$ in the ternary expansion of $x$. Writing $a\in [0,2]$ as
    \begin{align*}
      a &= \sum_{k=0}^{\infty}\frac{a_k}{3^{k}},
    \end{align*}
    where $a_k\in\set{0,1,2}$, we may then find $a_k$ at each ternary expansion slot for $k$ as follows:
    \begin{itemize}
      \item if $a_k = 0$, we may find $b_k,c_k\in \mathcal{C}$ such that $b_k = c_k = 0$
      \item if $a_k = 2$, we may find $b_k,c_k\in \mathcal{C}$ such that $b_k = 2$ and $c_k = 0$ or vice versa.
      \item if $a_k = 1$, we may find $b_k,c_k\in \mathcal{C}$ such that $b_{k+1} = c_{k+1} = 2$.
    \end{itemize}
    Therefore, since every digit of every ternary expansion in $[0,2]$ can be obtained from $\mathcal{C}$, we see that $\mathcal{C} + \mathcal{C} = [0,2]$.
  \item We may set $B$ to be the union of all integer translates of $\mathcal{C}$, and set $A = \mathcal{C}$. This yields closed subsets of $\R$ with Lebesgue measure zero that sum to $\R$.
\end{enumerate}
\subsection{Problem 2}%
Consider the sequence of functions
\begin{align*}
  f_n(x) &= n\1_{\left[ \frac{1}{n+1},\frac{1}{n} \right]},
\end{align*}
defined on $\left[ 0,1 \right]$. This sequence is pointwise convergent everywhere to zero, as $f_n(0) = 0$ and the Archimedean property give that for any $x\in (0,1]$, there is some $n$ large enough that gives $\frac{1}{n} < x$. Furthermore, we see that
\begin{align*}
  \int_{}^{} f_n\:d\mu &= n\left( \frac{1}{n}- \frac{1}{n+1} \right)\\
                       &= \frac{1}{n+1}\\
                       &\rightarrow 0.
\end{align*}
Finally, we see that by taking suprema, we have the integral
\begin{align*}
  \int_{}^{} \Phi\:d\mu &= \sum_{n=1}^{\infty}\frac{1}{n+1}\\
                        &\rightarrow \infty.
\end{align*}
\subsection{Problem 4}%
Suppose toward contradiction that both $f$ and $1/f$ are in $L_1\left( \R \right)$. Then, from Hölder's Inequality, we have
\begin{align*}
  \infty &= \int_{}^{} 1\:d\mu\\
         &\leq \left( \int_{}^{} f\:d\mu \right)^{1/2} \left( \int_{}^{} \frac{1}{f}\:d\mu \right)^{1/2}\\
         &< \infty,
\end{align*}
which is a contradiction.
\subsection{Problem 5}%
\begin{enumerate}[(a)]
  \item Let $f\in L_2\left( [-1,1] \right)$. We may find $g\in C\left( [-1,1] \right)$ such that $\norm{f-g}_{L_2} < \ve/2$. Similarly, we may find a polynomial $p$ such that $\norm{g-p}_{u} < \ve/4$, meaning that $\left\vert p(x)-g(x) \right\vert < \ve/4$ for all $x\in [-1,1]$. This yields
    \begin{align*}
      \norm{p-g}_{L_2} &= \left( \int_{-1}^{1} \left\vert p(x)-g(x) \right\vert^2\:dx \right)^{1/2}\\
                       &< \left( \int_{-1}^{1} \left( \frac{\ve}{4} \right)^2\:dx \right)^{1/2}\\
                       &= \left( \frac{\ve^2}{8} \right)^{1/2}\\
                       &< \frac{\ve}{2},
    \end{align*}
    so $\norm{f-p}_{L_2} < \ve$, meaning that the closed linear span of the monomials is dense in $L_2$, and the Legendre polynomials form an orthonormal system.
  \item We see that at every step in evaluating the expression
    \begin{align*}
      L_n(x) &= c_n \diff{^{n}}{x^{n}}\left( x^2-1 \right)^{n},\label{eq:legendre_expression}\tag{$\ast$}
    \end{align*}
    the degree of the polynomial increases by $1$, so each $L_n(x)$ has degree $n$. To verify that the polynomials generated from \eqref{eq:legendre_expression} are orthogonal to each other, we let $n > m$ without loss of generality, and use integration by parts to obtain
    \begin{align*}
      \iprod{L_n}{L_m} &= \int_{-1}^{1} \left( \diff{^{n}}{x^{n}}\left( x^2-1 \right)^{n} \right) \left( \diff{^{m}}{x^{m}}\left( x^2-1 \right)^{m} \right)\:dx\\
                       &= \diff{^{n-1}}{x^{n-1}}\left( x^2-1 \right)^{n}\diff{^{m}}{x^{m}}\left( x^2-1 \right)^{m}\biggr\vert_{-1}^{1} - \int_{-1}^{1} \diff{^{n-1}}{x^{n-1}}\left( x^2-1 \right)^{n}\diff{^{m+1}}{x^{m+1}}\left( x^2-1 \right)^{m} \:dx\\
                       &\vdots\\
                       &= \left( -1 \right)^{n} \int_{-1}^{1} \diff{^{m+n}}{x^{m+n}}\left( x^2-1 \right)^{m}\:dx\\
                       &= \left( -1 \right)^{n} \int_{-1}^{1} \diff{^{n}}{x^{n}}\left( \diff{^{m}}{x^{m}}\left( x^2-1 \right)^{m} \right)\:dx\\
                       &= \left( -1 \right)^{n} \int_{}^{} \diff{^{n}}{x^{n}}L_m(x)\:dx\\
                       &= 0,
    \end{align*}
    seeing as we are taking $n$ derivatives of a degree $m < n$ polynomial.
\end{enumerate}
\section{\href{https://math.virginia.edu/graduate/exams/analysis/2020Jan_real.pdf}{January 2020}}%
\subsection{Problem 1}%
\begin{enumerate}[(a)]
  \item This is false. If $A\subseteq [0,1]$ is the ``fat Cantor set'' constructed similar to the traditional Cantor, but obtained by deleting the middle fourth of each subinterval rather than the middle third, then $\mu(A) = \frac{1}{2}$, but $A$ is nowhere dense, meaning that if $U\subseteq A$ is open, then $U = \emptyset$.\newline

    To see that $A$ is nowhere dense, we see that $A$ is closed, so if $x\in A\subseteq [0,1]$, and $\ve > 0$, we may show that the interval $\left( x-\ve,x+\ve \right)$ is not contained in $A$. In the recursive construction of $A$, we may see that there is some step $n_1$ such that $\frac{1}{4^{n_1}} < 2\ve$, implying that $\left( x-\ve,x+\ve \right)$ is not contained in the recursive construction at $n_1$. Therefore $A^{\circ} = \emptyset$.
  \item This is true. By the definition of the Lebesgue outer measure, for any $\ve > 0$, there are $\set{\left( a_k,b_k \right)}_{k=1}^{\infty}$ such that
    \begin{align*}
      \mu(A) + \ve &< \mu\left( \bigcup_{k=1}^{\infty}\left( a_k,b_k \right) \right),
    \end{align*}
    so by setting
    \begin{align*}
      U &= \bigcup_{k=1}^{\infty}\left( a_k,b_k \right),
    \end{align*}
    we have that $U$ is open, meaning that by the definition of infimum, we get 
    \begin{align*}
      \mu(A) &= \inf\set{U | A\subseteq U,U\text{ open}}.
    \end{align*}
\end{enumerate}
\begin{remark}
  Part (a) can be solved by selecting $A = \R\setminus \Q\cap [0,1]$.
\end{remark}

\subsection{Problem 3}%
\begin{enumerate}[(a)]
  \item Consider the algebra of polynomials on $[0,1]$ without a constant term. Then, since linear combinations and multiplications still yield polynomials without constant term, and $f(x) = x$ separates points in $[0,1]$, this algebra satisfies the requirements of the question. Yet, since all elements of this algebra are equal to zero at $x= 0$, the uniform closure of the algebra yields all the continuous functions on $[0,1]$ with $f(0) = 0$.
  \item In order to satisfy the requirements of the Stone--Weierstrass theorem, we need the algebra $\mathcal{A}$ to include the constant functions.
\end{enumerate}
\subsection{Problem 4}%
We consider the signed measure on $ \mathcal{F} $ defined by
\begin{align*}
  \nu(E) &= \int_{E}^{} f\:d\mu,
\end{align*}
meaning that $\nu\ll \mu$, so the function $g \coloneq \diff{\nu}{\mu}$, where $\diff{\nu}{\mu}$ denotes the Radon--Nikodym derivative of $\nu$ with respect to $\mu$ (where we restrict $\mu$ to $\mathcal{F}$), is $\mathcal{F}$-measurable (by Radon--Nikodym) and in $L_1\left( \R,\mathcal{F},\mu \right)$. This gives, for all $E\in \mathcal{F}$,
\begin{align*}
  \int_{E}^{} g\:d\mu &= \int_{E}^{} \diff{\nu}{\mu}\:d\mu\\
                      &= \int_{E}^{} \:d\nu\\
                      &= \nu\left( E \right)\\
                      &= \int_{E}^{} f\:d\mu.
\end{align*}
\subsection{Problem 5}%
Let $M = \mu(X)$.\newline

Let $\left( f_n \right)_n\rightarrow f$ in measure, and let $\ve > 0$. If we let
\begin{align*}
  A &= \set{x | \left\vert f_n(x)-f(x) \right\vert > \ve/2M}\\
  B &= \set{x | \left\vert f_n(x)-f(x) \right\vert \leq \ve/2M},
\end{align*}
we have
\begin{align*}
  \int_{X}^{} \min\left( 1,\left\vert f_n-f \right\vert \right)\:d\mu &= \int_{A}^{} \min\left( 1,\left\vert f_n-f \right\vert \right) \:d\mu + \int_{B}^{} \min\left( 1,\left\vert f_n-f \right\vert \right) \:d\mu\\
                                                                      &\leq \mu\left( A \right) + \ve/2\\
                                                                      &< \ve/2 + \ve/2\\
                                                                      &= \ve.
\end{align*}
Meanwhile, if
\begin{align*}
  \int_{X}^{} \min\left( 1,\left\vert f_n-f \right\vert \right)\:d\mu &\rightarrow 0,
\end{align*}
then by Chebyshev's Inequality, we have, for a fixed $0 < \ve \leq 1$,
\begin{align*}
  \mu\left( \set{x | \left\vert f_n-f \right\vert \geq \ve} \right) &= \mu \left( \set{x | \min\left( 1,\left\vert f_n-f \right\vert \right) \geq \ve} \right)\\
                                                                    &\leq \frac{1}{\ve} \int_{X}^{} \min\left( 1,\left\vert f_n-f \right\vert \right)\:d\mu\\
                                                                    &\rightarrow 0,
\end{align*}
so $\left( f_n \right)_n\rightarrow f$ in measure.
\section{\href{https://math.virginia.edu/graduate/exams/analysis/2020Aug_real.pdf}{August 2020}}%
\subsection{Problem 1}%
This is false. To see this, let $ \mathfrak{C}(x) $ denote the Cantor--Lebesgue function, and let
\begin{align*}
  h(x) &= \sum_{n=-\infty}^{\infty} \mathfrak{C}\left( x - n \right) + n.
\end{align*}
Then, since $\mathfrak{C}(x)$ has derivative zero almost everywhere, the sum of a number of translates of $\mathfrak{C}(x)$ still has derivative zero almost everywhere. Then, setting
\begin{align*}
  f(x) &= h(x) + x,
\end{align*}
we get that $f(x)$ has derivative equal to $1$ almost everywhere. However, at the same time, $f(2) - f(1) = 2$.
\subsection{Problem 2}%
We show the inverse problem, which is that every closed set in $\R^2$ is $G_{\delta}$. To do this, we let $A\subseteq \R^2$ be closed, nonempty, and proper (if $A = \emptyset$ or $A = \R^2$ the answer is trivial).\newline

Then, there is some $x\in A^{c}$, and specifically there is $x\in A^{c}$ with rational coordinates (else, select $y\in \Q^2$ within the ball of radius $\ve$ that allows $A^{c}$ to be open). Furthermore, since $\R^2$ is a metric space, $\R^2$ is regular, so there are open $U_{x}$ and $V_x$ such that $A\subseteq U_x$, $x\in V_x$, and $U_x\cap V_x = \emptyset$.\newline

Therefore, we get
\begin{align*}
  A &= \bigcap \set{U_x | x\in \Q^2\setminus A},
\end{align*}
meaning that $A$ is $G_{\delta}$. Taking complements, we thus get that every open set is $F_{\sigma}$.
\subsection{Problem 3}%
\begin{enumerate}[(a)]
  \item We see that
    \begin{align*}
      \iprod{Pf_i}{f_j} &= \delta_{i+1,j}\\
                        &= \delta_{i,j-1}\\
                        &= \iprod{f_i}{f_{j-1}}\\
                        &= \iprod{f_i}{P^{\ast}f_j},
    \end{align*}
    so that $Pf_n = f_{n-1}$ if $n > 1$. Else, if $n = 1$, then $P^{\ast}f_n = 0$.
  \item We see that, acting on the orthonormal basis $\left( f_n \right)_n$, $P^{\ast}P\left( f_n \right) = f_n$, and
    \begin{align*}
      PP^{\ast}\left( f_n \right) &= \begin{cases}
        0 & n = 1\\
        1 & \text{else},
      \end{cases}
    \end{align*}
    so that $P^{\ast}P = I$ and $PP^{\ast}$ is as above.
\end{enumerate}
\subsection{Problem 4}%
We see that
\begin{align*}
  \mu\left( \set{x | f_n(x) > t} \right) &= \mu\left( X \right) - \mu\left( \set{x | f_n(x) \leq t} \right),
\end{align*}
so by taking limits, we find that
\begin{align*}
  \lim_{n\rightarrow\infty} \mu\left( \set{x | f_n(x) > t} \right) &= \begin{cases}
    1 & t < 0\\
    0 & t\geq 0
  \end{cases}.
\end{align*}
So, if $\ve > 0$, then
\begin{align*}
  \mu\left( \set{x | \left\vert f_n(x) \right\vert > \ve} \right) &= \mu\left( \set{x | f_n(x) < -\ve} \right) + \mu\left( \set{x | f_n(x) > \ve} \right)\\
                                                                  &\leq \mu\left( \set{x | f_n(x)\leq -\ve} \right) + \mu\left( \set{x | f_n(x) > \ve} \right)\\
                                                                  &\rightarrow 0.
\end{align*}
\section{\href{https://math.virginia.edu/graduate/exams/analysis/2022Aug_real.pdf}{August 2022}}%
\subsection{Problem 1}%
We note that
\begin{align*}
  \left\vert \frac{n\sin\left( x/n \right)}{x\left( 1+x^2 \right)}  \right\vert &\leq \left\vert \frac{n\left( x/n \right)}{x\left( 1+x^2 \right)} \right\vert\\
                                                                                &= \frac{1}{1+x^2},
\end{align*}
and since $\frac{1}{1+x^2}$ is integrable, we may use Dominated Convergence to switch limit and integral, giving
\begin{align*}
  \lim_{n\rightarrow\infty} \int_{0}^{\infty} \frac{n\sin\left( x/n \right)}{x\left( 1+x^2 \right)}\:dx &= \int_{0}^{\infty} \lim_{n\rightarrow\infty}\frac{n\sin\left( x/n \right)}{x\left( 1+x^2 \right)}\:dx\\
                                                                                                        &= \int_{0}^{\infty} \lim_{h\rightarrow 0} \frac{\frac{1}{h}\sin\left( hx \right)}{x\left( 1+x^2 \right)}\:dx\\
                                                                                                        &= \int_{0}^{\infty} \frac{x}{x\left( 1+x^2 \right)}\:dx\\
                                                                                                        &= \frac{\pi}{2}.
\end{align*}
\subsection{Problem 2}%
\begin{enumerate}[(a)]
  \item Let $f$ be Lipschitz, and let $M$ denote the Lipschitz constant --- i.e., $\left\vert f(x)-f(y) \right\vert\leq \left\vert x-y \right\vert$ for all $x,y\in [a,b]$. Set $\delta = \frac{\ve}{M}$. Then, if $\set{\left( a_j,b_j \right)}_{j=1}^{k}$ is a partition such that $\sum_{j=1}^{k}\left\vert b_j-a_j \right\vert < \delta$, we have
    \begin{align*}
      \sum_{j=1}^{k} \left\vert f\left(b_j\right)-f\left( a_j \right) \right\vert &\leq M\sum_{j=1}^{k} \left\vert b_j-a_j \right\vert\\
                                                                                  &< \ve.
    \end{align*}
    Thus, $f$ is absolutely continuous. Now, if $x,x+h\in [a,b]$, we have that
    \begin{align*}
      \left\vert \frac{f\left( x+h \right)-f\left( x \right)}{h} \right\vert &\leq M,
    \end{align*}
    meaning that
    \begin{align*}
      \left\vert f'(x) \right\vert &= \lim_{h\rightarrow 0} \left\vert \frac{f\left( x+h \right)-f(x)}{h} \right\vert\\
                                   &\leq M,
    \end{align*}
    and since $f'(x)$ exists for a.e. $x\in [a,b]$, we have that $\esssup_{x\in [a,b]} \left\vert f'(x) \right\vert \leq M$, so $f'\in L_{\infty}\left( [a,b] \right)$.\newline

    Let $f$ be absolutely continuous with bounded derivative. Then, if $M$ is the essential supremum of the $f'$, the fundamental theorem of calculus gives
    \begin{align*}
      \left\vert f(y)-f(x) \right\vert &= \left\vert \int_{x}^{y} f'(t)\:dt \right\vert\\
                                       &\leq \int_{x}^{y} \left\vert f'(t) \right\vert\:dt\\
                                       &\leq \int_{x}^{y} M\:dx\\
                                       &= M\left\vert y-x \right\vert,
    \end{align*}
    so $f$ is Lipschitz.
  \item If $f$ is such that $f'(x)$ exists, then for $x,x+h\in [a,b]$, we have
    \begin{align*}
      \left\vert \frac{f\left( x+h \right)-f(x)}{h} \right\vert &\leq \norm{f}_{\text{Lip}},
    \end{align*}
    so by taking limits, we have
    \begin{align*}
      \left\vert f'(x) \right\vert &\leq \norm{f}_{\text{Lip}}.
    \end{align*}
    Thus, this ordering must respect essential suprema, meaning
    \begin{align*}
      \norm{f'}_{L_{\infty}} &\leq \norm{f}_{\text{Lip}}.
    \end{align*}
    Furthermore, if $\ve > 0$, there are $x,y\in [a,b]$ with $x < y$  such that
    \begin{align*}
      \norm{f}_{\text{Lip}} - \ve &< \left\vert \frac{f(y)-f(x)}{y-x} \right\vert\\
                                  &= \frac{1}{\left\vert y-x \right\vert} \left\vert \int_{x}^{y} f'(t)\:dt \right\vert\\
                                  &\leq \frac{1}{\left\vert y-x \right\vert} \int_{x}^{y} \left\vert f'(t) \right\vert\:dt\\
                                  &\leq \frac{1}{\left\vert y-x \right\vert} \int_{x}^{y} \norm{f'}_{L_{\infty}}\:dt\\
                                  &= \norm{f'}_{L_{\infty}},
    \end{align*}
    and since $\ve$ is arbitrary, we have $\norm{f}_{\text{Lip}}\leq \norm{f'}_{L_{\infty}}$.
\end{enumerate}
\section{\href{https://math.virginia.edu/graduate/exams/analysis/2023Jan_real.pdf}{January 2023}}%
\subsection{Problem 1}%
By using Fatou's Lemma, and assuming WLOG that $\left( f_n \right)_n\rightarrow f$ pointwise everywhere, we get
\begin{align*}
  \int_{X}^{} \left\vert f \right\vert^{p}\:d\mu &= \int_{X}^{} \liminf_{n\rightarrow\infty} \left\vert f_n \right\vert^{p}\:d\mu\\
                                                 &\leq \liminf_{n\rightarrow\infty} \int_{X}^{} \left\vert f_n \right\vert^{p}\:d\mu\\
                                                 &\leq 1,
\end{align*}
so $\norm{f}_{L_p}\leq 1$.
\subsection{Problem 2}%
Let
\begin{align*}
  f(t) &= \mu\left( E\cap \left( -\infty,t \right) \right),
\end{align*}
and for any sequence $\left( t_n \right)_n$, define
\begin{align*}
  E_n &= E\cap \left( -\infty,t_n \right).
\end{align*}
We will show that $f$ is left- and right-continuous, hence continuous. To start, if $\left( t_n \right)_n\searrow t$, then
\begin{align*}
  \bigcap_{n\in \N} E_n &= E\cap \left( -\infty,t \right],
\end{align*}
so
\begin{align*}
  f(t) &= \mu\left( \bigcap_{n\in \N} E_n\setminus \set{t} \right)\\
       &= \mu\left( \bigcap_{n\in \N} E_n \right) -\mu\left( \set{t} \right).
\end{align*}
Since $\mu$ is atomless, we see that $\mu\left( \set{t} \right) = 0$, so since $\mu\left( E \right) < \infty$,
\begin{align*}
  f(t) &= \mu\left( \bigcap_{n\in\N} E_n \right)\\
       &= \lim_{n\rightarrow\infty} \mu\left( E_n \right)\\
       &= \lim_{n\rightarrow\infty} f\left( t_n \right).
\end{align*}
Thus, $f$ is right-continuous. Similarly, if $f$ is left-continuous, and $\left( t_n \right)_n \nearrow t$, then
\begin{align*}
  \bigcup_{n\in \N} E_n &= E\cap \left( -\infty,t \right),
\end{align*}
so by continuity from below,
\begin{align*}
  f(t) &= \mu\left( \bigcup_{n\in\N} E_n \right)\\
       &= \lim_{n\rightarrow\infty} \mu\left( E_n \right)\\
       &= \lim_{n\rightarrow\infty} f\left( t_n \right).
\end{align*}
Therefore, $f$ is continuous. Since
\begin{align*}
  \lim_{t\rightarrow -\infty} f(t) &= 0\\
  \lim_{t\rightarrow\infty} f(t) &= \mu\left( E \right)\\
                                 &> 0,
\end{align*}
the intermediate value theorem gives some $t_0\in \R$ such that
\begin{align*}
  f\left(t_0\right) &= \mu\left( E\cap \left( -\infty,t_0 \right) \right)\\
                    &= \frac{1}{2}\mu\left( E \right).
\end{align*}
\subsection{Problem 4}%
Let $\left( f_n \right)_n$ be Cauchy in $W_p\left( [0,1] \right)$. Then, for all $\ve > 0$, there is $N\in \N$ such that for all $m,n\geq N$,
\begin{align*}
  \norm{f_n-f_m}_{W_p} &= \left\vert f_n(0)-f_m(0) \right\vert + \norm{f_n' - f_m'}_{L_p}\\
                       &< \ve,
\end{align*}
meaning that both
\begin{align*}
  \left\vert f_n(0)-f_m(0) \right\vert &< \ve\\
  \norm{f_n'-f_m'}_{L_p} &<\ve.
\end{align*}
Since $\C$ and $L_p\left( [0,1] \right)$ are complete, there is $c\in \C$ and $g\in L_p\left( [0,1] \right)$ such that
\begin{align*}
  f_n(0) &\rightarrow c\\
  f_n' &\rightarrow g.
\end{align*}
Define
\begin{align*}
  f(x) &= c + \int_{0}^{x} g(t)\:dt.
\end{align*}
Then, we note that by the Fundamental Theorem of Calculus,
\begin{align*}
  f'(x) &= g(x)\\
        &\in L_p\left( [0,1] \right),
\end{align*}
so $f\in W_p\left( [0,1] \right)$. Finally, we see that
\begin{align*}
  \norm{f_n-f}_{W_p\left( [0,1] \right)} &= \left\vert f_n(0) - f(0) \right\vert + \norm{f_n' - f'}_{L_p}\\
                                         &= \left\vert f_n(0) - c \right\vert + \norm{f_n'-g}_{L_p}\\
                                         &\rightarrow 0,
\end{align*}
so $\left( f_n \right)_n\rightarrow f$ in $W_p$, meaning $W_p$ is complete.
\end{document}
