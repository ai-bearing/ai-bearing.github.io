\documentclass[10pt]{mypackage}

% sans serif font:
%\usepackage{cmbright}
%\usepackage{sfmath}
%\usepackage{bbold} %better blackboard bold

%\usepackage{homework}
\usepackage{notes}
\usepackage{mlmodern}
%\usepackage{newpxtext,eulerpx,eucal}
%\renewcommand*{\mathbb}[1]{\varmathbb{#1}}
\usepackage{parskip}

\fancyhf{}
\fancyhead[R]{Avinash Iyer}
\fancyhead[L]{Regular Values}
\fancyfoot[C]{\thepage}

\setcounter{secnumdepth}{0}

\begin{document}
\RaggedRight
A smooth map between manifolds $f\colon M\rightarrow N$ includes a certain family of local information; for instance, the derivative $D_pf\colon T_pM\rightarrow T_{f(p)}N$, which is a linear map between tangent spaces at $p$ and $q$, is defined on a coordinate chart $U\subseteq M$ for $p$ and a corresponding coordinate chart $V\subseteq N$ for $f(p)$. Yet, the properties of this linear map can give us information about the underlying map $f$.

To understand this, we need to dive into the world of regular and critical values.

Much of this document is based on the book \textit{Topology from the Differentiable Viewpoint} and assorted notes from my Differential Topology class.
\section{Sard's Theorem and the Regular Value Theorem}%
\begin{definition}
  Let $f\colon M\rightarrow N$ be a smooth map, and let $p\in M$. We say $p$ is a \textit{critical point} for $f$ if $D_pf$ does not have the same rank as the dimension of $T_{f(p)} N$. 

  If $D_pf$ has the same rank as the dimension of $T_{f(p)}N$, then we say that $p$ is a \textit{regular point} of $f$.

  We say $q\in N$ is a \textit{critical value} for $f$ if $f^{-1}\left( \set{q} \right)$ contains a critical point for $f$. Else,we say that $q$ is a \textit{regular value}.
\end{definition}
We start with the case of Sard's Theorem on $\R^{n}$. Then, we will expand this to the case of any arbitrary manifold by means of a technical lemma.
\begin{theorem}[Sard's Theorem]
  Let $f\colon \R^{n}\supseteq U\rightarrow \R^{m}$ be a smooth map. Then, if $C$ is the set of critical points for $f$, we have $f(C) \subseteq \R^{m}$ has measure zero.
\end{theorem}
The proof of Sard's Theorem is very technical, so we will not be showing the full proof. A proof can be found at \href{https://public.websites.umich.edu/~alexmw/Sard.pdf}{this link}.

A useful result used in conjunction with Sard's Theorem is the Regular Value Theorem. We will show some important results using these two theorems.
\begin{theorem}[Regular Value Theorem]
  Let $f\colon M\rightarrow N$ be a smooth map of manifolds with dimensions $m\geq n$. If $q\subseteq N$ is a regular value, then $f^{-1}\left( \set{q} \right)\subseteq M$ is a submanifold of dimension $m-n$.
\end{theorem}
\begin{proof}
  Let $p\in f^{-1}\left( \set{q} \right)$, and let $\left( U,\varphi \right)$ be a chart about $p$ where $\varphi(U)\cong \R^{m}\cong T_pM$ are identified  together. Since $D_pf$ is full rank, we have that $K = \ker\left( D_pf \right)$ is of codimension $n$, meaning that $K \cong \R^{m-n}$.

  Let $L\colon \R^{m}\rightarrow \R^{m-n}$ be a projection, and define $F\colon U\rightarrow N\times \R^{m-n}$ by $x\mapsto \left( f(x),L(x) \right)$. Then, since $L$ is a linear map and the matrix representation for $D_pF$ is block-diagonal, we have that $D_pF = \left( D_pf,L \right)$. In particular, $D_pF\colon \R^{n}\rightarrow \R^{n}$ is full rank, so by the \href{https://en.wikipedia.org/wiki/Inverse_function_theorem#Statements}{inverse function theorem}, $F$ is invertible on a neighborhood $V\times W\subseteq N\times \R^{m-n}$, where $W$ is a neighborhood of $0$. We may thus identify $U\cong V\times W$.

  By composing with the projection $\pi\colon N\times \R^{m-n}\rightarrow N$ given by $\left( q,W \right)\mapsto q$, we have that $f = \pi\circ F$, meaning $f^{-1}\left( \set{q} \right) = F^{-1}\left( \pi^{-1}\left( \set{q} \right) \right)$, so that $f^{-1}\left( \set{q} \right)\cong \R^{m-n}$.
\end{proof}
\begin{remark}
  If $M$ is compact and $N$ has the same dimension as $M$, $f^{-1}\left( \set{q} \right)$ is discrete. Additionally, the cardinality $\left\vert f^{-1}\left( \set{q} \right) \right\vert$ is a locally constant function of $q$.

  To see this, let $p_1,\dots,p_k$ be the elements of $f^{-1}\left( \set{q} \right)$ with corresponding disjoint open neighborhoods $U_1,\dots,U_k$. These neighborhoods are necessarily mapped diffeomorphically onto neighborhoods $V_1,\dots,V_k$ in $N$. If we let
  \begin{align*}
    V &= \left( V_1\cap\cdots\cap V_k \right) \setminus f\left( M\setminus \left( U_1\cup\cdots\cup U_k \right) \right),
  \end{align*}
  then for any $w\in V$, we have $\left\vert f^{-1}\left( \set{w} \right) \right\vert$ is equal to $\left\vert f^{-1}\left( \set{q} \right) \right\vert$.
\end{remark}
\section{The No-Retraction Theorem}%
One of the primary applications of the Regular Value Theorem is the No-Retraction Theorem, which is essentially a generalization of the smooth version of the Brouwer Fixed-Point Theorem.
\begin{theorem}
  Let $M$ be a compact smooth $n$-dimensional manifold with boundary $N = \partial M$. There does not exist any smooth surjective function $r\colon M\rightarrow N$ such that $r|_{N} = \id_N$.
\end{theorem}
\begin{proof}
  Suppose toward contradiction that there were a retraction. Let $X$ be the set of critical points for $r$ in $M$; by Sard's Theorem, $r(X)\subseteq N$ has measure zero, so there exists a regular value $y\in N$.

  By the Regular Value Theorem, $r^{-1}\left( \set{y} \right)$ is a smooth $1$-dimensional manifold, so $r^{-1}\left( \set{y} \right)$ is either $S^{1}$ or an open interval. If $r^{-1}\left( \set{y} \right)$ is $S^{1}$, then $r^{-1}\left( \set{y} \right)$ is necessarily contained in the interior of $M$, which would contradict the fact that $y\in \partial M$ and $y\in r^{-1}\left( \set{y} \right)$. Therefore, $r^{-1}\left( \set{y} \right)$ is an interval, and specifically it is one that has both of its endpoints on $N$. This follows from the fact that on the interior of $M$, such an interval must be identified to a $1$-dimensional topological subspace of $M$. Therefore, there is some $z \neq y\in N$ such that $z\in r^{-1}\left( \set{y} \right)$, implying that $r(z) = y\neq z$, which means $r$ is not a retraction.
\end{proof}
The usefulness of results like the No-Retraction Theorem is often amplified in homology theory, where one can find an analogous result through cohomology groups. 
\section{Mapping Degree Theory}%
The best use case for the Regular Value Theorem is in proving the equivalence between two ways to describe the ``wrapping'' of a map from one manifold about another.
\begin{definition}
  Let $f\colon M\rightarrow N$ be a smooth map between manifolds of the same dimension. Then, the degree of $f$ at $q\in N$, is given by
  \begin{align*}
    \deg_q(f) &= \sum_{p\in f^{-1}\left( \set{q} \right)}\sgn\left( D_pf \right),
  \end{align*}
  where $\sgn\left( D_pf \right)$ denotes the sign of the determinant of $D_pf$.
\end{definition}
\end{document}
