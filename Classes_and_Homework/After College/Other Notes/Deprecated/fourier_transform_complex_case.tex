\documentclass[10pt]{mypackage}

% sans serif font:
%\usepackage{cmbright}
%\usepackage{sfmath}
%\usepackage{bbold} %better blackboard bold

%\usepackage{homework}
\usepackage{notes}
\usepackage{newpxtext,eulerpx,eucal}
\renewcommand*{\mathbb}[1]{\varmathbb{#1}}

\fancyhf{}
\fancyhead[R]{Avinash Iyer}
\fancyhead[L]{The Fourier Transform: Complex Analysis Edition}
\fancyfoot[C]{\thepage}

\setcounter{secnumdepth}{0}

\begin{document}
\RaggedRight
These are some notes I'm taking on the Fourier transform for functions with domain on a particular type of subset of $\C$.
\begin{definition}
  Let $0 < a < \infty$. The class $\mathcal{F}_a$ is the family of holomorphic functions 
  \begin{align*}
    f\colon \set{z\in \C | \left\vert \im(z) \right\vert < a} &\rightarrow \C
  \end{align*}
  for which
  \begin{align*}
    A_{f,\ve} &\coloneq \sup_{\left\vert \im(z) \right\vert < a} \left( 1 + \left\vert \re(z) \right\vert \right)^{1+\ve}\left\vert f(z) \right\vert
  \end{align*}
  is finite for some $\ve > 0$.\newline

  We define
  \begin{align*}
    \mathcal{F} &= \bigcup_{0 < a < \infty} \mathcal{F}_a
  \end{align*}
\end{definition}
\begin{definition}
  If $f\in \mathcal{F}$, we define the \textit{Fourier transform} of $f$ by
  \begin{align*}
    \hat{f}\left( \xi \right) &= \int_{-\infty}^{\infty} f(x)e^{-2\pi i x \xi}\:dx.
  \end{align*}
\end{definition}
\begin{lemma}
  If $f\in \mathcal{F}_a$, then for any $0 \leq b < a$, we have
  \begin{align*}
    \hat{f}\left( \xi \right) &= \begin{cases}
      \int_{-\infty}^{\infty} f\left( x-ib \right)e^{-2\pi i \left( x-ib \right)\xi}\:dx & \xi \geq 0\\
      \int_{-\infty}^{\infty} f\left( x + ib \right)e^{-2\pi i \left( x + ib \right)\xi}\:dx & \xi \leq 0.
    \end{cases}
  \end{align*}
\end{lemma}
\begin{proof}
  For $0 < b < a$, define $g(z) = f(z)e^{-2\pi i z \xi}$. Let $\gamma_R$ be the rectangle with corners $-R,R,R-ib,-R-ib$ traversed \textit{clockwise}. As $R\rightarrow\infty$, the integral on the top horizontal line tends to $\hat{f}\left( \xi \right)$.\newline

  The integral over the left vertical line is estimated by
  \begin{align*}
    \left\vert i\int_{-b}^{0} f\left( -R + iy \right)e^{-2\pi i \left( -R + iy \right)\xi}\:dy \right\vert &\leq \frac{A_{f,\ve}}{R^{1+\ve}} \int_{-B}^{0} e^{-2\pi y \xi}\:dy\\
                                                                                                           &\leq \frac{A_{f,\ve}}{R^{1+\ve}},
  \end{align*}
  which tends to zero as $R\rightarrow\infty$. A similar bound holds for the integral over the right horizontal line. Since $f$ is holomorphic, Cauchy's Integral theorem implies that
  \begin{align*}
    0 &= \hat{f}\left( \xi \right) - \int_{-\infty}^{\infty} f\left( x-ib \right)e^{-2\pi i \left( x-ib \right)\xi}\:dx,
    \intertext{whence}
    \hat{f}\left( \xi \right) &= \int_{-\infty}^{\infty} f\left( x-ib \right)e^{-2\pi i \left( x-ib \right)\xi}\:dx.
  \end{align*}
  An analogous argument holds for $\xi \leq 0$ when the rectangle is taken over the upper half-plane.
\end{proof}
\begin{corollary}
  If $f\in \mathcal{F}_a$, then $ \hat{f}\left( \xi \right) $ has exponential decay, in the sense that there is some $B \geq 0$ such that
  \begin{align*}
    \left\vert \hat{f}\left( \xi \right) \right\vert &\leq Be^{-2\pi b \left\vert \xi \right\vert}
  \end{align*}
  for all $0\leq b < a$.
\end{corollary}
\begin{proof}
  For $0 < b < a$, and $\xi \geq 0$, we have
  \begin{align*}
    \hat{f}\left( \xi \right) &= \int_{-\infty}^{\infty} f\left( x-ib \right)e^{-2\pi i \left( x-ib \right)\xi}\:dx.
  \end{align*}
  Therefore, we have
  \begin{align*}
    \left\vert \hat{f}\left( \xi \right) \right\vert &\leq A_{f,\ve} e^{-2\pi b \xi} \int_{-\infty}^{\infty} \frac{1}{\left( 1 + \left\vert x \right\vert \right)^{1+\ve}}\:dx.
  \end{align*}
  Therefore, we get the desired bound for $\xi \geq 0$. A similar argument yields the desired bound for $\xi \leq 0$.
\end{proof}
\begin{theorem}[Fourier Inversion Formula]
  If $f\in \mathcal{F}$, we have
  \begin{align*}
    f(x) &= \int_{-\infty}^{\infty} \hat{f}\left( \xi \right)e^{2\pi i x \xi}\:d\xi.
  \end{align*}
\end{theorem}
\begin{proof}
  Write
  \begin{align*}
    \int_{-\infty}^{\infty} \hat{f}\left( \xi \right)e^{2\pi i x \xi}\:d\xi &= \int_{-\infty}^{0} \hat{f}\left( \xi \right)e^{2\pi i x \xi}\:d\xi + \int_{0}^{\infty} \hat{f}\left( \xi \right)e^{2\pi i x \xi}\:d\xi.
  \end{align*}
  We consider the second integral. Since $f\in \mathcal{F}_a$ for some $a$, we let $0 < b < a$. We use the lemma to write
  \begin{align*}
    \hat{f}\left( \xi \right) &= \int_{-\infty}^{\infty} f\left( u-ib \right)e^{-2\pi i \left( u-ib \right)\xi}\:du.
  \end{align*}
  Using the convergence of integration over $\xi$ and Fubini's Theorem,
  \begin{align*}
    \int_{0}^{\infty} \hat{f}\left( \xi \right)e^{2\pi i x \xi}\:d\xi &= \int_{0}^{\infty} \int_{-\infty}^{\infty} f\left( u-ib \right)e^{-2\pi i \left( u-ib \right)\xi}e^{2\pi i x \xi}\:du\:d\xi\\
                                                                      &= \int_{-\infty}^{\infty} \int_{0}^{\infty} e^{-2\pi i \left( u-ib-x \right)\xi}\:d\xi\:du\\
                                                                      &= \int_{-\infty}^{\infty} f\left( u-ib \right)\frac{1}{2\pi b + 2\pi \left( u-x \right)}\:du\\
                                                                      &= \frac{1}{2\pi i} \int_{-\infty}^{\infty} \frac{f\left( u-ib \right)}{u-ib-x}\:du\\
                                                                      &= \frac{1}{2\pi i} \int_{L_1}^{} \frac{f\left( \zeta \right)}{\zeta - x}\:d\zeta,
  \end{align*}
  where $L_1$ is the line $u-ib$ traversed from left to right. Similarly, we may write
  \begin{align*}
    \int_{-\infty}^{0} \hat{f}\left( \xi \right)e^{2\pi i x \xi}\:d\xi &= -\frac{1}{2\pi i} \int_{L_2}^{} \frac{f\left( \zeta \right)}{\zeta - x}\:d\zeta,
  \end{align*}
  where $L_2$ is the line $u+ib$ traversed from right to left. If we close up this contour into a rectangle, then we find that we get
  \begin{align*}
    f(x) &= \lim_{R\rightarrow\infty}\frac{1}{2\pi i} \int_{\gamma_R}^{} \frac{f\left( \zeta \right)}{\zeta - x}\:d\zeta\\
         &= \frac{1}{2\pi i}\int_{L_1}^{} \frac{f\left( \zeta \right)}{\zeta - x}\:d\zeta - \frac{1}{2\pi i} \int_{L_2}^{} \frac{f\left( \zeta \right)}{\zeta - x}\:d\zeta.
  \end{align*}
\end{proof}
The Fourier inversion formula gives access to a converse that provides structural properties of functions.
\begin{lemma}
  Let $f\colon \R\rightarrow \C$ be a continuous integrable function. Suppose its Fourier transform is such that there exists some $B\geq 0$ and $a > 0$ such that $ \left\vert \hat{f}\left( \xi \right) \right\vert \leq Be^{-2\pi a \left\vert \xi \right\vert} $. Then, there exists a holomorphic function $g\in \mathcal{F}_a$ such that $g(x) = f(x)$.
\end{lemma}
\begin{proof}
  Define the entire function
  \begin{align*}
    f_n(z) &= \int_{-n}^{n} \hat{f}\left( \xi \right)e^{2\pi i x \xi}\:d\xi.
  \end{align*}
  For $z\in \set{z\in \C | \left\vert \im(z) \right\vert < b}$, the function
  \begin{align*}
    g(z) &= \int_{-\infty}^{\infty} \hat{f}\left( \xi \right)e^{2\pi i z \xi}\:d\xi
  \end{align*}
  is well-defined, since the decay condition on $\hat{f}$ ensure the absolute convergence of the integral. By Fourier inversion, $g(x) = f(x)$ for all $x\in \R$. Finally, $g$ is holomorphic, as the sequence converges uniformly to $g$, as
  \begin{align*}
    \left\vert f_n(z) - g(z) \right\vert &\leq B\left( \int_{-\infty}^{-n} e^{-2\pi \left( a-b \right)\xi}\:d\xi + \int_{n}^{\infty} e^{-2\pi\left( a-b \right)\xi}\:d\xi \right)\\
                                         &= \frac{2Be^{-2\pi \left( a-b \right) n}}{2\pi \left( a-b \right)},
  \end{align*}
  which converges to $0$ uniformly as $n$ tends to infinity.
\end{proof}
\begin{theorem}[Paley--Wiener Theorem]
  Let $f\colon \R\rightarrow \R$ be continuous, integrable, and have integrable Fourier transform. Then, $\hat{f}$ is supported on an interval of the form $\left[ -M,M \right]$ for some $M > 0$ if and only if there exists an entire function $g\colon \C\rightarrow \C$ for which there exists some $A \geq 0$ such that $\left\vert g(z) \right\vert \leq Ae^{2\pi M \left\vert z \right\vert}$ and $g(x) = f(x)$ for all $x\in \R$.
\end{theorem}
\end{document}
