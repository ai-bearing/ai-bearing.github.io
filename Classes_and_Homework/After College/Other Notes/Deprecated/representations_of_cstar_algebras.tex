\documentclass[10pt]{mypackage}

% sans serif font:
%\usepackage{cmbright}
%\usepackage{sfmath}
%\usepackage{bbold} %better blackboard bold

%\usepackage{homework}
\usepackage{notes}
\usepackage{mlmodern}
\usepackage[ backend=bibtex,
style=alphabetic,
sorting=ynt ]{biblatex}
\addbibresource{all_references.bib}
\usepackage{parskip}

\fancyhf{}
\fancyhead[R]{Avinash Iyer}
\fancyhead[L]{Representations of $C^{\ast}$-Algebras}
\fancyfoot[C]{\thepage}

\setcounter{secnumdepth}{0}

\begin{document}
\RaggedRight

\section{Basics}%
\begin{definition}
  Let $A$ be a $C^{\ast}$-algebra. A \textit{representation} of $A$ is a $\ast$-homomorphism $\pi\colon A\rightarrow B\left( H \right)$ for some Hilbert space $H$. 
\end{definition}
\begin{definition}
  Two representations $\pi\colon A\rightarrow B\left( H_{\pi} \right)$ and $\rho\colon A\rightarrow B\left( H_{\rho} \right)$ are called unitarily equivalent if there is a unitary map $U\colon H_{\rho}\rightarrow H_{\pi}$ such that
  \begin{align*}
    \pi(a) &= U\rho(a)U^{\ast}
  \end{align*}
  for all $a\in A$.
\end{definition}
\begin{definition}
  If $\pi\colon A\rightarrow B\left( H_{\pi} \right)$ and $\rho\colon A\rightarrow B\left( H_{\rho} \right)$ be representations. Then, the formula
  \begin{align*}
    \pi\oplus\rho(a) \left( h,k \right) &\coloneq \left( \pi(a)h,\rho(a)k \right)
  \end{align*}
  defines the \textit{direct sum} of $\pi$ and $\rho$. If $\pi$ is unitarily equivalent to a direct sum $\rho_1\oplus \rho_2$, then we consider $\rho_1\oplus \rho_2$ to be a decomposition of $\pi$ in terms of the ``smaller'' representations.
\end{definition}
\begin{definition}
  A closed subspace $K$ of $H_{\pi}$ is \textit{invariant} under $\pi$ if $\pi(a)k\in K$ for all $a\in A$ and $k\in K$.
\end{definition}
Observe that if $K$ is an invariant subspace, then the orthogonal complement $K^{\perp}$ is also invariant. This follows from the fact that if $y\in K^{\perp}$, then
\begin{align*}
  \iprod{k}{\pi(a)y} &= \iprod{\pi(a)^{\ast}k}{y}\\
                     &= \iprod{\pi\left( a^{\ast} \right)k}{y}\\
                     &= 0
\end{align*}
for all $k\in K$.

Conversely, if $K$ is invariant, then we can recover $\pi = \pi|_{K} \oplus \pi|_{K^{\perp}}$, via the canonical unitary isomorphism $U\colon K\oplus K^{\perp}\rightarrow H_{\pi}$ given by $\left( k,y \right)\mapsto k + y$.
\begin{definition}
  A representation $\pi$ is \textit{irreducible} if there are no closed invariant subspaces apart from $\set{0}$ and $H_{\pi}$.
\end{definition}
\begin{lemma}
  A representation $\pi$ of a $C^{\ast}$-algebra $A$ is irreducible if and only if $\pi(A)' = \C I_{H}$, where $\pi(A)'$ denotes the commutant of $\pi(A)$.
\end{lemma}
\begin{proof}
  Suppose $V$ is a nontrivial invariant subspace for $\pi$. Then, the orthogonal projection $P_{V}$ commutes with every $\pi(A)$ and is not a scalar multiple of $I_{H}$.

  Now, suppose there is a non-scalar operator $T$ commuting with $\pi(A)$. Then, either the real or imaginary part of $T$ is a self-adjoint operator $S$ that commutes with $\pi(A)$. From the continuous functional calculus, since $\sigma(S)$ is not one point, there are some nonzero continuous $f,g\in C\left( \sigma(S) \right)$ such that $fg = 0$. Then, since $f(S),g(S)\in C^{\ast}\left( S \right)$, and $f(S),g(S)$ commute with $\pi(A)$, it follows that $ \overline{f(S)H} $ and $ \overline{g(S)H} $ are nonzero mutually orthogonal invariant subspaces, so $\pi$ is reducible.
\end{proof}
\begin{definition}
  If $\pi$ is a representation of the $C^{\ast}$-algebra $A$, then we call the subspace
  \begin{align*}
    \left[ \pi(A)H_{\pi} \right] &= \overline{\Span}\set{\pi(a)h | h\in H_{\pi}, a\in A}
  \end{align*}
  the \textit{essential subspace} of $H_{\pi}$. The representation $\pi$ is called \textit{nondegenerate} if the essential subspace $K$ is equal to $H_{\pi}$.
\end{definition}
Note that the representation $\pi$ being nondegenerate is equivalent to $\pi(1) = I_{H_{\pi}}$ if $A$ has an identity, or $\pi\left( e_i \right) \rightarrow I_{H_{\pi}}$ strongly for any approximate identity $\left( e_i \right)_{i\in I}$.

The essential subspace is always invariant, and $\pi$ is equivalent to $\pi|_{K}\oplus 0$. Generally, if $I$ is an ideal in $A$, then the subspace
\begin{align*}
  K &= \overline{\Span} \set{\pi(a)h | h\in H_{\pi},a\in I}
\end{align*}
is invariant, but $\pi$ is not zero on $K^{\perp}$ unless $I$ is an essential ideal.\footnote{An essential ideal is one that has nonzero intersection with any other closed ideal of $A$.} Any nondegenerate representation of an ideal $I$ extends canonically to a nondegenerate representation $\pi$ of $A$ on the same space.
\section{The Gelfand--Naimark--Segal Construction}%
The primary way we represent an abstract $C^{\ast}$-algebra is via the Gelfand--Naimark--Segal construction, which starts by using a special linear functional called a state to build a representation.
\begin{definition}
  A \textit{state} on a $C^{\ast}$-algebra $A$ is a linear functional $\varphi\colon A\rightarrow \C$ such that $\varphi(a) \geq 0$ whenever $a\geq 0$.
\end{definition}
\begin{lemma}
  If $\varphi$ is a positive linear functional, then $\varphi$ satisfies the Cauchy--Schwarz inequality,
  \begin{align*}
    \left\vert \varphi\left( b^{\ast}a \right) \right\vert^2 &\leq \varphi\left( a^{\ast}a \right)\varphi\left( b^{\ast}b \right).
  \end{align*}
\end{lemma}
\begin{proof}
  The pairing $\left[ a,b \right] = \varphi\left( b^{\ast}a \right)$ is a positive semidefinite sesquilinear form by the various operations. Therefore, it satisfies the Cauchy--Schwarz inequality.
\end{proof}
\begin{lemma}
  If $\varphi$ is a positive linear functional on a $C^{\ast}$-algebra $A$, then $\varphi$ is continuous.

  Furthermore, if $A$ is unital, then $\norm{\varphi} = \varphi(1)$, and if $A$ is non-unital with approximate unit $\left( e_{i} \right)_i$, then $\norm{\varphi} = \lim_{i\in I} \varphi\left( e_i \right)$.
\end{lemma}
\begin{proof}
  If $A$ is unital, then for any $0\leq a \leq 1$, we have $0\leq \varphi(a)\leq \varphi(1)$. For general $\norm{a}\leq 1$, we have
  \begin{align*}
    \left\vert \varphi(a) \right\vert^2 &= \left\vert \varphi\left( 1^{\ast}a \right) \right\vert^2\\
                                        &\leq \varphi\left( a^{\ast}a \right)\varphi(1)\\
                                        &\leq \varphi(1)^2,
  \end{align*}
  so that $\norm{\varphi} = \varphi(1)$.

  Now, if $A$ is not unital, we start by showing continuity on $A_{+}$. If it were not continuous, then we would have $a_n\geq 0$ with $\norm{a_n}\leq 2^{-n}$ and $\varphi\left( a_n \right) > 1$. If
  \begin{align*}
    a &= \sum_{n= 1}^{\infty}a_n,
  \end{align*}
  we have that $a$ is positive with $\norm{a}\leq 1$. Yet,
  \begin{align*}
    \varphi\left( a \right) &= \sum_{n=1}^{N} + \varphi\left( \sum_{n=N+1}^{\infty}a_n \right) > N
  \end{align*}
  for all $N\geq 1$, which cannot happen, so there is some constant $C$ such that $0\leq \varphi(a)\leq C\norm{a}$ for any $a\geq 0$. For general $a$, we may write
  \begin{align*}
    a &= a_1 - a_2 + ia_3 - ia_4
  \end{align*}
  with $0\leq a_i\leq \norm{a}$ for each $i$. Since $0\leq \varphi\left( a_i \right)\leq C\norm{a}$, we have
  \begin{align*}
    \left\vert \varphi\left( a \right) \right\vert^2 &= \left( \varphi\left( a_1 \right)-\varphi\left( a_2 \right) \right)^2 + \left( \varphi\left( a_3 \right) - \varphi\left( a_4 \right) \right)^2\\
                                                     &\leq 2C^2\norm{a}^2,
  \end{align*}
  so $\varphi$ is continuous with $\norm{\varphi}\leq C\sqrt{2}$.

  Let $\left( e_i \right)_i$ be an approximate identity for $A$. We have that
  \begin{align*}
    \varphi\left( e_j \right) &= \lim_{i\in I} \varphi\left( e_ie_je_i \right)\\
                              &\leq \liminf_{i\in I} \varphi\left( e_i^2 \right)\\
                              &\leq \liminf_{i\in I}\varphi\left( e_i \right),
  \end{align*}
  following from the fact that $0\leq e_i\leq 1$ for each $i$ and that $\varphi$ preserves order. Thus, we have
  \begin{align*}
    \limsup_{j\in I} \varphi\left( e_j \right) &\leq \liminf_{i\in I} \varphi\left( e_i \right),
  \end{align*}
  so we may define
  \begin{align*}
    M &\coloneq \lim_{i\in I}\varphi\left( e_i \right)
  \end{align*}
  with $M\leq \norm{\varphi}$ unambiguously. For any $a\in A$ with $\norm{a}\leq 1$, we may use the Cauchy--Schwarz inequality to find
  \begin{align*}
    \left\vert \varphi\left( a \right)^2 \right\vert &= \lim_{i\in I} \left\vert \varphi\left( e_i a \right) \right\vert^2\\
                                                     &\leq \limsup_{i\in I} \left\vert \varphi\left( a^{\ast}a \right) \right\vert\varphi\left( e_i^2 \right)\\
                                                     &\leq \norm{\varphi}M.
  \end{align*}
  Taking suprema, we find that $\norm{\varphi}\leq M$, so $\norm{\varphi} = M$.
\end{proof}
\nocite{blackadar_operator_algebras,morita_equivalence_cstar_algebras,davidson_functional_analysis}
\printbibliography 
\end{document}
