\documentclass[10pt]{mypackage}

\usepackage{mlmodern}
%\usepackage{newpxtext,eulerpx,eucal}
%\renewcommand*{\mathbb}[1]{\varmathbb{#1}}

%\usepackage{homework}
\usepackage{notes}

\usepackage[ backend=bibtex, style = alphabetic, sorting=ynt ]{biblatex}
\addbibresource{all_references.bib}

\usepackage{parskip}

\fancyhf{}
\fancyhead[R]{Avinash Iyer}
\fancyhead[L]{Von Neumann Algebras: Theory and Structure}
\fancyfoot[C]{\thepage}

\setcounter{secnumdepth}{0}

\begin{document}
\RaggedRight
\section{Basic Structure of von Neumann Algebras}%
We start by recalling some of the topologies on $B(H)$.
\begin{definition}
  Let $H$ be a Hilbert space, with $B(H)$ denoting the space of bounded operators on $H$.

  The \textit{strong operator topology}, or SOT, is the locally convex topology generated by the seminorms 
  \begin{align*}
    \set{\norm{Tv} | T\in B(H),v\in H}
  \end{align*}
  The \textit{weak operator topology}, or WOT, is the locally convex topology generated by the seminorms 
  \begin{align*}
    \set{ \left\vert \iprod{Tv}{w} \right\vert | T\in B(H),v,w\in H }
  \end{align*}
\end{definition}
\begin{theorem}
  Let $\phi\colon B(H)\rightarrow \C$ be a linear functional. The following are equivalent:
  \begin{enumerate}[(i)]
    \item there are $\xi_k,\eta_k\in H$ such that $\ds \phi(T) = \sum_{k=1}^{n} \iprod{T\xi_k}{\eta_k}$;
    \item $\phi$ is WOT-continuous;
    \item $\phi$ is SOT-continuous.
  \end{enumerate}
\end{theorem}
\begin{proof}
  The directions (i) implies (ii) implies (iii) are pretty much by definition. To see (iii) implies (i), we have $\xi_1,\dots,\xi_n$ such that, for all $T\in B(H)$, $\max\norm{T\xi_k}\leq 1$ implies $\phi(T)\leq 1$. Then, we have
  \begin{align*}
    \left\vert \phi(T) \right\vert &\leq \left( \sum_{k=1}^{n}\norm{T\xi_k}^2 \right)^{1/2}.
  \end{align*}
  Let
  \begin{align*}
    H^{(n)} &\coloneq \bigoplus_{k=1}^{n}H\\
    T^{(n)} &\coloneq \operatorname{diag}\left( T,\dots,T \right)\in B\left( H^{(n)} \right),
  \end{align*}
  and let $\xi = \left( \xi_1,\dots,\xi_n \right)\in H^{(n)}$. We see then that the linear functional $\psi\colon H\rightarrow \C$ given by
  \begin{align*}
    \psi\left( T^{(n)}\xi \right) &= \phi(T)
  \end{align*}
  defines a linear functional on the closed subspace of $K$ spanned by the vectors 
  \begin{align*}
    \set{T^{(n)}\xi | T\in B(H)},
  \end{align*}
  and has
  \begin{align*}
    \left\vert \psi\left( T^{(n)}\xi \right) \right\vert &\leq \norm{T^{(n)}\xi},
  \end{align*}
  so by the Riesz Representation Theorem for Hilbert Spaces, it follows there is $\eta = \left( \eta_1,\dots,\eta_n \right)$ such that
  \begin{align*}
    \phi(x) &= \iprod{T^{(n)}\xi}{\eta}\\
            &= \sum_{k=1}^{n} \iprod{T\xi_k}{\eta_k}.
  \end{align*}
\end{proof}
\begin{corollary}
  Every SOT-closed convex subset of $B(H)$ is WOT-closed.
\end{corollary}
\begin{proof}
  The closed convex subsets of a locally convex topological vector space are determined by the continuous linear functionals, as follows from an application of the Hahn--Banach separation.
\end{proof}
\begin{theorem}
  The unit ball of $B(H)$ is WOT-compact.
\end{theorem}
\begin{proof}
  Let $ \overline{\D} $ denote the closed unit disk of $\C$, and consider the set
  \begin{align*}
    K &= \prod_{x,y\in B_H} \overline{\D}.
  \end{align*}
  This space is compact by Tychonoff's theorem. Define the embedding $\phi\colon B_{B(H)}\rightarrow K$ given by
  \begin{align*}
    \phi(T) &= \left( \iprod{Tx}{y} \right)_{x,y}.
  \end{align*}
  By Cauchy--Schwarz, we have
  \begin{align*}
    \left\vert \iprod{Tx}{y} \right\vert &\leq \norm{T}_{\op}\norm{x}\norm{y}\\
                                         &\leq 1,
  \end{align*}
  so $\phi$ is well-defined. We see that $\phi$ is WOT-continuous by definition and injective, so we only need to show that $\img\left( \phi \right)$ is closed. Let $\left( T_i \right)_i\subseteq B_{B(H)}$ be a net with
  \begin{align*}
    \lim_{i\in I} \left( \iprod{T_ix}{y} \right)_{x,y} &= \left( z_{x,y} \right)_{x,y}.
  \end{align*}
  We have that $\left( z_{x,y} \right)_{x,y}\in K$ since $K$ is compact, and since the product topology is the topology of pointwise convergence, we have
  \begin{align*}
    \lim_{i\in I} \iprod{T_ix}{y} &= z_{x,y}
  \end{align*}
  defines a sesquilinear form $F\left( x,y \right)$. This means we may find $T\in B_{B(H)}$ such that $F\left( x,y \right) = \iprod{Tx}{y}$, and so $\left( T_i \right)_i\rightarrow T$ in WOT.
\end{proof}
\begin{definition}
  A \textit{partial isometry} is an operator $W\in B(H)$ such that for any $h\in \left( \ker(W) \right)^{\perp}$, we have $\norm{Wh} = \norm{h}$. The space $\left( \ker\left( W \right) \right)^{\perp}$ is called the \textit{initial space} of $W$, and the space $\img(W)$ is called the final space of $W$.
\end{definition}
\begin{proposition}
  If $W\in B(H)$, the following are equivalent:
  \begin{enumerate}[(i)]
    \item $W$ is a partial isometry;
    \item $W^{\ast}$ is a partial isometry;
    \item $W^{\ast}W$ is a projection (onto the initial space of $W$);
    \item $WW^{\ast}$ is a projection (onto the final space of $W$);
    \item $WW^{\ast}W = W$;
    \item $W^{\ast}WW^{\ast} = W^{\ast}$.
  \end{enumerate}
\end{proposition}
\begin{proof}
  The equivalence between (v) and (vi) follows from taking adjoints.

  Let $W$ be a partial isometry, meaning that $W$ is an isometry from $\left( \ker\left( W \right) \right)^{\perp}$ to $\img(W)$. Since $\img(W)$ is dense in $\ker\left( W^{\ast} \right)^{\perp}$, it follows that we only need to show that $W^{\ast}$ is an isometry on $\img(W)$. Let $k\in \img(W)$, so there is $h\in \left( \ker\left( W \right) \right)^{\perp}$ such that $Wh = k$. Then, we have
  \begin{align*}
    \iprod{Wh}{Wh} &= \iprod{h}{h}
      \intertext{so}
    \iprod{W^{\ast}Wh-h}{h} &= 0,
  \end{align*}
  meaning that $W^{\ast}W-I$ is zero on $\left( \ker\left( W \right) \right)^{\perp}$, so we have
  \begin{align*}
    \norm{W^{\ast}k} &= \norm{W^{\ast}Wh}\\
                     &= \norm{h}\\
                     &= \norm{Wh}\\
                     &= \norm{k},
  \end{align*}
  meaning $W^{\ast}$ is a partial isometry.

  By taking adjoints, we see that (i) and (ii) are equivalent. Let $x\in H$ have the decomposition $x = y + z$ where $y\in \ker\left( W \right)$ and $z\in \left( \ker\left( W \right) \right)^{\perp}$. We will show that $W^{\ast}Wx = z$. Observe that $Wx = Wz$, meaning that
  \begin{align*}
    \iprod{z-W^{\ast}Wx}{z} &= \iprod{z-W^{\ast}Wz}{z}\\
                            &= \iprod{z}{z} - \iprod{W^{\ast}Wz}{z}\\
                            &= \iprod{z}{z} - \iprod{Wz}{Wz}\\
                            &= 0,
  \end{align*}
  since $\norm{Wz} = \norm{z}$ by definition. In particular, following from the polarization identity, this means that for all $v\in H$, we have $ \iprod{z-W^{\ast}Wx}{v} = 0 $, so that $z = W^{\ast}Wx$. This shows that (i) implies (iii). By taking adjoints, we see that (ii) implies that $WW^{\ast}$ is a projection onto the initial space of $W^{\ast}$, which is equal to the final space of $W$.
\end{proof}
\subsection{Double Commutant Theorem}%
\begin{definition}
  Let $M\subseteq B(H)$. We define the \textit{commutant} to be
  \begin{align*}
    M' &\coloneq \set{S\in B(H) | TS = ST\text{ for all }T\in M}.
  \end{align*}
  The double commutant of $M$ is denoted $M''$, and has $M\subseteq M''$.
\end{definition}
We see that $M'$ is a WOT-closed subalgebra, and if $M'$ is self-adjoint, then $M'$ is a $C^{\ast}$-algebra. Additionally, if $M_1\subseteq M_2$, then $M_1'\supseteq M_1'$.
\begin{theorem}[Double Commutant Theorem]
  Let $M$ be a unital $C^{\ast}$-subalgebra of $B(H)$. The following are equivalent:
  \begin{enumerate}[(i)]
    \item $M = M''$;
    \item $M$ is WOT-closed;
    \item $M$ is SOT-closed.
  \end{enumerate}
\end{theorem}
\begin{proof}
  The implications (i) implies (ii) follows from the discussion above, and (ii) if and only (iii) follow from the definitions (as subalgebras are convex). We focus on showing that (iii) implies (i).

  For a fixed $\xi\in H$, let $P$ be the projection onto the closure of the subspace $\set{T\xi | T\in M}$. We see that $P\xi = \xi$, since $I_{H} \in M$. Additionally, $PTP = TP$ for each $T\in M$, so $P\in M'$. Letting $V\in M''$, we have that $PV = VP$, so $V\xi \in PH$. In particular, for each $\ve > 0$, there is $S\in M$ such that $\norm{\left( V-S \right)\xi} < \ve$.

  Let $\xi_1,\dots,\xi_n\in H$, and set $\xi = \left( \xi_1,\dots,\xi_n \right)$ in $H^{(n)}$. Letting $\rho\colon B(H)\hookrightarrow B\left(H^{(n)}\right)$ be the embedding defined by
  \begin{align*}
    T &\mapsto T^{(n)},
  \end{align*}
  we see that
  \begin{align*}
    \rho\left( M \right)' &= \set{S\in B(K)| S_{ij}\in M'}.
  \end{align*}
  Therefore, we have that $\rho(V)\in \rho(M)''$, meaning that using the same process as above in the amplified algebra, we have
  \begin{align*}
    \sum_{k=1}^{n} \norm{\left( V-T \right)\xi_k}^2 &= \norm{\left( \rho(V)-\rho(T) \right)\xi}^2\\
                                                    &< \ve^2,
  \end{align*}
  meaning that we can approximate $V$ in SOT from $M$, so $V\in M$.
\end{proof}
\begin{definition}
  A \textit{von Neumann algebra} is a unital SOT-closed (or WOT-closed) $C^{\ast}$-subalgebra of $B(H)$.
\end{definition}
The double commutant theorem says that $M = M''$ is a characterization of a von Neumann algebra.

Observe that if $T\in M$ is a normal operator in a von Neumann algebra $M$, then if $E$ denotes the spectral measure for $T$, and $S\in M'$, then $TS = ST$, so by Fuglede's Theorem, $T^{\ast}S = ST^{\ast}$, meaning that $Sf(T) = f(T)S$ for all $f\in B_{\infty}\left(\sigma(T)\right)$. In particular, this means that $E(S)\in M'' = M$. Since the closed linear span of the characteristic functions $\chi_{S}$ is equal to $B_{\infty}\left(\sigma(T)\right)$, it follows that, if $M$ is a von Neumann algebra, then $M$ is the (norm)-closed linear span of all of its projections.

To see this another way, let $a\in M_{\sa}$, and consider a partition $-\norm{a} = t_0 < t_1 < \cdots < t_n = \norm{a}$, where $t_{j+1}-t_j < \ve$ for each $j = 0,\dots,n-1$, and define projections
\begin{align*}
  P_i &= \chi_{\left[ t_{j-1},t_j \right)}
\end{align*}
for $j=1,\dots,n-1$, and $P_n = \chi_{\left[ t_{n-1},t_n \right]}$. Then, we necessarily have
\begin{align*}
  \norm{a - \sum_{j=1}^{n}t_jP_j}_{\op} < \ve,
\end{align*}
so every self-adjoint operator is in the norm-closed linear span of the projections of $M$. Since every element of $M$ can be written as a decomposition of self-adjoint operators, it follows that $M$ is the norm-closed linear span of its projections.
\section{Abelian von Neumann Algebras}%

\section{Kaplansky Density Theorem and Pedersen's Up-Down Theorem}%
\section{Comparison Theory of Projections}%

\nocite{pedersen_cstar_algebras_automorphism_groups,murphy_cstar_algebras_and_operator_theory,conway_operator_theory}
\end{document}
