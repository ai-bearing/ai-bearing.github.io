\documentclass[10pt]{mypackage}

\usepackage{mlmodern}
%\usepackage{newpxtext,eulerpx,eucal}
%\renewcommand*{\mathbb}[1]{\varmathbb{#1}}

%\usepackage{homework}
\usepackage{notes}

\usepackage[ backend=bibtex, style = alphabetic, sorting=ynt ]{biblatex}
\addbibresource{all_references.bib}

\usepackage{parskip}

\fancyhf{}
\fancyhead[R]{Avinash Iyer}
\fancyhead[L]{Spectral Theory for Bounded Normal Operators}
\fancyfoot[C]{\thepage}

\setcounter{secnumdepth}{0}

\begin{document}
\RaggedRight
We recall from linear algebra that a linear operator $T\colon V\rightarrow V$ is called diagonalizable if there is an orthonormal basis $\set{e_j}_{j\in J}$ and a bounded collection of elements $\set{\lambda_j}_{j\in J}$ such that for every $x\in V$, we have
\begin{align*}
  Tx &= \sum_{j\in J}\lambda_j \iprod{x}{e_j}e_j.
\end{align*}
When $V$ is a Hilbert space, there are a variety of generalizations. It will be useful to review the \href{https://ai-bearing.github.io/Classes_and_Homework/Other Notes/compact_and_fredholm_operators.pdf}{basic properties} of compact and Fredholm operators.
\section{Spectral Theory for Compact Normal Operators}%
The first, most basic version of the spectral theorem is the one for compact normal operators. We recall the different types of spectra.
\begin{definition}
  Let $T\in B(X)$, where $X$ is a Banach space.
  \begin{enumerate}[(i)]
    \item The \textit{point spectrum} of $T$ is the set
      \begin{align*}
        \sigma_{p}(T) &= \set{\lambda\in \C | \ker\left( T-\lambda I \right)\neq \set{0}},
      \end{align*}
      which are the eigenvalues of $T$.
    \item The \textit{approximate point spectrum} of $T$ is the set
      \begin{align*}
        \pi(T) &= \set{\lambda\in \C | T-\lambda I\text{ is not bounded below}}.
      \end{align*}
    \item The \textit{compression spectrum} of $T$ is
      \begin{align*}
        \gamma(T) &= \set{\lambda\in \C | \img\left( T-\lambda I \right)\text{ is not dense in }X}.
      \end{align*}
  \end{enumerate}
\end{definition}
There is a useful characterization of compact operators as follows.
\begin{lemma}
  The following for $T\in B(H)$ are equivalent:
  \begin{enumerate}[(i)]
    \item $T$ is compact;
    \item $T|_{B_H}$ is a weak--norm continuous function from $B_H$ into $H$.
  \end{enumerate}
\end{lemma}
\begin{proof}
  Suppose $T$ is compact. Then, if $\left( x_{i} \right)_{i\in I}$ is a weakly convergent net in $B_H$ with limit $x$, and $\ve > 0$, there is some finite-rank $S\in F(H)$ with $\norm{S-T}_{\op} < \ve/3$. We have
  \begin{align*}
    \norm{Tx_i - Tx} &\leq \norm{Tx_i - Sx_i} + \norm{Sx_i - Sx} + \norm{Sx - Tx}\\
                     &\leq 2\norm{T-S}_{\op} + \norm{Sx_i - Sx}.
  \end{align*}
  Every operator in $B(H)$ is weak--weak continuous, and since $\img(S)$ is finite-dimensional, all norms coincide, so that $Sx_i \rightarrow Sx$ in norm, giving that $\norm{Tx_i -Tx} < \ve/3$ for sufficiently large $i$. Thus, $T$ is weak--norm continuous.

  If $T$ is weak--norm continuous, then since $B_H$ is weakly compact, it follows that $T\left(B_H\right)$ is compact by continuity.
\end{proof}
\begin{lemma}
  A diagonalizable operator $T$ in $B(H)$ is compact if and only if its eigenvalues $\set{\lambda_j | j\in J}$ corresponding to an orthonormal basis $\set{e_j | j\in J}$ belongs to $c_0(J)$.
\end{lemma}
\begin{proof}
  Since $T$ is diagonalizable, we have
  \begin{align*}
    Tx &= \sum_{j\in J}\lambda_j \iprod{x}{e_j}e_j.
  \end{align*}
  If $T\in K(H)$, and $\ve > 0$, then we set
  \begin{align*}
    J_{\ve} &= \set{j\in J | \left\vert \lambda_j \right\vert \geq \ve}.
  \end{align*}
  If $J_{\ve}$ is infinite, then since $ \iprod{x}{e_j} \rightarrow 0 $ by Parseval's identity, we have that the net $\left( e_j \right)_{j\in J_{\ve}}$ converges weakly to zero. Yet, since $\norm{Te_j} = \left\vert \lambda_j \right\vert \geq \ve$ for any $j\in J_{\ve}$, this contradicts the fact that $T$ is weak--norm continuous. Thus, $J_{\ve}$ is finite for any $\ve > 0$, so $\left( \lambda_j \right)_{j\in J}$ vanishes at infinity.

  Now, if $J_{\ve}$ is finite for every $\ve > 0$, we may define $T_{\ve}\in F(H)$ by
  \begin{align*}
    T_{\ve} &= \sum_{j\in J_{\ve}}\lambda_j \iprod{\cdot}{e_j}e_j,
  \end{align*}
  and
  \begin{align*}
    \norm{\left( T-T_{\ve} \right)x}^2 &= \norm{\sum_{j\notin J_{\ve}} \lambda_j \iprod{x}{e_j}e_j}^2\\
                                       &= \sum_{j\in J_{\ve}} \left\vert \lambda_j \right\vert^2 \left\vert \iprod{x}{e_j} \right\vert^2\\
                                       &\leq \ve^2\norm{x}^2,
  \end{align*}
  so $\norm{T-T_{\ve}} \leq \ve$, meaning that $T\in \overline{F(H)} = K(H)$.
\end{proof}
Note that by some basic computations, if $T$ is diagonalizable, then we have
\begin{align*}
  T^{\ast} &= \sum_{j\in J} \overline{\lambda_j} \iprod{\cdot}{e_j}e_j\\
  T^{\ast}T &= \sum_{j\in J} \left\vert \lambda_j \right\vert^2 \iprod{\cdot}{e_j}e_j\\
            &= TT^{\ast}.
\end{align*}
Thus, in particular, we have that every diagonalizable operator is normal.
\begin{theorem}
  An operator $T\in B(H)$ is diagonalizable with eigenvalues vanishing at infinity if and only if it is a compact normal operator.
\end{theorem}
\begin{proof}
  Now we only need to show that every compact normal operator is diagonalizable. Since $T$ is compact, we know that the spectrum of $T$ consists of $0$ and a countable set of isolated points, and since $T$ is normal, its spectral radius is equal to the operator norm, meaning that there is some $\lambda$ such that $\left\vert \lambda \right\vert = \norm{T}_{\op}$. In particular, there is an eigenvector for $T$.

  Let $\mathcal{Z}$ be the family of orthonormal systems of eigenvectors of $T$, ordered by inclusion. Since we have established that this family is nonempty, and the union provides an upper bound for any chain in $\mathcal{Z}$, there is some maximal orthonormal system $\set{e_j}_{j\in J}$ with corresponding eigenvalues $\set{\lambda_j}_{j\in J}$. We let $P$ be the projection onto the closed subspace spanned by the $e_j$. For each $x\in H$, we have
  \begin{align*}
    TPx &= T\left( \sum_{j\in J} \iprod{x}{e_j}e_J \right)\\
        &= \sum_{j\in J} \lambda_j \iprod{x}{e_j}e_j\\
        &= \sum_{j\in J} \iprod{x}{ \overline{\lambda_j}e_j }e_j\\
        &= \sum_{j\in J} \iprod{x}{T^{\ast}e_j}e_j\\
        &= \sum_{j\in J} \iprod{Tx}{e_j}e_j\\
        &= PTx.
  \end{align*}
  Thus, the operator $\left( I-P \right)T$ is normal, and is also compact. If $P\neq I$, then either $\left( I-P \right)T = 0$, and every unit vector in $\left( I-P \right)(H)$ is an eigenvector for $T$ (contradicting maximality), or else $\left( I-P \right)T\neq 0$, in which case there is $e_0\in \left( I-P \right)(H)$ with $Te_0 = \lambda e_0$ and $\left\vert \lambda \right\vert = \norm{\left( I-P \right)T}_{\op}$, which once again contradicts maximality.

  Thus, $P = I$, and we are done.
\end{proof}
\section{Spectral Theory for Normal Operators}%
We now generalize from the special case of compact operators. Here, we cannot use the convenient properties of compact operators with respect to finite dimensionality/codimensionality.

First, we notice that if $T\in B\left( H \right)$ is a normal operator, then $C^{\ast}\left( T \right)$, the $C^{\ast}$-algebra generated by $T$, is abelian, so from \href{https://ai-bearing.github.io/Classes_and_Homework/After\%20College/Other\%20Notes/continuous_functional_calculus.pdf}{the Gelfand isomorphism}, we have that $C^{\ast}\left( T \right)\cong C\left(\sigma(T)\right)$ are isometrically $\ast$-isomorphic.

We will generalize this in a moment, but first we will apply the continuous functional calculus to show an important commutation relation. In $\M_n\left(\C\right)$, we know that an operator $S$ commutes with a normal operator $T$ if and only if all the eigenspaces for $T$ are invariant under $S$; since $T$ and $T^{\ast}$ commute, it then follows that $S$ commutes with $T^{\ast}$.

It turns out that this generalizes to infinite-dimensional spaces, but the proof requires the use of the continuous functional calculus.
\begin{proposition}[Fuglede's Theorem]
  If $S$ and $T$ are operators in $B(H)$, and $T$ is normal, then $ST = TS$ implies $ST^{\ast} = T^{\ast}S$.
\end{proposition}
\begin{proof}
  Define
  \begin{align*}
    e^{\lambda T} &= \sum_{n=0}^{\infty} \frac{\left( \lambda T \right)^{n}}{n!}.
  \end{align*}
  This is an element of $C^{\ast}(T)$ by the continuous functional calculus, and similarly, $e^{\lambda T^{\ast}}\in C^{\ast}\left( T \right)$, with
  \begin{align*}
    e^{\lambda T^{\ast}} &= e^{\lambda T^{\ast} - \overline{\lambda}T} e^{ \overline{\lambda}T }.
  \end{align*}
  There is some self-adjoint operator $R$ such that $\lambda T^{\ast} - \overline{\lambda}T = iR$, meaning that
  \begin{align*}
    U(\lambda) &= e^{\lambda T^{\ast}-  \overline{\lambda}T}
  \end{align*}
  is a unitary operator in $C^{\ast}(T)$ with $U(\lambda)^{\ast} = U(-\lambda)$.

  It follows from the expression for $e^{\lambda T}$ that $S$ commutes with $e^{\lambda T}$ for every $\lambda$, so that
  \begin{align*}
    e^{-\lambda T^{\ast}}S e^{\lambda T^{\ast}} &= U(-\lambda)SU(\lambda),
  \end{align*}
  with the operators uniformly bounded in norm by $\norm{S}$.

  Fixing $x,y\in H$, define $f\colon \C\rightarrow \C$ by
  \begin{align*}
    f(\lambda) &= \iprod{e^{-\lambda T^{\ast}}S e^{\lambda T^{\ast}}x}{y}.
  \end{align*}
  It follows that $f$ is an entire function with $\left\vert f(\lambda) \right\vert\leq \norm{S}$ for all $\lambda$, so that
  \begin{align*}
    \iprod{e^{-\lambda T^{\ast}}S e^{\lambda T^{\ast}}x}{y} - \iprod{Sx}{y} &= f(\lambda)-f(0)\\
                                                                            &= 0,
  \end{align*}
  so that
  \begin{align*}
    e^{-\lambda T^{\ast}}Se^{\lambda T^{\ast}}- S &= 0.
  \end{align*}
  Thus, $ST^{\ast} - T^{\ast}S = 0$.
\end{proof}
\subsection{Spectral Theorem I}%
In order to prove the spectral theorem for normal operators, we use the concept of a spectral measure.
\begin{definition}
  Let $\Omega$ be a compact Hausdorff space, and $H$ a Hilbert space. A \textit{spectral measure} $E$ relative to $\left( \omega,H \right)$ is a map $E$ from the Borel $\sigma$-algebra of $\Omega$ to the set of projections on $B(H)$ satisfying
  \begin{enumerate}[(i)]
    \item $E(\emptyset) = 0$, $E(\Omega) = I_H$;
    \item $E\left(S_1\cap S_2\right) = E\left( S_1 \right)E\left( S_2 \right)$;
    \item for all $x,y\in H$, the map $E_{x,y}\colon S\rightarrow \iprod{E(S)x}{y}$ is a regular complex Borel measure on $\Omega$.
  \end{enumerate}
\end{definition}
We will let $B_{\infty}(\Omega)$ be the set of bounded Borel functions on $\Omega$, and $M(\Omega)$ the space of regular Borel complex measures with the total variation norm.
\begin{example}
  Let $\Omega$ be a compact Hausdorff space, $\mu$ a positive regular Borel measure on $\Omega$. Let $M_{\varphi}\in B\left( L_2\left( \Omega,\mu \right) \right)$ be defined by
  \begin{align*}
    M_{\varphi}f &= \varphi f.
  \end{align*}
  We observe that
  \begin{align*}
    \norm{M_{\varphi}f}^2 &= \int_{}^{} \left\vert \varphi f \right\vert^2\:d\mu\\
                          &\leq \norm{\varphi}_{L_{\infty}} \int_{}^{} \left\vert f \right\vert^2\:d\mu.
  \end{align*}
  In particular, this means that $\norm{M_{\varphi}}_{\op}\leq \norm{\varphi}_{L_{\infty}}$.

  The map $L_{\infty}\left( \Omega,\mu \right)\rightarrow B\left( L_2\left( \Omega,\mu \right) \right)$ is thus a $\ast$-homomorphism of $C^{\ast}$-algebras, where $M_{\varphi}^{\ast} = M_{ \overline{\varphi} }$.

  In fact, since this map is injective, it is in fact an isometric $\ast$-homomorphism of $C^{\ast}$-algebras, following from the continuous functional calculus.
\end{example}
\begin{lemma}
  Let $\Omega$ be a compact Hausdorff space, $H$ a Hilbert space. Let $\mu_{x,y}\in M(\Omega)$ for each $x,y\in H$. Suppose that for each Borel set $S$ in $\Omega$, the function $\pi_S\colon H\times H\rightarrow \C$ given by $\left( x,y \right)\mapsto \mu_{x,y}\left( S \right)$ is a sesquilinear form.

  Then, for any $f\in B_{\infty}(\Omega)$, the function
  \begin{align*}
    \pi_f\colon H\times H&\rightarrow \C\\
    \left( x,y \right) &\mapsto \int_{}^{} f\:d\mu_{x,y}
  \end{align*}
  is a sesquilinear form.
\end{lemma}
\begin{proof}
  This is a standard boostrapping argument. We start by letting $f$ be a simple function, so we may write
  \begin{align*}
    f &= \sum_{j=1}^{n}\lambda_j\1_{S_j}
  \end{align*}
  for pairwise disjoint Borel subsets $S_1,\dots,S_j$ of $\Omega$ and complex numbers $\lambda_1,\dots,\lambda_n$. Then,
  \begin{align*}
    \int_{}^{} f\:d\mu_{x,y} &= \sum_{j=1}^{n} \lambda_j\mu_{x,y}\left( S_j \right),
  \end{align*}
  and since the (bounded) sesquilinear forms on $H$ are in one to one correspondence with $B(H)$, it follows that the case for simple functions follows.

  If $f\in B_{\infty}(\Omega)$ is arbitrary, then there is a sequence $\left( f_n \right)_n\rightarrow f$ of simple functions converging in the uniform norm. We observe that
  \begin{align*}
    \int_{}^{} \left\vert f_n-f \right\vert\:d\left\vert \mu_{x,y} \right\vert &\leq \norm{f_n-f}_{L_{\infty}} \left\vert \mu_{x,y} \right\vert(\Omega),
  \end{align*}
  so we may exchange limit and integral by dominated convergence, giving
  \begin{align*}
    \int_{}^{} f\:d\mu_{x,y} &= \lim_{n\rightarrow\infty} \int_{}^{} f_n\:d\mu_{x,y}
  \end{align*}
  for every $x,y\in H$. Thus, $\pi_f$ is a sesquilinear form on $H$.
\end{proof}
\begin{theorem}
  Let $\Omega$ be a compact Hausdorff space, $H$ a Hilbert space, and $E$ a spectral measure on $\left( \Omega,H \right)$. Then, for any $f\in B_{\infty}(\Omega)$, the map $\pi_f\colon H\times H\rightarrow \C$ given by
  \begin{align*}
    \left( x,y \right) &\mapsto \int_{}^{} f\:dE_{x,y}
  \end{align*}
  is a bounded sesquilinear form, with $\norm{\pi_f}\leq \norm{f}_{L_{\infty}}$.
\end{theorem}
\begin{proof}
  The previous lemma shows that $\pi_f$ is a sesquilinear form. We only need to show that $\norm{\pi_f}\leq \norm{f}_{L_{\infty}}$. Let $\Omega = S_1\cup\cdots\cup S_n$, with $S_1,\dots,S_n$ pairwise disjoint Borel sets. Then,
  \begin{align*}
    \sum_{j=1}^{n} \left\vert \iprod{E\left( S_j \right)x}{y} \right\vert &= \sum_{j=1}^{n}\left\vert \iprod{E\left( S_j \right)x}{E\left( S_j \right)y} \right\vert\\
                                                                          &\leq \left( \sum_{j=1}^{n}\norm{E\left( S_j \right)x}^2 \right)^{1/2}\left( \sum_{j=1}^{n} \norm{E\left( S_j \right)y}^2 \right)^{1/2}\\
                                                                          &= \norm{E(\Omega)x}\norm{E(\omega)y}\\
                                                                          &= \norm{x}\norm{y}.
  \end{align*}
  Thus, $\norm{E_{x,y}}\leq \norm{x}\norm{y}$, so
  \begin{align*}
    \left\vert \int_{}^{} f\:dE_{x,y} \right\vert &\leq \norm{f}_{L_{\infty}}\norm{E_{x,y}}\\
                                                  &\leq \norm{f}_{L_{\infty}}\norm{x}\norm{y}.
  \end{align*}
  Thus, $\norm{\pi_f}\leq \norm{f}_{L_{\infty}}$.
\end{proof}
Thus, paired with the correspondence of sesquilinear forms and bounded operators on a Hilbert space, we obtain the following result.
\begin{theorem}
  Let $\Omega$ be a compact Hausdorff space, $H$ a Hilbert space, and $E$ a spectral measure on $\left( \Omega,H \right)$. Then, for each $f\in B_{\infty}(\Omega)$, there is a unique bounded operator $T$ on $H$ such that
  \begin{align*}
    \iprod{Tx}{y} &= \int_{}^{} f\:dE_{x,y}.
  \end{align*}
\end{theorem}
We will define the \textit{integral} of $f\in B_{\infty}(\Omega)$ to be the (unique) operator such that for all $x,y\in H$,
\begin{align*}
  \iprod{ \left( \int_{}^{} f\:dE \right)x }{y} &= \int_{}^{} f\:dE_{x,y}.
\end{align*}
\begin{proposition}
  If $E$ is a spectral measure for $\left( \Omega,H \right)$, and we define
  \begin{align*}
    \rho\colon B_{\infty}(\Omega) &\rightarrow B(H)\\
    f &\mapsto \int_{}^{} f\:dE,
  \end{align*}
  then $\rho$ is a representation for the $C^{\ast}$-algebra $B_{\infty}(\Omega)$. That is, $\rho$ is a unital $\ast$-homomorphism.
\end{proposition}
\begin{proof}
  Linearity follows from a bootstrapping argument, and boundedness from the definition of the sesquilinear form.

  Thus, we only need to show multiplicativity. Similarly from bootstrapping, we only need to show the case when $f$ and $g$ are simple. Suppose $f = \1_{S}$ and $g = \1_{S'}$. Then,
  \begin{align*}
    \rho\left( fg \right) &= \int_{}^{} \1_{S}\1_{S'}\:dE\\
                             &= E\left( S\cap S' \right)\\
                             &= E(S)E(S')\\
                             &= \left( \int_{}^{} \1_{S}\:dE \right)\left( \int_{}^{} \1_{S'}\:dE \right)\\
                             &= \rho(f)\rho(g)
  \end{align*}
  and similarly projections are self-adjoint.
\end{proof}
Now, we've elucidated a lot of properties of spectral measures, but we still have not answered the question of their existence. This is the spectral theorem.
\begin{theorem}[Spectral Theorem for Bounded Normal Operators]
  Let $\Omega$ be a compact Hausdorff space, $H$ a Hilbert space, and let $\varphi\colon C(\Omega)\rightarrow B(H)$ be a unital $\ast$-homomorphism. Then, there is a unique spectral measure $E$ with respect to $\left( \Omega,H \right)$ such that
  \begin{align*}
    \varphi(f) &= \int_{}^{} f\:dE
  \end{align*}
  for all $f\in C(\Omega)$. Moreover, if $T\in B(H)$, then $T$ commutes with $\varphi(f)$ for all $f\in C(\Omega)$ if and only if $T$ commutes with $E(S)$ for all Borel $S\subseteq \Omega$.
\end{theorem}
\begin{proof}
  For any $x,y\in H$, the function $\tau_{x,y}\colon C(\Omega)\rightarrow \C$ given by
  \begin{align*}
    f &\mapsto \iprod{\varphi(f)x}{y}
  \end{align*}
  is linear with $\norm{\tau_{x,y}}_{\op}\leq \norm{x}\norm{y}$. From the Riesz Representation Theorem, there is a unique measure $\mu_{x,y}\in M(\Omega)$ such that 
  \begin{align*}
    \tau_{x,y}(f) &= \int_{}^{} f\:d\mu_{x,y}
  \end{align*}
  for all $f\in C(\Omega)$. We also have that $\norm{\mu_{x,y}} = \norm{\tau_{x,y}}_{\op}$. The function
  \begin{align*}
    \left( x,y \right) &\mapsto \iprod{\varphi(f)x}{y}
  \end{align*}
  is a sesquilinear map from $H$ to $M(\Omega)$ such that $x\mapsto \mu_{x,y}$ is linear and $y\mapsto \mu_{x,y}$ is conjugate-linear. Thus, for all $f\in B_{\infty}(\Omega)$, the map
  \begin{align*}
    \left( x,y \right) &\mapsto \int_{}^{} f\:d\mu_{x,y}
  \end{align*}
  is a sesquilinear form, with
  \begin{align*}
    \left\vert \int_{}^{} f\:d\mu_{x,y} \right\vert &\leq \norm{f}_{L_{\infty}}\norm{\mu_{x,y}}\\
                                                    &\leq \norm{f}_{L_{\infty}}\norm{x}\norm{y},
  \end{align*}
  so there is a unique bounded operator, $\psi(f)\in B(H)$ such that
  \begin{align*}
    \iprod{\psi(f)x}{y} &= \int_{}^{} f\:d\mu_{x,y}
  \end{align*}
  for all $x,y\in H$. If $f\in C(\Omega)$, then we have that
  \begin{align*}
    \iprod{\psi(f)x}{y} &= \int_{}^{} f\:d\mu_{x,y}\\
                        &= \tau_{x,y}(f)\\
                        &= \iprod{\varphi(f)x}{y},
  \end{align*}
  so $\psi(f) = \varphi(f)$.

  We now show that $\psi$ is a $\ast$-homomorphism. If $f\in C(\Omega)$ with $ \overline{f} = f $, then $\varphi(f)$ is self-adjoint, meaning that
  \begin{align*}
    \int_{}^{} f\:d\mu_{x,x} &= \iprod{\varphi(f)x}{x}
  \end{align*}
  is a real number, so $\mu_{x,x}$ is a real measure. Thus, if $f\in B_{\infty}(\Omega)$ is arbitrary, dominated convergence gives that
  \begin{align*}
    \iprod{\psi(f)x}{x} &= \int_{}^{} f\:d\mu_{x,x}
  \end{align*}
  is real. Thus, $\psi(f)$ is self-adjoint, so $\psi$ preserves involutions.

  If $f\in B_{\infty}$ and $x\in H$, then we claim that it is enough to show that
  \begin{align*}
    \iprod{\psi(fg)x}{x} &= \iprod{\psi(f)\psi(g)x}{x}\tag*{$(\ast)$}
  \end{align*}
  holds for any $g\in C(\Omega)$. A way to rewrite $(\ast)$ is by
  \begin{align*}
    \int_{}^{} gf\:d\mu_{x,x} &= \int_{}^{} g\:d\mu_{x,\psi( \overline{f} )x},
  \end{align*}
  so if $(\ast)$ holds for all $g\in C(\Omega)$, then the measures $f\:d\mu_{x,x}$ and $\mu_{x,\psi( \overline{f} )x}$ are equal since their corresponding linear functionals are necessarily equal. In particular, this holds for all such $g$.

  Since $\varphi$ is a $\ast$-homomorphism, the equation $(\ast)$ holds for all $f,g\in C(\Omega)$, so it holds if $f\in C(\Omega)$ and $g\in B_{\infty}(\Omega)$ by density. Similarly, by replacing $f$ and $g$ with their conjugates, we have
  \begin{align*}
    \iprod{\psi( \overline{fg} )x}{x} &= \iprod{\psi( \overline{f} ) \psi( \overline{g} )x}{x},
  \end{align*}
  so by taking conjugates and using the fact that $\psi$ is a homomorphism, we get
  \begin{align*}
    \iprod{\psi(gf)x}{x} &= \iprod{\psi(g)\psi(f)x}{x}\tag*{$(\ast\ast)$}
  \end{align*}
  for all $g\in B_{\infty}(\Omega)$. Using $(\ast)$ by interchanging $g$ and $f$, we obtain that $(\ast\ast)$ holds for all $f,g\in B_{\infty}(\Omega)$. Since $x\in H$ was arbitrary, we have $\psi(gf) = \psi(g)\psi(f)$, so $\psi$ is a homomorphism.

  Now, if $S$ is a Borel subset of $\Omega$, we let $E(S) = \psi\left(\1_S\right)$. We see that $E(S)$ is a projection on $H$, and that the map $E\colon S\rightarrow E(S)$ is a spectral measure, with $E_{x,y} = \mu_{x,y}\in M(\Omega)$, as
  \begin{align*}
    E_{x,y}(S) &= \iprod{E(S)x}{y}\\
               &= \iprod{\psi\left( \1_S \right)x}{y}\\
               &= \int_{}^{} \1_S\:d\mu_{x,y}.
  \end{align*}
  If $f\in B_{\infty}(\Omega)$, then from a bootstrapping argument, we have
  \begin{align*}
    \iprod{\left( \int_{}^{} f\:dE \right)x}{y} &= \int_{}^{} f\:dE_{x,y}\\
                                                &= \int_{}^{} f\:d\mu_{x,y}\\
                                                &= \iprod{\psi(f)x}{y},
  \end{align*}
  so that
  \begin{align*}
    \psi(f) &= \int_{}^{} f\:dE,
  \end{align*}
  and in particular, for all $f\in C(\Omega)$,
  \begin{align*}
    \varphi(f) &= \int_{}^{} f\:dE.
  \end{align*}
  Additionally, for all $x,y\in H$, if $E'$ is another spectral measure that satisfies
  \begin{align*}
    \varphi(f) &= \int_{}^{} f\:dE',
  \end{align*}
  then we have for all $x,y\in H$,
  \begin{align*}
    \int_{}^{} f\:dE'_{x,y} &= \iprod{\varphi(f)x}{y}\\
                            &= \int_{}^{} f\:dE_{x,y},
  \end{align*}
  so $E'_{x,y} = E_{x,y}$ for all $x,y$, meaning that for all Borel $S\subseteq \Omega$,
  \begin{align*}
    \iprod{E'(S)x}{y} &= \iprod{E(S)x}{y},
  \end{align*}
  meaning $E = E'$.

  Finally, if $T$ is an operator on $H$ commuting with all the elements of the range of $\varphi$, then if $f\in C(\Omega)$, we have
  \begin{align*}
    \int_{}^{} f\:d\mu_{Tx,y} &= \iprod{\psi(f)Tx}{y}\\
                              &= \iprod{T\psi(f)x}{y}\\
                              &= \iprod{\psi(f)x}{T^{\ast}y}\\
                              &= \int_{}^{} f\:d\mu_{x,T^{\ast}y},
  \end{align*}
  so that $E_{Tx,y} = E_{x,T^{\ast}y}$, and $E(S)T = TE(S)$ for all Borel $S\subseteq \Omega$. Conversely, if $T$ commutes with all the projections $E(S)$, then we have
  \begin{align*}
    \iprod{E(S)Tx}{y} &= \iprod{TE(S)x}{y}\\
                      &= \iprod{E(S)x}{T^{\ast}y},
  \end{align*}
  or that $E_{Tx,y} = E_{x,T^{\ast}y}$, so for all $f\in C(\Omega)$,
  \begin{align*}
    \int_{}^{} f\:dE_{Tx,y} &= \int_{}^{} f\:dE_{x,T^{\ast}y},
  \end{align*}
  or that
  \begin{align*}
    \iprod{\varphi(f)Tx}{y} &= \iprod{\varphi(f)x}{T^{\ast}y}\\
                           &= \iprod{T\varphi(f)x}{y},
  \end{align*}
  and since this holds for all $x,y\in H$, $\varphi(f)T = T\varphi(f)$.
\end{proof}
The most important case is when the $\ast$-homomorphism in question is a representation of the $C^{\ast}$-algebra generated by a normal operator, and is often known as \textit{the} spectral theorem.
\begin{theorem}
  Let $T$ be a normal operator on a Hilbert space $H$. There is a unique spectral measure $E$ relative to $\left( \sigma(T),H \right)$ such that
  \begin{align*}
    T &= \int_{}^{} \iota\:dE,
  \end{align*}
  where $\iota$ is the inclusion map of $\sigma(T)$ into $\C$.
\end{theorem}
\begin{proof}
  Let $\varphi\colon C(\sigma(T))\rightarrow B(H)$ be the functional calculus at $T$. There is then a unique spectral measure $E$ relative to $(\sigma(T),H)$ such that
  \begin{align*}
    \varphi(f) &= \int_{}^{} f\:dE
  \end{align*}
  for all $f\in C(\sigma(T))$. In particular, we have
  \begin{align*}
    T &= \varphi(\iota)\\
      &= \int_{}^{} \iota\:dE,
  \end{align*}
  and uniqueness following from the fact that $1$ and $\iota$ generate $C(\sigma(T))$ as a $C^{\ast}$-algebra.
\end{proof}
We call the spectral measure in this special case the \textit{resolution of the identity} for $T$. We have that for all $f\in B_{\infty}(\sigma(T))$, we may unambiguously define
\begin{align*}
  f(T) &= \int_{}^{} f\:dE.
\end{align*}
We call the unital $\ast$-homomorphism taking $f\mapsto f(T)$ the \textit{Borel functional calculus} at $T$.
\subsection{Examples}%
\begin{example}
  Let $\mu$ be a regular compactly supported Borel measure on $\C$. Define $M_z$ on $L_2\left( \mu \right)$ by $M_zf = zf$ for each $f\in L_2\left( \mu \right)$. Then, $M_z^{\ast}f = \overline{z}f$, and $M_z$ is normal.

  Now, we claim that $\sigma\left( M_z \right) = \supp\left( \mu \right)$. This follows from the fact that if $\lambda\in \C\setminus \supp\left( \mu \right)$, then the operator $S$ defined by
  \begin{align*}
    Sf &= \left( z-\lambda \right)^{-1}f
  \end{align*}
  has that $\norm{Sf} < \infty$ for all $f\in L_2\left( \mu \right)$.

  In particular, this means that for any bounded Borel function $\phi$, we may define $M_{\phi}f = \phi f$ and we have $\phi\left( M_z \right) = M_{\phi}$.
\end{example}
\begin{example}
  If $\left( X,\Omega,\mu \right)$ is a $\sigma$-finite measure space, and $H = L_2\left( X,\mu \right)$, we may define, for any $\phi\in L_{\infty}(\mu)$, the operator $M_{\phi}f = \phi f$. Then, $M_{\phi}$ is normal with $M_{\phi}^{\ast} = M_{ \overline{\phi} }$.

  The \textit{essential range} of $\phi$ is defined as
  \begin{align*}
    \essran(\phi) &= \bigcap\set{ \overline{\phi(S)} | S\in \Omega,\mu(X\setminus S) = 0}.
  \end{align*}
  Then, we have that $\sigma\left( M_{\phi} \right) = \essran(\phi)$. We see that if $\lambda\notin \essran(\phi)$, then there is a set $S$ in $\Omega$ with $\mu\left( X\setminus S \right) = 0$ and $\lambda\notin \overline{\phi(S)}$, so there is $\delta > 0$ so $ \left\vert \phi(x)-\lambda \right\vert\geq \delta $ for all $x\in S$. Therefore, we may define
  \begin{align*}
    M_{\psi} &= \left( M_{\phi} - \lambda \right)^{-1}
  \end{align*}
  with $\psi\in L_{\infty}(\mu)$.

  Now, if $\lambda\in \essran(\phi)$, then for every $n$, there is $S_n\in \Omega$ with $0 < \mu\left(S_n\right) < \infty$ and $ \left\vert \phi(x)-\lambda \right\vert < 1/n $ for all $x\in S_n$. Set
  \begin{align*}
    f_n &= \left( \mu\left( S_n \right) \right)^{-1/2}\1_{S_n},
  \end{align*}
  so $f_n\in L_2\left( \mu \right)$ and $\norm{f_n} = 1$. Yet, we have
  \begin{align*}
    \norm{\left( M_{\phi}-\lambda \right)f_n}^2 &= \frac{1}{\mu\left( S_n \right)} \int_{S_n}^{} \left\vert \phi-\lambda \right\vert^2\:d\mu\\
                                                &< \frac{1}{n^2},
  \end{align*}
  meaning that $\lambda$ is an element of the approximate point spectrum of $M_{\phi}$.
\end{example}
\begin{proposition}
  Let $N_k$ be a normal operator on $H_k$ for each $k\geq 1$, and assume that $\sup_{k} \norm{N_k}_{\op} < \infty$. Let $E_k$ be the spectral measure for $N_k$, and define
  \begin{align*}
    N &= \bigoplus_{k=1}^{\infty}N_k
  \end{align*}
  acting on
  \begin{align*}
    H &= \bigoplus_{k=1}^{\infty} H_k.
  \end{align*}
  Then,
  \begin{enumerate}[(a)]
    \item we have
      \begin{align*}
        \sigma(N) &= \overline{\bigcup_{k=1}^{\infty}\sigma\left( N_k \right)};
      \end{align*}
    \item and if $E$ is the spectral measure for $N$, for any Borel $S\subseteq \sigma(N)$,
      \begin{align*}
        E(S) &= \bigoplus_{k=1}^{\infty} E_k\left( S\cap \sigma\left( N_k \right) \right).
      \end{align*}
  \end{enumerate}
\end{proposition}
\begin{proof}\hfill
  \begin{enumerate}[(a)]
    \item We observe that $\lambda\in \rho\left( N \right)$ if and only if
      \begin{align*}
        N-\lambda I_{H} &= \bigoplus_{k=1}^{\infty} \left( N_k - \lambda I_k \right)
      \end{align*}
      is invertible, meaning that we must have $N_k - \lambda I_k$ is invertible for each $k$. In particular, this means that
      \begin{align*}
        \rho(N) &= \left( \bigcap_{k=1}^{\infty} \rho\left( N_k \right) \right)^{\circ},
      \end{align*}
      so that
      \begin{align*}
        \sigma(N) &= \overline{\bigcup_{k=1}^{\infty} \sigma\left( N_k \right)}.
      \end{align*}
    \item Any projection $P\colon H\rightarrow H$ necessarily decomposes as
      \begin{align*}
        P &= \bigoplus_{k=1}^{\infty} P_k P,
      \end{align*}
      where $P_k$ denotes the projection onto $H_k$. Therefore, we have
      \begin{align*}
        E(S) &= \bigoplus_{k=1}^{\infty}P_k E(S)\\
             &= \bigoplus_{k=1}^{\infty} E_k\left(\sigma\left(N_k\right)\right)E(S)\\
             &= \bigoplus_{k=1}^{\infty} E_k\left( S\cap \sigma\left( N_k \right) \right).
      \end{align*}
  \end{enumerate}
\end{proof}
\subsection{Spectral Theorem II}%
We can now reformulate the spectral theorem for the special case where the normal operator $N$ admits a special vector. Recall that a \textit{reducing subspace} for an operator $T$ on a Hilbert space $H$ is an invariant subspace $W$ such that both $W$ and $W^{\perp}$ are invariant.
\begin{definition}
  A vector $e_0\in H$ is \textit{star-cyclic} for an operator $A$ if $H$ is the smallest reducing subspace for $A$ that contains $e_0$. We say $A$ is star-cyclic if $A$ admits a star-cyclic vector.

  A vector $e_0$ is cyclic for $A$ if the smallest invariant subspace for $A$ that contains $e_0$ is $H$. We say $A$ is cyclic if $A$ has a cyclic vector.
\end{definition}
\begin{proposition}
  If $A$ has either a cyclic or star-cyclic vector, then $H$ is separable.
\end{proposition}
\begin{proof}
  By definition, both $C^{\ast}(A)$ and $\set{p(A) | p\in \C[x]}$ are separable subalgebras of $B(H)$.
\end{proof}
If $\mu$ is a compactly supported measure on $\C$, and $M_z$ is the multiplication by $z$ as defined above, then if we let $K = \supp(\mu)$, we have $C^{\ast}\left( M_z \right) = \set{M_{u} | u\in C(K)}$. Since $C(K)$ is dense in $L_2\left( \mu \right)$, it follows that $1$ is a star-cyclic vector for $M_z$.

As it turns out, the converse is also true.
\begin{theorem}
  A normal operator $N$ is star-cyclic if and only if $N$ is unitarily equivalent to $M_z$ for some compactly supported measure $\mu$ on $\C$. If $e_0$ is a star-cyclic vector for $N$, then we may choose $\mu$ to be such that there is an isomorphism $V\colon H\rightarrow L_2\left( \mu \right)$ with $Ve_0 = 1$ and $VNV^{-1} = M_z$.
\end{theorem}
\begin{proof}
  If $N\cong M_z$, then we already see that $N$ is star-cyclic.

  Now, suppose $N$ has a star-cyclic vector $e_0$. Letting $E$ be the spectral measure for $N$, we define
  \begin{align*}
    \mu\left( S \right) &= \iprod{E(S)e_0}{e_0}
  \end{align*}
  for every Borel subset $S$ of $\C$, and let $K = \supp\left( \mu \right)$. If $\phi\in B_{\infty}(K)$, then we have
  \begin{align*}
    \norm{\phi(N)e_0}^2 &= \iprod{\phi(N)e_0}{\phi(N)e_0}\\
                        &= \iprod{\left\vert \phi \right\vert^2(N)e_0}{e_0}\\
                        &= \int_{}^{} \left\vert \phi \right\vert^2\:dE_{e_0,e_0}\\
                        &= \int_{}^{} \left\vert \phi \right\vert^2\:d\mu,
  \end{align*}
  so, considering $B_{\infty}(K)$ as a subspace of $L_2\left( \mu \right)$, we may define an isometry $U\phi = \phi(N)e_0$ between $B_{\infty}(K)$ and $\set{\phi(N)e_0 | \phi\in B_{\infty}(K)}$. Yet, since $e_0$ is star-cyclic, the range of $U$ is dense in $H$, so $U$ extends to an isometric isomorphism $V\colon L_2\left( \mu \right)\rightarrow H$.

  Now, if $\phi\in B(K)$, then
  \begin{align*}
    VM_zV^{-1}\left( \phi(N)e_0 \right) &= VM_z(\phi)\\
                                            &= V\left( z\phi \right)\\
                                            &= N\phi(N)e_0,
  \end{align*}
  so that $VM_zV^{-1} = N$. By exchanging $V$ for $V^{-1}$, we get our desired result.
\end{proof}
It turns out then that any theorem about the multiplication operators applies to any star-cyclic normal operator. We may now give a complete unitary invariant of star-cyclic normal operators.
\begin{definition}
  Two measures $\mu_1$ and $\mu_2$ defined on the measurable space $\left( X,\Omega \right)$ are \textit{mutually absolutely continuous} if $\mu_1(S) = 0$ if and only if $\mu_2(S) = 0$. We write $\left[ \mu_1 \right] = \left[ \mu_2 \right]$.
\end{definition}
\begin{theorem}
  If we let $M_{\mu_1} = M_z\in B\left(L_2\left( \mu_1 \right)\right)$ and $M_{\mu_2} = M_z\in B\left( L_2\left( \mu_2 \right) \right)$, then we have $M_{\mu_1}\cong M_{\mu_2}$ are unitarily equivalent if and only if $\left[ \mu_1 \right] = \left[ \mu_2 \right]$.
\end{theorem}
\begin{proof}
  Let $\left[ \mu_1 \right] = \left[ \mu_2 \right]$, and take $\phi = \diff{\mu_1}{\mu_2}$. If $g\in L_1\left( \mu_1 \right)$, then $g\phi\in L_{1}\left( \mu_2 \right)$, with
  \begin{align*}
    \int_{}^{} g\phi\:d\mu_2 &= \int_{}^{} g\:d\mu_1.
  \end{align*}
  Therefore, if we let $\phi\in L_2\left( \mu_1 \right)$, we have $\sqrt{\phi}f\in L_2\left( \mu_2 \right)$ with $\norm{\sqrt{\phi}f} = \norm{f}$. That is, $U\colon L_2\left( \mu_1 \right)\rightarrow L_2\left( \mu_2 \right)$ given by $Uf = \sqrt{\phi}f$ is an isometry. For any $g\in L_2\left( \mu_2 \right)$, we may define $f = \phi^{-1/2}g\in L_2\left( \mu_1 \right)$ with $Uf = g$, so $U$ is surjective, with $UM_{\mu_1}U^{-1} = M_{\mu_2}$.

  Now, if we let $V\colon L_2\left( \mu_1 \right)\rightarrow L_2\left( \mu_2 \right)$ be an isomorphism with $VM_{\mu_1}V^{-1} = M_{\mu_2}$, we take $\psi = V(1)$, where we use the fact that $\mu_1$ and $\mu_2$ are compactly supported measures. Therefore, $\psi\in L_2\left( \mu_2 \right)$. 
\end{proof}
% Discuss application to the general case of representations of C*-algebras in states and representations document
\subsection{Some Applications}%
\begin{proposition}
  If $N$ is a normal operator, then $N$ is compact if and only if for every $\ve > 0$, $E\left( \set{z | \left\vert z \right\vert > \ve} \right)$ has finite rank.
\end{proposition}
\begin{proof}
  Let $\Delta_{\ve} = \set{z | \left\vert z \right\vert > \ve}$, with $E_{\ve} = \Delta_{\ve}$. Then,
  \begin{align*}
    N - NE_{\ve} &= \int_{\sigma(N)}^{} z-z\1_{\Delta_{\ve}}\:dE(z)\\
                 &= \int_{}^{} z\1_{\C\setminus \Delta_{\ve}}\:dE\\
                 &\eqcolon \phi(N).
  \end{align*}
  Observe that we have
  \begin{align*}
    \norm{ \int_{}^{} \phi\:dE } &\leq \sup_{z\in\sigma(N)}\left\vert \phi(z) \right\vert,
  \end{align*}
  so that $\norm{N-NE_{\ve}} \leq \ve$. In particular, if $E_{\ve}$ has finite rank for every $\ve > 0$, so too does $NE_{\ve}$, so $N\in K(H)$.

  Meanwhile, if $N$ is compact, then $\phi(z) = z^{-1}\1_{\Delta_{\ve}}(z)$ is a bounded Borel function on $\C$, and $N\phi(N)$ is compact. Yet,
  \begin{align*}
    N\phi(N) &= \int_{}^{} zz^{-1}\1_{\Delta_{\ve}}(z)\:dE(z)\\
             &= E_{\ve},
  \end{align*}
  so $E_{\ve}$ is a compact projection, hence of finite rank.
\end{proof}
\nocite{pedersen_analysis_now,blackadar_operator_algebras,davidson_functional_analysis,conway_functional_analysis}
\printbibliography
\end{document}
