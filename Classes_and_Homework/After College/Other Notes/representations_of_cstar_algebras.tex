\documentclass[10pt]{mypackage}

% sans serif font:
%\usepackage{cmbright}
%\usepackage{sfmath}
%\usepackage{bbold} %better blackboard bold

%\usepackage{homework}
\usepackage{notes}
\usepackage{mlmodern}
\usepackage[ backend=bibtex,
style=alphabetic,
sorting=ynt ]{biblatex}
\addbibresource{all_references.bib}
\usepackage{parskip}

\fancyhf{}
\fancyhead[R]{Avinash Iyer}
\fancyhead[L]{Representations of $C^{\ast}$-Algebras}
\fancyfoot[C]{\thepage}

\setcounter{secnumdepth}{0}

\begin{document}
\RaggedRight

\section{Basics}%
\begin{definition}
  Let $A$ be a $C^{\ast}$-algebra. A \textit{representation} of $A$ is a $\ast$-homomorphism $\pi\colon A\rightarrow B\left( \mathcal{H} \right)$ for some Hilbert space $\mathcal{H}$. 
\end{definition}
\begin{definition}
  Two representations $\pi\colon A\rightarrow B\left( \mathcal{H}_{\pi} \right)$ and $\rho\colon A\rightarrow B\left( \mathcal{H}_{\rho} \right)$ are called unitarily equivalent if there is a unitary map $U\colon \mathcal{H}_{\rho}\rightarrow \mathcal{H}_{\pi}$ such that
  \begin{align*}
    \pi(a) &= U\rho(a)U^{\ast}
  \end{align*}
  for all $a\in A$.
\end{definition}
\begin{definition}
  If $\pi\colon A\rightarrow B\left( \mathcal{H}_{\pi} \right)$ and $\rho\colon A\rightarrow B\left( \mathcal{H}_{\rho} \right)$ be representations. Then, the formula
  \begin{align*}
    \pi\oplus\rho(a) \left( h,k \right) &\coloneq \left( \pi(a)h,\rho(a)k \right)
  \end{align*}
  defines the \textit{direct sum} of $\pi$ and $\rho$. If $\pi$ is unitarily equivalent to a direct sum $\rho_1\oplus \rho_2$, then we consider $\rho_1\oplus \rho_2$ to be a decomposition of $\pi$ in terms of the ``smaller'' representations.
\end{definition}
\begin{definition}
  A closed subspace $\mathcal{K}$ of $\mathcal{H}_{\pi}$ is \textit{invariant} under $\pi$ if $\pi(a)k\in \mathcal{K}$ for all $a\in A$ and $k\in \mathcal{K}$.
\end{definition}
Observe that if $\mathcal{K}$ is an invariant subspace, then the orthogonal complement $\mathcal{K}^{\perp}$ is also invariant. This follows from the fact that if $y\in \mathcal{K}^{\perp}$, then
\begin{align*}
  \iprod{k}{\pi(a)y} &= \iprod{\pi(a)^{\ast}k}{y}\\
                     &= \iprod{\pi\left( a^{\ast} \right)k}{y}\\
                     &= 0
\end{align*}
for all $k\in \mathcal{K}$.

Conversely, if $\mathcal{K}$ is invariant, then we can recover $\pi = \pi|_{\mathcal{K}} \oplus \pi|_{\mathcal{K}^{\perp}}$, via the canonical unitary isomorphism $U\colon \mathcal{K}\oplus \mathcal{K}^{\perp}\rightarrow \mathcal{H}_{\pi}$ given by $\left( k,y \right)\mapsto k + y$.
\begin{definition}
  A representation $\pi$ is \textit{irreducible} if there are no closed invariant subspaces apart from $\set{0}$ and $\mathcal{H}_{\pi}$.
\end{definition}
\begin{lemma}
  A representation $\pi$ of a $C^{\ast}$-algebra $A$ is irreducible if and only if $\pi(A)' = \C I_{\mathcal{H}}$, where $\pi(A)'$ denotes the commutant of $\pi(A)$.
\end{lemma}
\begin{proof}
  Suppose $\mathcal{V}$ is a nontrivial invariant subspace for $\pi$. Then, the orthogonal projection $P_{\mathcal{V}}$ commutes with every $\pi(A)$ and is not a scalar multiple of $I_{\mathcal{H}}$.

  Now, suppose there is a non-scalar operator $T$ commuting with $\pi(A)$. Then, either the real or imaginary part of $T$ is a self-adjoint operator $S$ that commutes with $\pi(A)$. From the continuous functional calculus, since $\sigma(S)$ is not one point, there are some nonzero continuous $f,g\in C\left( \sigma(S) \right)$ such that $fg = 0$. Then, since $f(S),g(S)\in C^{\ast}\left( S \right)$, and $f(S),g(S)$ commute with $\pi(A)$, it follows that $ \overline{f(S)\mathcal{H}} $ and $ \overline{g(S)\mathcal{H}} $ are nonzero mutually orthogonal invariant subspaces, so $\pi$ is reducible.
\end{proof}
\begin{definition}
  If $\pi$ is a representation of the $C^{\ast}$-algebra $A$, then we call the subspace
  \begin{align*}
    \left[ \pi(A)\mathcal{H}_{\pi} \right] &= \overline{\Span}\set{\pi(a)h | h\in \mathcal{H}_{\pi}, a\in A}
  \end{align*}
  the \textit{essential subspace} of $\mathcal{H}_{\pi}$. The representation $\pi$ is called \textit{nondegenerate} if the essential subspace $\mathcal{K}$ is equal to $\mathcal{H}_{\pi}$.
\end{definition}
Note that the representation $\pi$ being nondegenerate is equivalent to $\pi(1) = I_{\mathcal{H}_{\pi}}$ if $A$ has an identity, or $\pi\left( e_i \right) \rightarrow I_{\mathcal{H}_{\pi}}$ strongly for any approximate identity $\left( e_i \right)_{i\in I}$.

The essential subspace is always invariant, and $\pi$ is equivalent to $\pi|_{\mathcal{K}}\oplus 0$. Generally, if $I$ is an ideal in $A$, then the subspace
\begin{align*}
  \mathcal{K} &= \overline{\Span} \set{\pi(a)h | h\in \mathcal{H}_{\pi},a\in I}
\end{align*}
is invariant, but $\pi$ is not zero on $\mathcal{K}^{\perp}$ unless $I$ is an essential ideal.\footnote{An essential ideal is one that has nonzero intersection with any other closed ideal of $A$.} Any nondegenerate representation of an ideal $I$ extends canonically to a nondegenerate representation $\pi$ of $A$ on the same space.
\section{The Gelfand--Naimark--Segal Construction}%
\begin{definition}
  An element $a$ of a $C^{\ast}$-algebra $A$ is called \textit{positive} if there is $b\in A$ with $a = b^{\ast}b$. Equivalently, $a$ is positive if and only if $\sigma(a)\subseteq [0,\infty)$.
\end{definition}
There are a few useful identities for positive elements. Specifically, the following hold:
\begin{align*}
  \norm{a}^21_A &\geq a^{\ast}a\\
  \norm{a}^2 b^{\ast}b - b^{\ast}a^{\ast}ab &\geq 0.
\end{align*}
\begin{definition}
  A linear functional $\rho\colon A\rightarrow \C$ is called \textit{positive} if $\rho(a)\geq 0$ whenever $a\geq 0$. A positive linear functional of norm $1$ is called a \textit{state}.
\end{definition}
\begin{lemma}
  Let $f$ be a positive linear functional on a $C^{\ast}$-algebra $A$. Then, for all $a,b\in A$, we have
  \begin{align*}
    f\left( b^{\ast}a \right) &= \overline{f\left( a^{\ast}b \right)}
    \intertext{and}
    \left\vert f\left( b^{\ast}a \right) \right\vert^2 &\leq f\left( b^{\ast}b \right)f\left( a^{\ast}a \right).
  \end{align*}
\end{lemma}
\begin{proof}
  To see the first identity, we let $\lambda\in \C$, and observe that
  \begin{align*}
    0 &\leq f\left( \left( \lambda a + b \right)^{\ast}\left( \lambda a + b \right) \right)\\
      &= \left\vert \lambda \right\vert^2 f\left( a^{\ast}a \right) + \overline{\lambda}f\left( a^{\ast}b \right) + \lambda f\left( b^{\ast}a \right) + f\left( b^{\ast}b \right).
  \end{align*}
  Now, since $\left\vert \lambda \right\vert^2 f\left( a^{\ast}a \right) + f\left( b^{\ast}b \right)$ is always real, we must have
  \begin{align*}
    \im\left( \overline{\lambda}f\left( a^{\ast}b \right) + \lambda f\left( b^{\ast}a \right) \right) &= 0
  \end{align*}
  for all $\lambda$. By taking $\lambda = 1$ and $\lambda = i$, we get equality of imaginary and real parts of $f\left( a^{\ast}b \right)$ and $ \overline{f\left( b^{\ast}a \right)} $.

  As for the Cauchy--Schwarz inequality, we observe that if $\lambda = x \overline{f\left( b^{\ast}a \right)}$ for some $x\in \R$, we have
  \begin{align*}
    0 &\leq x^2 \left\vert f\left( b^{\ast}a \right) \right\vert^2 f\left( a^{\ast}a \right) + x\left\vert f\left( a^{\ast}b \right) \right\vert^2 + x\left\vert f\left( b^{\ast}a \right) \right\vert^2 + f\left( b^{\ast}b \right)\\
      &= x^2 \left\vert f\left( b^{\ast}a \right) \right\vert^2f\left( a^{\ast}a \right) + 2x \left\vert f\left( b^{\ast}a \right) \right\vert^2 + f\left( b^{\ast}a \right).
  \end{align*}
  The right-hand side is a quadratic in $x$ that is always greater than or equal to $0$, so
  \begin{align*}
    4 \left\vert f\left( b^{\ast}a \right) \right\vert^4 - 4\left\vert f\left( b^{\ast}a \right) \right\vert^2f\left( a^{\ast}a \right)f\left( b^{\ast}b \right) &\leq 0.
  \end{align*}
\end{proof}
To understand the GNS construction, we start by taking a state $\tau$ on a $C^{\ast}$-algebra $A$. Then, defining
\begin{align*}
  N_{\tau} &= \set{a\in A | \tau\left( a^{\ast}a \right) = 0},
\end{align*}
we observe that $\tau\left( b^{\ast}a \right) = 0$ if either $a$ or $b$ are in $N_{\tau}$. In particular, we get the inner product on $A/N_{\tau}$ given by
\begin{align*}
  \iprod{a + N_{\tau}}{b + N_{\tau}} &= \tau\left( b^{\ast}a \right).
\end{align*}
Define $\mathcal{H}_{\tau}$ to be the Hilbert space completion of $A/N_{\tau}$. Since $\norm{a}^2 b^{\ast}b - b^{\ast}a^{\ast}ab$ is of the form $c^{\ast}c$, we have
\begin{align*}
  \norm{a\left( b + N_{\tau} \right)}^2 &= \tau\left( b^{\ast}a^{\ast}ab \right)\\
                                        &= \norm{a}^2 \tau\left( b^{\ast}b \right) - \tau\left( c^{\ast}c \right)\\
                                        &\leq \norm{a}^2 \tau\left( b^{\ast}b \right)\\
                                        &= \norm{a}^2 \norm{b + N_{\tau}}^2.
\end{align*}
In particular, this means that the elements of $A$ act as bounded operators on $A/N_{\tau}$, which we extend to operators $\pi_{\tau}(a)$ in the completion. This gives a nondegenerate representation $\pi_{\tau}$ of $A$ on the Hilbert space $\mathcal{H}_{\tau}$.
\begin{lemma}
  Suppose $A$ is a non-unital $C^{\ast}$-algebra, and $\rho\in S(A)$. Then, if $\left( e_i \right)_{i\in I}$ is an approximate identity for $A$, $\rho\left( e_i \right) \rightarrow 1$. Furthermore, the formula $\tau\left( a + \lambda 1 \right) = \rho(a) + \lambda$ defines a state $\tau$ on the unitization $ \widetilde{A} $.
\end{lemma}
\begin{proof}
  Since approximate identities in $C^{\ast}$-algebras are increasing, $\rho$ is positive, and $\rho\left( e_i \right)\leq 1$ for each $e_i$, it follows that $\left( \rho\left( e_i \right) \right)_{i\in I}$ converges to $L$ for some $L\leq 1$. Now, since
  \begin{align*}
    e_i^2 &= \left( e_i^{1/2} \right)^{\ast} e_i e_i^{1/2}\\
          &\leq \left( e_i^{1/2} \right)^{\ast}e_i^{1/2}\\
          &= e_i.
  \end{align*}
  Thus, if $a\in A$, Cauchy--Schwarz gives
  \begin{align*}
    \left\vert \rho\left( e_i a \right) \right\vert^2 &\leq \rho\left( e_i^2 \right) \rho\left( a^{\ast}a \right)\\
                                                      &\leq \rho\left( e_i \right) \norm{a}^2\\
                                                      &\leq L\norm{a}^2.
  \end{align*}
  This holds for all $a\in A$ if and only if $L\geq 1$ as $\norm{\rho} = 1$, so since $L\leq 1$ it follows that $\rho\left( e_i \right) \rightarrow 1$.

  By the earlier equation, we also have that $\left\vert \rho\left( a \right) \right\vert^2 \leq \rho\left( a^{\ast}a \right)$. Combining with the identity $\rho\left( a^{\ast} \right) = \overline{\rho\left( a \right)}$, we get
  \begin{align*}
    \tau\left( \left( \lambda 1 + a \right)^{\ast}\left( \lambda 1 + a \right) \right) &= \tau\left( \left\vert \lambda \right\vert^2 1 + \overline{\lambda}a + \lambda a^{\ast} + a^{\ast}a \right)\\
                                                                                       &= \left\vert \lambda \right\vert + 2\re\left( \overline{\lambda}\rho(a) \right) + \rho\left( a^{\ast}a \right)\\
                                                                                       &\geq \left\vert \lambda \right\vert - 2\left\vert \lambda \right\vert\left\vert \rho(a) \right\vert + \rho\left( a^{\ast}a \right)\\
                                                                                       &\geq \left\vert \lambda \right\vert^2 - 2\left\vert \lambda \right\vert\left\vert \rho(a) \right\vert + \left\vert \rho(a) \right\vert^2\\
                                                                                       &\geq \left( \left\vert \lambda \right\vert - \left\vert \rho(a) \right\vert \right)^2\\
                                                                                       &\geq 0.
  \end{align*}
  Thus, $\tau$ is positive and has $\tau(1) = 1$.
\end{proof}
\begin{definition}
  If $\pi\colon A\rightarrow B\left( \mathcal{H}_{\pi} \right)$ is a representation, then a vector $h\in \mathcal{H}_{\pi}$ is called cyclic for $\pi$ if the set
  \begin{align*}
    \left[ \pi(A)h \right] \coloneq \set{\pi(a)h | a\in A}
  \end{align*}
  spans a dense subspace of $\mathcal{H}_{\pi}$.
\end{definition}
\begin{proposition}
  If $\rho$ is a state on a $C^{\ast}$-algebra $A$, then there is a unit cyclic vector $h_{\rho}$ in $\mathcal{H}_{\rho}$ such that
  \begin{align*}
    \rho(a) &= \iprod{\pi_{\rho}(a) h_{\rho}}{h_{\rho}}
  \end{align*}
  for all $a\in A$.

  Conversely, if $h$ is a unit cyclic vector for a representation $\pi\colon A\rightarrow B\left( \mathcal{H}_{\pi} \right)$, then the map $\tau\colon A\rightarrow \C$ given by
  \begin{align*}
    \tau\left( a \right) &= \iprod{\pi(a)h}{h}
  \end{align*}
  is a state on $A$. The map $a\mapsto \pi(a)h$ induces a unitary isomorphism between $\mathcal{H}_{\tau}$ and $\mathcal{H}_{\pi}$ such that $\pi(a) = U\pi_{\tau}(a)U^{\ast}$.
\end{proposition}
\begin{proof}
  If $A$ is unital, then we set $h_{\rho} = 1 + N_{\rho}$ as our cyclic vector. It then follows that
  \begin{align*}
    \left[ \pi(A)h_{\rho} \right] &= \set{\pi(a) + N_{\rho} | a\in A}\\
                                  &= A/N_{\rho},
  \end{align*}
  which is norm-dense in $\mathcal{H}_{\rho}$.

  If $A$ is non-unital, we may use the lemma to extend $\rho$ to $\tau\colon \widetilde{A}\rightarrow \C$; the inclusion $A\hookrightarrow \widetilde{A}$ induces an isometry $V$ of $\mathcal{H}_{\rho}$ into $\mathcal{H}_{\tau}$ mapping $a + N_{\rho}$ to $a + N_{\tau}$. This isometry is such that
  \begin{align*}
    V\pi_{\rho}(a) &= \pi_{\tau}(a)V
  \end{align*}
  for all $a\in A$. We may identify $\mathcal{H}_{\rho}$ with the subspace $V\mathcal{H}_{\rho}$ of $\mathcal{H}_{\tau}$, since $\mathcal{H}_{\rho}$ is then essential subspace $\left[ \pi_{\tau}(a)\mathcal{H}_{\tau} \right]$ of $\pi_{\tau}|_{A}$, with $\pi_{\tau}|_{A} = \pi_{\rho}\oplus 0$. The projection of $1 + N_{\tau}$ onto $\mathcal{H}_{\rho}$ satisfies
  \begin{align*}
    \pi_{\rho}(a)h_{\rho} &= \pi_{\tau}(a) \left( 1 + N_{\tau} \right)\\\
                          &= a + N_{\tau}.
  \end{align*}
  Thus, $h_{\rho}$ is cyclic for $\pi_{\rho}$, and has
  \begin{align*}
    \iprod{\pi_{\rho}(a)h_{\rho}}{h_{\rho}} &= \iprod{\pi_{\tau}(a)\left( 1 + N_{\tau} \right)}{1 + N_{\tau}}\\
                                            &= \tau(a)\\
                                            &= \rho(a).
  \end{align*}
  Now, suppose $h$ is a unit cyclic vector for $\pi\colon A\rightarrow B\left( \mathcal{H}_{\pi} \right)$. Let $\tau$ be a positive functional of norm at most $1$; then, $\tau$ has norm equal to $\norm{h}^2 = 1$ since $\pi\left( e_i \right)h \rightarrow h$ for any approximate identity $\left( e_i \right)_{i\in I}$ of $A$. Next,
  \begin{align*}
    N_{\tau} &= \set{a\in A | \iprod{\pi\left( a^{\ast}a \right)h}{h} = 0}\\
             &= \set{a\in A | \pi(a)h = 0},
  \end{align*}
  meaning there is a well-defined linear map $U_0\colon A/N_{\tau}\rightarrow \mathcal{H}_{\pi}$ given by $U_0\left( a + N_{\tau} \right) = \pi(a)h$. The map $U_0$ is isometric, since
  \begin{align*}
    \iprod{U_0\left( a + N_{\tau} \right)}{ U_0\left( b + N_{\tau} \right) } &= \iprod{\pi\left( b^{\ast}a \right)h}{h}\\
                                                                             &= \tau\left( b^{\ast}a \right)\\
                                                                             &= \iprod{a + N_{\tau}}{b + N_{\tau}}.
  \end{align*}
  Thus, $U_0$ extends to an isometry $U$ on the completion $\mathcal{H}_{\tau}$ of $A/N_{\tau}$, and maps onto $ \overline{\Span}\set{\pi(a)h | a\in A} $, which is equal to $\mathcal{H}_{\pi}$ since $h$ is cyclic. Thus, $U$ is unitary.

  We find that
  \begin{align*}
    U\pi_{\tau}\left( a \right)\left( b + N_{\tau} \right) &= U\left( ab + N_{\tau} \right)\\
                                                           &= \pi\left( ab \right)h\\
                                                           &= \pi(a)\left( \pi(b)h \right)\\
                                                           &= \pi(a)U\left( b + N_{\tau} \right).
  \end{align*}
  Thus, $\pi_{\tau}(a)$ is unitarily equivalent to $\pi(a)$ for each $a$.
\end{proof}
\begin{remark}
  The completion of $A/N_{\phi}$, which we denoted $\mathcal{H}_{\phi}$, is often denoted as $L_2\left( A,\phi \right)$.
\end{remark}
\begin{proposition}
  Let $\phi$ and $\psi$ be positive linear functionals with $\psi\leq \phi$ inducing representations $\pi_{\phi}$ and $\pi_{\psi}$. Then, there is a unique operator $T\in \pi_{\phi}(A)'\subseteq B\left( H_{\phi} \right)$, with $0\leq T \leq I$, such that 
  \begin{align*}
    \psi(x) &= \phi\left( T\pi_{\phi}(x) \right)\\
            &= \iprod{T\pi_{\phi}(x)\xi_{\phi}}{\xi_{\phi}}_{\phi},
  \end{align*}
  where $\xi_{\phi}$ denotes the cyclic vector for $\phi$.
\end{proposition}
\begin{proof}
  We define a sesquilinear form $F$ on $\mathcal{H}_{\phi}$ such that
  \begin{align*}
    \psi\left( y^{\ast}x \right) &= F\left( \pi_{\phi}(x)\xi_{\phi},\pi_{\phi}(y)\xi_{\phi} \right).
  \end{align*}
  This is a bounded sesquilinear form that is necessarily positive. Thus, there is some $T\in B\left( \mathcal{H}_{\phi} \right)$ with $0\leq T \leq I$ such that $F\left( \eta,\zeta \right) = \iprod{T\eta}{\zeta}_{\phi}$ for all $\eta,\zeta\in \mathcal{H}_{\phi}$.

  Now, for $x,y,z\in A$, we have
  \begin{align*}
    \iprod{T\pi_{\phi}(x)\pi_{\phi}(z)\xi_{\phi}}{\pi_{\phi}(y)\xi_{\phi}}_{\phi} &= \psi\left( y^{\ast}\left( xz \right) \right)\\
                                                                                          &= \psi\left( \left( x^{\ast}y \right)^{\ast}z \right)\\
                                                                                          &= \iprod{T\pi_{\phi}(z)\xi_{\phi}}{\pi_{\phi}\left( x^{\ast} \right) \pi_{\phi}(y)\xi_{\phi} }_{\phi}\\
                                                                                          &= \iprod{\pi_{\phi}\left( x \right)T  \pi_{\phi}(z)\xi_{\phi} }{\pi_{\phi}(y)\xi_{\phi}}_{\phi}.
  \end{align*}
  Therefore, we have $T\pi_{\phi}(x) = \pi_{\phi}(x)T$.
\end{proof}
Notice that if $\psi\leq \phi$, then $\omega = \phi-\psi$ is positive linear functional such that $\phi = \psi + \omega$, meaning that $\pi_{\phi}$ is unitarily equivalent to
\begin{align*}
  \mathcal{H}_{\phi} = \left[ \left( \pi_{\psi}\oplus \pi_{\omega} \right)\left(A\right)\left( \xi_{\psi}\oplus \xi_{\omega} \right) \right],
\end{align*}
and $T$ is the compression of the projection onto the first coordinate.

Recall that the state space $S(A)$ is a $w^{\ast}$-compact convex subset of $A^{\ast}$, so it is equal to the closed convex hull of its extreme points by Krein--Milman. These extreme points are known as pure states.
\begin{proposition}
  If $\phi$ is a state on $A$, then $\pi_{\phi}$ is irreducible if and only if $\phi$ is a pure state.
\end{proposition}
\begin{proof}
  If $\pi_{\phi}$ is irreducible, then for any positive linear functional $\psi\leq \phi$, it must be the case that $\psi$ is a multiple of $\phi$ since there are no sub-representations for $\pi_{\phi}$. In particular, this means that $\psi = \phi$, so $\phi$ is pure.

  Now, let $\phi$ be pure. Suppose there is a projection $P$ in $\pi_{\phi}(A)'$ with $P\neq 0,I$.  Then, $P\xi_{\phi} \neq 0$, as else we would have
  \begin{align*}
    0 &= \pi_{\phi}(x)P\xi_{\phi}\\
      &= P\left[ \pi_{\phi}(x)\xi_{\phi} \right]
  \end{align*}
  for all $x\in A$, whence $P$ would be equal to zero. Similarly, we must have $\left( I-P \right)\xi_{\phi} \neq 0$. For each $x\in A$, let
  \begin{align*}
    \phi_1(x) &= \iprod{\pi_{\phi}P\xi_{\phi}}{P\xi_{\phi}}_{\phi}\\
              &= \phi\left( P\pi_{\phi}(x) \right)\\
    \phi_2(x) &= \iprod{\pi_{\phi}\left( I-P \right)\xi_{\phi}}{\left( I-P \right)\xi_{\phi}}\\
              &= \phi\left( \left( I-P \right)\pi_{\phi}(x) \right).
  \end{align*}
  Then, $\phi_1,\phi_2$ are positive linear functionals on $A$, with $\phi = \phi_1 + \phi_2$, $\phi_1 = \lambda\phi$, and $\phi_2 = \left( 1-\lambda \right)\phi$ for some $0 < \lambda < 1$, following from the fact that $\phi$ is pure. For any $\ve > 0$, there is $x\in A$ with $\norm{\pi_{\phi}\xi_{\phi} - P\xi_{\phi}} < \ve$, and we have
  \begin{align*}
    \phi\left( x^{\ast}x \right) &= \norm{\pi_{\phi}(x)\xi_{\phi}}^2\\
                                 &= \norm{P\xi_{\phi}}^2,
  \end{align*}
  whence
  \begin{align*}
    \norm{\left( I-P \right)\pi_{\phi}\xi_{\phi} - P\xi_{\phi}} < \ve,
  \end{align*}
  and
  \begin{align*}
    \left( 1-\lambda \right)\norm{P\xi_{\phi}}^2 &= \phi_2\left( x^{\ast}x \right)\\
                                                 &= \norm{\left( I-P \right)\pi_{\phi}\xi_{\phi}}^2\\
                                                 &< \ve^2,
  \end{align*}
  and since $\ve$ is arbitrary, we have $\lambda = 1$, which means no such $P$ exists. Thus, $\pi_{\phi}(A)' = \C I$, and so $\pi_{\phi}$ is irreducible.
\end{proof}
\begin{corollary}
  If $A$ is a $C^{\ast}$-algebra and $x\in A$, then there is an irreducible representation $\pi$ of $A$ with $\norm{\pi(x)} = \norm{x}$.
\end{corollary}
\begin{proof}
  Let $\pi$ be represented via the pure state $\phi$. There is a pure state on $A$ with $\phi\left( \left( xx^{\ast} \right)^2 \right) = \norm{\left( xx^{\ast} \right)^2}$, emerging from extending the isomorphism $C^{\ast}\left( \left( xx^{\ast} \right)^2 \right)\cong C_0\left( \sigma\left( \left( xx^{\ast} \right)^2 \right) \right)$. Thus,
  \begin{align*}
    \norm{x}^2 &= \iprod{xx^{\ast}}{xx^{\ast}}^{1/2}\\
               &= \norm{xx^{\ast}}\\
               &= \norm{\pi(x)x^{\ast}}\\
               &\leq \norm{\pi(x)}\norm{x}\\
               &\leq \norm{x}^2,
  \end{align*}
  so the inequalities are equalities, and $\norm{\pi(x)} = \norm{x}$.
\end{proof}
\section{Spectrum of a $C^{\ast}$-Algebra}%
\begin{definition}
  If $A$ is a $C^{\ast}$-algebra, the \textit{spectrum} of $A$, denoted $ \hat{A} $, is the set of unitary equivalence classes of irreducible representations of $A$.
\end{definition}
We start by understanding the irreducible representations of $K(\mathcal{H})$. Recall the characterization of $K\left( \mathcal{H} \right)$ as
\begin{align*}
  K\left( \mathcal{H} \right) &= \overline{\Span}\set{h\otimes \overline{k} | h,k\in \mathcal{H}},
\end{align*}
where
\begin{align*}
  h\otimes \overline{k}\left( \ell \right) &= \iprod{\ell}{k}h.
\end{align*}
Let $\pi\colon K\left( \mathcal{H} \right)\rightarrow B\left( \mathcal{H}_{\pi} \right)$ be an irreducible representation, and let $e\in \mathcal{H}$ be a unit vector. Then, $e\otimes \overline{e}$ is the projection of $\mathcal{H}$ onto the span of $e$, meaning $P = \pi\left( e\otimes \overline{e} \right)$ is the projection onto the closed subspace $P\mathcal{H}_{\pi}$. Let $\xi\in B\left( \mathcal{H}_{\pi} \right)$ be fixed, and consider
\begin{align*}
  \left[ \pi\left( K\left(\mathcal{H}\right) \right)\xi \right] &= \overline{\Span}\set{\pi(T)\xi | T\in K\left( \mathcal{H} \right)}.
\end{align*}
This is a nonzero invariant subspace, so by irreducibility, $\left[ \pi\left( K\left( \mathcal{H} \right) \right)\xi \right]$ is all of $\mathcal{H}_{\pi}$. Define $U\colon \mathcal{H}\rightarrow \mathcal{H}_{\pi}$ by
\begin{align*}
  Uh = \pi\left( h\otimes \overline{e} \right)\xi.
\end{align*}
Then, we have
\begin{align*}
  \iprod{Ug}{Uh} &= \iprod{\pi\left( g\otimes \overline{e} \right)\xi}{\pi\left( h\otimes \overline{e} \right)\xi}\\
                 &= \iprod{\xi}{\pi\left( \left( g\otimes \overline{e} \right)^{\ast}\left( h\otimes \overline{e} \right) \right)\xi}\\
                 &= \iprod{\xi}{\pi\left( \left( e\otimes \overline{g} \right)\left( h\otimes \overline{e} \right) \right)\xi}\\
                 &= \iprod{\xi}{\pi\left( \iprod{h}{g}\left( e\otimes \overline{e} \right) \right)\xi}\\
                 &= \iprod{g}{h} \iprod{\xi}{\pi\left( e\otimes \overline{e} \right)\xi}\\
                 &= \iprod{g}{h} \norm{\xi}^2\\
                 &= \iprod{g}{h}.
\end{align*}
Furthermore, since
\begin{align*}
  \pi\left( h\otimes \overline{k} \right)\xi &= \pi\left( h\otimes \overline{k} \right)\pi\left( e\otimes \overline{e} \right)\xi\\
                                             &= \iprod{e}{k}\pi\left( h\otimes \overline{e} \right)\xi\\
                                             &= \iprod{e}{k}Uh,
\end{align*}
we have that every element of the spanning set is in the range of $U$. Thus, $U$ is a unitary, and
\begin{align*}
  \pi\left( h\otimes \overline{k} \right)\left( Ug \right) &= \pi\left( \left( h\otimes \overline{k} \right)\left( g\otimes \overline{e} \right) \right)\xi\\
                                                           &= \iprod{g}{k}\pi\left( h\otimes \overline{e} \right)\xi\\
                                                           &= \iprod{g}{k} Uh\\
                                                           &= U\left( \iprod{g}{k}h \right)\\
                                                           &= U\left( \left( h\otimes \overline{k} \right)g \right).
\end{align*}
In particular, this means we have $\pi(T)U = U T$ for all $T\in K\left( \mathcal{H} \right)$. Thus, $\pi$ is unitarily equivalent to the identity, meaning that the spectrum of $K\left( \mathcal{H} \right)$ is simply the identity.

Similar to how the analysis of commutative $C^{\ast}$-algebras depends on the identification between characters and maximal ideals on the $C^{\ast}$-algebra, given by $\phi\mapsto \ker\left( \phi \right)$. Similarly, we are interested in the ideals that are the kernels of irreducible representations, known as primitive ideals.
\begin{proposition}
  Let $A$ be a $C^{\ast}$-algebra. Then,
  \begin{enumerate}[(a)]
    \item every closed ideal in $A$ is the intersection of the primitive ideals that contain it;
    \item if $I$ is a primitive ideal, and $J,K$ are ideals such that $J\cap K\subseteq I$, then either $J\subseteq I$ or $K\subseteq I$.
  \end{enumerate}
\end{proposition}
\begin{proof}\hfill
  \begin{enumerate}[(a)]
    \item Let $a\in A$ with $a\notin I$. Then, consider $a + I\in A/I$. There is an irreducible representation of $A/I$ such that $\norm{\pi(a+I)} = \norm{a+I}\neq 0$. Yet, composing with the quotient map, this yields an irreducible representation $\pi\circ q$ with $a\notin \ker\left( \pi\circ q \right)$. 
    \item Let $\pi$ be an irreducible representation with $\ker\left(\pi\right) = I$. If $J\nsubseteq I$, then $\pi(J)\neq 0$, so $\left[ \pi(J)\mathcal{H} \right] \eqcolon \mathcal{V}$ is nonzero. Since $J$ is an ideal, $\mathcal{V}$ is invariant, but this means $\left[ \pi(J)\mathcal{H} \right] = \mathcal{H}$. Yet, this yields
      \begin{align*}
        \pi(K)\left( \pi(J)\mathcal{H} \right) &\subseteq \pi\left( K\cap J \right)\mathcal{H}\\
                                               &\subseteq \pi(I)\\
                                               &= 0,
      \end{align*}
      whence $K\subseteq \ker\left( \pi \right)$.
  \end{enumerate}
\end{proof}
\nocite{blackadar_operator_algebras,morita_equivalence_cstar_algebras}
\printbibliography 
\end{document}
