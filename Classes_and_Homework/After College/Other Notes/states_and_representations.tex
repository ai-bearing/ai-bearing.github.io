\documentclass[10pt]{mypackage}

\usepackage{mlmodern}
%\usepackage{newpxtext,eulerpx,eucal}
%\renewcommand*{\mathbb}[1]{\varmathbb{#1}}

%\usepackage{homework}
\usepackage{notes}

\usepackage[ backend=bibtex, style = alphabetic, sorting=ynt ]{biblatex}
\addbibresource{all_references.bib}

\usepackage{parskip}

\fancyhf{}
\fancyhead[R]{Avinash Iyer}
\fancyhead[L]{States and Representations of $C^{\ast}$-Algebras}
\fancyfoot[C]{\thepage}

\setcounter{secnumdepth}{0}

\begin{document}
\RaggedRight
\section{States}%
\subsection{Positive Cone of a $C^{\ast}$-Algebra}%
In order to understand states and representations, we need to start by understanding positive elements of $C^{\ast}$-algebras. This will require a solid command of the \href{https://ai.avinash-iyer.com/Classes_and_Homework/After\%20College/Other\%20Notes/continuous_functional_calculus.pdf}{continuous functional calculus}.
\begin{definition}
  Let $A$ be a $C^{\ast}$-algebra. We say $a\in A$ is \textit{positive} if $a\in A_{\sa}$ and $\sigma(a)\subseteq \R_{+}$. We write $a\geq 0$, and say $a\in A_{+}$.
\end{definition}
For example, an element $f\in C(X)$ is positive if and only if $f(x)\geq 0$ for all $x\in X$, and $\phi\in L_{\infty}(\mu)$ is positive if and only if $\phi(x)\geq 0$ for $\mu$-a.e. $x$.
\begin{proposition}
  If $a\in A_{+}$, then there are unique positive elements $u$ and $v$ in $A$ such that $a = u-v$, $uv = vu = 0$.
\end{proposition}
\begin{proof}
  Letting $f(t) = \max(0,t)$ and $g(t) = -\min(0,t)$, we have $f,g\in C(\R)_{+}$ with $f(t) - g(t) = t$. Using the continuous functional calculus, let $u = f(a)$ and $v = g(a)$. Spectral mapping gives that $u,v\in A_{+}$ with $a = u-v$ and $uv = vu = 0$.
\end{proof}
\nocite{conway_operator_theory}
\printbibliography
\end{document}
