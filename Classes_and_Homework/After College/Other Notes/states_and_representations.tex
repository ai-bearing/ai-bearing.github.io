\documentclass[10pt]{mypackage}

\usepackage{mlmodern}
%\usepackage{newpxtext,eulerpx,eucal}
%\renewcommand*{\mathbb}[1]{\varmathbb{#1}}

%\usepackage{homework}
\usepackage{notes}

\usepackage[ backend=bibtex, style = alphabetic, sorting=ynt ]{biblatex}
\addbibresource{all_references.bib}

\usepackage{parskip}

\fancyhf{}
\fancyhead[R]{Avinash Iyer}
\fancyhead[L]{States and Representations of $C^{\ast}$-Algebras}
\fancyfoot[C]{\thepage}

\setcounter{secnumdepth}{0}

\begin{document}
\RaggedRight
\section{States}%
For now, we will assume that $M$ is a unital self-adjoint subspace of a $C^{\ast}$-algebra $A$. If $\rho$ is a linear functional on $M$, then the equation
\begin{align*}
  \rho^{\ast}\left( a \right) &= \overline{\rho\left( a^{\ast} \right)}
\end{align*}
defines another linear functional; if $\rho = \rho^{\ast}$, then we call $\rho$ hermitian. Equivalently, $\rho$ is hermitian if $\rho\left( a^{\ast} \right) = \overline{\rho(a)}$. If $\rho$ is a bounded hermitian functional on $M$, then we claim that
\begin{align*}
  \norm{\rho} &= \sup\set{\rho(a) | a\in M_{\sa},\norm{a}\leq 1}.
\end{align*}
This follows from the fact that if $\ve > 0$, then from the Riesz lemma, we may find $a$ in the unit ball of $M$ with $\left\vert \rho(a) \right\vert > \norm{\rho} - \ve$. For a suitable $\lambda$ with $\left\vert \lambda \right\vert = 1$, we have
\begin{align*}
  \norm{\rho} - \ve &< \left\vert \rho(a) \right\vert\\
                    &= \rho\left( \lambda a \right)\\
                    &= \overline{\rho\left( \lambda a \right)}\\
                    &= \rho\left( \left( \lambda a \right)^{\ast} \right).
\end{align*}
If $a_0 = \re\left( \lambda a \right)$, we have $\norm{a_0}\leq 1$ with $\rho\left( a_0 \right) > \norm{\rho} - \ve$. Thus,
\begin{align*}
  \norm{\rho} &\leq \sup\set{\rho(a) | a \in M_{\sa},\norm{a}\leq 1},
\end{align*}
with the reverse inequality being true by definition.

We say the linear functional $\rho$ is \textit{positive} if for any $a\in M_{+}$, $\rho(a)\geq 0$; if $\rho(1) = 1$, then we say $\rho$ is a state. In fact, if $\rho$ is positive, then $\rho$ is hermitian, since if $a = a^{\ast}$, then 

We start by considering a version of the Cauchy--Schwarz inequality for states.
\begin{proposition}
  If $\rho$ is a positive linear functional on a $C^{\ast}$-algebra $A$, then
  \begin{align*}
    \left\vert \rho(b^{\ast}a) \right\vert^2 &\leq \rho\left( a^{\ast}a \right)\rho\left( b^{\ast}b \right).
  \end{align*}
\end{proposition}
\begin{proof}
  With $a\in A$, we have $a^{\ast}a\in A_{+}$, so $\rho\left( a^{\ast}a \right)\geq 0$. Then, since $\rho$ is hermitian, we have that
  \begin{align*}
    \iprod{a}{b} &= \rho\left( b^{\ast}a \right)
  \end{align*}
  defines a positive sesquilinear form on $A$, so the traditional Cauchy--Schwarz inequality gives the desired result.
\end{proof}

\nocite{conway_operator_theory,kadison_and_ringrose_1,morita_equivalence_cstar_algebras}
\printbibliography
\end{document}
