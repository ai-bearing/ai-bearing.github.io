\documentclass[10pt]{mypackage}

\usepackage{mlmodern}
%\usepackage{newpxtext,eulerpx,eucal}
%\renewcommand*{\mathbb}[1]{\varmathbb{#1}}

%\usepackage{homework}
\usepackage{notes}

\usepackage[ backend=bibtex, style = alphabetic, sorting=ynt ]{biblatex}
\addbibresource{all_references.bib}

\usepackage{parskip}

\fancyhf{}
\fancyhead[R]{Avinash Iyer}
\fancyhead[L]{States and Representations of $C^{\ast}$-Algebras}
\fancyfoot[C]{\thepage}

\setcounter{secnumdepth}{0}

\begin{document}
\RaggedRight
\section{Representations}%
\begin{definition}
  If $A$ is a $C^{\ast}$-algebra, a representation of $A$ is a pair $\left( \pi,H \right)$ where $H$ is a Hilbert space and $\pi\colon A\rightarrow B(H)$ is a $\ast$-homomorphism. If $A$ is unital, then we require $\pi(1) = I$.
\end{definition}
Note that if $A$ does not have an identity, we can extend to the unitization $A_1 = A\oplus \C$ and define $ \widetilde{\pi}\left( a,\lambda \right) = \pi(a) + \lambda I $ for any $a\in A$ and $\lambda\in \C$.

Note that every representation is contractive and the range of any representation is closed.
\begin{example}\hfill
  \begin{enumerate}[(a)]
    \item If $A$ is a $C^{\ast}$-subalgebra of $B(H)$, then the inclusion map $A\hookrightarrow B(H)$ is a representation.
    \item If $\left( X,\Omega,\mu \right)$ is a $\sigma$-finite measure space, then $\pi\colon L_{\infty}(\mu) = B\left( L_2\left( \mu \right) \right)$, where $\pi(\phi) = M_{\phi}$, is a representation.
    \item If $X$ is compact, and $\mu$ is a positive Borel measure on $X$, then $\pi_{\mu}\colon C(X)\rightarrow B\left( L_2\left( \mu \right) \right)$ defined by $\pi_{\mu}(f) = M_f$ is a representation of $C(X)$.
  \end{enumerate}
\end{example}
\begin{definition}
  Let $A$ be a $C^{\ast}$-algebra.
  \begin{enumerate}[(i)]
    \item If $d$ is a cardinal number, $H$ a Hilbert space, we let $H^{(d)}$ denote the direct sum of $H$ with itself over $d$. If $T\in B(H)$, we let $T^{(d)}$ be the direct sum of $T$ with itself over $d$, which is known as the $d$-fold inflation of $T$.

      Given a representation $\pi\colon A\rightarrow B(H)$, we have $\pi^{(d)} \colon A\rightarrow B\left( H^{(d)} \right)$, defined by $\pi^{(d)}(a) = \pi(a)^{(d)}$ is a representation, which is known as the inflation of $\pi$.

      If $d = \aleph_{0}$, we will denote their respective inflations as $H^{(\infty)}$ and $\pi^{(\infty)}$.
    \item If $\set{\left( \pi_i,H_i \right)}_{i\in I}$ is a collection of representations of $A$, the direct sum of these representations is the representation
      \begin{align*}
        \bigoplus_{i\in I}\pi_i\colon A &\rightarrow B\left( \bigoplus_{i\in I}H_i \right)\\
        a &\mapsto \bigoplus_{i\in I}\pi_i(a).
      \end{align*}
  \end{enumerate}
\end{definition}
Note that since all representations are contractive, the direct sum is in fact a bounded operator. Furthermore, if $\pi$ is isometric (hence injective), then so too is its inflation.
\begin{example}
  If $X$ is a compact topological space, and $\left( \mu_n \right)_n$ is a sequence of positive Borel measures for $X$, with corresponding representations $\pi_n\colon C(X)\rightarrow B\left( L_2\left( \mu_n \right) \right)$ taking $f\mapsto M_f$, then $\bigoplus_{n\geq 1}\pi_n$ is also a representation.
\end{example}
\begin{definition}
  Two representations $\left( \pi_1,H_1 \right)$ and $\left( \pi_2,H_2 \right)$ are called equivalent if there is a unitary $U\colon H_1\rightarrow H_2$ such that $\pi_2(a) = U\pi_1(a)U^{-1}$.
\end{definition}
\begin{definition}
  If $A$ is a $C^{\ast}$-algebra, then a representation $\rho\colon A\rightarrow B(H)$ is \textit{non-degenerate} if
  \begin{align*}
    \left[ \rho(A)H \right] &= \overline{\set{\rho(a)h | a\in A,h\in H}}\\
                            &= H.
  \end{align*}
  That is, the reducing subspace $\left[ \rho(A)H \right]$ for $\rho(A)$ ``lives'' on the entire Hilbert space, or that the only $g\in H$ for which $\rho(a)g = 0$ for all $a\in A$ is $0$.
\end{definition}
\begin{definition}
  A representation $\rho\colon A\rightarrow B(H)$ is called \textit{cyclic} if there is some $v\in H$ such that
  \begin{align*}
    H &= \left[ \rho(A)v \right].
  \end{align*}
  We call the vector $v$ a \textit{cyclic vector} for $\rho$.
\end{definition}
\begin{theorem}
  Let $\pi$ be a representation for the $C^{\ast}$-algebra $A$. Then, there is a family of cyclic representations $\set{\pi_i}_{i\in I}$ for $A$ such that
  \begin{align*}
    \pi &\cong \bigoplus_{i\in I}\pi_i.
  \end{align*}
\end{theorem}
\begin{proof}
  Let $ \mathcal{E} $ be the family of sets of nonzero vectors in $H$ such that $\left[ \pi(A)e \right]\perp \left[ \pi(a)f \right]$ for any $e\neq f\in E\in \mathcal{E}$. Ordering $E$ by inclusion, using Zorn's Lemma gives us that $\mathcal{E}$ has a maximal element $E_0$. Define
  \begin{align*}
    H_0 &= \bigoplus_{e\in E_0} \left[ \pi(A)e \right].
  \end{align*}
  Let $h\in H_0^{\perp}$, so that $ 0 = \iprod{\pi(a)e}{h} $ for all $a\in A$ and $e\in E_0$. Therefore, if we have $a,b\in A$ with $e\in E_0$, we have
  \begin{align*}
    0 &= \iprod{\pi\left( b^{\ast}a \right)e}{h}\\
      &= \iprod{\pi(b)^{\ast}\pi(a)e}{h}\\
      &= \iprod{\pi(a)e}{\pi(b)h}.
  \end{align*}
  That is, $\pi(A)e\perp \pi(A)h$ for all $e\in E_0$, meaning that $E_0\cup\set{h}\in \mathcal{E}$, meaning that by maximality of $E_0$, we must have that $h = 0$ and $H = H_0$.

  Letting $H_e \coloneq \left[ \pi(A)e \right]$, then for any $a\in A$, we have $\pi(a)H_e\subseteq H_e$, and $\pi(a)^{\ast} = \pi\left( a^{\ast} \right)$, so that $H_e$ reduces $\pi(a)$. If we define $\pi_e\colon A\rightarrow B\left( H_e \right)$, we have that $\pi_e$ is a representation of $a$, with
  \begin{align*}
    \pi &= \bigoplus_{e\in E_0} \pi_e.
  \end{align*}
\end{proof}
\begin{definition}
  A representation $\pi$ of a $C^{\ast}$-algebra $A$ is called \textit{irreducible} if the only invariant subspaces for $\pi(A)$ are $0$ and $H$.
\end{definition}
The best example of an irreducible representation is a cyclic representation.
\begin{lemma}
  A representation of a $C^{\ast}$-algebra $A$ is irreducible if and only if the only operators commuting with $\pi(A)$ are multiples of the identity.
\end{lemma}
\begin{proof}
  If $\pi$ has a nontrivial invariant subspace $V$, then $P_V$ commutes with every $\pi(a)$ and is not a scalar. Conversely, if there is a non-scalar operator $T$ commuting with $\pi(A)$, then either the real or imaginary part of $T$ is a non-scalar self-adjoint operator $S$ commuting with $\pi(A)$, so there is some spectral projection $P$ for $S$ that is neither the $0$ projection or the identity that commutes with $\pi(A)$, meaning that $P(H)$ is an invariant subspace for $\pi(A)$.
\end{proof}
\section{States}%
For now, we will assume that $M$ is a unital self-adjoint subspace of a $C^{\ast}$-algebra $A$. If $\rho$ is a linear functional on $M$, then the equation
\begin{align*}
  \rho^{\ast}\left( a \right) &= \overline{\rho\left( a^{\ast} \right)}
\end{align*}
defines another linear functional; if $\rho = \rho^{\ast}$, then we call $\rho$ hermitian. Equivalently, $\rho$ is hermitian if $\rho\left( a^{\ast} \right) = \overline{\rho(a)}$. If $\rho$ is a bounded hermitian functional on $M$, then we claim that
\begin{align*}
  \norm{\rho} &= \sup\set{\rho(a) | a\in M_{\sa},\norm{a}\leq 1}.
\end{align*}
This follows from the fact that if $\ve > 0$, then from the Riesz lemma, we may find $a$ in the unit ball of $M$ with $\left\vert \rho(a) \right\vert > \norm{\rho} - \ve$. For a suitable $\lambda$ with $\left\vert \lambda \right\vert = 1$, we have
\begin{align*}
  \norm{\rho} - \ve &< \left\vert \rho(a) \right\vert\\
                    &= \rho\left( \lambda a \right)\\
                    &= \overline{\rho\left( \lambda a \right)}\\
                    &= \rho\left( \left( \lambda a \right)^{\ast} \right).
\end{align*}
If $a_0 = \re\left( \lambda a \right)$, we have $\norm{a_0}\leq 1$ with $\rho\left( a_0 \right) > \norm{\rho} - \ve$. Thus,
\begin{align*}
  \norm{\rho} &\leq \sup\set{\rho(a) | a \in M_{\sa},\norm{a}\leq 1},
\end{align*}
with the reverse inequality being true by definition.

We say the linear functional $\rho$ is \textit{positive} if for any $a\in M_{+}$, $\rho(a)\geq 0$; if $\rho(1) = 1$, then we say $\rho$ is a state. In fact, if $\rho$ is positive, then $\rho$ is hermitian (as we will see below).

We start by considering a version of the Cauchy--Schwarz inequality for states.
\begin{proposition}
  If $\rho$ is a positive linear functional on a $C^{\ast}$-algebra $A$, then
  \begin{align*}
    \left\vert \rho(b^{\ast}a) \right\vert^2 &\leq \rho\left( a^{\ast}a \right)\rho\left( b^{\ast}b \right).
  \end{align*}
\end{proposition}
\begin{proof}
  With $a\in A$, we have $a^{\ast}a\in A_{+}$, so $\rho\left( a^{\ast}a \right)\geq 0$. Then, since $\rho$ is hermitian, we have that
  \begin{align*}
    \iprod{a}{b} &= \rho\left( b^{\ast}a \right)
  \end{align*}
  defines a positive sesquilinear form on $A$, so the traditional Cauchy--Schwarz inequality gives the desired result.
\end{proof}
\begin{proposition}
  Let $\rho$ be a bounded linear functional on a $C^{\ast}$-algebra $A$. The following are equivalent:
  \begin{enumerate}[(i)]
    \item $\rho$ is positive;
    \item for every approximate unit $\left( e_i \right)_{i\in I}$, $\norm{\rho} = \lim_{i} \tau\left( e_i \right)$;
    \item for some approximate unit $\left( e_i \right)_{i\in I}$, $\norm{\rho} = \lim_{i}\tau\left( e_i \right)$. 
  \end{enumerate}
\end{proposition}
\begin{proof}
  We may assume that $\norm{\rho} = 1$. To see that (i) implies (ii), we assume $\rho$ is positive, and let $\left( e_i \right)_{i\in I}$ be an approximate unit for $A$. Then, $\left( \rho\left( e_i \right) \right)_{i\in I}$ is an increasing net in $\R$ that converges to its supremum, which is not greater than $1$, so $\lim_{i}\rho\left( e_i \right) \leq 1$.

  Now, we let $a\in A$ be such that $\norm{a}\leq 1$. We have
  \begin{align*}
    \left\vert \rho\left( e_i a \right) \right\vert^2 &\leq \rho\left( e_i^2 \right)\rho\left( a^{\ast}a \right)\\
                                                      &\leq \rho\left( e_i \right)\rho\left( a^{\ast}a \right)\\
                                                      &\leq \lim_{i}\rho\left( e_i \right),
  \end{align*}
  so $\left\vert \rho\left( a \right) \right\vert^2 \leq \lim_{i}\rho\left( e_i \right)$, meaning $1\leq \lim_{i}\rho\left( e_i \right)$.

  Showing that (ii) implies (iii) is pretty much by definition. For (iii) implies (i), let $\left( e_i \right)_i$ be an approximate unit with $1 = \lim_{i}\rho\left( e_i \right)$. Let $a\in A_{\sa}$ with $\norm{a}\leq 1$, and write $\rho(a) = \alpha + i\beta$ for $\alpha,\beta\in \R$. We may assume that $\beta\leq 0$, and we will show that $\beta = 0$. Letting $n$ be any positive integer, we have
  \begin{align*}
    \norm{a - ine_{i}}^2 &= \norm{\left( a+ine_{i} \right)\left( a-ine_{i} \right)}\\
                         &= \norm{a^2 + n^2e_i^2 - in\left( ae_{i}-e_{i}a \right)}\\
                         &\leq 1 + n^2 + n\norm{ae_i - e_ia},
  \end{align*}
  so that
  \begin{align*}
    \left\vert \rho\left( a-ine_{i} \right) \right\vert^2 &\leq 1 + n^2 + n\norm{ae_i - e_ia}.
  \end{align*}
  Yet, since $\lim_{i}\rho\left( a-ine_{i} \right) = \rho\left( a \right) - in$, with $\lim_{i}\norm{ae_i-e_ia} = 0$, in the limit, we get
  \begin{align*}
    \left\vert \alpha + i\beta - in \right\vert^2 &\leq 1 + n^2,
  \end{align*}
  so by expanding, we have
  \begin{align*}
    -2n\beta &\leq 1-\beta^2 - \alpha^2.
  \end{align*}
  Since $\beta\leq 0$, and this inequality holds for all positive $n$, it follows that $\beta = 0$. If $a$ is positive with $\norm{a}\leq 1$, we have $e_i-a$ is hermitian with $\norm{e_i-a}\leq 1$, so $\rho\left( e_i-a \right) \leq 1$. Then, $1-\rho(a)= \lim_{i}\rho\left( e_i-a \right)\leq 1$, meaning $\rho(a)\geq 0$. Thus, $\rho$ is positive.
\end{proof}
\begin{corollary}
  If $\rho$ is a bounded linear functional on a unital $C^{\ast}$-algebra $A$, then $\rho$ is positive if and only if $\rho(1) = \norm{\rho}$.
\end{corollary}
\begin{proof}
  The sequence consisting exclusively of $1$ is an approximate unit for $A$.
\end{proof}
The best-known example of a state is that of the \textit{vector state} on $B(H)$, given by $\rho_{v}\colon B(H)\rightarrow \C$,
\begin{align*}
  \rho_v(T) &= \iprod{Tv}{v}.
\end{align*}
for a unit vector $v\in H$.

In fact, we will show in the next section that this is, to an extent, ``every'' state on a $C^{\ast}$-algebra.
\subsection{The GNS Construction}%
The most important fact about states is that they allow us to represent any $C^{\ast}$-algebra as a subalgebra of $B(H)$.
\begin{theorem}[GNS Construction]
  Let $A$ be a $C^{\ast}$-algebra, and let $\rho\colon A\rightarrow \C$ be a state. Then, there is a representation $\pi_{\rho}\colon A\rightarrow B\left( H_{\rho} \right)$ with unit cyclic vector $\xi_{\rho}$ such that
  \begin{align*}
    \iprod{\pi_{\rho}(a)\xi_{\rho}}{\xi_{\rho}} &= \rho(a)
  \end{align*}
  for all $a\in A$.

  Furthermore, if $\xi$ is a unit cyclic vector for a representation $\pi\colon A\rightarrow B\left(H_{\pi}\right)$, then the vector state
  \begin{align*}
    \tau\colon A &\rightarrow \C\\
    a &\mapsto \iprod{\pi(a)\xi}{\xi}
  \end{align*}
  induces a unitary isomorphism of $H_{\rho}$ onto $H_{\pi}$ such that $\pi(a) = U\pi_{\tau}(a)U^{-1}$ for all $a\in A$.
\end{theorem}
\begin{proof}
  To start, we let $\rho$ be a state on a $C^{\ast}$-algebra $A$, and define the subspace
  \begin{align*}
    N_{\rho} &= \set{a\in A | \rho\left( a^{\ast}a \right) = 0}.
  \end{align*}
  We see that $\rho\left( b^{\ast}a \right) = 0$ if either $a$ or $b$ are in $N_{\rho}$, meaning there is a well-defined inner product on $A/N_{\rho}$ given by
  \begin{align*}
    \iprod{a + N_{\rho}}{b + N_{\rho}} &= \rho\left( b^{\ast}a \right).
  \end{align*}
  We may define $H_{\rho}$ to be the Hilbert space completion of $A/N_{\rho}$. We will show that $N_{\rho}$ is a left ideal, by taking, for $a\in A$ and $x\in N_{\rho}$, and using the identity $x^{\ast}a^{\ast}ax\leq \norm{a}^2x^{\ast}x$, to find
  \begin{align*}
    \iprod{ax}{ax} &= \phi\left( \left( ax \right)^{\ast}ax \right)\\
                   &\leq \phi\left( \norm{a}^2 x^{\ast}x \right)\\
                   &= 0,
  \end{align*}
  meaning that $ax\in N_{\rho}$. Furthermore, we see that
  \begin{align*}
    \norm{a\left( b + N_{\rho} \right)}^2 &= \rho\left( b^{\ast}a^{\ast}ab \right)\\
                                          &\leq \norm{a}^2\rho\left( b^{\ast}b \right)\\
                                          &= \norm{a}^2\norm{b + N_{\rho}}^2
  \end{align*}
  Therefore, we may uniquely extend elements of $a$ to bounded operators on $H_{\rho}$, which defines a representation $\pi_{\rho}\colon A\rightarrow B\left( H_{\rho} \right)$.

  Now, if $A$ is unital, we observe that $1 + N_{\rho}$ is cyclic for $\pi_{\rho}$, as
  \begin{align*}
    \left[ \pi_{\rho}(A)\xi_{\rho} \right] &= \overline{\set{\pi_{\rho}(a)\xi_{\rho} | a\in A}}\\
                                           &= \overline{\set{a\left( 1 + N_{\rho} \right) | a\in A}}\\
                                           &= \overline{A/N_{\rho}}\\
                                           &= H_{\rho},
  \end{align*}
  and we observe that
  \begin{align*}
    \iprod{\pi_{\rho}(a)\xi_{\rho}}{\xi_{\rho}} &= \iprod{a\left( 1 + N_{\rho} \right)}{1 + N_{\rho}}\\
                                                &= \iprod{a + N_{\rho}}{1 + N_{\rho}}\\
                                                &= \rho\left( a \right).
  \end{align*}
  Meanwhile, if $A$ is not unital, then we may extend $\rho$ to a state $\tau$ on the unitization $ \widetilde{A} $, which induces an isometry $V$ of $H_{\rho}$ to $H_{\tau}$ mapping $a + N_{\rho}$ to $a + N_{\tau}$ that intertwines $\pi_{\rho}$ and $\pi_{\tau}$, in the sense that $V\pi_{\rho}(a) = \pi_{\tau}(a)V$. We may identify $H_{\rho}$ with the subspace $VH_{\rho}\subseteq H_{\tau}$. This gives that $\pi_{\tau}|_{A} = \pi_{\rho}\oplus 0$ in $H_{\rho}\oplus H_{\rho}^{\perp}$. The projection of $1 + N_{\tau}$ onto $H_{\rho}$, which we may denote $h_{\rho}$, then satisfies
  \begin{align*}
    \pi_{\rho}(a)h_{\rho} &= \pi_{\tau}(a)\left( 1 + N_{\tau} \right)\\
                          &= a + N_{\tau},
  \end{align*}
  meaning that $h_{\rho}$ is cyclic for $\pi_{\rho}$, with
  \begin{align*}
    \iprod{\pi_{\rho}(a)h_{\rho}}{h_{\rho}} &= \iprod{\pi_{\tau}(a)\left( 1 + N_{\tau} \right)}{1 + N_{\tau}}\\
                                            &= \tau(a)\\
                                            &= \rho(a).
  \end{align*}
  Now, for the converse, we see that the linear functional $\tau$ defined by
  \begin{align*}
    \tau(a) &= \iprod{\pi(a)\xi}{\xi}
  \end{align*}
  is positive with norm at most $1$; in fact, since for any approximate identity we have $\pi\left( e_i \right)\xi\rightarrow \xi$, it follows that $\tau$ in fact has norm $1$. We have that
  \begin{align*}
    N_{\tau} &= \set{a\in A | \iprod{\pi\left( a^{\ast}a \right)\xi}{\xi}}\\
             &= \set{a\in A | \pi(a)\xi = 0},
  \end{align*}
  so there is a well-defined linear map $U_0\colon A/N_{\tau}\rightarrow H_{\pi}$ defined by $U_0\left( a + N_{\tau} \right) = \pi(a)\xi$, which is isometric, since
  \begin{align*}
    \iprod{U_0\left( a + N_{\tau} \right)}{U_0\left( b + N_{\tau} \right)} &= \iprod{\pi\left( b^{\ast}a \right)\xi}{\xi}\\
                                                                           &= \tau\left( b^{\ast}a \right)\\
                                                                           &= \iprod{a + N_{\tau}}{b + N_{\tau}},
  \end{align*}
  meaning that $U_0$ extends to an isometric linear map $U$ on $H_{\tau}$ which surjects onto $\left[ \pi(A)\xi \right] = H_{\pi}$ since $\xi$ is cyclic. Therefore, $U$ is unitary, and we have
  \begin{align*}
    U\pi_{\tau}(a)\left( b + N_{\tau} \right) &= U\left( ab + N_{\tau} \right)\\
                                              &= \pi\left( ab \right)\xi\\
                                              &= \pi(a)\pi(b)\xi\\
                                              &= \pi(a)U\left( b + N_{\tau} \right).
  \end{align*}
  Therefore, $U\pi_{\tau}(a)U^{-1} = \pi(a)$.
\end{proof}
\begin{corollary}
  If $\pi$ is an irreducible representation of a $C^{\ast}$-algebra $A$, and $\xi\in H_{\pi}$ is any unit vector, then $\pi$ is unitarily equivalent to the GNS representation $\pi_{\tau}$, where $\tau$ is the vector state $a\mapsto \iprod{\pi(a)\xi}{\xi}$.
\end{corollary}
\begin{proof}
  We observe that $K = \left[ \pi(A)\xi \right]$ is invariant under $\pi$, so by irreducibility we have that $K$  is either $0$ or $H_{\pi}$. Since $\pi$ is non-degenerate, it follows that $\pi\left( e_i \right)\xi \rightarrow \xi$ for some approximate unit $\left( e_i \right)_i$, so $\xi$ is a nonzero vector in $K$ and $K$ is all of $H_{\pi}$.
\end{proof}
\subsection{Representations and the Extremal Structure of the State Space}%
The state space, $S(A)$, can be seen to be convex, as if $A$ is unital with $\phi,\psi\in S(A)$, then
\begin{align*}
  \left( \left( 1-t \right)\phi + t\psi \right)\left( 1 \right) &= \left( 1-t \right)\phi(1) + t\psi(1)\\
                                                                &= 1.
\end{align*}
Furthermore, by taking a net $\left( \phi_i \right)_{i\in I}\subseteq S(A)$, we see that the state space is $w^{\ast}$-closed. Thus, from the \href{https://ai.avinash-iyer.com/Classes_and_Homework/After\%20College/Other\%20Notes/extreme_points_krein_milman_and_stone_weierstrass.pdf}{Krein--Milman Theorem}, it follows that the state space is equal to the $w^{\ast}$-closure of the convex hull of the extreme points of $S(A)$. The extreme points are known as \textit{pure states}, and they have a relationship with representations of $A$.
\begin{theorem}
  Let $\rho\in S(A)$. Then, the GNS representation $\pi_{\rho}$ is irreducible if and only if $\rho$ is a pure state.
\end{theorem}
\begin{proof}
  Let $\pi\coloneq \pi_{\rho}$ be non-irreducible. That is, there is an invariant subspace $K\subseteq H_{\rho}$ with corresponding projection $P$ such that $P,I-P\neq 0$. We will write $\rho$ as a nontrivial convex combination of states.

  Since $K$ is invariant, $\pi(a)P = P\pi(a) = P\pi(a)P$ for all $a\in A$. Since $\xi_{\rho}$ is cyclic, it follows that $\norm{P\xi_{\rho}}\neq 0$, as otherwise we would have
  \begin{align*}
    \pi(A)P\xi_{\rho} &= P\pi(A)\xi_{\rho}\\
                      &= PH_{\rho}\\
                      &= 0.
  \end{align*}
  Similarly, we have $\norm{\left( 1-P \right)\xi_{\rho}}\neq 0$, meaning that 
  \begin{align*}
    \phi(a) &\coloneq \frac{1}{\norm{P\xi_{\rho}}^2} \iprod{\pi(a)P\xi_{\rho}}{P\xi_{\rho}}\\
    \psi(a) &\coloneq \frac{1}{\norm{\left( 1-P \right)\xi_{\rho}}^2} \iprod{\pi(a)\left( 1-P \right)\xi_{\rho}}{\left( 1-P \right)\xi_{\rho}}
  \end{align*}
  define states for $A$ with $\rho(a) = \left( \lambda \right)\phi(a) + \left( 1-\lambda \right)\psi(a)$, where $\lambda = \norm{P\xi_{\rho}}^2$, and $0 < \lambda < 1$. Now, if we had $\rho = \phi$, then $P\pi(a)P = \pi(a)P$ would imply that
  \begin{align*}
    \iprod{\pi(a)\xi_{\rho}}{\xi_{\rho}} &= \frac{1}{\norm{P\xi_{\rho}}^2} \iprod{\pi(a)\xi_{\rho}}{P\xi_{\rho}}
  \end{align*}
  for all $a\in A$, but this is only possible if $\norm{P\xi_{\rho}}^2\xi_{\rho} = P\xi_{\rho}$, implying that $\norm{P\xi_{\rho}}^2P\xi_{\rho} = P\xi_{\rho}$, meaning $\norm{P\xi_{\rho}}^2 = 1$, contradicting $0 < \lambda < 1$. Similarly, $\rho\neq \psi$, $\rho$ thus admits a nontrivial convex decomposition, meaning $\rho$ is not an extreme point of $S(A)$.

  Now, suppose $\pi = \pi_{\rho}$ is irreducible, with $\rho(a) = \iprod{\pi(a)\xi}{\xi}$ for some $\xi\in H_{\pi}$. Suppose $\rho = \lambda\phi + \left( 1-\lambda \right)\psi$ for some states $\phi,\psi$ and some $0 < \lambda < 1$. Since $\psi$ and $\phi$ are positive, we have that
  \begin{align*}
    N_{\rho} &= \set{a\in A | \rho\left( a^{\ast}a \right) = 0}\\
             &\subseteq N_{\phi}\\
             &= \set{a\in A | \phi\left( a^{\ast}a \right) = 0}.
  \end{align*}
  Since $\pi(a)h = \pi(b)h$ whenever $a-b\in N_{\rho}$, the Cauchy--Schwarz inequality implies that the sesquilinear form
  \begin{align*}
    \left( \pi(a)h,\pi(b)h \right) &= \lambda\phi\left( b^{\ast}a \right)
  \end{align*}
  is well-defined on the dense subspace $\pi(A)h\subseteq H_{\pi}$. By the polarization identity, $\left( \cdot,\cdot \right)$ is bounded on $\pi(A)h$, so it can be extended to a bounded sesquilinear form $q$ on $H_{\pi}$, meaning there is some bounded operator $T\in B\left(H_{\pi}\right)$ such that $q\left( h,k \right) = \iprod{h}{Tk}$. In particular, we have
  \begin{align*}
    \iprod{\pi(a)h}{T\pi(b)h} &= \left( \pi(a)h,\pi(b)h \right)\\
                              &= \lambda \phi\left( b^{\ast}a \right).
  \end{align*}
  Since $\phi$ is positive and $\pi(A)h$ is dense, it follows that $T$ is a positive operator with norm at most $1$. Now, we claim that $T$ commutes with $\pi(A)$. If $a,b,c\in A$, then
  \begin{align*}
    \iprod{\pi(a)h}{T\pi(c)\pi(b)h} &= \lambda\phi\left( \left( cb \right)^{\ast}a \right)\\
                                    &= \lambda\phi\left( b^{\ast}\left( c^{\ast}a \right) \right)\\
                                    &= \iprod{\pi\left( c^{\ast}a \right)h}{T\pi(b)h}\\
                                    &= \iprod{\pi(a)h}{\pi(c)T\pi(b)h},
  \end{align*}
  so since $\pi(A)h$ is dense, it follows that $\pi(c)T = T\pi(c)$. Since $\pi$ is irreducible and $T$ is positive, it follows that there is some $z \geq 0$ such that $T = zI$. For any approximate identity $\left( e_i \right)_{i\in I}$ for $A$ and all $a\in A$, it then follows that
  \begin{align*}
    \lambda\phi(a) &= \lim_{i\in I}\lambda\phi\left( e_ia \right)\\
                   &= \lim_{i} \iprod{\pi(a)h}{T\pi\left( e_i \right)h}\\
                   &= z \iprod{\pi(a)h}{h}\\
                   &= z\rho(a),
  \end{align*}
  meaning that $\lambda = z$, so $\phi = \rho$.
\end{proof}
\begin{remark}
  The operator $T$ in the forward direction of the proof is akin to a non-commutative Radon--Nikodym derivative.
\end{remark}
\begin{lemma}
  For any $a\in A$, there is a \textit{pure} state on $A$ such that $\rho\left( a^{\ast}a \right) = \norm{a}^2$.
\end{lemma}
\begin{proof}
  Let $\Sigma$ be the set of states satisfying $\rho\left( a^{\ast}a \right) = \norm{a}^2$. This is nonempty as we established existence earlier, and the collection of all such $\rho$ form a $w^{\ast}$-compact convex subset of $A^{\ast}$. By Krein--Milman, there is an extreme point $\rho$ of $\Sigma$. We claim that $\rho$ is a pure state.

  Let $\phi,\psi\in S(A)$, $0 < \lambda < 1$, and $\rho = \lambda\phi + \left( 1-\lambda \right)\psi$. Then, we have
  \begin{align*}
    \rho\left( a^{\ast}a \right) &= \lambda \phi\left( a^{\ast}a \right) + \left( 1-\lambda \right)\psi\left( a^{\ast}a \right)\\
                                 &\leq \norm{a}^2\\
                                 &= \rho\left( a^{\ast}a \right),
  \end{align*}
  so the inequality is equality, meaning that $\phi\left( a^{\ast}a \right) = \norm{a}^2 = \psi\left( a^{\ast}a \right)$. Then, $\phi$ and $\psi$ are in $\Sigma$, and since $\rho$ is extreme in $\Sigma$, we have that $\rho = \phi = \psi$.
\end{proof}
\begin{theorem}
  Let $A$ be a $C^{\ast}$-algebra. Then, for each $a\in A$, there is an irreducible representation $\pi$ of $A$ with $\norm{\pi(a)} = \norm{a}$.
\end{theorem}
\begin{proof}
  Choose $\rho$ as in the previous lemma. Then, $\pi_{\rho}$ is irreducible, and
  \begin{align*}
    \norm{a}^2 &= \rho\left( a^{\ast}a \right)\\
               &= \iprod{\pi_{\rho}\left( a^{\ast}a \right)\xi_{\rho}}{\xi_{\rho}}\\
               &= \norm{\pi_{\rho}(a)\xi_{\rho}}^2\\
               &\leq \norm{\pi_{\rho}(a)}^2,
  \end{align*}
  which combined with the fact that $\ast$-homomorphisms are contractive, gives $\norm{\pi_{\rho}(a)} = \norm{a}$.
\end{proof}
\subsection{The Universal Representation}%
To progress further, we must ensure that the state space separates the points of a $C^{\ast}$-algebra, so that we may use the states of a $C^{\ast}$-algebra to construct the universal representation.
\begin{lemma}
  Let $A$ be a $C^{\ast}$-algebra, and let $a\in A_{\sa}$. Then, there is a state $\rho$ of $A$ such that $\left\vert \rho(a) \right\vert = \norm{a}$. In particular, for any $a\in A$, there is a state $\rho$ such that $\rho\left( a^{\ast}a \right) = \norm{a}^2$.
\end{lemma}
\begin{proof}
  We may assume that $A$ has a unit. If we let $B = C^{\ast}\left( a \right)$, then there is an isometric isomorphism from $C^{\ast}(a) \cong C(\hat{B})$, where $\hat{B}$ is the character space of $B$. In particular, since $\hat{a}$ is a continuous map on a compact space, there is some $\phi\in \hat{B}$ such that
  \begin{align*}
    \left\vert \phi(a) \right\vert &= \left\vert \hat{a}(\phi) \right\vert\\
                                   &= \norm{\hat{a}}\\
                                   &= \norm{a}.
  \end{align*}
  Therefore, there is some $\rho\in A^{\ast}$ such that $\rho|_{B} = \phi$ and $\norm{\rho} = \norm{\phi} = 1$. Since $\phi(1) = 1$, we have $\rho(1) = 1$, so $\rho$ is a state with $\left\vert \rho(a) \right\vert = \norm{a}$. Thus, $\rho$ is a state on $A$ with the desired property.
\end{proof}
\begin{proposition}
  Every $C^{\ast}$-algebra $A$ has a faithful non-degenerate representation.
\end{proposition}
\begin{proof}
  For each $a\neq 0$ in $A$, let $\rho_{a}$ be a state such that $\rho_a\left( a^{\ast}a \right)= \norm{a}^2$. Let $\pi_{\rho_a}$ be the corresponding GNS representation with cyclic vector $\xi_{\rho_a}$. Then, $\pi_{\rho_a}(a)\neq 0$, as
  \begin{align*}
    0 &< \norm{a}^2\\
      &= \rho_a\left( a^{\ast}a \right)\\
      &= \iprod{\pi_{\rho_a}\left( a^{\ast}a \right)\xi_{\rho_a}}{\xi_{\rho_a}}\\
      &= \norm{\pi_{\rho_a}(a)\xi_{\rho_a}}^2.
  \end{align*}
  Therefore, the representation
  \begin{align*}
    \pi\coloneq \bigoplus_{\substack{a\in A\\a\neq 0}}\pi_{\rho_a}
  \end{align*}
  is a nondegenerate faithful representation.
\end{proof}
\begin{definition}
  The \textit{universal representation} for $A$ is the pair $\left( H_u,\pi_u \right)$, where
  \begin{align*}
    H_u &= \bigoplus_{\rho\in S(A)}H_{\rho}\\
    \pi_u &= \bigoplus_{\rho\in S(A)} \pi_{\rho}.
  \end{align*}
\end{definition}
We observe that, as we showed in the lemma, the state space of $A$ separates the points of $A$, so this is in fact a faithful representation.

We will have more to discuss about the universal representation in the notes on von Neumann algebras.
\nocite{conway_operator_theory,kadison_and_ringrose_1,morita_equivalence_cstar_algebras,murphy_cstar_algebras_and_operator_theory}
\printbibliography
\end{document}
