\documentclass[10pt]{mypackage}

\usepackage{mlmodern}
%\usepackage{newpxtext,eulerpx,eucal}
%\renewcommand*{\mathbb}[1]{\varmathbb{#1}}

%\usepackage{homework}
\usepackage{notes}

\usepackage[ backend=bibtex, style = alphabetic, sorting=ynt ]{biblatex}
\addbibresource{all_references.bib}

\usepackage{parskip}

\fancyhf{}
\fancyhead[R]{Avinash Iyer}
\fancyhead[L]{States and Representations of $C^{\ast}$-Algebras}
\fancyfoot[C]{\thepage}

\setcounter{secnumdepth}{0}

\begin{document}
\RaggedRight
\section{Representations}%
\begin{definition}
  If $A$ is a $C^{\ast}$-algebra, a representation of $A$ is a pair $\left( \pi,H \right)$ where $H$ is a Hilbert space and $\pi\colon A\rightarrow B(H)$ is a $\ast$-homomorphism. If $A$ is unital, then we require $\pi(1) = I$.
\end{definition}
Note that if $A$ does not have an identity, we can extend to the unitization $A_1 = A\oplus \C$ and define $ \widetilde{\pi}\left( a,\lambda \right) = \pi(a) + \lambda I $ for any $a\in A$ and $\lambda\in \C$.

Note that every representation is contractive and the range of any representation is closed.
\begin{example}\hfill
  \begin{enumerate}[(a)]
    \item If $A$ is a $C^{\ast}$-subalgebra of $B(H)$, then the inclusion map $A\hookrightarrow B(H)$ is a representation.
    \item If $\left( X,\Omega,\mu \right)$ is a $\sigma$-finite measure space, then $\pi\colon L_{\infty}(\mu) = B\left( L_2\left( \mu \right) \right)$, where $\pi(\phi) = M_{\phi}$, is a representation.
    \item If $X$ is compact, and $\mu$ is a positive Borel measure on $X$, then $\pi_{\mu}\colon C(X)\rightarrow B\left( L_2\left( \mu \right) \right)$ defined by $\pi_{\mu}(f) = M_f$ is a representation of $C(X)$.
  \end{enumerate}
\end{example}
\begin{definition}
  Let $A$ be a $C^{\ast}$-algebra.
  \begin{enumerate}[(i)]
    \item If $d$ is a cardinal number, $H$ a Hilbert space, we let $H^{(d)}$ denote the direct sum of $H$ with itself over $d$. If $T\in B(H)$, we let $T^{(d)}$ be the direct sum of $T$ with itself over $d$, which is known as the $d$-fold inflation of $T$.

      Given a representation $\pi\colon A\rightarrow B(H)$, we have $\pi^{(d)} \colon A\rightarrow B\left( H^{(d)} \right)$, defined by $\pi^{(d)}(a) = \pi(a)^{(d)}$ is a representation, which is known as the inflation of $\pi$.

      If $d = \aleph_{0}$, we will denote their respective inflations as $H^{(\infty)}$ and $\pi^{(\infty)}$.
    \item If $\set{\left( \pi_i,H_i \right)}_{i\in I}$ is a collection of representations of $A$, the direct sum of these representations is the representation
      \begin{align*}
        \bigoplus_{i\in I}\pi_i\colon A &\rightarrow B\left( \bigoplus_{i\in I}H_i \right)\\
        a &\mapsto \bigoplus_{i\in I}\pi_i(a).
      \end{align*}
  \end{enumerate}
\end{definition}
Note that since all representations are contractive, the direct sum is in fact a bounded operator. Furthermore, if $\pi$ is isometric (hence injective), then so too is its inflation.
\begin{example}
  If $X$ is a compact topological space, and $\left( \mu_n \right)_n$ is a sequence of positive Borel measures for $X$, with corresponding representations $\pi_n\colon C(X)\rightarrow B\left( L_2\left( \mu_n \right) \right)$ taking $f\mapsto M_f$, then $\bigoplus_{n\geq 1}\pi_n$ is also a representation.
\end{example}
\begin{definition}
  Two representations $\left( \pi_1,H_1 \right)$ and $\left( \pi_2,H_2 \right)$ are called equivalent if there is a unitary $U\colon H_1\rightarrow H_2$ such that $\pi_2(a) = U\pi_1(a)U^{-1}$. 
\end{definition}

\section{States}%
For now, we will assume that $M$ is a unital self-adjoint subspace of a $C^{\ast}$-algebra $A$. If $\rho$ is a linear functional on $M$, then the equation
\begin{align*}
  \rho^{\ast}\left( a \right) &= \overline{\rho\left( a^{\ast} \right)}
\end{align*}
defines another linear functional; if $\rho = \rho^{\ast}$, then we call $\rho$ hermitian. Equivalently, $\rho$ is hermitian if $\rho\left( a^{\ast} \right) = \overline{\rho(a)}$. If $\rho$ is a bounded hermitian functional on $M$, then we claim that
\begin{align*}
  \norm{\rho} &= \sup\set{\rho(a) | a\in M_{\sa},\norm{a}\leq 1}.
\end{align*}
This follows from the fact that if $\ve > 0$, then from the Riesz lemma, we may find $a$ in the unit ball of $M$ with $\left\vert \rho(a) \right\vert > \norm{\rho} - \ve$. For a suitable $\lambda$ with $\left\vert \lambda \right\vert = 1$, we have
\begin{align*}
  \norm{\rho} - \ve &< \left\vert \rho(a) \right\vert\\
                    &= \rho\left( \lambda a \right)\\
                    &= \overline{\rho\left( \lambda a \right)}\\
                    &= \rho\left( \left( \lambda a \right)^{\ast} \right).
\end{align*}
If $a_0 = \re\left( \lambda a \right)$, we have $\norm{a_0}\leq 1$ with $\rho\left( a_0 \right) > \norm{\rho} - \ve$. Thus,
\begin{align*}
  \norm{\rho} &\leq \sup\set{\rho(a) | a \in M_{\sa},\norm{a}\leq 1},
\end{align*}
with the reverse inequality being true by definition.

We say the linear functional $\rho$ is \textit{positive} if for any $a\in M_{+}$, $\rho(a)\geq 0$; if $\rho(1) = 1$, then we say $\rho$ is a state. In fact, if $\rho$ is positive, then $\rho$ is hermitian, since if $a = a^{\ast}$, then 

We start by considering a version of the Cauchy--Schwarz inequality for states.
\begin{proposition}
  If $\rho$ is a positive linear functional on a $C^{\ast}$-algebra $A$, then
  \begin{align*}
    \left\vert \rho(b^{\ast}a) \right\vert^2 &\leq \rho\left( a^{\ast}a \right)\rho\left( b^{\ast}b \right).
  \end{align*}
\end{proposition}
\begin{proof}
  With $a\in A$, we have $a^{\ast}a\in A_{+}$, so $\rho\left( a^{\ast}a \right)\geq 0$. Then, since $\rho$ is hermitian, we have that
  \begin{align*}
    \iprod{a}{b} &= \rho\left( b^{\ast}a \right)
  \end{align*}
  defines a positive sesquilinear form on $A$, so the traditional Cauchy--Schwarz inequality gives the desired result.
\end{proof}

\nocite{conway_operator_theory,kadison_and_ringrose_1,morita_equivalence_cstar_algebras}
\printbibliography
\end{document}
