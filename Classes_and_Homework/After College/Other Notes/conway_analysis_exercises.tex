\documentclass[10pt]{mypackage}

\usepackage{mlmodern}
%\usepackage{newpxtext,eulerpx,eucal}
%\renewcommand*{\mathbb}[1]{\varmathbb{#1}}

\usepackage{homework}
\usepackage{notes}

\usepackage[ backend=bibtex, style = alphabetic, sorting=ynt ]{biblatex}
\addbibresource{all_references.bib}

\usepackage{parskip}

\fancyhf{}
\fancyhead[R]{Avinash Iyer}
\fancyhead[L]{\textit{A Course in Functional Analysis}: Exercises and Solutions}
\fancyfoot[C]{\thepage}

\setcounter{secnumdepth}{0}

\begin{document}
\RaggedRight
These are assorted exercises and solutions from Conway's \textit{A Course in Functional Analysis}.
\section{Spectral Theory}%

\begin{problem}[{\cite[Exercise IX.1.2]{conway_functional_analysis}}]
  Show that the unit ball of $B(H)$ is WOT-compact.
\end{problem}
\begin{solution}
  Consider the set
  \begin{align*}
    K &= \prod_{x,y\in B_{H}} \overline{\D},
  \end{align*}
  where $\D$ represents the complex unit disk and $B_H$ denotes the closed unit ball of $H$. The space $K$ is compact by Tychonoff's theorem. Let $\phi\colon B_{B(H)}\rightarrow K$ be defined by
  \begin{align*}
    \phi(T) &= \left( \iprod{Tx}{y} \right)_{x,y}.
  \end{align*}
  Observe that by Cauchy--Schwarz, we have that
  \begin{align*}
    \left\vert \iprod{Tx}{y} \right\vert &\leq \norm{T}_{\op}\norm{x}\norm{y}\\
                                         &\leq 1,
  \end{align*}
  so $\phi$ is indeed well-defined. Furthermore, $\phi$ is injective since for any two operators $T$ and $S$, we have that $T = S$ if and only if $ \iprod{Tx}{y} = \iprod{Sx}{y} $ for all $x,y\in B_{H}$, and $\phi$ is continuous by the definition of the weak operator topology. Therefore, we only need to show that $\phi$ has a closed range.

  Let $\left( T_i \right)_i$ be a net of operators in $B_{B(H)}$ such that
  \begin{align*}
    \lim_{i\in I}\left( \iprod{T_i x}{y} \right)_{x,y} &= \left( z_{x,y} \right)_{x,y}.
  \end{align*}
  Then, from the Cauchy--Schwarz inequality, it follows that $\left( z_{x,y} \right)_{x,y}\in K$, and by the definition of convergence in the product topology, we have, for each $x,y$,
  \begin{align*}
    \lim_{i\in I} \iprod{T_i x}{y} &= z_{x,y}.
  \end{align*}
  Therefore, we may define a semidefinite sesquilinear form $F\colon H\times H\rightarrow \C$ given by
  \begin{align*}
    F(x,y) &= \lim_{i\in I} \iprod{T_i x}{y}
  \end{align*}
  for each $x,y\in H$. From the structure of sesquilinear forms, it then follows that there is some $T\in B_{B(H)}$ such that $F(x,y) = \iprod{Tx}{y}$, and thus $\left( T_i \right)_{i\in I}\rightarrow T$ in WOT.
\end{solution}
\begin{problem}[{\cite[Exercise IX.1.13]{conway_functional_analysis}}]
  A representation $\rho\colon A\rightarrow B(H)$ is \textit{irreducible} if the only projections in $B(H)$ that commute with every $\rho(a)$ for $a\in A$ are $0$ and $1$. Prove that if $A$ is abelian and $\rho$ is an irreducible representation, then $\dim(H) = 1$. Find the corresponding spectral measure.
\end{problem}
\begin{solution}
  Since $A$ is abelian, so too is $\rho(A)$, meaning that $\rho(A)\subseteq \rho(A)'$. Since $\rho(A)' = \C 1$ by the assumption of irreducibility, it follows then that $\rho(A) = \C 1$, whence $H = [\rho(A) v] = \C v$.

  Without loss of generality, we may assume that $A = C_0(X)$ for some locally compact Hausdorff space $X$, and $\rho\colon C_0(X)\rightarrow \C$ is a character. The characters of $C_0(X)$ are given by evaluation at $x_0\in X$, meaning that their corresponding spectral measure is the Dirac mass $\delta_{x_0}$.
\end{solution}
\begin{problem}[{\cite[Exercise IX.2.1]{conway_functional_analysis}}]
  Show that $\lambda\in \sigma_p(N)$ if and only if $E\left( \set{\lambda} \right) \neq 0$. Moreover, if $\lambda\in \sigma_P(N)$, then $E\left(\set{\lambda}\right)$ is the orthogonal projection onto $\ker\left( N-\lambda I \right)$.
\end{problem}
\begin{solution}
  Suppose $E\left(\set{\lambda}\right)\neq 0$, meaning that
  \begin{align*}
    E(\set{\lambda}) &= \int_{\sigma(N)}^{} \1_{\set{\lambda}}\:dE\\
                     &\neq 0.
  \end{align*}
  Since, for all $x\in H$, we have
  \begin{align*}
    \left( N-\lambda I \right)E\left(\set{\lambda}\right) x &= \left( \int_{\sigma(N)}^{} \left( z-\lambda \right)\1_{\set{\lambda}}\:dE \right)x\\
                                                            &= E(\set{\lambda}) \left( N-\lambda I \right) x\\
                                                            &= 0,
  \end{align*}
  it follows that $E(\set{\lambda}) x\in \ker\left( N-\lambda I \right)$, so that $E(\set{\lambda})\leq P_{\lambda}$, where $P_{\lambda}$ is the projection onto $\ker\left( N-\lambda I \right)$. Since $E(\set{\lambda}) > 0$, it follows that $P_{\lambda} > 0$, so $P_{\lambda}$ is nontrivial, meaning $\ker\left( N-\lambda I \right)$ is nontrivial, so $\lambda\in \sigma_p(N)$.

  Now, let $\lambda\in \sigma_p(N)$. We start by supposing that $\lambda\neq 0$. If $x\in \ker\left( T-\lambda I \right)$ is nonzero, then we have
  \begin{align*}
    Tx &= \lambda x\\
       &= \left( \int_{\sigma(N)}^{} \lambda\:dE \right)x\\
       &= \left( \int_{\sigma(N)}^{} \1_{\set{\lambda}}z\:dE \right)x\\
       &= E(\set{\lambda}) Tx\\
       &= \lambda E(\set{\lambda})x,
  \end{align*}
  so $E(\set{\lambda})x = x$, meaning that $E(\set{\lambda}) \geq P_{\lambda}$. In particular, from what we have established above, this means that $E(\set{\lambda}) = P_{\lambda}$, so $E(\set{\lambda})\neq 0$. If $\lambda = 0$, then we shift $N$ by subtracting a factor of $I$, perform the same process, and shift back.
\end{solution}
\begin{problem}[{\cite[Exercise IX.2.10]{conway_functional_analysis}}]
  Let $A$ be a hermitian operator with spectral measure $E$ on a separable space. For each real number $t$, define a projection $P(t) = E\left( -\infty,t \right)$. Show:
  \begin{enumerate}[(a)]
    \item $P(s)\leq P(t)$ for $s\leq t$;
    \item if $t_n\leq t_{n+1}$ and $\left( t_n \right)_n\rightarrow t$, then $P\left( t_n \right)\rightarrow P(t)$ in SOT;
    \item for all but a countable number of points $t$, $P\left( t_n \right)\rightarrow P(t)$ in SOT if $\left( t_n \right)_n\rightarrow t$;
    \item for any $f\in C(\sigma(A))$, we have
      \begin{align*}
        f(A) &= \int_{-\infty}^{\infty} f(t)\:dP(t),
      \end{align*}
      where the integral is defined in the Riemann--Stieltjes sense.
  \end{enumerate}
\end{problem}
\begin{solution}
  Let $x\in H$. Then, we observe that
  \begin{align*}
    \iprod{P(s)x}{x} &= \int_{\sigma(A)}^{} \1_{\left( -\infty,s \right)}\:dE_{x,x}\\
                     &= \int_{\sigma(A)\cap \left( -\infty,s \right)}^{} \:dE_{x,x}\\
                     &\leq \int_{\sigma(A)\cap \left( -\infty,t \right)}^{} \:dE_{x,x}\\
                     &= \iprod{P(t)x}{x}.
  \end{align*}
  Since $x$ is arbitrary, and the measure $E_{x,x}$ is real by the assumption that $A$ is hermitian, it follows that $P(s)\leq P(t)$ whenever $s\leq t$. This shows (a).

  To show (b), we recall that a sequence of projections $\left( P_n \right)_n\rightarrow P$ in SOT if and only if it converges in WOT, as
  \begin{align*}
    \iprod{\left( P_n-P \right)x}{x} &= \iprod{\left( P_n-P \right)^2x}{x}\\
                                     &= \iprod{\left( P_n-P \right)x}{\left( P_n-P \right)x}\\
                                     &= \norm{\left( P_n-P \right)x}^2.
  \end{align*}
  We thus see that if $\left( t_n \right)_n \nearrow t$, we have for any $x\in H$,
  \begin{align*}
    \left\vert \iprod{\left( P\left( t_n \right)-P(t) \right)x}{x} \right\vert &= \left\vert \int_{\sigma(A)}^{} \1_{(-\infty,t_n)} - \1_{(-\infty,t)}\:dE_{x,x} \right\vert\\
                                                        &= \left\vert E_{x,x}\left(\sigma(A)\cap (-\infty,t_n)\right) - E_{x,x}\left( \sigma(A)\cap \left( -\infty,t \right) \right) \right\vert,
  \end{align*}
  which by continuity from below for measures converges to zero. Thus, $\left( P\left( t_n \right) \right)_n\rightarrow P(t)$ in WOT, so it converges in SOT.

  To establish (c), we fix $x\in H$. Then, the function
  \begin{align*}
    f(t) &= \iprod{P(t)x}{x}
  \end{align*}
  is a monotone increasing right-continuous function on $\R$. In particular, this means that $f(t)$ has at most a countable number of discontinuities, meaning that for any sequence $\left( t_n \right)_n\rightarrow t$, it follows that $P\left( t_n \right)\rightarrow P(t)$ in WOT, hence in SOT, everywhere outside these countable number of discontinuities.
\end{solution}
\begin{problem}[{\cite[Exercise IX.2.14]{conway_functional_analysis}}]
  Prove that if $A$ is hermitian, $\exp(iA)$ is unitary. Is the converse true?
\end{problem}
\begin{solution}
  Since $\sigma(A)\subseteq \R$, we have
  \begin{align*}
    \exp(iA) &= \int_{\sigma(A)}^{} e^{ix}\:dE\\
    \exp\left( iA \right)^{\ast} &= \int_{\sigma(A)}^{} e^{-ix}\:dE\\
    \exp\left( iA \right)\exp\left( iA \right)^{\ast} &= \int_{\sigma(A)}^{} \:dE\\
                                                      &= \exp\left( iA \right)^{\ast}\exp\left( iA \right),
  \end{align*}
  so that $\exp\left( iA \right)$ is unitary.

  Similarly, there is a continuous bijection $f\colon [0,2\pi)\rightarrow S^{1}$ given by $t\mapsto e^{it}$, with Borel-measurable inverse $g$, so that if $V$ is any unitary operator, we may define
  \begin{align*}
    A &= g\left( V \right),
  \end{align*}
  which has $\exp\left( iA \right) = V$. Since $g$ is real-valued, it follows that $A$ is hermitian.
\end{solution}
\begin{problem}[{\cite[Exercise IX.2.22]{conway_functional_analysis}}]
  Prove that if $U$ is any unitary operator on $H$, then there is a continuous function $u\colon [0,1]\rightarrow B(H)$ such that $u(0) = U$, $u(1) = I$, and $u(t)$ is unitary for each $t$.
\end{problem}
\begin{solution}
  Since $U$ is unitary, there is a hermitian operator $A$ such that $U = \exp(iA)$. We may then define the continuous map
  \begin{align*}
    u\colon [0,1] &\rightarrow B(H)\\
    t &\mapsto \exp\left( i\left( 1-t \right)A \right).
  \end{align*}
  Since $\left( 1-t \right)A$ is also a hermitian operator, and dominated convergence gives that this is a continuous map, this is thus our desired map.
\end{solution}
\begin{problem}[{\cite[Exercise IX.2.23]{conway_functional_analysis}}]
  If $N$ is normal, show that there is a sequence of invertible normal operators that converges to $N$.
\end{problem}
\begin{solution}
  We observe that the sequence of functions
  \begin{align*}
    f_n(z) &= z\1_{\sigma(N)\setminus \set{0}} + \frac{1}{n}\1_{\set{0}}
  \end{align*}
  converges pointwise to $z$, is nonzero everywhere on $\sigma(N)$, and is bounded above by the necessarily integrable function
  \begin{align*}
    h(z) &= \left( \sup_{z\in \sigma(N)}\left\vert z \right\vert + 1 \right)\1_{\sigma(N)}.
  \end{align*}
  So, the operator
  \begin{align*}
    f_n(N) &= \int_{\sigma(N)}^{} f_n(z)\:dE
  \end{align*}
  is integrable for each $N$, with $f_n(N)\rightarrow N$ by dominated convergence.
\end{solution}
\printbibliography
\end{document}
