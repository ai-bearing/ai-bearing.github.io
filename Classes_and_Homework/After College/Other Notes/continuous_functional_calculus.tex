\documentclass[10pt]{mypackage}

% sans serif font:
%\usepackage{cmbright}
%\usepackage{sfmath}
%\usepackage{bbold} %better blackboard bold

%\usepackage{homework}
\usepackage{notes}
\usepackage{mlmodern}

\usepackage[ backend=bibtex,style=alphabetic,sorting=ynt ]{biblatex}
\addbibresource{all_references.bib}

%\usepackage{newpxtext,eulerpx,eucal}
%\renewcommand*{\mathbb}[1]{\varmathbb{#1}}
\usepackage{parskip}

\fancyhf{}
\fancyhead[R]{Avinash Iyer}
\fancyhead[L]{The Continuous Functional Calculus}
\fancyfoot[C]{\thepage}

\setcounter{secnumdepth}{0}

\begin{document}
\RaggedRight
\section{The Spectrum and Gelfand Transform}%
We first recall some definitions and results from the theory of Banach algebras.
\begin{definition}
  Let $A$ be a Banach algebra, $x\in A$. The spectrum of $x$ in $A$ is given by
  \begin{align*}
    \sigma_A(x) &= \set{\lambda\in \C | x-\lambda 1 \text{ is not invertible in }\widetilde{A}},
  \end{align*}
  where $ \widetilde{A} $ denotes the unitization of $A$. If $A$ does not have a unit, then $0\in \sigma_A(x)$ for all $x\in A$. The complement of the spectrum is known as the resolvent, and is denoted $\rho_A(x)$.
\end{definition}
\begin{proposition}
  If $A$ is a Banach algebra, then for all $a\in A$, we have that $\sigma(a)\subseteq \C$ is compact. Furthermore, we have $\sigma(A)\subseteq B\left( 0,\norm{a} \right)$.
\end{proposition}
\begin{definition}
  The \textit{spectral radius} of $a\in A$ is denoted $r(a)$, and is given by
  \begin{align*}
    r(a) &= \sup_{\lambda\in \sigma(a)} \left\vert \lambda \right\vert.
  \end{align*}
\end{definition}
\begin{theorem}[Gelfand--Mazur]
  The only complex Banach division algebra is $\C$.
\end{theorem}
\begin{theorem}
  Let $A$ be a Banach algebra, and let $a\in A$. Then,
  \begin{align*}
    r(a) &= \lim_{n\rightarrow\infty}\norm{a^n}^{1/n}.
  \end{align*}
\end{theorem}
\begin{corollary}
  If $A$ is a $C^{\ast}$-algebra and $a\in A$ is a normal element, then $r(a) = \norm{a}$.
\end{corollary}
\begin{definition}
  A \textit{character} on a commutative Banach algebra $A$ is a nonzero unital algebra homomorphism. We denote the set of all characters on $A$ by $\hat{A}$.
\end{definition}
\begin{theorem}
  Every character in on $\hat{A}$ corresponds to a maximal ideal $I\subseteq A$.
\end{theorem}
\begin{proposition}
  If $\phi\in \hat{A}$, then $\phi(x)\in \sigma(x)$, and conversely, if $\lambda\in \sigma(x)$, there is $\phi\in \hat{A}$ with $\phi(x) = \lambda$. In particular, for every $\phi\in \hat{A}$, we have $\left\vert \phi(x) \right\vert\leq \norm{x}$, and thus $\norm{\phi} = 1$.
\end{proposition}
\begin{proof}
  We have $\phi\left( x-\phi(x)1 \right) = 0$, so $x-\phi(x)1$ is not invertible. Conversely, if $I$ is a maximal ideal in $A$ containing the ideal generated by $x-\lambda 1$, we have $A/I\cong \C$.
\end{proof}
We observe that if $A$ is a commutative unital Banach algebra, then we can identify $\hat{A}$ with a closed subset of the unit ball of $A^{\ast}$, and so we can endow $\hat{A}$ with the $w^{\ast}$-topology, which yields that $\hat{A}$ is a compact Hausdorff space. If $A$ is not unital, then $\hat{A}$ is a locally compact Hausdorff space, and the one-point compactification of $\hat{A}$ is the character space of $\widetilde{A}$.

Then, in particular, $C( \hat{A} )$ is a commutative $C^{\ast}$-algebra, which is equipped with a natural homomorphism $\Gamma\colon x\mapsto \hat{x}$ given by $\hat{x}(\phi) = \phi(x)$. The Gelfand transform is in fact a $\ast$-homomorphism.
\begin{theorem}
  If $A$ is a commutative $C^{\ast}$-algebra, then the Gelfand transform is an isometric $\ast$-isomorphism from $A$ onto $C_0( \hat{A} )$.
\end{theorem}
\begin{proof}
  By passing to the unitization, we may assume that $A$ is unital. We only need to show that $\Gamma$ is isometric, since then the range will be closed, and hence equal to $C_0( \hat{A} )$ by the Stone--Weierstrass theorem.

  We know that there is $\lambda\in \sigma(x)$ such that $\left\vert \lambda \right\vert = \norm{x}$, meaning that we can find $\phi$ with $\left\vert \hat{x}(\phi) \right\vert = \norm{x}$, so that $\norm{\hat{x}}\geq \norm{x}$.
\end{proof}

It has been well established that if $X$ and $Y$ are compact Hausdorff spaces, and $\phi\colon X\rightarrow Y$ is a continuous map, then we can define a $\ast$-homomorphism $\hat{\phi}\colon C(Y)\rightarrow C(X)$ by taking $\hat{\phi}(f) = f\circ\phi$. This is a contravariant map, in that if $\phi$ is injective, then $\hat{\phi}$ is surjective, and vice versa.

Conversely, if $\psi\colon C(Y)\rightarrow C(X)$ is a $\ast$-homomorphism, and $x\in X\cong \widehat{C(X)}$, then $\delta_x\circ\psi\in \widehat{C(Y)}\cong Y$, where $\delta_x(f) = f(x)$, and $\check{\psi}(x) = \delta_x\circ\psi$ defines an inverse to $\hat{\phi}$.
\begin{corollary}
  Let $A$ and $B$ be $C^{\ast}$-algebras, $\phi\colon A\rightarrow B$ an injective $\ast$-homomorphism. Then, $\phi$ is isometric.
\end{corollary}
\begin{proof}
  By passing to the unitization (and using the fact that the injection into the unitization is isometric), we may assume without loss of generality that $\phi$ is isometric. If $a\in A$, then $x = a^{\ast}a$ is a normal element with a normal element as its image.

  By restricting $\phi$ to $P = C^{\ast}\left( x,1 \right)$ and taking $Q = \phi(C)$, then we observe that the map defined by $\pi = \Gamma_Q\circ \phi \circ \Gamma_P^{-1}$ is an injective homomorphism between $C( \hat{P} )$ and $C( \hat{Q} )$. Thus, since $\Gamma_C$ and $\Gamma_D$ are isometric, as well as $\pi$, we must have $\phi$ is isometric.
\end{proof}
\nocite{blackadar_operator_algebras,morita_equivalence_cstar_algebras}
\printbibliography 
\end{document}
