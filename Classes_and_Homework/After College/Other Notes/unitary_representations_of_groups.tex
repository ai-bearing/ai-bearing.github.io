\documentclass[10pt]{mypackage}

\usepackage{mlmodern}
%\usepackage{newpxtext,eulerpx,eucal}
%\renewcommand*{\mathbb}[1]{\varmathbb{#1}}

%\usepackage{homework}
\usepackage{notes}

\usepackage[ backend=bibtex, style = alphabetic, sorting=ynt ]{biblatex}
\addbibresource{all_references.bib}

\usepackage{parskip}

\fancyhf{}
\fancyhead[R]{Avinash Iyer}
\fancyhead[L]{Unitary Representations of Locally Compact Groups}
\fancyfoot[C]{\thepage}

\setcounter{secnumdepth}{0}

\begin{document}
\RaggedRight
\section{Quasi-Review: Locally Compact Groups and the Banach $\ast$-algebra $L_1(G)$}%
\subsection{Basic Properties of Topological Groups}%
A topological group is a group $G$ equipped with a topology such that the operations
\begin{align*}
  \left( x,y \right) &\mapsto xy\\
  x &\mapsto x^{-1}
\end{align*}
are continuous. In general, we will let $1$ denote the identity of $G$.

We call $G$ a locally compact group if the topology of $G$ is locally compact. Equivalently, the topology of $G$ is locally compact if there there is a neighborhood system about $1$ consisting of pre-compact open sets.

We will refer to the following subset operations in $G$ regularly:
\begin{align*}
  Ax &= \set{ax | a\in A}\\
  xA &= \set{xa | a\in A}\\
  A^{-1} &= \set{a^{-1} | a\in A}\\
  AB &= \set{ab | a\in A, b\in B}.
\end{align*}
A subset $V$ is called \textit{symmetric} if $V = V^{-1}$.

These are some useful propositions.
\begin{proposition}
  Let $G$ be a topological group.
  \begin{enumerate}[(i)]
    \item The topology of $G$ is invariant under translations and inversion.
    \item For every neighborhood $U$ of $1$, there is a symmetric neighborhood $V$ of $1$ such that $VV\subseteq U$.
    \item If $H$ is a subgroup of $G$, then so is $ \overline{H} $.
    \item Every open subgroup of $G$ is closed.
    \item If $A$ and $B$ are compact subsets of $G$, then so is $AB$.
  \end{enumerate}
\end{proposition}
\begin{proposition}
  Suppose $H$ is a subgroup of the topological group $G$.
  \begin{enumerate}[(i)]
    \item If $H$ is closed, then $G/H$ is Hausdorff.
    \item If $G$ is locally compact, then so is $G/H$.
    \item If $H$ is normal, then $G/H$ is a topological group.
  \end{enumerate}
\end{proposition}
We will assume all the time that $G$ is Hausdorff, via the following proposition.
\begin{corollary}
  If $G$ is a T1 topological group, then $G$ is Hausdorff. If $G$ is not T1, then $ \overline{\set{1}} $ is a closed normal subgroup with $G/ \overline{\set{1}}$ is a Hausdorff topological group.
\end{corollary}
\begin{proposition}
  Every locally compact group $G$ has a subgroup $G_0$ that is open, closed, and $\sigma$-compact.
\end{proposition}
Considering various functions $f\colon G\rightarrow \C$, we define the left and right translates of $f$ as
\begin{align*}
  L_yf(x) &= f\left( y^{-1}x \right)\\
  R_yf(x) &= f\left( xy \right),
\end{align*}
and say that $f$ is left (right) uniformly continuous if $\norm{L_yf-f}_{u}\rightarrow 0$ ($\norm{R_yf-f}_u\rightarrow 0$) as $y\rightarrow 1$.
\begin{proposition}
  If $f\in C_c(G)$, then $f$ is left and right uniformly continuous.
\end{proposition}
A left \textit{Haar measure} is a nonzero Radon measure $\mu$ on $G$ such that $\mu\left( xE \right)=\mu(E)$ for every Borel subset $E\subseteq G$.
\begin{proposition}
  Every locally compact group $G$ admits a left Haar measure $\lambda$. This Haar measure is unique up to a constant multiple.
\end{proposition}
If we have a left Haar measure $\lambda$, then if we define
\begin{align*}
  \lambda_x(E) &= \lambda\left( Ex \right),
\end{align*}
we have that $\lambda_x$ is again a left Haar measure, so there is some number $\Delta(x)$ such that $\lambda_x = \Delta(x)\lambda$, where $\Delta(x)$ is independent of the original choice of $\lambda$.

The function $\Delta\colon G\rightarrow \left( 0,\infty \right)$ defined as such is known as the \textit{modular function} of $G$.
\begin{proposition}
  The function $\Delta$ is a continuous homomorphism from $G$ to $\R_{> 0}$, and for any $f\in L_1\left( \lambda \right)$, we have
  \begin{align*}
    \int_{}^{} R_yf\:d\lambda &= \Delta\left( y^{-1} \right) \int_{}^{} f\:d\lambda.
  \end{align*}
\end{proposition}
We call $G$ \textit{unimodular} if $\Delta\equiv 1$.
\begin{proposition}
  If $G/\left[ G,G \right]$ is compact, then $G$ is unimodular.
\end{proposition}
\subsection{Convolutions and $L_1(G)$}%
If $G$ is a locally compact group, we let $M(G)$ denote the space of complex-valued Radon measures on $G$. The convolution of two measures $\mu,\nu\in M(G)$ is given as follows. If we let
\begin{align*}
  I(\phi) &= \iint \phi(xy)\:d\mu(x)d\nu(y),
\end{align*}
then we observe that $I(\phi)$ is a linear functional on $C_0(G)$ that satisfies
\begin{align*}
  \left\vert I(\phi) \right\vert &\leq \norm{\phi}_{u}\norm{\mu}\norm{\nu},
\end{align*}
meaning that it is given by a measure $\mu\ast\nu\in M(G)$ with $\norm{\mu\ast\nu}\leq \norm{\mu}\norm{\nu}$. We call $\mu\ast\nu$ the convolution of $\mu$ and $\nu$.

Observe that if $\delta_x\in M(G)$ is the point mass at $x\in G$, then
\begin{align*}
  \int_{}^{} \phi\:d\left( \delta_x\ast\delta_y \right) &= \iint \phi(uv)\:d\delta_x(u)\delta_y(v)\\
                                                        &= \phi(xy)\\
                                                        &= \int_{}^{} \phi\:d\delta_{xy},
\end{align*}
meaning that $\delta_x\ast\delta_y = \delta_{xy}$.

The estimate $\norm{\mu\ast\nu}\leq \norm{\mu}\norm{\mu}$ gives that convolution makes $M(G)$ a Banach algebra, which we call the \textit{measure algebra} of $G$. Furthermore, $M(G)$ admits an involution
\begin{align*}
  \mu^{\ast}\left( E \right) &= \overline{\mu\left( E^{-1} \right)},
\end{align*}
so that
\begin{align*}
  \int_{}^{} \phi\:d\mu^{\ast} &= \int_{}^{} \phi\left( x^{-1} \right)\:d \overline{\mu(x)}.
\end{align*}
We may identify the space $L_1(G)$ to be the subspace of $M(G)$ where a function $f$ is identified with the measure $f(x)dx$. If $f,g\in L_1(G)$, then the convolution of $f$ and $g$ is the function
\begin{align*}
  f\ast g(x) &= \int_{}^{} f(y)g\left( y^{-1}x \right)\:dy.
\end{align*}
With convolution and the involution given by
\begin{align*}
  f^{\ast}(x)dx &= \overline{f\left( x^{-1} \right)}d\left( x^{-1} \right)\\
  f^{\ast}(x)&= \Delta\left( x^{-1} \right) \overline{f\left( x^{-1} \right)},
\end{align*}
we have that $L_1(G)$ is a Banach $\ast$-algebra known as the \textit{group algebra} of $G$.

Now, we observe that if $G$ is discrete, then if $\delta_e$ is the point mass at $1$, we have that $f\ast\delta = \delta\ast f = f$ for any function $f$. If $G$ is not discrete, we must use an \textit{approximate identity} for $G$. In particular, we can select a family of mollifiers $\set{\psi_U}_{U\in \mathcal{U}}$ such that
\begin{align*}
  \norm{\psi_U\ast f - f} &\rightarrow 0\\
  \norm{f\ast\psi_U - f} &\rightarrow 0
\end{align*}
if $f$ is uniformly continuous and $U\rightarrow \set{1}$ in a neighborhood system $\mathcal{U}$ of $1$.
\subsection{Homogeneous Spaces}%
If $G$ is a locally compact group, then $G$ can act on a locally compact Hausdorff space by homeomorphisms. Recall from algebra that the group action is transitive if there is one orbit. We call $S$ a $G$-space.

The standard example of a transitive $G$-space is the quotient space $G/H$ for some closed subgroup $H$ of $G$. These are, to an extent, the only $G$-spaces, as follows from the orbit-stabilizer theorem. If $S$ is a $G$-space, then we may define a map $\phi\colon G\rightarrow S$ by $\phi(x) = x\cdot s_0$, and take the quotient by the stabilizer subgroup
\begin{align*}
  H &= \set{x\in G | x\cdot s_0 = s_0},
\end{align*}
so that $\Phi\colon G/H\rightarrow S$ has $\Phi\circ q = \phi$ for the quotient map $q\colon G\rightarrow G/H$ is a continuous bijection.
\begin{proposition}
  If $G$ is $\sigma$-compact, then $\Phi$ is a homeomorphism. 
\end{proposition}
\begin{proof}
  It suffices to show that $\phi$ maps open sets in $G$ to open sets in $S$. Suppose $U$ is open in $G$, $x_0\in U$. Pick a compact symmetric neighborhood $V$ of $1$ such that $x_0VV\subseteq U$. Since $G$ is $\sigma$-compact, there is a countable $\set{y_n}_{n\geq 1}\subseteq G$ such that $\set{y_nV}_{n\geq 1}$ covers $G$. Then, we have
  \begin{align*}
    S &= \bigcup_{n=1}^{\infty} \phi\left( y_nV \right).
  \end{align*}
  The sets $\phi\left( y_nV \right)$ are homeomorphic to $\phi(V)$ since the map $s\mapsto y_n\cdot s$ is a homeomorphism of $S$, and all the $y_nV$ are compact, hence closed.

  By Baire Category Theorem for LCH spaces, it follows that $\phi(V)$ has an interior point, which we call $\phi\left(x_1\right)$ for $x_1\in V$. Then, $\phi\left(x_0\right)$ is an interior point of $\phi\left( x_0x_1^{-1}V \right)$, and $x_0x_1^{-1}V\in x_0VV\subseteq U$, so that $\phi\left(x_0\right)$ is an interior point of $\phi(U)$. Thus $\phi(U)$ is open.
\end{proof}
If $S$ is a transitive $G$-space that is isomorphic to a quotient space $G/H$, then will write $S\cong G/H$, and call $S$ a \textit{homogeneous space}. The identification is dependent on the choice of base point, but the identity $s_0' = x_0\cdot x_0$ induces a map $H' = x_0 H x_0^{-1}$, inducing a $G$-equivariant homeomorphism $G/H\cong G/H'$.

We will address the question of whether there is a $G$-invariant Radon measure on $G/H$ --- that is, a radon measure $\mu$ such that $\mu\left( xE \right) = \mu\left( E \right)$ for every $x\in G$.

We assume that $G$ is a locally compact group with left Haar measure $dx$, a $H$ is a closed subgroup of $G$ with left Haar measure $d\xi$, and $q\colon G\rightarrow G/H$ is the quotient map $q(x) = xH$, and $\Delta_G,\Delta_H$ the corresponding modular functions.

Let $P\colon C_c(G)\rightarrow C_c(G/H)$ be defined by
\begin{align*}
  Pf(xH) &= \int_{H}^{} f\left( x\xi \right)\:d\xi.
\end{align*}
This is well-defined by left-invariance of $d\xi$. If $\phi\in C\left( G/H \right)$, then
\begin{align*}
  P\left( \left( \phi\circ q \right)\cdot f \right) &= \phi\cdot Pf.
\end{align*}
\begin{lemma}
  If $E\subseteq G/H$ is compact, then there is a compact $K\subseteq G$ with $q(K) = E$.
\end{lemma}
\begin{proof}
  Let $V$ be an open neighborhood of $1$ in $G$ with compact closure. Since $q$ is an open map, $q\left( xV \right)$ is an open cover of $E$, so there is a finite subcover $q\left( x_jV \right)$. Let $K = q^{-1}(E)\cap \bigcup_{j=1}^{n}x \overline{V}$. Then, since $q^{-1}(E)$ is closed, $K$ is compact with $q(K) = E$.
\end{proof}
\begin{lemma}
  If $F\subseteq G/H$ is compact, then there is $f\geq 0$ in $C_c(G)$ with $Pf = 1$ on $F$.
\end{lemma}
\begin{proof}
  Let $E$ be a compact neighborhood of $F$ in $G/H$. We find $K\subseteq G$ compact such that $q(K) = E$. Select positive $g\in C_c(G)$ with $g > 0$ on $K$, and $\phi\in C_c(G/H)$ supported in $E$ with $\phi = 1$ on $F$. Set
  \begin{align*}
    f &= \frac{\phi\circ q}{Pg\circ q}g,
  \end{align*}
  with the fraction equal to zero whenever the numerator is zero. We have $f$ is continuous, since $Pg > 0$ on $\supp\left( \phi \right)$, has support contained in $\supp\left( g \right)$, and $Pf = \frac{\phi}{Pg}Pg = \phi$.
\end{proof}
\begin{proposition}
  If $\phi\in C_c(G/H)$, then there exists $f\in C_c(G)$ with $Pf = \phi$ and $q\left( \supp(f) \right) = \supp(\phi)$, and has $f\geq 0$ if $\phi \geq 0$.
\end{proposition}
\begin{proof}
  If $\phi \in C_c(G)$, then by the previous lemma, then there exists $g\geq 0$ in $C_c(G)$ with $Pg = 1$ on $\supp(\phi)$. Letting $f = \left( \phi\circ q \right)g$, then $Pf = \phi\left( Pg \right) = \phi$.
\end{proof}
\begin{theorem}
  Let $G$ be a locally compact group, $H$ a closed subgroup. There is a $G$-invariant Radon measure $\mu$ on $G/H$ if and only if $\Delta_G|_{H} = \Delta_H$. In this case, we have
  \begin{align*}
    \int_{G}^{} f(x)\:dx &= \int_{G/H}^{} Pf\:d\mu\\
                         &= \int_{G/H}^{} \int_{H}^{} f\left( x\xi \right)\:d\xi\:d\mu(xH)
  \end{align*}
  for any $f\in C_c(G)$.
\end{theorem}
\begin{proof}
  Suppose there is a $G$-invariant measure $\mu$. Then, $f\mapsto \int_{}^{} Pf\:d\mu$ is a nonzero left-invariant positive linear functional on $C_c(G)$, so by the uniqueness of Haar measure, we have $ \int_{}^{} Pf\:d\mu = c\int_{}^{} f(x)\:dx $ for some $c$.

  This formula fully determines $\mu$, meaning that $\mu$ is unique up to the arbitrary constant factor in Haar measure. We may assume that $c = 1$, so we have for any $\eta\in H$,
  \begin{align*}
    \Delta_G(\eta) \int_{G}^{} f(x)\:dx &= \int_{G}^{} f\left( x\eta^{-1} \right)\:dx\\
                                        &= \int_{G/H}^{} \int_{H}^{} f\left( x\xi\eta^{-1} \right)\:d\xi\:d\mu(xH)\\
                                        &= \Delta_H(\eta) \int_{G/H}^{} \int_{H}^{} f\left( x\xi \right)\:d\xi\:d\mu(xH)\\
                                        &= \Delta_H(\eta) \int_{G}^{} f(x)\:dx,
  \end{align*}
  so that $\Delta_G(\eta) = \Delta_H(\eta)$.
\end{proof}
\section{Representations of a Group and its Group $\ast$-Algebra}%
If $G$ is a locally compact group, then a \textit{unitary representation} of $G$ is a homomorphism $\pi\colon G\rightarrow \mathcal{U}\left( \mathcal{H}_{\pi} \right)$, where $\mathcal{U}\left( \mathcal{H}_{\pi} \right)$ denotes the unitary group of a Hilbert space $\mathcal{H}_{\pi}$. We call $\mathcal{H}_{\pi}$ the \textit{representation space} of $\pi$, and the dimension of $\mathcal{H}_{\pi}$ is called the dimension (or degree) of the representation.

We do not require $\pi$ to be continuous in the norm topology of $\mathcal{B}\left( \mathcal{H}_{\pi} \right)$, but as it turns out, both weak and strong continuity are equivalent as the WOT and SOT coincide on $\mathcal{U}\left( \mathcal{H}_{\pi} \right)$. If $\left( T_{\alpha} \right)_{\alpha}\rightarrow T$ is a net of unitary operators converging in WOT, then for any $u\in \mathcal{H}_{\pi}$, we have
\begin{align*}
  \norm{\left( T_{\alpha} - T\right)u}^2 &= \norm{T_{\alpha}u}^2 - 2\re \iprod{T_{\alpha}u}{Tu} + \norm{Tu}^2\\
                                         &= \norm{u}^2 - 2\re \iprod{T_{\alpha}u}{Tu},
\end{align*}
and the latter term converges to $2\norm{Tu}^2 = 2\norm{u}^2$, so that $\norm{T_{\alpha}u-Tu}\rightarrow 0$.

If $G$ acts on a locally compact Hausdorff space $S$, then $G$ acts on $C(S)$ by $\left( \pi(g)f \right)s = f\left( g^{-1}\cdot s \right)$. If $S$ has a $G$-invariant Radon measure, then $\pi$ defines a unitary representation on $L_2\left( \mu \right)$.

The most important representation is the representation on $L_2(G)$ induced by the action of $G$ on itself by left-multiplication. This defines $\left( \pi_{L}(g)f \right)(y) = f\left( x^{-1}y \right)$.

If $\pi_1$ and $\pi_2$ are unitary representations of $G$, then an intertwining operator for $\pi_1$ and $\pi_2$ is a bounded linear map $T\colon H_{1}\rightarrow H_2$ such that $T\pi_1(x) = \pi_2(x)T$ for all $x\in G$. We say that $\pi_1$ and $\pi_2$ are \textit{unitarily equivalent} if the set of intertwiners admits a unitary map.
\section{Functions of Positive Type}%

\nocite{folland_abstract_harmonic_analysis,kazhdan_property_t}
\printbibliography
\end{document}
