\documentclass[10pt]{mypackage}

\usepackage{mlmodern}
%\usepackage{newpxtext,eulerpx,eucal}
%\renewcommand*{\mathbb}[1]{\varmathbb{#1}}

%\usepackage{homework}
\usepackage{notes}

\usepackage[ backend=bibtex, style = alphabetic, sorting=ynt ]{biblatex}
\addbibresource{all_references.bib}

\usepackage{parskip}

\fancyhf{}
\fancyhead[R]{Avinash Iyer}
\fancyhead[L]{Order and Ideal Structure of $C^{\ast}$-Algebras}
\fancyfoot[C]{\thepage}

\setcounter{secnumdepth}{0}

\begin{document}
\RaggedRight
What follows will require a solid command of the \href{https://ai.avinash-iyer.com/Classes_and_Homework/After\%20College/Other\%20Notes/continuous_functional_calculus.pdf}{continuous functional calculus}.
\section{The Positive Cone of a $C^{\ast}$-Algebra}%
\begin{definition}
  Let $A$ be a $C^{\ast}$-algebra. We say $a\in A$ is \textit{positive} if $a\in A_{\sa}$ and $\sigma(a)\subseteq \R_{+}$. We write $a\geq 0$, and say $a\in A_{+}$.
\end{definition}
For example, an element $f\in C(X)$ is positive if and only if $f(x)\geq 0$ for all $x\in X$, and $\phi\in L_{\infty}(\mu)$ is positive if and only if $\phi(x)\geq 0$ for $\mu$-a.e. $x$.
\begin{proposition}
  If $a\in A_{+}$, then there are unique positive elements $u$ and $v$ in $A$ such that $a = u-v$, $uv = vu = 0$.
\end{proposition}
\begin{proof}
  Letting $f(t) = \max(0,t)$ and $g(t) = -\min(0,t)$, we have $f,g\in C(\R)_{+}$ with $f(t) - g(t) = t$. Using the continuous functional calculus, let $u = f(a)$ and $v = g(a)$. Spectral mapping gives that $u,v\in A_{+}$ with $a = u-v$ and $uv = vu = 0$.
\end{proof}
\begin{lemma}
  If $a\in A$ is a $C^{\ast}$-algebra, $t\in \R$ with $\norm{a}\leq t$, then $a\in A_{+}$ if and only if $\norm{a-t1}\leq t$.
\end{lemma}
\begin{proof}
  Since $\sigma(a)\subseteq [-t,t]$, and $a,1$ are self-adjoint, we have
  \begin{align*}
    \norm{a-t1} &= r\left( a-t1 \right)\\
                &= \sup_{s\in \sigma(a)} \left\vert s-t \right\vert\\
                &= \sup_{s\in \sigma(a)} \left( t-s \right),
  \end{align*}
  so we have $\norm{a-t1}\leq t$ if and only if $\sigma(a)\subseteq [0,\infty)$.
\end{proof}
\begin{theorem}
  Suppose $A$ is a $C^{\ast}$-algebra.
\end{theorem}
\begin{proposition}
  Let $A$ be a $C^{\ast}$-algebra, $a\in A$. The following are equivalent:
  \begin{enumerate}[(i)]
    \item $a\in A_{+}$
    \item there is $v\in A_{+}$ with $a = v^2$
    \item there is $b\in A$ with $a = b^{\ast}b$.
  \end{enumerate}
\end{proposition}
\begin{proof}
  To see that (i) implies (ii), we use the function $f(t) = t^{1/2}$ and apply the continuous functional calculus, which is well-defined since $\sigma(a)\subseteq [0,\infty)$. Similarly, since $v$ is self-adjoint, it follows that (ii) implies (iii).

  Now, if $a = b^{\ast}b$ for $b\in A$, we have the decomposition $a = a_{+} - a_{-}$ as above. If we let $C = ba_{-}$, then
  \begin{align*}
    c^{\ast}c &= a_{-} b^{\ast}ba_{-}\\
              &= a_{-}\left( a_{+}-a_{-} \right)a_{-}\\
              &= -\left( a_{-} \right)^{3}.
  \end{align*}
  Since $a_{-}\in A_{+}$, and $\sigma\left( a_{-} \right) = \set{t^3 | t\in \sigma\left(a_{-}\right)}$, it follows that $-c^{\ast}c = \left( a_{-} \right)^3$ is in $A_{+}$. Yet, this means that $c = 0$, so that $a = a_{+}\in A_{+}$
\end{proof}
The element $v$ such that $v^2 = a$ is known as the square root of $a$. We define the absolute value of an element $a$ in a $C^{\ast}$-algebra by $\left\vert a \right\vert = \left( a^{\ast}a \right)^{1/2}$.
\begin{corollary}
  If $A$ is a $C^{\ast}$-algebra, $a\in A_{+}$, and $b\in A$, then $b^{\ast}ab\in A_{+}$.
\end{corollary}
\begin{proof}
  We have that $b^{\ast}ab = \left( a^{1/2}b \right)^{\ast}a^{1/2}b$.
\end{proof}
\begin{proposition}
  The set $A_{+}$ is a norm-closed cone in $A$.
\end{proposition}
\begin{proof}
  Let $A_{+}\supseteq \left( a_n \right)_n\rightarrow a\in A$. Then, $a\in A_{\sa}$, and we have from above that, since $\sigma\left( a_n \right)\subseteq \left[ 0,\norm{a_n} \right]$,
  \begin{align*}
    \norm{a_n - \norm{a_n} 1} &\leq \norm{a_n},
  \end{align*}
  so that
  \begin{align*}
    \norm{a-\norm{a}} &\leq \norm{a},
  \end{align*}
  meaning that $a\geq 0$. It follows from spectral mapping that if $a \geq 0$ and $t\geq 0$, then $ta \geq 0$.

  Finally, we must show that if $a,b\geq 0$, then so is $a + b$. It suffices to show that $\norm{a}$ and $\norm{b}$ are both less than or equal to $1$. Yet, we then have
  \begin{align*}
    \norm{1-\frac{1}{2}\left( a+b \right)} &= \frac{1}{2}\norm{\left( 1-a \right) + \left( 1-b \right)}\\
                                           &\leq 1,
  \end{align*}
  so we have $\frac{1}{2}\left( a+b \right) \geq 0$.

  Finally, if $a\in A_{+}\cap \left( -A_{+} \right)$, then $\sigma(a) = 0$, so $r(a) = 0$, but $r(a) = \norm{a}$ since $a$ is self-adjoint, and thus normal.
\end{proof}
The fact that $A_{+}$ is a closed cone in $A$ allows us to define an order on $A_{\sa}$ given by $a\leq b$ if $b-a\in A_{+}$.

Note that an element $a\in A_{+}$ is invertible if and only if $a \geq t1$ for some $t > 0$. This follows from the fact that $a \geq t1$ if and only if $a-t1\in A_{+}$; by spectral mapping, this means that $\sigma\left( a-t1 \right) = \set{\lambda - t | \lambda\in \sigma(a)}$, which occurs only if $\sigma(a)\subseteq [t,\infty)$. In particular, this means that $0\notin \sigma(a)$, so that $a$ is invertible.
\begin{proposition}
  Let $a,b\in A_{\sa}$. The following hold:
  \begin{enumerate}[(i)]
    \item if $-b \leq a \leq b$, then $\norm{a}\leq \norm{b}$;
    \item if $0\leq a \leq b$, then $a^{1/2}\leq b^{1/2}$;
    \item if $0\leq a \leq b$ and $a$ is invertible, then $b$ is invertible and $b^{-1}\leq a^{-1}$.
  \end{enumerate}
\end{proposition}
\begin{proof}\hfill
  \begin{enumerate}[(i)]
    \item Since $-\norm{b}1 \leq -b \leq a \leq b \leq \norm{b}1$, it follows that $\norm{a}\leq \norm{b}$.
    \item From ($\ast$) in the proof of (iii),
      \begin{align*}
        \norm{b^{-1/4}a^{1/2}b^{-1/4}} &= r\left( b^{-1/4}a^{1/2}b^{-1/4} \right)\\
                                       &= r\left( a^{1/2}b^{-1/4}b^{-1/4} \right)\\
                                       &\leq \norm{a^{1/2}b^{-1/2}}\\
                                       &\leq 1,
      \end{align*}
      meaning that $b^{-1/4}a^{1/2}b^{-1/4}\leq 1$, and thus $a^{1/2}\leq b^{1/2}$. This shows the case where $a$ is invertible

      If $a$ is not invertible, then for any $\ve > 0$, we have that $\left( a + \ve 1 \right)^{1/2}\leq \left( b + \ve 1 \right)^{1/2}$. For some $c\in A_{\sa}$, define $f_{\ve}\in C(\sigma(c))$ by $f_{\ve}(t) = \left( t + \ve \right)^{1/2}$, meaning that $f_{\ve}\left( c \right)\in A_{+}$ and $\left( f_{\ve}(c) \right)^2 = c + \ve 1$. In particular, this means that $\left( c + \ve 1 \right)^{1/2} = f_{\ve}(c)$, and since $f_{\ve}(t)\rightarrow t^{1/2}$ uniformly as $\ve \rightarrow 0$, we have that $\norm{\left( c + \ve1 \right)^{1/2} - c^{1/2}}\rightarrow 0$, so we get $a^{1/2}\leq b^{1/2}$ in the general caseomte.
    \item If $0\leq a \leq b$ and $a$ is invertible, then $a\geq t1$ for some $t > 0$, so $b \geq t1$ meaning $b$ is invertible. Moreover, we see that
      \begin{align*}
        0 &\leq b^{-1/2}ab^{-1/2}\\
          &\leq b^{-1/2}bb^{-1/2}\\
          &= 1,
      \end{align*}
      so that $\norm{b^{-1/2}ab^{-1/2}} \leq 1$. Therefore, we have
      \begin{align*}
        \norm{a^{1/2}b^{-1/2}} &= \norm{\left( a^{1/2} b^{-1/2}\right)^{\ast}a^{1/2}b^{-1/2}}^{1/2}\tag*{$\ast$}\\
                               &= \norm{b^{-1/2}ab^{-1/2}}^{1/2}\\
                               &\leq 1,
      \end{align*}
      so we have
      \begin{align*}
        \norm{a^{1/2}b^{-1}a^{1/2}} &= \norm{a^{1/2}b^{-1/2}\left( a^{1/2}b^{-1/2} \right)^{\ast}}\\
                                    &\leq 1,
      \end{align*}
      so $a^{1/2}b^{-1}a^{1/2}\leq 1$, meaning $b^{-1}\leq a^{-1}$.
  \end{enumerate}
\end{proof}
We say a function like $f(t) = t^{1/2}$ defined on $S\subseteq \R$ is \textit{operator-monotone} if $f(a)\leq f(b)$ whenever $a\leq b$ in $A_{\sa}$ and $\sigma(a)\cup \sigma(b)\subseteq S$.
\section{Ideals in $C^{\ast}$-Algebras}%
We seek to understand the structure of ideals in a $C^{\ast}$-algebra, which will allow us to establish a $C^{\ast}$-algebraic version of the first isomorphism theorem. We will also use some of these ideas when discussing approximate identities in $C^{\ast}$-algebras next section.
\begin{proposition}
  If $I$ is a closed left (or right) ideal in $A$, with $a\in I$ and $a = a^{\ast}$, then if $f\in C(\sigma(a))$ is such that $f(0) = 0$, then $f(a)\in I$.
\end{proposition}
\begin{proof}
  We note that if $I$ is proper, we must have $0\in \sigma(a)$ since $a$ cannot be invertible. Since $\sigma(a)\subseteq \R$, the Stone--Weierstrass theorem implies there is a sequence $\left( p_n(t) \right)_n\rightarrow f(t)$ uniformly, so $p_n(0) \rightarrow f(0) = 0$. In particular, $q_n(t) = p_n(t) - p_n(0)$ converges to $f(t)$ uniformly on $\sigma(a)$ with $q_n(0) = 0$ for all $n$. Therefore, we see that $q_n(a) \in I$ for each $n$ by the definition of an ideal, and $\norm{q_n(a) - f(a)} \rightarrow 0$, so $f(a)\in I$.
\end{proof}
\begin{corollary}
  If $I$ is a closed left or right ideal, and $a\in I$ with $a = a^{\ast}$, then $a_{+},a_{-},\left\vert a \right\vert,\left\vert a \right\vert^{1/2}\in I$.
\end{corollary}
\begin{theorem}
  If $I$ is a closed ideal in $A$, then $I$ is self-adjoint.
\end{theorem}
\begin{proof}
  Fix $a\in I$. Since $I$ is an ideal, $a^{\ast}a\in I$. We will construct a sequence of continuous functions $\left( u_n \right)_n$ defined on $[0,\infty)$ such that $u_n(0) = 0$ and $u_n(t) \geq 0$ for all $t$, such that
  \begin{align*}
    \norm{au_n\left( a^{\ast}a \right) - a} &\rightarrow 0.
  \end{align*}
  If such a sequence can be constructed, then $u_n\left(a^{\ast}a\right)\in I$, $u_n\left(a^{\ast}a\right)\in I$, and $u_n\left( a^{\ast}a \right)a^{\ast}\in I$ as $I$ is an ideal, giving
  \begin{align*}
    \norm{u_n\left( a^{\ast}a \right)a^{\ast} - a^{\ast}} &= \norm{au_n\left( a^{\ast}a \right) - a}\\
                                                          &\rightarrow 0,
  \end{align*}
  so that $a^{\ast}\in I$ whenever $a\in I$. We observe that
  \begin{align*}
    \norm{au_n\left( a^{\ast}a \right) - a}^2 &= \norm{\left( au_n\left( a^{\ast}a \right) - a \right)^{\ast}\left( au_n\left( a^{\ast}a \right) - a \right)}\\
                                              &= \norm{u_n\left( a^{\ast}a \right)a^{\ast}au_n\left( a^{\ast}a \right) - a^{\ast}au_n\left( a^{\ast}a \right) - u_n\left( a^{\ast}a \right)a^{\ast}a + a^{\ast}a},
  \end{align*}
  meaning that if $b = a^{\ast}a$, we have that $bu_n(b) = u_n(b)b$ implies that, if we set
  \begin{align*}
    f_n(t) &= tu_n(t)^2 - 2tu_n(t) + t\\
           &= t\left( u_n(t) - 1 \right)^2,
  \end{align*}
  then we have $\norm{f_n(b)}\leq \sup_{t \geq 0}\left\vert f_n(t) \right\vert$. If we set $u_n(t) = nt$ for $0\leq t \leq \frac{1}{n}$ and $u(t) = 1$ for all $t \geq \frac{1}{n}$, then we have that
  \begin{align*}
    \sup_{t\geq 0} \left\vert f_n(t) \right\vert &= \frac{4}{27n},
  \end{align*}
  which gives our desired result.
\end{proof}
The function $u_n$ actually gives us a sequence $\left( e_n \right)_n$ of positive elements corresponding to each $a$ such that $e_1\leq e_2\leq \cdots$, $\norm{e_n}\leq 1$, and $\norm{ae_n - a}\rightarrow 0$. We will discuss a similar, more useful version in the following section.

Now, we discuss quotients of $C^{\ast}$-algebras. 
\begin{lemma}
  If $I$ is an ideal in $A$, and $a\in A$, then
  \begin{align*}
    \norm{a + I} &= \inf\set{\norm{a-ax} | x\in I, x\geq 0,\norm{x}\leq 1}.
  \end{align*}
\end{lemma}
\begin{proof}
  Let
  \begin{align*}
    \left( B_I \right)_{+} &= \set{x\in \left( B_X\cap I \right)\cap A_{+}}.
  \end{align*}
  Then, we have that
  \begin{align*}
    \norm{a + I} &\leq \inf_{x\in \left( B_I \right)_{+}} \norm{a-ax}.
  \end{align*}
  Now, if $\left( e_n \right)_n$ is a sequence in $\left( B_{I} \right)_{+}$ such that $\norm{y-ye_n}\rightarrow 0$ as $n\rightarrow\infty$, we have that $0\leq 1-e_n \leq 1$, meaning
  \begin{align*}
    \norm{\left( a + y \right)\left( 1-e_n \right)} &\leq \norm{a+y}.
  \end{align*}
  Hence,
  \begin{align*}
    \norm{a+y} &\geq \liminf_{n\rightarrow\infty} \norm{\left( a+y \right)\left( 1-e_n \right)}\\
               &= \liminf_{n\rightarrow\infty}\norm{a-ae_n + y-ye_n}\\
               &= \liminf_{n\rightarrow\infty} \norm{a-ae_n},
  \end{align*}
  meaning that
  \begin{align*}
    \norm{a+y} &\geq \inf_{n} \norm{a-ae_n}\\
               &\geq \inf_{x\in \left( B_{I} \right)_{+}} \norm{a-ax}.
  \end{align*}
  This gives our desired result.
\end{proof}
\begin{theorem}
  Let $A$ be a $C^{\ast}$-algebra, and let $I$ be a closed ideal of $A$. Defining $\left( a+I \right)^{\ast} = a^{\ast} + I$, we have that $A/I$ with the quotient norm is a $C^{\ast}$-algebra.
\end{theorem}
\begin{proof}
  Since $A$ is a $C^{\ast}$-algebra, it is a Banach algebra, so since $I$ is a closed subspace of $A$, it follows that $A/I$ is a Banach algebra. The main thing we must prove is the $C^{\ast}$-identity. Since $I$ is self-adjoint, it follows that
  \begin{align*}
    \norm{a^{\ast} + I} &= \norm{a + I}
  \end{align*}
  for all $a\in A$, so since $A/I$ is a Banach algebra, we have
  \begin{align*}
    \norm{a^{\ast}a + I} &= \norm{\left( a^{\ast} + I \right)\left( a+I \right)}\\
                         &\leq \norm{a^{\ast} + I} \norm{a + I}\\
                         &= \norm{a + I}^2.
  \end{align*}
  Meanwhile, from the lemma, we have
  \begin{align*}
    \norm{a+I}^2 &= \inf_{x\in \left( B_I \right)_{+}} \norm{a-ax}^2\\
                 &= \inf_{x\in \left( B_{I} \right)_{+}} \norm{a \left( 1-x \right)}^2\\
                 &= \inf_{x\in \left( B_{I} \right)_{+}} \norm{\left( 1-x \right)a^{\ast}a\left( 1-x \right)}\\
                 &\leq \inf_{x\in \left( B_I \right)_{+}}\norm{a^{\ast}a\left( 1-x \right)}\\
                 &= \inf_{x\in \left( B_{I} \right)_{+}} \norm{a^{\ast}a - a^{\ast}ax}\\
                 &= \norm{a^{\ast}}.
  \end{align*}
  This gives our desired result.
\end{proof}
\begin{theorem}
  If $A$ and $B$ are $C^{\ast}$-algebras, and $\rho\colon A\rightarrow B$ is a $\ast$-homomorphism, then $\rho$ is a contraction, and $\img(\rho)$ is closed in $B$. If $\rho$ is injective, then $\rho$ is an isometry.
\end{theorem}
\begin{proof}
  It suffices to assume that $A$ and $B$ are unital (else we simply use the unitization), so we know that $\sigma\left( \rho(x) \right)\subseteq \sigma(x)$ for each $x$, so $r\left( \rho(x) \right)\subseteq r(x)$. This gives the case for all normal elements, and by the $C^{\ast}$-identity, we get
  \begin{align*}
    \norm{\rho(a)}^2 &=\norm{\rho\left( a^{\ast}a \right)}\\
                     &= r\left( \rho\left( a^{\ast}a \right) \right)\\
                     &\leq r\left( a^{\ast}a \right)\\
                     &= \norm{a}^2.
  \end{align*}
\end{proof}
\nocite{conway_functional_analysis,kadison_and_ringrose_1,conway_operator_theory}
\printbibliography
\end{document}
