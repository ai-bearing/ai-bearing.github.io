\documentclass[10pt]{mypackage}

\usepackage{mlmodern}
%\usepackage{newpxtext,eulerpx,eucal}
%\renewcommand*{\mathbb}[1]{\varmathbb{#1}}

%\usepackage{homework}
%\usepackage{notes}

\usepackage[ backend=bibtex, style = alphabetic, sorting=ynt ]{biblatex}
\addbibresource{all_references.bib}

\usepackage{parskip}

\fancyhf{}
\fancyhead[R]{Avinash Iyer}
\fancyhead[L]{Order Structure of $C^{\ast}$-Algebras}
\fancyfoot[C]{\thepage}

\setcounter{secnumdepth}{0}

\begin{document}
\RaggedRight
What follows will require a solid command of the \href{https://ai.avinash-iyer.com/Classes_and_Homework/After\%20College/Other\%20Notes/continuous_functional_calculus.pdf}{continuous functional calculus}.
\section{The Positive Cone of a $C^{\ast}$-Algebra}%
\begin{definition}
  Let $A$ be a $C^{\ast}$-algebra. We say $a\in A$ is \textit{positive} if $a\in A_{\sa}$ and $\sigma(a)\subseteq \R_{+}$. We write $a\geq 0$, and say $a\in A_{+}$.
\end{definition}
For example, an element $f\in C(X)$ is positive if and only if $f(x)\geq 0$ for all $x\in X$, and $\phi\in L_{\infty}(\mu)$ is positive if and only if $\phi(x)\geq 0$ for $\mu$-a.e. $x$.
\begin{proposition}
  If $a\in A_{+}$, then there are unique positive elements $u$ and $v$ in $A$ such that $a = u-v$, $uv = vu = 0$.
\end{proposition}
\begin{proof}
  Letting $f(t) = \max(0,t)$ and $g(t) = -\min(0,t)$, we have $f,g\in C(\R)_{+}$ with $f(t) - g(t) = t$. Using the continuous functional calculus, let $u = f(a)$ and $v = g(a)$. Spectral mapping gives that $u,v\in A_{+}$ with $a = u-v$ and $uv = vu = 0$.
\end{proof}
\begin{lemma}
  If $a\in A$ is a $C^{\ast}$-algebra, $t\in \R$ with $\norm{a}\leq t$, then $a\in A_{+}$ if and only if $\norm{a-t1}\leq t$.
\end{lemma}
\begin{proof}
  Since $\sigma(a)\subseteq [-t,t]$, and $a,1$ are self-adjoint, we have
  \begin{align*}
    \norm{a-t1} &= r\left( a-t1 \right)\\
                &= \sup_{s\in \sigma(a)} \left\vert s-t \right\vert\\
                &= \sup_{s\in \sigma(a)} \left( t-s \right),
  \end{align*}
  so we have $\norm{a-t1}\leq t$ if and only if $\sigma(a)\subseteq [0,\infty)$.
\end{proof}
\begin{theorem}
  Suppose $A$ is a $C^{\ast}$-algebra.
\end{theorem}
\begin{proposition}
  Let $A$ be a $C^{\ast}$-algebra, $a\in A$. The following are equivalent:
  \begin{enumerate}[(i)]
    \item $a\in A_{+}$
    \item there is $v\in A_{+}$ with $a = v^2$
    \item there is $b\in A$ with $a = b^{\ast}b$.
  \end{enumerate}
\end{proposition}
\begin{proof}
  To see that (i) implies (ii), we use the function $f(t) = t^{1/2}$ and apply the continuous functional calculus, which is well-defined since $\sigma(a)\subseteq [0,\infty)$. Similarly, since $v$ is self-adjoint, it follows that (ii) implies (iii).

  Now, if $a = b^{\ast}b$ for $b\in A$, we have the decomposition $a = a_{+} - a_{-}$ as above. If we let $C = ba_{-}$, then
  \begin{align*}
    c^{\ast}c &= a_{-} b^{\ast}ba_{-}\\
              &= a_{-}\left( a_{+}-a_{-} \right)a_{-}\\
              &= -\left( a_{-} \right)^{3}.
  \end{align*}
  Since $a_{-}\in A_{+}$, and $\sigma\left( a_{-} \right) = \set{t^3 | t\in \sigma\left(a_{-}\right)}$, it follows that $-c^{\ast}c = \left( a_{-} \right)^3$ is in $A_{+}$. Yet, this means that $c = 0$, so that $a = a_{+}\in A_{+}$
\end{proof}
The element $v$ such that $v^2 = a$ is known as the square root of $a$. We define the absolute value of an element $a$ in a $C^{\ast}$-algebra by $\left\vert a \right\vert = \left( a^{\ast}a \right)^{1/2}$.
\begin{corollary}
  If $A$ is a $C^{\ast}$-algebra, $a\in A_{+}$, and $b\in A$, then $b^{\ast}ab\in A_{+}$.
\end{corollary}
\begin{proof}
  We have that $b^{\ast}ab = \left( a^{1/2}b \right)^{\ast}a^{1/2}b$.
\end{proof}
\begin{proposition}
  The set $A_{+}$ is a norm-closed cone in $A$.
\end{proposition}
\begin{proof}
  Let $A_{+}\supseteq \left( a_n \right)_n\rightarrow a\in A$. Then, $a\in A_{\sa}$, and we have from above that, since $\sigma\left( a_n \right)\subseteq \left[ 0,\norm{a_n} \right]$,
  \begin{align*}
    \norm{a_n - \norm{a_n} 1} &\leq \norm{a_n},
  \end{align*}
  so that
  \begin{align*}
    \norm{a-\norm{a}} &\leq \norm{a},
  \end{align*}
  meaning that $a\geq 0$. It follows from spectral mapping that if $a \geq 0$ and $t\geq 0$, then $ta \geq 0$.

  Finally, we must show that if $a,b\geq 0$, then so is $a + b$. It suffices to show that $\norm{a}$ and $\norm{b}$ are both less than or equal to $1$. Yet, we then have
  \begin{align*}
    \norm{1-\frac{1}{2}\left( a+b \right)} &= \frac{1}{2}\norm{\left( 1-a \right) + \left( 1-b \right)}\\
                                           &\leq 1,
  \end{align*}
  so we have $\frac{1}{2}\left( a+b \right) \geq 0$.

  Finally, if $a\in A_{+}\cap \left( -A_{+} \right)$, then $\sigma(a) = 0$, so $r(a) = 0$, but $r(a) = \norm{a}$ since $a$ is self-adjoint, and thus normal.
\end{proof}
The fact that $A_{+}$ is a closed cone in $A$ allows us to define an order on $A_{\sa}$ given by $a\leq b$ if $b-a\in A_{+}$.
\nocite{conway_functional_analysis,kadison_and_ringrose_1}
\printbibliography
\end{document}
