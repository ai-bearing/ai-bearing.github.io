\documentclass[10pt]{mypackage}

% sans serif font:
%\usepackage{cmbright,sfmath,bbold}
%\renewcommand{\mathcal}{\mathtt}

%Euler:
\usepackage{newpxtext,eulerpx,eucal,eufrak}
\renewcommand*{\mathbb}[1]{\varmathbb{#1}}
\renewcommand*{\hbar}{\hslash}

%\renewcommand{\mathbb}{\mathds}
%\usepackage{homework}
%\usepackage{exposition}

\pagestyle{fancy} %better headers
\fancyhf{}
\rhead{Avinash Iyer}
\lhead{Analysis Qualifier}

\setcounter{secnumdepth}{0}

\begin{document}
\RaggedRight
This is a notes document regarding essential problem-solving methods for the analysis qualifiers.
\section{Real Analysis}%
\subsection{\href{https://math.virginia.edu/graduate/exams/analysis/2019Aug_real.pdf}{August 2019}}%
\subsubsection{Problem 1}%
\begin{enumerate}[(a)]
  \item Recall that the Cantor set $\mathcal{C}$ is defined to consist of all $x\in [0,1]$ such that $x$ only contains $0$ and $2$ in the ternary expansion of $x$. Writing $a\in [0,2]$ as
    \begin{align*}
      a &= \sum_{k=0}^{\infty}\frac{a_k}{3^{k}},
    \end{align*}
    where $a_k\in\set{0,1,2}$, we may then find $a_k$ at each ternary expansion slot for $k$ as follows:
    \begin{itemize}
      \item if $a_k = 0$, we may find $b_k,c_k\in \mathcal{C}$ such that $b_k = c_k = 0$
      \item if $a_k = 2$, we may find $b_k,c_k\in \mathcal{C}$ such that $b_k = 2$ and $c_k = 0$ or vice versa.
      \item if $a_k = 1$, we may find $b_k,c_k\in \mathcal{C}$ such that $b_{k+1} = c_{k+1} = 2$.
    \end{itemize}
    Therefore, since every digit of every ternary expansion in $[0,2]$ can be obtained from $\mathcal{C}$, we see that $\mathcal{C} + \mathcal{C} = [0,2]$.
  \item We may set $B$ to be the union of all integer translates of $\mathcal{C}$, and set $A = \mathcal{C}$. This yields closed subsets of $\R$ with Lebesgue measure zero that sum to $\R$.
\end{enumerate}
\subsubsection{Problem 2}%
Consider the sequence of functions
\begin{align*}
  f_n(x) &= n\1_{\left[ \frac{1}{n+1},\frac{1}{n} \right]},
\end{align*}
defined on $\left[ 0,1 \right]$. This sequence is pointwise convergent everywhere to zero, as $f_n(0) = 0$ and the Archimedean property give that for any $x\in (0,1]$, there is some $n$ large enough that gives $\frac{1}{n} < x$. Furthermore, we see that
\begin{align*}
  \int_{}^{} f_n\:d\mu &= n\left( \frac{1}{n}- \frac{1}{n+1} \right)\\
                       &= \frac{1}{n+1}\\
                       &\rightarrow 0.
\end{align*}
Finally, we see that by taking suprema, we have the integral
\begin{align*}
  \int_{}^{} \Phi\:d\mu &= \sum_{n=1}^{\infty}\frac{1}{n+1}\\
                        &\rightarrow \infty.
\end{align*}
\subsubsection{Problem 4}%
Suppose toward contradiction that both $f$ and $1/f$ are in $L_1\left( \R \right)$. Then, from Hölder's Inequality, we have
\begin{align*}
  \infty &= \int_{}^{} 1\:d\mu\\
         &\leq \left( \int_{}^{} f\:d\mu \right)^{1/2} \left( \int_{}^{} \frac{1}{f}\:d\mu \right)^{1/2}\\
         &< \infty,
\end{align*}
which is a contradiction.
\subsubsection{Problem 5}%
\begin{enumerate}[(a)]
  \item Let $f\in L_2\left( [-1,1] \right)$. We may find $g\in C\left( [-1,1] \right)$ such that $\norm{f-g}_{L_2} < \ve/2$. Similarly, we may find a polynomial $p$ such that $\norm{g-p}_{u} < \ve/4$, meaning that $\left\vert p(x)-g(x) \right\vert < \ve/4$ for all $x\in [-1,1]$. This yields
    \begin{align*}
      \norm{p-g}_{L_2} &= \left( \int_{-1}^{1} \left\vert p(x)-g(x) \right\vert^2\:dx \right)^{1/2}\\
                       &< \left( \int_{-1}^{1} \left( \frac{\ve}{4} \right)^2\:dx \right)^{1/2}\\
                       &= \left( \frac{\ve^2}{8} \right)^{1/2}\\
                       &< \frac{\ve}{2},
    \end{align*}
    so $\norm{f-p}_{L_2} < \ve$, meaning that the closed linear span of the monomials is dense in $L_2$, and the Legendre polynomials form an orthonormal system.
  \item We see that at every step in evaluating the expression
    \begin{align*}
      L_n(x) &= c_n \diff{^{n}}{x^{n}}\left( x^2-1 \right)^{n},\label{eq:legendre_expression}\tag{$\ast$}
    \end{align*}
    the degree of the polynomial increases by $1$, so each $L_n(x)$ has degree $n$. To verify that the polynomials generated from \eqref{eq:legendre_expression} are orthogonal to each other, we let $n > m$ without loss of generality, and use integration by parts to obtain
    \begin{align*}
      \iprod{L_n}{L_m} &= \int_{-1}^{1} \left( \diff{^{n}}{x^{n}}\left( x^2-1 \right)^{n} \right) \left( \diff{^{m}}{x^{m}}\left( x^2-1 \right)^{m} \right)\:dx\\
                       &= \diff{^{n-1}}{x^{n-1}}\left( x^2-1 \right)^{n}\diff{^{m}}{x^{m}}\left( x^2-1 \right)^{m}\biggr\vert_{-1}^{1} - \int_{-1}^{1} \diff{^{n-1}}{x^{n-1}}\left( x^2-1 \right)^{n}\diff{^{m+1}}{x^{m+1}}\left( x^2-1 \right)^{m} \:dx\\
                       &\vdots\\
                       &= \left( -1 \right)^{n} \int_{-1}^{1} \diff{^{m+n}}{x^{m+n}}\left( x^2-1 \right)^{m}\:dx\\
                       &= \left( -1 \right)^{n} \int_{-1}^{1} \diff{^{n}}{x^{n}}\left( \diff{^{m}}{x^{m}}\left( x^2-1 \right)^{m} \right)\:dx\\
                       &= \left( -1 \right)^{n} \int_{}^{} \diff{^{n}}{x^{n}}L_m(x)\:dx\\
                       &= 0,
    \end{align*}
    seeing as we are taking $n$ derivatives of a degree $m < n$ polynomial.
\end{enumerate}
\subsection{\href{https://math.virginia.edu/graduate/exams/analysis/2020Jan_real.pdf}{January 2020}}%
\subsubsection{Problem 1}%
\begin{enumerate}[(a)]
  \item This is false. If $A\subseteq [0,1]$ is the ``fat Cantor set'' constructed similar to the traditional Cantor, but obtained by deleting the middle fourth of each subinterval rather than the middle third, then $\mu(A) = \frac{1}{2}$, but $A$ is nowhere dense, meaning that if $U\subseteq A$ is open, then $U = \emptyset$.
  \item This is true. By the definition of the Lebesgue outer measure, for any $\ve > 0$, there are $\set{\left( a_k,b_k \right)}_{k=1}^{\infty}$ such that
    \begin{align*}
      \mu(A) + \ve &< \mu\left( \bigcup_{k=1}^{\infty}\left( a_k,b_k \right) \right),
    \end{align*}
    so by setting
    \begin{align*}
      U &= \bigcup_{k=1}^{\infty}\left( a_k,b_k \right),
    \end{align*}
    we have that $U$ is open, meaning that by the definition of infimum, we get 
    \begin{align*}
      \mu(A) &= \inf\set{U | A\subseteq U,U\text{ open}}.
    \end{align*}
\end{enumerate}
\subsubsection{Problem 3}%
\begin{enumerate}[(a)]
  \item Consider the algebra of polynomials on $[0,1]$ without a constant term. Then, since linear combinations and multiplications still yield polynomials without constant term, and $f(x) = x$ separates points in $[0,1]$, this algebra satisfies the requirements of the question. Yet, since all elements of this algebra are equal to zero at $x= 0$, the uniform closure of the algebra yields all the continuous functions on $[0,1]$ with $f(0) = 0$.
  \item In order to satisfy the requirements of the Stone--Weierstrass theorem, we need the algebra $\mathcal{A}$ to include the constant functions.
\end{enumerate}
\subsubsection{Problem 4}%
We consider the signed measure on $ \mathcal{F} $ defined by
\begin{align*}
  \nu(E) &= \int_{E}^{} f\:d\mu,
\end{align*}
meaning that $\nu\ll \mu$, so the function $g \coloneq \diff{\nu}{\mu}$, where $\diff{\nu}{\mu}$ denotes the Radon--Nikodym derivative of $\nu$ with respect to $\mu$, is $\mathcal{F}$-measurable and in $L_1\left( \R,\mathcal{F},\mu \right)$. This gives, for all $E\in \mathcal{F}$,
\begin{align*}
  \int_{E}^{} g\:d\mu &= \int_{E}^{} \:d\nu\\
                      &= \nu\left( E \right)\\
                      &= \int_{E}^{} f\:d\mu.
\end{align*}
\subsection{\href{https://math.virginia.edu/graduate/exams/analysis/2020Aug_real.pdf}{August 2020}}%
\subsubsection{Problem 1}%
This is false. To see this, let $ \mathfrak{C}(x) $ denote the Cantor--Lebesgue function, and let
\begin{align*}
  h(x) &= \sum_{n=-\infty}^{\infty} \mathfrak{C}\left( x - n \right) + n.
\end{align*}
Then, since $\mathfrak{C}(x)$ has derivative zero almost everywhere, the sum of a number of translates of $\mathfrak{C}(x)$ still has derivative zero almost everywhere. Then, setting
\begin{align*}
  f(x) &= h(x) + x,
\end{align*}
we get that $f(x)$ has derivative equal to $1$ almost everywhere. However, at the same time, $f(2) - f(1) = 2$.
\subsubsection{Problem 2}%
We show the inverse problem, which is that every closed set in $\R^2$ is $G_{\delta}$. To do this, we let $A\subseteq \R^2$ be closed, nonempty, and proper (if $A = \emptyset$ or $A = \R^2$ the answer is trivial).\newline

Then, there is some $x\in A^{c}$, and specifically there is $x\in A^{c}$ with rational coordinates (else, select $y\in \Q^2$ within the ball of radius $\ve$ that allows $A^{c}$ to be open). Furthermore, since $\R^2$ is a metric space, $\R^2$ is regular, so there are open $U_{x}$ and $V_x$ such that $A\subseteq U_x$, $x\in V_x$, and $U_x\cap V_x = \emptyset$.\newline

Therefore, we get
\begin{align*}
  A &= \bigcap \set{U_x | x\in \Q^2\setminus A},
\end{align*}
meaning that $A$ is $G_{\delta}$. Taking complements, we thus get that every open set is $F_{\sigma}$.
\end{document}
