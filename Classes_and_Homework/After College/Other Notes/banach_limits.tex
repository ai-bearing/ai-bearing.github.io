\documentclass[12pt]{mypackage}

% sans serif font:
%\usepackage{cmbright}
%\usepackage{sfmath}
%\usepackage{bbold} %better blackboard bold

%\usepackage{homework}
\usepackage{notes}
%\usepackage{mlmodern}
%\usepackage{newpxtext,eulerpx,eucal}
%\renewcommand*{\mathbb}[1]{\varmathbb{#1}}
\usepackage{parskip}

\fancyhf{}
\fancyhead[R]{Avinash Iyer}
\fancyhead[L]{Generalized Limits}
\fancyfoot[C]{\thepage}

\setcounter{secnumdepth}{0}

\begin{document}
\RaggedRight
\section{Banach Limits}%
\begin{theorem}[Hahn--Banach--Minkowski]
  Let $X$ be a real vector space, and let $p\colon X\rightarrow \R$ be such that $p\left( x+y \right) \leq p(x) + p(y)$ for all $x,y\in X$ and $p\left( tx \right) = tp(x)$ for all $t \geq 0$. If $f\colon Y\rightarrow \R$ is a linear functional defined on a subspace $Y$ such that $f(x)\leq p(x)$ for all $x\in Y$, then there is an extension $F\colon X\rightarrow \R$ such that $F(x)\leq p(x)$ for all $x\in X$ and $F|_{Y} = f$.
\end{theorem}
\begin{corollary}
  If $X$ is a complex normed vector space with subspace $E\subseteq X$ and $\varphi\in E^{\ast}$, then there is $\Phi\colon X\rightarrow \C$ such that $\Phi|_{E} = \varphi$ and $\norm{\Phi} = \norm{\varphi}$.

  Additionally, if there is $x_0\in X\setminus E$, then there is $f\in X^{\ast}$ such that $f\left( x_0 \right) = \dist_{E}\left(x_0\right)$ and $f|_{E} = 0$.
\end{corollary}
One of the most important vector spaces is the space $\ell_{\infty}$ of bounded sequences $x\colon \N\rightarrow \C$, which admits a subspace of convergent subspaces, often denoted $c$. 
\begin{proposition}
  There exists a linear functional $L\colon \ell_{\infty}\rightarrow \C$ with
  \begin{enumerate}[(i)]
    \item $\norm{L} = 1$;
    \item for any $x\in c$, $L(x) = \lim_{n\rightarrow\infty}x_n$;
    \item for any $x\in \ell_{\infty}$ with $x_n\geq 0$  for each $n$, we have $L(x) \geq 0$;
    \item for any $x\in \ell_{\infty}$, with $\left( S(x) \right)_{n}\coloneq x_{n+1}$, we have $L\left(S(x)\right) = L(x)$.
  \end{enumerate}
\end{proposition}
We will construct this linear functional using the Hahn--Banach theorem(s) by following the construction in Conway's book. We consider the real vector space $\re\left(\ell_{\infty}\right)$, which we will write as $\ell_{\infty}$ for now.

To start we consider the subspace $M$ of $\ell_{\infty}$ given by
\begin{align*}
  M &= \set{x-S(x) | x\in \ell_{\infty}}.
\end{align*}
If $\1$ denotes the sequence of $1$s in $\ell_{\infty}$, then we see that $\dist_{M}\left(\1\right) = 1$. First, $0\in M$, so that $\dist_{M}\left(\1\right) \leq 1$. If there is $n$ such that $\left( x-S(x) \right)_n\leq 0$, then we see that
\begin{align*}
  \norm{\1 - \left( x-S(x) \right)} &\geq \left\vert \1-\left( x_n - \left( S(x) \right)_n \right) \right\vert\\
                                    &\geq 1.
\end{align*}
Else, if for all $n$, $0 \leq \left( x - S(x) \right)_n = x_n-x_{n+1}$, then $x_{n+1}\leq x_n$ for all $n$. Since $x\in \ell_{\infty}$, there is $\alpha = \lim_{n\rightarrow\infty}x_n$. Therefore, $\lim_{n\rightarrow\infty}\left( x_n-x_{n+1} \right) = 0$, so $\norm{\1 - \left( x-S(x) \right)} \geq 1$.

Thus, there is some linear functional $L\colon \ell_{\infty}\rightarrow \R$ such that $\norm{L} = 1$ and $L\left(x\right) = L\left(S(x)\right)$. This satisfies (i) and (iv) in the proposition.

Next, we show that $c_0\subseteq \ker\left( L \right)$. Since (in our current focus) $c = c_0 + \R\1$, we would then obtain (ii). To see this, let $x\in c_0$. Observe that $S^{n}(x) - x$ is contained in $M$, meaning that $L(x)= L\left( S^{n}(x) \right)$ for each $n$. If $\ve > 0$, there is some $N$ such that $\left\vert x_m \right\vert < \ve$ for all $m > N$. Therefore,
\begin{align*}
  \left\vert L(x) \right\vert &= \left\vert L\left( S^{n}(x) \right) \right\vert\\
                              &\leq \norm{S^n(x)}\\
                              &< \ve.
\end{align*}
Since $\ve$ is arbitrary, we thus have $x\in \ker\left( L \right)$.

Finally, to show (c), we let $x\in \ell_{\infty}$ be such that $x_n \geq 0$ for all $n$, and assume toward contradiction that $L(x) < 0$. Dividing out by $\norm{x}$, we have that $L(x) < 0$ and $0 \leq x_n \leq 1$ for all $n$. Yet, this would imply that $\norm{\1-x}\leq 1$ and $L\left(\1-x\right) = 1 - L(x) > 1$, contradicting (a).

To extend to $\C$, and rewriting the functional on $\R$ as $L_1$, we may write any element $x\in \ell_{\infty}$ as $x = x_1 + i x_2$. Observe then that
\begin{align*}
  L(x) &= L_1\left(x_1\right) + i L_1\left( x_2 \right)
\end{align*}
is a linear functional on $\ell_{\infty}$. Now, observe that $\norm{L} \geq 1$ almost by design. Since $L$ is a nonzero linear functional, we let $x$ be such that $L(x)\neq 0$, and set
\begin{align*}
  \alpha = \frac{\left\vert L(x) \right\vert}{L(x)}.
\end{align*}
We have that $\left\vert \alpha \right\vert = 1$ and $\alpha L(x) = \left\vert L(x) \right\vert$. We may then compute
\begin{align*}
  \left\vert L(x) \right\vert &= L\left( \alpha x \right)\\
                              &= \re\left( L\left( \alpha x \right) \right)\\
                              &= L_1\left( \alpha x \right)\\
                              &\leq \norm{L_1}\norm{\alpha x}\\
                              &= \norm{x}.
\end{align*}
In particular, this means $\norm{L}\leq 1$, so $\norm{L} = 1$.

We call such a shift-invariant extension of the limit to all of $\ell_{\infty}$ a \textit{Banach limit}. A quick observation gives that this limit functional cannot in fact be an algebra homomorphism. Considering the case of $a_n = \left( -1 \right)^{n}$, we have that $a_{n+1} = -a_n$, so that 
\begin{align*}
  L\left( S(a) \right) &= L\left( a \right)\\
                       &= L\left( -a \right)\\
                       &= - L(a),
\end{align*}
or that the Banach limit must be equal to $0$. However, if $L$ were instead an algebra homomorphism (where the multiplication operation on $\ell_{\infty}$ is given pointwise), we would have
\begin{align*}
  L\left( a^2 \right) &= L\left( \1 \right)\\
                      &= 1\\
                      &= \left( L\left( a \right) \right)^2,
\end{align*}
meaning that we would have $L\left( a \right) = \pm 1$, This means that such a limit that is an algebra homomorphism cannot be shift-invariant in the general case.

Regarding the case of $\left( -1 \right)^{n}$, we observe that from our work above that $0$ is the only Banach limit for this sequence. The sequences where every Banach limit assigns the same value for them are known as the \textit{almost convergent} sequences. Lorentz (1948) showed that an equivalent criterion for almost convergence is that, for all $\ve > 0$, there exists $p_0$ such that for all $p > p_0$ and all $n\in \N$, we have
\begin{align*}
  \left\vert \frac{x_n + x_{n+1} + \cdots + x_{n+p-1}}{p} - L \right\vert < \ve.
\end{align*}
Every convergent sequence is almost convergent by definition.
\section{Generalized Limits beyond Banach Limits}%
\subsection{Constructing Banach Limits from Generalized Limits}%
We have obtained one limit. However, there are lots of other extensions of the limit, each of which has norm $1$. In fact, if we consider the restriction of $\ell_{\infty}$ to the reals, we know from undergrad real analysis that
\begin{align*}
  p(x) &= \limsup_{n\rightarrow\infty} \left( \left( x_n \right)_n \right)
\end{align*}
is in fact a sublinear functional. Furthermore, since
\begin{align*}
  \liminf\left( \left( x_n \right)_n \right) &= -\limsup\left( \left( -x_n \right)_n \right),
\end{align*}
we can specify an extension to the limit functional to $L\colon \ell_{\infty}\rightarrow \R$ with $\norm{L} = 1$ such that for any $\left( x_n \right)_n\in \ell_{\infty}$, we have 
\begin{align*}
  \liminf_{n\rightarrow\infty}\left( \left( x_n \right)_n \right) \leq L(x)\leq \limsup_{n\rightarrow\infty}\left( \left( x_n \right)_n \right)
\end{align*}
However, these generalized limits may not align to the same requirement of shift-invariance as the aforementioned Banach limit. We can in fact construct a Banach limit from any generalized limit by using Césaro-type summation: letting $x = \left( x_n \right)_n\in \ell_{\infty}$, we may define
\begin{align*}
  c_n &= \frac{1}{n} \sum_{k=1}^{n} a_k,
\end{align*}
and then, for a generalized limit $L$, defining $L'$ by
\begin{align*}
  L'\left( a \right) &= L\left(c\right).
\end{align*}
From undergrad real analysis, we know that if $\left( a_n \right)_n\rightarrow a$ is convergent, then we have
\begin{align*}
  \left( \frac{1}{n}\sum_{k=1}^{n}a_k \right)_{n} &\rightarrow a
\end{align*}
as well. Thus, we see that
\begin{align*}
  L'\left( S(a) \right) - L'(a) &= L'\left( \left( a_{n+1} \right)_n \right) - L\left( \left( a_n \right)_n \right)\\
                                &= L\left( \left( \frac{1}{n}\sum_{k=1}^{n}a_{k+1} - \frac{1}{n}\sum_{k=1}^{n}a_k \right)_n \right)\\
                                &= L\left( \left( \frac{1}{n}\left( a_{n+1}-a_1 \right) \right)_n \right)\\
                                &= \lim_{n\rightarrow\infty} \left( \frac{1}{n}\left( a_{n+1}-a_1 \right) \right)\\
                                &= 0,
\end{align*}
with the second-to-last equality emerging from the fact that $a\in \ell_{\infty}$, so the sequence $\left( a_{n+1}-a_1 \right)_n$ is necessarily bounded.

In particular, this means that there are plenty of Banach limits, since we can take \textit{any} generalized extension of the limit to $\ell_{\infty}$, then use this Césaro-type summation to yield a Banach limit.
\subsection{Generalized Limits that are Algebra Homomorphisms}%
We know from undergraduate real analysis that the limit satisfies the various properties described above --- positive, unital, shift-invariant --- and used some Hahn--Banach magic to construct a similar type of limit on all of $\ell_{\infty}$. Yet, there is one more property from undergraduate real analysis that classical limits have and we did not make use of; namely, that such limits are algebra homomorphisms. We will now discuss how to construct such types of limits.

To start, we observe that $\ell_{\infty}$ and its subspace $c_0$ are $C^{\ast}$-algebras. In particular, if we endow $\N$ with the discrete topology, then
\begin{align*}
  \ell_{\infty} &= C_b\left( \N \right)\\
  c_0 &= C_0\left( \N \right).
\end{align*}
Since $\ell_{\infty}$ is a unital $C^{\ast}$-algebra,the Gelfand--Naimark theorem says that we may view $\ell_{\infty}$ as the space of continuous functions over the space of characters on the space. That is, we let $\Omega\left( \ell_{\infty} \right)\subseteq \ell_{\infty}^{\ast}$ be the space of all algebra homomorphisms $\phi\colon \ell_{\infty}\rightarrow \C$ satisfying $\phi\left( \1 \right) = 1$, and endow $\Omega\left( \ell_{\infty} \right)$ with the weak* topology. Such characters are automatically positive and unital, so we may view them, in a sense, as limits.

Characters are in one-to-one correspondence with maximal ideals (one direction is obvious, and the other direction follows from the Gelfand--Mazur theorem characterizing Banach algebras that are also division algebras). In particular, we may define a corresponding space of maximal ideals, $\beta\N$, endowed with the hull--kernel topology, where if $X\subseteq \beta\N$ is some set, then
\begin{align*}
  \overline{X} &= \set{p\in \beta\N | p\supseteq \bigcap_{q\in X}q}.
\end{align*}
It can be shown that $\beta\N$ satisfies the universal property of the \href{https://en.wikipedia.org/wiki/Stone\%E2\%80\%93\%C4\%8Cech_compactification}{Stone--Cech compactification}. From some more hard work (see the book \textit{Algebra in the Stone--Cech Compactification}), it can be shown that the set $\beta\N$ is the space of ultrafilters on $\N$. Furthermore, any character $\phi\colon \ell_{\infty}\rightarrow \C$ is in one-to-one correspondence with an ultrafilter by defining the set $A\subseteq \N$ to be an element of $\mathcal{U}_{\phi}$ if $\phi\left( \chi_A \right) = 1$.

Yet, this idea of viewing limits as every such character is not really accurate. For instance, the map $\phi\colon \ell_{\infty}\rightarrow \C$ taking $\phi\left( x \right) = x_1$ is a character, but it doesn't really evaluate a sequence at a limit if it has one. For this purpose, we desire that any $c_0$ sequence evaluates to $0$; that is, we want the set of all elements of $\phi\in \Omega\left( \ell_{\infty} \right)$ such that $c_0\subseteq \ker\left( \phi \right)$. This induces a character, $\varphi\colon \ell_{\infty}/c_0\rightarrow \C$, following from the first isomorphism theorem (for $C^{\ast}$-algebras). Finally, we may view these as generalized limits, by setting
\begin{align*}
  \lim_{\varphi}\left( \left( a_n \right)_n \right) &= \phi\left( \left( a_n \right)_n + c_0 \right).
\end{align*}
To find the ultrafilters that correspond to these particular characters, observe that if $A\subseteq \N$ is finite, then $\chi_A$ is a $c_0$ sequence; in particular, this means that it evaluates to zero. Therefore, if $\phi\colon \ell_{\infty}\rightarrow \C$ is a character with $c_0\subseteq \ker\left( \phi \right)$, then the ultrafilter $\mathcal{U}_{\phi}$ must not have any finite subsets --- i.e., it is a free ultrafilter (rather than a principal ultrafilter).
\end{document}
