\documentclass[12pt]{mypackage}

% sans serif font:
%\usepackage{cmbright}
%\usepackage{sfmath}
%\usepackage{bbold} %better blackboard bold

%\usepackage{homework}
\usepackage{notes}
\usepackage{mlmodern}
%\usepackage{newpxtext,eulerpx,eucal}
%\renewcommand*{\mathbb}[1]{\varmathbb{#1}}
\usepackage{parskip}

\fancyhf{}
\fancyhead[R]{Avinash Iyer}
\fancyhead[L]{Generalized Limits}
\fancyfoot[C]{\thepage}

\setcounter{secnumdepth}{0}

\begin{document}
\RaggedRight
\section{Banach Limits}%
\begin{theorem}[Hahn--Banach--Minkowski]
  Let $X$ be a real vector space, and let $p\colon X\rightarrow \R$ be such that $p\left( x+y \right) \leq p(x) + p(y)$ for all $x,y\in X$ and $p\left( tx \right) = tp(x)$ for all $t \geq 0$. If $f\colon Y\rightarrow \R$ is a linear functional defined on a subspace $Y$ such that $f(x)\leq p(x)$ for all $x\in Y$, then there is an extension $F\colon X\rightarrow \R$ such that $F(x)\leq p(x)$ for all $x\in X$ and $F|_{Y} = f$.

  Furthermore, if $v\in X\setminus Y$, the value of $F(v)$ can be designated to be in the closed interval defined by
  \begin{align*}
    m &= \sup_{w\in Y} \left( -p\left( -w-v \right) - f(w)\right)
  \end{align*}
  at the left endpoint, and
  \begin{align*}
    M &= \inf_{w\in Y} \left( p\left( w+v \right) - f(w) \right)
  \end{align*}
  at the right endpoint.
\end{theorem}
\begin{corollary}
  If $X$ is a complex normed vector space with subspace $E\subseteq X$ and $\varphi\in E^{\ast}$, then there is $\Phi\colon X\rightarrow \C$ such that $\Phi|_{E} = \varphi$ and $\norm{\Phi} = \norm{\varphi}$.

  Additionally, if there is $x_0\in X\setminus E$, then there is $f\in X^{\ast}$ such that $f\left( x_0 \right) = \dist_{E}\left(x_0\right)$ and $f|_{E} = 0$.
\end{corollary}
One of the most important vector spaces is the space $\ell_{\infty}$ of bounded sequences $x\colon \N\rightarrow \C$, which admits a subspace of convergent subspaces, often denoted $c$. 
\begin{proposition}
  There exists a linear functional $L\colon \ell_{\infty}\rightarrow \C$ with
  \begin{enumerate}[(i)]
    \item $\norm{L} = 1$;
    \item for any $x\in c$, $L(x) = \lim_{n\rightarrow\infty}x_n$;
    \item for any $x\in \ell_{\infty}$ with $x_n\geq 0$  for each $n$, we have $L(x) \geq 0$;
    \item for any $x\in \ell_{\infty}$, with $\left( S(x) \right)_{n}\coloneq x_{n+1}$, we have $L\left(S(x)\right) = L(x)$.
  \end{enumerate}
\end{proposition}
We will construct this linear functional using the Hahn--Banach theorem(s) by following the construction in Conway's book. We consider the real vector space $\re\left(\ell_{\infty}\right)$, which we will write as $\ell_{\infty}$ for now.

To start we consider the subspace $M$ of $\ell_{\infty}$ given by
\begin{align*}
  M &= \set{x-S(x) | x\in \ell_{\infty}}.
\end{align*}
If $\1$ denotes the sequence of $1$s in $\ell_{\infty}$, then we see that $\dist_{M}\left(\1\right) = 1$. First, $0\in M$, so that $\dist_{M}\left(\1\right) \leq 1$. If there is $n$ such that $\left( x-S(x) \right)_n\leq 0$, then we see that
\begin{align*}
  \norm{\1 - \left( x-S(x) \right)} &\geq \left\vert \1-\left( x_n - \left( S(x) \right)_n \right) \right\vert\\
                                    &\geq 1.
\end{align*}
Else, if for all $n$, $0 \leq \left( x - S(x) \right)_n = x_n-x_{n+1}$, then $x_{n+1}\leq x_n$ for all $n$. Since $x\in \ell_{\infty}$, there is $\alpha = \lim_{n\rightarrow\infty}x_n$. Therefore, $\lim_{n\rightarrow\infty}\left( x_n-x_{n+1} \right) = 0$, so $\norm{\1 - \left( x-S(x) \right)} \geq 1$.

Thus, there is some linear functional $L\colon \ell_{\infty}\rightarrow \R$ such that $\norm{L} = 1$ and $L\left(x\right) = L\left(S(x)\right)$. This satisfies (i) and (iv) in the proposition.

Next, we show that $c_0\subseteq \ker\left( L \right)$. Since (in our current focus) $c = c_0 + \R\1$, we would then obtain (ii). To see this, let $x\in c_0$. Observe that $S^{n}(x) - x$ is contained in $M$, meaning that $L(x)= L\left( S^{n}(x) \right)$ for each $n$. If $\ve > 0$, there is some $N$ such that $\left\vert x_m \right\vert < \ve$ for all $m > N$. Therefore,
\begin{align*}
  \left\vert L(x) \right\vert &= \left\vert L\left( S^{n}(x) \right) \right\vert\\
                              &\leq \norm{S^n(x)}\\
                              &< \ve.
\end{align*}
Since $\ve$ is arbitrary, we thus have $x\in \ker\left( L \right)$.

Finally, to show (c), we let $x\in \ell_{\infty}$ be such that $x_n \geq 0$ for all $n$, and assume toward contradiction that $L(x) < 0$. Dividing out by $\norm{x}$, we have that $L(x) < 0$ and $0 \leq x_n \leq 1$ for all $n$. Yet, this would imply that $\norm{\1-x}\leq 1$ and $L\left(\1-x\right) = 1 - L(x) > 1$, contradicting (a).

To extend to $\C$, and rewriting the functional on $\R$ as $L_1$, we may write any element $x\in \ell_{\infty}$ as $x = x_1 + i x_2$. Observe then that
\begin{align*}
  L(x) &= L_1\left(x_1\right) + i L_1\left( x_2 \right)
\end{align*}
is a linear functional on $\ell_{\infty}$. Now, observe that $\norm{L} \geq 1$ almost by design. Since $L$ is a nonzero linear functional, we let $x$ be such that $L(x)\neq 0$, and set
\begin{align*}
  \alpha = \frac{\left\vert L(x) \right\vert}{L(x)}.
\end{align*}
We have that $\left\vert \alpha \right\vert = 1$ and $\alpha L(x) = \left\vert L(x) \right\vert$. We may then compute
\begin{align*}
  \left\vert L(x) \right\vert &= L\left( \alpha x \right)\\
                              &= \re\left( L\left( \alpha x \right) \right)\\
                              &= L_1\left( \alpha x \right)\\
                              &\leq \norm{L_1}\norm{\alpha x}\\
                              &= \norm{x}.
\end{align*}
In particular, this means $\norm{L}\leq 1$, so $\norm{L} = 1$.

We call such a shift-invariant extension of the limit to all of $\ell_{\infty}$ a \textit{Banach limit}. A quick observation gives that this limit functional cannot in fact be an algebra homomorphism. Considering the case of $a_n = \left( -1 \right)^{n}$, we have that $a_{n+1} = -a_n$, so that 
\begin{align*}
  L\left( S(a) \right) &= L\left( a \right)\\
                       &= L\left( -a \right)\\
                       &= - L(a),
\end{align*}
or that the Banach limit must be equal to $0$. However, if $L$ were instead an algebra homomorphism (where the multiplication operation on $\ell_{\infty}$ is given pointwise), we would have
\begin{align*}
  L\left( a^2 \right) &= L\left( \1 \right)\\
                      &= 1\\
                      &= \left( L\left( a \right) \right)^2,
\end{align*}
meaning that we would have $L\left( a \right) = \pm 1$, This means that such a limit that is an algebra homomorphism cannot be shift-invariant in the general case.

Regarding the case of $\left( -1 \right)^{n}$, we observe that from our work above that $0$ is the only Banach limit for this sequence. The sequences where every Banach limit assigns the same value for them are known as the \textit{almost convergent} sequences. Lorentz (1948) showed that an equivalent criterion for almost convergence is that, for all $\ve > 0$, there exists $p_0$ such that for all $p > p_0$ and all $n\in \N$, we have
\begin{align*}
  \left\vert \frac{x_n + x_{n+1} + \cdots + x_{n+p-1}}{p} - L \right\vert < \ve.
\end{align*}
Every convergent sequence is almost convergent by definition.
\section{Generalized Limits beyond Banach Limits}%
We have obtained one limit. However, there are lots of other extensions of the limit, each of which has norm $1$. In fact, if we consider the restriction of $\ell_{\infty}$ to the reals, we know from undergrad real analysis that
\begin{align*}
  p(x) &= \limsup_{n\rightarrow\infty} \left( \left( x_n \right)_n \right)
\end{align*}
is in fact a sublinear functional. Furthermore, since
\begin{align*}
  \liminf\left( \left( x_n \right)_n \right) &= -\limsup\left( \left( -x_n \right)_n \right),
\end{align*}
we can specify an extension to the limit functional to $L\colon \ell_{\infty}\rightarrow \R$ with $\norm{L} = 1$ such that for any $\left( x_n \right)_n\in \ell_{\infty}$, we have 
\begin{align*}
  \liminf_{n\rightarrow\infty}\left( \left( x_n \right)_n \right) \leq L(x)\leq \limsup_{n\rightarrow\infty}\left( \left( x_n \right)_n \right)
\end{align*}
However, these generalized limits may not align to the same requirement of shift-invariance as the aforementioned Banach limit. We can in fact construct a Banach limit from any generalized limit by using Césaro-type means: letting $x = \left( x_n \right)_n\in \ell_{\infty}$, we may define
\begin{align*}
  c_n &= \frac{1}{n} \sum_{k=1}^{n} a_k,
\end{align*}
and then, for a generalized limit $L$, defining $L'$ by
\begin{align*}
  L'\left( a \right) &= L\left(c\right).
\end{align*}
From undergrad real analysis, we know that if $\left( a_n \right)_n\rightarrow a$ is convergent, then we have
\begin{align*}
  \left( \frac{1}{n}\sum_{k=1}^{n}a_k \right)_{n} &\rightarrow a
\end{align*}
as well. Thus, we see that
\begin{align*}
  L'\left( S(a) \right) - L'(a) &= L'\left( \left( a_{n+1} \right)_n \right) - L\left( \left( a_n \right)_n \right)\\
                                &= L\left( \left( \frac{1}{n}\sum_{k=1}^{n}a_{k+1} - \frac{1}{n}\sum_{k=1}^{n}a_k \right)_n \right)\\
                                &= L\left( \left( \frac{1}{n}\left( a_{n+1}-a_1 \right) \right)_n \right)\\
                                &= \lim_{n\rightarrow\infty} \left( \frac{1}{n}\left( a_{n+1}-a_1 \right) \right)\\
                                &= 0,
\end{align*}
with the second-to-last equality emerging from the fact that $a\in \ell_{\infty}$, so the sequence $\left( a_{n+1}-a_1 \right)_n$ is necessarily bounded.

In particular, this means that there are plenty of Banach limits, since we can take \textit{any} generalized extension of the limit to $\ell_{\infty}$, then use this Césaro-type summation to yield a Banach limit.

Now, the question becomes: what are all the generalized limits? To understand this, we need to take a tour of topology. First, we recall an important structural theorem.
\begin{theorem}[Riesz Representation Theorem]
  Let $X$ be a compact Hausdorff space. Then,
  \begin{align*}
    C(X)^{\ast}\cong M_r(X),
  \end{align*}
  where $M_r(X)$ denotes the space of complex regular Borel measures on $X$.
\end{theorem}
Now, we consider the fact that $\ell_{\infty} $ can be identified with $ C_b\left(\N\right)$ when $N$ is equipped with the discrete topology. In particular, there is some $R > 0$ such that, for any $x\in \ell_{\infty}$, we have $\img(x) \subseteq B\left( 0,R \right)$. The universal property of the \href{https://en.wikipedia.org/wiki/Stone–Čech_compactification}{Stone--\v{C}ech compactification} thus gives a unique extension $\beta x\colon \beta\N\rightarrow \C$ such that $\beta x |_{\N} = x$, where $\beta\N$ is the Stone--\v{C}ech compactification of the naturals. In particular, we may thus identify $\ell_{\infty}\cong C_b\left( \N \right)$ with $C\left( \beta\N \right)$, so $\ell_{\infty}^{\ast}\cong M_r\left( \beta \N \right)$.

The next question emerges: what exactly \textit{is} $\beta\N$?
\begin{proposition}
  The space $\beta\N$ is the space of all ultrafilters on $\N$.
\end{proposition}
Recall that a filter $\mathcal{F}$ on a set $X$ is a family of subsets $\mathcal{F}$ that is ``closed upwards,'' in the sense that if $A\in \mathcal{F}$ and $A\subseteq B$, then $B\in \mathcal{F}$, and is such that $X\in \mathcal{F}$ and, if $A,B\in \mathcal{F}$, then so is $A\cap B$.

An ultrafilter $\mathcal{U}$ on $X$ is a maximal \textit{proper} filter, and has the property that if $A\in P(X)$, then either $A\in \mathcal{U}$ or $A^{c}\in \mathcal{U}$. An ultrafilter on $\N$ is called principal if it contains a finite subset, else it is called a non-principal ultrafilter (or free ultrafilter). The principal ultrafilters in $\beta\N$ can be identified with $\N$, since they are all of the form
\begin{align*}
  \mathcal{F}_q &= \set{A\subseteq N | q\in A}.
\end{align*}
If $\mathcal{U}$ is an ultrafilter, then we can define a limit of a sequence $\left( x_n \right)_n$ along $\mathcal{U}$ by saying that $\lim_{\mathcal{U}}\left( x_n \right)_n = L$ if and only if, for every $\ve > 0$, we have
\begin{align*}
  \set{n | \left\vert x_n - L \right\vert \leq \ve} &\in \mathcal{U}.
\end{align*}
Limits along principal ultrafilters are very boring --- if $\mathcal{F}_p$ is the principal ultrafilter corresponding to $p$, then $\lim_{\mathcal{F}_p}\left( x_n \right)_n = x_p$. However, limits along non-principal ultrafilters are where things start to get interesting.

A limit along a non-principal ultrafilter can be viewed as a special type of linear functional on $\ell_{\infty}$ --- specifically, it is a character (unital algebra homomorphism with range $\C$) with kernel containing $c_0$.\footnote{That its kernel contains $c_0$ follows from the fact that every non-principal ultrafilter contains the cofinite filter.} Some references then define limits along non-principal ultrafilters to be characters on the quotient $C^{\ast}$-algebra $\ell_{\infty}/c_0$. The relation comes from the fact that the character space of a general $C^{\ast}$-algebra $C^b(X)$, which is the space of continuous bounded functions on $X$, is given by $\beta X$, and that the remainder $\beta X \setminus X$ is the character space of the quotient $C^{\ast}$-algebra $C^b(X)/C_0(X)$.

%In fact, we can consider all the generalized limits as some some probability measure on $\beta\N\setminus \N$, and the Banach limits as the probability measures that are invariant under the action of $\Z$-translation (i.e., elementwise addition on the ultrafilter). In particular, since the probability measures form a compact convex subset of $M_r\left( \beta\N \right)$, it follows that every generalized limit is in the $w^{\ast}$-closure of the extreme points of the probability measures, which are the limits along \textit{all} ultrafilters, including the principal ultrafilters.
\end{document}
