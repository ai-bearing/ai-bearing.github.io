\documentclass[10pt]{mypackage}

% sans serif font:
%\usepackage{cmbright}
%\usepackage{sfmath}
%\usepackage{bbold} %better blackboard bold

%serif font + different blackboard bold for serif font
%\usepackage{homework}
\usepackage{newpxtext,eulerpx}
\renewcommand*{\mathbb}[1]{\varmathbb{#1}}

\fancyhf{}
\rhead{Avinash Iyer}
\lhead{Locally Compact Groups}

\setcounter{secnumdepth}{0}

\begin{document}
\RaggedRight
These are some notes I have taken from Gerald B. Folland's \textit{A Course in Abstract Harmonic Analysis}, with some other textbooks for various sources.
\section{Basic Properties of Topological Groups}%
\begin{definition}
  A \textit{topological group} is a group $G$ with a topology such that the operation
  \begin{align*}
    m\colon G\times G &\rightarrow G\\
    \left( x,y \right)&\mapsto xy
  \end{align*}
  is continuous with respect to the product topology on $G\times G$ and the operation
  \begin{align*}
    i\colon G&\rightarrow G\\
    x &\mapsto x^{-1}
  \end{align*}
  is continuous with respect to the topology on $G$.\newline

  For a topological group $G$, we denote the unit element as $1_G$, and we set
  \begin{align*}
    Ax &= \set{yx | y\in A}\\
    xA &= \set{xy | y\in A}\\
    A^{-1} &= \set{y^{-1} | y\in A}\\
    AB &= \set{xy | x\in A,y\in B}
  \end{align*}
  for all subsets $A,B\subseteq G$ and elements $x\in G$.
\end{definition}
\begin{definition}
  A subset $A\subseteq G$ is called \textit{symmetric} if $A = A^{-1}$.
\end{definition}
\begin{proposition}
  Let $G$ be a topological group.
  \begin{enumerate}[(i)]
    \item The topology of $G$ is invariant under translations and inversion; that is, if $U$ is open, then $xU,Ux,U^{-1},AU,UA$ are open for any $x\in G$ and subset $A\subseteq G$.
    \item For every neighborhood $U$ of $1_G$, there is a symmetric neighborhood $V$ of $1_G$ such that $VV \subseteq U$.
    \item If $H$ is a subgroup of $G$, so is $ \overline{H} $.
    \item Every open subgroup of $G$ is closed.
    \item If $A$ and $B$ are compact sets in $G$, so is $AB$.
  \end{enumerate}
\end{proposition}
\begin{proof}\hfill
  \begin{enumerate}[(i)]
    \item This is equivalent to the separate continuity of $\left( x,y \right)\mapsto xy$ and $x\mapsto x^{-1}$; furthermore, 
      \begin{align*}
        AU &= \bigcup_{x\in A} xU\\
        UA &= \bigcup_{x\in A}Ux.
      \end{align*}
    \item Since $\left( x,y \right)\mapsto xy$ is continuous at $1_G$, then for every neighborhood $U$ of $1_G$, there are neighborhoods $W_1,W_2\subseteq U$. We may take $V = W_1\cap W_2\cap W_1^{-1}\cap W_2^{-1}$.
    \item For $x,y\in \overline{H}$, there are nets $\left( x_{\alpha} \right)_{\alpha}\rightarrow x$ and $\left( y_{\alpha} \right)_{\alpha}\rightarrow y$; since $\left( x_{\alpha}y_{\alpha} \right)\rightarrow xy$ and $\left( x_{\alpha}^{-1} \right)_{\alpha}\rightarrow x^{-1}$ by continuity of the operations, we have $xy,x^{-1}\in \overline{H}$.
    \item If $H$ is open, then so are all the cosets $xH$; since $G\setminus H$ is the union of all the cosets of $H$ except for $H$ itself, $G\setminus H$ is open, so $H$ is closed.
    \item Since $A\times B$ is compact, and $AB$ is the continuous image of $A\times B$ under $\left( x,y \right)\mapsto xy$, we have $AB$ is compact.
  \end{enumerate}
\end{proof}
Now, if $H$ is a subgroup of $G$, we let $G/H$ be the space of left cosets of $H$, and $q\colon G\rightarrow G/H$ is the canonical quotient map, we may impose the quotient topology on $G/H$, meaning that $U\subseteq G/H$ is open if and only if $q^{-1}(U)$ is open. Thus, $q$ maps open sets in $G$ to open sets in $G/H$, as if $V\subseteq G$ is open, $q^{-1}\left(q(V)\right) = VH$ is also open, so $q(V)$ is open.
\begin{proposition}
  Let $H$ be a subgroup of a topological group $G$.
  \begin{enumerate}[(i)]
    \item If $H$ is closed, then $G/H$ is Hausdorff.
    \item If $G$ is locally compact, so is $G/H$.
    \item If $H$ is normal, then $G/H$ is a topological group.
  \end{enumerate}
\end{proposition}
\begin{proof}\hfill
  \begin{enumerate}[(i)]
    \item If $ \overline{x} = q(x) $ and $ \overline{y} = q(y) $ are distinct points in $G/H$, and since $H$ is closed, $xHy^{-1}$ is a closed set that does not contain $1_G$. There is a symmetric neighborhood $U$ of $1_G$ such that $UU \cap xHy^{-1} = \emptyset$; since $U = U^{-1}$ and $H = HH$ ($H$ is a subgroup), we have $1_G\notin UxH\left( Uy \right)^{-1} = \left( UxH \right)\left( UyH \right)^{-1}$, so $UxH\cap UyH = \emptyset$. Therefore, $q\left( Ux \right)$ and $q\left( Uy \right)$ are disjoint neighborhoods of $ \overline{x} $ and $ \overline{y} $.
    \item If $U$ is a compact neighborhood of $1_G$, $q\left( Ux \right)$ is a compact neighborhood of $q(x)$ in $G/H$.
    \item If $x,y\in G$, and $U$ is a neighborhood of $G/H$, continuity of multiplication in $G$ implies that there are neighborhoods $V$ of $x$ and $W$ of $y$ such that $VW\subseteq q^{-1}(U)$. We see that $q(V)$ and $q(W)$ are neighborhoods of $q(x)$ and $q(y)$ such that $q(V)q(W)\subseteq U$, meaning multiplication is continuous in $G/H$. Similarly, inversion is continuous.
  \end{enumerate}
\end{proof}
\begin{corollary}
  If $G$ is T1, then $G$ is Hausdorff, and if $G$ is not T1, then $ \overline{\set{1_G}} $ is a closed normal subgroup, and $G/ \overline{\set{1_G}}$ is a Hausdorff topological group.
\end{corollary}
\begin{proof}
  Since singletons are closed in any T1 space, the first assertion follows from part (i) in the previous proposition by taking $H = \set{1_G}$.\newline

  To see the second assertion, we note that $ \overline{\set{1_G}} $ is a subgroup, and it is the smallest closed subgroup of $G$; it is normal, as otherwise we would obtain a smaller closed subgroup by intersection with one of the conjugates, meaning the result follows from parts (i) and (iii) in the previous proposition by taking $H = \overline{\set{1_G}}$.
\end{proof}
Thus, without loss of generality, we may assume that a topological group is Hausdorff (else take $G/ \overline{\set{1_G}}$), and when we talk about locally compact groups, we are talking about topological groups that are locally compact and Hausdorff.
\begin{proposition}
  Every locally compact  group $G$ has a subgroup $G_0$ that is open, closed, and $\sigma$-compact.
\end{proposition}
\begin{proof}
  Let $U$ be a symmetric compact neighborhood of $1_G$, let $U_n = \prod_{i=1}^{n}U$, and let
  \begin{align*}
    G_0 &= \bigcup_{n=1}^{\infty}U_n.
  \end{align*}
  Then, $G_0$ is the group generated by $U$, so it is a subgroup; $G_0$ is open since $U_{n+1}$ is a neighborhood of $U_n$ for all $n$, and so $G_0$ is closed as all open subgroups are closed. Finally, since each $U_n$ is a finite product of compact subsets of $G$, $G_0$ is $\sigma$-compact.
\end{proof}
We thus see that $G_0$ is the disjoint union of cosets of $G_0$, meaning $G$ is a disjoint union of $\sigma$-compact spaces. In particular, if $G$ is connected, then $G$ is necessarily $\sigma$-compact.
\begin{definition}
  Let $f\colon G\rightarrow \C$ be a function. The \textit{translates} of $f$ via $y\in G$ are defined by
  \begin{align*}
    L_yf(x) &= f\left( y^{-1}x \right)\\
    R_yf(x) &= f\left( xy \right).
  \end{align*}
  Note that the maps $y\mapsto L_y$ and $y\mapsto R_y$ are group homomorphisms.\newline

  The function $f$ is called left/right uniformly continuous if
  \begin{align*}
    \norm{L_yf-f}_{u} &\rightarrow 0\\
    \norm{R_yf - f}_{u} &\rightarrow 0
  \end{align*}
  as $y\rightarrow 1_{G}$ respectively.
\end{definition}
\begin{proposition}
  If $f\in C_c(G)$, then $f$ is left and right uniformly continuous.
\end{proposition}
\begin{proof}
  We will prove this for $R_yf$.\newline

  If $f\in C_c(G)$, and $\ve > 0$, then for every $x\in K = \supp(f)$, there is a neighborhood $U_x$ of $1_G$ such that
  \begin{align*}
    \left\vert f(xy) - f(x) \right\vert &< \frac{1}{2}\ve
  \end{align*}
  for any $y\in U_x$. Similarly, there is a symmetric neighborhood $V_x$ of $1_G$ such that $V_xV_x\subseteq U_x$; the sets $xV_x$ cover $K$, so there exist $x_1,\dots,x_n\in K$ such that $K\subseteq \bigcup_{j=1}^{n}x_jV_{x_j}$.\newline

  Let $V = \bigcap_{j=1}^{n}V_{x_j}$. If $x\in K$, then there is some $j$ such that $x_j^{-1}x\in V_{x_j}$, so $xy = x_j\left( x_{j}^{-1}x \right)y\in x_jU_{x_j}$, so
  \begin{align*}
    \left\vert f\left( xy \right) - f\left( x \right) \right\vert &\leq \left\vert f\left( xy \right) - f\left( x_j \right) \right\vert + \left\vert f\left( x_j \right) - f\left( x \right) \right\vert\\
                                                                  &< \frac{1}{2}\ve + \frac{1}{2}\ve\\
                                                                  &= \ve,
  \end{align*}
  for any $y\in V$, meaning that $\norm{R_yf-f}_{u} < \ve$. Similarly, if $xy\in K$, then $\left\vert f(xy) - f(x) \right\vert < \ve$; meanwhile, if $x,xy\notin K$, then $f(x) = f(xy) = 0$, so we are done.
\end{proof}
\section{Haar Measure}%
\begin{definition}
We define a subset of $C_c(G)$ to be
\begin{align*}
  C_c^{+}(G) &= \set{f\in C_c(G) | f\geq 0,f\neq 0}.
\end{align*}
\end{definition}
\begin{definition}
  A left/right Haar measure on $G$ is a nonzero Radon measure $\mu$ on $G$ such that $\mu\left( xE \right) = \mu\left( E \right)$ for every Borel $E\subseteq G$ and all $x\in G$.
\end{definition}
\begin{proposition}
  Let $\mu$ be a Radon measure on the locally compact group $G$, and let $ \widetilde{\mu}\left( E \right) = \mu\left( E^{-1} \right) $. Then, the following hold:
  \begin{enumerate}[(a)]
    \item $\mu$ is a left Haar measure if and only if $ \widetilde{\mu} $ is a right Haar measure.
    \item $\mu$ is a left Haar measure if and only i $ \int_{}^{} L_yf\:d\mu = \int_{}^{} f\:d\mu $ for all $f\in C_c^{+}\left( G \right)$ and every $y\in G$.
  \end{enumerate}
\end{proposition}
\begin{proof}
  The result in (a) follows from basic properties of the inverse.\newline

  To see (b), note that for any Radon measure $\mu$, one has $\int_{}^{} L_yf\:d\mu = \int_{}^{} f\:d\mu_y$, where $\mu_y(E) = \mu\left( yE \right)$, which follows from approximation via simple functions. Thus, if $\mu$ is a Haar measure, then $\int_{}^{} L_yf\:d\mu = \int_{}^{} f\:d\mu$ for all $f\in C_c^{+}(G)$, so it holds for all $f\in C_c\left( G \right)$. The measure $\mu$ is unique from the \href{https://desvl.xyz/2020/09/19/The-Riesz-Markov-Kakutani-Representation-Theorem/#This-post}{Riesz--Markov--Kakutani Representation Theorem}.
\end{proof}
Now, our focus turns to the question of establishing the existence and (essential) uniqueness of the Haar measure.
\begin{theorem}
  Every locally compact group $G$ possesses a left Haar measure $\lambda$.
\end{theorem}
\begin{proof}
  We will construct $\lambda$ as a linear functional on $C_c(G)$.\newline

  Let $f,\phi\in C_c^{+}(G)$. We define $\left( f:\phi \right)$ to be the infimum of all such finite sums $\sum_{j=1}^{n}c_j$ such that 
  \begin{align*}
    f\leq \sum_{j=1}^{n}c_jL_{x_j}\phi
  \end{align*}
  for some $x_1,\dots,x_n\in G$. Such a value necessarily exists as $\supp(f)$ can be covered by some finite number of translates of $\phi^{-1}\left( 1/2\norm{\phi}_u,\infty \right)$, meaning that $\left( f:\phi \right)\leq 2N\norm{f}_{u}/\norm{\phi}_u$. We see the following:
  \begin{enumerate}[(i)]
    \item $ \ds \left( f:\phi \right) = \left( L_yf:\phi \right) $;
    \item $ \ds \left( f_1 + f_2:\phi \right) \leq \left( f_1:\phi \right) + \left( f_2:\phi \right) $;
    \item $\ds \left( cf:\phi \right) = c\left( f:\phi \right)$ for any $c \geq 0$;
    \item $\ds \left( f_1:\phi \right) \leq \left( f_2:\phi \right)$ whenever $f_1\leq f_2$;
    \item $ \ds \left( f:\phi \right) \geq \norm{f}_{u}/\norm{\phi}_{u} $;
    \item $\ds \left( f:\phi \right)\leq \left( f:\psi \right)\left( \psi:\phi \right)$ for any $\psi\in C_c^{+}\left( G \right)$.
  \end{enumerate}
  To see (vi), notice that if $f\leq \sum_{i=1}^{n}c_iL_{x_i}\phi$ and $\psi\leq \sum_{j=1}^{m}b_jL_{y_j}\phi$, then $f\leq \sum_{i=1}^{n}\sum_{j=1}^{m}c_ib_jL_{x_jy_j}\phi$.\newline

  We fix a function $f_0\in C_c^{+}\left( G \right)$, and define
  \begin{align*}
    I_{\phi}\left( f \right) &= \frac{\left( f:\phi \right)}{\left( f_0:\phi \right)}.
  \end{align*}
  This functional is left-invariant, subadditive, homogeneous, and monotone, and also satisfies
  \begin{align*}
    \left( f_0:f \right)^{-1}\leq I_{\phi}\left( f \right)\leq \left( f:f_0 \right).
  \end{align*}
  Now, $I_{\phi}$ is not necessarily additive, but on a neighborhood it is very close to being so.
  \begin{lemma}
    If $f_1,f_2\in C_c^{+}\left( G \right)$, and $\ve > 0$, then there is a neighborhood $V$ of $1_G$ such that $I_{\phi}\left( f_1 \right) + I_{\phi}\left( f_2 \right) \leq I_{\phi}\left( f_1 + f_2 \right) + \ve$ whenever $\supp\left( \phi \right)\subseteq V$.
  \end{lemma}
  \begin{proof}[Proof of Lemma]
    Fix $g\in C_c^{+}\left( G \right)$ such that $g = 1$ on $\supp\left( f_1 + f_2 \right)$, and let $\delta$ be a (to be determined) positive number. Let $h = f_1 + f_2 + \delta g$, and let $h_i = f_i/h$ for each $i$; note that $h_i = 0$ whenever $f_i = 0$.\newline

    Then, we see that $h_i\in C_c^{+}\left( G \right)$, so there is a neighborhood $V$ of $1_G$ such that $\left\vert h_i\left( x \right) - h_i\left( y \right) \right\vert < \delta$ for each $i$ and all $y$ such that $y^{-1}x\in V$.\newline

    Suppose $\phi\in C_c^{+}\left( G \right)$ and $\supp\left( \phi \right) \subseteq V$. If $h\leq \sum_{j=1}^{n}c_jL_[x_j]\phi$, then
    \begin{align*}
      f_i(x) &= h(x)h_i(x)\\
             &\leq \sum_{j=1}^{m} c_j\phi\left( x_j^{-1}x \right)h_i(x)\\
             &\leq \sum_{j=1}^{m}c_j\phi\left( x_j^{-1}x \right)\left( h_i\left( x_j \right) + \delta \right),
    \end{align*}
    since $\left\vert h_i(x)-h_i\left( x_j \right) \right\vert < \delta$ whenever $x_j^{-1}x\in \supp\left( \phi \right)$. Since $h_1 + h_2 \leq 1$, we have
  \end{proof}
\end{proof}
\end{document}
