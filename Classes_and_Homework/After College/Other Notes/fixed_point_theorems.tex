\documentclass[10pt]{mypackage}

% sans serif font:
%\usepackage{cmbright}
%\usepackage{sfmath}
%\usepackage{bbold} %better blackboard bold

%\usepackage{homework}
\usepackage{notes}
\usepackage{mlmodern}
%\usepackage{newpxtext,eulerpx,eucal}
%\renewcommand*{\mathbb}[1]{\varmathbb{#1}}
\usepackage{parskip}

\fancyhf{}
\fancyhead[R]{Avinash Iyer}
\fancyhead[L]{Fixed-Point Theorems}
\fancyfoot[C]{\thepage}

\setcounter{secnumdepth}{0}

\begin{document}
\RaggedRight
Fixed-point theorems are a very important part of topology and analysis. We'll discuss and prove some from these fields, as well as some applications.
\section{Fixed-Point Theorems in Topology}%
The classic, most well-known fixed point theorem is, of course, the Brouwer fixed-point theorem. The proof we use proceeds via cohomology theory. Some background can be found in one of the sections of my \href{https://ai.avinash-iyer.com/Classes_and_Homework/After\%20College/Classes/Differential\%20Topology/differential_topology_notes.pdf}{differential topology notes}.
\begin{theorem}
  Let $f\colon B\left( 0,1 \right)\rightarrow B\left( 0,1 \right)$ be a continuous self-map of the closed unit ball in $\R^{n}$. Then, $f$ has a fixed point.
\end{theorem}
\begin{proof}
  Suppose toward contradiction that there is no fixed point. Then, for any $x\in B\left( 0,1 \right)$, we can draw a unique line from $x$ to $f(x)$ and extend it in a manner such that it hits the unit sphere, $S\left( 0,1 \right)$. This defines a continuous function $g\colon B\left( 0,1 \right)\rightarrow S\left( 0,1 \right)$ such that $g|_{S\left( 0,1 \right)} = \id$.

  Yet, since $B\left( 0,1 \right)$ is contractible, its cohomology outside of $H^{0}$ is trivial by the Poincaré Lemma, while the cohomology of $S\left( 0,1 \right)$ is nontrivial at $H^{n-1}$. Yet, since $g$ is a deformation retract, the existence of such a $g$ would imply that $S\left( 0,1 \right)$ and $B\left( 0,1 \right)$ are homotopy-equivalent, but the fact that their cohomologies differ shows that this cannot be the case.
\end{proof}
Brouwer's Fixed-Point Theorem can be proven in a variety of different ways, including Sperner's Lemma (which is a form of the Brouwer Fixed-Point Theorem for triangulations), heavier differential topology such as Stokes's Theorem, and a proof using the fact that the game of \href{https://en.wikipedia.org/wiki/Hex_(board_game)}{Hex} always ends.

However, we will be using this to prove a different, more general theorem that is again related to cohomology theory.
\begin{theorem}[Lefschetz Fixed-Point Theorem]
  Let $M$ be a triangulated, compact, smooth manifold, and let $f\colon M\rightarrow M$ be a smooth map that preserves the simplicial structure of $M$. Let $f_k^{\ast}\colon H^k_{\operatorname{DR}}\left( M \right)\rightarrow H^{k}_{\operatorname{DR}}\left( M \right)$ denote the induced map in cohomology. Then, if
  \begin{align*}
    L(f) &= \sum_{k=0}^{n} \left( -1 \right)^{k} \tr\left( f_k^{\ast} \right)\\
         &\neq 0,
  \end{align*}
  then $f$ has a fixed-point.
\end{theorem}
\begin{proof}
  By abuse of notation, we let $f^{\ast}\colon H^{\ast}\left( M;\R \right)\rightarrow H^{\ast}\left( M;\R \right)$ be the corresponding map in simplicial cohomology, following from the de Rham theorem.

  Let $\Delta\subseteq M$ be a simplex; then, by the definition of $f$, we observe that $f\left( \Delta \right)\subseteq M$ is also another simplex, which we call $\Lambda$. If $\Delta = \Lambda$, then we obtain smooth self-map of the simplex, which necessarily has a fixed point by the Brouwer fixed-point theorem.

  Now, we assume that $f$ has no fixed-points. We will show that this implies that the Lefschetz number $L(f)$ is equal to zero.

  From the de Rham theorem, we observe that simplicial cochain $I_{\omega}$ on $M$ is then defined by
  \begin{align*}
    I_{\omega}\left( \Delta \right) &= \int_{\Delta}^{} \omega,
    \intertext{and so by taking the pullback,}
    f^{\ast}\left( I_{\omega} \right)\left( \Delta \right) &= \int_{\Delta}^{} f^{\ast}\omega\\
                                                           &= I_{f^{\ast}\omega}\left( \Delta \right).
  \end{align*}
  By even more abuse of notation, we let $f^{\ast}\colon C^{k}\left( M;\R \right)\rightarrow C^{k}\left( M;\R \right)$ be this map from $I_{\omega}$ to $I_{f^{\ast}\omega}$. Observe that by our assumption, for any $k$-simplex $\sigma\subseteq M$, we have $f\left( \sigma \right)\nsubseteq \sigma$, so by selecting a $k$-form supported on $\sigma$, we have $I_{f^{\ast}\omega}\left( \sigma \right) = 0$. In particular, the map $f_k^{\ast}\colon C^k\left( M;\R \right)\rightarrow C^{k}\left( M;\R \right)$ has no eigenvectors, and thus $\tr\left( f_k^{\ast} \right) = 0$. Thus,
  \begin{align*}
    0 &= \sum_{k=0}^{n} \left( -1 \right)^{k}\tr\left( f_k^{\ast} \right).
  \end{align*}
  Now, observe that these $f_k^{\ast}$ are maps on the cochains, not the underlying cohomology. Therefore, we dedicate the rest of the proof to showing that this alternating sum is equal on cohomology. To do this, we use the fact that vector spaces are projective objects, so short exact sequences of the form
  \begin{center}
    % https://tikzcd.yichuanshen.de/#N4Igdg9gJgpgziAXAbVABwnAlgFyxMJZABgBpiBdUkANwEMAbAVxiRGJAF9T1Nd9CKAIzkqtRizYBBLjxAZseAkQBMo6vWatEIAEKzeigUQDM68VrYBhA-L5LByACznNknR05iYUAObwiUAAzACcIAFskMhAcCCQhbmCwyMQRGLjEFUSQUIikNXSkE2zclLNCxCcvTiA
\begin{tikzcd}
0 \arrow[r] & A \arrow[r] & B \arrow[r] & C \arrow[r] & 0
\end{tikzcd}
  \end{center}
  can be expressed as follows:
  \begin{center}
    % https://tikzcd.yichuanshen.de/#N4Igdg9gJgpgziAXAbVABwnAlgFyxMJZABgBpiBdUkANwEMAbAVxiRGJAF9T1Nd9CKAIzkqtRizYBBLjxAZseAkQBMo6vWatEIKQB09ENMzgACAMKzeigUQDM68VraXu1-spQAWR5sk6OTjEYKABzeCJQADMAJwgAWyQyEBwIJCE3EFiE9OpUpBVM7MTENRS0xDsiuJKHcqQvIM4gA
\begin{tikzcd}
0 \arrow[r] & A \arrow[r] & A\oplus C \arrow[r] & C \arrow[r] & 0
\end{tikzcd}
  \end{center}
  In particular, if we write commuting linear maps as follows:
  \begin{center}
    % https://tikzcd.yichuanshen.de/#N4Igdg9gJgpgziAXAbVABwnAlgFyxMJZABgBpiBdUkANwEMAbAVxiRGJAF9T1Nd9CKAIzkqtRizYBBLjxAZseAkQBMo6vWatEIKQB09ENMzgACAMKzeigUQDM68VraXu1-spQAWR5sk6ON3k+JUESUiExP212K2CbT2QRSI0JGJkghQ8wtRSnf10DIxMLOKzQ+wiotJcykNtvKtTnAK4xGCgAc3giUAAzACcIAFskMhAcCCQhIMGR6epJpBVZodHENQmpxDtV+Z3F7a899YBWQ6QANhOri8QAdhuHu4AOJ5e7gE4nhy2kD-yMQAFnE5utNktnoC2H1QWsFn9EJdmgVOm1OEA
\begin{tikzcd}
0 \arrow[r] & A \arrow[r] \arrow[d, "g"] & A\oplus C \arrow[r] \arrow[d, "f"] & C \arrow[r] \arrow[d, "h"] & 0 \\
0 \arrow[r] & A \arrow[r]                & A\oplus C \arrow[r]                & C \arrow[r]                & 0
\end{tikzcd}
  \end{center}
  then
  \begin{align*}
    \left[ f \right]_{\gamma} &= \begin{pmatrix}\left[ g \right]_{\alpha} & K \\ 0 & \left[ h \right]_{\beta}\end{pmatrix},
  \end{align*}
  so that $\tr\left( f \right) = \tr\left( g \right) + \tr\left( h \right)$.

  Now, relabeling $f_k^{\ast}\colon C^{k}\left( M;\R \right)\rightarrow C^{k}\left( M;\R \right)$ as $g_k^{\ast}$, we have the following commutative diagrams that emerge from the map $f$.
  \begin{center}
    % https://tikzcd.yichuanshen.de/#N4Igdg9gJgpgziAXAbVABwnAlgFyxMJZABgBpiBdUkANwEMAbAVxiRGJAF9T1Nd9CKAIzkqtRizYAtAHrAA1py48QGbHgJEATKOr1mrRCADCM+ct7qBRAMy7xBtgCE58gNRCl3S-00oALPb6kkYc3qp8GoIkpEJiwYbsFhFWfsgicXoSibLm4Wq+0TqZDiEmZskFUbax8dnOrh5eKlXWAbVZjqGVkW0xNnVdSfm9aSIDnWUueS2jRaQTpTkVI6nRdosJbAASK7NrRIGb9d2rhURk-oNlYfvnwqRXk4nTPQcoOk9L0ns+1Sh2L5bIy7GZ-PqBIEnYZiGBQADm8CIoAAZgAnCAAWyQZBAOAgSE8KnRWMJ1HxSC04RJ2MQOjxBMQNmpGNpdgZSH8LNJiAArOTGQA2bm0wUCpAAdhFkvFiAAHNL5bKAJyK9kUpXfIwAIwA+gomskaZTZRLnmx4br5HIADo2uhwHDNVGsskcxBirUgABeVqNrsQQlxGqERJdPNDsqEVOJAejUeZsYj6sZQi5SdpQn57qEwozhM9Ial+cDZpzCpLQjlUdVlZThOrXpRVtt9sdzpAxsD9KL5qMvrBnbjIhznuBID1eQonCAA
\begin{tikzcd}
0 \arrow[r] & Z^{k} \arrow[r] \arrow[d, "z_k"] & C^k \arrow[r] \arrow[d, "g_k^{\ast}"] & B^{k+1} \arrow[r] \arrow[d, "b_{k+1}"] & 0 \\
0 \arrow[r] & Z^k \arrow[r]                    & C^k \arrow[r]                         & B^{k+1} \arrow[r]                      & 0 \\
            &                                  &                                       &                                        &   \\
0 \arrow[r] & B^k \arrow[r] \arrow[d, "b_k"]   & Z^k \arrow[r] \arrow[d, "z_k"]        & H^k \arrow[r] \arrow[d, "f_k^{\ast}"]  & 0 \\
0 \arrow[r] & B^k \arrow[r]                    & Z^k \arrow[r]                         & H^k \arrow[r]                          & 0
\end{tikzcd}
  \end{center}
  Using the trace formula we established above, we get
  \begin{align*}
    \tr\left( g_k^{\ast} \right) &= \tr\left( b_{k+1} \right) + \tr\left( z_k \right)\\
    \tr\left( z_k \right) &= \tr\left( f_k^{\ast} \right) + \tr\left( b_k \right).
  \end{align*}
  Therefore, we get
  \begin{align*}
    0 &= \sum_{k=0}^{n} \left( -1 \right)^{k} \tr\left( g_k^{\ast} \right)\\
      &= \sum_{k=0}^{n} \left( -1 \right)^{k} \left( \tr\left( f_k^{\ast} \right) + \tr\left( b_k \right) + \tr\left( b_{k+1} \right) \right)\\
      &= \sum_{k=0}^{n} \left( -1 \right)^{k} \tr\left( f_k^{\ast} \right).
  \end{align*}
\end{proof}
\section{Fixed-Point Theorems in Analysis}%
There are a number of fixed-point theorems in analysis that are extraordinarily useful. The original fixed-point theorem is known as the contraction mapping theorem.
\begin{theorem}[Contraction Mapping Theorem]
  Let $X$ be a complete metric space, and let $f\colon X\rightarrow X$ be a self-map that satisfies, for all $x,y\in X$,
  \begin{align*}
    d\left( f(x),f(y) \right) &\leq \lambda d\left( x,y \right)
  \end{align*}
  for some $0 \leq \lambda < 1$. Then, there is some $x_0\in X$ such that $f\left(x_0\right) = x_0$.
\end{theorem}
\begin{proof}
  Let $x_1 \in X$ be some element. Define $x_{n+1} = f\left( x_n \right)$. Then, we have
  \begin{align*}
    d\left( x_3,x_2 \right) &\leq \lambda d\left( x_2,x_1 \right),
  \end{align*}
  whence
  \begin{align*}
    d\left( x_m,x_n \right) &\leq \frac{\lambda^{n-1}}{1-\lambda} d\left( x_2,x_1 \right).
  \end{align*}
  In particular, the sequence $\left( x_n \right)_n$ is Cauchy, so there is some $x_0$ such that $\left( x_n \right)_n\rightarrow x_0$ since $X$ is complete, so that
  \begin{align*}
    d\left( f\left( x_0 \right),x_0 \right) &= 0.
  \end{align*}
  Thus, we have a fixed point.
\end{proof}
The contraction mapping theorem can be extended to a more general case. This is known as the Schauder Fixed-Point Theorem. To prove it, we use a technical lemma.
\begin{lemma}
  Let $K$ be a compact subset of $X$, $\ve > 0$, and $A$ the finite subset such that
  \begin{align*}
    K &\subseteq \bigcup_{a\in A} U\left( a,\ve \right).
  \end{align*}
  Let $m_a\colon X\rightarrow \R$ be a function defined by $m_a(x) = 0$ if $\norm{x-a} \geq \ve$ and $m_a(x) = \ve - \norm{x-a}$ if $\norm{x-a}\leq \ve$. Then, if $\phi_A\colon K\rightarrow X$ is defined by
  \begin{align*}
    \phi_A(x) &= \frac{\sum_{a\in A} m_a(x)a}{\sum_{a\in A}m_a(x)},
  \end{align*}
  then $\phi_A$ is continuous with
  \begin{align*}
    \norm{\phi_A(x) - x} < \ve
  \end{align*}
  for all $x\in K$.
\end{lemma}
\begin{proof}
  For each $a\in A$, we have $m_a(x) \geq 0$, and $\sum_{a\in A} m_a(x) > 0$ for all $x\in K$. Therefore, $\phi_A$ is well-defined on $K$. The fact that $\phi_A$ is continuous follows from the fact that for each $a\in A$, $m_a\colon K\rightarrow [0,\ve]$ is continuous (since it is the composition of a collection of continuous maps on $X$).

  If $x\in K$, then
  \begin{align*}
    \phi_A(x) - x &= \frac{\sum_{a\in A}\left( m_a(x) \right)\left( a-x \right)}{\sum_{a\in A}m_a(x)}.
  \end{align*}
  If $m_a(x) > 0$, then $\norm{x-a} < \ve$, hence
  \begin{align*}
    \norm{\phi_A(x) - x} &\leq \frac{\sum_{a\in A}m_a(x)\norm{a-x}}{\sum_{a\in A}m_a(x)}\\
                         &< \ve.
  \end{align*}
\end{proof}
\begin{theorem}[Schauder Fixed-Point Theorem]
  Let $E$ be a nonempty, closed, convex subset of a normed vector space, and let $f\colon E\rightarrow E$ be such that $f(E)\subseteq E$ is contained in a compact subset. Then, $f$ admits a fixed-point.
\end{theorem}
\begin{proof}
  Let $K = \overline{f(E)}$, with $K\subseteq E$. For each $n$, define $A_n$ to be such that
  \begin{align*}
    K &\subseteq \bigcup_{a\in A_n} U\left( a,1/n \right).
  \end{align*}
  For each $n$, write $\phi_n = \phi_{A_n}$ as in the lemma. The definition of $\phi_n$ is such that $\phi_n(K)\subseteq \conv(K)\subseteq E$, where the latter emerges from the fact that $E$ is convex. Write $f_n = \phi_n\circ f$, which maps $E$ into $E$, so that
  \begin{align*}
    \norm{f_n(x) - f(x)} &< 1/n
  \end{align*}
  for all $x\in E$. Letting $X_n = \Span\left( A_n \right)$, and $E_n = E\cap X_n$. Then, $X_n$ is a finite-dimensional normed space, $E_n$ is a compact convex subset of $X_n$, and $f_n$ maps $E_n\rightarrow E_n$ and is continuous. By Brouwer's Fixed-Point Theorem, it follows that there is some $x_n$ in $E_n$ such that $f_n\left( x_n \right) = x_n$.

  Since $\left( f\left( x_n \right) \right)_n$ is a sequence in the compact set $K$, there is some point $x_0$ and a subsequence $\left( f\left( x_{n_j} \right) \right)_j$ such that $f\left( x_{n_j} \right)\rightarrow x_0$. Since $f_{n_j}\left( x_{n_j} \right) = x_{n_j}$, it follows that
  \begin{align*}
    \norm{x_{n_j} - x_0} &\leq \norm{f_{n_j}\left( x_{n_j} \right) - f\left( x_{n_j} \right)} + \norm{f\left( x_{n_j} \right) - x_0}\\
                         &\leq \frac{1}{n_j} + \norm{f \left( x_{n_j} \right) - x_0}.
  \end{align*}
  Therefore, $\left( x_{n_j} \right) \rightarrow x_0$, and thus, by continuity of $f$, $f\left( x_0 \right) = \lim_{j\rightarrow\infty}f\left( x_{n_j} \right) = x_0$.
\end{proof}
The next fixed-point theorem we will prove is the Markov--Kakutani Fixed-Point Theorem; we will use it to show a particularly useful method to find the amenability of a group, which we will discuss in the section on applications. It is also useful for proving a more high-powered fixed-point theorem known as the Ryll--Nardzewski Fixed-Point Theorem, which we will use to establish the existence of Haar measure on a group.
\begin{theorem}[Markov--Kakutani Fixed-Point Theorem]
  Let $K$ be a nonempty compact convex subset of a locally convex topological vector space $X$, and let $ \mathcal{F} $ be a commuting family of continuous affine maps of $K$ into itself. Then, there is $x_0\in K$ such that $T\left( x_0 \right) = x_0$ for all $T\in \mathcal{F}$.
\end{theorem}
\begin{proof}
  Let $T\in \mathcal{F}$, and for $n\geq 1$, define
  \begin{align*}
    T^{(n)} &= \frac{1}{n}\sum_{k=0}^{n-1} T^{k}.
  \end{align*}
  If $S,T\in \mathcal{F}$, the commutation of $S$ and $T$ implies that $S^{(n)}T^{(m)} = T^{(m)}S^{(n)}$. Let 
  \begin{align*}
    \mathcal{K} = \set{T^{(n)}(K) | T\in \mathcal{F},n\geq 1}.   
  \end{align*}
   Each of the $ \mathcal{K} $ is compact (by continuity) and convex (by the fact that each $T\in \mathcal{F}$ is affine).

  If $T_1,\dots,T_p\in \mathcal{F}$, and $n_1,\dots,n_p\geq 1$, then by commutativity of $ \mathcal{F} $, we have
  \begin{align*}
    T_1^{\left(n_1\right)}\cdots T_p^{\left( n_p \right)}\left( K \right) &\subseteq \bigcap_{j=1}^{p} T_j^{\left( n_j \right)}\left( K \right).
  \end{align*}
  In particular, this means that $\mathcal{K}$ has the finite intersection property. Therefore, by compactness, we have that there is
  \begin{align*}
    x_0 &\in \bigcap_{B\in \mathcal{K}} B.
  \end{align*}
  We claim that $x_0$ is the desired common fixed-point for maps in $\mathcal{F}$. If $T\in \mathcal{F}$, then $x_0\in T^{(n)}\left( K \right)$. Then, there is $x\in K$ such that
  \begin{align*}
    x_0 &= \frac{1}{n}\left( x + T(x) + \cdots + T^{n-1}(x) \right).
  \end{align*}
  In particular, we get
  \begin{align*}
    T\left( x_0 \right) - x_0 &= \frac{1}{n}\left( T(x) + \cdots + T^{n}(x) \right) - \frac{1}{n}\left( x + T(x) + \cdots + T^{n-1}(x) \right)\\
                              &= \frac{1}{n}\left( T^n(x)-x \right)\\
                              &\in \frac{1}{n} \left( K-K \right).
  \end{align*}
  Since $K$ is compact, so too is $K-K$. If $U$ is an open neighborhood of $0$ in $X$, there is $n\geq 1$ such that $\frac{1}{n}\left( K-K \right) \subseteq U$. In particular, this means that $T\left(x_0\right) - x_0\in U$ for every open neighborhood $U$ of $0$, so that $T\left( x_0 \right) - x_0 = 0$.
\end{proof}
\end{document}
