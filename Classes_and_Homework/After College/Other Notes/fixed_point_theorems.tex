\documentclass[10pt]{mypackage}

% sans serif font:
%\usepackage{cmbright}
%\usepackage{sfmath}
%\usepackage{bbold} %better blackboard bold

%\usepackage{homework}
\usepackage{notes}
\usepackage{mlmodern}
\usepackage[ backend=bibtex,
style=alphabetic,
sorting=ynt ]{biblatex}
\addbibresource{all_references.bib}
%\usepackage{newpxtext,eulerpx,eucal}
%\renewcommand*{\mathbb}[1]{\varmathbb{#1}}
\usepackage{parskip}

\fancyhf{}
\fancyhead[R]{Avinash Iyer}
\fancyhead[L]{Fixed-Point Theorems}
\fancyfoot[C]{\thepage}

\setcounter{secnumdepth}{0}

\begin{document}
\RaggedRight
Fixed-point theorems are a very important part of topology and analysis. We'll discuss and prove some from these fields, as well as some applications.
\section{Fixed-Point Theorems in Topology}%
The classic, most well-known fixed point theorem is, of course, the Brouwer fixed-point theorem. The proof we use proceeds via cohomology theory. Some background can be found in one of the sections of my \href{https://ai.avinash-iyer.com/Classes_and_Homework/After\%20College/Classes/Differential\%20Topology/differential_topology_notes.pdf}{differential topology notes}.
\subsection{Brouwer Fixed-Point Theorem}%
\begin{theorem}
  Let $f\colon B\left( 0,1 \right)\rightarrow B\left( 0,1 \right)$ be a continuous self-map of the closed unit ball in $\R^{n}$. Then, $f$ has a fixed point.
\end{theorem}
\begin{proof}
  Suppose toward contradiction that there is no fixed point. Then, for any $x\in B\left( 0,1 \right)$, we can draw a unique line from $x$ to $f(x)$ and extend it in a manner such that it hits the unit sphere, $S\left( 0,1 \right)$. This defines a continuous function $g\colon B\left( 0,1 \right)\rightarrow S\left( 0,1 \right)$ such that $g|_{S\left( 0,1 \right)} = \id$.

  Yet, since $B\left( 0,1 \right)$ is contractible, its cohomology outside of $H^{0}$ is trivial by the Poincaré Lemma, while the cohomology of $S\left( 0,1 \right)$ is nontrivial at $H^{n-1}$. Yet, since $g$ is a deformation retract, the existence of such a $g$ would imply that $S\left( 0,1 \right)$ and $B\left( 0,1 \right)$ are homotopy-equivalent, but the fact that their cohomologies differ shows that this cannot be the case.
\end{proof}
Brouwer's Fixed-Point Theorem can be proven in a variety of different ways, including Sperner's Lemma (which is a form of the Brouwer Fixed-Point Theorem for triangulations), heavier differential topology such as Stokes's Theorem, and a proof using the fact that the game of \href{https://en.wikipedia.org/wiki/Hex_(board_game)}{Hex} always ends.
\subsection{Lefschetz Fixed-Point Theorem}%
We will be using this to prove a different, more general theorem that is again related to cohomology theory.
\begin{theorem}[Lefschetz Fixed-Point Theorem]
  Let $M$ be a triangulated, compact, smooth manifold, and let $f\colon M\rightarrow M$ be a smooth map that preserves the simplicial structure of $M$. Let $f_k^{\ast}\colon H^k_{\operatorname{DR}}\left( M \right)\rightarrow H^{k}_{\operatorname{DR}}\left( M \right)$ denote the induced map in cohomology. Then, if
  \begin{align*}
    L(f) &= \sum_{k=0}^{n} \left( -1 \right)^{k} \tr\left( f_k^{\ast} \right)\\
         &\neq 0,
  \end{align*}
  then $f$ has a fixed-point.
\end{theorem}
\begin{proof}
  By abuse of notation, we let $f^{\ast}\colon H^{\ast}\left( M;\R \right)\rightarrow H^{\ast}\left( M;\R \right)$ be the corresponding map in simplicial cohomology, following from the de Rham theorem.

  Let $\Delta\subseteq M$ be a simplex; then, by the definition of $f$, we observe that $f\left( \Delta \right)\subseteq M$ is also another simplex, which we call $\Lambda$. If $\Delta = \Lambda$, then we obtain smooth self-map of the simplex, which necessarily has a fixed point by the Brouwer fixed-point theorem.

  Now, we assume that $f$ has no fixed-points. We will show that this implies that the Lefschetz number $L(f)$ is equal to zero.

  From the de Rham theorem, we observe that simplicial cochain $I_{\omega}$ on $M$ is then defined by
  \begin{align*}
    I_{\omega}\left( \Delta \right) &= \int_{\Delta}^{} \omega,
    \intertext{and so by taking the pullback,}
    f^{\ast}\left( I_{\omega} \right)\left( \Delta \right) &= \int_{\Delta}^{} f^{\ast}\omega\\
                                                           &= I_{f^{\ast}\omega}\left( \Delta \right).
  \end{align*}
  By even more abuse of notation, we let $f^{\ast}\colon C^{k}\left( M;\R \right)\rightarrow C^{k}\left( M;\R \right)$ be this map from $I_{\omega}$ to $I_{f^{\ast}\omega}$. Observe that by our assumption, for any $k$-simplex $\sigma\subseteq M$, we have $f\left( \sigma \right)\nsubseteq \sigma$, so by selecting a $k$-form supported on $\sigma$, we have $I_{f^{\ast}\omega}\left( \sigma \right) = 0$. In particular, the map $f_k^{\ast}\colon C^k\left( M;\R \right)\rightarrow C^{k}\left( M;\R \right)$ has no eigenvectors, and thus $\tr\left( f_k^{\ast} \right) = 0$. Thus,
  \begin{align*}
    0 &= \sum_{k=0}^{n} \left( -1 \right)^{k}\tr\left( f_k^{\ast} \right).
  \end{align*}
  Now, observe that these $f_k^{\ast}$ are maps on the cochains, not the underlying cohomology. Therefore, we dedicate the rest of the proof to showing that this alternating sum is equal on cohomology. To do this, we use the fact that vector spaces are projective objects, so short exact sequences of the form
  \begin{center}
    % https://tikzcd.yichuanshen.de/#N4Igdg9gJgpgziAXAbVABwnAlgFyxMJZABgBpiBdUkANwEMAbAVxiRGJAF9T1Nd9CKAIzkqtRizYBBLjxAZseAkQBMo6vWatEIAEKzeigUQDM68VrYBhA-L5LByACznNknR05iYUAObwiUAAzACcIAFskMhAcCCQhbmCwyMQRGLjEFUSQUIikNXSkE2zclLNCxCcvTiA
\begin{tikzcd}
0 \arrow[r] & A \arrow[r] & B \arrow[r] & C \arrow[r] & 0
\end{tikzcd}
  \end{center}
  can be expressed as follows:
  \begin{center}
    % https://tikzcd.yichuanshen.de/#N4Igdg9gJgpgziAXAbVABwnAlgFyxMJZABgBpiBdUkANwEMAbAVxiRGJAF9T1Nd9CKAIzkqtRizYBBLjxAZseAkQBMo6vWatEIKQB09ENMzgACAMKzeigUQDM68VraXu1-spQAWR5sk6OTjEYKABzeCJQADMAJwgAWyQyEBwIJCE3EFiE9OpUpBVM7MTENRS0xDsiuJKHcqQvIM4gA
\begin{tikzcd}
0 \arrow[r] & A \arrow[r] & A\oplus C \arrow[r] & C \arrow[r] & 0
\end{tikzcd}
  \end{center}
  In particular, if we write commuting linear maps as follows:
  \begin{center}
    % https://tikzcd.yichuanshen.de/#N4Igdg9gJgpgziAXAbVABwnAlgFyxMJZABgBpiBdUkANwEMAbAVxiRGJAF9T1Nd9CKAIzkqtRizYBBLjxAZseAkQBMo6vWatEIKQB09ENMzgACAMKzeigUQDM68VraXu1-spQAWR5sk6ON3k+JUESUiExP212K2CbT2QRSI0JGJkghQ8wtRSnf10DIxMLOKzQ+wiotJcykNtvKtTnAK4xGCgAc3giUAAzACcIAFskMhAcCCQhIMGR6epJpBVZodHENQmpxDtV+Z3F7a899YBWQ6QANhOri8QAdhuHu4AOJ5e7gE4nhy2kD-yMQAFnE5utNktnoC2H1QWsFn9EJdmgVOm1OEA
\begin{tikzcd}
0 \arrow[r] & A \arrow[r] \arrow[d, "g"] & A\oplus C \arrow[r] \arrow[d, "f"] & C \arrow[r] \arrow[d, "h"] & 0 \\
0 \arrow[r] & A \arrow[r]                & A\oplus C \arrow[r]                & C \arrow[r]                & 0
\end{tikzcd}
  \end{center}
  then
  \begin{align*}
    \left[ f \right]_{\gamma} &= \begin{pmatrix}\left[ g \right]_{\alpha} & K \\ 0 & \left[ h \right]_{\beta}\end{pmatrix},
  \end{align*}
  so that $\tr\left( f \right) = \tr\left( g \right) + \tr\left( h \right)$.

  Now, relabeling $f_k^{\ast}\colon C^{k}\left( M;\R \right)\rightarrow C^{k}\left( M;\R \right)$ as $g_k^{\ast}$, we have the following commutative diagrams that emerge from the map $f$.
  \begin{center}
    % https://tikzcd.yichuanshen.de/#N4Igdg9gJgpgziAXAbVABwnAlgFyxMJZABgBpiBdUkANwEMAbAVxiRGJAF9T1Nd9CKAIzkqtRizYAtAHrAA1py48QGbHgJEATKOr1mrRCADCM+ct7qBRAMy7xBtgCE58gNRCl3S-00oALPb6kkYc3qp8GoIkpEJiwYbsFhFWfsgicXoSibLm4Wq+0TqZDiEmZskFUbax8dnOrh5eKlXWAbVZjqGVkW0xNnVdSfm9aSIDnWUueS2jRaQTpTkVI6nRdosJbAASK7NrRIGb9d2rhURk-oNlYfvnwqRXk4nTPQcoOk9L0ns+1Sh2L5bIy7GZ-PqBIEnYZiGBQADm8CIoAAZgAnCAAWyQZBAOAgSE8KnRWMJ1HxSC04RJ2MQOjxBMQNmpGNpdgZSH8LNJiAArOTGQA2bm0wUCpAAdhFkvFiAAHNL5bKAJyK9kUpXfIwAIwA+gomskaZTZRLnmx4br5HIADo2uhwHDNVGsskcxBirUgABeVqNrsQQlxGqERJdPNDsqEVOJAejUeZsYj6sZQi5SdpQn57qEwozhM9Ial+cDZpzCpLQjlUdVlZThOrXpRVtt9sdzpAxsD9KL5qMvrBnbjIhznuBID1eQonCAA
\begin{tikzcd}
0 \arrow[r] & Z^{k} \arrow[r] \arrow[d, "z_k"] & C^k \arrow[r] \arrow[d, "g_k^{\ast}"] & B^{k+1} \arrow[r] \arrow[d, "b_{k+1}"] & 0 \\
0 \arrow[r] & Z^k \arrow[r]                    & C^k \arrow[r]                         & B^{k+1} \arrow[r]                      & 0 \\
            &                                  &                                       &                                        &   \\
0 \arrow[r] & B^k \arrow[r] \arrow[d, "b_k"]   & Z^k \arrow[r] \arrow[d, "z_k"]        & H^k \arrow[r] \arrow[d, "f_k^{\ast}"]  & 0 \\
0 \arrow[r] & B^k \arrow[r]                    & Z^k \arrow[r]                         & H^k \arrow[r]                          & 0
\end{tikzcd}
  \end{center}
  Using the trace formula we established above, we get
  \begin{align*}
    \tr\left( g_k^{\ast} \right) &= \tr\left( b_{k+1} \right) + \tr\left( z_k \right)\\
    \tr\left( z_k \right) &= \tr\left( f_k^{\ast} \right) + \tr\left( b_k \right).
  \end{align*}
  Therefore, we get
  \begin{align*}
    0 &= \sum_{k=0}^{n} \left( -1 \right)^{k} \tr\left( g_k^{\ast} \right)\\
      &= \sum_{k=0}^{n} \left( -1 \right)^{k} \left( \tr\left( f_k^{\ast} \right) + \tr\left( b_k \right) + \tr\left( b_{k+1} \right) \right)\\
      &= \sum_{k=0}^{n} \left( -1 \right)^{k} \tr\left( f_k^{\ast} \right).
  \end{align*}
\end{proof}
\section{Fixed-Point Theorems in Analysis}%
There are a number of fixed-point theorems in analysis that are extraordinarily useful. 
\subsection{Contraction Mapping Theorem}%
The original fixed-point theorem is known as the contraction mapping theorem.
\begin{theorem}[Contraction Mapping Theorem]
  Let $X$ be a complete metric space, and let $f\colon X\rightarrow X$ be a self-map that satisfies, for all $x,y\in X$,
  \begin{align*}
    d\left( f(x),f(y) \right) &\leq \lambda d\left( x,y \right)
  \end{align*}
  for some $0 \leq \lambda < 1$. Then, there is some $x_0\in X$ such that $f\left(x_0\right) = x_0$.
\end{theorem}
\begin{proof}
  Let $x_1 \in X$ be some element. Define $x_{n+1} = f\left( x_n \right)$. Then, we have
  \begin{align*}
    d\left( x_3,x_2 \right) &\leq \lambda d\left( x_2,x_1 \right),
  \end{align*}
  whence
  \begin{align*}
    d\left( x_m,x_n \right) &\leq \frac{\lambda^{n-1}}{1-\lambda} d\left( x_2,x_1 \right).
  \end{align*}
  In particular, the sequence $\left( x_n \right)_n$ is Cauchy, so there is some $x_0$ such that $\left( x_n \right)_n\rightarrow x_0$ since $X$ is complete, so that
  \begin{align*}
    d\left( f\left( x_0 \right),x_0 \right) &= 0.
  \end{align*}
  Thus, we have a fixed point.
\end{proof}
\subsection{Schauder Fixed-Point Theorem}%
The contraction mapping theorem can be extended to a more general case. This is known as the Schauder Fixed-Point Theorem. To prove it, we use a technical lemma.
\begin{lemma}
  Let $K$ be a compact subset of $X$, $\ve > 0$, and $A$ the finite subset such that
  \begin{align*}
    K &\subseteq \bigcup_{a\in A} U\left( a,\ve \right).
  \end{align*}
  Let $m_a\colon X\rightarrow \R$ be a function defined by $m_a(x) = 0$ if $\norm{x-a} \geq \ve$ and $m_a(x) = \ve - \norm{x-a}$ if $\norm{x-a}\leq \ve$. Then, if $\phi_A\colon K\rightarrow X$ is defined by
  \begin{align*}
    \phi_A(x) &= \frac{\sum_{a\in A} m_a(x)a}{\sum_{a\in A}m_a(x)},
  \end{align*}
  then $\phi_A$ is continuous with
  \begin{align*}
    \norm{\phi_A(x) - x} < \ve
  \end{align*}
  for all $x\in K$.
\end{lemma}
\begin{proof}
  For each $a\in A$, we have $m_a(x) \geq 0$, and $\sum_{a\in A} m_a(x) > 0$ for all $x\in K$. Therefore, $\phi_A$ is well-defined on $K$. The fact that $\phi_A$ is continuous follows from the fact that for each $a\in A$, $m_a\colon K\rightarrow [0,\ve]$ is continuous (since it is the composition of a collection of continuous maps on $X$).

  If $x\in K$, then
  \begin{align*}
    \phi_A(x) - x &= \frac{\sum_{a\in A}\left( m_a(x) \right)\left( a-x \right)}{\sum_{a\in A}m_a(x)}.
  \end{align*}
  If $m_a(x) > 0$, then $\norm{x-a} < \ve$, hence
  \begin{align*}
    \norm{\phi_A(x) - x} &\leq \frac{\sum_{a\in A}m_a(x)\norm{a-x}}{\sum_{a\in A}m_a(x)}\\
                         &< \ve.
  \end{align*}
\end{proof}
\begin{theorem}[Schauder Fixed-Point Theorem]
  Let $E$ be a nonempty, closed, convex subset of a normed vector space, and let $f\colon E\rightarrow E$ be such that $f(E)\subseteq E$ is contained in a compact subset. Then, $f$ admits a fixed-point.
\end{theorem}
\begin{proof}
  Let $K = \overline{f(E)}$, with $K\subseteq E$. For each $n$, define $A_n$ to be such that
  \begin{align*}
    K &\subseteq \bigcup_{a\in A_n} U\left( a,1/n \right).
  \end{align*}
  For each $n$, write $\phi_n = \phi_{A_n}$ as in the lemma. The definition of $\phi_n$ is such that $\phi_n(K)\subseteq \conv(K)\subseteq E$, where the latter emerges from the fact that $E$ is convex. Write $f_n = \phi_n\circ f$, which maps $E$ into $E$, so that
  \begin{align*}
    \norm{f_n(x) - f(x)} &< 1/n
  \end{align*}
  for all $x\in E$. Letting $X_n = \Span\left( A_n \right)$, and $E_n = E\cap X_n$. Then, $X_n$ is a finite-dimensional normed space, $E_n$ is a compact convex subset of $X_n$, and $f_n$ maps $E_n\rightarrow E_n$ and is continuous. By Brouwer's Fixed-Point Theorem, it follows that there is some $x_n$ in $E_n$ such that $f_n\left( x_n \right) = x_n$.

  Since $\left( f\left( x_n \right) \right)_n$ is a sequence in the compact set $K$, there is some point $x_0$ and a subsequence $\left( f\left( x_{n_j} \right) \right)_j$ such that $f\left( x_{n_j} \right)\rightarrow x_0$. Since $f_{n_j}\left( x_{n_j} \right) = x_{n_j}$, it follows that
  \begin{align*}
    \norm{x_{n_j} - x_0} &\leq \norm{f_{n_j}\left( x_{n_j} \right) - f\left( x_{n_j} \right)} + \norm{f\left( x_{n_j} \right) - x_0}\\
                         &\leq \frac{1}{n_j} + \norm{f \left( x_{n_j} \right) - x_0}.
  \end{align*}
  Therefore, $\left( x_{n_j} \right) \rightarrow x_0$, and thus, by continuity of $f$, $f\left( x_0 \right) = \lim_{j\rightarrow\infty}f\left( x_{n_j} \right) = x_0$.
\end{proof}
We can provide an alternative proof for a more general case. This requires a bit more familiarity with constructions from the theory of topological vector spaces.
\begin{theorem}[Schauder--Tychonoff Fixed-Point Theorem]
  Let $X$ be a locally convex topological vector space, $E\subseteq X$ a convex subset, and $f\colon E\rightarrow E$ a continuous map such that $f(E)\subseteq E$ is contained in a compact subset. Then, $f$ has a fixed point.
\end{theorem}
\begin{proof}
  Let $\mathcal{P}$ be the family of seminorms that define the topology on $X$, and define, for finite $F\subseteq \mathcal{P}$ and $r > 0$,
  \begin{align*}
    U_{F,r} &\coloneq \set{x\in X | p\in F,p(x) < r},
  \end{align*}
  which is the neighborhood base of balanced, absorbing, convex subsets. Observe then that the family
  \begin{align*}
    \Lambda &= \set{U_{F,r} | F\subseteq \mathcal{P}\text{ finite},r > 0}
  \end{align*}
  forms a directed set under $U_{F,r}\preceq U_{G,s}$ whenever $F\subseteq G$ and $r \geq s$. For any element $U\in \Lambda$, compactness of $K\coloneq f(E)$ provides $v_1,\dots,v_n$ such that
  \begin{align*}
    K\subseteq \bigcup_{i=1}^{n} v_i + U.
  \end{align*}
  Let $g_1,\dots,g_n$ be a partition of unity subordinate to the open cover $\set{v_i + U}_{i=1}^{n}$. Define
  \begin{align*}
    f_U(\xi) &= \bigcup_{i=1}^{n} g_i\left( f\left( \xi \right) \right) v_i
  \end{align*}
  for any $\xi\in E$. Then, $f_U$ is continuous and has range $K_U = \conv\set{v_1,\dots,v_n}$. The subset $K_U$ is contained in a finite dimensional subspace of $X$, and is thus compact, with $K_U\subseteq E$ by convexity. In particular, $f$ is defined on $K_U$, so by Brouwer's Fixed-Point Theorem, there is $\xi_U\in K_U$ such that $f_U\left( \xi_U \right) = \xi_U$.

  Now, we see that
  \begin{align*}
    f\left( \xi_U \right) - \xi_U &= f\left( \xi_U \right) - f_U\left( \xi_U \right)\\
                                  &= \sum_{i=1}^{n} g_i\left( f\left( \xi_U \right) \right)\left( f\left( \xi_U \right) - v_i \right).
  \end{align*}
  Since $g_i$ is supported on $v_i + U$ for each $i$, we have that the latter is a convex combination of points in $U$. In particular, since $U$ is convex, we have that $f\left( \xi_U \right) - \xi_U\in U$.

  The net $\left( f\left( \xi_U \right) \right)_{U\in \Lambda}$ is contained in $K$, so it has a convergent subnet $\Gamma$ with limit $\xi$. Let $V\in \Lambda$ be a neighborhood of $0$, and select $V_0\in \Lambda$ such that $V_0 - V_0\subseteq V$. By continuity, there is then $W\in \Lambda$ with $V_0\preceq W$ and $\xi - \eta\in W - W$ implies that $f(\xi) - f(\eta)\in V_0$. Finally, there is $U\in\Lambda$ with $W\preceq U$ and $f\left( \xi_U \right) - \eta\in W$. Thus,
  \begin{align*}
    \xi_U - \eta &= \left( f\left( \xi_U \right) - \eta \right) - \left( f\left( \xi_U \right) - \xi_U \right)\\
                 &\in W - U\\
                 &\subseteq W - W.
  \end{align*}
  Therefore, $f\left( \xi_U \right) - f\left( \eta \right)\in V_0$. Thus,
  \begin{align*}
    f\left( \eta \right) - \eta &= \left( f\left( \xi_U \right) - \eta \right) - \left( f\left( \xi_U \right) - f\left( \eta \right) \right)\\
                                &\in W - V_0\\
                                &\subseteq V_0 - V_0\\
                                &\subseteq V.
  \end{align*}
  Since $V$ was an arbitrary neighborhood of $0$, it follows that $f\left( \xi \right) = \xi$.
\end{proof}
\subsection{Markov--Kakutani Fixed-Point Theorem}%
The next fixed-point theorem we will prove is the Markov--Kakutani Fixed-Point Theorem; we will use it to show a particularly useful method to find the amenability of a group, which we will discuss in the section on applications. It is also useful for proving a more high-powered fixed-point theorem known as the Ryll--Nardzewski Fixed-Point Theorem, which we will use to establish the existence of Haar measure on a group.
\begin{theorem}[Markov--Kakutani Fixed-Point Theorem]
  Let $K$ be a nonempty compact convex subset of a locally convex topological vector space $X$, and let $ \mathcal{F} $ be a commuting family of continuous affine maps of $K$ into itself. Then, there is $x_0\in K$ such that $T\left( x_0 \right) = x_0$ for all $T\in \mathcal{F}$.
\end{theorem}
\begin{proof}
  Let $T\in \mathcal{F}$, and for $n\geq 1$, define
  \begin{align*}
    T^{(n)} &= \frac{1}{n}\sum_{k=0}^{n-1} T^{k}.
  \end{align*}
  If $S,T\in \mathcal{F}$, the commutation of $S$ and $T$ implies that $S^{(n)}T^{(m)} = T^{(m)}S^{(n)}$. Let 
  \begin{align*}
    \mathcal{K} = \set{T^{(n)}(K) | T\in \mathcal{F},n\geq 1}.   
  \end{align*}
   Each of the $ \mathcal{K} $ is compact (by continuity) and convex (by the fact that each $T\in \mathcal{F}$ is affine).

  If $T_1,\dots,T_p\in \mathcal{F}$, and $n_1,\dots,n_p\geq 1$, then by commutativity of $ \mathcal{F} $, we have
  \begin{align*}
    T_1^{\left(n_1\right)}\cdots T_p^{\left( n_p \right)}\left( K \right) &\subseteq \bigcap_{j=1}^{p} T_j^{\left( n_j \right)}\left( K \right).
  \end{align*}
  In particular, this means that $\mathcal{K}$ has the finite intersection property. Therefore, by compactness, we have that there is
  \begin{align*}
    x_0 &\in \bigcap_{B\in \mathcal{K}} B.
  \end{align*}
  We claim that $x_0$ is the desired common fixed-point for maps in $\mathcal{F}$. If $T\in \mathcal{F}$, then $x_0\in T^{(n)}\left( K \right)$. Then, there is $x\in K$ such that
  \begin{align*}
    x_0 &= \frac{1}{n}\left( x + T(x) + \cdots + T^{n-1}(x) \right).
  \end{align*}
  In particular, we get
  \begin{align*}
    T\left( x_0 \right) - x_0 &= \frac{1}{n}\left( T(x) + \cdots + T^{n}(x) \right) - \frac{1}{n}\left( x + T(x) + \cdots + T^{n-1}(x) \right)\\
                              &= \frac{1}{n}\left( T^n(x)-x \right)\\
                              &\in \frac{1}{n} \left( K-K \right).
  \end{align*}
  Since $K$ is compact, so too is $K-K$. If $U$ is an open neighborhood of $0$ in $X$, there is $n\geq 1$ such that $\frac{1}{n}\left( K-K \right) \subseteq U$. In particular, this means that $T\left(x_0\right) - x_0\in U$ for every open neighborhood $U$ of $0$, so that $T\left( x_0 \right) - x_0 = 0$.
\end{proof}
\subsection{Hahn Fixed-Point Theorem and Kakutani Fixed-Point Theorem}%
Finally, the most general fixed-point theorem applies to families that are ``sufficiently separating'' in the following sense. 
\begin{definition}
  Let $X$ be a locally convex topological vector space, and let $K\subseteq X$. A semigroup $\mathcal{S}$ of maps from $K$ to itself is called \textit{distal} if, for any two points $v\neq w $ in $ K$, we have
  \begin{align*}
    0\notin \overline{\set{Av - Aw | A\in \mathcal{S}}}.
  \end{align*}
\end{definition}
Equivalently, there is a basic open neighborhood
\begin{align*}
  U_{F,r} &= \set{v\in X | p_i(v)< r\text{ for all }p_i\in F}
\end{align*}
for some finite $F\subseteq \mathcal{P}$, where $ \mathcal{P} $ is the family of seminorms that determines the topology on $X$. In particular, the semigroup being distal is equivalent to the existence of a seminorm $p$ and $r > 0$ such that $\inf_{A\in \mathcal{S}}p\left( Av - Aw \right) \geq r$.
\begin{theorem}[Hahn Fixed-Point Theorem]
  Let $X$ be a locally convex topological vector space, and let $K$ be a nonempty compact convex subset of $X$. Let $\mathcal{S}$ be a semigroup of continuous affine maps of $K$ into itself that is distal. Then, there is a common fixed-point for $\mathcal{S}$.
\end{theorem}
\begin{proof}
  Consider the family of compact convex subsets $C\subseteq K$ that are $\mathcal{S}$-invariant. A decreasing chain of such sets has nonempty intersection, and the intersection is compact, convex, and $\mathcal{S}$-invariant. Thus, we may assume that $K$ is minimal. If $K$ is one point, we are done.

  Else, let $\xi,\eta$ be distinct points in $K$, and set $\zeta = \frac{1}{2}\left( \xi + \eta \right)$. Let $L = \overline{ \mathcal{S} \zeta }$, which is an $\mathcal{S}$-invariant compact subset of $K$. Then, $ \overline{\conv}(L) $ is closed, convex, and $\mathcal{S}$-invariant. Since we assume that $K$ is minimal, it is then the case that $ \overline{\conv}(L) = K $. From the Krein--Milman theorem, it follows that $K$ has an extreme point $\gamma$, with $\gamma\in L$.

  Therefore, we have a net $\left( A_{\lambda} \right)_{\lambda\in \Lambda}$ in $ \mathcal{S} $ such that
  \begin{align*}
    A_{\lambda}\zeta &= \frac{1}{2} \left( A_{\lambda}\xi + A_{\lambda}\eta \right)\\
                     &\rightarrow \gamma.
  \end{align*}
  From compactness, we thus have $\xi'$ and $\eta'$ such that $A_{\lambda}\xi\rightarrow \xi'$ and $A_{\lambda}\eta\rightarrow \eta'$. Since $\gamma$ is extreme, it then follows that $\gamma = \xi'=\eta'$. In particular, this means that
  \begin{align*}
    A_{\lambda}\xi - A_{\lambda}\eta &= \gamma - \gamma\\
                                     &= 0,
  \end{align*}
  which implies that $ \mathcal{S} $ is not distal. Therefore, this means that $K$ is a singleton.
\end{proof}
\begin{definition}
  If $K$ is a subset of a locally convex topological vector space $X$, a set $ \mathcal{S} $ of maps from $K$ to itself is called \textit{equicontinuous} if, for every $V\in \mathcal{O}_{0}$, there is an open $ W\in \mathcal{O}_{0} $ such that such that, whenever $\xi,\eta\in K$ with $\xi-\eta\in W$, we have $A\xi - A\eta \in V$ for all $A\in \mathcal{S}$.
\end{definition}
\begin{theorem}[Kakutani Fixed-Point Theorem]
  Let $K$ be a compact convex subset of a locally convex topological vector space $X$. Let $ \mathcal{G} $ be a \textit{group} of affine maps of $K$ into itself that is equicontinuous. Then, $\mathcal{G}$ has a fixed-point.
\end{theorem}
\begin{proof}
  It suffices to show that $ \mathcal{G} $ is distal on $K$. Let $\xi,\eta\in K$ be distinct points. Choose $V\in \mathcal{O}_{0}$ with $\eta\notin \xi + V$. By equicontinuity, we may find $W\in \mathcal{O}_{0}$ with $\xi',\eta'\in W$ and $\xi'-\eta'\in W$ implying $A\xi' - A\eta'\in V$ for all $A\in \mathcal{G}$.

  Suppose there were $A\in \mathcal{G}$ such that $A\xi - A\eta\in W$. Then,
  \begin{align*}
    \xi - \eta &= A^{-1}\left( A\xi \right) - A^{-1}\left( A\eta \right)\\
               &\in V,
  \end{align*}
  which is a contradiction. Thus, $ \mathcal{G} $ is distal, so $\mathcal{G}$ admits a fixed-point by the Hahn Fixed-Point Theorem.
\end{proof}
\subsection{Ryll-Nardzewski Fixed-Point Theorem}%
Thus far, we have required norm compactness of the subset $K$. Our final fixed-point theorem gives us the most general case.
\begin{theorem}[Ryll--Nardzewski Fixed-Point Theorem]
  Let $X$ be a locally convex topological vector space with topology $\tau$, and let $\tau_{w}$ be the weak topology on $X$. Suppose $K$ is a nonempty weakly compact convex subset of $X$. Let $ \mathcal{S} $ be a semigroup of weakly continuous affine maps of $K$ into itself that is distal with respect to $\tau$. Then, there is a common fixed point for $\mathcal{S}$.
\end{theorem}
\begin{proof}
  We start by reducing to the case of a finitely generated semigroup as follows. For each finite $F\subseteq \mathcal{S}$, we let $K_F$ be the set of common fixed-points for $F$.

  This is a compact subset, and the family $\set{K_F | F\subseteq \mathcal{S}\text{ finite}}$ has the finite intersection property. Therefore, there is a point in the intersection of all these subsets, which is non-empty by compactness. Thus, we may assume $\mathcal{S}$ is finitely generated, and in particular, countable.

  We may suppose that $K$ is a minimal weakly compact, convex, $\mathcal{S}$-invariant set, as in the proof of Hahn's theorem. For any $\xi\in K$, the set $ \overline{\conv}^{\tau}\left( \set{A\xi | A\in \mathcal{S}} \right) $ is an $\mathcal{S}$-invariant convex subset of $K$. By geometric Hahn--Banach, a convex norm-closed subset is weakly closed, so it is equal to $ K $, by minimality. Thus, $K$ is separable under $\tau$. Let $B$ be a countable $\tau$-dense subset of $K$.

  By a similar argument, there is a minimal, weakly compact (not necessarily convex) $\mathcal{S}$-invariant subset $L\subseteq K$. If $L$ is a singleton, then we are done, so suppose $L$ contains distinct points $\xi$ and $\eta$. Since $\mathcal{S}$ is distal, there is a convex $\tau$-open neighborhood $U\in \mathcal{O}_{0}$ such that $A\xi - A\eta\notin U$ for all $A\in \mathcal{S}$. Select a convex $\tau$-closed neighborhood $V$ of $0$ such that $V-V\subseteq U$. Then, $V$ is weakly closed (again by geometric Hahn--Banach). Since $B$ is $\tau$-dense, for any $\zeta\in K$, we have $\left( \zeta - V \right)\cap B \neq \emptyset$. Therefore, we have
  \begin{align*}
    L &= \bigcup_{\beta\in B} \left( \beta + V \right)\cap L.
  \end{align*}
  Since the sets $\left( \beta + V \right)\cap L$ are weakly closed, it follows that $L$ is not a countable union of weakly closed nowhere-dense sets, so there is a $\tau_{w}$-open set $W$ and $\beta_0\in B$ such that 
  \begin{align*}
    \emptyset &\neq L\cap W\\
              &\subseteq \left( \beta_0 + V \right)\cap L\\
              &\subseteq \beta_0 + V.
  \end{align*}
  Since $ \overline{\conv}^{\tau}(L) $ is a $\tau$-closed, convex, $\mathcal{S}$-invariant subset of $K$, it is equal to $K$ by minimality. Let $\gamma$ be an extreme point of $K$, so that $\gamma\in L$ by the Krein--Milman theorem. By the minimality of $L$, we have $ \overline{\mathcal{S}\gamma}^{\tau} = L $, so we have $A_0\in \mathcal{S}$ such that $A_0\gamma\in L\cap W$.

  Set $\zeta = \frac{1}{2}\left( \xi + \eta \right)$. By minimality of $K$, we have
  \begin{align*}
    K &= \overline{\conv}^{\tau} \left( \mathcal{S}\zeta \right)\\
      &= \overline{\conv}^{\tau}\left( \overline{\mathcal{S}\zeta}^{\tau_{w}} \right).
  \end{align*}
  Thus, by Krein--Milman, we have $\gamma\in \overline{\mathcal{S}\zeta}^{\tau_w}$, so there is a net $ \left( A_{\lambda} \right)_{\lambda\in\Lambda} $ in $ \mathcal{S} $ such that
  \begin{align*}
    \frac{1}{2}\left( A_{\lambda}\xi + A_{\lambda}\eta \right) &\xrightarrow{\tau_w} \gamma.
  \end{align*}
  By compactness, we may in fact take $A_{\lambda}\eta\xrightarrow{\tau_w} \eta'$ and $A_{\lambda}\xi\xrightarrow{\tau_w}\xi'$, whence $\gamma = \frac{1}{2} \left( \xi' + \eta' \right)$. Since $\gamma$ is extreme, it follows that $\gamma = \xi' = \eta'$. Thus, there is some $\lambda_0$ such that
  \begin{align*}
    A_0A_{\lambda_0}\xi &\in L\cap W\\
    A_0A_{\lambda_0}\eta &\in L\cap W,
  \end{align*}
  so that $A_0A_{\lambda_0}\xi - A_0A_{\lambda_0}\eta\in V-V\subseteq U$, contradicting the assumption that $ \mathcal{S} $ is distal. Thus, $L$ is a singleton, so $\mathcal{S}$ has a fixed-point.
\end{proof}
\subsection{Application: Haar Measure on Compact Groups}%
\begin{definition}
  A group $G$ is known as a topological group if it is equipped with a Hausdorff topology such that multiplication is jointly continuous and inversion is a homeomorphism of $G$ onto itself.
\end{definition}
If $G$ is a locally compact Hausdorff group, then we let $C_c(G)$ be the space of compactly supported continuous functions on $G$ with the supremum norm. Recall that a Radon measure on $G$ is a Borel measure on $G$ that is outer regular, inner regular on open sets, and finite on compact sets. By the Riesz Representation Theorem there is a one-to-one correspondence between positive linear functionals on $C_c(G)$ and Radon measures on $G$. In fact, there is a (essentially) unique positive Radon measure $\mu$ on $G$ such that $\mu\left( L_sf \right) = \mu(f)$ for every $s\in G$ and every $f\in C_c(G)$, which is known as Haar measure.

We will show the case for when $G$ is compact by using these fixed-point theorems.
\begin{theorem}
  Let $G$ be a compact group. Then, $G$ has a unique left-invariant probability measure $\mu$. 
\end{theorem}
\begin{proof}
  Since $G$ is compact, we have $C(G)^{\ast} = M(G)$ is the space of all complex regular Borel measures on $G$, via the Riesz Representation Theorem. Let $K$ be the $w^{\ast}$-compact convex set of probability measures on $G$. The family of operators $L_s^{\ast}$ are $w^{\ast}$-continuous isometries that map $K$ onto itself.

  Letting $\mathcal{G} = \set{L_s^{\ast} | s\in G}$ be this family, we see that $\mathcal{G}$ is isomorphic to $G$, as for any $\mu\in K$ and any $f\in C(G)$,
  \begin{align*}
    L_t^{\ast}L_s^{\ast}\mu(f) &= \mu\left( L_tL_sf \right)\\
                               &= \mu\left( L_{st}f \right)\\
                               &= L_{st}^{\ast}\mu(f).
  \end{align*}
  Now, we will show that $\mathcal{G}$ is equicontinuous on $K$. We have
  \begin{align*}
    U_F &= \set{\mu\in M(G) | \left\vert \mu(f) \right\vert < 1\text{ for all }f\in F}
  \end{align*}
  for finite $F\subseteq C(G)$ is a basic open neighborhood of $0$. The map $\Phi\colon G\rightarrow C(G)^{F}$ given by
  \begin{align*}
    \Phi(s) = \left( L_sf_1,\dots,L_sf_n \right)
  \end{align*}
  is continuous, so it has a norm-compact image. There is a finite subset $S\subseteq G$ such that $\Phi(S)$ is a $1/3$-net in $\Phi(G)$, so we may define a $w^{\ast}$-open neighborhood of $0$ in $M(G)$, given by
  \begin{align*}
    W &= \set{\mu\in M(G) | \left\vert \mu\left( L_sf_i \right) \right\vert < 1/3, s\in S, f_i\in F}.
  \end{align*}
  Let $\mu,\nu\in K$ with $\mu-\nu\in W$. Then, for any $t\in G$, there is some $s\in S$ such that $\norm{L_tf_i-L_sf_i} < 1/3$ for each $f_i\in F$. Therefore, for each $f_i\in F$, we have
  \begin{align*}
    \left\vert L_t^{\ast}\left( \mu-\nu \right)\left(f_i\right) \right\vert &= \left\vert \left( \mu-\nu \right)\left( L_tf_i \right) \right\vert\\
                                                                            &\leq \left\vert \left( \mu-\nu \right)\left( L_sf_i \right) \right\vert + \left\vert \left( \mu-\nu \right)\left( L_tf_i - L_sf_i \right) \right\vert\\
                                                                            &< \frac{1}{3} + \norm{\mu-\nu}\norm{L_tf_i - L_sf_i}\\
                                                                            &< \frac{1}{3} + \frac{2}{3}\\
                                                                            &= 1.
  \end{align*}
  In particular, this means $L_t^{\ast}\mu - L_t^{\ast}\nu \in U_F$, so $\mathcal{G}$ is equicontinuous. Thus, by Kakutani's Fixed-Point Theorem, there is some $\mu\in K$ such that $L_s^{\ast}\mu = \mu$ for all $s\in G$, which is our desired left-invariant measure.
\end{proof}
\nocite{davidson_functional_analysis,conway_functional_analysis}
\printbibliography
\end{document}
