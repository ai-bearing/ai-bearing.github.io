\documentclass[10pt]{mypackage}

\usepackage{mlmodern}
%\usepackage{newpxtext,eulerpx,eucal}
%\renewcommand*{\mathbb}[1]{\varmathbb{#1}}

%\usepackage{homework}
\usepackage{notes}

\usepackage[ backend=bibtex, style = alphabetic, sorting=ynt ]{biblatex}
\addbibresource{all_references.bib}

\usepackage{parskip}

\fancyhf{}
\fancyhead[R]{Avinash Iyer}
\fancyhead[L]{Von Neumann Algebras: Structure and Examples}
\fancyfoot[C]{\thepage}

\setcounter{secnumdepth}{0}

\begin{document}
\RaggedRight
Notationally, we will use $1$ to denote the identity operator.
\tableofcontents
\section{Preliminaries}%
We start by recalling some of the topologies on $B(H)$.
\begin{definition}
  Let $H$ be a Hilbert space, with $B(H)$ denoting the space of bounded operators on $H$.

  The \textit{strong operator topology}, or SOT, is the locally convex topology generated by the seminorms 
  \begin{align*}
    \set{\norm{Tv} | T\in B(H),v\in H}
  \end{align*}
  The \textit{weak operator topology}, or WOT, is the locally convex topology generated by the seminorms 
  \begin{align*}
    \set{ \left\vert \iprod{Tv}{w} \right\vert | T\in B(H),v,w\in H }
  \end{align*}
\end{definition}
\begin{theorem}
  Let $\phi\colon B(H)\rightarrow \C$ be a linear functional. The following are equivalent:
  \begin{enumerate}[(i)]
    \item there are $\xi_k,\eta_k\in H$ such that $\ds \phi(T) = \sum_{k=1}^{n} \iprod{T\xi_k}{\eta_k}$;
    \item $\phi$ is WOT-continuous;
    \item $\phi$ is SOT-continuous.
  \end{enumerate}
\end{theorem}
\begin{proof}
  The directions (i) implies (ii) implies (iii) are pretty much by definition. To see (iii) implies (i), we have $\xi_1,\dots,\xi_n$ such that, for all $T\in B(H)$, $\max\norm{T\xi_k}\leq 1$ implies $\phi(T)\leq 1$. Then, we have
  \begin{align*}
    \left\vert \phi(T) \right\vert &\leq \left( \sum_{k=1}^{n}\norm{T\xi_k}^2 \right)^{1/2}.
  \end{align*}
  Let
  \begin{align*}
    H^{(n)} &\coloneq \bigoplus_{k=1}^{n}H\\
    T^{(n)} &\coloneq \operatorname{diag}\left( T,\dots,T \right)\in B\left( H^{(n)} \right),
  \end{align*}
  and let $\xi = \left( \xi_1,\dots,\xi_n \right)\in H^{(n)}$. We see then that the linear functional $\psi\colon H\rightarrow \C$ given by
  \begin{align*}
    \psi\left( T^{(n)}\xi \right) &= \phi(T)
  \end{align*}
  defines a linear functional on the closed subspace of $K$ spanned by the vectors 
  \begin{align*}
    \set{T^{(n)}\xi | T\in B(H)},
  \end{align*}
  and has
  \begin{align*}
    \left\vert \psi\left( T^{(n)}\xi \right) \right\vert &\leq \norm{T^{(n)}\xi},
  \end{align*}
  so by the Riesz Representation Theorem for Hilbert Spaces, it follows there is $\eta = \left( \eta_1,\dots,\eta_n \right)$ such that
  \begin{align*}
    \phi(x) &= \iprod{T^{(n)}\xi}{\eta}\\
            &= \sum_{k=1}^{n} \iprod{T\xi_k}{\eta_k}.
  \end{align*}
\end{proof}
\begin{corollary}
  Every SOT-closed convex subset of $B(H)$ is WOT-closed.
\end{corollary}
\begin{proof}
  The closed convex subsets of a locally convex topological vector space are determined by the continuous linear functionals, as follows from an application of the Hahn--Banach separation.
\end{proof}
\begin{theorem}
  The unit ball of $B(H)$ is WOT-compact.
\end{theorem}
\begin{proof}
  Let $ \overline{\D} $ denote the closed unit disk of $\C$, and consider the set
  \begin{align*}
    K &= \prod_{x,y\in B_H} \overline{\D}.
  \end{align*}
  This space is compact by Tychonoff's theorem. Define the embedding $\phi\colon B_{B(H)}\rightarrow K$ given by
  \begin{align*}
    \phi(T) &= \left( \iprod{Tx}{y} \right)_{x,y}.
  \end{align*}
  By Cauchy--Schwarz, we have
  \begin{align*}
    \left\vert \iprod{Tx}{y} \right\vert &\leq \norm{T}_{\op}\norm{x}\norm{y}\\
                                         &\leq 1,
  \end{align*}
  so $\phi$ is well-defined. We see that $\phi$ is WOT-continuous by definition and injective, so we only need to show that $\img\left( \phi \right)$ is closed. Let $\left( T_i \right)_i\subseteq B_{B(H)}$ be a net with
  \begin{align*}
    \lim_{i\in I} \left( \iprod{T_ix}{y} \right)_{x,y} &= \left( z_{x,y} \right)_{x,y}.
  \end{align*}
  We have that $\left( z_{x,y} \right)_{x,y}\in K$ since $K$ is compact, and since the product topology is the topology of pointwise convergence, we have
  \begin{align*}
    \lim_{i\in I} \iprod{T_ix}{y} &= z_{x,y}
  \end{align*}
  defines a sesquilinear form $F\left( x,y \right)$. This means we may find $T\in B_{B(H)}$ such that $F\left( x,y \right) = \iprod{Tx}{y}$, and so $\left( T_i \right)_i\rightarrow T$ in WOT.
\end{proof}
\section{Structure of von Neumann Algebras}%
There are a variety of ways we will understand the structure of von Neumann algebras. We start with discussing the most basic characterization of von Neumann algebras (emerging from the Double Commutant Theorem), then go into more depth into the structure of abelian von Neumann algebras, and end with a discussion of a characterization of a von Neumann algebra as a dual space.
\subsection{Double Commutant Theorem}%
\begin{definition}
  Let $M\subseteq B(H)$. We define the \textit{commutant} to be
  \begin{align*}
    M' &\coloneq \set{S\in B(H) | TS = ST\text{ for all }T\in M}.
  \end{align*}
  The double commutant of $M$ is denoted $M''$, and has $M\subseteq M''$.
\end{definition}
We see that $M'$ is a WOT-closed subalgebra, and if $M'$ is self-adjoint, then $M'$ is a $C^{\ast}$-algebra. Additionally, if $M_1\subseteq M_2$, then $M_1'\supseteq M_1'$.
\begin{theorem}[Double Commutant Theorem]
  Let $M$ be a unital $C^{\ast}$-subalgebra of $B(H)$. The following are equivalent:
  \begin{enumerate}[(i)]
    \item $M = M''$;
    \item $M$ is WOT-closed;
    \item $M$ is SOT-closed.
  \end{enumerate}
\end{theorem}
\begin{proof}
  The implications (i) implies (ii) follows from the discussion above, and (ii) if and only (iii) follow from the definitions (as subalgebras are convex). We focus on showing that (iii) implies (i).

  For a fixed $\xi\in H$, let $P$ be the projection onto the closure of the subspace $\set{T\xi | T\in M}$. We see that $P\xi = \xi$, since $1 \in M$. Additionally, $PTP = TP$ for each $T\in M$, so $P\in M'$. Letting $V\in M''$, we have that $PV = VP$, so $V\xi \in PH$. In particular, for each $\ve > 0$, there is $S\in M$ such that $\norm{\left( V-S \right)\xi} < \ve$.

  Let $\xi_1,\dots,\xi_n\in H$, and set $\xi = \left( \xi_1,\dots,\xi_n \right)$ in $H^{(n)}$. Letting $\rho\colon B(H)\hookrightarrow B\left(H^{(n)}\right)$ be the embedding defined by
  \begin{align*}
    T &\mapsto T^{(n)},
  \end{align*}
  we see that
  \begin{align*}
    \rho\left( M \right)' &= \set{S\in B(K)| S_{ij}\in M'}.
  \end{align*}
  Therefore, we have that $\rho(V)\in \rho(M)''$, meaning that using the same process as above in the amplified algebra, we have
  \begin{align*}
    \sum_{k=1}^{n} \norm{\left( V-T \right)\xi_k}^2 &= \norm{\left( \rho(V)-\rho(T) \right)\xi}^2\\
                                                    &< \ve^2,
  \end{align*}
  meaning that we can approximate $V$ in SOT from $M$, so $V\in M$.
\end{proof}
\begin{definition}
  A \textit{von Neumann algebra} is a unital SOT-closed (or WOT-closed) $C^{\ast}$-subalgebra of $B(H)$.
\end{definition}
The double commutant theorem says that $M = M''$ is a characterization of a von Neumann algebra.

Observe that if $T\in M$ is a normal operator in a von Neumann algebra $M$, then if $E$ denotes the spectral measure for $T$, and $S\in M'$, then $TS = ST$, so by Fuglede's Theorem, $T^{\ast}S = ST^{\ast}$, meaning that $Sf(T) = f(T)S$ for all $f\in B_{\infty}\left(\sigma(T)\right)$. In particular, this means that $E(S)\in M'' = M$. Since the closed linear span of the characteristic functions $\chi_{S}$ is equal to $B_{\infty}\left(\sigma(T)\right)$, it follows that, if $M$ is a von Neumann algebra, then $M$ is the (norm)-closed linear span of all of its projections.

To see this another way, let $a\in M_{\sa}$, and consider a partition $-\norm{a} = t_0 < t_1 < \cdots < t_n = \norm{a}$, where $t_{j+1}-t_j < \ve$ for each $j = 0,\dots,n-1$, and define projections
\begin{align*}
  P_i &= \chi_{\left[ t_{j-1},t_j \right)}
\end{align*}
for $j=1,\dots,n-1$, and $P_n = \chi_{\left[ t_{n-1},t_n \right]}$. Then, we necessarily have
\begin{align*}
  \norm{a - \sum_{j=1}^{n}t_jP_j}_{\op} < \ve,
\end{align*}
so every self-adjoint operator is in the norm-closed linear span of the projections of $M$. Since every element of $M$ can be written as a decomposition of self-adjoint operators, it follows that $M$ is the norm-closed linear span of its projections.
\begin{proposition}
  Let $M$ be a von Neumann algebra, and let $A\in M$.
  \begin{enumerate}[(a)]
    \item If $A$ is normal, and $\phi$ is a bounded Borel function on $\sigma(A)$, then $\phi(A) \in M$.
    \item The operator $A$ is the linear combination of four unitaries in $M$.
    \item If $E$ and $F$ are the projections onto $ \overline{\img(A)} $ and $\ker (A)$ respectively, then $E,F\in M$.
    \item If $A = W \left\vert A \right\vert$ is the polar decomposition for $A$, then $W$ and $ \left\vert A \right\vert $ are in $M$.
  \end{enumerate}
\end{proposition}
\subsection{Abelian von Neumann Algebras}%
\begin{definition}
  Two subsets $M_1\subseteq B\left(H_1\right)$ and $M_2\subseteq B\left( H_2 \right)$ are said to be \textit{spatially isomorphic} if there is an isomorphism $U\colon H_1\rightarrow H_2$ such that $U M_1 U^{-1} = M_2$.
\end{definition}
\begin{definition}
  A vector $e_0$ is said to be separating for $S\subseteq B\left( H \right)$ if the only operator $T\in S$ for which $Te_0 = 0$ is the $0$ operator.
\end{definition}
\begin{proposition}
  If $S$ is a subspace of $B(H)$, then every cyclic vector for $S$ is separating for $S'$. If $A$ is a $C^{\ast}$-algebra of operators, then a vector is cyclic for $A$ if and only if it is separating for $A'$.
\end{proposition}
\begin{proof}
  If $e_0$ is cyclic for $S$, and $T\in S'$ with $Te_0 = 0$, then for every $L\in S$, we have $TLe_0 = LTe_0 = 0$, meaning that $T \left[ Se_0 \right] = 0$. Since $e_0$ is cyclic, this means $T = 0$.

  If $A$ is a unital $C^{\ast}$-subalgebra of $B(H)$, with $e_0$ separating for $A'$, we let $P$ be the projection onto $N = \left[ Ae_0 \right]^{\perp}$. Since $N$ reduces $A$, it follows that $P \in A'$, but since $e_0\perp N$, we have $Pe_0 = 0$. Since $e_0$ is separating for $A'$, it follows that $P = 0$, so $e_0$ is cyclic for $A$.
\end{proof}
\begin{corollary}
  If $A$ is an abelian algebra of operators, every cyclic vector for $A$ is separating.
\end{corollary}
\begin{theorem}
  If $H$ is separable, and $A$ is an unital, abelian $C^{\ast}$-subalgebra of $B(H)$, then the following are equivalent:
  \begin{enumerate}[(a)]
    \item $A$ is a maximal abelian von Neumann algebra;
    \item $A = A'$;
    \item $A$ is SOT-closed with a cyclic vector;
    \item there is a compact metric space $X$, a regular Borel measure $\mu$ supported on $X$, and an isomorphism $U\colon L_2\left( X,\mu \right)\rightarrow H$ such that $U A_{\mu}U^{-1} = A$, where $A_{\mu}$ is the representation of $L_{\infty}\left(X,\mu\right)$ as the space of multiplication operators acting on $L_{2}\left( X,\mu \right)$.
  \end{enumerate}
\end{theorem}
\begin{proof}
  If $A$ is a maximal abelian von Neumann algebra, then $A = A''$ and $A\subseteq A'$, or that $A'\supseteq A'' = A$, so $A = A'$. Similarly, if $A = A'$, then $A = A' = A''$, so that $A$ is a maximal abelian von Neumann algebra. Thus, (a) and (b) are equivalent.

  Now, assume $A = A'$, it follows that $A = A''$, so that $A$ is SOT-closed and contains the identity. Let $\set{e_n}_{n\geq 1}$ be a maximal sequence of unit vectors with $ \left[ Ae_n \right]\perp \left[ A e_m \right] $ whenever $n\leq m$. Then, by maximality, we have
  \begin{align*}
    H &= \bigoplus_{n\geq 1} \left[ Ae_n \right].
  \end{align*}
  Let $P_n = \left[ Ae_n \right]$, and set $e_0 = \sum_{n=1}^{\infty}2^{-n}e_n$. Since $P_n$ reduces $A$, $P_n \in A'$, so from (b), $P_n\in A$, meaning that $e_n = 2^{n}Pe_0\in \left[ Ae_0 \right]$, and thus $\left[ Ae_n \right]\subseteq \left[ Ae_0 \right]$ for each $n$. Thus, $e_0$ is cyclic for $A$. This shows (b) implies (c).

  Now, since $H$ is separable, $B_A$ is WOT-compact, meaning there is a countable WOT-dense subset. Let $A_1$ be the $C^{\ast}$-algebra generated by this WOT-dense subset; then, $A_1$ is a separable $C^{\ast}$-algebra that is WOT-dense in $A$. Let $X$ be the character space of $A_1$; since $A_1$ is separable, $X$ is metrizable, and let $\rho\colon C(X)\rightarrow A_1\subseteq A\subseteq B(H)$ be the inverse Gelfand transform. Then, $\rho$ is a representation of $C(X)$, so there is a spectral measure $E$ on $X$ such that
  \begin{align*}
    \rho(f) &= \int_{}^{} f\:dE.
  \end{align*}
  For every bounded Borel function, we then have
  \begin{align*}
    \widetilde{\rho}\left( \phi \right) &= \int_{}^{} \phi\:dE\\
                                        &\in A_1''\\
                                        &= A''\\
                                        &= A
  \end{align*}
  by the Double Commutant Theorem.

  Letting $e_0$ be a cyclic vector for $A$, set $\mu\left( B \right) = \iprod{E(B)e_0}{e_0}$ for any Borel $B\subseteq X$. We have
  \begin{align*}
    \iprod{\widetilde{\rho}(\phi)e_0}{e_0} &= \int_{}^{} \phi\:d\mu
  \end{align*}
  for every $\phi\in B_{\infty}(X)$, and
  \begin{align*}
    \norm{ \widetilde{\rho}(\phi)e_0 }^2 &= \iprod{\widetilde{\rho}(\phi)^{\ast}\widetilde{\rho}(\phi)e_0}{e_0}\\
                                         &= \int_{}^{} \left\vert \phi \right\vert^2\:d\mu.
  \end{align*}
  Therefore, $B_{\infty}(X)$, considered as a dense subspace of $L_2\left( X,\mu \right)$, admits the well-defined isometry $U\colon B_{\infty}(X)\rightarrow H$ given by $U\phi = \widetilde{\rho}(\phi)e_0$. We may extend $U$ to be an isometry on all of $L_2\left( X,\mu \right)$.

  Now, if $\phi\in B_{\infty}(X)$ and $\psi\in L_{\infty}(X,\mu)$, then
  \begin{align*}
    UM_{\psi}\phi &= U\left( \psi\phi \right)\\
                  &= \widetilde{\rho}\left( \psi\phi \right)e_0\\
                  &= \widetilde{\rho}(\psi) \widetilde{\rho}(\phi) e_0\\
                  &= \widetilde{\rho}(\psi) U\phi.
  \end{align*}
  That is, $UA_{\mu}U^{-1} = \widetilde{\rho}\left( L_{\infty}\left( X,\mu \right) \right)$. Yet, since $A_{\mu}$ is WOT-closed in $B\left( L_2\left( X,\mu \right) \right)$, we have $ \widetilde{\rho}\left( L_{\infty}(X,\mu) \right) $ is WOT-closed in $B(H)$. Furthermore, since $ \widetilde{\rho}\left( L_{\infty}\left( X,\mu \right) \right)\supseteq \rho\left( C(X) \right) = A_1 $, we have $UA_{\mu}U^{-1} = A$. This shows (c) implies (d).

  Finally, to show (d) implies (b), we show that $A_{\mu} = A_{\mu}'$. Let $T\in A_{\mu}'$. Since $X$ is compact and $\mu$ is regular, it follows that $\mu(X) < \infty$. Then, $1\in L_2\left( X,\mu \right)$, so we may set $L_2\left( X,\mu \right)\ni \phi = T(1)$. For any $\psi\in L_{\infty}\left( X,\mu \right)$, then $\psi\in L_2\left( X,\mu \right)$, with $T\psi = TM_{\psi}1 = M_{\psi}T(1) = \psi\phi$, with
  \begin{align*}
    \norm{\phi\psi} &= \norm{T\psi}\\
                    &\leq \norm{T}_{\op}\norm{\psi}.
  \end{align*}
  Set $\Delta_n = \set{x\in X | \left\vert \phi(x) \right\vert \geq n}$. Setting $\psi = \chi_{\Delta_n}$, we have
  \begin{align*}
    \norm{T}_{\op}^2\mu\left( \Delta_n \right) &= \norm{T}_{\op}^2\norm{\psi}^2\\
                                               &\geq \norm{\phi\psi}^2\\
                                               &= \int_{\Delta_n}^{} \left\vert \phi \right\vert^2\:d\mu\\
                                               &\geq n^2\mu\left( \Delta_n \right).
  \end{align*}
  Yet, since $T$ is bounded, for sufficiently large $n$ it follows that $\mu\left( \Delta_n \right) = 0$, meaning $\phi\in L_{\infty}\left( \mu \right)$, and since $T = M_{\phi}$ on $L_{\infty}(\mu)$, we have $T = M_{\phi}$.
\end{proof}
\subsection{Trace-Class Operators and the $\sigma$-Weak Operator Topology}%
In order to discuss a further structural characterization of von Neumann algebras, we start by discussing trace-class operators and characterizing $B(H)$ as a dual space.

An operator $T\in B(H)$ is called \textit{trace-class} if there exists an orthonormal basis $\left( e_i \right)_{i\in I}$ such that the quantity
\begin{align*}
  \tr\left( \left\vert T \right\vert \right) &\coloneq \sum_{i\in I} \iprod{\left\vert T \right\vert e_i}{e_i}\\
                                                           &< \infty.
\end{align*}
Similarly, an operator $T\in B(H)$ is called \textit{Hilbert--Schmidt} if the quantity $\tr\left( T^{\ast}T \right) < \infty$. The set of all trace-class operators is denoted $L_1\left( B(H) \right)$, while the set of Hilbert--Schmidt operators is denoted $L_2\left( B(H) \right)$. We list some essential properties of trace-class operators. The proofs can be found in \cite[Ch. 3, \S 18]{conway_operator_theory}.
\begin{proposition}[Properties of trace-class and Hilbert--Schmidt operators]
  Let $T_1\in L_1\left( B(H) \right)$ and $T_2\in L_2\left( B(H) \right)$. The following properties hold.
  \begin{enumerate}[(i)]
    \item The quantities 
      \begin{align*}
        \norm{T_1}_{1} &\coloneq \tr\left( \left\vert T_1 \right\vert \right)\\
        \norm{T_2}_{2} &\coloneq \tr\left( T_2^{\ast}T_2 \right)
      \end{align*}
      define norms for $T_1$ and $T_2$ respectively.
    \item For any $A\in B(H)$, we have $\tr\left( AT_1 \right) = \tr\left( T_1A \right)$, and $\left\vert \tr\left( AT_1 \right) \right\vert\leq \norm{A}_{\op}\norm{T_1}_{1}$.
    \item Both $L_1\left( B(H) \right)$ and $L_2\left( B(H) \right)$ are ideals in $B(H)$ satisfying
      \begin{align*}
        \norm{AT_{1,2}}_{1,2} &\leq \norm{A}_{\op}\norm{T_{1,2}}_{1,2}.
      \end{align*}
      Furthermore, both $L_1\left( B(H) \right)$ and $L_2\left( B(H) \right)$ are subsets of $K(H)$.
    \item The operator $T_1$ is the product of two Hilbert--Schmidt operators, and any operator $S$ is trace-class if and only if it is the product of two Hilbert--Schmidt operators.
    \item The pairing $ \iprod{A}{B} = \tr\left( B^{\ast}A \right) $ defines an inner product on $L_2\left( B(H) \right)$, and $L_2\left( B(H) \right)$ is a Hilbert space with respect to this inner product.
  \end{enumerate}
\end{proposition}
The main thing we are interested in is understanding the duality properties of trace-class operators. We observe that the following is an analogue of the duality $\left( c_0 \right)^{\ast} = \ell_1$.
\begin{theorem}
  For any $T\in L_1\left( B(H) \right)$, define the linear functional $\phi_T\colon K(H)\rightarrow \C$ by $\phi_T(A) = \tr\left( TA \right) = \tr\left( AT \right)$. Then, the map $T\mapsto \phi_T$ is an isometric isomorphism between $L_1\left( B(H) \right)$ and $\left( K(H) \right)^{\ast}$.
\end{theorem}
\begin{proof}
  We observe that
  \begin{align*}
    \sup\set{\left\vert \tr\left( AC \right) \right\vert | C\in K(H),\norm{C}_{\op}\leq 1} &\leq \norm{A}_{1},
  \end{align*}
  so that $\Phi_A$ is a bounded linear functional on $K(H)$ satisfying $\norm{\Phi_A} \leq \norm{A}_{1}$. Defining $\rho\colon L_1\left( B(H) \right)\rightarrow K(H)$ by $\rho(A) = \Phi_A$, we have that $\rho$ is a linear map with $\norm{\rho(A)}\leq \norm{A}_{1}$ for all $A\in L_1\left( B(H) \right)$.

  Now, we will show that $\rho$ is surjective with $\norm{\rho(A)}\geq \norm{A}_{1}$ for any $A\in L_1\left( B(H) \right)$. Define a sesquilinear form for $\Phi\in K(H)^{\ast}$ by $\left[ g,h \right] = \Phi\left( \theta_{g,h} \right)$, where $\theta_{g,h}$ is the rank-one bounded operator given by
  \begin{align*}
    \theta_{g,h}(k) &= \iprod{k}{h}g.
  \end{align*}
  We have that $\left\vert \left[ g,h \right] \right\vert\leq \norm{\Phi}\norm{g}\norm{h}$ for all $g$ and $h$, so $\left[ \cdot,\cdot \right]$ is bounded, so there is $A\in B(H)$ such that $\left[ g,h \right] = \iprod{Ag}{h}$. We will show that $A\in L_1\left( B(H) \right)$ with $\Phi = \Phi_A$.

  Let $C\in F(H)$ be given by
  \begin{align*}
    C &= \sum_{k=1}^{n}\theta_{g_k,h_k},
  \end{align*}
  Then,
  \begin{align*}
    \Phi(C) &= \Phi\left( \sum_{k=1}^{n}\theta_{g_k\otimes h_k} \right)\\
            &= \sum_{k=1}^{n} \iprod{Ag_k}{h_k}\\
            &= \sum_{k=1}^{n} \tr\left( A\theta_{g_k,h_k} \right)\\
            &= \tr\left( AC \right).
  \end{align*}
  If we can show that $A\in L_1\left( B(H) \right)$, then both $\Phi$ and $\Phi_A$ are bounded linear functionals on $K(H)$ that agree on $F(H)$.

  For this, let $A = W\left\vert A \right\vert$ be the polar decomposition of $A$, and let $\left( e_i \right)_{i\in I}$ be an orthonormal basis. For any finite subset $F\subseteq I$, we have
  \begin{align*}
    C_F\coloneq \left( \sum_{i\in F} \theta_{e_i,e_i} \right)W^{\ast}
  \end{align*}
  is a contraction in $F(H)$ with 
  \begin{align*}
    \norm{\Phi} &\geq \left\vert \Phi\left( C_F \right) \right\vert\\
                &= \left\vert \Phi\left( \sum_{i\in F} e_i\otimes We_i \right) \right\vert\\
                &= \sum_{i\in F} \left\vert \iprod{Ae_i}{We_i} \right\vert\\
                &= \sum_{i\in F} \iprod{\left\vert A \right\vert e_i}{e_i}.
  \end{align*}
  Letting $F$ grow arbitrarily gives $\norm{\Phi}\geq \norm{A}_1$, so $A\in L_1\left( B(H) \right)$, and $\Phi = \Phi_A$. Yet, this means $\norm{\Phi_A} \geq \norm{A}_1$, so $\rho$ is an isometry.
\end{proof}
Similarly, just as $\left( \ell_1 \right)^{\ast} = \ell_{\infty}$, the following holds.
\begin{theorem}
  Let $\Psi\colon L_1\left( B(H) \right)\rightarrow \C$ be given by
  \begin{align*}
    \Phi_B(A) &= \tr\left( AB \right).
  \end{align*}
  Then, the map $B\mapsto \Phi_B$ defines an isometric isomorphism of $B(H)$ onto $\ell_1\left( B(H) \right)^{\ast}$.
\end{theorem}
\begin{proof}
  That $\norm{\Psi_B}\leq \norm{B}$ follows from the fact that $\left\vert \tr\left( AB \right) \right\vert\leq \norm{A}_1\norm{B}_{\op}$. Defining $\rho(B) = \Psi_B$, we have $\rho$ is linear. If $\ve > 0$, we use the Riesz lemma to find a unit vector $g$ such that $\norm{Bg} > \norm{B}_{\op}-\ve$. Find a unit vector $h$ such that $ \iprod{Bg}{h} = \norm{Bg} $. Then, letting $C = \theta_{g,h}$, we have $C\in L_1\left( B(H) \right)$ with $\norm{C}_1 = 1$, with
  \begin{align*}
    \norm{\Psi_B} &\geq \left\vert \tr\left( BC \right) \right\vert\\
                  &= \iprod{Bg}{h}\\
                  &= \norm{Bg}\\
                  &> \norm{B}_{\op} - \ve.
  \end{align*}
  Since $\ve$ is arbitrary, we have $\norm{\Psi_B} = \norm{B}_{\op}$, and $\rho$ is an isometry.

  Now, let $\Psi\in L_1\left( B(H) \right)^{\ast}$. Then, there is an operator $B\in B(H)$ such that $ \iprod{Bg}{h} = \Psi\left( \theta_{g,h} \right) $ for all $g,h\in H$. Then, it follows that $\Psi(T) = \Psi_B(T)$ for every finite-rank operator $T$, so since $F(H)$ is dense in $L_1\left( B(H) \right)$, we have that both $\Psi$ and $\Psi_B$ are bounded linear functionals with $\Psi = \Psi_B$.
\end{proof}
Therefore, we can talk about the weak* topology on $B(H)$ induced by $L_1\left( B(H) \right)$. We discuss an alternative form of convergence known as $\sigma$-WOT and $\sigma$-SOT convergence.
\begin{definition}
  Let $H$ be a Hilbert space. The $\sigma$-strong operator topology on $B(H)$ is the locally convex topology defined by the family of seminorms
  \begin{align*}
    p_{\xi}\left( T \right) &= \norm{\left( T\otimes 1 \right)\xi}
  \end{align*}
  for all $\xi\in H\otimes \ell_2$. The norm is defined by
  \begin{align*}
    \norm{\left( T\otimes 1 \right)\xi} &= \left( \sum_{k=1}^{\infty}\norm{T\xi_k}^{2} \right)^{1/2}.
  \end{align*}
  The $\sigma$-weak operator topology on $B(H)$ is the locally convex topology defined by the family of seminorms
  \begin{align*}
    q_{\xi,\eta} &= \left\vert \iprod{\left( T\otimes 1 \right)\xi}{\eta} \right\vert
  \end{align*}
  for all $\xi,\eta\in H\otimes \ell_2$. The inner product is defined by
  \begin{align*}
    \left\vert \iprod{\left( T\otimes 1 \right)\xi}{\eta} \right\vert &= \left\vert \sum_{k=1}^{\infty} \iprod{T\xi_k}{\eta_k}\right\vert.
  \end{align*}
\end{definition}
We note that $\sigma$-WOT and WOT are equal on bounded subsets of $B(H)$. Furthermore, the following holds.
\begin{proposition}
  The weak* topology on $B(H)$ induced by $L_1\left( B(H) \right)$ and the $\sigma$-WOT are identical.
\end{proposition}
\begin{proof}
  First, we observe that for any sequences $\xi,\eta\in H\otimes \ell_2$, we have that the operator
  \begin{align*}
    T &= \sum_{k=1}^{\infty}\theta_{\xi_k,\eta_k}\label{eq:trace_class_form}\tag{$\ast$}
  \end{align*}
  is trace-class. Since multiplication by an element of $B(H)$ is continuous with respect to the trace-class norm, it follows that, whenever $\left( S_i \right)_i\rightarrow S$ is a $w^{\ast}$-convergent net, then
  \begin{align*}
    \sum_{k=1}^{\infty} \iprod{S_i\xi_k}{\eta_k} &= \tr\left( \sum_{k=1}^{\infty}\theta_{S_i\xi_k,\eta_k} \right)\\
                                                 &= \tr\left( S_iT \right)\\
                                                 &\rightarrow \tr\left( ST \right)\\
                                                 &= \sum_{k=1}^{\infty} \iprod{S\xi_k}{\eta_k}.
  \end{align*}
  Therefore, we have that each seminorm tends to $0$ for all $\xi,\eta\in H\otimes \ell_2$, meaning $\left( S_i \right)_{i}\rightarrow S$ in $\sigma$-WOT.

  Now, if $\left( S_i \right)_i\rightarrow S$ in $\sigma$-WOT, then since every trace-class operator is of the form in \eqref{eq:trace_class_form}, it follows that $\tr\left( S_iT \right) \rightarrow \tr\left( ST \right)$ for every $T\in L_1\left( B(H) \right)$, so $\left( S_i \right)_i\rightarrow S$ is $w^{\ast}$-convergent.
\end{proof}
\subsection{Normal Linear Functionals and Preduals of von Neumann Algebras}%
The existence of a predual for $B(H)$ extends to all von Neumann algebras.
\begin{theorem}
  Let $M\subseteq B(H)$ be a von Neumann algebra. Then, there is a Banach space $M_{\ast}$ such that $M$ is isometrically isomorphic to $\left( M_{\ast} \right)^{\ast}$, where the $w^{\ast}$ topology on $M$ is the $\sigma$-weak topology.
\end{theorem}
\begin{proof}
  Let $M^{\perp}$ be the annihilator of $M$ in $B(H)$, in that
  \begin{align*}
    M^{\perp} &= \set{A\in L_1\left( B(H) \right) | \tr\left( AT \right) = 0\text{ for all }T\in M}.
  \end{align*}
  Then, $M^{\perp}$ is a norm-closed subspace of $L_1\left( B(H) \right)$, so we form the Banach space $M_{\ast} = L_1\left( B(H) \right)/M^{\perp}$. Since $M$ is $\sigma$-WOT closed (as it is WOT-closed), it follows that $M = \left( M^{\perp} \right)_{\perp}$, where $N_{\perp}$ denotes the pre-annihilator. The quotient map $Q\colon L_1\left( B(H) \right)\rightarrow M_{\ast}$ is thus an isometric embedding of $\left( M_{\ast} \right)^{\ast}$ onto $M$ in $B(H)$.
\end{proof}
We specifically consider $M_{\ast}$ to be the collection of $\sigma$-WOT continuous linear functionals on $M$.
\section{Kaplansky Density Theorem and Pedersen's Up-Down Theorem}%
We start by discussing two extremely useful theorems.
\subsection{Kaplansky's Density Theorem}%
\begin{lemma}
  Let $\left( T_i \right)_{i\in I},\left( S_i \right)_{i\in I}\subseteq B(H)$ be nets with $\left( T_i \right)_i\rightarrow T, \left( S_i \right)_i\rightarrow S$ in SOT. If $\sup_{i\in I}\norm{T_i}< \infty$, then $\left( T_iS_i \right)_{i}\rightarrow TS$ in SOT.
\end{lemma}
\begin{proof}
  Set $R = \sup_{i}\norm{T_i}$. Then, for any $\xi\in H$,
  \begin{align*}
    \norm{TS\xi - T_iS_i\xi} &\leq \norm{\left( T-T_i \right)S\xi} + \norm{T_i\left( S-S_i \right)\xi}\\
                             &\leq \norm{\left( T-T_i \right)\xi} + R\norm{\left( S-S_i \right)\xi}\\
                             &\rightarrow 0.
  \end{align*}
\end{proof}
\begin{proposition}
  Let $f\in C(\C)$. Then, the map $T\mapsto f(T)$ on normal operators in $B(H)$ is SOT-continuous on bounded subsets of $B(H)$.
\end{proposition}
\begin{proof}
  Let $\left( T_i \right)_i$ be a uniformly bounded net of operators converging to $T$ in SOT, with $R = \sup_{i}\norm{T_i}$. By Stone--Weierstrass, we are able to approximate $f$ uniformly $B\left( 0,R \right)$ by a sequence of polynomials $\left( p_n \right)_n\subseteq \C\left[ z,\overline{z} \right]$. Since multiplication is SOT-continuous on bounded subsets, it follows that $\left( p_n\left( T_i,T_i^{\ast} \right) \right)_i\rightarrow p_n\left( T_i,T_i^{\ast} \right)$ in SOT.

  Fix $\xi\in H$, $\ve > 0$, and set $N$ to be such that
  \begin{align*}
    \sup_{z\in B(0,R)} \left\vert f(z)-p_N\left( z, \overline{z} \right) \right\vert &< \frac{\ve}{3\norm{\xi}},
  \end{align*}
  and $i_0$ to be such that for all $i\geq i_0$,
  \begin{align*}
    \norm{\left( p_N\left( T_i,T_i^{\ast} \right) - p_N\left( T,T^{\ast} \right) \right)\xi} &< \ve/3.
  \end{align*}
  Then,
  \begin{align*}
    \norm{\left( f(T)-f\left( T_i \right) \right)\xi} &\leq \norm{\left( f(T)-p_N\left( T,T^{\ast} \right) \right)}\norm{\xi} + \norm{\left( p_N\left( T,T^{\ast} \right)-p_N\left( T_i,T_i^{\ast} \right) \right)\xi} + \norm{\left( p_N\left( T_i,T_i^{\ast} \right)-f\left( T_i \right) \right)\xi}\\
                                                      &< \ve.
  \end{align*}
\end{proof}
Now, we observe that if $T\in B(H)_{\sa}$, then $\sigma(T)\subseteq \R$, meaning that $T + z1$ is invertible for any $z\in \C$ with $\im(z)\neq 0$.
\begin{definition}
  Let $T\in B(H)_{\sa}$. Then, the \textit{Cayley transform} of $T$ is given by the operator
  \begin{align*}
    c(T) &\coloneq \left( T-i1 \right)\left( T + i1 \right)^{-1}.
  \end{align*}
\end{definition}
Observe that the Cayley transform emerges from the continuous functional calculus on $c(z) = \frac{z-i}{z+i}$, meaning that $c(T)$ is a unitary operator, and $\left( T-i1 \right)\left( T+i1 \right)^{-1} = \left( T+i1 \right)^{-1}\left( T-i1 \right)$. This gives the following.
\begin{proposition}
  The Cayley Transform is SOT-continuous on $B(H)_{\sa}$.
\end{proposition}
\begin{proof}
  Let $\left( T_j \right)_j\rightarrow T$ be a net of self-adjoint operators. By continuous functional calculus, we have $\norm{\left( T_j + i1 \right)^{-1}}\leq 1$ for all $i\in I$. If $\xi\in H$, we have
  \begin{align*}
    \norm{c(T)\xi-c\left(T_j\right)\xi} &= \norm{c(T)\xi - \left( T_j+i1 \right)^{-1}\left( T_j-i1 \right)\xi}\\
                                        &= \norm{2i\left( T_j+i1 \right)^{-1}\left( T-T_j \right)\left( T-i1 \right)^{-1}\xi}\\
                                        &\leq 2\norm{\left( T-T_j \right)\left( T-i1 \right)^{-1}\xi}.
  \end{align*}
  Thus, SOT-convergence of $\left( T_j \right)_j$ to $T$ implies SOT convergence of the Cayley transform.
\end{proof}
\begin{corollary}
  If $f\in C_0\left(\R\right)$, then the map $T\mapsto f(T)$ is SOT-continuous on $B(H)_{\sa}$.
\end{corollary}
\begin{proof}
  Since $f$ vanishes at infinity, it follows that the function
  \begin{align*}
    g(z) &\coloneq \begin{cases}
      0 & z = 1\\
      f\left( i\frac{1+z}{1-z} \right) & \text{else}
    \end{cases}
  \end{align*}
  defines a continuous function on $S^1$. Since any continuous function on $\C$ is SOT-continuous on bounded sets, it follows that $g$ is SOT-continuous on unitary operators, so by composing $g$ with the Cayley transform, it follows that $f$ is SOT-continuous.
\end{proof}
For any subset $S\subseteq B(H)$, we define
\begin{align*}
  \left( S \right)_1 &\coloneq \set{T\in S | \norm{T}\leq 1}.
\end{align*}
\begin{theorem}[Kaplansky Density Theorem]
  Let $A\subseteq B(H)$ be a $\ast$-subalgebra. Then,
  \begin{align*}
    \overline{A_{\sa}}^{\text{SOT}} &= \left( \overline{A}^{\text{SOT}} \right)_{\sa}
      \intertext{and}
    \overline{\left( A \right)_1}^{\text{SOT}} &= \left( \overline{A}^{\text{SOT}} \right)_{1}.
  \end{align*}
\end{theorem}
\begin{proof}
  Denote $B = \overline{A}^{\text{SOT}}$. We start by showing that it suffices to show that $A$ is (operator) norm-closed. This follows from the fact that norm convergence implies SOT convergence, meaning that if $C$ denotes the norm closure of $A$, then $ \overline{C}^{\text{SOT}} = \overline{A}^{\text{SOT}} $.

  Since SOT convergence implies WOT convergence, it follows that $ \overline{A_{\sa}}^{\text{SOT}} \subseteq B_{\sa}$. If $T\in B_{\sa}$, then there exists a net $\left( T_i \right)_i\rightarrow T$ in SOT. Taking adjoints is WOT-continuous, so $ \left( \frac{T_i + T_i^{\ast}}{2} \right)_i\subseteq A_{\sa} $ converges to $T$ in WOT. Therefore, $T\in \overline{A_{\sa}}^{\text{WOT}}$, but since $A_{\sa}$ is convex, $ \overline{A_{\sa}}^{\text{WOT}} = \overline{A_{\sa}}^{\text{SOT}}$, meaning $B_{\sa} = \overline{A_{\sa}}^{\text{SOT}}$.

  Now, to show that $ \overline{\left( A \right)_1}^{\text{SOT}} = \left( B \right)_1$, we start by showing that the SOT closure of $\left( A_{\sa} \right)_1$ and $\left( B_{\sa} \right)_1$ coincide. Let $x\in \left( B_{\sa} \right)_1$, and let $\left( T_i \right)_i\subseteq A_{\sa}$ converge to $T$ in SOT. Let $f\in C_0\left( \R \right)$ be a function with $\norm{f}_{u} = 1$ and $f(t) = t$ for $\left\vert t \right\vert\leq 1$. Then, $\left( f\left( T_i \right) \right)_i\subseteq \left( A_{\sa} \right)_1$, converging to $f(T) = T$ in SOT, meaning $\left( A_{\sa} \right)_1$ is SOT dense in $\left( B_{\sa} \right)_1$.

  Next, we show that $ \overline{\M_2(A)}^{\text{SOT}} = \M_2(B)$. Fixing elements
  \begin{align*}
     \begin{pmatrix}S_{11} & S_{12} \\ S_{21} & S_{22}\end{pmatrix} &\in \M_2(B)\\
    \begin{pmatrix}\xi_1\\\xi_2\end{pmatrix} &\in H\oplus H,
  \end{align*}
  we use the fact that $ B = \overline{A}^{\text{SOT}} $, so for each $i,j$, we can find $T_{ij}\in A$ such that $\norm{ \left( T_{ij}-S_{ij} \right)\xi_j } < \ve$. In particular, this gives
  \begin{align*}
    \norm{ \begin{pmatrix}T_{11} - S_{11} & T_{12} - S_{12} \\ T_{21} - S_{21} & T_{22} - S_{22} \end{pmatrix} \begin{pmatrix}\xi_1\\\xi_2\end{pmatrix} }^2 &= \sum_{i=1}^{2} \norm{\left( T_{i1}-S_{i1} \right)\xi_1 + \left( T_{i2} - S_{i2} \right)\xi_2}^2\\
                                          &< 8\ve^2.
  \end{align*}
  Now, since we have $ \overline{\left( A \right)_1}^{\text{SOT}}\subseteq \left( B \right)_1 $, we then select $S\in \left( B \right)_1$, and consider
  \begin{align*}
    \overline{S} &= \begin{pmatrix}0 & S \\ S^{\ast} & \end{pmatrix}\\
                 &\in \left( \M_2(B) \right)_1,
  \end{align*}
  which is self-adjoint. Therefore, by applying the earlier result replacing $A$ and $B$ with $\M_2(A)$ and $\M_2(B)$, we have a net $\left( \overline{S_i} \right)_i\subseteq \left( \M_2(A)_{\sa} \right)_{1}$ converging to $ \overline{S} $ in SOT.

  Now, if $S_i$ denotes the $(1,2)$ entry of $ \overline{S_i} $, then we observe that $\norm{S_i}\leq 1$ and converges to $S$ in SOT upon application to the vector $\left( 0,\xi \right)$.
\end{proof}
Note that the choice of $1$ for the operator norm bound in the KDT is arbitrary; by introducing some factors, we find that for any $R$, we have $ \overline{\left( A \right)_R}^{\text{SOT}} = \left( B \right)_R $. The primary case will find use for is where $R = \norm{T}$ for some $T\in B$.
\begin{corollary}
  If $M\subseteq B(H)$ is a unital $\ast$-subalgebra, then the following are equal to each other:
  \begin{itemize}
    \item $ \overline{M}^{\text{$\sigma$-SOT}} $;
    \item $ \overline{M}^{\text{$\sigma$-WOT}} $;
    \item $ \overline{M}^{\text{SOT}} $;
    \item $ \overline{M}^{\text{WOT}} $;
    \item $ M'' $.
  \end{itemize}
  In particular, this means that $M$ is a von Neumann algebra if and only if it is $\sigma$-SOT or $\sigma$-WOT closed.
\end{corollary}
\begin{proof}
  The latter three equivalences follow from the Double Commutant Theorem. Now, since $\sigma$-SOT convergence implies $\sigma$-WOT convergence, which implies WOT convergence, it follows that all we need to show that $ \overline{M}^{\text{SOT}}\subseteq \overline{M}^{\text{$\sigma$-SOT}} $. For $T\in \overline{M}^{\text{SOT}}$, we may find $\left( T_i \right)_i\rightarrow T$, where the net is contained in $\left( M \right)_{\norm{T}}$, convergent in SOT. Since the net is uniformly bounded, and the $\sigma$-SOT and SOT coincide on bounded subsets, it follows that $ \overline{M}^{\text{SOT}}\subseteq \overline{M}^{\text{$\sigma$-SOT}} $.
\end{proof}
\subsection{Pedersen's Up-Down Theorem}%
If $M\subseteq B(H)$, we let $M_{\sigma}$/$M_{\delta}$ be the set of operators in $B(H)_{\sa}$ that can be obtained as SOT limits of monotone increasing/decreasing sequences from $M$. Similarly, let $M^{m}$ and $M_{m}$ be the sets obtained by monotone increasing/decreasing \textit{nets} from $M$. We have that $M\subseteq M_{\sigma}\subseteq M^{m}$, and $M_{\delta} = -\left( -M_{\sigma} \right)$. Furthermore, if $M$ is SOT-closed, then $M^{m} = M_{m} = M$.

We investigate the converse.
\begin{lemma}
  Let $A$ be a $C^{\ast}$-subalgebra of $B(H)$ with SOT closure $M$. If $p$ is a projection in $M$, then for any sequence $\left( \xi_n \right)_n$ of unit vectors in $H$, there is an element $y$ in $\left( \left( A_{+}^1 \right)_{\sigma} \right)_{\delta}$ such that $y\left( 1-p \right)\xi_n = 0$, and $\left( 1-y \right)p\xi_n = 0$ for all $n$. Here, $A^{1}$ denotes the unit ball of $A$.
\end{lemma}
\begin{proof}
  We will approximate $p$ by vectors of the form $p\xi_n$ and $\left( 1-p \right)\xi_n$. By the Kaplansky density theorem, we can find $\left( x_k \right)_k$ in $A_{+}^{1}$ such that $\norm{p\xi_n - x_k p \xi_n} < 1/k$, and $\norm{x_n \left( 1-p \right)\xi_i} < 2^{-n}/n$ for all $i\leq n$.

  For any $n < m$, define
  \begin{align*}
    y_{n,m} &= \left( 1 + \sum_{k=n}^{m}kx_k \right)^{-1}\left( \sum_{k=n}^{m}kx_k \right).
  \end{align*}
  From results in spectral theory, we see that $y_{nm}\in A_{+}^1$ with $y_{nm}\leq \sum_{k=n}^{m}kx_k$. Therefore, for any $i\leq n$, we have
  \begin{align*}
    \iprod{y_{nm}\left( 1-p \right)\xi_i}{\left( 1-p \right)\xi_i} &\leq \sum_{k=n}^{m} 2^{-k}\\
                                                                   &< 2^{-n+1}.
  \end{align*}
  Now, since $mx_m \leq \sum_{k=n}^{m}kx_k$, we have $\left( 1+mx_m \right)^{-1}mx_m \leq y_{nm}$, and thus
  \begin{align*}
    1-y_{nm} &\leq \left( 1 + m x_m \right)^{-1}\\
             &\leq \frac{1}{1+m} \left( 1 + m\left( 1-x_m \right) \right),
  \end{align*}
  so for $i\leq m$, we have $ \iprod{p\xi_i}{p\xi_i} \leq \frac{2}{1+m} $. For fixed $n$, the sequence $\left( y_{nm} \right)_m$ is monotone increasing and SOT-convergent to an element $y_n\in \left( A_+^1 \right)_{\sigma}$. Furthermore, since $y_{n+1,m}\leq y_{nm}$, we have that $y_{n+1}\leq y_n$, and so $\left( y_n \right)_n$ is monotone decreasing to an element $y$ in $\left( \left( A_+^1 \right)_{\sigma} \right)_{\delta}$. This gives
  \begin{align*}
    \iprod{y_n \left( 1-p \right)\xi_i}{\left( 1-p \right)\xi_i} &\leq 2^{-n+1}\\
    \iprod{\left( 1-y_n \right)p\xi_i}{p\xi_i} &\leq 0,
  \end{align*}
  so since $0\leq y \leq 1$, we have $y\left( 1-p \right)\xi_i = 0$ and $\left( 1-y \right)p\xi_i = 0$ for all $i$.
\end{proof}
\begin{theorem}
  Let $A$ be a $C^{\ast}$-subalgebra of $B(H)$ with SOT closure $M$. If $H$ is separable, then $M_{+}^{1} = \left( \left( A_{+}^1 \right)_{\sigma} \right)_{\delta}$ and $M_{\sa} = \left( \left( A_{\sa} \right)_{\sigma} \right)_{\delta}$.
\end{theorem}
\begin{proof}
  Let $\left( \xi_i \right)_i$ be a dense subsequence of the unit ball of $H$. Then, we have that each projection in $M$ belongs to $\left( \left( A_+^1 \right)_{\sigma} \right)_{\delta}$. If $A$ acts non-degenerately on $H$, we have that $1$ is the largest element in $M_+^1$, meaning that $1\in \left( A_+^1 \right)_{\sigma}$.

  For each $x\in M_+^1$, there is a sequence of spectral projections $\left( p_k \right)_k$ such that $x$ is the norm limit of $\sum_{k=1}^{n} 2^{-k}p_k$. This is given by letting $p_1 = (1/2,1]$, $p_2 = (1/4,1/2)\cup (3/4,1]$, etc.

  Let $\left( z_{km} \right)_{m}$ be a sequence in $\left( A_+^1 \right)_{\sigma}$ decreasing to $p_k$, and define
  \begin{align*}
    x_n &= \sum_{k=1}^{n} 2^{-k}z_{kn} + 2^{-n}.
  \end{align*}
  Since $\left( A_{+}^1 \right)_{\sigma}$ is convex, it follows that $x_n\in \left( A_+^1 \right)_{\sigma}$, and we have
  \begin{align*}
    x_n - x_{n+1} &= \sum_{k=1}^{n}2^{-k} \left( z_{kn} - z_{k,n+1} \right) + 2^{-n} - \left( 2^{-n-1}z_{n+1,n+1} + 2^{-n-1} \right)\\
                  &\geq 0,
  \end{align*}
  so that $\left( x_n \right)_n$ is decreasing. We have
  \begin{align*}
    x_n - x &\leq \sum_{k=1}^{n} 2^{-k} \left( z_{kn}-p_k \right) + 2^{-m}
  \end{align*}
  for any $n > m$, so $\left( x_n \right)_n\rightarrow x$ and $x\in \left( \left( A_+^1 \right)_{\sigma} \right)_{\delta}$.

  To show that $M_{\sa} = \left( \left( A_{\sa} \right)_{\sigma} \right)_{\delta}$, note that any $x\in M_{\sa}$ can be written as $\alpha y - \beta$ for $\alpha,\beta$ positive and $y\in M_{+}^{1}$.
\end{proof}
\begin{theorem}
  A $C^{\ast}$-subalgebra $M$ of $B(H)$ is a von Neumann algebra if and only if $\left( M_{\sa} \right)^m = M_{\sa}$.
\end{theorem}
\begin{proof}
  The forward direction is clear from the fact that SOT closure includes SOT closure includes the monotone nets.

  Now, in the reverse direction, suppose $M_{\sa}$ is monotone closed. By cutting with a projection, we may assume that $1\in M$. To show that $M$ is a von Neumann algebra, it suffices to show that any projection in the SOT closure of $M$ belongs to $M$. Let $\xi\in pH$ and $\eta\in \left( 1-p \right)H$. Then, there is an element $y\in M_{+}$ such that $y\xi = \xi$ and $y\eta = 0$. The range projection $p_{\xi\eta}$ of $y$ (emerging from the polar decomposition) belongs to $M$, and has $p_{\xi\eta}\xi = \xi$ and $p_{\xi\eta}\eta = 0$.

  The projections $\inf\set{p_{\xi\eta_1},\dots,p_{\xi\eta_n}}$ forms a decreasing net in $M_{+}$ as $\set{\eta_1,\dots,\eta_n}$ runs through the finite subsets of $\left( 1-p \right)H$. Therefore, we have the limit projection $p_{\xi}$ is less than or equal to $p$. We have that $p$ is the limit of the increasing net of projections $\sup\set{p_{\xi_1},\dots,p_{\xi_n}}$ as $\set{\xi_1,\dots,\xi_n}$ runs through finite subsets of $pH$. Therefore, $p\in M$.
\end{proof}
\section{Two Fundamental von Neumann Algebras}%
We focus now on two special von Neumann algebras.
\subsection{Group von Neumann Algebras}%
We start by discussing a little bit of theory of unitary representations.

Let $\Gamma$ be a discrete group. A unitary representation of $\Gamma$ is a homomorphism $\pi\colon \Gamma\rightarrow U(H)$. The trivial representation of $\Gamma$ is given by $\pi(g) = 1$. The left regular representation is $\lambda\colon \Gamma\rightarrow U\left( \ell_2\left( \Gamma \right) \right)$ given by $\left( \lambda(g)\xi \right)(x) = \xi\left( g^{-1}x \right)$. The right regular representation is $\rho\colon \Gamma\rightarrow U\left( \ell_2\left( \Gamma \right) \right)$, given by $\left( \rho(g)\xi \right)(x) = \xi\left( xg \right)$.

If $\Lambda < \Gamma$ is a subgroup, then the representation $\pi\colon \Gamma\colon \ell_2\left( \Gamma/\Lambda \right)$, given by $\left( \pi(g)\xi \right)(x) = \xi\left( g^{-1}x \right)$ is a \textit{quasi-regular} representation.

We say two representations $\pi_i\colon \Gamma\rightrarrow U\left( H_i \right)$, for $i=1,2$ are \textit{equivalent} if there exists a unitary $U\colon H_1\rightarrow H_2$ such that $U\pi_1(g) = \pi_2(g)U$ for all $g\in \Gamma$. Note that the left and right regular representations are equivalent under the unitary $U\colon \ell_2\left( \Gamma \right)\rightarrow \ell_2\left( \Gamma \right)$ given by $U\xi(x) = \xi\left( x^{-1} \right)$.

Given a unitary representation $\pi\colon \Gamma\rightarrow U\left( H \right)$, the adjoint representation $ \overline{\pi}\colon \Gamma\rightarrow U( \overline{H} ) $ is given by $ \overline{\pi(g)} \overline{\xi} = \overline{\pi(g)\xi} $. Note that $ \overline{\xi} = \iprod{\cdot}{\xi} $ is the linear functional.

If $\pi_{i}\colon \Gamma\rightarrow U\left( H_i \right)$, we define the direct sum representation by
\begin{align*}
  \left( \bigoplus_{i\in I}\pi_i \right)(g) &= \bigoplus_{i\in I}\pi_i(g),
\end{align*}
and if $I$ is finite, we have
\begin{align*}
  \left( \bigotimes_{i\in I}\pi_i \right)(g) &= \bigotimes_{i\in I}\pi_i(g).
\end{align*}
\begin{lemma}[Fell's Absorption Principle]
  Let $\pi\colon \Gamma\rightarrow U(H)$ be a unitary representation of a discrete group $\Gamma$, and let $1_H$ be the trivial representation of $\Gamma$ on $H$. Then, $\lambda\otimes \pi$ and $\lambda\otimes 1_H$ are equivalent.
\end{lemma}
\begin{proof}
  Consider the unitary in $U\left( \ell_2\left( \Gamma \right)\otimes H \right)$ given by $U\left( \delta_g\otimes \xi \right) = \delta_g\otimes \pi(g)\xi$ for all $g\in \Gamma$ and $\xi\in H$. Then, for all $h,g\in \Gamma$ and $\xi\in H$, we have
  \begin{align*}
    \left( U^{\ast}\left( \lambda\otimes \pi \right)(h)U \right)\left( \delta_g\otimes \xi \right) &= \left( U^{\ast}\left( \lambda\otimes \pi \right)(h) \right)\left( \delta_g\otimes \pi(g)\xi \right)\\
                                                                                                   &= U^{\ast}\left( \delta_{hg}\otimes \pi(h)\pi(g)\xi \right)\\
                                                                                                   &= \delta_{hg}\otimes \pi\left( hg \right)^{-1}\pi\left( h \right)\pi\left( g \right)\xi\\
                                                                                                   &= \left( \lambda\otimes 1_H \right)(h)\left( \delta_g\otimes \xi \right).
  \end{align*}
\end{proof}
For any $\xi,\eta\in \ell_2\left( \Gamma \right)$, the convolution of $\xi$ with $\eta$ is
\begin{align*}
  \xi\cdot\eta(x) &= \sum_{g\in \Gamma}\xi(g)\eta\left( g^{-1}x \right)\\
                  &= \sum_{g\in \Gamma} \xi\left( xg^{-1} \right)\eta(g).
\end{align*}
By Cauchy--Schwarz, we have $\xi\cdot\eta\in \ell_{\infty}(\Gamma)$ with $\norm{\xi\cdot\eta}_{\ell_{\infty}}\leq \norm{\xi}_{\ell_2}\norm{\eta}_{\ell_2}$. If $\xi,\eta\in \ell_1\left( \Gamma \right)$, we have $\norm{\xi\cdot\eta}_{\ell_1}\leq \norm{\xi}_{\ell_1}\norm{\eta}_{\ell_1}$.

Given $\xi\in \ell_2\left( \Gamma \right)$, we set
\begin{align*}
  D_{\xi} &= \set{\eta\in \ell_2\left( \Gamma \right) | \xi\cdot\eta\in \ell_2\left( \Gamma \right)};\\
  D_{\xi}' &= \set{\eta\in \ell_2\left( \Gamma \right) | \eta\cdot \xi\in \ell_2\left( \Gamma \right)},
\end{align*}
with
\begin{align*}
  L_{\xi}\eta &= \xi\cdot\eta\\
  R_{\xi}\eta &= \eta\cdot\xi
\end{align*}
acting on $D_{\xi}$ and $D_{\xi}'$ respectively.
\begin{lemma}
  The operators $L_{\xi}$ and $R_{\xi}$ have closed graphs in $\ell_2\left( \Gamma \right)\oplus \ell_2\left( \Gamma \right)$.
\end{lemma}
\begin{proof}
  Let $\left( \eta_n \right)_n\rightarrow \ell_2\left( \Gamma \right)$ be a sequence such that $\eta_n\rightarrow \eta\in \ell_2\left( \Gamma \right)$, and $L_{\xi}\eta_n\rightarrow \zeta\in \ell_2\left( \Gamma \right)$. Then, for any $x\in \Gamma$, we have $ \left\vert \zeta(x)-\left( \xi\cdot\eta \right)(x) \right\vert \leq \norm{\xi}_{\ell_2}\norm{\eta_n-\eta}_{\ell_2} $, so $\xi\cdot\eta = \zeta\in \ell_2\left( \Gamma \right)$, meaning $\eta\in D_{\xi}$ and $L_{\xi}\eta = \zeta$.
\end{proof}
A left convolver is a vector $\xi\in \ell_2\left( \Gamma \right)$ such that $\xi\cdot \ell_2\left( \Gamma \right)\subseteq \ell_2\left( \Gamma \right)$. If $\xi$ is a left convolver, then by the closed graph theorem, we have $L_{\xi},R_{\xi}\in B\left( \ell_2\left( \Gamma \right) \right)$ whenever $\xi$ is a left(/right resp.) convolver. The space of left/right convolvers contains $\delta_g$ for each $g\in \Gamma$.

Let $L\left( \Gamma \right)$ be the set of left convolvers, and $R\left( \Gamma \right)$ the space of right convolvers. Setting $ \overline{\xi}(x) = \overline{\xi\left( x^{-1} \right)} $, we have $L_{\xi}^{\ast} = L_{ \overline{\xi} }$, with $L_{\xi\cdot \eta} = L_{\xi}L_{\eta}$. This gives the structure of unital $\ast$-subalgebras for both $L\left( \Gamma \right)$ and $R\left( \Gamma \right)$. In fact, we have something more.
\begin{theorem}
  Let $\Gamma$ be a discrete group. Then, $L\left( \Gamma \right)$ and $R\left(\Gamma\right)$ are von Neumann algebras. Furthermore, $L\left( \Gamma \right) = \img\left(\rho\right)'$ and $R\left( \Gamma \right) = \img\left( \lambda \right)'$.
\end{theorem}
\subsection{Group Measure Space}%

\nocite{pedersen_cstar_algebras_automorphism_groups,murphy_cstar_algebras_and_operator_theory,conway_operator_theory,davidson_functional_analysis}
\printbibliography
\end{document}
