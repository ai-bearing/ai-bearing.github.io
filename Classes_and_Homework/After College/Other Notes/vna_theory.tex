\documentclass[10pt]{mypackage}

\usepackage{mlmodern}
%\usepackage{newpxtext,eulerpx,eucal}
%\renewcommand*{\mathbb}[1]{\varmathbb{#1}}

%\usepackage{homework}
\usepackage{notes}

\usepackage[ backend=bibtex, style = alphabetic, sorting=ynt ]{biblatex}
\addbibresource{all_references.bib}

\usepackage{parskip}

\fancyhf{}
\fancyhead[R]{Avinash Iyer}
\fancyhead[L]{Von Neumann Algebras: Theory and Structure}
\fancyfoot[C]{\thepage}

\setcounter{secnumdepth}{0}

\begin{document}
\RaggedRight
\section{Introduction and Preliminaries}%
We start by recalling some of the topologies on $B(H)$.
\begin{definition}
  Let $H$ be a Hilbert space, with $B(H)$ denoting the space of bounded operators on $H$.

  The \textit{strong operator topology}, or SOT, is the locally convex topology generated by the seminorms 
  \begin{align*}
    \set{\norm{Tv} | T\in B(H),v\in H}
  \end{align*}
  The \textit{weak operator topology}, or WOT, is the locally convex topology generated by the seminorms 
  \begin{align*}
    \set{ \left\vert \iprod{Tv}{w} \right\vert | T\in B(H),v,w\in H }
  \end{align*}
\end{definition}
\begin{theorem}
  Let $\phi\colon B(H)\rightarrow \C$ be a linear functional. The following are equivalent:
  \begin{enumerate}[(i)]
    \item there are $\xi_k,\eta_k\in H$ such that $\ds \phi(T) = \sum_{k=1}^{n} \iprod{T\xi_k}{\eta_k}$;
    \item $\phi$ is WOT-continuous;
    \item $\phi$ is SOT-continuous.
  \end{enumerate}
\end{theorem}
\begin{proof}
  The directions (i) implies (ii) implies (iii) are pretty much by definition. To see (iii) implies (i), we have $\xi_1,\dots,\xi_n$ such that, for all $T\in B(H)$, $\max\norm{T\xi_k}\leq 1$ implies $\phi(T)\leq 1$. Then, we have
  \begin{align*}
    \left\vert \phi(T) \right\vert &\leq \left( \sum_{k=1}^{n}\norm{T\xi_k}^2 \right)^{1/2}.
  \end{align*}
  Let
  \begin{align*}
    H^{(n)} &\coloneq \bigoplus_{k=1}^{n}H\\
    T^{(n)} &\coloneq \operatorname{diag}\left( T,\dots,T \right)\in B\left( H^{(n)} \right),
  \end{align*}
  and let $\xi = \left( \xi_1,\dots,\xi_n \right)\in H^{(n)}$. We see then that the linear functional $\psi\colon H\rightarrow \C$ given by
  \begin{align*}
    \psi\left( T^{(n)}\xi \right) &= \phi(T)
  \end{align*}
  defines a linear functional on the closed subspace of $K$ spanned by the vectors 
  \begin{align*}
    \set{T^{(n)}\xi | T\in B(H)},
  \end{align*}
  and has
  \begin{align*}
    \left\vert \psi\left( T^{(n)}\xi \right) \right\vert &\leq \norm{T^{(n)}\xi},
  \end{align*}
  so by the Riesz Representation Theorem for Hilbert Spaces, it follows there is $\eta = \left( \eta_1,\dots,\eta_n \right)$ such that
  \begin{align*}
    \phi(x) &= \iprod{T^{(n)}\xi}{\eta}\\
            &= \sum_{k=1}^{n} \iprod{T\xi_k}{\eta_k}.
  \end{align*}
\end{proof}
\begin{corollary}
  Every SOT-closed convex subset of $B(H)$ is WOT-closed.
\end{corollary}
\begin{proof}
  The closed convex subsets of a locally convex topological vector space are determined by the continuous linear functionals, as follows from an application of the Hahn--Banach separation.
\end{proof}
\begin{theorem}
  The unit ball of $B(H)$ is WOT-compact.
\end{theorem}
\begin{proof}
  Let $ \overline{\D} $ denote the closed unit disk of $\C$, and consider the set
  \begin{align*}
    K &= \prod_{x,y\in B_H} \overline{\D}.
  \end{align*}
  This space is compact by Tychonoff's theorem. Define the embedding $\phi\colon B_{B(H)}\rightarrow K$ given by
  \begin{align*}
    \phi(T) &= \left( \iprod{Tx}{y} \right)_{x,y}.
  \end{align*}
  By Cauchy--Schwarz, we have
  \begin{align*}
    \left\vert \iprod{Tx}{y} \right\vert &\leq \norm{T}_{\op}\norm{x}\norm{y}\\
                                         &\leq 1,
  \end{align*}
  so $\phi$ is well-defined. We see that $\phi$ is WOT-continuous by definition and injective, so we only need to show that $\img\left( \phi \right)$ is closed. Let $\left( T_i \right)_i\subseteq B_{B(H)}$ be a net with
  \begin{align*}
    \lim_{i\in I} \left( \iprod{T_ix}{y} \right)_{x,y} &= \left( z_{x,y} \right)_{x,y}.
  \end{align*}
  We have that $\left( z_{x,y} \right)_{x,y}\in K$ since $K$ is compact, and since the product topology is the topology of pointwise convergence, we have
  \begin{align*}
    \lim_{i\in I} \iprod{T_ix}{y} &= z_{x,y}
  \end{align*}
  defines a sesquilinear form $F\left( x,y \right)$. This means we may find $T\in B_{B(H)}$ such that $F\left( x,y \right) = \iprod{Tx}{y}$, and so $\left( T_i \right)_i\rightarrow T$ in WOT.
\end{proof}
\begin{definition}
  A \textit{partial isometry} is an operator $W\in B(H)$ such that for any $h\in \left( \ker(W) \right)^{\perp}$, we have $\norm{Wh} = \norm{h}$. The space $\left( \ker\left( W \right) \right)^{\perp}$ is called the \textit{initial space} of $W$, and the space $\img(W)$ is called the final space of $W$.
\end{definition}
\begin{proposition}
  If $W\in B(H)$, the following are equivalent:
  \begin{enumerate}[(i)]
    \item $W$ is a partial isometry;
    \item $W^{\ast}$ is a partial isometry;
    \item $W^{\ast}W$ is a projection (onto the initial space of $W$);
    \item $WW^{\ast}$ is a projection (onto the final space of $W$);
  \end{enumerate}
\end{proposition}
\begin{proof}
  Let $W$ be a partial isometry, meaning that $W$ is an isometry from $\left( \ker\left( W \right) \right)^{\perp}$ to $\img(W)$. Since $\img(W)$ is dense in $\ker\left( W^{\ast} \right)^{\perp}$, it follows that we only need to show that $W^{\ast}$ is an isometry on $\img(W)$. Let $k\in \img(W)$, so there is $h\in \left( \ker\left( W \right) \right)^{\perp}$ such that $Wh = k$. Then, we have
  \begin{align*}
    \iprod{Wh}{Wh} &= \iprod{h}{h}
      \intertext{so}
    \iprod{W^{\ast}Wh-h}{h} &= 0,
  \end{align*}
  meaning that $W^{\ast}W-I$ is zero on $\left( \ker\left( W \right) \right)^{\perp}$, so we have
  \begin{align*}
    \norm{W^{\ast}k} &= \norm{W^{\ast}Wh}\\
                     &= \norm{h}\\
                     &= \norm{Wh}\\
                     &= \norm{k},
  \end{align*}
  meaning $W^{\ast}$ is a partial isometry.

  By taking adjoints, we see that (i) and (ii) are equivalent. Let $x\in H$ have the decomposition $x = y + z$ where $y\in \ker\left( W \right)$ and $z\in \left( \ker\left( W \right) \right)^{\perp}$. We will show that $W^{\ast}Wx = z$. Observe that $Wx = Wz$, meaning that
  \begin{align*}
    \iprod{z-W^{\ast}Wx}{z} &= \iprod{z-W^{\ast}Wz}{z}\\
                            &= \iprod{z}{z} - \iprod{W^{\ast}Wz}{z}\\
                            &= \iprod{z}{z} - \iprod{Wz}{Wz}\\
                            &= 0,
  \end{align*}
  since $\norm{Wz} = \norm{z}$ by definition. In particular, following from the polarization identity, this means that for all $v\in H$, we have $ \iprod{z-W^{\ast}Wx}{v} = 0 $, so that $z = W^{\ast}Wx$. This shows that (i) implies (iii). By replacing all instances of $W$ with $W^{\ast}$, we see that (ii) implies that $WW^{\ast}$ is a projection onto the initial space of $W^{\ast}$, which is equal to the final space of $W$.
\end{proof}
\begin{theorem}[Polar Decomposition]
  Let $A\in B(H)$. Then, there is a partial isometry $W$ with initial space $\left( \ker\left( A \right) \right)^{\perp}$ and final space $ \overline{\img(A)} $ such that $A = W \left\vert A \right\vert$. Moreover, if $A = UP$, where $P$ is a positive operator and $U$ is a partial isometry with $\ker(U) = \ker(P)$, then $P = \left\vert A \right\vert$  and $U = W$.
\end{theorem}
\begin{proof}
  Let $h\in H$. Then, 
  \begin{align*}
    \norm{Ah} &= \iprod{A^{\ast}Ah}{h}\\
              &= \iprod{\left\vert A \right\vert h}{\left\vert A \right\vert h},
  \end{align*}
  so that
  \begin{align*}
    \norm{Ah} &= \norm{ \left\vert A \right\vert h }.
  \end{align*}
  We may thus define $W\colon \img\left(\left\vert A \right\vert\right) \rightarrow \img\left( A \right)$ by taking
  \begin{align*}
    W\left( \left\vert A \right\vert h \right) &= Ah.
  \end{align*}
  Then, from above, we know that $W$ is an isometry, so it can be extended to an isometry from $ \overline{\img\left(\left\vert A \right\vert\right)} $ to $ \overline{\img\left( A \right)} $. We may then extend $W$ to all of $H$ by defining it to be $0$ on $ \left( \img\left( \left\vert A \right\vert \right) \right)^{\perp} $. This makes $W$ a partial isometry with $W\left\vert A \right\vert = A$. We must verify that $W$ has the correct initial space. That is, we must show that $ \overline{\img\left( \left\vert A \right\vert \right)} = \left( \ker\left( A \right) \right)^{\perp} $.

  Suppose $f = A^{\ast}g$ for some $g\in \left( \ker\left( A^{\ast} \right) \right)^{\perp} = \overline{\img(A)}$. Then, $\img\left( A^{\ast}A \right)$ is dense in $ \left( \ker\left( A \right) \right)^{\perp} $. Yet, since $A^{\ast}Ak = \left\vert A \right\vert h$, where $h = \left\vert A \right\vert k$, it follows that $ \img\left( \left\vert A \right\vert \right) $ is dense in $ \left( \ker\left( A \right) \right)^{\perp} $.

  For uniqueness, we have that $A^{\ast}A = PU^{\ast}UP$, but since $U^{\ast}U$ is the projection onto the initial space, it follows that $\left( \ker\left( U \right) \right)^{\perp} = \left( \ker\left( P \right) \right)^{\perp} = \overline{\img(P)}$, meaning $A^{\ast}A = P^2$, so $P = \left\vert A \right\vert$ by the uniqueness in the continuous functional calculus. For any $h\in H$, we have $W \left\vert A \right\vert h = Ah = U \left\vert A \right\vert h$, meaning that $U$ and $W$ agree on a dense subset of their initial space, so $U = W$.
\end{proof}
\begin{corollary}
  If $T = W \left\vert T \right\vert$ is the polar decomposition for $T\in B(H)$, then $\left\vert T^{\ast} \right\vert = W\left\vert T \right\vert W^{\ast}$, and $T^{\ast} = W^{\ast}\left\vert T^{\ast} \right\vert$.
\end{corollary}
\begin{proof}
  We see that $W \left\vert T \right\vert W^{\ast}$ is positive, and
  \begin{align*}
    W \left\vert T \right\vert W^{\ast} W \left\vert T \right\vert W^{\ast} &= W \left\vert T \right\vert^2 W^{\ast}\\
                                                                            &= W T T^{\ast} W^{\ast}\\
                                                                            &= TT^{\ast}.T
  \end{align*}
  Therefore, by uniqueness, we have $W \left\vert T \right\vert W = \left\vert T^{\ast} \right\vert$. Furthermore, we see that
  \begin{align*}
    W^{\ast}\left\vert T^{\ast} \right\vert &= W^{\ast}W \left\vert T \right\vert W^{\ast}\\
                                            &= \left\vert T \right\vert W^{\ast}\\
                                            &= \left( W \left\vert T \right\vert \right)^{\ast}\\
                                            &= T^{\ast}.
  \end{align*}
\end{proof}
\section{Structure of von Neumann Algebras}%
There are a variety of ways we will understand the structure of von Neumann algebras. We start with discussing the most basic characterization of von Neumann algebras (emerging from the Double Commutant Theorem), then go into more depth into the structure of abelian von Neumann algebras, and end with a discussion of a characterization of a von Neumann algebra as a dual space.
\subsection{Double Commutant Theorem}%
\begin{definition}
  Let $M\subseteq B(H)$. We define the \textit{commutant} to be
  \begin{align*}
    M' &\coloneq \set{S\in B(H) | TS = ST\text{ for all }T\in M}.
  \end{align*}
  The double commutant of $M$ is denoted $M''$, and has $M\subseteq M''$.
\end{definition}
We see that $M'$ is a WOT-closed subalgebra, and if $M'$ is self-adjoint, then $M'$ is a $C^{\ast}$-algebra. Additionally, if $M_1\subseteq M_2$, then $M_1'\supseteq M_1'$.
\begin{theorem}[Double Commutant Theorem]
  Let $M$ be a unital $C^{\ast}$-subalgebra of $B(H)$. The following are equivalent:
  \begin{enumerate}[(i)]
    \item $M = M''$;
    \item $M$ is WOT-closed;
    \item $M$ is SOT-closed.
  \end{enumerate}
\end{theorem}
\begin{proof}
  The implications (i) implies (ii) follows from the discussion above, and (ii) if and only (iii) follow from the definitions (as subalgebras are convex). We focus on showing that (iii) implies (i).

  For a fixed $\xi\in H$, let $P$ be the projection onto the closure of the subspace $\set{T\xi | T\in M}$. We see that $P\xi = \xi$, since $I_{H} \in M$. Additionally, $PTP = TP$ for each $T\in M$, so $P\in M'$. Letting $V\in M''$, we have that $PV = VP$, so $V\xi \in PH$. In particular, for each $\ve > 0$, there is $S\in M$ such that $\norm{\left( V-S \right)\xi} < \ve$.

  Let $\xi_1,\dots,\xi_n\in H$, and set $\xi = \left( \xi_1,\dots,\xi_n \right)$ in $H^{(n)}$. Letting $\rho\colon B(H)\hookrightarrow B\left(H^{(n)}\right)$ be the embedding defined by
  \begin{align*}
    T &\mapsto T^{(n)},
  \end{align*}
  we see that
  \begin{align*}
    \rho\left( M \right)' &= \set{S\in B(K)| S_{ij}\in M'}.
  \end{align*}
  Therefore, we have that $\rho(V)\in \rho(M)''$, meaning that using the same process as above in the amplified algebra, we have
  \begin{align*}
    \sum_{k=1}^{n} \norm{\left( V-T \right)\xi_k}^2 &= \norm{\left( \rho(V)-\rho(T) \right)\xi}^2\\
                                                    &< \ve^2,
  \end{align*}
  meaning that we can approximate $V$ in SOT from $M$, so $V\in M$.
\end{proof}
\begin{definition}
  A \textit{von Neumann algebra} is a unital SOT-closed (or WOT-closed) $C^{\ast}$-subalgebra of $B(H)$.
\end{definition}
The double commutant theorem says that $M = M''$ is a characterization of a von Neumann algebra.

Observe that if $T\in M$ is a normal operator in a von Neumann algebra $M$, then if $E$ denotes the spectral measure for $T$, and $S\in M'$, then $TS = ST$, so by Fuglede's Theorem, $T^{\ast}S = ST^{\ast}$, meaning that $Sf(T) = f(T)S$ for all $f\in B_{\infty}\left(\sigma(T)\right)$. In particular, this means that $E(S)\in M'' = M$. Since the closed linear span of the characteristic functions $\chi_{S}$ is equal to $B_{\infty}\left(\sigma(T)\right)$, it follows that, if $M$ is a von Neumann algebra, then $M$ is the (norm)-closed linear span of all of its projections.

To see this another way, let $a\in M_{\sa}$, and consider a partition $-\norm{a} = t_0 < t_1 < \cdots < t_n = \norm{a}$, where $t_{j+1}-t_j < \ve$ for each $j = 0,\dots,n-1$, and define projections
\begin{align*}
  P_i &= \chi_{\left[ t_{j-1},t_j \right)}
\end{align*}
for $j=1,\dots,n-1$, and $P_n = \chi_{\left[ t_{n-1},t_n \right]}$. Then, we necessarily have
\begin{align*}
  \norm{a - \sum_{j=1}^{n}t_jP_j}_{\op} < \ve,
\end{align*}
so every self-adjoint operator is in the norm-closed linear span of the projections of $M$. Since every element of $M$ can be written as a decomposition of self-adjoint operators, it follows that $M$ is the norm-closed linear span of its projections.
\begin{proposition}
  Let $M$ be a von Neumann algebra, and let $A\in M$.
  \begin{enumerate}[(a)]
    \item If $A$ is normal, and $\phi$ is a bounded Borel function on $\sigma(A)$, then $\phi(A) \in M$.
    \item The operator $A$ is the linear combination of four unitaries in $M$.
    \item If $E$ and $F$ are the projections onto $ \overline{\img(A)} $ and $\ker (A)$ respectively, then $E,F\in M$.
    \item If $A = W \left\vert A \right\vert$ is the polar decomposition for $A$, then $W$ and $ \left\vert A \right\vert $ are in $M$.
  \end{enumerate}
\end{proposition}
\subsection{Abelian von Neumann Algebras}%
\begin{definition}
  Two subsets $M_1\subseteq B\left(H_1\right)$ and $M_2\subseteq B\left( H_2 \right)$ are said to be \textit{spatially isomorphic} if there is an isomorphism $U\colon H_1\rightarrow H_2$ such that $U M_1 U^{-1} = M_2$.
\end{definition}
\begin{definition}
  A vector $e_0$ is said to be separating for $S\subseteq B\left( H \right)$ if the only operator $T\in S$ for which $Te_0 = 0$ is the $0$ operator.
\end{definition}
\begin{proposition}
  If $S$ is a subspace of $B(H)$, then every cyclic vector for $S$ is separating for $S'$. If $A$ is a $C^{\ast}$-algebra of operators, then a vector is cyclic for $A$ if and only if it is separating for $A'$.
\end{proposition}
\begin{proof}
  If $e_0$ is cyclic for $S$, and $T\in S'$ with $Te_0 = 0$, then for every $L\in S$, we have $TLe_0 = LTe_0 = 0$, meaning that $T \left[ Se_0 \right] = 0$. Since $e_0$ is cyclic, this means $T = 0$.

  If $A$ is a unital $C^{\ast}$-subalgebra of $B(H)$, with $e_0$ separating for $A'$, we let $P$ be the projection onto $N = \left[ Ae_0 \right]^{\perp}$. Since $N$ reduces $A$, it follows that $P \in A'$, but since $e_0\perp N$, we have $Pe_0 = 0$. Since $e_0$ is separating for $A'$, it follows that $P = 0$, so $e_0$ is cyclic for $A$.
\end{proof}
\begin{corollary}
  If $A$ is an abelian algebra of operators, every cyclic vector for $A$ is separating.
\end{corollary}
\begin{theorem}
  If $H$ is separable, and $A$ is an unital, abelian $C^{\ast}$-subalgebra of $B(H)$, then the following are equivalent:
  \begin{enumerate}[(a)]
    \item $A$ is a maximal abelian von Neumann algebra;
    \item $A = A'$;
    \item $A$ is SOT-closed with a cyclic vector;
    \item there is a compact metric space $X$, a regular Borel measure $\mu$ supported on $X$, and an isomorphism $U\colon L_2\left( X,\mu \right)\rightarrow H$ such that $U A_{\mu}U^{-1} = A$, where $A_{\mu}$ is the representation of $L_{\infty}\left(X,\mu\right)$ as the space of multiplication operators acting on $L_{2}\left( X,\mu \right)$.
  \end{enumerate}
\end{theorem}
\begin{proof}
  If $A$ is a maximal abelian von Neumann algebra, then $A = A''$ and $A\subseteq A'$, or that $A'\supseteq A'' = A$, so $A = A'$. Similarly, if $A = A'$, then $A = A' = A''$, so that $A$ is a maximal abelian von Neumann algebra. Thus, (a) and (b) are equivalent.

  Now, assume $A = A'$, it follows that $A = A''$, so that $A$ is SOT-closed and contains the identity. Let $\set{e_n}_{n\geq 1}$ be a maximal sequence of unit vectors with $ \left[ Ae_n \right]\perp \left[ A e_m \right] $ whenever $n\leq m$. Then, by maximality, we have
  \begin{align*}
    H &= \bigoplus_{n\geq 1} \left[ Ae_n \right].
  \end{align*}
  Let $P_n = \left[ Ae_n \right]$, and set $e_0 = \sum_{n=1}^{\infty}2^{-n}e_n$. Since $P_n$ reduces $A$, $P_n \in A'$, so from (b), $P_n\in A$, meaning that $e_n = 2^{n}Pe_0\in \left[ Ae_0 \right]$, and thus $\left[ Ae_n \right]\subseteq \left[ Ae_0 \right]$ for each $n$. Thus, $e_0$ is cyclic for $A$. This shows (b) implies (c).

  Now, since $H$ is separable, $B_A$ is WOT-compact, meaning there is a countable WOT-dense subset. Let $A_1$ be the $C^{\ast}$-algebra generated by this WOT-dense subset; then, $A_1$ is a separable $C^{\ast}$-algebra that is WOT-dense in $A$. Let $X$ be the character space of $A_1$; since $A_1$ is separable, $X$ is metrizable, and let $\rho\colon C(X)\rightarrow A_1\subseteq A\subseteq B(H)$ be the inverse Gelfand transform. Then, $\rho$ is a representation of $C(X)$, so there is a spectral measure $E$ on $X$ such that
  \begin{align*}
    \rho(f) &= \int_{}^{} f\:dE.
  \end{align*}
  For every bounded Borel function, we then have
  \begin{align*}
    \widetilde{\rho}\left( \phi \right) &= \int_{}^{} \phi\:dE\\
                                        &\in A_1''\\
                                        &= A''\\
                                        &= A
  \end{align*}
  by the Double Commutant Theorem.

  Letting $e_0$ be a cyclic vector for $A$, set $\mu\left( B \right) = \iprod{E(B)e_0}{e_0}$ for any Borel $B\subseteq X$. We have
  \begin{align*}
    \iprod{\widetilde{\rho}(\phi)e_0}{e_0} &= \int_{}^{} \phi\:d\mu
  \end{align*}
  for every $\phi\in B_{\infty}(X)$, and
  \begin{align*}
    \norm{ \widetilde{\rho}(\phi)e_0 }^2 &= \iprod{\widetilde{\rho}(\phi)^{\ast}\widetilde{\rho}(\phi)e_0}{e_0}\\
                                         &= \int_{}^{} \left\vert \phi \right\vert^2\:d\mu.
  \end{align*}
  Therefore, $B_{\infty}(X)$, considered as a dense subspace of $L_2\left( X,\mu \right)$, admits the well-defined isometry $U\colon B_{\infty}(X)\rightarrow H$ given by $U\phi = \widetilde{\rho}(\phi)e_0$. We may extend $U$ to be an isometry on all of $L_2\left( X,\mu \right)$.

  Now, if $\phi\in B_{\infty}(X)$ and $\psi\in L_{\infty}(X,\mu)$, then
  \begin{align*}
    UM_{\psi}\phi &= U\left( \psi\phi \right)\\
                  &= \widetilde{\rho}\left( \psi\phi \right)e_0\\
                  &= \widetilde{\rho}(\psi) \widetilde{\rho}(\phi) e_0\\
                  &= \widetilde{\rho}(\psi) U\phi.
  \end{align*}
  That is, $UA_{\mu}U^{-1} = \widetilde{\rho}\left( L_{\infty}\left( X,\mu \right) \right)$. Yet, since $A_{\mu}$ is WOT-closed in $B\left( L_2\left( X,\mu \right) \right)$, we have $ \widetilde{\rho}\left( L_{\infty}(X,\mu) \right) $ is WOT-closed in $B(H)$. Furthermore, since $ \widetilde{\rho}\left( L_{\infty}\left( X,\mu \right) \right)\supseteq \rho\left( C(X) \right) = A_1 $, we have $UA_{\mu}U^{-1} = A$. This shows (c) implies (d).

  Finally, to show (d) implies (b), we show that $A_{\mu} = A_{\mu}'$. Let $T\in A_{\mu}'$. Since $X$ is compact and $\mu$ is regular, it follows that $\mu(X) < \infty$. Then, $1\in L_2\left( X,\mu \right)$, so we may set $L_2\left( X,\mu \right)\ni \phi = T(1)$. For any $\psi\in L_{\infty}\left( X,\mu \right)$, then $\psi\in L_2\left( X,\mu \right)$, with $T\psi = TM_{\psi}1 = M_{\psi}T(1) = \psi\phi$, with
  \begin{align*}
    \norm{\phi\psi} &= \norm{T\psi}\\
                    &\leq \norm{T}_{\op}\norm{\psi}.
  \end{align*}
  Set $\Delta_n = \set{x\in X | \left\vert \phi(x) \right\vert \geq n}$. Setting $\psi = \chi_{\Delta_n}$, we have
  \begin{align*}
    \norm{T}_{\op}^2\mu\left( \Delta_n \right) &= \norm{T}_{\op}^2\norm{\psi}^2\\
                                               &\geq \norm{\phi\psi}^2\\
                                               &= \int_{\Delta_n}^{} \left\vert \phi \right\vert^2\:d\mu\\
                                               &\geq n^2\mu\left( \Delta_n \right).
  \end{align*}
  Yet, since $T$ is bounded, for sufficiently large $n$ it follows that $\mu\left( \Delta_n \right) = 0$, meaning $\phi\in L_{\infty}\left( \mu \right)$, and since $T = M_{\phi}$ on $L_{\infty}(\mu)$, we have $T = M_{\phi}$.
\end{proof}
\subsection{Preduals and Duals of von Neumann Algebras}%
A theorem of Sakai says that a $C^{\ast}$-algebra $M\subseteq B(H)$ is a von Neumann algebra precisely whenever there exists a predual for $M$; we do not prove this here, but we start by seeking to understand the predual of certain von Neumann algebras. This will enable us to understand the dual space of a von Neumann algebra.
\begin{definition}
  An operator $T\in B(H)$ is called \textit{trace-class} if the quantity
  \begin{align*}
    \norm{T}_{1} &= \operatorname{Tr}\left( \left\vert T \right\vert \right)\\
                 &= \sum_{i\in I} \iprod{\left\vert T \right\vert e_i}{e_i}
  \end{align*}
  is finite, where $\left( e_i \right)_{i\in I}$ is an orthonormal basis. The set of all trace-class operators is denoted $L_1\left( B(H) \right)$.
\end{definition}
The well-definition of the trace-class norm follows from Parseval's identity: if $\left( e_i \right)_{i\in I}$ and $\left( f_j \right)_{j\in J}$ are orthonormal bases for $H$, then
\begin{align*}
  \sum_{i\in I} \iprod{\left\vert A \right\vert e_i}{e_i} &= \sum_{i\in I} \norm{ \left\vert A \right\vert^{1/2}e_i }^2\\
                                                          &= \sum_{i\in I}\sum_{j\in J} \left\vert \iprod{\left\vert A \right\vert^{1/2}e_i}{f_j} \right\vert^2\\
                                                          &= \sum_{i\in I} \sum_{j\in J} \left\vert \iprod{ \left( \left\vert A \right\vert^{1/2} \right)^{\ast}f_j }{e_i} \right\vert^2\\
                                                          &= \sum_{j\in J}\sum_{i\in I} \left\vert \iprod{\left\vert A \right\vert^{1/2}f_j}{e_i} \right\vert^2
                                                          &= \sum_{j\in J} \norm{\left\vert A \right\vert^{1/2}f_j}^2\\
                                                          &= \sum_{j\in J} \iprod{\left\vert A \right\vert f_j}{f_j}.
\end{align*}

\section{Kaplansky Density Theorem and Pedersen's Up-Down Theorem}%

\section{Projections, Factors, and the Type Decomposition}%

\section{Tracial von Neumann Algebras}%

\nocite{pedersen_cstar_algebras_automorphism_groups,murphy_cstar_algebras_and_operator_theory,conway_operator_theory}
\end{document}
