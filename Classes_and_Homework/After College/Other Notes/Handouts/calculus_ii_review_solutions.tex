\documentclass[12pt]{mypackage}

% sans serif font:
%\usepackage{cmbright}
%\usepackage{sfmath}
%\usepackage{bbold} %better blackboard bold

\usepackage{homework}
\usepackage{notes}
\usepackage{newpxtext,eulerpx,eucal}
\renewcommand*{\mathbb}[1]{\varmathbb{#1}}

\fancyhf{}
\fancyhead[C]{Calculus II Review and Exam Preparation: Solutions}
\fancyhead[L]{}
\fancyfoot[C]{\thepage}

\setcounter{secnumdepth}{0}

\begin{document}
\RaggedRight
Avoid looking at these solutions until you have genuinely given the practice problems an honest attempt. 
\tableofcontents
\pagebreak
\section{Problem 1}%
\begin{problem}\hfill
\begin{enumerate}[(a)]
  \item Consider the solid defined by rotating the region bounded by $x = 0$, $x = \pi/2$, $y = 0$, and $y = \cos(x)$. Set up integrals $I_1$ and $I_2$ for the volume and surface area of this solid respectively.
  \item Find the volume of the solid by resolving the integral. 
  \item Similarly, find an expression for the surface area of this solid. To find the antiderivative, you may find the following steps useful.
    \begin{enumerate}[(i)]
      \item Use a substitution to express the integral entirely in terms of square roots and polynomial expressions.
      \item Take $u = \tan(\theta)$ and use trigonometric identities to express the integral solely in terms of $\sec\left( \theta \right)$.
      \item Extract a factor of $\sec^2(\theta)$ and use integration by parts and a trigonometric identity to reduce this integral to that of $\sec(\theta)$.
      \item To evaluate the integral of $\sec(\theta)$, multiply top and bottom by $\sec(\theta) + \tan(\theta)$, then use a substitution.
    \end{enumerate}
\end{enumerate}
\end{problem}
\begin{solution}\hfill
  \begin{enumerate}[(a)]
    \item We start by considering the following graph.
\begin{center}
  \begin{tikzpicture}
    \begin{axis}[
        axis lines=center,
        xlabel={$x$},
        ylabel={$y$},
        domain=0:pi/2,
        samples=200,
        grid=major,
        grid style={line width=0.1pt, draw=gray!40},
        xmin=0, xmax=pi/1.9,
        ymin=0, ymax=1.2,
        xtick={0, pi/8, pi/4, 3*pi/8, pi/2},
        xticklabels={$0$, $\pi/8$, $\pi/4$, $3\pi/8$, $\pi/2$},
        ytick={0, 0.25, 0.5, 0.75, 1},
        yticklabels={$0$, $1/4$, $1/2$, $3/4$, $1$},
        width=8cm,
        height=8cm,
        thick
    ]
    % Plot sin²(x)
    \addplot[red, very thick, smooth] {cos(deg(x))};

    % Add function label
    \node[red] at (axis cs:3*pi/8,1) {$f(x) = \cos(x)$};

    \end{axis}
  \end{tikzpicture}
\end{center}
In order to write the volume of the solid, we see that at any $x$, the solid looks similar to a cylinder of height $dx$ and radius $\cos(x)$. In the limit, this gives that the volume can be written as
\begin{align*}
  A &= \int_{0}^{\pi/2} \pi \cos^2(x)\:dx.
\end{align*}
Similarly, the arc length is given by $\sqrt{dx^2 + dy^2}$, or $\sqrt{1 + \left( \diff{y}{x} \right)^2}dx$, meaning that, by using the surface area expression, we get
\begin{align*}
  S &= \int_{0}^{\pi/2} 2\pi \cos(x)\sqrt{1 + \sin^2(x)}\:dx.
\end{align*}
\item To evaluate the volume, we use the identity that $\cos(2x) = 2\cos^2(x)-1$. By rearranging, this gives that $\cos^2(x) = \frac{1 + \cos(2x)}{2}$. Therefore, we get
  \begin{align*}
    \int_{0}^{\pi/2} \pi\cos^2(x)\:dx &= \frac{\pi}{2} \int_{0}^{\pi/2} 1\:dx + \frac{\pi}{2} \int_{0}^{\pi/2} \cos(2x)\:dx.
    \intertext{By substituting with $u = 2x$ on the second integral and evaluating the first integral, we get}
                                      &= \frac{\pi^2}{4} + \frac{\pi}{4} \int_{0}^{\pi} \cos(u) \:du\\
                                      &= \frac{\pi^2}{4} + \frac{\pi}{4} \left( \sin(u)|_{0}^{\pi} \right)\\
                                      &= \frac{\pi^2}{4} + \frac{\pi}{4} \left( \sin(\pi) - \sin(0)\right)\\
                                      &= \frac{\pi^2}{4}.
  \end{align*}
\item Evaluating the surface area integral will be a bit more involved.\newline

  First, define $t = \sin(x)$, meaning $dt = \cos(x) dx$, and the substitution gives
  \begin{align*}
    \int_{0}^{\pi/2} \cos(x)\sqrt{1 + \sin^2(x)}\:dx &= \int_{0}^{1} \sqrt{1 + t^2}\:dt.
  \end{align*}
  We then define $t = \tan(\theta)$, or that $\theta = \arctan(t)$, meaning that we get $dt = \sec^2(\theta)\:d\theta$. This gives
  \begin{align*}
    \int_{0}^{1} \sqrt{1 + t^2}\:dt &= \int_{0}^{\pi/4} \sqrt{1 + \tan^2(\theta)}\sec^2(\theta)\:d\theta\\
                                  &= \int_{0}^{\pi/4} \sec^3(\theta)\:d\theta.
  \end{align*}
  To evaluate this particular integral, we take out a factor of $\sec^2\left( \theta \right)$, and use integration by parts. Since the derivative of $\tan(\theta)$ is $\sec^2\theta$, we let $dv = \sec^2\left( \theta \right) d\theta$, $u = \sec\left( \theta \right)$, and get
  \begin{align*}
    \int_{0}^{\pi/4} \sec^3(\theta)\:d\theta &= \sec\left( \theta \right)\tan\left( \theta \right)\biggr\vert_{0}^{\pi/4} - \int_{0}^{\pi/4} \sec(\theta)\tan^2\theta\:d\theta.
    \intertext{Substituting $\tan^2\theta = \sec^2\theta - 1$, we have}
                                             &= \sec(\theta)\tan(\theta)\biggr\vert_{0}^{\pi/4} - \int_{0}^{\pi/4} \sec^3\theta\:d\theta + \int_{0}^{\pi/4} \sec(\theta)\:d\theta.
  \end{align*}
  Therefore, we get that
  \begin{align*}
    \int_{0}^{\pi/4} \sec^3(\theta)\:d\theta &= \left(\frac{1}{2}\sec(\theta)\tan(\theta) \right)\biggr\vert_{0}^{\pi/4} + \frac{1}{2}\int_{0}^{\pi/4} \sec(\theta)\:d\theta.
  \end{align*}
  Finally, to evaluate the integral of $\sec(\theta)$, we multiply both numerator and denominator by $\sec(\theta) + \tan(\theta) $. All in all, this gives
  \begin{align*}
    \left( \frac{1}{2}\sec(\theta)\tan(\theta) \right)\biggr\vert_{0}^{\pi/4} + \frac{1}{2}\int_{0}^{\pi/4} \sec(\theta)\:d\theta &= \frac{1}{2}\sqrt{2} + \int_{0}^{\pi/4} \frac{\sec^2\theta + \sec(\theta)\tan(\theta)}{\sec(\theta) + \tan(\theta)}\:d\theta.
    \intertext{Using one final substitution, this time taking the dummy variable $q = \sec(\theta) + \tan(\theta)$, we get}
                                                                                                                                  &= \frac{1}{2}\sqrt{2} + \frac{1}{2}\int_{1}^{\sqrt{2} + 1} \frac{1}{q}\:dq\\
                                                                                                                                  &= \frac{1}{2}\sqrt{2} + \left( \frac{1}{2}\ln\left\vert q \right\vert \right) \biggr\vert_{1}^{\sqrt{2} + 1}\\
                                                                                                         &= \frac{1}{2}\sqrt{2} + \frac{1}{2}\ln\left( 1 + \sqrt{2} \right)
  \end{align*}
  This is our final surface area.
  \end{enumerate}
\end{solution}
\begin{remark}
  It is possible to use a different evaluation technique, where instead of substituting tangent, one instead substitutes a \href{https://en.wikipedia.org/wiki/Hyperbolic_functions}{hyperbolic trigonometric function}. The hyperbolic sine and cosine are given by
  \begin{align*}
    \sinh(x) &= \frac{e^{x} - e^{-x}}{2}\\
    \cosh(x) &= \frac{e^{x} + e^{-x}}{2}.
  \end{align*}
  Similar to how $\sin$ and $\cos$ are defined on the circle $x^2 + y^2 = 1$, with $\sin$ being the $y$-value and $\cos$ being the $x$-value, $\sinh$ and $\cosh$ emerge from the hyperbola $x^2 - y^2 = 1$, with $\sinh$ being the $y$-value and $\cosh$ being the $x$-value. The corresponding ``Pythagorean'' identity for $\sinh$ and $\cosh$ is then
  \begin{align*}
    \cosh^2(x) - \sinh^2(x) &= 1.
  \end{align*}
  Do you think you can rework the solution by using this identity for your substitution?
\end{remark}
\pagebreak
\section{Problem 2}%
\begin{problem}
Evaluate the convergence of the following series, determining whether a series is absolutely convergent, conditionally convergent, or divergent. Explicitly indicate which test(s) you are using to arrive at your conclusion, and carefully verify the hypotheses.
\begin{enumerate}[(a)]
  \item 
    \begin{align*}
      \sum_{n=1}^{\infty} \frac{\ln(n)}{n^2}.
    \end{align*}
  \item
    \begin{align*}
      \sum_{n=0}^{\infty} \frac{\left( -1 \right)^{n}}{2n+1}.
    \end{align*}
  \item 
    \begin{align*}
      \sum_{n=1}^{\infty} \frac{\sin\left( \frac{1}{n} \right)}{n^2}.
    \end{align*}
\end{enumerate}
\end{problem}
\begin{solution}\hfill
  \begin{enumerate}[(a)]
    \item First, observe that all the terms of this series are positive. To evaluate the test for divergence, one can use L'Hôpital's rule, but one can also see that, for all $x > 0$, we have that $x < e^{x}$, meaning that $\ln(x) < x$ for all $x > 1$. In particular, we get
      \begin{align*}
        \lim_{n\rightarrow\infty} \frac{\ln(n)}{n^2} &\leq \lim_{n\rightarrow\infty} \frac{n}{n^2}\\
                                                     &= 0.
      \end{align*}
      Thus, the test for divergence does not give us any information about the divergence or convergence of the series. Thus, we look for other tests.\newline

      It is not clear that we can easily use the limit comparison test for this series. Attempting the ratio test, we find that
      \begin{align*}
        \lim_{n\rightarrow\infty} \frac{\frac{\ln(n+1)}{\left( n+1 \right)^2}}{\frac{\ln(n)}{n^2}} &= \lim_{n\rightarrow\infty} \frac{\ln(n+1)}{\ln(n)}\frac{n^2}{\left( n+1 \right)^2}
        \intertext{and using L'Hôpital's Rule,}
                                                                                                   &= \lim_{n\rightarrow\infty} \frac{\frac{1}{n+1}}{\frac{1}{n}}\frac{n^2}{\left( n+1 \right)^2}\\
                                                                                                   &= \lim_{n\rightarrow \infty} \frac{n^3}{\left( n+1 \right)^3}\\
                                                                                                   &= 1.
      \end{align*}
      This means that the ratio test is inconclusive. Therefore, we must use the integral test. We observe that $f(x) = \frac{\ln(x)}{x^2}$ is positive and
      \begin{align*}
        \diff{f}{x} &= -2\frac{\ln(x)}{x^3} + \frac{1}{x^3}\\
                    &= \frac{1-2\ln(x)}{x^3}\\
                    &< 0
      \end{align*}
      for all $x > 2$. In particular, this means that the hypotheses for the integral test are satisfied, meaning that we may compute
      \begin{align*}
        \int_{1}^{\infty} \frac{\ln(x)}{x^2}\:dx &= -\frac{1}{x}\ln(x)\biggr\vert_{1}^{\infty} + \int_{1}^{\infty} \frac{1}{x^2}\:dx\\
                                                 &= -\lim_{t\rightarrow\infty} \frac{\ln(t)}{t} + \left( -\frac{1}{x} \right)\biggr\vert_{1}^{\infty}\\
                                                 &= -\lim_{t\rightarrow\infty} \left( -\lim_{t\rightarrow\infty} \frac{1}{t} + 1 \right)
                                                 \intertext{Evaluating these limits using a preferred technique gives}
                                                 &= 1.
      \end{align*}
      Since the integral of $\frac{\ln(x)}{x^2}$ from $1$ to $\infty$ converges, we find that the series converges. Additionally, since $f$ is always positive, the convergence is absolute.
    \item Since this is an alternating series, we write
      \begin{align*}
        \sum_{n=0}^{\infty} \frac{\left( -1 \right)^{n}}{2n + 1} &= \sum_{n=0}^{\infty} \left( -1 \right)^{n}a_n,
      \end{align*}
      where $a_n = \frac{1}{2n+1}$. Observe that the $a_n$ are strictly positive and decreasing. Furthermore,
      \begin{align*}
        \lim_{n\rightarrow\infty} \frac{1}{2n+1} &= 0.
      \end{align*}
      Thus, by the alternating series test, the series converges conditionally.\newline

      To evaluate absolute convergence, we use the limit comparison test with the series
      \begin{align*}
        \sum_{n=1}^{\infty} \frac{1}{n} &= \sum_{n=1}^{\infty} b_n.
      \end{align*}
      This gives
      \begin{align*}
        \lim_{n\rightarrow\infty} \frac{a_n}{b_n} &= \lim_{n\rightarrow\infty} \frac{\frac{1}{2n+1}}{\frac{1}{n}}\\
                                                  &= \lim_{n\rightarrow\infty} \frac{n}{2n+1}\\
                                                  &= \frac{1}{2}.
      \end{align*}
      Since this limit exists, and the series $\ds\sum_{n=1}^{\infty}\frac{1}{n}$ diverges, it follows that the original series is \textit{not} absolutely convergent.
    \item We observe that we cannot use the alternating series test. However, we do know that
      \begin{align*}
        \left\vert \sin(x) \right\vert &\leq 1
      \end{align*}
      for all $x$. Therefore, we consider the comparison series
      \begin{align*}
        \sum_{n=1}^{\infty} b_n &= \sum_{n=1}^{\infty}\frac{1}{n^2}.
      \end{align*}
      In particular, we observe that
      \begin{align*}
        \left\vert \frac{\sin\left( \frac{1}{n} \right)}{n^2} \right\vert &\leq \frac{1}{n^2}.
      \end{align*}
      Since the series $\ds \sum_{n=1}^{\infty}\frac{1}{n^2}$ converges by $p$-series, direct comparison gives that the series
      \begin{align*}
        \sum_{n=1}^{\infty} \left\vert \frac{\sin\left( \frac{1}{n} \right)}{n^2} \right\vert
      \end{align*}
      is absolutely convergent. Therefore, the original series is convergent.
  \end{enumerate}
\end{solution}
\pagebreak
\section{Problem 3}%
\begin{problem}
Find the \textit{interval} of convergence for the following power series.
\begin{enumerate}[(a)]
  \item
    \begin{align*}
      \sum_{n=1}^{\infty} \left( n^{5} +3n^2 + 2n + 3 \right) \frac{\left( x-3 \right)^{n}}{4^{n}}.
    \end{align*}
  \item 
    \begin{align*}
      \sum_{n=1}^{\infty} \frac{2^{3n}}{n^25^{2n}} \left( x-2 \right)^{n}.
    \end{align*}
  \item 
    \begin{align*}
      \sum_{n=1}^{\infty} \frac{1}{n2^{n/2}} x^{4n}.
    \end{align*}
\end{enumerate}
\end{problem}
\begin{solution}\hfill
  \begin{enumerate}[(a)]
    \item As a general rule, we evaluate the interval of convergence for
      \begin{align*}
        \sum_{n=1}^{\infty} n^{k} \frac{\left( x-3 \right)^{n}}{4^{n}}
      \end{align*}
      for each $k\geq 0$. By using the ratio test, we get
      \begin{align*}
        \lim_{n\rightarrow\infty} \left\vert \frac{\left( n+1 \right)^{k}\frac{\left( x-3 \right)^{n+1}}{4^{n+1}}}{n^{k}\frac{\left( x-3 \right)^{n}}{4^{n}}} \right\vert &< 1\\
        \left\vert \frac{x-3}{4} \right\vert &< 1\\
        \left\vert x-3 \right\vert &< 4.
      \end{align*}
      Therefore, the interval of convergence contains the open interval $\left( -1,7 \right)$. Evaluating each endpoint, we see that
      \begin{align*}
        \sum_{n=1}^{\infty} n^{k} \frac{\left( -1-3 \right)^{n}}{4^{n}} &= \sum_{n=1}^{\infty} \left( -1 \right)^{n}n^{k}
      \end{align*}
      diverges, while
      \begin{align*}
        \sum_{n=1}^{\infty} n^{k} \frac{\left( 7-3 \right)^{n}}{4^{n}} &= \sum_{n=1}^{\infty} n^{k},
      \end{align*}
      which diverges. Thus, the interval of convergence is $\left( -1,7 \right)$.
    \item Rewriting the series, we have
      \begin{align*}
        \sum_{n=1}^{\infty} \frac{2^{3n}}{n^25^{2n}} \left( x-2 \right)^{n} &= \sum_{n=1}^{\infty} \frac{8^{n}}{n^2 25^{n}} \left( x-2 \right).
      \end{align*}
      Using the ratio test, we get
      \begin{align*}
        \lim_{n\rightarrow\infty} \left\vert \frac{\frac{8^{n+1}}{(n+1)^2 25^{n+1}} \left( x-2 \right)^{n+1}}{\frac{8^{n}}{n^2 25^{n}} \left( x-2 \right)^{n}} \right\vert &< 1\\
        \lim_{n\rightarrow\infty} \left\vert \frac{\left( n+1 \right)^2}{n^2}\frac{8}{25} \left( x-2 \right)\right\vert &< 1\\
        \left\vert \frac{8}{25}\left( x-2 \right) \right\vert &< 1\\
        \left\vert x-2 \right\vert &< \frac{25}{8}.
      \end{align*}
      Thus, we find that the interval of convergence contains $\left( -9/8,41/8 \right)$.\newline

      Evaluating each of the endpoints, we have
      \begin{align*}
        \sum_{n=1}^{\infty} \frac{8^{n}}{n^2 25^{n}} \left( -9/8- 2\right)^{n} &= \sum_{n=1}^{\infty} \frac{\left( -1 \right)^{n}}{n^2}
      \end{align*}
      which converges by the alternating series test. Similarly,
      \begin{align*}
        \sum_{n=1}^{\infty} \frac{8^{n}}{n^2 25^{n}} \left( 41/8-2 \right)^{n} &= \sum_{n=1}^{\infty}\frac{1}{n^2}
      \end{align*}
      which converges by $p$-series. Thus, the interval of convergence is $ \left[ -9/8,41/8 \right] $.
    \item Using the ratio test, we evaluate
      \begin{align*}
        \lim_{n\rightarrow\infty} \left\vert \frac{\frac{1}{(n+1)2^{(n+1)/2}}x^{4n+4}}{\frac{1}{n2^{n/2}}x^{4n}} \right\vert &< 1\\
        \lim_{n\rightarrow\infty} \left\vert \frac{n}{n+1} 2^{1/2} x^{4} \right\vert &< 1\\
        \left\vert x^{4} 2^{-1/2} \right\vert &< 1\\
        \left\vert x^{4} \right\vert &< 2^{1/2}\\
        \left\vert x \right\vert &< 2^{1/8}.
      \end{align*}
      Thus, we have that the interval of convergence contains $\left( -2^{1/8},2^{1/8} \right)$. Evaluating each of the endpoints, we have
      \begin{align*}
        \sum_{n=1}^{\infty} \frac{1}{n2^{n/2}}\left( -2^{1/8} \right)^{4n} &= \sum_{n=1}^{\infty} \frac{1}{n2^{n/2}} \left( -1 \right)^{4n} 2^{n/2}\\
                                                                             &= \sum_{n=1}^{\infty} \frac{1}{n},
      \end{align*}
      which diverges. Similarly, we have
      \begin{align*}
        \sum_{n=1}^{\infty} \frac{n^2}{2^{n/2}} \left( 2^{1/8} \right)^{4n} &= \sum_{n=1}^{\infty} \frac{1}{n2^{n/2}} 2^{n/2}\\
                                                                            &= \sum_{n=1}^{\infty} \frac{1}{n},
      \end{align*}
      which diverges. Thus, the interval of convergence is $\left( -2^{1/8},2^{1/8} \right)$.
  \end{enumerate}
\end{solution}
\end{document}
