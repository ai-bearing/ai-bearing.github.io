\documentclass[12pt]{mypackage}

% sans serif font:
%\usepackage{cmbright}
%\usepackage{sfmath}
%\usepackage{bbold} %better blackboard bold

\usepackage{homework}
\usepackage{notes}
\usepackage{newpxtext,eulerpx,eucal}
\renewcommand*{\mathbb}[1]{\varmathbb{#1}}

\fancyhf{}
\fancyhead[C]{Calculus II Review and Exam Preparation: Solutions}
\fancyhead[L]{}
\fancyfoot[C]{\thepage}

\setcounter{secnumdepth}{0}

\begin{document}
\RaggedRight
\begin{center}
  \large
  Avoid looking at these solutions until you have genuinely given the practice problems an honest attempt. Each part will have solutions on a separate section, given in the table of contents, so even if you do want to see a solution, please only look at the solution for the part you are stuck on.
\end{center}
\tableofcontents
\pagebreak
\section{Problem 1}%
\begin{problem}\hfill
\begin{enumerate}[(a)]
  \item Consider the solid defined by rotating the region bounded by $x = 0$, $x = \pi/2$, $y = 0$, and $y = \cos(x)$. Set up integrals $I_1$ and $I_2$ for the volume and surface area of this solid respectively.
  \item Find the volume of the solid by resolving the integral. 
  \item Similarly, find an expression for the surface area of this solid. To find the antiderivative, you may find the following steps useful.
    \begin{enumerate}[(i)]
      \item Use a substitution to express the integral entirely in terms of square roots and polynomial expressions.
      \item Take $u = \tan(\theta)$ and use trigonometric identities to express the integral solely in terms of $\sec\left( \theta \right)$.
      \item Extract a factor of $\sec^2(\theta)$ and use integration by parts and a trigonometric identity to reduce this integral to that of $\sec(\theta)$.
      \item To evaluate the integral of $\sec(\theta)$, multiply top and bottom by $\sec(\theta) + \tan(\theta)$, then use a substitution.
    \end{enumerate}
\end{enumerate}
\end{problem}
\begin{solution}\hfill
  \begin{enumerate}[(a)]
    \item We start by considering the following graph.
\begin{center}
  \begin{tikzpicture}
    \begin{axis}[
        axis lines=center,
        xlabel={$x$},
        ylabel={$y$},
        domain=0:pi/2,
        samples=200,
        grid=major,
        grid style={line width=0.1pt, draw=gray!40},
        xmin=0, xmax=pi/1.9,
        ymin=0, ymax=1.2,
        xtick={0, pi/8, pi/4, 3*pi/8, pi/2},
        xticklabels={$0$, $\pi/8$, $\pi/4$, $3\pi/8$, $\pi/2$},
        ytick={0, 0.25, 0.5, 0.75, 1},
        yticklabels={$0$, $1/4$, $1/2$, $3/4$, $1$},
        width=8cm,
        height=8cm,
        thick
    ]
    % Plot sin²(x)
    \addplot[red, very thick, smooth] {cos(deg(x))};

    % Add function label
    \node[red] at (axis cs:3*pi/8,1) {$f(x) = \cos(x)$};

    \end{axis}
  \end{tikzpicture}
\end{center}
In order to write the volume of the solid, we see that at any $x$, the solid looks similar to a cylinder of height $dx$ and radius $\cos(x)$. In the limit, this gives that the volume can be written as
\begin{align*}
  A &= \int_{0}^{\pi/2} \pi \cos^2(x)\:dx.
\end{align*}
Similarly, the arc length is given by $\sqrt{dx^2 + dy^2}$, or $\sqrt{1 + \left( \diff{y}{x} \right)^2}dx$, meaning that, by using the surface area expression, we get
\begin{align*}
  S &= \int_{0}^{\pi/2} 2\pi \cos(x)\sqrt{1 + \sin^2(x)}\:dx.
\end{align*}
\item To evaluate the volume, we use the identity that $\cos(2x) = 2\cos^2(x)-1$. By rearranging, this gives that $\cos^2(x) = \frac{1 + \cos(2x)}{2}$. Therefore, we get
  \begin{align*}
    \int_{0}^{\pi/2} \pi\cos^2(x)\:dx &= \frac{\pi}{2} \int_{0}^{\pi/2} 1\:dx + \frac{\pi}{2} \int_{0}^{\pi/2} \cos(2x)\:dx.
    \intertext{By substituting with $u = 2x$ on the second integral and evaluating the first integral, we get}
                                      &= \frac{\pi^2}{4} + \frac{\pi}{4} \int_{0}^{\pi} \cos(u) \:du.
                                      &= \frac{\pi^2}{4} + \frac{\pi}{4} \left( \sin(u)|_{0}^{\pi} \right)\\
                                      &= \frac{\pi^2}{4} + \frac{\pi}{4} \left( \sin(\pi) - \sin(0)\right)\\
                                      &= \frac{\pi^2}{4}.
  \end{align*}
\item Evaluating the surface area integral will be a bit more involved. For now, we will disregard the bounds, and only evaluate them at the very end.\newline

  First, define $t = \sin(x)$, meaning $dt = \cos(x) dx$, and the substitution gives
  \begin{align*}
    \int_{0}^{\pi/2} \cos(x)\sqrt{1 + \sin^2(x)}\:dx &= \int_{0}^{1} \sqrt{1 + t^2}\:dt.
  \end{align*}
  We then define $t = \tan(\theta)$, or that $\theta = \arctan(t)$, meaning that we get $dt = \sec^2(\theta)\:d\theta$. This gives
  \begin{align*}
    \int_{0}^{1} \sqrt{1 + t^2}\:dt &= \int_{0}^{\pi/4} \sqrt{1 + \tan^2(\theta)}\sec^2(\theta)\:d\theta\\
                                  &= \int_{0}^{\pi/4} \sec^3(\theta)\:d\theta.
  \end{align*}
  To evaluate this particular integral, we take out a factor of $\sec^2\theta$, and use integration by parts. Since the derivative of $\tan(\theta)$ is $\sec^2\theta$, we let $dv = \sec^2\theta d\theta$, $u = \sec\theta$, and get
  \begin{align*}
    \int_{0}^{\pi/4} \sec^3(\theta)\:d\theta &= \sec\left( \theta \right)\tan\left( \theta \right)\vert_{0}^{\pi/4} - \int_{0}^{\pi/4} \sec(\theta)\tan^2\theta\:d\theta.
    \intertext{Substituting $\tan^2\theta = \sec^2\theta - 1$, we have}
                                             &= \sec(\theta)\tan(\theta)|_{0}^{\pi/4} - \int_{0}^{\pi/4} \sec^3\theta\:d\theta + \int_{0}^{\pi/4} \sec(\theta)\:d\theta.
  \end{align*}
  Therefore, we get that
  \begin{align*}
    \int_{0}^{\pi/4} \sec^3(\theta)\:d\theta &= \left(\frac{1}{2}\sec(\theta)\tan(\theta) \right)\biggr\vert_{0}^{\pi/4} + \frac{1}{2}\int_{0}^{\pi/4} \sec(\theta)\:d\theta.
  \end{align*}
  Finally, to evaluate the integral of $\sec(\theta)$, we multiply both numerator and denominator by $\sec(\theta) + \tan(\theta) $. All in all, this gives
  \begin{align*}
    \left( \frac{1}{2}\sec(\theta)\tan(\theta) \right)\biggr\vert_{0}^{\pi/4} + \frac{1}{2}\int_{0}^{\pi/4} \sec(\theta)\:d\theta &= \frac{1}{2}\sqrt{2} + \int_{0}^{\pi/4} \frac{\sec^2\theta + \sec(\theta)\tan(\theta)}{\sec(\theta) + \tan(\theta)}\:d\theta.
    \intertext{Using one final substitution, this time taking the dummy variable $q = \sec(\theta) + \tan(\theta)$, we get}
                                                                                                                                  &= \frac{1}{2}\sqrt{2} + \frac{1}{2}\int_{1}^{\sqrt{2} + 1} \frac{1}{q}\:dq\\
                                                                                                                                  &= \frac{1}{2}\sqrt{2} + \left( \frac{1}{2}\ln\left\vert q \right\vert \right) \biggr\vert_{1}^{\sqrt{2} + 1}\\
                                                                                                         &= \frac{1}{2}\sqrt{2} + \frac{1}{2}\ln\left( 1 + \sqrt{2} \right)
  \end{align*}
  This is our final surface area.
  \end{enumerate}
\end{solution}
\begin{remark}
  It is possible to use a different evaluation technique, where instead of substituting tangent, one instead substitutes a \href{https://en.wikipedia.org/wiki/Hyperbolic_functions}{hyperbolic trigonometric function}. The hyperbolic sine and cosine are given by
  \begin{align*}
    \sinh(x) &= \frac{e^{x} - e^{-x}}{2}\\
    \cosh(x) &= \frac{e^{x} + e^{-x}}{2}.
  \end{align*}
  Similar to how $\sin$ and $\cos$ are defined on the circle $x^2 + y^2 = 1$, with $\sin$ being the $y$-value and $\cos$ being the $x$-value, $\sinh$ and $\cosh$ emerge from the hyperbola $x^2 - y^2 = 1$, with $\sinh$ being the $y$-value and $\cosh$ being the $x$-value. The corresponding ``Pythagorean'' identity for $\sinh$ and $\cosh$ is then
  \begin{align*}
    \cosh^2(x) - \sinh^2(x) &= 1.
  \end{align*}
  Do you think you can rework the solution by using this identity for your substitution?
\end{remark}
\end{document}
