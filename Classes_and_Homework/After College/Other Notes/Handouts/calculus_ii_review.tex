\documentclass[12pt]{mypackage}

% sans serif font:
%\usepackage{cmbright}
%\usepackage{sfmath}
%\usepackage{bbold} %better blackboard bold

%\usepackage{homework}
%\usepackage{notes}
\usepackage{newpxtext,eulerpx,eucal}
\renewcommand*{\mathbb}[1]{\varmathbb{#1}}

\fancyhf{}
\fancyhead[C]{Calculus II Review and Exam Preparation}
\fancyfoot[C]{\thepage}

\setcounter{secnumdepth}{0}

\begin{document}
\RaggedRight
\section{Useful Results}%
These will almost certainly not be given to you on the final exam, but it is important to have these ones memorized to the point where if someone woke you up at 3am you would be able to recite these.
\begin{itemize}
  \item Pythagorean Identities:
    \begin{align*}
      \sin^2(x) + \cos^2(x) &= 1\\
      1 + \tan^2(x) &= \sec^2(x)\\
      1 + \cot^2(x) &= \csc^2(x).
    \end{align*}
  \item Double-Angle identities:
    \begin{align*}
      \sin\left( 2x \right) &= 2\sin(x)\cos(x)\\
      \cos\left( 2x \right) &= \cos^2(x)-\sin^2(x)\\
                            &= 1 - 2\sin^2(x)\\
                            &= 2\cos^2(x)-1.
    \end{align*}
  \item Limits:
    \begin{align*}
      \lim_{x\rightarrow 0} \frac{\sin(x)}{x} &= 1\\
      \lim_{n\rightarrow 0} \frac{1}{n} &= 0.
    \end{align*}
  \item Derivatives:
    \begin{align*}
      \diff{}{x}\left( \sin(x) \right) &= \cos(x)\\
      \diff{}{x}\left( \cos(x) \right) &= -\sin(x)\\
      \diff{}{x}\left( \tan(x) \right) &= \sec^2(x)\\
      \diff{}{x}\left( \cot(x) \right) &= -\csc^2(x)\\
      \diff{}{x}\left( \sec(x) \right) &= \sec(x)\tan(x)\\
      \diff{}{x}\left( \csc(x) \right) &= -\csc(x)\cot(x)\\
      \diff{}{x}\left( x^{n} \right) &= nx^{n-1}\\
      \diff{}{x}\left( e^{x} \right) &= e^{x}\\
      \diff{}{x}\left( \ln(x) \right) &= \frac{1}{x}\\
      \diff{}{x}\left( \arctan(x) \right) &= \frac{1}{1 + x^2}.
    \end{align*}
\end{itemize}
\pagebreak
\section{Practice Problems}%
These are some relatively involved multi-step practice problems that are similar (but likely more difficult) than the ones you may encounter on the final exam.
\subsection{Practice Problem 1}%
\begin{enumerate}[(a)]
  \item Consider the solid defined by rotating the region bounded by $x = 0$, $x = \pi/2$, $y = 0$, and $y = \cos(x)$. Set up integrals $I_1$ and $I_2$ for the volume and surface area of this solid respectively.
  \item Find the volume of the solid by resolving the integral. 
  \item Similarly, find an expression for the surface area of this solid. To find the antiderivative, you may find the following steps useful.
    \begin{enumerate}[(i)]
      \item Use a substitution to express the integral entirely in terms of square roots and polynomial expressions.
      \item Take $u = \tan(\theta)$ and use trigonometric identities to express the integral solely in terms of $\sec\left( \theta \right)$.
      \item Extract a factor of $\sec^2(\theta)$ and use integration by parts and a trigonometric identity to reduce this integral to that of $\sec(\theta)$.
      \item To evaluate the integral of $\sec(\theta)$, multiply top and bottom by $\sec(\theta) + \tan(\theta)$, then use a substitution.
    \end{enumerate}
\end{enumerate}
\subsection{Practice Problem 2}%
Evaluate the convergence of the following series, determining whether a series is absolutely convergent, conditionally convergent, or divergent. Explicitly indicate which test(s) you are using to arrive at your conclusion, and carefully verify the hypotheses.
\begin{enumerate}[(a)]
  \item 
    \begin{align*}
      \sum_{n=1}^{\infty} \frac{\ln(n)}{n^2}.
    \end{align*}
  \item
    \begin{align*}
      \sum_{n=0}^{\infty} \frac{\left( -1 \right)^{n}}{2n+1}.
    \end{align*}
  \item 
    \begin{align*}
      \sum_{n=0}^{\infty} \frac{\sin\left( \frac{1}{n} \right)}{n^2}.
    \end{align*}
\end{enumerate}
\subsection{Practice Problem 3}%
Find the \textit{interval} of convergence for the following power series.
\begin{enumerate}[(a)]
  \item
    \begin{align*}
      \sum_{n=1}^{\infty} \left( n^{5} +3n^2 + 2n + 3 \right) \frac{\left( x-3 \right)^{n}}{4^{n}}.
    \end{align*}
  \item 
    \begin{align*}
      \sum_{n=1}^{\infty} \frac{2^{3n}}{n^25^{2n}} \left( x-2 \right)^{n^2}.
    \end{align*}
  \item 
    \begin{align*}
      \sum_{n=1}^{\infty} \frac{n^2}{2^{n/2}} x^{4n}.
    \end{align*}
\end{enumerate}
\end{document}
