\documentclass[10pt]{mypackage}

% sans serif font:
%\usepackage{cmbright}
%\usepackage{sfmath}
%\usepackage{bbold} %better blackboard bold

%\usepackage{homework}
\usepackage{notes}
\usepackage{newpxtext,eulerpx,eucal}
\renewcommand*{\mathbb}[1]{\varmathbb{#1}}

\fancyhf{}
\fancyhead[R]{Avinash Iyer}
\fancyhead[L]{Generating Functions}
\fancyfoot[C]{\thepage}

\setcounter{secnumdepth}{0}

\begin{document}
\RaggedRight
These are some notes on generating functions I'm taking, primarily from \textit{The Art of Computer Programming}, Volume I, by Donald Knuth.
\begin{definition}
  If $\left( a_0,a_1,\dots \right)$ is a sequence of numbers, then the infinite series
  \begin{align*}
    G(z) &= a_0 + a_1 z + \cdots\\
         &= \sum_{n=0}^{\infty}a_nz^{n}
  \end{align*}
  is known as the \textit{generating function} for the sequence.
\end{definition}
We can try to understand some basic operations for the generating function.\newline

To start, if $G(z)$ is the generating function for $\left( a_n \right)_n$, then the generating function for $\left( a_{n-m} \right)_n$ for a fixed $m$ is given by $z^{m}G(z)$, as
\begin{align*}
  z^{m}G(z) &= z^{m}\sum_{n=0}^{\infty}a_nz^{n}\\
            &= \sum_{n=m}^{\infty}a_{n-m}z^{n}.
\end{align*}
Similarly, the generating function for $\left( a_{n+m} \right)_n$ is given by $z^{-m}\left( G(z) - \sum_{k=0}^{m-1}a_kz^{k} \right)$, as
\begin{align*}
  z^{-m}\sum_{n=m}^{\infty}a_nz^{n} &= \sum_{n=0}^{\infty}a_{n+m}z^{n}.
\end{align*}
\begin{example}
  Our first example is finding a generating function for the Fibonacci sequence. Letting $G(z)$ be the generating function for $\left( F_n \right)_n$, we have that $zG(z)$ is the generating function for $\left( F_{n-1} \right)_n$ and $z^2G(z)$ is the generating function for $\left( F_{n-2} \right)_n$. In particular, since $F_{n} = F_{n-1} + F_{n-2}$, we have that
  \begin{align*}
    \left( 1-z-z^2 \right)G(z) &= p(z),
  \end{align*}
  where $p(z)$ is some polynomial satisfying $p(0) = 0$ and $p(1) = 1$. In particular, we can show that this yields
  \begin{align*}
    G(z) &= \frac{z}{1-z-z^2}.
  \end{align*}
  In general, for a linear recurrence $a_n = c_1a_{n-1} + \cdots + c_ma_{n-m}$, the corresponding generating function will be of the form
  \begin{align*}
    G(z) &= \frac{p(z)}{1-c_1z-\cdots-c_mz^{m}}
  \end{align*}
  for some polynomial $p(z)$. In particular, for the simplest case, this yields
  \begin{align*}
    \frac{1}{1-z} &= 1 + z + z^2 + \cdots
  \end{align*}
  is the generating function for the sequence $\left( 1,1,1,\dots \right)$.
\end{example}
Next, we discuss multiplication. If we let
\begin{align*}
  G(z) &= \sum_{n=0}^{\infty}a_nz^{n}\\
  H(z) &= \sum_{n=0}^{\infty}b_nz^{n},
\end{align*}
then we observe that
\begin{align*}
  G(z)H(z) &= \sum_{n=0}^{\infty}\left( \sum_{k=0}^{n}a_kb_{n-k} \right) z^{n}.
\end{align*}
In the particular case where $b$ is the sequence of all $1$s, then
\begin{align*}
  \frac{1}{z}G(z) &= \sum_{n=0}^{\infty}\left( \sum_{k=0}^{n}a_k \right)z^{n}.
\end{align*}
When we want binomial coefficients of the form
\begin{align*}
  c_n &= \sum_{k=0}^{n} {n\choose k} a_kb_{n-k},
\end{align*}
we usually use the generating functions for $\left( \frac{a_n}{n!} \right)_n$ and $\left( \frac{b_n}{n!} \right)_n$, which yields the generating function for $\left( \frac{c_n}{n!} \right)_n$.\newline

Next, to extract particular terms of a generating function, we let $\omega = e^{2\pi i/m}$ be a primitive $m$th root of unity, and find
\begin{align*}
  \sum_{n\equiv r\text{ mod }m}a_nz^{n} &= \frac{1}{m} \sum_{k=0}^{m-1} \omega^{-kr}G\left( \omega^{k}z \right).
\end{align*}
For instance, if $m = 3$ and $r = 1$, then
\begin{align*}
  a_1 z + a_4 z^{4} + \cdots &= \frac{1}{3} \left( G(z) + \omega^{-1}G\left( \omega z \right) + \omega^{-2}G\left( \omega^2 z \right) \right).
\end{align*}
Differentiation and integration yield similar results:
\begin{align*}
  zG'(z) &= \sum_{n=0}^{\infty} na_nz^{n}\\
  \int_{0}^{z} G(t)\:dt &= \sum_{k=1}^{\infty}\frac{1}{k}a_{k-1} z^{k}
\end{align*}
We next write down some generating functions.
\begin{itemize}
  \item Binomial theorem:
    \begin{align*}
      \left( 1+z \right)^{r} &= \sum_{k=0}^{\infty} {r\choose k} z^{k}.
    \end{align*}
  \item Exponents:
    \begin{align*}
      e^{z} &= \sum_{k=0}^{\infty} \frac{1}{k!} z^{k}\\
      \left( e^{z}-1 \right)^{n} &= n!\sum_{k=0}^{\infty} S\left( n,k \right) \frac{z^{k}}{k!},
    \end{align*}
    where $S\left( n,k \right)$ denotes the Stirling numbers of the second kind. The Stirling numbers of the second kind denote the number of ways to partition a set of $n$ elements into $k$ disjoint (nonempty) subsets.
  \item Logarithms:
    \begin{align*}
      \ln\left( 1+z \right) &= \sum_{k=1}^{\infty}\frac{\left( -1 \right)^{k+1}}{k}z^{k}\\
      \frac{1}{\left( 1-z \right)^{m+1}}\ln\left( \frac{1}{1-z} \right) &= \sum_{k=1}^{\infty} \left( H_{m+k}-H_m \right) {m+k\choose k} z^{k},
    \end{align*}
    where $H_k$ denotes the $k$th harmonic number.
\end{itemize}
\begin{example}
  Consider the elementary symmetric functions on $n$ variables, given by
  \begin{align*}
    e_m &= \sum_{1\leq j_1 < \cdots < j_m \leq n} x_{j_1}\cdots x_{j_m}.
  \end{align*}
  Observe that the $e_m$ appear as the coefficient of $z^{m}$ in the polynomial
  \begin{align*}
    G_n(z) &= \prod_{i=1}^{n} \left( 1 + x_i z \right).
  \end{align*}

\end{example}
\end{document}
