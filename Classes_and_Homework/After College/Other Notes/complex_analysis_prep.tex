\documentclass[10pt]{mypackage}

% sans serif font:
%\usepackage{cmbright}
%\usepackage{sfmath}
%\usepackage{bbold} %better blackboard bold

%\usepackage{homework}
\usepackage{notes}
\usepackage{newpxtext,eulerpx,eucal}
\renewcommand*{\mathbb}[1]{\varmathbb{#1}}

\fancyhf{}
\rhead{Avinash Iyer}
\lhead{Complex Analysis Qualifier Preparation}

\setcounter{secnumdepth}{0}

\begin{document}
\RaggedRight
This is a collection of old complex analysis qualifier exam solutions, as well as some notes on useful results and proofs.
\section{Useful Results and Proofs}%
\subsection{Analytic Functions}%
\begin{definition}
  Let $U\subseteq \C$ be an open set. A function $f\colon U\rightarrow \C$ is called \textit{analytic} if, for any $z_0\in U$, there is $r > 0$ and $\left( a_k \right)_k\subseteq \C$ such that
  \begin{align*}
    f(z) &= \sum_{k=0}^{\infty}a_k\left( z-z_0 \right)^{k}
  \end{align*}
  for all $z\in U\left( z_0,r \right)$.
\end{definition}
Analytic functions form a $\C$-algebra.
\begin{theorem}[Identity Theorem]
  Let $f,g\colon U\rightarrow \C$ be analytic functions defined a connected open set (also known as a region). If
  \begin{align*}
    A &= \set{z\in \C | f(z) = g(z)}
  \end{align*}
  admits an accumulation point in $U$, then $f = g$ on $U$.
\end{theorem}
\begin{proof}
  To begin, we show that if $f\colon U\rightarrow \C$ is an analytic function that is not uniformly zero, then for any $z_0\in U$, there is $\rho > 0$ such that $f$ is nonzero on $\dot{U}\left( z_0,\rho \right)\subseteq U$. Towards this end, we may write
  \begin{align*}
    f(z) &= \sum_{k=0}^{\infty}a_k\left( z-z_0 \right)^{k},
  \end{align*}
  for all $z\in U\left( z_0,r \right)$, some $r > 0$, and since $f$ is not uniformly zero, there is some minimal $\ell$ such that $a_{\ell}\neq 0$. This yields
  \begin{align*}
    f(z) &= \left( z-z_0 \right)^{\ell}\sum_{k=0}^{\infty}a_{k + \ell}\left( z-z_0 \right)^{k};
  \end{align*}
  the function $h\colon U\left( z_0,r \right)\rightarrow \C$ given by
  \begin{align*}
    h(z) &= \sum_{k=0}^{\infty}a_{k + \ell}\left( z-z_0 \right)^{k}
  \end{align*}
  then has the same radius of convergence as $f$ and is not zero at $z_0$, so that $g$ is not zero on some $U\left( z_0,\rho \right)$ as $g$ is continuous.\newline

  Now, we let $V_1$ be the set of accumulation points of $A$ in $U$, and let $V_2 = U\setminus V_1$.\newline

  If $z\in V_2$, then there is some $r_1 > 0$ such that $\dot{U}\left( z_0,r_1 \right)\cap A = \emptyset$, or that $\dot{U}\left( z_0,r_1 \right) \subseteq A^{c}$. Meanwhile, since $U$ is open, there is some $r_2 > 0$ such that $U\left( z_0,r_2 \right)\subseteq U$, meaning that if $r = \min\set{r_1,r_2}$, then $U\left( z_0,r \right) \subseteq U\setminus A$. Thus, $V_2$ is open.\newline

  Meanwhile, if $z\in V_1$, then since $V_1\subseteq U$, it follows that there is $r > 0$ such that $U\left( z,r \right)$ and $\left( a_k \right)_k$ such that
  \begin{align*}
    f(w)- g(w) &= \sum_{k=0}^{\infty}a_k\left( w-z \right)^{k}
  \end{align*}
  for all $w\in U\left( z,r \right)$. We claim that $f(w) - g(w)$ is uniformly zero on $U\left( z,r \right)$. Else, if there were $w_0\in U\left( z,r \right)$ such that $f\left( w_0 \right)\neq g\left( w_0 \right)$, then it would follow that there is $0 < s\leq r$ such that $f(w)\neq g(w)$ for all $w\in \dot{U}\left( w_0,s \right)$. Yet, this would contradict the assumption that $z$ is an accumulation point, meaning that $V_1$ is open.\newline

  Since $V_1$ and $V_2$ are disjoint open sets whose union is equal to $U$, it follows that either $V_1 = U$ or $V_2 = U$. If $A \neq \emptyset$, then the identity theorem follows.
\end{proof}
\subsection{Differentiability}%
\begin{definition}
  If $U\subseteq \C$ is an open set, then we say $f$ is differentiable at $z_0\in U$ if
  \begin{align*}
    \lim_{w\rightarrow z_0} \frac{f\left( w \right)-f\left( z_0 \right)}{w-z_0}
  \end{align*}
  exists. We call this value the \textit{derivative} of $f$ at $z_0$, and usually write $f'\left(z_0\right)$.\newline

  If $f$ is differentiable at every $z_0\in U$, we say $f$ is differentiable on $U$.\newline

  If $f$ is continuous and admits a continuous derivative, then we say $f$ is \textit{holomorphic}.
\end{definition}
Note that the limit must be independent of direction. That is, for all $\ve > 0$, there is $\delta > 0$ such that
\begin{align*}
  \left\vert \frac{f\left( w \right)-f\left( z_0 \right)}{z-z_0} - f'\left( z_0 \right) \right\vert &< \ve
\end{align*}
whenever $0 < \left\vert z-z_0 \right\vert < \delta$.\newline

Now, given $U\subseteq \C$, write $z = x + iy$ and
\begin{align*}
  f\left( z \right) &= f\left( x + iy \right)\\
                    &= u\left( x,y \right) + iv\left( x,y \right),
\end{align*}
where $u = \re(f)$ and $v = \im(f)$. Observe then that if $f$ is differentiable at $x_0 + iy_0\in U$, then since the limit is independent of path, by taking the limit over real numbers, we have
\begin{align*}
  f'\left( z_0 \right) &= \lim_{h\rightarrow 0} \frac{\left( u\left( x + h,y \right) + iv\left( x + h,y \right) \right) - \left( u\left( x,y \right) + iv\left( x,y \right) \right)}{ h }\\
                       &= \pd{u}{x} + i \pd{v}{x},
\end{align*}
and by taking over the imaginary numbers,
\begin{align*}
  f'\left( z_0 \right) &= \lim_{h\rightarrow 0} \frac{\left( u\left( x,y+h \right) + iv\left( x,y+h \right) \right) - \left( u\left( x,y \right) + iv\left( x,y \right) \right)}{ih}\\
                       &= -i \pd{u}{y} + \pd{v}{y}.
\end{align*}
Thus, we obtain the following.
\begin{definition}
  The system of partial differential equations
  \begin{align*}
    \pd{u}{x} &= \pd{v}{y}\\
    \pd{u}{y} &= - \pd{v}{x}
  \end{align*}
  is known as the \textit{Cauchy--Riemann Equations}.
\end{definition}
Observe that if $f$ is differentiable, then the $u$ and $v$ in the definition of $f$ satisfy the Cauchy--Riemann equations. Yet, we desire to understand a bit more about when exactly $f$ is differentiable or holomorphic.
\subsection{Cauchy's Integral Formula and its Consequences}%

\section{Old Exams}%
\section{Notation}%
\begin{itemize}
  \item $\ds U\left( z_0,r \right) = \set{z\in \C | \left\vert z-z_0 \right\vert < r}$
  \item $\ds B\left( z_0,r \right) = \set{z\in \C | \left\vert z-z_0 \right\vert \leq r}$
  \item $\ds S\left( z_0,r \right) = \set{z\in \C | \left\vert z-z_0 \right\vert = r}$
  \item $\ds \dot{U}\left( z_0,r \right) = \set{z\in \C | 0 < \left\vert z-z_0 \right\vert < r}$
  \item $\ds A\left( z_0,r_1,r_2 \right) = \set{z\in \C | r_1 < \left\vert z-z_0 \right\vert < r_2}$
\end{itemize}
\end{document}
