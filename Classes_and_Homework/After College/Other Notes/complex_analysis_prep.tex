\documentclass[10pt]{mypackage}

% sans serif font:
%\usepackage{cmbright}
%\usepackage{sfmath}
%\usepackage{bbold} %better blackboard bold

%\usepackage{homework}
\usepackage{notes}
\usepackage{newpxtext,eulerpx,eucal}
\renewcommand*{\mathbb}[1]{\varmathbb{#1}}
\DeclareMathOperator{\ord}{ord}

\fancyhf{}
\rhead{Avinash Iyer}
\lhead{Complex Analysis Qualifier Preparation}

\setcounter{secnumdepth}{0}

\begin{document}
\RaggedRight
This is a collection of old complex analysis qualifier exam solutions, as well as some notes on useful results and proofs.
\section{Useful Results and Proofs}%
\subsection{Analytic Functions}%
\begin{definition}
  Let $U\subseteq \C$ be an open set. A function $f\colon U\rightarrow \C$ is called \textit{analytic} if, for any $z_0\in U$, there is $r > 0$ and $\left( a_k \right)_k\subseteq \C$ such that
  \begin{align*}
    f(z) &= \sum_{k=0}^{\infty}a_k\left( z-z_0 \right)^{k}
  \end{align*}
  for all $z\in U\left( z_0,r \right)$.
\end{definition}
Analytic functions form a $\C$-algebra.
\begin{theorem}[Identity Theorem]
  Let $f,g\colon U\rightarrow \C$ be analytic functions defined a connected open set (also known as a region). If
  \begin{align*}
    A &= \set{z\in \C | f(z) = g(z)}
  \end{align*}
  admits an accumulation point in $U$, then $f = g$ on $U$.
\end{theorem}
\begin{proof}
  To begin, we show that if $f\colon U\rightarrow \C$ is an analytic function that is not uniformly zero, then for any $z_0\in U$, there is $\rho > 0$ such that $f$ is nonzero on $\dot{U}\left( z_0,\rho \right)\subseteq U$. Towards this end, we may write
  \begin{align*}
    f(z) &= \sum_{k=0}^{\infty}a_k\left( z-z_0 \right)^{k},
  \end{align*}
  for all $z\in U\left( z_0,r \right)$, some $r > 0$, and since $f$ is not uniformly zero, there is some minimal $\ell$ such that $a_{\ell}\neq 0$. This yields
  \begin{align*}
    f(z) &= \left( z-z_0 \right)^{\ell}\sum_{k=0}^{\infty}a_{k + \ell}\left( z-z_0 \right)^{k};
  \end{align*}
  the function $h\colon U\left( z_0,r \right)\rightarrow \C$ given by
  \begin{align*}
    h(z) &= \sum_{k=0}^{\infty}a_{k + \ell}\left( z-z_0 \right)^{k}
  \end{align*}
  then has the same radius of convergence as $f$ and is not zero at $z_0$, so that $g$ is not zero on some $U\left( z_0,\rho \right)$ as $g$ is continuous.\newline

  Now, we let $V_1$ be the set of accumulation points of $A$ in $U$, and let $V_2 = U\setminus V_1$.\newline

  If $z\in V_2$, then there is some $r_1 > 0$ such that $\dot{U}\left( z_0,r_1 \right)\cap A = \emptyset$, or that $\dot{U}\left( z_0,r_1 \right) \subseteq A^{c}$. Meanwhile, since $U$ is open, there is some $r_2 > 0$ such that $U\left( z_0,r_2 \right)\subseteq U$, meaning that if $r = \min\set{r_1,r_2}$, then $U\left( z_0,r \right) \subseteq U\setminus A$. Thus, $V_2$ is open.\newline

  Meanwhile, if $z\in V_1$, then since $V_1\subseteq U$, it follows that there is $r > 0$ such that $U\left( z,r \right)$ and $\left( a_k \right)_k$ such that
  \begin{align*}
    f(w)- g(w) &= \sum_{k=0}^{\infty}a_k\left( w-z \right)^{k}
  \end{align*}
  for all $w\in U\left( z,r \right)$. We claim that $f(w) - g(w)$ is uniformly zero on $U\left( z,r \right)$. Else, if there were $w_0\in U\left( z,r \right)$ such that $f\left( w_0 \right)\neq g\left( w_0 \right)$, then it would follow that there is $0 < s\leq r$ such that $f(w)\neq g(w)$ for all $w\in \dot{U}\left( w_0,s \right)$. Yet, this would contradict the assumption that $z$ is an accumulation point, meaning that $V_1$ is open.\newline

  Since $V_1$ and $V_2$ are disjoint open sets whose union is equal to $U$, it follows that either $V_1 = U$ or $V_2 = U$. If $A \neq \emptyset$, then the identity theorem follows.
\end{proof}
\subsection{Differentiability}%
\begin{definition}
  If $U\subseteq \C$ is an open set, then we say $f$ is differentiable at $z_0\in U$ if
  \begin{align*}
    \lim_{w\rightarrow z_0} \frac{f\left( w \right)-f\left( z_0 \right)}{w-z_0}
  \end{align*}
  exists. We call this value the \textit{derivative} of $f$ at $z_0$, and usually write $f'\left(z_0\right)$.\newline

  If $f$ is differentiable at every $z_0\in U$, we say $f$ is differentiable on $U$.\newline

  If $f$ is continuous and admits a continuous derivative, then we say $f$ is \textit{holomorphic}.
\end{definition}
Note that the limit must be independent of direction. That is, for all $\ve > 0$, there is $\delta > 0$ such that
\begin{align*}
  \left\vert \frac{f\left( w \right)-f\left( z_0 \right)}{z-z_0} - f'\left( z_0 \right) \right\vert &< \ve
\end{align*}
whenever $0 < \left\vert z-z_0 \right\vert < \delta$.\newline

Now, given $U\subseteq \C$, write $z = x + iy$ and
\begin{align*}
  f\left( z \right) &= f\left( x + iy \right)\\
                    &= u\left( x,y \right) + iv\left( x,y \right),
\end{align*}
where $u = \re(f)$ and $v = \im(f)$. Observe then that if $f$ is differentiable at $x_0 + iy_0\in U$, then since the limit is independent of path, by taking the limit over real numbers, we have
\begin{align*}
  f'\left( z_0 \right) &= \lim_{h\rightarrow 0} \frac{\left( u\left( x + h,y \right) + iv\left( x + h,y \right) \right) - \left( u\left( x,y \right) + iv\left( x,y \right) \right)}{ h }\\
                       &= \pd{u}{x} + i \pd{v}{x},
\end{align*}
and by taking over the imaginary numbers,
\begin{align*}
  f'\left( z_0 \right) &= \lim_{h\rightarrow 0} \frac{\left( u\left( x,y+h \right) + iv\left( x,y+h \right) \right) - \left( u\left( x,y \right) + iv\left( x,y \right) \right)}{ih}\\
                       &= -i \pd{u}{y} + \pd{v}{y}.
\end{align*}
Thus, we obtain the following.
\begin{definition}
  The system of partial differential equations
  \begin{align*}
    \pd{u}{x} &= \pd{v}{y}\\
    \pd{u}{y} &= - \pd{v}{x}
  \end{align*}
  is known as the \textit{Cauchy--Riemann Equations}.
\end{definition}
Observe that if $f$ is differentiable, then the $u$ and $v$ in the definition of $f$ satisfy the Cauchy--Riemann equations. Yet, we desire to understand a bit more about when exactly $f$ is differentiable or holomorphic.
\begin{proposition}
  If $f = u + iv$ is a holomorphic function such that $u,v$ are in $C^{2}\left( U \right)$, then $u$ and $v$ are harmonic. That is, $u$ and $v$ satisfy Laplace's equation:
  \begin{align*}
    \pd{^2u}{x^2} + \pd{^2u}{y^2} &= 0.
  \end{align*}
\end{proposition}
We call $u$ and $v$ \textit{harmonic conjugates} for each other. That is, if $u\colon U\rightarrow \R$ is a harmonic function, then $v\in C^{1}\left( U \right)$ is called a harmonic conjugate if the Cauchy--Riemann equations hold for $u$ and $v$.
\begin{theorem}
  Let $U\subseteq \R^{2}$ be a ball or all of $\R^{2}$. Then, every harmonic function on $U$ has a harmonic conjugate. If $u\in C^{3}\left( U \right)$, then this conjugate is itself harmonic.
\end{theorem}
\begin{lemma}
  Let $g\colon U\left( \left( x_0,y_0 \right),R \right)\rightarrow \R$ be such that $g$ and $ \pd{g}{x} $ are continuous. Then, $G\colon U\left( \left( x_0,y_0 \right),R \right)\rightarrow \R$, given by
  \begin{align*}
    G\left(x,y\right) &= \int_{y_0}^{y} g\left( x,t \right)\:dt
  \end{align*}
  satisfies
  \begin{align*}
    \pd{G}{x} &= \int_{y_0}^{y} \pd{g}{x}\left( x,t \right)\:dt.
  \end{align*}
\end{lemma}
\begin{proof}[Proof of Lemma]
  Write
  \begin{align*}
    \frac{G\left( x + h,y \right) - G\left( x,y \right)}{h} - \int_{y_0}^{y} \pd{g}{x}\left( x,t \right)\:dt &= \int_{y_0}^{y} \left( \frac{g\left( x+h,t \right)-g\left( x,t \right)}{h} - \pd{g}{x}\left( x,t \right) \right)\:dt.
  \end{align*}
  By mean value theorem, the first term is equal to $ \pd{g}{x}\left( x_1,t \right) $ for some $x_1$ between $x$ and $x + h$. As $h\rightarrow 0$, $x_1\rightarrow x$, as $ \pd{g}{x} $ is uniformly continuous on a compact subset that contains $x$ and $x + h$. We may exchange limit and integral to obtain the desired result.
\end{proof}
\begin{proof}[Proof of Theorem]
  We prove for the case of $U = U\left( \left( x_0,y_0 \right),R \right)$. Define
  \begin{align*}
    v\left( x,y \right) &= \int_{y_0}^{y} \pd{u}{x}\left( x,t \right)\:dt + \phi(x),
  \end{align*}
  with $\phi(x)$ to be determined later. By the fundamental theorem of calculus, we have
  \begin{align*}
    \pd{v}{y} &= \pd{u}{x},
  \end{align*}
  while by differentiating under the integral sign, and using the fact that $u$ is harmonic, we have
  \begin{align*}
    \pd{v}{x} &= \int_{y_0}^{y} \pd{^2u}{x^2}\left( x,t \right)\:dt + \diff{\phi}{x}\\
              &= - \int_{y_0}^{y} \pd{^2u}{y^2}\left( x,t \right)\:dt + \diff{\phi}{x}\\
              &= - \pd{u}{y}\left( x,y \right) + \pd{u}{y}\left( x,y_0 \right) + \diff{\phi}{x}.
  \end{align*}
  Defining $\phi\colon \R\rightarrow \R$ by
  \begin{align*}
    \phi(x) &= -\int_{x_0}^{x} \pd{u}{y}\left( s,y_0 \right)\:ds,
  \end{align*}
  we see that $v$ thus satisfies all the necessary requirements to be a harmonic conjugate.\newline

  Now, if $u$ is $C^{3}$, then we defined $v$ via the derivative of $u$, so that $v$ is $C^{2}$, and thus $v$ is harmonic.
\end{proof}
\subsection{Cauchy's Integral Formula}%
\begin{proposition}
  Fix $z_0\in \C$, $R > 0$, and $f\colon U\left( z_0,R \right)\rightarrow \C$ holomorphic. For all $z\in U\left( z_0,R \right)$, we have
  \begin{align*}
    f(z) &= \frac{1}{2\pi i} \int_{S\left( z_0,R \right)}^{} \frac{f\left( \zeta \right)}{\zeta - z}\:d\zeta.
  \end{align*}
\end{proposition}
\begin{proof}
  It suffices to show that
  \begin{align*}
    \frac{1}{2\pi i} \int_{S\left( z_0,R \right)}^{} \frac{f\left( \zeta \right)-f\left( z \right)}{\zeta - z}\:d\zeta &= 0.
  \end{align*}
  By using the chain rule and fundamental theorem of calculus, we find
  \begin{align*}
    \frac{1}{2\pi i} \int_{S\left( z_0,R \right)}^{} \frac{f\left( \zeta \right)-f\left( z \right)}{\zeta - z}\:d\zeta &= \frac{1}{2\pi i}\int_{S\left( z_0,R \right)}^{} \frac{ \int_{0}^{1} f'\left( \left( 1-t \right)z + t\zeta \right)\left( \zeta - z \right)\:dt }{\zeta - z}\:d\zeta\\
                                                                                                                       &= \frac{1}{2\pi i} \int_{S\left( z_0,R \right)}^{} \int_{0}^{1} f'\left( \left( 1-t \right)z + t\zeta \right)\:dt \:d\zeta\\
                                                                                                                       &= \frac{1}{2\pi i} \int_{S\left( z_0,R \right)}^{} \diff{}{\zeta}\left( \frac{1}{t}f\left( \left( 1-t \right)z + t\zeta \right) \right)\:d\zeta\\
                                                                                                                       &= 0.
  \end{align*}
\end{proof}
\begin{proposition}
  Let $f\colon U\rightarrow \C$ be a holomorphic function. The following all hold:
  \begin{enumerate}[(i)]
    \item $f$ is analytic;
    \item $f$ is smooth with $f^{(n)}$ holomorphic;
    \item for all $z_0\in U$, if we let $R = \sup\set{r > 0 | U\left( z_0,r \right)\subseteq U}$, then there is $\left( a_n \right)_n\subseteq \C$ such that
      \begin{align*}
        f\left( z \right) &= \sum_{n=0}^{\infty} a_n\left( z-z_0 \right)^{n},
      \end{align*}
      where the power series has radius of convergence $R$.
  \end{enumerate}
\end{proposition}
\begin{proof}\hfill
  \begin{enumerate}[(i)]
    \item There exists $r < s$ with $U\left( z_0,s \right)\subseteq U$ and $r < r_1 < s$ such that $S\left( z_0,r_1 \right) \subseteq U$. By Cauchy's Integral Formula, and a power series expansion of $\frac{1}{\xi - z}$ about $z_0$, this gives
      \begin{align*}
        f(z) &= \frac{1}{2\pi i} \oint_{S\left( z_0,r_1 \right)}^{} \frac{f\left( \xi \right)}{\xi-z}\:d\xi\\
             &=\sum_{n=0}^{\infty} \left( z-z_0 \right)^n \underbrace{\left( \frac{1}{2\pi i} \oint_{S\left( z_0,r_1 \right)}^{} \frac{f\left( \xi \right)}{\left( \xi-z_0 \right)^{n+1}}\:d\xi \right)}_{\eqcolon a_n}\\
             &=\sum_{n=0}^{\infty}a_n\left( z-z_0 \right)^{n}.
      \end{align*}
    \item Analytic functions are automatically smooth, hence complex-differentiable with continuous derivative.
    \item If $r < r_1 < R$, then
      \begin{align*}
        f\left( z \right) &= \sum_{n=0}^{\infty}\left( z-z_0 \right)^{n} \left( \frac{1}{2\pi i} \int_{S\left( z_0,r_1 \right)}^{} \frac{f\left( \xi \right)}{\left( \xi-z \right)^{n+1}}\:d\xi \right),
      \end{align*}
      and since the series converges uniformly, we have
      \begin{align*}
        \frac{f^{(n)}(z)}{n!} &= \frac{1}{2\pi i} \oint_{S\left( z_0,r_1 \right)}^{} \frac{f\left( \xi \right)}{\left( \xi-z \right)^{n+1}}\:d\xi.
      \end{align*}
      Since $r$ was arbitrary, this holds for any $ 0 < r_1 < R $, whence
      \begin{align*}
        f\left( z \right) &= \sum_{n=0}^{\infty} a_n\left( z-z_0 \right)^{n}
      \end{align*}
      holds for all $z\in U\left( z_0,R \right)$.
  \end{enumerate}
\end{proof}
\begin{corollary}
  Let $U\subseteq \C$ be open, let $z_0\in U$, and $r > 0$ with $B\left( z_0,r \right)\subseteq U$. The following hold:
  \begin{enumerate}[(i)]
    \item for all $z\in U\left( z_0,r \right)$,
      \begin{align*}
        \frac{f^{(n)}\left(z\right)}{n!} &= \frac{1}{2\pi i} \int_{S\left( z_0,r \right)}^{} \frac{f\left( \xi \right)}{\left( \xi-z \right)^{n+1}}\:d\xi;
      \end{align*}
    \item for all $n  > 0$,
      \begin{align*}
        \left\vert f^{(n)}\left( z_0 \right) \right\vert &\leq \frac{n!}{r^{n}} \sup_{\zeta\in S\left( z_0,r \right)} \left\vert f\left(\zeta\right) \right\vert.
      \end{align*}
      This particular result is known as the \textit{Cauchy Estimate}.
  \end{enumerate}
\end{corollary}
\begin{theorem}[Liouville's Theorem]
  If $f\colon \C\rightarrow \C$ is holomorphic and bounded in modulus, then $f$ is constant.
\end{theorem}
Liouville's Theorem follows from applying Cauchy's estimate to $f$ and using the fact that $f$ is bounded to find that all higher derivatives of $f$ vanish.
\begin{theorem}[Fundamental Theorem of Algebra]
  If $p(z) = a_nz^{n} + \cdots + a_1 z + a_0$ has $n\geq 1$ and $a_n\neq 0$, then there is at least one $z_0$ such that $p\left( z_0 \right) = 0$.
\end{theorem}
\begin{proof}
  Suppose $p(z)$ were never zero. It would follow then that $\frac{1}{p(z)}$ is also an entire function.\newline
  
  Since $\lim_{|z|\rightarrow\infty}\left\vert p(z) \right\vert = \infty$, it follows that $\lim_{|z|\rightarrow\infty} \frac{1}{\left\vert p(z) \right\vert} = 0$, whence $ \left\vert \frac{1}{p(z)} \right\vert $ is an entire function that is bounded (as all functions that vanish at infinity are bounded). This means that $ \frac{1}{p(z)} $ is constant, so $p(z)$ is constant.
\end{proof}
\begin{corollary}
  Let $f\colon \C\rightarrow \C$ be a nonconstant entire function. Then, $f\left(\C\right)$ is dense in $\C$.
\end{corollary}
\begin{proof}
  Suppose there were $w\in\C$ and $r > 0$ such that $U\left( w,r \right)\cap f\left( \C \right) = \emptyset$. Then, $\left\vert f(z)-w \right\vert \geq r$ for all $z\in \C$, meaning that
  \begin{align*}
    g(z) &= \frac{1}{f(z)-w}
  \end{align*}
  is bounded and entire (the entirety following from the fact that $f(z)-w$ is nonvanishing).
\end{proof}
\subsection{Cycles, Winding Numbers, and Homology}%
Now, we may generalize some of these results related to Cauchy's Integral Formula.
\begin{proposition}
  Let $\gamma\colon [a,b]\rightarrow \C$ be a piecewise $C^{1}$ loop. For all $z\in \C\setminus \img\left( \gamma \right)$, we have
  \begin{align*}
    \frac{1}{2\pi i} \oint_{\gamma}^{} \frac{1}{\xi-z}\:d\xi &\in \Z.
  \end{align*}
\end{proposition}
\begin{proof}
  Let $\phi\colon [a,b]\rightarrow \C$ be defined by
  \begin{align*}
    \phi(t) &= \int_{a}^{t} \frac{\gamma'(s)}{\gamma(s)-z}\:ds.
  \end{align*}
  Then, we observe
  \begin{align*}
    \phi(b) &= \oint_{\gamma}^{} \frac{1}{\xi - z}\:d\xi.
  \end{align*}
  Then, define $\psi\colon [a,b]\rightarrow \C$ by
  \begin{align*}
    \psi(t) &= \frac{e^{\phi(t)}}{\gamma(t)-z}.
  \end{align*}
  By the fundamental theorem of calculus, we have
  \begin{align*}
    \phi'(t) &= \frac{\gamma'(t)}{\gamma(t)-z}\\
  \psi'(t) &= \frac{\phi'(t)e^{\phi(t)}}{\gamma(t)-z} - \frac{e^{\phi'(t)}\gamma'(t)}{\left( \gamma(t)-z \right)^2}\\
           &= 0,
  \end{align*}
  whence $\psi(t)$ is constant, and $\psi(t) = \psi(a)$, so
  \begin{align*}
    \psi(a) &= \frac{1}{\gamma(a) - z}.
  \end{align*}
  In particular, $\psi(b) = \psi(a)$, so
  \begin{align*}
    e^{\phi(b)} &= \psi(b)\left( \gamma(b)-z \right)\\
                &= \psi(a)\left( \gamma(a)-z \right)\\
                &= 1,
  \end{align*}
  so $\phi(b) = 2\pi i k$ for some $k\in \Z$.
\end{proof}
\begin{definition}
  Let $\gamma\colon [a,b]\rightarrow \C$ be a piecewise $C^{1}$ loop. For all $z\in \C\setminus\img\left( \gamma \right)$, define
  \begin{align*}
    n\left( \gamma;z \right) &= \frac{1}{2\pi i} \oint_{\gamma}^{} \frac{1}{\xi - z}\:d\xi
  \end{align*}
   to be the \textit{winding number} of $\gamma$ about $z$.
\end{definition}
\begin{definition}
  A piecewise $C^{1}$ \textit{cycle} is a formal sum
  \begin{align*}
    \Gamma &= \gamma_1 + \cdots + \gamma_n,
  \end{align*}
  where the $\gamma_j\colon \left[ a_j,b_j \right]\rightarrow \C$ are piecewise $C^{1}$ loops. The \textit{length} of $\Gamma$ is the sum of the lengths of the respective $\gamma_j$.\newline

  Given a piecewise $C^{1}$ cycle $\Gamma$, define
  \begin{align*}
    \oint_{\Gamma}^{} f(z)\:dz &= \sum_{j=1}^{n} \oint_{\gamma_j}^{} f(z)\:dz,
  \end{align*}
  and
  \begin{align*}
    n\left( \Gamma;z \right) &= \sum_{j=1}^{n}n\left( \gamma_j;z \right).
  \end{align*}
\end{definition}
\begin{proposition}
  The following hold for the winding number $n\left( \gamma;z \right)$:
  \begin{enumerate}[(i)]
    \item the function $n\left( \Gamma;\cdot \right)\colon \C\setminus \img\left( \gamma \right)\rightarrow \Z$ is continuous;
    \item $n\left( \Gamma;z \right)$ is constant on each connected component of $\C\setminus \img\left( \Gamma \right)$;
    \item there exists a unique unbounded connected component with $n\left( \Gamma;z \right) = 0$ for all $z$ in this unbounded connected component.
  \end{enumerate}
\end{proposition}
\begin{proof}\hfill
  \begin{enumerate}[(i)]
    \item Since $ \img\left( \Gamma \right) $ is compact, any $z\notin \img\left( \Gamma \right)$ admits a strictly positive
      \begin{align*}
        \dist_{\img\left( \Gamma \right)}\left( z \right) &= \inf_{w\in\img\left( \Gamma \right)} \left\vert w-z \right\vert.
      \end{align*}
      Let $w\in\C$ be such that
      \begin{align*}
        \left\vert w-z \right\vert &< \frac{1}{2}\dist_{\img\left( \Gamma \right)}\left( z \right),
      \end{align*}
      so that $w\in \C\setminus \img\left( \Gamma \right)$. Observe then that
      \begin{align*}
        \left\vert n\left( \Gamma;z \right)-n\left( \Gamma;w \right) \right\vert &= \left\vert \frac{1}{2\pi i} \oint_{\Gamma}^{} \frac{1}{\xi-z} - \frac{1}{\xi-w}\:d\xi \right\vert\\
                                                                                 &\leq \frac{1}{2\pi} \sum_{j=1}^{n} \oint_{\gamma_j}^{} \left\vert \frac{1}{\xi-z} - \frac{1}{\xi-w} \right\vert \:\left\vert d\xi \right\vert\\
                                                                                 &= \frac{1}{2\pi} \sum_{j=1}^{n} \oint_{\gamma_j}^{} \left\vert \frac{z-w}{\left( \xi-z \right)\left( \xi-w \right)} \right\vert\:\left\vert d\xi \right\vert\\
                                                                                 &\leq \frac{1}{2\pi} \left( \frac{2}{\dist_{\img\left( \Gamma \right)}(z)} \right)^2 \ell\left( \Gamma \right)\left\vert z-w \right\vert,
      \end{align*}
      whence $\left\vert n\left( \Gamma;z \right)-n\left( \Gamma;w \right) \right\vert$ is sufficiently small whenever $\left\vert z-w \right\vert$ is sufficiently small.
    \item If $C$ is a connected component of $\C\setminus\img\left( \Gamma \right)$, and $n\left( \Gamma;\cdot \right)\colon C\rightarrow \Z$ is continuous, then since $\Z$ is discrete, $n\left( \Gamma;\cdot \right)$ is constant on $C$.
    \item For uniqueness, if there are unbounded connected components $C_1$ and $C_2$ of $\C\setminus \img\left( \Gamma \right)$, then there exists $M > \sup_{z\in\img\left( \Gamma \right)}\left\vert z \right\vert$ and $w_1\in C_1,w_2\in C_2$ such that $\left\vert w_1 \right\vert > 2M$ and $\left\vert w_2 \right\vert > 2M$. Since $\C\setminus U\left( 0,2M \right)$ is path connected, there exists $\gamma\colon [0,1]\rightarrow \C$ with $\left\vert \gamma(t) \right\vert \geq 2M$ and $\gamma(0) = w_1$, $\gamma(1) = w_2$. Therefore, $w_1$ and $w_2$ are in the same connected component.\newline

      Existence then follows from $\img\left( \Gamma \right)$ being compact.\newline

      Finally, let $\left( z_n \right)_n\subseteq C$, where $C$ is the unbounded connected component, be such that $\lim_{n\rightarrow\infty} \left\vert z_n \right\vert = \infty$. For $M > \sup_{z\in\img\left( \gamma \right)}\left\vert z \right\vert$, there exists $m\in \N$ such that $\left\vert z_m \right\vert > M$. Then, we have
      \begin{align*}
        \left\vert n\left( \Gamma;z_m \right) \right\vert &= \left\vert \frac{1}{2\pi i} \oint_{\Gamma}^{} \frac{1}{\xi-z}\:d\xi \right\vert\\
                                                          &\leq \frac{1}{2\pi} \sum_{j=1}^{k} \oint_{\gamma_j}^{} \frac{1}{\left\vert \xi-z \right\vert}\:\left\vert d\xi \right\vert\\
                                                          &\leq \frac{1}{2\pi} \sum_{j=1}^{k} \oint_{\gamma_j}^{} \frac{1}{\left\vert z_m \right\vert - M}\:\left\vert d\xi \right\vert\\
                                                          &= \frac{\ell\left( \Gamma \right)}{2\pi \left( \left\vert z_m \right\vert - M \right)},
      \end{align*}
      whence $\lim_{m\rightarrow\infty} n\left( \Gamma;z_m \right) = 0$, meaning that there exists $N$ such that $\left\vert n\left( \Gamma;z_m \right) \right\vert < 1$ for all $m\geq N$, meaning $n\left( \Gamma;z_m \right) = 0$ for all sufficiently large $m$. Since $C$ is connected, it thus follows that $n\left( \Gamma;z \right) = 0$ for all $z\in C$.
  \end{enumerate}
\end{proof}
\begin{definition}
  Let $U\subseteq \C$ be open. A cycle $\Gamma$ is \textit{homologous to zero in $U$} if $\img\left( \Gamma \right)\subseteq U$ and for all $z\in \C\setminus U$, $n\left( \Gamma;z \right) = 0$.
\end{definition}
\begin{theorem}[Cauchy's Integral Formula, General Case]
  Let $\Gamma = \gamma_1 + \cdots + \gamma_k$ be a piecewise $C^1$ cycle homologous to zero in $U$, and $f\colon U\rightarrow \C$ holomorphic. Then, for all $z\in U\setminus \img\left( \Gamma \right)$,
  \begin{align*}
    n\left( \Gamma;z \right)f\left( z \right) &= \frac{1}{2\pi i} \oint_{\Gamma}^{} \frac{f\left( \xi \right)}{\xi - z}\:d\xi
  \end{align*}
\end{theorem}
\begin{theorem}[Cauchy's Integral Theorem]
  Let $U\subseteq \C$ be open, $f\colon U\rightarrow \C$ holomorphic, and $\Gamma$ homologous to zero in $U$. Then,
  \begin{align*}
    \oint_{\Gamma}^{} f(z)\:dz &= 0.
  \end{align*}
\end{theorem}
\begin{definition}
  A region $U\subseteq \C$ is called \textit{simply connected} if its complement in the extended complex plane is connected.
\end{definition}
\begin{theorem}
  If $U\subseteq \C$ is simply connected, then every loop in $U$ is homologous to zero.
\end{theorem}
\begin{proof}
  Extend the function $n\left( \gamma;\cdot \right)$ to the extended complex plane by defining $n\left( \gamma;\infty \right) = 0$. This extended function is continuous on $\hat{\C}\setminus U$, as $n\left( \gamma;\cdot \right)$ is zero on the unique unbounded connected component of $\C\setminus \img\left( \gamma \right)$. It follows that $n\left( \gamma;z \right)$ is equal to zero on $\hat{\C}\setminus U$, whence $\gamma$ is homologous to zero in $U$.
\end{proof}
\begin{proposition}
  Let $U\subseteq \C$ be a region, $f\colon U\rightarrow \C$ holomorphic. The following are equivalent:
  \begin{enumerate}[(i)]
    \item there exists a holomorphic function $F\colon U\rightarrow \C$ such that $F'(z) = f(z)$;
    \item for every piecewise $C^{1}$ loop $\gamma$ with $\img\left( \gamma \right)\subseteq U$, we have
      \begin{align*}
        \oint_{\gamma}^{} f(z)\:dz &= 0.
      \end{align*}
  \end{enumerate}
\end{proposition}
\begin{proof}
  The direction (i) $\Rightarrow$ (ii) follows immediately from the fundamental theorem of calculus. In the reverse direction, we define $F\colon U\rightarrow \C$ by
  \begin{align*}
    f(z) &= \int_{\sigma\left( z_0,z \right)}^{} f\left( \xi \right)\:d\xi,
  \end{align*}
  where $\sigma\left( z_0,z \right)\colon [0,1]\rightarrow U$ is a piecewise $C^{1}$ curve with $\sigma(0) = z_0$ and $\sigma(1) = z$. Such a curve always exists as $U$ is open and connected (hence path-connected). The integral is well-defined, since if $\gamma_1$ and $\gamma_2$ are any two such paths, then $\Gamma = \gamma_1\setminus \gamma_2$ is a piecewise $C^1$ loop. Additionally, $F$ is continuous.\newline

  Now, we evaluate the derivative of $F$. Let $z\in U$, $r > 0$ such that $U\left( z,r \right)\subseteq U$, and $h\in \C$ be such that $z + h\in U\left( z,r \right)$. Then,
  \begin{align*}
    \frac{F\left( z+h \right) - F\left( z \right)}{h} &= \frac{1}{h} \int_{\sigma\left( z_0,z_0 + h \right)}^{} f\left( \xi \right)\:d\xi - \frac{1}{h} \int_{\sigma\left( z_0,z \right)}^{} f\left( \xi \right)\:d\xi\\
                                                      &= \frac{1}{h} \int_{\sigma\left( z,z+h \right)}^{} f\left( \xi \right)\:d\xi.
  \end{align*}
  We may assume that $\sigma\left( z,z+h \right)$ is a straight line, so that
  \begin{align*}
    \int_{\sigma\left( z,z+h \right)}^{} f\left( \xi \right)\:d\xi &= hf(z),
  \end{align*}
  meaning that
  \begin{align*}
    \left\vert \frac{F\left( z+h \right)-F\left( z \right)}{h} - f(z) \right\vert &= \frac{1}{\left\vert h \right\vert} \left\vert \int_{\sigma\left( z,z+h \right)}^{} f(\xi)\:d\xi - f\left( z \right) \right\vert\\
                                                                                  &\leq \sup_{w\in \img\left( \sigma\left( z,z+h \right) \right)} \left\vert f(w)-f(z) \right\vert.
  \end{align*}
  Since $f$ is continuous, it follows that the right hand side goes to zero as $\left\vert h \right\vert$ becomes small. Thus, $F'$ is continuous, so $f$ is holomorphic.
\end{proof}
Observe that $\C\setminus \set{0}$ is not simply connected, meaning that, for instance, the function
\begin{align*}
  f(z) &= \frac{1}{z}
\end{align*}
does not have a holomorphic antiderivative defined on the entirety $\C\setminus \set{0}$, as
\begin{align*}
  \int_{S^1}^{} f(z)\:dz &= 2\pi i.
\end{align*}
Yet, if we restrict $f(z)$ to a simply connected subset of $\C$, there \textit{is} a holomorphic antiderivative. Choosing such a simply connected subset of $\C$ is known as choosing a \textit{branch} of the logarithm. In fact, there is more that we can say.
\begin{corollary}
  Let $U\subseteq \C$ be simply connected, and let $f\colon U\rightarrow \C\setminus \set{0}$ be a nonvanishing holomorphic function. For each fixed pair $z_0\in U$ and $w_0\in \C$ for which $e^{w_0} = f\left( z_0 \right)$, there exists a unique holomorphic function $g\colon U\rightarrow \C$ for which $g\left(z_0\right) = w_0$ and $e^{g(z)} = f(z)$.\newline

  We call $g$ the logarithm of $f$, written $g(z) = \log\left( f(z) \right)$.
\end{corollary}
\begin{proof}
  Since $f$ is nonvanishing and $U$ is simply connected, it follows that $\frac{f'}{f}$ is holomorphic on $U$, meaning there is $ \widetilde{g}\colon U\rightarrow \C $ such that $\widetilde{g}'(z) = \frac{f'(z)}{f(z)}$. Thus, there is some constant $K$ such that
  \begin{align*}
    f(z) &= Ke^{\widetilde{g}(z)}.
  \end{align*}
  Define
  \begin{align*}
    g(z) &= \log(K) + \widetilde{g}(z).
  \end{align*}
\end{proof}
\begin{theorem}[Morera's Theorem]
  Let $U\subseteq \C$ be open, $f\colon U\rightarrow \C$ continuous. Suppose
  \begin{align*}
    \oint_{T}^{} f(z)\:dz &= 0
  \end{align*}
  for all triangles $T\subseteq U$ homologous to zero. Then, $f$ is holomorphic.
\end{theorem}
\begin{proof}
  Since $U$ is open, if $z_0\in U$, there is $r$ such that $U\left( z_0,r \right)\subseteq U$. Define $F\colon U\left( z_0,r \right)\rightarrow \C$ by
  \begin{align*}
    F(z) &= \int_{\sigma\left( z_0,z \right)}^{} f\left( \xi \right)\:d\xi,
  \end{align*}
  where $\sigma$ is the straight line from $z_0$ to $z$. For $0 < \left\vert h \right\vert < r-\left\vert z-z_0 \right\vert$, we construct the straight lines $\sigma\left( z,z+h \right)$ and $\sigma\left( z_0,z+h \right)$, such that
  \begin{align*}
    T &= \sigma\left( z_0,z \right) + \sigma\left( z,z+h \right) - \sigma\left( z_0,z+h \right),
  \end{align*}
  and
  \begin{align*}
    \oint_{T}^{} f(z)\:dz &= 0\\
                          &= \int_{\sigma\left( z_0,z \right)}^{} f\left( \xi \right)\:d\xi + \int_{\sigma\left( z,z+h \right)}^{} f\left( \xi \right)\:d\xi - \int_{\sigma\left( z_0,z+h \right)}^{} f\left( \xi \right)\:d\xi\\
                          &= F(z) - F\left( z+h \right) + \int_{\sigma\left( z,z+h \right)}^{} f\left( \xi \right)\:d\xi,
                          \intertext{meaning}
    F\left( z+h \right) - F\left( z \right) &= \int_{\sigma\left( z,z+h \right)}^{} f\left( \xi \right)\:d\xi\\
    \frac{F\left( z+h \right) - F\left( z \right)}{h} &= \frac{1}{h} \int_{\sigma\left( z,z+h \right)}^{} f\left( \xi \right)\:d\xi\\
    \left\vert \frac{F\left( z+h \right) - F\left( z \right)}{h} - f(z) \right\vert &= \left\vert \frac{1}{h}\int_{\sigma\left( z,z+h \right)}^{} \left( f(\xi) - f(z) \right)\:d\xi \right\vert\\
                                                                                    &\leq \frac{1}{\left\vert h \right\vert}\left\vert h \right\vert\sup_{w\in \img\left( \sigma\left( z,z+h \right) \right)} \left\vert f(w) - f(z) \right\vert\\
                                                                                    &= \sup_{w\in \img\left( \sigma\left( z,z+h \right) \right)} \left\vert f(w)-f(z) \right\vert.
  \end{align*}
  Since $f$ is continuous, it follows that for sufficiently small $\left\vert h \right\vert$, the right-hand-side goes to zero, whence $F'(z) = f(z)$, meaning $F$ is holomorphic, so $F$ is analytic, meaning $f$ is analytic, so $f$ is holomorphic.
\end{proof}
\begin{corollary}
  Let $U\subseteq \C$ be open, $\gamma\colon \left[ a,b \right]\rightarrow U$ a piecewise $C^{1}$ curve, and $g\colon U\times \img\left( \gamma \right)\rightarrow \C$ continuous. Suppose that for each $w\in \img\left( \gamma \right)$, the function $g\left( \cdot,w \right)$ is holomorphic. Then,
  \begin{align*}
    f(z)\coloneq \int_{\gamma}^{} g\left( z,w \right)\:dw
  \end{align*}
  is holomorphic.
\end{corollary}
\begin{proof}
  Let $T$ be a triangle in $U$ homologous to zero. Then, by Fubini's Theorem,
  \begin{align*}
    \oint_{T}^{} f(z)\:dz &= \oint_{T}^{} \int_{\gamma}^{} g\left( z,w \right)\:dw\:dz\\
                          &= \int_{\gamma}^{} \oint_{T}^{} g\left( z,w \right)\:dz\:dw.
  \end{align*}
  The interior integral vanishes for every $w$ as $g\left( \cdot,w \right)$ is holomorphic. Thus, $f$ is holomorphic.
\end{proof}
\begin{definition}
  Let $U\subseteq \C$ be open, $\gamma_1,\gamma_2$ piecewise $C^{1}$ loops in $U$. We say $\gamma_1$ and $\gamma_2$ are homotopic in $U$ if there is a continuous function
  \begin{align*}
    H\colon [a,b]\times [0,1]\rightarrow U
  \end{align*}
  such that
  \begin{align*}
    H\left( s,0 \right) &= \gamma_1(s)\\
    H\left( s,1 \right) &= \gamma_2(s)\\
    H\left( a,t \right) &= H\left( b,t \right).
  \end{align*}
  For each $t$, $H\left( \cdot,t \right)$ is a continuous loop. We call $H$ a homotopy between $\gamma_0$ and $\gamma_1$.
\end{definition}
\begin{theorem}
  If $\gamma_0$ and $\gamma_1$ are homotopic in $U$, then $\Gamma = \gamma_1-\gamma_0$ is homologous to zero in $U$.
\end{theorem}
\begin{theorem}
  If $K\subseteq U$ is compact and $U$ is connected, then there is some cycle $\Gamma$ homologous to zero in $U$ such that $n\left( \Gamma;z \right) = 1$ for all $z\in K$.
\end{theorem}
\begin{corollary}
  Let $U$ be a region. The following are equivalent:
  \begin{enumerate}[(i)]
    \item $U$ is simply connected;
    \item for every nonvanishing holomorphic function $f\colon U\rightarrow \C\setminus \set{0}$, there is a holomorphic function $g\colon U\rightarrow \C$ such that $f(z) = e^{g(z)}$;
    \item for all cycles $\Gamma$ with $\img\left( \Gamma \right)\subseteq U$, $\Gamma$ is homologous to zero in $U$.
  \end{enumerate}
\end{corollary}
\subsection{Maximum Modulus Principle}%
\begin{theorem}[Mean Value Property]
  Let $U\subseteq \C$ be open, $f\colon U\rightarrow \C$ holomorphic, with $z_0\in U$ and $r > 0$ such that $B\left( z_0,r \right)\subseteq U$. Then,
  \begin{align*}
    f\left( z_0 \right) &= \frac{1}{2\pi} \int_{0}^{2\pi} f\left( z_0 + re^{i\theta} \right)\:d\theta.
  \end{align*}
\end{theorem}
\begin{proof}
  By the Cauchy Integral Formula, we have
  \begin{align*}
    f\left( z_0 \right) &= \frac{1}{2\pi i} \int_{S\left( z_0,r \right)}^{} \frac{f\left( \xi \right)}{\xi - z}\:d\xi.
  \end{align*}
  Parametrizing $\gamma\left( \theta \right) = z_0 + re^{i\theta}$, we get
  \begin{align*}
    f\left( z_0 \right) &= \frac{1}{2\pi i} \int_{0}^{2\pi} \frac{f\left( z_0 + re^{i\theta} \right)}{re^{i\theta}}ire^{i\theta}\:d\theta\\
                        &= \frac{1}{2\pi} \int_{0}^{2\pi} f\left( z_0 + re^{i\theta} \right)\:d\theta.
  \end{align*}
\end{proof}
\begin{corollary}
  If $u\colon \R^{2}\supseteq U\rightarrow \R$ is harmonic, $\left( x_0,y_0 \right)\in U$, and $r > 0$ is such that $B\left( \left( x_0,y_0 \right),r \right)\subseteq U$, then
  \begin{align*}
    u\left( x_0,y_0 \right) &= \frac{1}{2\pi} \int_{0}^{2\pi} u\left( x_0 + r\cos\left( \theta \right),y_0 + r\sin\left( \theta \right) \right)\:d\theta.
  \end{align*}
\end{corollary}
\begin{proof}
  Take real parts of the mean value property for holomorphic $f = u + iv$.
\end{proof}
Observe then that the triangle inequality implies that
\begin{align*}
  \left\vert u\left( x_0,y_0 \right) \right\vert &\leq \frac{1}{2\pi} \int_{0}^{2\pi} \left\vert u\left( x_0 + r\cos\left( \theta \right),y_0 + r\sin\left( \theta \right) \right) \right\vert\:d\theta.
\end{align*}
Functions that satisfy this weaker criterion are known as \textit{subharmonic}. It is subharmonic functions for which the most general case of the \textit{maximum modulus principle} hold.
\begin{theorem}[Maximum Modulus Principle]
  Let $U\subseteq \R^{2}$ be open and connected, and let $u\colon U\rightarrow \R$ be subharmonic. Suppose there exists $\left( x_0,y_0 \right)\in U$ such that $u\left( x_0,y_0 \right)\geq u\left( x,y \right)$ for all $x,y\in U$. Then, $u$ is constant.
\end{theorem}
\begin{proof}
  Let $\lambda = u\left( x_0,y_0 \right)$, and let $E = \set{\left( x,y \right) | u\left( x,y \right) = \lambda} = u^{-1}\left( \set{\lambda} \right)$. We see immediately that $E$ is closed; we claim that $E$ is also open.\newline

  Fix $\left( x_1,y_1 \right)\in E$. Then, $u\left( x_1,y_1 \right) = \lambda$. Take $r > 0$ such that $U\left( \left( x_1,y_1 \right),r \right)\subseteq U$. Then, for all $0 < s < r$, we have $S\left( \left( x_1,y_1 \right),s \right)\subseteq U$, meaning that
  \begin{align*}
    \lambda &= u\left( x_1,y_1 \right) \\
            &\leq \frac{1}{2\pi} \int_{0}^{2\pi} u\left( x_1 + s\cos\left( \theta \right),y_1+s\sin\left( \theta \right) \right)\:d\theta\\
            &\leq \lambda,
  \end{align*}
  with the latter inequality following from the fact that $\lambda$ is a local maximum. Therefore, $u\left( x_1 + s\cos\left( \theta \right),y_1 + s\sin\left( \theta \right) \right) = \lambda$ for all $0 < s < r$, whence $U\left( \left( x_1,y_1 \right),r \right)\subseteq E$. Thus, $E$ is open, so since $U$ is connected, it follows that $E$ is all of $U$, meaning $u$ is constant.
\end{proof}
\begin{corollary}
  If $U\subseteq \R^2$ is bounded and $u\colon \overline{U}\rightarrow \R$ is continuous with $u|_{U}$ subharmonic, then there exists $\left( x_0,y_0 \right)\in \partial U$ such that $u\left( x_0,y_0 \right) = \sup_{\left( x,y \right)\in U} u\left( x,y \right)$.
\end{corollary}
\begin{corollary}
  If $U\subseteq \C$ is open and connected, with $f\colon U\rightarrow \C$ holomorphic, then if $\left\vert f \right\vert\colon U\rightarrow \R$ has a local maximum at $z_0\in U$, then $f$ is constant.
\end{corollary}
\begin{proof}
  Let $r > 0$ be such that $U\left( z_0,r \right)\subseteq U$. Then, restricting $\left\vert f \right\vert$ to $U\left( z_0,r \right)$, we see that $\left\vert f \right\vert$ restricted to $U\left( z_0,r \right)$ is subharmonic viewed as a function on $U\left( z_0,r \right)$, hence $\left\vert f \right\vert$ is constant on $U\left( z_0,r \right)$.\newline

  Now, by the mean value property and triangle inequality, it follows that for all $0 < s < r$, we have
  \begin{align*}
    \left\vert f\left( z_0 \right) \right\vert &\leq \frac{1}{2\pi} \int_{0}^{2\pi} \left\vert f\left( z_0 + se^{i\theta} \right) \right\vert\:d\theta\\
                                               &= \left\vert f\left( z_0 \right) \right\vert,
  \end{align*}
  meaning that these are equalities. In particular, there exists some $\theta_s$ such that $e^{i\theta_s} f\left( z_0 + se^{i\theta} \right) \geq 0$, meaning that for this value of $s$, we have
  \begin{align*}
    \left\vert f\left( z_0 \right) \right\vert &= e^{i\theta_s} \int_{0}^{2\pi} f\left( z_0 + se^{i\theta} \right)\:d\theta\\
                                               &= e^{i\theta_s} f\left( z_0 \right),
  \end{align*}
  with the latter equality following from the mean value property. Since this holds for any $s$, it follows that $\theta_s$ is independent of $s$, meaning that $f(z)e^{i\theta_s} \geq 0$ for all $z\in U\left( z_0,r \right)$, meaning that $\im\left( e^{i\theta_s}f\left( z \right) \right) = 0$ on $U\left( z_0,r \right)$, whence $f(z)e^{i\theta_s}$ is constant, meaning $f$ is constant on $U\left( z_0,r \right)$.\newline

  Finally, by the identity theorem, it follows that $f$ is constant on $U$.
\end{proof}
\begin{definition}
  Let $U\subseteq \R^{2}$ be an open set. We say a sequence $U\supseteq \left( \left( x_n,y_n \right) \right)_n\rightarrow \partial U$ if, for every compact $K\subseteq U$, the set $\set{n\in \N | \left( x_n,y_n \right)\in K}$ is finite.
\end{definition}
\begin{definition}
  Let $U\subseteq \R^{2}$ be an open set. Given a function $u\colon U\rightarrow \R$, define
  \begin{align*}
    \limsup_{\left( x,y \right)\rightarrow \partial U} u\left( x,y \right) &\coloneq \inf_{\substack{K\subseteq U\\K\text{ compact}}} \sup_{\left( x,y \right)\in U\setminus K} u\left( x,y \right).
  \end{align*}
\end{definition}
These definitions allow us to extend the maximum modulus principle for subharmonic functions even further.
\begin{theorem}
  Let $U\subseteq \C$ be a region, $u\colon U\rightarrow \R$ a nonconstant subharmonic function. If $\left( \left( x_n,y_n \right) \right)_n\subseteq U$ is such that $u\left( x_n,y_n \right)\rightarrow \sup_{x,y\in U}u\left( x,y \right)$, then $\left( \left( x_n,y_n \right) \right)_n\rightarrow \partial U$. Moreover, $\limsup_{\left( x,y \right)\rightarrow \partial U}u\left( x,y \right) = \sup_{\left( x,y \right)\in U} u\left( x,y \right)$.
\end{theorem}
\begin{proof}
  Suppose toward contradiction that $\left( \left( x_n,y_n \right) \right)_n\nrightarrow \partial U$, so there exists a compact subset $K\subseteq U$ and a subset $\left( \left( x_{n_k},y_{n_k} \right) \right)_k$ wholly contained in $K$. Since $K$ is compact, there is a subsequence of $\left( \left( x_{n_k},y_{n_k} \right) \right)_k$ converging to $\left( x_0,y_0 \right)\in U$. Therefore, $u\left( x_0,y_0 \right) = \sup_{\left( x,y \right)\in U}u\left( x,y \right)$, so $u$ is constant by the maximum modulus principle, which is a contradiction.\newline

  Finally, $\limsup_{\left( x,y \right)\rightarrow \partial U}u\left( x,y \right)\leq \sup_{\left( x,y \right)\in U}u\left( x,y \right)$, while if $\left( \left( x_n,y_n \right) \right)_n\rightarrow \partial U$ is such that $u\left( x_n,y_n \right)$ converges to $\sup_{\left( x,y \right)\in U}u\left( x,y \right)$, then $\sup_{\left( x,y \right)\in U}u\left( x,y \right) = \lim_{n\rightarrow\infty}u\left( x_n,y_n \right) \leq \limsup_{\left( x,y \right)\rightarrow \partial U}u\left( x,y \right)$.
\end{proof}
\begin{theorem}[Open Mapping Principle]
  Let $U\subseteq \C$ be a region, and let $f\colon U\rightarrow \C$ be a nonconstant holomorphic function. Then, $f\left( V \right)\subseteq \C$ is open.
\end{theorem}
\begin{proof}
  Let $z_0\in U$ and $r > 0$ be such that $ B\left( z_0,r \right)\subseteq U $. We will show that there exists $R$ such that $U\left( f\left( z_0 \right),R \right)\subseteq f\left( U\left( z_0,r \right) \right)\subseteq U$, whence $f(U)$ is open.\newline

  Since $U$ is a region and $f$ is nonconstant, the zeros of $g(z)\coloneq f(z) - f\left( z_0 \right)$ are isolated, so there exists some $0 < s < r$ such that
  \begin{align*}
    \delta &= \inf_{\left\vert z-z_0 \right\vert = s} \left\vert f(z)-f\left( z_0 \right) \right\vert
  \end{align*}
  is strictly greater than zero. We claim that $U\left( f\left( z_0 \right),\delta/2 \right) \subseteq f\left( U\left( z_0,r \right) \right)$. Suppose this were not the case, meaning there would be some $\xi\in U\left( f\left( z_0 \right),\delta/2 \right)\setminus f\left( U\left( z_0,r \right) \right)$, and define $ h\colon B\left( z_0,s \right)\rightarrow \C $ by
  \begin{align*}
    h(z) &= \frac{1}{f\left( z \right) - \xi}.
  \end{align*}
  Since $\xi\notin f\left( U\left( z_0,r \right) \right)$, this is holomorphic, while $\xi\in U\left( f\left( z_0 \right),\delta/2 \right)$ implies
  \begin{align*}
    \sup_{\left\vert z-z_0 \right\vert = s} \left\vert h(z) \right\vert &= \sup_{\left\vert z-z_0 \right\vert = s} \frac{1}{\left\vert f\left( z \right) - \xi \right\vert}\\
                                                                        &\leq \sup_{\left\vert z-z_0 \right\vert = s} \frac{1}{\left\vert f\left( z \right)-f\left( z_0 \right) \right\vert - \left\vert f\left( z_0 \right) - \xi \right\vert}\\
                                                                        &\leq \frac{1}{\delta - \delta/2}\\
                                                                        &= \frac{2}{\delta}.
  \end{align*}
  Yet,
  \begin{align*}
    \left\vert h\left( z_0 \right) \right\vert &= \frac{1}{\left\vert f\left( z_0 \right) - \xi \right\vert}\\
                                               &> \frac{2}{\delta},
  \end{align*}
  contradicting the maximum modulus principle. Thus, $U\left( f\left( z_0 \right),\delta/2 \right)\subseteq f\left( U\left( z_0,r \right) \right)$.
\end{proof}
In the proof of the Hadamard Three-Lines Theorem, we used the function $h_{\ve}(z) = \frac{1}{ 1 + \ve (z-a) }$ for this purpose.
\subsection{Classification of Singularities}%
The classification of singularities seeks to answer two fundamental questions: if $U\subseteq \C$ is open, $z_0\in U$, and $f\colon U\setminus \set{z_0}\rightarrow \C$ is holomorphic, 
\begin{itemize}
  \item does $f$ have a holomorphic extension to $U$ including $z_0$;
  \item and what else can we say about the behavior of $f$ at $z_0$?
\end{itemize}
\begin{definition}
  Let $U\subseteq \C$ be open, $z_0\in U$, $f\colon U\setminus \set{z_0}\rightarrow \C$ holomorphic.
  \begin{itemize}
    \item If there exists a holomorphic $g\colon U\rightarrow \C$ with $g = f$ on $U\setminus \set{z_0}$, then we say $z_0$ is a \textit{removable singularity}.
    \item If $\lim_{z\rightarrow z_0} \left\vert f(z) \right\vert = \infty$, then we say $f$ has a \textit{pole} at $z_0$.
    \item Else, we say $f$ has an \textit{essential singularity} at $z_0$.
  \end{itemize}
\end{definition}
\begin{theorem}[Riemann's Theorem on Removable Singularities]
  Let $U\subseteq \C$ be open, $z_0\in U$, and $f\colon U\setminus \set{z_0}\rightarrow \C$ holomorphic. Then, $z_0$ is a removable singularity if and only if $\lim_{z\rightarrow z_0}f(z) = 0$.
\end{theorem}
\begin{proof}
  If $z_0$ is removable, then $g(z)$ is a holomorphic function with $g(z) = f(z)$ on $U\setminus \set{z_0}$, and since $g$ is continuous, it follows that $\lim_{z\rightarrow z_0}g(z) = g\left(z_0\right)$, whence $\lim_{z\rightarrow z_0}\left( z-z_0 \right)g(z) = \lim_{z\rightarrow z_0}\left( z-z_0 \right)f(z) = 0$.\newline

  Now, if $\lim_{z\rightarrow z_0}\left( z-z_0 \right)f(z) = 0$, then there is $r$ such that $B\left( z_0,r \right)\subseteq U$, and since $f$ is locally bounded around $z_0$, it follows that
  \begin{align*}
    f(z) &= \frac{1}{2\pi i} \int_{S\left( z_0,r \right)}^{} \frac{f\left( \zeta \right)}{\zeta - z}\:d\zeta
  \end{align*}
  holds for all $z\in \dot{U}\left( z_0,r \right)$. Yet, the formula extends to $z_0$ as it is bounded, whence we may define the holomorphic extension for $f$ by
  \begin{align*}
    g\left( z \right) &= \begin{cases}
      f(z) & z\neq z_0\\
      \frac{1}{2\pi i} \int_{S\left( z_0,r \right)}^{} \frac{g\left( \zeta \right)}{\zeta - z}\:d\zeta & z= z_0
    \end{cases}.
  \end{align*}
\end{proof}
\begin{proposition}[Existence of Laurent Series]
  Suppose $f\colon A\left( z_0,r,R \right)\rightarrow \C$ is holomorphic, with $0 \leq r < R$. Then, there exist holomorphic functions
  \begin{align*}
    g_1\colon U\left( z_0,R \right)\rightarrow \C\\
    g_2\colon \C\setminus B\left( z_0,r \right)\rightarrow \C
  \end{align*}
  such that $f = g_1 + g_2$ on $A\left( z_0,r,R \right)$. Moreover, there exists $\left( a_n \right)_{n\in\Z}\subseteq \C$ such that
  \begin{align*}
    f(z) &= \sum_{n=-\infty}^{\infty} a_n\left( z-z_0 \right)^{n}
  \end{align*}
  for all $z$, and the series converges uniformly on $A\left( z_0,\rho,s \right)$ with $r < \rho < s < R$.
\end{proposition}
\begin{proof}
  Fix $z\in A\left( z_0,r,R \right)$. Then, for $r < \rho_1,\rho_2 < \left\vert z-z_0 \right\vert$, the cycle
  \begin{align*}
    \Gamma_1 &= S\left( z_0,\rho_1 \right) - S\left( z_0,\rho_2 \right)
  \end{align*}
  is homologous to zero in $A\left( z_0,r,\left\vert z-z_0 \right\vert \right)$. By Cauchy's Integral Theorem, it then follows that
  \begin{align*}
    \oint_{S\left( z_0,\rho_1 \right)}^{} \frac{f\left( \xi \right)}{\xi-z}\:d\xi &= \oint_{S\left( z_0,\rho_2 \right)}^{} \frac{f\left( \xi \right)}{\xi-z}\:d\xi.
  \end{align*}
  Similarly, for $\left\vert z-z_0 \right\vert < s_1,s_2 < R$, we have
  \begin{align*}
    \oint_{S\left( z_0,s_1 \right)}^{} \frac{f\left( \xi \right)}{\xi-z}\:d\xi &= \oint_{S\left( z_0,s_2 \right)}^{} \frac{f\left( \xi \right)}{\xi-z}\:d\xi.
  \end{align*}
  Define $g_1\colon U\left( z_0,R \right)\rightarrow \C$ by
  \begin{align*}
    g_1(z) &= \frac{1}{2\pi i} \oint_{S\left( z_0,s \right)}^{} \frac{f\left( \xi \right)}{\xi - z}\:d\xi,
  \end{align*}
  where $ \left\vert z-z_0 \right\vert < s < R $. This function is holomorphic by Morera's Theorem. Similarly, we may define $g\colon \C\setminus B\left( z_0,r \right)\rightarrow \C$ by
  \begin{align*}
    g_2\left( z \right) &= -\frac{1}{2\pi i} \oint_{S\left( z_0,\rho \right)}^{} \frac{f\left( \xi \right)}{\xi-z}\:d\xi,
  \end{align*}
  where $r < \rho < \left\vert z-z_0 \right\vert$. We claim that $f = g_1 + g_2$ on $A\left( z_0,r,R \right)$.\newline

  For $z\in A\left( z_0,r,R \right)$, we may find, for any $r < \rho < \left\vert z-z_0 \right\vert < s < R$, we let
  \begin{align*}
    \Gamma &= S\left( z_0,s \right) - S\left( z_0,\rho \right),
  \end{align*}
  homologous to zero in $A\left( z_0,r,R \right)$, whence
  \begin{align*}
    f\left( z \right) &= \frac{1}{2\pi i} \left( \oint_{S\left( z_0,s \right)}^{} \frac{f\left( \xi \right)}{\xi-z}\:d\xi - \int_{S\left( z_0,\rho \right)}^{} \frac{f\left( \xi \right)}{\xi - z}\:d\xi \right)\\
                      &= g_1\left( z \right) + g_2\left( z \right).
  \end{align*}
\end{proof}
\begin{theorem}
  Let $U\subseteq \C$, $f\colon U\setminus \set{z_0}\rightarrow \C$ be holomorphic with Laurent series
  \begin{align*}
    f(z) &= \sum_{n=-\infty}^{\infty}a_n\left( z-z_0 \right)^{n}
  \end{align*}
  on $\dot{U}\left( z_0,R \right)$ for some $R$ with $U\left( z_0,R \right)\subseteq U$. Then,
  \begin{enumerate}[(i)]
    \item $f$ has a removable singularity at $z_0$ if and only if $a_n = 0$ for all $n < 0$;
    \item $f$ has a pole at $z_0$ if and only if 
      \begin{align*}
        1 \leq \left\vert \set{n < 0 | a_n\neq 0} \right\vert <\infty.
      \end{align*}
    \item $f$ has an essential singularity at $z_0$ if and only if
      \begin{align*}
        \left\vert \set{n < 0 | a_n\neq 0} \right\vert &= \infty.
      \end{align*}
  \end{enumerate}
\end{theorem}
\begin{proof}\hfill
  \begin{enumerate}[(i)]
    \item If $a_n = 0$ for all $n < 0$, then $\lim_{z\rightarrow z_0} \left( z-z_0 \right)f(z) = 0$, so $f$ has a removable singularity at $z_0$.\newline

      Conversely, if $f$ has a removable singularity at $z_0$, then for $n < 0$, we have
      \begin{align*}
        a_n &= \frac{1}{2\pi i} \oint_{S\left( z_0,\rho \right)}^{} \frac{f\left( \xi \right)}{\left( \xi - z_0 \right)^{n+1}}\:d\xi
      \end{align*}
      for any $0 < \rho < R$. Since $\lim_{z\rightarrow z_0}\left( z-z_0 \right)f(z) = 0$, then for any $\ve > 0$, there is sufficiently small $\rho$ such that
      \begin{align*}
        \left\vert a_n \right\vert &= \left\vert \frac{1}{2\pi i} \oint_{S\left( z_0,\rho \right)}^{} \frac{f\left( \xi \right)}{\left( \xi-z_0 \right)^{n+1}}\:d\xi \right\vert\\
                                   &\leq \rho^{-1-n} \sup_{\left\vert \xi-z_0 \right\vert = \rho}\left\vert \left( \xi-z_0 \right)f(z) \right\vert\\
                                   &\leq \ve.
      \end{align*}
      Thus, $\left\vert a_n \right\vert = 0$ for all $n < 0$.
    \item If $a_n\neq 0$ for a nonempty finite collection of $n < 0$, we let $m$ be the largest number such that $a_{-m} < 0$, so that $f(z) = \sum_{n=-m}^{\infty}a_n\left( z-z_0 \right)^{n}$. It follows that $\lim_{z\rightarrow z_0}\left( z-z_0 \right)^{m}f(z) = a_{-m}\neq 0$. In particular, there is some small $\delta$ such that
      \begin{align*}
        \left\vert \left( z-z_0 \right)^{m}f(z) - a_{-m} \right\vert &< \frac{\left\vert a_{-m} \right\vert}{2},
      \end{align*}
      whence for a sufficiently small $\ve$,
      \begin{align*}
        \left\vert f(z) \right\vert &> \frac{\left\vert a_{-m} \right\vert-\left\vert \left( z-z_0 \right)^{m}f(z)-a_{-m} \right\vert}{\left\vert z-z_0 \right\vert^{m}}\\
                                    &> \frac{\left\vert a_{-m} \right\vert}{2\left\vert z-z_0 \right\vert^{m}}\\
                                    &> \frac{1}{\ve},
      \end{align*}
      so $f$ has a pole at $z_0$.\newline

      Conversely, if $f$ has a pole at $z_0$, there is some $r > 0$ such that $\left\vert f(z) \right\vert \geq 1$ whenever $z\in \dot{U}\left( z_0,r \right)$. Define $g\colon \dot{U}\left( z_0,r \right)\rightarrow \C$ by
      \begin{align*}
        g(z) &= \frac{1}{f(z)},
      \end{align*}
      whence $g$ is holomorphic and bounded on $\dot{U}\left( z_0,r \right)$. By the classification of singularities, $z_0$ is a removable singularity of $g$, so there exists a holomorphic function $h\colon U\left( z_0,r \right)\rightarrow \C$ that is equal to $ \frac{1}{f(z)} $ for $z\neq z_0$ and $\lim_{z\rightarrow z_0}\frac{1}{f(z)}$ if $z= z_0$. We write
      \begin{align*}
        h(z) &= \sum_{n=0}^{\infty} b_n\left( z-z_0 \right)^{n},
      \end{align*}
      where we must have $b_0 = 0$. Let $m$ be the smallest positive integer such that $h(z) = \left( z-z_0 \right)^{m}\widetilde{h}(z)$, where $\widetilde{h}(z) = \sum_{n=0}^{\infty}b_{n+m}\left( z-z_0 \right)^{n}$ and $b_m\neq 0$. The function $\widetilde{h}$ is holomorphic on $U\left( z_0,r \right)$ and nonzero on some $U\left( z_0,\rho \right)$ with $0 < \rho \leq r$, so that $ \widetilde{f}(z) = \frac{1}{\widetilde{h}(z)} $ is holomorphic on $U\left( z_0,\rho \right)$, so
      \begin{align*}
        f(z) &= \left( z-z_0 \right)^{-m}\widetilde{f}(z)
      \end{align*}
      on $\dot{U}\left( z_0,\rho \right)$. Writing
      \begin{align*}
        \widetilde{f}(z) &= \sum_{n=0}^{\infty}c_n\left( z-z_0 \right)^{n},
      \end{align*}
      we deduce that
      \begin{align*}
        f(z) &= \sum_{n=-m}^{\infty}c_{n+m}\left( z-z_0 \right)^{n}.
      \end{align*}
    \item Follows from (i) and (ii).
  \end{enumerate}
\end{proof}
\begin{definition}
  Let $U\subseteq \C$ be open.
  \begin{itemize}
    \item If $f\colon U\setminus \set{z_0}\rightarrow \C$ has a pole or a removable singularity at $z_0$, then the order of the pole or singularity is the smallest $m \geq 0$ such that
      \begin{align*}
        f(z) &= \left( z-z_0 \right)^{-m} g(z)
      \end{align*}
      with $g\colon U\rightarrow \C$ is holomorphic and has $g\left( z_0 \right) \neq 0$.
    \item If $f\colon U\rightarrow \C$ has a zero at $z_0\in U$, then the order of the zero at $z_0$ is the smallest $m \geq 0$ such that
      \begin{align*}
        f(z) &= \left( z-z_0 \right)^{m}g(z)
      \end{align*}
      where $g\colon U\rightarrow \C$ is holomorphic and has $g\left( z_0 \right) \neq 0$.
  \end{itemize}
  If $z_0\in U$ is either a pole or a zero (or a removable singularity that is a zero when $f$ is extended), then the \textit{order} of $f$ is the unique $m\in\Z$ such that
  \begin{align*}
    f(z)  &= \left( z-z_0 \right)^{m}g(z)
  \end{align*}
  with $g\left(z_0\right)\neq 0$.
\end{definition}
\begin{theorem}[Casorati--Weierstrass]
  Let $U\subseteq \C$ be an open set, and let $f\colon U\setminus \set{z_0}\rightarrow \C$ be a holomorphic function. If $z_0$ is an essential singularity of $f$, then for any $r > 0$ with $U\left( z_0,r \right)\subseteq U$, $f\left( \dot{U}\left( z_0,r \right) \right)\subseteq \C$ is dense.
\end{theorem}
\begin{proof}
  Suppose $f\left( \dot{U}\left( z_0,r \right) \right)$ is not dense in $\C$. Then, there exists $\ve > 0$ and $w\in \C$ such that $\left\vert f(z) - w \right\vert \geq \ve$ for all $z\in \dot{U}\left( z_0,r \right)$. We will show that $z_0$ is either removable or a pole.\newline

  Define
  \begin{align*}
    g\colon \dot{U}\left( z_0,r \right)\rightarrow \C
  \end{align*}
  by
  \begin{align*}
    g(z) &= \frac{1}{f(z) - w}.
  \end{align*}
  Observe that $g$ is definitionally bounded, so $z_0$ is a removable singularity of $g$. Thus, there is a holomorphic function $h\colon U\left( z_0,r \right)\rightarrow \C$ such that $h(z) = g(z)$ for $z\neq z_0$ and $h(z) = \lim_{z\rightarrow z_0}g(z)$. We write
  \begin{align*}
    h(z) &= \sum_{n=0}^{\infty}b_n\left( z-z_0 \right)^{n},
  \end{align*}
  and take
  \begin{align*}
    h(z) &= \left( z-z_0 \right)^{m} \widetilde{h}(z),
  \end{align*}
  which exists as $g$ is not uniformly zero, and where $ \widetilde{h}\left( z_0 \right) \neq 0 $. Consequently, there is $0 < \rho \leq r$ such that $ \widetilde{f}\colon U\left( z_0,\rho \right)\rightarrow \C $ given by
  \begin{align*}
    \widetilde{f}(z) &= \frac{1}{ \widetilde{h}(z) },
  \end{align*}
  which is holomorphic. Thus,
  \begin{align*}
    f(z) &= w + \left( z-z_0 \right)^{-m} \widetilde{f}(z).
  \end{align*}
  Thus, we get
  \begin{align*}
    f(z) &= \sum_{n=-m}^{\infty} a_n\left( z-z_0 \right)^{n},
  \end{align*}
  whence $z_0$ is either removable or a pole.
\end{proof}
\subsection{The Argument Principle and Rouché's Theorem}%
\begin{definition}
  If $U\subseteq \C$ is open, and $V\subseteq U$ is an open subset such that $U\setminus V$ consists solely of isolated points, then a function $f\colon V\rightarrow \C$ is \textit{meromorphic} if it is holomorphic on $V$ and every $z_0\in U\setminus V$ is either a pole or a removable singularity. We say $f$ is meromorphic on $U$.
\end{definition}
\begin{theorem}
  Let $U\subseteq \C$ be an open set, $\Gamma$ a piecewise $C^{1}$ cycle homologous to zero in $U$. Let $f$ be meromorphic on $U$ with no poles or zeros on $\img\left( \Gamma \right)$.
  \begin{enumerate}[(i)]
    \item The set $\set{z_0 \in U | \operatorname{ord}_{z_0}(f)\neq 0,n\left( \Gamma;z_0 \right) \neq 0}$ is finite.
    \item We have
      \begin{align*}
        \frac{1}{2\pi i} \oint_{\Gamma}^{} \frac{f'(z)}{f(z)}\:dz &= \sum_{\substack{z_0\in U \\ \operatorname{ord}{z_0}(f)\neq 0\\n\left( \Gamma;z_0 \right)}} n\left( \Gamma;z_0 \right)\ord_{z_0}(f).
      \end{align*}
  \end{enumerate}
\end{theorem}
\begin{proof}\hfill
  \begin{enumerate}[(i)]
    \item Let $K = \set{z\in U | n\left( \Gamma;z \right) \neq 0}\cup \img\left( \Gamma \right)$. We know that for $R > \sup_{w\in\img\left( \Gamma \right)}\left\vert w \right\vert$, we have $n\left( \Gamma;z \right) = 0$ for all $z\in \C\setminus B\left( 0,R \right)$, meaning that $K$ is bounded. Furthermore, if $\left( z_n \right)_n\rightarrow z\in \C$, then either $z\in \img\left( \Gamma \right)\subseteq K$ or $z\in \C\setminus \img\left( \Gamma \right)$, so that $n\left( \Gamma;z \right) \neq 0$ by the continuity of the map $w\mapsto n\left( \Gamma;w \right)$. Thus, $z\in K$ as $\Gamma$ is homologous to zero in $U$. It follows thus that $K$ is compact.\newline

      Let $E = \set{z_0\in K | \operatorname{ord}_{z_0}\left( f \right)\neq 0}$, meaning that $E$ consists of poles and zeros of $f$. These points are isolated, meaning $E$ is a closed subset of $K$, hence compact. Since $E$ is compact and contains isolated points only, it follows that $E$ is finite.
    \item For each $z\in K$, select $\delta_z > 0$ such that $U\left( z,\delta_z \right)\subseteq U$, and $U\left( z,\delta_z \right)\cap E \subseteq \set{z}$. Such a $\delta_z$ exists since $z$ is isolated in $E$ and $U$ is open. The collection $\set{U\left( z,\delta_z \right) | z\in K}$ is an open cover of $K$, so there is a finite subcover $\set{U\left( z_1,\delta_1 \right),\dots,U\left( z_m,\delta_m \right)}$. Define
      \begin{align*}
        V &= \bigcup_{j=1}^{m}U\left( z_j,\delta_j \right),
      \end{align*}
      so that $V$ is open with $K\subseteq V \subseteq U$. Since $\set{z\in U | n\left( \Gamma;z \right)\neq 0}\subseteq K\subseteq V$, it follows that $\Gamma$ is homologous to zero in $V$.\newline

      Define
      \begin{align*}
        g(z) &= f(z)\prod_{z\in E}\left( z-z_0 \right)^{-\operatorname{ord}_{z_0}(f)}.
      \end{align*}
      Since $\operatorname{ord}_z(g) = 0$ for all $z\in V$, all the singularities of $g$ are removable, and $\frac{g'}{g}$ is holomorphic on $V$ and satisfies
      \begin{align*}
        \frac{g'}{g} &= \frac{f'}{f} - \sum_{z_0\in E}\frac{1}{z-z_0}\operatorname{ord}_{z_0}(f).
      \end{align*}
      Cauchy's Integral theorem provides the desired result.
  \end{enumerate}
\end{proof}
\begin{theorem}[Rouché's Theorem]
  Let $U\subseteq \C$ be an open set, $\Gamma$ a piecewise $C^{1}$  cycle homologous to zero in $U$. Let $f$ and $g$ be meromorphic on $U$ with no poles or zeros on $\img\left( \Gamma \right)$. If $\left\vert f(z) - g(z) \right\vert < \left\vert f(z) \right\vert + \left\vert g(z) \right\vert$ for all $z\in \img\left( \Gamma \right)$, then
  \begin{align*}
    \frac{1}{2\pi i}\oint_{\Gamma}^{} \frac{f'(z)}{f(z)}\:dz &= \frac{1}{2\pi i}\oint_{\Gamma}^{} \frac{g'(z)}{g(z)}\:dz.
  \end{align*}
\end{theorem}
\begin{proof}
  Since $\left\vert f(z) - g(z) \right\vert < \left\vert f(z) \right\vert + \left\vert g(z) \right\vert$ on $\img\left( \Gamma \right)$, coupled with the fact that $\operatorname{ord}_{z}(f) = \operatorname{ord}_{z}(g) = 0$ on $\img\left( \gamma \right)$ implies that
  \begin{align*}
    \left\vert \frac{f(z)}{g(z)} - 1 \right\vert &< \left\vert \frac{f(z)}{g(z)} \right\vert + 1
  \end{align*}
  for all $z\in \img\left( \Gamma \right)$. This only holds if $\frac{f(z)}{g(z)} \in \C\setminus \left( -\infty,0 \right]$ for $z\in \img\left( \Gamma \right)$. Since $\img\left( \Gamma \right)$ is compact, there exists some $\ve > 0$ such that
  \begin{align*}
    \dist_{\left( -\infty,0 \right]}\left\vert \frac{f}{g}\left( \img\left( \Gamma \right) \right) \right\vert &\geq \ve.
  \end{align*}
  Since $\frac{f}{g}$ is continuous and $\img\left( \Gamma \right)$ is compact, there also exists some $\delta > 0$ such that whenever $\dist_{\img\left( \Gamma \right)}\left( z \right) < \delta$, we have $\frac{f(z)}{g(z)} \in \C\setminus \left( -\infty,0 \right]$.\newline

  Setting $V = \set{z\in U | \dist_{\img\left( \Gamma \right)}(z) < \delta}$, we let $h\colon V\rightarrow \C$ be defined by
  \begin{align*}
    h(z) &= \log\left( \frac{f(z)}{g(z)} \right)
  \end{align*}
  for the branch of the logarithm that excludes $\left( -\infty,0 \right]$, which is well-defined as $\frac{f}{g}\notin \left( -\infty,0 \right]$ on $V$. This satisfies
  \begin{align*}
    \frac{h'}{h} &= \frac{f'}{f} - \frac{g'}{g},
  \end{align*}
  whence by Cauchy's Integral Theorem,
  \begin{align*}
    0 &= \frac{1}{2\pi i} \oint_{\Gamma}^{} \frac{h'(z)}{h(z)}\:dz\\
      &= \frac{1}{2\pi i} \left( \oint_{\Gamma}^{} \frac{f'(z)}{f(z)}\:dz - \oint_{\Gamma}^{} \frac{g'(z)}{g(z)}\:dz \right).
  \end{align*}
\end{proof}
\begin{remark}
  Most use cases for Rouché's Theorem involve finding $g(z)$ such that $\left\vert f(z) - g(z)\right\vert < \left\vert g(z) \right\vert$ on $\img\left( \Gamma \right)$, where both $f(z)$ and $g(z)$ have no zeros or poles on $\img\left( \Gamma \right)$.
\end{remark}
\section{Worked Examples and Problem-Solving Methods}%
\begin{example}
  Suppose $U$ is a region in $\C$ that contains $0$, and suppose $f\colon U\rightarrow \C$ is a holomorphic function satisfying
  \begin{align*}
    \left\vert f\left( \frac{1}{n} \right) \right\vert &< e^{n}
  \end{align*}
  for sufficiently large $n$. We will show that this means $f$ is $0$ everywhere.\newline

  Toward this end, since $U$ is open, there is some $r > 0$ such that $U\left( 0,r \right)\subseteq U$. Since $f$ is holomorphic, on $U\left( 0,r \right)$, we may write
  \begin{align*}
    f\left( z \right) &= \sum_{n=0}^{\infty}a_nz^{n}
  \end{align*}
  for some sequence $\left( a_n \right)_n\subseteq \C$. Now, we also observe that
  \begin{align*}
    \left\vert f(0) \right\vert &= \lim_{n\rightarrow\infty} \left\vert f\left( \frac{1}{n} \right) \right\vert\\
                                &\leq \lim_{n\rightarrow\infty} e^{-n}\\
                                &= 0.
  \end{align*}
  Suppose toward contradiction that $f$ were nonconstant. Then, there would be some minimal positive value $\ell$ such that
  \begin{align*}
    f(z) &= z^{\ell}\sum_{n=0}^{\infty}a_{n + \ell}z^{n}
  \end{align*}
  has $a_{\ell}\neq 0$. Thus, defining
  \begin{align*}
    g(z) &= \sum_{n=0}^{\infty}a_{n + \ell}z^{n},
  \end{align*}
  we observe that $g\left( 0 \right) \neq 0$, meaning that on some sufficiently small ball $U\left( 0,\delta \right)\subseteq U\left( 0,r \right)$, we have $\left\vert g(z) \right\vert > \left\vert \frac{a_{\ell}}{2} \right\vert$ for all $z\in U\left( 0,\delta \right)$. In particular, this means that for $n$ with $\frac{1}{n} < \delta$,
  \begin{align*}
    e^{-n} &\geq \left\vert f\left( \frac{1}{n} \right) \right\vert\\
           &= n^{-\ell}\left\vert g\left( \frac{1}{n} \right) \right\vert\\
           &\geq \frac{\left\vert a_{\ell} \right\vert}{2n^{\ell}},
  \end{align*}
  whence
  \begin{align*}
    \left\vert a_{\ell} \right\vert &\leq \frac{2n^{\ell}}{e^{n}}.
  \end{align*}
  Yet, since $n$ is arbitrary and $\ell$ is constant, this implies that $\left\vert a_{\ell} \right\vert = 0$, contradicting the assumption that there were such a $g$. Thus, in particular, we have that $f(z) = 0$ on $U\left( 0,r \right)$, whence $f$ is zero everywhere by the identity theorem.
\end{example}
\subsection{Cauchy Estimate Problems}%
\begin{example}
  Suppose $f$ is an entire function, and suppose there exists a constant $C$ such that for all $z\in \C$,
  \begin{align*}
    \left\vert f(z) \right\vert &\leq C\left( 1 + \left\vert z \right\vert \right)^{1/2}.
  \end{align*}
  We will show that $f$ is then constant. Toward this end, we will be able to use the Cauchy estimate by taking
  \begin{align*}
    \left\vert f^{(n)}(z) \right\vert &\leq \frac{n!}{R^{n}} \sup_{\left\vert z \right\vert = R} \left\vert f(z) \right\vert\\
                                      &\leq \frac{Cn!}{R^{n}} \sup_{\left\vert z \right\vert = R} \left( 1 + \left\vert z \right\vert \right)^{1/2}\\
                                      &= \frac{Cn!}{R^{n}} \left( 1 + R \right)^{1/2},
  \end{align*}
  whence for all $n\geq 1$, since $R$ is arbitrary, we have $\left\vert f^{(n)}\left( z \right) \right\vert = 0$, so $f$ is constant.
\end{example}
\subsection{Maximum Modulus Principle Problems}%

\subsubsection{The Phragmén--Lindelöf Method}%
The maximum modulus principle is primarily useful in the case where $f$ is continuous on the closure of a bounded open set $U$ and holomorphic on the interior. Yet, this fails to be true if $U$ is unbounded.\newline

For instance, if
\begin{align*}
  U &= \set{z\in \C | -\frac{\pi}{2} < \im(z) < \frac{\pi}{2}},
\end{align*}
and $f(z) = e^{e^{z}}$, then 
\begin{align*}
  f\left( x \pm \frac{\pi}{2}i \right) &= e^{\pm ie^{x}},
\end{align*}
whence $\left\vert f(z) \right\vert = 1$ for $z\in \partial U$. Yet, $f(z)\rightarrow \infty$ very rapidly along the positive real axis, which is contained in $U$.\newline

Yet, all hope is not lost in the case that $U$ is unbounded. If $U$ is unbounded and there is $g\colon U\rightarrow \C$  such that $\left\vert f \right\vert < \left\vert g \right\vert$, and $g\rightarrow\infty$ ``slowly'' (so to speak) as $z\rightarrow\infty$, then it turns out that $f$ is actually bounded in $U$, and we can use the maximum modulus principle to obtain other conclusions about $f$.\newline

Finding such a $g$ is part of the \textit{Phragmén--Lindelöf} method, which we expand upon here.
\begin{example}
  From the Cauchy estimates, we know that if $f$ is entire and
  \begin{align*}
    \left\vert f(z) \right\vert \leq C\left( 1 + \left\vert z \right\vert^{1/2} \right),
  \end{align*}
  then $f$ is constant.
\end{example}
\begin{theorem}[Hadamard Three-Lines Theorem]
  Let $a,b\in \R$ be fixed with $ a < b $. Let $U = \set{z | a < \re(z) < b}$. Suppose $\left\vert f(z) \right\vert < B$ for all $z\in U$ and some fixed $B < \infty$. Define
  \begin{align*}
    M(x) &= \sup\set{\left\vert f(z) \right\vert | z\in \overline{U}}.
  \end{align*}
  Then,
  \begin{align*}
    M(x)^{b-a} &\leq M(a)^{b-x}M(b)^{x-a}.
  \end{align*}
\end{theorem}
\begin{proof}
  Suppose $M(a) = M(b) = 1$. Our task now is to show that $\left\vert f(z) \right\vert \leq 1$ for all $z\in U$. Toward this end, define
  \begin{align*}
    h_{\ve}(z) &= \frac{1}{1 + \ve\left( z-a \right)}
  \end{align*}
  for $z\in \overline{U}$. We have $\left\vert h_{\ve} \right\vert \leq 1$ in $ \overline{U} $, so that
  \begin{align*}
    \left\vert f(z)h_{\ve}(z) \right\vert \leq 1
  \end{align*}
  for all $z\in \partial U$. Furthermore, since $\left\vert 1 + \ve\left( z-a \right) \right\vert \geq \ve\left\vert \im(z) \right\vert$, we have
  \begin{align*}
    \left\vert f(z)h_{\ve}(z) \right\vert &\leq \frac{B}{\ve\left\vert \im(z) \right\vert}
  \end{align*}
  for all $z\in \overline{U}$. Cut out a (closed) rectangle $R$ from $ \overline{U} $ via the lines $\im(z) = \pm \frac{B}{\ve}$. Thus, along $\partial R$, we have $\left\vert f(z)h_{\ve}(z) \right\vert \leq 1$, so that $\left\vert f(z)h_{\ve}(z) \right\vert \leq 1$ on $R$ by the maximum modulus principle.\newline

  Yet, since $\left\vert f(z)h_{\ve}(z) \right\vert \leq \frac{B}{\ve\left\vert \im(z) \right\vert}$ on the entirety of $ \overline{U} $, and $ \frac{B}{\ve\left\vert \im(z) \right\vert} < 1 $ outside $R$, it follows that $\left\vert fh_{\ve} \right\vert\leq 1$ on $ \overline{U} $, so $ \left\vert f(z)h_{\ve}(z) \right\vert \leq 1 $ for all $z\in U$ and all $\ve > 0$. Taking the limit as $\ve \rightarrow 0$, we obtain the desired result, that $\left\vert f(z) \right\vert \leq 1$.\newline

  In the general case, we define
  \begin{align*}
    g(z) &= M(a)^{(b-z)/(b-a)}M(b)^{(z-a)/(b-a)},
  \end{align*}
  where for all $M > 0$ and complex $w$, we have $M^{w} = e^{w\ln(M)}$. Then, $g$ is entire, $g$ is always nonzero, $\frac{1}{g}$ is bounded on $ \overline{U} $, and has
  \begin{align*}
    \left\vert g\left( a + iy \right) \right\vert &= M(a)\\
    \left\vert g\left( b + iy \right) \right\vert &= M(b),
  \end{align*}
  meaning that $\frac{f}{g}$ satisfies the previous assumptions, so that $ \left\vert f/g \right\vert \leq 1 $ in $U$.
\end{proof}
In the Phragmén--Lindelöf method, we seek to find a particular $\ve$-dependent function $h_{\ve}\colon U\rightarrow \C$ such that the following hold:
\begin{itemize}
  \item $\left\vert fh_{\ve}(z) \right\vert \leq M$ for all $z\in \partial U$;
  \item $\lim_{\ve\rightarrow 0} h_{\ve}(z) = 1$;
  \item there exists a \textit{bounded} $V\subseteq U$ such that $\left\vert fh_{\ve} \right\vert\leq M$ on $\partial V$ and on $U\setminus \overline{V}$.
\end{itemize}
\section{Old Exams}%
\section{Notation}%
\begin{itemize}
  \item $\ds U\left( z_0,r \right) = \set{z\in \C | \left\vert z-z_0 \right\vert < r}$
  \item $\ds B\left( z_0,r \right) = \set{z\in \C | \left\vert z-z_0 \right\vert \leq r}$
  \item $\ds S\left( z_0,r \right) = \set{z\in \C | \left\vert z-z_0 \right\vert = r}$
  \item $\ds \dot{U}\left( z_0,r \right) = \set{z\in \C | 0 < \left\vert z-z_0 \right\vert < r}$
  \item $\ds A\left( z_0,r_1,r_2 \right) = \set{z\in \C | r_1 < \left\vert z-z_0 \right\vert < r_2}$
\end{itemize}
\end{document}
