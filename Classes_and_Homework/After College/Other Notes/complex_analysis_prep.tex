\documentclass[10pt]{mypackage}

% sans serif font:
%\usepackage{cmbright}
%\usepackage{sfmath}
%\usepackage{bbold} %better blackboard bold

\usepackage{homework}
\usepackage{notes}
\usepackage{mlmodern}
\usepackage[skip=10pt plus 1pt]{parskip}
%\usepackage{newpxtext,eulerpx,eucal}
%\renewcommand*{\mathbb}[1]{\varmathbb{#1}}


\fancyhf{}
\rhead{Avinash Iyer}
\lhead{Complex Analysis Qualifier Preparation}

\setcounter{secnumdepth}{0}

\begin{document}
\RaggedRight
This is a collection of old complex analysis qualifier exam solutions, as well as some notes on useful results and proofs.
\section{Useful Results and Proofs}%
\subsection{Analytic Functions}%
\begin{definition}
  Let $U\subseteq \C$ be an open set. A function $f\colon U\rightarrow \C$ is called \textit{analytic} if, for any $z_0\in U$, there is $r > 0$ and $\left( a_k \right)_k\subseteq \C$ such that
  \begin{align*}
    f(z) &= \sum_{k=0}^{\infty}a_k\left( z-z_0 \right)^{k}
  \end{align*}
  for all $z\in U\left( z_0,r \right)$.
\end{definition}
Analytic functions form a $\C$-algebra.
\begin{theorem}[Identity Theorem]
  Let $f,g\colon U\rightarrow \C$ be analytic functions defined a connected open set (also known as a region). If
  \begin{align*}
    A &= \set{z\in \C | f(z) = g(z)}
  \end{align*}
  admits an accumulation point in $U$, then $f = g$ on $U$.
\end{theorem}
\begin{proof}
  To begin, we show that if $f\colon U\rightarrow \C$ is an analytic function that is not uniformly zero, then for any $z_0\in U$, there is $\rho > 0$ such that $f$ is nonzero on $\dot{U}\left( z_0,\rho \right)\subseteq U$. Towards this end, we may write
  \begin{align*}
    f(z) &= \sum_{k=0}^{\infty}a_k\left( z-z_0 \right)^{k},
  \end{align*}
  for all $z\in U\left( z_0,r \right)$, some $r > 0$, and since $f$ is not uniformly zero, there is some minimal $\ell$ such that $a_{\ell}\neq 0$. This yields
  \begin{align*}
    f(z) &= \left( z-z_0 \right)^{\ell}\sum_{k=0}^{\infty}a_{k + \ell}\left( z-z_0 \right)^{k};
  \end{align*}
  the function $h\colon U\left( z_0,r \right)\rightarrow \C$ given by
  \begin{align*}
    h(z) &= \sum_{k=0}^{\infty}a_{k + \ell}\left( z-z_0 \right)^{k}
  \end{align*}
  then has the same radius of convergence as $f$ and is not zero at $z_0$, so that $g$ is not zero on some $U\left( z_0,\rho \right)$ as $g$ is continuous.%\newline

  Now, we let $V_1$ be the set of accumulation points of $A$ in $U$, and let $V_2 = U\setminus V_1$.%\newline

  If $z\in V_2$, then there is some $r_1 > 0$ such that $\dot{U}\left( z_0,r_1 \right)\cap A = \emptyset$, or that $\dot{U}\left( z_0,r_1 \right) \subseteq A^{c}$. Meanwhile, since $U$ is open, there is some $r_2 > 0$ such that $U\left( z_0,r_2 \right)\subseteq U$, meaning that if $r = \min\set{r_1,r_2}$, then $U\left( z_0,r \right) \subseteq U\setminus A$. Thus, $V_2$ is open.%\newline

  Meanwhile, if $z\in V_1$, then since $V_1\subseteq U$, it follows that there is $r > 0$ such that $U\left( z,r \right)$ and $\left( a_k \right)_k$ such that
  \begin{align*}
    f(w)- g(w) &= \sum_{k=0}^{\infty}a_k\left( w-z \right)^{k}
  \end{align*}
  for all $w\in U\left( z,r \right)$. We claim that $f(w) - g(w)$ is uniformly zero on $U\left( z,r \right)$. Else, if there were $w_0\in U\left( z,r \right)$ such that $f\left( w_0 \right)\neq g\left( w_0 \right)$, then it would follow that there is $0 < s\leq r$ such that $f(w)\neq g(w)$ for all $w\in \dot{U}\left( w_0,s \right)$. Yet, this would contradict the assumption that $z$ is an accumulation point, meaning that $V_1$ is open.%\newline

  Since $V_1$ and $V_2$ are disjoint open sets whose union is equal to $U$, it follows that either $V_1 = U$ or $V_2 = U$. If $A \neq \emptyset$, then the identity theorem follows.
\end{proof}
\subsection{Differentiability}%
\begin{definition}
  If $U\subseteq \C$ is an open set, then we say $f$ is differentiable at $z_0\in U$ if
  \begin{align*}
    \lim_{w\rightarrow z_0} \frac{f\left( w \right)-f\left( z_0 \right)}{w-z_0}
  \end{align*}
  exists. We call this value the \textit{derivative} of $f$ at $z_0$, and usually write $f'\left(z_0\right)$.%\newline

  If $f$ is differentiable at every $z_0\in U$, we say $f$ is differentiable on $U$.%\newline

  If $f$ is continuous and admits a continuous derivative, then we say $f$ is \textit{holomorphic}.
\end{definition}
Note that the limit must be independent of direction. That is, for all $\ve > 0$, there is $\delta > 0$ such that
\begin{align*}
  \left\vert \frac{f\left( w \right)-f\left( z_0 \right)}{z-z_0} - f'\left( z_0 \right) \right\vert &< \ve
\end{align*}
whenever $0 < \left\vert z-z_0 \right\vert < \delta$.%\newline

Now, given $U\subseteq \C$, write $z = x + iy$ and
\begin{align*}
  f\left( z \right) &= f\left( x + iy \right)\\
                    &= u\left( x,y \right) + iv\left( x,y \right),
\end{align*}
where $u = \re(f)$ and $v = \im(f)$. Observe then that if $f$ is differentiable at $x_0 + iy_0\in U$, then since the limit is independent of path, by taking the limit over real numbers, we have
\begin{align*}
  f'\left( z_0 \right) &= \lim_{h\rightarrow 0} \frac{\left( u\left( x + h,y \right) + iv\left( x + h,y \right) \right) - \left( u\left( x,y \right) + iv\left( x,y \right) \right)}{ h }\\
                       &= \pd{u}{x} + i \pd{v}{x},
\end{align*}
and by taking over the imaginary numbers,
\begin{align*}
  f'\left( z_0 \right) &= \lim_{h\rightarrow 0} \frac{\left( u\left( x,y+h \right) + iv\left( x,y+h \right) \right) - \left( u\left( x,y \right) + iv\left( x,y \right) \right)}{ih}\\
                       &= -i \pd{u}{y} + \pd{v}{y}.
\end{align*}
Thus, we obtain the following.
\begin{definition}
  The system of partial differential equations
  \begin{align*}
    \pd{u}{x} &= \pd{v}{y}\\
    \pd{u}{y} &= - \pd{v}{x}
  \end{align*}
  is known as the \textit{Cauchy--Riemann Equations}.
\end{definition}
Observe that if $f$ is differentiable, then the $u$ and $v$ in the definition of $f$ satisfy the Cauchy--Riemann equations. Yet, we desire to understand a bit more about when exactly $f$ is differentiable or holomorphic.
\begin{proposition}
  If $f = u + iv$ is a holomorphic function such that $u,v$ are in $C^{2}\left( U \right)$, then $u$ and $v$ are harmonic. That is, $u$ and $v$ satisfy Laplace's equation:
  \begin{align*}
    \pd{^2u}{x^2} + \pd{^2u}{y^2} &= 0.
  \end{align*}
\end{proposition}
We call $u$ and $v$ \textit{harmonic conjugates} for each other. That is, if $u\colon U\rightarrow \R$ is a harmonic function, then $v\in C^{1}\left( U \right)$ is called a harmonic conjugate if the Cauchy--Riemann equations hold for $u$ and $v$.
\begin{theorem}
  Let $U\subseteq \R^{2}$ be a ball or all of $\R^{2}$. Then, every harmonic function on $U$ has a harmonic conjugate. If $u\in C^{3}\left( U \right)$, then this conjugate is itself harmonic.
\end{theorem}
\begin{lemma}
  Let $g\colon U\left( \left( x_0,y_0 \right),R \right)\rightarrow \R$ be such that $g$ and $ \pd{g}{x} $ are continuous. Then, $G\colon U\left( \left( x_0,y_0 \right),R \right)\rightarrow \R$, given by
  \begin{align*}
    G\left(x,y\right) &= \int_{y_0}^{y} g\left( x,t \right)\:dt
  \end{align*}
  satisfies
  \begin{align*}
    \pd{G}{x} &= \int_{y_0}^{y} \pd{g}{x}\left( x,t \right)\:dt.
  \end{align*}
\end{lemma}
\begin{proof}[Proof of Lemma]
  Write
  \begin{align*}
    \frac{G\left( x + h,y \right) - G\left( x,y \right)}{h} - \int_{y_0}^{y} \pd{g}{x}\left( x,t \right)\:dt &= \int_{y_0}^{y} \left( \frac{g\left( x+h,t \right)-g\left( x,t \right)}{h} - \pd{g}{x}\left( x,t \right) \right)\:dt.
  \end{align*}
  By mean value theorem, the first term is equal to $ \pd{g}{x}\left( x_1,t \right) $ for some $x_1$ between $x$ and $x + h$. As $h\rightarrow 0$, $x_1\rightarrow x$, as $ \pd{g}{x} $ is uniformly continuous on a compact subset that contains $x$ and $x + h$. We may exchange limit and integral to obtain the desired result.
\end{proof}
\begin{proof}[Proof of Theorem]
  We prove for the case of $U = U\left( \left( x_0,y_0 \right),R \right)$. Define
  \begin{align*}
    v\left( x,y \right) &= \int_{y_0}^{y} \pd{u}{x}\left( x,t \right)\:dt + \phi(x),
  \end{align*}
  with $\phi(x)$ to be determined later. By the fundamental theorem of calculus, we have
  \begin{align*}
    \pd{v}{y} &= \pd{u}{x},
  \end{align*}
  while by differentiating under the integral sign, and using the fact that $u$ is harmonic, we have
  \begin{align*}
    \pd{v}{x} &= \int_{y_0}^{y} \pd{^2u}{x^2}\left( x,t \right)\:dt + \diff{\phi}{x}\\
              &= - \int_{y_0}^{y} \pd{^2u}{y^2}\left( x,t \right)\:dt + \diff{\phi}{x}\\
              &= - \pd{u}{y}\left( x,y \right) + \pd{u}{y}\left( x,y_0 \right) + \diff{\phi}{x}.
  \end{align*}
  Defining $\phi\colon \R\rightarrow \R$ by
  \begin{align*}
    \phi(x) &= -\int_{x_0}^{x} \pd{u}{y}\left( s,y_0 \right)\:ds,
  \end{align*}
  we see that $v$ thus satisfies all the necessary requirements to be a harmonic conjugate.%\newline

  Now, if $u$ is $C^{3}$, then we defined $v$ via the derivative of $u$, so that $v$ is $C^{2}$, and thus $v$ is harmonic.
\end{proof}
\subsection{Cauchy's Integral Formula}%
\begin{proposition}
  Fix $z_0\in \C$, $R > 0$, and $f\colon U\left( z_0,R \right)\rightarrow \C$ holomorphic. For all $z\in U\left( z_0,R \right)$, we have
  \begin{align*}
    f(z) &= \frac{1}{2\pi i} \int_{S\left( z_0,R \right)}^{} \frac{f\left( \zeta \right)}{\zeta - z}\:d\zeta.
  \end{align*}
\end{proposition}
\begin{proof}
  It suffices to show that
  \begin{align*}
    \frac{1}{2\pi i} \int_{S\left( z_0,R \right)}^{} \frac{f\left( \zeta \right)-f\left( z \right)}{\zeta - z}\:d\zeta &= 0.
  \end{align*}
  By using the chain rule and fundamental theorem of calculus, we find
  \begin{align*}
    \frac{1}{2\pi i} \int_{S\left( z_0,R \right)}^{} \frac{f\left( \zeta \right)-f\left( z \right)}{\zeta - z}\:d\zeta &= \frac{1}{2\pi i}\int_{S\left( z_0,R \right)}^{} \frac{ \int_{0}^{1} f'\left( \left( 1-t \right)z + t\zeta \right)\left( \zeta - z \right)\:dt }{\zeta - z}\:d\zeta\\
                                                                                                                       &= \frac{1}{2\pi i} \int_{S\left( z_0,R \right)}^{} \int_{0}^{1} f'\left( \left( 1-t \right)z + t\zeta \right)\:dt \:d\zeta\\
                                                                                                                       &= \frac{1}{2\pi i} \int_{S\left( z_0,R \right)}^{} \diff{}{\zeta}\left( \frac{1}{t}f\left( \left( 1-t \right)z + t\zeta \right) \right)\:d\zeta\\
                                                                                                                       &= 0.
  \end{align*}
\end{proof}
\begin{proposition}
  Let $f\colon U\rightarrow \C$ be a holomorphic function. The following all hold:
  \begin{enumerate}[(i)]
    \item $f$ is analytic;
    \item $f$ is smooth with $f^{(n)}$ holomorphic;
    \item for all $z_0\in U$, if we let $R = \sup\set{r > 0 | U\left( z_0,r \right)\subseteq U}$, then there is $\left( a_n \right)_n\subseteq \C$ such that
      \begin{align*}
        f\left( z \right) &= \sum_{n=0}^{\infty} a_n\left( z-z_0 \right)^{n},
      \end{align*}
      where the power series has radius of convergence $R$.
  \end{enumerate}
\end{proposition}
\begin{proof}\hfill
  \begin{enumerate}[(i)]
    \item There exists $r < s$ with $U\left( z_0,s \right)\subseteq U$ and $r < r_1 < s$ such that $S\left( z_0,r_1 \right) \subseteq U$. By Cauchy's Integral Formula, and a power series expansion of $\frac{1}{\xi - z}$ about $z_0$, this gives
      \begin{align*}
        f(z) &= \frac{1}{2\pi i} \oint_{S\left( z_0,r_1 \right)}^{} \frac{f\left( \xi \right)}{\xi-z}\:d\xi\\
             &=\sum_{n=0}^{\infty} \left( z-z_0 \right)^n \underbrace{\left( \frac{1}{2\pi i} \oint_{S\left( z_0,r_1 \right)}^{} \frac{f\left( \xi \right)}{\left( \xi-z_0 \right)^{n+1}}\:d\xi \right)}_{\eqcolon a_n}\\
             &=\sum_{n=0}^{\infty}a_n\left( z-z_0 \right)^{n}.
      \end{align*}
    \item Analytic functions are automatically smooth, hence complex-differentiable with continuous derivative.
    \item If $r < r_1 < R$, then
      \begin{align*}
        f\left( z \right) &= \sum_{n=0}^{\infty}\left( z-z_0 \right)^{n} \left( \frac{1}{2\pi i} \int_{S\left( z_0,r_1 \right)}^{} \frac{f\left( \xi \right)}{\left( \xi-z \right)^{n+1}}\:d\xi \right),
      \end{align*}
      and since the series converges uniformly, we have
      \begin{align*}
        \frac{f^{(n)}(z)}{n!} &= \frac{1}{2\pi i} \oint_{S\left( z_0,r_1 \right)}^{} \frac{f\left( \xi \right)}{\left( \xi-z \right)^{n+1}}\:d\xi.
      \end{align*}
      Since $r$ was arbitrary, this holds for any $ 0 < r_1 < R $, whence
      \begin{align*}
        f\left( z \right) &= \sum_{n=0}^{\infty} a_n\left( z-z_0 \right)^{n}
      \end{align*}
      holds for all $z\in U\left( z_0,R \right)$.
  \end{enumerate}
\end{proof}
\begin{corollary}
  Let $U\subseteq \C$ be open, let $z_0\in U$, and $r > 0$ with $B\left( z_0,r \right)\subseteq U$. The following hold:
  \begin{enumerate}[(i)]
    \item for all $z\in U\left( z_0,r \right)$,
      \begin{align*}
        \frac{f^{(n)}\left(z\right)}{n!} &= \frac{1}{2\pi i} \int_{S\left( z_0,r \right)}^{} \frac{f\left( \xi \right)}{\left( \xi-z \right)^{n+1}}\:d\xi;
      \end{align*}
    \item for all $n  > 0$,
      \begin{align*}
        \left\vert f^{(n)}\left( z_0 \right) \right\vert &\leq \frac{n!}{r^{n}} \sup_{\zeta\in S\left( z_0,r \right)} \left\vert f\left(\zeta\right) \right\vert.
      \end{align*}
      This particular result is known as the \textit{Cauchy Estimate}.
  \end{enumerate}
\end{corollary}
\begin{theorem}[Liouville's Theorem]
  If $f\colon \C\rightarrow \C$ is holomorphic and bounded in modulus, then $f$ is constant.
\end{theorem}
Liouville's Theorem follows from applying Cauchy's estimate to $f$ and using the fact that $f$ is bounded to find that all higher derivatives of $f$ vanish.
\begin{theorem}[Fundamental Theorem of Algebra]
  If $p(z) = a_nz^{n} + \cdots + a_1 z + a_0$ has $n\geq 1$ and $a_n\neq 0$, then there is at least one $z_0$ such that $p\left( z_0 \right) = 0$.
\end{theorem}
\begin{proof}
  Suppose $p(z)$ were never zero. It would follow then that $\frac{1}{p(z)}$ is also an entire function.%\newline
  
  Since $\lim_{|z|\rightarrow\infty}\left\vert p(z) \right\vert = \infty$, it follows that $\lim_{|z|\rightarrow\infty} \frac{1}{\left\vert p(z) \right\vert} = 0$, whence $ \left\vert \frac{1}{p(z)} \right\vert $ is an entire function that is bounded (as all functions that vanish at infinity are bounded). This means that $ \frac{1}{p(z)} $ is constant, so $p(z)$ is constant.
\end{proof}
\begin{corollary}
  Let $f\colon \C\rightarrow \C$ be a nonconstant entire function. Then, $f\left(\C\right)$ is dense in $\C$.
\end{corollary}
\begin{proof}
  Suppose there were $w\in\C$ and $r > 0$ such that $U\left( w,r \right)\cap f\left( \C \right) = \emptyset$. Then, $\left\vert f(z)-w \right\vert \geq r$ for all $z\in \C$, meaning that
  \begin{align*}
    g(z) &= \frac{1}{f(z)-w}
  \end{align*}
  is bounded and entire (the entirety following from the fact that $f(z)-w$ is nonvanishing).
\end{proof}
\subsection{Cycles, Winding Numbers, and Homology}%
Now, we may generalize some of these results related to Cauchy's Integral Formula.
\begin{proposition}
  Let $\gamma\colon [a,b]\rightarrow \C$ be a piecewise $C^{1}$ loop. For all $z\in \C\setminus \img\left( \gamma \right)$, we have
  \begin{align*}
    \frac{1}{2\pi i} \oint_{\gamma}^{} \frac{1}{\xi-z}\:d\xi &\in \Z.
  \end{align*}
\end{proposition}
\begin{proof}
  Let $\phi\colon [a,b]\rightarrow \C$ be defined by
  \begin{align*}
    \phi(t) &= \int_{a}^{t} \frac{\gamma'(s)}{\gamma(s)-z}\:ds.
  \end{align*}
  Then, we observe
  \begin{align*}
    \phi(b) &= \oint_{\gamma}^{} \frac{1}{\xi - z}\:d\xi.
  \end{align*}
  Then, define $\psi\colon [a,b]\rightarrow \C$ by
  \begin{align*}
    \psi(t) &= \frac{e^{\phi(t)}}{\gamma(t)-z}.
  \end{align*}
  By the fundamental theorem of calculus, we have
  \begin{align*}
    \phi'(t) &= \frac{\gamma'(t)}{\gamma(t)-z}\\
  \psi'(t) &= \frac{\phi'(t)e^{\phi(t)}}{\gamma(t)-z} - \frac{e^{\phi'(t)}\gamma'(t)}{\left( \gamma(t)-z \right)^2}\\
           &= 0,
  \end{align*}
  whence $\psi(t)$ is constant, and $\psi(t) = \psi(a)$, so
  \begin{align*}
    \psi(a) &= \frac{1}{\gamma(a) - z}.
  \end{align*}
  In particular, $\psi(b) = \psi(a)$, so
  \begin{align*}
    e^{\phi(b)} &= \psi(b)\left( \gamma(b)-z \right)\\
                &= \psi(a)\left( \gamma(a)-z \right)\\
                &= 1,
  \end{align*}
  so $\phi(b) = 2\pi i k$ for some $k\in \Z$.
\end{proof}
\begin{definition}
  Let $\gamma\colon [a,b]\rightarrow \C$ be a piecewise $C^{1}$ loop. For all $z\in \C\setminus\img\left( \gamma \right)$, define
  \begin{align*}
    n\left( \gamma;z \right) &= \frac{1}{2\pi i} \oint_{\gamma}^{} \frac{1}{\xi - z}\:d\xi
  \end{align*}
   to be the \textit{winding number} of $\gamma$ about $z$.
\end{definition}
\begin{definition}
  A piecewise $C^{1}$ \textit{cycle} is a formal sum
  \begin{align*}
    \Gamma &= \gamma_1 + \cdots + \gamma_n,
  \end{align*}
  where the $\gamma_j\colon \left[ a_j,b_j \right]\rightarrow \C$ are piecewise $C^{1}$ loops. The \textit{length} of $\Gamma$ is the sum of the lengths of the respective $\gamma_j$.%\newline

  Given a piecewise $C^{1}$ cycle $\Gamma$, define
  \begin{align*}
    \oint_{\Gamma}^{} f(z)\:dz &= \sum_{j=1}^{n} \oint_{\gamma_j}^{} f(z)\:dz,
  \end{align*}
  and
  \begin{align*}
    n\left( \Gamma;z \right) &= \sum_{j=1}^{n}n\left( \gamma_j;z \right).
  \end{align*}
\end{definition}
\begin{proposition}
  The following hold for the winding number $n\left( \gamma;z \right)$:
  \begin{enumerate}[(i)]
    \item the function $n\left( \Gamma;\cdot \right)\colon \C\setminus \img\left( \gamma \right)\rightarrow \Z$ is continuous;
    \item $n\left( \Gamma;z \right)$ is constant on each connected component of $\C\setminus \img\left( \Gamma \right)$;
    \item there exists a unique unbounded connected component with $n\left( \Gamma;z \right) = 0$ for all $z$ in this unbounded connected component.
  \end{enumerate}
\end{proposition}
\begin{proof}\hfill
  \begin{enumerate}[(i)]
    \item Since $ \img\left( \Gamma \right) $ is compact, any $z\notin \img\left( \Gamma \right)$ admits a strictly positive
      \begin{align*}
        \dist_{\img\left( \Gamma \right)}\left( z \right) &= \inf_{w\in\img\left( \Gamma \right)} \left\vert w-z \right\vert.
      \end{align*}
      Let $w\in\C$ be such that
      \begin{align*}
        \left\vert w-z \right\vert &< \frac{1}{2}\dist_{\img\left( \Gamma \right)}\left( z \right),
      \end{align*}
      so that $w\in \C\setminus \img\left( \Gamma \right)$. Observe then that
      \begin{align*}
        \left\vert n\left( \Gamma;z \right)-n\left( \Gamma;w \right) \right\vert &= \left\vert \frac{1}{2\pi i} \oint_{\Gamma}^{} \frac{1}{\xi-z} - \frac{1}{\xi-w}\:d\xi \right\vert\\
                                                                                 &\leq \frac{1}{2\pi} \sum_{j=1}^{n} \oint_{\gamma_j}^{} \left\vert \frac{1}{\xi-z} - \frac{1}{\xi-w} \right\vert \:\left\vert d\xi \right\vert\\
                                                                                 &= \frac{1}{2\pi} \sum_{j=1}^{n} \oint_{\gamma_j}^{} \left\vert \frac{z-w}{\left( \xi-z \right)\left( \xi-w \right)} \right\vert\:\left\vert d\xi \right\vert\\
                                                                                 &\leq \frac{1}{2\pi} \left( \frac{2}{\dist_{\img\left( \Gamma \right)}(z)} \right)^2 \ell\left( \Gamma \right)\left\vert z-w \right\vert,
      \end{align*}
      whence $\left\vert n\left( \Gamma;z \right)-n\left( \Gamma;w \right) \right\vert$ is sufficiently small whenever $\left\vert z-w \right\vert$ is sufficiently small.
    \item If $C$ is a connected component of $\C\setminus\img\left( \Gamma \right)$, and $n\left( \Gamma;\cdot \right)\colon C\rightarrow \Z$ is continuous, then since $\Z$ is discrete, $n\left( \Gamma;\cdot \right)$ is constant on $C$.
    \item For uniqueness, if there are unbounded connected components $C_1$ and $C_2$ of $\C\setminus \img\left( \Gamma \right)$, then there exists $M > \sup_{z\in\img\left( \Gamma \right)}\left\vert z \right\vert$ and $w_1\in C_1,w_2\in C_2$ such that $\left\vert w_1 \right\vert > 2M$ and $\left\vert w_2 \right\vert > 2M$. Since $\C\setminus U\left( 0,2M \right)$ is path connected, there exists $\gamma\colon [0,1]\rightarrow \C$ with $\left\vert \gamma(t) \right\vert \geq 2M$ and $\gamma(0) = w_1$, $\gamma(1) = w_2$. Therefore, $w_1$ and $w_2$ are in the same connected component.%\newline

      Existence then follows from $\img\left( \Gamma \right)$ being compact.%\newline

      Finally, let $\left( z_n \right)_n\subseteq C$, where $C$ is the unbounded connected component, be such that $\lim_{n\rightarrow\infty} \left\vert z_n \right\vert = \infty$. For $M > \sup_{z\in\img\left( \gamma \right)}\left\vert z \right\vert$, there exists $m\in \N$ such that $\left\vert z_m \right\vert > M$. Then, we have
      \begin{align*}
        \left\vert n\left( \Gamma;z_m \right) \right\vert &= \left\vert \frac{1}{2\pi i} \oint_{\Gamma}^{} \frac{1}{\xi-z}\:d\xi \right\vert\\
                                                          &\leq \frac{1}{2\pi} \sum_{j=1}^{k} \oint_{\gamma_j}^{} \frac{1}{\left\vert \xi-z \right\vert}\:\left\vert d\xi \right\vert\\
                                                          &\leq \frac{1}{2\pi} \sum_{j=1}^{k} \oint_{\gamma_j}^{} \frac{1}{\left\vert z_m \right\vert - M}\:\left\vert d\xi \right\vert\\
                                                          &= \frac{\ell\left( \Gamma \right)}{2\pi \left( \left\vert z_m \right\vert - M \right)},
      \end{align*}
      whence $\lim_{m\rightarrow\infty} n\left( \Gamma;z_m \right) = 0$, meaning that there exists $N$ such that $\left\vert n\left( \Gamma;z_m \right) \right\vert < 1$ for all $m\geq N$, meaning $n\left( \Gamma;z_m \right) = 0$ for all sufficiently large $m$. Since $C$ is connected, it thus follows that $n\left( \Gamma;z \right) = 0$ for all $z\in C$.
  \end{enumerate}
\end{proof}
\begin{definition}
  Let $U\subseteq \C$ be open. A cycle $\Gamma$ is \textit{homologous to zero in $U$} if $\img\left( \Gamma \right)\subseteq U$ and for all $z\in \C\setminus U$, $n\left( \Gamma;z \right) = 0$.
\end{definition}
\begin{theorem}[Cauchy's Integral Formula, General Case]
  Let $\Gamma = \gamma_1 + \cdots + \gamma_k$ be a piecewise $C^1$ cycle homologous to zero in $U$, and $f\colon U\rightarrow \C$ holomorphic. Then, for all $z\in U\setminus \img\left( \Gamma \right)$,
  \begin{align*}
    n\left( \Gamma;z \right)f\left( z \right) &= \frac{1}{2\pi i} \oint_{\Gamma}^{} \frac{f\left( \xi \right)}{\xi - z}\:d\xi
  \end{align*}
\end{theorem}
\begin{theorem}[Cauchy's Integral Theorem]
  Let $U\subseteq \C$ be open, $f\colon U\rightarrow \C$ holomorphic, and $\Gamma$ homologous to zero in $U$. Then,
  \begin{align*}
    \oint_{\Gamma}^{} f(z)\:dz &= 0.
  \end{align*}
\end{theorem}
\begin{definition}
  A region $U\subseteq \C$ is called \textit{simply connected} if its complement in the extended complex plane is connected.
\end{definition}
\begin{theorem}
  If $U\subseteq \C$ is simply connected, then every loop in $U$ is homologous to zero.
\end{theorem}
\begin{proof}
  Extend the function $n\left( \gamma;\cdot \right)$ to the extended complex plane by defining $n\left( \gamma;\infty \right) = 0$. This extended function is continuous on $\hat{\C}\setminus U$, as $n\left( \gamma;\cdot \right)$ is zero on the unique unbounded connected component of $\C\setminus \img\left( \gamma \right)$. It follows that $n\left( \gamma;z \right)$ is equal to zero on $\hat{\C}\setminus U$, whence $\gamma$ is homologous to zero in $U$.
\end{proof}
\begin{proposition}
  Let $U\subseteq \C$ be a region, $f\colon U\rightarrow \C$ holomorphic. The following are equivalent:
  \begin{enumerate}[(i)]
    \item there exists a holomorphic function $F\colon U\rightarrow \C$ such that $F'(z) = f(z)$;
    \item for every piecewise $C^{1}$ loop $\gamma$ with $\img\left( \gamma \right)\subseteq U$, we have
      \begin{align*}
        \oint_{\gamma}^{} f(z)\:dz &= 0.
      \end{align*}
  \end{enumerate}
\end{proposition}
\begin{proof}
  The direction (i) $\Rightarrow$ (ii) follows immediately from the fundamental theorem of calculus. In the reverse direction, we define $F\colon U\rightarrow \C$ by
  \begin{align*}
    f(z) &= \int_{\sigma\left( z_0,z \right)}^{} f\left( \xi \right)\:d\xi,
  \end{align*}
  where $\sigma\left( z_0,z \right)\colon [0,1]\rightarrow U$ is a piecewise $C^{1}$ curve with $\sigma(0) = z_0$ and $\sigma(1) = z$. Such a curve always exists as $U$ is open and connected (hence path-connected). The integral is well-defined, since if $\gamma_1$ and $\gamma_2$ are any two such paths, then $\Gamma = \gamma_1\setminus \gamma_2$ is a piecewise $C^1$ loop. Additionally, $F$ is continuous.%\newline

  Now, we evaluate the derivative of $F$. Let $z\in U$, $r > 0$ such that $U\left( z,r \right)\subseteq U$, and $h\in \C$ be such that $z + h\in U\left( z,r \right)$. Then,
  \begin{align*}
    \frac{F\left( z+h \right) - F\left( z \right)}{h} &= \frac{1}{h} \int_{\sigma\left( z_0,z_0 + h \right)}^{} f\left( \xi \right)\:d\xi - \frac{1}{h} \int_{\sigma\left( z_0,z \right)}^{} f\left( \xi \right)\:d\xi\\
                                                      &= \frac{1}{h} \int_{\sigma\left( z,z+h \right)}^{} f\left( \xi \right)\:d\xi.
  \end{align*}
  We may assume that $\sigma\left( z,z+h \right)$ is a straight line, so that
  \begin{align*}
    \int_{\sigma\left( z,z+h \right)}^{} f\left( \xi \right)\:d\xi &= hf(z),
  \end{align*}
  meaning that
  \begin{align*}
    \left\vert \frac{F\left( z+h \right)-F\left( z \right)}{h} - f(z) \right\vert &= \frac{1}{\left\vert h \right\vert} \left\vert \int_{\sigma\left( z,z+h \right)}^{} f(\xi)\:d\xi - f\left( z \right) \right\vert\\
                                                                                  &\leq \sup_{w\in \img\left( \sigma\left( z,z+h \right) \right)} \left\vert f(w)-f(z) \right\vert.
  \end{align*}
  Since $f$ is continuous, it follows that the right hand side goes to zero as $\left\vert h \right\vert$ becomes small. Thus, $F'$ is continuous, so $f$ is holomorphic.
\end{proof}
Observe that $\C\setminus \set{0}$ is not simply connected, meaning that, for instance, the function
\begin{align*}
  f(z) &= \frac{1}{z}
\end{align*}
does not have a holomorphic antiderivative defined on the entirety $\C\setminus \set{0}$, as
\begin{align*}
  \int_{S^1}^{} f(z)\:dz &= 2\pi i.
\end{align*}
Yet, if we restrict $f(z)$ to a simply connected subset of $\C$, there \textit{is} a holomorphic antiderivative. Choosing such a simply connected subset of $\C$ is known as choosing a \textit{branch} of the logarithm. In fact, there is more that we can say.
\begin{corollary}
  Let $U\subseteq \C$ be simply connected, and let $f\colon U\rightarrow \C\setminus \set{0}$ be a nonvanishing holomorphic function. For each fixed pair $z_0\in U$ and $w_0\in \C$ for which $e^{w_0} = f\left( z_0 \right)$, there exists a unique holomorphic function $g\colon U\rightarrow \C$ for which $g\left(z_0\right) = w_0$ and $e^{g(z)} = f(z)$.%\newline

  We call $g$ the logarithm of $f$, written $g(z) = \log\left( f(z) \right)$.
\end{corollary}
\begin{proof}
  Since $f$ is nonvanishing and $U$ is simply connected, it follows that $\frac{f'}{f}$ is holomorphic on $U$, meaning there is $ \widetilde{g}\colon U\rightarrow \C $ such that $\widetilde{g}'(z) = \frac{f'(z)}{f(z)}$. Thus, there is some constant $K$ such that
  \begin{align*}
    f(z) &= Ke^{\widetilde{g}(z)}.
  \end{align*}
  Define
  \begin{align*}
    g(z) &= \log(K) + \widetilde{g}(z).
  \end{align*}
\end{proof}
\begin{theorem}[Morera's Theorem]
  Let $U\subseteq \C$ be open, $f\colon U\rightarrow \C$ continuous. Suppose
  \begin{align*}
    \oint_{T}^{} f(z)\:dz &= 0
  \end{align*}
  for all triangles $T\subseteq U$ homologous to zero. Then, $f$ is holomorphic.
\end{theorem}
\begin{proof}
  Since $U$ is open, if $z_0\in U$, there is $r$ such that $U\left( z_0,r \right)\subseteq U$. Define $F\colon U\left( z_0,r \right)\rightarrow \C$ by
  \begin{align*}
    F(z) &= \int_{\sigma\left( z_0,z \right)}^{} f\left( \xi \right)\:d\xi,
  \end{align*}
  where $\sigma$ is the straight line from $z_0$ to $z$. For $0 < \left\vert h \right\vert < r-\left\vert z-z_0 \right\vert$, we construct the straight lines $\sigma\left( z,z+h \right)$ and $\sigma\left( z_0,z+h \right)$, such that
  \begin{align*}
    T &= \sigma\left( z_0,z \right) + \sigma\left( z,z+h \right) - \sigma\left( z_0,z+h \right),
  \end{align*}
  and
  \begin{align*}
    \oint_{T}^{} f(z)\:dz &= 0\\
                          &= \int_{\sigma\left( z_0,z \right)}^{} f\left( \xi \right)\:d\xi + \int_{\sigma\left( z,z+h \right)}^{} f\left( \xi \right)\:d\xi - \int_{\sigma\left( z_0,z+h \right)}^{} f\left( \xi \right)\:d\xi\\
                          &= F(z) - F\left( z+h \right) + \int_{\sigma\left( z,z+h \right)}^{} f\left( \xi \right)\:d\xi,
                          \intertext{meaning}
    F\left( z+h \right) - F\left( z \right) &= \int_{\sigma\left( z,z+h \right)}^{} f\left( \xi \right)\:d\xi\\
    \frac{F\left( z+h \right) - F\left( z \right)}{h} &= \frac{1}{h} \int_{\sigma\left( z,z+h \right)}^{} f\left( \xi \right)\:d\xi\\
    \left\vert \frac{F\left( z+h \right) - F\left( z \right)}{h} - f(z) \right\vert &= \left\vert \frac{1}{h}\int_{\sigma\left( z,z+h \right)}^{} \left( f(\xi) - f(z) \right)\:d\xi \right\vert\\
                                                                                    &\leq \frac{1}{\left\vert h \right\vert}\left\vert h \right\vert\sup_{w\in \img\left( \sigma\left( z,z+h \right) \right)} \left\vert f(w) - f(z) \right\vert\\
                                                                                    &= \sup_{w\in \img\left( \sigma\left( z,z+h \right) \right)} \left\vert f(w)-f(z) \right\vert.
  \end{align*}
  Since $f$ is continuous, it follows that for sufficiently small $\left\vert h \right\vert$, the right-hand-side goes to zero, whence $F'(z) = f(z)$, meaning $F$ is holomorphic, so $F$ is analytic, meaning $f$ is analytic, so $f$ is holomorphic.
\end{proof}
\begin{corollary}
  Let $U\subseteq \C$ be open, $\gamma\colon \left[ a,b \right]\rightarrow U$ a piecewise $C^{1}$ curve, and $g\colon U\times \img\left( \gamma \right)\rightarrow \C$ continuous. Suppose that for each $w\in \img\left( \gamma \right)$, the function $g\left( \cdot,w \right)$ is holomorphic. Then,
  \begin{align*}
    f(z)\coloneq \int_{\gamma}^{} g\left( z,w \right)\:dw
  \end{align*}
  is holomorphic.
\end{corollary}
\begin{proof}
  Let $T$ be a triangle in $U$ homologous to zero. Then, by Fubini's Theorem,
  \begin{align*}
    \oint_{T}^{} f(z)\:dz &= \oint_{T}^{} \int_{\gamma}^{} g\left( z,w \right)\:dw\:dz\\
                          &= \int_{\gamma}^{} \oint_{T}^{} g\left( z,w \right)\:dz\:dw.
  \end{align*}
  The interior integral vanishes for every $w$ as $g\left( \cdot,w \right)$ is holomorphic. Thus, $f$ is holomorphic.
\end{proof}
\begin{definition}
  Let $U\subseteq \C$ be open, $\gamma_1,\gamma_2$ piecewise $C^{1}$ loops in $U$. We say $\gamma_1$ and $\gamma_2$ are homotopic in $U$ if there is a continuous function
  \begin{align*}
    H\colon [a,b]\times [0,1]\rightarrow U
  \end{align*}
  such that
  \begin{align*}
    H\left( s,0 \right) &= \gamma_1(s)\\
    H\left( s,1 \right) &= \gamma_2(s)\\
    H\left( a,t \right) &= H\left( b,t \right).
  \end{align*}
  For each $t$, $H\left( \cdot,t \right)$ is a continuous loop. We call $H$ a homotopy between $\gamma_0$ and $\gamma_1$.
\end{definition}
\begin{theorem}
  If $\gamma_0$ and $\gamma_1$ are homotopic in $U$, then $\Gamma = \gamma_1-\gamma_0$ is homologous to zero in $U$.
\end{theorem}
\begin{theorem}
  If $K\subseteq U$ is compact and $U$ is connected, then there is some cycle $\Gamma$ homologous to zero in $U$ such that $n\left( \Gamma;z \right) = 1$ for all $z\in K$.
\end{theorem}
\begin{corollary}
  Let $U$ be a region. The following are equivalent:
  \begin{enumerate}[(i)]
    \item $U$ is simply connected;
    \item for every nonvanishing holomorphic function $f\colon U\rightarrow \C\setminus \set{0}$, there is a holomorphic function $g\colon U\rightarrow \C$ such that $f(z) = e^{g(z)}$;
    \item for all cycles $\Gamma$ with $\img\left( \Gamma \right)\subseteq U$, $\Gamma$ is homologous to zero in $U$.
  \end{enumerate}
\end{corollary}
\subsection{Maximum Modulus Principle}%
\begin{theorem}[Mean Value Property]
  Let $U\subseteq \C$ be open, $f\colon U\rightarrow \C$ holomorphic, with $z_0\in U$ and $r > 0$ such that $B\left( z_0,r \right)\subseteq U$. Then,
  \begin{align*}
    f\left( z_0 \right) &= \frac{1}{2\pi} \int_{0}^{2\pi} f\left( z_0 + re^{i\theta} \right)\:d\theta.
  \end{align*}
\end{theorem}
\begin{proof}
  By the Cauchy Integral Formula, we have
  \begin{align*}
    f\left( z_0 \right) &= \frac{1}{2\pi i} \int_{S\left( z_0,r \right)}^{} \frac{f\left( \xi \right)}{\xi - z}\:d\xi.
  \end{align*}
  Parametrizing $\gamma\left( \theta \right) = z_0 + re^{i\theta}$, we get
  \begin{align*}
    f\left( z_0 \right) &= \frac{1}{2\pi i} \int_{0}^{2\pi} \frac{f\left( z_0 + re^{i\theta} \right)}{re^{i\theta}}ire^{i\theta}\:d\theta\\
                        &= \frac{1}{2\pi} \int_{0}^{2\pi} f\left( z_0 + re^{i\theta} \right)\:d\theta.
  \end{align*}
\end{proof}
\begin{corollary}
  If $u\colon \R^{2}\supseteq U\rightarrow \R$ is harmonic, $\left( x_0,y_0 \right)\in U$, and $r > 0$ is such that $B\left( \left( x_0,y_0 \right),r \right)\subseteq U$, then
  \begin{align*}
    u\left( x_0,y_0 \right) &= \frac{1}{2\pi} \int_{0}^{2\pi} u\left( x_0 + r\cos\left( \theta \right),y_0 + r\sin\left( \theta \right) \right)\:d\theta.
  \end{align*}
\end{corollary}
\begin{proof}
  Take real parts of the mean value property for holomorphic $f = u + iv$.
\end{proof}
Observe then that the triangle inequality implies that
\begin{align*}
  \left\vert u\left( x_0,y_0 \right) \right\vert &\leq \frac{1}{2\pi} \int_{0}^{2\pi} \left\vert u\left( x_0 + r\cos\left( \theta \right),y_0 + r\sin\left( \theta \right) \right) \right\vert\:d\theta.
\end{align*}
Functions that satisfy this weaker criterion are known as \textit{subharmonic}. It is subharmonic functions for which the most general case of the \textit{maximum modulus principle} hold.
\begin{theorem}[Maximum Modulus Principle]
  Let $U\subseteq \R^{2}$ be open and connected, and let $u\colon U\rightarrow \R$ be subharmonic. Suppose there exists $\left( x_0,y_0 \right)\in U$ such that $u\left( x_0,y_0 \right)\geq u\left( x,y \right)$ for all $x,y\in U$. Then, $u$ is constant.
\end{theorem}
\begin{proof}
  Let $\lambda = u\left( x_0,y_0 \right)$, and let $E = \set{\left( x,y \right) | u\left( x,y \right) = \lambda} = u^{-1}\left( \set{\lambda} \right)$. We see immediately that $E$ is closed; we claim that $E$ is also open.%\newline

  Fix $\left( x_1,y_1 \right)\in E$. Then, $u\left( x_1,y_1 \right) = \lambda$. Take $r > 0$ such that $U\left( \left( x_1,y_1 \right),r \right)\subseteq U$. Then, for all $0 < s < r$, we have $S\left( \left( x_1,y_1 \right),s \right)\subseteq U$, meaning that
  \begin{align*}
    \lambda &= u\left( x_1,y_1 \right) \\
            &\leq \frac{1}{2\pi} \int_{0}^{2\pi} u\left( x_1 + s\cos\left( \theta \right),y_1+s\sin\left( \theta \right) \right)\:d\theta\\
            &\leq \lambda,
  \end{align*}
  with the latter inequality following from the fact that $\lambda$ is a local maximum. Therefore, $u\left( x_1 + s\cos\left( \theta \right),y_1 + s\sin\left( \theta \right) \right) = \lambda$ for all $0 < s < r$, whence $U\left( \left( x_1,y_1 \right),r \right)\subseteq E$. Thus, $E$ is open, so since $U$ is connected, it follows that $E$ is all of $U$, meaning $u$ is constant.
\end{proof}
\begin{corollary}
  If $U\subseteq \R^2$ is bounded and $u\colon \overline{U}\rightarrow \R$ is continuous with $u|_{U}$ subharmonic, then there exists $\left( x_0,y_0 \right)\in \partial U$ such that $u\left( x_0,y_0 \right) = \sup_{\left( x,y \right)\in U} u\left( x,y \right)$.
\end{corollary}
\begin{corollary}
  If $U\subseteq \C$ is open and connected, with $f\colon U\rightarrow \C$ holomorphic, then if $\left\vert f \right\vert\colon U\rightarrow \R$ has a local maximum at $z_0\in U$, then $f$ is constant.
\end{corollary}
\begin{proof}
  Let $r > 0$ be such that $U\left( z_0,r \right)\subseteq U$. Then, restricting $\left\vert f \right\vert$ to $U\left( z_0,r \right)$, we see that $\left\vert f \right\vert$ restricted to $U\left( z_0,r \right)$ is subharmonic viewed as a function on $U\left( z_0,r \right)$, hence $\left\vert f \right\vert$ is constant on $U\left( z_0,r \right)$.%\newline

  Now, by the mean value property and triangle inequality, it follows that for all $0 < s < r$, we have
  \begin{align*}
    \left\vert f\left( z_0 \right) \right\vert &\leq \frac{1}{2\pi} \int_{0}^{2\pi} \left\vert f\left( z_0 + se^{i\theta} \right) \right\vert\:d\theta\\
                                               &= \left\vert f\left( z_0 \right) \right\vert,
  \end{align*}
  meaning that these are equalities. In particular, there exists some $\theta_s$ such that $e^{i\theta_s} f\left( z_0 + se^{i\theta} \right) \geq 0$, meaning that for this value of $s$, we have
  \begin{align*}
    \left\vert f\left( z_0 \right) \right\vert &= e^{i\theta_s} \int_{0}^{2\pi} f\left( z_0 + se^{i\theta} \right)\:d\theta\\
                                               &= e^{i\theta_s} f\left( z_0 \right),
  \end{align*}
  with the latter equality following from the mean value property. Since this holds for any $s$, it follows that $\theta_s$ is independent of $s$, meaning that $f(z)e^{i\theta_s} \geq 0$ for all $z\in U\left( z_0,r \right)$, meaning that $\im\left( e^{i\theta_s}f\left( z \right) \right) = 0$ on $U\left( z_0,r \right)$, whence $f(z)e^{i\theta_s}$ is constant, meaning $f$ is constant on $U\left( z_0,r \right)$.%\newline

  Finally, by the identity theorem, it follows that $f$ is constant on $U$.
\end{proof}
\begin{definition}
  Let $U\subseteq \R^{2}$ be an open set. We say a sequence $U\supseteq \left( \left( x_n,y_n \right) \right)_n\rightarrow \partial U$ if, for every compact $K\subseteq U$, the set $\set{n\in \N | \left( x_n,y_n \right)\in K}$ is finite.
\end{definition}
\begin{definition}
  Let $U\subseteq \R^{2}$ be an open set. Given a function $u\colon U\rightarrow \R$, define
  \begin{align*}
    \limsup_{\left( x,y \right)\rightarrow \partial U} u\left( x,y \right) &\coloneq \inf_{\substack{K\subseteq U\\K\text{ compact}}} \sup_{\left( x,y \right)\in U\setminus K} u\left( x,y \right).
  \end{align*}
\end{definition}
These definitions allow us to extend the maximum modulus principle for subharmonic functions even further.
\begin{theorem}
  Let $U\subseteq \C$ be a region, $u\colon U\rightarrow \R$ a nonconstant subharmonic function. If $\left( \left( x_n,y_n \right) \right)_n\subseteq U$ is such that $u\left( x_n,y_n \right)\rightarrow \sup_{x,y\in U}u\left( x,y \right)$, then $\left( \left( x_n,y_n \right) \right)_n\rightarrow \partial U$. Moreover, $\limsup_{\left( x,y \right)\rightarrow \partial U}u\left( x,y \right) = \sup_{\left( x,y \right)\in U} u\left( x,y \right)$.
\end{theorem}
\begin{proof}
  Suppose toward contradiction that $\left( \left( x_n,y_n \right) \right)_n\nrightarrow \partial U$, so there exists a compact subset $K\subseteq U$ and a subset $\left( \left( x_{n_k},y_{n_k} \right) \right)_k$ wholly contained in $K$. Since $K$ is compact, there is a subsequence of $\left( \left( x_{n_k},y_{n_k} \right) \right)_k$ converging to $\left( x_0,y_0 \right)\in U$. Therefore, $u\left( x_0,y_0 \right) = \sup_{\left( x,y \right)\in U}u\left( x,y \right)$, so $u$ is constant by the maximum modulus principle, which is a contradiction.%\newline

  Finally, $\limsup_{\left( x,y \right)\rightarrow \partial U}u\left( x,y \right)\leq \sup_{\left( x,y \right)\in U}u\left( x,y \right)$, while if $\left( \left( x_n,y_n \right) \right)_n\rightarrow \partial U$ is such that $u\left( x_n,y_n \right)$ converges to $\sup_{\left( x,y \right)\in U}u\left( x,y \right)$, then $\sup_{\left( x,y \right)\in U}u\left( x,y \right) = \lim_{n\rightarrow\infty}u\left( x_n,y_n \right) \leq \limsup_{\left( x,y \right)\rightarrow \partial U}u\left( x,y \right)$.
\end{proof}
\begin{theorem}[Open Mapping Principle]
  Let $U\subseteq \C$ be a region, and let $f\colon U\rightarrow \C$ be a nonconstant holomorphic function. Then, $f\left( V \right)\subseteq \C$ is open.
\end{theorem}
\begin{proof}
  Let $z_0\in U$ and $r > 0$ be such that $ B\left( z_0,r \right)\subseteq U $. We will show that there exists $R$ such that $U\left( f\left( z_0 \right),R \right)\subseteq f\left( U\left( z_0,r \right) \right)\subseteq U$, whence $f(U)$ is open.%\newline

  Since $U$ is a region and $f$ is nonconstant, the zeros of $g(z)\coloneq f(z) - f\left( z_0 \right)$ are isolated, so there exists some $0 < s < r$ such that
  \begin{align*}
    \delta &= \inf_{\left\vert z-z_0 \right\vert = s} \left\vert f(z)-f\left( z_0 \right) \right\vert
  \end{align*}
  is strictly greater than zero. We claim that $U\left( f\left( z_0 \right),\delta/2 \right) \subseteq f\left( U\left( z_0,r \right) \right)$. Suppose this were not the case, meaning there would be some $\xi\in U\left( f\left( z_0 \right),\delta/2 \right)\setminus f\left( U\left( z_0,r \right) \right)$, and define $ h\colon B\left( z_0,s \right)\rightarrow \C $ by
  \begin{align*}
    h(z) &= \frac{1}{f\left( z \right) - \xi}.
  \end{align*}
  Since $\xi\notin f\left( U\left( z_0,r \right) \right)$, this is holomorphic, while $\xi\in U\left( f\left( z_0 \right),\delta/2 \right)$ implies
  \begin{align*}
    \sup_{\left\vert z-z_0 \right\vert = s} \left\vert h(z) \right\vert &= \sup_{\left\vert z-z_0 \right\vert = s} \frac{1}{\left\vert f\left( z \right) - \xi \right\vert}\\
                                                                        &\leq \sup_{\left\vert z-z_0 \right\vert = s} \frac{1}{\left\vert f\left( z \right)-f\left( z_0 \right) \right\vert - \left\vert f\left( z_0 \right) - \xi \right\vert}\\
                                                                        &\leq \frac{1}{\delta - \delta/2}\\
                                                                        &= \frac{2}{\delta}.
  \end{align*}
  Yet,
  \begin{align*}
    \left\vert h\left( z_0 \right) \right\vert &= \frac{1}{\left\vert f\left( z_0 \right) - \xi \right\vert}\\
                                               &> \frac{2}{\delta},
  \end{align*}
  contradicting the maximum modulus principle. Thus, $U\left( f\left( z_0 \right),\delta/2 \right)\subseteq f\left( U\left( z_0,r \right) \right)$.
\end{proof}
In the proof of the Hadamard Three-Lines Theorem, we used the function $h_{\ve}(z) = \frac{1}{ 1 + \ve (z-a) }$ for this purpose.
\subsection{Classification of Singularities}%
The classification of singularities seeks to answer two fundamental questions: if $U\subseteq \C$ is open, $z_0\in U$, and $f\colon U\setminus \set{z_0}\rightarrow \C$ is holomorphic, 
\begin{itemize}
  \item does $f$ have a holomorphic extension to $U$ including $z_0$;
  \item and what else can we say about the behavior of $f$ at $z_0$?
\end{itemize}
\begin{definition}
  Let $U\subseteq \C$ be open, $z_0\in U$, $f\colon U\setminus \set{z_0}\rightarrow \C$ holomorphic.
  \begin{itemize}
    \item If there exists a holomorphic $g\colon U\rightarrow \C$ with $g = f$ on $U\setminus \set{z_0}$, then we say $z_0$ is a \textit{removable singularity}.
    \item If $\lim_{z\rightarrow z_0} \left\vert f(z) \right\vert = \infty$, then we say $f$ has a \textit{pole} at $z_0$.
    \item Else, we say $f$ has an \textit{essential singularity} at $z_0$.
  \end{itemize}
\end{definition}
\begin{theorem}[Riemann's Theorem on Removable Singularities]
  Let $U\subseteq \C$ be open, $z_0\in U$, and $f\colon U\setminus \set{z_0}\rightarrow \C$ holomorphic. Then, $z_0$ is a removable singularity if and only if $\lim_{z\rightarrow z_0}f(z) = 0$.
\end{theorem}
\begin{proof}
  If $z_0$ is removable, then $g(z)$ is a holomorphic function with $g(z) = f(z)$ on $U\setminus \set{z_0}$, and since $g$ is continuous, it follows that $\lim_{z\rightarrow z_0}g(z) = g\left(z_0\right)$, whence $\lim_{z\rightarrow z_0}\left( z-z_0 \right)g(z) = \lim_{z\rightarrow z_0}\left( z-z_0 \right)f(z) = 0$.%\newline

  Now, if $\lim_{z\rightarrow z_0}\left( z-z_0 \right)f(z) = 0$, then there is $r$ such that $B\left( z_0,r \right)\subseteq U$, and since $f$ is locally bounded around $z_0$, it follows that
  \begin{align*}
    f(z) &= \frac{1}{2\pi i} \int_{S\left( z_0,r \right)}^{} \frac{f\left( \zeta \right)}{\zeta - z}\:d\zeta
  \end{align*}
  holds for all $z\in \dot{U}\left( z_0,r \right)$. Yet, the formula extends to $z_0$ as it is bounded, whence we may define the holomorphic extension for $f$ by
  \begin{align*}
    g\left( z \right) &= \begin{cases}
      f(z) & z\neq z_0\\
      \frac{1}{2\pi i} \int_{S\left( z_0,r \right)}^{} \frac{g\left( \zeta \right)}{\zeta - z}\:d\zeta & z= z_0
    \end{cases}.
  \end{align*}
\end{proof}
\begin{proposition}[Existence of Laurent Series]
  Suppose $f\colon A\left( z_0,r,R \right)\rightarrow \C$ is holomorphic, with $0 \leq r < R$. Then, there exist holomorphic functions
  \begin{align*}
    g_1\colon U\left( z_0,R \right)\rightarrow \C\\
    g_2\colon \C\setminus B\left( z_0,r \right)\rightarrow \C
  \end{align*}
  such that $f = g_1 + g_2$ on $A\left( z_0,r,R \right)$. Moreover, there exists $\left( a_n \right)_{n\in\Z}\subseteq \C$ such that
  \begin{align*}
    f(z) &= \sum_{n=-\infty}^{\infty} a_n\left( z-z_0 \right)^{n}
  \end{align*}
  for all $z$, and the series converges uniformly on $A\left( z_0,\rho,s \right)$ with $r < \rho < s < R$.
\end{proposition}
\begin{proof}
  Fix $z\in A\left( z_0,r,R \right)$. Then, for $r < \rho_1,\rho_2 < \left\vert z-z_0 \right\vert$, the cycle
  \begin{align*}
    \Gamma_1 &= S\left( z_0,\rho_1 \right) - S\left( z_0,\rho_2 \right)
  \end{align*}
  is homologous to zero in $A\left( z_0,r,\left\vert z-z_0 \right\vert \right)$. By Cauchy's Integral Theorem, it then follows that
  \begin{align*}
    \oint_{S\left( z_0,\rho_1 \right)}^{} \frac{f\left( \xi \right)}{\xi-z}\:d\xi &= \oint_{S\left( z_0,\rho_2 \right)}^{} \frac{f\left( \xi \right)}{\xi-z}\:d\xi.
  \end{align*}
  Similarly, for $\left\vert z-z_0 \right\vert < s_1,s_2 < R$, we have
  \begin{align*}
    \oint_{S\left( z_0,s_1 \right)}^{} \frac{f\left( \xi \right)}{\xi-z}\:d\xi &= \oint_{S\left( z_0,s_2 \right)}^{} \frac{f\left( \xi \right)}{\xi-z}\:d\xi.
  \end{align*}
  Define $g_1\colon U\left( z_0,R \right)\rightarrow \C$ by
  \begin{align*}
    g_1(z) &= \frac{1}{2\pi i} \oint_{S\left( z_0,s \right)}^{} \frac{f\left( \xi \right)}{\xi - z}\:d\xi,
  \end{align*}
  where $ \left\vert z-z_0 \right\vert < s < R $. This function is holomorphic by Morera's Theorem. Similarly, we may define $g\colon \C\setminus B\left( z_0,r \right)\rightarrow \C$ by
  \begin{align*}
    g_2\left( z \right) &= -\frac{1}{2\pi i} \oint_{S\left( z_0,\rho \right)}^{} \frac{f\left( \xi \right)}{\xi-z}\:d\xi,
  \end{align*}
  where $r < \rho < \left\vert z-z_0 \right\vert$. We claim that $f = g_1 + g_2$ on $A\left( z_0,r,R \right)$.%\newline

  For $z\in A\left( z_0,r,R \right)$, we may find, for any $r < \rho < \left\vert z-z_0 \right\vert < s < R$, we let
  \begin{align*}
    \Gamma &= S\left( z_0,s \right) - S\left( z_0,\rho \right),
  \end{align*}
  homologous to zero in $A\left( z_0,r,R \right)$, whence
  \begin{align*}
    f\left( z \right) &= \frac{1}{2\pi i} \left( \oint_{S\left( z_0,s \right)}^{} \frac{f\left( \xi \right)}{\xi-z}\:d\xi - \int_{S\left( z_0,\rho \right)}^{} \frac{f\left( \xi \right)}{\xi - z}\:d\xi \right)\\
                      &= g_1\left( z \right) + g_2\left( z \right).
  \end{align*}
\end{proof}
\begin{theorem}
  Let $U\subseteq \C$, $f\colon U\setminus \set{z_0}\rightarrow \C$ be holomorphic with Laurent series
  \begin{align*}
    f(z) &= \sum_{n=-\infty}^{\infty}a_n\left( z-z_0 \right)^{n}
  \end{align*}
  on $\dot{U}\left( z_0,R \right)$ for some $R$ with $U\left( z_0,R \right)\subseteq U$. Then,
  \begin{enumerate}[(i)]
    \item $f$ has a removable singularity at $z_0$ if and only if $a_n = 0$ for all $n < 0$;
    \item $f$ has a pole at $z_0$ if and only if 
      \begin{align*}
        1 \leq \left\vert \set{n < 0 | a_n\neq 0} \right\vert <\infty.
      \end{align*}
    \item $f$ has an essential singularity at $z_0$ if and only if
      \begin{align*}
        \left\vert \set{n < 0 | a_n\neq 0} \right\vert &= \infty.
      \end{align*}
  \end{enumerate}
\end{theorem}
\begin{proof}\hfill
  \begin{enumerate}[(i)]
    \item If $a_n = 0$ for all $n < 0$, then $\lim_{z\rightarrow z_0} \left( z-z_0 \right)f(z) = 0$, so $f$ has a removable singularity at $z_0$.%\newline

      Conversely, if $f$ has a removable singularity at $z_0$, then for $n < 0$, we have
      \begin{align*}
        a_n &= \frac{1}{2\pi i} \oint_{S\left( z_0,\rho \right)}^{} \frac{f\left( \xi \right)}{\left( \xi - z_0 \right)^{n+1}}\:d\xi
      \end{align*}
      for any $0 < \rho < R$. Since $\lim_{z\rightarrow z_0}\left( z-z_0 \right)f(z) = 0$, then for any $\ve > 0$, there is sufficiently small $\rho$ such that
      \begin{align*}
        \left\vert a_n \right\vert &= \left\vert \frac{1}{2\pi i} \oint_{S\left( z_0,\rho \right)}^{} \frac{f\left( \xi \right)}{\left( \xi-z_0 \right)^{n+1}}\:d\xi \right\vert\\
                                   &\leq \rho^{-1-n} \sup_{\left\vert \xi-z_0 \right\vert = \rho}\left\vert \left( \xi-z_0 \right)f(z) \right\vert\\
                                   &\leq \ve.
      \end{align*}
      Thus, $\left\vert a_n \right\vert = 0$ for all $n < 0$.
    \item If $a_n\neq 0$ for a nonempty finite collection of $n < 0$, we let $m$ be the largest number such that $a_{-m} < 0$, so that $f(z) = \sum_{n=-m}^{\infty}a_n\left( z-z_0 \right)^{n}$. It follows that $\lim_{z\rightarrow z_0}\left( z-z_0 \right)^{m}f(z) = a_{-m}\neq 0$. In particular, there is some small $\delta$ such that
      \begin{align*}
        \left\vert \left( z-z_0 \right)^{m}f(z) - a_{-m} \right\vert &< \frac{\left\vert a_{-m} \right\vert}{2},
      \end{align*}
      whence for a sufficiently small $\ve$,
      \begin{align*}
        \left\vert f(z) \right\vert &> \frac{\left\vert a_{-m} \right\vert-\left\vert \left( z-z_0 \right)^{m}f(z)-a_{-m} \right\vert}{\left\vert z-z_0 \right\vert^{m}}\\
                                    &> \frac{\left\vert a_{-m} \right\vert}{2\left\vert z-z_0 \right\vert^{m}}\\
                                    &> \frac{1}{\ve},
      \end{align*}
      so $f$ has a pole at $z_0$.%\newline

      Conversely, if $f$ has a pole at $z_0$, there is some $r > 0$ such that $\left\vert f(z) \right\vert \geq 1$ whenever $z\in \dot{U}\left( z_0,r \right)$. Define $g\colon \dot{U}\left( z_0,r \right)\rightarrow \C$ by
      \begin{align*}
        g(z) &= \frac{1}{f(z)},
      \end{align*}
      whence $g$ is holomorphic and bounded on $\dot{U}\left( z_0,r \right)$. By the classification of singularities, $z_0$ is a removable singularity of $g$, so there exists a holomorphic function $h\colon U\left( z_0,r \right)\rightarrow \C$ that is equal to $ \frac{1}{f(z)} $ for $z\neq z_0$ and $\lim_{z\rightarrow z_0}\frac{1}{f(z)}$ if $z= z_0$. We write
      \begin{align*}
        h(z) &= \sum_{n=0}^{\infty} b_n\left( z-z_0 \right)^{n},
      \end{align*}
      where we must have $b_0 = 0$. Let $m$ be the smallest positive integer such that $h(z) = \left( z-z_0 \right)^{m}\widetilde{h}(z)$, where $\widetilde{h}(z) = \sum_{n=0}^{\infty}b_{n+m}\left( z-z_0 \right)^{n}$ and $b_m\neq 0$. The function $\widetilde{h}$ is holomorphic on $U\left( z_0,r \right)$ and nonzero on some $U\left( z_0,\rho \right)$ with $0 < \rho \leq r$, so that $ \widetilde{f}(z) = \frac{1}{\widetilde{h}(z)} $ is holomorphic on $U\left( z_0,\rho \right)$, so
      \begin{align*}
        f(z) &= \left( z-z_0 \right)^{-m}\widetilde{f}(z)
      \end{align*}
      on $\dot{U}\left( z_0,\rho \right)$. Writing
      \begin{align*}
        \widetilde{f}(z) &= \sum_{n=0}^{\infty}c_n\left( z-z_0 \right)^{n},
      \end{align*}
      we deduce that
      \begin{align*}
        f(z) &= \sum_{n=-m}^{\infty}c_{n+m}\left( z-z_0 \right)^{n}.
      \end{align*}
    \item Follows from (i) and (ii).
  \end{enumerate}
\end{proof}
\begin{definition}
  Let $U\subseteq \C$ be open.
  \begin{itemize}
    \item If $f\colon U\setminus \set{z_0}\rightarrow \C$ has a pole or a removable singularity at $z_0$, then the order of the pole or singularity is the smallest $m \geq 0$ such that
      \begin{align*}
        f(z) &= \left( z-z_0 \right)^{-m} g(z)
      \end{align*}
      with $g\colon U\rightarrow \C$ is holomorphic and has $g\left( z_0 \right) \neq 0$.
    \item If $f\colon U\rightarrow \C$ has a zero at $z_0\in U$, then the order of the zero at $z_0$ is the smallest $m \geq 0$ such that
      \begin{align*}
        f(z) &= \left( z-z_0 \right)^{m}g(z)
      \end{align*}
      where $g\colon U\rightarrow \C$ is holomorphic and has $g\left( z_0 \right) \neq 0$.
  \end{itemize}
  If $z_0\in U$ is either a pole or a zero (or a removable singularity that is a zero when $f$ is extended), then the \textit{order} of $f$ is the unique $m\in\Z$ such that
  \begin{align*}
    f(z)  &= \left( z-z_0 \right)^{m}g(z)
  \end{align*}
  with $g\left(z_0\right)\neq 0$.
\end{definition}
\begin{theorem}[Casorati--Weierstrass]
  Let $U\subseteq \C$ be an open set, and let $f\colon U\setminus \set{z_0}\rightarrow \C$ be a holomorphic function. If $z_0$ is an essential singularity of $f$, then for any $r > 0$ with $U\left( z_0,r \right)\subseteq U$, $f\left( \dot{U}\left( z_0,r \right) \right)\subseteq \C$ is dense.
\end{theorem}
\begin{proof}
  Suppose $f\left( \dot{U}\left( z_0,r \right) \right)$ is not dense in $\C$. Then, there exists $\ve > 0$ and $w\in \C$ such that $\left\vert f(z) - w \right\vert \geq \ve$ for all $z\in \dot{U}\left( z_0,r \right)$. We will show that $z_0$ is either removable or a pole.%\newline

  Define
  \begin{align*}
    g\colon \dot{U}\left( z_0,r \right)\rightarrow \C
  \end{align*}
  by
  \begin{align*}
    g(z) &= \frac{1}{f(z) - w}.
  \end{align*}
  Observe that $g$ is definitionally bounded, so $z_0$ is a removable singularity of $g$. Thus, there is a holomorphic function $h\colon U\left( z_0,r \right)\rightarrow \C$ such that $h(z) = g(z)$ for $z\neq z_0$ and $h(z) = \lim_{z\rightarrow z_0}g(z)$. We write
  \begin{align*}
    h(z) &= \sum_{n=0}^{\infty}b_n\left( z-z_0 \right)^{n},
  \end{align*}
  and take
  \begin{align*}
    h(z) &= \left( z-z_0 \right)^{m} \widetilde{h}(z),
  \end{align*}
  which exists as $g$ is not uniformly zero, and where $ \widetilde{h}\left( z_0 \right) \neq 0 $. Consequently, there is $0 < \rho \leq r$ such that $ \widetilde{f}\colon U\left( z_0,\rho \right)\rightarrow \C $ given by
  \begin{align*}
    \widetilde{f}(z) &= \frac{1}{ \widetilde{h}(z) },
  \end{align*}
  which is holomorphic. Thus,
  \begin{align*}
    f(z) &= w + \left( z-z_0 \right)^{-m} \widetilde{f}(z).
  \end{align*}
  Thus, we get
  \begin{align*}
    f(z) &= \sum_{n=-m}^{\infty} a_n\left( z-z_0 \right)^{n},
  \end{align*}
  whence $z_0$ is either removable or a pole.
\end{proof}
\subsection{The Argument Principle and Rouché's Theorem}%
\begin{definition}
  If $U\subseteq \C$ is open, and $V\subseteq U$ is an open subset such that $U\setminus V$ consists solely of isolated points, then a function $f\colon V\rightarrow \C$ is \textit{meromorphic} if it is holomorphic on $V$ and every $z_0\in U\setminus V$ is either a pole or a removable singularity. We say $f$ is meromorphic on $U$.
\end{definition}
\begin{theorem}
  Let $U\subseteq \C$ be an open set, $\Gamma$ a piecewise $C^{1}$ cycle homologous to zero in $U$. Let $f$ be meromorphic on $U$ with no poles or zeros on $\img\left( \Gamma \right)$.
  \begin{enumerate}[(i)]
    \item The set $\set{z_0 \in U | \operatorname{ord}_{z_0}(f)\neq 0,n\left( \Gamma;z_0 \right) \neq 0}$ is finite.
    \item We have
      \begin{align*}
        \frac{1}{2\pi i} \oint_{\Gamma}^{} \frac{f'(z)}{f(z)}\:dz &= \sum_{\substack{z_0\in U \\ \operatorname{ord}{z_0}(f)\neq 0\\n\left( \Gamma;z_0 \right)}} n\left( \Gamma;z_0 \right)\ord_{z_0}(f).
      \end{align*}
  \end{enumerate}
\end{theorem}
\begin{proof}\hfill
  \begin{enumerate}[(i)]
    \item Let $K = \set{z\in U | n\left( \Gamma;z \right) \neq 0}\cup \img\left( \Gamma \right)$. We know that for $R > \sup_{w\in\img\left( \Gamma \right)}\left\vert w \right\vert$, we have $n\left( \Gamma;z \right) = 0$ for all $z\in \C\setminus B\left( 0,R \right)$, meaning that $K$ is bounded. Furthermore, if $\left( z_n \right)_n\rightarrow z\in \C$, then either $z\in \img\left( \Gamma \right)\subseteq K$ or $z\in \C\setminus \img\left( \Gamma \right)$, so that $n\left( \Gamma;z \right) \neq 0$ by the continuity of the map $w\mapsto n\left( \Gamma;w \right)$. Thus, $z\in K$ as $\Gamma$ is homologous to zero in $U$. It follows thus that $K$ is compact.%\newline

      Let $E = \set{z_0\in K | \operatorname{ord}_{z_0}\left( f \right)\neq 0}$, meaning that $E$ consists of poles and zeros of $f$. These points are isolated, meaning $E$ is a closed subset of $K$, hence compact. Since $E$ is compact and contains isolated points only, it follows that $E$ is finite.
    \item For each $z\in K$, select $\delta_z > 0$ such that $U\left( z,\delta_z \right)\subseteq U$, and $U\left( z,\delta_z \right)\cap E \subseteq \set{z}$. Such a $\delta_z$ exists since $z$ is isolated in $E$ and $U$ is open. The collection $\set{U\left( z,\delta_z \right) | z\in K}$ is an open cover of $K$, so there is a finite subcover $\set{U\left( z_1,\delta_1 \right),\dots,U\left( z_m,\delta_m \right)}$. Define
      \begin{align*}
        V &= \bigcup_{j=1}^{m}U\left( z_j,\delta_j \right),
      \end{align*}
      so that $V$ is open with $K\subseteq V \subseteq U$. Since $\set{z\in U | n\left( \Gamma;z \right)\neq 0}\subseteq K\subseteq V$, it follows that $\Gamma$ is homologous to zero in $V$.%\newline

      Define
      \begin{align*}
        g(z) &= f(z)\prod_{z\in E}\left( z-z_0 \right)^{-\operatorname{ord}_{z_0}(f)}.
      \end{align*}
      Since $\operatorname{ord}_z(g) = 0$ for all $z\in V$, all the singularities of $g$ are removable, and $\frac{g'}{g}$ is holomorphic on $V$ and satisfies
      \begin{align*}
        \frac{g'}{g} &= \frac{f'}{f} - \sum_{z_0\in E}\frac{1}{z-z_0}\operatorname{ord}_{z_0}(f).
      \end{align*}
      Cauchy's Integral theorem provides the desired result.
  \end{enumerate}
\end{proof}
\begin{theorem}[Rouché's Theorem]
  Let $U\subseteq \C$ be an open set, $\Gamma$ a piecewise $C^{1}$  cycle homologous to zero in $U$. Let $f$ and $g$ be meromorphic on $U$ with no poles or zeros on $\img\left( \Gamma \right)$. If $\left\vert f(z) - g(z) \right\vert < \left\vert f(z) \right\vert + \left\vert g(z) \right\vert$ for all $z\in \img\left( \Gamma \right)$, then
  \begin{align*}
    \frac{1}{2\pi i}\oint_{\Gamma}^{} \frac{f'(z)}{f(z)}\:dz &= \frac{1}{2\pi i}\oint_{\Gamma}^{} \frac{g'(z)}{g(z)}\:dz.
  \end{align*}
\end{theorem}
\begin{proof}
  Since $\left\vert f(z) - g(z) \right\vert < \left\vert f(z) \right\vert + \left\vert g(z) \right\vert$ on $\img\left( \Gamma \right)$, coupled with the fact that $\operatorname{ord}_{z}(f) = \operatorname{ord}_{z}(g) = 0$ on $\img\left( \gamma \right)$ implies that
  \begin{align*}
    \left\vert \frac{f(z)}{g(z)} - 1 \right\vert &< \left\vert \frac{f(z)}{g(z)} \right\vert + 1
  \end{align*}
  for all $z\in \img\left( \Gamma \right)$. This only holds if $\frac{f(z)}{g(z)} \in \C\setminus \left( -\infty,0 \right]$ for $z\in \img\left( \Gamma \right)$. Since $\img\left( \Gamma \right)$ is compact, there exists some $\ve > 0$ such that
  \begin{align*}
    \dist_{\left( -\infty,0 \right]}\left\vert \frac{f}{g}\left( \img\left( \Gamma \right) \right) \right\vert &\geq \ve.
  \end{align*}
  Since $\frac{f}{g}$ is continuous and $\img\left( \Gamma \right)$ is compact, there also exists some $\delta > 0$ such that whenever $\dist_{\img\left( \Gamma \right)}\left( z \right) < \delta$, we have $\frac{f(z)}{g(z)} \in \C\setminus \left( -\infty,0 \right]$.%\newline

  Setting $V = \set{z\in U | \dist_{\img\left( \Gamma \right)}(z) < \delta}$, we let $h\colon V\rightarrow \C$ be defined by
  \begin{align*}
    h(z) &= \log\left( \frac{f(z)}{g(z)} \right)
  \end{align*}
  for the branch of the logarithm that excludes $\left( -\infty,0 \right]$, which is well-defined as $\frac{f}{g}\notin \left( -\infty,0 \right]$ on $V$. This satisfies
  \begin{align*}
    \frac{h'}{h} &= \frac{f'}{f} - \frac{g'}{g},
  \end{align*}
  whence by Cauchy's Integral Theorem,
  \begin{align*}
    0 &= \frac{1}{2\pi i} \oint_{\Gamma}^{} \frac{h'(z)}{h(z)}\:dz\\
      &= \frac{1}{2\pi i} \left( \oint_{\Gamma}^{} \frac{f'(z)}{f(z)}\:dz - \oint_{\Gamma}^{} \frac{g'(z)}{g(z)}\:dz \right).
  \end{align*}
\end{proof}
\begin{remark}
  Most use cases for Rouché's Theorem involve finding $g(z)$ such that $\left\vert f(z) - g(z)\right\vert < \left\vert g(z) \right\vert$ on $\img\left( \Gamma \right)$, where both $f(z)$ and $g(z)$ have no zeros or poles on $\img\left( \Gamma \right)$.
\end{remark}
It is possible to use Rouché's Theorem to prove the fundamental theorem of algebra.
\begin{corollary}
  Let $P\colon \C\rightarrow \C$ be a nonconstant polynomial. Then, there exists $z_0\in \C$ with $P\left( z_0 \right) = 0$.
\end{corollary}
\begin{proof}
  Write $P(z) = a_nz^{n} + \cdots + a_1 z + a_0$, where $n\geq 1$, $a_n\neq 0$, and $a_1,\dots,a_n\in \C$.%\newline

  Let $Q(z) = a_nz^{n}$, $R$ sufficiently large, and $\Gamma = S\left( 0,R \right)$. Then,
  \begin{align*}
    \left\vert P(z) - Q(z) \right\vert &= \left\vert a_{n-1}z^{n-1} + \cdots + a_1 z + a_0 \right\vert\\
                                       &\leq \left\vert a_n \right\vert R^{n} \left( \frac{\left\vert a_{n-1} \right\vert}{\left\vert a_n \right\vert R} + \cdots + \frac{\left\vert a_0 \right\vert}{\left\vert a_n \right\vert R^{n}} \right)\\
                                       &< \left\vert a_n \right\vert R^{n}\\
                                       &\leq \left\vert P(z) \right\vert + \left\vert Q(z) \right\vert.
  \end{align*}
  Thus, $P$ has the same number of zeros counted with multiplicity as $Q$.
\end{proof}
\begin{corollary}
  Let $U\subseteq \C$ be an open set, $f\colon U\rightarrow \C$ be a holomorphic injective function. Then, $f'(z)\neq 0$ for all $z\in U$, whence $f$ admits a holomorphic inverse.
\end{corollary}
\begin{proof}
  Suppose there is some $z_0\in U$ such that $f'\left(z_0\right)= 0$. By the identity theorem, there must be some $r > 0$ such that $B\left( z_0,r \right) \subseteq U$ and $f'(z)\neq 0$ for all $z\in \dot{U}\left( z_0,r \right)$, else $f$ would be equal to a constant on some nonempty open set, hence not injective.%\newline

  Let $g(z) = f(z) - f\left( z_0 \right)$. Define $m\coloneq \operatorname{ord}_{z_0}(g)$. We observe that $m\geq 2$, as $g\left( z_0 \right) = 0$ and $g'\left( z_0 \right) = 0$. We may write
  \begin{align*}
    g(z) &= \lambda \left( z-z_0 \right)^{m} + \left( z-z_0 \right)^{k+1}h(z)
  \end{align*}
  for some holomorphic function $h\colon U\rightarrow \C$ and a constant $\lambda\in \C\setminus \set{0}$, meaning
  \begin{align*}
    f(z) &= f\left( z_0 \right) + \lambda \left( z-z_0 \right)^{m} + \left( z-z_0 \right)^{k+1}h(z)
  \end{align*}
  for all $z\in U$.%\newline

  Let $C = \sup_{z\in B\left( z-0,r \right)}\left\vert h(z) \right\vert$, which is finite as $B\left( z_0,r \right)$ is compact. Letting $\rho = \min\left( r,\frac{\left\vert \lambda \right\vert}{2C} \right)$ and $\eta = \frac{\left\vert \lambda \right\vert \rho^{m}}{2}$, and fixing $w\in \dot{U}\left( f\left( z_0, \right),\eta \right)$, we observe that if $z\in S\left( z-0,\rho \right)$,
  \begin{align*}
    \left\vert f(z) - w - \lambda \left( z-z_0 \right)^{m} \right\vert &= \left\vert f\left( z_0 \right) - w + \left( z-z_0 \right)^{m+1}h(z) \right\vert\\
                                                                       &\leq \left\vert f\left( z_0 \right) - w \right\vert + \left\vert z-z_0 \right\vert^{m+1} \left\vert h(z) \right\vert\\
                                                                       &< \eta + \rho^{m+1}C\\
                                                                       &< \rho^{m} \left\vert \lambda \right\vert\\
                                                                       &= \left\vert \left( z-z_0 \right)^{m}\lambda \right\vert.
  \end{align*}
  Therefore, by Rouché's theorem, the number of zeros in $U\left( z_0,\rho \right)$ for $f(z) - w$ is equal to the number of zeros counted with multiplicity in $U\left( z_0,\rho \right)$ of $\left( z-z_0 \right)^{m}\lambda$. Since the latter $m\geq 2$, it follows that the former is also $m\geq 2$. Since $f'(z)\neq 0$ for all $z\in U\left( z_0,\rho \right)$, no zero of $f(z) - w$ can have order at least $2$, meaning that there are at least two distinct zeros of $f(z) - w$ in $U\left( z_0,\rho \right)$, whence $f$ is not injective.
\end{proof}
\subsection{Residues}%
\begin{definition}
  Let $U\subseteq \C$, $z_0\in U$, $r > 0$ such that $U\left( z_0,r \right)\subseteq U$, and
  \begin{align*}
    f(z) &= \sum_{n=-\infty}^{\infty} a_n\left( z-z_0 \right)^{n}
  \end{align*}
  on $\dot{U}\left( z_0,r \right)$. The \textit{residue} of $f$ at $z_0$ is the coefficient $a_{-1}$ of $\left( z-z_0 \right)^{-1}$. We write $\res\left( f;z_0 \right)$.
\end{definition}
\begin{proposition}
  Let $U\subseteq \C$ be an open set, and let $f$ be meromorphic on $U$ with a pole of order $m\geq 1$ at $z_0\in U$. Then,
  \begin{align*}
    \res\left( f;z_0 \right) &= \frac{1}{\left( m-1 \right)!} \lim_{z\rightarrow z_0} \diff{^{m-1}}{z^{m-1}} \left( z-z_0 \right)^{m}f(z).
  \end{align*}
\end{proposition}
\begin{proof}
  Write
  \begin{align*}
    f(z) &= \sum_{n=-\infty}^{\infty} a_n\left( z-z_0 \right)^{n}.
  \end{align*}
  Differentiating term by term, we find that
  \begin{align*}
    \frac{1}{\left( m-1 \right)!} \lim_{z\rightarrow z_0} \diff{^{m-1}}{z^{m-1}} \left( z-z_0 \right)^{m}f(z) &= a_{-1}.
  \end{align*}
\end{proof}
\begin{proposition}
  Let $f,g\colon U\rightarrow \C$ be holomorphic with $z_0$ a simple zero for $g$ that is not a zero or pole for $f$. Then,
  \begin{align*}
    \res\left( \frac{f}{g};z_0 \right) &= \frac{f\left(z_0\right)}{g'\left( z_0 \right)}.
  \end{align*}
\end{proposition}
\begin{proof}
  We compute the residue directly, using the fact that $g\left( z_0 \right) = 0$ to find
  \begin{align*}
    \lim_{z\rightarrow z_0} \left( z-z_0 \right) \frac{f(z)}{g(z)} &= \frac{f\left( z_0 \right)}{\lim_{z\rightarrow z_0} \frac{g\left( z \right) - g\left( z_0 \right)}{z-z_0}}\\
                                                                   &= \frac{f\left( z_0 \right)}{g'\left( z_0 \right)}.
  \end{align*}
\end{proof}
\begin{example}\hfill
  \begin{itemize}
    \item If $f(z) = \frac{e^{iz}}{z}$, then
      \begin{align*}
        \res\left( f;0 \right) &= 1.
      \end{align*}
    \item If $f(z) = \pi \cot\left( \pi z \right)$, then
      \begin{align*}
        \res\left( f;n \right) &= \frac{\pi \cos\left( \pi n \right)}{ \diff{}{z}|_{z = n} \sin\left( \pi z \right) }\\
                               &= \frac{\pi \cos\left( \pi n \right)}{\pi \cos\left( \pi n \right)}\\
                               &= 1.
      \end{align*}
    \item Let $f(z) = \frac{e^{3z}}{\left( z-2 \right)^2}$. To compute the residue at $z = 2$, we may directly find
      \begin{align*}
        \res\left( f;2 \right) &= \lim_{z\rightarrow 2} \diff{}{z}\left( z-2 \right)^2 \frac{e^{3z}}{\left( z-2 \right)^2}\\
                               &= 3e^{6}.
      \end{align*}
  \end{itemize}
\end{example}
\begin{theorem}
  Let $f\colon \dot{U}\left( z_0,r \right)\rightarrow \C$ be holomorphic. Then, for all $0 < \rho < r$, we have
  \begin{align*}
    \oint_{S\left( z_0,\rho \right)}^{} f(z)\:dz &= 2\pi i \res\left( f;z_0 \right).
  \end{align*}
\end{theorem}
\begin{proof}
  Write
  \begin{align*}
    f(z) &= \sum_{n=-\infty}^{\infty} a_n\left( z-z_0 \right)^{n}.
  \end{align*}
  The series converges uniformly on compact sets, so we may exchange order of integration and summation to find
  \begin{align*}
    \oint_{S\left( z_0,\rho \right)}^{} f(z)\:dz &= \sum_{n=-\infty}^{\infty}a_n \oint_{S\left( z_0,\rho \right)}^{} \left( z-z_0 \right)^{n}\:dz\\
                                                 &= 2\pi i a_{-1}.
  \end{align*}
\end{proof}
\begin{theorem}
  Let $U\subseteq \C$ be an open set, $\Gamma$ a piecewise $C^{1}$ cycle homologous to zero in $U$. Let $f$ be meromorphic on $U$ with no poles on $\img\left( \Gamma \right)$. Then,
  \begin{align*}
    \oint_{\Gamma}^{} f(z)\:dz &= 2\pi i \sum_{z_0\in E} n\left( \Gamma;z_0 \right)\res\left( f;z_0 \right),
  \end{align*}
  where
  \begin{align*}
    E &= \set{z_0\in U | \ord_{z_0}(f) < 0, n\left( \Gamma;z_0 \right) \neq 0}.
  \end{align*}
\end{theorem}
\begin{proof}
  As in the proof of the argument principle, we see that $E$ is finite, which we write $\set{z_1,\dots,z_m}$. Select $r > 0$ such that $ B\left( z_j,r \right) $ are pairwise disjoint and contained in $U$. Set $ \widetilde{\Gamma} = \Gamma - \sum_{j=1}^{n} S\left( z_j,r \right) $. Then, $ \widetilde{\Gamma} $ is homologous  to zero in $U\setminus \set{z_1,\dots,z_m}$, while $f$ is holomorphic on $U\setminus \set{z_1,\dots,z_m}$, whence
  \begin{align*}
    0 &= \oint_{ \widetilde{\Gamma} }^{} f(z)\:dz\\
      &= \oint_{\Gamma}^{} f(z)\:dz - \sum_{j=1}^{m} \oint_{S\left( z_j,r \right)}^{} f(z)\:dz\\
      &= \oint_{\Gamma}^{} f(z)\:dz - 2\pi i \sum_{z_0\in E}n\left( \Gamma;z_0 \right)\res\left( f;z_0 \right).
  \end{align*}
\end{proof}
\subsection{Conformal Maps and Spaces of Holomorphic Functions}%
Given an open subset $U\subseteq \C$, we define $H(U)$ to be the set of all holomorphic functions $f\colon U\rightarrow \C$. Similarly, we write $C(U)$ for the continuous functions $f\colon U\rightarrow \C$.
\begin{definition}
  A sequence $\left( f_n \right)_n\subseteq C(U)$ converges uniformly on compacts to $f\in C(U)$ if, for every compact $K\subseteq U$, the sequence $\left( f_n|_{K} \right)_{n}\rightarrow f|_{K}$ uniformly.
\end{definition}
\begin{proposition}
  Let $U\subseteq \C$ be open. If a sequence $\left( f_n \right)_n\subseteq H(U)$ converges uniformly on compacts to $f\in C(U)$, then $f\in H(U)$. Moreover, the sequence $\left( f_n' \right)_{n}\rightarrow f'$.
\end{proposition}
\begin{proof}
  Since any $\left( f_n \right)_n$ converges on compacts to $f$, it converges uniformly on any triangle $T$ homologous to zero in $U$, so that
  \begin{align*}
    0 &= \lim_{n\rightarrow\infty} \oint_{T}^{} f_n(z)\:dz\\
      &= \oint_{T}^{} \lim_{n\rightarrow\infty}f_n(z)\:dz\\
      &= \oint_{T}^{} f(z)\:dz.
  \end{align*}
  To show that $\left( f_n' \right)_n\rightarrow f'$ converges uniformly on compacts, we use Cauchy's integral formula to find that
  \begin{align*}
    f_n'(z)-f'(z) &= \frac{1}{2\pi i} \oint_{S\left( z_0,R \right)}^{} \frac{f_n\left( w \right)-f\left( w \right)}{\left( w - z \right)^2}\:dw,
  \end{align*}
  where $z\in U\left( z_0,R \right)$ and $B\left( z_0,R \right)\subseteq U$. Cauchy's estimate then show that this tends to $0$ as $n\rightarrow\infty$ uniformly for all $z\in U\left( z_0,R \right)$. Since compact sets are totally bounded, it thus follows that the convergence is uniform on compact subsets.
\end{proof}
\begin{definition}
  Let $U\subseteq \C$ be an open set. An exhaustion of $U$ is a collection of compacts $\left( K_m \right)_m$ for which $K_m\subseteq K_{m+1}^{\circ}$, and
  \begin{align*}
    U &= \bigcup_{m=1}^{\infty}K_m.
  \end{align*}
\end{definition}
The primary example we will use is
\begin{align*}
  K_m &\coloneq \set{z\in U | \left\vert z \right\vert\leq m,\dist_{\C\setminus U}\left( z \right) \geq \frac{1}{m}}.
\end{align*}
\begin{definition}
  Let $U\subseteq \C$ be an open set, and let $\left( K_m \right)_m$ be an Exhaustion of $U$. For $f,g\in C\left( U \right)$, we define
  \begin{align*}
    d\left( f,g \right) &= \sum_{m=1}^{\infty}2^{-m} \frac{\norm{f-g}_{K_m}}{1 + \norm{f-g}_{K_m}}.
  \end{align*}
\end{definition}
Now, despite the fact that the metric space $\left( C(U),d \right)$ depends on the choice of exhaustion, it can be shown that any two metrics based on exhaustions $\left( K_m \right)_m$ and $\left( K_m' \right)_m$ are uniformly equivalent.
\begin{theorem}
  Let $U\subseteq \C$ be an open set, $\left( K_m \right)_m$ an exhaustion of $U$, and let
  \begin{align*}
    d\left( f,g \right) &= \sum_{m=1}^{\infty}2^{-m} \frac{\norm{f-g}_{K_m}}{1 + \norm{f-g}_{K_m}}
  \end{align*}
  for $f,g\in C(U)$. Then, $\left( H\left( U \right),d \right)$ is a complete metric space.
\end{theorem}
\begin{proposition}
  The topology on $\left( H\left( U \right),d \right)$ is equal to the topology of uniform convergence on compact subsets.
\end{proposition}
\begin{proof}
  Let $\left( K_m \right)_m$ be an exhaustion of $U$. Let $\left( X_m,d_m \right) = \left( C\left( K_m, \right),\norm{\cdot-\cdot}_{K_m} \right)$, and set
  \begin{align*}
    X &= \prod_{m=1}^{\infty}C\left( K_m \right).
  \end{align*}
  The isometry $\iota\colon C\left( U \right)\rightarrow X$, given by $\iota(f) = \left( f|_{K_m} \right)_m$ yields that a sequence $\left( f_n \right)_n\rightarrow f$ if and only if $\left( f_n|_{K_m} \right)_n\rightarrow f|_{K_m}$ for each $K_m$. In particular, if $\left( f_n \right)_n\rightarrow f$ uniformly on compact subsets, then it converges uniformly on each $K_m$, hence $\lim_{n\rightarrow\infty}d\left( f_n,f \right) = 0$.%\newline

  Conversely, if $K\subseteq \C$, $\set{K_m^{\circ}}_{m=1}^{\infty}$ is an open cover of $K$, so there exists a finite subcover. Since $K_m\subseteq K_{m+1}^{\circ}$, this means there is some $M\in \N$ such that $K\subseteq K_M$. In particular, if $\lim_{n\rightarrow\infty}d\left( f_n,f \right)= 0$, it then follows that $\left( f_n \right)_n\rightarrow f$ uniformly on $K$.
\end{proof}
\begin{definition}
  Let $U\subseteq \C$ be an open set. A family $\mathcal{F}\subseteq H\left( U \right)$ is called \textit{normal} if its closure $ \overline{\mathcal{F}} $ is compact in $H(U)$.
\end{definition}
\begin{theorem}
  Let $U\subseteq \C$ be an open set, $\mathcal{F}\subseteq H(U)$ a family of holomorphic functions. The following are equivalent:
  \begin{enumerate}[(i)]
    \item $\mathcal{F}$ is normal;
    \item for each $K\subseteq U$, the family $\mathcal{F}|_{K}$ has compact closure in $C(K)$:
    \item for each $z\in U$, there exists a bounded open set $W_z\subseteq U$ containing $z$ such that $\mathcal{F}|_{W_z}$ has compact closure in $C\left( \overline{W_z} \right)$.
  \end{enumerate}
\end{theorem}
\begin{proof}
  We start by showing that (i) implies (iii). For $z\in U$, let $R = R_z > 0$ such that $B\left( z,R \right)\subseteq U$, and let $g_n\in \mathcal{F}|_{ B\left( z,R \right) }$. Choose $f_n\in \mathcal{F}$ such that $f_n|_{ B\left( z,R \right) } = g_n$. If $\mathcal{F}$ is normal, there exists a subsequence $\left( f_{n_k} \right)_k\rightarrow f\in \overline{\mathcal{F}}$. In particular, $f_{n_k}|_{B\left( z,R \right)}\rightarrow f|_{B\left( z,R \right)}$. Thus, $\mathcal{F}|_{B\left( z,R \right)}$ has compact closure in $C\left( B\left( z,R \right) \right)$.%\newline

  Next, we show that (iii) implies (ii). Let $W_z$ be as above. Given $K\subseteq U$, the collection $\set{W_z}_{z\in U}$ is an open cover of $K$ that has a finite subcover $\set{W_{1},\dots,W_{\ell}}$. Given $g_n\in \mathcal{F}|_{K}$, write $g_n = f_n|_{K}$ for some $f_n\in \mathcal{F}$. There then exists a subsequence $\left( f_{n_k} \right)_k$ such that $\left( f_{n_k}|_{W_{j}} \right)\rightarrow f_j\in C\left( \overline{W_j} \right)$ for each $j$. The function $f \coloneq \lim_{k\rightarrow\infty}f_{n_k}(z)$ is well-defined and satisfies $f|_{ \overline{W_{j}} } = f_j$ for each $j$. Moreover, since
  \begin{align*}
    \limsup_{k\rightarrow\infty}\norm{f_{n_k}-h}_{K} &\leq \limsup_{k\rightarrow\infty}\max_{1\leq j \leq \ell} \norm{f_{n_k} - f_j}|_{ \overline{W_j} }\\
                                                     &= 0,
  \end{align*}
  it follows that $\mathcal{F}|_{K}$ has compact closure in $C\left( K \right)$.%\newline

  Finally, we show that (ii) implies (i). Let $\left( K_m \right)_m$ be an exhaustion of $U$. By Tychonoff's Theorem, $\prod_{m=1}^{\infty} \overline{\set{f|_{K_m} | f\in \mathcal{F}}}$ is compact. Thus, given a sequence $\left( f_n \right)_n$ in $\mathcal{F}$, there is a subsequence $\left( f_{n_k} \right)_k$ such that for each $m\in \N$, there is some $g_m\in C\left( K_m \right)$ such that $f_{n_k}|_{K_m}$ converges uniformly to $g_m$. Thus, the function $f(z) = \lim_{k\rightarrow\infty}f_{n_k}(z)$ is well-defined and satisfies $f|_{K_m} = g_m$ for all $m$. Given a compact $K\subseteq U$, there is some $M\in \N$ such that $K\subseteq K_M$, and consequently, $f_{n_k}|_{K}\rightarrow f|_{K}$ uniformly, so $\mathcal{F}$ is normal.
\end{proof}
We will now create a much more workable criterion for normality.
\begin{definition}
  Let $\left( X,d \right)$ be a compact metric space. A family $\mathcal{F}\subseteq C(X)$ is (uniformly) equicontinuous if, for all $\ve > 0$, there is $\delta > 0$ such that for all $f\in \mathcal{F}$, $ \left\vert f(x) - f(y) \right\vert < \ve $ whenever $d\left( x,y \right) < \delta$.
\end{definition}
Recall the Arzelà--Ascoli theorem.
\begin{theorem}[Arzelà--Ascoli]
  Let $K\subseteq \C$ be compact, and let $\mathcal{F}\subseteq C(K)$ be a family of continuous functions. The following are equivalent:
  \begin{enumerate}[(i)]
    \item $ \overline{\mathcal{F}} $ is compact;
    \item $\mathcal{F}$ is bounded and, for all $z\in K$, $ \overline{\set{f(z) | f\in \mathcal{F}}} \subseteq \C$ is compact.
  \end{enumerate}
\end{theorem}
\begin{definition}
  Let $U\subseteq \C$ be an open set. A family $\mathcal{F}\subseteq H(U)$ is called locally (uniformly) bounded on $U$ if, for all $z_0\in U$, there is some $\delta > 0$ such that $U\left( z_0,\delta \right)\subseteq U$ and there is some $C \geq 0$ such that $\left\vert f(z) \right\vert\leq C$ for all $z\in U\left( z_0,\delta \right)$ and all $f\in \mathcal{F}$.
\end{definition}
\begin{proposition}
  Let $U\subseteq \C$ be open, $\mathcal{F}\subseteq H(U)$ a family of holomorphic functions. The following are equivalent:
  \begin{enumerate}[(i)]
    \item $\mathcal{F}$ is locally bounded;
    \item for every compact $K\subseteq U$, $\sup_{f\in \mathcal{F}}\norm{f}_{K}$ is finite.
  \end{enumerate}
\end{proposition}
\begin{proof}
  We start by showing that (i) implies (ii). For each $z\in U$, there is $\delta_z > 0$ and $C_z > 0$ such that $U\left( z,\delta_z \right)\subseteq U$ and $\left\vert f(w) \right\vert\leq C_z$ for all $w\in U\left( z,\delta_z \right)$. Let $K\subseteq U$ be compact. There is a finite subcover $\set{U\left( z_1,\delta_1 \right),\dots,U\left( z_k,\delta_k \right)}$, with corresponding bounds $C_1,\dots,C_k$, so by defining $C = \max\set{C_{1},\dots,C_k}$, it follows that $\sup_{f\in \mathcal{F}}\norm{f}_K\leq C < \infty$.%\newline

  Now, we show that (ii) implies (i). Given $z_0\in U$, there is $r > 0$ such that $B\left( z_0,r \right)\subseteq U$. By taking $K = B\left( z_0,r \right)$, we are done.
\end{proof}
\begin{theorem}[Montel's Theorem]
  Let $U\subseteq \C$ be open, $\mathcal{F}\subseteq H(U)$ a family of holomorphic functions. The following are equivalent:
  \begin{enumerate}[(i)]
    \item $\mathcal{F}$ is normal;
    \item $\mathcal{F}$ is locally bounded.
  \end{enumerate}
\end{theorem}
\begin{proof}
  We start by showing that (i) implies (ii). Given $K\subseteq U$, the map $\norm{\cdot}_{K}$ is continuous, so $\sup_{f\in \mathcal{F}}\norm{f}_{K}\leq \sup_{f\in \overline{\mathcal{F}}}\norm{f}_{K}$, which is finite as $ \overline{\mathcal{F}} $ is compact.%\newline

  Now, we show (ii) implies (i). Fix $z_0\in U$, and choose $R_0 > 0$ such that $B\left( z_0,R_0 \right)\subseteq U$. Let $W_0 = U\left( z_0,R_0/2 \right)$. It suffices to show that $\mathcal{F}|_{ \overline{W_0} }$ has compact closure in $C\left( \overline{W_0} \right)$, which by the Arzelà--Ascoli theorem, is equivalent to showing that $\mathcal{F}|_{ \overline{W_0} }$ is equicontinuous. Since $ \mathcal{F} $ is locally bounded, there is some $C_0 \geq 0$ such that $\sup_{f\in \mathcal{F}}\norm{f}_{ B\left( z_0,R_0 \right) }\leq C_0$. By Cauchy's Integral Formula, we then have for all $z\in \overline{W_0}$ and all $f\in \mathcal{F}$,
  \begin{align*}
    \left\vert f'\left( z \right) \right\vert &= \left\vert \frac{1}{2\pi i} \oint_{S\left( z_0,R_0 \right)}^{} \frac{f\left( \xi \right)}{\left( \xi-z \right)^2}\:d\xi \right\vert\\
                                              &\leq \frac{C_0}{2\pi} \oint_{S\left( z_0,R_0 \right)}^{} \frac{1}{\left\vert \xi-z \right\vert^2}\:\left\vert d\xi \right\vert\\
                                              &\leq \frac{4C_0}{R_0}\\
                                              &\eqcolon A_0.
  \end{align*}
  We have thus shown that $\left\vert f'\left( z \right) \right\vert\leq A_0$ for all $z\in \overline{W_0}$, meaning that each $f|_{ \overline{W_0} }\in \mathcal{F}|_{ \overline{W_0} }$ is $A_0$-Lipschitz, hence $\mathcal{F}|_{ \overline{W_0} }$ is equicontinuous, hence normal.
\end{proof}
\section{Worked Examples and Problem-Solving Methods}%
\begin{example}
  Suppose $U$ is a region in $\C$ that contains $0$, and suppose $f\colon U\rightarrow \C$ is a holomorphic function satisfying
  \begin{align*}
    \left\vert f\left( \frac{1}{n} \right) \right\vert &< e^{n}
  \end{align*}
  for sufficiently large $n$. We will show that this means $f$ is $0$ everywhere.%\newline

  Toward this end, since $U$ is open, there is some $r > 0$ such that $U\left( 0,r \right)\subseteq U$. Since $f$ is holomorphic, on $U\left( 0,r \right)$, we may write
  \begin{align*}
    f\left( z \right) &= \sum_{n=0}^{\infty}a_nz^{n}
  \end{align*}
  for some sequence $\left( a_n \right)_n\subseteq \C$. Now, we also observe that
  \begin{align*}
    \left\vert f(0) \right\vert &= \lim_{n\rightarrow\infty} \left\vert f\left( \frac{1}{n} \right) \right\vert\\
                                &\leq \lim_{n\rightarrow\infty} e^{-n}\\
                                &= 0.
  \end{align*}
  Suppose toward contradiction that $f$ were nonconstant. Then, there would be some minimal positive value $\ell$ such that
  \begin{align*}
    f(z) &= z^{\ell}\sum_{n=0}^{\infty}a_{n + \ell}z^{n}
  \end{align*}
  has $a_{\ell}\neq 0$. Thus, defining
  \begin{align*}
    g(z) &= \sum_{n=0}^{\infty}a_{n + \ell}z^{n},
  \end{align*}
  we observe that $g\left( 0 \right) \neq 0$, meaning that on some sufficiently small ball $U\left( 0,\delta \right)\subseteq U\left( 0,r \right)$, we have $\left\vert g(z) \right\vert > \left\vert \frac{a_{\ell}}{2} \right\vert$ for all $z\in U\left( 0,\delta \right)$. In particular, this means that for $n$ with $\frac{1}{n} < \delta$,
  \begin{align*}
    e^{-n} &\geq \left\vert f\left( \frac{1}{n} \right) \right\vert\\
           &= n^{-\ell}\left\vert g\left( \frac{1}{n} \right) \right\vert\\
           &\geq \frac{\left\vert a_{\ell} \right\vert}{2n^{\ell}},
  \end{align*}
  whence
  \begin{align*}
    \left\vert a_{\ell} \right\vert &\leq \frac{2n^{\ell}}{e^{n}}.
  \end{align*}
  Yet, since $n$ is arbitrary and $\ell$ is constant, this implies that $\left\vert a_{\ell} \right\vert = 0$, contradicting the assumption that there were such a $g$. Thus, in particular, we have that $f(z) = 0$ on $U\left( 0,r \right)$, whence $f$ is zero everywhere by the identity theorem.
\end{example}
\subsection{Cauchy Estimate Problems}%
\begin{example}
  Suppose $f$ is an entire function, and suppose there exists a constant $C$ such that for all $z\in \C$,
  \begin{align*}
    \left\vert f(z) \right\vert &\leq C\left( 1 + \left\vert z \right\vert \right)^{1/2}.
  \end{align*}
  We will show that $f$ is then constant. Toward this end, we will be able to use the Cauchy estimate by taking
  \begin{align*}
    \left\vert f^{(n)}(z) \right\vert &\leq \frac{n!}{R^{n}} \sup_{\left\vert z \right\vert = R} \left\vert f(z) \right\vert\\
                                      &\leq \frac{Cn!}{R^{n}} \sup_{\left\vert z \right\vert = R} \left( 1 + \left\vert z \right\vert \right)^{1/2}\\
                                      &= \frac{Cn!}{R^{n}} \left( 1 + R \right)^{1/2},
  \end{align*}
  whence for all $n\geq 1$, since $R$ is arbitrary, we have $\left\vert f^{(n)}\left( z \right) \right\vert = 0$, so $f$ is constant.
\end{example}
\subsection{Maximum Modulus Principle Problems}%
\begin{example}
  We show that if $f\colon U\rightarrow \C$ is holomorphic on a connected open set, and
  \begin{align*}
    u\left( x,y \right) &= \left\vert f\left( x + iy \right) \right\vert
  \end{align*}
  is harmonic on $U$, then $f$ is constant.%\newline

  Toward this end, we let $z_0\in U$ and $r > 0$ such that $B\left( z_0,r \right)\subseteq U$. For any $0 < s < r$, the mean value property gives
  \begin{align*}
    \left\vert f\left( z_0 \right) \right\vert &\leq \frac{1}{2\pi} \int_{0}^{2\pi} \left\vert f\left( z_0 + se^{i\theta} \right) \right\vert\:d\theta\\
                                               &= \left\vert f\left( z_0 \right) \right\vert.
  \end{align*}
  In particular, for any $0 < s < r$, we have the equality
  \begin{align*}
    \left\vert f\left( z_0 \right) \right\vert &= \frac{1}{2\pi} \int_{0}^{2\pi} \left\vert f\left( z_0 + se^{i\theta} \right) \right\vert\:d\theta.
  \end{align*}
  Since $f$ is continuous, there is some $\theta_s$ such that $\left\vert f\left( z_0 + se^{i\theta} \right) \right\vert = e^{i\theta_s}f\left( z_0 + se^{i\theta} \right)$, whence
  \begin{align*}
    \left\vert f\left( z_0 \right) \right\vert &= e^{i\theta_s} \frac{1}{2\pi} \int_{0}^{2\pi} f\left( z_0 + se^{i\theta} \right)\:d\theta\\
                                               &= e^{i\theta_s} f\left( z_0 \right),
  \end{align*}
  meaning that $\theta_s \eqcolon \theta_0$ is independent of $s$. Yet, this means that $e^{i\theta_0}f\left( z \right)$ is holomorphic on $U\left( z_0,r \right)$ and has $\im\left( e^{i\theta}f(z) \right) = 0$, meaning that by the open mapping principle, $f(z)$ is constant on $U\left( z_0,r \right)$, and so $f$ is constant on $U$ by the identity theorem.
\end{example}
\subsubsection{The Phragmén--Lindelöf Method}%
The maximum modulus principle is primarily useful in the case where $f$ is continuous on the closure of a bounded open set $U$ and holomorphic on the interior. Yet, this fails to be true if $U$ is unbounded.%\newline

For instance, if
\begin{align*}
  U &= \set{z\in \C | -\frac{\pi}{2} < \im(z) < \frac{\pi}{2}},
\end{align*}
and $f(z) = e^{e^{z}}$, then 
\begin{align*}
  f\left( x \pm \frac{\pi}{2}i \right) &= e^{\pm ie^{x}},
\end{align*}
whence $\left\vert f(z) \right\vert = 1$ for $z\in \partial U$. Yet, $f(z)\rightarrow \infty$ very rapidly along the positive real axis, which is contained in $U$.%\newline

Yet, all hope is not lost in the case that $U$ is unbounded. If $U$ is unbounded and there is $g\colon U\rightarrow \C$  such that $\left\vert f \right\vert < \left\vert g \right\vert$, and $g\rightarrow\infty$ ``slowly'' (so to speak) as $z\rightarrow\infty$, then it turns out that $f$ is actually bounded in $U$, and we can use the maximum modulus principle to obtain other conclusions about $f$.%\newline

Finding such a $g$ is part of the \textit{Phragmén--Lindelöf} method, which we expand upon here.
\begin{example}
  From the Cauchy estimates, we know that if $f$ is entire and
  \begin{align*}
    \left\vert f(z) \right\vert \leq C\left( 1 + \left\vert z \right\vert^{1/2} \right),
  \end{align*}
  then $f$ is constant.
\end{example}
\begin{theorem}[Hadamard Three-Lines Theorem]
  Let $a,b\in \R$ be fixed with $ a < b $. Let $U = \set{z | a < \re(z) < b}$. Suppose $\left\vert f(z) \right\vert < B$ for all $z\in U$ and some fixed $B < \infty$. Define
  \begin{align*}
    M(x) &= \sup\set{\left\vert f(z) \right\vert | z\in \overline{U}}.
  \end{align*}
  Then,
  \begin{align*}
    M(x)^{b-a} &\leq M(a)^{b-x}M(b)^{x-a}.
  \end{align*}
\end{theorem}
\begin{proof}
  Suppose $M(a) = M(b) = 1$. Our task now is to show that $\left\vert f(z) \right\vert \leq 1$ for all $z\in U$. Toward this end, define
  \begin{align*}
    h_{\ve}(z) &= \frac{1}{1 + \ve\left( z-a \right)}
  \end{align*}
  for $z\in \overline{U}$. We have $\left\vert h_{\ve} \right\vert \leq 1$ in $ \overline{U} $, so that
  \begin{align*}
    \left\vert f(z)h_{\ve}(z) \right\vert \leq 1
  \end{align*}
  for all $z\in \partial U$. Furthermore, since $\left\vert 1 + \ve\left( z-a \right) \right\vert \geq \ve\left\vert \im(z) \right\vert$, we have
  \begin{align*}
    \left\vert f(z)h_{\ve}(z) \right\vert &\leq \frac{B}{\ve\left\vert \im(z) \right\vert}
  \end{align*}
  for all $z\in \overline{U}$. Cut out a (closed) rectangle $R$ from $ \overline{U} $ via the lines $\im(z) = \pm \frac{B}{\ve}$. Thus, along $\partial R$, we have $\left\vert f(z)h_{\ve}(z) \right\vert \leq 1$, so that $\left\vert f(z)h_{\ve}(z) \right\vert \leq 1$ on $R$ by the maximum modulus principle.%\newline

  Yet, since $\left\vert f(z)h_{\ve}(z) \right\vert \leq \frac{B}{\ve\left\vert \im(z) \right\vert}$ on the entirety of $ \overline{U} $, and $ \frac{B}{\ve\left\vert \im(z) \right\vert} < 1 $ outside $R$, it follows that $\left\vert fh_{\ve} \right\vert\leq 1$ on $ \overline{U} $, so $ \left\vert f(z)h_{\ve}(z) \right\vert \leq 1 $ for all $z\in U$ and all $\ve > 0$. Taking the limit as $\ve \rightarrow 0$, we obtain the desired result, that $\left\vert f(z) \right\vert \leq 1$.%\newline

  In the general case, we define
  \begin{align*}
    g(z) &= M(a)^{(b-z)/(b-a)}M(b)^{(z-a)/(b-a)},
  \end{align*}
  where for all $M > 0$ and complex $w$, we have $M^{w} = e^{w\ln(M)}$. Then, $g$ is entire, $g$ is always nonzero, $\frac{1}{g}$ is bounded on $ \overline{U} $, and has
  \begin{align*}
    \left\vert g\left( a + iy \right) \right\vert &= M(a)\\
    \left\vert g\left( b + iy \right) \right\vert &= M(b),
  \end{align*}
  meaning that $\frac{f}{g}$ satisfies the previous assumptions, so that $ \left\vert f/g \right\vert \leq 1 $ in $U$.
\end{proof}
In the Phragmén--Lindelöf method, we seek to find a particular $\ve$-dependent function $h_{\ve}\colon U\rightarrow \C$ such that the following hold:
\begin{itemize}
  \item $\left\vert fh_{\ve}(z) \right\vert \leq M$ for all $z\in \partial U$;
  \item $\lim_{\ve\rightarrow 0} h_{\ve}(z) = 1$;
  \item there exists a \textit{bounded} $V\subseteq U$ such that $\left\vert fh_{\ve} \right\vert\leq M$ on $\partial V$ and on $U\setminus \overline{V}$.
\end{itemize}
\subsection{Rouché's Theorem Problems}%
\begin{example}
  We show that if $f$ and $g$ are holomorphic on a neighborhood of $B\left( 0,1 \right)$, and $f(z)$ has a simple zero at $z = 0$ and no other zero in $B\left( 0,1 \right)$, then $f_{\ve}(z) = f(z) + \ve g(z)$ has exactly one zero in $\D$ for sufficiently small $\ve$.%\newline

  To show this, we start by showing that the conditions of Rouché's Theorem are satisfied for both $f_{\ve}(z)$ and $f(z)$. It is clear from the fact that $f$ has no other zeros in $B\left( 0,1 \right)$ that $f$ has no zeros on $S\left( 0,1 \right)$, while we may find $\ve$ small enough such that
  \begin{align*}
    \left\vert f_{\ve}(z) \right\vert &\geq \left\vert f(z) \right\vert - \ve \left\vert g(z) \right\vert\\
                                      &> 0
  \end{align*}
  by selecting $\ve$ such that $\ve \inf_{z\in S\left( 0,1 \right)}\left\vert g(z) \right\vert < \inf_{z\in S\left( 0,1 \right)}\left\vert f(z) \right\vert$. Thus, if we set $m_1 = \inf_{z\in S\left( 0,1 \right)}\left\vert f(z) \right\vert$ and $m_2 = \sup_{z\in S\left( 0,1 \right)}\left\vert g(z) \right\vert$, we have for $\ve < \frac{m_1}{m_2}$ and all $z\in S\left( 0,1 \right)$,
  \begin{align*}
    \left\vert f_{\ve}(z) - f(z) \right\vert &\leq \left\vert \ve g(z) \right\vert\\
                                             &\leq \ve m_2\\
                                             &< m_1\\
                                             &< \left\vert f(z) \right\vert,
  \end{align*}
  whence $f(z)$ and $f_{\ve}(z)$ have the same number of zeros in $ \D $ counted with multiplicity.
\end{example}
\subsection{Residue Integrals}%
\begin{example}
  We will evaluate
  \begin{align*}
    I &= \int_{0}^{\infty} \frac{1}{1 + x^{n}}\:dx
  \end{align*}
  via contour integration. Toward this end, let $f(z) = \frac{1}{1 + z^{n}}$. We observe that $z^{n} + 1$ has roots for $e^{i\theta}$ at values $\theta = \frac{\pi + 2\pi k}{n}$ for $0\leq k < n$.%\newline

  We take the closed contour $\gamma_R$ given by
  \begin{align*}
    \oint_{\gamma_R}^{} f(z)\:dz &= \int_{0}^{R} f(x)\:dx + \int_{0}^{\frac{2\pi}{n}} f\left( Re^{i\theta} \right)\:d\left( Re^{i\theta} \right) + \int_{R}^{0} f\left( xe^{i\left( 2\pi/n \right)} \right)\:d\left( xe^{i\left( 2\pi/n \right)} \right).
  \end{align*}
  The left-hand side encloses the residue of $f$ at $e^{i\pi/n}$, so by the Residue Theorem, since $f$ has a simple pole at $e^{i\pi/n}$,
  \begin{align*}
    \oint_{\gamma_R}^{} f(z)\:dz &= 2\pi i \res\left( f;e^{i\pi/n} \right)\\
                                 &= 2\pi i \frac{1}{n\left( e^{i\pi/n} \right)^{n-1}}\\
                                 &= 2\pi i \frac{1}{n\left( e^{i\frac{\left( n-1 \right)\pi}{n}} \right)}\\
                                 &= \frac{2\pi}{n} \frac{e^{i\pi/2}}{e^{i\pi\frac{n-1}{n}}}\\
                                 &= \frac{2\pi}{n} e^{i\frac{\pi}{n} - \frac{\pi}{2}}.
  \end{align*}
  We now observe that the original integral expression equals
  \begin{align*}
    \oint_{\gamma_R}^{} f(z)\:dz &= \int_{0}^{R} \frac{1}{1 + x^{n}}\:dx + \int_{0}^{\frac{2\pi}{n}} \frac{1}{1 + R^{n}e^{in\theta}}iRe^{i\theta}\:d\theta + e^{i\left( 2\pi/n \right)}\int_{R}^{0} \frac{1}{1 + x^{n}}\:dx\\
                                 &= \left( 1-e^{i\left( 2\pi/n \right)} \right) I + \int_{0}^{2\pi/n} \frac{1}{1 + R^{n}e^{in\theta}}iRe^{i\theta}\:d\theta.
  \end{align*}
  Estimating the second integral, we get for $R > 1$,
  \begin{align*}
    \left\vert \int_{0}^{R} \frac{iRe^{i\theta}}{1 + R^{n}e^{in\theta}}\:d\theta \right\vert &\leq \frac{\frac{2\pi}{n}R}{R^{n}-1},
  \end{align*}
  so that the integral tends to $0$ as $R\rightarrow\infty$. Thus,
  \begin{align*}
    2\pi i \res\left( f;e^{i\pi/n} \right) &= \lim_{R\rightarrow\infty} \oint_{\gamma_R}^{} f(z)\:dz\\
                                           &= \left( 1-e^{i\left( 2\pi/n \right)} \right) I,
  \end{align*}
  whence
  \begin{align*}
    I &= \frac{2\pi}{n} \frac{e^{i\left( \frac{\pi}{n}-\frac{\pi}{2} \right)}}{1-e^{i\left( 2\pi/n \right)}}\\
      &= \frac{2\pi}{n} \frac{e^{-i\frac{\pi}{2}}}{e^{-i\pi/n} - e^{i\pi/n}}\\
      &= \frac{2\pi}{n} \frac{-i}{-2i \sin\left( \frac{\pi}{n} \right)}\\
      &= \frac{\pi}{n\sin\left( \frac{\pi}{n} \right)}.
  \end{align*}
\end{example}
\section{Old Exams}%
\subsection{\href{https://math.virginia.edu/graduate/exams/analysis/2019Aug_complex.html}{August 2019}}%
\begin{problem}[Problem 1]
  Let $\xi$ be a nonnegative real number. Compute
  \begin{align*}
    \int_{-\infty}^{\infty} \frac{e^{ix\xi}}{x^2 + 1}\:dx.
  \end{align*}
\end{problem}
\begin{solution}
  We consider
  \begin{align*}
    f(z) &= \frac{e^{i\xi z}}{z^2 + 1}\\
    \int_{-\infty}^{\infty} \frac{e^{i\xi x}}{x^2 + 1}\:dx &= \lim_{R\rightarrow\infty} \int_{-R}^{R} \frac{e^{i\xi x}}{x^2 + 1}\:dx.
  \end{align*}
  We consider the contour $\gamma_R$ given closing with the semicircle of radius $R$ in the upper half-plane, parametrized by $\set{Re^{i\theta} | 0\leq \theta \leq \pi}$. Then,
  \begin{align*}
    \oint_{\gamma_R}^{} f(z)\:dz &= 2\pi i \res\left( f;i \right)\\
                                 &= \int_{-R}^{R} \frac{e^{i\xi x}}{x^2 + 1}\:dx + \int_{0}^{\pi} \frac{e^{i\xi Re^{i\theta}}}{R^2e^{2i\theta} + 1}iRe^{i\theta}\:d\theta.
  \end{align*}
  On the circular integral, we observe that for $R > 1$,
  \begin{align*}
    \left\vert \int_{0}^{\pi} \frac{iRe^{i\theta}e^{i\xi R\left( \cos\left( \theta \right) + i\sin\left( \theta \right) \right)}}{R^2 e^{2i\theta} + 1}\:dz \right\vert &\leq \pi \frac{Re^{-R\xi \sin\left( \theta \right)}}{R^2 - 1}\\
                                                                                                                                                                              &\rightarrow 0
  \end{align*}
  as $R\rightarrow\infty$. Therefore, since the pole at $i$ is simple, we get
  \begin{align*}
    2\pi i \res\left( f;i \right) &= 2\pi i \left( \lim_{z\rightarrow i} \frac{\left( z-i \right)e^{i\xi x}}{\left( z-i \right)\left( z+i \right)} \right)\\
                                  &= \pi e^{-\xi},
  \end{align*}
  whence
  \begin{align*}
    \pi e^{-\xi} &= \lim_{R\rightarrow\infty} \oint_{\gamma_R}^{} f(z)\:dz\\
                 &= \lim_{R\rightarrow\infty} \int_{-R}^{R} \frac{e^{i\xi x}}{x^2 + 1}\:dx\\
                 &= \int_{-\infty}^{\infty} \frac{e^{i\xi x}}{x^2 + 1}\:dx.
  \end{align*}
\end{solution}
\begin{problem}[Problem 2]
  Let $f$ be an entire function, and suppose there is some $\alpha\in \left( 0,\infty \right)$ such that
  \begin{align*}
    \left\vert f(z) \right\vert &\leq C\left\vert z \right\vert^{\alpha}
  \end{align*}
  for all $z\geq 1$. Show that $f$ is a polynomial.
\end{problem}
\begin{solution}
  From the Archimedean property, we know that there is some natural number $N$ such that $N > \alpha$. We observe then that, from Cauchy's estimates,
  \begin{align*}
    \left\vert f^{(N)}\left( z \right) \right\vert &\leq \frac{N!}{r^{N}} \sup_{|z| = r} \left\vert f(z) \right\vert\\
                                                   &\leq \frac{N!}{r^{N}} \sup_{|z| = r}C\left\vert z \right\vert^{\alpha}\\
                                                   &= \frac{CN!}{r^{N-\alpha}}\\
                                                   &\rightarrow 0
  \end{align*}
  as $r\rightarrow\infty$, whence the Taylor expansion for $f$ about $0$ terminates at some $N$. In particular, this means that $f$ is a polynomial.
\end{solution}
\begin{problem}[Problem 3]
  Let $f$ be an entire function. Suppose that $\lim_{z\rightarrow\infty}f(z) = \infty$. Show that $f$ is a polynomial.
\end{problem}
\begin{solution}
  Consider the transformation $z\mapsto 1/z$, giving
  \begin{align*}
    \lim_{z\rightarrow 0} f\left(1/z\right) &= \lim_{z\rightarrow\infty} f(z)\\
                               &= \infty.
  \end{align*}
  In particular, from the classification of singularities, this means that $f\left( 1/z \right)$ has a pole at $0$. This gives some $n$ such that
  \begin{align*}
    f\left( 1/z \right) &= \sum_{k=0}^{n} a_k z^{-k},
  \end{align*}
  whence
  \begin{align*}
    f(z) &= \sum_{k=0}^{n} a_kz^{k},
  \end{align*}
  so $f$ is a polynomial.
\end{solution}
\begin{problem}[Problem 4]
  Let $\Omega = \set{z\in \C | \re\left( z \right) > 0}$. Suppose $ f\colon \overline{\Omega}\rightarrow \C $ be continuous with $f|_{\Omega}$ holomorphic. Suppose $\left\vert f\left( iy \right) \right\vert\leq 1$ for all $y\in \R$ and $\left\vert f\left( z \right) \right\vert\leq 2$ for all $z\in \Omega$. Show that in fact $\left\vert f(z) \right\vert\leq 1$ for all $z\in \Omega$.
\end{problem}
\begin{solution}
  Consider the function
  \begin{align*}
    f_{\ve}(z) &= \frac{f(z)}{1 + \ve z}.
  \end{align*}
  We observe that
  \begin{align*}
    \left\vert f_{\ve}(z) \right\vert &= \frac{\left\vert f(z) \right\vert}{\left\vert 1 + \ve z \right\vert}\\
                                      &\leq \frac{2}{\left\vert 1 + \ve z \right\vert}\\
                                      &\leq \frac{2}{\ve \left\vert \im(z) \right\vert}.
  \end{align*}
  Now, we observe that for $z$ in the rectangle with corners $i2/\ve$, $-i2/\ve$, $1/\ve + i2/\ve$, and $1/\ve - i2/\ve$, that
  \begin{align*}
    \left\vert f_{\ve}(z) \right\vert &\leq 1
  \end{align*}
  for all $z$ on this rectangle, so by the maximum modulus principle, the inequality holds on the interior of the rectangle, and
  \begin{align*}
    \left\vert f_{\ve}(z) \right\vert &\leq 1
  \end{align*}
  for all $z$ in $\Omega$ outside this this rectangle, so that
  \begin{align*}
    \left\vert f(z) \right\vert &= \lim_{\ve\rightarrow 0} \left\vert f(z) \right\vert\\
                                &\leq 1
  \end{align*}
  for all $z\in \Omega$.
\end{solution}
\begin{problem}[Problem 5]
  Let $\D = \set{z | \left\vert z \right\vert < 1}$. Let $\mathcal{F}$ be a family of holomorphic functions on $\D$, and that $\sup_{f\in \mathcal{F}}\left\vert f(0) \right\vert < \infty$. Show that $\mathcal{F}$ is normal if and only if $\set{f' | f\in \mathcal{F}}$ is normal.
\end{problem}
\begin{solution}
  Call the family $\mathcal{G} = \set{f' | f\in \mathcal{F}}$. First, we observe that if $\left( f_n \right)_n\subseteq \mathcal{F}$ is a sequence with convergent subsequence $\left( f_{n_k} \right)_k\rightarrow f\colon \D\rightarrow \C$ uniformly on compact sets, then it has been well-established that $\left( f_{n_k}' \right)_k\rightarrow f'$ uniformly on compact sets, whence $\left( f_n' \right)_n\subseteq \mathcal{G}$ admits a convergent subsequence.%\newline

  Now, let $\left( f_n \right)_n\subseteq \mathcal{F}$, so that $\left( f_n' \right)_n\subseteq \mathcal{G}$. Then, $\left( f_n' \right)_n$ admits a subsequence $\left( f_{n_k}' \right)_k\rightarrow g\colon \D\rightarrow \C$.%\newline

  First, we observe that since $\D$ is simply connected, $g$ admits an antiderivative $f\colon \D\rightarrow\C$. We will show that $\left( f_{n_k} \right)_k\rightarrow f$ uniformly on compacts.%\newline

  Let $K\subseteq \D$ be compact, and let $z\in K$. Let $\left( K_m \right)_m$ be an exhaustion of $\D$ by closed balls of radius $\frac{m}{m+1}$. Then, there is some $M$ such that $K\subseteq K_M^{\circ}$. We observe that the path $\gamma\colon [0,1]\rightarrow \D$ given by $\gamma(t) = tz$ is then contained wholly in $K_M$. Furthermore, we have
  \begin{align*}
    \left\vert f_{n_k}(z) - f(z) \right\vert &\leq \left\vert \int_{0}^{1} z\left( f_{n_k}'(tz) - g(tz) \right)\:dt \right\vert\\
                                             &\leq \left\vert z \right\vert \sup_{t\in [0,1]}\left\vert f_{n_k}'(tz) - g(tz) \right\vert\\
                                             &\leq \sup_{z\in K_M} \left\vert f_{n_k}'(z) - g(z) \right\vert\\
                                             &\rightarrow 0
  \end{align*}
  whence
  \begin{align*}
    \sup_{z\in K} \left\vert f_{n_k}(z) - f(z) \right\vert &\leq \sup_{z\in K_M} \left\vert f_{n_k}(z) - f(z) \right\vert\\
                                                           &\leq \sup_{z\in K_M} \left\vert f_{n_k}'(z) - g(z) \right\vert\\
                                                           &\rightarrow 0
  \end{align*}
  so that $\left( f_{n_k} \right)_k\rightarrow f$ uniformly on $K$. Thus, $\mathcal{F}$ is normal.
\end{solution}
\subsection{\href{https://math.virginia.edu/graduate/exams/analysis/2020Jan_complex.html}{January 2020}}%
\begin{problem}[Problem 1]
  Compute
  \begin{align*}
    \lim_{R\rightarrow\infty} \int_{-R}^{R} \frac{\sin\left( x \right)}{x}\:dx
  \end{align*}
  using residue theory.
\end{problem}
\begin{solution}
  We start by taking $f(z) = \frac{e^{iz}}{z}$, and let $\gamma_R$ be the contour defined by a semicircle below the real axis with radius $\delta $ centered at the origin and an outer semicircle with radius $R$, alongside rays from $-R$ to $-\delta$ and $\delta$ to $R$. We observe then that
  \begin{align*}
    \oint_{\gamma_R}^{} f(z)\:dz &= \int_{-R}^{-\delta} \frac{e^{ix}}{x}\:dx + \int_{\delta}^{R} \frac{e^{ix}}{x}\:dx\\
                                 &+ i\int_{\pi}^{2\pi} \frac{e^{i\delta e^{i\theta}}}{\delta e^{i\theta}}\delta e^{i\theta}\:d\theta\\
                                 &+ \int_{0}^{\pi} \frac{e^{iRe^{i\theta}}}{Re^{i\theta}}iRe^{i\theta}\:d\theta.
  \end{align*}
  The left-hand side evaluates to
  \begin{align*}
    2\pi i \res\left( f;0 \right) &= 2\pi i \lim_{z\rightarrow 0} \left( \frac{ze^{iz}}{z} \right)\\
                                  &= 2\pi i.
  \end{align*}
  Now, estimating the integral in $R$, we have
  \begin{align*}
    \left\vert \frac{iRe^{i\theta}e^{iRe^{i\theta}}}{Re^{i\theta}} \right\vert &= \left\vert e^{iR\left( \cos\left( \theta \right) + i\sin\left( \theta \right) \right)} \right\vert\\
                                                                               &\leq e^{-R},
  \end{align*}
  whence the integral over $R$ goes to zero. Finally, we observe that
  \begin{align*}
    \left\vert e^{i\delta e^{i\theta}}\right\vert &= \left\vert e^{i\delta\left( \cos\left( \theta \right) + i\sin\left( \theta \right) \right)} \right\vert\\
                                                  &= e^{-\delta\sin\left( \theta \right)}
  \end{align*}
  For a fixed $\theta$, we observe that $e^{-\delta\sin\left( \theta \right)}\rightarrow 0$ uniformly as $\delta\rightarrow 0$, whence upon taking these various limits, we get
  \begin{align*}
    2\pi i &= i \lim_{R\rightarrow\infty}\int_{-R}^{R} \frac{\sin\left( x \right)}{x}\:dx + i\pi,
  \end{align*}
  so that
  \begin{align*}
    \lim_{R\rightarrow \infty} \int_{-R}^{R} \frac{\sin\left( x \right)}{x}\:dx &= \pi.
  \end{align*}
\end{solution}
\begin{problem}[Problem 2]
  Let $f$ be a nonconstant entire function. Show that $f\left( \C \right)$ is dense in $\C$.
\end{problem}
\begin{solution}
  Suppose toward contradiction that there were some $w\in \C$ such that there exists $r > 0$ with $U\left( w,r \right)\notin f\left( \C \right)$. Then, we observe that the function
  \begin{align*}
    g(z) &= \frac{1}{f(z) - w}
  \end{align*}
  is entire since $w\notin f\left( \C \right)$ and has
  \begin{align*}
    \left\vert g(z) \right\vert &= \frac{1}{\left\vert f(z) - w \right\vert}\\
                                &\leq \frac{1}{r},
  \end{align*}
  since $\left\vert f(z)- w \right\vert\geq r$. Thus, $g$ is constant by Liouville's Theorem, so $f$ is constant, which contradicts the assumption that $f$ is nonconstant.
\end{solution}
\begin{problem}[Problem 3]
  Let $f$ be an entire function. Assume there is a sequence $\left( r_n \right)_n\subseteq \left( 0,\infty \right)$ with $r_n\rightarrow\infty$ and constants $C,\alpha\in \left( 0,\infty \right)$ such that
  \begin{align*}
    \sup_{|z| = r_n} \left\vert f(z) \right\vert &\leq Cr_n^{\alpha}.
  \end{align*}
  Show that $f$ is a polynomial.
\end{problem}
\begin{solution}
  Let $N\in \N$ be such that $\alpha < N$. Then, by Cauchy's estimate, we have
  \begin{align*}
    \left\vert f^{(N)}\left( 0 \right) \right\vert &\leq \frac{N!}{r_n^{N}} \sup_{|z| = r_n} \left\vert f(z) \right\vert\\
                                                   &\leq \frac{CN!}{r_n^{N-\alpha}},
  \end{align*}
  and since $r_n\rightarrow\infty$, so too does $r_n^{N-\alpha}$, whence $\left\vert f^{(N)}\left( 0 \right) \right\vert = 0$. In particular, this means that the power series expansion of $f$ about $0$, which is equal to $f$ on $\C$ as $f$ is entire, terminates past $N$, meaning that $f$ is a polynomial.
\end{solution}
\begin{problem}[Problem 4]
  Let $\left( f_n \right)_n$ be a sequence of holomorphic functions, and suppose $\left( f_n \right)_n\rightarrow f$ uniformly on compact subsets of $\D$ to a function $f$. Suppose $f$ has no zeros on $S\left(0,1/2\right)$. Show that there exists an $N\in \N$ such that if $n,m\geq N$, then $f_m$ and $f_n$ have the same number of zeros in $U\left(0,1/2\right)$, counted with multiplicity.
\end{problem}
\begin{solution}
  Since $S\left( 0,1/2 \right)$ is compact, it follows that $\left( f_n \right)_n\rightarrow f$ uniformly on $S\left( 0,1/2 \right)$. Let $\ve_0 > 0$ be such that $\inf_{z\in S\left( 0,1/2 \right)}\left\vert f(z) \right\vert \geq \ve_0 > 0$. There exists $N\in \N$ such that for all $n\geq N$, we have
  \begin{align*}
    \sup_{|z| = 1/2}\left\vert f_n(z) - f(z) \right\vert &< \frac{\ve_0}{2},
  \end{align*}
  whence
  \begin{align*}
    \left\vert f_n(z) \right\vert &= \left\vert f_n(z) - f(z) + f(z) \right\vert\\
                                  &\geq \left\vert f(z) \right\vert - \left\vert f_n(z) - f(z) \right\vert\\
                                  &\geq \frac{\ve_0}{2}\\
                                  &> 0.
  \end{align*}
  In particular, for every $n\geq N$, we have
  \begin{align*}
    \left\vert f_n(z) - f(z) \right\vert &< \frac{\ve_0}{2}\\
                                         &< \ve_0\\
                                         &\leq \left\vert f(z) \right\vert\\
                                         &\leq \left\vert f(z) \right\vert + \left\vert f_n(z) \right\vert,
  \end{align*}
  so by Rouché's Theorem, it follows that for every $n\geq N$, $f_n$ has the same number of zeros in $U\left( 0,1/2 \right)$ as $f$, whence for all $m,n\geq N$, $f_m$ and $f_n$ have the same number of zeros on $U\left( 0,1/2 \right)$.
\end{solution}
\begin{problem}[Problem 5]
  Let $U\subseteq \C$ be open, and let $f_n\colon U\rightarrow \C$ be a sequence of analytic functions. Suppose that $f_n$ converge pointwise to a function $f\colon U\rightarrow \C$. If $\left\vert f_n \right\vert\leq 1$ for all $n$, show that $f$ is analytic.
\end{problem}
\begin{solution}
  By Montel's Theorem, the collection $\mathcal{F} = \set{f_n | n\in\N}$ is normal as it is locally (globally, even) bounded by $1$.%\newline

  Our goal is to show that under this condition, $\left( f_n \right)_n\rightarrow f$ uniformly on compact subsets. The completeness of the space $H\left( U \right)$ under the topology of uniform convergence on compact subsets would then give us our desired result.%\newline

  Since $\mathcal{F}$ is normal, it follows that for any subsequence $\left( f_{n_k} \right)_k$, there is a sub-subsequence $\left( f_{n_{k_{j}}} \right)_j\rightarrow h_k$ uniformly on compact subsets, where $h_k$ is necessarily holomorphic. Yet, since every subsequence of $\left( f_n \right)_n$ converges pointwise to $f$, it follows that $h_k = f$ for all $k$. In particular, this means that every subsequence $\left( f_n \right)_k$ admits a sub-subsequence $\left( f_{n_{k_j}} \right)_j\rightarrow f$ uniformly on compact subsets. We show that this implies that $\left( f_n \right)_n\rightarrow f$ uniformly on compact subsets.%\newline

  Suppose this were not the case. Then, there would be some compact $K\subseteq U$, some $\ve_0 > 0$, and some subsequence $\left( f_{n_k} \right)_k$ such that $\norm{f_{n_k} - f}_{K}\geq \ve_0$ for all $k$. Yet, since $\left( f_{n_k} \right)_k$ admits a convergent subsequence $\left( f_{n_{k_j}} \right)_j$, it follows that for some $J$, we have $\norm{f_{n_{k_j}} - f}_{K} < \ve_0$ for all $j\geq J$, which is a contradiction. In particular, this means that $\left( f_n \right)_n\rightarrow f$ uniformly on compact subsets, so $f$ is holomorphic.
\end{solution}
\begin{solution}[Correction of the Final Paragraph]
  Suppose $\left( f_n \right)_n$ does not converge uniformly on compacts to $f$. Then, there exists some $\ve_0 > 0$, some compact $K\subseteq U$, some subsequence $\left( f_{n_j} \right)_j$, and some points $z_j\in K$ such that
  \begin{align*}
    \left\vert f_{n_j}\left(z_j\right) - f\left( z_j \right) \right\vert \geq \ve_0
  \end{align*}
  for all $j$. Since $\left( f_{n_j} \right)$ admits a subsequence that converges uniformly on compact subsets to a holomorphic function $g$, but pointwise convergence gives that $g(z) = f(z)$, which means that at $z_j$, the above inequality cannot hold.
\end{solution}
\subsection{\href{https://math.virginia.edu/graduate/exams/analysis/2020Aug_complex.html}{August 2020}}%
\begin{problem}[Problem 1]
  Let $f$ be an entire function, and assume that
  \begin{align*}
    \sup_{|z| = r} \left\vert f(z) \right\vert &\leq 10 \ln\left( r \right)
  \end{align*}
  for all $r \geq 100$. Show that $f$ is constant.
\end{problem}
\begin{solution}
  If $R > 100$, then by Cauchy's estimates, we have
  \begin{align*}
    \left\vert f'(0) \right\vert &\leq \frac{1}{R} \sup_{|z| = R} \left\vert f(z) \right\vert\\
                                 &\leq \frac{1}{R} \left( 10\ln(R) \right)\\
                                 &\rightarrow 0
  \end{align*}
  as $R\rightarrow\infty$. In particular, this means that the power series expansion for $f(z)$ centered at $0$ terminates after the constant term, whence $f$ is constant.
\end{solution}
\begin{problem}[Problem 2]
  Let $\Omega = \set{z\in \C | 0 < |z| < 1}$. Suppose that
  \begin{enumerate}[(i)]
    \item $\left\vert f(z) \right\vert > 1$ for all $z\in \Omega$;
    \item there exists $\left( z_k \right)_k\subseteq \Omega$ with $\left( z_k \right)_k\rightarrow 0$ and $\left\vert f\left( z_k \right) \right\vert \leq 10$ for all $k$.
  \end{enumerate}
  Is $0$ a removable singularity?
\end{problem}
\begin{solution}
  First, since $\left\vert f(z) \right\vert > 1$, we observe that the function
  \begin{align*}
    g(z) &= \frac{1}{f(z)}
  \end{align*}
  is then holomorphic on $\Omega$, and is bounded above by $1$, meaning that by Riemann's Theorem on removable singularities, there is a holomorphic extension $ \widetilde{g} $ defined on $\D$.%\newline

  Observe that $ \widetilde{g} $ is continuous at $z = 0$, so that with the aforementioned sequence $\left( z_k \right)_k$, we have
  \begin{align*}
    \left\vert \widetilde{g}(0) \right\vert &= \lim_{k\rightarrow\infty} \left\vert \widetilde{g}\left( z_k \right) \right\vert\\
                                            &= \lim_{k\rightarrow\infty} \left\vert g\left( z_k \right) \right\vert\\
                                            &= \lim_{k\rightarrow\infty} \left\vert \frac{1}{f\left( z_k \right)} \right\vert\\
                                            &\geq \frac{1}{10}.
  \end{align*}
   In particular, $\widetilde{g}(0)\neq 0$, so that the extension
   \begin{align*}
     \widetilde{f}(z) &= \frac{1}{\widetilde{g}(z)}
   \end{align*}
   is a holomorphic extension of $f$ to $\D$, whence $0$ is a removable singularity.
\end{solution}
\begin{problem}[Problem 3]
  Compute
  \begin{align*}
    I &= \int_{0}^{\infty} \frac{\ln\left( x \right)}{x^2 + 1}\:dx.
  \end{align*}
  Show all estimates.
\end{problem}
\begin{solution}
  Consider the branch of the complex logarithm that excludes the positive real axis, and let $f(z) = \frac{\ln(z)}{z^2 + 1}$. Then, using the contour hugging the positive real axis with imaginary part $\ve$, a semicircle of radius $R$, returning along the negative real axis to $-\ve$, we then get
  \begin{align*}
    \oint_{\gamma_R}^{} f(z)\:dz &= \int_{i\ve}^{R + i\ve} \frac{\ln\left( x \right)}{x^2 + 1}\:dx + \int_{-R}^{\ve} \frac{\ln\left( x \right)}{x^2 + 1}\:dx\\
                                                   &+ \int_{\lambda}^{\pi} \frac{\ln\left( Re^{i\theta} \right)}{R^2e^{2i\theta} + 1}iRe^{i\theta}\:d\theta,
  \end{align*}
  so that upon taking $\ve\rightarrow 0$ and using a $u$-substitution on the negative real axis integral, we get
  \begin{align*}
    \oint_{\gamma_R}^{} f(z)\:dz &= 2 \int_{0}^{R} \frac{\ln(x)}{x^2 + 1}\:dx + i\pi \int_{0}^{R} \frac{1}{x^2 + 1}\:dx\\
                                 &+ \int_{0}^{\pi} \frac{\ln\left( R \right) + i\theta}{R^2e^{2i\theta} + 1}iRe^{i\theta}\:d\theta.
  \end{align*}
  Estimating the second integral, for $R > 1$, we get
  \begin{align*}
    \left\vert \int_{0}^{\pi} \frac{\ln\left( R \right) + i\theta}{R^2e^{2i\theta} + 1}iRe^{i\theta}\:d\theta \right\vert &\leq \pi\left( \frac{R\ln\left( R \right)}{R^2 - 1} + \frac{R\theta}{R^2 - 1} \right)\\
                                                                                                                                  &\rightarrow 0,
  \end{align*}
  so that
  \begin{align*}
    2\pi i \res\left( f;i \right) &= 2\pi i \left( \lim_{z\rightarrow i} \frac{\left( z-i \right)\ln\left( i \right)}{2i} \right)\\
                                  &= \frac{\pi^2}{2}i\\
                                  &= 2 \int_{0}^{\infty} \frac{\ln\left( x \right)}{x^2 + 1}\:dx + i\pi \left( \frac{\pi}{2} \right),
  \end{align*}
  meaning that
  \begin{align*}
    \int_{0}^{\pi} \frac{\ln\left( x \right)}{x^2 + 1}\:dx &= 0.
  \end{align*}
\end{solution}
\begin{problem}[Problem 5]
  Let $\mathcal{F}$ be a family of entire functions with the property that for all circles $C$ there is a constant $M_C$ such that
  \begin{align*}
    \sup_{f\in \mathcal{F}}\sup_{|z| \in C} \left\vert f(z) \right\vert &\leq M_C.
  \end{align*}
  \begin{enumerate}[(i)]
    \item Show that there exists a sequence $\left( f_n \right)_n\subseteq \mathcal{F}$ converging to an entire function uniformly on compact subsets.
    \item Suppose that $f_n(z) \neq 7$ for all $n\geq 77$ and all $z\in \D$. Can it be the case that $g\left( \frac{1+i}{2} \right) = 7$?
  \end{enumerate}
\end{problem}
\begin{solution}\hfill
  \begin{enumerate}[(i)]
    \item We start by showing that the family $\mathcal{F}$ is a normal family. Letting $z\in \C$, take $r > 0$ such that $z\in U\left( 0,r \right)$, and let $\delta > 0$ be such that $U\left( z,\delta \right)\subseteq U\left( 0,r \right)$. By the assumptions of the problem, we have $M_r$ such that
      \begin{align*}
        \sup_{|z| = r} \left\vert f(z) \right\vert &\leq M_r
      \end{align*}
      for all $f\in \mathcal{F}$. By the maximum modulus principle, since every element of $\mathcal{F}$ is entire, and thus has a continuous extension to $B\left( 0,r \right)$, we have
      \begin{align*}
        \sup_{|z| \leq r} \left\vert f(z) \right\vert &= \sup_{|z| = r} \left\vert f(z) \right\vert,
      \end{align*}
      whence
      \begin{align*}
        \sup_{w\in U\left( z,\delta \right)} \left\vert f(z) \right\vert &\leq \sup_{|z|\leq r} \left\vert f(z) \right\vert\\
                                                                         &\leq M_r,
      \end{align*}
      for all $f\in \mathcal{F}$. Therefore, the family $\mathcal{F}$ is locally bounded, and is thus normal by Montel's Theorem.%\newline

      Therefore, if $\left( f_n \right)_n\subseteq \mathcal{F}$, there is a subsequence $\left( f_{n_k} \right)_k\rightarrow g$ for some $g\in H\left( \C \right)$ uniformly on compact subsets. We will relabel this subsequence to $\left( f_n \right)_n$ for the following question.
    \item Let $\frac{1}{\sqrt{2}} < r < 1$. Consider the sequence of functions given by $h_n = f_n - 7$. Since $7$ is a constant function, it follows that $h_n\rightarrow g - 7$ uniformly on compact subsets.%\newline

      By the assumptions of the problem, $h_n(z) \neq 0$ on $S\left( 0,r \right)$ and the $h_n$ each have no zeros in $U\left( 0,r \right)$. Thus, from Hurwitz's Theorem, it follows that either $g-7$ has no zeros on $U\left( 0,r \right)$ or is uniformly zero on $U\left( 0,r \right)$. If it is the latter case, then since $\frac{1+i}{2}\in U\left( 0,r \right)$, we have that $g\left( \frac{1+i}{2} \right) = 7$.
  \end{enumerate}
\end{solution}
\subsection{\href{https://math.virginia.edu/graduate/exams/analysis/2021Jan_complex.html}{January 2021}}%
\begin{problem}[Problem 1]
  Compute, for $\xi > 0$,
  \begin{align*}
    I &= \int_{-\infty}^{\infty} \frac{e^{ix\xi}}{x^2 - 2x + 2}\:dx.
  \end{align*}
  Show all estimates.
\end{problem}
\begin{solution}
  We consider the semicircular contour $\gamma_R$ ranging from $-R$ to $R$ centered at $0$. Letting
  \begin{align*}
    f(z) &= \frac{e^{iz\xi}}{z^2 - 2z + 2},
  \end{align*}
  we get
  \begin{align*}
    \oint_{\gamma_R}^{} f(z)\:dz &= \int_{-R}^{R} \frac{e^{ix\xi}}{x^2 - 2x + 2}\:dx + \int_{0}^{\pi} \frac{e^{i\xi Re^{i\theta}}}{R^2e^{2i\theta}-Re^{i\theta} + 2}iRe^{i\theta}\:d\theta.
  \end{align*}
  Estimating the second integral, we find that for $R > 2$
  \begin{align*}
    \left\vert \int_{0}^{\pi} \frac{iRe^{i\theta}e^{i\xi Re^{i\theta}}}{R^2e^{2i\theta}-Re^{i\theta} + 2}\:d\theta \right\vert &\leq \frac{\pi R e^{-R}}{R^2 - 2R - 2}\\
                                                                                                                                          &\rightarrow 0
  \end{align*}
  as $R\rightarrow\infty$. As a result, we get
  \begin{align*}
    \int_{-R}^{R} \frac{e^{ix\xi}}{x^2-2x+2}\:dx &= \lim_{R\rightarrow\infty} \int_{\gamma_R}^{} f(z)\:dz\\
                                                 &= 2\pi i \res\left( f;1+i \right)\\
                                                 &= \pi e^{i\xi-1}.
  \end{align*}
\end{solution}
\begin{problem}
  Suppose that $f\colon \C\rightarrow \C$ is entire, and
  \begin{align*}
    \lim_{|z|\rightarrow\infty} \frac{\left\vert f(z) \right\vert}{\left\vert z \right\vert} &= 0.
  \end{align*}
  Show that $f$ is constant.
\end{problem}
\begin{solution}
  Since $f$ is entire, and we have
  \begin{align*}
    \lim_{|z|\rightarrow\infty} \frac{\left\vert f(z) \right\vert}{\left\vert z \right\vert} &= 0,
  \end{align*}
  we then have
  \begin{align*}
    \lim_{|w|\rightarrow 0} \left\vert wf\left( 1/w \right) \right\vert &= 0,
  \end{align*}
  where we substitute $w = \frac{1}{z}$. In particular, this means
  \begin{align*}
    \lim_{w\rightarrow 0} wf\left( 1/w \right) &= 0,
  \end{align*}
  so by Riemann's theorem on removable singularities, it follows that
  \begin{align*}
    f(w) &= \sum_{n=0}^{\infty}a_nw^{-n}
  \end{align*}
  is bounded in a neighborhood of $0$. In particular, $0$ is not a pole for $f\left( 1/w \right)$, meaning that the Laurent series expansion for $f$ centered at $0$ terminates at $a_0$. Thus, $f$ is constant.
\end{solution}
\begin{problem}[Problem 3]
  Suppose that $f\colon \C\rightarrow \C$ is entire, and that there exist constants $C,R > 0$ such that $\left\vert f(z) \right\vert \geq C$ whenever $\left\vert z \right\vert \geq R$. Show that $f$ is a polynomial.
\end{problem}
\begin{solution}
  Consider the function $g\colon \C\setminus \set{0}\rightarrow \C$ given by
  \begin{align*}
    g(z) &= f\left( 1/z \right).
  \end{align*}
  We then have that $\left\vert g(z) \right\vert\geq C$ whenever $\left\vert z \right\vert \leq \frac{1}{R}$. There are then two cases.
  \begin{itemize}
    \item If $\left\vert g \right\vert$ is bounded above, then $f\left( z \right)$ is bounded on $\C\setminus B\left( 0,R \right)$, whence $f$ is bounded and entire, hence constant by Liouville's Theorem.
    \item If $\left\vert g \right\vert$ is not bounded above, then by the converse to the Casorati--Weierstrass theorem, since
      \begin{align*}
        U\left( 0,C/2 \right) \not\subseteq g\left( U\left( 0,1/R \right)\setminus \set{0} \right),
      \end{align*}
      it follows that $g$ has a pole at $0$. In particular, this means that the Laurent series expansion for $g$ is of the form
      \begin{align*}
        g(z) &= \sum_{n=0}^{m} a_nz^{-m},
      \end{align*}
      meaning that $f$ is a polynomial.
  \end{itemize}
\end{solution}
\begin{problem}[Problem 4]
  Let $U\subseteq \C$ be nonempty and open, and let $\left( f_n \right)_n\colon U\rightarrow \D$ be a sequence of holomorphic functions converging uniformly on compact sets to $f\colon U\rightarrow \C$. Suppose there is $p\in U$ such that $\left\vert f(p) \right\vert = 1$. Show that $f$ is constant.
\end{problem}
\begin{solution}
  To start, we observe that $f$ is holomorphic since the metric space $H(U)$ is complete in the topology of uniform convergence on compact subsets. Additionally, $\left( f_n \right)_n\rightarrow f$ pointwise since it converges uniformly on compact subsets. This means in particular that $f\left( U \right)\subseteq \overline{\D}$.%\newline

  Since there is some $z\in \partial \D$ with $z\in \img\left( f \right)$, it then follows that $\img\left( f \right)$ is not open, meaning that $f$ is constant by the converse of the open mapping principle.
\end{solution}
\begin{problem}[Problem 5]
  Let $f_n\colon \D\rightarrow \D$ be a sequence of holomorphic functions that converges pointwise to $0$ on $B\left( 0,1/2 \right)$.
  \begin{enumerate}[(a)]
    \item Suppose $f\colon \D\rightarrow \C$ is a limit (uniformly on compact subsets) of a subsequence of $\left( f_n \right)_n$. Show that $f = 0$.
    \item Show that $\left( f_n \right)_n\rightarrow 0$ uniformly on compact subsets.
  \end{enumerate}
\end{problem}
\begin{solution}\hfill
  \begin{enumerate}[(a)]
    \item Since $f$ is the limit of a subsequence of $\left( f_n \right)_n$ uniformly on compact subsets, it follows that $f$ is holomorphic by the completeness of $H\left( \D \right)$ in the topology of uniform convergence on compact subsets. Since $f \equiv 0$ on $B\left( 0,1/2 \right)$, the identity theorem gives that $f\equiv 0$.
    \item First, we observe that the family $\set{f_n | n\in\N}$ is a normal family by Montel's Theorem since every function in this family is bounded above by $1$, meaning that any subsequence of $\left( f_{n} \right)_n$ admits a subsequence converging to some holomorphic function $f\colon \D\rightarrow \overline{\D}$. We have earlier shown that any subsequence of $\left( f_n \right)_n$ that converges to some holomorphic function will converge to $0$, meaning that every subsequence of $\left( f_n \right)_n$ admits a subsequence converging to the same point.%\newline

      Therefore, from undergrad real analysis, we observe that if $\left( f_n \right)_n$ were not to converge to $0$, we would have some $\ve_0 > 0$ and some subsequence $\left( f_{n_k} \right)_k$ such that
      \begin{align*}
        d_K\left( f_{n_k},0 \right) &\geq \ve_0
      \end{align*}
      for all $k$, but this would contradict the fact that $\left( f_{n_k} \right)_k$ has a subsequence converging to $0$. Thus, it follows that $\left( f_n \right)_n\rightarrow 0$ uniformly on compact subsets.
  \end{enumerate}
\end{solution}
\subsection{\href{https://math.virginia.edu/graduate/exams/analysis/2022Jan_complex.html}{January 2022}}%
\begin{problem}[Problem 1]
  Compute, for $0 < \alpha <1$,
  \begin{align*}
    I &= \int_{0}^{\infty} \frac{1}{x^{\alpha}\left( x+1 \right)}\:dx.
  \end{align*}
\end{problem}
\begin{solution}
  Consider the branch of the logarithm defined on $\C\setminus \left[ 0,\infty \right)$. Define a keyhole contour $\gamma_R$ with height/inner radius $\ve$ and outer radius $R$ hugging the branch cut.%\newline

  Writing out $\gamma_R$ as $\ve\rightarrow 0$, we get
  \begin{align*}
    \int_{\gamma_R}^{} f(z)\:dz &= \int_{0}^{R} \frac{1}{x^{\alpha}\left( x+1 \right)}\:dx + \int_{0}^{2\pi} \frac{1}{R^{\alpha}e^{i\alpha\theta}\left( 1 + Re^{i\theta} \right)}iRe^{i\theta}\:d\theta + \int_{R}^{0} \frac{1}{\left( xe^{2\pi i} \right)^{\alpha}\left( 1 + x \right)}\:dx\\
                                &= \left( 1-e^{-2\pi i \alpha} \right) \int_{0}^{R} \frac{1}{x^{\alpha}\left( 1 + x \right)}\:dx + \int_{0}^{2\pi} \frac{iRe^{i\theta}}{R^{\alpha}e^{i\alpha\theta}\left( 1 + Re^{i\theta} \right)}\:d\theta.
  \end{align*}
  Estimating the second integral, we have, for $R > 1$,
  \begin{align*}
    \left\vert \int_{0}^{2\pi} \frac{iRe^{i\theta}}{R^{\alpha}e^{i\alpha\theta}\left( 1 + Re^{i\theta} \right)}\:d\theta \right\vert &\leq 2\pi \frac{R}{R^{\alpha}\left( R-1 \right)}\\
                                                                                                                                                    &\rightarrow 0.
  \end{align*}
  Thus, we have
  \begin{align*}
    \left( 1-e^{-2\pi i \alpha} \right)I &= 2\pi i \res\left( f;-1 \right)\\
                                         &= 2\pi i \frac{1}{e^{i\pi \alpha}},
  \end{align*}
  whence
  \begin{align*}
    I &= \frac{2\pi i}{2i\sin\left( \pi\alpha \right)}\\
      &= \frac{\pi}{\sin\left( \pi\alpha \right)}.
  \end{align*}
\end{solution}
\begin{problem}[Problem 2]
  Suppose that $f$ is an entire function such that there are constants $a,b > 0$ with
  \begin{align*}
    \left\vert f(z) \right\vert &\leq a + b\left\vert z \right\vert
  \end{align*}
  for all $z\in \C$. Show that $f$ is a polynomial of degree at most $1$.
\end{problem}
\begin{solution}
  Using Cauchy's Estimate, we find that
  \begin{align*}
    \left\vert f''(0) \right\vert &\leq \frac{2}{R^2}\sup_{|z| = R} \left\vert f(z) \right\vert\\
                                  &\leq \frac{2}{R^2} \sup_{|z| = R} \left( a + b\left\vert z \right\vert \right)\\
                                  &= \frac{2\left( a + bR \right)}{R^2}\\
                                  &\rightarrow 0.
  \end{align*}
  Thus, the power series expansion for $f$ about $0$ terminates after the expression $a_0 + a_1 z$, meaning $f$ is linear.
\end{solution}
\begin{problem}[Problem 3]
  Give an explicit example of an unbounded harmonic function $u\colon \D\rightarrow \left( 0,\infty \right)$ such that
  \begin{align*}
    \lim_{z\rightarrow \zeta}u(z) &= 0
  \end{align*}
  for all $\zeta\in S^{1}$ with $\zeta\neq 1$.
\end{problem}
\begin{solution}
  The Cayley Transform, $q\colon \mathbb{H}\rightarrow \D$, given by
  \begin{align*}
    q(z) &= \frac{z-i}{z+i}
  \end{align*}
  is a conformal map of $ \mathbb{H} $ to $\D$. The inverse, then,
  \begin{align*}
    q^{-1}(z) &= -i\frac{z+1}{z-1}
  \end{align*}
  is then a conformal map of $ \D$ to $\mathbb{H} $ . Furthermore, $\im\left( q^{-1}(z) \right)$ is then harmonic as it is the imaginary part of a holomorphic function, with for any $z = e^{i\theta}$ in $S^{1}\setminus \set{1}$
  \begin{align*}
    q^{-1}\left( e^{i\theta} \right) &= -i\frac{e^{i\theta} + 1}{e^{i\theta} - 1}\\
                                     &= -i\frac{i\sin\left( \theta \right)}{1-\cos\left( \theta \right)}\\
                                     &= \frac{\sin\left( \theta \right)}{1-\cos\left( \theta \right)}\\
                                     &\in \R,
  \end{align*}
  whence $\im\left( q^{-1}\left( e^{i\theta} \right) \right) = 0$ for all elements of $S^{1}$ that are not equal to $1$.
\end{solution}
\begin{problem}[Problem 4]
  Let $S = \set{z\in \C | 0 < \re(z) < 1}$. Suppose $f\colon \overline{S}\rightarrow \C$ is bounded, $f|_{S}$ is holomorphic, and
  \begin{align*}
    \sup_{t\in \R} \left( \max\left( \left\vert f\left( it \right) \right\vert,\left\vert f\left( 1 + it  \right) \right\vert \right) \right) &\leq 1.
  \end{align*}
  Show that $\left\vert f(z) \right\vert\leq 1$ for all $z\in S$.
\end{problem}
\begin{solution}
  Let $f_{\ve}\left( z \right) = \frac{f(z)}{1 + \ve z}$, and let $K = \sup_{z\in \overline{S}} \left\vert f(z) \right\vert$. Then, we observe that
  \begin{align*}
    \left\vert f_{\ve}(z) \right\vert &\leq \frac{\left\vert f(z) \right\vert}{\ve \left\vert \im(z) \right\vert}.
  \end{align*}
  In particular, if $\left\vert \im(z) \right\vert = K/\ve$, we have $ \left\vert f_{\ve}(z) \right\vert\leq 1 $ on the rectangle with vertices at $\pm iK/\ve,1\pm iK/\ve$, meaning that by the maximum modulus principle, $\left\vert f_{\ve}(z) \right\vert \leq 1$ on this rectangle, while $\left\vert f_{\ve}(z) \right\vert \leq 1$ outside the rectangle by the fact that $\left\vert f \right\vert$ is bounded by $K$.%\newline

  Thus, since $\ve > 0$ is arbitrary, we have $\left\vert f(z) \right\vert\leq 1$ on $ \overline{S} $.
\end{solution}
\begin{problem}[Problem 5]
  Let $\left( M_n \right)_{n=0}^{\infty}$ be a sequence of positive numbers such that $\sum_{n=0}^{\infty}M_nz^{n}$ has a radius of convergence at least $1$.%\newline

  Let $\mathcal{F}$ be the set of holomorphic functions on $\D$ that satisfy
  \begin{align*}
    \left\vert \frac{f^{(n)}(0)}{n!} \right\vert &\leq M_n.
  \end{align*}
  Show that $\mathcal{F}$ is normal.
\end{problem}
\begin{solution}
  Let $z\in \D$, $\delta > 0$ such that $B\left( z,\delta \right)\subseteq \D$. For $f\in \mathcal{F}$ and $w\in \D$, uniform convergence and the triangle inequality provide
  \begin{align*}
    \left\vert f(w) \right\vert &\leq \sum_{n=0}^{\infty} \left\vert \frac{f^{(n)}(0)}{n!} \right\vert \left\vert z \right\vert^{n}\\
                                &\leq \sum_{n=0}^{\infty} \left\vert \frac{f^{(n)}(0)}{n!} \right\vert \left( \left\vert z \right\vert + \delta \right)^{n}\\
                                &\leq \sum_{n=0}^{\infty} M_n\left( \left\vert z \right\vert + \delta \right)^{n}\\
                                &\eqcolon C.
  \end{align*}
  In particular, this means that $\mathcal{F}$ is locally bounded, so $\mathcal{F}$ is normal by Montel's Theorem.
\end{solution}
\subsection{\href{https://math.virginia.edu/graduate/exams/analysis/2022Aug_complex.html}{August 2022}}%
\begin{problem}[Problem 1]
  Compute, for $\xi > 0$, $a > 0$, and $b > 0$, the quantity
  \begin{align*}
    I &= \int_{-\infty}^{\infty} \frac{e^{ix\xi}}{\left( x-a \right)^2 + b}\:dx.
  \end{align*}
  Show all estimates.
\end{problem}
\begin{solution}
  Let $\gamma_R$ be the semicircle in the upper half-plane with radius $R$, so that
  \begin{align*}
    \oint_{\gamma_R}^{} \frac{e^{iz\xi}}{\left( z-a \right)^2 + b}\:dz &= \int_{-R}^{R} \frac{e^{ix\xi}}{\left( x-a \right)^2 + b}\:dx + \int_{0}^{\pi} \frac{e^{i\xi Re^{i\theta}}}{\left( Re^{i\theta}-a \right)^2 + b}iRe^{i\theta}\:d\theta.
  \end{align*}
  Estimating the second integral, we have that for sufficiently large $R$,
  \begin{align*}
    \left\vert \int_{0}^{\pi} \frac{e^{i\xi R\left( \cos\theta + i\sin\theta \right)}}{R^2e^{2i\theta}-2aRe^{i\theta} + a^2 + b}\:d\theta \right\vert &\leq \pi \frac{Re^{-\xi R}}{R^2-2aR-a^2-b}\\
                                                                                                                                                            &\rightarrow 0,
  \end{align*}
  whence
  \begin{align*}
    \int_{-\infty}^{\infty} \frac{e^{ix\xi}}{\left( x-a \right)^2 + b}\:dx &= \oint_{\gamma_R}^{} \frac{e^{iz\xi}}{\left( z-a \right)^2 + b}\:dz\\
                                                                           &= 2\pi i \res\left( \frac{e^{iz\xi}}{\left( z-a \right)^2 + b}; a+bi \right)\\
                                                                           &= 2\pi i \left( \frac{e^{i\xi\left( a + bi \right)}}{2bi} \right)\\
                                                                           &= \frac{\pi e^{-b\xi}}{b}e^{ia\xi}.
  \end{align*}
\end{solution}
\begin{problem}[Problem 2]
  Let $f$ be an entire function such that
  \begin{align*}
    \lim_{R\rightarrow\infty}\left( \sup_{|z| > R} \frac{\left\vert f(z) \right\vert}{\left\vert z \right\vert} \right) &= 0.
  \end{align*}
  Show that $f$ is constant.
\end{problem}
\begin{solution}
  For any $\ve > 0$, there is sufficiently large $R$ such that for all $|z| > R$, we have
  \begin{align*}
    \left\vert \frac{f(z)}{z} \right\vert &\leq \ve.
  \end{align*}
  Now, taking the coordinate transformation $z\mapsto \frac{1}{z}$, we have that for all $z\in \dot{U}\left( 0,1/R \right)$,
  \begin{align*}
    \left\vert zf(1/z) \right\vert &\leq \ve,
  \end{align*}
  so that $\lim_{z\rightarrow 0}zf(1/z) = 0$. In particular, this means that $f(1/z)$ has a removable singularity at $0$.

  Writing
  \begin{align*}
    f(z) &= \sum_{n=0}^{\infty}a_nz^{n}\\
    f(1/z) &= \sum_{n=0}^{\infty} a_nz^{-n},
  \end{align*}
  meaning that $f(1/z) = a_0$, so $f$ is constant.
\end{solution}
\begin{problem}[Problem 3]
  Suppose $f\colon \C\setminus B\left( 0,R \right)\rightarrow \C$ is analytic, and $\lim_{z\rightarrow\infty}f(z) = 0$. Show that $\lim_{z\rightarrow\infty}zf(z)$ exists.
\end{problem}
\begin{solution}
  If we let $g(z) = f(1/z)$, we observe that $g\colon \dot{U}(0,1/R)\rightarrow \C$ has $\lim_{z\rightarrow 0} g(z) = 0$. In particular, this means that $g$ is bounded in a neighborhood of $0$, so there is an extension with $ \widetilde{g}(z) = z^{\ell}h(z) $ for some holomorphic function $h\colon U(0,1/R)\rightarrow \C$, $h(z) \neq 0$, and $\ell \geq 1$.

  We have that
  \begin{align*}
    \lim_{z\rightarrow\infty} zf(z) &= \lim_{z\rightarrow 0} z^{-1}g(z)\\
                                    &= \lim_{z\rightarrow 0} z^{- 1} \widetilde{g}(z)\\
                                    &= \lim_{z\rightarrow 0} z^{\ell - 1}h(z)\\
                                    &\neq \infty,
  \end{align*}
  seeing as $\ell \geq 1$.
\end{solution}
\begin{problem}[Problem 4]
  Suppose $\left( f_n \right)_n$ is a sequence of entire functions that converges uniformly on compact subsets of $\C$ to a polynomial $p$ of degree $d$. Show that there is some $N$ such that for all $n\geq N$, the function $f_n$ has at least $d$ zeros.
\end{problem}
\begin{solution}
  Let $R$ be so large such that $p$ has no zeros on $S(0,R)$ and all the zeros of $p$ are contained in $U(0,R)$. Letting $m = \inf_{|z| = R} \left\vert p(z) \right\vert$, uniform convergence on compact subsets implies that there is $\ell$ such that for all $|z| = R$ and all $n\geq \ell$, we have 
  \begin{align*}
    \left\vert f_n(z) - p(z) \right\vert < m/3.
  \end{align*}
  In particular, we have that for all $|z| = R$ and all $n\geq \ell$,
  \begin{align*}
    \left\vert f_n(z)  \right\vert &\geq \left\vert p(z) \right\vert - \left\vert f_n(z) - p(z) \right\vert\\
                                   &\geq 2m/3\\
                                   &> 0,
  \end{align*}
  whence $f_n(z)$ has no zeros on $S(0,R)$ for all such $\ell$. Finally, we have for all $n\geq \ell$
  \begin{align*}
    \left\vert f_n(z) - p(z) \right\vert &< m/3\\
                                         &< m\\
                                         &\leq \left\vert p(z) \right\vert,
  \end{align*}
  so by Rouché's Theorem, $f_n(z)$ and $p(z)$ have $d$ zeros in $U(0,R)$ for all $n\geq \ell$.
\end{solution}
\begin{problem}[Problem 5]
  Let $U$ be open and connected and fix $p\in U$. Let $\mathcal{F}$ be the family of holomorphic functions $f\colon U\rightarrow \mathbb{H}$ such that $f(p) = i$ for all $f\in \mathcal{F}$. Show that $\mathcal{F}$ is normal.
\end{problem}
\begin{solution}
  Let $q\colon \mathbb{H}\rightarrow \D$ be the Cayley transform. Then, if $\left( f_n \right)_n\subseteq \mathcal{F}$, then 
  \begin{align*}
    \left( q\circ f_n \right)_{n} &\subseteq \set{g\colon U\rightarrow \D | g(p) = 0}.
  \end{align*}
  This is a normal family since it is locally bounded, it follows that there is a subsequence $\left( q\circ f_{n_k} \right)_k\rightarrow g\colon U\rightarrow \overline{\D}$. Since $g(p) = 0$, it follows that $g$ is a nonconstant holomorphic function. Since $q$ is conformal, we thus have
  \begin{align*}
    \left( f_{n_k} \right)_k &= \left( q^{-1}\circ q \circ f_{n_k} \right)_k\\
                             &\rightarrow q^{-1}\circ g,
  \end{align*}
  which is holomorphic, so $ \mathcal{F} $ is normal.
\end{solution}
\subsection{\href{https://math.virginia.edu/graduate/exams/analysis/2023Jan_complex.html}{January 2023}}%
\begin{problem}[Problem 1]
  Compute
  \begin{align*}
    I &= \int_{-\infty}^{\infty} \frac{\sin^2\left( x \right)}{x^2 + 1}\:dx.
  \end{align*}
\end{problem}
\begin{solution}
  Take $\sin^2(x) = \frac{1}{2} - \frac{1}{2}\cos\left( 2x \right)$. Writing the integral out, we have
  \begin{align*}
    \int_{-\infty}^{\infty} \frac{\sin^2(x)}{x^2 + 1}\:dx &= \frac{1}{2} \int_{-\infty}^{\infty} \frac{1}{x^2 + 1}\:dx\\
                                                          &+ \lim_{R\rightarrow\infty} \re\left( \int_{-R}^{R} \frac{e^{2ix}}{x^2 + 1}\:dx \right).
  \end{align*}
  To evaluate the second integral, we let $\gamma_R$ be the semicircle with radius $R$ centered at $0$ in the upper half-plane. Then, parametrizing, we get
  \begin{align*}
    \oint_{\gamma_R}^{} \frac{e^{2iz}}{z^2 + 1}\:dz &= \int_{-R}^{R} \frac{e^{2ix}}{x^2 + 1}\:dx + \int_{0}^{\pi} \frac{e^{2iRe^{i\theta}}}{R^2e^{2i\theta} + 1}iRe^{i\theta}\:d\theta.
  \end{align*}
  Estimating the second integral, we have, for $R > 1$,
  \begin{align*}
    \left\vert \int_{0}^{\pi} \frac{iRe^{i\theta}e^{2iRe^{i\theta}}}{R^2e^{2i\theta} - 1}\:d\theta \right\vert &\leq \pi\frac{Re^{-2R}}{R^2 - 1},
  \end{align*}
  which goes to zero as $R\rightarrow\infty$. Thus,
  \begin{align*}
    \oint_{\gamma_R}^{} \frac{e^{2iz}}{z^2 + 1}\:dz &= 2\pi i \res\left( f;i \right)\\
                                                    &= \lim_{R\rightarrow\infty} \int_{-R}^{R} \frac{e^{2ix}}{x^2 + 1}\:dx\\
                                                    &= 2\pi i \left( \frac{e^{-2}}{2i} \right)\\
                                                    &= \frac{\pi}{e^2}.
  \end{align*}
  Thus, we get
  \begin{align*}
    \int_{-\infty}^{\infty} \frac{\sin^2(x)}{x^2 + 1}\:dx &= \frac{\pi}{2} - \frac{\pi}{2e^2}.
  \end{align*}
\end{solution}
\begin{problem}[Problem 2]
  Let $f\colon \C\rightarrow \C$ be an entire function for which there exist $M,R \geq 0$ such that $\left\vert f(z) \right\vert \geq M\left\vert z \right\vert^2$ for all $\left\vert z \right\vert > R$. Show that $f$ is a polynomial of degree at least $2$.
\end{problem}
\begin{solution}
  We observe that an equivalent formulation of the condition is that
  \begin{align*}
    \left\vert f\left( \frac{1}{z} \right) \right\vert \geq \frac{M}{\left\vert z \right\vert^2}
  \end{align*}
  for all $0 < \left\vert z \right\vert < \frac{1}{R}$. In particular, this means that $\lim_{z\rightarrow 0} \left\vert f\left( \frac{1}{z} \right) \right\vert = \infty$, meaning that $f\left( \frac{1}{z} \right)$ has a pole at $z = 0$.%\newline

  Since $f$ is entire, we thus get
  \begin{align*}
    f\left( z \right) &= \sum_{n=0}^{\infty}a_nz^{n}
    \intertext{and since $f$ has a pole at $z = 0$,}
    f\left( \frac{1}{z} \right) &= \sum_{n=0}^{m} a_nz^{-n},
    \intertext{whence}
    f\left( z \right) &= \sum_{n=0}^{m} a_nz^{n}.
  \end{align*}
  Writing $f\left( \frac{1}{z} \right) = z^{-m}h(z)$ for some holomorphic function $h\colon U\left( 0,1/R \right)\rightarrow \C$, we get for all $0 < \left\vert z \right\vert < \frac{1}{R}$,
  \begin{align*}
    \left\vert z^{-m}h\left( z \right) \right\vert \geq M\left\vert z \right\vert^{-2}\\
    \left\vert z^{-m+2}h\left( z \right) \right\vert &\geq M,
  \end{align*}
  which implies that $m\geq 2$, as else we would have that $\lim_{z\rightarrow 0} \left\vert z^{-m + 2}h(z) \right\vert = 0 < M$, implying there would be some $0 < \left\vert z \right\vert < \frac{1}{R}$ such that $\left\vert z^{-m + 2}h(z) \right\vert < M$.
\end{solution}
\begin{problem}[Problem 3]
  Let $f,g\colon \C\rightarrow \C$ be entire functions such that there exists $\lambda\in\R$ with
  \begin{align*}
    \im\left( f(z) \right) &\leq \lambda \im\left( g(z) \right)
  \end{align*}
  for all $z\in \C$. Show that there exist constants $a,b\in \C$ such that $f(z) = a g(z) + b$.
\end{problem}
\begin{solution}
  Since $\im\left( f(z) \right) \leq \lambda \im\left( g(z) \right)$, it follows that $\im\left( f(z) - \lambda g(z) \right) \leq 0$. Now, we have
  \begin{align*}
    \left\vert e^{i\left( \lambda g(z) - f(z) \right)} \right\vert &= e^{\im\left( f(z) - \lambda g(z) \right)}\\
                                                                   &\leq 1.
  \end{align*}
  Since $e^{i\left( \lambda g(z) - f(z) \right)}$ is entire and bounded, it follows from Liouville that there is some $a\in \C$ such that
  \begin{align*}
    e^{i\lambda g(z) - if(z)} &= e^{a}
  \end{align*}
  for all $z\in \C$. We thus have a branch of the logarithm such that
  \begin{align*}
    i\lambda g(z) - if(z) &= a,
  \end{align*}
  whence
  \begin{align*}
    f(z) &= \lambda g(z) + ia.
  \end{align*}
\end{solution}
\begin{problem}[Problem 4]
  Let $P(z) = z^{9} + z^{5} - 8z^3 + 2z + 1$. Determine the number of zeros for $P$ counted with multiplicity in the annulus $A\left( 0,1,2 \right)$.
\end{problem}
\begin{solution}
  Let $\gamma = S\left( 0,2 \right)$. Observe that for all $z\in S\left( 0,2 \right)$, we have
  \begin{align*}
    \left\vert P(z) \right\vert&\geq \left\vert z \right\vert^{9} - \left\vert z \right\vert^{5} - \left\vert 8z^3 \right\vert - 2\left\vert z \right\vert - 1\\
                               &= 512-32-64-4-1\\
                               &> 0,
  \end{align*}
  and that
  \begin{align*}
    \left\vert P(z) - z^{9} \right\vert &\leq \left\vert z \right\vert^{5} + \left\vert 8z^3 \right\vert + \left\vert 2z \right\vert + 1\\
                                        &< 512\\
                                        &= \left\vert z \right\vert^{9}
  \end{align*}
  on $S\left( 0,2 \right)$. By Rouché's Theorem, it follows that all the zeros of $P$ lie in $U\left( 0,2 \right)$.%\newline

  Meanwhile, on $S\left( 0,1 \right)$, we have
  \begin{align*}
    \left\vert P(z) \right\vert &\geq 8\left\vert z \right\vert^3 - \left\vert z \right\vert^9 - \left\vert z \right\vert^5 - 2\left\vert z \right\vert - 1\\
                                &= 8-1-1-2-1\\
                                &> 0,
  \end{align*}
  and
  \begin{align*}
    \left\vert P(z) - \left( -8z^3 \right) \right\vert &\leq \left\vert z \right\vert^9 + \left\vert z \right\vert^5 + 2\left\vert z \right\vert + 1\\
                                                       &< 8\\
                                                       &= \left\vert -8z^3 \right\vert,
  \end{align*}
  so there are three zeros inside $U\left( 0,1 \right)$ by Rouché's Theorem. Thus, there are six zeros in $A\left( 0,1,2 \right)$.
\end{solution}
\begin{problem}[Problem 5]
  Let $\D$ be the unit disk, and let $\mathcal{F}$ be the family of holomorphic functions $f\colon \D\rightarrow \C$ satisfying $\left\vert f(z) \right\vert > 1$ for all $z\in \D$ and $f\left( 0 \right) = 2i$. Show that $\mathcal{F}$ is normal.
\end{problem}
\begin{solution}
  Let $\left( f_{n} \right)_n\subseteq \mathcal{F}$ be a sequence in $\mathcal{F}$. We observe that the family
  \begin{align*}
    \mathcal{G} &= \set{\frac{1}{f} | f\in \mathcal{F}}
  \end{align*}
  consists of holomorphic functions from $\D$ to $\D$, which is globally bounded by $1$, whence $\mathcal{G}$ is normal. In particular, this means that there is some holomorphic function $g\colon \D\rightarrow \overline{\D}$ and a subsequence $\left( \frac{1}{f_{n_k}} \right)_k\rightarrow g$ uniformly on compact subsets. Notice that $g\left( 0 \right) = \frac{1}{2i}$, meaning that $g$ is nonconstant; additionally, since each of the $\frac{1}{f_{n_k}}$ do not take on the value $0$, and $g\left( 0 \right)\neq 0$, it follows that $g$ has no zeros in $\D$, meaning that $\frac{1}{g}$ is a well-defined holomorphic function. Since $\frac{1}{z}\colon \D\setminus \set{0}\rightarrow \C\setminus B\left( 0,1 \right)$ is conformal, it follows that $\left( f_{n_k} \right)_k\rightarrow \frac{1}{g}$ uniformly on compact subsets. Thus, $\mathcal{F}$ is normal.
\end{solution}
\subsection{\href{https://math.virginia.edu/graduate/exams/analysis/2023Aug_complex.html}{August 2023}}%
\begin{problem}[Problem 1]
  For $\xi\in \R$, compute
  \begin{align*}
    \int_{-\infty}^{\infty} \frac{\cos\left( \xi x \right)}{x^2 + 4x + 5}\:dx.
  \end{align*}
  You must justify all estimates.
\end{problem}
\begin{solution}
  Let $\gamma_R$ be the semicircle with radius $R$ determined as follows:
  \begin{itemize}
    \item if $\xi > 0$, then the semicircle is in the upper half-plane;
    \item if $\xi < 0$, then the semicircle is in the lower half-plane;
    \item if $\xi = 0$, then
      \begin{align*}
        \int_{-\infty}^{\infty} \frac{1}{x^2 + 4x + 5}\:dx &= \int_{-\infty}^{\infty} \frac{1}{\left( x + 2 \right)^2 + 1}\:dx\\
                                                           &= \pi.
      \end{align*}
  \end{itemize}
  Let $\xi > 0$. Observe that
  \begin{align*}
    \re \left( \oint_{\gamma_R}^{} \frac{e^{iz}}{z^2 + 4z + 5}\:dz \right) &= \re\left( \int_{-\infty}^{\infty} \frac{e^{i \xi x}}{x^2 + 4x + 5}\:dx \right)\\
                                                           &+ \re\left( \int_{0}^{\pi} \frac{e^{i\xi Re^{i\theta}}}{R^2 e^{2i\theta} + 4Re^{i\theta} + 5}iRe^{i\theta}\:d\theta \right)\\
                                                           &= \int_{-R}^{R} \frac{\cos\left( \xi x \right)}{x^2 + 4x + 5}\:dx + \re\left( \int_{0}^{\pi} \frac{iRe^{i\theta}e^{i\xi Re^{i\theta}}}{R^2e^{2i\theta} + 4Re^{i\theta} + 5}\:d\theta \right).
  \end{align*}
  Estimating the second integral, we have for $R > \sqrt{5}$,
  \begin{align*}
    \left\vert \int_{0}^{\pi} \frac{iRe^{i\theta}e^{i\xi Re^{i\theta}}}{R^2e^{2i\theta} + 4Re^{i\theta} + 5}\:d\theta \right\vert &\leq \pi \frac{Re^{-\xi R}}{R^2-4R-5}\\
                                                                                                                                           &\rightarrow 0
  \end{align*}
  as $R\rightarrow\infty$. Thus,
  \begin{align*}
    \int_{-\infty}^{\infty} \frac{\cos\left( \xi x \right)}{x^2 + 4x +5}\:dx &= \re\left( \lim_{R\rightarrow\infty} \oint_{\gamma_R}^{} \frac{e^{i\xi z}}{z^2 + 4z + 5}\:dz \right)\\
                                                                             &= \re\left( 2\pi i \res\left[ \frac{e^{i\xi z}}{\left( z-\left( -2 + i \right) \right)\left( z-\left( -2-i \right) \right)};-2 + i \right] \right)\\
                                                                             &= \re\left( \pi e^{-2i\xi - \xi} \right)\\
                                                                             &= \pi e^{-\xi}\cos\left( 2\xi \right).
  \end{align*}
  If $\xi < 0$, then we have
  \begin{align*}
    \re\left( \oint_{\gamma_R}^{} f\:dz \right) &= \re\left( \int_{-R}^{R} \frac{e^{i\xi x}}{x^2 + 4x + 5}\:dx \right) \\
                                                &+ \re\left( \int_{0}^{\pi} \frac{e^{i\xi Re^{-i\theta}}}{R^2e^{-2i\theta} + 4Re^{-i\theta} + 5} \left( -iRe^{-i\theta} \right)\:d\theta \right).
  \end{align*}
  A similar technique yields that the second integral tends to zero, so that
  \begin{align*}
    \int_{-\infty}^{\infty} \frac{\cos\left( \xi x \right)}{x^2 + 4x + 5}\:dx &= \pi e^{\xi}\cos\left( 2\xi \right).
  \end{align*}
\end{solution}
\begin{problem}[Problem 2]
  Let $f\colon \C\rightarrow \C$ be an entire function whose derivative satisfies
  \begin{align*}
    \left\vert f'(z) \right\vert &\leq e^{\sqrt{\ln\left(\left\vert z \right\vert + 1\right)}}
  \end{align*}
  for all $z\in \C$. Show that there exist constants $a,b\in \C$ with $\left\vert a \right\vert\leq 1$ such that $f(z) = az + b$.
\end{problem}
\begin{solution}
  First, we observe that
  \begin{align*}
    \left\vert f'(0) \right\vert &\leq e^{\sqrt{\ln\left( 0 + 1 \right)}}\\
                                 &= 1.
  \end{align*}
  Next, from Cauchy's estimates, we have
  \begin{align*}
    \left\vert f''(0) \right\vert &\leq \frac{1}{R} \sup_{|z| = R} \left\vert f'(z) \right\vert\\
                                  &\leq \frac{1}{R} \sup_{|z| = R} e^{\sqrt{\ln\left( \left\vert z \right\vert + 1 \right)}}\\
                                  &= \frac{1}{R} e^{\sqrt{\ln\left( R + 1 \right)}}.
  \end{align*}
  Next,
  \begin{align*}
    \ln\left\vert f''(0) \right\vert &\leq \ln\left( \frac{1}{R} e^{\sqrt{\ln\left( R + 1 \right)}} \right)\\
                                     &= \sqrt{\ln\left( R + 1 \right)} - \ln\left( R \right),
  \end{align*}
  and for sufficiently large $R$, we have $\ln\left( R \right) > \sqrt{\ln\left( R + 1 \right)}$, whence $\ln\left\vert f''(0) \right\vert \rightarrow -\infty$ as $R\rightarrow\infty$, so $\left\vert f''(0) \right\vert = 0$. In particular, this means that $f(z) = az + b$ is the power series expansion for $f$ centered at zero, with $\left\vert a \right\vert = \left\vert f'(0) \right\vert \leq 1$.
\end{solution}
\begin{problem}[Problem 3]
  Let $U = \set{z\in \C | \re(z) > 0}$, and suppose $f\colon \overline{U}\rightarrow \C$ is continuous with $\sup_{z\in \partial U} \leq 1$ and $f|_{U}$ holomorphic. Suppose there exist $M,\alpha > 0$ such that $\left\vert f(z) \right\vert \leq Me^{\alpha \sqrt{\left\vert z \right\vert}}$ for all $z\in U$. Show that $\left\vert f(z) \right\vert \leq 1$ for all $z\in U$.
\end{problem}
\begin{solution}
  Let $f_{\ve}(z) = f(z)e^{-\ve z^{3/4}}$. Then,
  \begin{align*}
    \left\vert f_{\ve}(z) \right\vert &= \left\vert f(z) \right\vert e^{-\ve \re\left( z^{3/4} \right)}\\
                                      &\leq Me^{\alpha z^{1/2} - \ve \left\vert z \right\vert^{3/4}\cos\left( 3\pi/8 \right)}.
  \end{align*}
  Thus, we have that for sufficiently large $t_0$, we have
  \begin{align*}
    \lim_{t\rightarrow\infty} Me^{\alpha t^{1/2} - \ve t^{3/4}\cos\left( 3\pi/8\right)} &\leq 1,
  \end{align*}
  so that on the semicircle with radius $t_0$ centered at $0$ on the right half-plane, we have $\left\vert f_{\ve}(z) \right\vert \leq 1$. By the maximum modulus principle, this bound holds on the interior of the semicircle, and it holds on the exterior by the definition definition of $t_0$. In particular, we have that $\left\vert f_{\ve}(z) \right\vert \leq 1$ on $U$. Yet, since $\ve$ is arbitrary, we have that $\left\vert f(z) \right\vert\leq 1$ on $U$.
\end{solution}

\subsection{\href{https://math.virginia.edu/graduate/exams/analysis/2025Aug_complex.html}{August 2025}}%
\begin{problem}[Problem 1]
  Evaluate
  \begin{align*}
    \int_{0}^{2\pi} \frac{1}{5 + 4\cos\left( \theta \right)}\:d\theta.
  \end{align*}
\end{problem}
\begin{solution}
  Let $z = e^{i\theta}$. Upon a change of variables, we get
  \begin{align*}
    \int_{0}^{2\pi} \frac{1}{5 + 4\cos\left( \theta \right)}\:d\theta &= \oint_{S^1}^{} \frac{1}{iz\left( 5 + 2\left( z + \frac{1}{z} \right) \right)}\:dz\\
                                                                      &= \oint_{S^1}^{} \frac{1}{i\left( 2z^2 + 5z + 2 \right)}\:dz\\
                                                                      &= \oint_{S^{1}}^{} \frac{1}{i\left( 2z + 1 \right)\left( z+2 \right)}\:dz\\
                                                                      &= 2\pi i \res\left( f;-\frac{1}{2} \right)\\
                                                                      &= 2\pi i \left( \frac{1}{2i\left( -\frac{1}{2} + 2 \right)} \right)\\
                                                                      &= \frac{2}{3}\pi
  \end{align*}
\end{solution}
\begin{problem}[Problem 2]
  Let $f(z)$ be an entire function satisfying $\left\vert f\left( 2z \right) \right\vert\leq 2\left\vert f(z) \right\vert$. Show that either $f(z) = f(0)$ or $f(z) = f'(0)z$ for all $z\in \C$.
\end{problem}
\begin{solution}
  From Cauchy's Estimate, we have
  \begin{align*}
    \left\vert f''(0) \right\vert &\leq \frac{2}{2^{2k}}\sup_{|z| = 2^{k}} \left\vert f(z) \right\vert\\
                                  &\leq \frac{2}{2^{2k}}\sup_{|z| = 2^{k}} \left\vert 2^{k}f\left( z2^{-k} \right) \right\vert\\
                                  &= \frac{1}{2^{k-1}} \sup_{|z| = 1}\left\vert f(z) \right\vert.
  \end{align*}
  Since $k$ is arbitrary, it follows that $\left\vert f''(0) \right\vert = 0$, whence the power series expansion for $f$ centered at $0$ is given by $f(z) = f(0) + f'(0)z$.%\newline

  We claim that at least one of these must be zero. If it were the case that both $f(0)$ and $f'(0)$ were nonzero, then it would follow that there were some isolated zero $z_0\neq 0$ for $f$, whence
  \begin{align*}
    0 &< \left\vert f\left( 2z_0 \right) \right\vert\\
      &\leq 2\left\vert f\left( z_0 \right) \right\vert\\
      &= 0,
  \end{align*}
  which is a contradiction. Thus, if $f(0) = 0$, then $f(z) = f'(0)z$, and if $f'(0) = 0$, then $f(z) = f(0)$.
\end{solution}
\begin{problem}[Problem 3]
  Show that the only solution to $e^{z} = 2z+1$ inside $\D$ is $z = 0$.
\end{problem}
\begin{solution}
  We let $f(z) = e^{z}-2z-1$ and $g(z) = 2z$. On $S^{1}$, both $f$ and $g$ are nonzero. Additionally,
  \begin{align*}
    \left\vert e^{z}-1 \right\vert &= \left\vert \sum_{k=1}^{\infty}\frac{z^{k}}{k!} \right\vert\\
                                   &\leq \sum_{k=1}^{\infty} \frac{\left\vert z \right\vert^{k}}{k!}\\
                                   &= e-1\\
                                   &< 2\\
                                   &= \left\vert 2z \right\vert,
  \end{align*}
  whence $f$ and $g$ have the same number of zeros in $\D$, namely $1$ at $z = 0$.
\end{solution}
\begin{problem}[Problem 4]
  Let $U = \set{z\in \C | 0 < \re(z) < 1}$. Let $f\colon U\rightarrow \C$ be continuous and bounded with $f|_{U}$ holomorphic. Suppose there exist constants $M_0$ and $M_1$ such that
  \begin{align*}
    \sup_{\re(z) = 0} \left\vert f(z) \right\vert &\leq M_0\\
    \sup_{\re(z) = 1}\left\vert f(z) \right\vert &\leq M_1.
  \end{align*}
  Show that for $0 \leq r \leq 1$,
  \begin{align*}
    \sup_{\re(z)  =r} \left\vert f(z) \right\vert &\leq M_0^{1-r}M_1^{r}.
  \end{align*}
\end{problem}
\begin{solution}
  Let $\ve > 0$, and define $f_{\ve}(z) = f(z)M_0^{z-1}M_1^{-z}e^{\ve\left( z^2-1 \right)}$. We will show that $\sup_{z\in \overline{U}}\left\vert f_{\ve}(z) \right\vert\leq 1$.%\newline

  Observe that $\re\left( z^2 - 1 \right) = \left( \re(z)^2-1 \right)-\im(z)^2$; since $0 < \re(z) < 1$, it follows that this quantity is less than or equal to $-\im(z)^2$. Thus, we have
  \begin{align*}
    \left\vert f_{\ve}(z) \right\vert &\leq \left\vert f(z) \right\vert M_0^{\re(z)-1}M-1^{-\re(z)}e^{-\ve\im(z)^2}.
  \end{align*}
  Since $f$ is bounded, it follows that as $\im(z)\rightarrow\infty$, $\left\vert f_{\ve}(z) \right\vert\rightarrow 0$. Let $A$ be such that $\left\vert f_{\ve}(z) \right\vert\leq 1$ for all $\left\vert \im(z) \right\vert \geq A$. Then, on the rectangle with corners at $Ai,1+Ai,-Ai,1-Ai$ such that $f_{\ve}(z)\leq 1$ on the boundary of this rectangle. From the maximum modulus principle and the fact that $\left\vert f_{\ve} \right\vert\leq 1$ outside the rectangle, it follows that $\left\vert f_{\ve}(z) \right\vert\leq 1$ for all $z\in \overline{U}$. Thus, since $\ve > 0$ is arbitrary, we have
  \begin{align*}
    \lim_{\ve\rightarrow 0} \left\vert f_{\ve}(z) \right\vert &\leq 1,
  \end{align*}
  and thus
  \begin{align*}
    \left\vert f(z) \right\vert &\leq M_0^{1-\re(z)}M_1^{\re(z)}.
  \end{align*}
  The desired result follows.
\end{solution}
\begin{problem}[Problem 5]
  Let $\mathcal{F}$ be the set of holomorphic functions on $U\left( 0,2 \right)$ such that 
  \begin{align*}
    \int_{0}^{2\pi} \left\vert f\left( e^{i\theta} \right) \right\vert\:d\theta &\leq 1.
  \end{align*}
  Let
  \begin{align*}
    \mathcal{G} &= \set{g\colon \D\rightarrow \C | g = f|_{\D}\text{ for some }f\in \D}.
  \end{align*}
  Show that $\mathcal{G}$ is normal.
\end{problem}
\begin{solution}
  Let $z\in \D$. We observe then that for every $f\in \mathcal{F}$, we have
  \begin{align*}
    \left\vert f(z) \right\vert &\leq \int_{0}^{2\pi} \frac{\left\vert f\left( e^{i\theta} \right) \right\vert}{\left\vert e^{i\theta}-z \right\vert}\left\vert ie^{i\theta} \right\vert\:d\theta\\
                                &\leq \frac{1}{1-\left\vert z \right\vert} \int_{0}^{2\pi} \left\vert f\left( e^{i\theta} \right) \right\vert\:d\theta\\
                                &= \frac{1}{1-\left\vert z \right\vert}.
  \end{align*}
  In particular, if we let $\delta$ be such that $U\left( z,\delta \right)\subseteq \D$, then for all $w\in U\left( z,\delta \right)$ and all $f\in \mathcal{G}$, we have
  \begin{align*}
    \left\vert f(w) \right\vert &\leq \frac{1}{1-\left\vert w \right\vert}\\
                                &\leq \frac{1}{1-\left\vert z \right\vert-\delta}\\
                                &\eqcolon C.
  \end{align*}
  Thus, $\mathcal{G}$ is locally bounded, hence normal.
\end{solution}

\section{Notation}%
\begin{itemize}
  \item $\ds U\left( z_0,r \right) = \set{z\in \C | \left\vert z-z_0 \right\vert < r}$
  \item $\ds B\left( z_0,r \right) = \set{z\in \C | \left\vert z-z_0 \right\vert \leq r}$
  \item $\ds S\left( z_0,r \right) = \set{z\in \C | \left\vert z-z_0 \right\vert = r}$
  \item $\ds \dot{U}\left( z_0,r \right) = \set{z\in \C | 0 < \left\vert z-z_0 \right\vert < r}$
  \item $\ds A\left( z_0,r_1,r_2 \right) = \set{z\in \C | r_1 < \left\vert z-z_0 \right\vert < r_2}$
\end{itemize}
\end{document}
