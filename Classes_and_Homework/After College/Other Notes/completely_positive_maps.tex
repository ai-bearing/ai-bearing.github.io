\documentclass[10pt]{mypackage}

%\usepackage{mlmodern}
%\usepackage{newpxtext,eulerpx,eucal}
%\renewcommand*{\mathbb}[1]{\varmathbb{#1}}

%\usepackage{homework}
\usepackage{notes}

\usepackage[ backend=bibtex, style = alphabetic, sorting=ynt ]{biblatex}
\addbibresource{all_references.bib}

\usepackage{parskip}

\fancyhf{}
\fancyhead[R]{Avinash Iyer}
\fancyhead[L]{Completely Bounded and Completely Positive Maps}
\fancyfoot[C]{\thepage}

\setcounter{secnumdepth}{0}

\begin{document}
\RaggedRight
\section{Positive Maps}%
We will start by focusing our discussion of positive maps on a subclass of linear subspaces of $C^{\ast}$-algebras.
\begin{definition}
  Let $\mathcal{A}$ be a $C^{\ast}$-algebra, and let $\mathcal{S}\subseteq \mathcal{A}$ be a self-adjoint linear subspace that contains $1$. We call such an $\mathcal{S}$ an \textit{operator system}.
\end{definition}
Note that if $h$ is a self-adjoint element of $\mathcal{S}$, then it is possible to write $h$ as the difference of two positive elements in $\mathcal{S}$,
\begin{align*}
  h &= \frac{1}{2}\left( \norm{h}1 + h \right) - \frac{1}{2}\left( \norm{h}1-h \right).
\end{align*}
\begin{definition}
  If $\mathcal{S}\subseteq \mathcal{A}$ is an operator system, $\mathcal{B}$ is a $C^{\ast}$-algebra, and $\phi\colon \mathcal{S}\rightarrow \mathcal{B}$ is a linear map, then we say $\phi$ is positive if it maps positive elements of $\mathcal{S}$ to positive elements of $\mathcal{B}$.
\end{definition}
In the special case where the $C^{\ast}$-algebra $\mathcal{B}$ is the complex numbers (i.e., $\phi$ is a positive linear functional), then we know from results in $C^{\ast}$-algebra theory that $\norm{\phi} = \phi(1)$. If $\mathcal{B}$ is an arbitrary $C^{\ast}$-algebra, it turns out that $\phi$ is still positive, but that the bound is different.
\begin{proposition}
  If $\phi\colon \mathcal{S}\rightarrow \mathcal{B}$ is a positive map, then $\norm{\phi}\leq 2\norm{\phi(1)}$.
\end{proposition}
\begin{proof}
  If $p$ is positive, then since $0\leq p \leq \norm{p}1$, it follows that $0\leq \phi(p)\leq \norm{p}\phi(1)$, so that $\norm{\phi(p)}\leq \norm{p}\norm{\phi(1)}$.

  If $p_1$ and $p_2$ are positive, then $\norm{p_1-p_2}\leq \max\left( \norm{p_1},\norm{p_2} \right)$, so if $h$ is self-adjoint in $\mathcal{S}$, we have
  \begin{align*}
    \phi(h) &= \frac{1}{2}\phi\left( \norm{h}1 + h \right) - \frac{1}{2}\phi\left( \norm{h}1 - h \right),
  \end{align*}
  giving
  \begin{align*}
    \norm{\phi(h)} &\leq \frac{1}{2}\max\left( \norm{\phi\left( \norm{h}1 + h \right)}, \norm{\phi\left( \norm{h}1 - h \right)}\right)\\
                   &\leq \norm{h}\norm{\phi(1)}.
  \end{align*}
  Finally, if $a$ is an arbitrary element of $\mathcal{S}$, then we may write the Cartesian decomposition $a = h + ik$, and find
  \begin{align*}
    \norm{\phi(a)} &\leq \norm{\phi(h)} + \norm{\phi(k)}\\
                   &\leq 2\norm{a}\norm{\phi(1)}.
  \end{align*}
\end{proof}
It turns out that this bound is strict.
\section{Completely Positive Maps}%

\section{Dilations and Extensions}%

\section{Nuclearity and Exactness}%

\section{Application to Amenability}%

\nocite{paulsen_completely_bounded_maps,brown_and_ozawa}
\printbibliography
\end{document}
