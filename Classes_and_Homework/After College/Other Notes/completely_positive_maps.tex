\documentclass[10pt]{mypackage}

%\usepackage{mlmodern}
%\usepackage{newpxtext,eulerpx,eucal}
%\renewcommand*{\mathbb}[1]{\varmathbb{#1}}

%\usepackage{homework}
\usepackage{notes}

\usepackage[ backend=bibtex, style = alphabetic, sorting=ynt ]{biblatex}
\addbibresource{all_references.bib}

\usepackage{parskip}

\fancyhf{}
\fancyhead[R]{Avinash Iyer}
\fancyhead[L]{Completely Bounded and Completely Positive Maps}
\fancyfoot[C]{\thepage}

\setcounter{secnumdepth}{0}

\begin{document}
\RaggedRight
\section{Positive Maps}%
We will start by focusing our discussion of positive maps on a subclass of linear subspaces of $C^{\ast}$-algebras.
\begin{definition}
  Let $\mathcal{A}$ be a $C^{\ast}$-algebra, and let $\mathcal{S}\subseteq \mathcal{A}$ be a self-adjoint linear subspace that contains $1$. We call such an $\mathcal{S}$ an \textit{operator system}.
\end{definition}
Note that if $h$ is a self-adjoint element of $\mathcal{S}$, then it is possible to write $h$ as the difference of two positive elements in $\mathcal{S}$,
\begin{align*}
  h &= \frac{1}{2}\left( \norm{h}1 + h \right) - \frac{1}{2}\left( \norm{h}1-h \right).
\end{align*}
\begin{definition}
  If $\mathcal{S}\subseteq \mathcal{A}$ is an operator system, $\mathcal{B}$ is a $C^{\ast}$-algebra, and $\phi\colon \mathcal{S}\rightarrow \mathcal{B}$ is a linear map, then we say $\phi$ is positive if it maps positive elements of $\mathcal{S}$ to positive elements of $\mathcal{B}$.
\end{definition}
In the special case where the $C^{\ast}$-algebra $\mathcal{B}$ is the complex numbers (i.e., $\phi$ is a positive linear functional), then we know from results in $C^{\ast}$-algebra theory that $\norm{\phi} = \phi(1)$. If $\mathcal{B}$ is an arbitrary $C^{\ast}$-algebra, it turns out that $\phi$ is still positive, but that the bound is different.
\begin{proposition}
  If $\phi\colon \mathcal{S}\rightarrow \mathcal{B}$ is a positive map, then $\norm{\phi}\leq 2\norm{\phi(1)}$.
\end{proposition}
\begin{proof}
  If $p$ is positive, then since $0\leq p \leq \norm{p}1$, it follows that $0\leq \phi(p)\leq \norm{p}\phi(1)$, so that $\norm{\phi(p)}\leq \norm{p}\norm{\phi(1)}$.

  If $p_1$ and $p_2$ are positive, then $\norm{p_1-p_2}\leq \max\left( \norm{p_1},\norm{p_2} \right)$, so if $h$ is self-adjoint in $\mathcal{S}$, we have
  \begin{align*}
    \phi(h) &= \frac{1}{2}\phi\left( \norm{h}1 + h \right) - \frac{1}{2}\phi\left( \norm{h}1 - h \right),
  \end{align*}
  giving
  \begin{align*}
    \norm{\phi(h)} &\leq \frac{1}{2}\max\left( \norm{\phi\left( \norm{h}1 + h \right)}, \norm{\phi\left( \norm{h}1 - h \right)}\right)\\
                   &\leq \norm{h}\norm{\phi(1)}.
  \end{align*}
  Finally, if $a$ is an arbitrary element of $\mathcal{S}$, then we may write the Cartesian decomposition $a = h + ik$, and find
  \begin{align*}
    \norm{\phi(a)} &\leq \norm{\phi(h)} + \norm{\phi(k)}\\
                   &\leq 2\norm{a}\norm{\phi(1)}.
  \end{align*}
\end{proof}
It turns out that this bound is strict.
\begin{example}
  Consider the subspace $\mathcal{S}\subseteq C\left( S^1 \right)$ spanned by $1,z,\bar{z}$. Then, we may define $\phi\colon \mathcal{S}\rightarrow \M_2$ given by
  \begin{align*}
    \phi\left( a + bz + c \bar{z} \right) &= \begin{pmatrix}a & 2b \\ 2c & a\end{pmatrix}.
  \end{align*}
  It follows that an element of $\mathcal{S}$ is positive if and only if $c = \overline{b}$ and $a \geq 2\left\vert b \right\vert$, while a self-adjoint element of $\M_2$ is positive if and only if its diagonal entries and determinant are positive real numbers. Therefore, it follows that $\phi$ is a positive map.

  Yet,
  \begin{align*}
    2\norm{\phi(1)} &= 2\\
                    &= \norm{\phi(z)}\\
                    &\leq \norm{\phi},
  \end{align*}
  meaning that $\norm{\phi} = 2\norm{\phi(1)}$.
\end{example}
\section{Completely Positive Maps}%
\begin{definition}
  If $\mathcal{B}$ is a $C^{\ast}$-algebra and $\phi\colon \mathcal{S}\rightarrow \mathcal{B}$ is a linear map, then we may define $\phi_n\colon \M_n\left( \mathcal{S} \right)\rightarrow \M_n\left( \mathcal{B} \right)$ by $\phi_n\left( \left( a_{ij} \right)_{i,j} \right) = \left( \phi\left( a_{ij} \right) \right)_{i,j}$.

  We call $\phi$ $n$-positive if $\phi_n$ is positive, and we call $\phi$ completely positive if $\phi$ is $n$-positive for all $n$.

  We call $\phi$ completely bounded if
  \begin{align*}
    \norm{\phi}_{\text{cb}} &\coloneq \sup_{n}\norm{\phi_n}
  \end{align*}
  is finite. If $\norm{\phi}_{\text{cb}}\leq 1$, then we call $\phi$ completely contractive.
\end{definition}
\begin{lemma}
  Let $\mathcal{A}$ be a unital $C^{\ast}$-algebra, and let $a,b\in \mathcal{A}$. Then,
  \begin{enumerate}[(i)]
    \item $\norm{a}\leq 1$ if and only if
      \begin{align*}
        \begin{pmatrix}1 & a \\ a^{\ast} & 1\end{pmatrix}
      \end{align*}
      is positive in $\M_2\left(\mathcal{A}\right)$.
    \item The matrix
      \begin{align*}
        \begin{pmatrix}1 & a \\ a^{\ast} & b\end{pmatrix}
      \end{align*}
      is positive in $\M_2(\mathcal{A})$ if and only if $a^{\ast}a\leq b$.
  \end{enumerate}
\end{lemma}
\begin{proof}
  Faithfully represent $\mathcal{A}$ on $\mathcal{H}$ via $\pi\colon \mathcal{A}\rightarrow B\left( \mathcal{H} \right)$, and set $A = \pi(a)$. Then, if $\norm{A}\leq 1$, for any $x,y\in \mathcal{H}$, we have
  \begin{align*}
    \iprod{ \begin{pmatrix}1 & A \\ A^{\ast} & 1\end{pmatrix} \begin{pmatrix}x\\y\end{pmatrix} }{ \begin{pmatrix}x\\y\end{pmatrix} } &= \iprod{x}{y} + \iprod{Ay}{x} + \iprod{x}{Ay} + \iprod{y}{y}\\
                             &\geq \norm{x}^2 - 2\norm{A}\norm{x}\norm{y} + \norm{y}^2\\
                             &\geq 0.
  \end{align*}
  Conversely, if $\norm{A} > 1$< then there are unit vectors $x$ and $y$ such that $ \iprod{Ay}{x} < -1 $, so the inner product above would be negative.

  Now, if we let $B = \pi(b)$, then if we let $B\geq A^{\ast}A$, then $B-A^{\ast}A\geq 0$, so that $ \iprod{By}{y}\geq \iprod{Ay}{Ay} $ for all $y\in \mathcal{H}$, meaning that for all $x,y\in \mathcal{H}$, we have
  \begin{align*}
    \iprod{\begin{pmatrix}1 & A \\ A^{\ast} & B\end{pmatrix} \begin{pmatrix}x\\y\end{pmatrix}}{ \begin{pmatrix}x\\y\end{pmatrix} } &= \iprod{x}{x} + \iprod{Ax}{y} + \iprod{A^{\ast}x}{y} + \iprod{By}{y}\\
                            &\geq \iprod{x}{x} + 2\re \iprod{Ax}{y} + \iprod{Ay}{Ay}\\
                            &\geq \norm{x}^2 + \norm{Ay}^2 - 2\norm{Ay}\norm{x}\\
                            &\geq 0.
  \end{align*}
  In the case that $B\not\geq A^{\ast}A$, then there is some unit vector $y$ such that $ \iprod{By}{y} < \norm{Ay}^2 $, which would yield the analogous outcome as in the proof of (i).
\end{proof}
\section{Dilations and Extensions}%
\begin{theorem}[Stinespring's Dilation Theorem]
  Let $\mathcal{A}$ be a unital $C^{\ast}$-algebra, and let $\phi\colon \mathcal{A}\rightarrow B\left( \mathcal{H} \right)$ be a completely positive map. Then, there exists a Hilbert space $\mathcal{K}$, a unital $\ast$-homomorphism $\pi\colon \mathcal{A}\rightarrow B\left( \mathcal{K} \right)$, and a bounded operator $V\colon \mathcal{H}\rightarrow \mathcal{K}$ with $\norm{\phi(1)} = \norm{V}^2$ such that
  \begin{align*}
    \phi(a) &= V^{\ast}\pi(a)V.
  \end{align*}
\end{theorem}
\begin{proof}
  Let $A\odot \mathcal{H}$ be the algebraic tensor product, and define a symmetric bilinear map
  \begin{align*}
    \left( a\otimes x,b\otimes y \right) &= \iprod{\phi\left( b^{\ast}a \right)x}{y},
  \end{align*}
  and extend linearly, where $ \iprod{\cdot}{\cdot} $ is the inner product on $\mathcal{H}$. Then, since $\phi$ is completely positive, it follows that $ \left( \cdot,\cdot \right) $ is positive semidefinite, with
  \begin{align*}
    \left( \sum_{j=1}^{n}a_j\otimes x_j,\sum_{i=1}^{n}a_i\otimes x_i \right) &= \iprod{\phi_n\left( \left( a_i^{\ast}a_j \right)_{i,j} \right) \begin{pmatrix}x_1\\\vdots\\x_n\end{pmatrix}}{ \begin{pmatrix}x_1\\\vdots\\x_n\end{pmatrix} }\\
                                                                             &\geq 0.
  \end{align*}
  Since positive semidefinite bilinear forms satisfy the Cauchy--Schwarz inequality, we may define the subspace
  \begin{align*}
    \mathcal{N} &= \set{u\in \mathcal{A}\odot \mathcal{H} | \left( u,u \right) = 0}\\
                &= \set{u\in \mathcal{A}\odot \mathcal{H} | \left( u,v \right) = 0\text{ for all }v\in \mathcal{A}\odot \mathcal{H}},
  \end{align*}
  with an induced bilinear form on the quotient space $A\odot \mathcal{H}/\mathcal{N}$ defined by
  \begin{align*}
    \iprod{u + \mathcal{N}}{ v + \mathcal{N} } &= \left( u,v \right).
  \end{align*}
  Define $\mathcal{K}$ to be the Hilbert space completion of $\mathcal{A}\odot \mathcal{H}/\mathcal{N}$. Now, define a linear map $\pi(a)\colon \mathcal{A}\odot \mathcal{H}\rightarrow \mathcal{A}\odot \mathcal{H}$ by
  \begin{align*}
    \pi(a)\left( \sum_{i=1}^{n}a_i\otimes x_i \right) &= \sum_{i=1}^{n} \left( aa_i \right)\otimes x_i.
  \end{align*}
  We have that the inequality in $\M_n\left( \mathcal{A} \right)$ given by
  \begin{align*}
    \left( a_i^{\ast}a^{\ast}aa_j \right)_{i,j} &\leq \norm{a^{\ast}a}\left( a_i^{\ast}a_j \right)_{i,j}
  \end{align*}
  is satisfied, giving
  \begin{align*}
    \left( \pi(a)\left( \sum_{j=1}^{n}a_j\otimes x_j \right),\pi(a)\left( \sum_{i=1}^{n}a_i\otimes x_i \right) \right) &= \sum_{i,j=1}^{n} \iprod{\phi\left( a_i^{\ast}a^{\ast}aa_j \right)x_j}{x_i}\\
                                                                                                                                &\leq \norm{a^{\ast}a}\sum_{i,j=1}^{n} \iprod{\phi\left( a_i^{\ast}a_j \right)x_j}{x_i}\\
                                                                                                                                                                                                                                &= \norm{a}^2 \left( \sum_{j=1}^{n}a_j\otimes x_j,\sum_{i=1}^{n}a_i\otimes x_i \right).
  \end{align*}
  Therefore, $\pi(a)$ is invariant under $\mathcal{N}$, so induces a quotient map on $\mathcal{A}\otimes \mathcal{H}/\mathcal{N}$, which we will also denote by $\pi(a)$. We have that (this new) $\pi(a)$ is bounded with $\norm{\pi(a)}\leq \norm{a}$, meaning that it extends to a bounded linear map on $\mathcal{K}$.

  We define $V\colon \mathcal{H}\rightarrow \mathcal{K}$ by
  \begin{align*}
    V(x) &= 1\otimes x + \mathcal{N}.
  \end{align*}
  Then,
  \begin{align*}
    \norm{Vx}^2 &= \left( 1\otimes x,1\otimes x \right)\\
                &= \iprod{\phi(1)x}{x}\\
                &\leq \norm{\phi(1)}\norm{x}^2,
  \end{align*}
  and $\norm{V}^2 = \norm{\phi(1)}$.

  Finally, we observe that
  \begin{align*}
    \iprod{V^{\ast}\pi(a)Vx}{y} &= \left( \pi(a)1\otimes x,1\otimes y \right)\\
                                &= \iprod{\phi(a)x}{y}
  \end{align*}
  for all $x$ and $y$, so $V^{\ast}\pi(a)V = \phi(a)$.
\end{proof}
We observe that if $\phi$ is unital, then $V$ is an isometry, and we may identify $\mathcal{H}$ with the subspace $V\mathcal{H}$ of $\mathcal{K}$, and that $\phi(a) = P\pi(a)P$, where $P$ is the projection onto $\mathcal{H}$. In particular, this means that every unital completely positive map is the compression of a $\ast$-homomorphism.

This construction is very similar to the GNS representation, and we call the triple $\left( \pi,V,\mathcal{K} \right)$ a Stinespring representation for $\phi$.
\section{Nuclearity and Exactness}%
\begin{definition}
  We call a map $\theta\colon A\rightarrow B$ between $C^{\ast}$-algebras \textit{nuclear} if there are contractive completely positive maps $\varphi_n\colon A\rightarrow \M_{k(n)}\left( \C \right)$ and $\psi_n\colon \M_{k(n)}\left( \C \right)\rightarrow B$ such that $\psi_n\circ\varphi_n\rightarrow\theta$ pointwise in the norm topology:
  \begin{align*}
    \norm{\psi_n\circ\varphi_n(a)\rightarrow\theta(a)}&\rightarrow 0
  \end{align*}
  for all $a\in A$.
\end{definition}
\begin{definition}
  If $A$ is a $C^{\ast}$-algebra, and $N$ is a von Neumann algebra, then a map $\theta\colon A\rightarrow N$ is called \textit{weakly nuclear} if there exist contractive completely positive maps $\varphi_n\colon A\rightarrow \M_{k(n)}\left( \C \right)$ and $\psi_n\colon \M_{k(n)}\left( \C \right)\rightarrow N$ such that $\psi_n\circ\varphi_n\rightarrow \theta$ pointwise in the ultraweak topology:
  \begin{align*}
    \eta\left( \psi_n\circ\varphi_n(a) \right)\rightarrow\eta\left( \theta(a) \right)
  \end{align*}
  for all $a\in A$ and normal functionals $\eta\in N_{\ast}$.
\end{definition}
By uniqueness of preduals, if $N\subseteq B\left(\mathcal{H}\right)$ is a faithful normal representation, then it suffices to observe that $\psi_n\circ\varphi_n\rightarrow\theta$ pointwise in the ultraweak topology if and only if
\begin{align*}
  \iprod{\psi_n\circ\varphi_n(a)v}{w} &\rightarrow \iprod{\theta(a)v}{w}
\end{align*}
for all $a\in A$ and $v,w\in \Omega$ for some collection $\Omega$ of vectors whose linear span is dense in $\mathcal{H}$.

One of the interesting aspects of nuclear maps is that whether a map is nuclear or not depends on the range.
\begin{proposition}
  Let $M\subseteq B(H)$ be a von Neumann algebra. The natural inclusion map $M\hookrightarrow B(H)$ is always weakly nuclear.
\end{proposition}
\begin{proof}
  Let $\set{P_i}_{i\in I}$ be a net of finite rank projections increasing to the identity. If $P_i$ has rank $k(i)$, then we may define maps $\varphi_i\colon M\rightarrow \M_{k(i)}\left( \C \right)\cong P_i B(H)P_i$ by compression, and let $\psi_i\colon \M_{k(i)}\left( \C \right) \rightarrow B(H)$ be natural inclusion maps.

  Since the predual of $B(H)$ is the trace class operators, we have that these maps converge weakly to the identity on $B(H)$, it follows that $M\hookrightarrow B(H)$ is weakly nuclear.
\end{proof}
\begin{proposition}
  A map $\theta\colon A\rightarrow B$ is nuclear if and only if for every finite $F\subseteq A$ and every $\ve > 0$, there exist $n\in \N$ and contractive completely positive maps $\varphi\colon A\rightarrow \M_n\left( \C \right)$, $\psi\colon \M_n\left( \C \right)\rightarrow B$ such that $\norm{\theta(a)-\psi\circ\varphi(a)} < \ve$ for all $a\in F$.
\end{proposition}
\begin{proof}
  Define the set $\mathcal{F} = \set{\left( F,\ve \right) | F\subseteq A\text{ finite, }\ve > 0}$, directed by
  \begin{align*}
    \left( F_1,\ve_1 \right)\preceq \left( F_2,\ve_2 \right) &\Leftrightarrow F_1\subseteq F_2\text{ and }\ve_2 \leq \ve_1.
  \end{align*}
  It can then be verified that convergence in the point-norm topology in the definition for nuclearity is equivalent to convergence via this directed set, and vice versa.
\end{proof}

\begin{definition}
  A $C^{\ast}$-algebra $A$ is called \textit{nuclear} if the identity map is nuclear.
\end{definition}
\begin{definition}
  A $C^{\ast}$-algebra $A$ is called \textit{exact} if there exists a faithful representation $\pi\colon A\rightarrow B(H)$ such that $\pi$ is nuclear.
\end{definition}
\begin{definition}
   A von Neumann algebra $M$ is called \textit{semidiscrete} if the identity map is weakly nuclear.
\end{definition}
We will show now that if the double dual of a $C^{\ast}$-algebra is semidiscrete, then the $C^{\ast}$-algebra is nuclear.
\begin{lemma}
  Let $A$ be a Banach space, and let $B(A)$ be the space of all bounded linear maps from $A$ to $A$, and let $C\subseteq B(A)$ be a convex set. Then, the point-weak and point-norm closures of $C$ coincide.
\end{lemma}
\section{Application to Amenability}%

\nocite{paulsen_completely_bounded_maps,brown_and_ozawa}
\printbibliography
\end{document}
