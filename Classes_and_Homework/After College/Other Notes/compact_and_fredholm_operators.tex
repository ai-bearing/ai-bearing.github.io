\documentclass[10pt]{mypackage}

\usepackage{mlmodern}

%\usepackage{homework}
\usepackage{notes}

\usepackage[ backend=bibtex,
style=alphabetic,
sorting=ynt ]{biblatex}
\addbibresource{all_references.bib}

\usepackage{parskip}

\fancyhf{}
\fancyhead[R]{Avinash Iyer}
\fancyhead[L]{Compact and Fredholm Operators}
\fancyfoot[C]{\thepage}

\setcounter{secnumdepth}{0}

\begin{document}
\RaggedRight
\section{Compact Operators}%
\begin{definition}
  A linear map $T\colon X\rightarrow Y$ between Banach spaces is called \textit{compact} if $ T\left( B_X \right)\subseteq Y $ has compact closure, where $B_X$ denotes the closed unit ball of $X$. We denote the space of compact operators $K(X,Y)$.
\end{definition}
The theory of compact operators (and the soon to arise Fredholm operators) arose from the analysis of integral equations. To start, let $I = [0,1]$, and consider the Banach space $C(I)$ with the supremum norm. Letting $k\in C\left( I\times I \right)$, we define $u\in B(X)$ by taking
\begin{align*}
  Tf(x) &= \int_{0}^{1} k(x,y)f(y)\:dy.
\end{align*}
The fact that $Tf\in X$ follows from an application of the Dominated Convergence Theorem and the fact that, since $k(x,y)$ is jointly continuous, it is also separately continuous (see \cite[Theorem 2.27]{folland_real_analysis}). In fact, we can show something even stronger: we claim that the family $T\left( B_X \right)$ is in fact equicontinuous. This follows from the fact that, $I^2$ is compact, so if $\ve > 0$, there is $\delta$ such that whenever $\max\set{\left\vert x-x' \right\vert,\left\vert y-y' \right\vert} < \delta$, we have $\left\vert k\left( x,y \right) - k\left( x',y' \right) \right\vert < \ve$. Therefore,
\begin{align*}
  \left\vert Tf(x) - Tf\left(x'\right) \right\vert &= \left\vert \int_{0}^{1} \left( k\left( x,y \right) - k\left( x',y \right) \right)f(y)\:dy \right\vert\\
                                                   &\leq \int_{0}^{1} \left\vert k\left( x,y \right) - k\left( x',y \right) \right\vert\left\vert f(y) \right\vert\:dy\\
                                                   &\leq \sup_{y\in I} \left\vert k\left( x,y \right) - k\left( x',y \right) \right\vert \norm{f}_{u}\\
                                                   &\leq \ve \norm{f}_{u}.
\end{align*}
Furthermore, since
\begin{align*}
  \left\vert Tf(x) \right\vert &\leq \norm{k}_{u}\norm{f}_{u},
\end{align*}
we have that $T\left(B_X\right)$ is pointwise bounded. Thus, by the Arzelà--Ascoli theorem, it follows that $T\left( B_X \right)$ is totally bounded, so $T$ is a compact operator. We call the function $k$ the \textit{kernel} of the operator $T$.

Similarly, the operator $V\in B(X)$ given by
\begin{align*}
  Vf(x) &= \int_{0}^{x} f(y)\:dy
\end{align*}
is such that $V\left(B_X\right)$ is totally bounded by Arzelà--Ascoli, so $V$ is also compact. In fact, $V$ has no eigenvalues as well. This follows from the fact that, if there were $\lambda\in \C\setminus \set{0}$ with $V(f) = \lambda f$, then $f(0) = 0$ and $f'(t) = 1/\lambda f(t)$, so that $f(t) = f(0) e^{t/\lambda} = 0$, meaning $f = 0$.

We call the operator $V$ the \textit{Volterra integral operator} on $X$.

We can see that $K(X)$ is in fact an algebraic ideal in $B(X)$ (by continuity). In fact, there is a topological dimension to $K(X)\subseteq B(X)$.
\begin{proposition}
  If $X,Y$ are Banach spaces, then $K(X,Y)$ is a closed subspace of $B(X,Y)$.
\end{proposition}
\begin{proof}
  Let $\left( T_n \right)_n$ converge to $T\in B(X,Y)$. Let $\ve > 0$, and select $N$ such that $\norm{T_N - T} < \ve/3$. Since $T_N\left(B_X\right)$ is totally bounded, there are $x_1,\dots,x_n\in B_X$ such that for each $x\in S$, we have
  \begin{align*}
    \norm{T_Nx - T_Nx_j} < \ve/3
  \end{align*}
  for some $j$. Therefore, we have
  \begin{align*}
    \norm{Tx-Tx_j } &\leq \norm{Tx-T_Nx} + \norm{T_Nx-T_Nx_j} + \norm{T_Nx_j - Tx_j}\\
                    &< \ve.
  \end{align*}
  Therefore, $T\left( B_X \right)$ is totally bounded, so $T\in K(X,Y)$.
\end{proof}
Therefore, we see that $ \overline{F(X,Y)}\subseteq K(X,Y) $ is, where $F(X,Y)$ denotes the finite-rank operators, but this inclusion may be strict. In the cases where $ \overline{F(X)} = K(X) $, we say the Banach space $X$ has the approximation property. There are Banach spaces that do not have the approximation property.

Recall that if $T\colon X\rightarrow Y$ is a bounded linear map between Banach spaces, the transpose is defined by $T^{\ast}\colon Y^{\ast}\rightarrow X^{\ast}$, given by $T^{\ast}\varphi = \varphi\circ T$.
\begin{theorem}
  If $X$ and $Y$ are Banach spaces with $T\in K\left( X,Y \right)$, then $T^{\ast}\in K\left( Y^{\ast},X^{\ast} \right)$.
\end{theorem}
\begin{proof}
  Let $\ve > 0$. Since $T\left(B_X\right)$ is totally bounded, there exist elements $x_1,\dots,x_n$ such that if $x\in B_X$, then $\norm{Tx - Tx_i} < \ve/3$ for some $i$. Let $V\in B\left( Y^{\ast},\C^n \right)$ be defined by $V\varphi = \left( \varphi\left( Tx_1 \right),\dots,\varphi\left( Tx_n \right) \right)$. Since $V$ has finite rank, $V$ is compact, so $V\left( B_{X^{\ast}} \right)$ is totally bounded. Thus, there exist $\varphi_1,\dots,\varphi_m$ such that if $\varphi\in T$, then $\norm{V\varphi - V\varphi_j} = \max_{i=1}^{n} \left\vert T^{\ast}\varphi\left(x_i\right) - T^{\ast}\varphi_j(x_i) \right\vert$.

  Now, if $x\in B_X$, then $\norm{Tx - Tx_i} < \ve/3$ for some $i$, so thus $ \left\vert T^{\ast}\varphi\left( x_i \right) - T^{\ast}\varphi_j\left( x_i \right) \right\vert < \ve/3$. Thus,
  \begin{align*}
    \left\vert T^{\ast}\varphi(x) - T^{\ast}\varphi_j\left( x \right) \right\vert &\leq \left\vert T^{\ast}\varphi(x) - T^{\ast}\varphi\left(x_i\right) \right\vert + \left\vert T^{\ast}\varphi\left( x_i \right) - T^{\ast}\varphi_j\left( x_i \right) \right\vert + \left\vert T^{\ast}\varphi_j\left( x_i \right) - T^{\ast}\varphi_j\left( x \right) \right\vert\\
                                                                                  &< \ve,
  \end{align*}
  whence $\norm{T^{\ast}\varphi - T^{\ast}\varphi_j}\leq \ve$, meaning $T^{\ast}\left( B_{X^{\ast}} \right)$ is totally bounded, hence $T^{\ast}$ compact.
\end{proof}
Recall that a linear map $T\colon X\rightarrow Y$ is called bounded below if there is $\delta > 0$ such that $\norm{Tx}\geq \delta \norm{x}$ for all $x$. In this case, $T\left( X \right)\subseteq Y$ is necessarily closed. Every invertible linear map is bounded below, as is every isometry.

Equivalently, a map $T\colon X\rightarrow Y$ is not bounded below if and only if there is a sequence of unit vectors $\left( x_n \right)_n\subseteq X$ such that $\lim_{n\rightarrow \infty}Tx_n = 0$.
\begin{theorem}
  Let $T$ be a compact operator on a Banach space $X$, and let $\lambda\in \C\setminus \set{0}$.
  \begin{enumerate}[(i)]
    \item The space $\ker\left( T-\lambda \id_X \right)$ is finite-dimensional.
    \item The space $\left( T-\lambda \id_X \right)(X)$ is closed and has finite codimension in $X$.
  \end{enumerate}
\end{theorem}
\begin{proof}
  Let $Z = \ker\left( T-\lambda \id_X \right)$. Then, $T(Z)\subseteq Z$, and the restriction $T|_{Z}$ is in $K(Z)$. Since $T|_{Z} = \lambda \id_Z$ with $\lambda\neq 0$, it follows that $\id|_{Z}$ is compact, meaning $Z$ is finite-dimensional.

  Since $Z$ is finite-dimensional, there is a closed subspace $Y$ of $X$ such that $X = Z\oplus Y$. 

  Observe that $\left( T - \lambda \id_X \right)X = \left( T - \lambda\id_X \right)Y$, so to show that $\left( T-\lambda\id_X \right)X$ is closed, it suffices to show that the restriction $\left( T - \lambda\id_X \right)|_{Y}$ is bounded below.

  Suppose otherwise. Then, there is a sequence $\left( x_n \right)_n$ of unit vectors in $Y$ such that $\lim_{n\rightarrow\infty}\norm{Tx_n - \lambda x_n} = 0$. We may assume without loss of generality that $\left( Tx_n \right)_n$ is convergent. It follows then that, since $x_n = \frac{1}{\lambda} \left( Tx_n - \left( T - \lambda\id_X \right)x_n \right)$, we have that $\left( x_n \right)_n\rightarrow x$ for some $x\in Y$, as $Y$ is closed. Since $Tx = \lambda x$, we have $x\in Y\cap \ker\left( T-\lambda\id_X \right)$, meaning $x = 0$. Yet, $x$ is the limit of unit vectors, and so is also a unit vector, which means we reach a contradiction. Thus, $\left( T - \lambda\id_X \right)|_{Y}$ is bounded below.

  Let $W = X/(T-\lambda\id_X)X$. To show that $\left( T-\lambda\id_X \right)X$ has finite codimension, we show that $W$ is finite-dimensional, by showing that $W^{\ast}$ is finite-dimensional. Let $\pi\colon X\rightarrow W$ be the quotient map. Then, $\ker\left( \pi^{\ast} \right)\subseteq \ker\left( T^{\ast}-\lambda\id_{X^{\ast}} \right)$. Letting $\sigma\in \ker\left( T^{\ast}-\lambda\id_{X^{\ast}} \right)$, we have that $\sigma$ annihilates $\left( T-\lambda\id_X \right)X$, so it induces a bounded linear functional $\tau\colon W\rightarrow \C$ such that $\sigma = \tau\circ\pi = \pi^{\ast}(\tau)$. Since $T^{\ast}$ is compact, $\ker\left( T^{\ast}-\lambda\id_{X^{\ast}} \right)$ is finite-dimensional, so $\pi^{\ast}$ has finite-dimensional range, and since $\pi^{\ast}$ is injective, $W^{\ast}$ is thus finite-dimensional, so $W$ is finite-dimensional. 
\end{proof}
\nocite{murphy_cstar_algebras_and_operator_theory}
\printbibliography
\end{document}
