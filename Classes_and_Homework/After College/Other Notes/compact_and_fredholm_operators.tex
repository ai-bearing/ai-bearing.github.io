\documentclass[10pt]{mypackage}

\usepackage{mlmodern}

%\usepackage{homework}
\usepackage{notes}

\usepackage[ backend=bibtex,
style=alphabetic,
sorting=ynt ]{biblatex}
\addbibresource{all_references.bib}

\usepackage{parskip}

\fancyhf{}
\fancyhead[R]{Avinash Iyer}
\fancyhead[L]{Compact and Fredholm Operators}
\fancyfoot[C]{\thepage}

\setcounter{secnumdepth}{0}

\begin{document}
\RaggedRight
\section{Compact Operators}%
\begin{definition}
  A linear map $T\colon X\rightarrow Y$ between Banach spaces is called \textit{compact} if $ T\left( B_X \right)\subseteq Y $ has compact closure, where $B_X$ denotes the closed unit ball of $X$. We denote the space of compact operators $K(X,Y)$.
\end{definition}
The theory of compact operators (and the soon to arise Fredholm operators) arose from the analysis of integral equations. To start, let $I = [0,1]$, and consider the Banach space $C(I)$ with the supremum norm. Letting $k\in C\left( I\times I \right)$, we define $u\in B(X)$ by taking
\begin{align*}
  Tf(x) &= \int_{0}^{1} k(x,y)f(y)\:dy.
\end{align*}
The fact that $Tf\in X$ follows from an application of the Dominated Convergence Theorem and the fact that, since $k(x,y)$ is jointly continuous, it is also separately continuous (see \cite[Theorem 2.27]{folland_real_analysis}). In fact, we can show something even stronger: we claim that the family $T\left( B_X \right)$ is in fact equicontinuous. This follows from the fact that, $I^2$ is compact, so if $\ve > 0$, there is $\delta$ such that whenever $\max\set{\left\vert x-x' \right\vert,\left\vert y-y' \right\vert} < \delta$, we have $\left\vert k\left( x,y \right) - k\left( x',y' \right) \right\vert < \ve$. Therefore,
\begin{align*}
  \left\vert Tf(x) - Tf\left(x'\right) \right\vert &= \left\vert \int_{0}^{1} \left( k\left( x,y \right) - k\left( x',y \right) \right)f(y)\:dy \right\vert\\
                                                   &\leq \int_{0}^{1} \left\vert k\left( x,y \right) - k\left( x',y \right) \right\vert\left\vert f(y) \right\vert\:dy\\
                                                   &\leq \sup_{y\in I} \left\vert k\left( x,y \right) - k\left( x',y \right) \right\vert \norm{f}_{u}\\
                                                   &\leq \ve \norm{f}_{u}.
\end{align*}
Furthermore, since
\begin{align*}
  \left\vert Tf(x) \right\vert &\leq \norm{k}_{u}\norm{f}_{u},
\end{align*}
we have that $T\left(B_X\right)$ is pointwise bounded. Thus, by the Arzelà--Ascoli theorem, it follows that $T\left( B_X \right)$ is totally bounded, so $T$ is a compact operator. We call the function $k$ the \textit{kernel} of the operator $T$.

Similarly, the operator $V\in B(X)$ given by
\begin{align*}
  Vf(x) &= \int_{0}^{x} f(y)\:dy
\end{align*}
is such that $V\left(B_X\right)$ is totally bounded by Arzelà--Ascoli, so $V$ is also compact. In fact, $V$ has no eigenvalues as well. This follows from the fact that, if there were $\lambda\in \C\setminus \set{0}$ with $V(f) = \lambda f$, then $f(0) = 0$ and $f'(t) = 1/\lambda f(t)$, so that $f(t) = f(0) e^{t/\lambda} = 0$, meaning $f = 0$.

We call the operator $V$ the \textit{Volterra integral operator} on $X$.

We can see that $K(X)$ is in fact an algebraic ideal in $B(X)$ (by continuity). In fact, there is a topological dimension to $K(X)\subseteq B(X)$.
\begin{proposition}
  If $X,Y$ are Banach spaces, then $K(X,Y)$ is a closed subspace of $B(X,Y)$.
\end{proposition}
\begin{proof}
  Let $\left( T_n \right)_n$ converge to $T\in B(X,Y)$. Let $\ve > 0$, and select $N$ such that $\norm{T_N - T} < \ve/3$. Since $T_N\left(B_X\right)$ is totally bounded, there are $x_1,\dots,x_n\in B_X$ such that for each $x\in S$, we have
  \begin{align*}
    \norm{T_Nx - T_Nx_j} < \ve/3
  \end{align*}
  for some $j$. Therefore, we have
  \begin{align*}
    \norm{Tx-Tx_j } &\leq \norm{Tx-T_Nx} + \norm{T_Nx-T_Nx_j} + \norm{T_Nx_j - Tx_j}\\
                    &< \ve.
  \end{align*}
  Therefore, $T\left( B_X \right)$ is totally bounded, so $T\in K(X,Y)$.
\end{proof}
Therefore, we see that $ \overline{F(X,Y)}\subseteq K(X,Y) $ is, where $F(X,Y)$ denotes the finite-rank operators, but this inclusion may be strict. In the cases where $ \overline{F(X)} = K(X) $, we say the Banach space $X$ has the approximation property. There are Banach spaces that do not have the approximation property.

Recall that if $T\colon X\rightarrow Y$ is a bounded linear map between Banach spaces, the transpose is defined by $T^{\ast}\colon Y^{\ast}\rightarrow X^{\ast}$, given by $T^{\ast}\varphi = \varphi\circ T$.
\begin{theorem}
  If $X$ and $Y$ are Banach spaces with $T\in K\left( X,Y \right)$, then $T^{\ast}\in K\left( Y^{\ast},X^{\ast} \right)$.
\end{theorem}
\begin{proof}
  Let $\ve > 0$. Since $T\left(B_X\right)$ is totally bounded, there exist elements $x_1,\dots,x_n$ such that if $x\in B_X$, then $\norm{Tx - Tx_i} < \ve/3$ for some $i$. Let $V\in B\left( Y^{\ast},\C^n \right)$ be defined by $V\varphi = \left( \varphi\left( Tx_1 \right),\dots,\varphi\left( Tx_n \right) \right)$. Since $V$ has finite rank, $V$ is compact, so $V\left( B_{X^{\ast}} \right)$ is totally bounded. Thus, there exist $\varphi_1,\dots,\varphi_m$ such that if $\varphi\in T$, then $\norm{V\varphi - V\varphi_j} = \max_{i=1}^{n} \left\vert T^{\ast}\varphi\left(x_i\right) - T^{\ast}\varphi_j(x_i) \right\vert$.

  Now, if $x\in B_X$, then $\norm{Tx - Tx_i} < \ve/3$ for some $i$, so thus $ \left\vert T^{\ast}\varphi\left( x_i \right) - T^{\ast}\varphi_j\left( x_i \right) \right\vert < \ve/3$. Thus,
  \begin{align*}
    \left\vert T^{\ast}\varphi(x) - T^{\ast}\varphi_j\left( x \right) \right\vert &\leq \left\vert T^{\ast}\varphi(x) - T^{\ast}\varphi\left(x_i\right) \right\vert + \left\vert T^{\ast}\varphi\left( x_i \right) - T^{\ast}\varphi_j\left( x_i \right) \right\vert + \left\vert T^{\ast}\varphi_j\left( x_i \right) - T^{\ast}\varphi_j\left( x \right) \right\vert\\
                                                                                  &< \ve,
  \end{align*}
  whence $\norm{T^{\ast}\varphi - T^{\ast}\varphi_j}\leq \ve$, meaning $T^{\ast}\left( B_{X^{\ast}} \right)$ is totally bounded, hence $T^{\ast}$ compact.
\end{proof}
Recall that a linear map $T\colon X\rightarrow Y$ is called bounded below if there is $\delta > 0$ such that $\norm{Tx}\geq \delta \norm{x}$ for all $x$. In this case, $T\left( X \right)\subseteq Y$ is necessarily closed. Every invertible linear map is bounded below, as is every isometry.

Equivalently, a map $T\colon X\rightarrow Y$ is not bounded below if and only if there is a sequence of unit vectors $\left( x_n \right)_n\subseteq X$ such that $\lim_{n\rightarrow \infty}Tx_n = 0$.
\begin{theorem}
  Let $T$ be a compact operator on a Banach space $X$, and let $\lambda\in \C\setminus \set{0}$.
  \begin{enumerate}[(i)]
    \item The space $\ker\left( T-\lambda \id_X \right)$ is finite-dimensional.
    \item The space $\left( T-\lambda \id_X \right)(X)$ is closed and has finite codimension in $X$.
  \end{enumerate}
\end{theorem}
\begin{proof}
  Let $Z = \ker\left( T-\lambda \id_X \right)$. Then, $T(Z)\subseteq Z$, and the restriction $T|_{Z}$ is in $K(Z)$. Since $T|_{Z} = \lambda \id_Z$ with $\lambda\neq 0$, it follows that $\id|_{Z}$ is compact, meaning $Z$ is finite-dimensional.

  Since $Z$ is finite-dimensional, there is a closed subspace $Y$ of $X$ such that $X = Z\oplus Y$. 

  Observe that $\left( T - \lambda \id_X \right)X = \left( T - \lambda\id_X \right)Y$, so to show that $\left( T-\lambda\id_X \right)X$ is closed, it suffices to show that the restriction $\left( T - \lambda\id_X \right)|_{Y}$ is bounded below.

  Suppose otherwise. Then, there is a sequence $\left( x_n \right)_n$ of unit vectors in $Y$ such that $\lim_{n\rightarrow\infty}\norm{Tx_n - \lambda x_n} = 0$. We may assume without loss of generality that $\left( Tx_n \right)_n$ is convergent. It follows then that, since $x_n = \frac{1}{\lambda} \left( Tx_n - \left( T - \lambda\id_X \right)x_n \right)$, we have that $\left( x_n \right)_n\rightarrow x$ for some $x\in Y$, as $Y$ is closed. Since $Tx = \lambda x$, we have $x\in Y\cap \ker\left( T-\lambda\id_X \right)$, meaning $x = 0$. Yet, $x$ is the limit of unit vectors, and so is also a unit vector, which means we reach a contradiction. Thus, $\left( T - \lambda\id_X \right)|_{Y}$ is bounded below.

  Let $W = X/(T-\lambda\id_X)X$. To show that $\left( T-\lambda\id_X \right)X$ has finite codimension, we show that $W$ is finite-dimensional, by showing that $W^{\ast}$ is finite-dimensional. Let $\pi\colon X\rightarrow W$ be the quotient map. Then, $\ker\left( \pi^{\ast} \right)\subseteq \ker\left( T^{\ast}-\lambda\id_{X^{\ast}} \right)$. Letting $\sigma\in \ker\left( T^{\ast}-\lambda\id_{X^{\ast}} \right)$, we have that $\sigma$ annihilates $\left( T-\lambda\id_X \right)X$, so it induces a bounded linear functional $\tau\colon W\rightarrow \C$ such that $\sigma = \tau\circ\pi = \pi^{\ast}(\tau)$. Since $T^{\ast}$ is compact, $\ker\left( T^{\ast}-\lambda\id_{X^{\ast}} \right)$ is finite-dimensional, so $\pi^{\ast}$ has finite-dimensional range, and since $\pi^{\ast}$ is injective, $W^{\ast}$ is thus finite-dimensional, so $W$ is finite-dimensional. 
\end{proof}
Note that if $T\colon X\rightarrow X$ is a linear map on a vector space, then the sequence of spaces $\left( \ker\left( T^n \right) \right)_{n}$ is increasing; if $\ker\left( T^n \right) \neq \ker\left( T^{n+1} \right)$ for all $n$, we say that $T$ has infinite \textit{ascent}, and write $\operatorname{asc}\left( T \right) = \infty$. Otherwise, we say $T$ has finite ascent, and define $\operatorname{asc}(T)$ to be the smallest $p$ such that $\ker\left( T^p \right) = \ker\left( T^{p+1} \right)$ for all $n \geq p$.

Similarly, the sequence of spaces $T^{n}(X)$ is decreasing. We say $T$ has infinite \textit{descent} if $T^{n}(X)\neq T^{n+1}(X)$ for all $n$, and we write $\operatorname{desc}(T) = \infty$. Else, we say $T$ has finite descent, and define $\operatorname{desc}(T)$ to be the smallest $p$ such that $T^{p+1}(X) = T^{p}(X)$.

To prove the next theorem, we recall the Riesz lemma.
\begin{lemma}[Riesz Lemma]
  Let $Y$ be a proper closed subspace of a normed vector space $X$. Then, for any $\ve > 0$, there is a unit vector $x\in X$ such that $\norm{x + Y} > 1-\ve$.
\end{lemma}
\begin{theorem}
  Let $T$ be a compact operator on a Banach space $X$. Suppose $\lambda\in \C\setminus \set{0}$. Then, $T- \lambda I$ has finite ascent and descent.
\end{theorem}
\begin{proof}
  Suppose toward contradiction that the ascent is infinite. Letting $N_n = \ker\left( T-\lambda I \right)^{n}$, we observe then that $N_{n-1}$ is a proper subspace of $N_n$, so by the Riesz Lemma, there is a unit vector $x_n\in N_n$ such that $\norm{x_n + N_{n-1}} \geq 1/2$. For any $m < n$, we have
  \begin{align*}
    Tx_n - Tx_m &= \lambda x_n + \left( T-\lambda \right)x_n - \left( T-\lambda \right)x_m - \lambda x_m\\
                &= \lambda x_n - z
  \end{align*}
  for some $z\in N_{n-1}$. Thus, $\norm{Tx_n - Tx_m} = \norm{\lambda x_n - z} \geq \left\vert \lambda \right\vert/2$. It follows that $\left( Tx_n \right)_n$ has no convergent subsequence, which contradicts the compactness of $T$.

  Similarly, if we let $V_n = \im\left( T-\lambda I \right)^{n}$, and suppose toward contradiction that $\operatorname{desc}\left( T-\lambda I \right) = \infty$, then we have that $V_{n}\leq V_{n-1}$ is a proper subspace, so there is some unit vector $x_n$ such that $\norm{x_n + V_{n-1}}\geq 1/2$. By a similar process, we find a sequence $\left( Tx_n \right)_n$ with no convergent subsequence, contradicting the assumption of compactness of the operator $T$.
\end{proof}
Note that this result also gives us that, if we let $N_{\lambda}\coloneq \ker\left( T - \lambda I \right)^{\operatorname{asc}(T)}$, we must have $\sigma\left( T|_{N_{\lambda}} \right) = \set{\lambda}$, seeing as the restriction $\left( T-\lambda I \right)|_{N_{\lambda}}$ is nilpotent on a finite-dimensional subspace.
\section{Fredholm Operators and Connections}%
\begin{definition}
  Let $X,Y$ be Banach spaces, and let $T\in B(X,Y)$. We say $T$ is \textit{Fredholm} if both $\dim\ker(T)$ and $\dim\coker(T)$ are finite. The index of $T$ is given by
  \begin{align*}
  \ind(T) &= \dim\ker(T)-\dim\coker(T).
  \end{align*}
\end{definition}
\begin{theorem}
  Let $X,Y$ be Banach spaces, and let $T\in B(X,Y)$. Suppose there is a closed subspace $Z$ of $Y$ such that $T(X)\oplus Z = Y$. Then, $T(X)$ is closed in $Y$.
\end{theorem}
\begin{proof}
  From the first isomorphism theorem, we may descend to the map $X/\ker(T)\rightarrow Y$ given by $x + \ker(T) \mapsto Tx$, so we may assume without loss of generality that $T$ is injective.

  The map
  \begin{align*}
    V\colon X\oplus Z &\rightarrow Y\\
    \left( x,z \right) &\mapsto Tx + z
  \end{align*}
  is a continuous isomorphism between Banach spaces, so by the open mapping theorem, $v^{-1}$ is also continuous. Letting $x\in X$, we have $\norm{x} = \norm{V^{-1}Tx}\leq \norm{V^{-1}}\norm{Tx}$, so that $\norm{Tx} \geq \norm{V^{-1}}^{-1}\norm{x}$, meaning $T$ is bounded below, and thus $T(X)$ is closed in $Y$.
\end{proof}
\begin{theorem}
  Let 
  \begin{center}
    % https://tikzcd.yichuanshen.de/#N4Igdg9gJgpgziAXAbVABwnAlgFyxMJZABgBpiBdUkANwEMAbAVxiRAA0QBfU9TXfIRQBGclVqMWbAJrdeIDNjwEiAJjHV6zVohAAtbuJhQA5vCKgAZgCcIAWyRkQOCElETtbACpyrth4juLkjqHlK6AMqGXEA
    \begin{tikzcd}
    X \arrow[r, "T"] & Y \arrow[r, "S"] & Z
    \end{tikzcd}
  \end{center}
  be Fredholm linear maps between Banach spaces $X,Y,Z$. Then, $ST$ is Fredholm with
  \begin{align*}
    \ind(ST) &= \ind(S) + \ind(T).
  \end{align*}
\end{theorem}
\begin{proof}
  Set $Y_2 = \ker(S)\cap T(X)$, and let $Y_1,Y_3,Y_4$ be such that $T(X) = Y_2\oplus Y_3$, $\ker(S) = Y_1 \oplus Y_2$, and $Y = T_1\oplus T(X)\oplus Y_4$, where $Y_1,Y_2,Y_4$ are finite-dimensional. We have that the map $\ker\left( ST \right)\rightarrow Y_2$, $x\mapsto Tx$ is surjective and has the same kernel as $T$, so $\ker\left( ST \right)$ is finite-dimensional with $\dim\ker(ST) = \dim\ker(T) + \dim\left(Y_2\right)$.

  Next, since $S(Y) = S\left( Y_3 \right)\oplus S\left( Y_4 \right)$ and $S\left( Y_3 \right) = ST(X)$, we have $S(Y) = ST(X)\oplus S\left(Y_4\right)$. Let $Z'$ be a finite-dimensional subspace of $Z$ such that $S(Y)\oplus Z' = Z$, so $Z = ST(X)\oplus S\left( Y_4 \right)\oplus Z'$. Since $S\left( Y_4 \right)\oplus Z'$ is finite-dimensional, $ST(X)$ has finite codimension in $Z$, so $ST$ is Fredholm.

  The map $Y_4\rightarrow S\left(Y_4\right)$ given by $y\mapsto Sy$ is a linear isomorphism, so $\dim\left( Y_4 \right) = \dim\left( S\left( Y_4 \right) \right)$, and thus 
  \begin{align*}
    \dim\coker\left( ST \right) &= \dim\left( Y_4 \right) + \dim\left( Z' \right)\\
                                &= \dim\left( Y_4 \right) + \dim\coker\left( S \right).
  \end{align*}
  Thus, we have
  \begin{align*}
    \dim\ker\left( ST \right) + \dim\coker\left( T \right) + \dim\coker\left( S \right) &= \dim\ker\left( T \right) + \dim\ker\left( S \right) + \dim\coker\left( ST \right),
  \end{align*}
  so that $\ind\left( ST \right) = \ind(S) + \ind(T)$.
\end{proof}
\begin{theorem}
  Let $T$ be a compact operator on a Banach space $X$, and let $\lambda\in \C\setminus \set{0}$.
  \begin{enumerate}[(i)]
    \item The operator $T - \lambda I$ is Fredholm of index $0$.
    \item If $p$ is the ascent of $T-\lambda I$, then
      \begin{align*}
        X &= \ker\left( T-\lambda I \right)^{p}\oplus \left( T-\lambda I \right)^{p}(X).
      \end{align*}
  \end{enumerate}
\end{theorem}
\begin{proof}\hfill
  \begin{enumerate}[(i)]
    \item We know from the above theorem that $\ker\left( T-\lambda I \right)$ and $\coker\left( T-\lambda I \right)$ are finite-dimensional. If $m,n$ are integers greater than the maximum of the ascent and descent of $T-\lambda I$, then we have $\dim\ker\left( T - \lambda I \right)^{n} = \dim\ker\left( T-\lambda I \right)^{m}$, and analogously for the cokernel, whence $\ind\left( T-\lambda I \right)^{m} = \ind\left( T-\lambda I \right)^{n}$ for all such $m,n$. Thus, $m\ind\left( T-\lambda I \right) = n\ind\left( T-\lambda I \right)$, so that $\ind\left( T-\lambda I \right) = 0$.
    \item Let $x\in \ker\left( T-\lambda I \right)^{p}\cap \left( T-\lambda I \right)^{p}(X)$. Then, there is $y\in X$ such that $x = \left( T-\lambda I \right)^{p} y$ with $\left( T-\lambda I \right)^{2p} y = 0$. Since $\ker\left( T-\lambda I \right)^{p} = \ker\left( T-\lambda I \right)^{2p}$, it follow that $\left( T-\lambda I \right)^{p} y = 0$, whence $x= 0$. Moreover, since $\dim\ker\left( T-\lambda I \right)^{p} = \dim\coker\left( T-\lambda I \right)^{p}$, we have $X = \ker\left( T-\lambda I \right)^{p}\oplus \left( T-\lambda I \right)^{p}(X)$.
  \end{enumerate}
\end{proof}
\begin{corollary}[Fredholm Alternative]
  The operator $T-\lambda I$ is injective if and only if it is surjective.
\end{corollary}
\begin{proof}
  We have that $\dim\ker\left( T-\lambda I \right) = 0$ if and only if $\dim\coker\left( T-\lambda I \right) = 0$, meaning that $T-\lambda I$ is injective if and only if it is surjective.
\end{proof}
Now, we will understand the structure of the Fredholm operators in the space of all bounded operators.
\begin{proposition}
  The space of Fredholm operators in $B\left( X,Y \right)$ is open. The index is a continuous integer-valued function on the space of Fredholm operators.
\end{proposition}
\begin{proof}
  Let $T$ be Fredholm, with $N = \ker(T)$. Let $V$ be a complement for $N$ such that $X = N\oplus V$, so that $T(V) = T(X)$. Since $T(X)$ is closed and has finite codimension, we may find a subspace $W$ of $Y$ such that $T(X)\oplus W = Y$, and define $ \widetilde{T}\colon V\oplus W \rightarrow Y $ by $ \widetilde{T}(v,w) = Tv + w $.

  Note that
  \begin{align*}
    \norm{ \widetilde{T}(v,w) } &\leq \norm{Tv} + \norm{w}\\
                                &\leq \max\left(\norm{T}_{\op},1\right)\left( \norm{v} + \norm{w} \right),
  \end{align*}
  meaning that $ \widetilde{T} $ is continuous, and is necessarily a bijection by the fact that exact sequences of vector spaces split.

  Let $ S\in B(X,Y) $ be with $\norm{S-T}_{\op} < 1/\norm{ \widetilde{T}^{-1} }$. Define $ \widetilde{S}\colon V\oplus W \rightarrow Y $ by $ \widetilde{S}\left( v,w \right) = Sv + w $. Then, $ \widetilde{S} $ is continuous, with
  \begin{align*}
    \norm{ \widetilde{S} - \widetilde{T} }_{\op} &= \norm{ \left( S-T \right)|_{V} }_{\op}\\
                                                 &< 1/\norm{ \widetilde{T}^{-1} }.
  \end{align*}
  Thus, $ \widetilde{S} $ is invertible, meaning that $ S(V) = \widetilde{S}(V) $ is closed, and $Y = S(V) + W$.

  In particular, this means that $\dim(Y/S(X))\leq \dim(Y/S(V)) = \dim(W) < \infty$, and since $\ker(S)\cap V = \set{0}$, the quotient map $Q\colon X\rightarrow X/V$ is injective on $\ker(S)$, so that
  \begin{align*}
    \dim\ker(S) &\leq \dim(X/V)\\
                &= \dim(N)\\
                &< \infty,
  \end{align*}
  so $S$ is Fredholm. Thus, the space of Fredholm operators is open in $B(X,Y)$.

  The subspace $V + \ker(S) = Q^{-1}\left( \left( \ker(S) + V \right)/V \right)$ is closed and of finite codmension. Choose a finite-dimensional $Z$ such that $X = V + \ker(S) + Z$. Then, $S(X) = S(V) + S(Z)$, since $S$ is injective on $V + Z$ and $Z$ is finite-dimensional. Letting $P\colon Y = S(V) + W \rightarrow W$ be the projection onto $W$ with kernel $S(V)$, we have
  \begin{align*}
    P(S(X)) &= P(S(Z))\\
            &\cong Z
  \end{align*}
  since $S$ is injective on $Z$ and $P$ is injective on $S(Z)$. Thus,
  \begin{align*}
    Y/S(X) &= \left( Y/S(V) \right)/\left( S(X)/S(V) \right)\\
           &\cong W/P(S(Z)).
  \end{align*}
  We may compute
  \begin{align*}
    \ind(S) &= \dim\ker(S) - \dim\coker(S)\\
            &= \dim\ker(S) - (\dim(W) - \dim(Z))\\
            &= \dim(\ker(S) \oplus Z) - \dim(W)\\
            &= \dim\ker(T) - \dim\coker(T)\\
            &= \ind(T),
  \end{align*}
  so the index is constant on an open ball around $T$, whence it is locally constant.
\end{proof}
We know that $K(X)$ is a (closed) ideal in $B(X)$, so we can consider the quotient $B(X)/K(X)$. It turns out that there is a characterization of Fredholm operators in the quotient $B(X)/K(X)$.
\begin{theorem}[Atkinson]
  An operator $T\in B(X)$ is Fredholm if and only if $ \pi(T)\in B(X)/K(X) $ is invertible.
\end{theorem}
\begin{proof}
  Suppose $T$ is Fredholm. Let $N = \ker(T)$, $V$ a complement for $N$ so that $X = N\oplus V$. Since $T$ is Fredholm, we have a finite-dimensional complement $W$ such that $X = T(V) \oplus W$. If we let $T|_{V}\in B(V,T(V))$, then we have that $T|_{V}$ is a continuous bijection. By the bounded inverse theorem, there is a continuous inverse $S\in B(T(V),V)$. Let $ \overline{S}\in B(X) $ be such that $ \overline{S}(Tv + w) = v $ for $v\in V$ and $w\in W$.

  Then, we have $ \overline{S}T(u + v) = \overline{S}Tv = v $ for any $u\in N$ and $v\in V$. That is, $ \overline{S}T$ is the projection of $X$ onto $V$, $P_V$. Since $I - P_V = P_N$ is finite-rank, we ahve
  \begin{align*}
    \pi( \overline{S} )\pi(T) &= \pi\left( I-P_N \right)\\
                              &= \pi(I).
  \end{align*}
  Similarly, we have $T \overline{S}\left( Tv + w \right) = Tv = P_{T(V)}(Tv + w)$ is the projection onto $T(V)$ with kernel $W$, where $I-P_{T(V)} = P_W$ is finite-rank, meaning that $ \pi(T) \pi( \overline{S} ) = \pi(I)$, meaning $\pi(T)$ is invertible.

  Now, if $\pi(T)$ is invertible, with inverse $\pi(S)$, we have compact operators $K$ and $L$ such that $ST = I + K$ and $TS = I + L$. Since $\ker(T) \subseteq \ker\left( I+K \right)$, $\ker(T)$ is finite-dimensional, and since $\img(T) \supseteq \img\left(I + L\right)$, we have that $\img(T)$ is closed with finite codimension. Thus, $T$ is Fredholm.
\end{proof}
Finally, our final application of the theory of compact and Fredholm operators is an important structural result. Recall the definition of the commutant and bicommutant.
\begin{lemma}
  Let $T\in K(X)$. Suppose there is a sequence of closed subspaces $V_0\subsetneq V_1\subsetneq \cdots$ and there are scalars $\lambda_i$ such that $\left( T-\lambda_i I \right)V_i\subseteq V_{i-1}$ for each $i$. Then, $\lim_{i\rightarrow\infty}\lambda_i = 0$.
\end{lemma}
\begin{proof}
  Using the Riesz Lemma, we may select unit vectors $x_i\in V_i$ with $\dist_{V_{i-1}}\left(x_i\right)\geq 1/2$ for each $i\geq 1$. Then,
  \begin{align*}
    Tx_i &= \lambda_i x_i + y_i
  \end{align*}
  for some $y_i\in V_{i-1}$. If there is a subsequence with $\left\vert \lambda_{i_k} \right\vert \geq \delta > 0$, then for $m < n$, we have $y_{i_n}-Tx_{i_n} \in V_{i_n-1}$, and
  \begin{align*}
    \norm{Tx_{i_n} - Tx_{i_m}} &= \norm{\lambda_{i_n}x_{i_n} + \left( Tx_{i_m} - y_{i_n} \right)}\\
                               &\geq \frac{1}{2}\left\vert \lambda_{i_n} \right\vert\\
                               &\geq \frac{1}{2}\delta,
  \end{align*}
  meaning that $ \overline{T\left( B_X \right)} $ is not compact, which is a contradiction.
\end{proof}
\begin{definition}
  If $S\subseteq B(X)$ is a collection of operators, the \textit{commutant} of $S$ is the set
  \begin{align*}
    S' &= \set{T\in B(X) | TP = PT\text{ for all }P\in S}.
  \end{align*}
  The bicommutant of $S$, $S''$, is $(S')'$. Note that $S\subseteq S''$ always.
\end{definition}
\begin{theorem}
  Let $X$ be an infinite-dimensional Banach space, and let $T\in K(X)$. Then, the following hold.
  \begin{enumerate}[(i)]
    \item We have $0\in \sigma(T)$. If $0\neq \lambda\neq\in \sigma(T) $, then $\lambda$ is in the point spectrum (that is, $\lambda$) is an isolated point.
    \item The spectrum of $T$ is either finite or is equal to a countable set $\set{\lambda_n}_{n\geq 1}\cup \set{0}$ with $\lim_{n\rightarrow\infty}\lambda_n = 0$.
    \item For each $\lambda\in \sigma(T)\setminus \set{0}$, there is a decomposition
      \begin{align*}
        X &= N_{\lambda} \oplus R_{\lambda},
      \end{align*}
      with
      \begin{align*}
        N_{\lambda} &= \ker\left( T-\lambda I \right)^{n_{\lambda}}\\
        R_{\lambda} &= \img\left( T-\lambda I \right)^{n_{\lambda}},
      \end{align*}
      where $n_{\lambda} = \operatorname{asc}(T) = \operatorname{desc}(T)$.
    \item We have $\sigma\left( T|_{N_{\lambda}} \right) = \set{\lambda}$ and $\sigma\left( T|_{R_{\lambda}} \right) = \sigma(T)\setminus \set{\lambda}$.
    \item There is a unique finite-rank projection $E_{\lambda}$ in the bicommutant $\set{T}''$ such that $\img\left( E_{\lambda} \right) = N_{\lambda}$ and $\ker\left( E_{\lambda} \right) = R_{\lambda}$. Furthermore, both $N_{\lambda}$ and $R_{\lambda}$ are invariant for $\set{T}'$.
    \item If $\lambda,\mu\in \sigm(T)\setminus \set{0}$ are distinct, then $E_{\lambda}E_{\mu} = 0$.
  \end{enumerate}
\end{theorem}
\begin{proof}
  Compact operators cannot be surjective, since by the open mapping theorem, there is $r$ such that $U(0,r)\subseteq T\left(U_X\right)$, which would contradict compactness. Thus, $0\in \sigma(T)$. By the Fredholm alternative, we know that if $\lambda\neq 0$ then either $\ker\left( T-\lambda I \right)\neq \set{0}$ or $T-\lambda I$ is invertible, so $\sigma(T) = \sigma_p(T)\cup \set{0}$ by the definition of the point spectrum, which consists of all the points such that $\ker\left( T-\lambda I \right)\neq \set{0}$. This shows (i).

  If $\lambda\in \sigma(T)\setminus \set{0}$, then we know from an earlier result that $X = N_{\lambda} + R_{\lambda}$, and that $N_{\lambda}$ and $R_{\lambda}$ are $\set{T}'$-invariant. In particular, we have
    \begin{align*}
      \sigma(T) &= \sigma\left( T|_{N_{\lambda}} \right)\cup \sigma\left( T|_{R_{\lambda}} \right)\\
                &= \set{\lambda}\cup \sigma\left( T|_{R_{\lambda}} \right). 
    \end{align*}
    This shows (iv).

    Furthermore, since $\left( T-\lambda I \right)|_{R_{\lambda}}$ is invertible (again by the Fredholm alternative), we have that $\sigma\left( T|_{R_{\lambda}} \right) = \sigma\left( T \right)\setminus \set{\lambda}$, meaning that $\lambda$ is isolated. In particular, this means that the spectrum is at most countably infinite.

    For each $\lambda_n\in \sigma(T)\setminus \set{0}$, select a unit vector $x_n\in \ker\left( T-\lambda_n I \right)$, and take $V_n = \Span\set{x_1,\dots,x_n}$.Then, $V_{n-1}\subsetneq V_n$, and $\left( T-\lambda I \right) V_n\subseteq V_{n-1}$, so that $\lim_{n\rightarrow\infty}\lambda_n = 0$ by the above lemma. This shows (ii).

    Now, there is a unique projection with range $N_{\lambda}$ and kernel $R_{\lambda}$, so if $S\in \set{T}'$, then $S$ leaves both the range and kernel of $E_{\lambda}$ invariant. This means
    \begin{align*}
      SE_{\lambda} &= E_{\lambda}S E_{\lambda}\\
      S\left( I-E_{\lambda} \right) &= \left( I-E_{\lambda} \right)S\left( I-E_{\lambda} \right),
    \end{align*}
    so that $SE_{\lambda} = E_{\lambda}SE_{\lambda} = E_{\lambda}S$, meaning that $E_{\lambda}\in \set{T}''$.

    Now, if $\mu$ is another nonzero element of $\sigma(T)$, then $E_{\mu}\in \set{T}''\subseteq \set{T}'$, meaning that $E_{\lambda}E_{\mu} = E_{\mu}E_{\lambda}$, while
    \begin{align*}
      \img\left( E_{\mu}E_{\lambda} \right) &\subseteq N_{\lambda}\cap N_{\mu}\\
                                            &= \set{0}.
    \end{align*}
    Thus, $E_{\lambda}E_{\mu} = 0$.
\end{proof}
\nocite{murphy_cstar_algebras_and_operator_theory,davidson_functional_analysis}
\printbibliography
\end{document}
