\documentclass[10pt]{mypackage}

% sans serif font:
%\usepackage{cmbright}
%\usepackage{sfmath}
%\usepackage{bbold} %better blackboard bold

%\usepackage{homework}
\usepackage{notes}
\usepackage{newpxtext,eulerpx,eucal}
\renewcommand*{\mathbb}[1]{\varmathbb{#1}}

\fancyhf{}
\rhead{Avinash Iyer}
\lhead{Algebra I: Notes and Review}

\setcounter{secnumdepth}{0}

\begin{document}
\RaggedRight
These are some notes from my Algebra I class. We use the textbook \textit{Abstract Algebra} by Dummit and Foote, and will cover rings, groups, and modules.
\section{PIDs, UFDs and All That}%
We always assume here that $R$ is commutative and unital.
\subsection{Preliminaries}%
\begin{definition}
  If $a_1,\dots,a_n\in R$, then the \textit{ideal generated by} $a_1,\dots,a_n$ is given by
  \begin{align*}
    \left( a_1,\dots,a_n \right) \coloneq \bigcap\set{I | a_1,\dots,a_n\in I,I\text{ is an ideal in }R}.
  \end{align*}
  An ideal is called \textit{principal} if $I = \left( a \right)$ for some $a\in I$. We may write $I = a\cdot R$ in this case.
\end{definition}
\begin{definition}
  If $I$ and $J$ are ideals in $R$, then $IJ$ is given by
  \begin{align*}
    IJ &= \set{\sum_{i=1}^{n}x_iy_i | x_i\in I, y_i\in J,n\in \N}.
  \end{align*}
\end{definition}
\begin{theorem}[Isomorphism Theorems]\hfill
  \begin{description}
    \item[First Isomorphism Theorem:] Let $\varphi\colon R\rightarrow S$ be a ring homomorphism. Then, $ \overline{\varphi}\colon R/\ker\left( \varphi \right)\rightarrow \img\left( \varphi \right) $ is an isomorphism given by $ \overline{\varphi}\left( a + \ker\left( \varphi \right) \right) = \varphi\left( a \right) $.
    \item[Second Isomorphism Theorem:] Let $R$ be a ring, $S\subseteq R$ a subring, and let $I\subseteq R$ be an ideal. Then,
      \begin{enumerate}[(i)]
        \item $I + S$ is a subring of $R$;
        \item $I$ is an ideal of $I + S$;
        \item $I\cap S$ is an ideal of $S$;
        \item $S/I\cap S \cong I + S/I$.
      \end{enumerate}
    \item[Third Isomorphism Theorem:] Let $R$ be a ring, $I,J$ ideals of $R$ with $I\subseteq J$. Then, $J/I$ is an ideal of $R/I$, and we have $\left( R/I \right)/\left( J/I \right) \cong R/J$.
    \item[Fourth Isomorphism Theorem:] If $R$ is a ring and $I$ is an ideal, then there is a one-to-one correspondence between subrings of $R/I$ and subrings of $R$ containing $I$.
  \end{description}
\end{theorem}
\begin{definition}
  Let $M$ be an ideal in $R$.
  \begin{enumerate}[(i)]
    \item We say $M$ is prime if $M\neq R$ and, for any $ab\in M$, we have either $a\in M$ or $b\in M$.
    \item We say $M$ is maximal if $M\neq R$ and if $M\subseteq I\subseteq R$ where $I$ is an ideal, then either $I = M$ or $I = R$.
  \end{enumerate}
\end{definition}
\begin{theorem}
  Let $M$ be an ideal in $R$.
  \begin{enumerate}[(i)]
    \item $M$ is prime if and only if $R/M$ is an integral domain.
    \item $M$ is maximal if and only if $R/M$ is a field.
  \end{enumerate}
\end{theorem}
\begin{proof}\hfill
  \begin{enumerate}[(i)]
    \item Let $M$ be maximal, with $a + M \in R/M$, $a + M \neq 0 + M$. Then, $a\notin M$, so that the ideal $\left( a \right) + M$ strictly contains $M$. Therefore, $1 + M\in \left( a \right) + M$, meaning there is some $r + M$ such that $ \left( r + M \right)\left( a + M \right) = 1 + M $. Thus, an inverse exists.\newline

      Now, if $R/M$ is a field, and $M \subsetneq I \subseteq R$, then $I/M$ is an ideal of $R/M$, and since $I\supsetneq M$, we have $I/M\neq 0 + M$. Since $R/M$ is a field, its only ideals are either $ 0 + M $ and $R/M$, so $I/M = R/M$, meaning $I = R$.
    \item We have $P\subseteq R$ is prime if and only if $ab\in P$ implies $a\in P$ or $b\in P$. Yet, means that $ab + P = 0 + P$ if and only if $ a = 0 + P $ or $b = 0 + P$.
  \end{enumerate}
\end{proof}
\subsection{Chinese Remainder Theorem}%
\begin{definition}
  We say two ideals $I$ and $J$ are \textit{coprime} if $I + J = R$, or that there exist $x\in I$ and $y\in J$ such that $x + y = 1$.
\end{definition}
\begin{theorem}[Chinese Remainder Theorem]
  Let $I_1,\dots,I_n$ be pairwise coprime ideals of $R$. Then, for any $a_1,\dots,a_n\in R$, there exists $x\in R$ with $x\equiv a_i$ modulo $I_i$ for all $i$. In other words, there a solution to the system of congruences given by
  \begin{align*}
    x + I_1 &= a_1 + I_1\\
    x + I_2 &= a_2 + I_2\\
            &\vdots\\
    x + I_n &= a_n + I_n.
  \end{align*}
\end{theorem}
\begin{proof}
  It suffices to construct elements $y_1,\dots,y_n$ such that $y_i\equiv 1$ modulo $I_i$ and $0$ otherwise. Then, we will be able to set $x = \sum_{i} a_iy_i$ as our desired solution.\newline

  We construct $y_1$ as follows. From our assumption, $I_1 + I_j = R$ for all $j \geq 2$, so for each $j\geq 2$, there exists $u_j\in I_1$ and $v_j\in I_j$ such that $u_j + v_j = 1$. Taking the product, we find that
  \begin{align*}
    \prod_{j=2}^{n} \left( u_j + v_j \right) &= 1\\
                                             &= \underbrace{v_2 \cdots v_n}_{\eqcolon y_1} \underbrace{+ \cdots + u_2\cdots u_n}_{\eqcolon x_1}.
  \end{align*}
  We verify that $y_1$ does the job, which we can see by the fact that $y_1 \equiv 0$ modulo $I_j$ for $j\neq 1$, as $v_2\cdots v_j\in I_2\cdots I_j\subseteq I_j$ for each $j\geq 2$. Similarly, each summand in $x_1$ contains at least one $u_j$, so $x_1\equiv 0$ modulo $I_1$.\newline

  The rest of the $y_i$ follow analogously.
\end{proof}
We can restate the Chinese Remainder Theorem in a variety of ways.
\begin{theorem}[Chinese Remainder Theorem, Alternative Versions]
  Let $I_1,\dots,I_n$ be pairwise coprime ideals.
  \begin{enumerate}[(i)]
    \item There exists a surjective homomorphism 
      \begin{align*}
        \varphi\colon R&\rightarrow  R/I_1\times\cdots\times R/I_n\\
        r &\mapsto \left( r + I_1,\dots, r + I_n \right).
      \end{align*}
      This homomorphism induces an isomorphism
      \begin{align*}
        \overline{\varphi}\colon R/\left( I_1\cap\cdots\cap I_n \right) \rightarrow R/I_1\times\cdots\times R/I_n.
      \end{align*}
    \item If $I_1,\dots,I_n$ are pairwise coprime, then
      \begin{align*}
        R/I_1\cdots I_n &\cong R/I_1\times\cdots\times R/I_n
      \end{align*}
      are isomorphic.
  \end{enumerate}
\end{theorem}
\begin{example}
  We observe that if $R = \Z$, and $p_1,\dots,p_r$ are distinct primes with $\ell_1,\dots,\ell_r$ positive integers, then
  \begin{align*}
    \Z/p_1^{\ell_1}\cdots p_{r}^{\ell_r}\Z &\cong \Z/p_{1}^{\ell_1}\Z \times\cdots\times \Z/p_{r}^{\ell_r}\Z.
  \end{align*}
\end{example}
\begin{example}[Polynomial Interpolation]
  If we let
  \begin{align*}
    p_i(x) &= x-\alpha_i,
  \end{align*}
  where $\alpha_i\in \F$, we observe that there is a surjective evaluation homomorphism
  \begin{align*}
    \operatorname{ev}\colon \frac{\F\left[ x \right]}{\left( p_i(x) \right)} &\rightarrow \F,
  \end{align*}
  given by $f(x) \mapsto f\left( \alpha_i \right)$. In particular, if $\alpha_1,\dots,\alpha_r$ are distinct, then
  \begin{align*}
    \frac{\F\left[ x \right]}{\left( p_1(x),\dots,p_r(x) \right)} &\cong \F\times\cdots\times \F,
  \end{align*}
  so that, for all $\beta_1,\dots,\beta_r\in \F$, there is some $f(x)\in \F\left[ x \right]$ such that $f\left( \alpha_i \right) = \beta_i$ for $ i = 1,\dots,r $.
\end{example}
\subsection{Field of Fractions and Localization}%
Given a ring $R$, how can we find maximal ideals in $R$? More specifically, given a commutative ring $R$ with $1$, and prime ideal $ P\subseteq R $, we want to construct a new ring $R_{\mathfrak{p}}$ with unique maximal ideal $P$.\newline

Toward this end, we start by reviewing a useful construction known as the field of fractions.
\begin{definition}
  Let $R$ be an integral domain. We define the field $K = \operatorname{frac}\left( R \right)$ to be the unique field with an injection
  \begin{align*}
    \iota\colon R&\hookrightarrow K\\
    1_{R} &\mapsto 1_{K},
  \end{align*}
  satisfying the following universal property.\newline

  Given any embedding into a field, $\sigma\colon R\hookrightarrow L$, such that $1_{R}\mapsto 1_{L}$, there is a unique extension $ \widetilde{\sigma}\colon K\rightarrow L $ such that the following diagram commutes.
  \begin{center}
    % https://tikzcd.yichuanshen.de/#N4Igdg9gJgpgziAXAbVABwnAlgFyxMJZABgBpiBdUkANwEMAbAVxiRACUQBfU9TXfIRQAmUsKq1GLNgBluvEBmx4CRUZWr1mrRCADS3CTCgBzeEVAAzAE4QAtkjIgcEJKMna2AHS-4cdEGp-LAY2AAsICABreStbB0QnFyQARmoGOgAjGAYABX4VIRBrLBMwnECPaV0fbBM7AJ44+zcg10Q0kAzsvILBNgYYSwrNKR0QHwB3LFg8BlhgWtKGrkMuIA
\begin{tikzcd}
R \arrow[rr, "\iota", hook] \arrow[rrdd, "\sigma"'] &  & K \arrow[dd, "\widetilde{\sigma}"] \\
                                                    &  &                                    \\
                                                    &  & L                                 
\end{tikzcd}
  \end{center}
\end{definition}
In order to construct $K$, we let $S\subseteq R\times R$ be defined by
\begin{align*}
  S &= \set{\left( a,b \right) | b\neq 0}.
\end{align*}
We impose an equivalence relation on $S$ by saying $\left( a,b \right)\sim \left( c,d \right)$ if and only if $ ad - bc = 0 $. Clearly, this relation is reflexive and symmetric. To see that it is transitive, we let $\left( a,b \right)\sim \left( c,d \right)$, and $ \left( c,d \right)\sim \left( e,f \right) $, meaning $ad-bc=  0$ and $cf-de = 0$. Multiplying the first equation by $f$ and the second equation by $b$, then subtracting, we get $ adf - bde = 0 $, meaning $d\left( af-be  \right)= 0$. Since $R$ admits no zero divisors, this means that $ af-be = 0 $, so the relation is transitive.\newline

We write $ \left[ \left( a,b \right) \right] = \frac{a}{b} $ for $K$, with operations
\begin{align*}
  \frac{a}{b} + \frac{c}{d} &= \frac{ad + bc}{bd}\\
  \frac{a}{b}\cdot \frac{c}{d} &= \frac{ac}{bd}.
\end{align*}
These operations are well-defined and do satisfy the universal property. Verifying this is a pain, but it can be done.\newline

Now, we may extend this to all unital commutative rings, not just integral domains.
\begin{definition}
  Let $R$ be a unital commutative ring, and let $S\subseteq R$. We say $S$ is \textit{multiplicative} if
  \begin{itemize}
    \item $1\in S$;
    \item $0\notin S$;
    \item for any $x,y\in S$, $xy\in S$.
  \end{itemize}
\end{definition}
\begin{example}\hfill
  \begin{enumerate}[(i)]
    \item If $R$ is an integral domain, then $R\setminus \set{0}$ is multiplicative.
    \item If $z\in R$ is such that $z$ is not nilpotent, then $S = \set{z^{n} | n\geq 0}$ is multiplicative.
    \item If $P$ is a prime ideal, then $S = R\setminus P$ is multiplicative.
  \end{enumerate}
\end{example}
We will use (iii) to construct a ring with a unique maximal ideal. First, though, we construct a ring of fractions using multiplicative sets.
\begin{definition}
  Let $R$ be a unital commutative ring, and let $S\subseteq R$ be multiplicative. We construct a ring $S^{-1}R$ by taking an equivalence relation on $R\times S$ as follows:
  \begin{align*}
    \left( a,s \right)\sim \left( b,t \right) &\Leftrightarrow \exists s'\in S\text{ such that }s'\left( at-bs \right) = 0.
  \end{align*}
  We write
  \begin{align*}
    S^{-1}R &= \set{\left[ \left( a,s \right) \right] | a\in R,s\in S},
  \end{align*}
  and denote
  \begin{align*}
    \left[ \left( a,s \right) \right] &= \frac{a}{s}.
  \end{align*}
  This becomes a ring under the operations
  \begin{align*}
    \frac{a}{s} + \frac{b}{t} &= \frac{at + bs}{st}\\
    \frac{a}{s}\cdot \frac{b}{t} &= \frac{ab}{st}.
  \end{align*}
  We call $S^{-1}R$ the \textit{localization of $R$ with respect to $S$}.
\end{definition}
We can see some basic properties of the localization. 
\begin{proposition}
  Let $R$ be a unital commutative ring, $S\subseteq R$ multiplicative, and let $S^{-1}R$ be the corresponding localization.
  \begin{itemize}
    \item The additive identity in $S^{-1}R$ is $\frac{0}{1}$.
    \item The additive inverse of $ \frac{a}{s} $ in $S^{-1}R$ is $\frac{-a}{s}$.
    \item For all $a\in R$ and all $s,s'\in S$, we have $ \frac{as'}{ss'} = \frac{a}{s} $.
    \item Every element of the form $ \frac{s}{t} $ where both $s,t\in S$ is invertible, with corresponding inverse $ \frac{t}{s} $.
    \item The map $\iota_S\colon R\rightarrow S^{-1}R$ given by $ r\mapsto \frac{r}{1} $ is an injective ring homomorphism such that $\iota_S\left( S \right)\subseteq \left( S^{-1}R \right)^{\times}$, where $\left( S^{-1}R \right)^{\times}$ denotes the group of invertible elements in $S^{-1}R$.
  \end{itemize}
\end{proposition}
\end{document}
