\documentclass[10pt]{mypackage}

% sans serif font:
%\usepackage{cmbright}
%\usepackage{sfmath}
%\usepackage{bbold} %better blackboard bold

%\usepackage{homework}
\usepackage{notes}
\usepackage{newpxtext,eulerpx,eucal}
%\usepackage{mlmodern}
%\renewcommand{\mathbb}[1]{\mathds{#1}}
\renewcommand*{\mathbb}[1]{\varmathbb{#1}}

\fancyhf{}
\rhead{Avinash Iyer}
\lhead{Algebra I: Notes and Review}

\setcounter{secnumdepth}{0}

\begin{document}
\RaggedRight
These are some notes from my Algebra I class. We use the textbook \textit{Abstract Algebra} by Dummit and Foote, and will cover rings, groups, and modules.
\section{PIDs, UFDs and All That}%
We always assume here that $R$ is commutative and unital.
\subsection{Preliminaries}%
\begin{definition}
  If $a_1,\dots,a_n\in R$, then the \textit{ideal generated by} $a_1,\dots,a_n$ is given by
  \begin{align*}
    \left( a_1,\dots,a_n \right) \coloneq \bigcap\set{I | a_1,\dots,a_n\in I,I\text{ is an ideal in }R}.
  \end{align*}
  An ideal is called \textit{principal} if $I = \left( a \right)$ for some $a\in I$. We may write $I = a\cdot R$ in this case. A ring where every ideal is principal is called a \textit{principal ideal domain}.
\end{definition}
\begin{definition}
  If $I$ and $J$ are ideals in $R$, then $IJ$ is given by
  \begin{align*}
    IJ &= \set{\sum_{i=1}^{n}x_iy_i | x_i\in I, y_i\in J,n\in \N}.
  \end{align*}
\end{definition}
\begin{theorem}[Isomorphism Theorems]\hfill
  \begin{description}
    \item[First Isomorphism Theorem:] Let $\varphi\colon R\rightarrow S$ be a ring homomorphism. Then, $ \overline{\varphi}\colon R/\ker\left( \varphi \right)\rightarrow \img\left( \varphi \right) $ is an isomorphism given by $ \overline{\varphi}\left( a + \ker\left( \varphi \right) \right) = \varphi\left( a \right) $.
    \item[Second Isomorphism Theorem:] Let $R$ be a ring, $S\subseteq R$ a subring, and let $I\subseteq R$ be an ideal. Then,
      \begin{enumerate}[(i)]
        \item $I + S$ is a subring of $R$;
        \item $I$ is an ideal of $I + S$;
        \item $I\cap S$ is an ideal of $S$;
        \item $S/I\cap S \cong I + S/I$.
      \end{enumerate}
    \item[Third Isomorphism Theorem:] Let $R$ be a ring, $I,J$ ideals of $R$ with $I\subseteq J$. Then, $J/I$ is an ideal of $R/I$, and we have $\left( R/I \right)/\left( J/I \right) \cong R/J$.
    \item[Fourth Isomorphism Theorem:] If $R$ is a ring and $I$ is an ideal, then there is a one-to-one correspondence between subrings of $R/I$ and subrings of $R$ containing $I$.
  \end{description}
\end{theorem}
\begin{definition}
  Let $M$ be an ideal in $R$.
  \begin{enumerate}[(i)]
    \item We say $M$ is prime if $M\neq R$ and, for any $ab\in M$, we have either $a\in M$ or $b\in M$.
    \item We say $M$ is maximal if $M\neq R$ and if $M\subseteq I\subseteq R$ where $I$ is an ideal, then either $I = M$ or $I = R$.
  \end{enumerate}
\end{definition}
\begin{theorem}
  Let $M$ be an ideal in $R$.
  \begin{enumerate}[(i)]
    \item $M$ is prime if and only if $R/M$ is an integral domain.
    \item $M$ is maximal if and only if $R/M$ is a field.
  \end{enumerate}
\end{theorem}
\begin{proof}\hfill
  \begin{enumerate}[(i)]
    \item Let $M$ be maximal, with $a + M \in R/M$, $a + M \neq 0 + M$. Then, $a\notin M$, so that the ideal $\left( a \right) + M$ strictly contains $M$. Therefore, $1 + M\in \left( a \right) + M$, meaning there is some $r + M$ such that $ \left( r + M \right)\left( a + M \right) = 1 + M $. Thus, an inverse exists.\newline

      Now, if $R/M$ is a field, and $M \subsetneq I \subseteq R$, then $I/M$ is an ideal of $R/M$, and since $I\supsetneq M$, we have $I/M\neq 0 + M$. Since $R/M$ is a field, its only ideals are either $ 0 + M $ and $R/M$, so $I/M = R/M$, meaning $I = R$.
    \item We have $P\subseteq R$ is prime if and only if $ab\in P$ implies $a\in P$ or $b\in P$. Yet, means that $ab + P = 0 + P$ if and only if $ a = 0 + P $ or $b = 0 + P$.
  \end{enumerate}
\end{proof}
\subsection{Chinese Remainder Theorem}%
\begin{definition}
  We say two ideals $I$ and $J$ are \textit{coprime} if $I + J = R$, or that there exist $x\in I$ and $y\in J$ such that $x + y = 1$.
\end{definition}
\begin{theorem}[Chinese Remainder Theorem]
  Let $I_1,\dots,I_n$ be pairwise coprime ideals of $R$. Then, for any $a_1,\dots,a_n\in R$, there exists $x\in R$ with $x\equiv a_i$ modulo $I_i$ for all $i$. In other words, there a solution to the system of congruences given by
  \begin{align*}
    x + I_1 &= a_1 + I_1\\
    x + I_2 &= a_2 + I_2\\
            &\vdots\\
    x + I_n &= a_n + I_n.
  \end{align*}
\end{theorem}
\begin{proof}
  It suffices to construct elements $y_1,\dots,y_n$ such that $y_i\equiv 1$ modulo $I_i$ and $0$ otherwise. Then, we will be able to set $x = \sum_{i} a_iy_i$ as our desired solution.\newline

  We construct $y_1$ as follows. From our assumption, $I_1 + I_j = R$ for all $j \geq 2$, so for each $j\geq 2$, there exists $u_j\in I_1$ and $v_j\in I_j$ such that $u_j + v_j = 1$. Taking the product, we find that
  \begin{align*}
    \prod_{j=2}^{n} \left( u_j + v_j \right) &= 1\\
                                             &= \underbrace{v_2 \cdots v_n}_{\eqcolon y_1} \underbrace{+ \cdots + u_2\cdots u_n}_{\eqcolon x_1}.
  \end{align*}
  We verify that $y_1$ does the job, which we can see by the fact that $y_1 \equiv 0$ modulo $I_j$ for $j\neq 1$, as $v_2\cdots v_j\in I_2\cdots I_j\subseteq I_j$ for each $j\geq 2$. Similarly, each summand in $x_1$ contains at least one $u_j$, so $x_1\equiv 0$ modulo $I_1$.\newline

  The rest of the $y_i$ follow analogously.
\end{proof}
We can restate the Chinese Remainder Theorem in a variety of ways.
\begin{theorem}[Chinese Remainder Theorem, Alternative Versions]
  Let $I_1,\dots,I_n$ be pairwise coprime ideals.
  \begin{enumerate}[(i)]
    \item There exists a surjective homomorphism 
      \begin{align*}
        \varphi\colon R&\rightarrow  R/I_1\times\cdots\times R/I_n\\
        r &\mapsto \left( r + I_1,\dots, r + I_n \right).
      \end{align*}
      This homomorphism induces an isomorphism
      \begin{align*}
        \overline{\varphi}\colon R/\left( I_1\cap\cdots\cap I_n \right) \rightarrow R/I_1\times\cdots\times R/I_n.
      \end{align*}
    \item If $I_1,\dots,I_n$ are pairwise coprime, then
      \begin{align*}
        R/I_1\cdots I_n &\cong R/I_1\times\cdots\times R/I_n
      \end{align*}
      are isomorphic.
  \end{enumerate}
\end{theorem}
\begin{example}
  We observe that if $R = \Z$, and $p_1,\dots,p_r$ are distinct primes with $\ell_1,\dots,\ell_r$ positive integers, then
  \begin{align*}
    \Z/p_1^{\ell_1}\cdots p_{r}^{\ell_r}\Z &\cong \Z/p_{1}^{\ell_1}\Z \times\cdots\times \Z/p_{r}^{\ell_r}\Z.
  \end{align*}
\end{example}
\begin{example}[Polynomial Interpolation]
  If we let
  \begin{align*}
    p_i(x) &= x-\alpha_i,
  \end{align*}
  where $\alpha_i\in \F$, we observe that there is a surjective evaluation homomorphism
  \begin{align*}
    \operatorname{ev}\colon \frac{\F\left[ x \right]}{\left( p_i(x) \right)} &\rightarrow \F,
  \end{align*}
  given by $f(x) \mapsto f\left( \alpha_i \right)$. In particular, if $\alpha_1,\dots,\alpha_r$ are distinct, then
  \begin{align*}
    \frac{\F\left[ x \right]}{\left( p_1(x),\dots,p_r(x) \right)} &\cong \F\times\cdots\times \F,
  \end{align*}
  so that, for all $\beta_1,\dots,\beta_r\in \F$, there is some $f(x)\in \F\left[ x \right]$ such that $f\left( \alpha_i \right) = \beta_i$ for $ i = 1,\dots,r $.
\end{example}
\subsection{Field of Fractions and Localization}%
Given a ring $R$, how can we find maximal ideals in $R$? More specifically, given a commutative ring $R$ with $1$, and prime ideal $ P\subseteq R $, we want to construct a new ring $R_{\mathfrak{p}}$ with unique maximal ideal $P$.\newline

Toward this end, we start by reviewing a useful construction known as the field of fractions.
\begin{definition}
  Let $R$ be an integral domain. We define the field $K = \operatorname{frac}\left( R \right)$ to be the unique field with an injection
  \begin{align*}
    \iota\colon R&\hookrightarrow K\\
    1_{R} &\mapsto 1_{K},
  \end{align*}
  satisfying the following universal property.\newline

  Given any embedding into a field, $\sigma\colon R\hookrightarrow L$, such that $1_{R}\mapsto 1_{L}$, there is a unique extension $ \widetilde{\sigma}\colon K\rightarrow L $ such that the following diagram commutes.
  \begin{center}
    % https://tikzcd.yichuanshen.de/#N4Igdg9gJgpgziAXAbVABwnAlgFyxMJZABgBpiBdUkANwEMAbAVxiRACUQBfU9TXfIRQAmUsKq1GLNgBluvEBmx4CRUZWr1mrRCADS3CTCgBzeEVAAzAE4QAtkjIgcEJKMna2AHS-4cdEGp-LAY2AAsICABreStbB0QnFyQARmoGOgAjGAYABX4VIRBrLBMwnECPaV0fbBM7AJ44+zcg10Q0kAzsvILBNgYYSwrNKR0QHwB3LFg8BlhgWtKGrkMuIA
\begin{tikzcd}
R \arrow[rr, "\iota", hook] \arrow[rrdd, "\sigma"'] &  & K \arrow[dd, "\widetilde{\sigma}"] \\
                                                    &  &                                    \\
                                                    &  & L                                 
\end{tikzcd}
  \end{center}
\end{definition}
In order to construct $K$, we let $S\subseteq R\times R$ be defined by
\begin{align*}
  S &= \set{\left( a,b \right) | b\neq 0}.
\end{align*}
We impose an equivalence relation on $S$ by saying $\left( a,b \right)\sim \left( c,d \right)$ if and only if $ ad - bc = 0 $. Clearly, this relation is reflexive and symmetric. To see that it is transitive, we let $\left( a,b \right)\sim \left( c,d \right)$, and $ \left( c,d \right)\sim \left( e,f \right) $, meaning $ad-bc=  0$ and $cf-de = 0$. Multiplying the first equation by $f$ and the second equation by $b$, then subtracting, we get $ adf - bde = 0 $, meaning $d\left( af-be  \right)= 0$. Since $R$ admits no zero divisors, this means that $ af-be = 0 $, so the relation is transitive.\newline

We write $ \left[ \left( a,b \right) \right] = \frac{a}{b} $ for $K$, with operations
\begin{align*}
  \frac{a}{b} + \frac{c}{d} &= \frac{ad + bc}{bd}\\
  \frac{a}{b}\cdot \frac{c}{d} &= \frac{ac}{bd}.
\end{align*}
These operations are well-defined and do satisfy the universal property. Verifying this is a pain, but it can be done.\newline

Now, we may extend this to all unital commutative rings, not just integral domains.
\begin{definition}
  Let $R$ be a unital commutative ring, and let $S\subseteq R$. We say $S$ is \textit{multiplicative} if
  \begin{itemize}
    \item $1\in S$;
    \item $0\notin S$;
    \item for any $x,y\in S$, $xy\in S$.
  \end{itemize}
\end{definition}
\begin{example}\hfill
  \begin{enumerate}[(i)]
    \item If $R$ is an integral domain, then $R\setminus \set{0}$ is multiplicative.
    \item If $z\in R$ is such that $z$ is not nilpotent, then $S = \set{z^{n} | n\geq 0}$ is multiplicative.
    \item If $P$ is a prime ideal, then $S = R\setminus P$ is multiplicative.
  \end{enumerate}
\end{example}
We will use (iii) to construct a ring with a unique maximal ideal. First, though, we construct a ring of fractions using multiplicative sets.
\begin{definition}
  Let $R$ be a unital commutative ring, and let $S\subseteq R$ be multiplicative. We construct a ring $S^{-1}R$ by taking an equivalence relation on $R\times S$ as follows:
  \begin{align*}
    \left( a,s \right)\sim \left( b,t \right) &\Leftrightarrow \exists s'\in S\text{ such that }s'\left( at-bs \right) = 0.
  \end{align*}
  We write
  \begin{align*}
    S^{-1}R &= \set{\left[ \left( a,s \right) \right] | a\in R,s\in S},
  \end{align*}
  and denote
  \begin{align*}
    \left[ \left( a,s \right) \right] &= \frac{a}{s}.
  \end{align*}
  This becomes a ring under the operations
  \begin{align*}
    \frac{a}{s} + \frac{b}{t} &= \frac{at + bs}{st}\\
    \frac{a}{s}\cdot \frac{b}{t} &= \frac{ab}{st}.
  \end{align*}
  We call $S^{-1}R$ the \textit{localization of $R$ with respect to $S$}.
\end{definition}
We can see some basic properties of the localization. 
\begin{proposition}
  Let $R$ be a unital commutative ring, $S\subseteq R$ multiplicative, and let $S^{-1}R$ be the corresponding localization.
  \begin{itemize}
    \item The additive identity in $S^{-1}R$ is $\frac{0}{1}$.
    \item The additive inverse of $ \frac{a}{s} $ in $S^{-1}R$ is $\frac{-a}{s}$.
    \item For all $a\in R$ and all $s,s'\in S$, we have $ \frac{as'}{ss'} = \frac{a}{s} $.
    \item Every element of the form $ \frac{s}{t} $ where both $s,t\in S$ is invertible, with corresponding inverse $ \frac{t}{s} $.
    \item The map $\iota_S\colon R\rightarrow S^{-1}R$ given by $ r\mapsto \frac{r}{1} $ is an injective ring homomorphism such that $\iota_S\left( S \right)\subseteq \left( S^{-1}R \right)^{\times}$, where $\left( S^{-1}R \right)^{\times}$ denotes the group of invertible elements in $S^{-1}R$.
  \end{itemize}
\end{proposition}
\subsection{Unique Factorization Domains}%
\begin{definition}
  A ring $R$ is called \textit{Noetherian} if, for any ascending chain of ideals $I_1\subseteq I_2\subseteq\cdots$, there is some index $N$ such that for all $m\geq N$, $I_m = I_N$.
\end{definition}
\begin{proposition}
  The following are equivalent:
  \begin{itemize}
    \item $R$ is Noetherian;
    \item every ideal in $R$ is finitely generated.
  \end{itemize}
\end{proposition}
\begin{proof}
  Let $R$ be Noetherian. Suppose toward contradiction that there exists $I$ that is not finitely generated. Then, $I$ is nonzero, so there is $\alpha_1\in I$ such that $I_1 = \left( \alpha_1 \right)$ is nonzero. Since $I$ is not finitely generated, $I\neq I_1$, so there is $\alpha_2\in I\setminus I_1$, so that $I_2 = \left( \alpha_1,\alpha_2 \right)$ is such that $I_1\subseteq I_2$. Inductively, we generate $I_n = \left( \alpha_1,\dots,\alpha_n \right)$ such that $I_{n-1}\subsetneq I_n$, implying that we have a strictly ascending chain of ideals, which is a contradiction.\newline

  Suppose every ideal in $R$ is finitely generated. Let $I_1\subseteq I_2\subseteq\cdots$ be an ascending chain of ideals, and set $I = \bigcup I_n$ be their union. By assumption, $I$ is finitely generated, so we have $I = \left( \alpha_1,\dots,\alpha_N \right)$ for some $\alpha_1,\dots,\alpha_N\in R$. Yet, since $I$ is the union of all these ideals, there is some $M$ such that $\alpha_1,\dots,\alpha_N\in I_M$, meaning the chain stabilizes.
\end{proof}
\begin{corollary}
  If $R$ is a principal ideal domain, then $R$ is Noetherian.
\end{corollary}
\begin{definition}
  Let $R$ be an integral domain.
  \begin{enumerate}[(i)]
    \item Two elements $a,b\in R$ are called \textit{associated} if $a = bu$ for some unit (invertible) element $u\in R$. Equivalently, $a$ and $b$ are associated if $\left( a \right) = \left( b \right)$
    \item An element $a\in R$ is called \textit{irreducible} if
      \begin{itemize}
        \item $a$ is not a unit element;
        \item whenever $a = bc$ for some $b,c\in R$, then one of $b$ or $c$ is a unit.
      \end{itemize}
    \item An element $a$ is called \textit{prime} if $a\neq 0$, $a\notin R^{\times}$, and $\left( a \right)$ is prime. Equivalently, $a$ is prime if, whenever $a | bc$, it follows that $a | b$ or $a | c$, where divisibility in $R$ is defined traditionally (i.e., there exists $z\in R$ such that $az = b$).
  \end{enumerate}
\end{definition}
\begin{note}
  Prime elements are irreducible, but not necessarily vice versa.
\end{note}
The question then arises: when are irreducibles prime?
\begin{definition}
  We say $a\in R$ with $a\neq 0$, $a\notin R^{\times}$ has a \textit{unique factorization} into irreducibles if
  \begin{enumerate}[(i)]
    \item we may write $a = u p_1\cdots p_r$, where $u$ is a unit and $p_1,\dots,p_r$ are irreducible;
    \item for any other such factorization
      \begin{align*}
        a &= u\prod_{i=1}^{r}p_i\\
          &= v\prod_{j=1}^{s} q_j,
      \end{align*}
      where $p_i,q_j$ are irreducible and $u,v$ are units, we have
      \begin{itemize}
        \item $r = s$;
        \item upon permutation of factors, $p_i$ and $q_i$ are associated.
      \end{itemize}
  \end{enumerate}
  We call $R$ a \textit{unique factorization domain} if, for any $a\in R$ with $a\neq 0$, $a\notin R^{\times}$, $a$ has unique factorization into irreducibles.
\end{definition}
\begin{proposition}
  If $R$ a Noetherian ring, then every $a\in R$ with $a\neq 0$ and $a\notin R^{\times}$ admits a factorization into irreducibles.
\end{proposition}
\begin{proof}
  First, we show that every such $a$ has an irreducible factor or divisor. If $a$ is itself irreducible, then we are done. Else, there are $b,c\in R$ with $a = bc$ and neither $a$ nor $b$ a unit. In particular, this means that $\left( a \right)\subsetneq \left( b \right)$. Inductively, if $b$ is not irreducible, then we may find $b_2,c_2$ such that $b = b_2 c_2$, meaning that $\left( b \right)\subsetneq \left( b_2 \right)$, and so on and so forth.\newline

  This gives a chain of ideals
  \begin{align*}
    \left( a \right)\subsetneq \left( b \right)\subsetneq \left( b_2 \right)\subsetneq \cdots
  \end{align*}
  that eventually stabilizes, meaning that there is some $b_N$ such that $b_N$ is irreducible.\newline

  Now, we may show that $a$ admits a factorization. If $a = bc$ with $b$ irreducible (as we showed previously), then if $c$ is not irreducible, we may take $c = b_1c_1$ and create this same chain of ideals
  \begin{align*}
    \left( c \right)\subsetneq \left( c_1 \right)\subsetneq \left( c_2 \right)\subsetneq\cdots
  \end{align*}
  using the Noetherian condition to end up at an irreducible or a unit.
\end{proof}
The main issue facing general Noetherian rings is that the uniqueness of the factorization may go awry.
\begin{example}
For instance, in the ring $R = \Z\left[ \sqrt{-5} \right]$, there is not unique factorization. For instance, we may write
\begin{align*}
  6 &= \left( 2 \right)\left( 3 \right)\\
    &= \left( 1 + \sqrt{-5} \right)\left( 1 + \sqrt{-5} \right),
\end{align*}
where we may see that all of these are irreducible as follows. Define a norm on $\Z\left[ \sqrt{-5} \right]\subseteq \C$ by $N\left( a + b\sqrt{-5} \right) = a^2 + 5b^2$, where this norm is multiplicative as it is inherited from $\C$.
\begin{lemma}
  If $N$ is a norm on the ring $R = \Z\left[ \sqrt{-D} \right]$, where $D$ is a square-free positive integer, then $u\in R$ is an invertible (or unit) element if and only if $N(u) = 1$.
\end{lemma}
\begin{proof}[Proof of Lemma]
  If $v\in R$ is such that $uv = 1$, then $N(uv) = N(u)N(v) = 1$, meaning that both $N(u)$ and $N(v)$ are $1$.\newline

  Meanwhile, if $N(u) = 1$, then $1 = u \overline{u}$, meaning that $ \overline{u} = u^{-1} $.
\end{proof}
We may show that $2$ is irreducible relatively quickly. Observe that if there were a factorization of $2 = ab$ into irreducibles, then $4 = N(a)N(b)$ would hold, with neither $N(a)$ nor $N(b)$ being equal to $1$. This would mean that $N(a) = 2$ for some $a = x + y\sqrt{-5}$, or that $x^2 + 5y^2 = 2$. Yet, reducing modulo $5$, this implies that $x^2 \equiv 2$ modulo $5$, yet the only squares in $\Z/5\Z$ are $1$ and $4$.
\end{example}
%\begin{proposition}
%  In a unique factorization domain, every irreducible element is prime.
%\end{proposition}
%\begin{proof}
%  Let $p$ be an irreducible element such that $p | ab$, where $a,b\in R$. Writing the factorizations into irreducibles
%  \begin{align*}
%    a &= p_1\cdots p_r\\
%    b &= q_1\cdots q_s,
%  \end{align*}
%  unique up to associates, we then get
%  \begin{align*}
%    pk &= ab\\
%       &= p_1\cdots p_r q_1\cdots q_s.
%  \end{align*}
%  Since the factorization of $ab$ is unique up to associates, it follows that $p$ must be associated to one of $p_1,\dots,p_r$ or $q_1,\dots,q_s$. If it's the former, then $p | a$, and if it's the latter, then $p | b$. Thus, $p$ is prime.
%\end{proof}
Given a factorization, there is a simple way to classify the uniqueness of the factorization.
\begin{proposition}
  Let $a\in R$ be such that $a\neq 0$ and $a\notin R^{\times}$. If $a$ admits a factorization
  \begin{align*}
    a &= u p_1\cdots p_r,
  \end{align*}
  with $p_1,\dots,p_n$ \textit{prime}, then this factorization is unique (up to associates).
\end{proposition}
\begin{proof}
  Suppose $a$ admits another factorization,
  \begin{align*}
    a &= vq_1\cdots q_s,
  \end{align*}
  where $q_1,\dots,q_s$ are irreducible and $v$ is a unit. Then, we have
  \begin{align*}
    up_1\cdots p_r &= vq_1\cdots q_s,
  \end{align*}
  meaning that $p_1$ divides $vq_1\cdots q_s$. Since $p_1$ is prime, $p_1 | q_j$ for some $j$, meaning that $q_j = v_1p_1$ for some $v_1\in R$. Yet, since $q_j$ is irreducible, it follows that $v_1$ is a unit. By permuting elements, we may say that $p_1$ and $q_1$ are associated, so we have
  \begin{align*}
    up_1\cdots p_r &= vv_1p_1q_2\cdots q_s.
  \end{align*}
  Now, since $R$ is a domain, it admits the cancellation property, so we may then write
  \begin{align*}
    up_2\cdots p_r &= vv_1q_2\cdots q_s.
  \end{align*}
  Proceeding in this fashion, we observe first that $r \leq s$, as else, we would have $p_i$ dividing a unit for $R$, which is not allowed. Thus, we find
  \begin{align*}
    u &= vv_1\cdots v_r q_{r+1}\cdots q_s.
  \end{align*}
  Similarly, this means there cannot be any more $q_j$, or else the $q_j$ would be a unit. Thus, these are the same factorizations (up to associates).
\end{proof}
\begin{theorem}
  If a domain $R$ is a principal ideal domain, then $R$ is a unique factorization domain.
\end{theorem}
\begin{proof}
  First, we show that if $a\in R$ is irreducible, then $a$ is prime.\newline

  Observe that $\left( a \right)$ is then contained in a maximal ideal $M$, where $M = \left( p \right)$ for some $p\in R$ with $p$ not a unit. Since $M$ is maximal, $M$ is prime, so that $p$ is prime, and $\left( a \right)\subseteq \left( p \right)$. Observe then that $a = pu$ for some $u\in R$; since $a$ is irreducible and $p$ is not a unit, it must be the case that $u$ is a unit. Thus, $\left( a \right) = \left( p \right)$, so that $a$ is prime.\newline

  Now, since $R$ is a principal ideal domain, every element in $R$ admits a factorization into irreducibles, and all irreducibles are prime. Therefore, the factorization is unique by the above lemma.
\end{proof}
\subsection{Euclidean Domains}%
\begin{definition}
  An integral domain $R$ is called a \textit{Euclidean Domain} if there exists $N\colon R\setminus \set{0}\rightarrow \Z_{\geq 0}$ such that for all $a,b\in R$, with $b\neq 0$, there exist $q,r\in R$ such that
  \begin{itemize}
    \item $a = qb + r$;
    \item either $r = 0$ or $N(r) < N(b)$.
  \end{itemize}
\end{definition}
\begin{example}\hfill
  \begin{itemize}
    \item Any field admits the vacuous norm, $N(k) = 0$ for all $k\in F\setminus \set{0}$.
    \item The ring $R = \Z$ is Euclidean with the norm $N(n) = \left\vert n \right\vert$. 
    \item The ring $R = \F[x]$, where $\F$ is a field, is Euclidean with norm $N\colon \F\left[ x \right]\setminus \set{0}\rightarrow \N$ given by $N(f) = \deg(f)$.
  \end{itemize}
\end{example}
\begin{theorem}
  If $R$ is Euclidean, then $R$ is a principal ideal domain.
\end{theorem}
\begin{proof}
  Let $I\subseteq R$ be an ideal. If $I = \set{0}$, then $I$ is principal and we are done.\newline

  Else, suppose $I\neq 0$. There exists $\alpha\in I$ with $\alpha\neq 0$, so that $N(\alpha)$ is well-defined. Let $b\in I$ be such that $N(b)$ is minimal for all possible elements of $I$.\newline

  We claim that $I = \left( b \right)$. Let $a\in I$ be arbitrary, and perform Euclidean division on $a$ by $b$, yielding
  \begin{align*}
    a &= qb + r,
  \end{align*}
  where $r = 0$ or $N(r) < N(b)$.\newline

  If $r\neq 0$, then $N(r) < N(b)$, but $r = a - bq\in I$, which would contradict minimality of $N(b)$, so that $r = 0$, and thus $a = bq\in \left( b \right)$.
\end{proof}
\begin{theorem}
  The Gaussian integers, $\Z\left[ i \right]$, are Euclidean with norm
  \begin{align*}
    N\left(a + bi\right) &= a^2 + b^2.
  \end{align*}
\end{theorem}
\begin{proof}
  Observe that $N$ is multiplicative. If we let $\alpha = a + bi$ and $\beta = c + di$ with $\alpha,\beta\neq 0$, we want to show that there exist $\gamma$ and $\delta$ such that $\alpha = \beta \gamma + \delta$ and $\delta = 0$ or $N(\delta) < N(\beta)$.\newline

  Consider $\frac{\alpha}{\beta}\in \C$, so that
  \begin{align*}
    \frac{\alpha}{\beta} &= \frac{\left( a + bi \right)\left( c - di \right)}{c^2 + d^2}\\
                         &= \frac{\left( a + bi \right)\left( c -di \right)}{N(\beta)}\\
                         &\eqcolon x + yi,
  \end{align*}
  so that $ \frac{\alpha}{\beta}\in \Q\left[ i \right] $.\newline

  Now, we can find $x_0,y_0\in \Z$ such that $\left\vert x-x_0 \right\vert\leq \frac{1}{2}$ and $\left\vert y-y_0 \right\vert\leq \frac{1}{2}$. Setting $\delta = x_0 + y_0 i$, we have that $\delta = \alpha - \beta\gamma\in \Z\left[ i \right]$. We claim that if $\delta \neq 0$, then $N(\delta) < N(\beta)$.\newline

  Observe that since $N$ is multiplicative, this condition is equivalent to $N\left( \frac{\delta}{\beta} \right) < 1$. We observe that
  \begin{align*}
    N\left( \frac{\delta}{\beta} \right) &= N\left( \frac{\alpha - \beta \gamma}{\beta} \right)\\
                                         &= N\left( \frac{\alpha}{\beta} - \gamma \right)\\
                                         &= \left( x-x_0 \right)^2 + \left( y-y_0 \right)^2\\
                                         &\leq \frac{1}{2}\\
                                         &< 1.
  \end{align*}
\end{proof}
\begin{remark}
  While the remainder in Euclidean division for $\Z$ and $\F\left[ x \right]$ \textit{is} unique, this is not the case for general Euclidean domains. For instance, if we want to divide $a = 1 + i$ by $b = 2$ in $\Z\left[ i \right]$ with our previously specified norm, we find that 
  \begin{align*}
    1 + i &= 2\cdot 0 + \left( 1 + i \right) \\
          &= 2\cdot 1 + \left( -1 + i \right),
  \end{align*}
  both of which satisfy the conditions for Euclidean division.
\end{remark}
Now, in any PID (really, any UFD), we can talk about a greatest common divisor. In a principal ideal domain, the GCD for $a,b\in R$ is given by the unique (up to associates) element $d$ such that
\begin{align*}
  \left( a,b \right) &= \left( d \right).
\end{align*}
Meanwhile, greatest common divisors in a UFD are slightly more complicated. If we have two elements $a,b\in R$ with prime factorizations
\begin{align*}
  a &= up_1^{v_1}p_2^{v_2}\cdots p_n^{v_n}\\
  b &= vp_1^{w_1}p_2^{w_2}\cdots p_n^{w_n},
\end{align*}
then the greatest common divisor is given by
\begin{align*}
  \gcd\left( a,b \right) &= \prod_{i=1}^{n} p_i^{\operatorname{min}\left( v_i,w_i \right)}.
\end{align*}
This is defined up to associates, similar to how the factorization of any element is defined up to associates.
\subsection{Unique Factorization in Polynomial Rings}%
Our goal is to prove that if $R$ is a UFD, then $R[x]$ is a UFD.\newline

We do this by first discussing irreducibility in $R[x]$, including a full characterization of irreducible elements.
\begin{definition}
  Assume $R$ is a unique factorization domain, and let $0\neq f(x)\in R[x]$. Writing
  \begin{align*}
    f(x) &= a_nx^n + a_{n-1}x^{n-1} + \cdots + a_1 x + a_0,
  \end{align*}
  we define the \textit{content} of $f$, written $\operatorname{c}(f)$, to be
  \begin{align*}
    \operatorname{c}(f) &= \operatorname{gcd}\left( a_0,a_1,\dots,a_n \right).
  \end{align*}
\end{definition}
\begin{proposition}[Gauss's Lemma]
  Let $R$ be a UFD, and let $f(x),g(x)\in R[x]$ be nonzero polynomials. Then,
  \begin{align*}
    \operatorname{c}\left( f g \right) &= \operatorname{c}\left( f \right)\operatorname{c}\left( g \right).
  \end{align*}
\end{proposition}
\begin{proof}
  For any nonzero polynomial $h\in R[x]$, we may write
  \begin{align*}
    h(x) &= \operatorname{c}\left( h \right) z(x),
  \end{align*}
  where $\operatorname{c}\left( z \right) = 1$, simply by factoring. Thus, writing
  \begin{align*}
    f(x) &= \operatorname{c}\left( f \right) u(x)\\
    g(x) &= \operatorname{c}\left( g \right) v(x),
  \end{align*}
  where $\operatorname{c}\left( u \right) = \operatorname{c}\left( v \right) = 1$, hence
  \begin{align*}
    \operatorname{c}\left( fg \right) &= \operatorname{c}\left( \operatorname{c}\left( f \right)\operatorname{c}\left( g \right)u v \right)\\
                                      &= \operatorname{c}\left( f \right)\operatorname{c}\left( g \right)\operatorname{c}\left( uv \right).
  \end{align*}
  We want to show that $\operatorname{c}\left( u(x)v(x) \right) = 1$ (up to associates).\newline

  Suppose not. Since $ \operatorname{c}\left( uv \right) $ is nonzero and (assumed to be) not a unit, we may find a prime $p$ such that $ p | \operatorname{c}\left( u v \right) $. That is, we may find $p$ such that $p$ divides all coefficients of $u(x)v(x)$.\newline

  Consider now the reduction homomorphism
  \begin{align*}
    \pi\colon R[x]\rightarrow \left( R/\left( p \right) \right)[x],
  \end{align*}
  where we reduce all coefficients modulo $\left( p \right)$. Since $p$ is prime, $ \left( p \right) $ is prime, so that $R/\left( p \right)$ is an integral domain, meaning that $\left( R/\left( p \right) \right)[x]$ is an integral domain.\newline

  Since $ \operatorname{c}\left( u \right) = \operatorname{c}\left( v \right) = 1 $, it follows that $\pi\left( u(x) \right) \neq 0$ and $\pi\left( v(x) \right) \neq 0$, as at least one coefficient in $u(x)$ or $v(x)$ is not divisible by $p$. Thus, in $\left( R/\left( p \right) \right)[x]$ is a domain, it follows that $\pi\left( u(x) \right)\pi\left( v(x) \right)\neq 0$. Yet, since $\pi$ is a homomorphism, it follows that $0 = \pi\left( u(x)v(x) \right) = \pi\left( u(x) \right)\pi\left( v(x) \right)$, since we assumed that $p$ divides all the coefficients of $u(x)v(x)$.
\end{proof}
\begin{corollary}[Gauss's Lemma, Redux]
  Let $R$ be a UFD, and let $F = \operatorname{frac}\left( R \right)$. Let $f(x)\in R[x]$, and assume $f(x)$ is reducible in $F[x]$. Then, $f(x)$ is reducible in $R[x]$.
\end{corollary}
\begin{proof}
  Let $f(x)$ be reducible in $F[x]$, so that $f(x) = g(x)h(x)$, where $g(x)$ and $h(x)$ are nonconstant polynomials in $F[x]$.\newline

  By factoring, we have
  \begin{align*}
    g(x) &= \frac{a}{b}u(x)\\
    h(x) &= \frac{c}{d}v(x),
  \end{align*}
  where $a,b,c,d\in R\setminus \set{0}$, $u(x),v(x)\in R[x]$, and $ \operatorname{c}\left( u \right) = \operatorname{c}\left( v \right) = 1 $.\newline

  Substituting this information into the expression for $f(x)$, we have
  \begin{align*}
    f(x) &= \frac{ac}{bd}u(x)v(x)\\
    bd f(x) &= ac u(x)v(x),
  \end{align*}
  so that
  \begin{align*}
    bd \operatorname{c}\left( f \right)&= ac \operatorname{c}\left( u \right)\operatorname{c}\left( v \right).
  \end{align*}
  meaning
  \begin{align*}
    bd \operatorname{c}\left( f \right) &= ac.
  \end{align*}
  In particular, this means that $\frac{ac}{bd}$ is a valid representative for $\operatorname{c}\left( f \right)$, so that $ \frac{ac}{bd}\in R $. Therefore,
  \begin{align*}
    f(x) &= \left( \frac{ac}{bd}u(x) \right)v(x),
  \end{align*}
  both nonconstant and in $R[x]$, meaning $f(x)$ has a nontrivial factorization in $R[x]$, and thus $f$ is reducible.
\end{proof}
\begin{corollary}[Classification of Irreducibles]
  Let $R$ be a UFD, let $F = \operatorname{frac}\left( R \right)$, and let $f(x)\neq 0\in R[x]$.
  \begin{enumerate}[(i)]
    \item If $f(x)$ is constant, then $f$ is irreducible in $R[x]$ if and only if $f(x)$ is irreducible in $R$.
    \item If $f(x)$ is not constant, then $f$ is irreducible in $R[x]$ if and only if $\operatorname{c}\left( f \right) = 1$ and $f(x)$ is irreducible in $F[x]$.
  \end{enumerate}
\end{corollary}
\begin{proof}\hfill
  \begin{enumerate}[(i)]
    \item Observe that $R[x]$ and $R$ have the same units (since $R$ is an integral domain, and so admits no nilpotent elements), meaning that the product of two nonzero polynomials is a constant if and only if the polynomials themselves are constant.
    \item Let $f$ be nonconstant. If $f$ is irreducible in $R[x]$, then we may write
      \begin{align*}
        f(x) &= \operatorname{c}\left( f \right)u(x),
      \end{align*}
      where $u(x)$ is nonconstant and has $\operatorname{c}(u) = 1$. Yet, since $f$ is irreducible, it also follows that $\operatorname{c}(f) = 1$. Additionally, $f$ is irreducible in $F[x]$ by the contrapositive of Gauss's Lemma.\newline

      If $f$ is irreducible in $F[x]$, and has content $1$, then for any factorization
      \begin{align*}
        f(x) &= g(x)h(x),
      \end{align*}
      where $g(x),h(x)\in F[x]$, either $g$ or $h$ must be a constant. Now, since $f$ is contained in $R[x]$, we may take a common denominator to yield
      \begin{align*}
        f(x) &= a u(x),
      \end{align*}
      where $u(x)$ is nonconstant and has content $1$, with $a\in R$. Since $f$ has content $1$, it follows that $a$ is a unit element, meaning that any factorization of $f$ must contain a unit, so that $f$ is irreducible in $R[x]$.
  \end{enumerate}
\end{proof}
\begin{theorem}
  If $R$ is a UFD, then $R[x]$ is a UFD.
\end{theorem}
\begin{proof}
  Let $F = \operatorname{frac}(R)$, and let $f(x)\in R[x]$ be a nonzero, non-unit element. If $f(x)\in R$, then $f$ is a product of irreducibles in $R$ by part (i) of the classification, meaning the product is automatically unique up to permutation and associates as $R$ is a UFD.\newline

  Now, if $f$ is nonconstant, then $f(x)\in F[x]$ is nonzero and non-unit, as the units in $F[x]$ are the elements of $F$. Since $F[x]$ is a principal ideal domain (as $F[x]$ is a Euclidean domain, following from the division algorithm), $F[x]$ is a UFD, so we may write
  \begin{align*}
    f(x) &= \prod_{i=1}^{n}g_i(x),
  \end{align*}
  where the $g_i(x)$ are irreducible in $F[x]$. Writing
  \begin{align*}
    g_i(x) &= \frac{a_i}{b_i} u_i(x),
  \end{align*}
  where the $u_i(x)\in R[x]$ with $\operatorname{c}\left( u_i \right) = 1$ for each $i$, we have
  \begin{align*}
    \prod_{i=1}^{n}\frac{a_i}{b_i} &\in R,
  \end{align*}
  as $f(x)\in R[x]$, so we may write
  \begin{align*}
    f(x) &= \prod_{i=1}^{n} \frac{a_i}{b_i} \prod_{i=1}^{n}u_i(x).
  \end{align*}
  Each of the $u_i(x)$ are irreducible in $R[x]$ by the classification, and the product $\prod_{i=1}^{n}\frac{a_i}{b_i}\in R$ is either a unit or a product of irreducibles. This gives the existence of such a factorization for $f$.\newline

  To see uniqueness, if
  \begin{align*}
    f(x) &= \left( \prod_{i=1}^{k}a_i \right)\left( \prod_{i=1}^{m}p_i(x) \right)\\
         &= \left( \prod_{j=1}^{\ell}b_j \right)\left( \prod_{j=1}^{m}q_j(x) \right)
  \end{align*}
  are factorizations where $a_i,b_j$ are irreducible in $R$, and $p_i,q_j$ are nonconstant and irreducible with content $1$, then we may take the content of both sides, yielding
  \begin{align*}
    \operatorname{c}\left( \left( \prod_{i=1}^{k}a_i \right)\left( \prod_{i=1}^{k}p_i(x) \right) \right) &= \prod_{i=1}^{k}a_i\\
    \operatorname{c}\left( \left( \prod_{j=1}^{\ell}b_j \right)\left( \prod_{j=1}^{\ell}q_j(x) \right) \right) &= \prod_{j=1}^{\ell}b_j.
  \end{align*}
  Since contents are only well-defined up to associates, the most we can say is that
  \begin{align*}
    \prod_{i=1}^{k}a_i &= u\prod_{j=1}^{\ell}b_j,
  \end{align*}
  where $u\in R^{\times}$. Since there is at least one $q_j$, we may replace $q_1$ by $uq_1$, then divide, so that we find
  \begin{align*}
    \prod_{i=1}^{k}a_i &= \prod_{j=1}^{\ell}b_j.
  \end{align*}
  Since both of these are products of irreducibles in $R$, it follows that $k = \ell$ and, after permutation of factors, $b_i = u_i a_i$ for some $u_i\in R^{\times}$. Additionally, we also have the equality
  \begin{align*}
    \prod_{i=1}^{m}p_i &= \prod_{j=1}^{n}q_j.
  \end{align*}
  Since all of these factors are irreducible in $F[x]$, and $F[x]$ is a PID, we find that $ n= m $ and, upon permutation of factors, we have $q_i(x) = \gamma_ip_i(x)$ for some $\gamma_i\in F\setminus \set{0}$. Write
  \begin{align*}
    \gamma_i &= \frac{c_i}{d_i},
  \end{align*}
  where $c_i,d_i\in R\setminus \set{0}$, so that
  \begin{align*}
    d_i q_i(x) &= c_ip_i(x).
  \end{align*}
  Taking the content of both sides, we then get that $v_id_i = c_i$ for some $v_i\in R^{\times}$, so that $\gamma_i = v_i\in R^{\times}$, meaning that $p_i$ and $q_i$ are associates in $R[x]$.
\end{proof}
Unique factorization in polynomial rings having the rigidity laid out in the classification theorem makes for very useful criteria to understand irreducibility.
\begin{theorem}[Eisenstein's Criterion]
  Let $R$ be a UFD, and let $p\in R$ be a prime element. If we write $f(x)\in R[x]$ as
  \begin{align*}
    f(x) &= \sum_{i=0}^{n}a_ix^{i},
  \end{align*}
  then if
  \begin{itemize}
    \item $a_0\neq 0$;
    \item $p \nmid a_n$;
    \item $p | a_i$ for $0 \leq i \leq n-1$;
    \item and $p^2 \nmid a_0$,
  \end{itemize}
  then $f(x)$ is irreducible in $F[x]$. If, in addition, $\operatorname{c}(f) = 1$, then $f$ is irreducible in $R[x]$.
\end{theorem}
\begin{remark}
  This is the more general formulation of the case when $R = \Z$ and $f$ is monic that we see in undergrad abstract algebra.
\end{remark}
\begin{proof}
  Suppose toward contradiction that $f$ is reducible in $F[x]$, where we may write
  \begin{align*}
    f(x) &= g(x)h(x)
  \end{align*}
  with $g(x),h(x)\in R[x]$ nonconstant as in the proof of Gauss's Lemma. The reduction map $\pi\colon R[x]\rightarrow \left( R/\left( p \right) \right)[x]$ is a homomorphism, so that
  \begin{align*}
    \overline{f}(x) &= \overline{g}(x) \overline{h}(x).
  \end{align*}
  Thus, by our assumptions, we have
  \begin{align*}
    \overline{f}(x) &= \overline{a_n}x^{n},
  \end{align*}
  with $ \overline{a_n}\neq \overline{0} $. Since the degree of $ \overline{f} $ remains the same upon reduction, it follows that $ \overline{g} $ and $ \overline{h} $ have the same degrees as they had originally.\newline

  Observe that in a domain, the product of the highest-degree terms is the highest-degree term of the product, and similarly for the lowest-degree terms. Therefore, we must have $ \overline{g}(x) $ and $ \overline{h}(x) $ are monomials, as their product is a monomial. Writing
  \begin{align*}
  \overline{g}(x) &= \beta x^{k}\\
  \overline{h}(x) &= \gamma x^{\ell},
  \end{align*}
  with $\gamma,\beta\in R$ and $k = \deg(g)$, $\ell = \deg(h)$, we then get
  \begin{align*}
    g(x) &= bx^{k} + p u(x)\\
    h(x) &= cx^{\ell} + p v(x),
  \end{align*}
  where $k,\ell > 0$ and $u(x),v(x)\in R[x]$. Then,
  \begin{align*}
    f(x) &= \left( bx^{k} + pu(x) \right)\left( cx^{\ell} + pv(x) \right),
  \end{align*}
  whence the constant term of this product is divisible by $p^2$.
\end{proof}
\section{Modules}%
For this section, a ring $R$ may not be commutative nor unital.
\begin{definition}
  Let $R$ be a ring. A \textit{left $R$-module} is a set $M$ with operations
  \begin{align*}
    +\colon M\times M&\rightarrow M\\
    \left( m,n \right) &\mapsto m+n\\
    .\colon R\times M&\rightarrow M\\
    \left( r,m \right) &\mapsto r.m,
  \end{align*}
  satisfying the following axioms:
  \begin{description}[font=\normalfont]
    \item[(M0)] $\left( M,+ \right)$ is an abelian group;
    \item[(M1)] $\left( r+s \right).m = r.m + s.m$ for all $r,s\in R$ and $m\in M$;
    \item[(M2)] $\left( rs \right).m = r.\left( s.m \right)$ for all $r,s\in R$ and $m\in M$;
    \item[(M3)] $r.\left( m+n \right) = r.m + r.n$ for all $r\in R$ and $m,n\in M$;
    \item[(M4)] if $R$ is unital, then $1.m = m$ for all $m\in M$.
  \end{description}
\end{definition}
\end{document}
