\documentclass[10pt]{mypackage}

% sans serif font:
%\usepackage{cmbright}
%\usepackage{sfmath}
%\usepackage{bbold} %better blackboard bold

%\usepackage{homework}
\usepackage{notes}
\usepackage{newpxtext,eulerpx,eucal}
\renewcommand*{\mathbb}[1]{\varmathbb{#1}}

\fancyhf{}
\rhead{Avinash Iyer}
\lhead{Algebra I: Notes and Review}

\setcounter{secnumdepth}{0}

\begin{document}
\RaggedRight
\section{Principal Ideal Domains}%
We always assume here that $R$ is commutative and unital.
\begin{definition}
  If $a_1,\dots,a_n\in R$, then the \textit{ideal generated by} $a_1,\dots,a_n$ is given by
  \begin{align*}
    \left( a_1,\dots,a_n \right) \coloneq \bigcap\set{I | a_1,\dots,a_n\in I,I\text{ is an ideal in }R}.
  \end{align*}
\end{definition}
\begin{definition}
  If $I$ and $J$ are ideals in $R$, then $IJ$ is given by
  \begin{align*}
    IJ &= \set{\sum_{i=1}^{n}x_iy_i | x_i\in I, y_i\in J,n\in \N}.
  \end{align*}
\end{definition}
\begin{definition}
  Let $M$ be an ideal in $R$.
  \begin{enumerate}[(i)]
    \item We say $M$ is prime if $M\neq R$ and, for any $ab\in M$, we have either $a\in M$ or $b\in M$.
    \item We say $M$ is maximal if $M\neq R$ and if $M\subseteq I\subseteq R$ where $I$ is an ideal, then either $I = M$ or $I = R$.
  \end{enumerate}
\end{definition}
\begin{theorem}
  Let $M$ be an ideal in $R$.
  \begin{enumerate}[(i)]
    \item $M$ is prime if and only if $R/M$ is an integral domain.
    \item $M$ is maximal if and only if $R/M$ is a field.
  \end{enumerate}
\end{theorem}
\end{document}
