\documentclass[10pt]{mypackage}

% sans serif font:
%\usepackage{cmbright}
%\usepackage{sfmath}
%\usepackage{bbold} %better blackboard bold

%serif font + different blackboard bold for serif font
\usepackage{homework}
\usepackage{newpxtext,eulerpx}
\renewcommand*{\mathbb}[1]{\varmathbb{#1}}

\fancyhf{}
\rhead{Avinash Iyer}
\lhead{Review 1}

\setcounter{secnumdepth}{0}

\begin{document}
\RaggedRight
\begin{problem}[Problem 1]
  Let $T\colon V\rightarrow W$ be a linear transformation between $\F$-vector spaces. Show that $T$ is injective if and only if $T$ maps $\F$-linearly independent subsets of $V$ to $\F$-linearly independent subsets of $W$.
\end{problem}
\begin{solution}
  Let $T$ be injective. We claim that if $\set{v_1,\dots,v_n}$ is linearly independent in $V$, then $\set{Tv_1,\dots,Tv_n}$ is linearly independent in $W$. We see that if
  \begin{align*}
    \sum_{j=1}^{n}a_jTv_j &= 0_{W},
  \end{align*}
  then
  \begin{align*}
    T\left( \sum_{j=1}^{n}a_jv_j \right) &= 0_W,
  \end{align*}
  meaning that
  \begin{align*}
    \sum_{j=1}^{n}a_jv_j &\in \ker\left( T \right).
  \end{align*}
  Now, since $T$ is injective, $\ker\left( T \right) = \set{0_V}$, meaning that $\sum_{j=1}^{n}a_jv_j = 0_V$. Yet, since $\set{v_1,\dots,v_n}$ is linearly independent, this means $a_j = 0$ for each $j$, so $\set{Tv_1,\dots,Tv_n}$ is linearly independent in $W$.\newline

  Now, let $T$ map linearly independent subsets of $V$ to linearly independent subsets of $W$. 
\end{solution}
\begin{problem}[Problem 2]
  Let $P_{n+1}\left( \R \right)$ be the space of polynomials with real coefficients of degree $\leq n+1$. Prove that for any $n$ points $a_1,\dots,a_n\in \R$, there exists a nonzero polynomial $f\in P_{n+1}\left( \R \right)$ such that $f\left( a_j \right) = 0$ for each $j$, and $\sum_{j=1}^{n}f'\left(a_j\right) = 0$.
\end{problem}
\begin{solution}
  Based on the first condition, we see that the product $\prod_{j=1}^{n}\left( x-a_j \right)$ must divide the polynomial $f$, and since $f$ has degree at most $n+1$, we must have $f(x) = \left( Ax+B \right)\prod_{j=1}^{n} \left( x-a_j \right)$ for some $a,b\in \R$. Writing $f'(x)$, we see that
  \begin{align*}
    f'(x) &= A\prod_{j=1}^{n}\left( x-a_j \right) + \left( Ax+B \right)\sum_{i=1}^{n} \prod_{j\neq i}\left( x-a_j \right),
  \end{align*}
\end{solution}
\begin{problem}[Problem 7]\hfill
  \begin{enumerate}[(a)]
    \item Let $A\in \Mat_{n}\left( \C \right)$ be a matrix such that $A^2 = I_{n}$. Show that $A$ is diagonalizable.
    \item Give an example of of $A\in \Mat_{2}\left( \C \right)$ satisfying $A^2 = \mathbf{0}_{2}$ (the zero matrix) which is not diagonalizable.
  \end{enumerate}
\end{problem}
\begin{solution}\hfill
  \begin{enumerate}[(a)]
    \item Since $A^2- I_n = \mathbf{0}_{n}$, we see that the minimal polynomial of $A$ is $m_A(t) = t^2 - 1$, which splits over $\C$ to yield $m_A(t) = \left( t-1 \right)\left( t+1 \right)$. In particular, since the minimal polynomial splits into a product of distinct linear factors, $A$ is diagonalizable.
    \item The matrix
      \begin{align*}
        A &= \begin{pmatrix}0 & 1 \\ 0 & 0\end{pmatrix}
      \end{align*}
      satisfies $A^2 = \mathbf{0}_{2}$, but since $A \neq \mathbf{0}_{2}$, we see that $m_{A}(t) = t^2$. Since $m_A(t)$ does not split into distinct linear factors over $\C$, we see that $A$ is necessarily not diagonalizable.
  \end{enumerate}
\end{solution}
\begin{problem}[Problem 8]
  Let $A\in \Mat_{n}\left( \C \right)$ be a matrix such that $A^2$ has $n$ distinct eigenvalues. Show that $A$ is diagonalizable.
\end{problem}
\end{document}
