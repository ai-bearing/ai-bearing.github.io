\documentclass[10pt]{mypackage}

% sans serif font:
%\usepackage{cmbright}
%\usepackage{sfmath}
%\usepackage{bbold} %better blackboard bold

\usepackage{homework}
%\usepackage{notes}
\usepackage{newpxtext,eulerpx,eucal}
\renewcommand*{\mathbb}[1]{\varmathbb{#1}}

\fancyhf{}
\rhead{Avinash Iyer}
\lhead{Algebra I: Assignment 3}

\setcounter{secnumdepth}{0}

\begin{document}
\RaggedRight
\begin{problem}[Problem 1]
  Let $a_1,\dots,a_n\in \R$. Suppose that for each $i\in \set{1,\dots,n}$, we are given $m_i \geq 0$ and $m+1$ numbers $b_{i0},\dots,b_{im_i}\in \R$. Use the Chinese Remainder Theorem to show that there exists a polynomial $f(x)\in \R[x]$ such that
  \begin{align*}
    f\left( a_i \right) &= b_{i0}\\
    f'\left( a_i \right) &= b_{i1}\\
                         &\vdots\\
    f^{\left(m_i\right)} &= b_{im_i}.
  \end{align*}
\end{problem}
\begin{solution}
  We observe that if we take
  \begin{align*}
    f(x) &= q_{01}(x)\left( x-a_1 \right) + b_{10},
  \end{align*}
  then
  \begin{align*}
    f'(x) &= q_{01}(x) + q_{01}'(x)\left( x-a_1 \right),
  \end{align*}
  so that
  \begin{align*}
    f'\left( a_1 \right) &= q_{01}\left( a_1 \right)
    \intertext{and}
    f'(x) &= q_{11}\left( x \right)\left( x-a_1 \right) + b_{11},
    \intertext{meaning}
    f(x) &= \left( q_{11}\left( x \right)\left( x-a_1 \right) + b_{11} \right)\left( x-a_1 \right) + b_{10}.
  \end{align*}
  Inductively, we thus get the system of congruences
  \begin{align*}
    f(x) &\equiv b_{10} + b_{11}\left( x-a_1 \right) + \cdots + b_{1m_1}\left( x-a_1 \right)^{m_1 - 1}\text{ mod } \left( x-a_1 \right)^{m_1}\\
         &\equiv b_{20} + b_{21}\left( x-a_2 \right) + \cdots + b_{2m_2}\left( x-a_2 \right)^{m_2 - 1}\text{ mod } \left( x-a_2 \right)^{m_2}\\
         &\vdots\\
         &\equiv b_{n0} + b_{n1}\left( x-a_n \right) + \cdots + b_{nm_n}\left( x-a_n \right)^{m_n-1}\text{ mod } \left( x-a_n \right)^{m_n}.
  \end{align*}
  Since the family of ideals $\set{\left( \left( x-a_1 \right)^{m_1} \right),\dots, \left( \left( x-a_n \right)^{m_n} \right)}$ are pairwise coprime, the Chinese Remainder Theorem implies that some $f(x)\in \R[x]$ satisfies this system of congruences.
\end{solution}
\begin{problem}[Problem 4]\hfill
  \begin{enumerate}[(a)]
    \item Let $R,S$ be commutative rings with $1$, and let $f\colon R\rightarrow S$ be a ring homomorphism such that $f\left( 1_R \right) = 1_S$. Show that for any prime ideal $ P\subseteq S $, the preimage $f^{-1}\left( P \right)$ is a prime ideal of $R$.
    \item Give an example of a ring homomorphism $f\colon R\rightarrow S$ with $f\left( 1_{R} \right) = 1_S$ and a maximal ideal $M\subseteq S$ such that $f^{-1}\left( M \right)$ is not a maximal ideal of $R$.
  \end{enumerate}
\end{problem}
\begin{solution}\hfill
  \begin{enumerate}[(a)]
    \item 
    \item Let $R = \Z$ and $S = \Q$, with $f\colon \Z\hookrightarrow \Q$ being the natural inclusion. Since $\Q$ is a field, the only maximal ideal of $\Q$ is $\set{0}$, but $\set{0} = f^{-1}\left( \set{0} \right)$ is not maximal in $\Z$ since there are other proper ideals in $\Z$.
  \end{enumerate}
\end{solution}
\end{document}
