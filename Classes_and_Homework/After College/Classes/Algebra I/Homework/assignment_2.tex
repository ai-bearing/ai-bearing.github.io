\documentclass[10pt]{mypackage}

% sans serif font:
%\usepackage{cmbright}
%\usepackage{sfmath}
%\usepackage{bbold} %better blackboard bold

\usepackage{homework}
%\usepackage{notes}
\usepackage{newpxtext,eulerpx,eucal}
\renewcommand*{\mathbb}[1]{\varmathbb{#1}}

\fancyhf{}
\rhead{Avinash Iyer}
\lhead{Algebra I: Assignment 2}

\setcounter{secnumdepth}{0}

\begin{document}
\RaggedRight
\begin{problem}[Problem 1]
  For two ideals $I,J\subseteq R$, prove the following.
  \begin{enumerate}[(a)]
    \item The intersection $I\cap J$ is an ideal of $R$.
    \item The product $IJ\subseteq I\cap J$.
    \item Let $f\colon R\rightarrow R/(IJ)$ be the canonical homomorphism. Then, for any $x\in I\cap J$, the image $f(x)$ is nilpotent.
    \item If $I + J = R$, then $IJ = I\cap J$.
  \end{enumerate}
\end{problem}
\begin{solution}\hfill
  \begin{enumerate}[(a)]
    \item If $x,y\in I\cap J$, then $x - y\in I\cap J$ since $x-y\in I$ and $x-y\in J$. Furthermore, if $r\in R$, then $rx\in I$ and $rx\in J$, so $rx\in I\cap J$, so $I\cap J$ is an ideal.
    \item We observe that for any $q\in IJ$, we may express
      \begin{align*}
        q &= \sum_{k=1}^{n}x_ky_k,
      \end{align*}
      where $x_k\in I$ and $y_k\in J$. In particular, each $x_ky_k\in I\cap J$, so $q\in I\cap J$, meaning $IJ\subseteq I\cap J$.
    \item Let $x\in I\cap J$. Then, following from the well-definedness of operations in the quotient ring, we see that $\left( x+IJ \right)^{n} = x^{n} + IJ$. In particular, if $n = 2$, then $x^2$ is a linear combination of an element of $I$ multiplied by an element of $J$, so $x^2\in IJ$, meaning that $\left( x + IJ \right)^{2} = x^2 + IJ = IJ = 0 + IJ$, meaning that $x$ is nilpotent.
    \item We will show that if $q\in I\cap J$, then $q$ can be written as a linear combination of elements of $I$ multiplied by elements of $J$. In particular, we start by letting $i\in I$ and $j\in J$ be such that $i + j = 1$. Then, $q\left( i+j \right) = q$, meaning that $qi + qj = q$, and since $q\in I\cap J$, we have expressed $q$ as a linear combination of elements of $I$ multiplied by elements of $J$. Thus, $I\cap J \subseteq IJ$, meaning $IJ = I\cap J$.
  \end{enumerate}
\end{solution}
\end{document}
