\documentclass[10pt]{mypackage}

% sans serif font:
%\usepackage{cmbright}
%\usepackage{sfmath}
%\usepackage{bbold} %better blackboard bold

\usepackage{homework}
%\usepackage{notes}
\usepackage{newpxtext,eulerpx,eucal}
\renewcommand*{\mathbb}[1]{\varmathbb{#1}}

\fancyhf{}
\rhead{Avinash Iyer}
\lhead{Algebra I: Assignment 2}

\setcounter{secnumdepth}{0}

\begin{document}
\RaggedRight
\begin{problem}[Problem 1]
  For two ideals $I,J\subseteq R$, prove the following.
  \begin{enumerate}[(a)]
    \item The intersection $I\cap J$ is an ideal of $R$.
    \item The product $IJ\subseteq I\cap J$.
    \item Let $f\colon R\rightarrow R/(IJ)$ be the canonical homomorphism. Then, for any $x\in I\cap J$, $f(x)$ is nilpotent.
    \item If $I + J = R$, then $IJ = I\cap J$.
  \end{enumerate}
\end{problem}
\begin{solution}\hfill
  \begin{enumerate}[(a)]
    \item If $x,y\in I\cap J$, then $x - y\in I\cap J$ since $x-y\in I$ and $x-y\in J$. Furthermore, if $r\in R$, then $rx\in I$ and $rx\in J$, so $rx\in I\cap J$, so $I\cap J$ is an ideal.
    \item We observe that for any $q\in IJ$, we may express
      \begin{align*}
        q &= \sum_{k=1}^{n}x_ky_k,
      \end{align*}
      where $x_k\in I$ and $y_k\in J$. In particular, each $x_ky_k\in I\cap J$, so $q\in I\cap J$, meaning $IJ\subseteq I\cap J$.
    \item Let $x\in I\cap J$. Then, following from the well-definedness of operations in the quotient ring, we see that $\left( x+IJ \right)^{n} = x^{n} + IJ$. In particular, if $n = 2$, then $x^2$ is a linear combination of an element of $I$ multiplied by an element of $J$, so $x^2\in IJ$, meaning that $\left( x + IJ \right)^{2} = x^2 + IJ = IJ = 0 + IJ$, meaning that $x$ is nilpotent.
    \item Let $q\in I\cap J$, and find $i\in I$, $j\in J$ such that $i + j = 1$. Then, $q = q(1) = qi + qj$, so $q\in IJ$, meaning $IJ = I\cap J$.
  \end{enumerate}
\end{solution}
\begin{problem}[Problem 3]
  Let $R = \Z\left[ i \right]$ be the ring of Gaussian integers.
  \begin{enumerate}[(a)]
    \item Show that every nonzero ideal $I\subseteq R$ contains a nonzero integer.
    \item Identify the quotient $R/I$ where $I = \left( 2+i \right)$ is the principal ideal generated by $2 + i$.
  \end{enumerate}
\end{problem}
\begin{solution}\hfill
  \begin{enumerate}[(a)]
    \item Let $I\subseteq R$ be a nonzero ideal, and let $a + ib\in I$ with $a,b\in \Z\setminus\set{0}$. Since multiplication by any element of $R$ yields another element in $I$, we see that
      \begin{align*}
        \left( a + ib \right)\left( a-ib \right) &= a^2 + b^2\\
                                                 &\in R,
      \end{align*}
      and since $a,b\neq 0$, so too is $a^2 + b^2$, so any nonzero ideal of $R$ contains a nonzero integer.
    \item Consider the map $\varphi\colon \Z\rightarrow R/I$ given by $z\mapsto z + I$. Since this is a composition of the inclusion map $\Z\hookrightarrow \Z\left[ i \right]$ and the projection map $\pi\colon \Z\left[ i \right]\rightarrow \Z\left[ i \right]/\left( 2+i \right)$, this is a ring homomorphism. We will show that this ring homomorphism is surjective.\newline

      Let $\left( a + bi \right) + I\in R/I$. We will show that there is some $k\in \Z$ such that $k - \left( a + bi \right) \in \left( 2 + i \right)$. For this purpose, let
      \begin{align*}
        \left( x + yi \right)\left( 2+i \right) &= \left( a-k \right) + bi,
      \end{align*}
      so that
      \begin{align*}
        2x - y &= \left( a-k \right)\\
        2y  + x &= b.
      \end{align*}
      We thus get that
      \begin{align*}
        5x &= 2a + b - 2k\\
        5y &= 2b - a + k.
      \end{align*}
      Reducing modulo $5$, we thus have that
      \begin{align*}
        0 &\equiv 2a + b - 2k\\
          &\equiv 2b - a + k,
      \end{align*}
      meaning that $k = 3b + a$ (modulo $5$). We thus have that
      \begin{align*}
        \left( 3b + a \right) - \left( a + bi \right) &= 3b - bi\\
                                                      &= b\left( 3-i \right)\\
                                                      &= b\left( 1-i \right)\left( 2+i \right),
      \end{align*}
      so $z\mapsto z + I$ is surjective. We observe furthermore that $5\Z\subseteq \ker\left( \varphi \right)$, and since $5$ is prime, it is a subset of no other ideal, and since the homomorphism $\varphi$ is nontrivial, we thus have that $\ker\left( \varphi \right) = 5\Z$, so by the first isomorphism theorem, $\Z\left[ i \right]/\left( 2+i \right) \cong \Z/5\Z$.
  \end{enumerate}
\end{solution}
\begin{problem}[Problem 4]
  Let $R_1$ and $R_2$ be rings. We consider the Cartesian product $R = R_1\times R_2$ and introduce the operations
  \begin{align*}
    \left( a_1,b_1 \right) + \left( a_2,b_2 \right) &= \left( a_1 + a_2,b_1 + b_2 \right)\\
    \left( a_1,a_2 \right)\left( b_1,b_2 \right) &= \left( a_1a_2,b_1b_2 \right).
  \end{align*}
  Show that $R$ is a ring with these operations.
\end{problem}
\begin{solution}
  If $a_1,b_1\in R_1$ and $a_2,b_2\in R_2$, then since $a_1 - a_2\in R_1$ and $b_1-b_2\in R_2$, it is clear that $R_1\times R_2$ endowed with the $+$ operation is an abelian group with additive identity $\left( 0,0 \right)$ as we have endowed $R_1\times R_2$ with coordinate-wise operations inherited from $R_1$ and $R_2$.\newline

  Similarly, if $c_1\in R_1$ and $c_2\in R_2$, then since multiplication is associative in $R_1$ and $R_2$, we have
  \begin{align*}
    \left( a_1,a_2 \right)\cdot \left( \left( b_1,b_2 \right)\cdot \left( c_1,c_2 \right) \right) &= \left( a_1\cdot \left( b_1\cdot c_1 \right),a_2\cdot \left( b_2\cdot c_2 \right) \right)\\
                                                                                                  &= \left( \left( a_1\cdot b_1 \right)\cdot c_1,\left( a_2\cdot b_2 \right)\cdot c_2 \right)\\
                                                                                                  &= \left( \left( a_1,a_2 \right)\cdot \left( b_1,b_2 \right) \right)\cdot \left( c_1,c_2 \right),
  \end{align*}
  so multiplication is associative in $R_1\times R_2$. Finally, since multiplication distributes over addition in $R_1$ and in $R_2$, we have that 
  \begin{align*}
    \left( a_1,a_2 \right)\cdot \left( \left( b_1,b_2 \right) + \left( c_1,c_2 \right) \right) &= \left( a_1\cdot \left( b_1 + c_1 \right),a_2\cdot \left( b_2 + c_2 \right) \right)\\
                                                                                               &= \left( a_1\cdot b_1 + a_1\cdot c_1,a_2\cdot b_2 + a_2\cdot c_2 \right)\\
                                                                                               &= \left( a_1\cdot b_1,a_2\cdot b_2 \right) + \left( a_1\cdot c_1,a_2\cdot c_2 \right)\\
                                                                                               &= \left( a_1,a_2 \right)\cdot \left( b_1,b_2 \right) + \left( a_1,a_2 \right)\cdot \left( c_1,c_2 \right),
  \end{align*}
  meaning that multiplication distributes over addition in $R_1\times R_2$.
\end{solution}
\begin{problem}[Problem 7]
  Let $I,J$ be ideals such that $I + J = R$ and $IJ = 0$. Show that the map
  \begin{align*}
    f\colon R\rightarrow \R/I\times \R/J
  \end{align*}
  given by $x\mapsto \left( x+I,x+J \right)$ is a ring isomorphism.
\end{problem}
\begin{solution}
  By the result from Problem 1, we know that since $I + J = R$ and $IJ = 0$, then $I\cap J = \set{0}$. Therefore, since $r\in \ker\left( f \right)$ if and only if $r + I = 0 + I$ and $r + J = 0 + J$, or that $r\in I$ and $r\in J$, we have that $\ker\left( f \right) = 0$.\newline

  Furthermore, for any $\left( r+I,s+J \right)\in R/I\times R/J$, we use the fact that $r + I\neq 0 + I$ if and only if $r\in J$, and $s + J \neq 0 + J$ if and only if $s\in I$, meaning that $x = r + s$ satisfies $x + I = r + I$ and $x + J = r + J$. Moreover, if $r + I = 0 + I$ and $s + J \neq 0 + J$, then $x = s$ satisfies the desired result, while if $r + I \neq 0$ and $s + J = 0$, then $x = r$ satisfies the desired result. Thus, $f$ is surjective, hence an isomorphism.
\end{solution}
\end{document}
