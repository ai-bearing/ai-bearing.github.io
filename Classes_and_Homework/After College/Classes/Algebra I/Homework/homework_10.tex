\documentclass[10pt]{mypackage}

% sans serif font:
%\usepackage{cmbright}
%\usepackage{sfmath}
%\usepackage{bbold} %better blackboard bold

\usepackage{homework}
%\usepackage{notes}
\usepackage{newpxtext,eulerpx,eucal}
\renewcommand*{\mathbb}[1]{\varmathbb{#1}}

\fancyhf{}
\fancyhead[R]{Avinash Iyer}
\fancyhead[L]{Algebra I: Homework 10}
\fancyfoot[C]{\thepage}

\setcounter{secnumdepth}{0}

\begin{document}
\RaggedRight
\begin{problem}[Problem 1]\hfill
  \begin{enumerate}[(a)]
    \item Let $G$ be a finite group. Show that for any subgroup $H\leq G$, we have $n_p\left( H \right)\leq n_p\left( G \right)$.
    \item Let $f\colon G\rightarrow G'$ be a surjective homomorphism of finite groups, and let $p$ be a prime. Show that every $p$-Sylow subgroup $P'$ of $G'$ is the image of some $p$-Sylow subgroup $P$ of $G$.
  \end{enumerate}
\end{problem}
\begin{solution}\hfill
  \begin{enumerate}[(a)]
    \item Suppose $\left\vert G \right\vert = p^{r}m$ and $\left\vert H \right\vert = p^{s}\ell$, with $p\nmid m,\ell$.\newline

      First, we observe that if $s = r$, then any $p$-Sylow subgroup of $H$ is a $p$-Sylow subgroup of $G$ that is contained in $H$, whence $n_p(H)\leq n_p(G)$.\newline

      Now, let $s < r$. We observe that if $P\leq H\leq G$ is a $p$-Sylow subgroup of $H$, then by the second Sylow theorem, $P$ is contained in some $p$-Sylow subgroup, $P'\leq G$. We claim that any two distinct $p$-Sylow subgroups of $H$ must be contained in distinct $p$-Sylow subgroups of $G$. This follows from the fact that, if $P_1,P_2\leq H$ are two distinct $p$-Sylow subgroups, and $P_1,P_2\leq P'$, then the subgroup $\left\langle P_1,P_2 \right\rangle$ generated in $H$ is contained in both $H$ and $P'$, but has strictly larger order than either $P_1$ or $P_2$, which contradicts the maximality of the orders of $P_1$ and $P_2$ respectively. Thus, any $p$-Sylow subgroup of $H$ is of the form $P'\cap H$ for some $p$-Sylow subgroup of $G$, whence $n_p(H)\leq n_p(G)$.
    \item Let $N = \ker\left( f \right)$, and let $P_0$ be a $p$-Sylow subgroup of $N$. By the second Sylow theorem, there is a $p$-Sylow subgroup of $G$, $P_1$, such that $P_0\subseteq P_1$. From the first isomorphism theorem, we know that $f$ induces an isomorphism $\hat{f}\colon G/N\rightarrow G'$, so we will establish the result in $G/N$.\newline

      First, observe that $\pi\colon G\rightarrow G/N$ induces the subgroup $P_1N/N$ in $G/N$. By the second isomorphism theorem, it then follows that
      \begin{align*}
        P_1N/N &\cong P_1/P_1\cap N,
      \end{align*}
      whence
      \begin{align*}
        \left\vert P_1N/N \right\vert &= \left\vert P_1/P_1\cap N \right\vert\\
                                      &= \frac{\left\vert P_1 \right\vert}{\left\vert P_1\cap N \right\vert}\\
                                      &= \frac{\left\vert P-1 \right\vert}{\left\vert P_0 \right\vert},
      \end{align*}
      meaning that $f\left( P_1 \right)$ is a $p$-group. Furthermore, since $P_1$ is a maximal $p$-group, it follows that $p\nmid \left[ G:P_1 \right]$, and since $P_1\subseteq P_1N$, we have that $p\nmid \left[ G:P_1N \right]$. Yet, this means that $p$ does not divide
      \begin{align*}
        \frac{\left\vert G \right\vert}{\left\vert P_1N \right\vert} &= \frac{\left\vert G/N \right\vert}{\left\vert P_1N/N \right\vert}\\
                                                                     &= \left[ G/N : P_1N/N \right],
      \end{align*}
      meaning that $P_1N/N$ is in fact a maximal $p$-subgroup of $G/N$.\newline

      Now, we show that any $p$-Sylow subgroup of $G/N$ arises in this fashion. In particular, suppose $P'\leq G'\cong G/N$ is a $p$-Sylow subgroup. Then, by the fourth isomorphism theorem, it follows that $P'$ corresponds to a subgroup $P_2\in G$ with $N\leq P_2$. Such a $P_2$ must be a $p$-subgroup by Lagrange's theorem, as $P_2$ surjects onto $P'$. Our goal is to show that $p\nmid \left[ G:P_2 \right]$, which as we showed before, is equivalent to showing that $p\nmid \left[ G/N : P_2N/N \right]$.\newline

      Now, since $P_2$ is a $p$-subgroup, there is some conjugate $gP_2g^{-1}$ such that $gP_2g^{-1}$ is a subgroup of $ P_1$ by the second Sylow theorem.\newline

      Yet, this means that $\left( gN \right) \left( P_2N/N \right) \left( gN \right)^{-1}\leq P_1N/N$, meaning that $\left\vert P_2N/N \right\vert = \left\vert P_1N/N \right\vert$ as we have stated that $P'\cong P_2N/N$ is a $p$-Sylow subgroup in and of itself. Therefore, it follows that $\left[ G/N:P_2N/N \right] = \left[ G/N : P_1N/N \right]$, so that $p\nmid \left[ G : P_2 \right]$, thus $P_2$ is a $p$-Sylow subgroup.
  \end{enumerate}
\end{solution}
\begin{problem}[Problem 2]
  Let $F = \Z/p\Z$ be the field with $p$ elements. Show that the group of upper unitriangular matrices,
  \begin{align*}
    U &= \set{\left( a_{ij} \right)_{i,j}\in \GL_n\left( F \right) | a_{ij} = 0\text{ for }i > j\text{ and }a_{ii} = 1}
  \end{align*}
  is a $p$-Sylow subgroup of $G = \GL_n\left( F \right)$.
\end{problem}
\begin{solution}
  To start, we observe that the group of upper unitriangular matrices has an order of $p^{n\left( n-1 \right)/2}$, as follows from the fact that all elements in the strict upper triangle of any given matrix can be pulled from $\Z/p\Z$.\newline

  The order of $\GL_n\left( F \right)$ can be seen to be
  \begin{align*}
    \left\vert \GL_n\left( F \right) \right\vert &= \left( p^{n}-1 \right)\left( p^{n}-p \right)\cdots \left( p^{n}-p^{n-1} \right).
  \end{align*}
  Taking out powers of $p$ from each of the factors that divides $p$, we find that we get
  \begin{align*}
    \left\vert \GL_n\left( F \right) \right\vert &= \left( p^{n} -1 \right) \left( p \right)\left( p^{n-1}-1 \right) \cdots \left( p^{n-1} \right)\left( p-1 \right)\\
                                                 &= p\cdot p^2 \cdots p^{n-1}\left( p^{n}-1 \right)\left( p^{n-1}-1 \right)\cdots \left( p-1 \right)\\
                                                 &= p^{n\left( n-1 \right)/2}\left( p^{n}-1 \right)\left( p^{n-1}-1 \right)\cdots \left( p-1 \right).
  \end{align*}
  Now, we observe that each of the factors of the form $p^{k}-1$ are coprime to $p^{k}$, meaning that $p$ necessarily cannot divide $p^{k}-1$ for each $k$, so that any $p$-Sylow subgroup of $\GL_n\left( F \right)$ has the order $p^{n\left( n-1 \right)/2}$.
\end{solution}
\begin{problem}[Problem 4]
  Let $G$ be a finite group of order $p^{n}$ with $n\geq 1$.\newline

  Show that for every $m = 0,1,\dots,n$, the group $G$ has a subgroup of order $p^{m}$.
\end{problem}
\begin{solution}
  We prove using induction. If $n = 1$, then $G$ is a group of the form $\Z/p\Z$, meaning that $G$ has a subgroup of order $1$, which is $\set{e}$, and a group of order $p$, which is the group itself.\newline

  Now, let $\left\vert G \right\vert = p^{n}$, and suppose that for any group with order $p^{k}$ with $k < n$, we have that said group contains subgroups of all prime orders from $0$ to $k$. Letting $G$ act on itself via conjugation, we obtain from the class equation that
  \begin{align*}
    \left\vert G \right\vert &= \left\vert Z(G) \right\vert + \sum_{a\in A} \left[ G:Z_G(a) \right].
  \end{align*}
  Taking residues modulo $p$, we observe that $\left\vert Z(G) \right\vert \equiv \left\vert G \right\vert$ modulo $p$, and since $\left\vert Z(G) \right\vert \geq 1$ as $\set{e}\in Z(G)$, it follows that $\left\vert Z(G) \right\vert \geq p$.\newline

  Now, if $Z(G) = G$, then $G$ is abelian, so by the structure of finite abelian groups, we have
  \begin{align*}
    G\cong \Z/p^{n_1}\Z\times\cdots\times \Z/p^{n_k}\Z,
  \end{align*}
  where we may use powers of $p$ as the sole factors by virtue of the fact that $G$ is a $p$ group. Taking residue classes modulo $\Z/p\Z\times \set{0}\times \cdots \times \set{0}$, we observe then that the quotient group $G/\left( \Z/p\Z \right)$ has subgroups of orders up to $n-1$ all of which contain $\Z/p\Z$ by the fourth isomorphism theorem, whence $G$ has subgroups of all orders up to $n$.\newline

  Meanwhile, if $Z(G)\neq G$, then $Z(G)$ is a $p$-group of order less than $p^{n}$, and the quotient group $G/Z(G)$ has order strictly less than $p^{n}$ as well, meaning that the former contains $p$-subgroups up to the order of $Z(G)$, while the latter contains $p$-subgroups up to the order of $G/Z(G)$, each of which contains $Z(G)$, so that $G$ contains $p$-subgroups of all orders up to $p^{n}$.
\end{solution}
\begin{problem}[Problem 5]
  Show that a group of order $351$ always has a normal $p$-Sylow subgroup for some prime $p$ dividing the order.
\end{problem}
\begin{solution}
  The prime factorization of $351$ yields $3^{3}\cdot 13$. We observe that the number of $13$-Sylow subgroups is congruent to $1$ modulo $13$ and divides $27$; in particular, we have the cases of $1$ and $27$. If there is one $13$-Sylow subgroup, this subgroup is normal, and we are done. Else, if there are $27$ $13$-Sylow subgroups, these subgroups intersect at the identity (as each is isomorphic to $\Z/13\Z$), and thus there are $ 324 $ elements of order $13$, giving $ 27 $ elements with order not equal to $13$. Since, by the first Sylow theorem, there is at least one $3$-Sylow subgroup, this is the $3$-Sylow subgroup, which is necessarily normal.
\end{solution}
\begin{problem}[Problem 8]
  Let $G$ be a group of order $3\cdot 5^2 \cdot 17$.
  \begin{enumerate}[(a)]
    \item Show that $n_{17}(G) = 1$. That is, a $17$-Sylow subgroup $H$ is normal.
    \item The conjugation action of $G$ on $H$ defines a group homomorphism $G\rightarrow \aut(H)$. Show that this homomorphism is trivial, and conclude that $H\subseteq Z(G)$.
  \end{enumerate}
\end{problem}
\begin{solution}\hfill
  \begin{enumerate}[(a)]
    \item By the third Sylow theorem, we know that $n_{17}(G) | 75$ and $n_{17}(G) \equiv 1$ modulo $17$. Writing out the possibilities for $n_{17}$ under the second condition explicitly gives
      \begin{align*}
        n_{17}\left( G \right) &= 1,18,35,52,69,86,\dots
      \end{align*}
      of which only $1$ divides $75$. Thus, there is only one $17$-Sylow subgroup.
    \item Let $H$ be the $17$-Sylow subgroup of $G$. Since $gHg^{-1} = H$ for all $g\in G$, it follows that the map $g\mapsto \iota_g$ defines a group homomorphism $f\colon G\rightarrow \aut\left( H \right)$. Since $H$ is abelian, $H\leq \ker\left( f \right)$, so by the first isomorphism theorem, there is an induced homomorphism $ \overline{f}\colon G/H\rightarrow \aut\left( H \right) $.\newline

      Now, we observe that, since $H$ has prime order, $H\cong \Z/17\Z$, meaning that $\aut\left( H \right)\cong \left( \Z/17\Z \right)^{\times}$, which is a group of order $16$, while $\left\vert G/H \right\vert = 75$. Yet, since $16\nmid 75$, it follows that $ \overline{f} $ must in fact be the trivial homomorphism, meaning that $ ghg^{-1} = g $ for each $g\in G$ (as it is already true for all $g\in H$ and any representative for $gH\in G/H$). Therefore, $H\subseteq Z(G)$.
  \end{enumerate}
\end{solution}
\end{document}
