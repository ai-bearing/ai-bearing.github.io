\documentclass[10pt]{mypackage}

% sans serif font:
%\usepackage{cmbright}
%\usepackage{sfmath}
%\usepackage{bbold} %better blackboard bold

\usepackage{homework}
%\usepackage{notes}
\usepackage{newpxtext,eulerpx,eucal}
\renewcommand*{\mathbb}[1]{\varmathbb{#1}}

\fancyhf{}
\fancyhead[R]{Avinash Iyer}
\fancyhead[L]{Algebra I: Homework 10}
\fancyfoot[C]{\thepage}

\setcounter{secnumdepth}{0}

\begin{document}
\RaggedRight
\begin{problem}[Problem 1]\hfill
  \begin{enumerate}[(a)]
    \item Let $G$ be a finite group. Show that for any subgroup $H\leq G$, we have $n_p\left( H \right)\leq n_p\left( G \right)$.
    \item Let $f\colon G\rightarrow G'$ be a surjective homomorphism of finite groups, and let $p$ be a prime. Show that every $p$-Sylow subgroup $P'$ of $G'$ is the image of some $p$-Sylow subgroup $P$ of $G$.
  \end{enumerate}
\end{problem}
\begin{solution}\hfill
  \begin{enumerate}[(a)]
    \item Suppose $\left\vert G \right\vert = p^{r}m$ and $\left\vert H \right\vert = p^{s}\ell$, with $p\nmid m,\ell$.\newline

      First, we observe that if $s = r$, then any $p$-Sylow subgroup of $H$ is a $p$-Sylow subgroup of $G$ that is contained in $H$, whence $n_p(H)\leq n_p(G)$.\newline

      Now, let $s < r$. We observe that if $P\leq H\leq G$ is a $p$-Sylow subgroup of $H$, then by the second Sylow theorem, $P$ is contained in some $p$-Sylow subgroup, $P'\leq G$. We claim that any two distinct $p$-Sylow subgroups of $H$ must be contained in distinct $p$-Sylow subgroups of $G$. This follows from the fact that, if $P_1,P_2\leq H$ are two distinct $p$-Sylow subgroups, and $P_1,P_2\leq P'$, then the subgroup $\left\langle P_1,P_2 \right\rangle$ generated in $H$ is contained in both $H$ and $P'$, but has strictly larger order than either $P_1$ or $P_2$, which contradicts the maximality of the orders of $P_1$ and $P_2$ respectively. Thus, any $p$-Sylow subgroup of $H$ is of the form $P'\cap H$ for some $p$-Sylow subgroup of $G$, whence $n_p(H)\leq n_p(G)$.
  \end{enumerate}
\end{solution}
\begin{problem}[Problem 8]
  Let $G$ be a group of order $3\cdot 5^2 \cdot 17$.
  \begin{enumerate}[(a)]
    \item Show that $n_{17}(G) = 1$. That is, a $17$-Sylow subgroup $H$ is normal.
    \item The conjugation action of $G$ on $H$ defines a group homomorphism $G\rightarrow \aut(H)$. Show that this homomorphism is trivial, and conclude that $H\subseteq Z(G)$.
  \end{enumerate}
\end{problem}
\begin{solution}\hfill
  \begin{enumerate}[(a)]
    \item By the third Sylow theorem, we know that $n_{17}(G) | 75$ and $n_{17}(G) \equiv 1$ modulo $17$. Writing out the possibilities for $n_{17}$ under the second condition explicitly gives
      \begin{align*}
        n_{17}\left( G \right) &= 1,18,35,52,69,86,\dots
      \end{align*}
      of which only $1$ divides $75$. Thus, there is only one $17$-Sylow subgroup.
  \end{enumerate}
\end{solution}
\end{document}
