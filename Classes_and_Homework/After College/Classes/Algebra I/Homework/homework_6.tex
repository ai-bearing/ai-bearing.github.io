\documentclass[10pt]{mypackage}

% sans serif font:
%\usepackage{cmbright}
%\usepackage{sfmath}
%\usepackage{bbold} %better blackboard bold

\usepackage{homework}
%\usepackage{notes}
\usepackage{newpxtext,eulerpx,eucal}
\renewcommand*{\mathbb}[1]{\varmathbb{#1}}

\fancyhf{}
\rhead{Avinash Iyer}
\lhead{Algebra I: Homework 6}

\setcounter{secnumdepth}{0}

\begin{document}
\RaggedRight
\begin{problem}[Problem 1]
  Let $R$ be a ring and $M$ a left $R$-module.
  \begin{enumerate}[(a)]
    \item Prove that for every $m\in M$, the map $r\mapsto r\cdot m$ from $R$ to $M$ is a homomorphism of $R$-modules.
    \item Assume that $R$ is commutative and $M$ an $R$-module. Prove that there is an isomorphism $\hom_{R}\left( R,M \right)\cong M$ as left $R$-modules.
  \end{enumerate}
\end{problem}
\begin{solution}\hfill
  \begin{enumerate}[(a)]
    \item Let $m\in M$ be fixed, and define $\varphi_m\colon R\rightarrow M$ by
      \begin{align*}
        \varphi_m(r) &= r\cdot m.
      \end{align*}
      It follows from the axioms of left $R$-modules that
      \begin{align*}
        \varphi_m\left( r + s \right) &= \left( r+s \right)\cdot m\\
                                      &= r\cdot m + s\cdot m\\
                                      &= \varphi_m(r) + \varphi_m(s),
      \end{align*}
      and
      \begin{align*}
        \varphi_m\left( rs \right) &= \left( rs \right)\cdot m\\
                                   &= r\cdot \left( s\cdot m \right)\\
                                   &= r\cdot\left( \varphi_m(s) \right),
      \end{align*}
      so that $\varphi_m$ is a homomorphism of left $R$-modules.
    \item If $\varphi_m\colon R\rightarrow M$ is the homomorphism as defined in part (a), we define a map $\varphi\colon M\rightarrow \hom_{R}\left( R,M \right)$ by
      \begin{align*}
        \varphi(m)(r) &= \varphi_m(r).
      \end{align*}
      First, we verify that $\varphi$ is a homomorphism. If $r\in R$ is arbitrary, then
      \begin{align*}
        \varphi\left( m + n \right)\left( r \right) &= \varphi_{m+n}\left( r \right)\\
                                                    &= r\cdot\left( m + n \right)\\
                                                    &= r\cdot m + r\cdot n\\
                                                    &= \varphi_m\left( r \right) + \varphi_n\left( r \right)\\
                                                    &= \left( \varphi(m) + \varphi(n) \right)(r).
      \end{align*}

      To see that $\varphi$ is injective, we see that $\ker\left( \varphi \right)$ consists of all elements $m\in M$ such that $\varphi(m) = \varphi_0$, where $\varphi_0\colon R\rightarrow M$ takes $r\mapsto 0$ for all $r\in R$. In particular, since $1\in R$, it follows that $1\cdot m = m = 0$, meaning that $\ker\left( \varphi \right) = \set{0}$.\newline


  \end{enumerate}
\end{solution}
\end{document}
