\documentclass[10pt]{mypackage}

% sans serif font:
%\usepackage{cmbright}
%\usepackage{sfmath}
%\usepackage{bbold} %better blackboard bold

\usepackage{homework}
%\usepackage{notes}
\usepackage{newpxtext,eulerpx,eucal}
\renewcommand*{\mathbb}[1]{\varmathbb{#1}}

\fancyhf{}
\rhead{Avinash Iyer}
\lhead{Algebra I: Homework 6}

\setcounter{secnumdepth}{0}

\begin{document}
\RaggedRight
\begin{problem}[Problem 1]
  Let $R$ be a ring and $M$ a left $R$-module.
  \begin{enumerate}[(a)]
    \item Prove that for every $m\in M$, the map $r\mapsto r\cdot m$ from $R$ to $M$ is a homomorphism of $R$-modules.
    \item Assume that $R$ is commutative and $M$ an $R$-module. Prove that there is an isomorphism $\hom_{R}\left( R,M \right)\cong M$ as left $R$-modules.
  \end{enumerate}
\end{problem}
\begin{solution}\hfill
  \begin{enumerate}[(a)]
    \item Let $m\in M$ be fixed, and define $\varphi_m\colon R\rightarrow M$ by
      \begin{align*}
        \varphi_m(r) &= r\cdot m.
      \end{align*}
      It follows from the axioms of left $R$-modules that
      \begin{align*}
        \varphi_m\left( r + s \right) &= \left( r+s \right)\cdot m\\
                                      &= r\cdot m + s\cdot m\\
                                      &= \varphi_m(r) + \varphi_m(s),
      \end{align*}
      and
      \begin{align*}
        \varphi_m\left( rs \right) &= \left( rs \right)\cdot m\\
                                   &= r\cdot \left( s\cdot m \right)\\
                                   &= r\cdot\left( \varphi_m(s) \right),
      \end{align*}
      so that $\varphi_m$ is a homomorphism of left $R$-modules.
    \item If $\varphi_m\colon R\rightarrow M$ is the homomorphism as defined in part (a), we define a map $\varphi\colon M\rightarrow \hom_{R}\left( R,M \right)$ by
      \begin{align*}
        \varphi(m)(r) &= \varphi_m(r).
      \end{align*}
      First, we verify that $\varphi$ is a homomorphism. If $r\in R$ is arbitrary, then
      \begin{align*}
        \varphi\left( m + n \right)\left( r \right) &= \varphi_{m+n}\left( r \right)\\
                                                    &= r\cdot\left( m + n \right)\\
                                                    &= r\cdot m + r\cdot n\\
                                                    &= \varphi_m\left( r \right) + \varphi_n\left( r \right)\\
                                                    &= \left( \varphi(m) + \varphi(n) \right)(r).
      \end{align*}

      To see that $\varphi$ is injective, we see that $\ker\left( \varphi \right)$ consists of all elements $m\in M$ such that $\varphi(m) = \varphi_0$, where $\varphi_0\colon R\rightarrow M$ takes $r\mapsto 0$ for all $r\in R$. In particular, since $1\in R$, it follows that for all $m\in \ker\left( \varphi \right)$, we have $1\cdot m = m = 0$, so $\ker\left( \varphi \right) = \set{0}$.\newline

      To see that $\varphi$ is surjective, we observe that for any $\psi\in \hom_{R}\left( R,M \right)$, $\psi$ is fully determined by where it maps $1$, as
      \begin{align*}
        \psi(r) &= r\cdot \psi(1).
      \end{align*}
      Therefore, if $\psi\in \hom_{R}\left( R,M \right)$, then we may find $m\in M$ corresponding to $\psi$ by taking
      \begin{align*}
        m &\coloneq \psi(1).
      \end{align*}
      Thus, $M\cong \hom_{R}\left( R,M \right)$.
  \end{enumerate}
\end{solution}
\begin{problem}[Problem 3]
  Let $R$ be a ring, and $M$ a left $R$-module.
  \begin{enumerate}[(a)]
    \item Let $N$ be a subset of $M$. The \textit{annihilator} of $N$ is defined to be the set
      \begin{align*}
        \operatorname{ann}_R\left( N \right) &= \set{r\in R | r\cdot n = 0\text{ for all }n\in N}.
      \end{align*}
      Prove that $\operatorname{ann}_R(N)$ is a left-ideal of $R$.
    \item Show that if $N$ is an $R$-submodule of $M$, then $\operatorname{ann}_R\left( N \right)$ is a two-sided ideal of $R$.
    \item For a subset $I$ of $R$, the \textit{annihilator} of $I$ in $M$ is defined to be the set
      \begin{align*}
        \operatorname{ann}_M\left( I \right) &= \set{m\in M | x\cdot m = 0\text{ for all }x\in I}.
      \end{align*}
      Find a natural condition on $I$ that guarantees $\operatorname{ann}_M(I)$ is a submodule of $M$.
    \item Let $R$ be an integral domain. Prove that every finitely generated torsion $R$-module has a nonzero annihilator.
  \end{enumerate}
\end{problem}
\begin{solution}\hfill
  \begin{enumerate}[(a)]
    \item First, we observe that $\operatorname{ann}_R(N)$ is nonempty, as $0\in \operatorname{ann}_R(N)$. Additionally, if $s,t\in \operatorname{ann}_R(N)$, then for all $n\in N$,
      \begin{align*}
        \left( s-t \right)\cdot n &= s\cdot n - t\cdot n\\
                                  &= 0,
      \end{align*}
      so that $N$ is closed under subtraction. Finally, if $r\in R$ and $s\in \operatorname{ann}_R\left( N \right)$, then for all $n\in N$,
      \begin{align*}
        \left( rs \right)\cdot n &= r\cdot \left( s\cdot n \right)\\
                                 &= r\cdot 0\\
                                 &= 0,
      \end{align*}
      meaning that $rs\in \operatorname{ann}_R\left( N \right)$, or that $\operatorname{ann}_R\left( N \right)$ is a left-ideal of $R$.
    \item Let $N$ be an $R$-submodule of $M$, and let $s\in \operatorname{ann}_R\left( N \right)$. If $r\in R$, then for all $n\in N$, $r\cdot n\in N$, so that $\left( sr \right)\cdot n = s\cdot \left( r\cdot n \right) = 0$, meaning that $sr\in \operatorname{ann}_R\left( N \right)$. Thus, $\operatorname{ann}_R\left( N \right)$ is a right-ideal, hence a two-sided ideal for $R$.
    \item We observe to start that $\operatorname{ann}_M(I)$ contains $0$ and is additively closed, since if $m,n\in \operatorname{ann}_M\left( I \right)$ and $x\in I$ are arbitrary, then
      \begin{align*}
        x\cdot \left( m + n \right) &= x\cdot m + x\cdot n\\
                                    &= 0.
      \end{align*}
      Therefore, if we desire for $\operatorname{ann}_M(I)$ to be a submodule of $M$, we would need $r\cdot m\in \operatorname{ann}_M(I)$ for all $m\in \operatorname{ann}_M(I)$ and all $r\in R$, which would mean $r\cdot m$ would have to satisfy the condition
      \begin{align*}
        0 &= x\cdot \left( r\cdot m \right)\\
          &= \left( xr \right)\cdot m
      \end{align*}
      for all $x\in I$, meaning that we would require $xr\in \operatorname{ann}_M(I)$. In other words, this means that $\operatorname{ann}_M(I)$ would have to be a right-ideal for $R$.
    \item Let $M = \left\langle a_1,\dots,a_n \right\rangle$ be a finitely generated torsion $R$-module. Since $M$ has torsion, for each $a_i$, there is some $0\neq r_i\in R$ such that $r_i\cdot a_i = 0$. The product
      \begin{align*}
        r &= \prod_{i=1}^{n}r_i
      \end{align*}
      is necessarily nonzero as $R$ is an integral domain, and satisfies $r\cdot a_i = 0$ for all $i$ by rearrangement of factors, so that $\left( r \right)\subseteq \operatorname{ann}_{R}\left( M \right)$ as $\operatorname{ann}_R\left( M \right)$ is an ideal containing $r$. Thus, $\operatorname{ann}_R\left( M \right)$ is a nonzero ideal.
  \end{enumerate}
\end{solution}
\begin{problem}[Problem 4]
  An $R$-module $M$ is called \textit{simple} if its only submodules are $\set{0}$ and $M$. An $R$-module $M$ is called \textit{indecomposable} if $M$ is not isomorphic to $N\oplus Q$ for some nonzero submodules $N$ and $Q$. Show that every simple $R$-module is indecomposable, but the converse is not true.
\end{problem}
\begin{solution}
  If $R$ is simple, then $R$ does not admit any nonzero proper submodules, meaning that $R$ cannot be isomorphic to the direct sum of any nonzero proper submodules.\newline

  Now, if we let $R = \Z$ be our ring, then we observe that all the nonzero proper ideals (i.e., $\Z$-submodules) of $\Z$ are of the form $\left( a \right)$ for some $a\in \Z$, as $\Z$ is a Euclidean domain (hence principal ideal domain). Observe that we can only write $\Z$ as a sum of submodules
  \begin{align*}
    \Z &= \left( a \right) + \left( b \right)
  \end{align*}
  when $\operatorname{gcd}\left( a,b \right) = 1$. Yet, these ideals necessarily do not intersect nontrivially, as $0\neq ab\in \left( a \right)\cap\left( b \right)$ meaning that $\Z$ is indecomposable. Meanwhile, $\Z$ is not simple since $\Z$ admits nonzero proper ideals.
\end{solution}
\begin{problem}[Problem 5]
  Let $R$ be a ring. An $R$-module $M$ is called cyclic if it is generated as an $R$-module by a single element. That is, $M = R\cdot m$ for some $m\in M$.
  \begin{enumerate}[(a)]
    \item Prove that every cyclic $R$-module is of the form $R/I$ for some left-ideal $I$ of $R$.
    \item Show that the simple $R$-modules are precisely the ones which are isomorphic to $R/\mathfrak{m}$ for some maximal left-ideal $\mathfrak{m}$.
    \item Show that any nonzero homomorphism of simple $R$-modules is an isomorphism. Deduce that if $M$ is simple, then its endomorphism ring
      \begin{align*}
        \operatorname{end}_R\left( M \right) &\coloneq \hom_{R}\left( M,M \right)
      \end{align*}
      is a division ring. This result is known as Schur's Lemma.
  \end{enumerate}
\end{problem}
\begin{solution}\hfill
  \begin{enumerate}[(a)]
    \item Let $M = \left\langle m \right\rangle$ be a cyclic $R$-module. Consider the map
      \begin{align*}
        \varphi\colon R\rightarrow M
      \end{align*}
      given by $r\mapsto r\cdot m$. Since $M$ is cyclic, this map is surjective, and admits the kernel $\operatorname{ann}_R\left( \set{m} \right)$. The annihilator is a left-ideal of $R$ as specified above, so that any such module is of the form $R/I$ for some left-ideal $I$ of $R$.
    \item If $M$ is a simple $R$-module, then if $0\neq m\in M$, we have that $R\cdot m = M$, as $R\cdot m$ is a submodule of $M$ that contains a nonzero element. Thus, we observe that $M$ is cyclic, so $M \cong R/I$ for some left-ideal $I$ of $R$. By the fourth isomorphism theorem and the correspondence between left-$R$-submodules of $R$ and left-ideals of $R$, we know that submodules of $M$ correspond to left-ideals of $R$ containing $I$; yet, since $M$ does not contain any proper submodules, it follows that any submodule of $M$ must either be isomorphic to $I$ or to $R$, meaning that $I$ is a maximal left-ideal.
    \item Let $\varphi\colon M\rightarrow N$ be a nonzero homomorphism of simple $R$-modules. Let $m\in M$ be nonzero, and let $\varphi(m) = n$ with $n\neq 0$. Then, for any $r\in R$, we have $\varphi\left( r\cdot m \right) = r\cdot n$. Since $M$ and $N$ are simple, and $m$ and $n$ are nonzero, it follows that $M = \left\langle m \right\rangle$ and $N = \left\langle n \right\rangle$, meaning that $\varphi$ is necessarily surjective, as for any element $r\cdot n\in N$, we may find $r\cdot m\in M$ such that $\varphi\left( r\cdot m \right) = r\cdot n$. Now, considering $\ker\left( \varphi \right)\subseteq M$, we observe that $\ker\left( \varphi \right)$ is a submodule; it follows that $\ker\left( \varphi \right) = \set{0}$ or $\ker\left( \varphi \right) = M$, but we know that it cannot be the latter as $\varphi$ is nonzero. Thus, $\varphi$ is an isomorphism.\newline

      If $M$ is simple, then if $\varphi\in \operatorname{end}_R\left( M \right)$ is nonzero, $\varphi$ is necessarily an automorphism as we have shown that nonzero homomorphisms of simple $R$-modules are isomorphisms, so that $\varphi$ admits an inverse. Thus, $ \operatorname{end}_R\left( M \right) $ is a division ring.
  \end{enumerate}
\end{solution}
\end{document}
