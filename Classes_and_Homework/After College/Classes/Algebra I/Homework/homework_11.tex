\documentclass[10pt]{mypackage}

% sans serif font:
%\usepackage{cmbright}
%\usepackage{sfmath}
%\usepackage{bbold} %better blackboard bold

\usepackage{homework}
%\usepackage{notes}
\usepackage{newpxtext,eulerpx,eucal}
\renewcommand*{\mathbb}[1]{\varmathbb{#1}}

\fancyhf{}
\fancyhead[R]{Avinash Iyer}
\fancyhead[L]{Algebra I: Homework 11}
\fancyfoot[C]{\thepage}

\setcounter{secnumdepth}{0}

\begin{document}
\RaggedRight
\begin{problem}[Problem 1]
  Show that every element of order $2$ in $A_n$ is the square of an element of order $4$ in $S_n$.
\end{problem}
\begin{solution}
  Let $\alpha\in A_n$ be written as a product of disjoint cycles
  \begin{align*}
    \alpha &= \sigma_1\cdots \sigma_r,
  \end{align*}
  such that $\alpha^2 = e$. Since $\alpha = \alpha^{-1}$, we then have that
  \begin{align*}
    \alpha &= \sigma_1^{-1}\cdots \sigma_r^{-1},
  \end{align*}
  whence each of $\sigma_1,\dots,\sigma_r$ is of order $2$. In particular, this means that $\alpha$ is in fact a product of an even number of disjoint transpositions, which we will rewrite as
  \begin{align*}
    \alpha &= \tau_1\cdots \tau_{2k}.
  \end{align*}
  Pairing up these transpositions, we observe that
  \begin{align*}
    \tau_1\tau_2 &= \left( a_1,b_1 \right)\left( a_2,b_2 \right)\\
                 &= \left( a_1,a_2,b_1,b_2 \right)^2,
  \end{align*}
  whence we have $k$ $4$-cycles $\zeta_1,\dots,\zeta_k$ given by
  \begin{align*}
    \zeta_i^2 &= \tau_{2i-1}\tau_{2i}
  \end{align*}
  Each of these $\zeta_i$ are disjoint, of order $4$, and we have
  \begin{align*}
    \gamma &= \zeta_1\cdots \zeta_k
  \end{align*}
  is of order $4$ in $S_n$ and is such that
  \begin{align*}
    \gamma^2 &= \alpha.
  \end{align*}
\end{solution}
\begin{problem}[Problem 2]
  Let $G = \left\langle x \right\rangle$ be a cyclic group, $H$ an arbitrary group. Let $\varphi_1,\varphi_2\colon G\rightarrow \aut(H)$ be homomorphisms such that $\img\left( \varphi_1 \right)$ and $\img\left( \varphi_2 \right)$ are conjugate. If $G$ is infinite, also assume that $\varphi_1$ and $\varphi_2$ are injective. Prove that the semidirect products $H\rtimes_{\varphi_1}G$ and $H\rtimes_{\varphi_2}G$ are isomorphic.
\end{problem}
\begin{problem}[Problem 3]\hfill
  \begin{enumerate}[(a)]
    \item Construct a nonabelian group of order $75$.
    \item Show that up to isomorphism there are three groups of order $75$.
  \end{enumerate}
\end{problem}
\begin{solution}\hfill
  \begin{enumerate}[(a)]
    \item We observe that $75 = 3\cdot 5^2$, so by the result on subgroups of the form $p^2 q$, with $q < p$, we have a unique $5$-Sylow subgroup. Suppose this $5$-Sylow subgroup is of the form $\Z/5\Z\times \Z/5\Z$. Then, this is in fact a $2$-dimensional vector space over $\Z/5\Z$, meaning that
      \begin{align*}
        \aut\left( \Z/5\Z\times \Z/5\Z \right) &\cong \GL_2\left( \Z/5\Z \right),
      \end{align*}
      which has order $24\cdot 20 = 480$. In particular, there is some nontrivial automorphism from $\Z/3\Z\rightarrow \aut\left( \Z/5\Z\times \Z/5\Z \right)$, which we can find by selecting an element of order $3$ from $\GL_2\left( \Z/5\Z \right)$, which emerges from the fact that $480 = 3\cdot 2^5\cdot 5$ admits a $3$-Sylow subgroup. This gives the nonabelian group $\left( \Z/5\Z\times \Z/5\Z \right)\rtimes_{f}\Z/3\Z$.
    \item We observe that there are two abelian groups of order $75$, given by
      \begin{align*}
        G_1 &= \Z/3\Z\times \Z/5\Z\times \Z/5\Z\\
        G_2 &= \Z/3\Z\times \Z/5^2\Z.
      \end{align*}
      The reason $G_1$ and $G_2$ are not isomorphic is that there are no elements of order $25$ in $G_1$, while (for example), $(0,3)$ has order $5^2$ in $G_2$.
  \end{enumerate}
\end{solution}
\end{document}
