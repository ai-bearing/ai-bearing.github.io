\documentclass[10pt]{mypackage}

% sans serif font:
%\usepackage{cmbright}
%\usepackage{sfmath}
%\usepackage{bbold} %better blackboard bold

\usepackage{homework}
%\usepackage{notes}
\usepackage{newpxtext,eulerpx,eucal}
\renewcommand*{\mathbb}[1]{\varmathbb{#1}}

\fancyhf{}
\fancyhead[R]{Avinash Iyer}
\fancyhead[L]{Algebra I: Homework 11}
\fancyfoot[C]{\thepage}

\setcounter{secnumdepth}{0}

\begin{document}
\RaggedRight
\begin{problem}[Problem 1]
  Show that every element of order $2$ in $A_n$ is the square of an element of order $4$ in $S_n$.
\end{problem}
\begin{solution}
  Let $\alpha\in A_n$ be written as a product of disjoint cycles
  \begin{align*}
    \alpha &= \sigma_1\cdots \sigma_r,
  \end{align*}
  such that $\alpha^2 = e$. Since $\alpha = \alpha^{-1}$, we then have that
  \begin{align*}
    \alpha &= \sigma_1^{-1}\cdots \sigma_r^{-1},
  \end{align*}
  whence each of $\sigma_1,\dots,\sigma_r$ is of order $2$. In particular, this means that $\alpha$ is in fact a product of an even number of disjoint transpositions, which we will rewrite as
  \begin{align*}
    \alpha &= \tau_1\cdots \tau_{2k}.
  \end{align*}
  Pairing up these transpositions, we observe that
  \begin{align*}
    \tau_1\tau_2 &= \left( a_1,b_1 \right)\left( a_2,b_2 \right)\\
                 &= \left( a_1,a_2,b_1,b_2 \right)^2,
  \end{align*}
  whence we have $k$ $4$-cycles $\zeta_1,\dots,\zeta_k$ given by
  \begin{align*}
    \zeta_i^2 &= \tau_{2i-1}\tau_{2i}
  \end{align*}
  Each of these $\zeta_i$ are disjoint, of order $4$, and we have
  \begin{align*}
    \gamma &= \zeta_1\cdots \zeta_k
  \end{align*}
  is of order $4$ in $S_n$ and is such that
  \begin{align*}
    \gamma^2 &= \alpha.
  \end{align*}
\end{solution}
\begin{problem}[Problem 2]
  Let $G = \left\langle x \right\rangle$ be a cyclic group, $H$ an arbitrary group. Let $\varphi_1,\varphi_2\colon G\rightarrow \aut(H)$ be homomorphisms such that $\img\left( \varphi_1 \right)$ and $\img\left( \varphi_2 \right)$ are conjugate. If $G$ is infinite, also assume that $\varphi_1$ and $\varphi_2$ are injective. Prove that the semidirect products $H\rtimes_{\varphi_1}G$ and $H\rtimes_{\varphi_2}G$ are isomorphic.
\end{problem}
\begin{solution}
  Let $M_1 = \varphi_1(G)$ and $M_2 = \varphi_2(G)$. We start with the assumption that $\varphi_1$ and $\varphi_2$ are injective and that $G$ is infinite. It then follows that $M_1 = \left\langle \varphi_1\left( x \right) \right\rangle$ and $M_2 = \left\langle \varphi_2\left( x \right) \right\rangle$ by injectivity. Since $M_1$ and $M_2$ are conjugate, it follows that there is some $g\in \aut(H)$ such that $gM_1g^{-1} = M_2$. Conjugation is an isomorphism, so this means that $g\varphi_1\left( x \right)g^{-1} = \varphi_2(x)^{\ell}$ for some integer $\ell$. Yet, since $G$ is infinite, it must be the case that $\ell = 1$.\newline

  Define the map $\psi\colon H\rtimes_{\varphi_1}G \rightarrow H\rtimes_{\varphi_2}G$ by taking
  \begin{align*}
    \left( x,y \right) &\mapsto \left( g(x),y \right),
  \end{align*}
  where $g$ is the automorphism discussed earlier. Since $g$ is an automorphism, it follows that $\psi$ is a bijection of sets, so we only need to show that it is a homomorphism, which we do below:
  \begin{align*}
    \psi\left( \left( x_1,y_1 \right)\left( x_2,y_2 \right) \right) &= \psi\left( x_1\varphi_1\left( y_1 \right)\left( x_2 \right),y_1y_2 \right)\\
                                                                    &= \left( g\left( x_1 \right)g\left( \varphi_1\left( y_1 \right)\left( x_2 \right) \right),y_1y_2 \right)\\
                                                                    &= \left( g\left( x_1 \right)\varphi_2\left( y_1 \right)\left( g\left( x_2 \right) \right),y_1y_2 \right)\\
                                                                    &= \left( g\left( x_1 \right),y_1 \right) \cdot \left( g\left( x_2 \right),y_2 \right).
  \end{align*}
  Now, suppose $G$ is of finite order. Via a similar process, we observe that there is $g\in \aut\left( H \right)$ such that $g\varphi_1(x)g^{-1} = \varphi_2(x)^{a} = \varphi_2\left( x^{a} \right)$ for some $a\in \Z$. Observe that $\gcd\left( a,m \right) = 1$, where $m$ is the order of the image of $G$ under both $\varphi_1$ and $\varphi_2$. If $a$ is already coprime to $n$, then $\left\langle x^{a} \right\rangle = G$. Else, we use the fact established via the Chinese Remainder Theorem that the surjection $\left[ w \right]_n\rightarrow \left[ w \right]_m$ induces a surjection from $\left( \Z/n\Z \right)^{\times}\rightarrow \left( \Z/m\Z \right)^{\times}$ whenever $m | n$. In particular, this means there is some $a'\in \Z/n\Z$ such that $a'\equiv a$ modulo $m$ and $\gcd\left( a',n \right) = 1$.\newline

  Define the map
  \begin{align*}
    \Psi\colon H\rtimes_{\varphi_1}G &\rightarrow H\rtimes_{\varphi_2}(G)\\
    \left( x,y \right) &\mapsto \left( g(x),y^{a'} \right).
  \end{align*}
  Since $G = \left\langle x \right\rangle = \left\langle x^{a'} \right\rangle$ and $g$ is an automorphism, it follows that $\Psi$ is a bijective set map, so all we need to verify is that $\Psi$ is a group homomorphism. This can be seen by taking
  \begin{align*}
    \Psi\left( \left( x_1,y_1 \right)\cdot \left( x_2,y_2 \right) \right) &= \Psi\left( x_1\varphi_1\left( y_1 \right)\left( x_2 \right),y_1y_2 \right)\\
                                                                          &= \left( g\left( x_1 \right)g\varphi_1\left( y_1 \right)\left( x_2 \right),y_1^{a'}y_2^{a'} \right)\\
                                                                          &= \left( g\left( x_1 \right)\varphi_2\left( y_1 \right)^{a}g\left( x_2 \right),y_1^{a'}y_2^{a'} \right)\\
                                                                          &= \left( g\left( x_1 \right) \varphi_2\left( y_1^{a} \right)g\left( x_2 \right),y_1^{a'}y_2^{a'} \right)\\
                                                                          &= \left( g\left( x_1 \right)\varphi_2\left( y_1^{a'} \right)g\left( x_2 \right),y_1^{a'}y_2^{a'} \right)\\
                                                                          &= \left( g\left( x_1 \right),y_1^{a'} \right)\cdot \left( g\left( x_2 \right),y_2^{a'} \right)\\
                                                                          &= \Psi\left( x_1,y_1 \right)\cdot \Psi\left( x_2,y_2 \right).
  \end{align*}
\end{solution}
\begin{problem}[Problem 3]\hfill
  \begin{enumerate}[(a)]
    \item Construct a nonabelian group of order $75$.
    \item Show that up to isomorphism there are three groups of order $75$.
  \end{enumerate}
\end{problem}
\begin{solution}\hfill
  \begin{enumerate}[(a)]
    \item We observe that $75 = 3\cdot 5^2$, so by the result on subgroups of the form $p^2 q$, with $q < p$, we have a unique $5$-Sylow subgroup. Suppose this $5$-Sylow subgroup is of the form $\Z/5\Z\times \Z/5\Z$. Then, this is in fact a $2$-dimensional vector space over $\Z/5\Z$, meaning that
      \begin{align*}
        \aut\left( \Z/5\Z\times \Z/5\Z \right) &\cong \GL_2\left( \Z/5\Z \right),
      \end{align*}
      which has order $480$. In particular, there is some nontrivial automorphism from $\Z/3\Z\rightarrow \aut\left( \Z/5\Z\times \Z/5\Z \right)$, which we can find by selecting an element of order $3$ from $\GL_2\left( \Z/5\Z \right)$, which emerges from the fact that $480 = 2^5\cdot 3\cdot 5$ admits a $3$-Sylow subgroup. This gives the nonabelian group $\left( \Z/5\Z\times \Z/5\Z \right)\rtimes_{f}\Z/3\Z$.
    \item We observe that there are two abelian groups of order $75$, given by
      \begin{align*}
        G_1 &= \Z/3\Z\times \Z/5\Z\times \Z/5\Z\\
        G_2 &= \Z/3\Z\times \Z/5^2\Z.
      \end{align*}
      The reason $G_1$ and $G_2$ are not isomorphic is that there are no elements of order $25$ in $G_1$, while (for example), $(0,3)$ has order $5^2$ in $G_2$.\newline

      In order to show that any two non-abelian groups of order $75$ are isomorphic to each other, we start by showing that any non-abelian group of order $75$ is of the form above. Since there is one $5$-Sylow subgroup, we observe that said $5$-Sylow subgroup is a group of order $p^2$, meaning that it has two forms. Either it is $\Z/5\Z\times \Z/5\Z$ or $\Z/25\Z$ by the classification of finite abelian groups. In the former case, we showed that $\Z/3\Z$ admits a nontrivial automorphism to $\aut\left( \Z/5\Z\times\Z/5\Z \right)$. On the other hand, we observe that $\aut\left( \Z/25\Z \right) = \left( \Z/25\Z \right)^{\times}$, which has 20 elements. Yet, this means there is no nontrivial homomorphism from $\Z/3\Z$ to $\left( \Z/25\Z \right)^{\times}$ by Lagrange's Theorem. Therefore we only need to consider homomorphisms from $\Z/3\Z\rightarrow \aut\left( \Z/5\Z\times\Z/5\Z \right)$.\newline

      Now, suppose we have two nontrivial homomorphisms $f_1\colon \Z/3\Z\rightarrow \aut\left( \Z/5\Z\times \Z/5\Z \right)$ and $f_2\colon \Z/3\Z\rightarrow \aut\left( \Z/5\Z\times\Z/5\Z \right)$. Since these homomorphisms are nontrivial, they are injective (by Lagrange's Theorem), so $P_1\coloneq \img\left( f_1 \right)$ and $P_2\coloneq \img\left( f_2 \right)$ are $3$-Sylow subgroups. Let $m_1 = f_1\left( 1 \right)$ and $m_2 = f_2\left( 1 \right)$ be generators for $P_1$ and $P_2$ respectively. Then, there is some $g\in \aut\left( \Z/5\Z\times\Z/5\Z \right)$ such that for all $\ell\in \Z/3\Z$, we have $gf_1(\ell)g^{-1} = f_2(\ell)$. In particular, since the automorphisms $f_1$ and $f_2$ are conjugate, it follows from the result in Problem 2 that these two semidirect products are isomorphic.\newline

      Thus, there are exactly three groups of order $75$ up to isomorphism.
  \end{enumerate}
\end{solution}
\begin{problem}[Problem 4]
  Let $G$ be a group of order $n = 2k$, where $k$ is odd. Show that $G$ is not simple. You can use the following steps.
  \begin{enumerate}[(a)]
    \item Consider the injection $\rho\colon G\hookrightarrow S_n$ given by Cayley's Theorem. Let $H = \left\langle x \right\rangle$ be a subgroup of $G$ of order $2$. Show that $\rho(x)$ is an odd permutation. Deduce that $\img\left( \rho \right)\cap A_n$ is a normal subgroup of $\img\left( \rho \right)$ of index $2$.
    \item Show that $G$ has a subgroup of index $2$.
  \end{enumerate}
\end{problem}
\begin{solution}
  Let $\rho\colon G\hookrightarrow S_n$ be the permutation representation for the left-regular action given by Cayley's theorem. If $H = \left\langle x \right\rangle$ is a subgroup of order $2$, then $x$ has order $2$. Since injective homomorphisms preserve order, it follows that $\rho(x)$ has order $2$.

  Decomposing $\rho(x)$ into disjoint cycles,
  \begin{align*}
    \rho(x) &= \tau_1\cdots \tau_r,
  \end{align*}
  we observe that each of the $\tau_i$ are transpositions since $\rho(x)$ has order $2$. Furthermore, since the action of $G$ on itself is free, it follows that we cannot have any identity permutations in the cycle decomposition of $\rho(x)$ (as $x$ is not identity), meaning that there are $k$ such transpositions. Since $k$ is odd, $\rho(x)$ is thus an odd permutation.\newline

  In particular, since identity is an even permutation and $\rho(x)$ is an odd permutation, it follows that the composition $\sgn\circ \rho\colon G\rightarrow \set{-1,1}$ is a surjective group homomorphism, so the kernel of this map is a proper normal subgroup of $G$. Thus, $G$ is not simple.
\end{solution}
\begin{problem}[Problem 5]
  Classify all groups up of order 28.
\end{problem}
\begin{solution}
  We start by analyzing the nature of the Sylow subgroups. Writing $28 = 2^2\cdot 7$, we observe that there is at least one $7$-Sylow subgroup and at least one $2$-Sylow subgroup. Yet, if there were more than one $7$-Sylow subgroup, then there would have to be at least $8$, but $8\nmid 4$. Thus, there is a unique $7$-Sylow subgroup. Similarly, since the number of $2$-Sylow subgroups is odd and divides $7$, it follows that there must be a unique $2$-Sylow subgroup. This subgroup is also normal.\newline

  Thus, right away, we are able to use the classification of finite abelian groups to find two such groups of order $28$, given by
  \begin{align*}
    G_1 &= \Z/4\Z\times \Z/7\Z\\
    G_2 &= \Z/2\Z\times\Z/2\Z\times \Z/7\Z.
  \end{align*}
  These groups are not isomorphic to each other since $G_1$ contains an element of order $4$ and $G_2$ does not.\newline

  Next, we consider all homomorphisms from $\Z/4\Z$ and $\Z/2\Z\times \Z/2\Z$ respectively into $\left( \Z/7\Z \right)^{\times}$. Observe that $\left( \Z/7\Z \right)^{\times}\cong \Z/6\Z$. There is thus an element of order $2$ in $\left( \Z/7\Z \right)^{\times}$, the equivalence class $[6]$, so we define homomorphisms $\varphi \colon\Z/4\Z\rightarrow \left( \Z/7\Z \right)^{\times}$ and $\psi\colon \Z/2\Z\times \Z/2\Z\rightarrow \left( \Z/7\Z \right)^{\times}$ by taking $1\mapsto [6]$ and $(0,1)\mapsto [6],\left( 1,0 \right)\mapsto [1]$ respectively. The corresponding semidirect products $\Z/7\Z\rtimes_{\varphi}\Z/4\Z$ and $\Z/7\Z\rtimes_{\psi}\Z/2\Z\times\Z/2\Z$ are not isomorphic to each other since the former contains the element $\left( 0,1 \right)$ that has order $4$, while the latter does not contain any element of order $4$.\newline

  The only case we need to verify is for the semidirect product $\Z/7\Z\rtimes_{\psi}\Z/2\Z\times\Z/2\Z$, since we may also define $\left( 1,0 \right)\mapsto [6],\left( 0,1 \right)\mapsto [1]$ as another map from $\Z/2\Z\times\Z/2\Z\rightarrow \left( \Z/7\Z \right)^{\times}$. Yet, these are isomorphic under the map
  \begin{align*}
    \Psi\colon \Z/7\Z\rtimes_{\psi_1}\Z/2\Z\times\Z/2\Z &\rightarrow \Z/7\Z\rtimes_{\psi_2}\Z/2\Z\times\Z/2\Z\\
    \left( x,\left( y,z \right) \right) &\mapsto \left( x,\left( z,y \right) \right).
  \end{align*}
  Thus, the four isomorphism classes for groups of order $28$ are given by the following:
  \begin{itemize}
    \item $\Z/7\Z\times\Z/2\Z\times\Z/2\Z$
    \item $\Z/7\Z\times\Z/4\Z$
    \item $\Z/7\Z\rtimes \Z/4\Z$
    \item $\Z/7\Z\rtimes \Z/2\Z\times\Z/2\Z$.
\end{itemize}
\end{solution}
\end{document}
