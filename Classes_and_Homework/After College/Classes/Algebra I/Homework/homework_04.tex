\documentclass[10pt]{mypackage}

% sans serif font:
%\usepackage{cmbright}
%\usepackage{sfmath}
%\usepackage{bbold} %better blackboard bold

\usepackage{homework}
%\usepackage{notes}
\usepackage{newpxtext,eulerpx,eucal}
\renewcommand*{\mathbb}[1]{\varmathbb{#1}}

\fancyhf{}
\rhead{Avinash Iyer}
\lhead{Algebra I: Assignment 4}

\setcounter{secnumdepth}{0}

\begin{document}
\RaggedRight
\begin{problem}[Problem 1]
  Let $I,J,K$ be ideals of $R$.
  \begin{enumerate}[(a)]
    \item Show that $\left( IJ \right)K = I\left( JK \right)$.
    \item Show that $\left( I + J \right)K = IK + JK$.
  \end{enumerate}
\end{problem}
\begin{solution}\hfill
  \begin{enumerate}[(a)]
    \item Let $u\in \left( IJ \right)K$. Then, $u$ is of the form
      \begin{align*}
        u &= \sum_{k=1}^{n}u_kz_k,
      \end{align*}
      where the $u_k\in IJ$ and the $z_k\in K$. Since each $u_k$ is an element of $IJ$, we may write
      \begin{align*}
        u_k &= \sum_{i=1}^{m}x_{k_i}y_{k_i},
      \end{align*}
      where the $x_{k_i}\in I$ and the $y_{k_i}\in J$. This yields an expression
      \begin{align*}
        u &= \sum_{k=1}^{n}\left( \sum_{i=1}^{m} x_{k_i}y_{k_i} \right)z_k\\
          &= \sum_{k=1}^{n}\sum_{i=1}^{m}x_{k_i}y_{k_i}z_k.
      \end{align*}
      We observe that, for a fixed $k$, $y_{k_i}z_k\in JK$. So, $x_{k_i}\left( y_{k_i}z_k \right)\in I\left( JK \right)$ for a fixed $k$, meaning that $u\in I\left( JK \right)$. A similar argument holds in the reverse direction.
    \item Elements of $ I + J $ are of the form $x_i + y_i$, where $x_i\in I$ and $y_i\in J$. This means that elements of $ \left( I + J \right)K $ are of the form
      \begin{align*}
        u &= \sum_{k=1}^{n} \sum_{i=1}^{m} \left( x_i + y_i \right)z_k\\
          &= \sum_{k=1}^{n} \underbrace{\left( \sum_{i=1}^{m} x_i \right)}_{\eqcolon x_k}z_k + \sum_{k=1}^{n}\underbrace{\left( \sum_{i=1}^{m} y_i \right)}_{\eqcolon y_k}z_k\\
          &= \sum_{k=1}^{n} x_kz_k + \sum_{k=1}^{n} y_kz_k.
      \end{align*}
      Thus, we find that $u$ is in $IK + JK$, and vice versa.
  \end{enumerate}
\end{solution}
\begin{problem}[Problem 4]
  Let $S_1\subseteq S_2$ be multiplicative subsets of $R$, and let $\iota_{S_i}\colon R\rightarrow S_i^{-1}R$ be the corresponding localization homomorphisms. Use the universal property of localization to show that there exists a unique ring homomorphism $\iota'\colon S_1^{-1}R\rightarrow S_2^{-1}R$ such that $\iota'\circ \iota_{S_1} = \iota_{S_2}$. Provide an explicit description of this ring homomorphism. Use this to show that if $R$ is an integral domain and $S$ an arbitrary multiplicative subset of $R$, then $S^{-1}R$ injects into the fraction field $K = \operatorname{frac}\left( R \right)$.
\end{problem}
\begin{solution}
  We observe that $\iota_{S_2}\colon R\rightarrow S_{2}^{-1}R$ maps elements of $S_1$ to units in $S_2^{-1}R$, as the units in $S_2^{-1}R$ are elements of the form $ \frac{s}{s'} $ with $s,s'\in S_2$, so by the universal property, there is a unique ring homomorphism $\iota'\colon S_1^{-1}R\rightarrow S_2^{-1}R$ such that $\iota'\circ \iota_{S_1} = \iota_{S_2}$. In particular, this is the map $ \left[ \frac{r}{1} \right]_{S_1^{-1}R} \mapsto \left[ \frac{r}{1} \right]_{S_2^{-1}R} $.\newline

  Since any arbitrary multiplicative subset $S\subseteq R$ of an integral domain is contained in $R\setminus \set{0}$, it follows that $S^{-1}R$ injects into $ \left( R\setminus \set{0} \right)^{-1}R \eqcolon \operatorname{frac}\left( R \right) $.
\end{solution}
\begin{problem}[Problem 5]
  Let $R = \Q\times \Q$ and $S = \set{\left( 1,1 \right)}\cup \left( \Q^{\times}\times \set{0} \right)$. The goal of this problem is to identify the localization $S^{-1}R$.
  \begin{enumerate}[(a)]
    \item Describe explicitly when $\frac{\left( a_1,a_2 \right)}{\left( s_1,s_2 \right)}$ is equal to $\frac{\left( b_1,b_2 \right)}{\left( t_1,t_2 \right)}$ in $S^{-1}R$.
    \item Use your result from part (a) to show that the localization $S^{-1}R$ is isomorphic to the localization $T^{-1}\Q$, where $T = \Q\setminus \set{0}$, hence is isomorphic to $\R$.
    \item Find the kernel of the localization homomorphism $\iota_S\colon R\rightarrow S^{-1}R$.
  \end{enumerate}
\end{problem}
\begin{solution}\hfill
  \begin{enumerate}[(a)]
    \item By the definition of the equivalence relation, we must have an element $\left( r_1,r_2 \right)\in S$ such that
      \begin{align*}
        \left( r_1\left( a_1t_1-b_1s_1 \right),r_2\left( a_2t_2-b_2s_2 \right) \right) &= \left( 0,0 \right).
      \end{align*}
      In particular, since $r_1\in \Q^{\times}$, and we may always select $r_2 = 0$, it follows that
      \begin{align*}
        r_1\left( a_1t_1-b_1s_1 \right) &= 0,
      \end{align*}
      so that $a_1t_1 - b_1s_1 = 0$ (as $\Q$ is an integral domain).
    \item We consider the map $\pi_1\colon \Q\times \Q\rightarrow \Q$, which maps $\left( a_1,a_2 \right)\mapsto a_1$. Observe then that $S^{-1}R$ satisfies the universal property for localization, as we may write $S = \left( \Q^{\times}\times \set{0} \right)\cup \left( \Q^{\times}\times \set{1} \right)$, which maps to $\Q^{\times}\subseteq \Q$ under this projection map.\newline

      In particular, we see that the induced map $\widetilde{\pi_1}\colon S^{-1}R\rightarrow \Q$ is given by
      \begin{align*}
        \widetilde{\pi_1} \left( \frac{\left( a_1,a_2 \right)}{\left( s_1,s_2 \right)} \right) &= a_1s_1^{-1}
      \end{align*}
      for $s_1\in \Q^{\times}$ and $a_1\in \Q$.\newline

      Now, we observe that the map $\operatorname{id}\circ \pi_1 = \pi_1$, and that $ T^{-1}\Q $ satisfies the universal property for localization with respect to $ \operatorname{id} $, inducing the homomorphism $ \widetilde{\operatorname{id}} $ that takes
      \begin{align*}
        \widetilde{\operatorname{id}} \left( \frac{a}{s} \right) &= as^{-1}
      \end{align*}
      for $s\in \Q^{\times}$. Yet, we also observe that, if we set $\iota_{T}' = \iota_{T}\circ \widetilde{\pi_1}\circ \iota_S$, that
      \begin{align*}
        \widetilde{\operatorname{id}}\circ \iota_{T}'\left( a_1,a_2 \right) &= \widetilde{\operatorname{id}}\circ \iota_{T}\circ \widetilde{\pi_1}\circ \iota_S \left( a_1,a_2 \right)\\
                                                                            &= \widetilde{\operatorname{id}}\circ \iota_T \circ \widetilde{\pi_1} \left( \frac{\left( a_1,a_2 \right)}{\left( 1,1 \right)} \right)\\
                                                                                                                &= \widetilde{\operatorname{id}}\circ \iota_{T}\left( a_1 \right)\\
                                                                                                                &= \widetilde{\operatorname{id}}\left( \frac{a_1}{1} \right)\\
                                                                                                                &= a_1\\
                                                                                                                &= \pi_1\left( a_1,a_2 \right).
      \end{align*}
      Thus, $T^{-1}\Q$ also satisfies the universal property for localization, implying that $T^{-1}\Q$ and $S^{-1}R$ are isomorphic.
      \begin{center}
% https://tikzcd.yichuanshen.de/#N4Igdg9gJgpgziAXAbVABwnAlgFyxMJZABgBoBGAXVJADcBDAGwFcYkQAdDgW3pwAsARoOABFAL5c83eF14DhY8SHGl0mXPkIoATBWp0mrdnL5CRElWpAZseAkXKliBhizaIQAZQB6wALTk4gBKVup2WkR6AMyuRh4gACp+gZI8ZoqW4gYwUADm8ESgAGYAThDcSE4gOBBVNIwQEGhEAJxkxUxwMAaM9IIwjAAKGvbaIKVYefw4IDRuxp5czTClfBClYPQywFhQyqol5ZWI0TS19SADYFBI-tFkIH0Dw6ORnpPTs-PxJhwA7nsYHhGLBgMs0Kt1pttjBdvsDtYyhUkHoanVENUFgkuIDYCCwVw0FgAPpBMIgZEnaoXU40a63RD3R7PQYjCIOD5TGZzQzuP74HD0EnARKIo4oxCPWlo7ECiBCkVecWU45IaUY6qs14c8afHk-flLDjEskUqnq84Ys5Pfpst6cibc758xacDiC4Wi8QAchUlHEQA
\begin{tikzcd}
                                                                                                    & S^{-1}R \arrow[rd, "\widetilde{\pi_1}"] &                                                                                                                  \\
\mathbb{Q}\times\mathbb{Q} \arrow[ru, "\iota_{S}"] \arrow[rr, "\pi_1"'] \arrow[rrdd, "\iota_{T}'"'] &                                         & \mathbb{Q} \arrow["\operatorname{id}"', loop, distance=2em, in=35, out=325] \arrow[dd, "\iota_{T}"', bend right] \\
                                                                                                    &                                         &                                                                                                                  \\
                                                                                                    &                                         & T^{-1}\mathbb{Q} \arrow[uu, "\widetilde{\operatorname{id}}"', bend right]                                       
\end{tikzcd}
      \end{center}
    \item We see that an element $\left( a,b \right)$ in $S^{-1}R$ is equivalent to $\left( 0,0 \right)$ in $S^{-1}R$ if and only if there is $ \left( r_1,r_2 \right)\in \left( \Q^{\times}\times \set{0}  \right) \cup \left( \Q^{\times}\times \set{1} \right) $ such that
      \begin{align*}
        \left( r_1a,r_2b \right) &= 0.
      \end{align*}
      Since we may select $r_2 = 0$ for all $b\in \Q$, it follows that we must have $a = 0$, so that the kernel of $\iota_S$ is $ \set{0}\times \Q $.
  \end{enumerate}
\end{solution}
\begin{problem}[Problem 7]
  Let $S\subseteq R$ be a multiplicative subset, and let $\iota_S\colon R\rightarrow S^{-1}R$ be the corresponding localization homomorphism. Consider the map
  \begin{align*}
    \alpha\colon \set{P'| P'\text{ is a prime ideal of }S^{-1}R}&\rightarrow \set{P | P\text{ is a prime ideal of }R\text{ such that }S\cap P = \emptyset}\\
    P'&\mapsto \iota_S^{-1}\left( P' \right).
  \end{align*}
  \begin{enumerate}[(a)]
    \item Verify that $\alpha$ is well-defined.
    \item Define an inverse map $\beta$ by $\beta(P) = P\cdot S^{-1}R$. Show that $\beta$ is well-defined. That is, $\beta(P)$ is a prime ideal of $S^{-1}R$.
    \item Show that $\alpha$ and $\beta$ are mutual inverses.
  \end{enumerate}
\end{problem}
\begin{solution}\hfill
  \begin{enumerate}[(a)]
    \item We observe that $\iota_S$ takes $1_{R}$ to $\frac{1}{1}\equiv 1_{S^{-1}R}$, the latter equality coming from the fact that $ \frac{a}{1} \cdot \frac{1}{1} = \frac{a}{1} $, so that if $P'$ is a prime ideal in $S^{-1}R$, then $\iota_S^{-1}\left( P' \right)$ is a prime ideal in $S^{-1}R$. Additionally, we observe that $\iota_S^{-1}\left( P' \right)$ does not contain any element of $S$, as otherwise $P'$ would contain an invertible element in $S^{-1}R$, and thus $P'$ would not be prime.
    \item Let $P$ be a prime ideal in $R$ such that $P\cap S = \emptyset$. Elements of $P\cdot S^{-1}R$ are of the form $q \cdot \frac{r}{t}$, where $q\in P$, $r\in R$, and $ t\in S $. Equivalently, we may write this element as $ \left( qr \right)\cdot \frac{1}{t} $, where $qr\in P$ and $ \frac{1}{t}\in S^{-1}R $. We observe that if $ \frac{a}{s}\cdot \frac{b}{t}\in P\cdot S^{-1}R $, then $ ab\in P $ and $ \frac{1}{st}\in S^{-1}R $, so that either $a\in P$ or $b\in P$, as $P$ is prime. Thus, since $P\cdot S^{-1}R$ is an ideal, we have $ \frac{a}{s}\in P\cdot S^{-1}R $ or $ \frac{b}{t}\in P\cdot S^{-1}R $.
    \item We will show that if $P'$ is a prime ideal in $S^{-1}R$, then $\iota_{S}^{-1}\left( P' \right)\cdot S^{-1}R = P'$. Let $a\cdot \frac{b}{s}\in \iota_{S}^{-1}\left( P' \right)\cdot S^{-1}R$, where $a\in \iota_{S}^{-1}\left( P' \right)$ and $\frac{b}{s}\in S^{-1}R$. We may write $\left( ab \right)\frac{1}{s}\in \iota_{S}^{-1}\left( P' \right)\cdot S^{-1}R$, meaning that $ab\in \iota_{S}^{-1}\left( P' \right)$, so that $\frac{ab}{1}\in P'$, meaning that $\frac{ab}{s}\in P'$, giving one direction of inclusion. The other direction of inclusion follows from the fact that if $\frac{a}{s}\in P'$, then $ \frac{a}{1}\in P' $, meaning $a\in \iota_{S}^{-1}\left( P' \right)$, and thus $\frac{a}{s}\in \iota_{S}^{-1}\left( P' \right)\cdot S^{-1}R$. This gives that $\beta\circ \alpha$ is identity on the set of prime ideals of $S^{-1}R$.\newline

      If $P$ is a prime ideal of $S^{-1}R$ such that $P\cap S = \emptyset$, and if $a\in P$, then $ a \cdot \frac{b}{s}\in P\cdot S^{-1}R $ for any $\frac{b}{s}\in S^{-1}R$. In particular, this holds for $ b = s = 1 $, meaning that $ \frac{a}{1}\in P\cdot S^{-1}R $, so that $ a\in \iota_{S}^{-1}\left( P\cdot S^{-1}R \right) $, so one inclusion holds. The other inclusion holds by the fact that if $ a\in \iota_{S}^{-1}\left( P\cdot S^{-1}R \right) $, then $\frac{a}{1}\in P\cdot S^{-1}R$, so that $ a\cdot \frac{1}{1}\in P\cdot S^{-1}R $, meaning that $ a\in P $. Thus, $\alpha$ and $\beta$ are mutual inverses.
  \end{enumerate}
\end{solution}
\end{document}
