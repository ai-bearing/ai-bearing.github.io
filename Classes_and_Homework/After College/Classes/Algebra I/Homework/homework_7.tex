\documentclass[10pt]{mypackage}

% sans serif font:
%\usepackage{cmbright}
%\usepackage{sfmath}
%\usepackage{bbold} %better blackboard bold

\usepackage{homework}
%\usepackage{notes}
\usepackage{newpxtext,eulerpx,eucal}
\renewcommand*{\mathbb}[1]{\varmathbb{#1}}

\fancyhf{}
\fancyhead[R]{Avinash Iyer}
\fancyhead[L]{Algebra I: Homework 7}
\fancyfoot[C]{\thepage}

\setcounter{secnumdepth}{0}

\begin{document}
\RaggedRight
\begin{problem}[Problem 1]\hfill
  \begin{enumerate}[(a)]
    \item Show that $\R$ is not a free $\Z$-module.
    \item Compute $\hom_{\Z}\left( \Q,\Z \right)$ and $\hom_{\Z}\left( \R,\Z \right)$.
  \end{enumerate}
\end{problem}
\begin{solution}\hfill
  \begin{enumerate}[(a)]
    \item Suppose toward contradiction that $\R$ were a free $\Z$-module. Then, there would be some unique $\Z$-linear combination
      \begin{align*}
        1 &= z_1b_1 + \cdots + z_nb_n,
      \end{align*}
      with $b_1,\dots,b_n\in B$, where $B$ is the basis for $\R$. We observe now that for any $k\in \Z_{ > 0 }$,
      \begin{align*}
        \frac{1}{k} &= z_1'b_1' + \cdots + z_m'b_m'
      \end{align*}
      for some other basis elements $b_1',\dots,b_m'\in B$ and integers $z_1',\dots,z_m'$. Suppose toward contradiction that there were some $b_i'$ such that $b_i'\notin \set{b_1,\dots,b_n}$. Then, we would have
      \begin{align*}
        1 &= k\left( z_1'b_1' + \cdots + z_m'b_m' \right)\\
          &= kz_1'b_1' + \cdots + kz_m'b_m',
      \end{align*}
      implying that $1$ has a non-unique expression of integral linear combinations of basis elements, contradicting the assumption that $\R$ is free over $\Z$.\newline

      There is some submodule $Y\supseteq \Q$ of $\R$ defined by $\Z \left\langle b_1,\dots,b_n \right\rangle$. The map
      \begin{align*}
        v\colon \Z^{n}&\rightarrow Y\\
        \left( z_1,\dots,z_n \right) &\mapsto z_1b_1 + \cdots + z_nb_n
      \end{align*}
      is thus an isomorphism, as it is injective by the assumption that $B$ is a basis and surjective by definition. Now, since $\Q\subseteq Y$ is a submodule, we observe that $v^{-1}\left( \Q \right)\subseteq \Z^{n}$ is a submodule, as for any $w_1,w_2\in v^{-1}\left( \Q \right)$, we have $v\left( w_1 \right),v\left( w_2 \right)\in \Q$, whence $v\left( w_1 + w_2 \right)\in \Q$, so that $w_1 + w_2\in v^{-1}\left( \Q \right)$, and $v\left( zw_1 \right) = zv\left( w_1 \right)\in \Q$ for any $z\in \Z$, whence $zw_1\in v^{-1}\left( \Q \right)$.\newline

      Now, since each $\Z$ is a PID (hence Noetherian), it follows that every $\Z$-submodule(/ideal) of $\Z^{n}$ is also finitely generated, as it is of the form $I_1\times\cdots\times I_n$ for ideals $I_1,\dots,I_n\in \Z$. Thus, it follows that $\Q\cong v^{-1}\left( \Q \right)$, whence $\Q$ is then isomorphic to a finitely generated $\Z$-module, which is a contradiction as it has been well-established that $\Q$ is not finitely generated as a $\Z$-module.
    \item We claim that both $\hom_{\Z}\left( \Q,\Z \right)$ and $\hom_{\Z}\left( \R,\Z \right)$ are zero. Toward this end, observe that
      \begin{align*}
        \varphi\left( \frac{a}{b} \right) &= k\varphi\left( \frac{a}{kb} \right)
      \end{align*}
      for all $\frac{a}{b}\in \Q$ with $\frac{a}{b}\neq 0$ and all $k\in \Z_{ > 0 }$. Yet, this can only be the case if $\varphi\left( \frac{a}{b} \right)= 0$, whence $\hom_{\Z}\left( \Q,\Z \right) \cong \set{0}$. Similarly, if $r\in \R$ is real with $r\neq 0$, then
      \begin{align*}
        \varphi\left( r \right) &= k\varphi\left( \frac{r}{k} \right),
      \end{align*}
      for all $k\in \Z_{ > 0 }$, so that $\varphi\left( r \right) = 0$, and thus $\hom_{\Z}\left( \R,\Z \right)\cong \set{0}$.
  \end{enumerate}
\end{solution}
\begin{problem}[Problem 2]
  Let $R$ be a commutative ring with $1$. Suppose there are integers $m_1$ and $m_2$ such that $R^{m_1}\cong R^{m_2}$. Prove that $m_1 = m_2$.
\end{problem}
\begin{solution}
  Let $I$ be a maximal ideal of $R$, and let $K = R/I$. We claim that if $M_1\cong M_2$ are isomorphic $R$-modules, then $M_1/IM_1 \cong M_2/IM_2$ are isomorphic as $R/I$-vector spaces. Toward this end, we let
  \begin{align*}
    \psi\colon M_1\rightarrow M_2/IM_2
  \end{align*}
  be a surjective homomorphism of $R$-modules defined by $M_1\xrightarrow{\varphi}M_2\xrightarrow{\pi} M_2/IM_2$, whence $\ker\left( \psi \right) = IM_1$, as
  \begin{align*}
    \psi\left( v_1 \right) &= 0 + IM_2
  \end{align*}
  if and only if $\varphi\left( v_1 \right) \in IM_2$, whence $\varphi\left( v_1 \right) = i \varphi\left( w_1 \right)$ with $i\in I$, or that $\varphi\left( iw_1 \right)\in IM_2$, so $iw_1\in IM_1$. The reverse inclusion follows from the first isomorphism theorem, as $IM_1\subseteq \ker\left( \psi \right)$ by observation. Thus, we have an isomorphism $ \overline{\psi}\colon M_1/IM_1\rightarrow M_2/IM_2 $.\newline

  We claim that the action
  \begin{align*}
    \left( r + I \right)\cdot \left( m + IM_1 \right) &= r\cdot m + IM_1
  \end{align*}
  is a well-defined action of $R/I$ on $M_1/IM_1$. Toward this end, we let $r_1 + I = r_2 + I$, whence $r_1-r_2\in I$. For any $m + IM_1\in M_1/IM_1$, we have (as the quotient module $M_1/IM_1$ is well-defined)
  \begin{align*}
    \left( r_1 + I \right)\cdot \left( m + IM_1 \right) &= r_1\cdot m + IM_1\\
                                                        &= \left( r_1 - r_2 + r_2 \right)\cdot m + IM_1\\
                                                        &= \left( \left( r_1 - r_2 \right)\cdot m + IM_1 \right) + \left( r_2\cdot m + IM_1 \right)\\
                                                        &= \left( 0 + IM_1 \right) + \left( r_2\cdot m + IM_1 \right)\\
                                                        &= r_2\cdot m + IM_1.
  \end{align*}
  The rest of the axioms for the action of $R/I$ on $M_1/IM_1$ follow from the axioms of $R$-modules.\newline

  Thus, it follows that if $R^{m_1}\cong R^{m_2}$, then we have
  \begin{align*}
    R^{m_1}/IR^{m_1} &\cong R^{m_2}/IR^{m_2}\\
    K^{m_1} &\cong K^{m_2},
  \end{align*}
  whence $m_1=m_2$ by the invariance of dimension for vector spaces.
\end{solution}
\begin{problem}[Problem 4]
  Let $R$ be a local ring with maximal ideal $I$.
  \begin{enumerate}[(a)]
    \item Show that if $M$ is a finitely generated module with $I\cdot M = M$, then $M = \set{0}$.
    \item If $M$ is a finitely generated $R$-module, and $y_1,\dots,y_m\in M$ are such that $ \overline{y_1},\dots, \overline{y_m}\in M/IM $ generate $M/IM$, then $y_1,\dots,y_m$ generate $M$.
  \end{enumerate}
\end{problem}
\begin{solution}\hfill
  \begin{enumerate}[(a)]
    \item Let $M = \left\langle x_1,\dots,x_n \right\rangle$, and suppose $IM = M$. Then, it follows that there are $v_1,\dots,v_n\in I$ such that
      \begin{align*}
        x_n &= v_1\cdot x_1 + \cdots + v_n\cdot x_n,
      \end{align*}
      whence
      \begin{align*}
        \left( 1 - v_n \right)\cdot x_n &= v_1\cdot x_1 + \cdots + v_{n-1}\cdot x_{n-1},
      \end{align*}
      whence, since $I$ is a local ring,
      \begin{align*}
        x_n &= \left( 1-v_n \right)^{-1}\left( v_1\cdot x_1 + \cdots + v_{n-1}\cdot x_{n-1} \right),
      \end{align*}
      meaning that $M = \left\langle x_1,\dots,x_{n-1} \right\rangle$. Inductively, any generating subset of $M$ can be reduced in this fashion until $M = \set{0}$.
    \item Let $N = \left\langle y_1,\dots,y_m \right\rangle$. We wish to show that
      \begin{align*}
        M &= N + IM.
      \end{align*}
      Toward this end, let $v\in M$. If $v\in IM$, then we are done. Else, if $v\notin IM$, it follows that $ v + IM \neq 0 + IM $, so there are $\alpha_1,\dots,\alpha_m$ such that
      \begin{align*}
        v + IM &= \alpha_1\cdot \left( y_1 + IM \right) + \cdots + \alpha_m\cdot \left( y_1 + IM \right)\\
               &= \left( \alpha_1\cdot y_1 + \cdots + \alpha_m\cdot y_m \right) + IM.
      \end{align*}
      In particular, this means there is some $q\in IM$ such that
      \begin{align*}
        v &= \left( \alpha_1\cdot y_1 + \cdots + \alpha_m\cdot y_m \right) + q,
      \end{align*}
      whence $M = N + IM$.\newline

      Consider the subspace $I\left( M/N \right)$ of $M/N$. We seek to show that $I\left( M/N \right) = M/N$. Let $v + N\in M/N$. Since $v\in M$, it follows that there are $r_1,\dots,r_n\in I$ and $q\in IM$ such that
      \begin{align*}
        v &= \sum_{i=1}^{n} r_i\cdot y_i + q.
      \end{align*}
      In particular, this means that $v + N = q + N$. Since $q + N = ip + N$ for some $p\in M$, we have $i\left( p + N \right) = v + N$, whence $I\left( M/N \right) = M/N$, meaning that by part (a), we have $M/N\cong \set{0}$, or that $M = N$. Thus, $y_1,\dots,y_n$ generate $N$.
  \end{enumerate}
\end{solution}
\begin{problem}[Problem 6]
  Let $R$ be a ring, $M$ an $R$-module, and let $S\subseteq R$ be multiplicative.
  \begin{enumerate}[(a)]
    \item Mimic the construction of the localization $S^{-1}R$ to define the localization $S^{-1}M$ making it into an $R$-module.
    \item Show that $S^{-1}M$ gains the structure of an $S^{-1}R$-module.
  \end{enumerate}
\end{problem}
\begin{solution}\hfill
  \begin{enumerate}[(a)]
    \item Let $ \overline{M} = M\times S $, and define a relation $\left( m_1,s_1 \right)\sim \left( m_2,s_2 \right)$ if there exists $s\in S$ such that $s\left( s_2m_1 - s_1m_2 \right) = 0$.\newline

      We claim that this is an equivalence relation.
      \begin{itemize}
        \item Reflexivity is clear from the fact that we may choose $s = 1$, whence $\left( m_1,s_1 \right)\sim \left( m_1,s_1 \right)$ if and only if $s_1m_1 - s_1m_1 = 0$.
        \item Symmetry follows from the fact that, if $\left( m_1,s_1 \right)\sim \left( m_2,s_2 \right)$, then
          \begin{align*}
            s\left( s_2m_1 - s_1m_2 \right) &= 0\\
                                            &= \left( -1 \right)\cdot 0\\
                                            &= \left( -1 \right)\left( s\left( s_2m_1 - s_1m_2 \right) \right)\\
                                            &= s\left( s_1m_2 - s_2m_1 \right),
          \end{align*}
          meaning that $\left( m_2,s_2 \right)\sim \left( m_1,s_1 \right)$.
        \item Finally, for transitivity, we let $\left( m_1,s_1 \right)\sim \left( m_2,s_2 \right)$ and $\left( m_2,s_2 \right)\sim \left( m_3,s_3 \right)$. Then,
          \begin{align*}
            s\left( s_2m_1 - s_1m_2 \right) &= 0\\
            t\left( s_3m_2 - s_2m_3 \right) &= 0.
          \end{align*}
          We seek to find $r\in S$ such that $r\left( s_3m_1 - s_1m_3 \right) = 0$. Toward this end, we multiply the first equation by $ts_3$ and the second equation by $ss_1$. This gives
          \begin{align*}
            sts_3\left( s_2m_1 - s_1m_2 \right) &= 0\\
            sts_1\left( s_3m_2 - s_2m_3 \right) &= 0.
          \end{align*}
          Distributing these sums out, we get
          \begin{align*}
            sts_2s_3m_1 - sts_1s_3m_2 &= 0\\
            sts_1s_3m_2 - sts_1s_2m_3 &= 0.
          \end{align*}
          Adding, we get
          \begin{align*}
            sts_2s_3m_1 - sts_2s_1m_3 &= 0,
          \end{align*}
          whence
          \begin{align*}
            sts_2\left( s_3m_1 - s_1m_3 \right) &= 0,
          \end{align*}
          and since $s,t,s_2\in S$, so too is $sts_2$, whence $\left( m_1,s_1 \right)\sim \left( m_3,s_3 \right)$.
      \end{itemize}
      We write $\frac{m}{s}\equiv \left[ \left( m,s \right) \right]$. We may define $R$-operations by taking
      \begin{align*}
        r\cdot \left( \frac{m}{s} \right) &= \frac{rm}{s}\\
        \frac{m_1}{s_1} + \frac{m_2}{s_2} &= \frac{s_2m_1 + s_1m_2}{s_1s_2}.
      \end{align*}
      We claim that both of these operations are well-defined. To start, if $\left( m_1,s_1 \right)\sim \left( m_2,s_2 \right)$, then
      \begin{align*}
        s\left( s_2m_1 - s_1m_2 \right) &= 0\\
        rs\left( s_2m_1 - s_1m_2 \right) &= 0\\
        s\left( s_2rm_1 - s_1rm_2 \right) &= 0,
      \end{align*}
      whence $\frac{rm_1}{s_1} = \frac{rm_2}{s_2}$.\newline

      Now, we observe that addition as defined is commutative, so we only need to check well-definedness in the case of one summand. Therefore, if $\left( m_1,s_1 \right)\sim \left( n_1,t_1 \right)$, we have some $v\in S$ such that
      \begin{align*}
        v\left( t_1m_1 - s_1n_1 \right) &= 0.
      \end{align*}
      We claim now that
      \begin{align*}
        \frac{s_2m_1 + s_1m_2}{s_1s_2} &= \frac{s_2n_1 + t_1m_2}{t_1s_2}.
      \end{align*}
      Indeed, we observe that
      \begin{align*}
        v\left( t_1s_2^2m_1 + t_1s_1s_2m_2 - s_1s_2^2n_1 - t_1s_1s_2m_2 \right) &= vs_2^2\left( t_1m_1 - s_1n_1 \right)\\
                                                                                &= s_2^2\left( v\left( t_1m_1 - s_1n_1 \right) \right)\\
                                                                                &= 0.
      \end{align*}
      Now, addition is associative, since
      \begin{align*}
        \frac{m_1}{s_1} + \left( \frac{m_2}{s_2} + \frac{m_3}{s_3} \right) &= \frac{m_1}{s_1} + \left( \frac{s_3m_2 + s_2m_3}{s_2s_3} \right)\\
                                                                           &= \frac{s_2s_3m_1 + s_1s_3m_2 + s_1s_2m_3}{s_1s_2s_3}\\
                                                                           &= \frac{s_2m_1 + s_1m_2}{s_1s_2} + \frac{m_3}{s_3}\\
                                                                           &= \left( \frac{m_1}{s_1} + \frac{m_2}{s_2} \right) + \frac{m_3}{s_3}.
      \end{align*}
      Furthermore, we observe that scalar multiplication comports with addition in both $R$ and $S^{-1}M$, as
      \begin{align*}
        \left( r_1 + r_2 \right)\frac{m}{s} &= \frac{\left( r_1 + r_2 \right)m}{s}\\
                                            &= \frac{r_1m + r_2m}{s}\\
                                            &= \frac{r_1m}{s} + \frac{r_2 m}{s}\\
                                            &= r_1 \frac{m}{s} + r_2 \frac{m}{s}\\
        r\left( \frac{m_1}{s_1} + \frac{m_2}{s_2} \right) &= r\left( \frac{s_2m_1 + s_1m_2}{s_1s_2} \right)\\
                                                          &= \frac{rs_2m_1 + rs_1m_2}{s_1s_2}\\
                                                          &= \frac{s_2rm_1 + s_1rm_2}{s_1s_2}\\
                                                          &= \frac{rm_1}{s_1} + \frac{rm_2}{s_2}\\
                                                          &= r\frac{m_1}{s_1} + r\frac{m_2}{s_2}.
      \end{align*}
      Finally, we observe that
      \begin{align*}
        \frac{m_1}{s_1} + \frac{0}{1} &= \frac{m_1}{s_1},
      \end{align*}
      whence $\frac{0}{1}$ is the additive identity in $S^{-1}M$, and
      \begin{align*}
        1\frac{m}{s} &= \frac{m}{s},
      \end{align*}
      whence $1\cdot v = v$ in $S^{-1}M$. Thus, we find that $S^{-1}M$ takes on a structure as an $R$-module.
    \item To extend the structure of $S^{-1}M$ to yield an $S^{-1}M$-module, we take the scalar multiplication
      \begin{align*}
        \frac{r}{s} \cdot \frac{m}{t} &= \frac{rm}{st}.
      \end{align*}
      Now, we observe that
      \begin{align*}
        \left( \frac{r_1}{s_1} + \frac{r_2}{s_2} \right) \frac{m}{t} &= \left( \frac{r_1s_2 + r_2s_1}{s_1s_2} \right)\frac{m}{t}\\
                                                                     &= \frac{\left( r_1s_2 + r_2s_1 \right)m}{s_1s_2 t}\\
                                                                     &= \frac{r_1s_2 m + r_2s_1 m}{s_1s_2 t}\\
                                                                     &= \frac{r_1 m}{s_1 t} + \frac{r_2 m}{s_2 t}\\
                                                                     &= \frac{r_1}{s_1} \frac{m}{t} + \frac{r_2}{s_2}\frac{m}{t}\\
        \frac{r}{s}\left( \frac{m_1}{t_1} + \frac{m_2}{t_2} \right) &= \frac{r}{s}\left( \frac{t_2m_1 + t_1m_2}{t_1t_2} \right)\\
                                                                    &= \frac{rt_2m_1 + rt_1m_2}{st_1t_2}\\
                                                                    &= \frac{rm_1}{st_1} + \frac{rm_2}{st_2}\\
                                                                    &= \frac{r}{s}\frac{m_1}{t_1} + \frac{r}{s}\frac{m_2}{t_2},
      \end{align*}
      meaning that scalar multiplication by elements of $S^{-1}R$ comports with addition in $S^{-1}M$ and vice versa. Finally, we also observe that
      \begin{align*}
        \frac{0}{1} \frac{m}{s} &= \frac{0}{s}\\
                                &= \frac{0}{1}\\
        \frac{1}{1} \frac{m}{s} &= \frac{m}{s}.
      \end{align*}
      Thus, $S^{-1}M$ takes on the structure of an $S^{-1}R$-module.
  \end{enumerate}
\end{solution}
\end{document}
