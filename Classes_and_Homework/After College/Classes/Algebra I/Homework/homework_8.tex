\documentclass[10pt]{mypackage}

% sans serif font:
%\usepackage{cmbright}
%\usepackage{sfmath}
%\usepackage{bbold} %better blackboard bold

\usepackage{homework}
%\usepackage{notes}
\usepackage{newpxtext,eulerpx,eucal}
\renewcommand*{\mathbb}[1]{\varmathbb{#1}}

\fancyhf{}
\fancyhead[R]{Avinash Iyer}
\fancyhead[L]{Algebra I: Homework 8}
\fancyfoot[C]{\thepage}

\setcounter{secnumdepth}{0}

\begin{document}
\RaggedRight
\begin{problem}[Problem 1]
  Show that a sequence of $R$-modules
  \begin{center}
    % https://tikzcd.yichuanshen.de/#N4Igdg9gJgpgziAXAbVABwnAlgFyxMJZABgBpiBdUkANwEMAbAVxiRGJAF9T1Nd9CKAIzkqtRizYAZLjxAZseAkQBMo6vWatEIALKzeigUQDM68VrYA5A-L5LByACznNknR05iYUAObwiUAAzACcIAFskMhAcCCQhbmCwyMQRGLjENQt3ECDbUIikLNikM2ztEF985NLqEsQnL04gA
\begin{tikzcd}
0 \arrow[r] & L \arrow[r, "f"] & M \arrow[r, "g"] & N
\end{tikzcd}
  \end{center}
  is exact if and only if the sequence
  \begin{center}
    % https://tikzcd.yichuanshen.de/#N4Igdg9gJgpgziAXAbVABwnAlgFyxMJZABgBpiBdUkANwEMAbAVxiRGJAF9T1Nd9CKAIzkqtRizYAdKQAsIAWwAUABVIAZAJRceIDNjwEiAJlHV6zVohAz5ytQFlt3XgYFEAzGfGXpcxaqkAHLOYjBQAObwRKAAZgBOikhkIDgQSEIuIAlJiCKp6YimPpLWsQB6wDJ0cDicOnGJCkjFaUheJVYgEZXVtfWcFJxAA
\begin{tikzcd}
0 \arrow[r] & {\hom(P,L)} \arrow[r, "f_{\ast}"] & {\hom(P,M)} \arrow[r, "g_{\ast}"] & {\hom(P,N)}
\end{tikzcd}
  \end{center}
  is exact.
\end{problem}
\begin{solution}
  Suppose that the sequence of $R$-modules
  \begin{center}
    % https://tikzcd.yichuanshen.de/#N4Igdg9gJgpgziAXAbVABwnAlgFyxMJZABgBpiBdUkANwEMAbAVxiRGJAF9T1Nd9CKAIzkqtRizYAZLjxAZseAkQBMo6vWatEIALKzeigUQDM68VrYA5A-L5LByACznNknR05iYUAObwiUAAzACcIAFskMhAcCCQhbmCwyMQRGLjENQt3ECDbUIikLNikM2ztEF985NLqEsQnL04gA
\begin{tikzcd}
0 \arrow[r] & L \arrow[r, "f"] & M \arrow[r, "g"] & N
\end{tikzcd}
  \end{center}
  is exact. That is, $f$ is injective, and $\img\left( f \right) = \ker\left( g \right)$. Now, let $\varphi\in \hom\left( P,L \right)$, and suppose $\varphi\in \ker\left( f_{\ast} \right)$. Then, it follows that $f_{\ast}\left( \varphi \right) \equiv 0$, whence for all $v\in P$, we have
  \begin{align*}
    f_{\ast}\left( \varphi \right)\left( v \right) &= f\left( \varphi\left( v \right) \right).
  \end{align*}
  for all $v\in L$. Yet, since $f$ has kernel equal to $0$, this means that $\varphi\left( v \right) = 0$ for all $v\in L$, so that $\varphi \equiv 0$.\newline

  Now, we consider the relationship between $\img\left( f_{\ast} \right)$ and $\ker\left( g_{\ast} \right)$. First, we observe that $g_{\ast}\circ f_{\ast}(\varphi) = \left( g\circ f \right)_{\ast}(\varphi)$, but since $\ker\left( g \right) = \img\left( f \right)$, it follows that $g\circ f$ is $0$, as the original sequence is exact. Therefore, $\img\left( f_{\ast} \right)\subseteq \ker\left( g_{\ast} \right)$. Now, suppose $\psi\in \ker\left( g_{\ast} \right)$. That is, for all $v\in P$, we have $g_{\ast}\left( \psi \right)\left( v \right) = 0$. In particular, this means that we have
  \begin{align*}
    g\left( \psi\left( v \right) \right) &= 0.
  \end{align*}
  It follows then that $\psi\left( v \right)\in \img\left( f \right)$, as we assume that the original sequence of $R$-modules is exact. In particular, there is some $w\in L$ such that $\psi(v) = f(w)$. Note that since $f$ is injective, such a $w$ is uniquely determined, whence the map $\tau\colon P\rightarrow L$ defined by $v\mapsto w$ is well-defined. In particular, we also have
  \begin{align*}
    f_{\ast}\left( \tau \right)(v) &= f\left( \tau\left( v \right) \right)\\
                                   &= f\left( w \right)\\
                                   &= \psi\left( v \right)
  \end{align*}
  for all $v\in P$, so that $f_{\ast}\left( \tau \right) = \psi$. In particular, this gives $\img\left( f_{\ast} \right) = \ker\left( g_{\ast} \right)$.\newline

  Now, let the hom sequence 
  \begin{center}
    % https://tikzcd.yichuanshen.de/#N4Igdg9gJgpgziAXAbVABwnAlgFyxMJZARgBoAGAXVJADcBDAGwFcYkQAdDgCwgFsAFAAVSAGQCUIAL6l0mXPkIpyFanSat25abJAZseAkQBMqmgxZtEnHv2GkAspJlyDiogGYz6y+y69BEQA5ZzUYKABzeCJQADMAJ34kMhAcCCRyFxAEpMQVVPTEUx9Na1iAfWAuejgcKR04xL4kYrSkLxKrEAjK6tr6qUopIA
\begin{tikzcd}
0 \arrow[r] & {\hom(P,L)} \arrow[r, "f_{\ast}"] & {\hom(P,M)} \arrow[r, "g_{\ast}"] & {\hom(P,N)}
\end{tikzcd}
  \end{center}
  be exact. Since the hom sequence is exact, it follows that $f_{\ast}\left( \varphi \right) = 0$ if and only if $\varphi = 0$. In particular, if $v\in P$, then $f\left( \varphi(v) \right) = 0$ if and only if $\varphi(v) = 0$, whence $\ker\left( f \right) = 0$. Thus, $f$ is injective.\newline

  We start by showing that $\img\left( f \right)\subseteq \ker\left( g \right)$. If $q\in \img\left( f \right)$, then there is some $r\in L$ such that $f\left( r \right) = q$. Now, let $P = \left\langle r \right\rangle$ be the $R$-module generated by $r$, and let $\varphi\colon P\hookrightarrow L$ be the inclusion of $P$ into $L$. Then, we observe that $f\left( \iota\left( r \right) \right) = q$, whence $g_{\ast}\circ f_{\ast}\left( \iota \right)(r) = 0 = g(q)$, by the exactness of the hom sequence.\newline

  Finally, let $q\in \ker\left( g \right)$. We observe that $\iota\colon \left\langle q \right\rangle\hookrightarrow M$ is an inclusion of $R$-modules, so that there is some $\varphi\colon \left\langle q \right\rangle\rightarrow L$ such that $f_{\ast}\left( \varphi \right) = \iota$. In particular, this means that, as $\varphi\left( q \right)\in L$, we have
  \begin{align*}
    f\left( \varphi\left( q \right) \right) &= q,
  \end{align*}
  whence $q\in \img\left( f \right)$. Thus, $\img\left( f \right) = \ker\left( g \right)$, so the original sequence of $R$-modules is exact.
\end{solution}
\begin{problem}[Problem 2]
  Let $R$ be a local ring with maximal ideal $M$. Show that every finitely generated projective $R$-module $P$ is free.
\end{problem}
\begin{solution}
  Let $P$ be a finitely generated projective module. We observe that $P/MP$ can be viewed as a vector space over $K \coloneq R/M$, and since $P$ is a finitely generated module, so too is $P/MP$, whence there is a basis $ \overline{x_1},\dots, \overline{x_n}$ for $P/MP$. By Nakayama's Lemma, it follows that $P = \left\langle x_1,\dots,x_n \right\rangle$ as an $R$-module.\newline

  In particular, we have the module homomorphism $f\colon F = R^{n}\rightarrow P$ given by
  \begin{align*}
    \left( a_1,\dots,a_n \right) &\mapsto a_1\cdot x_1 + \cdots + a_n\cdot x_n
  \end{align*}
  is surjective. Now, we see then that
  \begin{align*}
    P &\cong F/\ker\left( f \right)
  \end{align*}
  by the First Isomorphism Theorem. Since $P$ is projective, it follows that the sequence
  \begin{center}
    % https://tikzcd.yichuanshen.de/#N4Igdg9gJgpgziAXAbVABwnAlgFyxMJZABgBpiBdUkANwEMAbAVxiRGJAF9T1Nd9CKAIzkqtRizYBFLjxAZseAkQBMo6vWatEIAGKzeigUQDM68VrYAFA-L5LByACznNknR05iYUAObwiUAAzACcIAFskERAcCCQVbmCwyMQ1GLjEMwt3ECDbUIikLNikJ0Tc5KQydKjygpTijLSAIxgwKCLqhiwwbRAoOjgACx8QagY6VoYre2MdBhggnDHsvoBHLgpOIA
\begin{tikzcd}
0 \arrow[r] & \ker\left( f \right) \arrow[r] & F \arrow[r, "f"] & P \arrow[r] \arrow[l, "q", dashed, bend left] & 0
\end{tikzcd}
  \end{center}
  admits a section $q\colon P\rightarrow F$, whence $F\cong P\oplus Q$ where $Q = \ker\left( f \right)$.\newline

  Taking residues modulo $M$, we have that $F/MF\cong P/MP\cong Q/MQ$, and since $P/MP$ and $F/MF$ are finite-dimensional $K$-vector spaces with the same dimension, it follows that $Q/MQ \cong \set{0}$ by the invariance of dimension.\newline

  Finally, we observe that any element $v\in F$ has a decomposition $v = p + q$ for a unique $q\in Q$ and unique $p\in P$. In particular, we have the surjection $\pi\colon F\rightarrow Q$ given by $v\mapsto q$. In particular, we observe that $\ker\left( \pi \right) \cong P$, meaning that $Q = F/P$ is a quotient of two finitely generated modules, hence finitely generated. Nakayama's Lemma thus gives $Q = \set{0}$.
\end{solution}
\begin{problem}[Problem 8]\hfill
  \begin{enumerate}[(a)]
    \item Let $G$ be a group such that $\left\vert G \right\vert = p^{n}$ for some prime $p$ and some $n\geq 1$. Let $X$ be a finite $G$-set, and let $X^{G}$ be the set of all fixed points of the action. Show that $\left\vert X \right\vert = \left\vert X^{G} \right\vert$ mod $p$.
    \item Show that every group $G$ of order $p^2$ is abelian.
  \end{enumerate}
\end{problem}
\begin{solution}\hfill
  \begin{enumerate}[(a)]
    \item We observe that, by definition, $X^{G}$ is the set of all elements of $X$ with trivial orbit. That is, $X\setminus X^{G}$ consists of all the nontrivial orbits of $X$. Letting $x_1,\dots,x_\ell$ be representatives for each of these orbits, we observe that
      \begin{align*}
        \left\vert X \right\vert &= \left\vert X^{G} \right\vert + \sum_{k=1}^{\ell} \left\vert G\cdot x_k \right\vert.
      \end{align*}
      From the orbit--stabilizer theorem, it follows that
      \begin{align*}
        \left\vert X \right\vert &= \left\vert X^{G} \right\vert + \sum_{k=1}^{\ell} \left[ G:\stab_G\left(x_k\right) \right].
      \end{align*}
      Since each of the $G\cdot x_k$ are nontrivial orbits, it follows that $\left[ G:\stab_G\left( x_k \right) \right]\neq 1$, whence each index is a power of $p$. Thus, we obtain
      \begin{align*}
        \left\vert X \right\vert &= \left\vert X^{G} \right\vert\text{ modulo }p.
      \end{align*}
    \item Let $G$ act on itself via conjugation, so that $Z(G)$ is the set of fixed points under this action. Therefore, we get the equation
      \begin{align*}
        \left\vert G \right\vert &= \left\vert Z(G) \right\vert + \sum_{k=1}^{\ell} \left\vert G\cdot x_k \right\vert
      \end{align*}
      for some orbit representatives $x_1,\dots,x_{\ell}$. We claim that there is no nontrivial orbit.\newline

      First, we observe that $\left\vert Z(G) \right\vert \geq 1$ as $Z(G)$ is a subgroup and thus contains the identity element. Now, if $Z(G) = 1$, then the sum of the sizes of the orbits $\left\vert G\cdot x_k \right\vert$ is $p^2 - 1$, implying
      \begin{align*}
        \sum_{k=1}^{\ell} \left[ G : \stab_G\left( x_k \right) \right] &= p^2 - 1,
      \end{align*}
      but $p$ divides each nontrivial index, implying that $p\mid p^2 - 1$, which is a contradiction as $p^2 - 1$ and $p^2$ are coprime. Next, if $\left\vert Z(G) \right\vert = p$, we see that $G/Z(G)$ has order $p$, whence $G/Z(G)$ is cyclic, contradicting the result from Problem 7. Therefore, $\left\vert Z(G) \right\vert = p^2$, whence $ghg^{-1} = h$ for all $g\in G$ and all $h\in G$, or that $gh = hg$, so $G$ is abelian.
  \end{enumerate}
\end{solution}
\end{document}
