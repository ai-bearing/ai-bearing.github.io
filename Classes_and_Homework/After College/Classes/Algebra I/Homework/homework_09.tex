\documentclass[10pt]{mypackage}

% sans serif font:
%\usepackage{cmbright}
%\usepackage{sfmath}
%\usepackage{bbold} %better blackboard bold

\usepackage{homework}
%\usepackage{notes}
\usepackage{newpxtext,eulerpx,eucal}
\renewcommand*{\mathbb}[1]{\varmathbb{#1}}

\fancyhf{}
\fancyhead[R]{Avinash Iyer}
\fancyhead[L]{Algebra I: Homework 9}
\fancyfoot[C]{\thepage}

\setcounter{secnumdepth}{0}

\begin{document}
\RaggedRight
\begin{problem}[Problem 1]
  Let $F$ be a finite field with $q$ elements.
  \begin{enumerate}[(a)]
    \item Find the order of the general linear group
      \begin{align*}
        \operatorname{GL}_n\left( F \right) &= \set{A\in \Mat_n\left( F \right) | \det\left( A \right)\neq 0}.
      \end{align*}
    \item Find the order of the special linear group
      \begin{align*}
        \operatorname{SL}_n\left( F \right) &= \set{A\in \Mat_n\left( F \right) | \det\left( A \right) = 1}.
      \end{align*}
  \end{enumerate}
\end{problem}
\begin{solution}\hfill
  \begin{enumerate}[(a)]
    \item In order to find the order of the general linear group, we let $A\in \GL_n\left( F \right)$ be an arbitrary matrix. The first column of $A$ can consist $q^{n}-1$ nonzero vectors in $F^{n}$.\newline

      To determine the second column, we observe that it cannot be a nonzero element element of the $1$-dimensional linear subspace spanned by the first column of $A$. In particular, this means that there are $q^{n}-1 - \left( q-1 \right)$ such possible elements, or $q^{n}-q$. Inductively, we see that if we have determined the first $k-1$ columns, then the choices of column $k$ consist of all nonzero vectors in $F^{n}$ that are not of the form
      \begin{align*}
        v &= c_1 v_1 + \cdots + c_{k-1}v_{k-1}
      \end{align*}
      for $c_1,\dots,c_{k-1}\in F$; there are $q^{k-1}-1$ such nonzero elements of the linear subspace spanned by $v_1,\dots,v_{k-1}$, so that
      \begin{align*}
        \left\vert \GL_n\left( F \right) \right\vert &= \prod_{i=0}^{n-1} \left( q^{n}-q^{i} \right).
      \end{align*}
    \item By using the determinant homomorphism $\det\colon \GL_n\left( F \right)\rightarrow F$, we find that
      \begin{align*}
        \GL_n\left( F \right)/\SL_n\left( F \right) &\cong F^{\times},
      \end{align*}
      whence
      \begin{align*}
        \left\vert \GL_n\left( F \right) \right\vert &= \left( q-1 \right) \left\vert \SL_n\left( F \right) \right\vert,
      \end{align*}
      or
      \begin{align*}
        \left\vert \SL_n\left( F \right) \right\vert &= \frac{1}{q-1} \prod_{i=0}^{n-1} \left( q^{n}-q^{i} \right).
      \end{align*}
  \end{enumerate}
\end{solution}
\begin{problem}[Problem 2]
  Let $F$ be a field with $q$ elements, $V = F^{n}$ an $n$-dimensional $F$-vector space, and $m\leq n$. The purpose of this problem is to determine the cardinality of the set $T(m)$, the set of all $m$-dimensional subspaces $W$ of $V$.
  \begin{enumerate}[(a)]
    \item Show that the standard action of $G = \GL_n\left( F \right)$ on $V$ induces a natural action of $G$ on $T(m)$. Furthermore, show that this action is transitive.
    \item Let $W\in T(m)$ be the subspace spanned by the first $m$ elements of the standard basis $\set{e_1,\dots,e_n}$ of $V$. Identify explicitly the stabilizer $\Stab_G(W)$.
    \item Combine these facts with the formulas from Problem 1 (a) to determine $\left\vert T(m) \right\vert$.
  \end{enumerate}
\end{problem}
\begin{solution}\hfill
  \begin{enumerate}[(a)]
    \item We observe that $G$ acts on $V$ by mapping $0\neq v\mapsto Av\neq 0$ for $A\in G$. We can extend this to an action on a $m$-dimensional subspace with ordered basis $\left( v_1,\dots,v_m \right)$ by mapping
      \begin{align*}
        S\cdot \left( v_1,\dots,v_m \right) &= \left( Sv_1,\dots,Sv_m \right).
      \end{align*}
      We observe that $\id\cdot \left( v_1,\dots,v_m \right) = \left( v_1,\dots,v_m \right)$, and
      \begin{align*}
        S\cdot \left( T\cdot \left( v_1,\dots,v_m \right) \right) &= S\cdot \left( Tv_1,\dots,Tv_m \right)\\
                                                                  &= \left( STv_1,\dots,STv_m \right)\\
                                                                  &= ST\cdot \left( v_1,\dots,v_m \right).
      \end{align*}
      Finally, to see that this action is transitive, we observe that for any two ordered bases $\left( v_1,\dots,v_m \right)$ and $\left( w_1,\dots,w_m \right)$ that define elements of $T(m)$, each can be extended to bases for $V$, $\left( v_1,\dots,v_m,v_{m+1},\dots,v_{n} \right)$ and $\left( w_1,\dots,w_m,w_{m+1},\dots,w_n \right)$, and we can specify a linear map $T\colon V\rightarrow V$ taking $v_i\mapsto w_i$ for each $i$. This specifies an element of $\GL_n\left( F \right)$ by taking the matrix representation of this linear map. Therefore, the action is transitive.
    \item We observe that if $\left( e_1,\dots,e_m \right)$ is the basis for $W$, and $T\in \GL_n\left( F \right)$, then
      \begin{align*}
        Te_i &= c_1e_1 + \cdots + c_me_m + c_{m+1}e_{m+1} + \cdots + c_ne_n
      \end{align*}
      for some constants $c_1,\dots,c_n$. In order for $T$ to stabilize $W$, then we must have
      \begin{align*}
        Te_i = c_1e_1 + \cdots + c_me_m
      \end{align*}
      for each $i=1,\dots,m$. In particular, $T$ is an invertible block matrix of the form
      \begin{align*}
        T &= \begin{pmatrix}A & \ast \\ 0 & C\end{pmatrix},
      \end{align*}
      where $A\in \GL_m\left( F \right)$, $\ast$ is an arbitrary $m\times \left( n-m \right)$ matrix, and $C\in \GL_{n-m}\left( F \right)$.
    \item By the orbit-stabilizer theorem, and since the action of $\GL_n\left( F \right)$ on $T(m)$ is transitive, we know that
      \begin{align*}
        \left\vert T(m) \right\vert &= \left[ \GL_n\left( F \right) : \Stab_G(W) \right]\\
                                    &= \frac{\left\vert \GL_n\left( F \right) \right\vert}{\left\vert \Stab_G\left( W \right) \right\vert}.
      \end{align*}
      Our task now is to compute the order of the stabilizer. We observe that any element $T$ of $\stab_G\left( W \right)$ consists of
      \begin{itemize}
        \item an arbitrary $m\times \left( n-m \right)$ matrix over $F$;
        \item an element of $\GL_m\left( F \right)$;
        \item and an element of $\GL_{n-m}\left( F \right)$.
      \end{itemize}
      Therefore, we find that
      \begin{align*}
        \left\vert \stab_G\left( W \right) \right\vert &= \left( \prod_{i=0}^{m} \left( q^{m}-q^{i} \right) \right)\left( \prod_{i=0}^{n-m}\left( q^{n-m}-q^{i} \right) \right) q^{n\left( m-n \right)}.
      \end{align*}
      Thus,
      \begin{align*}
        \left\vert T(m) \right\vert &= \frac{\prod_{i=0}^{n}\left( q^{n}-q^{i} \right)}{\left( \prod_{i=0}^{m}\left( q^{m}-q^{i} \right) \right) \left( \prod_{i=0}^{n-m}\left( q^{n-m} -q^{i}\right) \right)\left( q^{m\left( n-m \right)} \right)}.
      \end{align*}
  \end{enumerate}
\end{solution}
\begin{problem}[Problem 6]
  Suppose a finite group $G$ acts on a finite set $X$. For $g\in G$, let $X^{g}$ be the set of all $x\in X$ that are fixed by $g$. Prove that
  \begin{align*}
    \left\vert G\cdot X \right\vert &= \frac{1}{\left\vert G \right\vert} \sum_{g\in G} \left\vert X^{g} \right\vert>
  \end{align*}
  That is, the number of orbits equals the ``average'' number of fixed points of elements of $G$.
\end{problem}
\begin{solution}
  We start by showing that
  \begin{align*}
    \left\vert G\cdot X \right\vert &= \frac{1}{\left\vert G \right\vert} \sum_{x\in X} \left\vert \Stab_G(x) \right\vert.
  \end{align*}
  From the orbit-stabilizer theorem, we know that
  \begin{align*}
    \left\vert \Stab_G\left( x \right) \right\vert &= \frac{\left\vert G \right\vert}{\left\vert G\cdot x \right\vert}
  \end{align*}
  for each $x\in X$. Therefore, we observe that
  \begin{align*}
    \sum_{x\in X}\left\vert \Stab_G\left( x \right) \right\vert &= \left\vert G \right\vert\sum_{x\in X} \frac{1}{\left\vert G\cdot x \right\vert}.
  \end{align*}
  Since the orbits partition $X$, we observe that we may split
  \begin{align*}
    \sum_{x\in X} \frac{1}{\left\vert G\cdot x \right\vert} &= \sum_{i=1}^{r} \left\vert G\cdot x_i \right\vert \frac{1}{\left\vert G\cdot x_i \right\vert},
  \end{align*}
  since for any $x\in X$ with $x\in G\cdot x_i$, there are $\left\vert G\cdot x_i \right\vert$ total elements in the same orbit, whence
  \begin{align*}
    \sum_{x\in X} \left\vert \Stab_G\left( x \right) \right\vert &= \left\vert G \right\vert \sum_{i=1}^{r} 1\\
                                                                 &= \left\vert G \right\vert\left\vert G\cdot X \right\vert.
  \end{align*}
  Now, let $Y = \set{\left( g,x \right)\in G\times X | g\cdot x = x}$. Letting $\pi_1$ and $\pi_2$ be the projections on the first and second coordinate, we observe that for a specific $g_0$ and $x_0$, we have
  \begin{align*}
    \pi_1^{-1}\left( \set{g_0} \right) &= \set{\left( g_0,x \right)\in Y | g_0\cdot x = x}\\
    \pi_2^{-2}\left( \set{x_0} \right) &= \set{\left( g,x_0 \right)\in Y | g\cdot x_0 = x_0}.
  \end{align*}
  In particular, we have
  \begin{align*}
    \left\vert \pi_1^{-1}\left( \set{g_0} \right) \right\vert &= \left\vert X^{g_0} \right\vert\\
    \left\vert \pi_2^{-1}\left( \set{x_0} \right) \right\vert &= \left\vert \Stab_{G}\left( x_0 \right) \right\vert,
  \end{align*}
  and
  \begin{align*}
    Y &= \bigsqcup_{g\in G} \pi_1^{-1}\left( \set{g} \right)\\
      &= \bigsqcup_{x\in X} \pi_2^{-1} \left( \set{x} \right),
  \end{align*}
  whence
  \begin{align*}
    \left\vert Y \right\vert &= \sum_{g\in G} \left\vert X^{g} \right\vert\\
                             &= \sum_{x\in X} \left\vert \Stab_G\left( x \right) \right\vert\\
                             &= \left\vert G \right\vert\left\vert G\cdot X \right\vert.
  \end{align*}
  Thus,
  \begin{align*}
    \left\vert G\cdot X \right\vert &= \frac{1}{\left\vert G \right\vert} \sum_{g\in G} \left\vert X^{g} \right\vert.
  \end{align*}
\end{solution}
\begin{problem}[Problem 8]
  Recall that the symmetric group $S_3$ consists of the following permutations: $e$, the transpositions $\left( 1,2 \right)$, $\left( 1,3 \right)$, and $\left( 2,3 \right)$, and the two $3$-cycles $\left( 1,2,3 \right)$ and $\left( 1,3,2 \right)$. Also, recall that every $\sigma\in S_3$ can be written as a product of transpositions.
  \begin{enumerate}[(a)]
    \item Show that the center of $S_3$ is trivial, and hence $\operatorname{inn}\left( S_3 \right) \cong S_3$.
    \item Show that $\aut\left( G \right)\cong G$, and hence every automorphism of $G$ is inner.
  \end{enumerate}
\end{problem}
\end{document}
