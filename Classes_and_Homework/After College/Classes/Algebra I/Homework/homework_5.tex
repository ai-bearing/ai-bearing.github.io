\documentclass[10pt]{mypackage}

% sans serif font:
%\usepackage{cmbright}
%\usepackage{sfmath}
%\usepackage{bbold} %better blackboard bold

\usepackage{homework}
%\usepackage{notes}
\usepackage{newpxtext,eulerpx,eucal}
\renewcommand*{\mathbb}[1]{\varmathbb{#1}}

\fancyhf{}
\rhead{Avinash Iyer}
\lhead{Algebra I: Homework 5}

\setcounter{secnumdepth}{0}

\begin{document}
\RaggedRight
\begin{problem}[Problem 1]
  Let $R$ be a Euclidean domain with norm $N$, and let
  \begin{align*}
    m &= \min\set{N(x) | x\in R\setminus \set{0}}.
  \end{align*}
  Show that any $u\in R\setminus \set{0}$ satisfying $N(u) = m$ is invertible.
\end{problem}
\begin{solution}
  Let $u$ satisfy $N(u) = m$. Applying the division algorithm, we find that
  \begin{align*}
    1 &= uq + r,
  \end{align*}
  where $r = 0$ or $N(r) < N(u)$. In the former case, we find that $q = u^{-1}$, while the latter case violates the assumption that $N(u)$ is of minimal value.
\end{solution}
\begin{problem}[Problem 2]
  Show that in a UFD every irreducible element is prime. Conclude that if $R$ is a Noetherian domain, then $R$ is a UFD if and only if every irreducible element is prime.
\end{problem}
\begin{solution}
  Let $R$ be a UFD, and let $h$ be an irreducible element such that $h | ab$ for some $a,b\in R$.\newline

  Write the unique (up to associates) factorizations into irreducibles for $a$ and $b$, giving
  \begin{align*}
    a &= a_1a_2\cdots a_r\\
    b &= b_1b_2\cdots b_s.
  \end{align*}
  Therefore, for some $k\in R$, we have
  \begin{align*}
    hk &= \left( a_1a_2\cdots a_r \right) \left( b_1b_2\cdots b_s \right).
  \end{align*}
  Since $h$ is irreducible, and the factorizations for $a$ and $b$ are unique up to associates, there is some $u_j\in R^{\times}$ such that $h=u_ja_j$ or some $v_k\in R^{\times}$ such that $h = v_kb_k$ (else we would have a different factorization for $ab$ into irreducibles). Thus, $h | a$ or $h | b$ depending on which of these hold, so that $h$ is prime.\newline

  Since we already know that primes are irreducible, it follows that, in a Noetherian domain, since every element has at least one factorization into irreducibles, such a factorization is unique if and only if every irreducible element is prime.
\end{solution}
\begin{problem}[Problem 4]
  Let $R$ be a domain in which every prime ideal is principal. Show that $R$ is a PID by using the following suggestions.
  \begin{enumerate}[(i)]
    \item Assume that the set of nonprincipal ideals is nonempty. Then, use Zorn's Lemma to find a maximal element $I$ in it.
    \item Since $I$ is not prime, there exist $a,b\in R$ such that $ab\in I$ but $a,b\notin I$. Let $I_a = I + \left( a \right)$, and let $J$ be defined by
      \begin{align*}
        J &= \set{x\in R | xI_{a}\subseteq I}.
      \end{align*}
      Verify that $J$ is an ideal of $R$. Deduce a contradiction by showing that $I = I_aJ$.
  \end{enumerate}
\end{problem}
\begin{solution}
  Let $ \mathcal{X} $ be the set of all nonprincipal ideals of $R$, ordered by inclusion. Suppose toward contradiction that $ \mathcal{X} $ were nonempty. Let $ \set{K_{\alpha}}_{\alpha\in A} = \mathcal{C}\subseteq \mathcal{X} $ be a chain in $ \mathcal{X} $, and let $I = \bigcup_{\alpha\in A}K_{\alpha}$, which is an upper bound for $ \mathcal{C} $. We claim that $I$ is nonprincipal.\newline

  Suppose not. Then, $I = \left( v \right)$ for some $v\in R$; since $v\in I$, it follows that $v\in K_{\alpha}$ for some $\alpha\in A$, meaning that $ \left( v \right)\subseteq K_{\alpha} $, or that $ K_{\alpha} = I = \left( v \right) $, which would contradict the assumption that $K_{\alpha}$ is nonprincipal.\newline

  Since $I$ is nonprincipal, $I$ is not prime, so there exists some $ab\in I$ with $a\notin I$ and $b\notin I$. Letting $I_a = I + \left( a \right)$, since $ I\subsetneq I_a $, we must $I_a = \left( u \right)$ for some $u\in R$.\newline

  Let
  \begin{align*}
    J &= \set{x\in R | x\left( I + \left( a \right) \right)\subseteq I}.
  \end{align*}
  Observe that $J$ is closed under subtraction, since if $x,y\in J$, we have
  \begin{align*}
    \left( x-y \right)\left( I + \left( a \right) \right) &= x\left( I + \left( a \right) \right) - y\left( I + \left( a \right) \right)\\
                                                          &\subseteq I,
  \end{align*}
  since $I$ is closed under subtraction. Similarly, if $r\in R$, then
  \begin{align*}
    rx\left( I + \left( a \right) \right) &= r\left( x \left( I + \left( a \right) \right) \right)\\
                                          &\subseteq I,
  \end{align*}
  since $I$ is closed under multiplication by elements from $R$. Thus, $J$ is an ideal. In particular, since $J$ contains $I$ and $b\notin I$, $J$ must be a principal ideal of the form $\left( v \right)$, so that $I_aJ = \left( uv \right)$ is principal as well.\newline

  Now, we observe that elements of $I_aJ$ are of the form
  \begin{align*}
    \sum_{k=1}^{n} \left( x_k + r_ka \right)\left( s_k v \right) &= \sum_{k=1}^{n} x_k\left( s_k v \right) + s_kv\left( r_ka \right)\\
                                                                 &\in I,
  \end{align*}
  so that $I_aJ\subseteq I$.\newline

  If $x\in I$, then since $x\in I_a$, and $I_a = \left( u \right)$, it follows that $x = \ell u$ for some $\ell \in R$. Additionally, since $rx\in I$ for arbitrary $r\in R$, it follows that $ r\ell u = \ell ru \in I $, meaning that $ \ell \left( u \right) \subseteq I$, meaning that $\ell \in J$. Thus, $x\in I_aJ$, implying that $I = I_aJ$, meaning $I$ is principal, which is a contradiction of the fact that $I$ is (allegedly) not principal.
\end{solution}
\begin{problem}[Problem 5]
  Consider the following factorization into irreducibles in the ring $R = \Z\left[ \sqrt{-5} \right]$:
  \begin{align*}
    2\cdot 3 &= \left( 1 + \sqrt{-5} \right)\left( 1 - \sqrt{-5} \right).
  \end{align*}
  \begin{enumerate}[9a]
    \item Show that the elements $2,3,1\pm \sqrt{-5}$ are irreducible but not prime.
    \item Next, consider the following 4 ideals:
      \begin{align*}
        P_1 &= \left( 2,1+\sqrt{-5} \right)\\
        P_2 &= \left( 2,1-\sqrt{-5} \right)\\
        P_3 &= \left( 3,1+\sqrt{-5} \right)\\
        P_4 &= \left( 3,1-\sqrt{-5} \right).
      \end{align*}
      Show that all of $P_1,P_2,P_3,P_4$ are prime ideals.
    \item Show that $P_1P_3 = \left( 1 + \sqrt{-5} \right)$, $P_2P_4 = \left( 1-\sqrt{-5} \right)$, $P_3P_4 = \left( 3 \right)$, and $P_1 = P_2$.
  \end{enumerate}
\end{problem}
\begin{solution}\hfill
  \begin{enumerate}[(a)]
    \item We consider the norm $N\colon \Z\left[ \sqrt{-5} \right]\rightarrow \N$, given by $a + b\sqrt{-5}\mapsto a^2 + 5b^2$. This norm is multiplicative, so we may establish the irreducibility of the elements $2,3,1\pm\sqrt{-5}$ using this norm. If there were a factorization of $2$ into non-units $ab$, then
      \begin{align*}
        4 &= N(2)\\
          &= N(a)N(b),
      \end{align*}
      implying that $N(a)=N(b)=2$ (as elements have norm $1$ if and only if they are units). Yet, there are no $x,y\in \Z$ such that $x^2 + 5y^2 = 2$, as we would have $2 = x^2$ modulo $5$, but the only squares in $\Z/5\Z$ are $1$ and $4$.\newline

      Similarly, if there were a factorization of $3$ into non-units $ab$, then
      \begin{align*}
        9 &= N(a)N(b),
      \end{align*}
      meaning that $N(a) = N(b) = 3$, so by a similar reasoning, if $x^2 + 5y^2 = 3$, then $x^2 = 3$ modulo $5$, which cannot happen by a similar reasoning.\newline

      Finally, if there were a factorization of $1\pm \sqrt{-5}$ into non-units $ab$, then
      \begin{align*}
        6 &= N(a)N(b),
      \end{align*}
      meaning $N(a) = 2$ and $N(b) = 3$ or vice versa. By similar reasoning, this cannot happen.\newline

      Additionally, observe that the units of $\Z\left[ \sqrt{-5} \right]$ are $\pm 1$, as $x^2 + 5y^2 = 1$ for $x,y\in \Z$ if and only if $x = \pm 1$.\newline

      Now, to see that $2, 3, 1\pm\sqrt{-5}$ are not prime, observe that $2 | 6 = \left( 1+\sqrt{-5} \right)\left( 1-\sqrt{-5} \right)$, but $2$ does not divide either $1\pm\sqrt{-5}$, as we have just established that they are irreducible, and similarly for $3$ and vice versa.
    \item By the third isomorphism theorem, we see that
      \begin{align*}
        \Z\left[ \sqrt{-5} \right]/P_1 &\cong \frac{\Z\left[ \sqrt{-5} \right]/\left( 2 \right)}{ P_1/\left( 2 \right) }.
      \end{align*}
      Focusing our attention on $ \Z\left[ \sqrt{-5} \right]/\left( 2 \right) $, we observe that an arbitrary $a + b\sqrt{-5}\in \Z\left[ \sqrt{-5} \right]$ has the pair $\left( a,b \right)$ satisfying either $\left( 0,0 \right)$, $\left( 0,1 \right)$, $\left( 1,0 \right)$, or $\left( 1,1 \right)$ modulo $2$, meaning that $\Z\left[ \sqrt{-5} \right]/\left( 2 \right)$ is an abelian group with four elements, none of which has order greater than $2$. Hence, as abelian groups, we have
      \begin{align*}
        \Z\left[ \sqrt{-5} \right]/\left( 2 \right)\cong \Z/2\Z\times \Z/2\Z.
      \end{align*}
      Additionally, we see that $P_1/\left( 2 \right)$ has elements of the form $a + a\sqrt{-5}$, where $a\equiv 0$ or $1$ modulo $2$, hence $P_1/\left( 2 \right)$ is isomorphic to $\Z/2\Z$ (once again as abelian groups). By the third isomorphism theorem, it then follows that $\Z\left[ \sqrt{-5} \right]/P_1\cong \Z/2\Z$ as abelian groups. Yet, since $\Z/2\Z$ is a field, it also follows that $P_1$ is maximal, and thus prime.\newline

      Similarly, since
      \begin{align*}
        \Z\left[ \sqrt{-5} \right]/P_2 &\cong \frac{\Z\left[ \sqrt{-5} \right]/\left( 2 \right)}{ P_2/\left( 2 \right) },
      \end{align*}
      we may use the same process as we showed for $P_1$, but with $a-a\sqrt{-5}$ instead of $a+a\sqrt{-5}$, to show that $\Z\left[ \sqrt{-5} \right]/P_2\cong \Z/2\Z$.\newline

      Concerning $P_3$ and $P_4$, we use a similar process but with $\left( 3 \right)$ replacing $\left( 2 \right)$. We see then that arbitrary $a + b\sqrt{-5}\in \Z\left[ \sqrt{-5} \right]$ has $\left( a,b \right)$ modulo $3$ isomorphic to $\Z/3\Z\times \Z/3\Z$ as abelian groups; Since $a \pm a\sqrt{-5}\in \Z\left[ \sqrt{-5} \right]$ has $a\equiv 0,1,2$ modulo $3$, it follows that $\left( 3, 1 \pm \sqrt{-5} \right)/\left( 3 \right)\cong \Z/3\Z$ as abelian groups, meaning that $\Z\left[ \sqrt{-5} \right]/P_3\cong \Z/3\Z$ and $\Z\left[ \sqrt{-5} \right]/P_4\cong \Z/3\Z$, so that both $P_3$ and $P_4$ are maximal, hence prime.
    \item First, we observe that $2 + \left( -1 \right)\left( 1+\sqrt{-5} \right)= 1-\sqrt{-5}$, meaning that the generators of $P_1$ are contained in $P_2$ and vice versa. Thus, $P_1 = P_2$.\newline

      Next, by taking products of ideals, we see that
      \begin{align*}
        P_1P_3 &= \left( 6,3+3\sqrt{-5},2+2\sqrt{-5},-4+2\sqrt{-5} \right)\\
        P_2P_4 &= \left( 6,3-3\sqrt{-5},2-2\sqrt{-5},-4-2\sqrt{-5} \right)\\
        P_3P_4 &= \left( 9,6,3+3\sqrt{-5},3-3\sqrt{-5} \right).
      \end{align*}
      Immediately, we see that $3\in P_3P_4$, and that we may write $3 = 9-6$, so that $P_3P_4 = \left( 3 \right)$.\newline

      Concerning $P_1P_3$ and $P_2P_4$, we see that $1 + \sqrt{-5}\in P_1P_3$ by taking $3 + 3\sqrt{-5} + \left( -1 \right)\left( 2 + 2\sqrt{-5} \right)$, while the generators of $P_1P_3$ can be found by evaluating
      \begin{align*}
        6 &= \left( 1+\sqrt{-5} \right)\left( 1-\sqrt{-5} \right)\\
        3 + 3\sqrt{-5} &= \left( 1 + \sqrt{-5} \right)\left( 3 \right)\\
        2 + 2\sqrt{-5} &= \left( 1 + \sqrt{-5} \right)\left( 2 \right)\\
        -4 + 2\sqrt{-5} &= \left( 1+\sqrt{-5} \right)\left( 1 + \sqrt{-5} \right).
      \end{align*}
      Thus, $P_1P_3 = \left( 1 + \sqrt{-5} \right)$.\newline

      Similarly, we may find that $1 - \sqrt{-5}\in P_2P_4$ by taking $ 3 - 3\sqrt{-5} + \left( -1 \right)\left( 2-2\sqrt{-5} \right) $, while the generators of $P_2P_4$ can be found by evaluating
      \begin{align*}
        6 &= \left( 1-\sqrt{-5} \right)\left( 1 + \sqrt{-5} \right)\\
        3 - 3\sqrt{-5} &= \left( 1-\sqrt{-5} \right)\left( 3 \right)\\
        2-2\sqrt{-5} &= \left( 1-\sqrt{-5} \right)\left( 2 \right)\\
        -4-2\sqrt{-5} &= \left( 1-\sqrt{-5} \right)\left( 1-\sqrt{-5} \right).
      \end{align*}
      Thus, $P_2P_4 = \left( 1-\sqrt{-5} \right)$.
  \end{enumerate}
\end{solution}
\end{document}
