\documentclass[10pt]{mypackage}

% sans serif font:
%\usepackage{cmbright}
%\usepackage{sfmath}
%\usepackage{bbold} %better blackboard bold

\usepackage{homework}
%\usepackage{notes}
\usepackage{newpxtext,eulerpx,eucal}
\renewcommand*{\mathbb}[1]{\varmathbb{#1}}

\fancyhf{}
\rhead{Avinash Iyer}
\lhead{Algebra I: Homework 5}

\setcounter{secnumdepth}{0}

\begin{document}
\RaggedRight
\begin{problem}[Problem 1]
  Let $R$ be a Euclidean domain with norm $N$, and let
  \begin{align*}
    m &= \min\set{N(x) | x\in R\setminus \set{0}}.
  \end{align*}
  Show that any $u\in R\setminus \set{0}$ satisfying $N(u) = m$ is invertible.
\end{problem}
\begin{solution}
  Let $u$ satisfy $N(u) = m$. Applying the division algorithm, we find that
  \begin{align*}
    1 &= uq + r,
  \end{align*}
  where $r = 0$ or $N(r) < N(u)$. In the former case, we find that $q = u^{-1}$, while the latter case violates the assumption that $N(u)$ is of minimal value.
\end{solution}
\begin{problem}[Problem 2]
  Show that in a UFD every irreducible element is prime. Conclude that if $R$ is a Noetherian domain, then $R$ is a UFD if and only if every irreducible element is prime.
\end{problem}
\begin{solution}
  Let $R$ be a UFD, and let $h$ be an irreducible element such that $h | ab$ for some $a,b\in R$.\newline

  Write the unique (up to associates) factorizations into irreducibles for $a$ and $b$, giving
  \begin{align*}
    a &= a_1a_2\cdots a_r\\
    b &= b_1b_2\cdots b_s.
  \end{align*}
  Therefore, for some $k\in R$, we have
  \begin{align*}
    hk &= \left( a_1a_2\cdots a_r \right) \left( b_1b_2\cdots b_s \right).
  \end{align*}
  Since $h$ is irreducible, and the factorizations for $a$ and $b$ are unique up to associates, there is some $u_j\in R^{\times}$ such that $h=u_ja_j$ or some $v_k\in R^{\times}$ such that $h = v_kb_k$ (else we would have a different factorization for $ab$ into irreducibles). Thus, $h | a$ or $h | b$ depending on which of these hold, so that $h$ is prime.\newline

  Since we already know that primes are irreducible, it follows that, in a Noetherian domain, since every element has at least one factorization into irreducibles, such a factorization is unique if and only if every irreducible element is prime.
\end{solution}
\begin{problem}[Problem 4]
  Let $R$ be a domain in which every prime ideal is principal. Show that $R$ is a PID by using the following suggestions.
  \begin{enumerate}[(i)]
    \item Assume that the set of nonprincipal ideals is nonempty. Then, use Zorn's Lemma to find a maximal element $I$ in it.
    \item Since $I$ is not prime, there exist $a,b\in R$ such that $ab\in I$ but $a,b\notin I$. Let $I_a = I + \left( a \right)$, and let $J$ be defined by
      \begin{align*}
        J &= \set{x\in R | xI_{a}\subseteq I}.
      \end{align*}
      Verify that $J$ is an ideal of $R$. Deduce a contradiction by showing that $I = I_aJ$.
  \end{enumerate}
\end{problem}
\begin{solution}
  Let $ \mathcal{X} $ be the set of all nonprincipal ideals of $R$, ordered by inclusion. Suppose toward contradiction that $ \mathcal{X} $ were nonempty. Let $ \set{K_{\alpha}}_{\alpha\in A} = \mathcal{C}\subseteq \mathcal{X} $ be a chain in $ \mathcal{X} $, and let $I = \bigcup_{\alpha\in A}K_{\alpha}$, which is an upper bound for $ \mathcal{C} $. We claim that $I$ is nonprincipal.\newline

  Suppose not. Then, $I = \left( v \right)$ for some $v\in R$; since $v\in I$, it follows that $v\in K_{\alpha}$ for some $\alpha\in A$, meaning that $ \left( v \right)\subseteq K_{\alpha} $, or that $ K_{\alpha} = I = \left( v \right) $, which would contradict the assumption that $K_{\alpha}$ is nonprincipal.\newline

  Since $I$ is nonprincipal, $I$ is not prime, so there exists some $ab\in I$ with $a\notin I$ and $b\notin I$. Letting $I_a = I + \left( a \right)$, since $ I\subsetneq I_a $, we must $I_a = \left( u \right)$ for some $u\in R$.\newline

  Let
  \begin{align*}
    J &= \set{x\in R | x\left( I + \left( a \right) \right)\subseteq I}.
  \end{align*}
  Observe that $J$ is closed under subtraction, since if $x,y\in J$, we have
  \begin{align*}
    \left( x-y \right)\left( I + \left( a \right) \right) &= x\left( I + \left( a \right) \right) - y\left( I + \left( a \right) \right)\\
                                                          &\subseteq I,
  \end{align*}
  since $I$ is closed under subtraction. Similarly, if $r\in R$, then
  \begin{align*}
    rx\left( I + \left( a \right) \right) &= r\left( x \left( I + \left( a \right) \right) \right)\\
                                          &\subseteq I,
  \end{align*}
  since $I$ is closed under multiplication by elements from $R$. Thus, $J$ is an ideal. In particular, since $J$ contains $I$ and $b\notin I$, $J$ must be a principal ideal of the form $\left( v \right)$, so that $I_aJ = \left( uv \right)$ is principal as well.\newline

  Now, we observe that elements of $I_aJ$ are of the form
  \begin{align*}
    \sum_{k=1}^{n} \left( x_k + r_ka \right)\left( s_k v \right) &= \sum_{k=1}^{n} x_k\left( s_k v \right) + s_kv\left( r_ka \right)\\
                                                                 &\in I,
  \end{align*}
  so that $I_aJ\subseteq I$.\newline

  If $x\in I$, then since $x\in I_a$, and $I_a = \left( u \right)$, it follows that $x = \ell u$ for some $\ell \in R$. Additionally, since $rx\in I$ for arbitrary $r\in R$, it follows that $ r\ell u = \ell ru \in I $, meaning that $ \ell \left( u \right) \subseteq I$, meaning that $\ell \in J$. Thus, $x\in I_aJ$, implying that $I = I_aJ$, meaning $I$ is principal, which is a contradiction of the fact that $I$ is (allegedly) not principal.
\end{solution}
\end{document}
