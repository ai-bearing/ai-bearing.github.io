\documentclass[10pt]{mypackage}

% sans serif font:
%\usepackage{cmbright}
%\usepackage{sfmath}
%\usepackage{bbold} %better blackboard bold

\usepackage{homework}
%\usepackage{notes}
\usepackage{newpxtext,eulerpx,eucal}
\renewcommand*{\mathbb}[1]{\varmathbb{#1}}

\fancyhf{}
\rhead{Avinash Iyer}
\lhead{Algebra I: Assignment 4}

\setcounter{secnumdepth}{0}

\begin{document}
\RaggedRight
\begin{problem}[Problem 1]
  Let $I,J,K$ be ideals of $R$.
  \begin{enumerate}[(a)]
    \item Show that $\left( IJ \right)K = I\left( JK \right)$.
    \item Show that $\left( I + J \right)K = IK + JK$.
  \end{enumerate}
\end{problem}

\begin{problem}[Problem 4]
  Let $S_1\subseteq S_2$ be multiplicative subsets of $R$, and let $\iota_{S_i}\colon R\rightarrow S_i^{-1}R$ be the corresponding localization homomorphisms. Use the universal property of localization to show that there exists a unique ring homomorphism $\iota'\colon S_1^{-1}R\rightarrow S_2^{-1}R$ such that $\iota'\circ \iota_{S_1} = \iota_{S_2}$. Provide an explicit description of this ring homomorphism. Use this to show that if $R$ is an integral domain and $S$ an arbitrary multiplicative subset of $R$, then $S^{-1}R$ injects into the fraction field $K = \operatorname{frac}\left( R \right)$.
\end{problem}
\begin{solution}
  We observe that $\iota_{S_2}\colon R\rightarrow S_{2}^{-1}R$ maps elements of $S_1$ to units in $S_2^{-1}R$, as the units in $S_2^{-1}R$ are elements of the form $ \frac{s}{s'} $ with $s,s'\in S_2$, so by the universal property, there is a unique ring homomorphism $\iota'\colon S_1^{-1}R\rightarrow S_2^{-1}R$ such that $\iota'\circ \iota_{S_1} = \iota_{S_2}$. In particular, this is the map $ \left[ \frac{r}{1} \right]_{S_1^{-1}R} \mapsto \left[ \frac{r}{1} \right]_{S_2^{-1}R} $.\newline

  Since any arbitrary multiplicative subset $S\subseteq R$ of an integral domain is contained in $R\setminus \set{0}$, it follows that $S^{-1}R$ injects into $ \left( R\setminus \set{0} \right)^{-1}R \eqcolon \operatorname{frac}\left( R \right) $.
\end{solution}
\begin{problem}[Problem 5]
  Let $R = \Q\times \Q$ and $S = \set{\left( 1,1 \right)}\cup \left( \Q^{\times}\times \set{0} \right)$. The goal of this problem is to identify the localization $S^{-1}R$.
  \begin{enumerate}[(a)]
    \item Describe explicitly when $\frac{\left( a_1,a_2 \right)}{\left( s_1,s_2 \right)}$ is equal to $\frac{\left( b_1,b_2 \right)}{\left( t_1,t_2 \right)}$ in $S^{-1}R$.
    \item Use your result from part (a) to show that the localization $S^{-1}R$ is isomorphic to the localization $T^{-1}\Q$, where $T = \Q\setminus \set{0}$, hence is isomorphic to $\R$.
    \item Find the kernel of the localization homomorphism $\iota_S\colon R\rightarrow S^{-1}R$.
  \end{enumerate}
\end{problem}
\begin{solution}\hfill
  \begin{enumerate}[(a)]
    \item By the definition of the equivalence relation, we must have an element $\left( r_1,r_2 \right)\in S$ such that
      \begin{align*}
        \left( r_1\left( a_1t_1-b_1s_1 \right),r_2\left( a_2t_2-b_2s_2 \right) \right) &= \left( 0,0 \right).
      \end{align*}
      In particular, since $r_1\in \Q^{\times}$, and we may always select $r_2 = 0$, it follows that
      \begin{align*}
        r_1\left( a_1t_1-b_1s_1 \right) &= 0,
      \end{align*}
      so that $a_1t_1 - b_1s_1 = 0$ (as $\Q$ is an integral domain).
    \item We consider the map $\pi_1\colon \Q\times \Q\rightarrow \Q$, which maps $\left( a_1,a_2 \right)\mapsto a_1$. Observe then that $S^{-1}R$ satisfies the universal property for localization, as we may write $S = \left( \Q^{\times}\times \set{0} \right)\cup \left( \Q^{\times}\cup \set{1} \right)$, which clearly maps to $\Q^{\times}\subseteq \Q$ under this projection map. Additionally, we see that $T^{-1}\Q$ satisfies the universal property for localization when restricted to the first coordinate; yet, this restriction to the first coordinate is exactly our original homomorphism, so both $T^{-1}\Q$ and $S^{-1}R$ satisfy the universal property for localization. Thus, they must be isomorphic.
    \item 
  \end{enumerate}
\end{solution}
\begin{problem}[Problem 7]
  Let $S\subseteq R$ be a multiplicative subset, and let $\iota_S\colon R\rightarrow S^{-1}R$ be the corresponding localization homomorphism. Consider the map
  \begin{align*}
    \alpha\colon \set{P'| P'\text{ is a prime ideal of }S^{-1}R}&\rightarrow \set{P | P\text{ is a prime ideal of }R\text{ such that }S\cap P = \emptyset}\\
    P'&\mapsto \iota_S^{-1}\left( P' \right).
  \end{align*}
  \begin{enumerate}[(a)]
    \item Verify that $\alpha$ is well-defined.
    \item Define an inverse map $\beta$ by $\beta(P) = P\cdot S^{-1}R$. Show that $\beta$ is well-defined. That is, $\beta(P)$ is a prime ideal of $S^{-1}R$.
    \item Show that $\alpha$ and $\beta$ are mutual inverses.
  \end{enumerate}
\end{problem}
\end{document}
