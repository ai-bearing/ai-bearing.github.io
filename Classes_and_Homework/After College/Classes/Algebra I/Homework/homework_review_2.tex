\documentclass[10pt]{mypackage}

% sans serif font:
%\usepackage{cmbright}
%\usepackage{sfmath}
%\usepackage{bbold} %better blackboard bold

%serif font + different blackboard bold for serif font
\usepackage{homework}
\usepackage{newpxtext,eulerpx}
\renewcommand*{\mathbb}[1]{\varmathbb{#1}}

\fancyhf{}
\rhead{Avinash Iyer}
\lhead{Review 2}

\setcounter{secnumdepth}{0}

\begin{document}
\RaggedRight
\begin{problem}[Problem 1]
  Let $F$ be a field, and for $n\geq 1$, let $\Mat_{n}\left( F \right)$ be the set of $n\times n$ matrices with entries in $F$.
  \begin{enumerate}[(a)]
    \item Show that $\GL_n(F) \coloneq \set{x\in \Mat_n(F) | \det(x)\neq 0}$ is a group under matrix multiplication.
    \item Show that $\SL_n(F) \coloneq \set{x\in \Mat_n(F) | \det(x) = 1}$ is a normal subgroup of $\GL_n(F)$, and identify the quotient $\GL_n(F)/\SL_n(F)$.
  \end{enumerate}
\end{problem}
\begin{solution}\hfill
  \begin{enumerate}[(a)]
    \item We see that if $a,b\in \GL_n(F)$, then since $\det(a)\neq 0$, the properties of the determinant yield $0\neq \det\left( a \right)^{-1} = \det\left( a^{-1} \right)$, meaning that $a^{-1}\in \GL_n(F)$, and $0\neq \det(a)\det(b) = \det(ab)$, meaning that $ab\in \GL_n(F)$, since fields have no zero-divisors.
    \item If $a\in \SL_n(F)$, then for any $x\in \GL_n(F)$, we have
      \begin{align*}
        \det\left( xax^{-1} \right) &= \det\left( x \right)\det\left( a \right)\det\left( x^{-1} \right)\\
                                    &= \det\left( x \right)\det\left( a \right)\det\left( x \right)^{-1}\\
                                    &= \det\left( a \right)\\
                                    &= 1,
      \end{align*}
      meaning that $xax^{-1}\in \SL_n(F)$ for any $x\in \GL_n(F)$. In particular, we note that the map
      \begin{align*}
        \det\colon \GL_n(F) \rightarrow F\setminus\set{0},
      \end{align*}
      given by $a \mapsto \det(a)$ is a group homomorphism, as has been established by the properties of the determinant, and it is surjective, as the matrix $ \operatorname{diag}\left( a,1_F,\dots,1_F \right) $ has determinant $a$, for any $a\in F$. Finally, we see that $\det^{-1}\left( \set{1_F} \right)$ is $\SL_n(F)$, meaning that by the First Isomorphism Theorem, $\GL_n(F)/\SL_n(F) \cong F\setminus \set{0}$.
  \end{enumerate}
\end{solution}
\begin{problem}[Problem 2]
  Let $G$ be a group, and let $H_1,H_2\leq G$ be subgroups. Show that if $H_1\cup H_2$ is a subgroup, then either $H_1\subseteq H_2$ or $H_2\subseteq H_1$.
\end{problem}
\begin{solution}
  Suppose toward contradiction that there were some $x\in H_1\setminus H_2$ and $y\in H_2\setminus H_1$. Since $xy\in H_1\cup H_2$, it follows that $xy\in H_1$ or $xy\in H_2$. If $xy\in H_1$, then so too is $x^{-1}xy$, meaning $y\in H_1$, which is a contradiction. Similarly, if $xy\in H_2$, then so too is $xyy^{-1}$, implying $x\in H_2$, again a contradiction. Thus, either $H_1\setminus H_2$ or $H_2\setminus H_1$ is empty, so that $H_1\subseteq H_2$ or $H_2\subseteq H_1$.
\end{solution}
\begin{problem}[Problem 3]
  Let $G$ be a group, and let $H_1,H_2\leq G$ be subgroups.
  \begin{enumerate}[(a)]
    \item Show that if $H_1$ and $H_2$ are finite, with $\gcd\left(\left\vert H_1 \right\vert,\left\vert H_2 \right\vert\right) = 1$, then $H_1\cap H_2 = \set{e}$.
    \item Show that if both $H_1$ and $H_2$ are normal subgroups, and $H_1\cap H_2 = \set{e}$, then $h_1h_2 = h_2h_1$ for all $h_1\in H_1$ and $h_2\in H_2$.
  \end{enumerate}
\end{problem}
\begin{solution}\hfill
  \begin{enumerate}[(a)]
    \item Let $g\in H_1\cap H_2$. Then, we see that $\operatorname{ord}\left( g \right) | \left\vert H_1 \right\vert$ and $\operatorname{ord}\left( g \right) | \left\vert H_2 \right\vert$, so $\operatorname{ord}(g) | \gcd\left( \left\vert H_1 \right\vert,\left\vert H_2 \right\vert \right)$; yet, since $\gcd\left( \left\vert H_1 \right\vert,\left\vert H_2 \right\vert \right) = 1$, this means that $\operatorname{ord}\left( g \right) = 1$, meaning $g = \set{e}$.
    \item If $H_1$ and $H_2$ are normal subgroups, then for $h_1\in H_1$ and $h_2\in H_2$, we consider the commutator $c = h_1h_2h_1^{-1}h_2^{-1}$. Notice that by grouping as $\left( h_1h_2h_1^{-1} \right)h_2^{-1}$, since $H_2$ is a normal subgroup, $c\in H_2$. Similarly, by grouping as $h_1\left( h_2h_1^{-1}h_2^{-1} \right)$, since $H_1$ is normal, we see that $c\in H_1$. Since $H_1\cap H_2 = \set{e}$, we see that $h_1h_2h_1^{-1}h_2^{-1} = e$, so $h_1h_2 = h_2h_1$.
  \end{enumerate}
\end{solution}
\begin{problem}[Problem 4]
  Let $g\in G$ be an element with $\operatorname{ord}\left( g \right) = n < \infty$.
  \begin{enumerate}[(a)]
    \item Show that if $g^{m} = e$, then $n | m$.
    \item If $d | n$, then $\operatorname{ord}\left( g^{d} \right) = n/d$.
    \item Show that for any integer $m\neq 0$, $\left\langle g^{m} \right\rangle = \left\langle g^{\gcd\left( m,n \right)} \right\rangle$.
    \item Use (b) and (c) to conclude that $\operatorname{ord}\left( g^{m} \right) = \frac{n}{\gcd\left( m,n \right)}$ for any $m\neq 0$.
  \end{enumerate}
\end{problem}
\begin{solution}\hfill
  \begin{enumerate}[(a)]
    \item We see that if $g^{m} = e$, then $g^{m} = \left( g^{n} \right)^{k}$, as $\operatorname{ord}\left( g \right) = n < \infty$, so that $g^{m} = g^{nk}$, and thus $n | m$.
    \item Let $d | n$. Then, $n = ad$ for some $a\in \Z$, so $e = g^{n} = \left( g^{d} \right)^{a}$, meaning $\operatorname{ord}\left( g^{d} \right) = a = n/d$.
    \item The inclusion $ \left\langle g^{m} \right\rangle \subseteq \left\langle g^{\gcd\left( m,n \right)} \right\rangle $ immediately follows from the fact that $\gcd\left( m,n \right) | m$. For the reverse direction, we observe that by the Bezout identity, $\gcd\left( m,n \right) = am + bn$ for some $a,b\in \Z$, meaning that if $h\in \left\langle g^{\gcd\left( m,n \right)} \right\rangle$, then $h = g^{c\gcd\left( m,n \right)}$, so $h = g^{acm}$, so $h\in \left\langle g^{m} \right\rangle$.
    \item Since $\left\langle g^{m} \right\rangle = \left\langle g^{\gcd\left( m,n \right)} \right\rangle$, it follows that $\operatorname{ord}\left( g^{ m } \right) = \operatorname{ord}\left( g^{\gcd\left( m,n \right)} \right)$, so $\operatorname{ord}\left( g^{m} \right) = n/\left( \gcd\left( m,n \right) \right)$.
  \end{enumerate}
\end{solution}
\begin{problem}[Problem 5]
  Let $g$ and $h$ be commuting elements of a group $G$ having finite orders $m$ and $n$. If $m$ and $n$ are relatively prime, then $ \operatorname{ord}\left( gh \right) = mn $.
\end{problem}
\begin{solution}
  Let $k$ be such that $\left( gh \right)^{k} = e$, meaning that $g^{k}h^{k} = e$, so that $g^{k} = h^{-k}$. In particular, this means that $h^{-k}\in \left\langle g \right\rangle$, implying that $h^{k}\in \left\langle g \right\rangle$, and similarly, $ g^{k}\in \left\langle h \right\rangle $.\newline

  It follows that $g^{k}$ and $h^{k}$ are contained in $ \left\langle g \right\rangle \cap \left\langle h \right\rangle $; yet, since $m$ and $n$ are coprime, we know from Problem 3 that $ \left\langle g \right\rangle \cap \left\langle h \right\rangle = \set{e}$, so that $g^{k} = e = h^{k}$. Therefore, $ m | k $ and $ n | k $, meaning that $ \operatorname{lcm}\left( m,n \right) | k $. Yet, since $m$ and $n$ are relatively prime, this means $ mn | k $. Finally, since $g^{mn}h^{mn} = e$, it follows that $ \operatorname{ord}\left( gh \right) = mn $.
\end{solution}
\begin{problem}[Problem 6]
  Let $G$ be a finite group of even order. Then, $G$ contains an element of order $2$.
\end{problem}
\begin{solution}
  Suppose not. Then, for any $e\neq g\in G$, $g \neq g^{-1}$. By pairing off each non-identity $g$ with its corresponding $g^{-1}$, we see that $G$ can be partitioned as
  \begin{align*}
    G &= \set{\set{e},\set{g_1,g_1^{-1}},\dots,\set{g_{k},g_{k}^{-1}}},
  \end{align*}
  since $G$ is finite. Yet, this means that $G$ is of odd order, which is a contradiction.
\end{solution}
\begin{problem}[Problem 7]
  Let $G = \set{g_1,\dots,g_n}$ be a finite abelian group. Show that the product $g_1g_2\cdots g_n$ is an element of order $\leq 2$.
\end{problem}
\begin{solution}
  Clearly, $g_1g_2\dots g_n$ is an element of $G$; furthermore, we see that if we square this value, then
  \begin{align*}
    \left( g_1g_2\cdots g_n \right)^2 &= g_1g_2\cdots g_n g_1g_2\cdots g_n.
  \end{align*}
  Since $G$ is abelian, we may pair each $g_i$ with its corresponding $g_j$ such that $g_ig_j = e_{G}$. Therefore, we see that $\left( g_1g_2\cdots g_n \right)^2 = e_G$, so $g_1g_2\cdots g_n$ has order at most $2$.
\end{solution}
\begin{problem}[Problem 8]
  Construct an explicit isomorphism between the group $\left( \R_{> 0},\cdot \right)$ of strictly positive real numbers under multiplication and the group $\left( \R,+ \right)$ of all real numbers under addition.\newline

  On the other hand, show that the group $\left( \Q_{ > 0 },\cdot \right)$ of strictly positive rational numbers under multiplication is not isomorphic to the group $\left( \Q,+ \right)$ of all rational numbers under addition.
\end{problem}
\begin{solution}
  To see an isomorphism between $\left( \R_{ > 0 },\cdot \right)$ and $\left( \R,+ \right)$, we define the map $r\mapsto \ln\left( r \right)$. Notice that by the definition of the logarithm, $\ln\left( pr \right) = \ln\left( p \right) + \ln\left( r \right)$ (so $\ln$ preserves their respective group structures), and that $\ln$ admits an inverse, $\exp$, so we have an isomorphism between $\left( \R_{ > 0 },\cdot \right)$ and $\left( \R,+ \right)$.\newline

  On the other hand, we see that if $\varphi\colon \left( \Q,+ \right)\rightarrow \left( \Q_{> 0},\cdot \right)$ is any structure-preserving map, then $\varphi\left( 2a \right) = \varphi\left( a \right)^2$, meaning that $\varphi\left( \frac{1}{2}a \right) = \varphi\left( a \right)^{1/2}$. Yet, since $\Q_{> 0}$ is not closed under the taking of roots, such a map cannot be a homomorphism.
\end{solution}
\begin{problem}[Problem 9]
  Use Zorn's Lemma to prove that every (nontrivial) finitely generated group has a maximal proper subgroup.
\end{problem}
\begin{solution}
  Let $G = \left\langle g_1,\dots,g_n \right\rangle$, and let
  \begin{align*}
    \mathcal{H} &= \set{H\leq G | H\text{ is a subgroup, }H\neq G}
  \end{align*}
  be ordered by inclusion. We claim that $ \mathcal{H} $ satisfies the necessary requirements of Zorn's Lemma. To start, we see that $\set{e}$ is a proper subgroup of $G$, meaning that $ \set{e}\in \mathcal{H} $, so $ \mathcal{H} $ is nonempty. Furthermore, if $ \mathcal{C} = \set{H_i}_{i\in I} $ is a chain in $ \mathcal{H} $, then we claim that
  \begin{align*}
    H &= \bigcup_{i\in I}H_i
  \end{align*}
  is an upper bound that lies in $ \mathcal{H} $. First, we observe that, since $ \mathcal{C} $ is totally ordered by inclusion, the union of an arbitrary number of elements of $ \mathcal{H} $ is also a subgroup, as we had shown earlier. Additionally, if it were not the case that $ H\in \mathcal{H} $ (i.e., $G = H$), then since $G$ is finitely generated, it would follow that each of its generators, $ g_1,\dots,g_n $ are in $H$. Therefore, there would be some $H_i$ such that all of $g_1,\dots,g_n$ are in $H_i$, which would contradict the fact that $ \mathcal{C} $ is a chain in $H$.\newline

  Therefore, the conditions of Zorn's Lemma are satisfied, and so $G$ admits a maximal proper subgroup.
\end{solution}
\begin{problem}[Problem 10]\hfill
  \begin{enumerate}[(a)]
    \item Show that only a cyclic group of prime order does not have any proper subgroups, and derive that if $H$ is a maximal proper subgroup of an abelian group $G$, then the quotient $ G / H $ is a cyclic group of prime order.
    \item Use (a) to conclude that the additive group of rationals, $\left( Q,+ \right)$, does not have any maximal proper subgroups, and hence the finitely generated assumption in the previous problem was necessary.
  \end{enumerate}
\end{problem}
\begin{solution}\hfill
  \begin{enumerate}[(a)]
    \item Let $ e\neq a\in G $. Then, since $G$ does not admit any nontrivial proper subgroups, it follows that $ \left\langle a \right\rangle = G $, meaning that $G$ is a cyclic group. We see that this means $G$ must be finite, since else, $G\cong \Z$ by corresponding powers of $ a $ to the integers, and the integers contain proper subgroups. This implies that $ \operatorname{ord}\left( a \right) = n < \infty $, meaning that for any $m\neq 0$, $ \operatorname{ord}\left( a^{m} \right) = \frac{n}{\operatorname{gcd}\left( m,n \right)} $ from an earlier problem; yet, since $ \left\langle a^{m} \right\rangle = G$ as well, it follows that $\operatorname{gcd}\left( m,n \right) = 1$ for any $m\neq 0$, so that $n$ is prime.\newline

      From the fourth isomorphism theorem, it follows that if $H$ is a maximal proper subgroup of an abelian group $G$, then $G/H$ cannot contain any proper subgroups (or else there would be a proper subgroup of $G$ containing $H$, which would contradict maximality).
    \item If $H\leq \Q$ is a proper subgroup, then there is some $ \frac{m}{n}\notin H $, so that $ \frac{m}{n} + H\in \Q/H $. This implies that $ \frac{1}{n}\Z + H\subseteq \Q/H $, meaning that $ \Q/H $ is infinite for any proper subgroup of $H$. Since all quotients of $\Q$ by proper subgroups are infinite, it follows that none of them can be isomorphic to $\Z/p\Z$ for any prime $p$, so that $\Q$ does not have any maximal proper subgroups.
  \end{enumerate}
\end{solution}
\end{document}
