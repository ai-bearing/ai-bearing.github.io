\documentclass[10pt]{mypackage}

% sans serif font:
%\usepackage{cmbright,sfmath,bbold}
%\renewcommand{\mathcal}{\mathtt}

%Euler:
\usepackage{newpxtext,eulerpx,eucal,eufrak}
\renewcommand*{\mathbb}[1]{\varmathbb{#1}}
\renewcommand*{\hbar}{\hslash}

%\renewcommand{\mathbb}{\mathds}
%\usepackage{homework}
%\usepackage{exposition}

\pagestyle{fancy} %better headers
\fancyhf{}
\rhead{Avinash Iyer}
\lhead{Math 5305: Homework 1}

\setcounter{secnumdepth}{0}

\begin{document}
\RaggedRight
\section{Problem 1}%
\begin{enumerate}[(i)]
  \item Let $f\colon X\rightarrow Y$ be a function, and let $A\subseteq X$. Then, if $x\in A$, we must have $f(x)\in f(A)$, so that $x\in f^{-1}\left( f\left( A \right) \right)$, meaning $A\subseteq f^{-1}\left( f(A) \right)$. If $f$ is injective, then for any $z\in f^{-1}\left( f\left( X \right) \right)$, there is a unique $y\in f(X)$ such that $f(z) = y$ (by injectivity), meaning that $f$ is left-invertible; therefore, $f^{-1}\left( f\left( A \right) \right) = A$ for all such $A\subseteq X$.
  \item If $B\subseteq Y$, then for any $f(x)\in f\left( f^{-1}\left( B \right) \right)$, $x\in f^{-1}\left( B \right)$, or that $f(x)\in B$, meaning that $f\left( f^{-1}\left( B \right) \right) \subseteq B$. If $f$ is surjective, then for any $y\in Y$, there is some $z\in X$ such that $f(z) = y$, or that $z\in f^{-1}\left( \set{y} \right)$; therefore, we may select a right-inverse for $f$, meaning that $f\left( f^{-1}\left( X \right) \right) = Y$, or $f\left( f^{-1}\left( B \right) \right) = B$ for all subsets $B\subseteq Y$.
\end{enumerate}
\section{Problem 6}%
Let $X\neq\emptyset$ and $\mathcal{C}\subseteq \mathcal{P}\left( X \right)$. We define the topology $\tau_{\mathcal{C}}$ on $X$ by
\begin{align*}
  \tau_{\mathcal{C}} &= \bigcap \set{\tau | \tau\subseteq \mathcal{P}\left( X \right)\text{ is a topology},\mathcal{C}\subseteq \tau}.
\end{align*}
The intersection is indeed well-defined, as the discrete topology includes $\mathcal{C}$, and it is the smallest such topology as any other topology on $X$ that contains $\mathcal{C}$ is included in the intersection.
\section{Problem 9}%
Let $\tau_1,\tau_2$ be topologies on $X$ with respective bases $\mathcal{B}_1,\mathcal{B}_2$.\newline

Let $\tau_1\subseteq \tau_2$, $x\in X$, and $B_1\in \mathcal{B}_1$ with $x\in B_1$. Since $B_1\in \mathcal{B}_1$, $B_1\in \tau_1$, so $B_1\in \tau_2$, meaning that $B_1 = \bigcup_{i\in I}U_i$ for some family $\set{U_i}_{i\in I}\subseteq \mathcal{B}_2$. There is at least one such $U_i\in \mathcal{B}_2$ with $x\in U_i$; setting $B_2\coloneq U_i$, we have $x\in B_2\subseteq B_1$.\newline

Suppose now that for any $x\in X$, $B_1\in \mathcal{B}_1$ with $x\in B_1$, there is $B_2\in \mathcal{B}_2$ with $x\in B_2\subseteq B_1$. Let $U\in \tau_1$. Then, there is some family $\set{B_{i}}_{i\in I}\subseteq \mathcal{B}_1$ such that $\bigcup_{i\in I}B_i = U$. For each $x\in B_i$, we find $V_{x,i}\in \mathcal{B}_2$ such that $x\in V_{x,i}\subseteq B_{i}$. Then, we get $\bigcup\set{V_{x,i}| x\in U,i\in I} = U$, whence $U\in \tau_2$.
\section{Problem 14}%
Let $f\colon X\rightarrow Y$ be a map of topological spaces.
\begin{enumerate}[(i)]
  \item Let $f$ be open. Then, for any $A\subseteq X$, we see that $f\left( A^{\circ} \right)$ is open in $Y$. Since $A^{\circ}\subseteq A$, we see that $f\left( A^{\circ} \right)\subseteq f\left( A \right)$, so since $f\left( A^{\circ} \right)$ is open, it is contained in $f\left( A \right)^{\circ}$, meaning that $f\left( A^{\circ} \right)\subseteq f\left( A \right)^{\circ}$.\newline

    Now, let $f$ be such that $f\left( A^{\circ} \right)\subseteq f\left( A \right)^{\circ}$ for all $A\subseteq X$. Let $U\subseteq X$ be open, so $U^{\circ} = U$. Then, since $f\left( U^{\circ} \right)\subseteq f\left( U \right)^{\circ}$, we see that $f\left( U \right)\subseteq f\left( U \right)^{\circ}$, and since $f\left( U \right)^{\circ}\subseteq f\left( U \right)$ necessarily, we have $f\left( U \right) = f\left( U \right)^{\circ}$, or that $f$ is an open map.
  \item Let $f$ be closed. Then, for any $A\subseteq X$, we have that $f\left( \overline{A} \right)$ is closed in $X$. Since $A\subseteq \overline{A}$, we have that $f\left( A \right)\subseteq f\left( \overline{A} \right)$, and since $f\left( \overline{A} \right)$ is closed, we also have $ \overline{f\left( A \right)}\subseteq f\left( \overline{A} \right) $.\newline

    Now, let $f$ be such that $ \overline{f\left( A \right)}\subseteq f\left( \overline{A} \right) $, and let $C\subseteq X$ be closed, meaning $ \overline{C} = C $. Then, as assumed, we have $ \overline{f\left( C \right)}\subseteq f\left( \overline{C} \right)$, but since $C = \overline{C}$, we have $f\left( \overline{C} \right) = f\left( C \right) \subseteq \overline{f\left( C \right)}$, meaning $ \overline{f\left( C \right)} = f\left( C \right) $, or that $f\left( C \right)$ is closed.
\end{enumerate}
\section{Problem 18}%
\begin{enumerate}[(i)]
  \item Let 
    \begin{align*}
      \mathcal{B}_{\text{prod}} &= \set{\prod_{i\in I}U_i | U_i = B_i\in \mathcal{B}_i\text{ for finitely many $i$}}.
    \end{align*}
    We show that $\mathcal{B}_{\text{prod}}$ is a basis. First, for each $\mathcal{B}_i$, $\bigcup \mathcal{B}_i = X_i$, so by taking products we find that $\bigcup \mathcal{B}_{\text{prod}} = \prod_{i\in I}X_i$. Then, for any two elements $U$ and $V$ of $\mathcal{B}_{\text{prod}}$, there are finitely many $\set{U_{i_k}}_{k=1}^{n}$ and $\set{V_{i_j}}_{j=1}^{n}$ in $\mathcal{B}_{i_k}$ and $\mathcal{B}_{i_j}$ such that $U$ and $V$ are the respective products with $X_i$ elsewhere. We have three cases.
    \begin{itemize}
      \item If $U_{i_j}\neq X_{i_j}$ and $V_{i_j} = X_{i_j}$ at index $i_j$, then the intersection $U\cap V$ has $U_{i_j}$ at index $i_j$, meaning that their intersection is yet again a product of basis elements with all but finitely many equal to $X_i$.
      \item Similarly, if $U_{i_j} = X_{i_j}$ and $V_{i_j}\neq X_{i_j}$, the intersection $U\cap V$ yields yet another product of basis with elements with all but finitely many equal to $X_i$.
      \item If $U_{i_j},V_{i_j}\neq X_{i_j}$, then there is some $W_{i_j}\in \mathcal{B}_{i_j}$ such that $W_{i_j}\subseteq U_{i_j}\cap V_{i_j}$, meaning that the intersection $U\cap V$ contains a product of basis elements with all but finitely many equal to $X_i$.
    \end{itemize}
  \item Let
    \begin{align*}
      \mathcal{B}_{\text{box}} &= \set{\prod_{i\in I} B_i | B_i\in \mathcal{B}_i}.
    \end{align*}
    We can see immediately that $\mathcal{B}_{\text{box}}$ is a basis, as $\bigcup \mathcal{B}_i = X_i$ for each $i$, so by taking products, we get that $\bigcup \mathcal{B}_{\text{box}} = \prod_{i\in I}X_i$. Similarly, since the finite intersection of basis elements always contains a basis element, taking $U,V\in \mathcal{B}_{\text{box}}$, we have $\left( \prod_{i\in I}U_i \right)\cap \left( \prod_{i\in I}V_i \right) = \prod_{i\in I}\left( U_i\cap V_i \right)$, which contains some box $\prod_{i\in I} W_i$ where $W_i\in \mathcal{B}_i$.
  \item If the indexing set $I$ is finite, we see that $\tau_{\text{box}} = \tau_{\text{prod}}$. Else, if $I$ is infinite, we have that $\tau_{\text{prod}}\subsetneq \tau_{\text{box}}$.
\end{enumerate}
\end{document}
