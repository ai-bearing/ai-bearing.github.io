\documentclass[10pt]{mypackage}

% sans serif font:
%\usepackage{cmbright,sfmath,bbold}
%\renewcommand{\mathcal}{\mathtt}

%Euler:
\usepackage{newpxtext,eulerpx,eucal,eufrak}
\renewcommand*{\mathbb}[1]{\varmathbb{#1}}
\renewcommand*{\hbar}{\hslash}

%\renewcommand{\mathbb}{\mathds}
%\usepackage{homework}
%\usepackage{exposition}

\pagestyle{fancy} %better headers
\fancyhf{}
\rhead{Avinash Iyer}
\lhead{Math 5305: Homework 3}

\setcounter{secnumdepth}{0}

\begin{document}
\RaggedRight
\section{Problem 7}%
From the Stone--Weierstrass Theorem, we know that there is some sequence $\left( p_n \right)_n$ of polynomials such that $\left( p_n \right)_n\rightarrow f$ uniformly.\newline

Now, since
\begin{align*}
  \int_{a}^{b} f(x)x^{k}\:dx &= 0
\end{align*}
for each $k$, linearity means that this holds for all polynomials $p(x)$, so this means that for all $n$,
\begin{align*}
  \int_{a}^{b} f(x)p_n(x)\:dx &= 0
\end{align*}
for each $x$. Thus, since $\left( p_n \right)_n\rightarrow f$ uniformly, we may exchange limit and integral, yielding
\begin{align*}
  \int_{a}^{b} f(x)^2\:dx &= \int_{a}^{b} \lim_{n\rightarrow\infty}f(x)p_n(x)\:dx\\
                          &= \lim_{n\rightarrow\infty} \int_{a}^{b} f(x)p_n(x)\:dx\\
                          &= 0,
\end{align*}
so $f(x) = 0$ as $f$ is continuous and the positive function $f^2$ integrates to zero.
\section{Problem 8}%
We see that the family $\mathcal{F}$ is bounded in the sup norm. We show now that $\mathcal{F}$ is closed.\newline

If $\left( f_n \right)_n\subseteq \mathcal{F}$, and $\left( f_n \right)_n\rightarrow f$, then since the closed unit ball in $C\left( [a,b] \right)$ is closed, we know that $\norm{f}_{\sup} \leq 1$. Now, we show that $\left[ f \right]_{\alpha}\leq 1$. Now, for all $n$, we have
\begin{align*}
  \sup_{x\neq y\in [a,b]} \frac{\left\vert f_n(x)-f_n(y) \right\vert}{\left\vert x-y \right\vert^{\alpha}} &\leq 1,
\end{align*}
meaning that by taking limits, we have that
\begin{align*}
  \sup_{x\neq y\in [a,b]} \frac{\left\vert f(x)-f(y) \right\vert}{\left\vert x-y \right\vert^{\alpha}} &\leq 1,
\end{align*}
so $\mathcal{F}$ is closed.\newline

Finally, we show that $\mathcal{F}$ is equicontinuous.\newline

To see this, if $\ve > 0$, then by setting $\delta = \ve^{1/\alpha}$, Hölder continuity gives, whenever $\left\vert x-y \right\vert < \delta$,
\begin{align*}
  \left\vert f(y)-f(x) \right\vert &\leq [f]_{\alpha}\left\vert x-y \right\vert^{\alpha}\\
                                   &< \delta^{\alpha}\\
                                   &= \left( \ve^{1/\alpha} \right)^{\alpha}\\
                                   &= \ve.
\end{align*}
Since this holds for all $f\in \mathcal{F}$, the family $\mathcal{F}$ is equicontinuous, so by the Arzela--Ascoli theorem, $\mathcal{F}$ is compact.
\section{Problem 10}%
\begin{enumerate}[(i)]
  \item We may write all functions $f\in C\left( X,\C \right)$ as $f = \re\left( f \right) + i\im(f)$.\newline

    Now, note that if $f\in \mathcal{A}$, then since $\mathcal{A}$ is self adjoint, so too are $\re\left( f \right)$ and $\im\left( f \right)$, as
    \begin{align*}
      \re\left( f \right) &= \frac{1}{2}\left( f + \overline{f} \right)\\
      \im\left( f \right) &= \frac{1}{2i}\left( f - \overline{f} \right).
    \end{align*}
    So, to see that $\mathcal{A}$ separates points in $\R$, it suffices to see that if $x\neq y$ and $f\in \mathcal{A}$ is such that $f(x)\neq f(y)$, either $\re\left( f \right)(x)\neq \re\left( f \right)(y)$ or $\im\left( f \right)(x)\neq \im\left( f \right)\left( y \right)$.\newline

    Finally, since $\mathcal{A}$ separates $x\in X$ from $0$, this means either $\re\left( f \right)\left( x \right)\neq 0$ or $\im\left( f \right)\left( x \right)\neq 0$, so $\mathcal{A}$ is nowhere vanishing as a real subalgebra.\newline

    For $f\in C\left( X,\C \right)$, we may find $a,b\in \mathcal{A}$ with $a$ and $b$ real-valued such that $\norm{\re\left( f \right) - a} < \ve/2$ and $\norm{\im\left( f \right) - b} < \ve/2$, owing to the Stone--Weierstrass Theorem. Therefore, setting $g = a + ib$, we have
    \begin{align*}
      \norm{f-g} &= \norm{\re\left( f \right) - a + i\left( \im\left( f \right) - b \right)}\\
                 &\leq \norm{\re\left( f \right) - a} + \norm{\im\left( f \right) - b}\\
                 &< \ve.
    \end{align*}
  \item We can see that elements of $\mathcal{A}$ form polynomials in $z$ and $z^{-1} = \overline{z}$, so $\mathcal{A}$ is a subalgebra of $C\left( \mathbb{T},\C \right)$ that contains the constants. We must show that $\mathcal{A}$ separates points, is self-adjoint.\newline

    To see this, we note that $p(z) = z$ is an element of $\mathcal{A}$ and separates distinct elements of $\mathbb{T}$. Finally, if
    \begin{align*}
      p(z) &= \sum_{i=m}^{n} a_iz^{i},
      \intertext{then}
      \overline{p(z)} &= \sum_{i=m}^{n} \overline{a_i}z^{-i},
    \end{align*}
    meaning that $\mathcal{A}$ is self-adjoint. Thus, by the complex Stone--Weierstrass Theorem, $\mathcal{A}$ is dense in $C\left( \mathbb{T},\C \right)$.
\end{enumerate}
\section{Problem 11}%
\begin{enumerate}[(i)]
  \item We see that for any $m$, both $\inf_{n\geq m} a_n$ and $\sup_{n\geq m}$ exist (by the completeness of the extended real numbers) and are contained in $[-\infty,\infty]$. If we consider the sequences $\left( \inf_{n\geq m}a_n \right)_{m},\left( \sup_{n\geq m}a_n \right)_{m}$, we notice that the sequence $\left( \inf_{n\geq m}a_n \right)_{m}$ is \textit{increasing}, while $\left( \sup_{n\geq m}a_n \right)_{m}$ is \textit{decreasing}, as the supremum of a tail sequence is less than or equal to the supremum of the whole sequence, and similarly the infimum of a tail sequence is greater than or equal to the infimum of the whole sequence.\newline

    Since these sequences are monotonic, they must converge in $\left[ -\infty,\infty \right]$.\newline

    Furthermore, for all $n$, we have
    \begin{align*}
      \inf_{n\geq m}a_n &\leq \sup_{n\geq m}a_n,
    \end{align*}
    so by taking limits on both sides, we get
    \begin{align*}
      \liminf_{n\rightarrow\infty} a_n &\leq \limsup_{n\rightarrow\infty}a_n.
    \end{align*}
  \item If $\left( a_n \right)_n\rightarrow a$, then the sequence $\left( a_m \right)_{m\geq n}$ converges to $a$ for each $n$; therefore, both $\left( \inf_{m\geq n}a_m \right)_{m}\rightarrow a$ and $\left( \sup_{m\geq n}a_m \right)_{m}\rightarrow a$, so $\limsup_{n\rightarrow\infty}a_n = a$ and $\liminf_{n\rightarrow\infty}a_n = a$.\newline

    Now, if $\liminf_{n\rightarrow\infty} a_n = a = \limsup_{n\rightarrow\infty} a_n$, then we know that the sequences
    \begin{align*}
      \left( \inf_{n\geq m}a_n \right)_{m} &\rightarrow a\\
      \left( \sup_{n\geq m}a_n \right)_{m} &\rightarrow a,
    \end{align*}
    so since
    \begin{align*}
      \inf_{n\geq m}a_n \leq a_m \leq \sup_{n\geq m}a_n,
    \end{align*}
    the squeeze theorem gives that $\left( a_m \right)_m\rightarrow a$.
  \item Let $\left( a_n \right)_n$ be bounded. To show that these items hold, we quickly show that
    \begin{align*}
      \limsup_{n\rightarrow\infty} a_n &= \inf_{m\geq 1}\left( \sup_{n\geq m}a_n \right)\\
      \liminf_{n\rightarrow\infty} a_n &= \sup_{m\geq 1}\left( \inf_{n\geq m} a_n \right).
    \end{align*}
    To see these, note that since $\left( a_n \right)_n$ is bounded, both $\left( \sup_{n\geq m}a_n \right)_{m}$ and $\left( \inf_{n\geq m}a_n \right)_{m}$ are bounded. We discussed earlier that these sequences are monotone, with the supremum sequence decreasing and the infimum sequence increasing, so the monotone convergence theorem for sequences gives that their limit is equal to the infimum and supremum respectively.
    \begin{itemize}
      \item If $a = \limsup_{n\rightarrow\infty}a_n$, then by the discussion from earlier, we get that for any $\ve > 0$, there is some $M$ such that for all $n\geq M$,
        \begin{align*}
          a_n &< a + \ve,
        \end{align*}
        as one of the characterizations of infimum. Thus, for all $\ve > 0$, the set $\set{n | a_n \geq a + \ve}$ is contained in $\set{1,\dots,M}$.\newline

        Next, note that since $\sup_{n\geq M}a_n \geq a$, the characterization of supremum yields yields an increasing sequence $\left( M_k \right)_k$ such that $\sup_{n\geq M_k}a_n > a-\ve/k$, which means there is an infinite set of $n$ such that $a_n > a-\ve$.\newline

        Now, let $a$ be such that $\set{n | a_n > a + \ve}$ is finite and $\set{n | a_n > a-\ve}$ is infinite. Then, by the first condition, there is some $M$ such that for all $n\geq M$, $a_n < a + \ve$, meaning that $\limsup_{n\rightarrow\infty} a_n \leq a + \ve$. Since $\ve$ is arbitrary, we have that $\limsup_{n\rightarrow\infty} a_n \leq a$.\newline

        Similarly, by the second condition, there is some increasing sequence $\left( m_k \right)_k$ such that $a_{m_k} > a-\ve$ for all $m_k$, so $\limsup_{k\rightarrow\infty} a_{m_k} \geq a-\ve$. Since $\limsup_{n\rightarrow\infty} a_n \geq \limsup_{k\rightarrow\infty}a_{m_k}\geq a-\ve$, and $\ve$ is arbitrary, we have $\limsup_{n\rightarrow\infty}a_n \geq a$, so $\limsup_{n\rightarrow\infty} a_n = a$.
      \item If $a = \liminf_{n\rightarrow\infty} a_n$, then by the discussion from earlier, for any $\ve > 0$, there is some $M$ such that for all $n\geq M$,
        \begin{align*}
          a_n > a-\ve,
        \end{align*}
        by the characterization of the supremum. Thus, for all $\ve > 0$, the set $\set{n | a_n > a-\ve}$ is contained in $\set{1,\dots,M}$.\newline

        Next, since $\inf_{n\geq M}a_n \leq a$, the characterization of the infimum yields an increasing sequence $\left( M_k \right)_k$ such that $\inf_{n\geq M_k} a_n < a + \ve/k$, meaning that $\set{n | a_n < a + \ve}$ is infinite.\newline

        Now, if $a$ is such that $\set{n | a_n < a-\ve}$ is finite, then there is some $M$ such that for all $n\geq M$, $a_n \geq a-\ve$, meaning that $\liminf_{n\rightarrow\infty}a_n \geq a-\ve$, so since $\ve$ is arbitrary, $\liminf_{n\rightarrow\infty}a_n \geq a$.\newline

        Similarly, if $\set{n | a_n < a+\ve}$ is infinite, then there is some increasing sequence $\left( m_k \right)_k$ such that $a_{m_k} < a+\ve$ for all $a$, so $\liminf_{k\rightarrow\infty}a_{m_k} \leq a+\ve$. Since $\liminf_{n\rightarrow\infty}a_n \leq \liminf_{k\rightarrow\infty}a_{m_k} \leq a + \ve$, and $\ve$ is arbitrary, we have that $\liminf_{n\rightarrow\infty}a_n \leq a$.
    \end{itemize}
\end{enumerate}
\section{Problem 14}%
\begin{enumerate}[(a)]
  \item If $z_0\in U\left( 0,R \right)$, we may select $r$ such that $U\left( z_0,r \right)\subseteq U\left( 0,R \right)$, as open balls are open in $\C$. Since, for all $z\in U\left( 0,R \right)$, $f(z)=\sum_{n=0}^{\infty}a_nz^{n}$ is defined, so too is it the case for all $z\in U\left( z_0,r \right)$.\newline

    Note that the power series representation of $f$ is equal to the Taylor series,
    \begin{align*}
      f(z) &= \sum_{n=0}^{\infty} \frac{f^{(n)}(0)}{n!} z^{n},
    \end{align*}
    meaning that by taking the translation operation $z\mapsto z-z_0$, we get
    \begin{align*}
      f(z) &= \sum_{n=0}^{\infty}\frac{f^{(n)}\left( z_0 \right)}{n!} \left( z-z_0 \right)^{n}\\
           &\eqcolon \sum_{n=0}^{\infty} c_n\left( z-z_0 \right)^{n},
    \end{align*}
    so $f$ is analytic.
  \item Letting $f^{(n)}\left(z_0\right) = 0$, we see that since
    \begin{align*}
      f(z) &= \sum_{n=0}^{\infty}\frac{f^{(n)}\left( z_0 \right)}{n!}\left( z-z_0 \right)^{n}
    \end{align*}
    on $U\left( z_0,r \right)$, we have that $f(z) = 0$ on $U\left( z_0,r \right)$.\newline

    Now, consider the set $L\subseteq U\left( 0,R \right)$ of all points $w$ such that $f^{(n)}(w) = 0$ for all $n\geq 0$. This subset is nonempty, as $z_0\in L$. Furthermore, it is open as we have shown earlier that if $f^{(n)}(w) = 0$ for all $n$, then $f = 0$ on $U\left( w,r_{w} \right)$ for some $r_{w} > 0$. Finally, if we have a sequence $\left( z_n \right)_n\subseteq L$ such that $\left( z_n \right)_n\rightarrow z$, then since all derivatives of $f$ are continuous, we must have $f^{(k)}\left( z_n \right) \rightarrow f^{(k)}\left( z \right)$ for all $k\geq 0$. This means that $f^{(k)}\left( z \right) = 0$ for all $k\geq 0$, meaning that $L$ is closed. By connectedness, this means $L = U\left( 0,R \right)$, so $f = 0$.
\end{enumerate}
\end{document}
