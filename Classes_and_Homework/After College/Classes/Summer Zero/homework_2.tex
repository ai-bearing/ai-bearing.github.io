\documentclass[10pt]{mypackage}

% sans serif font:
%\usepackage{cmbright,sfmath,bbold}
%\renewcommand{\mathcal}{\mathtt}

%Euler:
\usepackage{newpxtext,eulerpx,eucal,eufrak}
\renewcommand*{\mathbb}[1]{\varmathbb{#1}}
\renewcommand*{\hbar}{\hslash}

%\renewcommand{\mathbb}{\mathds}
%\usepackage{homework}
%\usepackage{exposition}
\usepackage{microtype}

\pagestyle{fancy} %better headers
\fancyhf{}
\rhead{Avinash Iyer}
\lhead{Homework 2}

\setcounter{secnumdepth}{0}

\begin{document}
\RaggedRight
\section{Problem 2}%
Let $\set{U_i}_{i\in I}$ be an open cover of $\left[ a,b \right]$. Since the open intervals are a base for $\tau_{\text{st}}$ on $\R$, we may assume that all the $U_i$ are open intervals. There are then $U_1,U_2\subseteq \R$ open such that $a\in U_1$ and $b\in U_2$.\newline

Let $c_1 = \sup\left( U_1\cap [a,b] \right)$ and $d_1 = \inf\left( U_2\cap [a,b] \right)$. If $c_1 < d_1$, we may apply a similar procedure to the case of $\left[ c_1,d_1 \right]\subseteq \left[ a,b \right]$, choosing $U_3,U_4$ such that $c_2 = \sup\left(  U_3\cap \left[c_1,d_1\right] \right)$ and $d_2 = \sup\left(  U_4\cap \left[ c_1,d_1 \right] \right)$, and so on and so forth. We claim that this process must stop eventually (i.e., that there is some $c_k,d_k$ such that $d_k < c_k$).\newline

Suppose not; then, we have a sequence of nested closed intervals
\begin{align*}
  \left[ a,b \right] \supseteq \left[ c_1,d_1 \right]\supseteq \left[ c_2,d_2 \right]\supseteq \cdots,
\end{align*}
so by the nested intervals property, there would be some $x\in \left[ a,b \right]$ such that $x\in \left[ a,b \right]\cap \bigcap_{i=1}^{\infty}\left[ c_i,d_i \right]$. This $x$ is necessarily not covered by any such $U_i\in \set{U_i}_{i\in I}$, contradicting the assumption that $\set{U_i}_{i\in I}$ is an open cover of $\left[ a,b \right]$.
\section{Problem 17}%
Write
\begin{align*}
  \sum_{i=1}^{m} \left\vert x_i + y_i \right\vert^{p} &= \sum_{i=1}^{m} \left\vert x_i \right\vert\left\vert x_i + y_i \right\vert^{p-1} + \left\vert y_i \right\vert\left\vert x_i + y_i \right\vert^{p-1}.
\end{align*}
Then, by applying Hölder's Inequality, we have
\begin{align*}
  \sum_{i=1}^{m} \left\vert x_i \right\vert\left\vert x_i + y_i \right\vert^{p-1} &\leq \left( \sum_{i=1}^{m} \left\vert x_i \right\vert^{p} \right)^{1/p}\left( \sum_{i=1}^{m} \left\vert x_i + y_i \right\vert^{\left( p-1 \right)q} \right)^{q}\\
  \sum_{i=1}^{m} \left\vert y_i \right\vert\left\vert x_i + y_i \right\vert^{p-1} &\leq \left( \sum_{i=1}^{m} \left\vert y_i \right\vert^{p} \right)^{1/p}\left( \sum_{i=1}^{m} \left\vert x_i + y_i \right\vert^{\left( p-1 \right)q} \right)^{q}.
\end{align*}
Since $\left( p-1 \right)q = p$, we then have
\begin{align*}
  \sum_{i=1}^{m} \left\vert x_i + y_i \right\vert^{p} &\leq \left( \sum_{i=1}^{m}\left\vert x_i + y_i \right\vert^{p} \right)^{q} \left( \left( \sum_{i=1}^{m}\left\vert x_i \right\vert^{p} \right)^{1/p} + \left( \sum_{i=1}^{m} \left\vert y_i \right\vert^{p} \right)^{1/p} \right),
\end{align*}
and dividing, we get
\begin{align*}
  \left( \sum_{i=1}^{m}\left\vert x_i + y_i \right\vert^{p} \right)^{1/p} \leq \left( \sum_{i=1}^{m}\left\vert x_i \right\vert^{p} \right)^{1/p} + \left( \sum_{i=1}^{m}\left\vert y_i \right\vert^{p} \right)^{1/p}.
\end{align*}
\section{Problem 19}%
\begin{enumerate}[(i)]
  \item We see that
    \begin{align*}
      \sup_{i\in \N} \left\vert x_i \right\vert &= 0
    \end{align*}
    if and only if $\left\vert x_i \right\vert \leq 0$ for all $i$, meaning that $\left( x_i \right)_{i}$ is the zero sequence. Similarly,
    \begin{align*}
      \norm{\alpha x} &= \sup_{i\in \N} \left\vert \alpha x_i \right\vert\\
                      &= \left\vert \alpha \right\vert\sup_{i\in I}\left\vert x_i \right\vert\\
                      &= \left\vert \alpha \right\vert\norm{x}.
    \end{align*}
    Finally,
    \begin{align*}
      \norm{x + y} &= \sup_{i\in \N} \left\vert x_i + y_i \right\vert\\
                   &\leq \sup_{i\in \N} \left( \left\vert x_i \right\vert + \left\vert y_i \right\vert \right)\\
                   &\leq \sup_{i\in\N} \left\vert x_i \right\vert + \sup_{j\in\N} \left\vert y_j \right\vert\\
                   &= \norm{x} + \norm{y},
    \end{align*}
    meaning that $\norm{\cdot}$ is a bona fide norm.
  \item Let $B=  \set{x\in X | \norm{x} \leq 1}$. Let $\left( x_n \right)_{n}\subseteq B$ converge to $x\in \ell_{\infty}$ in the $\ell_{\infty}$ norm.\newline

    Note that for all $n$, $\sup_{i\in \N} \left\vert x_n(i) \right\vert \leq 1$, meaning that since
    \begin{align*}
      \sup_{i\in \N} \left\vert x(i) - x_n(i) \right\vert &\rightarrow 0,
    \end{align*}
    we have that
    \begin{align*}
      \left\vert x(i) - x_n(i) \right\vert &\rightarrow 0
    \end{align*}
    for each $i$, so
    \begin{align*}
      x_n(i) &\rightarrow x(i)
    \end{align*}
    for all $i$. Thus, $\left\vert x(i) \right\vert \leq 1$ for all $i$, meaning $\sup_{i\in \N}\left\vert x(i) \right\vert \leq 1$, so $\norm{x} \in B$.
  \item Let $\ve = 1/2$, and consider the collection $\left( e_n \right)_n$ of sequences in $\ell_{\infty}$ consisting of $1$ at position $n$ and zero elsewhere. Then, $\left( e_n \right)_n\subseteq B$, but since $\sup_{i\in \N}\left\vert e_n(i)-e_m(i) \right\vert = 1$ for all $n\neq m$, we cannot have balls of radius $1/2$ cover the family $\left( e_n \right)_n$ with finitely many such balls, meaning that $B$ is not totally bounded.
\end{enumerate}
\section{Problem 20}%
\begin{enumerate}[(i)]
  \item We see that $d\left( x,y \right) = 0$ if and only if $x_n = y_n$ for each $n$, since each $d_n$ is a metric; therefore, $d\left( x,y \right) = 0$ if and only if $x = y$.\newline

    Furthermore, we have that for all $x = \left( x_n \right)_n$, $y = \left( y_n \right)_n$, and $z = \left( z_n \right)_n$, $\frac{1}{2^{n}}d\left( x_n,z_n \right) \leq \frac{1}{2^{n}}d\left( x_n,y_n \right) + \frac{1}{2^{n}}d\left( y_n,z_n \right)$. Therefore, we get
    \begin{align*}
      d\left( x,z \right) &= \sum_{n=1}^{\infty} \frac{1}{2^{n}}d\left( x_n,z_n \right)\\
                          &\leq \sum_{n=1}^{\infty}\frac{1}{2^{n}}d\left( x_n,y_n \right) + \frac{1}{2^{n}}d\left( y_n,z_n \right)\\
                          &\leq d\left( x,y \right) + d\left( y,z \right).
    \end{align*}
    Since $d\left( x_n,y_n \right),d\left( y_n,z_n \right) \leq 1$, these sums must converge, so $d\left( x,y \right)$ is indeed a metric.
  \item We will show that a sequence $\left( y_n \right)_n\subseteq X$ converges to $y\in X$ with the given distance metric if and only if it does so pointwise. This will show that the metric $d$ induces the topology of pointwise convergence, which is exactly the topology $\tau_{\text{prod}}$.\newline

    To start, let $\left( y_n \right)_n\rightarrow y$ in the given distance metric. Then, for all $\ve > 0$, there is some $N\in \N$ such that for all $n \geq N$, we have
    \begin{align*}
      d\left( y_n,y \right) &= \sum_{j=1}^{\infty}\frac{1}{2^{j}} d_j\left( y_n(j),y(j) \right)\\
                            &< \ve,
    \end{align*}
    so we see that for each $j$, $d\left( y_n(j),y(j) \right) < \ve$, meaning that $y_n(j)\rightarrow y(j)$ for each $j$.\newline

    Let $\left( y_n \right)_n\rightarrow y$ pointwise. If $\ve > 0$, convergence of series gives some $J$ such that $\sum_{j=J+1}^{\infty}\frac{1}{2^{j}} < \ve/2$, meaning that
    \begin{align*}
       \sum_{j=J+1}^{\infty}\frac{1}{2^{j}}d_j\left( y_n(j),y(j) \right) < \ve/2
    \end{align*}
    For $j = 1,\dots,J$, we find $N_1,\dots,N_J$ such that for all $n \geq N_j$, $d_j\left( y_n(j),y(j) \right) < \ve/2$. Therefore, for $n\geq \max\left( N_1,\dots,N_J \right)$, we have
    \begin{align*}
      d\left( y_n,y \right) &= \sum_{j=1}^{\infty}\frac{1}{2^{j}}d_j\left( y_n(j),y(j) \right)\\
                            &= \sum_{j=1}^{J} \frac{1}{2^{j}}d_j\left( y_n(j),y(j) \right) + \sum_{j=J+1}^{\infty}\frac{1}{2^{j}}d_j\left( y_n(j),y(j) \right)\\
                            &< \sum_{j=1}^{J} \frac{\ve}{2^{j+1}} + \frac{\ve}{2}\\
                            &< \frac{\ve}{2} + \frac{\ve}{2}\\
                            &= \ve.
    \end{align*}
    Therefore, $\left( y_n \right)_n\rightarrow y$ in our given distance metric.\newline

    Since convergence of sequences in our given distance metric is given by pointwise convergence, the induced topologies must be equal, so $\tau_{d} = \tau_{\text{prod}}$.
  \item We prove that $X$ is complete if and only if $X_n$ is complete for all $n$.\newline

    To see this, note that $\left( y_n \right)_n\subseteq X$ is Cauchy if and only if $\left( y_n(j) \right)_n\subseteq X_j$ is Cauchy for each $j$, as for all $\ve > 0$ and $m,n\geq N$ with $d\left( y_n,y_m \right) < \ve$, then
    \begin{align*}
      \sum_{j=1}^{\infty}\frac{1}{2^{j}} d_j\left( y_n(j),y_m(j) \right) &< \ve,
    \end{align*}
    meaning this holds for all $j$, and in the reverse direction, we use the same $\frac{\ve}{2}$ method from part (ii).\newline

    The sequence $\left( y_n \right)_n$ thus converges in $X$ if and only if every $y_n(j)$ converges in $X_j$ (as $\tau_d$ is the topology of pointwise convergence), meaning that $X$ is complete if and only if each $X_j$ is complete.
\end{enumerate}
\section{Problem 21}%
Let $X$ be complete, and let $\left( C_n \right)_n\subseteq P(X)$ be nonempty, decreasing, closed sets with $\operatorname{diam}\left( C_n \right)\rightarrow 0$.\newline

Let $\left( x_n \right)_n$ be defined by $x_n\in C_n$ for each $n$. Then, for any $\ve > 0$, we may find $C_N$ such that $\operatorname{diam}\left( C_N \right) < \ve$, meaning that for all $n,m\geq N$, we have that $x_n,x_m\in C_N$, so $d\left( x_n,x_m \right) < \ve$, meaning that $\left( x_n \right)_n$ is Cauchy. Since $X$ is complete, $\left( x_n \right)_n\rightarrow x$ for some $x\in X$. This point must be in all such $C_n$, meaning that
\begin{align*}
  \bigcap_{n=1}^{\infty}C_n &= \set{x}.
\end{align*}
Now, let $X$ be a metric space such that for any $\left( C_n \right)_n\subseteq P(X)$ nonempty, decreasing, and closed with $\operatorname{diam}\left( C_n \right)\rightarrow 0$, there is some $x\in X$ with $\bigcap_{n=1}^{\infty}C_n = \set{x}$. Let $\left( x_n \right)_n$ be a Cauchy sequence in $X$.\newline

Define a family of closed sets by
\begin{align*}
  C_n &= \overline{\set{x_n,x_{n+1},\dots}}.
\end{align*}
We note the following:
\begin{itemize}
  \item each of the $C_n$ is closed;
  \item $C_n \supseteq C_{n+1}$ by construction, since $\set{x_n,x_{n+1},\dots}\supseteq \set{x_{n+1},x_{n+2},\dots}$, and closures respect set inclusion;
  \item $\operatorname{diam}\left( C_n \right)\rightarrow 0$, as $\left( x_n \right)_n$ is Cauchy, so if $\ve > 0$, there is some $N$ such that for all $n,m \geq N$, $d\left( x_n,x_m \right) < \ve$, meaning that the diameter of the closure of the set $\set{x_{N},x_{N+1},\dots}$ is no more than $\ve$.
\end{itemize}
Therefore, there is some $x\in X$ such that
\begin{align*}
  \bigcap_{n=1}^{\infty}C_n &= \set{x},
\end{align*}
meaning that $\left( x_n \right)_n \rightarrow x$, and $X$ is complete.
\end{document}
