\documentclass[10pt]{mypackage}

\usepackage{mlmodern}
%\usepackage{newpxtext,eulerpx,eucal}
%\renewcommand*{\mathbb}[1]{\varmathbb{#1}}

\usepackage{homework}
%\usepackage{notes}

%\usepackage[ backend=bibtex, style = alphabetic, sorting=ynt ]{biblatex}
%\addbibresource{  }

\usepackage{parskip}

\fancyhf{}
\fancyhead[R]{Avinash Iyer}
\fancyhead[L]{Algebraic Topology: Homework 10}
\fancyfoot[C]{\thepage}

\setcounter{secnumdepth}{0}

\begin{document}
\RaggedRight
\begin{problem}
  Construct infinitely many non-homotopic retractions $S^1\vee S^1\rightarrow S^1$.
\end{problem}
\begin{solution}
  We will let $\left( A,a_0 \right)$ be the first (pointed) copy of $S^1$ and $\left( B,b_0 \right)$ the second copy. By the universal property of the disjoint union, we observe that any pair of self-maps $f_a\colon A\rightarrow A$ and $f_b\colon B\rightarrow B$ induce a self-map on the disjoint union, defined by
  \begin{align*}
    f\left( x \right) &= \begin{cases}
      f_a(x), & x\in A;\\
      f_b(x), & x\in B.
    \end{cases}
  \end{align*}
  If these maps preserve their respective basepoints, then we obtain a self-map on the wedge sum of $A$ and $B$ upon the identification $a_0\sim b_0$.

  Therefore, a self-map $A\vee B \rightarrow A\vee B$ is a retraction into $S^1$ if $f_a$ is a retraction onto $a_0$ or $f_b$ is a retraction onto $b_0$. Without loss of generality, we will assume that $f_b$ is a retraction of $B$ onto $b_0$. We may then define non-homotopic self-maps of $A$, which we will denote $\phi_n\colon A\rightarrow A$ to be representatives of the homotopy classes $\left[ \omega_n \right]$ looping around $S^1$ a total of $n$ times in a manner that preserves $a_0$.

  Since we have established that, whenever $n\neq m$, we have that $\left[ \omega_n \right]\neq \left[ \omega_m \right]$, it follows that upon collapsing with the wedge sum, we have that $A\wedge B$ retracts onto $A$, with each map being non-homotopic to the next map.
\end{solution}
\end{document}
