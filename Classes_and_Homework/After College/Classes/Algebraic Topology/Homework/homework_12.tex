\documentclass[10pt]{mypackage}

\usepackage{mlmodern}
%\usepackage{newpxtext,eulerpx,eucal}
%\renewcommand*{\mathbb}[1]{\varmathbb{#1}}

\usepackage{homework}
%\usepackage{notes}

%\usepackage[ backend=bibtex, style = alphabetic, sorting=ynt ]{biblatex}
%\addbibresource{  }

\usepackage{parskip}

\fancyhf{}
\fancyhead[R]{Avinash Iyer}
\fancyhead[L]{Algebraic Topology: Homework 12}
\fancyfoot[C]{\thepage}

\setcounter{secnumdepth}{0}

\begin{document}
\RaggedRight
\section{Revised Problems}%
\begin{problem}[Homework 8, Problem 1]
  Let $A$ be a path-connected subspace of a topological space $X$, and let $i\colon A\rightarrow X$ be the inclusion of $A$ into $X$. Show that for any $x_0\in A$, the induced map $i_{\ast}\colon \pi_1\left( A,x_0 \right)\rightarrow \pi_1\left( X,x_0 \right)$ is surjective if and only if every path in $X$ with endpoints in $A$ is homotopic to a path in $A$.
\end{problem}
\begin{solution}
  Suppose $i_{\ast}\colon \pi_1\left( A,x_0 \right)\rightarrow \pi_1\left( X,x_0 \right)$ is surjective. Then, by definition, any loop in $X$ based at $x_0$ is homotopic to a loop in $A$ based at $x_0$.

  Let $\gamma$ be a path in $X$ from $x_1\in A$ to $x_2\in A$. There are paths $\sigma_1$ from $x_0$ to $x_1$ and $\sigma_2$ from $x_2$ to $x_0$ in $A$. Composing $\sigma_1\cdot \gamma\cdot \sigma_2$ gives a loop in $X$ based at $x_0$. Therefore, this loop is homotopic to a loop in $A$, which we will call $\eta$. We start by showing that $\eta$ is homotopic to a loop that passes through both $x_1$ and $x_2$.

  Let $x_1' = \eta\left( 1/3 \right)$, $x_2' = \eta\left( 2/3 \right)$, and define $\eta|_{[0,1/3]}(3t) = \omega_1(t)$, $\eta|_{[2/3,1]}(3t-2) = \omega_2(t)$, and $\kappa(t) = \eta|_{[1/3,2/3]}(3t-2)$. Then, there are paths $\zeta_{1,2}\colon [0,1]\rightarrow A$ that go from $x_{1,2}'$ to $x_{1,2}$. We observe that, as maps, $\omega_1$ is homotopic to the path $\omega_1\cdot \zeta_1\cdot \overline{\zeta_1}$, and similarly, $\omega_2$ is homotopic to the path $ \zeta_2\cdot \overline{\zeta_2}\cdot \omega_2 $. Therefore, if we take the full concatenation
  \begin{align*}
    \eta' &= \left( \omega_1\cdot \zeta_1 \cdot \overline{\zeta_1} \right) \cdot \kappa \cdot \left( \zeta_2 \cdot \overline{\zeta_2}\cdot \omega_2 \right),
  \end{align*}
  we observe that it is homotopic to $\eta$ via a reparametrization, and it passes through $x_1$ and $x_2$. Therefore, by composing homotopies, we may assume that the original loop $\sigma_1\cdot \gamma\cdot \sigma_2$ is homotopic to a loop, $\chi$, passing through $x_1$ and $x_2$ in $A$. Using a reparametrization such that $\chi(1/3) = x_1$ and $\chi(2/3) = x_2$, this allows us to determine that $\chi|_{[1/3,2/3]}$ is homotopic as a path to $\gamma$.

  In the reverse direction, we observe that since any loop in $X$ with an endpoint in $A$ is homotopic to a loop in $A$, it follows that every homotopy class of loops in $X$ based at $x_0$ contains a representative that is a loop in $A$, so the induced homomorphism is surjective.
\end{solution}
\section{Current Problems}%
\begin{problem}
  Suppose that $Y$ is obtained from a space $X$ by attaching cells of dimension $n$, where $n\geq 3$. Prove that the inclusion $X\hookrightarrow Y$ induces an isomorphism $\pi_1(X)\cong \pi_1(Y)$.
\end{problem}
\begin{solution}
  We let $Z$ be a space given by attaching $I\times I$ ``on top of'' $\gamma_{\alpha}$ going from $x_0\in X$ to points $y_{\alpha}$ on $Y\cap X$, identifying $I\times \set{0}$ to the paths $\gamma_{\alpha}$, then identifying $\set{0}\times I$ with each other and $\set{1}\times I$ with sub-arcs of $D^n_{\alpha}\cap X$. Observe that $Z$ deformation retracts onto $Y$.

  Within each open cell $e^n_{\alpha}$ of $Y$, we select points $w_{\alpha}$ not on the sub-arcs described above, and define subsets
  \begin{align*}
    A &= Z\setminus \left( \bigcup_{\alpha}\set{w_{\alpha}} \right)\\
    B &= Z\setminus X.
  \end{align*}
  We see that since each $Y$ is contractible, each interval $I$, and each path, it follows that $B$ is contractible, meaning that $\pi_1\left( Z \right)$ is given by $\pi_1(A)/N$, where $N$ is the normal subgroup generated by the inclusion of $\pi_1(A\cap B)$ into $\pi_1(A)$. Note that $A$ deformation retracts onto $X$, meaning we only need to show that $\pi_1(A\cap B)$ is trivial.

  This follows from applying the van Kampen theorem to the open cover of $A\cap B$ by the subsets $V_{\alpha} = A\cap B \setminus \bigcup_{\beta\neq \alpha}e^n_{\beta}$. Each $A_{\alpha}$ deformation retracts onto $e^n_{\alpha}\setminus \set{w_{\alpha}}$, which deformation retracts to $S^{n-1}$. Since, for any $k\geq 2$, $S^2$ is contractible, it then follows that each of the $A_{\alpha}$ has trivial fundamental group, so $A\cap B$ has trivial fundamental group. Tracing back these equivalences gives
  \begin{align*}
    \pi_1(Y) &\cong \pi_1(Z)\\
             &\cong \left( \pi_1(A)\ast \pi_1(B) \right)/\left\langle \pi_1\left(A\cap B\right) \right\rangle\\
             &\cong \pi_1(A)\\
             &\cong \pi_1(X).
  \end{align*}
\end{solution}
\begin{problem}[Problem 2]
  Give three covering spaces of $S^1\vee S^2$.
\end{problem}
\begin{solution}
  We consider three covering spaces of $S^1\vee S^2$ given by $\R\sqcup S^2$ glued at $\left( 1,0,0 \right)\sim 0$, $\R\sqcup S^2\sqcup S^2$ glued at $\left( 1,0,0 \right)\sim 0, \left( 1,0,0 \right)\sim 1$, and $\R\sqcup S^2\sqcup S^2\sqcup S^2$ glued at $\left( 1,0,0 \right)\sim 0,\left( 1,0,0 \right)\sim 1,\left( 1,0,0 \right)\sim -1$.

  To consider the covering maps, we use the map $t\mapsto e^{2\pi i t}$ as the covering map taking $\R$ onto $S^{1}$, and consider any open subset of $S^2$ to have its preimage mapped to identical copies of the open subset of the copies of $S^2$ for the respective number of said copies.
\end{solution}
\begin{problem}[Problem 3]
  Prove that if $p_1\colon \widetilde{X}_1\rightarrow X_1$ and $p_2\colon \widetilde{X}_2\rightarrow X_2$ are covering spaces, then so is their product
  \begin{align*}
    p_1\times p_2\colon \widetilde{X}_1\times \widetilde{X}_2\rightarrow X_1\times X_2.
  \end{align*}
\end{problem}
\begin{solution}
  We let $\set{U_{\alpha}}_{\alpha}$ be an open cover of $X_1$ satisfying the covering map criteria, and similarly for $\set{V_{\beta}}_{\beta}$ and $X_2$. Then, we observe that $\set{U_{\alpha}\times V_{\beta}}_{\alpha,\beta}$ forms an open cover of $X_1\times X_2$. We claim that this map satisfies the covering map criteria. We observe that for arbitrary $\alpha$ and $\beta$, the definition of the product topology gives
  \begin{align*}
    \left( p_1\times p_2 \right)^{-1}\left( U_{\alpha}\times V_{\beta} \right) &= p_1^{-1}\left( U_{\alpha} \right)\times p_2^{-1}\left( V_{\beta} \right).
  \end{align*}
  This gives rise to a disjoint union
  \begin{align*}
    p_1^{-1}\left( U_{\alpha} \right)\times p_2^{-1}\left( V_{\beta} \right) &= \left( \bigsqcup_{i\in I}Y_i \right)\times \left( \bigsqcup_{j\in J} W_j \right),
  \end{align*}
  with each $W_j$ is homeomorphic to $V_{\beta}$ and each $Y_i$ is homeomorphic to $U_{\alpha}$. This means that for some distinguished $Y_i\times W_j$, we have that $Y_i\times W_j$ is homeomorphic to $U_{\alpha}\times V_{\beta}$, giving that $p_1\times p_2$ is a covering map.
\end{solution}
\end{document}
