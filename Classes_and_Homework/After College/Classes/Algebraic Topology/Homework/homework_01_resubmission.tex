\documentclass[10pt]{mypackage}

\usepackage{mlmodern}
%\usepackage{newpxtext,eulerpx,eucal}
%\renewcommand*{\mathbb}[1]{\varmathbb{#1}}

\usepackage{homework}
%\usepackage{notes}

%\usepackage[ backend=bibtex, style = alphabetic, sorting=ynt ]{biblatex}
%\addbibresource{  }

\usepackage{parskip}

\fancyhf{}
\fancyhead[R]{Avinash Iyer}
\fancyhead[L]{Homework 1: Resubmission}
\fancyfoot[C]{\thepage}

\setcounter{secnumdepth}{0}

\begin{document}
\RaggedRight
\begin{problem}[Problem 1 (a)]
  Fix topological spaces $X$ and $Y$, and consider the set of all continuous maps $X\rightarrow Y$. Define a relation on this set by saying that $f$ is related to $g$ whenever $f$ is homotopic to $g$. Prove that this relation is an equivalence relation.
\end{problem}
\begin{solution}
For reflexivity, we may select the identity homotopy $F\colon X\times I\rightarrow Y$, given by $F(x,t) = f(x)$ for all $t\in I$ and all $x\in X$.

For symmetry, if $F\colon X\times I\rightarrow Y$ is a homotopy with $F(x,0) = f(x)$ and $F(x,1) = g(x)$, then we may define the homotopy $G\colon X\times I\rightarrow Y$ by taking $G(x,t) = F(x,1-t)$. This is a composition of continuous maps, so it is continuous, and has $G(x,0) = g(x)$ and $G(x,1) = f(x)$, so the relation is symmetric.

For transitivity, we consider two homotopies $F\colon X\times I\rightarrow Y$ and $G\colon X\times I\rightarrow Y$. Define a homotopy $H\colon X\times I \rightarrow Y$ by
\begin{align*}
  H\left( x,t \right) &= \begin{cases}
    F\left( x,2t \right) & 0\leq t \leq 1/2\\
    G\left( x,2t-1 \right) & 1/2\leq t \leq 1.
  \end{cases}
\end{align*}
We claim that this map is continuous. To see this, we observe that if $U\subseteq Y$ is closed, then both $F^{-1}(U)\subseteq X\times I$ and $G^{-1}(U)\subseteq X\times I$ are closed; by taking intersections with the closed subsets $X\times [0,1/2]$ and $X\times [1/2,1]$, we see then that
\begin{align*}
  H^{-1}(U) &= \left( F^{-1}(U)\cap \left( X\times [0,1/2] \right) \right)\cup \left( G^{-1}(U)\cap \left( X\times [1/2,1] \right) \right),
\end{align*}
which is a finite union of closed subsets, so it is closed. In particular, this means $H$ is continuous.
\end{solution}
\begin{remark}
  The first two parts of this proof were sound, so I only edited my solution for the transitivity part of this problem.
\end{remark}
\end{document}
