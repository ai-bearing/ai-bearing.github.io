\documentclass[10pt]{mypackage}

\usepackage{mlmodern}
%\usepackage{newpxtext,eulerpx,eucal}
%\renewcommand*{\mathbb}[1]{\varmathbb{#1}}

\usepackage{homework}
%\usepackage{notes}

%\usepackage[ backend=bibtex, style = alphabetic, sorting=ynt ]{biblatex}
%\addbibresource{  }

\usepackage{parskip}

\fancyhf{}
\fancyhead[R]{Avinash Iyer}
\fancyhead[L]{Algebraic Topology: Homework 7ish}
\fancyfoot[C]{\thepage}

\setcounter{secnumdepth}{0}

\begin{document}
\RaggedRight
\begin{problem}[Homework 1, Problem 1 (c)]
  Show that ``is homotopic to'' is a transitive relation.
\end{problem}
\begin{solution}
  Suppose $f\simeq g$ and $g\simeq h$ are homotopic with homotopies defined by $F\colon I\times I\rightarrow X$ and $G\colon I\times I\rightarrow X$. We claim that the map $H\colon I\times I\rightarrow X$ defined by
  \begin{align*}
    H(s,t) &= \begin{cases}
      F\left( s,2t \right) & 0\leq t \leq 1/2\\
      G\left( s,2t-1 \right) & 1/2\leq t \leq 1
    \end{cases}
  \end{align*}
  is continuous. To see this, observe that by definition, $H$ is continuous on $I\times [0,1/2]$ (as it is equal to $F$ on that interval) and $H$ is continuous on $I\times [1/2,1]$ (as it is equal to $G$ on that interval). Since $I\times [0,1/2]$ and $I\times [1/2,1]$ are closed under the product topology, and $F(s,1) = G(s,0)$ by assumption, it follows from the pasting lemma that $H$ is continuous as both constituents of the piecewise definition for $H$ are continuous on closed subsets and are equal on their intersection.
\end{solution}
\begin{problem}[Homework 4, Problem 2]
  Prove that, if $f\colon X\rightarrow Y$ is a map, then $Y$ is a deformation retract of the mapping cylinder $M_f$.
\end{problem}
\begin{solution}
  Recall that the definition of the mapping cylinder is given by
  \begin{align*}
    M_f &= \left( X\times [0,1] \right)\sqcup Y/\left( \left( x,1 \right)\simeq f(x) \right).
  \end{align*}
  We consider the set
  \begin{align*}
    W &= \left( X\times [0,1] \right)\sqcup Y,
  \end{align*}
  and let $q\colon W\rightarrow M_f$ be the quotient map. By the universal property of the quotient map, it follows that if we are able to find a deformation retract from $W$ to $X\times \set{1}\sqcup Y$, then by composing with the quotient map, we will obtain a deformation retract from $M_f$ to $Y$.

  Define the homotopy
  \begin{align*}
    H\colon W\times [0,1] &\rightarrow W
  \end{align*}
  on each component separately by taking
  \begin{align*}
    H\left( w,t \right) &= \begin{cases}
      \left( p,\max\left( s,t \right) \right) & w = \left( p,s \right)\in X\times [0,1]\\
      y & w \in y
    \end{cases}.
  \end{align*}
  Since $H$ is defined on a disjoint union, it follows that if we are able to show that $H$ is continuous on eac component, then $H$ is continuous. First, we observe that the maximum function is continuous, so it follows that $H$ is continuous on $\left( X\times [0,1] \right)\times [0,1]$. Similarly, since $H$ is constant in $t$ along $Y\times [0,1]$, it follows that $H$ is continuous. Furthermore, we observe that the image of $H$ is equal to $X\times \set{1}\sqcup Y$, and the subset $X\times \set{1}\sqcup Y$ is constant along $H$ as $1\geq t$ for all $0\leq t \leq 1$. 

  Therefore, upon composing with the quotient map, we have that $q\circ H$ is a deformation retract from $M_f$ to $Y$.
\end{solution}
\begin{problem}
  Prove that if $f\colon X\rightarrow Y$ is an inclusion map, then the mapping cone $C_f$ is homotopy equivalent to the quotient $Y/f(X)$.
\end{problem}
\end{document}
