\documentclass[12pt]{mypackage}

%\usepackage{mlmodern}
%\usepackage{newpxtext,eulerpx,eucal}
%\renewcommand*{\mathbb}[1]{\varmathbb{#1}}

\usepackage{homework}
%\usepackage{notes}

%\usepackage[ backend=bibtex, style = alphabetic, sorting=ynt ]{biblatex}
%\addbibresource{  }

\usepackage{parskip}

\fancyhf{}
\fancyhead[R]{Avinash Iyer}
\fancyhead[L]{Algebraic Topology: Homework 11}
\fancyfoot[C]{\thepage}

\setcounter{secnumdepth}{0}

\begin{document}
\RaggedRight
\section{Revised Problems}%
\begin{problem}[Homework 5, Problem 1]
  If $X$ is a connected space that is a union of a finite number of $2$-spheres, any two of which intersect in at most one point, show that $X$ is homotopy-equivalent to a wedge sum of $1$-spheres and $2$-spheres.
\end{problem}
\begin{solution}
  Let $A_1,\dots,A_n$ be the spheres in $X$; we will define a graph $\Gamma$ where $V(\Gamma)$ denotes each of the spheres of $X$ and $E(\Gamma)$ is the edge set defined defined by $\set{v_i,v_j}\in E(\Gamma)$ if and only if $A_i$ and $A_j$ are connected. Note that by our assumption, we have that $\Gamma$ is a simple and connected graph.

  First, we show that if $\Gamma$ is a tree, then $X$ is homotopy-equivalent to a wedge sum of $2$-spheres. First, assign $\Gamma$ a distinguished root vertex. Endow $X$ with a CW complex structure by assigning
  \begin{itemize}
    \item $0$-cells at each intersection point;
    \item extra $0$-cells at each leaf and at the root of the tree, the latter of which we will denote $a_0$;
    \item $1$-cells along the equator connecting all the $0$-cells;
    \item $2$-cells completing the respective spheres.
  \end{itemize}
  Since $\Gamma$ is a tree, there is a unique path from the root of the vertex to each leaf. Traversing along a path from $a_0$ to the extra $0$-cell on the leaf via $1$-cells connecting to each intersection point on the intermediate spheres, we obtain a contractible subcomplex of $X$. Upon taking the quotient, we find that $X$ is homotopy-equivalent to a wedge sum of all the spheres along this path (including the leaf) with the rest of $X$ with connection point at $a_0$. Continuing in this fashion then gives that, if $\Gamma$ is a tree, then $X$ is homotopy-equivalent to a wedge sum of spheres.

  If $\Gamma$ is not a tree, then $\Gamma$ admits spanning tree, and the rest of $\Gamma$ is then given by a collection of edges that complete some number of cycles. Therefore, we show that if $\Gamma$ is a cycle, then $X$ is homotopy-equivalent to a wedge of $2$-spheres (corresponding to each vertex) and a single $1$-sphere. For this, observe that if $\Gamma$ is a cycle, then $X$ is homotopy-equivalent to a line of spheres connected by $0$-cells at their equator with one extra $1$-cell connecting $0$-cells at the endpoints of the line. Collapsing along this equator will then give all the $2$-spheres as a wedge sum identified with a single point,a swell as both ends of the extra $1$-cell, meaning that we have that, in this case, $X$ is a wedge sum of these $2$-spheres with the $1$-sphere corresponding to the extra $1$-cell.

  Therefore, in the general case, we may start by collapsing $X$ along its spanning tree, which will necessarily collapse along every cycle, giving that $X$ is a wedge sum of $1$-spheres and $2$-spheres.
\end{solution}
\section{Current Problems}%
\begin{problem}[Problem 1]
  Consider the quotient space of $S^2$ obtained by identifying the north and south poles to a single point. Compute its fundamental group.
\end{problem}
\begin{solution}
  We observe in the figure below that the quotient can be expressed by homotopy-equivalent quotients of the sphere union a $1$-cell $A$ connecting between the north pole and the south pole. Therefore, this quotient space is homotopy-equivalent to $S^2\vee S^1$, so that $\pi_1\left( X \right) = \Z\ast \set{e} = \Z$.
  \begin{center}
    \includegraphics[width=10cm]{images/deformation_retraction_sphere.png}
  \end{center}
\end{solution}
\end{document}
