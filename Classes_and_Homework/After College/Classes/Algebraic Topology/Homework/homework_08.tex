\documentclass[12pt]{mypackage}

\usepackage{mlmodern}
%\usepackage{newpxtext,eulerpx,eucal}
%\renewcommand*{\mathbb}[1]{\varmathbb{#1}}

\usepackage{homework}
%\usepackage{notes}

%\usepackage[ backend=bibtex, style = alphabetic, sorting=ynt ]{biblatex}
%\addbibresource{  }

\usepackage{parskip}

\fancyhf{}
\fancyhead[R]{Avinash Iyer}
\fancyhead[L]{Algebraic Topology: Homework 8}
\fancyfoot[C]{\thepage}

\setcounter{secnumdepth}{0}

\begin{document}
\RaggedRight
\section{Revised Problems}%
\begin{problem}[Homework 3, Problem 3 (b)]
  Prove that $S^{\infty}$ is contractible.
\end{problem}
\begin{solution}
  We view $S^{\infty}$ as a topological subspace of $\R^{\infty}$ (finitely supported real sequences) equipped with the Euclidean norm; i.e., if $\left( x_n \right)\in \R^{\infty}$, then
  \begin{align*}
    \norm{\left( x_n \right)} &= \left( \sum_{i=0}^{\infty} x_i^2 \right)^{1/2},
  \end{align*}
  where the sum is finite by definition. We consider $S^{\infty}\subseteq \R^{\infty}$ to be the space of all finitely supported sequences $\left( x_n \right)$ such that
  \begin{align*}
    \norm{\left( x_n \right)} &= 1.
  \end{align*}
  We consider the continuous $1$-parameter family given by
  \begin{align*}
    f_t\left( x_n \right) &= \left( 1-t \right)\left( x_n \right) + tV\left( x_n \right),
  \end{align*}
  where $V$ denote the right unilateral shift mapping $\left( x_0,x_1,\dots \right)$ to $\left( 0,x_0,x_1,\dots \right)$. To show that $f_t\left( x_n \right)$ is never zero, we start by considering $\left( x_0,\dots,x_k,0 \right)$ viewed in $S^{k+1}$, and observe then that $f_t$, restricted to $S^{k+1}$, yields
  \begin{align*}
    f_t\left( \left( x_0,\dots,x_k,0 \right) \right) &= \left( \left( 1-t \right)x_0,\left( 1-t \right)x_1 + tx_0,\dots,\left( 1-t \right)x_{k} + t x_{k-1}, tx_k \right).
  \end{align*}
  Without loss of generality, we may consider $x_0$ as being nonzero; then, we observe that the second coordinate $1 + t\left( x_0-x_1 \right)$ will be nonzero for all $0\leq t \leq 1$ if $\left\vert x_0-x_1 \right\vert < 1$, and will be zero at $t = 1$ only when $\left\vert x_0-x_1 \right\vert = 1$, but that can only happen if either $x_0$ or $x_1$ is $1$ and every other coordinate is $0$; yet, in such a scenario, it is necessarily the case that $f_t$ is nonzero. Therefore, $\norm{f_t}$ is nonzero for all $0\leq t \leq 1$ acting on $S^{\infty}$.

  In particular, when we consider the homotopy $H\colon S^{\infty}\times [0,1]\rightarrow S^{\infty}$ given by
  \begin{align*}
    H\left( \left( x_n \right),t \right) &= \begin{cases}
      \left( 1-t \right)\left( x_n \right) + 2tV\left( x_n \right) & 0\leq t \leq 1/2\\
      \left( 2-2t \right)V\left( x_n \right) + \left( 2t-1 \right)\left( 1,0,\dots \right) & 1/2\leq t \leq 1
    \end{cases},
  \end{align*}
  we observe that $H$ is continuous along each of $S^{\infty}\times [0,1/2]$ and $S^{\infty}\times [1/2,1]$, and is equal at $t=1/2$, so by the pasting lemma, $H$ is continuous along $[0,1]$. Since $H\left( \cdot,t \right)/\norm{H\left( \cdot,t \right)}$ is contained in $S^{\infty}$ (with well-definedness following from the earlier discussion), and is a homotopy between the identity and a constant map, it follows that the identity is null-homotopic, so $S^{\infty}$ is contractible.
\end{solution}
\begin{problem}[Homework 6, Problem 2]
  Prove that, for a path-connected space $X$, the fundamental group $\pi_1(X)$ is abelian if and only if all the change-of-basepoint isomorphisms $\beta_h$ depend only on the endpoints of the path $h$, not on the precise path.
\end{problem}
\begin{solution}
  Let $X$ be path-connected, and suppose $\pi_1(X)$ is abelian. Let $x_0,x_1$ be distinct points in $X$ with distinct paths $h_1$ and $h_2$ connecting $x_0$ and $x_1$. We show that $\beta_{ \overline{h_2} }\beta_{h_1}$ is identity on $\pi_1\left( X,x_0 \right)$.

  If $f$ is a loop based at $x_0$, then we have
  \begin{align*}
    \beta_{ \overline{h_2} }\beta_{ h_1 } \left[ f \right] &= \beta_{ \overline{h_2} } \left[ \overline{h_1}\cdot f \cdot h_1 \right]\\
                                                           &= \left[ h_2\cdot \overline{h_1}\cdot f \cdot h_1 \cdot \overline{h_2} \right]\\
                                                           &= \left[ f \right]\left[ h_2\cdot \overline{h_1} \right]\left[ h_1\cdot \overline{h_2} \right]\\
                                                           &= \left[ f \right] \left[ h_2\cdot \overline{h_1}\cdot h_1\cdot \overline{h_2} \right]\\
                                                           &= \left[ f \right] \left[ h_2\cdot c\cdot \overline{h_2}  \right]\\
                                                           &= \left[ f \right]\left[ h_2\cdot \overline{h_2} \right]\\
                                                           &= \left[ f \right] \left[ h_2\cdot \overline{h_2} \right]\\
                                                           &= \left[ f \right]\left[ c \right]\\
                                                           &= \left[ f \right].
  \end{align*}
  Therefore, $\beta_{h_1} = \beta_{h_2}$.

  Now, suppose all change of basepoint isomorphisms are independent of the path. This necessarily holds for loops, so if $ \left[ \gamma \right], \left[ f \right]\in \pi_1\left( X,x_0 \right) $, since $\beta_{\gamma}$ does not change the basepoint, we must have that $\left[ f \right] = \beta_{\gamma}\left[ f \right]$, or that
  \begin{align*}
    \left[ f \right] &= \left[ \overline{\gamma}\cdot f \cdot \gamma \right]\\
                     &= \left[ \gamma \right]^{-1}\left[ f \right]\left[ \gamma \right],
  \end{align*}
  giving that $\left[ \gamma \right]\left[ f \right] = \left[ f \right]\left[ \gamma \right]$.
\end{solution}
\section{Current Problems}%
\begin{problem}[Problem 1]
  Let $A$ be a path-connected subspace of a topological space $X$, and let $i\colon A\rightarrow X$ be inclusion. Show that for any $x_0\in A$, the induced map $ i_{\ast}\colon \pi_1\left( A,x_0 \right)\rightarrow \pi_1\left( X,x_0 \right) $ is surjective if and only if every path in $X$ with endpoints in $A$ is homotopic to a path in $A$.
\end{problem}
\begin{solution}
  Suppose $i_{\ast}\colon \pi_1\left( A,x_0 \right)\rightarrow \pi_1\left( X,x_0 \right)$ is surjective. That is, for any homotopy class $\left[ f \right]\in \pi_1\left( X,x_0 \right)$ of a loop, there is some homotopy class $\left[ g \right]\in \pi_1\left( A,x_0 \right)$ such that $i_{\ast}\left[ g \right] = \left[ f \right]$. In particular, this means that 
  \begin{align*}
    i_{\ast}\left[ g \right] &= \left[ i\circ g \right]\\
                             &= \left[ g \right]\\
                             &= \left[ f \right],
  \end{align*}
  so that any loop based at $x_0$ in $X$ is homotopic to a loop based at $x_0$ in $A$.

  Now, let $\gamma$ be a path in $X$ with endpoints $x_0$ and $x_1$, and let $\zeta$ be a path from $x_1$ to $x_0$ in $A$. Then, the concatenation $\gamma\cdot \zeta$ is a loop in $X$ based at $x_0$, so it is homotopic to a loop $\gamma'$ in $A$ based at $x_0$. Let $x_2 = \gamma'(1/2)$; then, since $A$ is path-connected, it follows that the path $\eta(t) = \gamma'(t/2)$ is homotopic (as a map) to a path in $A$ that has an endpoint at $x_1$, which we will call $\eta'$. Similarly, the path $\xi(t) = \gamma'\left((t+1)/2\right)$ is homotopic (as a map) to $\zeta$, meaning that we have the chain of homotopies
  \begin{align*}
    \gamma\cdot \zeta &\simeq \gamma'\\
                      &= \eta\cdot \xi\\
                      &\simeq \eta'\cdot \xi\\
                      &\simeq \eta'\cdot \zeta,
  \end{align*}
  so when we concatenate $ \overline{\zeta} $ on both sides (which is kosher as we have fixed endpoints $x_0$ and $x_1$), we have that $\gamma\cdot c\simeq \eta'\cdot c$, or that $\gamma\simeq \eta$. For the general case of any two endpoints, $y_0$ and $y_1$, we create the desired loop, use the change-of-basepoint isomorphism, then use the inverse change-of-basepoint isomorphism.

  In the reverse direction, we observe that since any loop in $X$ with an endpoint in $A$ is necessarily homotopic to a loop in $A$, meaning that any homotopy class of loops in $X$ based at $x_0$ includes a representative that is a loop in $A$, meaning that the induced homomorphism is surjective.
\end{solution}
\begin{problem}[Problem 2]
  Show that there is no retraction $r\colon S^1\times D^2\rightarrow S^1\times S^1$.
\end{problem}
\begin{solution}
  Suppose such a retraction existed. We see that the sequence $S^1\times S^1 \xrightarrow{i} S^1\times D^2\xrightarrow{r}S^1\times S^1$ induces homomorphisms $ \Z^{2}\xrightarrow{i_{\ast}}\Z\xrightarrow{r_{\ast}}\Z^2 $ such that $r_{\ast}\circ i_{\ast} = \id$. The map $r_{\ast}$ is then a surjective homomorphism between $\Z$ and $\Z^2$, meaning that we have $\Z/\ker\left( r_{\ast} \right)\cong \Z^2$, but since $\Z$ is a principal ideal domain, we have that $\ker\left( r_{\ast} \right) = \left( v \right)$ for some $v\in \Z$; since $\Z^2$ is infinite, and $\Z/\left( v \right)$ is finite whenever $v\neq 0$, it follows that we have that $v$ must be equal to $0$, so $r_{\ast}$ is thus injective. Yet, this implies the existence of an isomorphism between $\Z$ and $\Z^2$, which violates the structure of finitely generated abelian groups.
\end{solution}
\end{document}
