\documentclass[10pt]{mypackage}

\usepackage{mlmodern}
%\usepackage{newpxtext,eulerpx,eucal}
%\renewcommand*{\mathbb}[1]{\varmathbb{#1}}

\usepackage{homework}
%\usepackage{notes}

%\usepackage[ backend=bibtex, style = alphabetic, sorting=ynt ]{biblatex}
%\addbibresource{  }

\usepackage{parskip}

\fancyhf{}
\fancyhead[R]{Avinash Iyer}
\fancyhead[L]{Algebraic Topology: Homework 4}
\fancyfoot[C]{\thepage}

\setcounter{secnumdepth}{0}

\begin{document}
\RaggedRight
\begin{problem}[Problem 1]
  Let $\left( X,x_0 \right)$ be a pointed CW complex. Prove that $\Sigma X = X\wedge S^1$.
\end{problem}
\begin{solution}
  The reduced suspension yields the identifications
  \begin{align*}
    \Sigma X &= X\times [0,1]/\left(X\times \set{0}\sim \set{x_0},X\times \set{1}\sim \set{x_0},\set{x_0}\times [0,1]\sim \set{x_0}\right)\\
             &= X\times [0,1]/\left( X\times \set{1}\sim \set{x_0}\times [0,1],X\times \set{1}\sim \set{x_0}\times [0,1] \right).
  \end{align*}
  Now, if we consider $\set{0} = \set{1}$ to be the basepoint of $S^1$, this yields our desired smash product
  \begin{align*}
    X\wedge S^1 &= X\times S^1/\left( X\times \set{0}\sim \set{x_0}\times S^1 \right)\\
                &= \Sigma X.
  \end{align*}
\end{solution}
\begin{problem}[Problem 2]
  Prove that, if $f\colon X\rightarrow Y$ is a map, then $Y$ is a deformation retract of the mapping cylinder $M_f$.
\end{problem}
\begin{solution}
  Recalling the definition of the mapping cylinder, we have
  \begin{align*}
    M_f &= X\times [0,1]\coprod Y/\left( \left( x,1 \right)\sim f(x) \right).
  \end{align*}
  In fact this allows us to define a homotopy from the identity on $M_f$ into $Y$ by taking
  \begin{align*}
    f_t(p) &= \begin{cases}
      p & p\in Y\\
      \left( p,t \right) & p\in X
    \end{cases}.
  \end{align*}
  We observe that this is a constant homotopy when restricted to $Y$, that it is continuous in $t$ as it is just a progression along $\set{p}\times [0,1]$, and that
  \begin{align*}
    f_1(p) &= \begin{cases}
      p & p\in Y\\
      f(p) & p\in X
    \end{cases},
  \end{align*}
  so this is a deformation retract.
\end{solution}
\begin{problem}[Problem 3]
  Prove that, if $f\colon X\hookrightarrow Y$ is an inclusion map, then the mapping cone $C_f$ is homotopy equivalent to the quotient $Y/f(X)$.
\end{problem}
\begin{solution}
  First, we observe that $f$ is an inclusion map, so $X\subseteq Y$ and $f(x) = x$ for all $x\in X$. Additionally, the mapping cylinder $M_f$ is given by
  \begin{align*}
    M_f &= X\times [0,1]\sqcup Y/\left( \left( x,1 \right)\sim x \right),
  \end{align*}
  and that $Y$ is homotopy equivalent to $M_f$ from Problem 2. We observe then that the mapping cone $C_f$ is the quotient $M_f/X\times \set{0}$, while we see that
  \begin{align*}
    M_f/X\times \set{1} &= \left( X\times [0,1]\sqcup Y/X \right)/\left( \left( x,1 \right)\sim x \right),
  \end{align*}
  which we can view as an ``inverted'' mapping cone, of sorts. In particular, this ``inverted mapping cone'' deformation retracts via straight line homotopy to $[x]\in Y/X$ for each $x\in X$, which is a single point. Therefore, we may define a homotopy (with corresponding homotopy inverse)
  \begin{align*}
    H\colon C_f\times [0,1]\rightarrow Y/f(X)
  \end{align*}
  given by
  \begin{align*}
    H_t(p) &= [p],
  \end{align*}
  where the equivalence class for $p$ is considered in $M_f/\left( X\times \set{t} \right)$. This map is continuous by the definition of the quotient topology, so this is our desired homotopy equivalence.
\end{solution}
\end{document}
