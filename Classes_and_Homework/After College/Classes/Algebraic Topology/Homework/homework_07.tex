\documentclass[10pt]{mypackage}

\usepackage{mlmodern}
%\usepackage{newpxtext,eulerpx,eucal}
%\renewcommand*{\mathbb}[1]{\varmathbb{#1}}

\usepackage{homework}
%\usepackage{notes}

%\usepackage[ backend=bibtex, style = alphabetic, sorting=ynt ]{biblatex}
%\addbibresource{  }

\usepackage{parskip}

\fancyhf{}
\fancyhead[R]{Avinash Iyer}
\fancyhead[L]{Algebraic Topology: Homework 7}
\fancyfoot[C]{\thepage}

\setcounter{secnumdepth}{0}

\begin{document}
\RaggedRight
\section{Revised Problems}%
\begin{problem}[Homework 5, Problem 1]
  If $X$ is a connected space that is a union of a finite number of $2$-spheres, any two of which intersect in at most one point, show that $X$ is homotopy-equivalent to a wedge sum of $1$-spheres and $2$-spheres.
\end{problem}
\begin{solution}
  We start by constructing a space that is homotopy-equivalent to $X$ as follows. Label the spheres in $X$ by $A_1,\dots,A_n$, and equip each $A_i$ with the cell complex structure consisting of two $0$-cells, two $1$-cells, and two $2$-cells. Line up these spheres such that $A_i$ and $A_{i+1}$ have non-empty intersection for each $0\leq i \leq n-1$, renumbering as necessary; such an arrangement must exist since $X$ is connected. In this arrangement, we consider $A_{i}$ and $A_{i+1}$ to be adjoined at a common $0$-cell.

  Beyond this arrangement, if there are $A_i$ and $A_j$ adjacent to each other in $X$ but not adjacent to each other in this line of spheres, insert a $0$-cell along the equator of both $A_i$ and $A_j$, then insert a $1$-cell connecting them. Observe that upon completion of this process, this new space is homotopy-equivalent to $X$, as $X$ is homeomorphic to the quotient by this collection of inserted $1$-cells. Without loss of generality, we may assume that all these newly inserted $1$-cells are connected along the same side of the equator. The figure below displays this idea.
  \begin{center}
    \includegraphics[width=7cm]{images/cw_equivalence_1.png}
  \end{center}
  Upon collapsing via the subcomplex described in the following figure, we observe that, since this subcomplex is contractible, as it is a collection of $0$-cells and $1$-cells that does not admit any sub-subcomplex homeomorphic to $S^1$, we get a homotopy equivalence where each of the spheres and each of the extra $1$-cells are all identified with a single point, which is a wedge sum of $1$-spheres and $2$-spheres.
  \begin{center}
    \includegraphics[width=7cm]{images/cw_equivalence_2.png}
  \end{center}
\end{solution}
\section{Current Problems}%
\begin{problem}[Problem 1]
  Prove that for any space $X$, the following definitions are equivalent:
  \begin{enumerate}[(a)]
    \item any map $S_1\rightarrow X$ is homotopic to a constant map;
    \item any map $S_1\rightarrow X$ can be extended to a map $D^2\rightarrow X$;
    \item $\pi_1\left( X,x_0 \right) = 0$ for any $x_0\in X$.
  \end{enumerate}
\end{problem}
\begin{solution}
  Let $F\colon S^1\times [0,1]\rightarrow X$ be a homotopy between an arbitrary $f\colon S^1\rightarrow X$ and a constant map $c$. Since $D^2$ is convex, we may use the straight-line homotopy $H\left( s,t \right) = \left( 1-t \right)e^{is}$ that maps from $S^{1}$ to $0\in D^2$.

  The extension $\hat{f}\colon D^2\rightarrow X$, given by
  \begin{align*}
    \hat{f}\left( re^{is} \right) &= F\left( s,1-r \right),
  \end{align*}
  where $s\in [0,2\pi)$ and $r\in [0,1]$, is a composition of two continuous functions, $F\left( s,t \right)$ and $r\mapsto 1-r$, so it is continuous with $\hat{f}\left( e^{is} \right) = F\left( s,0 \right) = f\left( e^{is} \right)$.

  Now, suppose any map $S^1\rightarrow X$ has an extension to a map $D^2\rightarrow X$. Consider a pointed map from $S^1$ to $x_0\in X$ taking $1 = e^{i0}\mapsto x_0$, which we call $f$. Then, $f$ has an extension to the closed unit disk, which we call $ \hat{f} $, which maps from $D^{2}$ to $X$ and takes $S_1\mapsto f\left( S^1 \right)$. Using the straight-line homotopy (and the triangle inequality), we have that $H\left( re^{is},t \right) = t + \left( 1-t \right)re^{is}$ is such that every point in $D^2$ maps to $1$, so upon composing this homotopy with $\hat{f}$, we find that $f$ is homotopic to the constant map at $x_0$, meaning that $\pi_1\left( X,x_0 \right) = 0$.

  If $\pi_1\left( X,x_0 \right) = 0$ for any $x_0\in X$, then for any $x_0\in X$, it follows that any map $f\colon S^1\rightarrow X$ has the same homotopy class as the constant map at $x_0$, giving (a).
\end{solution}
\begin{problem}[Problem 2]
  We have talked about how $\pi_1\left( X,x_0 \right)$ can be thought of as homotopy classes of basepoint-preserving maps $\left( S^1,s_0 \right)\rightarrow \left( X,x_0 \right)$. Consider now the set of homotopy classes of non-basepoint-preserving maps $S^1\rightarrow X$, denoted $\left[ S^1,X \right]$. Forgetting basepoints, we obtain a map $\Psi\colon \pi_1\left( X,x_0 \right)\rightarrow \left[ S^1,X \right]$.
  \begin{enumerate}[(a)]
    \item Prove that if $X$ is path-connected, then $\Psi$ is onto.
    \item Prove that $\Psi\left[ f \right] = \Psi\left[ g \right]$ if and only if $\left[ f \right]$ and $\left[ g \right]$ are conjugate in $\pi_1\left( X,x_0 \right)$.
  \end{enumerate}
\end{problem}
\begin{solution}\hfill
  \begin{enumerate}[(a)]
    \item Let $X$ be path-connected, and let $\left[ f \right]$ be a homotopy class of a map $f\colon S^1\rightarrow X$; define $y_0 = f\left( s_0 \right)$. Since $X$ is path-connected, there is a path $\gamma$ from $x_0$ to $y_0$; the path defined by $\gamma\cdot f \cdot \overline{\gamma}$ is then a loop based at $x_0$, so it admits a homotopy class $\left[ \gamma\cdot f \cdot \overline{\gamma} \right]\in \pi_1\left( X,x_0 \right)$. We claim that, in fact, $\left[ \gamma\cdot f \cdot \gamma\right] = \left[ f \right]$.

      To see this, consider the homotopy
      \begin{align*}
        H\left( s,t \right) &= \begin{cases}
          \gamma(t) & 0\leq s \leq t\\
          \gamma(s) & t\leq s \leq 1
        \end{cases},
      \end{align*}
      which is continuous (since $\gamma$) is continuous, and has $\left[ \gamma \right] = \left[ c_{y_0} \right]$. By defining $\gamma_t\coloneq H\left( \cdot,t \right)$, we then have a continuous $1$-parameter family that yields the desired homotopy $\left[ \gamma\cdot f \cdot \overline{\gamma} \right]\simeq \left[ c_{y_0}\cdot f \cdot \overline{c_{y_0}} \right] = \left[ f \right]$. Thus, when $X$ is path-connected, $\Psi$ is onto.
    \item Without loss of generality, we let $1 = e^{i0}$ be the basepoint for $S^1$. Suppose $\Psi\left[ f \right] = \Psi\left[ g \right]$, where $f$ and $g$ are pointed maps taking $1\mapsto x_0$. Let $H$ be a homotopy between $f$ and $g$, considered in $\left[ S^1,X \right]$; when we restrict our view to $H\left( 1,t \right)$, we observe that $H\left( 1,t \right)$ is a path in $X$, where $H\left( 1,0 \right) = f(1) = x_0$ and $H\left( 1,1 \right) = g(1) = x_0$, meaning that $H\left( 1,t \right)$ is a loop based at $x_0$. Call it $\gamma(t)$.

      I'm not exactly sure where to go from here, but I think the idea would be to show that either $f\cdot\gamma \simeq \gamma\cdot g$ or $\gamma\cdot f \simeq g\cdot\gamma$, which would show that $\left[ f \right]$ and $\left[ g \right]$ are conjugate in $\pi_1\left( X,x_0 \right)$.
  \end{enumerate}
\end{solution}
\end{document}
