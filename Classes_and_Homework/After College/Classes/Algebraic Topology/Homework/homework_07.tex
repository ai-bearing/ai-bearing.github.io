\documentclass[10pt]{mypackage}

\usepackage{mlmodern}
%\usepackage{newpxtext,eulerpx,eucal}
%\renewcommand*{\mathbb}[1]{\varmathbb{#1}}

\usepackage{homework}
%\usepackage{notes}

%\usepackage[ backend=bibtex, style = alphabetic, sorting=ynt ]{biblatex}
%\addbibresource{  }

\usepackage{parskip}

\fancyhf{}
\fancyhead[R]{Avinash Iyer}
\fancyhead[L]{Algebraic Topology: Homework 7}
\fancyfoot[C]{\thepage}

\setcounter{secnumdepth}{0}

\begin{document}
\RaggedRight
\section{Revised Problems}%
\begin{problem}[Homework 5, Problem 1]
  If $X$ is a connected space that is a union of a finite number of $2$-spheres, any two of which intersect in at most one point, show that $X$ is homotopy-equivalent to a wedge sum of $1$-spheres and $2$-spheres.
\end{problem}
\begin{solution}
  We start by constructing a space that is homotopy-equivalent to $X$ as follows. Label the spheres in $X$ by $A_1,\dots,A_n$, and equip each $A_i$ with the cell complex structure consisting of two $0$-cells, two $1$-cells, and two $2$-cells. Line up these spheres such that $A_i$ and $A_{i+1}$ have non-empty intersection for each $0\leq i \leq n-1$, renumbering as necessary; such an arrangement must exist since $X$ is connected. In this arrangement, we consider $A_{i}$ and $A_{i+1}$ to be adjoined at a common $0$-cell.

  Beyond this arrangement, if there are $A_i$ and $A_j$ adjacent to each other in $X$ but not adjacent to each other in this line of spheres, insert a $0$-cell along the equator of both $A_i$ and $A_j$, then insert a $1$-cell connecting them. Observe that upon completion of this process, this new space is homotopy-equivalent to $X$, as $X$ is homeomorphic to the quotient by this collection of inserted $1$-cells. Without loss of generality, we may assume that all these newly inserted $1$-cells are connected along the same side of the equator.
\end{solution}
\end{document}
