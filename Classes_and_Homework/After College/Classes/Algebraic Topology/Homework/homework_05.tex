\documentclass[12pt]{mypackage}

\usepackage{mlmodern}
%\usepackage{newpxtext,eulerpx,eucal}
%\renewcommand*{\mathbb}[1]{\varmathbb{#1}}

\usepackage{homework}
%\usepackage{notes}

%\usepackage[ backend=bibtex, style = alphabetic, sorting=ynt ]{biblatex}
%\addbibresource{  }

\usepackage{parskip}

\fancyhf{}
\fancyhead[R]{Avinash Iyer}
\fancyhead[L]{Algebraic Topology: Homework 5}
\fancyfoot[C]{\thepage}

\setcounter{secnumdepth}{0}

\begin{document}
\RaggedRight
\begin{problem}[Problem 1]
  If $X$ is a connected space that is a union of a finite number of $2$-spheres, any two of which intersect in at most one point, show that $X$ is homotopy-equivalent to a wedge sum of $1$-spheres and $2$-spheres.
\end{problem}
\begin{solution}
  We consider $X$ to have a CW complex structure consisting of $0$-cells at each intersection point, $1$-cells connecting each intersection point, and $2$-cells ``constructing'' the sphere. Our primary method of construction will be via collapsing along select $1$-cells, which since these $1$-cells are contractible, will yield a homotopy-equivalent CW complex.

  If we have a straight line of spheres connected one after another by their equator, we collapse along one side of this equator to yield a wedge sum of $2$-spheres (as in Example 0.9 in Hatcher).

  Else, we create a homotopy equivalent CW complex consisting of a straight line of spheres as in the previous case, introducing $1$-cells along one side of the equator to denote identifications among spheres adjacent in $X$ that are not adjacent in the straight line. By collapsing along the equator of this straight line as in the previous case, this yields a wedge sum of $1$-spheres (the extra $1$-cells that are not in the original straight line of spheres) and $2$-spheres.
\end{solution}
\end{document}
