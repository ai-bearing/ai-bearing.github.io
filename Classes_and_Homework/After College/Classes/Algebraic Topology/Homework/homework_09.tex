\documentclass[10pt]{mypackage}

\usepackage{mlmodern}
%\usepackage{newpxtext,eulerpx,eucal}
%\renewcommand*{\mathbb}[1]{\varmathbb{#1}}

\usepackage{homework}
%\usepackage{notes}

%\usepackage[ backend=bibtex, style = alphabetic, sorting=ynt ]{biblatex}
%\addbibresource{  }

\usepackage{parskip}

\fancyhf{}
\fancyhead[R]{Avinash Iyer}
\fancyhead[L]{Algebraic Topology: Homework 9}
\fancyfoot[C]{\thepage}

\setcounter{secnumdepth}{0}

\begin{document}
\RaggedRight
\section{Revised Problem}%
\begin{problem}[Homework 4, Problem 2]
  Prove that, if $f\colon X\rightarrow Y$ is a map, then $Y$ is a deformation retract of the mapping cylinder $M_f$.
\end{problem}
\begin{solution}
  We consider the set $W = \left( X\times [0,1] \right)\sqcup Y$, where we let $q\colon W\rightarrow M_f$ be the quotient map that identifies $\left( x,1 \right)\sim f(x)$. We will show that $W$ admits a deformation retract to $X\times \set{1}\sqcup Y$, as by composing with the quotient map and using the universal property, we then obtain a deformation retract from $M_f$ to $Y$.

  Now, define the homotopy 
  \begin{align*}
    H\colon W\times [0,1] \rightarrow W
  \end{align*}
  on each component separately, taking
  \begin{align*}
    H\left( w,t \right) &= \begin{cases}
      \left( p,\max\left( s,t \right) \right), & w = \left( p,s \right)\in X\times[ 0,1 ];\\
      y, & w= y\in Y.
    \end{cases}
  \end{align*}
  Since $H$ is defined on a disjoint union, we only need to show that $H$ is continuous on each component, as then $H$ is continuous. First, since the maximum function is continuous, it follows that $H$ is continuous on $\left( X\times [0,1] \right)\times [0,1]$. Furthermore, since $H$ is constant in $t$ along $Y\times [0,1]$ and is equal to the identity on each $Y\times \set{t}$, it follows that $H$ is continuous on the disjoint union $W\times [0,1]$.

  Now, when $t = 0$, we have that $H\left( w,0 \right) = \left( p,s \right)$ whenever $\left( p,s \right)\in X\times [0,1]$, and $H\left( w,0 \right) = y$ whenever $w = y\in Y$. In particular, this means that $H\left( \cdot,0 \right)$ is the identity on $X\times [0,1]\sqcup Y$. Similarly, if $t = 1$, then we have $H\left( w,1 \right) = \left( p,1 \right)$ whenever $w = \left( p,s \right)\in X\times [0,1]$, and $H\left( w,1 \right) = y$ whenever $w = y\in Y$. Additionally, since $\left( p,1 \right)\in X\times [0,1]$ is fixed as $t$ ranges from $0$ to $1$, it follows that $H$ is a homotopy relative to $X\times \set{1}\sqcup Y$ with its image equal to the piecewise-defined map
  \begin{align*}
    v\colon X\times [0,1]\sqcup Y &\rightarrow X\times [0,1]\sqcup Y\\
    v\left( w \right) &= \begin{cases}
      \left( p,1 \right), & w = \left( p,s \right)\in X\times [0,1];\\
      y, & w = y\in Y.
    \end{cases}
  \end{align*}
  Thus, $H$ is a deformation retraction onto $X\times \set{1}$, so since $q$ maps $X\times \set{1}$ to $f(X)\subseteq Y$, it follows that upon composition, we have that $q\circ H$ is a deformation retraction onto $q\left( X\times \set{1}\sqcup Y \right)\subseteq Y$.
\end{solution}
\section{Current Problems}%
\begin{problem}[Problem 1]
  Consider the $n$-dimensional torus $T^n = S^1\times\cdots\times S^1$, an $n$-fold Cartesian product. Prove that if $T^n$ is homeomorphic to $T^m$, then $n = m$.
\end{problem}
\begin{solution}
  Suppose $T^n$ is homeomorphic to $T^m$. Then, $T^n$ and $T^m$ are homotopy equivalent, meaning that they admit an isomorphism of fundamental groups $\varphi\colon \pi_1\left( T^n \right)\rightarrow \pi_1\left( T^m \right)$. Since taking fundamental groups commutes with direct products, it follows that we have $\varphi\colon \Z^n\rightarrow \Z^m$ is an isomorphism of $\Z$-modules. By taking the tensor product with the fraction field $\Q$, we see that we have a linear isomorphism of $\Q$-vector spaces $\id\otimes \varphi\colon \Q^n\rightarrow \Q^m$, meaning that $n = m$ by the invariance of dimension.
\end{solution}
\begin{problem}[Problem 2]
  Give a presentation of $\Z\times \Z$ by generators and relations.
\end{problem}
\begin{solution}
  Consider the surjection from the free group $F_2$ onto $\Z\times \Z$ given by $a\mapsto \left( 1,0 \right)$ and $b\mapsto \left( 0,1 \right)$. Then, since we must have $ab = ba$ as $\Z\times \Z$ is abelian, it follows that a presentation for $\Z\times \Z$ is given by $ \left\langle a,b \mid aba^{-1}b^{-1} = e \right\rangle $.
\end{solution}
\end{document}
