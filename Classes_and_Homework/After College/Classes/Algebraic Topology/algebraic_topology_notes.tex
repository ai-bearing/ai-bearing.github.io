\documentclass[10pt]{mypackage}

\usepackage{mlmodern}
%\usepackage{newpxtext,eulerpx,eucal}
%\renewcommand*{\mathbb}[1]{\varmathbb{#1}}

%\usepackage{homework}
\usepackage{notes}

%\usepackage[ backend=bibtex, style = alphabetic, sorting=ynt ]{biblatex}
%\addbibresource{  }

\usepackage{parskip}

\fancyhf{}
\fancyhead[R]{Avinash Iyer}
\fancyhead[L]{Algebraic Topology: Class Notes}
\fancyfoot[C]{\thepage}

\setcounter{secnumdepth}{0}

\begin{document}
\RaggedRight
\section{Homotopies and the Fundamental Group}%
The fundamental goal of algebraic topology (and really topology in general) is to determine when two topological spaces $X$ and $Y$ are homeomorphic to each other. We will define a homeomorphism in a more uniform way for what we will discuss later; two topological spaces $X$ and $Y$ are homeomorphic if there are maps
\begin{align*}
  f\colon X \rightarrow Y\\
  g\colon Y\rightarrow X
\end{align*}
such that $f\circ g = \id_Y$ and $g\circ f = \id_X$.

There is one issue though: this is a very hard problem. Most of the time, we restrict our view in some fashion or the other. For instance, \href{https://ai.avinash-iyer.com/Classes_and_Homework/After\%20College/Classes/Differential\%20Topology/differential_topology_notes.pdf}{differential topology} deals with smooth manifolds and tries to understand equivalence through diffeomorphisms.

We unfortunately cannot do this for the general case, so we will instead try to relax the conditions on the functions $f$ and $g$ as follows.
\begin{definition}
  A homotopy is a continuous map
  \begin{align*}
    F\colon X\times [0,1]\rightarrow Y
  \end{align*}
  with $F\left( \cdot,t \right)\colon X\rightarrow Y$ a continuous map. We say two maps $f_0$ and $f_1$ are \textit{homotopic} if there is a homotopy between them, and we write $f_0\simeq f_1$.
\end{definition}
We will write $[0,1]$ and $I$ interchangeably.
\begin{example}
  Consider the maps $\id\colon \R^{n}\rightarrow \R^{n}$, taking $x\mapsto x$, and $c\colon \R^{n}\rightarrow \R^{n}$, taking $x\mapsto 0$.

  Then, $\id$ and $c$ are homotopic via the homotopy
  \begin{align*}
    F\left( x,t \right) &= \left( 1-t \right)x.
  \end{align*}
\end{example}
Homotopies that fix subspaces are often quite useful.
\begin{definition}
  Let $F\colon X\times [0,1]\rightarrow Y$ be a homotopy. We say that $f_0 = F(\cdot,0)$ and $f_1 = F(\cdot,1)$ are homotopic relative to $A\subseteq X$ if $F|_{A\times I}$ is constant. We write $f_0\simeq f_1$ rel $A$.
\end{definition}
\begin{example}
  If $f_0 = \id$ on $\R^{n}$ and $f_1$ is given by
  \begin{align*}
    f_1(x) &= \begin{cases}
      x & x\in D^n\\
      \frac{x}{\norm{x}} & x\in \R^{n}\setminus D^n
    \end{cases},
  \end{align*}
  where $D^n$ denotes the closed unit ball in $\R^{n}$, then we have that $f_0\simeq f_1$ rel $D^n$.
\end{example}
\begin{definition}
  We say two spaces $X$ and $Y$ are \textit{homotopy equivalent} if there are continuous maps
  \begin{align*}
    f\colon X\rightarrow Y\\
    g\colon Y\rightarrow X
  \end{align*}
  such that $g\circ f \simeq \id_X$ and $f\circ g \simeq \id_Y$. In this case, we say $X$ and $Y$ have the same homotopy type, and write $X\simeq Y$.

  The function $g$ is called a homotopy inverse to $f$.
\end{definition}
\begin{example}
  If we let $f\colon S^1\hookrightarrow \R^{2}\setminus \set{0}$, then $g\colon \R^{2}\setminus \set{0}\rightarrow S^{1}$ given by $x\mapsto \frac{x}{\norm{x}}$ is a homotopy inverse to $f$. Though $\R^{2}\setminus \set{0}$ and $S^{1}$ are of different dimensions (as manifolds), they are still homotopy equivalent, in particular meaning that homotopy equivalence is strictly weaker than homeomorphism.
\end{example}
\begin{definition}
  A \textit{deformation retraction} of a space $X$ onto $A\subseteq X$ is a homotopy rel $A$ from $\id_X$ to some map $r\colon X\rightarrow X$ with $r(X) = A$.

  The map $r$, where $r|_{A} = \id_A$ and $r(X) = A$, is called a retraction of $X$ onto $A$, and we call $A$ a retract of $X$.
\end{definition}
There are retractions that are not deformation retractions. For instance, if we let $r\colon S^1\rightarrow S^1$ be defined by $\left( x,y \right)\mapsto \left( 1,0 \right)$, then we have that $r\left(S^1\right) = A = \set{(1,0)}$, but there is no homotopy from $r$ to $\id_{S^1}$.
\begin{definition}
  A space with the homotopy type of a point is called contractible.

  In other words, there are maps $f\colon X\rightarrow \ast$ and $g\colon \ast\rightarrow X$ such that $f\circ g = \id_{\ast}$ and $g\circ f = \id_X$, where $\ast$ denotes the one point space.
\end{definition}
We have shown that $\R^{n}$ is contractible via straight line homotopy, as well as $D^n$, but neither are homeomorphic for $n\geq 0$. However, $S^n$ is not contractible.
\subsection{Cell Complexes}%
Now, we can refine our original goal; we now seek to classify spaces up to homotopy equivalence. However, this is still a very hard question, which is where we introduce algebra. The question now is whether we can create an algebraic object (group, ring, module, etc.) assigned to a space such that, if two spaces are homotopy equivalent, the associated algebraic objects are isomorphic.

We call these objects \textit{algebraic invariants}, and there are many of these objects. One of these we learned in differential topology was the de Rham cohomology for a differentiable manifold. However, we must note that these invariants lose information.\footnote{For instance, the Poincaré Lemma from differential topology shows that $S^n$ and $\R^d\times S^n$ have the same cohomology groups, and so are homotopy equivalent, but they are most certainly not homeomorphic or diffeomorphic to each other.} They are not \textit{complete} invariants, but they are invariants nonetheless.

The two invariants we will discuss in this class are:
\begin{itemize}
  \item the fundamental group, $\pi_1(X)$;
  \item and the homology groups, $H_n(X)$.
\end{itemize}
We must first introduce the primary setting we will examine these groups: cell complexes.

Consider the one-holed and two-holed tori. They can be created from quotients of a square and an octagon as follows respectively.
\begin{center}
  \begin{tikzpicture}
  \draw[very thick] (0,0) -- (2,0) -- (2,2) -- (0,2) -- cycle;
  
  % Arrow decorations
  \draw[-{Stealth[length=3mm]},very thick] (0,0) -- (1,0);
  \draw[-{Stealth[length=3mm]},very thick] (1,0) -- (2,0);
  
  \draw[-{Stealth[length=3mm]},very thick] (2,0) -- (2,1);
  \draw[-{Stealth[length=3mm]},very thick] (2,1) -- (2,2);
  
  \draw[-{Stealth[length=3mm]},very thick] (2,2) -- (1,2);
  \draw[-{Stealth[length=3mm]},very thick] (1,2) -- (0,2);
  
  \draw[-{Stealth[length=3mm]},very thick] (0,2) -- (0,1);
  \draw[-{Stealth[length=3mm]},very thick] (0,1) -- (0,0);
  
  % Labels
  \node at (1,-0.3) {$a$};
  \node at (2.3,1) {$b$};
  \node at (1,2.3) {$a$};
  \node at (-0.3,1) {$b$};
  
  % Vertex
  \fill (0,0) circle (2pt);
  
  \node at (1,-0.8) {One-holed Torus};
\end{tikzpicture}
\end{center}
\begin{center}
  
\begin{tikzpicture}
  % Define the octagon vertices
  \def\n{8}
  \def\radius{2}
  
  \foreach \i in {1,...,\n} {
    \coordinate (p\i) at ({90+360/\n * (\i-1)}:\radius);
  }
  
  % Draw edges with arrows
  \draw[-{Stealth[length=3mm]},very thick] (p1) -- (p2);
  \draw[-{Stealth[length=3mm]},very thick] (p2) -- (p3);
  \draw[-{Stealth[length=3mm]},very thick] (p3) -- (p4);
  \draw[-{Stealth[length=3mm]},very thick] (p4) -- (p5);
  \draw[-{Stealth[length=3mm]},very thick] (p5) -- (p6);
  \draw[-{Stealth[length=3mm]},very thick] (p6) -- (p7);
  \draw[-{Stealth[length=3mm]},very thick] (p7) -- (p8);
  \draw[-{Stealth[length=3mm]},very thick] (p8) -- (p1);
  
  % Labels
  \node at ({90+360/8*0.5}:\radius+0.4) {$a$};
  \node at ({90+360/8*1.5}:\radius+0.4) {$b$};
  \node at ({90+360/8*2.5}:\radius+0.4) {$c$};
  \node at ({90+360/8*3.5}:\radius+0.4) {$d$};
  \node at ({90+360/8*4.5}:\radius+0.4) {$a$};
  \node at ({90+360/8*5.5}:\radius+0.4) {$b$};
  \node at ({90+360/8*6.5}:\radius+0.4) {$c$};
  \node at ({90+360/8*7.5}:\radius+0.4) {$d$};
  
  % Vertex
  \fill (p1) circle (2pt);
  
  \node at (0,-2.8) {Two-holed Torus};
\end{tikzpicture}
\end{center}
We start with a discrete set of points, which are known as $0$-cells and form the $0$-skeleton, denoted $X^0$. Inductively, we form the $n$-skeleton, written $X^n$, from the space $X^{n-1}$ by attaching $n$-cells ($n$-dimensional disks), denoted $D^n_{\alpha}$, via attaching maps $\varphi_{\alpha}\colon S^{n-1}_{\alpha}\rightarrow X^{n-1}$, and forming via
\begin{align*}
  X^{n} &= X^{n-1}\sqcup \left( \bigsqcup_{\alpha}D_{\alpha}^{n} / x\sim \varphi_{\alpha}(x) \right).
\end{align*}
This process may terminate at some $n$, or it continues to infinity, in which case we write $X = \bigcup X^n$. If $X = X^n$ for some $n$, the smallest such $n$ is called the \textit{dimension} of $X$. Sometimes we may denote the $n$-cells by $e^n$.

We also refer to these types of spaces as CW complexes. Note that all CW complexes are Hausdorff, so in particular, they cannot be the backbone of every topological space.
\begin{example}
  The space $S^n$ can be formed as an $n$-dimensional CW complex in two different ways. Either we may take one $0$-cell and attach an $n$-cell via the constant map, or take two $0$-cells, and inductively attach two $n$-cells to $S^{n-1}$.
\end{example}
\begin{example}
  A more intricate example is that of $ \mathbb{RP}^n$, real projective space in $n$-dimensions, which is given by
  \begin{align*}
    \mathbb{RP}^n &= S^n/x\sim -x.
  \end{align*}
  The cell decomposition of $ \mathbb{RP}^n $ is given by a quotient of the second method we discussed for constructing $S^n$, by identifying ``opposite cells'' of dimension $n$. It then follows that
  \begin{align*}
    \mathbb{RP}^n &= e^0\cup e^1 \cup \cdots \cup e^n
  \end{align*}
  with satisfactory attaching maps.
\end{example}

\end{document}
