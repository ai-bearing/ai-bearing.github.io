\documentclass[10pt]{mypackage}

\usepackage{mlmodern}
%\usepackage{newpxtext,eulerpx,eucal}
%\renewcommand*{\mathbb}[1]{\varmathbb{#1}}

\usepackage{homework}
\usepackage{notes}

%\usepackage[ backend=bibtex, style = alphabetic, sorting=ynt ]{biblatex}
%\addbibresource{  }

\usepackage{parskip}

\fancyhf{}
\fancyhead[R]{Avinash Iyer}
\fancyhead[L]{Random Matrices Exercises}
\fancyfoot[C]{\thepage}

\setcounter{secnumdepth}{0}

\begin{document}
\RaggedRight
\begin{problem}[Exercise 1.17]
  In this exercise, you will show that the moments of a standard Gaussian variable count pair partitions.
  \begin{enumerate}[(i)]
    \item Let $X$ be a standard Gaussian variable. Prove that
      \begin{align*}
        \mathbb{E}\left( X^{2k} \right) &= \left( 2k-1 \right)!!\\
        \mathbb{E}\left( X^{2k-1} \right) &= 0.
      \end{align*}
    \item Prove that $\left\vert P_2(2k) \right\vert = \left( 2k-1 \right)!!$ by putting $P_2\left( 2k \right)$ in explicit bijection with a set of cardinality $\left( 2k-1 \right)\left\vert P_2\left( 2k-2 \right) \right\vert$.
  \end{enumerate}
\end{problem}
\begin{remark}
  I had done the first part of this exercise earlier in a separate notes document, but had not written it up here for submission.
\end{remark}
\begin{solution}\hfill
  \begin{enumerate}[(i)]
    \item We see that
        \begin{align*}
          E\left[ Z^{m} \right] &= \int_{-\infty}^{\infty} x^{m}e^{-x^2/2}\:dx\\
                                &= -x^{m}e^{-x^2/2}\bigr\vert_{-\infty}^{\infty} + \left( m-1 \right) \int_{-\infty}^{\infty} x^{m-2}e^{-x^2/2}\:dx\\
                                &= \left( m-1 \right)E\left[ Z^{m-2} \right].
        \end{align*}
        Therefore, we recover the recursion relation for $\left( 2k-1 \right)!!$ whenever $m = 2k$ and $0$ otherwise.
      \item Considering the set $\left[ 2k \right] = \set{1,2,\dots,2k}$, we see that there are $2k-1$ ways to pair $1$ with any other element, and there are then $P_2\left( 2k-2 \right)$ pair partitions of the remaining $2k-2$ elements. This gives our desired bijection.
  \end{enumerate}
\end{solution}
\begin{problem}[Exercise 2.23]\hfill
  \begin{enumerate}[(i)]
    \item Find a recursion for $\left\vert NC_2\left( 2k \right) \right\vert$.
    \item Show that $\operatorname{Cat}\left( k \right) = \frac{1}{k+1} {2k\choose k}$ satisfies the same recursion relation shown in (i).
    \item Let $d\mu = f(x)\:dx$, where
      \begin{align*}
        f(x) &= \begin{cases}
          \frac{1}{2\pi}\sqrt{4-x^2} & -2\leq x \leq 2\\
          0 & \text{else}
        \end{cases}.
      \end{align*}
      Show that
      \begin{align*}
        \int_{\R}^{} x^{2k}\:d\mu &= \operatorname{Cat}(k)\\
        \int_{\R}^{} x^{2k-1}\:d\mu &= 0.
      \end{align*}
  \end{enumerate}
\end{problem}
\begin{solution}\hfill
  \begin{enumerate}[(i)]
    \item To find a recursion for $NC_2(2k)$, we start by counting the number of valid pairings of $1$ with any other element of $[2k]$. For this, we observe that there must be an even number of elements between $1$ and whatever it is paired with, meaning there are $k$ valid pairings of $1$.

      For each of these pairings, there are two separate ``sub-blocks'' that we use to count the non-crossing partitions, the ones ``between'' $\set{1,2\ell}$as $\ell$ ranges from $1$ to $k$, and the ones ``outside'' the pairing $\set{1,2\ell}$. Combined, this gives the recurrence relation
      \begin{align*}
        \left\vert NC_2(2k) \right\vert &= \sum_{i=1}^{k} \left\vert NC_2\left( 2i-2 \right) \right\vert\left\vert NC_2\left( 2k-2i \right) \right\vert.
      \end{align*}
    \item Using the binomial theorem, we observe that
      \begin{align*}
        \frac{1-\sqrt{1-4z}}{2z} &= \frac{1}{z}
      \end{align*}
  \end{enumerate}
\end{solution}
\end{document}
