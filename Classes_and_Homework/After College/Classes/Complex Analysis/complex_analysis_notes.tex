\documentclass[10pt]{mypackage}

% sans serif font:
%\usepackage{cmbright}
%\usepackage{sfmath}
%\usepackage{bbold} %better blackboard bold

%\usepackage{homework}
\usepackage{notes}
\usepackage{newpxtext,eulerpx,eucal}
\renewcommand*{\mathbb}[1]{\varmathbb{#1}}

\fancyhf{}
\rhead{Avinash Iyer}
\lhead{Complex Analysis Notes}

\setcounter{secnumdepth}{0}

\begin{document}
\RaggedRight
\section{The Extended Complex Plane and Linear Fractional Transformations}%
\begin{definition}
  Let $\C$ be the complex plane. We define $\hat{\C} \coloneq \C\cup\set{\infty}$ to be the one-point compactification of $\C$.\newline

  Open sets in $\hat{\C}$ look like either:
  \begin{itemize}
    \item open sets $U\subseteq \C$;
    \item sets $U\subseteq \hat{\C}$ where $\infty\in U$, and $U\setminus\set{\infty}\subseteq \C$ is open, where additionally, there exists $R$ such that the set $\set{z\in \C | |z| > R}\subseteq U\setminus \set{\infty}$.
  \end{itemize}
\end{definition}
\begin{proposition}
  There is a continuous bijection between $\hat{\C}$ and the unit sphere $S^{2}$ in $\R^3$ given by
  \begin{align*}
    z \mapsto \left( \frac{2\re\left( z \right)}{\left\vert z \right\vert^2+1},\frac{2\im\left( z \right)}{\left\vert z \right\vert^2 + 1}, \frac{\left\vert z \right\vert^2-1}{\left\vert z \right\vert^2+1} \right)
  \end{align*}
  if $z\neq \infty$, and $z\mapsto \left( 0,0,1 \right)$ if $z = \infty$.
\end{proposition}
\begin{proposition}
  Given the metric $\rho$ on $S^{2}$ given by
  \begin{align*}
    \rho\left( \left( x_1,x_2,x_3 \right),\left( y_1,y_2,y_3 \right) \right) &= \sqrt{2\left( 1-x_1y_1-x_2y_2-x_3y_3 \right)},
  \end{align*}
  we have a corresponding natural metric on $\hat{\C}$ given by
  \begin{align*}
    d\left( z,w \right) &= \begin{cases}
      \frac{2\left\vert z-w \right\vert}{\sqrt{\left( 1+\left\vert z \right\vert^2 \right)\left( 1 + \left\vert w \right\vert^2 \right)}} & z,w\in \C\\
      \frac{2}{\sqrt{1 + \left\vert z \right\vert^2}}& w = \infty,z\in \C\\
      \frac{2}{\sqrt{1 + \left\vert w \right\vert^2}} & z=\infty,w\in\C\\
      0 & w=z=\infty.
    \end{cases}
  \end{align*}
  Furthermore, the open sets as defined above are exactly the ones induced by this metric.
\end{proposition}
\begin{definition}
  A subset $C\subseteq \hat{\C}$ is called a \textit{Riemann Circle} if
  \begin{itemize}
    \item $C$ is a union of a straight line with $\infty$,
      \begin{align*}
        C = \set{z\in \C | a\re\left( z \right) + b\im\left( z \right) = c} \cup \set{\infty};
      \end{align*}
    \item $C$ is a Euclidean circle,
      \begin{align*}
        C &= \set{z\in \C | \left\vert z-z_0 \right\vert = r}.
      \end{align*}
  \end{itemize}
  A subset $D\subseteq \hat{\C}$ is called a \textit{Riemann disc} if either
  \begin{itemize}
    \item $D$ is an open half-plane in $\C$;
    \item $D$ is an open disc in $\C$;
    \item $D$ is the complement of a closed disc in $\C$.
  \end{itemize}
\end{definition}
\begin{note}
  The Riemann circles correspond to circles drawn on $S^{2}$, while Riemann discs correspond to circular caps on $S^{2}$.
\end{note}
\end{document}
