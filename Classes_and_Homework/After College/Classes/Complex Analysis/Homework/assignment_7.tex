\documentclass[11pt]{mypackage}

% sans serif font:
%\usepackage{cmbright}
%\usepackage{sfmath}
%\usepackage{bbold} %better blackboard bold

\usepackage{homework}
%\usepackage{notes}
\usepackage{newpxtext,eulerpx,eucal}
\renewcommand*{\mathbb}[1]{\varmathbb{#1}}

\fancyhf{}
\fancyhead[R]{Avinash Iyer}
\fancyhead[L]{Complex Analysis: Assignment 7}
\fancyfoot[C]{\thepage}

\setcounter{secnumdepth}{0}

\begin{document}
\RaggedRight

\begin{problem}[Problem 1]
  Let $0\leq r < R \leq \infty$. Suppose $\left( a_n \right)_n,\left( b_n \right)_n\subseteq \C$ are such that the series $\sum_{n=-\infty}^{\infty}a_n\left( z-z_0 \right)^{n}$ and $\sum_{n=-\infty}^{\infty}b_n\left( z-z_0 \right)^{n}$ converge in $A\left( z_0,r,R \right)$, and are such that
  \begin{align*}
    \sum_{n=-\infty}^{\infty} a_n\left( z-z_0 \right)^{n} &= \sum_{n=-\infty}^{\infty}b_n\left( z-z_0 \right)^{n}
  \end{align*}
  for all $z\in A\left( z_0,r,R \right)$. Show that $a_n = b_n$ for all $n$.
\end{problem}
\begin{solution}
  Suppose we have the functions
  \begin{align*}
    f(z) &= \sum_{n=-\infty}^{\infty} a_n\left( z-z_0 \right)^{n}\\
         &= f_1(z) + f_2(z)\\
    g(z) &= \sum_{n=-\infty}^{\infty}b_n\left( z-z_0 \right)^{n}\\
         &= g_1(z) + g_2(z)
  \end{align*}
  are written so that $f_1,g_1$ are holomorphic defined on $U\left( z_0,R \right)$ while $f_2,g_2$ are holomorphic defined on $\C\setminus B\left( z_0,r \right)$. The assumption that $f(z) = g(z)$ on $A\left( z_0,r,R \right)$ gives $f_1(z) - g_1(z) = g_2(z) - f_2(z)$, or
  \begin{align*}
    \sum_{n=0}^{\infty} \left( a_n-b_n \right)\left( z-z_0 \right)^{n} &= \sum_{n=-\infty}^{-1} \left( b_n-a_n \right)\left( z-z_0 \right)^{n}
  \end{align*}
  on $A\left( z_0,r,R \right)$. This means that we may define a function $h(z)$ by letting $r < \rho < R$ and taking
  \begin{align*}
    h(z) &= \begin{cases}
      \sum_{n=0}^{\infty}\left( a_n-b_n \right)\left( z-z_0 \right)^{n} & \left\vert z-z_0 \right\vert\leq \rho\\
      \sum_{n=-\infty}^{-1}\left( b_n-a_n \right)\left( z-z_0 \right)^{n} & \left\vert z-z_0 \right\vert > \rho
    \end{cases},
  \end{align*}
  which we observe is holomorphic on the entirety of $\C$ as a result of the fact that the separate power series expansions $\sum_{n=0}^{\infty}\left( a_n-b_n \right)\left( z-z_0 \right)^{n}$ and $\sum_{n=-\infty}^{-1}\left( b_n-a_n \right)\left( z-z_0 \right)^{n}$ are holomorphic on their respective domains of definition, while they are equal on $A\left( z_0,r,R \right)$.\newline

  Furthermore, we see that $\lim_{z\rightarrow\infty} \left\vert h(z) \right\vert = 0$, whence $h$ is a bounded entire function, so $h\equiv K$ for some constant $K$. This means that, for $\left\vert z-z_0 \right\vert < \rho$,
  \begin{align*}
    \sum_{n=0}^{\infty}\left( a_n-b_n \right)\left( z-z_0 \right)^{n} &= K,
  \end{align*}
  meaning that $a_0-b_0 = K$ and $a_{n\geq 1} - b_{n\geq 1} = 0$. Yet, for $\left\vert z-z_0 \right\vert > \rho$, we must have
  \begin{align*}
    \sum_{n=1}^{\infty}\left( a_{-n}-b_{-n} \right) \left( z-z_0 \right)^{-n} &= K,
  \end{align*}
  but there are no constant terms in this series expansion (while $z$ is arbitrary), meaning that $a_{n \leq -1} - b_{n \leq -1} = 0$, and that $K = 0$. Thus, we have $a_0 - b_0 = 0$, and we are done.
\end{solution}
\begin{problem}[Problem 2]\hfill
  \begin{enumerate}[(a)]
    \item Determine the Laurent series expansion of the function
      \begin{align*}
        f(z) &= \frac{z}{\left( z-3 \right)^2\left( z-4 \right)}
      \end{align*}
      that converges on $A\left( 0,3,4 \right)$. 
    \item Show that there does not exist a holomorphic function $f\colon \C\setminus \set{0}\rightarrow \C$ satisfying $ \left\vert f(z) \right\vert\geq \left\vert z \right\vert^{-2/3} $.
  \end{enumerate}
\end{problem}
\begin{solution}\hfill
  \begin{enumerate}[(a)]
    \item We start by taking a partial fraction decomposition of $f$ to yield
      \begin{align*}
        f(z) &= \frac{4}{z-4} - \frac{4}{z-3} -\frac{3}{\left( z-3 \right)^2}\\
             &= \frac{4}{z-4} - \frac{4}{z-3} + 3 \diff{}{z}\left( \frac{1}{z-3} \right) 
      \end{align*}
      We seek to expand about $z = 0$ within the ball $U\left( 0,4 \right)$ and outside the closed ball $B\left( 0,3 \right)$. This means that the first term in our partial fraction expansion becomes
      \begin{align*}
        a_1(z) &= -\frac{1}{1-\frac{z}{4}}\\
               &= -\sum_{n=0}^{\infty} \frac{z^{n}}{4^{n}},
      \end{align*}
      which converges on $U\left( 0,4 \right)$. The expansion in the second and third terms will require a little bit more work. Dividing out by $z$, we find that the second term becomes
      \begin{align*}
        a_2(z) &= -\frac{4}{z\left( 1-\frac{3}{z} \right)}\\
               &= -\frac{4}{z} \sum_{n=0}^{\infty} \frac{3^{n}}{z^{n}}\\
               &= -\sum_{n=1}^{\infty}\frac{4\cdot 3^{n-1}}{z^{n}},
      \end{align*}
      which converges outside the closed ball $B\left( 0,3 \right)$. Finally, for the third term, we observe that, using term-by-term differentiation (allowable as the series is uniformly convergent outside $B\left( 0,3 \right)$), we have
      \begin{align*}
        3\diff{}{z}\left( \frac{1}{z-3} \right) &= 3\diff{}{z}\left( \sum_{n=1}^{\infty} 3^{n-1}z^{-n} \right)\\
                                                &= \sum_{n=1}^{\infty} -\frac{n3^{n}}{z^{n+1}}.
      \end{align*}
      This yields a Laurent series expansion of
      \begin{align*}
        f(z) &= -\sum_{n=0}^{\infty} \frac{z^{n}}{4^{n}} + \sum_{n=1}^{\infty} \frac{\left( \frac{4}{3} z - n \right)3^{n}}{z^{n+1}}
      \end{align*}
    \item Suppose toward contradiction that there were such an $f(z)$. Since $\left\vert z \right\vert^{-2/3}$ is strictly greater than zero along its domain, it would follow that $\left\vert f(z) \right\vert$ would not have any zero along its domain. This means that $ g(z) = \frac{1}{f(z)}\colon\C\setminus \set{0}\rightarrow\C $ would be defined on its entire domain. Furthermore, we would have
      \begin{align*}
        \left\vert g(z) \right\vert &\leq \left\vert z \right\vert^{2/3},
      \end{align*}
      and on $U\left( 0,\ve \right)$, we know that $\left\vert z \right\vert^{2/3}$ is bounded above by $\ve^{2/3}$ as $\left\vert z \right\vert^{2/3}\colon \C\rightarrow \R_{\geq 0}$ is an increasing function. Thus, since $g$ would be locally bounded around $0$, it would follow that $g$ has a removable singularity at $0$. This means that there is a holomorphic extension $h\colon \C\rightarrow \C$ that agrees with $g$ on $\C\setminus \set{0}$. In particular, we would have $\left\vert h(z) \right\vert \leq \left\vert z \right\vert^{2/3}$ for all $z\in \C\setminus\set{0}$.\newline

      Now, let $R > 0$. Using the Cauchy estimate on $S\left( 0,R \right)$, we have, for any fixed $n > 0$,
      \begin{align*}
        \left\vert h^{(n)}(z) \right\vert &\leq \frac{n!}{R^{n}} \sup_{\left\vert z \right\vert = R} \left\vert h(z) \right\vert\\
                                          &\leq \frac{n!}{R^{n}} \sup_{\left\vert z \right\vert = R} \left\vert z \right\vert^{2/3}\\
                                          &= \frac{n!}{R^{n-2/3}}.
      \end{align*}
      Yet, since $R$ is arbitrary, it follows that $\left\vert h^{(n)}(z) \right\vert = 0$ for all $n > 0$, whence $h$ is constant. Yet, since $\left\vert h(z) \right\vert \leq \left\vert z \right\vert^{2/3}$ for all $z \in \C\setminus \set{0}$, it follows that $\left\vert h(z) \right\vert \leq \ve^{2/3}$ for any $\ve > 0$, whence $\left\vert h(z) \right\vert = 0$ for all $z\in \C$. At the same time, we explicitly defined $g(z)$ in a manner such that it could never equal zero, meaning that such an $f$ cannot exist.
  \end{enumerate}
\end{solution}
\begin{problem}[Problem 3]
  Let $0 < r < R$. Show that there does not exist a holomorphic bijection $f\colon \mathbb{D}\setminus \set{0}\rightarrow A\left( 0,r,R \right)$.
\end{problem}
\begin{solution}
  Suppose there were a holomorphic bijection $f\colon \D\setminus \set{0}\rightarrow A\left( 0,r,R \right)$. Since $\left\vert f(z) \right\vert \leq R$ for all $z\in \D\setminus \set{0}$, it follows that the singularity at $0$ is removable, so there is a holomorphic function $g\colon \D\rightarrow A\left( 0,r,R \right)$.\newline

  Considering $g(0)$, we observe that $g(0) = \lim_{z\rightarrow 0}f(z)$, meaning that $g(0)\in \overline{A}\left( 0,r,R \right)$ as $g(0)$ is a limit point of the image $f\left( \D\setminus \set{0} \right)$, where $f$ is continuous. However, it cannot be the case that $g(0)\in \partial A\left( 0,r,R \right)$, as $g$ is holomorphic so this would contradict the open mapping principle. Thus, we must have $g(0)\in A\left( 0,r,R \right)$, meaning that there is some $z_0\in \D\setminus \set{0}$ such that $f\left(z_0\right) = g(0)$.\newline

  Let $\left( z_n \right)_n\subseteq \D\setminus \set{0}$ be a sequence with $z_n\rightarrow 0$. Observe then that $\lim_{n\rightarrow\infty}f\left( z_n \right) = g\left(0\right)$ as $g$ is the unique holomorphic extension of $f$. However, since $f$ is a holomorphic bijection, the open mapping principle means that $f$ has a continuous inverse, meaning that $f^{-1}\left( f\left( z_n \right) \right) = z_n$ is continuous, with $\lim_{n\rightarrow\infty}f^{-1}\left( f\left( z_n \right) \right) = f^{-1}\left( g(0) \right) = z_0$, but $\left( z_n \right)_n\rightarrow 0$, meaning that by uniqueness of limits, $z_0 = 0$. Therefore, it cannot be the case that such a holomorphic $f$ exists.
\end{solution}
\begin{solution}[Special Case]
  \footnotesize
  Suppose there were a holomorphic bijection $f\colon \D\setminus \set{0}\rightarrow A\left( 0,r,R \right)$ with holomorphic inverse. Notice that for all $z\in \mathbb{D}\setminus \set{0}$, we would then have $\left\vert f(z) \right\vert < R$, meaning that $f$ is necessarily locally bounded close to $0$. Thus, the singularity at $0$ is removable, so there is a unique holomorphic function $g\colon \mathbb{D}\rightarrow \C$ with $g|_{ \mathbb{D}\setminus \set{0} } = f$.\newline

  We notice that $ g $ is an injection, as $g|_{ \mathbb{D}\setminus \set{0} }$ is a bijection and $g(0)$ is uniquely defined. As a result, it follows that the restriction $g\colon \mathbb{D}\rightarrow \img\left( g \right)$ is a holomorphic bijection. Furthermore, we also notice that
  \begin{align*}
    \lim_{z\rightarrow 0} \left\vert g(z) \right\vert &= \lim_{z\rightarrow 0} \left\vert f(z) \right\vert\\
                                                      &\geq r\\
                                                      &> 0,
  \end{align*}
  meaning that $g$ is nonvanishing on $ \mathbb{D} $. In particular, there is a logarithm $h(z)\colon \mathbb{D}\rightarrow \C$ such that 
  \begin{align*}
    g(z) &= e^{h(z)},
  \end{align*}
  and $f(z) = e^{h(z)}$ when restricted to $ \mathbb{D}\setminus \set{0} $. Now, since the identity map $\operatorname{id}\colon A\left( 0,r,R \right)\rightarrow A\left( 0,r,R \right)$ is a bijective holomorphic map with holomorphic inverse, it follows that
  \begin{align*}
    e^{h(z)} &= \operatorname{id}\left( f(z) \right).
  \end{align*}
  Yet, this means that
  \begin{align*}
    \operatorname{id}(z) &= e^{h\left( f^{-1}(z) \right)},
  \end{align*}
  meaning that $ \operatorname{id} $ admits a logarithm. Yet, $A\left( 0,r,R \right)$ is not simply connected, while $ \operatorname{id} $ is nonvanishing, which is a contradiction. Thus, no such $f$ exists.
\end{solution}
\begin{problem}[Problem 4]
  Show that if $f$ is entire and satisfies $\lim_{z\rightarrow\infty} f\left( z \right) = \infty$, then $f$ is a polynomial.
\end{problem}
\begin{solution}
  Consider the function $g\colon \C\setminus \set{0} \rightarrow \C$ given by $g(z) = f\left( \frac{1}{z} \right)$. Since $f$ is entire and $\frac{1}{z}$ is holomorphic on $\C\setminus \set{0}$, it follows that, given the power series expansion
  \begin{align*}
    f(z) &= \sum_{n=0}^{\infty}a_nz^{n},
  \end{align*}
  we have the Laurent series expansion
  \begin{align*}
    g(z) &= \sum_{n=0}^{\infty}a_nz^{-n},
  \end{align*}
  where $g\colon \C\setminus \set{0}\rightarrow \C$ has a singularity at $0$.\newline

  Observe that the limit $\lim_{z\rightarrow\infty} f(z)$ is equivalent to $\lim_{z\rightarrow 0}f\left( \frac{1}{z} \right)$, whence $\lim_{z\rightarrow 0} g(z) = \infty$. Therefore, $g$ has a pole of order $k$ at $0$, so by the classification of singularities, we have
  \begin{align*}
    g(z) &= \sum_{n=0}^{k}a_nz^{-n}.
  \end{align*}
  Since $g\left( \frac{1}{z} \right) = f(z)$, it then follows that
  \begin{align*}
    f(z) &= \sum_{n=0}^{k} a_nz^{n}.
  \end{align*}
\end{solution}
\begin{problem}[Problem 5]
  Let $r > 0$, $f,g\colon \dot{U}\left( 0,r \right)\rightarrow \C$ be holomorphic functions such that $g(z)\neq 0$ for all $z\in \dot{U}\left( 0,r \right)$. Show that the singularity at $0$ is essential for $f$ if and only if the singularity for $h \coloneq \frac{f}{g}$ at $0$ is essential.
\end{problem}
\begin{solution}
  Since $g\neq 0$ on $\dot{U}\left( 0,r \right)$ and $g$ does not have an essential singularity at $0$, it follows that that the singularity for $g(z)$ at $0$ is either a pole or removable. This allows us to write $g(z) = z^{-m} \widetilde{g}(z)$, where $m \geq 0$ is a positive integer and $\widetilde{g}(z)$ is holomorphic (and necessarily nonzero) on $U\left( 0,r \right)$. Note that if $m = 0$, then the singularity at $0$ is removable, and if $m > 0$, then the singularity at $0$ is a pole of order $m$.\newline

  Now, we may write
  \begin{align*}
    h(z) &= z^{m}\frac{f(z)}{\widetilde{g}(z)},
  \end{align*}
  where $ \widetilde{g}(z) $ is never zero, hence $h(z)\colon \dot{U}\left( 0,r \right)\rightarrow \C$ is holomorphic. In particular, since $f$ is also holomorphic, it follows that $f$ has a Laurent series expansion
  \begin{align*}
    f(z) &= \sum_{n=-\infty}^{\infty} a_n z^{n},
  \end{align*}
  so we may write
  \begin{align*}
    h(z) &= \frac{1}{\widetilde{g}(z)} \sum_{n=-\infty}^{\infty} a_nz^{m+n}\\
         &= \frac{1}{\widetilde{g}(z)} \sum_{n=-\infty}^{\infty} a_{n-m}z^{n}
  \end{align*}
  Observe then that the singularity at $0$ for $f$ is essential if and only if the set of all $n < 0$ for which $a_{n}\neq 0$ is unbounded below. Since $m$ is constant, it follows that the set of $n$ for which $a_{n-m}\neq 0$ is unbounded below, meaning that the singularity at $0$ for $h$ is essential, and vice versa.
\end{solution}
\end{document}
