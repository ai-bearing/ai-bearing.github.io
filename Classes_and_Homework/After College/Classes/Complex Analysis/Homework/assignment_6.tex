\documentclass[10pt]{mypackage}

% sans serif font:
%\usepackage{cmbright}
%\usepackage{sfmath}
%\usepackage{bbold} %better blackboard bold

\usepackage{homework}
%\usepackage{notes}
\usepackage{newpxtext,eulerpx,eucal}
\renewcommand*{\mathbb}[1]{\varmathbb{#1}}

\fancyhf{}
\rhead{Avinash Iyer}
\lhead{Complex Analysis: Assignment 6}

\setcounter{secnumdepth}{0}

\begin{document}
\RaggedRight
\begin{problem}[Problem 1]
  Let $U\subseteq \C$ be a nonempty open set.\newline

  Given a sequence $\left( z_n \right)_n\subseteq U$, we write $z_n\rightarrow \partial U$ if, for every compact subset $K\subseteq U$, there exists some $N = N(K)\in \N$ such that $z_n\notin K$ whenever $n\geq N$.\newline

  Given a function $u\colon U\rightarrow \R$, define
  \begin{align*}
    \limsup_{z\rightarrow \partial U} u(z) &= \inf_{\substack{K\subseteq U\\K\text{ compact}}} \sup_{z\in U\setminus K} u(z).
  \end{align*}
  \begin{enumerate}[(a)]
    \item For each positive integer $n\in\N$, define
      \begin{align*}
        K_n &\coloneq \set{z\in U | \left\vert z \right\vert \leq n, \dist_{\C\setminus U}\left( z \right) \geq \frac{1}{n}}.
      \end{align*}
      Show that:
      \begin{enumerate}[(i)]
        \item each $K_n$ is compact;
        \item $K_n\subseteq K_{n+1}^{\circ}$;
        \item $U = \bigcup_{n=1}^{\infty}K_n$.
      \end{enumerate}
    \item Let $L \coloneq \limsup_{z\rightarrow \partial U} u(z)$.
      \begin{enumerate}[(i)]
        \item Show that for each $S > L$, there is a compact subset $K\subseteq U$ such that $u(z)\leq S$ for all $z\in U\setminus K$.
        \item Show that there exists a sequence $\left( z_n \right)_n$ in $U$ with $z_n\rightarrow \partial U$ and $\limsup_{n\rightarrow\infty}u\left( z_n \right)\leq L$.
      \end{enumerate}
    \item Prove that
      \begin{align*}
        \limsup_{z\rightarrow \partial U} u(z) &= \sup_{\substack{\left( z_n \right)_n\subseteq U\\z_n\rightarrow \partial U}}\limsup_{n\rightarrow\infty} u\left( z_n \right),
      \end{align*}
      where the supremum is over all sequences $\left( z_n \right)_n$ with $\left( z_n \right)_n\rightarrow \partial U$.
  \end{enumerate}
\end{problem}
\begin{solution}\hfill
  \begin{enumerate}[(a)]
    \item We claim that the set
      \begin{align*}
        C_n &= \set{z\in U | \operatorname{dist}_{\C\setminus U}\left( z \right) \geq \frac{1}{n}}
      \end{align*}
      is closed. Then, we observe that $K_n = B\left( 0,n \right) \cap C_n$ would then be an intersection of a closed set with a closed and bounded set, hence a closed and bounded set, hence compact. Towards this end, suppose $\left( w_k \right)_k\subseteq C_n$ converges to $w\in \C$. Then, for each $k$, we have
      \begin{align*}
        \inf_{z\in \C\setminus U}\left\vert w_k - z \right\vert &\geq \frac{1}{n}.
      \end{align*}
      Observe then that for any $z\in \C\setminus U$, we have
      \begin{align*}
        \left\vert w_k-z \right\vert &\geq \frac{1}{n}
      \end{align*}
      for each $k$, meaning that
      \begin{align*}
        \lim_{k\rightarrow\infty} \left\vert w_k - z \right\vert &\geq \frac{1}{n},
      \end{align*}
      or that
      \begin{align*}
        \left\vert w-z \right\vert\geq \frac{1}{n}.
      \end{align*}
      In particular, it must be the case that $w\in U$, and that
      \begin{align*}
        \inf_{z\in \C\setminus U} \left\vert w-z \right\vert &\geq \frac{1}{n},
      \end{align*}
      so that $w\in C_n$, and thus $C_n$ is closed, and $K_n$ is compact.\newline

      To see that $K_n\subseteq K_{n+1}^{\circ}$, we show that $C_n\subseteq C_{n+1}^{\circ}$ by understanding the picture of $C_{n}^{\circ}$. Towards this end, we see that $z\in C_n^{\circ}$ if and only if $z\in U$ and there is some $r > 0$ such that $\operatorname{dist}_{\C\setminus U}\left( w \right) \geq \frac{1}{n}$ for all $w\in U\left( z,r \right)$.\newline

      Observe that if $\ve > 0$, then if $z$ satisfies $\dist_{\C\setminus U}\left( z \right) \geq \frac{1}{n} + \ve$, then if $w\in \C\setminus U$ and $\zeta\in U\left( z, \ve/2 \right)$, we have
      \begin{align*}
        \frac{1}{n} + \ve &\leq \left\vert z-w \right\vert\\
                          &\leq \left\vert z-\zeta \right\vert + \left\vert \zeta - w \right\vert\\
                          &< \ve/2 + \left\vert \zeta-w \right\vert,
      \end{align*}
      meaning that $\left\vert \zeta - w \right\vert \geq \frac{1}{n} + \ve/2$ for all $w\in \C\setminus U$, so that $\dist_{\C\setminus U}\left( \zeta \right) \geq \frac{1}{n}$. In particular, this means that $C_n^{\circ}$ consists of all $z$ for which there exists $\ve$ such that $\dist_{\C\setminus U}\left( z \right) \geq \frac{1}{n} + \ve$, or more succinctly,
      \begin{align*}
        C_n^{\circ} &= \set{z\in U | \dist_{\C\setminus U}\left( z \right) > \frac{1}{n}}.
      \end{align*}
      In particular, since $ \frac{1}{n} > \frac{1}{n + 1} $, it follows that $C_n\subseteq C_{n+1}^{\circ}$. Paired with the fact that $B\left( 0,n \right)\subseteq U\left( 0,n+1 \right)$, we obtain that
      \begin{align*}
        K_n &= B\left( 0,n \right) \cap C_n\\
            &\subseteq U\left( 0,n+1 \right)\cap C_{n+1}^{\circ}\\
            &= \left( B\left( 0,n+1 \right)\cap C_n \right)^{\circ}\\
            &= K_{n+1}^{\circ}.
      \end{align*}
      Finally, to show that $U = \bigcup_{n = 1}^{\infty}K_n$, we write
      \begin{align*}
        \bigcup_{n= 1}^{\infty}K_n &= \bigcup_{n = 1} ^{\infty} \left( B\left( 0,n \right) \cap C_n \right)\\
                                   &= \left( \bigcup_{n=1}^{\infty}B\left( 0,n \right) \right)\cap \left( \bigcup_{n=1}^{\infty} C_n \right),
      \end{align*}
      and since $\bigcup_{n=1}^{\infty}B\left( 0,n \right) = \C$, it follows that we must show that
      \begin{align*}
        \bigcup_{n=1}^{\infty}C_n &= U.
      \end{align*}
      Towards this end, we prove that if $A\subseteq \C$ is any subset, then $ \dist_{A}(z) = 0 $ if and only if $z\in \overline{A}$. Towards this end, if $\dist_{A}(z) = 0$, then for any $k$, there is $w\in A$ such that $\left\vert w-z \right\vert < \frac{1}{n}$, so that we may construct a sequence $\left( w_n \right)_n$ in $A$ such that $\left( w_n \right)_n\rightarrow z$, or that $z\in \overline{A}$. Similarly, if $z\in \overline{A}$, then if $\left( w_n \right)_n$ is a sequence in $A$ converging to $z$, and $\ve > 0$, it follows that $\left\vert w_n - z \right\vert < \ve$ for sufficiently large $n$, so that $\inf_{w\in Z}\left\vert w-z \right\vert = 0$.\newline

      Since $U$ is open, it follows that for any $z\in \C\setminus U$, since $\C\setminus U = \overline{\C\setminus U}$, $\dist_{\C\setminus U} (z) = 0$. Equivalently, if $z\in U$, we must have $\dist_{\C\setminus U}(z) > 0$, so that there exists $n$ sufficiently large such that $\dist_{\C\setminus U}(z) \geq 1/n$; this means $z\in C_n$, so that
      \begin{align*}
        U &\subseteq \bigcup_{n=1}^{\infty}C_n.
      \end{align*}
      Meanwhile, if $z\in \bigcup_{n=1}^{\infty}C_n$, then there is some $N$ such that $\dist_{\C\setminus U}(z) \geq 1/N$, meaning that $\dist_{\C\setminus U}(z) > 0$, meaning $z\notin \C\setminus U$, so that $z\in U$.
    \item\hfill
      \begin{enumerate}[(i)]
        \item If $S = L + \ve$ for $\ve > 0$, it follows by the definition of the infimum that there exists a compact subset $K\subseteq U$ such that $\sup_{z\in U\setminus K} u(z) \leq S$. Therefore, for all $z\in U\setminus K$, $u(z)\leq S$.
        \item Let $L_n = L + \frac{1}{n}$. We find $K_{j_n}\subseteq U$ that satisfies
          \begin{itemize}
            \item $u(z)\leq L_n$ for all $z\in U\setminus K_{j_n}$;
            \item $\left\vert z \right\vert \leq j_n$ for all $z\in K_{j_n}$;
            \item $\dist_{\C\setminus U}(z) \geq \frac{1}{j_n}$.
          \end{itemize}
          The existence of such a $K_{j_n}$ follows from the proof in (i) and the definitions in part (a). We may find $z_n\in U\setminus K_{j_n}$, so that $u\left(z_n\right)\leq L_n$.\newline

          The sequence $\left( z_n \right)_n$ thus escapes all the $K_{j_n}$, and since any $K\subseteq U$ is contained in some sufficiently large $K_{j_n}$, it follows that $\left( z_n \right)_n\rightarrow \partial U$. Furthermore, since $u\left( z_n \right)\leq L_n$ for each $n$, we have
          \begin{align*}
            \limsup_{n\rightarrow\infty} u\left( z_n \right) &\leq \limsup_{n\rightarrow\infty}L_n\\
                                                             &= L.
          \end{align*}
      \end{enumerate}
    \item 
  \end{enumerate}
\end{solution}
\begin{problem}[Problem 2]
  Let
  \begin{align*}
    U &= \set{z\in \C | \left\vert z \right\vert < 1,\im(z) > 0}.
  \end{align*}
  \begin{enumerate}[(a)]
    \item Construct a conformal map from $U$ to $ \mathbb{H} = \set{z\in\C | \im(z) > 0} $.
    \item Construct an unbounded harmonic function $u\colon U\rightarrow (0,\infty)$ such that for all $\left( x_0,y_0 \right)\in \partial U \setminus \set{(1,0)}$, we have that $\lim_{\left( x,y \right)\rightarrow \left( x_0,y_0 \right)} u\left( x,y \right) = 0$.
    \item Suppose that $v\colon U\rightarrow (0,\infty)$ is an unbounded harmonic function such that for all $\left( x_0,y_0 \right)\in \partial U\setminus \set{\left( 1,0 \right)}$, we have that $\lim_{\left( x,y \right)\rightarrow \left( x_0,y_0 \right)} v\left( x,y \right) = 0$. Show that there exists a sequence $\left( \left( x_n,y_n \right) \right)_n$ in $U$ converging to $\left( 1,0 \right)$ and $\lim_{n\rightarrow\infty}v\left( x_n,y_n \right) = \infty$.
  \end{enumerate}
\end{problem}
\begin{solution}\hfill
  \begin{enumerate}[(a)]
    \item We start by taking the Cayley transform, mapping $ \mathbb{H} $ to $ \mathbb{D} $, given by $\frac{z-i}{z+i}$. The inverse Cayley transform, which maps $ \mathbb{D} $ to $ \mathbb{H} $, is then given by the inverse transform, which takes
      \begin{align*}
        Q(z) &= i\frac{1+z}{1-z}.
      \end{align*}
      By taking $a + bi\in U$ with $b > 0$ and $a^2 + b^2 \leq 1$, we find that
      \begin{align*}
        i\frac{1 + \left( a + bi \right)}{1-a-bi} &= \frac{1}{\left( 1-a \right)^2 + b^2} \left( -2b + i\left( 1-a^2-b^2 \right) \right).
      \end{align*}
      Therefore, we observe that the inverse transform maps $U$ to the second quadrant, admitting arguments between $\frac{\pi}{2}$ and $\pi$. By squaring, we have
      \begin{align*}
        \left( Q(z) \right)^2 &= -\left( \frac{z+1}{1-z} \right)^2,
      \end{align*}
      which maps to complex numbers with arguments between $\pi$ and $2\pi$. Multiplying by $-1$, we get
      \begin{align*}
        H(z) &= \left( \frac{z+1}{1-z} \right)^2
      \end{align*}
      mapping from $U$ to the upper half-plane. Since we composed a series of bijective holomorphic maps (and, within a correct domain for the case of square root, ones that have holomorphic inverse), it follows that $H$ is a bijective holomorphic map with holomorphic inverse, hence conformal.
    \item Consider the function
      \begin{align*}
        u\left( x,y \right) &= \im\left( H\left( x + yi \right) \right).
      \end{align*}
      We observe that $u$ is the imaginary part of a holomorphic function, so it is harmonic. Since $H$ maps $U$ conformally to the upper half-plane, it follows that $u$ maps $U$ to $\left( 0,\infty \right)$, and that $u$ is unbounded, as $H$ is unbounded. It remains to show that $u$ maps $\partial U$ to $0$ in limit save for $\left( 1,0 \right)$. Towards this end, we split the case into two parts.\newline

      If $x_0 + iy_0 = e^{i\theta}$ for some $0 < \theta_0 < \pi$, then
      \begin{align*}
        \frac{e^{i\theta} + 1}{1-e^{i\theta}} &= \frac{\left( 1 + \cos\left( \theta \right) + i\sin\left( \theta \right) \right)\left( 1 - \cos\left( \theta \right) + i\sin\left( \theta \right) \right)}{2-2\cos\left( \theta \right)}\\
                                              &= \frac{1}{2-2\cos\left( \theta \right)} \left( 1-\cos^2\left( \theta \right)-\sin^2\left( \theta \right) + 2i\sin\left( \theta \right) \right)\\
                                              &= \frac{2i\sin\left( \theta \right)}{2-2\cos\left( \theta \right)}\\
                                              &= i\cot\left( \theta/2 \right).
      \end{align*}
      Squaring, we then get
      \begin{align*}
        \left( \frac{e^{i\theta} + 1}{1-e^{i\theta}} \right)^2 &= -\cot^2\left( \theta/2 \right)\\
                                                               &\in \R,
      \end{align*}
      so that $u\left( x_0,y_0 \right) = 0$ whenever $x_0 + iy_0 = e^{i\theta}$ for some $0 < \theta_0 < \pi$.\newline

      Meanwhile, if $x_0 + iy_0 = x_0$ with $x_0\neq 1$, then
      \begin{align*}
        H\left( x_0 + iy_0 \right) &= \left( \frac{x_0 + 1}{1-x_0} \right)^2\\
                                   &\in \R,
      \end{align*}
      so that $u\left( x_0,y_0 \right) = 0$ yet again.
    \item We let $v \equiv u$, where $u$ is defined as above. Since $u$ is unbounded, it follows that for each $N\geq 1$, there is $\left( x_N,y_N \right)\in U$ such that $u\left( x_N,y_n \right) \geq N$. Inductively, this allows us to construct a sequence $\left( x_n,y_n \right)\subseteq U$ such that $u\left( x_n,y_n \right) \geq n$, meaning that $\lim_{n\rightarrow\infty}u\left( x_n,y_n \right) =\infty$.\newline

      Since $u$ is harmonic, it is subharmonic, so by a previously established theorem, since $\sup\left( u \right) = \infty$, it follows that $\left( \left( x_n,y_n \right) \right)_n\rightarrow \partial U$. Yet, this sequence cannot converge to any element of $\partial U\setminus \set{\left( 1,0 \right)}$, as otherwise, we would have $u\left( x_n,y_n \right)\rightarrow 0$, which would contradict the fact that $u$ is continuous as it is harmonic. Therefore, we have $\left( \left( x_n,y_n \right) \right)_{n}\rightarrow \left( 1,0 \right)$.
  \end{enumerate}
\end{solution}
\begin{problem}[Problem 3]
  Let 
  \begin{align*}
    U &= \set{z\in \C | 0 < \re(z) < 1}.
  \end{align*}
  Let $f\colon \overline{U}\rightarrow \C$ be a continuous bounded function for which $ f|_{U} $ is holomorphic. Suppose there exist constants $M_0\geq 0$ and $M_1\geq 0$ such that
  \begin{align*}
    \sup_{\re(z) = 0} \left\vert f(z) \right\vert &\leq M_0\\
    \sup_{\re(z) = 1} \left\vert f(z) \right\vert &\leq M_1.
  \end{align*}
  Show that for all $r\in [0,1]$,
  \begin{align*}
    \sup_{\re(z) = r} \left\vert f(z) \right\vert \leq M_0^{1-r}M_1^{r}.
  \end{align*}
\end{problem}
\begin{solution}
  Let $\ve > 0$ be fixed. Define
  \begin{align*}
    f_{\ve}(z) &= f(z) M_0^{z-1}M_1^{-z}e^{\ve\left( z^2 - 1 \right)}.
  \end{align*}
  We will show that $\sup_{z\in \overline{U}}\left\vert f_{\ve}(z) \right\vert \leq 1$. Towards this end, if $\re(z) = 0$, we have $z = bi$ for some $b \in \R$; since $M_0,M_1\in \R_{\geq 0}$, we then get
  \begin{align*}
    \left\vert f(z)M_0^{z-1}M_1^{-z}e^{\ve \left( z^2 - 1 \right)} \right\vert &= \left\vert f(z)M_0^{bi-1}M_1^{-bi}e^{-\ve\left( b^2 + 1 \right)} \right\vert\\
                                                                               &= \left\vert f(z)M_0^{-1}e^{-\ve\left( b^2 + 1 \right)} \right\vert\\
                                                                               &\leq \left\vert f(z)M_0^{-1} \right\vert\\
                                                                               &\leq 1.
  \end{align*}
  Similarly, if $\re(z) = 1$, then we have $z = 1 + bi$ for some $b\in \R$, and since $M_0,M_1\in \R_{\geq 0}$, we have
  \begin{align*}
    \left\vert f(z) M_0^{z-1}M_1^{-z}e^{\ve\left( z^2 - 1 \right)} \right\vert &= \left\vert f(z) M_0^{bi}M_1^{-bi-1}e^{\ve\left( -2bi-b^2 \right)} \right\vert\\
                                                                               &= \left\vert f(z)M_1^{-1}e^{-b^2\ve} \right\vert\\
                                                                               &\leq \left\vert f(z)M_1^{-1} \right\vert\\
                                                                               &\leq 1.
  \end{align*}
  Since $\left\vert f_{\ve}(z) \right\vert \leq 1$ holds on both $\re(z) = 0$ and $\re(z) = 1$, it follows by the maximum modulus principle that we must have $\left\vert f_{\ve}(z) \right\vert \leq 1$ on the interior. In particular, this means that
  \begin{align*}
    \sup_{z\in \overline{U}} \left\vert f_{\ve}(z) \right\vert \leq 1.
  \end{align*}
  Since $\ve > 0$ is arbitrary, it follows that
  \begin{align*}
    \left\vert f(z)M_0^{z-1}M_1^{-z} \right\vert &\leq 1
  \end{align*}
  for all $z\in \overline{U}$, so that
  \begin{align*}
    \left\vert f(z) \right\vert&\leq \left\vert M_0^{1-z} \right\vert \left\vert M_1^{z} \right\vert\\
                               &= M_0^{1-\re(z)} M_1^{\re(z)}.
  \end{align*}
  In particular, this means that for $\re(z) = r$, we have
  \begin{align*}
    \left\vert f(z) \right\vert &\leq M_0^{1-r}M_1^{r},
  \end{align*}
  meaning this holds for the supremum over all $z$ with $\re(z) = r$, yielding
  \begin{align*}
    \sup_{\re(z) = r} \left\vert f(z) \right\vert &\leq M_0^{1-r}M_1^{r}.
  \end{align*}
\end{solution}
\begin{problem}
  Let $U\subseteq \C$ be a region, $f\colon U\rightarrow \C\setminus \set{0}$ a holomorphic function. We say that $f$ has a logarithm in $U$ if there exists a holomorphic function $g\colon U\rightarrow \C$ satisfying $f(z) = e^{g(z)}$ for all $z\in U$.
  \begin{enumerate}[(a)]
    \item Show that $f$ has a logarithm in $U$ if and only if for every piecewise $C^{1}$ cycle $\Gamma$ in $U$, we have
      \begin{align*}
        \oint_{\Gamma}^{} \frac{f'}{f}\:dz &= 0.
      \end{align*}
    \item If $\gamma$ is a piecewise $C^{1}$ loop in $U$, show that
      \begin{align*}
        \oint_{\gamma}^{} \frac{f'}{f}\:dz &= 2\pi i n\left( f\circ\gamma;0 \right).
      \end{align*}
    \item Let $J\subseteq \N$ be a countably infinite set. Show that $f$ has a logarithm in $U$ if and only if it has $k$th roots for every $k\in J$, so that for each $k\in J$, there is a holomorphic function $h\colon U\rightarrow \C\setminus \set{0}$ satisfying $h(z)^{k} = f(z)$ for all $z\in U$.
  \end{enumerate}
\end{problem}
\begin{solution}\hfill
  \begin{enumerate}[(a)]
    \item Let $f(z)$ have the logarithm $e^{g(z)}$. If $\gamma$ is any piecewise $C^{1}$ loop in $U$, we wish to show that
      \begin{align*}
        \oint_{\gamma}^{} \frac{f'(z)}{f(z)}\:dz &= 0,
      \end{align*}
      meaning that for any piecewise $C^{1}$ cycle $\Gamma = \gamma_1 + \cdots + \gamma_n$, we would have this sum add up to zero.\newline

      Toward this end, we observe that
      \begin{align*}
        \oint_{\gamma}^{} \frac{f'(z)}{f(z)}\:dz &= \oint_{\gamma}^{} g'(z)\:dz,
      \end{align*}
      meaning that this integral is equal to zero over $\gamma$.\newline

      Now, if
      \begin{align*}
        \int_{\Gamma}^{} \frac{f'(z)}{f(z)}\:dz &= 0
      \end{align*}
      for all piecewise $C^{1}$ cycles $\Gamma$ in $U$, then it certainly follows for a cycles consisting of a single $C^{1}$ loop, $\gamma$, where $\img(\gamma)\subseteq U$. Therefore, by an established proposition, there is some holomorphic function $g\colon U\rightarrow \C$ such that
      \begin{align*}
        g'(z) &= \frac{f'(z)}{f(z)}.
      \end{align*}
      Following a similar argument to the specific case of $U$ being a simply connected region and $f$ nonvanishing, we see that if we define
      \begin{align*}
        h(z) &= f(z)e^{-g(z)},
      \end{align*}
      that
      \begin{align*}
        h'(z) &= f'(z)e^{-g(z)} - f(z)g'(z)e^{-g(z)}\\
              &= f'(z)e^{-g(z)} - f(z)\frac{f'(z)}{f(z)}e^{-g(z)}\\
              &= 0,
      \end{align*}
      so that $h(z)$ is some constant $k\in \C$, following from the fundamental theorem of calculus as, if $z_0\in U$, $r> 0$, and $z\in U\left( z_0,r \right)$, we get
      \begin{align*}
        h(z) - h\left(z_0\right) &= \int_{0}^{1} h'\left( \left( 1-t \right)z_0 + tz \right)\left( z-z_0 \right)\:dt\\
                                 &= 0,
      \end{align*}
      meaning $h$ is constant on $U\left( z_0,r \right)$, hence constant by the identity theorem.\newline

      Since $f$ is never zero, it follows that $k\neq 0$, so that there is some $k^{\ast}\in \C$ with $e^{k^{\ast}} = k$, and
      \begin{align*}
        f(z) &= e^{k^{\ast}g(z)}.
      \end{align*}
      Since $k^{\ast}g(z)$ is also holomorphic, we thus find that $f$ has a logarithm.
    \item Let $\gamma\colon [a,b]\rightarrow U$ be a piecewise $C^{1}$ loop. We observe then that
      \begin{align*}
        \oint_{\gamma}^{} \frac{f'(z)}{f(z)}\:dz &= \int_{a}^{b} \frac{\gamma'(t)f'\left( \gamma(t) \right)}{f\left(\gamma(t)\right)}\:dt\\
                                                 &= \int_{a}^{b} \frac{1}{\left( f\circ\gamma \right)(t)}\left( f\circ\gamma \right)'(t)\:dt\\
                                                 &= \oint_{f\circ\gamma}^{} \frac{1}{w}\:dw\\
                                                 &= 2\pi i n\left( f\circ\gamma;0 \right).
      \end{align*}
    \item If $f$ has a logarithm in $U$, then we have a holomorphic $g(z)$ such that $f(z) = e^{g(z)}$. Then, if $k\in J$, we have
      \begin{align*}
        h(z) &= e^{\frac{1}{k}g(z)}
      \end{align*}
      satisfies $h(z)^{k} = \left( e^{\frac{1}{k}g(z)} \right)^{k} = e^{g(z)}$, so that $f(z) = h(z)^{k}$.\newline

      Suppose that for all $k\in J$, there is $h_k(z)$ such that $h_k(z)^{k} = f(z)$. Now, since $J$ is a countably infinite subset of $N$, $J$ is necessarily unbounded. Letting $\gamma\colon [a,b]\rightarrow U$ be a satisfactory piecewise $C^{1}$ loop, we then observe that for arbitrary $k\in J$, we must have
      \begin{align*}
        \oint_{\gamma}^{} \frac{f'(z)}{f(z)}\:dz &= k \oint_{\gamma}^{} \frac{h_k'(z)}{h_k(z)}\:dz.
      \end{align*}
      Since the integral on the left is single-valued, and $k$ is arbitrary and unbounded, it follows that both sides are necessarily equal to zero, so that
      \begin{align*}
        \oint_{\gamma}^{} \frac{f'(z)}{f(z)}\:dz &= 0
      \end{align*}
      for any piecewise $C^{1}$ loop. Thus, by part (a), we see that $f$ has a logarithm.
  \end{enumerate}
\end{solution}
\begin{problem}[Problem 5]\hfill
  \begin{enumerate}[(a)]
    \item Let $U\subseteq \C$ be a region, $f\colon U\rightarrow \C\setminus \set{0}$ a holomorphic function, and let $g$ be a logarithm of $f$. Suppose there exist $z,w\in U$ such that $f(z) = f(w)$. Show that for any piecewise $C^{1}$ curve $\gamma\colon [a,b]\rightarrow U$ for which $\gamma(a) = z$ and $\gamma(b) = w$, we have $g(z) - g(w) = 2\pi i n \left( f\circ\gamma;0 \right)$.
    \item Let $V\subseteq \C$ be a simply connected region. Fix a finite collection of points $a_1,\dots,a_k\in V$ and define $U = V\setminus \set{a_1,\dots,a_k}$. Let $\delta > 0$ be sufficiently small such that $ B\left( a_i,\delta \right)\cap B\left( a_j,\delta \right) = \emptyset $ for $i\neq j$, and let $\gamma_j\colon [0,2\pi]\rightarrow U$ be given by $\gamma_j\left( \theta \right)= a_j + \delta e^{i\theta}$.\newline

      Show that a holomorphic function $f\colon U\rightarrow \C\setminus \set{0}$ admits a logarithm if and only if $n\left( f\circ\gamma_j;0 \right) = 0$ for all $j$.
  \end{enumerate}
\end{problem}
\begin{solution}\hfill
  \begin{enumerate}[(a)]
    \item Let $\gamma\colon [a,b]\rightarrow U$ be a piecewise $C^{1}$ curve with $\gamma(a) = z$  and $\gamma(b) = w$. Writing $f(z) = e^{g(z)}$, we observe that, by the fundamental theorem of calculus,
      \begin{align*}
        \int_{\gamma}^{} \frac{f'(z)}{f(z)}\:dz &= \int_{\gamma}^{} g'(z)\:dz\\
                                                &= \int_{a}^{b} g'\left( \gamma(t) \right)\gamma'(t)\:dt\\
                                                &= \int_{a}^{b} \left( g\circ\gamma \right)'(t)\:dt\\
                                                &= \left( g\circ\gamma \right)(b) - \left( g\circ\gamma \right)(a)\\
                                                &= g(w)-g(z).
      \end{align*}
      Meanwhile, by direct substitution, and using the fact that $f(z) = f(w)$, we obtain
      \begin{align*}
        \int_{\gamma}^{} \frac{f'(z)}{f(z)}\:dz &= \int_{a}^{b} \frac{f'\left( \gamma(t) \right)}{f\left( \gamma(t) \right)} \gamma'(t)\:dt\\
                                                &= \int_{a}^{b} \frac{1}{f\left( \gamma(t) \right)}\left( f\circ\gamma \right)'(t)\:dt\\
                                                &= \oint_{f\circ\gamma}^{} \frac{1}{\zeta}\:d\zeta\\
                                                &= 2\pi i n\left( f\circ\gamma;0 \right).
      \end{align*}
    \item We use the condition in Problem 4 that $f$ has a logarithm if and only if for every piecewise $C^{1}$ cycle in $U$, we have
      \begin{align*}
        \oint_{\gamma}^{} \frac{f'(z)}{f(z)}\:dz &= 0.
      \end{align*}
      Since each of the $\gamma_j$ are piecewise $C^{1}$, it follows that if $f$ has a logarithm in $U$, then
      \begin{align*}
        \oint_{\gamma_j}^{} \frac{f'(z)}{f(z)}\:dz &= 2\pi i n\left( f\circ\gamma_j;0 \right)\\
                                                   &= 0.
      \end{align*}
      meaning that $n\left( f\circ\gamma_j;0 \right) = 0$ for each $\gamma_j$.\newline

      Now, suppose $f$ satisfies $n\left( f\circ\gamma_j;0 \right) = 0$ for each $j$. Let $\gamma$ be any piecewise $C^{1}$ loop in $U$, and define a piecewise $C^{1}$ cycle
      \begin{align*}
        \Gamma &= \gamma - \sum_{j=1}^{k}n\left( \gamma;a_j \right) \gamma_j.
      \end{align*}
      We claim that $\Gamma$ is homologous to zero in $U$. Toward this end, we observe that $\C\setminus U$ consists of two disjoint sets: $\C\setminus V$ and $\set{a_1,\dots,a_k}$. We observe that $\Gamma$ is a cycle in $V$ as $\Gamma$ is a cycle in $U$, so $n\left( \Gamma;z \right) = 0$ for any $z\in\C\setminus V$ as $V$ is simply connected. Meanwhile, we have
      \begin{align*}
        n\left( \Gamma;a_i \right) &= \oint_{\Gamma}^{} \frac{1}{z-a_i}\:dz\\
                                   &= \oint_{\gamma}^{} \frac{1}{z-a_i}\:dz - \sum_{j=1}^{k} n\left( \gamma;a_j \right)\oint_{\gamma_j}^{} \frac{1}{z-a_i}\:dz\\
                                   &= n\left( \gamma;a_i \right) - n\left( \gamma;a_i \right)\\
                                   &= 0,
      \end{align*}
      where we use the fact that $n\left( \gamma_j;a_j \right) = 1$ by a previously established result.\newline

      Thus, we see that $\Gamma$ is homologous to zero in $U$. By the generalization of Cauchy's Integral Theorem, it follows that, since $f(z)\neq 0$ for all $z\in U$, we have
      \begin{align*}
        0 &= \oint_{\Gamma}^{} \frac{f'(z)}{f(z)}\:dz\\
          &= \oint_{\gamma}^{} \frac{f'(z)}{f(z)}\:dz - \sum_{j=1}^{k} n\left( \gamma;a_j \right) \oint_{\gamma_j}^{} \frac{f'(z)}{f(z)}\:dz\\
          &= \oint_{\gamma}^{} \frac{f'(z)}{f(z)}\:dz - \sum_{j=1}^{k} n\left( \gamma;a_j \right)n\left( f\circ\gamma_j;0 \right)\\
          &= \oint_{\gamma}^{} \frac{f'(z)}{f(z)}\:dz.
      \end{align*}
      Since $\gamma$ is an arbitrary piecewise $C^{1}$ loop in $U$, this holds for all such formal linear combinations of piecewise $C^{1}$ loops, hence for all piecewise $C^{1}$ cycles in $U$. Thus, $f$ admits a logarithm in $U$.
  \end{enumerate}
\end{solution}
\end{document}
