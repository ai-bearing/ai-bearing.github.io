\documentclass[10pt]{mypackage}

% sans serif font:
%\usepackage{cmbright}
%\usepackage{sfmath}
%\usepackage{bbold} %better blackboard bold

\usepackage{homework}
%\usepackage{notes}
\usepackage{newpxtext,eulerpx,eucal}
\renewcommand*{\mathbb}[1]{\varmathbb{#1}}

\fancyhf{}
\fancyhead[R]{Avinash Iyer}
\fancyhead[L]{Complex Analysis: Assignment 11}
\fancyfoot[C]{\thepage}

\setcounter{secnumdepth}{0}

\begin{document}
\RaggedRight
\begin{problem}[Problem 1]
  Let $U\subseteq \C$ be a region, and let $V\coloneq \set{re^{i\theta}\in \C | -\pi/4 < \theta < \pi/4,r > 0}$. Fix $z_0\in U$, and let $\mathcal{F} \coloneq \set{f\in H\left( U \right) | f\left( z_0 \right) = 1, \img\left( f \right) \subseteq V}$. Show that $\mathcal{F}$ is normal.
\end{problem}
\begin{solution}
  We observe that a function $f\in H\left( U \right)$ if and only if $f\left( z_0 \right) = 1$ and $\img\left( f \right)\subseteq V$, or equivalently, that $e^{i\pi/4}f\left( z_0 \right) = e^{i\pi/4}$ and $\img\left( e^{i\pi/4}f \right)$ is a subset of the upper half-plane. In particular, this means that we seek to establish the normality of the family
  \begin{align*}
    \mathcal{G} &= \set{f\in H\left( U \right) | \img(f)\subseteq \mathbb{H},f\left( z_0 \right) = e^{i\pi/4}}.
  \end{align*}
  Toward this end, we use the Cayley transform, $q(z) = \frac{z-i}{z+i}$ to conformally map the upper half-plane to the unit disk, establishing that the family
  \begin{align*}
    \mathcal{D} &= \set{q\circ g | g\in \mathcal{G}}
  \end{align*}
  is locally bounded (indeed, globally bounded by $1$). Furthermore, every element of $\mathcal{D}$ has the property that
  \begin{align*}
    q\circ f \left( z_0 \right) &= \frac{e^{i\pi/4} - i}{e^{i\pi/4} + i}\\
                                &\in \D.
  \end{align*}
  Now, let $\left( f_n \right)_n\subseteq \mathcal{F}$. Then, $g_n\coloneq e^{i\pi/4}f_n$ is a sequence in $\mathcal{G}$, and $h_n\coloneq q\circ g_n$ is a sequence in $\mathcal{D}$. Since $\mathcal{D}$ is normal, there is a subsequence $\left( h_{n_k} \right)_k\rightarrow h\colon U\rightarrow \overline{\D}$ satisfying $h\left( z_0 \right) = \frac{e^{i\pi/4}-i}{e^{i\pi/4} + i}$, meaning that $h$ is a holomorphic function mapping $U\rightarrow \D$. Since $q$ and multiplication by $e^{i\pi/4}$ are conformal maps on their respective domains, it then follows that
  \begin{align*}
    \left( f_{n_k} \right)_k &= \left( e^{-i\pi/4}q^{-1}\circ h_{n_k} \right)_k\\
                             &\rightarrow e^{-i\pi/4}q^{-1}\circ h\\
                             &\in H\left( U \right),
  \end{align*}
  meaning that $\mathcal{F}$ is a normal family.
\end{solution}
\begin{problem}[Problem 2]
  Let $\mathcal{F} = \set{f\in H\left( \D \right) | \img\left( f \right)\subseteq \D}$. Fix $z_0\in \D$. Show that there exists a holomorphic function $g\colon \D\rightarrow \C$ with $\img\left( g \right)\subseteq \D$, $\left\vert g'\left( z_0 \right) \right\vert = \max_{f\in \mathcal{F}}\left\vert f'\left( z_0 \right) \right\vert$, and $g\left( z_0 \right) = 0$.
\end{problem}
\begin{solution}
  From Montel's Theorem, we know that the set $\mathcal{F}$ is normal, meaning that $ \overline{\mathcal{F}} $ is compact in $H\left( \D \right)$.\newline

  Now, we start by showing that the family 
  \begin{align*}
    \mathcal{F}' &= \set{f' | f\in \mathcal{F}}
  \end{align*}
  is normal, by showing that it is locally bounded. Let $z\in \D$, let $B\left( z,r \right)\subseteq \D$, and let $m = \sup_{f\in \mathcal{F}} \norm{f}_{B\left( z,r \right)}$. Note that by the extended maximum modulus principle,
  \begin{align*}
    \sup_{z\in B\left( z,r \right)} \left\vert f(z) \right\vert &= \sup_{z \in S\left( z,r \right)}\left\vert f(z) \right\vert.
  \end{align*}
  For a given $f\in \mathcal{F}$, Cauchy's estimate gives
  \begin{align*}
    \left\vert f'(z) \right\vert &\leq \frac{1}{r}\sup_{\left\vert \xi - z \right\vert = r} \left\vert f(\xi) \right\vert\\
                                 &= \frac{1}{r} \sup_{\xi\in S\left( z,r \right)} \left\vert f(\xi) \right\vert\\
                                 &\leq \frac{m}{r},
  \end{align*}
  meaning in particular that
  \begin{align*}
    \sup_{f'\in \mathcal{F}'} \left\vert f'(z) \right\vert &\leq \frac{m}{r},
  \end{align*}
  whence $\mathcal{F}'$ is locally bounded. Thus, by Montel's Theorem, it follows that $\mathcal{F}'$ is normal. Since both evaluation and the modulus are continuous operations, we observe then that the map
  \begin{align*}
    s\colon \overline{\mathcal{F}'} &\rightarrow \R\\
    f' &\mapsto \left\vert f'\left( z_0 \right) \right\vert
  \end{align*}
  is a continuous map whose domain is a compact space, so there is some $h\in \overline{\mathcal{F}'}$ such that 
  \begin{align*}
    \left\vert h\left( z_0 \right) \right\vert = \sup_{f\in \mathcal{F}} \left\vert f'\left( z_0 \right) \right\vert
  \end{align*}
  Since $\D$ is simply connected, there is some holomorphic antiderivative for $h$, given by $g\in H\left( \D \right)$. We claim that it must be the case that $g\in \overline{\mathcal{F}}$. Since $h\in \overline{\mathcal{F}'}$, there is some sequence of function $\left( f_n' \right)_{n}\rightarrow h = g'$ uniformly on compacts. Fix an exhaustion $\left( K_m \right)_m$ given by
  \begin{align*}
    K_m &= B\left( 0,\frac{m}{m+1} \right).
  \end{align*}
  Then, we have $tz\in K_m$ for all $z\in K_m$ and all $0 \leq t \leq 1$. In particular, this means that
  \begin{align*}
    \left\vert f_n(z) - g(z) \right\vert &= \left\vert \int_{0}^{1} z \left( f_n'(tz)- g'(tz)\right)\:dt \right\vert\\
                                         &\leq \int_{0}^{1} \left\vert z \left( f_n'\left( tz \right)-g'\left( tz \right) \right) \right\vert\:dt\\
                                         &\leq \int_{0}^{1} \left\vert f_n'\left( tz \right) - g'\left( tz \right) \right\vert\:dt\\
                                         &\leq \int_{0}^{1} \sup_{t\in [0,1]}\left\vert f_n'\left( tz \right) - g'\left( tz \right) \right\vert\:dt\\
                                         &\leq \int_{0}^{1} \sup_{w\in K_m}\left\vert f_n'\left( w \right) - g'\left( w \right) \right\vert\:dt\\
                                         &= \sup_{w\in K_m} \left\vert f_n'\left( w \right) - g'\left( w \right) \right\vert\\
                                         &= \norm{f_n' - g'}_{K_m},
  \end{align*}
  so it follows that $\norm{f_n - g}_{K_m} \leq \norm{f_n' - g'}_{K_m}$. In particular, since the latter tends to zero as $K_m$ is compact, it follows that the former tends to zero as well. In particular, this means that $\norm{f_n - g}_{H\left( \D \right)}\rightarrow 0$, whence $f_n \rightarrow g$ uniformly on compacts. Thus, it follows that $g\in \overline{\mathcal{F}}$, so that $\img\left( g \right)\subseteq \overline{\D}$.\newline

  Furthermore, since $f(z) = z\in \mathcal{F}$, it follows that $\left\vert g'\left( z_0 \right) \right\vert \geq 1$, meaning that $g$ is a nonconstant holomorphic function, meaning in particular that since $g\left( \D \right)\subseteq \overline{\D}$ already, we must indeed have $g\left( \D \right)\subseteq \D$ by the open mapping principle.\newline

  Now, we claim that $g\left( z_0 \right) = 0$. Suppose this were not the case. Then, there would be some $0 < r < 1$ with $\left\vert g\left( z_0 \right) \right\vert = r$. We have established on a previous assignment that the map
  \begin{align*}
    h_0(z) &= \frac{z-g\left( z_0 \right)}{1- \overline{g\left( z_0 \right)}z}
  \end{align*}
  is a bijective holomorphic mapping of $\D$ to itself, meaning that
  \begin{align*}
    h(z) &= \frac{g(z) - g\left( z_0 \right)}{1- \overline{g\left( z_0 \right)}g(z)}
  \end{align*}
  maps $\D$ to $\D$, so that $h\in \mathcal{F}$, with
  \begin{align*}
    h'\left( z \right) &= \frac{g'\left( z \right)}{1- \overline{g\left( z_0 \right)}g\left( z \right)} + \overline{g\left( z_0 \right)}g'\left( z \right) \frac{g\left( z \right) - g\left( z_0 \right)}{\left( 1- \overline{g\left( z_0 \right)}g(z) \right)^2}\\
    \left\vert h'\left( z_0 \right) \right\vert &= \left\vert g'\left( z_0 \right) \right\vert \frac{1}{1-\left\vert g\left( z_0 \right) \right\vert^2}\\
                                                &= \left\vert g'\left( z_0 \right) \right\vert \frac{1}{1-r^2}\\
                                                &> \left\vert g'\left( z_0 \right) \right\vert,
  \end{align*}
  which contradicts the maximality of $\left\vert g'\left( z_0 \right) \right\vert$. Thus, it must be the case that $g\left( z_0 \right) = 0$.
\end{solution}
\begin{problem}[Problem 3]
  Let $\left( a_n \right)_n$ be a sequence of nonnegative real numbers such that the radius of convergence of
  \begin{align*}
    \sum_{n=0}^{\infty} a_nz^{n}
  \end{align*}
  is at least $1$. Let
  \begin{align*}
    \mathcal{F} &\coloneq \bigcap_{n=0}^{\infty} \set{f\in H\left( \D \right) | \left\vert \frac{f^{(n)}(0)}{n!} \right\vert \leq a_n}.
  \end{align*}
  Show that $\mathcal{F}$ is a normal family.
\end{problem}
\begin{solution}
  Suppose $z\in \D$. We wish to establish some $\delta > 0$ and some $M > 0$ such that $U\left( z,\delta \right)\subseteq \D$, for all $f\in \mathcal{F}$,
  \begin{align*}
    \left\vert f(z) \right\vert &\leq M.
  \end{align*}
  Now, let $r > 0$ be such that $B\left( z,r \right)\subseteq \D$. We observe that for $f\in \mathcal{F}$, we have
  \begin{align*}
    f(z) &= \sum_{n=0}^{\infty}\frac{f^{(n)(0)}}{n!}z^{n},
  \end{align*}
  with $\left\vert \frac{f^{(n)}(0)}{n!} \right\vert\leq a_n$. By uniform convergence, we then see that for any $f\in \mathcal{F}$ and any $w\in B\left( z,r \right)$,
  \begin{align*}
    \left\vert f(w) \right\vert &\leq \sum_{n=0}^{\infty} \left\vert \frac{f^{(n)}(0)}{n!} \right\vert \left\vert w \right\vert^{n}\\
                                &\leq \sum_{n=0}^{\infty} a_n\left\vert w \right\vert^{n}\\
                                &\leq \sum_{n=0}^{\infty} a_n \left( \left\vert z \right\vert + r \right)^{n}\\
                                &\eqcolon C,
  \end{align*}
  since $\left\vert z \right\vert + r$ is less than $1$, while $\sum_{n=1}^{\infty}a_nz^{n}$ has radius of convergence at least $1$. Since this holds for all $f\in \mathcal{F}$, it follows that the family $\mathcal{F}$ is locally bounded, hence normal by Montel's Theorem.
\end{solution}
\begin{problem}[Problem 4]\hfill
  \begin{enumerate}[(a)]
    \item Fix $z_0\in \C$ and $r > 0$. Suppose $f\colon B\left( z_0,r \right)\rightarrow \C$ is continuous, and $f|_{U\left( z_0,r \right)}$ is holomorphic. Fix $0 < \rho < r$. Show that for all $z\in U\left( z_0,\rho \right)$,
      \begin{align*}
        \left\vert f(z) \right\vert &\leq \frac{1}{\pi\left( r-\rho \right)^2} \iint_{U\left( z_0,r \right)}^{} \left\vert f\left( x + iy \right) \right\vert\:dxdy.
      \end{align*}
    \item Fix $M\geq 0$, let $U\subseteq \C$ be open, and let $\mathcal{F}\subseteq H\left( U \right)$ be the family of holomorphic functions for which
      \begin{align*}
        \iint_{U}^{} \left\vert f\left( x + iy \right) \right\vert\:dxdy&\leq M.
      \end{align*}
      Show that $\mathcal{F}$ is normal.
  \end{enumerate}
\end{problem}
\begin{solution}\hfill
  \begin{enumerate}[(a)]
    \item For each $\rho \leq t \leq r$, we parametrize $S\left( z_0,t \right)$ as $\gamma(t) = z_0 + te^{i\theta}$. In particular, by Cauchy's Integral Formula, we get
      \begin{align*}
        f(z) &= \frac{1}{2\pi i} \int_{S\left( z_0,t \right)}^{} \frac{f\left( w \right)}{w - z}\:dw\\
             &= \frac{1}{2\pi} \int_{0}^{2\pi} \frac{f\left( z_0 + te^{i\theta} \right) te^{i\theta}}{\left( z_0 - z \right) + te^{i\theta}}\:d\theta.
      \end{align*}
      Introducing a factor of $1$, then using Fubini's Theorem thus gives
      \begin{align*}
        f(z) &= \frac{1}{2\pi \left( r-\rho \right)} \int_{\rho}^{r} \int_{0}^{2\pi} \frac{f\left( z_0 + te^{i\theta} \right)te^{i\theta}}{\left( z_0 - z \right) + te^{i\theta}}\:d\theta\:dt\\
             &= \frac{1}{2\pi \left( r - \rho \right)} \int_{0}^{2\pi} \int_{\rho}^{r} \frac{f\left( z_0 + te^{i\theta} \right) te^{i\theta}}{\left( z_0-z \right) + te^{i\theta}}\:dt\:d\theta
      \end{align*}
      Estimating the integral, we find that for $\rho < t \leq r$,
      \begin{align*}
        \left\vert \frac{1}{\left( z_0-z \right) + te^{i\theta}} \right\vert &\leq \frac{1}{\left\vert t \right\vert - \left\vert z_0 - z \right\vert}\\
                                                                             &\leq \frac{1}{\left\vert t \right\vert - \rho}.
      \end{align*}
      Since this holds for any $\rho < t \leq r$, it certainly holds for $t = r$, whence using this estimate and the triangle inequality gives
      \begin{align*}
        \left\vert f(z) \right\vert &\leq \frac{1}{2\pi \left( r - \rho \right)^2} \int_{0}^{2\pi} \int_{\rho}^{r} \left\vert f\left( z_0 + te^{i\theta} \right)te^{i\theta} \right\vert\:dt\:d\theta\\
                                    &\leq \frac{1}{\pi \left( r- \rho \right)^2} \int_{0}^{2\pi} \int_{0}^{r} t \left\vert f\left( z_0 + te^{i\theta} \right) \right\vert\:dt\:d\theta\\
                                    &= \frac{1}{\pi\left( r-\rho \right)^2} \iint_{U\left( z_0,r \right)}\left\vert f\left( x + iy \right) \right\vert\:dxdy.
      \end{align*}
    \item Let $z_0\in U$. Fix some $r > 0$ such that $B\left( z_0,r \right)\subseteq U$. In particular, we have
      \begin{align*}
        \iint_{U\left( z_0,r \right)} \left\vert f\left( x + iy \right) \right\vert\:dxdy &\leq \iint_{U} \left\vert f\left( x + iy \right) \right\vert\:dxdy\\
                                                                                          &\leq M,
      \end{align*}
      so for a fixed $0 < \rho < r$, we have for all $z\in U\left( z_0,\rho \right)$ and all $f\in \mathcal{F}$,
      \begin{align*}
        \left\vert f(z) \right\vert &\leq \frac{1}{\pi\left( r-\rho \right)^2} \iint_{U\left( z_0,r \right)} \left\vert f\left( x + iy \right) \right\vert\:dxdy\\
                                    &\leq \frac{1}{\pi \left( r-\rho \right)^2} M.
      \end{align*}
      In particular, this means that for all $f\in \mathcal{F}$, $f$ is bounded on $U\left( z_0,\rho \right)$, whence $\mathcal{F}$ is locally bounded, hence normal by Montel's Theorem.
  \end{enumerate}
\end{solution}
\begin{problem}[Problem 5]
  Let $\left( f_n \right)_n$ be a sequence of holomorphic functions from $\D$ to $\C$ that is locally bounded, and suppose there exists a holomorphic function $f\colon \D\rightarrow \C$ such that the set $\set{z\in \D | \lim_{n\rightarrow\infty}f_n(z) = f(z)}$ has a limit point in $\D$. Show that $\left( f_n \right)_n$ converges uniformly on compact sets to $f$.
\end{problem}
\begin{solution}
  Since $\left( f_n \right)_n$ is locally bounded, it follows that the family $\set{f_n | n\in \N}$ is a normal family, by Montel's theorem. In particular, this means that for any subsequence $\left( f_{n_k} \right)_k$, there is a subsequence of $\left( n_k \right)_k$, which we call $\left( n_{k_j} \right)_j$ and a holomorphic function $g_k\colon \D\rightarrow \C$ such that
  \begin{align*}
    \left( f_{n_{k_j}} \right)_j \rightarrow g_k
  \end{align*}
  on compact subsets. Yet, since uniform convergence on compact subsets implies pointwise convergence, we have that $\set{z\in \D | g_k(z) = f(z)}$ has an accumulation point in $\D$, whence each of the $g_k$ are equal to $f$ by the identity theorem.\newline

  Now, if it were not the case that $\left( f_n \right)_n\rightarrow f$ uniformly on compacts, then we would be able to find some subsequence $\left( f_{n_k} \right)_k$ with $\norm{f_{n_{k}} - f}\geq \ve_0$ for some $\ve_0 > 0$ and all $k$. Yet, since this is a subsequence of $\left( f_n \right)_n$, it admits a subsequence converging to $f$, contradicting the assertion that $\norm{f_{n_k} - f}\geq \ve_0$ for all $k$. Therefore, we have that $\left( f_n \right)_n\rightarrow f$ uniformly on compacts.
\end{solution}
\end{document}
