\documentclass[10pt]{mypackage}

% sans serif font:
%\usepackage{cmbright}
%\usepackage{sfmath}
%\usepackage{bbold} %better blackboard bold

\usepackage{homework}
%\usepackage{notes}
\usepackage{newpxtext,eulerpx,eucal}
\renewcommand*{\mathbb}[1]{\varmathbb{#1}}

\fancyhf{}
\fancyhead[R]{Avinash Iyer}
\fancyhead[L]{Complex Analysis: Assignment 8}
\fancyfoot[C]{\thepage}

\setcounter{secnumdepth}{0}

\begin{document}
\RaggedRight
\begin{problem}[Problem 1]
  Determine and classify each of the singularities in $\C$ of the following functions:
  \begin{enumerate}[(a)]
    \item $\ds \frac{z+1}{\sin^2\left( \pi/z \right)}$;
    \item $\ds \frac{1}{z^2 - 1} \cos\left( \frac{\pi z}{z+1} \right)$;
    \item $\ds \cot\left( z \right) - \frac{1}{z}$.
  \end{enumerate}
\end{problem}
\begin{solution}\hfill
  \begin{enumerate}[(a)]
    \item We observe that
      \begin{align*}
        \lim_{z\rightarrow -1} \sin^2\left( \pi/z \right) &= 0,
      \end{align*}
      and
      \begin{align*}
        \lim_{z\rightarrow -1} \left( z+1 \right)\frac{z+1}{\sin^2\left( \pi/z \right)} &= \lim_{z\rightarrow -1} \frac{\left( z+1 \right)^2}{\sin^2\left( \pi/z \right)}\\
                                                                                        &= \lim_{w\rightarrow 0} \frac{w^2}{\sin^2\left( w \right)}\\
                                                                                        &= 1,
      \end{align*}
      meaning that $f(z) = \frac{z+1}{\sin^2\left( \pi/z \right)}$ has a pole of order $1$ at $z = -1$. Next, we observe that
      \begin{align*}
        f(z) &= \frac{z+1}{\sin^2\left( \pi/z \right)}\\
             &= \left( z+1 \right)\csc^2\left( \pi/z \right)
      \end{align*}
      has singularities at every other $ z =  \frac{1}{n} $ for all $n \in\Z$ with $n\neq 0,-1$; for any such satisfactory $z = \frac{1}{n}$, we have
      \begin{align*}
        \lim_{z\rightarrow \frac{1}{n}} \left( z - \frac{1}{n} \right)^2\csc^2\left( \pi/z \right) \left( z+1 \right) &= 1 + \frac{1}{n},
      \end{align*}
      meaning that at each $z = \frac{1}{n}$ with $n\in\Z$ and $n\neq 0,-1$, we have a pole of order $2$.\newline

      Finally, there is no isolated singularity at $0$ because $0$ is an accumulation point of the sequence $\left( \frac{1}{n} \right)_{n\geq 1}$.
    \item Simplifying, we have
      \begin{align*}
        \frac{1}{z^2 - 1} \cos\left( \frac{\pi z}{z+1} \right) &= \frac{1}{\left( z-1 \right)\left( z+1 \right)} \cos\left( \pi - \frac{\pi}{z+1} \right)\\
                                                               &= \frac{1}{\left( z-1 \right)\left( z+1 \right)} \left( -\cos\left( \frac{\pi}{z+1} \right) \right)\\
                                                               &= -\frac{1}{\left( z-1 \right)\left( z+1 \right)} \cos\left( \frac{\pi}{z+1} \right).
      \end{align*}
      We observe that
      \begin{align*}
        \lim_{z\rightarrow 1} \left( z-1 \right)f(z) &= \lim_{z\rightarrow 1}\frac{-\cos\left( \frac{\pi}{z+1} \right)}{\left( z+1 \right)}\\
                                                     &= 0,
      \end{align*}
      meaning that the singularity at $z = 1$ is removable. Additionally, we observe that the Laurent expansion about $-1$ for the function is
      \begin{align*}
        -\frac{1}{2}\left( \frac{1}{z-1} - \frac{1}{z+1} \right)\cos\left( \frac{\pi}{\left( z+1 \right)} \right) &= -\frac{1}{2} \left( \frac{1}{z-1} - \frac{1}{z+1} \right) \left( \sum_{k=0}^{\infty}\frac{\left( -1 \right)\pi^{2k+1}}{\left( 2k+1 \right)!}\frac{1}{\left( z+1 \right)^{2k+1}} \right),
      \end{align*}
      meaning that there are infinitely many negative-power terms in this Laurent expansion, so that the singularity at $-1$ is essential.
    \item We observe that
      \begin{align*}
        \lim_{z\rightarrow 0} z\left( \frac{\cos(z)}{\sin(z)} - \frac{1}{z} \right) &= \lim_{z\rightarrow 0} \frac{z\cos\left( z \right)}{\sin\left( z \right)} - 1\\
                                                                                    &= 0,
      \end{align*}
      meaning that the singularity at $0$ is removable. Additionally, we see that for any $n\in\Z$ with $n\neq 0$,
      \begin{align*}
        \lim_{z\rightarrow n\pi}\left( z-n\pi \right)\frac{\cos\left( z \right)}{\sin\left( z \right)} &= \left( -1 \right)^{n} \lim_{z\rightarrow n\pi} \frac{\left( z-n\pi \right)}{\sin\left( z \right)}\\
                                                                                                       &= \left( -1 \right)^{n},
      \end{align*}
      whence the function has poles of order $1$ at $n\pi$ when $n\neq 0$.
  \end{enumerate}
\end{solution}
\begin{problem}[Problem 2]
  Let $f\colon \C\rightarrow \C$ be entire.
  \begin{enumerate}[(a)]
    \item Suppose there is a bounded set $U\subseteq \C$ such that $f\left( \C\setminus U \right)\subseteq \C$ is not dense. Show that $f$ is a polynomial.
    \item Suppose that $f$ is injective. Show that $f(z) = az + b$ for some $a\in \C\setminus \set{0}$ and $b\in \C$.
  \end{enumerate}
\end{problem}
\begin{solution}\hfill
  \begin{enumerate}[(a)]
    \item Since $U\subseteq \C$ is bounded, there is some $R > 0$ such that $U\subseteq B\left( 0,R \right)$. In particular, this means that $f\left( \C\setminus B\left( 0,R \right) \right)\subseteq \C$ is not dense. Consider now the set
      \begin{align*}
        V &= \set{\frac{1}{z} | z\in \C\setminus B\left( 0,R \right)}\\
          &\subseteq \C\setminus \set{0}.
      \end{align*}
      This set is open as $\frac{1}{z}$ is holomorphic on the open set $\C\setminus B\left( 0,R \right)$. Furthermore, for $\ve > 0$, we can see that $\dot{U}\left( 0,\ve \right)\subseteq V$, as if $\ve < \frac{1}{R}$, then for
      \begin{align*}
        \dot{U}\left( 0,\ve \right) &= \set{re^{i\theta} | 0 < r < \ve,0 \leq \theta < 2\pi}
      \end{align*}
      we see that
      \begin{align*}
        \frac{1}{\dot{U}\left( 0,\ve \right)} &= \set{se^{-i\theta} | s > \frac{1}{\ve},-2\pi < \theta\leq 0 }\\
                                              &\subseteq \C\setminus B\left( 0,R \right).
      \end{align*}
      In particular, since $f\left( \C\setminus B\left( 0,R \right) \right)$ is not dense in $\C$, if we define
      \begin{align*}
        g(z) &= f\left( \frac{1}{z} \right),
      \end{align*}
      then we observe that
      \begin{align*}
        g\left( \dot{U}\left( 0,\ve \right) \right) &\subseteq \C
      \end{align*}
      is not dense, meaning that $0$ is not an essential singularity of $g$ by the contrapositive to the Casorati--Weierstrass Theorem. Thus, we may write $g$ in the form
      \begin{align*}
        g(z) &= \sum_{k=0}^{n} a_k z^{-k}
      \end{align*}
      for some $n \geq 0$. Thus, since $f(z) = g\left( \frac{1}{z} \right)$, it follows that
      \begin{align*}
        f(z) &= \sum_{k=0}^{n}a_kz^{k},
      \end{align*}
      or that $f$ is a polynomial (constant if $n = 0$).
    \item Let $f$ be an injective entire function. First, we note that $f$ cannot be a constant function by definition.\newline

      Since $f$ is injective, it follows that $f\left( \C\setminus B\left( 0,1 \right) \right)\cap f\left( U\left( 0,1 \right) \right) = \emptyset$, while since $\C\setminus B\left( 0,1 \right)$ and $U\left( 0,1 \right)$ are both open, and $f$ is nonconstant, it follows from the open mapping principle that there is some $z_0\in f\left( \C\setminus B\left( 0,1 \right) \right)$ and some $r > 0$ such that $U\left( z_0,r \right)\subseteq \C\setminus B\left( 0,1 \right)$ and $U\left( z_0,r \right)\cap f\left( U\left( 0,1 \right) \right) = \emptyset$. Thus, in particular, $f$ is a polynomial, as follows from part (a).\newline

      Finally, we observe that $f$ cannot have degree greater than $1$, since $f$ can only have one value map to zero, and any functions of the form
      \begin{align*}
        f(z) &= a_n\left( z-\alpha \right)^{n}
      \end{align*}
      would have $\alpha + 1$ and $\alpha + e^{2i\pi/n}$ map to the same value, once again violating injectivity. Thus, $f$ is a nonconstant polynomial with degree at most $1$, whence $f(z) = az + b$ for some $a\in \C\setminus \set{0}$ and $b\in\C$.
  \end{enumerate}
\end{solution}
\begin{problem}[Problem 3]
  We say a function $h\colon \hat{\C}\rightarrow \hat{\C}$ is holomorphic if the following hold:
  \begin{enumerate}[(i)]
    \item for every $z_0\in \C$ with $h\left( z_0 \right)\neq \infty$, there is $r > 0$ such that $h\left( U\left( z_0,r \right) \right)\subseteq \C$ and $h$ is holomorphic on $U\left( z_0,r \right)$;
    \item for every $z_0\in \C$ with $h\left( z_0 \right) = \infty$, there exists some $r > 0$ such that $ \widetilde{h}(z) = \frac{1}{h(z)} $ has $\widetilde{h}\left( \dot{U}\left( z_0,r \right) \right)\subseteq \C$, $ \widetilde{h} $ is holomorphic on $\dot{U}\left( z_0,r \right)$, and $z_0$ is removable for $ \widetilde{h} $;
    \item if $h\left( \infty \right) \neq \infty$, then there exists some $r > 0$ such that $ \widetilde{h}(z) = h\left( \frac{1}{z} \right) $ has $\widetilde{h}\left( \dot{U}\left( 0,r \right) \right)\subseteq \C$, $\widetilde{h}$ is holomorphic on $\dot{U}\left( 0,r \right)$, and $0$ is removable for $\widetilde{h}$;
    \item if $h\left( \infty \right) = \infty$, then there exists some $r > 0$ such that $\widetilde{h} = \frac{1}{h\left( \frac{1}{z} \right)}$ is such that $ \widetilde{h}\left( \dot{U}\left( 0,r \right) \right)\subseteq \C $, $\widetilde{h}$ is holomorphic on $\dot{U}\left( 0,r \right)$, and $0$ is removable for $\widetilde{h}$.
  \end{enumerate}
  Show that if $h\colon \hat{\C}\rightarrow \hat{\C}$ is injective and holomorphic, then $h$ is a linear fractional transformation.
\end{problem}
\begin{solution}
  Using this definition, we claim that a function $h\colon \hat{\C}\rightarrow \hat{\C}$ is meromorphic when restricted to $\C$, and that if $h\left( \infty \right) = \infty$, then $h$ is meromorphic on $\C$ with a pole at $\infty$.\newline

  To start, if $z_0\in \C$ is such that $h\left( z_0 \right) \neq \infty$, then by condition, $h$ there is some $r > 0$ such that $h$ is holomorphic on $U\left( z_0,r \right)$. Now, if $z_0\in \C$ is such that $h\left( z_0 \right) = \infty$, then we observe that, on $\dot{U}\left( z_0,r \right)$, we have the function
  \begin{align*}
    \widetilde{h}(z) &= \frac{1}{h(z)}
  \end{align*}
  is such that
  \begin{align*}
    \lim_{z\rightarrow z_0} \left( z-z_0 \right)\widetilde{h}(z) &= 0,
  \end{align*}
  and also that
  \begin{align*}
    \widetilde{h}\left( z_0 \right) &= \frac{1}{h\left( z_0 \right)}\\
                                    &= 0,
  \end{align*}
  whence $\widetilde{h}$ has a holomorphic extension $g$ to $\dot{U}\left( z_0,r \right)$ with $g\left( z_0 \right) = 0$. Thus, on $\dot{U}\left( z_0,r \right)$, we have
  \begin{align*}
    \widetilde{h}(z) &= \left( z-z_0 \right)^{k} \widetilde{g}(z)
  \end{align*}
  with $\widetilde{g}\left( z_0 \right)\neq 0$, as all zeros of $ \widetilde{h} $ are isolated. Thus, from real analysis, there is some $0 < s < r$ such that $ \widetilde{g}\neq 0 $ on $U\left( z_0,s \right)$, meaning that on $\dot{U}\left( z_0,s \right)$, we have
  \begin{align*}
    h(z) &= \left( z-z_0 \right)^{-k} \frac{1}{\widetilde{g}(z)}, 
  \end{align*}
  whence $h$ has a pole of order $k$ at $z_0$. In particular, this means that the non-removable singularities of $h$ are exclusively poles, so $h$ is meromorphic. Additionally, this also means that a meromorphic function is holomorphic on $\hat{\C}$ upon defining the function to equal $\infty$ at its poles.\newline

  Now, if $h$ is such that $h\left( \infty \right) = \infty$, then we observe from the definition that the function
  \begin{align*}
    \widetilde{h}\left( z \right) &= h\left( \frac{1}{z} \right)
  \end{align*}
  has $\widetilde{h}\left( 0 \right) = \infty$, meaning that $\widetilde{h}$ has a pole at $0$ from earlier, so that $h\left( \infty \right) = \infty$ implies that $h$ has a pole at $\infty$.\newline

  Finally, if $h\left( \infty \right)\neq \infty$, we start by showing that either there is some $z_0$ with $h\left( z_0 \right) = \infty$ or $h$ is constant. If there is no such $z_0$, then $h\colon \hat{\C}\rightarrow \C$ is holomorphic and bounded (following from the extreme value theorem, as $\hat{\C}$ is compact), meaning that $h|_{\C}\colon \C\rightarrow \C$ is holomorphic and bounded, so that $h$ is constant. Thus, there is some $z_0$ such that $h\left( z_0 \right) = \infty$, whence $h$ is meromorphic when restricted to $\C$.\newline

  Now, suppose $h$ is an injective holomorphic function on $\hat{\C}$. Suppose $h\left( \infty \right) = \infty$. Then, for all $z_0\in \C$, it follows that $h\left( z_0 \right)\neq \infty$, whence $h$ is holomorphic on $\C$ by condition (i). We have already shown thus far that $h\left( \infty \right) = \infty$ implies that $h$ has a pole at $\infty$, meaning that $h$ is an entire function that has a pole at infinity, hence $h$ is a polynomial. Finally, since $h$ is injective, we know from Problem 2 (b) that this means $h\left( z \right) = az + b$, hence $h$ is a fractional linear transformation.\newline

  Now, if $h$ is an injective holomorphic function on $\hat{\C}$ that does not have $h\left( \infty \right) = \infty$, then $h\left( \infty \right) = k$ for some $k\in \C$, and $h$ is still a meromorphic function as seen above. Then, we observe that
  \begin{align*}
    p(z) &= \frac{1}{z-k}
  \end{align*}
  is meromorphic on $\C$, hence
  \begin{align*}
    \left( p\circ h \right)(z) &= \frac{1}{h(z) - k}
  \end{align*}
  is also meromorphic on $\C$ (hence holomorphic on $\hat{\C}$), injective, and has $\left( p\circ h \right)\left( \infty \right) = \infty$, so that $\left( p\circ h \right)\left( z \right) = az + b$, whence
  \begin{align*}
    \frac{1}{h\left( z \right)-k} &= az + b\\
    h\left( z \right) &= \frac{1}{az + b} + k,
  \end{align*}
  which is yet again a fractional linear transformation.
\end{solution}
\begin{problem}[Problem 4]
  Let $P\colon \C\rightarrow \C$ be a polynomial not uniformly zero.
  \begin{enumerate}[(a)]
    \item Show that $\sum_{n=0}^{\infty}P(n)z^{n}$ has radius of convergence exactly $1$.
    \item Show that if $f = \sum_{n=0}^{\infty}P(n)z^{n}$ then there exists a meromorphic function $g\colon \C\rightarrow \C$ that is rational and satisfies $g(z) = f(z)$ on $\D$.
    \item Show that at least one of the poles of $g$ satisfies $\left\vert z_0 \right\vert = 1$.
  \end{enumerate}
\end{problem}
\begin{solution}\hfill
  \begin{enumerate}[(a)]
    \item We write $P(z) = a_kz^{k} + \cdots + a_1 z + a_0$ where $ k > 1 $ and not all terms zero. Then, if $P$ is constant, we observe that
      \begin{align*}
        \sum_{n=0}^{\infty}a_0z^{n} &= a_0\sum_{n=1}^{\infty}z^{n},
      \end{align*}
      which has a radius of convergence of exactly $1$ as it is a geometric series. Therefore, if $P$ is nonconstant, we have
      \begin{align*}
        P(n) &= n^{k}\left( a_k + a_{k-1}n^{-1} + \cdots + a_1 n^{-k+1} + a_0n^{-k} \right),
      \end{align*}
      so that
      \begin{align*}
        \frac{1}{R} &= \limsup_{n\rightarrow\infty} \left( P(n) \right)^{1/n}\\
                    &= \limsup_{n\rightarrow\infty} \left( n^{k} \right)^{1/n}\left( a_k + a_{k-1}n^{-1} + \cdots + a_1n^{-k-1} + a_0n^{-k} \right)^{1/n}\\
                    &= \limsup_{n\rightarrow\infty} \left( n^{1/n} \right)^{k}\left( a_k \right)^{1/n}\\
                    &= 1.
      \end{align*}
      Therefore, the radius of convergence of the power series $\sum_{n=0}^{\infty}P(n)z^{n}$ is exactly $1$.
    \item From uniform convergence, we observe that on $\D$, whenever $k > 0$, we have
      \begin{align*}
        \sum_{n=1}^{\infty}n^{k}z^{n} &= \frac{d}{dz}\left( \sum_{n=1}^{\infty} \frac{n^{k}}{n+1} z^{n+1} \right)\\
                                      &= \diff{}{z}\left( \sum_{n=1}^{\infty} n^{k-1}z^{n+1} \right) - \diff{}{z}\left( \sum_{n=1}^{\infty} \frac{n^{k-1}}{n+1}z^{n+1} \right)\\
                                      &= \diff{}{z}\left( z\sum_{n=1}^{\infty}n^{k-1}z^{n} \right) - \sum_{n=1}^{\infty} n^{k-1}z^{n}\\
                                      &= z\diff{}{z}\left( \sum_{n=1}^{\infty} n^{k-1}z^{n} \right).
      \end{align*}
      Therefore, if we write
      \begin{align*}
        q_k(z) &= \sum_{n=1}^{\infty} n^{k}z^{n},
      \end{align*}
      we observe that we have the recurrence relation
      \begin{align*}
        q_k &= zq_{k-1}',
        \intertext{where}
        q_0(z) &= \sum_{n=1}^{\infty}z^{n}\\
               &= \sum_{n=0}^{\infty}z^{n+1}\\
               &= \frac{z}{1-z},
      \end{align*}
      so by solving this recurrence relation for each of the $k > 0$ terms in
      \begin{align*}
        f(z) &= \sum_{n=0}^{\infty}\left( a_kn^{k} + \cdots + a_1n + a_0 \right)z^{n}\\
             &= a_k\sum_{n=1}^{\infty}n^{k}z^{n} + \cdots + a_1\sum_{n=1}^{\infty}nz^{n} + a_0\sum_{n=0}^{\infty}z^{n},
      \end{align*}
      and using the identity 
      \begin{align*}
        \sum_{n=0}^{\infty} z^{n} = \frac{1}{1-z}
      \end{align*}
      on the term with $a_0$, we observe that this gives us an expression for $f$ entirely in terms of rational functions, whence $f$ has a meromorphic extension to $\C$.
    \item Suppose toward contradiction that for all poles $z_0$ of $g$, it were the case that $\left\vert z_0 \right\vert > 1$. Considering the Taylor expansion of $g$ about $0$, we would have
      \begin{align*}
        g(z) &= \sum_{n=0}^{\infty}a_nz^n,
      \end{align*}
      on $U\left( 0,r \right)$ with $r > 1$, as the Cauchy Integral Formula provides. Yet, since this new Taylor expansion agrees with $f(z) = \sum_{n=0}^{\infty}P(n)z^{n}$ on $\D$, it follows that these two expansions are equal on $U\left( 0,r \right)$ by the identity theorem. Yet, we have already shown that $f$ has radius of convergence exactly $1$, which contradicts the assertion that there is any pole of $g$ with $\left\vert z_0 \right\vert > 1$.\newline

      Alternatively, we can use the construction from part (b) to show that $g$ has exactly one pole at $1$. Toward this end, observe that 
      \begin{align*}
        \diff{}{z}\left( \frac{p(z)}{\left( 1-z \right)^{k}} \right) &= \frac{p'(z)}{\left( 1-z \right)^{k}} + \frac{kp(z)}{\left( 1-z \right)^{k+1}}.
      \end{align*}
      In particular, taking the derivative of a rational function whose denominator is some power of $\left( 1-z \right)^{k}$ does not alter the fact that this rational function has a pole at $1$, and neither does multiplying this rational function by a polynomial that has no factors in common with $\left( 1-z \right)^{k}$. Since we constructed $g$ by taking a series of linear combinations of derivatives $\frac{1}{1-z}$ and multiplications by $z$, it follows that $g$ has a pole exclusively at $1$.
  \end{enumerate}
\end{solution}
\begin{problem}[Problem 5]
  For $m\in\N$, evaluate the integral
  \begin{align*}
    \oint_{S^{1}}^{} \frac{z^{m-1}}{2z^{m}-1}\:dz.
  \end{align*}
\end{problem}
\begin{solution}
  We will use the argument principle to evaluate this integral. Toward this end, we see that
  \begin{align*}
    \oint_{S^{1}}^{} \frac{z^{m-1}}{2z^{m}-1}\:dz &= \frac{1}{2m} \oint_{S^{1}}^{} \frac{mz^{m-1}}{z^{m}-\left( 1/2 \right)}\:dz\\
                                                  &= \frac{1}{2m} \int_{S^{1}}^{} \frac{f'(z)}{f(z)}\:dz\\
                                                  &= \frac{1}{2m} \left( 2\pi i \right)\sum_{z_0} n\left( S^{1};z_0 \right)\operatorname{ord}_{z_0}\left( f \right)\\
                                                  &= \frac{\pi i}{m} \left( m \right)\\
                                                  &= \pi i.
  \end{align*}
\end{solution}
\end{document}
