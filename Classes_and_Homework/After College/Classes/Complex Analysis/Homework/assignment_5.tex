\documentclass[10pt]{mypackage}

% sans serif font:
%\usepackage{cmbright}
%\usepackage{sfmath}
%\usepackage{bbold} %better blackboard bold

\usepackage{homework}
%\usepackage{notes}
\usepackage{newpxtext,eulerpx,eucal}
\renewcommand*{\mathbb}[1]{\varmathbb{#1}}

\fancyhf{}
\rhead{Avinash Iyer}
\lhead{Complex Analysis: Assignment 5}

\setcounter{secnumdepth}{0}

\begin{document}
\RaggedRight
\begin{problem}[Problem 1]
  Let $U\subseteq \C$ be a bounded region, $f\colon \overline{U}\rightarrow \C$ continuous such that $ f|_{U} $ is holomorphic. Suppose $f$ is nonvanishing in $U$, and that there exists $c > 0$ such that $ \left\vert f(z) \right\vert = c $ for all $z\in \partial U$. Prove that there exists some $\theta\in \R$ such that $f(z) = ce^{i\theta}$ for all $z\in \overline{U}$.
\end{problem}
\begin{solution}
  Since $f$ is holomorphic on the connected, bounded, open set $U$, it follows from the maximum modulus principle that for all $z\in U$, we have $\left\vert f(z) \right\vert \leq \left\vert f\left( w \right) \right\vert$ for all $w\in \partial U$. In particular, we must have $\left\vert f(z) \right\vert \leq c$ for all $z\in U$. Since $\left\vert f(z) \right\vert\neq 0$ for all $z\in U$, it follows that $ \frac{1}{\left\vert f(z) \right\vert} \geq \frac{1}{c} $ for all $z\in U$. Yet, at the same time, since $ \frac{1}{f(z)} $ is holomorphic, we must have $\frac{1}{\left\vert f(z) \right\vert} \leq \frac{1}{\left\vert f(w) \right\vert}$ for all $w\in \partial U$, meaning that $ \frac{1}{\left\vert f(z) \right\vert} \leq \frac{1}{c} $, so that $ \left\vert \frac{1}{f} \right\vert = \frac{1}{c} $, or that $ \left\vert f(z) \right\vert = c $ for all $z\in U$.\newline

  In particular, for all $z\in U$, we have $ \left\vert f(z) \right\vert \geq \left\vert f(w) \right\vert $ for all $z\in U$, the maximum modulus principle gives that $f$ is constant. Since $ \left\vert f(z) \right\vert = c $, we thus have $f(z) = ce^{i\theta}$ for some $\theta\in \R$.
\end{solution}
\begin{problem}[Problem 2]
  For $0 < r < R$, let $A\left( z_0,r,R \right) = \set{z\in \C | r < \left\vert z-z_0 \right\vert < R}$. Suppose that there exists a continuous $f\colon \overline{A\left( z_0,r,R \right)}\rightarrow \C$ such that $f|_{ A\left( z_0,r,R \right) }$ is holomorphic, and that there exist constants $C_{r}$ and $C_{R}$ in $\R$ such that $\re\left( f(z) \right) = C_{r}$ on $S\left( z_0,r \right)$, and $\re\left( f(z) \right) = C_R$ on $S\left( z_0,R \right)$> Show that $C_r = C_R$, and that $f$ is constant for all $z\in \overline{A\left( z_0,r,R \right)}$.
\end{problem}
\begin{solution}
  Without loss of generality, since we may take $g(z) = f\left( z-z_0 \right)$, we may assume that $z_0 = 0$, so that we let $u\left( x,y \right)\colon \overline{A\left( 0,r,R \right)}\rightarrow \R$ be given by $ u\left( x,y \right) = \re\left( f\left( x-x_0 + i\left( y-y_0 \right) \right) \right) $. Since $u$ is the real part of a holomorphic function, $u$ is necessarily harmonic, so by the extended maximum modulus principle, $u$ takes on its maximum on either $S\left( 0,r \right)$ or $S\left( 0,R \right)$. In other words, it is the case that the maximum for $u$ is either $C_r$ or $C_R$.\newline

  Now, consider the function
  \begin{align*}
    w\left( x,y \right) &= u\left( x,y \right) - C_r - \left( C_R - C_r \right)\frac{\ln\left( x^2 + y^2 \right) - \ln\left( r^2 \right)}{\ln\left( R^2 \right) - \ln\left( r^2 \right)}.
  \end{align*}
  We start by verifying that $w$ is harmonic. Towards this end, since Laplace's equation is linear, we only need to evaluate the expression of $\ln$, as we already know that $u$ satisfies Laplace's equation. This gives
  \begin{align*}
    \pd{w}{x} &= -\frac{C_R - C_r}{\ln\left( R^2 \right) - \ln\left( r^2 \right)}\frac{2x}{x^2 + y^2}\\
    \pd{^2w}{x^2} &= -\frac{C_R - C_r}{\ln\left( R^2 \right) - \ln\left( r^2 \right)} \left( \frac{2}{x^2 + y^2} - 2x\left( \frac{2x}{\left( x^2 + y^2 \right)^2} \right) \right)\\
                  &= -\frac{C_R - C_r}{\ln\left( R^2 \right) - \ln\left( r^2 \right)} \frac{2y^2 - 2x^2}{\left( x^2 + y^2 \right)^2}\\
    \pd{^2w}{y^2} &= -\frac{C_R - C_r}{\ln\left( R^2 \right) - \ln\left( r^2 \right)} \frac{2x^2 - 2y^2}{\left( x^2 + y^2 \right)^2},
  \end{align*}
  which means that the sum is zero, and thus $w$ is harmonic. In particular, it also satisfies the extended maximum modulus principle, meaning that $w$ attains its maxima and minima on the boundary of the annulus. Yet, since $w$ equals $0$ on both the outer circle and inner circle of the annulus, it follows that $w$ is identically zero.\newline

  Thus, we have
  \begin{align*}
    u\left( x,y \right) &= C_r + \left( C_R - C_r \right)\frac{\ln\left( x^2 + y^2 \right) - \ln\left( r^2 \right)}{\ln\left( R^2 \right) - \ln\left( r^2 \right)}.
  \end{align*}
  Yet, this implies that
  \begin{align*}
    \re\left( f(z) \right) &= C_r + \frac{C_R - C_r}{\ln\left( R^2 \right) - \ln\left( r^2 \right)} \left( \ln\left( \left\vert z \right\vert^2 \right) - \ln\left( r^2 \right) \right).
  \end{align*}
  Since $f$ is holomorphic, we must have
  \begin{align*}
    0 &= \pd{f}{ \overline{z} }\\
      &= \pd{\re(f)}{ \overline{z} } + i \pd{ \im(f) }{ \overline{z} }\\
      &= \frac{C_R - C_r}{ \ln\left( R^2 \right) - \ln\left( r^2 \right) } \left( \frac{z}{\left\vert z \right\vert} \right)^2 + i \pd{\im(f)}{ \overline{z} }
  \end{align*}
  for all $z\in A\left( 0,r,R \right)$. In particular, this must also hold for $ z = \re(z) $, so that
  \begin{align*}
    0 &= \frac{C_R - C_r}{\ln\left( R^2 \right) - \ln\left( r^2 \right)} + i \pd{\im(f)}{ \overline{z} }.
  \end{align*}
  Now, since $ \overline{z} = \re\left( z \right) $, it follows that
  \begin{align*}
    0 &= \frac{C_R - C_r}{\ln\left( R^2 \right) - \ln\left( r^2 \right)} + i \pd{v}{ x },
  \end{align*}
  where $ f\left( x + iy \right) = u\left( x,y \right) + i v\left( x,y \right) $. Yet, since the first term in this equation is purely real, and $ i \pd{v}{x} $ is purely imaginary, it follows that both terms must be equal to zero, so that $C_R = C_r$.\newline

  This means we may take $u\left( x,y \right) = C$ for some $C$ such that $f(z) = C + i v\left( x,y \right)$. Thus, by Cauchy--Riemann, we must have
  \begin{align*}
    \pd{v}{x} &= 0\\
    \pd{v}{y} &= 0,
  \end{align*}
  so that $v\left( x,y \right)$ is constant, and thus $f$ is constant.
\end{solution}
\begin{problem}[Problem 3]
  Let $f\colon \C\rightarrow \C$ be an entire function such that
  \begin{align*}
    \sup_{M_1,M_2 \geq 0} \int_{-M_2}^{M_2} \int_{-M_1}^{M_1} \left\vert f\left( x + iy \right) \right\vert\:dx\:dy
  \end{align*}
  is finite. Show that $f(z) = 0$ for all $z\in \C$.
\end{problem}
\begin{solution}
  Letting $\left( x_0,y_0 \right)\in \R^{2}$, we observe that for any $r > 0$, we have
  \begin{align*}
    \left\vert f(x_0 + iy_0) \right\vert &\leq \frac{1}{2\pi} \int_{0}^{2\pi} \left\vert f\left( x_0 + r\cos\left( \theta \right),y_0 + r\sin\left( \theta \right) \right) \right\vert\:d\theta\\
                                         &= \frac{1}{2\pi r} \int_{0}^{r} \int_{0}^{2\pi} \left\vert f\left( x_0 + r\cos\left( \theta \right),y_0 + r\sin\left( \theta \right) \right) \right\vert\:d\theta\:dr.
  \end{align*}
  We observe that there is a closed square containing the closed disk $B\left( z_0,r \right)$ given by the set of all $z\in \C$ such that $ \left\vert \re(z) - \re\left( z_0 \right) \right\vert \leq r $ and $ \left\vert \im\left( z \right) - \im\left( z_0 \right) \right\vert \leq r $. Since the double integral is evaluating over a positive function, the integral over this square is larger than the integral over the corresponding disk, so that we have
  \begin{align*}
    \frac{1}{2\pi r} \int_{0}^{r} \int_{0}^{2\pi} \left\vert f\left( x_0 + r\cos\left( \theta \right),y_0 + r\sin\left( \theta \right) \right) \right\vert\:d\theta\:dr &\leq \frac{1}{2\pi r} \int_{y_0 - r}^{y_0 + r} \int_{x_0 - r}^{x_0 + r} \left\vert f\left( x,y \right) \right\vert\:dx\:dy\\
                                                                                                                                                                                   &\leq \frac{1}{2\pi r} \sup_{M_1,M_2\geq 0} \int_{-M_2}^{M_2} \int_{-M_1}^{M_1} \left\vert f\left( x,y \right) \right\vert\:dx\:dy.
  \end{align*}
  Since the quantity in the supremum is finite, $f$ is entire, and $r$ was arbitrary, it follows that we may take the limit as $r\rightarrow \infty$, so that $f\left( x_0 + iy_0 \right) = 0$. Since $x_0$ and $y_0$ are arbitrary, this thus holds for all $z\in \C$, so $f \equiv 0$.
\end{solution}
\begin{problem}[Problem 4]
  Let $U\subseteq \C$ be a region, and let $f\colon U\rightarrow \C$ be a holomorphic function. Show that if $ u\left( x,y \right) = \left\vert f\left( x + iy \right) \right\vert $ is a harmonic function, then $f$ is constant.
\end{problem}
\begin{solution}
  We know that $u\left( x,y \right)$ is a subharmonic function as it is the modulus of a holomorphic function, so that if $z_0 = x_0 + iy_0\in U$ and $B\left( z_0,r \right)\subseteq U$, then
  \begin{align*}
    u\left( x_0,y_0 \right) &\leq \frac{1}{2\pi}\int_{0}^{2\pi} u\left( x_0 + r\cos t, y_0 + r\sin t \right)\:dt.
  \end{align*}
\end{solution}
\end{document}
