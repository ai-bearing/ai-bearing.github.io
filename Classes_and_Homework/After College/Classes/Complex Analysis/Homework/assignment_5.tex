\documentclass[10pt]{mypackage}

% sans serif font:
%\usepackage{cmbright}
%\usepackage{sfmath}
%\usepackage{bbold} %better blackboard bold

\usepackage{homework}
%\usepackage{notes}
\usepackage{newpxtext,eulerpx,eucal}
\renewcommand*{\mathbb}[1]{\varmathbb{#1}}

\fancyhf{}
\rhead{Avinash Iyer}
\lhead{Complex Analysis: Assignment 5}

\setcounter{secnumdepth}{0}

\begin{document}
\RaggedRight
\begin{problem}[Problem 1]
  Let $U\subseteq \C$ be a bounded region, $f\colon \overline{U}\rightarrow \C$ continuous such that $ f|_{U} $ is holomorphic. Suppose $f$ is nonvanishing in $U$, and that there exists $c > 0$ such that $ \left\vert f(z) \right\vert = c $ for all $z\in \partial U$. Prove that there exists some $\theta\in \R$ such that $f(z) = ce^{i\theta}$ for all $z\in \overline{U}$.
\end{problem}
\begin{solution}
  Since $f$ is holomorphic on the connected, bounded, open set $U$, it follows from the maximum modulus principle that for all $z\in U$, we have $\left\vert f(z) \right\vert \leq \left\vert f\left( w \right) \right\vert$ for all $w\in \partial U$. In particular, we must have $\left\vert f(z) \right\vert \leq c$ for all $z\in U$. Since $\left\vert f(z) \right\vert\neq 0$ for all $z\in U$, it follows that $ \frac{1}{\left\vert f(z) \right\vert} \geq \frac{1}{c} $ for all $z\in U$. Yet, at the same time, since $ \frac{1}{f(z)} $ is holomorphic, we must have $\frac{1}{\left\vert f(z) \right\vert} \leq \frac{1}{\left\vert f(w) \right\vert}$ for all $w\in \partial U$, meaning that $ \frac{1}{\left\vert f(z) \right\vert} \leq \frac{1}{c} $, so that $ \left\vert \frac{1}{f} \right\vert = \frac{1}{c} $, or that $ \left\vert f(z) \right\vert = c $ for all $z\in U$.\newline

  In particular, for all $z\in U$, we have $ \left\vert f(z) \right\vert \geq \left\vert f(w) \right\vert $ for all $z\in U$, the maximum modulus principle gives that $f$ is constant. Since $ \left\vert f(z) \right\vert = c $, we thus have $f(z) = ce^{i\theta}$ for some $\theta\in \R$.
\end{solution}
\begin{problem}[Problem 2]
  For $0 < r < R$, let $A\left( z_0,r,R \right) = \set{z\in \C | r < \left\vert z-z_0 \right\vert < R}$. Suppose that there exists a continuous $f\colon \overline{A\left( z_0,r,R \right)}\rightarrow \C$ such that $f|_{ A\left( z_0,r,R \right) }$ is holomorphic, and that there exist constants $C_{r}$ and $C_{R}$ in $\R$ such that $\re\left( f(z) \right) = C_{r}$ on $S\left( z_0,r \right)$, and $\re\left( f(z) \right) = C_R$ on $S\left( z_0,R \right)$> Show that $C_r = C_R$, and that $f$ is constant for all $z\in \overline{A\left( z_0,r,R \right)}$.
\end{problem}
\begin{solution}
  Without loss of generality, since we may take $g(z) = f\left( z-z_0 \right)$, we may assume that $z_0 = 0$, so that we let $u\left( x,y \right)\colon \overline{A\left( 0,r,R \right)}\rightarrow \R$ be given by $ u\left( x,y \right) = \re\left( f\left( x-x_0 + i\left( y-y_0 \right) \right) \right) $. Since $u$ is the real part of a holomorphic function, $u$ is necessarily harmonic, so by the extended maximum modulus principle, $u$ takes on its maximum modulus on either $S\left( 0,r \right)$ or $S\left( 0,R \right)$. In other words, it is the case that the maximum modulus for $u$ is either $\left\vert C_r \right\vert$ or $\left\vert C_R \right\vert$.\newline

  Now, consider the function
  \begin{align*}
    w\left( x,y \right) &= u\left( x,y \right) - C_r - \left( C_R - C_r \right)\frac{\ln\left( x^2 + y^2 \right) - \ln\left( r^2 \right)}{\ln\left( R^2 \right) - \ln\left( r^2 \right)}.
  \end{align*}
  We start by verifying that $w$ is harmonic. Towards this end, since Laplace's equation is linear, we only need to evaluate the expression in the fraction free of the constants.
  \begin{align*}
    \pd{w}{x} &= \frac{C_R - C_r}{\ln\left( R^2 \right) - \ln\left( r^2 \right)}\frac{2x}{x^2 + y^2}\\
    \pd{^2w}{x^2} &= \frac{C_R - C_r}{\ln\left( R^2 \right) - \ln\left( r^2 \right)} \left( \frac{2}{x^2 + y^2} - 2x\left( \frac{2x}{\left( x^2 + y^2 \right)^2} \right) \right)\\
                  &= \frac{C_R - C_r}{\ln\left( R^2 \right) - \ln\left( r^2 \right)} \frac{2y^2 - 2x^2}{x^2 + y^2}\\
    \pd{^2w}{y^2} &= \frac{C_R - C_r}{\ln\left( R^2 \right) - \ln\left( r^2 \right)} \frac{2x^2 - 2y^2}{x^2 + y^2},
  \end{align*}
  which means that the sum is zero, and thus $w$ is harmonic. In particular, it also satisfies the extended maximum modulus principle, meaning that $w$ attains its maxima on the boundary of the annulus. Since, for $x + iy \in S\left( 0,r \right)$, we have $u\left( x,y \right) = C_r$ and $x^2 + y^2 = r^2$, we thus get that $w = 0$, and similarly, $w = 0$ on $S\left( 0,R \right)$, meaning that $w$ is identically zero on $ \overline{A\left( 0,r,R \right)} $.\newline

  Thus, we find that
  \begin{align*}
    u\left( x,y \right) &= C_r + \left( C_R - C_r \right) \frac{\ln\left( x^2 + y^2 \right) - \ln\left( r^2 \right)}{\ln\left( R^2 \right) - \ln\left( r^2 \right)}.
  \end{align*}
\end{solution}
\end{document}
