\documentclass[10pt]{mypackage}

% sans serif font:
%\usepackage{cmbright}
%\usepackage{sfmath}
%\usepackage{bbold} %better blackboard bold

\usepackage{homework}
%\usepackage{notes}
\usepackage{newpxtext,eulerpx,eucal}
\renewcommand*{\mathbb}[1]{\varmathbb{#1}}

\fancyhf{}
\rhead{Avinash Iyer}
\lhead{Complex Analysis: Assignment 2}

\setcounter{secnumdepth}{0}

\begin{document}
\RaggedRight
\begin{problem}[Problem 1]\hfill
  \begin{enumerate}[(a)]
    \item Show that the power series $\sum_{n=0}^{\infty}\frac{z^{n}}{n!}$ converges for all $z\in \C$, in which it defines an analytic function, which we denote $e^{z}$.
    \item With this as the definition of $e^{z}$, prove that $e^{z}e^{w} = e^{z+w}$.
    \item Show that for $\theta\in \R$, we have that $e^{i\theta} = \cos\left( \theta \right) + i\sin\left( \theta \right)$, where $\cos\left( \theta \right)$ and $\sin\left( \theta \right)$ are defined via their usual power series representations.
  \end{enumerate}
\end{problem}
\begin{solution}\hfill
  \begin{enumerate}[(a)]
    \item To compute
      \begin{align*}
        \rho &= \limsup_{n\rightarrow\infty} \left( \frac{1}{n!} \right)^{1/n},
      \end{align*}
      we start by noticing that
      \begin{align*}
        \lim_{n\rightarrow\infty} \frac{\frac{1}{\left( n+1 \right)!}}{\frac{1}{n!}} &= \lim_{n\rightarrow\infty} \frac{1}{\left( n+1 \right)}\\
                                                                                     &= 0.
      \end{align*}
      In particular, for $\ve > 0$, there is some $N$ such that for all $n\geq N$,
      \begin{align*}
        \frac{\frac{1}{\left( n+1 \right)!}}{\frac{1}{n!}} < \ve,
      \end{align*}
      so
      \begin{align*}
        \frac{1}{\left( n+1 \right)!} &< \frac{\ve}{n!},
      \end{align*}
      and by inductively using this approximation, we get that for any $n\geq N$,
      \begin{align*}
        \frac{1}{n!} &<  \frac{\ve^{n-N}}{N!}\\
                     &= \ve^{n} \left( \frac{1}{\ve^{N}N!} \right)
      \end{align*}
      so that
      \begin{align*}
        \limsup_{n\rightarrow\infty} \left( \frac{1}{n!} \right)^{1/n} &\leq \ve,
      \end{align*}
      meaning that $\rho = 0$, and thus the radius of convergence for the power series is infinite.
    \item Computing $e^{z}e^{w}$, we get
      \begin{align*}
        \left( \sum_{k=0}^{\infty}\frac{z^{k}}{k!} \right)\left( \sum_{\ell = 0}^{\infty}\frac{w^{k}}{k!} \right) &= \sum_{k=0}^{\infty}\sum_{\ell = k}^{\infty} \frac{1}{\left( \ell - k \right)!}\frac{1}{k!}w^{k}z^{\ell - k}\\
                                                                                                                  &= \sum_{\ell = 0}^{\infty} \frac{1}{\ell!} \sum_{k=0}^{\ell} \frac{1}{k!\left( \ell-k \right)!} w^{k}z^{\ell - k}\\
                                                                                                                  &= \sum_{\ell = 0}^{\infty} \frac{1}{\ell!} \left( z+w \right)^{\ell}\\
                                                                                                                  &= e^{z+w}.
      \end{align*}
    \item Computing $e^{i\theta}$ by direct substitution, we find that
      \begin{align*}
        e^{i\theta} &= \sum_{k=0}^{\infty} \frac{\left( i\theta \right)^{k}}{k!}\\
                    &= \sum_{k\text{ even}} \frac{\left( -1 \right)^{(k/2)}\theta^{k}}{k!} + i\sum_{k\text{ odd}} \frac{\left( -1 \right)^{\left( k-1 \right)/2}\theta^{k}}{k!}\\
                    &= \cos\left( \theta \right) + i\sin\left( \theta \right).
      \end{align*}
  \end{enumerate}
\end{solution}
\begin{problem}[Problem 2]
  Let $U\subseteq \C$ be an open set, $f\colon U\rightarrow \C$ an analytic function. Since $f$ is analytic, given $z_0\in U$, there is $r > 0$ and a sequence $\left( a_n \right)_n$ such that $f(z) = \sum_{n=0}^{\infty}a_n\left( z-z_0 \right)^{n}$ for all $z\in U\left( z_0,r \right)$.\newline

  Suppose there exists $R > r$ such that $U\left( z_0,R \right) \subseteq U$ and $\sum_{n=0}^{\infty}a_n\left( z-z_0 \right)^{n}$ has radius of convergence at least $R$. Show that $f(z) = \sum_{n=0}^{\infty}a_n\left( z-z_0 \right)^{n}$ for all $z\in U\left( z_0,R \right)$.
\end{problem}
\begin{solution}
  On the connected open set $V = U\left( z_0,R \right)$, define
  \begin{align*}
    g(z) &= \sum_{n=0}^{\infty}a_n\left( z-z_0 \right)^{n}.
  \end{align*}
  Observe that $f|_{V}$ and $g$ agree on the open subset $U\left( z_0,r \right)\subseteq U\left( z_0,R \right)$. By the identity theorem, this means that $f = g$ on $U\left( z_0,R \right)$.
\end{solution}
\begin{problem}[Problem 3]
  Let $U\subseteq \C$ be a region, and let $f\colon U\rightarrow \C$ be an analytic function.
  \begin{enumerate}[(a)]
    \item Suppose $f$ is nonconstant, $z_0\in U$. Show that there exists some $r > 0$ for which $U\left( z_0,r \right)\subseteq U$, a positive integer $k\in \N$, an analytic function $g\colon U\left( z_0,r \right)\rightarrow \C$, and a nonconstant $\lambda\in \C\setminus \set{0}$ such that for $z\in U\left( z_0,r \right)$,
      \begin{align*}
        f(z) &= f\left(z_0\right) + \lambda\left( z-z_0 \right)^{k} + \left( z-z_0 \right)^{k+1} g(z).
      \end{align*}
    \item Suppose that $f$ is nonconstant, and $z_0\in U$ is such that $f\left( z_0 \right) \neq 0$. Show that there exists some $s > 0$ such that $U\left( z_0,s \right)\subseteq U$, and $w_1,w_2\in U\left( z_0,s \right)$ such that $\left\vert f\left( w_1 \right) \right\vert > \left\vert f\left( z_0 \right) \right\vert > \left\vert f\left( w_2 \right) \right\vert$.
    \item Show that if $\left\vert f \right\vert$ is constant, then $f$ is constant.
  \end{enumerate}
\end{problem}
\begin{solution}\hfill
  \begin{enumerate}[(a)]
    \item Since $f$ is analytic, we may find $r > 0$ and a sequence $\left( a_n \right)_n$ such that
      \begin{align*}
        f(z) &= \sum_{n=0}^{\infty} a_n\left( z-z_0 \right)^{n}.
        \intertext{Observe that $f\left( z_0 \right) = a_0$, so}
             &= f\left( z_0 \right) + \sum_{n=1}^{\infty} a_n\left( z-z_0 \right)^{n}.
        \intertext{Next, we find the minimum value of $n$ such that $a_n\neq 0$, which we define to be $k$. Such a value must exist since $f$ is a nonconstant function, and if it were to not exist, the identity theorem would give $f$ as a constant function on $U\left( z_0,r \right)$. This gives}
             &= f\left( z_0 \right) + a_k\left( z-z_0 \right)^{k} + \sum_{n=k+1}^{\infty}a_n\left( z-z_0 \right)^{n}.
             \intertext{Finally, by reindexing the sum and factoring out $\left( z-z_0 \right)^{k+1}$, we get}
             &= f\left( z_0 \right) + a_k\left( z-z_0 \right)^{k} + \left( z-z_0 \right)^{k+1}\sum_{n=0}^{\infty}a_{n+k+1}\left( z-z_0 \right)^{n}.
        \intertext{Define $g(z)$ to be equal to the sum, and define $\lambda = a_k$. Notice that since the radius of convergence of a power series is a limiting case, $g$ and $f$ have the same radius of convergence. This gives}
             &= f\left( z_0 \right) + \lambda \left( z-z_0 \right)^{k} + \left( z-z_0 \right)^{k+1}g(z).
      \end{align*}
    \item Let $f$ be a nonconstant analytic function with $f\left( z_0 \right) \neq 0$. Since $f$ is nonconstant, we see that $\lambda$ in the previous problem is nonzero, meaning that $\left\vert \lambda \right\vert$ is nonzero, in addition to $\left\vert f\left(z_0\right) \right\vert$.\newline

      We start by considering the case where $f(z) = f\left(z_0\right) + \lambda \left( z-z_0 \right)^{k}$. We will reintroduce $g(z)$ later, but first we work on establishing the existence of $w_1$ and $w_2$ in this scenario. Writing $\left( z-z_0 \right) = \left\vert z-z_0 \right\vert e^{i\varphi}$, we thus get that
      \begin{align*}
        f(z) &= \left\vert f\left( z_0 \right) \right\vert e^{i\theta_0} + \left\vert \lambda \right\vert \left\vert z-z_0 \right\vert^{k} e^{i\left( \theta_{\lambda} + k\varphi \right)}
      \end{align*}
      for all $z\in U\left( z_0,r \right)$. Note that the phases $\theta_0$ and $\theta_{\lambda} + k\varphi$ ``add'' if and only if $\varphi = \frac{1}{k} \left( \theta_{0} - \theta_{\lambda} \right)$. Therefore, if $\omega_1\in U\left( z_0,r \right)\setminus \set{z_0}$ is such that $\omega_1 - z_0 = \left\vert \omega_1 - z_0 \right\vert e^{i \varphi_1}$ with $\varphi_2$ satisfying this condition, we then have
      \begin{align*}
        \left\vert f\left( \omega \right) \right\vert &= \left\vert f\left( z_0 \right) \right\vert + \left\vert \lambda \right\vert \left\vert \omega_1 - z_0 \right\vert^{k},
      \end{align*}
      implying that $\left\vert f\left( \omega \right) \right\vert > \left\vert f\left( z_0 \right) \right\vert$. Similarly, if $\varphi_2$ is such that $\varphi_2 = \frac{1}{k} \left( \theta_{0} - \theta_{\lambda} + \pi \right)$, then if $\omega_2\in U\left( z_0,r \right)\setminus \set{z_0}$ is such that 
      \begin{align*}
        \left\vert f\left( \omega_2 \right) \right\vert &= \left\vert f\left( z_0 \right) \right\vert - \left\vert \lambda \right\vert\left\vert \omega_2-z_0 \right\vert^{k}.
      \end{align*}
      Thus, in this case, we have found $\omega_1$ and $\omega_2$ satisfying $\left\vert f\left( \omega_1 \right) \right\vert > \left\vert f\left( z_0 \right) \right\vert > \left\vert f\left( \omega_2 \right) \right\vert$.\newline

      Now, reintroducing our term $\left( z-z_0 \right)^{k+1}g(z)$, which we write in polar form as $\left\vert z-z_0 \right\vert \left\vert g(z) \right\vert e^{i\psi}$, we notice that for a fixed $0 < s_0 < r$ such that $B\left( z_0,s_0 \right)\subseteq U\left( z_0,r \right)$, $\left\vert g \right\vert$ is bounded on $B\left( z_0,s_0 \right)$, as $g$ is analytic and thus continuous. Call this bound $M$.\newline

      We may then find $0 < s < s_0$ small enough with $w_1,w_2\in U\left( z_0,s \right)$ and arguments $\varphi_1$ and $\varphi_2$ as in the case of $\omega_1$ and $\omega_2$ defined earlier such that
      \begin{align*}
        \left\vert f\left(z_0\right) + \lambda \left( w_2-z_0 \right)^{k} \right\vert - Ms^{k+1} &> \left\vert f\left( z_0 \right) \right\vert\\
        \left\vert f\left(z_0\right) + \lambda \left( w_2-z_0 \right)^{k} \right\vert + Ms^{k+1} &< \left\vert f\left( z_0 \right) \right\vert.
      \end{align*}
      Then, by the triangle inequality, we see that
      \begin{align*}
        \left\vert f\left( w_1 \right) \right\vert &= \left\vert f\left( z_0 \right) + \lambda \left( w_1-z_0 \right)^{k} + \left( w_1-z_0 \right)^{k+1}g(z) \right\vert\\
                                                   &\geq \left\vert f\left( z_0 \right) + \lambda \left( w_1-z_0 \right)^{k} \right\vert - \left\vert w_1-z_0 \right\vert^{k+1}\left\vert g(z) \right\vert\\
                                                   &\geq \left\vert f\left( z_0 \right) + \lambda \left( w_1-z_0 \right)^{k} \right\vert - Ms^{k+1}\\
                                                   &> \left\vert f\left( z_0 \right) \right\vert,
      \end{align*}
      and similarly,
      \begin{align*}
        \left\vert f\left( w_2 \right) \right\vert &= \left\vert f\left( z_0 \right) + \lambda \left( w_2-z_0 \right)^{k} + \left( w_2-z_0 \right)^{k+1}g(z) \right\vert \\
                                                   &\leq \left\vert f\left( z_0 \right) + \lambda \left( w_2-z_0 \right)^{k} \right\vert + \left\vert g\left( z \right) \right\vert \left\vert w_1-z_0 \right\vert^{k+1}\\
                                                   &\leq \left\vert f\left( z_0 \right) + \lambda \left( w_2-z_0 \right)^{k} \right\vert + Ms^{k+1}\\
                                                   &< \left\vert f\left( z_0 \right) \right\vert.
      \end{align*}
    \item Let $\left\vert f \right\vert$ be constant. Via the contrapositive of the previous part, $\left\vert f\left( w \right) \right\vert = \left\vert f\left( z_0 \right) \right\vert$ for all $w\in U\left( z_0,s \right)$. In particular, this means that either $f\left( z_0 \right) = 0$ or $f$ is constant; note that if $f\left( z_0 \right) = 0$, then since $\left\vert f\left( w \right) \right\vert = \left\vert f\left( z_0 \right) \right\vert = 0$ for all $w\in U\left( z_0,s \right)$, the identity theorem means that $f = 0$, so either way, $f$ is constant.
  \end{enumerate}
\end{solution}
\begin{problem}[Problem 5]
  Let $U\subseteq \C$ be an open set, and let $V = \set{z\in \C | \overline{z}\in U}$.
  \begin{enumerate}[(a)]
    \item Show that if $f\colon U\rightarrow \C$ is analytic, then $g\colon V\rightarrow \C$ defined by $g(z) = \overline{f\left( \overline{z} \right)}$ is analytic.
    \item Show that if $f\colon U\rightarrow \C$ is holomorphic, then $g\colon V\rightarrow \C$ defined by $g\left( z \right) = \overline{f\left( \overline{z} \right)}$ is holomorphic.
  \end{enumerate}
\end{problem}
\begin{solution}\hfill
  \begin{enumerate}[(a)]
    \item Let $z_0\in V$, so that there exists $r > 0$ such that $U\left( \overline{z_0},r \right)\subseteq U$ and $\left( a_n \right)_n\subseteq \C$ with
      \begin{align*}
        f(z) &= \sum_{n=0}^{\infty} a_n\left( z- \overline{z_0} \right)^{n}.
      \end{align*}
      Observe that the sum uniformly converges on all compact subsets of $U\left( \overline{z_0},r \right)$, meaning that
      \begin{align*}
        f\left( \overline{z} \right) &= \sum_{n=0}^{\infty} a_n \left( \overline{z} - \overline{z_0} \right)^{n}
      \end{align*}
      uniformly converges on all compact subsets of $ U\left( z_0,r \right)\subseteq V $, as conjugation is continuous. Thus, we may exchange the sum and conjugation during the following series of operations that we carry out on $ f\left( \overline{z} \right) $.
      \begin{align*}
        f\left( \overline{z} \right) &= \sum_{n=0}^{\infty} a_n \left( \overline{z}- \overline{z_0} \right)^{n}\\
                                     &= \sum_{n=0}^{\infty} a_n \overline{\left( z-z_0 \right)^{n}}\\
                                     &= \sum_{n=0}^{\infty} \overline{ \overline{a_n} \left( z-z_0 \right)^{n} }\\
                                     &= \overline{\sum_{n=0}^{\infty} \overline{a_n} \left( z-z_0 \right)^{n}}.
      \end{align*}
      Finally, since conjugation is an involution, we have that
      \begin{align*}
        g(z) &= \overline{f\left( \overline{z} \right)}\\
             &= \overline{ \left( \overline{\sum_{n=0}^{\infty} \overline{a_n} \left( z-z_0 \right)^{n}} \right) }\\
             &= \sum_{n=0}^{\infty} \overline{a_n} \left( z-z_0 \right)^{n}.
      \end{align*}
      Notice that $g$ is defined on $U\left( z_0,r \right)$ since $U\left( z_0,r \right)\subseteq U\left( z_0,R \right)$, where $R$ is the radius of convergence, and the radius of convergence for a power series is unchanged if all its corresponding values of $\left( a_n \right)_n$ are conjugated. Thus, $g$ is analytic.
    \item We know that $f$ is holomorphic, so $f'(z)$ exists and is continuous on $U$. If $z\in V$, we notice that $w\rightarrow z$ in $V$ if and only if $ \overline{w}\rightarrow \overline{z} $ in $U$, so
      \begin{align*}
        \lim_{w\rightarrow z} \frac{g\left( w \right)-g\left( z \right)}{w-z} &= \lim_{w\rightarrow z} \overline{\frac{f\left( \overline{w} \right)-f\left( \overline{z} \right)}{ w-z }}\\
                                                                              &= \lim_{ w\rightarrow z }  \frac{f\left( \overline{w} \right) - f\left( \overline{z} \right)}{ \overline{w} - \overline{z} }\\
                                                                              &= \lim_{ \overline{w}\rightarrow \overline{z} } \frac{f\left( \overline{w} \right) - f\left( \overline{z} \right)}{ \overline{w} - \overline{z} }\\
                                                                              &= f'\left( \overline{z} \right),
      \end{align*}
      meaning that $g'(z)$ exists and is defined as $f'\left( \overline{z} \right)$ whenever $z\in V$. Since $f'$ is continuous and conjugation is continuous, so too is $g'$, meaning $g$ is holomorphic.
  \end{enumerate}
\end{solution}
\begin{problem}[Problem 6]\hfill
  \begin{enumerate}[(a)]
    \item For $a\in \mathbb{D}$, define $f_a(z) = \frac{z-a}{1- \overline{a}z}$. Prove that $f_a$ is a bijection from $\D$ to $\D$.
    \item For $a_1,a_2\in \D$, prove that there is a holomorphic bijection $f\colon \D\rightarrow \D$ satisfying $f\left( a_1 \right) = a_2$.
  \end{enumerate}
\end{problem}
\begin{solution}\hfill
  \begin{enumerate}[(a)]
    \item We will show that $f_a$ is a bijection from $\D$ to $\D$ by showing that $f_a$ is defined for all $z\in \D$, that if $z\in \D$, then $f_a(z) \in \D$, then by showing that $f_a$ admits an inverse. First, we observe that $f_a$ is defined so long as $1- \overline{a}z \neq 0$, meaning that $f_a$ is undefined if
      \begin{align*}
        1- \overline{a}z &= 0\\
        z &= \frac{1}{ \overline{a} }\\
          &= \frac{a}{\left\vert a \right\vert^2}\\
          &= \frac{1}{\left\vert a \right\vert} \left( \sgn(a) \right),
      \end{align*}
      which necessarily has modulus greater than $1$, as $\left\vert a \right\vert < 1$ and $\left\vert \sgn(a) \right\vert = 1$ if $a \neq 0$. Next, we see that $f_a(z)$ is a Möbius transformation that is uniquely determined by
      \begin{align*}
        a &\mapsto 0\\
        0 &\mapsto -a\\
        -a &\mapsto \frac{-2a}{1 + \left\vert a \right\vert^2},
      \end{align*}
      all of which stay within the unit disk (for $a\neq 0$ and $a\in \D$). Finally, observe that by taking
      \begin{align*}
        w &= \frac{z-a}{1- \overline{a}z}
      \end{align*}
      and solving for $w$, we obtain
      \begin{align*}
        z &= \frac{w+a}{1 + \overline{a}w}.
      \end{align*}
      This is a left and right inverse, as
      \begin{align*}
        f_a^{-1}\left( f_a(z) \right) &= \frac{\frac{z-a}{1- \overline{a}z} + a}{1 + \overline{a}\frac{z-a}{1- \overline{a}z}}\\
                                      &= z,
      \end{align*}
      and
      \begin{align*}
        f_a\left( f_a^{-1}\left( w \right) \right) &= \frac{\frac{w+a}{1 + \overline{a}w} - a}{1 - \overline{a}\frac{w+a}{1 + \overline{a}w}}\\
                                                   &= w.
      \end{align*}
      Thus, $f$ is a bijection from $\D$ to $\D$.
    \item Considering the $f_a$ of the previous example, we observe that $f_a$ is holomorphic. To see this, note that
      \begin{align*}
        f_a'(z) &= \lim_{h\rightarrow 0} \frac{\frac{\left( z+h \right) - a}{1 - \overline{a}\left( z+h \right)} - \frac{z-a}{1- \overline{a}z}}{h}\\
                &= \lim_{h\rightarrow 0} \frac{\left( \left( z+h \right) - a \right)\left( 1- \overline{a}z \right) - \left( z-a \right)\left( 1 - \overline{a}\left( z+h \right) \right)}{h\left( 1- \overline{a}z \right)\left( 1- \overline{a}\left( z+h \right) \right)}\\
                &= \frac{\left( 1 + \left\vert a \right\vert^2 \right) - \overline{a}z}{\left( 1- \overline{a}z \right)^2},
      \end{align*}
      which is continuous on $ \mathbb{D} $ as it is a rational function that is not undefined on $ \mathbb{D} $. Since $f_a$ is holomorphic, it follows that the composition of $f_a$ with any other such $f_b$ is also holomorphic by chain rule. Finally, note that from our above calculations, $f_{a}^{-1} = f_{-a}$, so we may take
      \begin{align*}
        f &= f_{-a_2}\circ f_{a_1}
      \end{align*}
      to be our holomorphic bijection from $ \mathbb{D} $ to $ \mathbb{D} $ that maps $a_1$ to $a_2$.
  \end{enumerate}
\end{solution}
\end{document}
