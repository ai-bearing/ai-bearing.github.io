\documentclass[10pt]{mypackage}

% sans serif font:
%\usepackage{cmbright}
%\usepackage{sfmath}
%\usepackage{bbold} %better blackboard bold

\usepackage{homework}
%\usepackage{notes}
\usepackage{newpxtext,eulerpx,eucal}
\renewcommand*{\mathbb}[1]{\varmathbb{#1}}

\fancyhf{}
\rhead{Avinash Iyer}
\lhead{Complex Analysis: Assignment 2}

\setcounter{secnumdepth}{0}

\begin{document}
\RaggedRight
\begin{problem}[Problem 1]\hfill
  \begin{enumerate}[(a)]
    \item Show that the power series $\sum_{n=0}^{\infty}\frac{z^{n}}{n!}$ converges for all $z\in \C$, in which it defines an analytic function, which we denote $e^{z}$.
    \item With this as the definition of $e^{z}$, prove that $e^{z}e^{w} = e^{z+w}$.
    \item Show that for $\theta\in \R$, we have that $e^{i\theta} = \cos\left( \theta \right) + i\sin\left( \theta \right)$, where $\cos\left( \theta \right)$ and $\sin\left( \theta \right)$ are defined via their usual power series representations.
  \end{enumerate}
\end{problem}
\begin{solution}\hfill
  \begin{enumerate}[(a)]
    \item To compute
      \begin{align*}
        \rho &= \limsup_{n\rightarrow\infty} \left( \frac{1}{n!} \right)^{1/n},
      \end{align*}
      we take ordinary natural logarithms and use the fact that logarithms are increasing functions to find that
      \begin{align*}
        \ln\left( \rho \right) &= \limsup_{n\rightarrow\infty} \left( -n\sum_{k=1}^{n}\ln\left( k \right) \right)\\
                               &= -\infty,
      \end{align*}
      meaning that $\rho = 0$, or that $R = \frac{1}{\rho}$ is infinite.
    \item Computing $e^{z}e^{w}$, we get
      \begin{align*}
        \left( \sum_{k=0}^{\infty}\frac{z^{k}}{k!} \right)\left( \sum_{\ell = 0}^{\infty}\frac{w^{k}}{k!} \right) &= \sum_{k=0}^{\infty}\sum_{\ell = k}^{\infty} \frac{1}{\left( \ell - k \right)!}\frac{1}{k!}w^{k}z^{\ell - k}\\
                                                                                                                  &= \sum_{\ell = 0}^{\infty} \frac{1}{\ell!} \sum_{k=0}^{\ell} \frac{1}{k!\left( \ell-k \right)!} w^{k}z^{\ell - k}\\
                                                                                                                  &= \sum_{\ell = 0}^{\infty} \frac{1}{\ell!} \left( z+w \right)^{\ell}\\
                                                                                                                  &= e^{z+w}.
      \end{align*}
    \item Computing $e^{i\theta}$ by direct substitution, we find that
      \begin{align*}
        e^{i\theta} &= \sum_{k=0}^{\infty} \frac{\left( i\theta \right)^{k}}{k!}\\
                    &= \sum_{k\text{ even}} \frac{\left( -1 \right)^{(k/2)}\theta^{k}}{k!} + i\sum_{k\text{ odd}} \frac{\left( -1 \right)^{\left( k-1 \right)/2}\theta^{k}}{k!}\\
                    &= \cos\left( \theta \right) + i\sin\left( \theta \right).
      \end{align*}
  \end{enumerate}
\end{solution}
\begin{problem}[Problem 2]
  Let $U\subseteq \C$ be an open set, $f\colon U\rightarrow \C$ an analytic function. Since $f$ is analytic, given $z_0\in U$, there is $r > 0$ and a sequence $\left( a_n \right)_n$ such that $f(z) = \sum_{n=0}^{\infty}a_n\left( z-z_0 \right)^{n}$ for all $z\in U\left( z_0,r \right)$.\newline

  Suppose there exists $R > r$ such that $U\left( z_0,R \right) \subseteq U$ and $\sum_{n=0}^{\infty}a_n\left( z-z_0 \right)^{n}$ has radius of convergence at least $R$. Show that $f(z) = \sum_{n=0}^{\infty}a_n\left( z-z_0 \right)^{n}$ for all $z\in U\left( z_0,R \right)$.
\end{problem}
\begin{solution}
  On the connected open set $V = U\left( z_0,R \right)$, define
  \begin{align*}
    g(z) &= \sum_{n=0}^{\infty}a_n\left( z-z_0 \right)^{n}.
  \end{align*}
  Observe that $f|_{V}$ and $g$ agree on the open subset $U\left( z_0,r \right)\subseteq U\left( z_0,R \right)$. By the identity theorem, this means that $f = g$ on $U\left( z_0,R \right)$.
\end{solution}
\begin{problem}[Problem 3]
  Let $U\subseteq \C$ be a region, and let $f\colon U\rightarrow \C$ be an analytic function.
  \begin{enumerate}[(a)]
    \item Suppose $f$ is nonconstant, $z_0\in U$. Show that there exists some $r > 0$ for which $U\left( z_0,r \right)\subseteq U$, a positive integer $k\in \N$, an analytic function $g\colon U\left( z_0,r \right)\rightarrow \C$, and a nonconstant $\lambda\in \C\setminus \set{0}$ such that for $z\in U\left( z_0,r \right)$,
      \begin{align*}
        f(z) &= f\left(z_0\right) + \lambda\left( z-z_0 \right)^{k} + \left( z-z_0 \right)^{k+1} g(z).
      \end{align*}
    \item Suppose that $f$ is nonconstant, and $z_0\in U$ is such that $f\left( z_0 \right) \neq 0$. Show that there exists some $s > 0$ such that $U\left( z_0,s \right)\subseteq U$, and $w_1,w_2\in U\left( z_0,s \right)$ such that $\left\vert f\left( w_1 \right) \right\vert > \left\vert f\left( z_0 \right) \right\vert > \left\vert f\left( w_2 \right) \right\vert$.
    \item Show that if $\left\vert f \right\vert$ is constant, then $f$ is constant.
  \end{enumerate}
\end{problem}
\begin{solution}\hfill
  \begin{enumerate}[(a)]
    \item Since $f$ is analytic, we may find $r > 0$ and a sequence $\left( a_n \right)_n$ such that
      \begin{align*}
        f(z) &= \sum_{n=0}^{\infty} a_n\left( z-z_0 \right)^{n}.
        \intertext{Observe that $f\left( z_0 \right) = a_0$, so}
             &= f\left( z_0 \right) + \sum_{n=1}^{\infty} a_n\left( z-z_0 \right)^{n}.
        \intertext{Next, we find the minimum value of $n$ such that $a_n\neq 0$, which we define to be $k$. Such a value must exist since $f$ is a nonconstant function. This gives}
             &= f\left( z_0 \right) + a_k\left( z-z_0 \right)^{k} + \sum_{n=k+1}^{\infty}a_n\left( z-z_0 \right)^{n}.
             \intertext{Finally, by reindexing the sum and factoring out $\left( z-z_0 \right)^{k+1}$, we get}
             &= f\left( z_0 \right) + a_k\left( z-z_0 \right)^{k} + \left( z-z_0 \right)^{k+1}\sum_{n=0}^{\infty}a_{n+k+1}\left( z-z_0 \right)^{n}.
        \intertext{Define $g(z)$ to be equal to the sum, and define $\lambda = a_k$. Notice that since the radius of convergence of a power series is a limiting case, $g$ and $f$ have the same radius of convergence. This gives}
             &= f\left( z_0 \right) + \lambda \left( z-z_0 \right)^{k} + \left( z-z_0 \right)^{k+1}g(z).
      \end{align*}
    \item Let $f$ be a nonconstant analytic function with $f\left( z_0 \right) \neq 0$. Since $f$ is nonconstant, we see that $\lambda$ in the previous problem is nonzero, meaning that $\left\vert \lambda \right\vert$ is nonzero, in addition to $\left\vert f\left(z_0\right) \right\vert$.
  \end{enumerate}
\end{solution}
\begin{problem}[Problem 6]\hfill
  \begin{enumerate}[(a)]
    \item For $a\in \mathbb{D}$, define $f_a(z) = \frac{z-a}{1- \overline{a}z}$. Prove that $f_a$ is a bijection from $\D$ to $\D$.
    \item For $a_1,a_2\in \D$, prove that there is a holomorphic bijection $f\colon \D\rightarrow \D$ satisfying $f\left( a_1 \right) = a_2$.
  \end{enumerate}
\end{problem}
\begin{solution}\hfill
  \begin{enumerate}[(a)]
    \item We will show that $f_a$ is a bijection from $\D$ to $\D$ by showing that $f_a$ is defined for all $z\in \D$, that if $z\in \D$, then $f_a(z) \in \D$, then by showing that $f_a$ admits an inverse. First, we observe that $f_a$ is defined so long as $1- \overline{a}z \neq 0$, meaning that $f_a$ is undefined if
      \begin{align*}
        1- \overline{a}z &= 0\\
        z &= \frac{1}{ \overline{a} }\\
          &= \frac{a}{\left\vert a \right\vert^2}\\
          &= \frac{1}{\left\vert a \right\vert} \left( \sgn(a) \right),
      \end{align*}
      which necessarily has modulus greater than $1$, as $\left\vert a \right\vert < 1$ and $\sgn(a) = 1$ if $a \neq 0$. Next, we see that $f_a(z)$ is a Möbius transformation that is uniquely determined by
      \begin{align*}
        a &\mapsto 0\\
        0 &\mapsto -a\\
        -a &\mapsto \frac{-2a}{1 + \left\vert a \right\vert^2},
      \end{align*}
      all of which stay within the unit disk (for $a\neq 0$ and $a\in \D$). Finally, observe that by taking
      \begin{align*}
        w &= \frac{z-a}{1- \overline{a}z}
      \end{align*}
      and solving for $w$, we obtain
      \begin{align*}
        z &= \frac{w+a}{1 + \overline{a}w}.
      \end{align*}
      This is a left and right inverse, as
      \begin{align*}
        f_a^{-1}\left( f_a(z) \right) &= \frac{\frac{z-a}{1- \overline{a}z} + a}{1 + \overline{a}\frac{z-a}{1- \overline{a}z}}\\
                                      &= z,
      \end{align*}
      and
      \begin{align*}
        f_a\left( f_a^{-1}\left( w \right) \right) &= \frac{\frac{w+a}{1 + \overline{a}w} - a}{1 - \overline{a}\frac{w+a}{1 + \overline{a}w}}\\
                                                   &= w.
      \end{align*}
      Thus, $f$ is a bijection from $\D$ to $\D$.
    \item Considering the $f_a$ of the previous example, we observe that $f_a$ is holomorphic, as it is Möbius transformation that is undefined at $ \frac{1}{\left\vert a \right\vert} \sgn\left( a \right) $, which is outside $\D$. By using the Möbius transformation characterization from earlier, we observe that the composition
      \begin{align*}
        f &= f_{a_2}^{-1}\circ f_{a_1}
      \end{align*}
      is holomorphic (as it is a composition of Möbius transformations) and maps $a_1$ to $a_2$.
  \end{enumerate}
\end{solution}
\end{document}
