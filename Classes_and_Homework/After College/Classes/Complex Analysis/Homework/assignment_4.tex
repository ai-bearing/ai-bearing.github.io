\documentclass[10pt]{mypackage}

% sans serif font:
%\usepackage{cmbright}
%\usepackage{sfmath}
%\usepackage{bbold} %better blackboard bold

\usepackage{homework}
%\usepackage{notes}
\usepackage{newpxtext,eulerpx,eucal}
\renewcommand*{\mathbb}[1]{\varmathbb{#1}}

\fancyhf{}
\rhead{Avinash Iyer}
\lhead{Complex Analysis: Assignment 4}

\setcounter{secnumdepth}{0}

\begin{document}
\RaggedRight
\begin{problem}[Problem 1]
  Let $\left( a_n \right)_n$ be a sequence for which $\sum_{n=0}^{\infty}\left\vert a_n \right\vert^2$ is finite. For each positive $N$, define $f_N(z) = \sum_{n=0}^{\infty}a_nz^{n}$, and define $f(z) = \sum_{n=0}^{\infty}a_nz^{n}$.
  \begin{enumerate}[(a)]
    \item Show that $f$ is holomorphic on $ \mathbb{D} $.
    \item For each $r\in \left( 0,1 \right)$, determine in terms of $ \left( a_n \right)_n $ the integral
      \begin{align*}
        \frac{1}{2\pi} \int_{0}^{2\pi} \left\vert f_N\left( re^{i\theta} \right) \right\vert^2\:d\theta.
      \end{align*}
    \item For each $r\in \left( 0,1 \right)$, determine in terms of $ \left( a_n \right)_n $ the integral
      \begin{align*}
        \frac{1}{2\pi} \int_{0}^{2\pi} \left\vert f\left( re^{i\theta} \right) \right\vert^{2}\:d\theta.
      \end{align*}
    \item Determine in terms of $ \left( a_n \right)_n $ the limit
      \begin{align*}
        \lim_{r\nearrow 1} \frac{1}{2\pi} \int_{0}^{2\pi} \left\vert f\left( re^{i\theta} \right) \right\vert^2\:d\theta.
      \end{align*}
  \end{enumerate}
\end{problem}
\begin{solution}\hfill
  \begin{enumerate}[(a)]
    \item Let $ 0 < r < 1 $. Since each $f_N$ is analytic, we can use the Cauchy Integral Formula to compute $a_N$ explicitly, yielding
      \begin{align*}
        \left\vert a_N \right\vert &= \left\vert \frac{1}{2\pi i} \int_{0}^{2\pi} \frac{f_N\left( \xi \right)}{\xi^{N+1}}\:d\xi \right\vert\\
                                   &\leq \frac{1}{r^{N}} \sup_{|z| = r} \left\vert f_N\left( z \right) \right\vert.
      \end{align*}
      Therefore, if we are able to show that the value 
      \begin{align*}
        \sup_{|z| = r}\left\vert f_N\left( z \right) \right\vert
      \end{align*}
      is uniformly bounded by a constant for all $N$, we will be able to use the Cauchy--Hadamard theorem to show that $\limsup_{N\rightarrow\infty}\left\vert a_N \right\vert^{1/N}\leq 1$. Toward this end, we use the Cauchy--Schwarz inequality, which yields
      \begin{align*}
        \sup_{|z| = r} \left\vert f_N(z) \right\vert &= \sup_{|z| = r} \left\vert \sum_{n=0}^{N}a_nz^{n} \right\vert\\
                                                     &\leq \sup_{|z| = r} \left( \sum_{n=0}^{N} \left\vert a_n \right\vert^2 \right)^{1/2} \left( \sup_{m=0}^{N} \left\vert z \right\vert^{2m} \right)^{1/2}\\
                                                     &\leq \sup_{|z| = r} \underbrace{\left( \sum_{n=0}^{\infty}\left\vert a_n \right\vert^2 \right)^{1/2}}_{\eqcolon K} \left( \sum_{m=0}^{\infty}\left\vert z \right\vert^{2m} \right)^{1/2}\\
                                                     &= \frac{K}{\left( 1-|r|^2 \right)^{1/2}}.
      \end{align*}
      Since we have established this uniform bound, we thus find that $ \sum_{n=0}^{\infty}a_nz^{n} $ has radius of convergence at least $1$, so $f$ is analytic on $ \mathbb{D} $, hence holomorphic on $ \mathbb{D} $.
    \item We write out the integral to yield
      \begin{align*}
        \frac{1}{2\pi}\int_{0}^{2\pi} \left\vert f_N\left( re^{i\theta} \right) \right\vert^2\:d\theta &= \frac{1}{2\pi} \int_{0}^{2\pi} \left( \sum_{n=0}^{N} a_nr^{n}e^{in\theta} \right) \overline{\left( \sum_{m=0}^{N} a_mr^{m}e^{im\theta} \right)}\:d\theta\\
                                                                                                       &= \frac{1}{2\pi} \sum_{n=0}^{N}\sum_{m=0}^{N} a_n \overline{a_m} r^{m + n} \int_{0}^{2\pi} e^{i\left( n-m \right)\theta}\:d\theta\\
                                                                                                       &= \sum_{n=0}^{N} \left\vert a_n \right\vert^2 r^{2n}.
      \end{align*}
    \item Since $f$ is holomorphic with radius of convergence at least $1$, the series expression on $ S\left( 0,r \right) $ converges uniformly, so that we may exchange sum and integral. This yields
      \begin{align*}
        \frac{1}{2\pi} \int_{0}^{2\pi} \left\vert f\left( re^{i\theta} \right) \right\vert^2\:d\theta &= \frac{1}{2\pi}\sum_{n=0}^{\infty}\sum_{m=0}^{\infty}a_n \overline{a_m} r^{m + n} \int_{0}^{2\pi} e^{i\left( n-m \right)\theta}\:d\theta\\
                                                                                                      &= \sum_{n=0}^{\infty}\left\vert a_n \right\vert^2 r^{2n}.
      \end{align*}
    \item Since the sequence $\left( a_n \right)_n$ is square-summable, the limit is well-defined, and we get
      \begin{align*}
        \lim_{r\nearrow 1} \frac{1}{2\pi} \int_{0}^{2\pi} \left\vert f\left( re^{i\theta} \right) \right\vert^2\:d\theta &= \lim_{r\nearrow 1} \sum_{n=0}^{\infty}\left\vert a_n \right\vert^2 r^{2n}\\
                                                                                                                         &= \sum_{n=0}^{\infty}\left\vert a_n \right\vert^2.
      \end{align*}
  \end{enumerate}
\end{solution}
\begin{problem}[Problem 2]
  Let $\varphi\colon \left[ 0,1 \right]\rightarrow \C$ be continuous, and define $f\colon \C\setminus \left[ 0,1 \right]\rightarrow \C$ by
  \begin{align*}
    f(z) &= \int_{0}^{1} \frac{\varphi(t)}{t-z}\:dt.
  \end{align*}
  Show that $f$ is holomorphic and determine the derivative of $f$ in terms of $\varphi$.
\end{problem}
\begin{problem}[Problem 3]
  Let $f\colon \C\rightarrow \C$ be entire.
  \begin{enumerate}[(a)]
    \item Suppose there exist $C,R > 0$ and $n\in \N$ such that $\left\vert f(z) \right\vert \leq C\left\vert z \right\vert^{n}$ for all $\left\vert z \right\vert > R$. Show that $f$ is a polynomial of degree at most $n$.
    \item Suppose that $g\colon \C\rightarrow \C$ is also entire and $ \left\vert f(z) \right\vert \leq \left\vert g(z) \right\vert $ for all $z\in \C$. Show that there exists some $\alpha\in \C$ with $\left\vert \alpha \right\vert \leq 1$ such that $f(z) = \alpha g(z)$ for all $z\in \C$.
    \item Suppose that there exists some $\theta\in \R$ such that $f\left( \C \right)\cap \set{re^{i\theta} | r > 0} = \emptyset$. Show that $f$ is constant.
  \end{enumerate}
\end{problem}
\begin{solution}\hfill
  \begin{enumerate}[(a)]
    \item Let $r > R$. Then, by the Cauchy estimate, we get that
      \begin{align*}
        \left\vert f^{\left(n+1\right)}(0) \right\vert &\leq \frac{\left( n+1 \right)!}{r^{n+1}}\sup_{|z| = r} \left\vert f(z) \right\vert\\
                                                       &\leq \frac{\left( n+1 \right)!}{r^{n+1}} \sup_{|z| = r} \left( C \left\vert z \right\vert^{n} \right)\\
                                                       &= \frac{C\left( n+1 \right)!}{r},
      \end{align*}
      so since $r$ is arbitrary and $f$ is entire, we find that $ f^{(n+1)}(0) = 0 $, so that the power series expansion of $f$ about $0$ terminates beyond $n + 1$, meaning that $f$ is a polynomial of degree at most $n$.
    \item If $g$ is $0$, then we are done. Else, assume that $g$ is not identically zero. Observe that if $g$ is everywhere non-vanishing, then the function $ \frac{f(z)}{g(z)} $ is entire, and satisfies
      \begin{align*}
        \left\vert \frac{f(z)}{g(z)} \right\vert \leq 1,
      \end{align*}
      hence $ \frac{f(z)}{g(z)} = \alpha $ for some $\alpha$ with $\left\vert \alpha \right\vert \leq 1$.\newline

      Now, if $g(z)$ does admit zeros, they must be isolated zeros, or else by the identity theorem, we would have that $g$ is identically zero on $\C$. We observe that if $a\in \C$ is a zero for $g$, we may then write
      \begin{align*}
        g(z) &= \left( z-a \right)^{n}g^{\ast}\left( z \right),
      \end{align*}
      with $g^{\ast}\left( z \right)$ holomorphic and $g^{\ast}\left( a \right)\neq 0$. Additionally, since $\left\vert f(z) \right\vert \leq \left\vert g(z) \right\vert$, we must have $f(a) = 0$, so that, similarly,
      \begin{align*}
        f(z) &= \left( z-a \right)^{m}f^{\ast}\left( z \right)
      \end{align*}
      with $f^{\ast}\left( z \right)$ holomorphic and $f^{\ast}\left( a \right) \neq 0$. We observe that, since $\left\vert f(z) \right\vert \leq \left\vert g(z) \right\vert$, in a sufficiently small deleted neighborhood of $a$, that $f^{\ast}\left( z \right)$ and $g^{\ast}\left( z \right)$ are both approximately constant, meaning that, necessarily, $\left\vert z-a \right\vert^{m}\leq \left\vert z-a \right\vert^{n} \frac{g^{\ast}\left( a \right)}{f^{\ast}\left( a \right)}$, so that for sufficiently small $\left\vert z-a \right\vert$, it follows that $ m \geq n $.\newline

      We define $k(z) = \frac{f(z)}{g(z)}$, and define a holomorphic extension of $k(z)$ by
      \begin{align*}
        h(z) &= \begin{cases}
          k(z) & g(z)\neq 0\\
          \lim_{z\rightarrow a} \left( z-a \right) k(z) & g(a) = 0.
        \end{cases}
      \end{align*}
      This is a holomorphic extension of $k$, as
      \begin{align*}
        \lim_{z\rightarrow a} \frac{h(z)-h(a)}{z-a} &= \lim_{z\rightarrow a} h'(z),
      \end{align*}
      so that we have $\left\vert h(z) \right\vert \leq 1$ for all $z$. Thus, $h$ is a bounded entire function, hence constant, so $ \frac{f(z)}{g(z)} = \alpha $ where defined with $\left\vert \alpha \right\vert \leq 1$, and $f(z) = \alpha g(z)$.
    \item By adding a sufficient multiple of $2\pi k$ to $\theta$, we may assume that $\theta > 0$. In particular, this means that 
      \begin{align*}
        \log_{\theta}\colon \C\setminus \set{re^{i\theta} | r > 0} \rightarrow \set{z | \theta < \im(z) < \theta + 2\pi}
      \end{align*}
      is holomorphic. Finally, we observe that the Cayley Transform,
      \begin{align*}
        \varphi(z) &= \frac{z-i}{z+i}
      \end{align*}
      takes the upper half-plane to the unit disk. Therefore, the composition $\varphi\circ \log_{\theta} \circ f\colon \C\rightarrow \mathbb{D}$ is an entire function that is bounded, hence constant. Since $\varphi$ and $\log_{\theta}$ are non-constant, it follows that $f$ is constant.
  \end{enumerate}
\end{solution}
\begin{problem}[Problem 4]
  Let $U = \set{z\in \C | -1 < \im(z) < 1}$. Suppose $f\colon U\rightarrow \C$ is holomorphic, and there exists $C > 0$ and $ \eta\in \R $ such that
  \begin{align*}
    \left\vert f(z) \right\vert \leq C\left( 1 + \left\vert z \right\vert \right)^{\eta}
  \end{align*}
  for all $z\in U$. Show that for each $n \geq 0$, there exists a constant $ C_{n,\eta} \geq 0 $ dependent only on $n$ and $\eta$ such that
  \begin{align*}
    \left\vert f^{(n)}(x) \right\vert \leq C_{n,\eta} \left( 1 + \left\vert x \right\vert \right)^{\eta}
  \end{align*}
  for all $x\in \R$.
\end{problem}
\begin{solution}
  Let $x\in \R$, $0 < r < 1$, and to start, assume $\eta \geq 0$. Then, from Cauchy's estimate, a bunch of triangle inequalities, and the fact that $\eta \geq 0$ and $ r < 1 $, we find that
  \begin{align*}
    \left\vert f^{(n)}\left( x \right) \right\vert &\leq \frac{n!}{r^{n}} \sup_{\left\vert w-x \right\vert= r} \left\vert f(w) \right\vert\\
                                                   &\leq \frac{n!}{r^{n}} \sup_{\left\vert w-x \right\vert = r} \left( C\left( 1 + \left\vert w \right\vert \right)^{\eta} \right)\\
                                                   &\leq \frac{Cn!}{r^{n}} \sup_{\left\vert w-x \right\vert = r} \left( 1 + \left\vert w-\frac{3}{2} x\right\vert + \frac{3}{2}\left\vert x \right\vert \right)^{\eta}\\
                                                   &\leq \frac{Cn!}{r^{n}} \sup_{\left\vert w-x \right\vert = r} \left( 1 + \left\vert w-x \right\vert + 2\left\vert x \right\vert \right)^{\eta}\\
                                                   &\leq \frac{Cn!}{r^{n}}\sup_{\left\vert w-x \right\vert = r} \left( 1 + r + 2\left\vert x \right\vert \right)^{\eta}\\
                                                   &\leq \frac{Cn!}{r^{n}} \left( 2 + 2\left\vert x \right\vert \right)^{\eta}\\
                                                   &\leq \frac{C2^{\eta}n!}{r^{n}} \left( 1 + \left\vert x \right\vert \right)^{\eta}.
  \end{align*}
  In particular, since this inequality holds for every $0 < r < 1$, it necessarily for $r = 1$, so that $C_{n,\eta} = C2^{\eta}n!$.\newline

  Now, if $\eta < 0$.
\end{solution}
\begin{problem}[Problem 5]
  Let $P(z) = a_nz^{n} + a_{n-1}z^{n-1} + \cdots + a_1z + a_0$ be a polynomial of degree $n \geq 1$, where $a_0,\dots,a_n\in \C$ with $a_n\neq 0$.
  \begin{enumerate}[(a)]
    \item Show that there exist $n$ complex numbers $\alpha_1,\dots,\alpha_n\in\C$ not necessarily distinct such that $P(z) = a_n \left( z-\alpha_1 \right)\cdots \left( z-\alpha_n \right)$.
    \item Suppose $\left\vert \alpha_0 \right\vert > \left\vert \alpha_n \right\vert$. Show that there exists some $\alpha\in \C$ for which $\left\vert \alpha \right\vert > 1$ and $P\left(\alpha\right) = 0$.
  \end{enumerate}
\end{problem}
\begin{solution}\hfill
  \begin{enumerate}[(a)]
    \item Dividing out by $a_n$, we take
      \begin{align*}
        P(z) &= a_n \left( z^{n} + \frac{a_{n-1}}{a_n}z^{n-1} + \cdots + \frac{a_1}{a_n}z + \frac{a_0}{a_n} \right).
      \end{align*}
      By the fundamental theorem of algebra, we can find some $\alpha_1$ such that $P\left( \alpha_1 \right) = 0$. Therefore, by polynomial division, we have a monic polynomial $q(z)$ with degree $n-1$ such that
      \begin{align*}
        P(z) &= a_n q(z)\left( z-\alpha_1 \right).
      \end{align*}
      If $q(z)$ is a constant polynomial, it is necessarily equal to $1$ and we are done. Else, inductively, we may find $\alpha_2,\dots,\alpha_n\in \C$ such that $q(z) = \left( z-\alpha_2 \right)\cdots \left( z-\alpha_n \right)$, meaning that
      \begin{align*}
        P(z) &= a_n\left( z-\alpha_1 \right)\left( z-\alpha_2 \right)\cdots \left( z-\alpha_n \right).
      \end{align*}
    \item 
  \end{enumerate}
\end{solution}
\end{document}
