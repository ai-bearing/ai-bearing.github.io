\documentclass[10pt]{mypackage}

% sans serif font:
%\usepackage{cmbright}
%\usepackage{sfmath}
%\usepackage{bbold} %better blackboard bold

\usepackage{homework}
%\usepackage{notes}
\usepackage{newpxtext,eulerpx,eucal}
\renewcommand*{\mathbb}[1]{\varmathbb{#1}}

\fancyhf{}
\rhead{Avinash Iyer}
\lhead{Complex Analysis: Assignment 4}

\setcounter{secnumdepth}{0}

\begin{document}
\RaggedRight
\begin{problem}[Problem 1]
  Let $\left( a_n \right)_n$ be a sequence for which $\sum_{n=0}^{\infty}\left\vert a_n \right\vert^2$ is finite. For each positive $N$, define $f_N(z) = \sum_{n=0}^{\infty}a_nz^{n}$, and define $f(z) = \sum_{n=0}^{\infty}a_nz^{n}$.
  \begin{enumerate}[(a)]
    \item Show that $f$ is holomorphic on $ \mathbb{D} $.
    \item For each $r\in \left( 0,1 \right)$, determine in terms of $ \left( a_n \right)_n $ the integral
      \begin{align*}
        \frac{1}{2\pi} \int_{0}^{2\pi} \left\vert f_N\left( re^{i\theta} \right) \right\vert^2\:d\theta.
      \end{align*}
    \item For each $r\in \left( 0,1 \right)$, determine in terms of $ \left( a_n \right)_n $ the integral
      \begin{align*}
        \frac{1}{2\pi} \int_{0}^{2\pi} \left\vert f\left( re^{i\theta} \right) \right\vert^{2}\:d\theta.
      \end{align*}
    \item Determine in terms of $ \left( a_n \right)_n $ the limit
      \begin{align*}
        \lim_{r\nearrow 1} \frac{1}{2\pi} \int_{0}^{2\pi} \left\vert f\left( re^{i\theta} \right) \right\vert^2\:d\theta.
      \end{align*}
  \end{enumerate}
\end{problem}
\begin{solution}\hfill
  \begin{enumerate}[(a)]
    \item Let $ 0 < r < 1 $. Since each $f_N$ is analytic, we can use the Cauchy Integral Formula to compute $a_N$ explicitly, yielding
      \begin{align*}
        \left\vert a_N \right\vert &= \left\vert \frac{1}{2\pi i} \int_{0}^{2\pi} \frac{f_N\left( \xi \right)}{\xi^{N+1}}\:d\xi \right\vert\\
                                   &\leq \frac{1}{r^{N}} \sup_{|z| = r} \left\vert f_N\left( z \right) \right\vert.
      \end{align*}
      Therefore, if we are able to show that the value 
      \begin{align*}
        \sup_{|z| = r}\left\vert f_N\left( z \right) \right\vert
      \end{align*}
      is uniformly bounded by a constant for all $N$, we will be able to use the Cauchy--Hadamard theorem to show that $\limsup_{N\rightarrow\infty}\left\vert a_N \right\vert^{1/N}\leq 1$. Toward this end, we use the Cauchy--Schwarz inequality, which yields
      \begin{align*}
        \sup_{|z| = r} \left\vert f_N(z) \right\vert &= \sup_{|z| = r} \left\vert \sum_{n=0}^{N}a_nz^{n} \right\vert\\
                                                     &\leq \sup_{|z| = r} \left( \sum_{n=0}^{N} \left\vert a_n \right\vert^2 \right)^{1/2} \left( \sup_{m=0}^{N} \left\vert z \right\vert^{2m} \right)^{1/2}\\
                                                     &\leq \sup_{|z| = r} \underbrace{\left( \sum_{n=0}^{\infty}\left\vert a_n \right\vert^2 \right)^{1/2}}_{\eqcolon K} \left( \sum_{m=0}^{\infty}\left\vert z \right\vert^{2m} \right)^{1/2}\\
                                                     &= \frac{K}{\left( 1-|r|^2 \right)^{1/2}}.
      \end{align*}
      Since we have established this uniform bound, we thus find that $ \sum_{n=0}^{\infty}a_nz^{n} $ has radius of convergence at least $1$, so $f$ is analytic on $ \mathbb{D} $, hence holomorphic on $ \mathbb{D} $.
    \item We write out the integral to yield
      \begin{align*}
        \frac{1}{2\pi}\int_{0}^{2\pi} \left\vert f_N\left( re^{i\theta} \right) \right\vert^2\:d\theta &= \frac{1}{2\pi} \int_{0}^{2\pi} \left( \sum_{n=0}^{N} a_nr^{n}e^{in\theta} \right) \overline{\left( \sum_{m=0}^{N} a_mr^{m}e^{im\theta} \right)}\:d\theta\\
                                                                                                       &= \frac{1}{2\pi} \sum_{n=0}^{N}\sum_{m=0}^{N} a_n \overline{a_m} r^{m + n} \int_{0}^{2\pi} e^{i\left( n-m \right)\theta}\:d\theta\\
                                                                                                       &= \sum_{n=0}^{N} \left\vert a_n \right\vert^2 r^{2n}.
      \end{align*}
    \item Since $f$ is holomorphic with radius of convergence at least $1$, the series expression on $ S\left( 0,r \right) $ converges uniformly, so that we may exchange sum and integral. This yields
      \begin{align*}
        \frac{1}{2\pi} \int_{0}^{2\pi} \left\vert f\left( re^{i\theta} \right) \right\vert^2\:d\theta &= \frac{1}{2\pi}\sum_{n=0}^{\infty}\sum_{m=0}^{\infty}a_n \overline{a_m} r^{m + n} \int_{0}^{2\pi} e^{i\left( n-m \right)\theta}\:d\theta\\
                                                                                                      &= \sum_{n=0}^{\infty}\left\vert a_n \right\vert^2 r^{2n}.
      \end{align*}
    \item Since the sequence $\left( a_n \right)_n$ is square-summable, the limit is well-defined, and we get
      \begin{align*}
        \lim_{r\nearrow 1} \frac{1}{2\pi} \int_{0}^{2\pi} \left\vert f\left( re^{i\theta} \right) \right\vert^2\:d\theta &= \lim_{r\nearrow 1} \sum_{n=0}^{\infty}\left\vert a_n \right\vert^2 r^{2n}\\
                                                                                                                         &= \sum_{n=0}^{\infty}\left\vert a_n \right\vert^2.
      \end{align*}
  \end{enumerate}
\end{solution}
\begin{problem}[Problem 2]
  Let $\varphi\colon \left[ 0,1 \right]\rightarrow \C$ be continuous, and define $f\colon \C\setminus \left[ 0,1 \right]\rightarrow \C$ by
  \begin{align*}
    f(z) &= \int_{0}^{1} \frac{\varphi(t)}{t-z}\:dt.
  \end{align*}
  Show that $f$ is holomorphic and determine the derivative of $f$ in terms of $\varphi$.
\end{problem}
\begin{problem}[Problem 3]
  Let $f\colon \C\rightarrow \C$ be entire.
  \begin{enumerate}[(a)]
    \item Suppose there exist $C,R > 0$ and $n\in \N$ such that $\left\vert f(z) \right\vert \leq C\left\vert z \right\vert^{n}$ for all $\left\vert z \right\vert > R$. Show that $f$ is a polynomial of degree at most $n$.
    \item Suppose that $g\colon \C\rightarrow \C$ is also entire and $ \left\vert f(z) \right\vert \leq \left\vert g(z) \right\vert $ for all $z\in \C$. Show that there exists some $\alpha\in \C$ with $\left\vert \alpha \right\vert \leq 1$ such that $f(z) = \alpha g(z)$ for all $z\in \C$.
    \item Suppose that there exists some $\theta\in \R$ such that $f\left( \C \right)\cap \set{re^{i\theta} | r > 0} = \emptyset$. Show that $f$ is constant.
  \end{enumerate}
\end{problem}
\begin{solution}\hfill
  \begin{enumerate}[(a)]
    \item Let $r > R$. Then, by the Cauchy estimate, we get that
      \begin{align*}
        \left\vert f^{\left(n+1\right)}(0) \right\vert &\leq \frac{\left( n+1 \right)!}{r^{n+1}}\sup_{|z| = r} \left\vert f(z) \right\vert\\
                                                       &\leq \frac{\left( n+1 \right)!}{r^{n+1}} \sup_{|z| = r} \left( C \left\vert z \right\vert^{n} \right)\\
                                                       &= \frac{C\left( n+1 \right)!}{r},
      \end{align*}
      so since $r$ is arbitrary and $f$ is entire, we find that $ f^{(n+1)}(0) = 0 $, so that the power series expansion of $f$ about $0$ terminates beyond $n + 1$, meaning that $f$ is a polynomial of degree at most $n$.
    \item If $g$ is $0$, then we are done. Else, assume that $g$ is not identically zero. Observe that if $g$ is everywhere non-vanishing, then the function $ \frac{f(z)}{g(z)} $ is entire, and satisfies
      \begin{align*}
        \left\vert \frac{f(z)}{g(z)} \right\vert \leq 1,
      \end{align*}
      hence $ \frac{f(z)}{g(z)} = \alpha $ for some $\alpha$ with $\left\vert \alpha \right\vert \leq 1$.\newline

      Now, if $g(z)$ does admit zeros, they must be isolated zeros, or else by the identity theorem, we would have that $g$ is identically zero on $\C$. Letting
      \begin{align*}
        h(z) &= \frac{f(z)}{g(z)},
      \end{align*}
      we see that $h$ admits isolated singularities at the zeros of $g$. In punctured neighborhoods of these zeros, $h$ is bounded by assumption, so each of these singularities is removable. Therefore, $h$ can be extended to an entire function, $k(z)$, satisfying
      \begin{align*}
        \left\vert k(z) \right\vert &\leq 1
      \end{align*}
      for all $z\in \C$. Thus, by Liouville's Theorem, it follows that $k(z) = \alpha$ for some $\alpha\in \C$ with $\left\vert \alpha \right\vert \leq 1$. In particular, whenever $g(z) \neq 0$, we have $f(z) = \alpha g(z)$, and clearly if $g$ is zero, so too is $f$. Thus, we have established the desired result.
    \item Suppose $f\left( \C \right)\cap \set{re^{i\theta} | r > 0} = \emptyset$. For $s > 0$, Cauchy's Estimate gives
      \begin{align*}
        \left\vert f'(0) \right\vert &\leq \frac{1}{s^2} \sup_{|z| = s} \left\vert f(z) \right\vert.
      \end{align*}
  \end{enumerate}
\end{solution}
\end{document}
