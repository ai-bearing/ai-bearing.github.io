\documentclass[10pt]{mypackage}

% sans serif font:
%\usepackage{cmbright}
%\usepackage{sfmath}
%\usepackage{bbold} %better blackboard bold

\usepackage{homework}
%\usepackage{notes}
\usepackage{newpxtext,eulerpx,eucal}
\renewcommand*{\mathbb}[1]{\varmathbb{#1}}

\fancyhf{}
\fancyhead[R]{Avinash Iyer}
\fancyhead[L]{Complex Analysis: Assignment 12}
\fancyfoot[C]{\thepage}

\setcounter{secnumdepth}{0}

\begin{document}
\RaggedRight
\begin{problem}[Problem 1]
  Let $U\subseteq \C$ be a region. Fix $z_0\in U$. Let 
  \begin{align*}
    \mathcal{F} &= \set{f\in H(U) | \img\left( f \right)\subseteq \C\setminus B\left( 0,1 \right),f\left( z_0 \right) = 2i}.
  \end{align*}
  Show that $\mathcal{F}$ is normal.
\end{problem}
\begin{solution}
  Let $\left( f_{n} \right)_n$ be a sequence in $\mathcal{F}$. We use the conformal map $z\mapsto \frac{1}{z}$ to map $\C\setminus B\left( 0,1 \right)$ to $\D$, giving that the family
  \begin{align*}
    \mathcal{G} &= \set{\frac{1}{f} | f\in \mathcal{F}}
  \end{align*}
  is locally bounded (indeed, globally bounded) by $1$. Thus, it follows that there is a subsequence
  \begin{align*}
    \left( \frac{1}{f_{n_k}} \right)_{k} &\rightarrow g\colon U\rightarrow \D
  \end{align*}
  for some holomorphic function $g\colon U\rightarrow \D$. Now, since $\frac{1}{f_n}$ has no zeros for each $n$, it follows from Hurwitz's theorem that either $g$ is uniformly $0$ or $g$ also has no zeros. Yet, since $g\left( z_0 \right) = -\frac{i}{2}\neq 0$, it thus follows that $\frac{1}{g}$ is holomorphic on $U$, whence
  \begin{align*}
    \left( f_{n_k} \right)_k\rightarrow \frac{1}{g}.
  \end{align*}
  Thus, $\mathcal{F}$ is normal.
\end{solution}
\begin{problem}[Problem 2]\hfill
  \begin{enumerate}[(a)]
    \item Using the Schwarz--Pick lemma, show that given $w\in \D$, there exists a holomorphic function $f\colon \D\rightarrow \D$ satisfying
      \begin{align*}
        f(w) &= 0\\
        \left\vert f'(w) \right\vert &= \sup_{\substack{g\in H(\D)\\g(\D)\subseteq \D}}\left\vert g'(w) \right\vert.
      \end{align*}
    \item Show that if $f\colon \D\rightarrow \C$ is holomorphic and bounded, then
      \begin{align*}
        \sup_{z\in \D} \left( 1-\left\vert z \right\vert^2 \right)\left\vert f'(z) \right\vert &\leq \sup_{z\in \D} \left\vert f(z) \right\vert.
      \end{align*}
    \item Let $f\colon \D\rightarrow \D$ be a holomorphic function. Show that $f$ either has at most $1$ fixed point or $f$ is the identity.
  \end{enumerate}
\end{problem}
\begin{solution}\hfill
  \begin{enumerate}[(a)]
    \item We know that the map
      \begin{align*}
        \psi_w\left( z \right) &= \frac{w-z}{1- \overline{w}z}
      \end{align*}
      is a conformal map that takes $\psi_w(w) = 0$. Now, we know that
      \begin{align*}
        \left\vert \psi_w'(w) \right\vert &= \frac{1}{1-\left\vert w \right\vert^2}.
      \end{align*}
      From the Schwarz--Pick Lemma, we have for all holomorphic functions $f\colon \D\rightarrow \D$
      \begin{align*}
        \frac{\left\vert f'(w) \right\vert}{1-\left\vert f(w) \right\vert^2} &\leq \frac{1}{1-\left\vert w \right\vert^2}.
      \end{align*}
      In particular, since $0 \leq \left\vert f(w) \right\vert < 1$, we have
      \begin{align*}
        \left\vert f'(w) \right\vert &\leq \frac{1}{1-\left\vert w \right\vert^2},
      \end{align*}
      whence $\psi_w(z)$ satisfies
      \begin{align*}
        \psi_w(w) &= 0\\
        \left\vert \psi_w'(w) \right\vert &= \sup_{\substack{g\in H(\D) \\ g(\D)\subseteq \D}} \left\vert g'(w) \right\vert.
      \end{align*}
    \item Let $K = \sup_{z\in \D}\left\vert f(z) \right\vert$. By the maximum modulus principle, $\left\vert f(z) \right\vert < K$ for all $z\in \D$, so it follows that $g(z) \coloneq \frac{f(z)}{K}$ is a self-map of the unit disk. By the Schwarz--Pick lemma, it then follows that
      \begin{align*}
        \frac{\left\vert g'(z) \right\vert}{1-\left\vert g(z) \right\vert^2} &\leq \frac{1}{1-\left\vert z \right\vert^2}.
      \end{align*}
      Simplifying, we then get
      \begin{align*}
        \left( 1-\left\vert z \right\vert^2 \right) \left\vert f'(z) \right\vert &\leq K\left( 1-\frac{\left\vert f(z) \right\vert^2}{K^2} \right)\\
                                                                                 &\leq K,
      \end{align*}
      so that
      \begin{align*}
        \sup_{z\in \D} \left( 1-\left\vert z \right\vert^2 \right)\left\vert f'(z) \right\vert &\leq \sup_{z\in \D} \left\vert f(z) \right\vert.
      \end{align*}
    \item The statement is equivalent to showing that if $f\colon \D\rightarrow \D$ is a holomorphic self-map with two fixed points, then $f$ is the identity map. Let $f$ be one of these maps, and let $\xi\neq \eta\in \D$ be such that $f\left( \xi \right) = \xi$ and $f\left( \eta \right) = \eta$.\newline

      We want to find some holomorphic self-map of $\D$ that sends $0\mapsto 0$. We consider the maps
      \begin{align*}
        \psi_{\xi} &= \frac{\xi - z}{1- \overline{\xi}z},
      \end{align*}
      which takes $0\mapsto \xi$ and $\xi\mapsto 0$. Notice that $\psi_{\xi}\circ \psi_{\xi} = \id$. Therefore,
      \begin{align*}
        g &= \psi_{\xi}\circ f \circ \psi_{\xi}
      \end{align*}
      is a holomorphic self-map that sends $0\mapsto 0$, so by Schwarz's Lemma, we have
      \begin{align*}
        \left\vert g(z) \right\vert \leq \left\vert z \right\vert
      \end{align*}
      for all $z\in \D$. Yet, we also have
      \begin{align*}
        g\left( \psi_{\xi}\left( \eta \right) \right) &= \psi_{\xi}\circ f \circ \psi_{\xi}\circ \psi_{\xi}\left( \eta \right)\\
                                                      &= \psi_{\xi}\left( \eta \right).
      \end{align*}
      In particular, this means that
      \begin{align*}
        \left\vert g\left( \psi_{\xi}\left( \eta \right) \right) \right\vert &= \left\vert \psi_{\xi}\left( \eta \right) \right\vert,
      \end{align*}
      so there exists $\D\ni w\coloneq \psi_{\xi}\left( \eta \right)$ such that $\left\vert g\left( w \right) \right\vert= \left\vert w \right\vert$, so that $g(w) = e^{i\theta}w$. Yet, since the identity relation holds for $\psi_{\xi}\left( \eta \right)$, it follows that $\theta = 0$, so $g\left( w \right) = w$. In particular, this means
      \begin{align*}
        \psi_{\xi}\circ f \circ \psi_{\xi}(z) &= z\\
        f\circ \psi_{\xi}\left( z \right) &= \psi_{\xi}\left( z \right).
      \end{align*}
      Yet, since $\psi_{\xi}$ is an automorphism, it follows that this relation holds for all $z\in \D$, so that $f\left( w \right) = w$ for all $w\in \D$, whence $f = \id$.
  \end{enumerate}
\end{solution}
\begin{problem}[Problem 3]
  Let $f\colon \D\rightarrow \D$ be a holomorphic function with $f\left( 0 \right)= 0$.
  \begin{enumerate}[(a)]
    \item Show that $\left\vert f(z) + f\left( -z \right) \right\vert \leq 2\left\vert z \right\vert^2$ for all $z\in \D$.
    \item Show that $\left\vert f(z) + f\left( -z \right) \right\vert = 2\left\vert z \right\vert^2$ for some $z\in \D\setminus \set{0}$ if and only if $f(z) = e^{i\theta}z^2$.
  \end{enumerate}
\end{problem}
\begin{solution}\hfill
  \begin{enumerate}[(a)]
    \item We seek to show that the function
      \begin{align*}
        k(z) &= \frac{f\left( z \right) + f\left( -z \right)}{2z}
      \end{align*}
      maps $\D\setminus \set{0}\rightarrow \D\setminus \set{0}$. We may safely assume that $z\neq 0$, as the desired inequality is certainly true for $z = 0$. We observe that since $f$ is a self-map of $\D$ with $f(0) = 0$, Schwarz's Lemma gives
      \begin{align*}
        \left\vert f(z) \right\vert &\leq \left\vert z \right\vert,
      \end{align*}
      or that
      \begin{align*}
        \frac{\left\vert f(z) \right\vert}{\left\vert z \right\vert} &\leq 1
      \end{align*}
      A similar fact holds for $f\left( -z \right)$. For all $z\in \D$, we thus have
      \begin{align*}
        \left\vert \frac{f\left( z \right) + f\left( -z \right)}{2z} \right\vert &\leq \frac{1}{2} \left( \left\vert \frac{f(z)}{z} \right\vert + \left\vert \frac{f\left( -z \right)}{z} \right\vert \right)\\
                                                                                 &< 1.
      \end{align*}
      Therefore, since $k$ is a self-map of $\D$ with $k(0) = 0$, Schwarz's Lemma gives
      \begin{align*}
        \left\vert f(z) + f\left( -z \right) \right\vert &\leq 2\left\vert z \right\vert^2.
      \end{align*}
    \item Equivalently, we are assuming that
      \begin{align*}
        \left\vert \frac{f\left( z \right) + f\left( -z \right)}{2z} \right\vert &= \left\vert z \right\vert
      \end{align*}
      for some $z\in \D\setminus \set{0}$. From Schwarz's Lemma, we then have that
      \begin{align*}
        \frac{f(z) + f\left( -z \right)}{2z} &= e^{i\theta}z
      \end{align*}
      for some $\theta\in \R$. This gives
      \begin{align*}
        \frac{1}{2}\left( f\left( z \right) + f\left( -z \right) \right) &= e^{i\theta}z^2.
      \end{align*}
      Writing out the power series expansion for $f$, we get
      \begin{align*}
        \frac{1}{2}\left( \sum_{n=0}^{\infty}a_nz^{n} + \sum_{n=0}^{\infty}a_n\left( -1 \right)^{n}z^{n} \right) &= e^{i\theta}z^2\\
        \sum_{n=0}^{\infty}a_{2n}z^{2n} &= e^{i\theta}z^2.
      \end{align*}
      Thus, we may write
      \begin{align*}
        f(z) &= e^{i\theta}z^2 + \sum_{n=0}^{\infty} a_{2n+1}z^{2n+1}.
      \end{align*}
      I don't know where to go from here.
  \end{enumerate}
\end{solution}
\begin{problem}[Problem 4]\hfill
  \begin{enumerate}[(a)]
    \item Show that if $f\colon \mathbb{H}\rightarrow \D$ is a conformal map, then there exists some $\theta\in \R$ and $\beta\in \mathbb{H}$ such that
      \begin{align*}
        f(z) &= e^{i\theta z} \frac{z-\beta}{z - \overline{\beta}}.
      \end{align*}
    \item Show that if $f\colon \mathbb{H}\rightarrow \mathbb{H}$ is a conformal map, then there exists some
      \begin{align*}
        \begin{pmatrix}a & b \\ c & d\end{pmatrix} &\in \SL_2\left( \R \right)
      \end{align*}
      such that
      \begin{align*}
        f(z) &= \frac{az + b}{cz + d}.
      \end{align*}
  \end{enumerate}
\end{problem}
\begin{solution}\hfill
  \begin{enumerate}[(a)]
    \item Let $f\colon \mathbb{H}\rightarrow \D$ be a conformal map, and let $\beta = f^{-1}(0)$. It suffices to show that
      \begin{align*}
        g(z) &= \frac{z- \beta}{z- \overline{\beta}}
      \end{align*}
      is a conformal map from $ \mathbb{H} $ to $\D$ that takes $\beta\mapsto 0$. The essential uniqueness of conformal maps from simply connected domains to the unit disk will give us our desired result.\newline

      Letting $\beta = a + bi$ with $ b > 0 $, we observe that the expression of $g$ can be rewritten as
      \begin{align*}
        g(z) &= \frac{\left( \frac{z-a}{b} \right) - i}{ \left( \frac{z-a}{b} \right) + i }\\
             &= q\circ L\left( z \right),
      \end{align*}
      where $L\colon \mathbb{H}\rightarrow \mathbb{H}$ takes $z\mapsto \frac{z-a}{b}$ (and is a uniquely determined automorphism of $ \mathbb{H} $ that takes $a + bi$ to $i$), while $q\colon \mathbb{H}\rightarrow \D$ is the Cayley transform. In particular, this is a composition of conformal maps, hence conformal, maps $ \beta\mapsto 0 $, and has $\left\vert g'\left( \beta \right) \right\vert \neq 0$, meaning that it must be the case that a general conformal map from $ \mathbb{H} $ to $\D$ that maps $\beta\mapsto 0$ must be of the form
      \begin{align*}
        f &= e^{i\theta} g(z).
      \end{align*}
    \item We start by showing that all conformal maps $f\colon \mathbb{H}\rightarrow \mathbb{H}$ that fix $i$ can be expressed by fractional linear transformations from matrices in $\SL_2\left( \R \right)$. We observe then that $q\circ f\colon \mathbb{H}\rightarrow \D$, where $q$ is the Cayley Transform, is necessarily of the form
      \begin{align*}
        q\circ f &= e^{i\theta}\frac{z-i}{z+i},
      \end{align*}
      following from the uniqueness we showed in part (a). This gives
      \begin{align*}
        f &= -i\frac{z\left( 1 + e^{i\theta} \right) + i\left( 1-e^{i\theta} \right)}{z\left( e^{i\theta}-1 \right) - i\left( 1 + e^{i\theta} \right)}\\
          &= -i\frac{e^{i\theta/2}\left( z\left( 2\cos\left( \theta/2 \right) \right) + 2\sin\left( \theta/2 \right) \right)}{e^{i\theta/2}\left( z\left( 2i\sin\left( \theta/2 \right) \right) - 2i\cos\left( \theta/2 \right) \right)}\\
          &= \frac{z\cos\left( \theta/2 \right) + \sin\left( \theta/2 \right)}{-z\sin\left( \theta/2 \right) + \cos\left( \theta/2 \right)}.
      \end{align*}
      Since the rotation map
      \begin{align*}
        \begin{pmatrix}\cos\left( \theta/2 \right) & \sin\left( \theta/2 \right) \\ -\sin\left( \theta/2 \right) & \cos\left( \theta/2 \right)\end{pmatrix} &\in \SL_2\left( \R \right),
      \end{align*}
      it follows that any conformal map of $ \mathbb{H} $ that fixes $i$ can be expressed in this form.\newline

      In the general case, we observe that we can translate an arbitrary element of the form $z = a + bi$ with $b > 0$ to $i$ by taking $L = \frac{z-a}{b}$, which admits a representation as an element of $\SL_2\left( \R \right)$ via the matrix
      \begin{align*}
        \begin{pmatrix}1/\sqrt{b} & -a/\sqrt{b} \\ 0 & \sqrt{b}\end{pmatrix} &\in \SL_2\left( \R \right).
      \end{align*}
      In particular, if $f\colon \mathbb{H}\rightarrow \mathbb{H}$ is a conformal map, then there is a unique element $a + bi$ that maps to $i$, meaning that we may necessarily write $f$ as
      \begin{align*}
        f &= \begin{pmatrix}\cos\left( \theta/2 \right) & \sin\left( \theta/2 \right) \\ -\sin\left( \theta/2 \right) & \cos\left( \theta/2 \right)\end{pmatrix} \begin{pmatrix}1/\sqrt{b} & -a/\sqrt{b} \\ 0 & \sqrt{b}\end{pmatrix}\cdot z,
      \end{align*}
      whence any conformal map from $ \mathbb{H} $ to $ \mathbb{H} $ can be expressed in this fashion.
  \end{enumerate}
\end{solution}
\begin{problem}[Problem 5]
  Let $f$ be an entire function satisfying $\left\vert f(z) \right\vert = 1$ for all $z\in S^{1}$. Show that there exists some $\theta\in \R$ and a nonnegative integer $n$ such that $f(z) = e^{i\theta}z^{n}$ for all $z\in \C$.
\end{problem}
\begin{solution}
  To start, if $f$ is constant, then it follows that $\left\vert f(z) \right\vert = 1$ for all $z\in \C$, meaning that $f(z) = e^{i\theta}$ for some $\theta$.\newline

  Now, let $f$ be nonconstant. We claim that $\inf_{z\in B\left( 0,1 \right)}\left\vert f(z) \right\vert = 0$. If it were not the case, then by applying the maximum modulus principle to both $f$ and $1/f$, we would reach a contradiction claiming that $f = e^{i\theta}$ on $\D$, hence on all of $\C$ by the identity theorem, contradicting the fact that $f$ is nonconstant.\newline

  Since all the zeros of $f$ are isolated, we have finitely many contained in $B\left( 0,1 \right)$, hence finitely many in $\D$. Call these zeros $\set{z_j}_{j=1}^{n}$ (with multiplicity). Let
  \begin{align*}
    B(z) &= \prod_{j=1}^{n} \frac{z_j-z}{1- \overline{z_j}z}.
  \end{align*}
  Since $B$ is a Blaschke product, it follows that $\left\vert B(z) \right\vert = 1$ on $S^{1}$. Furthermore, evaluating
  \begin{align*}
    \lim_{z\rightarrow z_j} \frac{f(z)}{B(z)} &= \lim_{z\rightarrow z_j} \frac{\left( z-z_j \right)^{k}g(z)}{\left( -1 \right)^{k}\left( z-z_j \right)^{k}H(z)}\\
                                              &= \left( -1 \right)^{k} \frac{g\left( z_j \right)}{H\left( z_j \right)},
  \end{align*}
  where $g$ and $H$ are holomorphic functions that are nonzero on $\D$. In particular, this means that the function $\frac{f}{B}$ has no zeros in $\D$ and $\left\vert \frac{f}{B} \right\vert = 1$. Therefore, by the reasoning above, we must have that 
  \begin{align*}
    f(z) &= e^{i\theta}B(z).
  \end{align*}
  The only Blaschke factors that are holomorphic on $\C$ are the ones of the form $z^{n}$, meaning that we have $f(z) = e^{i\theta}z^{n}$.
\end{solution}
\begin{problem}[Problem 6]\hfill
  \begin{enumerate}[(a)]
    \item Show that there does not exist a continuous function $f\colon \overline{ \mathbb{H} }\rightarrow \C$ for which $f|_{ \mathbb{H} }$ is holomorphic, $f\left( \R \right)\subseteq \left( -\infty,0 \right)$, and $f\left( \mathbb{H} \right)\subseteq \mathbb{H}$.
    \item Let $f\colon \overline{ \mathbb{H} }\rightarrow \C$ be a continuous function for which $f|_{ \mathbb{H} }$ is holomorphic, $f\left( \R \right)\subseteq \R$, and $ 0 \leq \im\left( f(z) \right)\leq \re\left( f(z) \right) $ for all $ z\in \overline{ \mathbb{H} } $. Show that $f$ is constant.
  \end{enumerate}
\end{problem}
\begin{solution}\hfill
  \begin{enumerate}[(a)]
    \item We extend $f$ to a holomorphic function on all of $\C$ using the Schwarz reflection principle. Observe then that the range of the extension for $f$ (which we call $g$) is contained in $ \C\setminus \left[ 0,\infty \right) $. We may thus define a branch of the square root that maps $\C\setminus \left[ 0,\infty \right)$ to the upper half-plane, so that
      \begin{align*}
        v(z) &= \frac{\sqrt{g(z)} - i}{\sqrt{g(z)} + i}
      \end{align*}
      is an entire function whose range is contained within $ \D $. Thus, it follows that $g$ (and thus $f$) is constant by Liouville's Theorem. Yet, this would lead to a contradiction, since the condition that $f\left( \R \right)\subseteq \left( 0,\infty \right)$ would imply that the constant value for $f$ is some element of $\R$, while the condition that $f\left( \mathbb{H} \right)\subseteq \mathbb{H}$ would imply that the constant value for $f$ is an element of $ \mathbb{H} $, which cannot be in $ \R $.
    \item Using the Schwarz reflection principle, we extend $f$ to be holomorphic on $\C$; we call this extension $g$. To understand the behavior of $g$ with respect to the alternative condition on $f$, we observe that on the lower half-plane, we have that $g\left( z \right) = \overline{f\left( \overline{z} \right)}$; in particular, we have
      \begin{align*}
        \im\left( \overline{f\left( \overline{z} \right)} \right)\leq 0 \leq \re\left( \overline{f\left( \overline{z} \right)} \right),
      \end{align*}
      so we have that $g$ maps $\C$ to the right half-plane. Thus, we have that $U\left( -1,1/2 \right)\nsubseteq g\left( \C \right)$, meaning that by the converse to a corollary of Liouville's Theorem, we have that $g$ is constant. Since $g$ is a holomorphic extension of $f$, it follows that $f$ is thus constant.
  \end{enumerate}
\end{solution}
\end{document}
