\documentclass[10pt]{mypackage}

% sans serif font:
%\usepackage{cmbright}
%\usepackage{sfmath}
%\usepackage{bbold} %better blackboard bold

\usepackage{homework}
%\usepackage{notes}
\usepackage{newpxtext,eulerpx,eucal}
\renewcommand*{\mathbb}[1]{\varmathbb{#1}}

\fancyhf{}
\fancyhead[R]{Avinash Iyer}
\fancyhead[L]{Complex Analysis: Assignment 12}
\fancyfoot[C]{\thepage}

\setcounter{secnumdepth}{0}

\begin{document}
\RaggedRight
\begin{problem}[Problem 1]
  Let $U\subseteq \C$ be a region. Fix $z_0\in U$. Let 
  \begin{align*}
    \mathcal{F} &= \set{f\in H(U) | \img\left( f \right)\subseteq \C\setminus B\left( 0,1 \right),f\left( z_0 \right) = 2i}.
  \end{align*}
  Show that $\mathcal{F}$ is normal.
\end{problem}
\begin{solution}
  Let $\left( f_{n} \right)_n$ be a sequence in $\mathcal{F}$. We use the conformal map $z\mapsto \frac{1}{z}$ to map $\C\setminus B\left( 0,1 \right)$ to $\D$, giving that the family
  \begin{align*}
    \mathcal{G} &= \set{\frac{1}{f} | f\in \mathcal{F}}
  \end{align*}
  is locally bounded (indeed, globally bounded) by $1$. Thus, it follows that there is a subsequence
  \begin{align*}
    \left( \frac{1}{f_{n_k}} \right)_{k} &\rightarrow g\colon U\rightarrow \D
  \end{align*}
  for some holomorphic function $g\colon U\rightarrow \D$. Now, since $\frac{1}{f_n}$ has no zeros for each $n$, it follows from Hurwitz's theorem that either $g$ is uniformly $0$ or $g$ also has no zeros. Yet, since $g\left( z_0 \right) = -\frac{i}{2}\neq 0$, it thus follows that $\frac{1}{g}$ is holomorphic on $U$, whence
  \begin{align*}
    \left( f_{n_k} \right)_k\rightarrow \frac{1}{g}.
  \end{align*}
  Thus, $\mathcal{F}$ is normal.
\end{solution}
\begin{problem}[Problem 2]\hfill
  \begin{enumerate}[(a)]
    \item Using the Schwarz--Pick lemma, show that given $w\in \D$, there exists a holomorphic function $f\colon \D\rightarrow \D$ satisfying
      \begin{align*}
        f(w) &= 0\\
        \left\vert f'(w) \right\vert &= \sup_{\substack{g\in H(\D)\\g(\D)\subseteq \D}}\left\vert g'(w) \right\vert.
      \end{align*}
    \item Show that if $f\colon \D\rightarrow \C$ is holomorphic and bounded, then
      \begin{align*}
        \sup_{z\in \D} \left( 1-\left\vert z \right\vert^2 \right)\left\vert f'(z) \right\vert &\leq \sup_{z\in \D} \left\vert f(z) \right\vert.
      \end{align*}
    \item Let $f\colon \D\rightarrow \D$ be a holomorphic function. Show that $f$ either has at most $1$ fixed point or $f$ is the identity.
  \end{enumerate}
\end{problem}
\begin{solution}\hfill
  \begin{enumerate}[(a)]
    \item We know that the map
      \begin{align*}
        \psi_w\left( z \right) &= \frac{w-z}{1- \overline{w}z}
      \end{align*}
      is a conformal map that takes $\psi_w(w) = 0$. Now, we know that
      \begin{align*}
        \left\vert \psi_w'(w) \right\vert &= \frac{1}{1-\left\vert w \right\vert^2}.
      \end{align*}
      From the Schwarz--Pick Lemma, we have for all holomorphic functions $f\colon \D\rightarrow \D$
      \begin{align*}
        \frac{\left\vert f'(w) \right\vert}{1-\left\vert f(w) \right\vert^2} &\leq \frac{1}{1-\left\vert w \right\vert^2}.
      \end{align*}
      In particular, since $0 \leq \left\vert f(w) \right\vert < 1$, we have
      \begin{align*}
        \left\vert f'(w) \right\vert &\leq \frac{1}{1-\left\vert w \right\vert^2},
      \end{align*}
      whence $\psi_w(z)$ satisfies
      \begin{align*}
        \psi_w(w) &= 0\\
        \left\vert \psi_w'(w) \right\vert &= \sup_{\substack{g\in H(\D) \\ g(\D)\subseteq \D}} \left\vert g'(w) \right\vert.
      \end{align*}
    \item Let $K = \sup_{z\in \D}\left\vert f(z) \right\vert$. By the maximum modulus principle, $\left\vert f(z) \right\vert < K$ for all $z\in \D$, so it follows that $g(z) \coloneq \frac{f(z)}{K}$ is a self-map of the unit disk. By the Schwarz--Pick lemma, it then follows that
      \begin{align*}
        \frac{\left\vert g'(z) \right\vert}{1-\left\vert g(z) \right\vert^2} &\leq \frac{1}{1-\left\vert z \right\vert^2}.
      \end{align*}
      Simplifying, we then get
      \begin{align*}
        \left( 1-\left\vert z \right\vert^2 \right) \left\vert f'(z) \right\vert &\leq K\left( 1-\frac{\left\vert f(z) \right\vert^2}{K^2} \right)\\
                                                                                 &\leq K,
      \end{align*}
      so that
      \begin{align*}
        \sup_{z\in \D} \left( 1-\left\vert z \right\vert^2 \right)\left\vert f'(z) \right\vert &\leq \sup_{z\in \D} \left\vert f(z) \right\vert.
      \end{align*}
    \item The statement is equivalent to showing that if $f\colon \D\rightarrow \D$ is a holomorphic self-map with two fixed points, then $f$ is the identity map. Let $f$ be one of these maps, and let $\xi\neq \eta\in \D$ be such that $f\left( \xi \right) = \xi$ and $f\left( \eta \right) = \eta$.\newline

      We want to find some holomorphic self-map of $\D$ that sends $0\mapsto 0$. We consider the maps
      \begin{align*}
        \psi_{\xi} &= \frac{\xi - z}{1- \overline{\xi}z},
      \end{align*}
      which takes $0\mapsto \xi$ and $\xi\mapsto 0$. Notice that $\psi_{\xi}\circ \psi_{\xi} = \id$. Therefore,
      \begin{align*}
        g &= \psi_{\xi}\circ f \circ \psi_{\xi}
      \end{align*}
      is a holomorphic self-map that sends $0\mapsto 0$, so by Schwarz's Lemma, we have
      \begin{align*}
        \left\vert g(z) \right\vert \leq \left\vert z \right\vert
      \end{align*}
      for all $z\in \D$. Yet, we also have
      \begin{align*}
        g\left( \psi_{\xi}\left( \eta \right) \right) &= \psi_{\xi}\circ f \circ \psi_{\xi}\circ \psi_{\xi}\left( \eta \right)\\
                                                      &= \psi_{\xi}\left( \eta \right).
      \end{align*}
      In particular, this means that
      \begin{align*}
        \left\vert g\left( \psi_{\xi}\left( \eta \right) \right) \right\vert &= \left\vert \psi_{\xi}\left( \eta \right) \right\vert,
      \end{align*}
      so there exists $\D\ni w\coloneq \psi_{\xi}\left( \eta \right)$ such that $\left\vert g\left( w \right) \right\vert= \left\vert w \right\vert$, so that $g(w) = e^{i\theta}w$. Yet, since the identity relation holds for $\psi_{\xi}\left( \eta \right)$, it follows that $\theta = 0$, so $g\left( w \right) = w$. In particular, this means
      \begin{align*}
        \psi_{\xi}\circ f \circ \psi_{\xi}(z) &= z\\
        f\circ \psi_{\xi}\left( z \right) &= \psi_{\xi}\left( z \right).
      \end{align*}
      Yet, since $\psi_{\xi}$ is an automorphism, it follows that this relation holds for all $z\in \D$, so that $f\left( w \right) = w$ for all $w\in \D$, whence $f = \id$.
  \end{enumerate}
\end{solution}
\begin{problem}[Problem 3]
  Let $f\colon \D\rightarrow \D$ be a holomorphic function with $f\left( 0 \right)= 0$.
  \begin{enumerate}[(a)]
    \item Show that $\left\vert f(z) + f\left( -z \right) \right\vert \leq 2\left\vert z \right\vert^2$ for all $z\in \D$.
    \item Show that $\left\vert f(z) + f\left( -z \right) \right\vert = 2\left\vert z \right\vert^2$ for some $z\in \D\setminus \set{0}$ if and only if $f(z) = e^{i\theta}z^2$.
  \end{enumerate}
\end{problem}
\begin{solution}\hfill
  \begin{enumerate}[(a)]
    \item We seek to show that the function
      \begin{align*}
        k(z) &= \frac{f\left( z \right) + f\left( -z \right)}{2z}
      \end{align*}
      maps $\D\setminus \set{0}\rightarrow \D\setminus \set{0}$. We may safely assume that $z\neq 0$, as the desired inequality is certainly true for $z = 0$. We observe that since $f$ is a self-map of $\D$ with $f(0) = 0$, Schwarz's Lemma gives
      \begin{align*}
        \left\vert f(z) \right\vert &\leq \left\vert z \right\vert,
      \end{align*}
      or that
      \begin{align*}
        \frac{\left\vert f(z) \right\vert}{\left\vert z \right\vert} &\leq 1
      \end{align*}
      A similar fact holds for $f\left( -z \right)$. For all $z\in \D$, we thus have
      \begin{align*}
        \left\vert \frac{f\left( z \right) + f\left( -z \right)}{2z} \right\vert &\leq \frac{1}{2} \left( \left\vert \frac{f(z)}{z} \right\vert + \left\vert \frac{f\left( -z \right)}{z} \right\vert \right)\\
                                                                                 &< 1.
      \end{align*}
      Therefore, since $k$ is a self-map of $\D$ with $k(0) = 0$, Schwarz's Lemma gives
      \begin{align*}
        \left\vert f(z) + f\left( -z \right) \right\vert &\leq 2\left\vert z \right\vert^2.
      \end{align*}
    \item Equivalently, we are assuming that
      \begin{align*}
        \left\vert \frac{f\left( z \right) + f\left( -z \right)}{2z} \right\vert &= \left\vert z \right\vert
      \end{align*}
      for some $z\in \D\setminus \set{0}$. From Schwarz's Lemma, we then have that
      \begin{align*}
        \frac{f(z) + f\left( -z \right)}{2z} &= e^{i\theta}z
      \end{align*}
      for some $\theta\in \R$. This gives
      \begin{align*}
        \frac{1}{2}\left( f\left( z \right) + f\left( -z \right) \right) &= e^{i\theta}z^2.
      \end{align*}
      Now, we observe that
      \begin{align*}
        f(z) &= \frac{1}{2}\left( f(z) + f\left( -z \right) \right) + \frac{1}{2}\left( f\left( z \right) - f\left( -z \right) \right).
      \end{align*}
      First, we observe that 
      \begin{align*}
        h(z) &= \frac{1}{2}\left( f(z) - f\left( -z \right) \right)
      \end{align*}
      has $\left\vert h(z) \right\vert < 1$ for all $z\in \D$, $h(0) = 0$, and
      \begin{align*}
        \left\vert h'(0) \right\vert &= \left\vert f'(0) \right\vert,
      \end{align*}
      meaning that there is $\rho$ such that $\frac{1}{2}\left( f(z) - f\left( -z \right) \right) = e^{i\rho} f(z)$, by a corollary to the Riemann Mapping Theorem and Schwarz's Lemma.
  \end{enumerate}
\end{solution}
\end{document}
