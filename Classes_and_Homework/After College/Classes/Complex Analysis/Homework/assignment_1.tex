\documentclass[10pt]{mypackage}

% sans serif font:
%\usepackage{cmbright}
%\usepackage{sfmath}
%\usepackage{bbold} %better blackboard bold

\usepackage{homework}
%\usepackage{notes}
\usepackage{newpxtext,eulerpx,eucal}
\renewcommand*{\mathbb}[1]{\varmathbb{#1}}

\fancyhf{}
\rhead{Avinash Iyer}
\lhead{Complex Analysis: Assignment 1}

\setcounter{secnumdepth}{0}

\begin{document}
\RaggedRight
\begin{problem}[Problem 1]
  Given $z = x + iy\in \C$, define
  \begin{align*}
    z^{\ast} &= \left( \frac{2x}{x^2 + y^2 + 1}, \frac{2y}{x^2 + y^2 + 1}, \frac{x^2 + y^2 - 1}{x^2 + y^2 + 1} \right).
  \end{align*}
  \begin{enumerate}[(a)]
    \item Show that $z^{\ast}\in S^2$.
    \item Prove that if $\left( x_1,x_2,x_3 \right)\in S^2\setminus \set{(0,0,1)}$, then there exists a unique $z\in \C$ such that $z^{\ast} = \left( x_1,x_2,x_3 \right)$.
    \item A circle in $S^2$ is the intersection of a plane in $\R^3$ with $S^2$, provided this intersection is nonempty. Prove that if $C$ is a circle in $S^{2}$, then there exists a set $\widetilde{C}\subseteq \C$ that is either a circle or a straight line such that $C\setminus \set{(0,0,1)} = \set{z^{\ast}\in \R^3 | z\in \widetilde{C}}$.
  \end{enumerate}
\end{problem}
\begin{solution}\hfill
  \begin{enumerate}[(a)]
    \item Via brute force calculation, we see that
      \begin{align*}
        \frac{4x^2}{\left( x^2 + y^2 + 1 \right)^2} + \frac{4y^2}{\left( x^2 + y^2 + 1 \right)^2} + \frac{\left( x^2 + y^2 - 1 \right)^2}{\left( x^2 + y^2 + 1 \right)^2} &= \frac{\left( x^2 + y^2 \right)^1 + 1 - 2\left( x^2 + y^2 \right) + 4\left( x^2 + y^2 \right)}{\left( x^2 + y^2 + 1 \right)^2}\\
                                                                                                                                                                                                  &= \frac{\left( x^2 + y^2 \right)^1 + 1 + 2\left( x^2 + y^2 \right)}{\left( x^2 + y^2 + 1 \right)^2}\\
                                                                                                                                                                                                                                                                                                                                                                                                                                                                                                                                                                                   &= 1.
      \end{align*}
    \item Let $z^{\ast} = \left( x_1,x_2,x_3 \right)\in S^{2}\setminus \set{(0,0,1)}$, and let $L\colon [0,\infty)\rightarrow \R^3$ be the line parametrized such that $L(1) = \left( x_1,x_2,x_3 \right)$ and $L(0) = \left( 0,0,1 \right)$, which is given by
      \begin{align*}
        L(t) &= \left( tx_1,tx_2, tx_3 + \left( 1-t \right) \right).
      \end{align*}
      Note then that $\norm{L(t)} = 1$ only when $t = 0$ or $t = 1$, meaning that $L(t)$ intersects $S^{2}\setminus \set{(0,0,1)}$ exactly once. By identifying $\C$ with $x + iy \mapsto \left( x,y,0 \right)$, we may find $z\in \C$ that uniquely maps to $\left( x_1,x_2,x_3 \right)$ under the $z^{\ast}$ identification by taking
      \begin{align*}
        tx_3 + \left( 1-t \right) &= 0\\
        1 + t\left( x_3 - 1 \right) &= 0\\
        t &= \frac{1}{1-x_3},
      \end{align*}
      so that
      \begin{align*}
        x+iy &= \frac{x_1}{1-x_3} + i\frac{x_2}{1-x_3}
      \end{align*}
      maps to $z^{\ast}$ under the given identification.
    \item Let $\left( x_1,x_2,x_3 \right)\in S^{2}$ lie on the plane $ax_1 + bx_2 + cx_3 = d$. By substituting $z = x+ iy \mapsto z^{\ast}$, we get
      \begin{align*}
        a \frac{2x}{x^2 + y^2 + 1} + b\frac{2y}{x^2 + y^2 + 1} + c \frac{x^2 + y^2 - 1}{x^2 + y^2 + 1} &= d\\
        2ax + 2by + c\left( x^2 + y^2 - 1 \right) &= d\left( x^2 + y^2 + 1 \right)\\
        \left( c-d \right)x^2 + 2ax + \left( c-d \right)y^2 + 2by &= c + d.
      \end{align*}
      This gives two cases. If $c = d$, then we get the line
      \begin{align*}
        ax + by &= c.
      \end{align*}
      Else, if $c \neq d$, we get the circle
      \begin{align*}
        x^2 + \frac{2a}{c-d}x + y^2 + \frac{2b}{c-d}y &= \frac{c+d}{c-d}\\
        \left( x-\frac{a}{c-d} \right)^2 + \left( y - \frac{b}{c-d} \right)^2 &= \frac{a^2 + b^2 + c^2 - d^2}{\left( c-d \right)^2}.
      \end{align*}
      Thus, circles in $S^{2}$ correspond to either circles or lines in $\C$.
  \end{enumerate}
\end{solution}
\begin{problem}[Problem 2]
  Define $f\colon \C\setminus \set{1}\rightarrow \C$ by $f(z) = \left( \frac{z+1}{z-1} \right)^2$.
  \begin{enumerate}[(a)]
    \item Is $f$ injective on $ \mathbb{D} $? Why or why not?
    \item Determine $f\left( \mathbb{D} \right)$.
  \end{enumerate}
\end{problem}
\begin{solution}\hfill
  \begin{enumerate}[(a)]
    \item We consider $q(z) = \frac{z+1}{z-1}$ as a fractional linear transformation on $\hat{\C}$. We see that
      \begin{align*}
        q\left( e^{i\theta} \right) &= \frac{e^{i\theta} + 1}{e^{i\theta} - 1}\\
                                    &= \frac{\left( 1+\cos\left( \theta \right) \right) + i\sin\left( \theta \right)}{\left( \cos\left( \theta \right) - 1 \right) + i\sin\left( \theta \right)}\\
                                    &= \frac{\left( \left( \cos\left( \theta \right) +1 \right) + i\sin\left( \theta \right) \right)\left( \left( \cos\left( \theta \right)-1 \right) - i\sin\left( \theta \right)  \right)}{\left( 1-\cos\left( \theta \right) \right)^2 + \sin^2\left( \theta \right)}\\
                                    &= \frac{\left( \cos^2\left( \theta \right) - 1 \right) + \sin^2\left( \theta \right) + i\sin\left( \theta \right)\left( \cos\left( \theta \right) - 1 - \left( \cos\left( \theta \right) + 1 \right) \right)}{2-2\cos\left( \theta \right)}\\
                                    &= i\frac{\sin\left( \theta \right)}{\cos\left( \theta \right)-1},
      \end{align*}
      and since $\frac{\sin\left( \theta \right)}{\cos\left( \theta \right)-1}$ maps $(0,2\pi) $ to $ \R$ bijectively, we see that $q$ maps $S^{1}\setminus \set{1}$ into the imaginary axis. We also see that $q(0) = -1$, so $q$ maps $\mathbb{D}$ bijectively onto the left half-plane, $ \mathbb{L} = \set{z | \re(z) < 0} $.\newline

      Now, notice that the function $h(z) = z^2$ is injective when defined on a half-plane, as the arguments $\left( \pi/2,3\pi/2 \right)$ map injectively to $\left( \pi,3\pi \right)$, and the function $\left\vert z \right\vert^2$ is clearly injective on $(0,\infty)$, so $f = h\circ q$ is injective on $ \mathbb{D} $.
    \item Since $f = h\circ q$, where $q$ maps $ \mathbb{D} $ to the left half-plane, and $h$ maps the left half-plane to the full complex plane save for $\left( -\infty,0 \right]$, we have that $f$ maps $\C$ to $\C\setminus (-\infty,0]$.
  \end{enumerate}
\end{solution}
\begin{problem}[Problem 3]
  Prove that there exists a linear fractional transformation that maps the first quadrant in $\C$ bijectively to the top half of the unit disc, and satisfies $f(2) = i$.
\end{problem}
\begin{solution}
  We start from the Cayley transform,
  \begin{align*}
    f_1(z) &= \frac{z-i}{z+i},
  \end{align*}
  which bijectively maps the upper half-plane to the unit disc. Yet, by taking $z = x + iy$ for $ x,y > 0$, we see that
  \begin{align*}
    f_1\left( x+iy \right) &= \frac{1}{x^2 + \left( y+1 \right)^2}\left( \left( x^2 + y^2 - 1 \right) + i\left( -2x \right) \right),
  \end{align*}
  implying that the first quadrant is mapped to the \textit{lower} half of the unit disc. Therefore, we flip about the origin by taking $f_2(z) = -f_1(z)$, so that
  \begin{align*}
    f_2(z) &= -\frac{z-i}{z+i},
  \end{align*}
  which maps the first quadrant of the upper half plane to the top half of the unit disc. Next, we see that
  \begin{align*}
    f_2(1) &= -\frac{1-i}{1+i}\\
           &= i,
  \end{align*}
  so to ensure that $f(2) = i$, we may define $f(z) = f_2(z/2)$, or
  \begin{align*}
    f(z) &= -\frac{z-2i}{z+2i}.
  \end{align*}
\end{solution}
\begin{problem}[Problem 4]
  Let $f\colon \C\rightarrow \C$ be a function. We say that $\lim_{z\rightarrow\infty}f(z) = \infty$ if, for all $M > 0$, there exists $R > 0$ such that $\left\vert f(z)  \right\vert > M$ whenever $\left\vert z \right\vert > R$.
  \begin{enumerate}[(a)]
    \item Show that if $f\colon \C\rightarrow \C$ is a nonconstant polynomial, then $\lim_{z\rightarrow\infty}f(z) = \infty$.
    \item Suppose that $f\colon \C\rightarrow \C$ is a continuous function satisfying $\lim_{z\rightarrow\infty}f(z) = \infty$. Show that there exists some $z_0\in \C$ for which $\left\vert f(z_0) \right\vert = \inf_{z\in \C} \left\vert f(z) \right\vert$.
  \end{enumerate}
\end{problem}
\begin{solution}\hfill
  \begin{enumerate}[(a)]
    \item If $f(z) = \sum_{k=0}^{n} a_kz^{k}$, with $n > 1$ and $a_n \neq 0$, then by a corollary of the triangle inequality, we see that
      \begin{align*}
        \left\vert f(z) \right\vert &= \left\vert \sum_{k=0}^{n}a_kz^{k} \right\vert\\
                                    &\geq \left\vert a_nz^{n} \right\vert - \sum_{k=0}^{n-1}\left\vert a_kz^{k} \right\vert.
      \end{align*}
      Now, we notice a few things. First, since $\left\vert a_n \right\vert$ is nonzero, we may divide by $\left\vert a_n \right\vert$, giving
      \begin{align*}
        \frac{1}{\left\vert a_n \right\vert} \left\vert f(z) \right\vert &\geq \left\vert z \right\vert^{n} - \frac{1}{\left\vert a_n \right\vert} \sum_{k=0}^{n-1} \left\vert a_k \right\vert \left\vert z \right\vert^{k}.
      \end{align*}
      Now, from real analysis, we know that
      \begin{align*}
        \lim_{|z|\rightarrow\infty} |z|^{n} &= \infty,
      \end{align*}
      as we may select $R = M^{1/n}$ to achieve this purpose. So, by using the limit comparison test, we see that
      \begin{align*}
        \lim_{|z|\rightarrow\infty} \frac{\left\vert z \right\vert^{n} - \sum_{k=0}^{n-1} \left\vert a_k/a_n \right\vert \left\vert z \right\vert^{k}}{\left\vert z \right\vert^{n}} &= 1,
      \end{align*}
      so
      \begin{align*}
        \lim_{\left\vert z \right\vert\rightarrow\infty} \frac{1}{\left\vert a_n \right\vert} \left\vert f(z) \right\vert &= \infty,
      \end{align*}
      so
      \begin{align*}
        \lim_{z\rightarrow\infty} \left\vert f(z) \right\vert &= \infty.
      \end{align*}
    \item Let $M > 0$ be sufficiently large such that the set $\set{z\in \C | \left\vert f(z) \right\vert \leq M}$ is not empty. Since $\lim_{z\rightarrow\infty} f(z) = \infty$, there exists $R$ such that $\left\vert f(z) \right\vert > M$ whenever $\left\vert z \right\vert > R$.\newline

      We see that on $B\left( 0,R \right)$, the closed disk of radius $R$ centered at $0$, the function $f$ is continuous, and so is the function $\left\vert f(z) \right\vert$, as the modulus is also a continuous function. Since $B(0,R)$ is compact, there is some $z_0\in B\left( 0,R \right)$ such that $\left\vert f\left(z_0\right) \right\vert = \inf_{z_0\in B\left( 0,R \right)}\left\vert f(z) \right\vert$. In particular, we note that $\left\vert f\left(z_0\right) \right\vert \leq M$, as we have specifically selected $M$ to be such that $\set{z\in \C | \left\vert f(z) \right\vert \leq M}$ is nonempty, meaning that $\left\vert f\left( z_0 \right) \right\vert = \inf_{z\in \C} \left\vert f(z) \right\vert$, as we have selected $R$ such that $ \left\vert f(z) \right\vert > M $ for all $z \in \C\setminus B\left( 0,R \right)$.
  \end{enumerate}
\end{solution}
\end{document}
