\documentclass[10pt]{mypackage}

% sans serif font:
%\usepackage{cmbright}
%\usepackage{sfmath}
%\usepackage{bbold} %better blackboard bold

\usepackage{homework}
%\usepackage{notes}
\usepackage{newpxtext,eulerpx,eucal}
\renewcommand*{\mathbb}[1]{\varmathbb{#1}}

\fancyhf{}
\rhead{Avinash Iyer}
\lhead{Complex Analysis: Assignment 1}

\setcounter{secnumdepth}{0}

\begin{document}
\RaggedRight
\begin{problem}[Problem 2]
  Define $f\colon \C\setminus \set{1}\rightarrow \C$ by $f(z) = \left( \frac{z+1}{z-1} \right)^2$.
  \begin{enumerate}[(a)]
    \item Is $f$ injective on $ \mathbb{D} $? Why or why not?
    \item Determine $f\left( \mathbb{D} \right)$.
  \end{enumerate}
\end{problem}
\begin{solution}\hfill
  \begin{enumerate}[(a)]
    \item We consider $q(z) = \frac{z+1}{z-1}$ as a fractional linear transformation on $\hat{\C}$. We see that
      \begin{align*}
        q\left( e^{i\theta} \right) &= \frac{e^{i\theta} + 1}{e^{i\theta} - 1}\\
                                    &= \frac{\left( 1+\cos\left( \theta \right) \right) + i\sin\left( \theta \right)}{\left( \cos\left( \theta \right) - 1 \right) + i\sin\left( \theta \right)}\\
                                    &= \frac{\left( \left( \cos\left( \theta \right) +1 \right) + i\sin\left( \theta \right) \right)\left( \left( \cos\left( \theta \right)-1 \right) - i\sin\left( \theta \right)  \right)}{\left( 1-\cos\left( \theta \right) \right)^2 + \sin^2\left( \theta \right)}\\
                                    &= \frac{\left( \cos^2\left( \theta \right) - 1 \right) + \sin^2\left( \theta \right) + i\sin\left( \theta \right)\left( \cos\left( \theta \right) - 1 - \left( \cos\left( \theta \right) + 1 \right) \right)}{2-2\cos\left( \theta \right)}\\
                                    &= i\frac{\sin\left( \theta \right)}{\cos\left( \theta \right)-1},
      \end{align*}
      and since $\frac{\sin\left( \theta \right)}{\cos\left( \theta \right)-1}$ maps $(0,2\pi)\rightarrow \R$ bijectively, we see that $q$ maps the unit circle into the imaginary axis. We also see that $q(0) = -1$, so $\mathbb{D}$ maps $\mathbb{D}$ bijectively onto the left half-plane, $ \mathbb{L} = \set{z | \re(z) < 0} $.\newline

      Now, notice that the function $h(z) = z^2$ is injective when defined on a half-plane (the arguments $\left( \pi/2,3\pi/2 \right)$ map injectively to $\left( \pi,3\pi \right)$, and the function $\left\vert z \right\vert^2$ is clearly injective on $(0,\infty)$), so since $f = h\circ q$ is injective on $ \mathbb{D} $.
    \item Since $f = h\circ q$, where $q$ maps $ \mathbb{D} $ to the left half-plane, and $h$ maps the left half-plane to the full complex plane save for $\left( -\infty,0 \right]$, we have that $f$ maps $\C$ to $\C\setminus (-\infty,0]$.
  \end{enumerate}
\end{solution}
\end{document}
