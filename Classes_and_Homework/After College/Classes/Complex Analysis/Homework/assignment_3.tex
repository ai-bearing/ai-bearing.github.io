\documentclass[10pt]{mypackage}

% sans serif font:
%\usepackage{cmbright}
%\usepackage{sfmath}
%\usepackage{bbold} %better blackboard bold

\usepackage{homework}
%\usepackage{notes}
\usepackage{newpxtext,eulerpx,eucal}
\renewcommand*{\mathbb}[1]{\varmathbb{#1}}

\fancyhf{}
\rhead{Avinash Iyer}
\lhead{Complex Analysis: Assignment 3}

\setcounter{secnumdepth}{0}

\begin{document}
\RaggedRight
\begin{problem}[Problem 1]\hfill
  \begin{enumerate}[(a)]
    \item Determine every holomorphic function $f\colon \C\rightarrow \C$ satisfying $\re\left( f\left( z \right) \right) = \re\left( z \right)^2 - \im\left( z \right)^2$.
    \item Let $f\colon \C\rightarrow \C$ be given by
      \begin{align*}
        f(z) &\coloneq \sqrt{\left\vert \re\left( z \right)\im\left( z \right) \right\vert}.
      \end{align*}
      Show that the Cauchy--Riemann equations are satisfied for $f$ at $z = 0$, but $f$ is not differentiable at $z = 0$.
  \end{enumerate}
\end{problem}
\begin{solution}\hfill
  \begin{enumerate}[(a)]
    \item We want to determine $f\colon \C\rightarrow \C$ such that
      \begin{align*}
        f\left( x + iy \right) &= u\left( x,y \right) + iv\left( x,y \right)
      \end{align*}
      satisfies
      \begin{align*}
        u\left( x,y \right) &= x^2 - y^2,
      \end{align*}
      and the Cauchy--Riemann equations:
      \begin{align*}
        \pd{u}{x} &= \pd{v}{y}\\
        \pd{u}{y} &= -\pd{v}{x}.
      \end{align*}
      First, we must verify that $u$ is indeed harmonic. This follows from the fact that
      \begin{align*}
        \pd{^2u}{x^2} &= 2\\
        \pd{^2u}{y^2} &= -2.
      \end{align*}
      Furthermore, we see that $u$ is $C^{3}$, as all of the third partial derivatives are equal to zero. Therefore, a harmonic conjugate of $u$ exists and ensures that $f$ is holomorphic on $\C$. By evaluating the Cauchy--Riemann equations separately, we find that
      \begin{align*}
        \pd{v}{y} &= 2x,
      \end{align*}
      or $ v = 2xy + K(x)$, and
      \begin{align*}
        -\pd{v}{x} &= -2y,
      \end{align*}
      or $v = 2xy + L(y)$. These are only in harmony when $v = 2xy + c$, where $c\in \C$ is a constant. Thus, we find that
      \begin{align*}
        f\left( x + iy \right) &= \left( x^2 - y^2 \right) + i\left( 2xy \right) + c
      \end{align*}
      is necessarily (up to a constant) unique.
    \item We write $f$ as
      \begin{align*}
        f\left( x + iy \right) &= \sqrt{\left\vert xy \right\vert}.
      \end{align*}
  \end{enumerate}
\end{solution}
\begin{problem}[Problem 2]
  Let $U\subseteq \C$ be a region, and let $f\colon U\rightarrow \C$ be a function.
  \begin{enumerate}[(a)]
    \item Suppose that $f$ and $ \overline{f} $ are both holomorphic. Show that $f$ is constant.
    \item Suppose that $f$ is holomorphic and $\re\left( f \right)$ is constant. Show that $f$ is constant.
  \end{enumerate}
\end{problem}
\begin{solution}\hfill
  \begin{enumerate}[(a)]
    \item Write $f\left( x + iy \right) = u\left( x,y \right) + i v\left( x,y \right)$. Since $f$ is holomorphic, we thus get
      \begin{align*}
        \pd{u}{x} &= \pd{v}{y}\\
        \pd{u}{y} &= - \pd{v}{x}.
      \end{align*}
      Now, since $ \overline{f} $ is also holomorphic, we have
      \begin{align*}
        \overline{f\left( x + iy \right)} &= u\left( x,y \right) - i v\left( x,y \right),
      \end{align*}
      meaning that
      \begin{align*}
        \pd{u}{x} &= - \pd{v}{y}\\
        \pd{u}{y} &= \pd{v}{x}
      \end{align*}
      or that
      \begin{align*}
        \pd{u}{x} &= \pm \pd{v}{y}\\
        \pd{u}{y} &= \pm \pd{v}{x}.
      \end{align*}
      Considering the first equation, we then get that $ \pd{u}{x} = \pd{v}{y} = 0 $, or that
      \begin{align*}
        u &= c_1(y)\\
        v &= d_1(x),
      \end{align*}
      while in the second equation, we get that $ \pd{v}{x} = 0 $ and $ \pd{u}{y} = 0 $, meaning that $u$ and $v$ are thus constant. Therefore, $f$ is constant.
    \item If $f$ is holomorphic and $\re\left( f \right)$ is constant, then $i \im(f) = f-\re(f)$ is holomorphic as it is the difference of two holomorphic functions, so $-i\im(f)$ is holomorphic as it is a constant multiple of a holomorphic function, and thus $ \re(f) - i\im(f) $ is holomorphic as it is the sum of two holomorphic functions. This gives $ \overline{f} $ is holomorphic, so $f$ is constant.
  \end{enumerate}
\end{solution}
\begin{problem}[Problem 3]
  Let $U,V\subseteq \C$ be open sets, $f\colon V\rightarrow U$ holomorphic for which $\re\left( f \right),\im\left( f \right)\in C^2\left( V \right)$, and $u\colon U\rightarrow \R$ harmonic. Show that $u\circ f\colon V\rightarrow \R$ is a harmonic function.
\end{problem}
\begin{solution}
  We write $f\left( x + iy \right) = k\left( x,y \right) + \ell \left( x,y \right)$, so that $u\circ f\left( x + iy \right) = u\left( k\left( x,y \right),\ell\left( x,y \right) \right)$. Observe then that this means $u\circ f$ is in $C^{2}\left( V \right)$, and that $u$ is harmonic as a function of $k$ and $\ell$.\newline

  Using the fact that $u\circ f$ is in $C^{2}\left( V \right)$, we use the chain rule by taking
  \begin{align*}
    \pd{^2\left( u\circ f \right)}{x^2} + \pd{^2\left( u\circ f \right)}{y^2} &= \pd{}{x}\left( \pd{\left( u\circ f \right)}{x} \right) + \pd{}{y}\left( \pd{\left( u\circ f \right)}{y} \right)\\
                                  &= \pd{}{x}\left( \pd{u}{k}\pd{k}{x} + \pd{u}{\ell}\pd{\ell}{x} \right) + \pd{}{y}\left( \pd{u}{k}\pd{k}{y} + \pd{u}{\ell}\pd{\ell}{y} \right)\\
                                  &= \pd{u}{k}\pd{^2k}{x^2} + \pd{u}{\ell}\pd{^2\ell}{x^2} + \pd{u}{k}\pd{^2k}{y^2} + \pd{u}{\ell}\pd{^2\ell}{y^2} \\
                                  &+ \pd{k}{x}\left( \pd{k}{x}\pd{}{k} + \pd{\ell}{x}\pd{}{\ell} \right) \left( \pd{u}{k} \right) + \pd{\ell}{x}\left( \pd{k}{x}\pd{}{k} + \pd{\ell}{x}\pd{}{\ell} \right)\left( \pd{u}{\ell} \right) \\
                                  &+ \pd{k}{y}\left( \pd{k}{y}\pd{}{k} + \pd{\ell}{y}\pd{}{\ell} \right)\left( \pd{u}{k} \right) + \pd{\ell}{y} \left( \pd{k}{y}\pd{}{k} + \pd{\ell}{y}\pd{}{\ell} \right)\left( \pd{u}{\ell} \right)\\
                                  &= \pd{u}{k}\pd{^2k}{x^2} + \pd{u}{\ell}\pd{^2\ell}{x^2} + 2 \pd{^2u}{k\partial \ell} \pd{k}{x}\pd{\ell}{x} + \pd{u}{k}\pd{^2k}{y^2} + \pd{u}{\ell}\pd{^2\ell}{y^2} + 2 \pd{^2u}{k\partial \ell} \pd{k}{y}\pd{\ell}{y}\\
                                  &+ \pd{^2u}{k^2}\left( \pd{k}{x} \right)^2 + \pd{^2u}{\ell^2} \left( \pd{\ell}{x} \right)^2 + \pd{^2u}{k^2} \left( \pd{k}{y} \right)^2 + \pd{^2u}{\ell^2} \left( \pd{\ell}{y} \right)^2,
                                  \intertext{where we first used the fact that the mixed partials of $u$ are continuous as $u$ is harmonic. Since $k$ and $\ell$ are $C^{2}$ real/imaginary components of a holomorphic function, they are harmonic, so by reducing via the Cauchy--Riemann equations, we find}
                                  &= \pd{u}{k}\left( \pd{^2k}{x^2} + \pd{^2k}{y^2} \right) + \pd{u}{\ell}\left( \pd{^2\ell}{x^2} + \pd{^2\ell}{y^2} \right)\\
                                  &+ \pd{^2u}{k\partial \ell} \left( \pd{\ell}{y} \right)\pd{\ell}{x} + \pd{^2u}{k\partial \ell} \left( - \pd{\ell}{x} \right) \pd{\ell}{y}\\
                                  &+ \left( \pd{k}{x} \right)^2\left( \pd{^2u}{k^2}+ \pd{^2u}{\ell^2} \right) + \left( \pd{k}{y} \right)^2 \left( \pd{^2u}{k^2} + \pd{^2u}{\ell^2} \right)\\
                                  &= 0,
  \end{align*}
  so $u\circ f$ is harmonic.
\end{solution}
\end{document}
