\documentclass[10pt]{mypackage}

% sans serif font:
%\usepackage{cmbright}
%\usepackage{sfmath}
%\usepackage{bbold} %better blackboard bold

\usepackage{homework}
%\usepackage{notes}
\usepackage{newpxtext,eulerpx,eucal}
\renewcommand*{\mathbb}[1]{\varmathbb{#1}}

\fancyhf{}
\rhead{Avinash Iyer}
\lhead{Complex Analysis: Assignment 3}

\setcounter{secnumdepth}{0}

\begin{document}
\RaggedRight
\begin{problem}[Problem 1]\hfill
  \begin{enumerate}[(a)]
    \item Determine every holomorphic function $f\colon \C\rightarrow \C$ satisfying $\re\left( f\left( z \right) \right) = \re\left( z \right)^2 - \im\left( z \right)^2$.
    \item Let $f\colon \C\rightarrow \C$ be given by
      \begin{align*}
        f(z) &\coloneq \sqrt{\left\vert \re\left( z \right)\im\left( z \right) \right\vert}.
      \end{align*}
      Show that the Cauchy--Riemann equations are satisfied for $f$ at $z = 0$, but $f$ is not differentiable at $z = 0$.
  \end{enumerate}
\end{problem}
\begin{solution}\hfill
  \begin{enumerate}[(a)]
    \item We want to determine $f\colon \C\rightarrow \C$ such that
      \begin{align*}
        f\left( x + iy \right) &= u\left( x,y \right) + iv\left( x,y \right)
      \end{align*}
      satisfies
      \begin{align*}
        u\left( x,y \right) &= x^2 - y^2,
      \end{align*}
      and the Cauchy--Riemann equations:
      \begin{align*}
        \pd{u}{x} &= \pd{v}{y}\\
        \pd{u}{y} &= -\pd{v}{x}.
      \end{align*}
      First, we must verify that $u$ is indeed harmonic. This follows from the fact that
      \begin{align*}
        \pd{^2u}{x^2} &= 2\\
        \pd{^2u}{y^2} &= -2.
      \end{align*}
      Furthermore, we see that $u$ is $C^{3}$, as all of the third partial derivatives are equal to zero. Therefore, a harmonic conjugate of $u$ exists and ensures that $f$ is holomorphic on $\C$. By evaluating the Cauchy--Riemann equations separately, we find that
      \begin{align*}
        \pd{v}{y} &= 2x,
      \end{align*}
      or $ v = 2xy + K(x)$, and
      \begin{align*}
        -\pd{v}{x} &= -2y,
      \end{align*}
      or $v = 2xy + L(y)$. These are only in harmony when $v = 2xy + c$, where $c\in \C$ is a constant. Thus, we find that
      \begin{align*}
        f\left( x + iy \right) &= \left( x^2 - y^2 \right) + i\left( 2xy \right) + c
      \end{align*}
      is necessarily (up to a constant) unique.
    \item We write $f$ as
      \begin{align*}
        f\left( x + iy \right) &= \sqrt{\left\vert xy \right\vert}.
      \end{align*}
      In particular, we see that $f\left( x + iy \right) = u\left( x,y \right)$ where $u\left( x,y \right) = \sqrt{\left\vert xy \right\vert}$. Evaluating the Cauchy--Riemann equations for $f$ at $0$, we have
      \begin{align*}
        \pd{u}{x}\biggr\vert_{(0,0)} &= \lim_{h\rightarrow 0} \frac{\sqrt{\left\vert 0+h \right\vert\left\vert 0 \right\vert} - \sqrt{\left\vert 0 \right\vert\left\vert 0 \right\vert}}{h}\\
                                     &= 0\\
                                     &= \pd{v}{y}\\
        \pd{u}{y}\biggr\vert_{(0,0)} &= \lim_{h\rightarrow 0} \frac{\sqrt{\left\vert 0 \right\vert\left\vert 0 + h \right\vert} - \sqrt{\left\vert 0 \right\vert\left\vert 0 \right\vert}}{ h }\\
                                     &= 0\\
                                     &= -\pd{v}{x}.
      \end{align*}
      Yet, we observe that if we let $h\rightarrow 0$ along the line $ h + ih $ with $h > 0$, then
      \begin{align*}
        f'\left( 0,0 \right) &= \lim_{h\rightarrow 0} \frac{ \sqrt{\left\vert h \right\vert^2} - \sqrt{\left\vert 0 \right\vert} }{h}\\
                             &= \lim_{h\rightarrow 0} \frac{h}{h}\\
                             &= \lim_{h\rightarrow 0} 1\\
                             &= 1,
      \end{align*}
      meaning that, while the partial derivatives $ \pd{u}{x}, \pd{v}{x} $ and $ \pd{u}{y}, \pd{v}{y}$ exist and satisfy the Cauchy--Riemann equations at $ \left( 0,0 \right) $, the limit defining the complex derivative doesn't exist at $\left( 0,0 \right)$.
  \end{enumerate}
\end{solution}
\begin{problem}[Problem 2]
  Let $U\subseteq \C$ be a region, and let $f\colon U\rightarrow \C$ be a function.
  \begin{enumerate}[(a)]
    \item Suppose that $f$ and $ \overline{f} $ are both holomorphic. Show that $f$ is constant.
    \item Suppose that $f$ is holomorphic and $\re\left( f \right)$ is constant. Show that $f$ is constant.
  \end{enumerate}
\end{problem}
\begin{solution}\hfill
  \begin{enumerate}[(a)]
    \item Write $f\left( x + iy \right) = u\left( x,y \right) + i v\left( x,y \right)$. Since $f$ is holomorphic, we thus get
      \begin{align*}
        \pd{u}{x} &= \pd{v}{y}\\
        \pd{u}{y} &= - \pd{v}{x}.
      \end{align*}
      Now, since $ \overline{f} $ is also holomorphic, we have
      \begin{align*}
        \overline{f\left( x + iy \right)} &= u\left( x,y \right) - i v\left( x,y \right),
      \end{align*}
      meaning that
      \begin{align*}
        \pd{u}{x} &= - \pd{v}{y}\\
        \pd{u}{y} &= \pd{v}{x}
      \end{align*}
      or that
      \begin{align*}
        \pd{u}{x} &= \pm \pd{v}{y}\\
        \pd{u}{y} &= \pm \pd{v}{x}.
      \end{align*}
      Considering the first equation, we then get that $ \pd{u}{x} = \pd{v}{y} = 0 $, or that
      \begin{align*}
        u &= c_1(y)\\
        v &= d_1(x),
      \end{align*}
      while in the second equation, we get that $ \pd{v}{x} = 0 $ and $ \pd{u}{y} = 0 $, meaning that $u$ and $v$ are thus constant. Therefore, $f$ is constant.
    \item If $f$ is holomorphic and $\re\left( f \right)$ is constant, then $i \im(f) = f-\re(f)$ is holomorphic as it is the difference of two holomorphic functions, so $-i\im(f)$ is holomorphic as it is a constant multiple of a holomorphic function, and thus $ \re(f) - i\im(f) $ is holomorphic as it is the sum of two holomorphic functions. This gives $ \overline{f} $ is holomorphic, so $f$ is constant.
  \end{enumerate}
\end{solution}
\begin{problem}[Problem 3]
  Let $U,V\subseteq \C$ be open sets, $f\colon V\rightarrow U$ holomorphic for which $\re\left( f \right),\im\left( f \right)\in C^2\left( V \right)$, and $u\colon U\rightarrow \R$ harmonic and $u\in C^{2}\left( U \right)$. Show that $u\circ f\colon V\rightarrow \R$ is a harmonic function.
\end{problem}
\begin{solution}
  We write $f\left( x + iy \right) = k\left( x,y \right) + \ell \left( x,y \right)$, so that $u\circ f\left( x + iy \right) = u\left( k\left( x,y \right),\ell\left( x,y \right) \right)$. Observe then that this means $u\circ f$ is in $C^{2}\left( V \right)$, and that $u$ is harmonic as a function of $k$ and $\ell$.\newline

  Using the fact that $u\circ f$ is in $C^{2}\left( V \right)$, we use the chain rule by taking
  \begin{align*}
    \pd{^2\left( u\circ f \right)}{x^2} + \pd{^2\left( u\circ f \right)}{y^2} &= \pd{}{x}\left( \pd{\left( u\circ f \right)}{x} \right) + \pd{}{y}\left( \pd{\left( u\circ f \right)}{y} \right)\\
                                  &= \pd{}{x}\left( \pd{u}{k}\pd{k}{x} + \pd{u}{\ell}\pd{\ell}{x} \right) + \pd{}{y}\left( \pd{u}{k}\pd{k}{y} + \pd{u}{\ell}\pd{\ell}{y} \right)\\
                                  &= \pd{u}{k}\pd{^2k}{x^2} + \pd{u}{\ell}\pd{^2\ell}{x^2} + \pd{u}{k}\pd{^2k}{y^2} + \pd{u}{\ell}\pd{^2\ell}{y^2} \\
                                  &+ \pd{k}{x}\left( \pd{k}{x}\pd{}{k} + \pd{\ell}{x}\pd{}{\ell} \right) \left( \pd{u}{k} \right) + \pd{\ell}{x}\left( \pd{k}{x}\pd{}{k} + \pd{\ell}{x}\pd{}{\ell} \right)\left( \pd{u}{\ell} \right) \\
                                  &+ \pd{k}{y}\left( \pd{k}{y}\pd{}{k} + \pd{\ell}{y}\pd{}{\ell} \right)\left( \pd{u}{k} \right) + \pd{\ell}{y} \left( \pd{k}{y}\pd{}{k} + \pd{\ell}{y}\pd{}{\ell} \right)\left( \pd{u}{\ell} \right)\\
                                  &= \pd{u}{k}\pd{^2k}{x^2} + \pd{u}{\ell}\pd{^2\ell}{x^2} + 2 \pd{^2u}{k\partial \ell} \pd{k}{x}\pd{\ell}{x} + \pd{u}{k}\pd{^2k}{y^2} + \pd{u}{\ell}\pd{^2\ell}{y^2} + 2 \pd{^2u}{k\partial \ell} \pd{k}{y}\pd{\ell}{y}\\
                                  &+ \pd{^2u}{k^2}\left( \pd{k}{x} \right)^2 + \pd{^2u}{\ell^2} \left( \pd{\ell}{x} \right)^2 + \pd{^2u}{k^2} \left( \pd{k}{y} \right)^2 + \pd{^2u}{\ell^2} \left( \pd{\ell}{y} \right)^2,
                                  \intertext{where we first used the fact that the mixed partials of $u$ are equal by Clairaut's Theorem as $u$ is in $C^{2}$. Since $k$ and $\ell$ are $C^{2}$ real/imaginary components of a holomorphic function, they are harmonic, so by reducing via the Cauchy--Riemann equations, we find}
                                  &= \pd{u}{k}\left( \pd{^2k}{x^2} + \pd{^2k}{y^2} \right) + \pd{u}{\ell}\left( \pd{^2\ell}{x^2} + \pd{^2\ell}{y^2} \right)\\
                                  &+ \pd{^2u}{k\partial \ell} \left( \pd{\ell}{y} \right)\pd{\ell}{x} + \pd{^2u}{k\partial \ell} \left( - \pd{\ell}{x} \right) \pd{\ell}{y}\\
                                  &+ \left( \pd{k}{x} \right)^2\left( \pd{^2u}{k^2}+ \pd{^2u}{\ell^2} \right) + \left( \pd{k}{y} \right)^2 \left( \pd{^2u}{k^2} + \pd{^2u}{\ell^2} \right)\\
                                  &= 0,
  \end{align*}
  so $u\circ f$ is harmonic.
\end{solution}
\begin{problem}[Problem 4]
  Define $g\colon \C\setminus \set{1}\rightarrow \C$ by $g(z) = \frac{z+1}{z-1}$ and $f(z) = e^{g(z)}$.
  \begin{enumerate}[(a)]
    \item Prove that $f$ is bounded in $ \mathbb{D} $.
    \item Compute $\lim_{t\searrow 0} f\left( t + \left( 1-t \right)a \right)$ for all $a\in \partial \mathbb{D}\setminus \set{1}$.
    \item Compute $\lim_{\theta\searrow 0} f\left( e^{i\theta} \right)$.
    \item Compute $\lim_{\theta \nearrow 0} f\left( e^{i\theta} \right)$.
  \end{enumerate}
\end{problem}
\begin{solution}\hfill
  \begin{enumerate}[(a)]
    \item We start by observing that
      \begin{align*}
        \left\vert f(z) \right\vert &= \left\vert e^{g(z)} \right\vert\\
                                    &= e^{\re\left( g(z) \right)}.
      \end{align*}
      Therefore, to establish that $f(z)$ is bounded, we must establish an upper bound on $\re\left( g(z) \right)$ when $z\in \mathbb{D}$. To this end, we establish that $g$ maps $ \mathbb{D} $ to the left half-plane, $\set{z\in \C | \re(z) < 0}$.\newline

      We start with the Cayley transform, 
      \begin{align*}
        h_1(z) &= \frac{z-i}{z+i},
      \end{align*}
      which bijectively maps the upper half-plane to the unit disc. Therefore, the inverse of the Cayley transform, given by
      \begin{align*}
        h_2(z) &= \frac{iz + i}{-z + 1}\\
               &= \frac{i\left( z+1 \right)}{-\left( z - 1 \right)}\\
               &= -i \frac{z+1}{z-1}
      \end{align*}
      bijectively maps the unit disc to the upper half-plane (since Möbius transformations are holomorphic bijections where defined, as follows from computing the derivative). Since
      \begin{align*}
        g(z) &= i h_2(z),
      \end{align*}
      it follows that $g(z)$ bijectively maps $ \mathbb{D} $ to the left half-plane, as if $x + iy$ is such that $y > 0$, then $ix - y$ is in the left half-plane, meaning that $\re\left( g(z) \right) < 0$ for all $z\in \mathbb{D}$, so $f$ is bounded on $ \mathbb{D} $.
    \item Since $e^{w}$ is defined for all $w\in \C$, we may evaluate the limit in $g$, then apply the exponential to obtain our desired result. Additionally, $g$ is continuous whenever $a\neq 1$, so it follows that
      \begin{align*}
        \lim_{t\rightarrow 0} g\left( t + \left( 1-t \right)a \right) &= \frac{a+1}{a-1},
      \end{align*}
      and
      \begin{align*}
        \lim_{t\rightarrow 0} f\left( t + \left( 1-t \right)a \right) &= e^{\frac{a+1}{a-1}}.
      \end{align*}
    \item By computing $g\left( e^{i\theta} \right)$, we find that we get
      \begin{align*}
        g\left( e^{i\theta} \right) &= \frac{\left( \cos\left( \theta \right) + 1 \right) + i\sin\left( \theta \right)}{ \left( \cos\left( \theta \right)-1 + i\sin\left( \theta \right) \right)}\\
                                    &= \frac{\left( \cos\left( \theta \right)+1 + i\sin\left( \theta \right) \right)\left( \cos\left( \theta \right)-1 - i\sin\left( \theta \right) \right)}{2-2\cos\left( \theta \right)}\\
                                    &= \frac{\cos^2\left( \theta \right) - 1 + \sin^2\left( \theta \right) - 2i\sin\left( \theta \right)}{2-2\cos\left( \theta \right)}\\
                                    &= -i\frac{\sin\left( \theta \right)}{1-\cos\left( \theta \right)}\\
                                    &= -i\cot\left( \theta/2 \right).
      \end{align*}
      Therefore, 
      \begin{align*}
        \lim_{\theta \searrow 0} f\left( e^{i\theta} \right) &= \lim_{\theta \searrow 0} e^{-i\cot\left( \theta/2 \right)}\\
                                                             &= \text{DNE},
      \end{align*}
      as $e^{i\cot\left( \theta/2 \right)}$ is periodic, and $\lim_{\theta\searrow 0} \cot\left( \theta/2 \right) = -\infty$.
    \item Similarly as above, we see that
      \begin{align*}
        \lim_{\theta\nearrow 0} f\left( e^{i\theta} \right) &= \lim_{\theta \nearrow 0} e^{-i\cot\left( \theta/2 \right)}\\
                                                            &= \text{DNE},
      \end{align*}
      as $\lim_{\theta\nearrow 0}\cot\left( \theta/2 \right) = \infty$.
  \end{enumerate}
\end{solution}
\begin{problem}[Problem 5]
  Define $f\colon \C\setminus 0 \rightarrow \C$ by
  \begin{align*}
    f(z) &= \frac{1}{2}\left( z + \frac{1}{z} \right).
  \end{align*}
  \begin{enumerate}[(a)]
    \item Let $C_r$ denote the circle of radius $r > 0$ centered at the origin.
      \begin{enumerate}[(i)]
        \item Show that $f\left( C_r \right)$ is an ellipse if $r\neq 1$.
        \item Find the center and equation of this ellipse.
        \item Show that $f\left( C_1 \right) = \left[ -1,1 \right]$.
      \end{enumerate}
    \item Show that $f|_{C\setminus \overline{ \mathbb{D} }}$ is injective, and $ f\left( \C\setminus \overline{D} \right) = \C\setminus \left[ -1,1 \right] $.
    \item Use $f$ to find a conformal map from $\C\setminus \left[ -1,1 \right]$ to $ \mathbb{D}\setminus \set{0} $.
    \item Show that $ f\left( \set{re^{i\theta} | r > 0} \right) $ is a hyperbola for each $\theta\in \R\setminus \frac{\pi}{2}\Z$, and $f\left( \set{re^{i\theta} | r > 0} \right)$ is a ray for each $\theta\in \frac{\pi}{2}\Z$.
  \end{enumerate}
\end{problem}
\begin{solution}\hfill
  \begin{enumerate}[(a)]
    \item We write
      \begin{align*}
        C_r &= \set{x + iy | x^2 + y^2 = r^2}.
      \end{align*}
      \begin{enumerate}[(i)]
        \item Letting $z = x + iy$ where $z\in C_r$ with $\neq 1$, we find that
          \begin{align*}
            f\left( z \right) &= f\left( x + iy \right)\\
                              &= \frac{1}{2}\left( x + iy + \frac{1}{x + iy}  \right)\\
                              &= \frac{1}{2}\left( x + iy + \frac{x - iy}{r^2} \right)\\
                              &= \frac{1}{2}\left( \frac{\left( r^2 + 1 \right)x + \left( r^2 - 1 \right)iy}{r^2} \right)\\
                              &= \frac{1}{2r^2} \left( \left( r^1 + 1 \right)x + \left( r^2 - 1 \right)iy \right),
          \end{align*}
          meaning that if we write a scaling transformation $g\colon \R^{2}\rightarrow \R^2$ by $g\left( x , y \right) = \left( \re\left( f \left( x + iy \right) \right), \im\left( f\left( x + iy \right) \right) \right)$ if $\left( x,y \right)\neq \left( 0,0 \right)$ and $\left( 0,0 \right)$ otherwise, we find that
          \begin{align*}
            g\left( z \right) &= \left( \frac{r^2 + 1}{2r^2} x, \frac{r^2 - 1}{2r^2}y \right)\\
                              &= \left( s_1(r)x,s_2(r)y \right),
          \end{align*}
          where $s_1$ and $s_2$ are nonzero scaling factors (constants that depend on $r$) for $x$ and $y$. Thus, $f\left( C_r \right)$ is an ellipse.
        \item Since there are no translations in the transformation $\R^{2}\rightarrow \R^{2}$ that $g$ defines, the center of $f\left( C_r \right)$ is zero. Therefore, the transformations $x\mapsto \frac{r^2 + 1}{2r^2} x$ and $y\mapsto \frac{r^2 -1}{2r^2}y$ induce the transformation on the ellipse given by
          \begin{align*}
            x^2 + y^2 &= r^2
            \intertext{maps to}
            \left( \frac{2r^2}{r^2 + 1} x \right)^2 + \left( \frac{2r^2}{r^2 - 1} y \right)^2 &= r^2
            \intertext{which equals}
            \frac{x^2}{\left( r^2 + 1 \right)^2} + \frac{y^2}{\left( r^2 - 1 \right)^2} &= \frac{1}{4r^2}.
          \end{align*}
        \item We observe that in the transformation that, if $x^2 + y^2 = 1$ , that since $r^2 - 1 = 0$, we have that for $z = x + iy$ contained on $S^{1}$,
          \begin{align*}
            g\left( z \right) &= \left( x,0 \right).
          \end{align*}
          Since the $x$ coordinate in $x + iy$ ranges from $-1$ to $1$ inclusive, we have that $f\left( z \right) = \left[ -1,1 \right]$.
      \end{enumerate}
    \item Consider a circle $C_r$ with $r > 1$. From above, we know that $g\left( C_r \right)$ is an ellipse in $\R^{2}$ defined by the equation
      \begin{align*}
        \frac{x^2}{\left( r^2 + 1 \right)^2} + \frac{y^2}{\left( r^2 - 1 \right)^2} &= \frac{1}{4r^2}.
      \end{align*}
      In particular, since $r > 1$, the maps $r\mapsto r^2 - 1$ and $r \mapsto r^2 + 1$ are injective, so the ellipse defined $f\left( C_r \right)\subseteq \C$ is uniquely defined. It remains to be shown that if there is $w\in \C\setminus \left[ -1,1 \right]$, there is a unique $z\in \C\setminus \overline{\mathbb{D}}$ such that $f(z) = w$. Toward this end, we simply compute $z$, yielding
      \begin{align*}
        w &= \frac{1}{2}\left( z + \frac{1}{z} \right)\\
        z^2 - 2wz &= -1\\
        \left( z-w \right)^2 &= w^2 - 1\\
        z &= w + \sqrt{w^2 - 1}.
      \end{align*}
      Notice that the square root has branch points at $-1$ and $1$, meaning that it is not well-defined along the line $\left[ -1,1 \right]$. Else, we may take the standard branch of the logarithm that defines the square root function, so that the square root is well-defined.
    \item We observe that $f|_{\C\setminus \overline{ \mathbb{D} }}$ is conformally equivalent to $\C\setminus \left[ -1,1 \right]$, so there is a well-defined holomorphic inverse, which we call $g$, where $g\colon \C\setminus \left[ -1,1 \right] \rightarrow \C\setminus \overline{ \mathbb{D} }$. We observe that, for $re^{i\theta}\in \C\setminus \mathbb{D}$, the function $q(z) = \frac{1}{z}$ is holomorphic, and
      \begin{align*}
        \frac{1}{re^{i\theta}} &= \frac{1}{r}e^{-i\theta},
      \end{align*}
      meaning that $ \frac{1}{z} $ is a bijection to $\D\setminus \set{0}$. In particular, it has the holomorphic inverse $\frac{1}{z}\colon \mathbb{D}\setminus \set{0}\rightarrow \C\setminus \overline{ \mathbb{D} }$. Therefore, we have $h\colon \C\setminus [-1,1]\rightarrow  \mathbb{D}\setminus \set{0}$ given by $\frac{1}{g}$ where $g$ is defined as above.
    \item Let $\theta\in \R\setminus \frac{\pi}{2}\Z$. Then,
      \begin{align*}
        f\left( re^{i\theta} \right) &= \frac{1}{2} \left( r\cos\left( \theta \right) + ir\sin\left( \theta \right) + \frac{1}{r\cos\left( \theta \right) + ir\sin\left( \theta \right)} \right)\\
                                     &= \frac{1}{2} \left( r\cos\left( \theta \right) + ir\sin\left( \theta \right) \frac{\cos\left( \theta \right) - i\sin\left( \theta \right)}{r} \right)\\
                                     &= \frac{1}{2}\left( \frac{r^2\cos\left( \theta \right) + ir^2\sin\left( \theta \right) + \cos\left( \theta \right) - i\sin\left( \theta \right)}{r} \right)\\
                                     &= \frac{1}{2} \left( \cos\left( \theta \right)\frac{\left( r^2 + 1 \right)}{r} + i\sin\left( \theta \right)\frac{r^2 -1}{r} \right).
      \end{align*}
      This yields a curve in $\C \cong \R^{2}$ parametrized by
      \begin{align*}
        \gamma\left( r \right)  &= \left( \cos\left( \theta \right)\frac{r^2 + 1}{2r} , \sin\left( \theta \right)\frac{r^2 - 1}{2r} \right).
      \end{align*}
      If we let $x$ and $y$ be as in those two coordinates, we desire to find a relationship between $x$ and $y$ in the form of a hyperbola. Toward this end, we examine
      \begin{align*}
        \cos^2\left( \theta \right)\frac{\left( r^2 + 1 \right)^2}{4r^2} - \sin^2\left( \theta \right)\frac{\left( r^2 - 1 \right)^2}{4r^2} &= \frac{\cos^2\left( \theta \right)\left( r^4 + 2r^2 + 1 \right) - \sin^2\left( \theta \right)\left( r^4 - 2r^2 + 1 \right)}{4r^{4}}\\
                                                                                                                                            &= \frac{\left( \cos^2\left( \theta \right) - \sin^2\left( \theta \right) \right) 4r^2}{4r^2}\\
                                                                                                                                            &= \cos\left( 2\theta \right)
      \end{align*}
      meaning that, since $\theta$ is fixed and is such that $\cos\left( 2\theta \right)\neq 0$, these coordinates for $\gamma\left( r \right)$ do indeed satisfy
      \begin{align*}
        x^2 - y^2 &= 1,
      \end{align*}
      so that $\im\left( \gamma \right)$ is a hyperbola.\newline

      If $\theta\in \frac{\pi}{2}\Z$, then we have two cases.
      \begin{itemize}
        \item If $\theta = \pi k$ for some $k\in \Z$, then $\cos\left( \theta \right) = \left( -1 \right)^{k}$ and $\sin\left( \theta \right) = 0$, so that
          \begin{align*}
            f\left( re^{i\theta} \right) &= \frac{1}{2}\left( r\left( -1 \right)^{k} + \frac{1}{r\left( -1 \right)^{k}} \right)\\
                                         &= \frac{\left( -1 \right)^{k}}{2} \frac{r^2 + 1}{r}
          \end{align*}
          which is a ray in $\C$ so long as $ r > 0 $.
        \item Similarly, if $\theta = \frac{\pi}{2} + \pi k$ for some $k\in \Z$, then $\sin\left( \theta \right) = \left( -1 \right)^{k}$ and $\cos\left( \theta \right) = 0$, so that
          \begin{align*}
            f\left( re^{i\theta} \right) &= \frac{1}{2}\left( ir\left( -1 \right)^{k} + \frac{1}{ir\left( -1 \right)^{k}} \right)\\
                                         &= i\frac{\left( -1 \right)^{k}}{2} \frac{r^2 - 1}{r},
          \end{align*}
          which is yet again a ray in $\C$ so long as $r > 0$.
      \end{itemize}
  \end{enumerate}
\end{solution}
\end{document}
