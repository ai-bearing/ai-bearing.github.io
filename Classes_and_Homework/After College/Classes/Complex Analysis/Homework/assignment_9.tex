\documentclass[10pt]{mypackage}

% sans serif font:
%\usepackage{cmbright}
%\usepackage{sfmath}
%\usepackage{bbold} %better blackboard bold

\usepackage{homework}
%\usepackage{notes}
\usepackage{newpxtext,eulerpx,eucal}
\renewcommand*{\mathbb}[1]{\varmathbb{#1}}

\fancyhf{}
\fancyhead[R]{Avinash Iyer}
\fancyhead[L]{Complex Analysis: Assignment 9}
\fancyfoot[C]{\thepage}

\setcounter{secnumdepth}{0}

\begin{document}
\RaggedRight
\begin{problem}[Problem 1]
  Show that if $1 < \lambda < \infty$, then the equation
  \begin{align*}
    ze^{\lambda-z} &= 1
  \end{align*}
  has precisely one solution in $ \mathbb{D} $.
\end{problem}
\begin{solution}
  Write $f(z) = ze^{\lambda - z} - 1$. Our task is to show that $f(z)$ has exactly one solution in $ \D $. Consider the function
  \begin{align*}
    g(z) &= ze^{\lambda - z}.
  \end{align*}
  We observe that $g(0) = 0$, and for any $z\neq 0$, $g(z) \neq 0$. Furthermore, since $e^{\lambda - z} \neq 0$ for all $z\in \mathbb{D}$, we observe that $g$ has exactly one zero at $z = 0\in \mathbb{D}$.\newline

  Let $\Gamma = S^{1} = \partial \D$. We then observe that $g$ and $f$ are never zero on $S^{1}$, and that
  \begin{align*}
    \left\vert f(z) - g(z) \right\vert &= 1\\
                                       &< e^{\lambda - 1}\\
                                       &< e^{\lambda - \re\left( z \right)}\\
                                       &= \left\vert ze^{\lambda - z} \right\vert\\
                                       &= \left\vert g(z) \right\vert,
  \end{align*}
  whence $f$ and $g$ have the same number of zeros in $ \D $. 
\end{solution}
\begin{problem}[Problem 2]\hfill
  \begin{enumerate}[(a)]
    \item Prove that for any constants $a_0,a_1,a_2\in \C$, the following inequality holds:
      \begin{align*}
        \max_{\left\vert z \right\vert = 1} \left\vert z^7 + a_2z^2 + a_1z + a_0 \right\vert &\geq 1.
      \end{align*}
    \item Let $U\subseteq \C$ be open with $B\left( 0,1 \right)\subseteq U$, and let $f\colon U\rightarrow \C$ be holomorphic. Suppose that
      \begin{align*}
        \max_{\left\vert z \right\vert = 1}\left\vert f(z) - \frac{1}{z^2} \right\vert &< 1.
      \end{align*}
      Show that $f$ is not a polynomial.
  \end{enumerate}
\end{problem}
\begin{solution}\hfill
  \begin{enumerate}[(a)]
    \item Suppose toward contradiction that this were not true. That is, there are $a_0,a_1,a_2\in \C$ such that for all $z\in S^{1}$,
      \begin{align*}
        \left\vert z^{7} + a_2 z^2 + a_1 z + a_0 \right\vert &< l.
      \end{align*}
      Write $g(z) = -z^{7}$ and $f(z) = a_2 z^2 + a_1 z + a_0$. We observe immediately that this condition implies that $f$ cannot have any zeros on $S^{1}$, as $\left\vert g(z) \right\vert = 1$ on $S^{1}$ always. Therefore, writing
      \begin{align*}
        \left\vert z^{7} + a_2 z^2 + a_1 z + a_0 \right\vert &= \left\vert f(z) - g(z) \right\vert\\
                                                             &< \left\vert g(z) \right\vert,
      \end{align*}
      we observe that the conditions of Rouché's theorem are satisfied, so $f$ and $g$ have the same number of zeros in $\D$. Yet, this is absurd, since $f$ can have at most two zeros when counted with multiplicity, while $g$ has $7$ zeros in $\D$ when counted with multiplicity.
    \item We observe that on $S^{1}$, $\left\vert \frac{1}{z^2} \right\vert = 1$, whence the condition equals
      \begin{align*}
        \left\vert f(z) - \frac{1}{z^2} \right\vert &< \left\vert \frac{1}{z^2} \right\vert.
      \end{align*}
      We claim that the conditions of Rouché's theorem are satisfied. Toward this end, observe that, from the reverse triangle inequality,
      \begin{align*}
        \left\vert f(z) \right\vert &< 2.
      \end{align*}
      If $f$ were to have a pole at $z_0\in S^{1}$, there would be some $r > 0$ such that $\left\vert f(z) \right\vert > 2$ for all $z\in \dot{U}\left( z_0,r \right)$, which would contain some other element of $S^{1}$. Additionally, we observe that $f$ has no zeros on $S^{1}$, as we would then have
      \begin{align*}
        \left\vert -\frac{1}{z^2} \right\vert &< \left\vert \frac{1}{z^2} \right\vert
      \end{align*}
      on $S^{1}$, another contradiction. Therefore, the sum of all orders in $\D$ for $f$ and $\frac{1}{z^2}$ is the same; since $\ord_{0}\left( \frac{1}{z^2} \right) = -2$, it follows that $f$ has two poles (counted with multiplicity) inside $\D$, whence $f$ is not a polynomial (as it is not even holomorphic on $U$).
  \end{enumerate}
\end{solution}
\begin{problem}[Problem 3]
  Let $U\subseteq \C$ be open containing $B\left( 0,1 \right)$, and let $f,g\colon U\rightarrow \C$ be holomorphic such that $\operatorname{ord}_0(f) = 1$ and $\operatorname{ord}_z(f) = 0$ for all $z\in B\left( 0,1 \right)\setminus \set{0}$. For $w\in \C$, define $f_w(z) = f(z) + w g(z)$.
  \begin{enumerate}[(a)]
    \item Show that there exists some $r > 0$ dependent on $g$ such that if $w\in U\left( 0,r \right)$, then $f_w$ has a unique zero in $ B\left( 0,1 \right) $, which we call $z_w$.
    \item Show that $\lim_{w\rightarrow 0} z_w = 0$.
    \item Show that
      \begin{align*}
        z_w &= \frac{1}{2\pi i} \oint_{S\left( 0,1 \right)}^{} \frac{f_w'\left(\xi\right)}{f_w\left(\xi\right)}\xi\:d\xi.
      \end{align*}
  \end{enumerate}
\end{problem}
\begin{solution}\hfill
  \begin{enumerate}[(a)]
    \item To start, we ignore the case where $g(z)$ is identically zero, as it then holds for all $w\in \C$. We observe that if we let $\Gamma = S^{1}$, we desire to find $w$ such that the conditions for Rouché's Theorem hold:
      \begin{align*}
        \left\vert f_w(z) - f(z) \right\vert &< \left\vert f(z) \right\vert\\
        \left\vert wg(z) \right\vert &< \left\vert f(z) \right\vert\\
        \left\vert w \right\vert\left\vert g(z) \right\vert &< \left\vert f(z) \right\vert
      \end{align*}
      Now, we observe a few things. First, since $S^{1}$ is compact, $f$ is holomorphic, and we assume that $\ord_z(f) = 0$ for all $z\in B\left( 0,1 \right)\setminus \set{0}$, we see that there is some $K > 0$ such that $\left\vert f(z) \right\vert \geq K$ for all $z\in S^{1}$. Furthermore, since $g$ is holomorphic, there is some $M > 0$ such that $\left\vert g(z) \right\vert \leq M$ for all $z\in S^{1}$. Therefore, we let $r = \frac{K}{M}$.\newline

      Then, if $\left\vert w \right\vert < r$, we have
      \begin{align*}
        \left\vert f_w(z) - f(z) \right\vert &= \left\vert w \right\vert\left\vert g(z) \right\vert\\
                                             &< r \left\vert g(z) \right\vert\\
                                             &\leq rM\\
                                             &= K\\
                                             &\leq \left\vert f(z) \right\vert
      \end{align*}
      on $S^{1}$. Additionally, to verify that the conditions of Rouche's Theorem still hold for $r = \frac{K}{M}$, we see that
      \begin{align*}
        \left\vert f(z) + wg(z) \right\vert &\geq \left\vert f(z) \right\vert - \left\vert w \right\vert\left\vert g(z) \right\vert\\
                                            &\geq K - \left\vert w \right\vert M\\
                                            &> K - rM\\
                                            &= 0,
      \end{align*}
      whence $f_w(z)$ has no zeros on $S^{1}$. Thus, given these conditions, $f_w(z)$ and $f(z)$ have the same number of zeros in $\D$, whence $f_w(z)$ has a unique zero.
    \item Let $\left( w_n \right)_n\rightarrow 0$ be a sequence in $U\left( 0,r \right)$ with corresponding sequence $\left( z_n \right)_n\subseteq \D$. We observe then that
      \begin{align*}
        \lim_{n\rightarrow \infty} \left( f_{w_n}\left( z_n \right) \right) &= \lim_{n\rightarrow\infty} \left( f\left( z_n \right) + w_ng\left( z_n \right) \right).
      \end{align*}
      Since $g$ is bounded above by $M$, we observe that
      \begin{align*}
        \lim_{n\rightarrow\infty} \left\vert w_n g\left( z_n \right) \right\vert &\leq \lim_{n\rightarrow\infty} \left\vert w_n \right\vert M\\
                                                                                 &= 0,
      \end{align*}
      whence
      \begin{align*}
        \lim_{n\rightarrow\infty} \left\vert f\left( z_n \right) \right\vert &= \lim_{n\rightarrow\infty} \left\vert \left( f_{w_n}\left( z_n \right) - w_ng\left( z_n \right) \right) \right\vert\\
                                                                             &\leq \lim_{n\rightarrow\infty} \left\vert w_n \right\vert \left\vert g\left( z_n \right) \right\vert\\
                                                                             &= 0,
      \end{align*}
      so that $\lim_{n\rightarrow\infty}f\left( z_n \right) = 0$. Since $\left( z_n \right)_n\subseteq \overline{\D}$, $\left( z_n \right)_n$ admits a subsequence $\left( z_{n_k} \right)_k\rightarrow z\in \overline{\D}$. Yet, this means $f\left( z \right) = 0$, meaning $z = 0$ as $f$ has exactly one zero in $ \overline{\D} $. Since $\left( w_n \right)_n\rightarrow 0$, it follows that $\left( w_{n_k} \right)_k\rightarrow 0$, whence $\lim_{w\rightarrow 0}z_w = 0$.
    \item We write
      \begin{align*}
        f_w(z) &= \left( z-z_w \right) h(z),
      \end{align*}
      where $h(z)$ is holomorphic and has $h(z)\neq 0$ for all $z\in \overline{\D}$. We see then that
      \begin{align*}
        \frac{f_w'(z)}{f_w(z)} &= \frac{1}{z-z_w} + \frac{h'(z)}{h(z)},
      \end{align*}
      meaning that the expression $\frac{h'(z)}{h(z)}$ is holomorphic. Therefore,
      \begin{align*}
        \oint_{S^{1}}^{} \frac{f'_w\left( \xi \right)}{f_w\left( \xi \right)}\xi\:d\xi &= \oint_{S^{1}}^{} \frac{\xi}{\xi - z_w}\:d\xi + \oint_{S^{1}}^{} \xi \frac{h'\left( \xi \right)}{h\left( \xi \right)}\:d\xi.
        \intertext{Since $h\left( \xi \right)\neq 0$ on $\D$, it follows that the second integral vanishes by Cauchy's Integral Theorem, whence by Cauchy's Integral Formula, we have}
                                                                                       &= z_w
      \end{align*}
  \end{enumerate}
\end{solution}
\begin{problem}[Problem 4]
  For all $n\in \N$ find the residue at $z = 0$ for each of the following functions.
  \begin{enumerate}[(a)]
    \item $\ds \frac{e^{z^2}}{z^{2n + 1}}$;
    \item $\ds z^{-n}e^{\alpha z}$ for $\alpha\in \Z$;
    \item $\ds \frac{z^{n-1}}{\sin^{n}\left( z \right)}$.
  \end{enumerate}
\end{problem}
\begin{solution}\hfill
  \begin{enumerate}[(a)]
    \item Using the Taylor expansion for $e^{z^2}$, we find that
      \begin{align*}
        \frac{1}{z^{2n+1}} e^{z^2} &= \frac{1}{z^{2n+1}}\sum_{k=0}^{\infty} \frac{z^{2k}}{k!}\\
                                   &= \sum_{k=0}^{\infty} \frac{z^{2k - 2n - 1}}{k!},
      \end{align*}
      meaning that the coefficient at $a_{-1}$ is $\frac{1}{n!}$.
    \item We have a pole of order $n$ at $z = 0$, as $e^{\alpha z}\neq 0$ for all $z$. Thus, computing the residue directly, we find
      \begin{align*}
        \res\left( f;0 \right) &= \frac{1}{\left( n-1 \right)!} \lim_{z\rightarrow 0} \diff{^{n-1}}{z^{n-1}} \left( e^{\alpha z} \right)\\
                               &= \frac{\alpha^{n-1}}{\left( n-1 \right)!}.
      \end{align*}
    \item We observe that the order of the numerator at $z = 0$ is $n-1$, while the order in the denominator at $z = 0$ is $n$, meaning that we have a simple pole at $z = 0$. Therefore, we compute
      \begin{align*}
        \res\left( f;0 \right) &= \lim_{z\rightarrow 0} \frac{z^{n}}{\sin^{n}\left( z \right)}\\
                               &= \left( \lim_{z\rightarrow 0} \frac{z}{\sin\left( z \right)} \right)^{n}\\
                               &= 1.
      \end{align*}
  \end{enumerate}
\end{solution}
\begin{problem}[Problem 5]
  For each positive $n\in \N$, let $\gamma_N$ be the loop consisting of the square with vertices at $\left( N + \frac{1}{2} \right) \left( -1-i \right)$, $\left( N + \frac{1}{2} \right)\left( 1-i \right)$, $\left( N + \frac{1}{2} \right)\left( 1 + i \right)$, and $\left( N + \frac{1}{2} \right)\left( -1 + i \right)$.\newline

  Let $f(z) = \frac{\pi \cot\left( \pi z \right)}{z^{4}}$. By evaluating $ \oint_{\gamma_N}^{} f(z)\:dz $, determine
  \begin{align*}
    \sum_{n=1}^{\infty} \frac{1}{n^{4}}.
  \end{align*}
\end{problem}
\begin{solution}
  We observe that the poles of $f(z)$ are at $-N,-N +1,\dots,0,\dots,N-1,N$. To compute the residue at each of these poles, we separate into the case of $z = 0$ and of $z\neq 0$. For the case with $z\neq 0$, we find that $f$ has a simple pole at $z = k$ for each such $k$, whence
  \begin{align*}
    \res\left( f;k \right) &= \lim_{z\rightarrow k} \frac{\pi\cos\left( \pi z \right)}{z^{4} \diff{}{z}\bigr\vert_{z = n} \pi \sin\left( \pi z \right)}\\
                           &= \frac{1}{k^{4}}.
  \end{align*}
  Since $z^{4}\sin\left( \pi z \right)$ has a zero of order $5$ at $0$, and $\cos\left(  \pi z \right)$ does not have a zero at $z = 0$, it follows that
  \begin{align*}
    f(z) &= \frac{\pi \cos\left( \pi z \right)}{z^{4}\sin\left( \pi z \right)}
  \end{align*}
  has a pole of order $5$ at $0$. We compute
  \begin{align*}
    \res\left( f;0 \right) &= \frac{1}{4!} \lim_{z\rightarrow 0} \diff{^{4}}{z^{4}}\left( z^{5}f(z) \right)\\
                           &= \frac{1}{4!} \lim_{z\rightarrow 0} \diff{^{4}}{z^{4}} \left( \pi z \cot\left( \pi z \right) \right).
  \end{align*}
  Upon tedious computation, we find that
  \begin{align*}
    \res\left( f;0 \right) &= -\frac{\pi^{4}}{45}.
  \end{align*}
  Therefore, we find that
  \begin{align*}
    \frac{1}{2\pi i}\oint_{\gamma_N}^{} f(z)\:dz &= 2\sum_{k=1}^{N} \frac{1}{k^{4}} - \frac{\pi^{4}}{45}.
  \end{align*}
  Now, we want to evaluate
  \begin{align*}
    \lim_{N\rightarrow\infty} \frac{1}{2\pi i}\oint_{\gamma_N}^{} f(z)\:dz.
  \end{align*}
  We observe that
  \begin{align*}
    \left\vert f(z) \right\vert &= \frac{\left\vert \pi\cot\left( \pi z \right) \right\vert}{\left\vert z^{4} \right\vert}\\
                                &\leq \frac{\left\vert \pi\cot\left( \pi z \right) \right\vert}{\left( N + \frac{1}{2} \right)^{4}}.
  \end{align*}
  Now, we want to establish an upper bound for $\left\vert \pi\cot\left( \pi z \right) \right\vert$ on the square $\gamma_N$. We observe that $\cot$ is an odd function, so we only need to concern ourselves with establishing this bound on the upper half-plane. Toward this end, we see that
  \begin{align*}
    \pi\cot\left( \pi\left( N + \frac{1}{2} \right) \right) &= 0.
  \end{align*}
  Computing, for $0 < k \leq N + \frac{1}{2}$,
  \begin{align*}
    \pi\cot\left( \pi\left( N + \frac{1}{2} \right) + \pi ki \right)&= \pi \frac{\cos\left( \pi\left( N + \frac{1}{2} \right) + \pi ki \right)}{\sin\left( \pi\left( N + \frac{1}{2} \right) + ki \right)}\\
                                                                    &= \pi \frac{\pm i \sinh\left( \pi k \right)}{ \pm \cosh\left( \pi k \right) }\\
                                                                    &= \pm \pi i \tanh\left( \pi k \right),
  \end{align*}
  whence
  \begin{align*}
    \left\vert \pi \coth\left( \pi z \right) \right\vert &\leq 1
  \end{align*}
  for all $z\in \gamma_N$. In particular, this yields
  \begin{align*}
    \left\vert \oint_{\gamma_N}^{} f(z)\:dz \right\vert &\leq 4\left( 2N + 1 \right)\frac{1}{\left( N + \frac{1}{2} \right)^4},
  \end{align*}
  so that
  \begin{align*}
    \lim_{N\rightarrow \infty} \oint_{\gamma_N}^{} f(z)\:dz &= 0.
  \end{align*}
  In particular, this yields
  \begin{align*}
    0 &= \lim_{N\rightarrow\infty}\oint_{\gamma_N}^{} f(z) \:dz\\
      &= \lim_{N\rightarrow\infty} \left( 2\sum_{k=1}^{N} \frac{1}{k^{4}} - \frac{\pi^{4}}{45} \right),
  \end{align*}
  so that
  \begin{align*}
    \sum_{k=1}^{\infty} \frac{1}{k^{4}} &= \frac{\pi^{4}}{90}.
  \end{align*}
\end{solution}
\end{document}
