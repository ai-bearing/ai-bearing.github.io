\documentclass[10pt]{mypackage}

% sans serif font:
%\usepackage{cmbright}
%\usepackage{sfmath}
%\usepackage{bbold} %better blackboard bold

\usepackage{homework}
%\usepackage{notes}
\usepackage{newpxtext,eulerpx,eucal}
\renewcommand*{\mathbb}[1]{\varmathbb{#1}}

\fancyhf{}
\fancyhead[R]{Avinash Iyer}
\fancyhead[L]{Complex Analysis: Assignment 10}
\fancyfoot[C]{\thepage}

\setcounter{secnumdepth}{0}

\begin{document}
\RaggedRight
\begin{problem}[Problem 1]
  For all $n\in \N$, find the residue of $f(z) = \left( 1-e^{-z} \right)^{n}$ at $z = 0$ via Cauchy's residue theorem.
\end{problem}
\begin{solution}
  Choose a square contour $\gamma$ defined by
  \begin{align*}
    \gamma &= \gamma_1 + \gamma_2 + \gamma_3 + \gamma_4\\
    \gamma_1 &= 1 + iy\\
    \gamma_2 &= i-x\\
    \gamma_3 &= -1-iy\\
    \gamma_4 &= -i + x
  \end{align*}
  with $-1\leq x \leq 1$ and $-1\leq y\leq 1$. Then,
  \begin{align*}
    2\pi i \res\left( f;0 \right) &= \oint_{\gamma}^{} f(z)\:dz\\
                                  &= \int_{\gamma_1}^{} f(z)\:dz + \int_{\gamma_2}^{} f(z)\:dz + \int_{\gamma_3}^{} f(z)\:dz + \int_{\gamma_4}^{} f(z)\:dz.
  \end{align*}
  We compute
  \begin{align*}
    \int_{\gamma_1}^{} f(z)\:dz &= \int_{-1}^{1} \frac{i}{\left( 1-e^{-1-iy} \right)^n}\:dy.
    \intertext{Taking $u = e^{-1-iy}$, we get}
                                &= -\int_{u(-1)}^{u(1)} \frac{1}{u\left( 1-u \right)^{n}}\:du\\
                                &= -\int_{e^{-1+i}}^{e^{-1-i}} \frac{1}{e^{-1-iy}} + \frac{p\left(e^{-1-iy}\right)}{\left( 1-e^{-1-iy} \right)^{n}}\:dy,
                                \intertext{where $p(u) = \sum_{k=1}^{n} {n\choose k} \left( -1 \right)^{k-1}u^{k-1}$.}
    \int_{\gamma_2}^{} f(z)\:dz &= \int_{-1}^{1} \frac{-1}{\left( 1-e^{-i + x} \right)^{n}}\:dx.
    \intertext{Taking $v = e^{-i + x}$}
                                &= - \int_{v\left( -1 \right)}^{v\left( 1 \right)} \frac{1}{v} + \frac{p(v)}{\left( 1-v \right)^{n}}\:dv\\
                                &= - \int_{e^{-1-i}}^{e^{1-i}} \frac{1}{e^{-i+x}} + \frac{p\left( e^{-i+x} \right)}{\left( 1-e^{-i + x} \right)^{n}}\:dx
  \end{align*}
\end{solution}
\begin{problem}[Problem 2]
  Evaluate
  \begin{align*}
    \int_{-\infty}^{\infty} \frac{\sin^2\left( x \right)}{x^2 + 1}\:dx &= \lim_{R\rightarrow\infty} \int_{-R}^{R} \frac{\sin^2\left( x \right)}{x^2 + 1}\:dx.
  \end{align*}
\end{problem}
\begin{solution}
  We compute
  \begin{align*}
    \int_{-R}^{R} \frac{\sin^2\left( x \right)}{x^2 + 1}\:dx &= \frac{1}{2}\int_{-R}^{R} \frac{1}{x^2 + 1}\:dx - \frac{1}{2} \int_{-R}^{R} \frac{\cos\left( 2x \right)}{x^2 + 1} \:dx.
  \end{align*}
  Calling the latter integral $I$, we take
  \begin{align*}
    f(z) &= \frac{e^{2iz}}{z^2 + 1},
  \end{align*}
  close the contour $\gamma$ in the upper half-plane with the half-circle $C_R = \set{Re^{i\theta} | 0 \leq \theta \leq \pi}$. This gives
  \begin{align*}
    \re\oint_{\gamma}^{} f(z)\:dz &= \re(I) + \re \int_{C_R}^{} f(z)\:dz\\
                               &= \re(I) + \re \int_{0}^{\pi} \frac{e^{2iRe^{i\theta}}}{R^2e^{2i\theta} + 1}iRe^{i\theta}\:d\theta.
  \end{align*}
  Estimating the integrand on the second integral, we see that for $R > 1$,
  \begin{align*}
    \left\vert \frac{iRe^{i\theta}e^{2iRe^{i\theta}}}{R^2e^{2i\theta} + 1} \right\vert &\leq \frac{R}{R^2 - 1}\left\vert e^{2i R\left( \cos\left( \theta \right) + i\sin\left( \theta \right) \right)} \right\vert\\
                                                                                       &\leq \frac{R}{\left( R^2 - 1 \right)\left( e^{2R\sin\left( \theta \right)} \right)}\\
                                                                                       &\leq \frac{R}{R^2 - 1}
  \end{align*}
  whence
  \begin{align*}
    \left\vert \int_{C_R}^{} f(z)\:dz \right\vert &\leq \pi \frac{R}{R^2 - 1}\\
                                                  &\rightarrow 0.
  \end{align*}
  Therefore, by Cauchy's residue theorem,
  \begin{align*}
    \lim_{R\rightarrow\infty} \int_{-R}^{R} \frac{\cos\left( 2x \right)}{x^2 + 1}\:dx &= \re\left( 2\pi i \res\left( f;i \right) \right)\\
                                  &= \re\left( 2\pi i \lim_{z\rightarrow i} \frac{\left( z-i \right)e^{2iz}}{\left( z-i \right)\left( z+i \right)} \right)\\
                                  &= \frac{\pi}{e^2}.
  \end{align*}
  Thus, we find that
  \begin{align*}
    \int_{-\infty}^{\infty} \frac{\sin^2\left( x \right)}{x^2 + 1}\:dx &= \frac{\pi}{2} - \frac{\pi}{2 e^2}.
  \end{align*}
\end{solution}
\begin{problem}[Problem 3]
  For $\xi\in \R$, evaluate
  \begin{align*}
    \int_{-\infty}^{\infty} \frac{\cos\left( \xi x \right)}{x^2 + 4x + 5}\:dx &= \lim_{R\rightarrow\infty} \frac{\cos\left( \xi x \right)}{x^2 + 4x + 5}.
  \end{align*}
\end{problem}
\begin{solution}
  First, if $\xi = 0$, then
  \begin{align*}
    \int_{-\infty}^{\infty} \frac{1}{x^2 + 4x + 5}\:dx &= \int_{-\infty}^{\infty} \frac{1}{\left( x + 2 \right)^2 + 1}\:dx\\
                                                       &= \pi
  \end{align*}
  upon a $u$-substitution.\newline

  Now, let $\xi> 0$. Using $f(z) = \frac{e^{i \xi z}}{z^2 + 4z + 5}$ and closing the contour 
  \begin{align*}
    \gamma_R = \left[ -R,R \right] + \set{Re^{i\theta} | 0 \leq \theta \leq \pi}
  \end{align*}
  in the upper half plane, we find that we get
  \begin{align*}
    \oint_{\gamma_R}^{} f(z)\:dz &= \underbrace{\int_{-R}^{R} f(x)\:dx}_{\eqcolon I} + \int_{C_R}^{} f(z)\:dz.
    \intertext{Parametrizing the integral over $C_R$ by $z = Re^{i\theta}$, we get}
                                 &= I + \int_{0}^{\pi} \frac{e^{i \xi Re^{i\theta}}}{\left( Re^{i\theta} + 2 \right)^2 + 1} iRe^{i\theta}\:d\theta.
  \end{align*}
  Estimating the second integral, we see that for $R > 5$,
  \begin{align*}
    \left\vert \frac{iRe^{i\theta}e^{i\xi Re^{i\theta}}}{\left( Re^{i\theta} + 2 \right)^2 + 1} \right\vert &\leq \frac{R}{R^2 - 4R - 5}\left\vert e^{i\xi R\left( \cos\left( \theta \right) + i\sin\left( \theta \right) \right)} \right\vert\\
                                                                                                            &\leq \frac{R}{\left( R^2 - 4R-5 \right)\left( e^{\xi R \sin\left( \theta \right)} \right)}\\
                                                                                                            &\leq \frac{R}{R^2 - 4R - 5}
  \end{align*}
  meaning that
  \begin{align*}
    \left\vert \int_{C_R}^{} f(z)\:dz \right\vert &\leq \pi \frac{R}{R^2 - 4R - 5}\\
                                                  &\rightarrow 0.
  \end{align*}
  Therefore, we find that
  \begin{align*}
    2\pi i \res\left( -2 + i \right) &= \lim_{R\rightarrow\infty}\oint_{\gamma_R}^{} f(z)\:dz\\
                                     &= \int_{-\infty}^{\infty} \frac{e^{i\xi x}}{x^2 + 4x + 5} \:dx\\
                                     &= 2\pi i \lim_{z\rightarrow -2 + i} \frac{\left( z-\left( -2 + i \right) \right)e^{i\xi z}}{\left( z-\left( -2 + i \right) \right)\left( z-\left( -2-i \right) \right)}\\
                                     &= 2\pi i \frac{e^{i\xi \left( -2 + i \right)}}{2i}\\
                                     &= \frac{\pi}{e^{\xi}} e^{-2i\xi}\\
                                     &= \frac{\pi}{e^{\xi}} \left( \cos\left( 2\xi \right) - i\sin\left( 2\xi \right)\right)\\
                                     &= \frac{\pi}{e^{\xi}} \cos\left( 2\xi \right) - i \frac{\pi}{e^{\xi}}\sin\left( 2\xi \right).
  \end{align*}
  Therefore, we find
  \begin{align*}
    \int_{-\infty}^{\infty} \frac{\cos\left( \xi x \right)}{x^2 + 4x + 5}\:dx &= \re \int_{-\infty}^{\infty} \frac{e^{i\xi x}}{x^2 + 4x + 5}\:dx\\
                                                                              &= \frac{\pi}{e^{\xi}} \cos\left( 2\xi \right).
  \end{align*}
  Now, let $\xi < 0$. We take $\eta_R$ to be the contour
  \begin{align*}
    \eta_R &= \left[ -R,R \right] + \set{Re^{-i\theta} | 0\leq \theta \leq \pi}.
  \end{align*}
  We find that
  \begin{align*}
    \oint_{\eta_R}^{} f(z)\:dz &= \int_{-R}^{R} f(x)\:dx + \int_{C_R}^{} f(z)\:dz\\
                               &= I + \int_{0}^{\pi} \frac{e^{i\xi\left( Re^{-i\theta} \right)}}{\left( Re^{-i\theta} + 2 \right)^2 + 1}\left( -iRe^{-i\theta} \right)\:d\theta.
  \end{align*}
  Estimating the second integrand, we have for $R > 5$
  \begin{align*}
    \left\vert \frac{-iRe^{i\theta}e^{i\xi\left( Re^{-i\theta} \right)}}{\left( Re^{-i\theta} + 2 \right)^2 + 1} \right\vert &\leq \frac{R}{R^2 - 4R - 5} \left\vert e^{i\xi R\left( \cos\left( \theta \right) - i\sin\left( \theta \right) \right)} \right\vert\\
                                                                                                                                   &\leq \frac{R}{R^2 - 4R - 5} e^{\xi R\sin\left( \theta \right)}\\
                                                                                                                                                                                                                                  &\leq \frac{R}{R^2 - 4R - 5}.
  \end{align*}
  Thus,
  \begin{align*}
    \left\vert \int_{C_R}^{} f(z)\:dz \right\vert &\leq \pi\frac{R}{R^2 - 4R - 5},
  \end{align*}
  whence the integral over $C_R$ goes to zero as $R\rightarrow\infty$. Therefore, we have
  \begin{align*}
    -2\pi i \res\left( f;-2-i \right) &= \lim_{R\rightarrow\infty}\int_{\eta_R}^{} f(z)\:dz\\
                                      &= I + \lim_{R\rightarrow\infty} \int_{C_R}^{} f(z)\:dz\\
                                      &= I\\
                                      &= -2\pi i \lim_{z\rightarrow -2-i} \frac{\left( z-\left( -2-i \right) \right)e^{i\xi z}}{\left( z-\left( -2-i \right) \right)\left( z-\left( -2 + i \right) \right)}\\
                                      &= -2\pi i \frac{e^{i\xi\left( -2-i \right)}}{-2i}\\
                                      &= \pi e^{i\xi \left( -2-i \right)}\\
                                      &= \pi e^{\xi} \left( \cos\left( 2\xi \right) - i\sin\left( 2\xi \right) \right)\\
                                      &= \pi e^{\xi} \cos\left( 2\xi \right) - i\pi e^{\xi}\sin\left( 2\xi \right).
  \end{align*}
  Therefore,
  \begin{align*}
    \int_{-\infty}^{\infty} \frac{\cos\left( \xi x \right)}{x^2 + 4x + 5}\:dx &= \re\left( I \right)\\
                                                                              &= \pi e^{\xi}\cos\left( 2\xi \right).
  \end{align*}
\end{solution}
\begin{problem}[Problem 4]
  Evaluate
  \begin{align*}
    \int_{0}^{\infty} \frac{\left( \log x \right)^2}{x^2 + 1}\:dx.
  \end{align*}
\end{problem}
\begin{solution}
  Select the branch of the logarithm that ignores $[0,\infty)$, so that $\arg(z)\in \left(0,2\pi\right)$ for all $z\in \C\setminus [0,\infty)$. Draw a keyhole contour $\gamma_{\delta,\ve,R}$ with an inner \textit{semi}circle of radius $\delta$, an outer semicircle of radius $R$, and returning along the negative real axis to the start of the semicircle of radius $\delta$.\newline

  Set $f(z) = \frac{\left( \log z \right)^2}{z^2 + 1}$, and observe that for $0 < \ve < \delta < 1 < R$, we have
  \begin{align*}
    \oint_{\gamma_{\delta,\ve,R}}^{} f(z)\:dz &= 2\pi i \left( \res\left( f;i \right) \right)\\
                                              &= 2\pi i \left( \lim_{z\rightarrow i} \left( z-i \right)\frac{\left( \log(z) \right)^2}{\left( z-i \right)\left( z+i \right)}  \right)\\
                                              &= -\frac{\pi^3}{4}.
  \end{align*}
  Meanwhile, we observe that in the limit as $\ve \rightarrow 0$, we are left with a few integrals
  \begin{align*}
    \oint_{\gamma_{\delta,\ve,R}}^{} f(z)\:dz &= \int_{\delta}^{R} \frac{\left( \log(x) \right)^2}{x^2 + 1}\:dx + \int_{-R}^{-\delta} \frac{\left(\log(x) \right)^2}{x^2 + 1}\:dx\tag{$\ast$}\\
                                              &+ \int_{0}^{\pi} \frac{\log\left( \delta e^{-i\theta} \right)^2}{\delta^2e^{-2i\theta} + 1} \left( -i\delta e^{-i\theta} \right)\:d\theta + \int_{0}^{\pi} \frac{\log\left( R e^{i\theta} \right)^2}{R^2e^{2i\theta} + 1} iR e^{i\theta}\:d\theta \tag{$\ast\ast$}
  \end{align*}
  We start by estimating the integrals in $(\ast\ast)$ by the circles $\delta e^{-i\theta}$ and $Re^{i\theta}$. Towards this end, we observe that
  \begin{align*}
    \left\vert \frac{-i\delta e^{-i\theta} \left( \ln\left( \delta \right) - i\theta \right)^2}{\delta^2 e^{-2i\theta} + 1} \right\vert &\leq \frac{\delta \left\vert \ln\left( \delta \right) \right\vert^2 + 2\theta \delta \left\vert \ln\left( \delta \right) \right\vert + \theta^2\delta}{1-\delta^2}\\
                                                                                                                                                  &\leq \frac{\delta \left\vert \ln\left( \delta \right) \right\vert^2 + 4\pi\delta \left\vert \ln\left( \delta \right) \right\vert + 4\pi^2 \delta}{1-\delta^2}\\
                                                                                                                                                                                                                                                                                                                                                &\rightarrow 0
  \end{align*}
  as $\delta \rightarrow 0$. Thus,
  \begin{align*}
    \left\vert \int_{0}^{2\pi} \frac{-i\delta e^{-i\theta}\left( \ln\left( \delta \right) - i\theta \right)}{\delta^2e^{2i\theta} + 1}\:d\theta \right\vert &\leq \pi \frac{\delta \left\vert \ln\left( \delta \right) \right\vert^2 + 4\pi\delta \left\vert \ln\left( \delta \right) \right\vert + 4\pi^2 \delta}{1-\delta^2}\\
                                                                                                                                                                               &\rightarrow 0.
  \end{align*}
  Similarly, 
  \begin{align*}
    \left\vert \frac{Re^{i\theta}\left( \ln\left( R \right) + i\theta \right)^2}{R^2e^{2i\theta} + 1} \right\vert &\leq \frac{R\left\vert \ln\left( R \right) \right\vert^2 + 2\theta R \left\vert \ln\left( R \right) \right\vert + \theta^2 R}{R^2 - 1}\\
                                                                                                                                   &\leq \frac{R \left\vert \ln\left( R \right) \right\vert^2}{R^2 - 1} + \frac{2\pi R}{R^2 - 1} + \frac{4\pi^2}{R^2 - 1}\\
                                                                                                                                                                                                                                  &= \frac{\left\vert \ln\left( R \right) \right\vert^2}{R - \frac{1}{R}}\frac{2\pi R}{R^2 - 1} + \frac{4\pi^2}{R^2 - 1}\\
                                                                                                                                                                                                                                                                                                                                                                                                                                                                                                                                                                                                                                                                                                                                                                                                                                                                                                                           &\rightarrow 0
  \end{align*}
  as $R\rightarrow \infty$, so the corresponding integral also goes to zero.\newline

  Now, we turn our attention to $(\ast)$. We observe that by the coordinate change $x\mapsto -x$, we get
  \begin{align*}
    \int_{\delta}^{R} \frac{\ln\left( x \right)}{x^2 + 1}\:dx + \int_{-R}^{-\delta} \frac{\left( \ln\left( x \right) \right)^2}{x^2 + 1}\:dx &= 2 \int_{\delta}^{R} \frac{\left( \ln\left( x \right) \right)^2}{x^2 + 1}\:dx + 2\pi i \int_{\delta}^{R} \frac{\ln\left( x \right)}{x^2 + 1}\:dx - \pi^2 \int_{\delta}^{R} \frac{1}{x^2 + 1}\:dx.
  \end{align*}
  As we take the limit as $\delta \rightarrow 0$ and $R\rightarrow\infty$, we observe that we get the equation
  \begin{align*}
    \frac{\pi^3}{4} &= 2 \underbrace{\int_{0}^{\infty} \frac{\left( \ln\left( x \right) \right)^2}{x^2 + 1}\:dx}_{\eqcolon I_1} + 2\pi i \underbrace{\int_{0}^{\infty} \frac{\ln\left( x \right)}{x^2 + 1}\:dx}_{\eqcolon I_0}
  \end{align*}
  Now, to evaluate $I_0$, we use the same contour for $g(z) = \frac{\ln(z)}{z^2 + 1}$, giving
    \begin{align*}
      \int_{\gamma_{\delta,\ve,R}}^{} g(z)\:dz &= \int_{\delta}^{R} \frac{\ln(x)}{x^2 + 1}\:dx  + \int_{-R}^{-\delta} \frac{\ln\left( x \right)}{x^2 + 1}\:dx\\
                                             &+ \int_{0}^{\pi} \frac{\ln\left( Re^{i\theta} \right)}{R^2e^{2i\theta} + 1}iRe^{i\theta} \:d\theta + \int_{0}^{\pi} \frac{\ln\left( \delta e^{-i\theta} \right)}{\delta^2e^{2i\theta} + 1}\left( -i\delta e^{-i\theta} \right)\:d\theta.
    \end{align*}
    The circle integrands may be estimated by
    \begin{align*}
      \left\vert \frac{iRe^{i\theta}\left\vert ln\left( R \right) + i\theta \right\vert}{R^2e^{2i\theta}} \right\vert &\leq \frac{R\ln(R) + R\theta}{R^2 - 1}\\
                                                                                                                                     &\leq \frac{R\ln(R) + \pi R}{R^2 - 1}\\
                                                                                                                                     &\rightarrow 0
    \end{align*}
    as $R\rightarrow \infty$, so that
    \begin{align*}
      \left\vert \int_{0}^{\pi} \frac{\ln\left( Re^{i\theta} \right)}{R^2e^{2i\theta} + 1}iRe^{i\theta} \:d\theta \right\vert &\leq \pi \frac{R\ln(R) + \pi R}{R^2 - 1}\\
                                                                                                                                             &\rightarrow 0.
    \end{align*}
    Similarly,
    \begin{align*}
      \left\vert \frac{-i\delta e^{-i\theta}\left( \ln\left( \delta \right)-i\theta \right)}{\delta^2e^{2i\theta}+1} \right\vert &\leq \frac{\delta \left\vert \ln\left( \delta \right) \right\vert + \pi \delta}{1-\delta^2}\\
                                                                                                                                                                      &\rightarrow 0
    \end{align*}
    as $\delta\rightarrow\infty$, so that
    \begin{align*}
      \left\vert \int_{0}^{\pi} \frac{\ln\left( \delta e^{-i\theta} \right)}{\delta^2e^{2i\theta} + 1}\left( -i\delta e^{-i\theta} \right)\:d\theta \right\vert &\leq \pi \frac{\delta \left\vert \ln\left( \delta \right) \right\vert + \pi \delta}{1-\delta^2}\\
                                                                                                                                                                           &\rightarrow 0.
    \end{align*}
    Thus, we must evaluate the first two integrals. Yet, by using the substitution $x\mapsto -x$, we see that
    \begin{align*}
      \int_{\delta}^{R} \frac{\ln\left( x \right)}{x^2 + 1}\:dx + \int_{-R}^{-\delta} \frac{\ln\left( x \right)}{x^2 + 1}\:dx &= 2 \int_{\delta}^{R} \frac{\ln\left( x \right)}{x^2 + 1}\:dx + i\pi \int_{\delta}^{R} \frac{1}{x^2 + 1}\:dx.
    \end{align*}
    Taking limits and evaluating residues gives
    \begin{align*}
      2\pi i \res\left( g;i \right) &= 2\pi i \left( \frac{i\pi/2}{2i} \right)\\
                                    &= i\frac{\pi^2}{2}\\
                                    &= 2 \int_{0}^{\infty} \frac{\ln(x)}{x^2 + 1}\:dx + i\pi \int_{0}^{\infty} \frac{1}{x^2 + 1}\:dx\\
                                    &= 2 \int_{0}^{\infty} \frac{\ln\left( x \right)}{x^2 + 1}\:dx + i\frac{\pi^2}{2},
    \end{align*}
    whence the integral for $g(z)$ is zero.\newline

  Thus, we find that
  \begin{align*}
    \int_{0}^{\infty} \frac{\left( \ln\left( x \right) \right)^2}{x^2 + 1}\:dx &= \frac{\pi^3}{8}
  \end{align*}
\end{solution}
\begin{problem}[Problem 5]
  For $\xi\in \R$, evaluate
  \begin{align*}
    \operatorname{p.v.} \int_{-\infty}^{\infty} \frac{x^3}{\left( x^2 + 1 \right)^2} e^{-2\pi i x \xi}\:dx.
  \end{align*}
\end{problem}
\begin{solution}
  We write
  \begin{align*}
    \int_{-\infty}^{\infty} \frac{x^{3}}{\left( x^2 + 1 \right)^2}e^{-2\pi i x \xi}\:dx &= \lim_{R\rightarrow\infty} \int_{-R}^{R} \frac{x^3}{\left( x-i \right)^2\left( x+i \right)^2}e^{-2\pi i x \xi}\:dx.
  \end{align*}
  Write $f(z) = \frac{z^3}{\left( z^2 + 1 \right)^2}e^{-2\pi i z \xi}$.\newline

  Suppose $\xi \geq 0$. Let $\gamma_{R}$ be the square contour in the lower half-plane with side length $R$ sitting on the real axis. Then,
  \begin{align*}
    -2\pi i \res\left( f; -i\right) &= \int_{\gamma_R}^{} f(z)\:dz\\
                                   &= \int_{-R}^{R} f(x)\:dx + \int_{0}^{R} f\left(R-iy\right)\:d\left( R - iy \right) + \int_{R}^{-R} f\left( x-iR \right)\:d\left( x - iR \right) + \int_{0}^{R} f\left( -R + iR + iy \right)\:d\left( -R-iR + iy \right)
  \end{align*}
  Writing each of the integrals not equal to the original integral, we get
  \begin{align*}
    \int_{0}^{R} f\left( R-iy \right)\:d\left( R-iy \right) &= -i \int_{0}^{R} \frac{\left( R-iy \right)^3}{\left( \left( R-iy \right)^2 + 1 \right)^2}e^{-2\pi i \xi \left( R - iy \right)}\:dy\\
    \int_{R}^{-R} f\left( x - iR \right)\:d\left( x-iR \right) &= 
  \end{align*}
\end{solution}
\end{document}
