\documentclass[10pt]{mypackage}

\usepackage{mlmodern}
%\usepackage{newpxtext,eulerpx,eucal}
%\renewcommand*{\mathbb}[1]{\varmathbb{#1}}

%\usepackage{homework}
\usepackage{notes}

%\usepackage[ backend=bibtex, style = alphabetic, sorting=ynt ]{biblatex}
%\addbibresource{  }

\usepackage{parskip}

\fancyhf{}
\fancyhead[R]{Avinash Iyer}
\fancyhead[L]{Algebra II: Class Notes}
\fancyfoot[C]{\thepage}

\setcounter{secnumdepth}{0}

\begin{document}
\RaggedRight
The primary text for Algebra II is Dummit and Foote's \textit{Abstract Algebra}, and will cover the following topics:
\begin{itemize}
  \item modules and advanced linear algebra;
  \item representation theory of finite groups;
  \item field theory and Galois theory.
\end{itemize}
\tableofcontents
\section{Modules and Advanced Linear Algebra}%
\subsection{Tensor Products of Modules}%
To motivate tensor products, we recall a basic fact from linear algebra. If we assume that $R$ is a field, and $M,N$ are finite-dimensional $R$-vector spaces, then the following equation necessarily holds:
\begin{align*}
  \dim\left( M\oplus N \right) &= \dim\left( M \right) + \dim\left( N \right).
\end{align*}
We want to construct a similar operation on vector spaces, $M\otimes N$, that satisfies
\begin{align*}
  \dim\left( M\otimes N \right) &= \dim\left( M \right)\dim\left( N \right).
\end{align*}
For now, we will label this by $M\bar{\otimes}N$, where we use the $\bar{\otimes}$ to refer to the fact that this is a temporary definition. Naively, we might seek to define $M\bar\otimes N$ as follows. If we let $\set{x_1,\dots,x_k}$ be a basis for $M$ and $\set{y_1,\dots,y_{\ell}}$ a basis for $N$, then we will define $M\bar\otimes N$ to be all the formal $R$-linear combinations over the basis
\begin{align*}
  B &= \set{x_i\otimes y_j | 1\leq i \leq k,1\leq j \leq \ell}.
\end{align*}
While this is technically correct --- as in, this does yield a vector space with
\begin{align*}
  \dim\left( M\bar\otimes N \right) &= \dim\left( M \right)\dim\left( N \right),
\end{align*}
the issue is that this definition is not canonical, in that it depends on chosen bases for $M$ and $N$. Furthermore, it is not clear how one may generalize from this definition to modules over arbitrary rings, which do not necessarily have bases. To resolve this issue, we will go about defining a construction that ``extends,'' in a sense, this definition of $M\bar\otimes N$.

To start, we define the simple tensor $m\otimes n$ for any $m\in M$ and $n\in N$. If we let
\begin{align*}
  m &= \sum_{i=1}^{k}\lambda_i x_i\\
  n &= \sum_{j=1}^{\ell}\mu_j y_j,
\end{align*}
then we will define
\begin{align*}
  m\otimes n &= \sum_{i=1}^{k}\sum_{j=1}^{\ell} \lambda_i\mu_j \left( x_i\otimes y_j \right).
\end{align*}
We observe that every element of $M\bar\otimes N$ is a sum (i.e., an \textit{integral} linear combination) of simple tensors, as by regrouping we may take
\begin{align*}
  \sum_{i=1}^{k}\sum_{j=1}^{\ell}\lambda_{ij}\left( x_i\otimes y_j \right) &= \sum_{i=1}^{k}\sum_{j=1}^{\ell} \left( \lambda_{ij}x_i \right)\otimes y_j.
\end{align*}
The simple tensors satisfy the following relations:
\begin{enumerate}[(R1)]
  \item $\left( m_1 + m_2 \right)\otimes n = m_1\otimes n + m_2\otimes n$;
  \item $m\otimes \left( n_1 + n_2 \right) = m\otimes n_1 + m\otimes n_2$;
  \item $\left( \alpha m \right)\otimes n = m\otimes \left( \alpha n \right)$
\end{enumerate}
for $m,m_1,m_2\in M$, $n,n_1,n_2\in N$, and $\alpha\in R$.
\begin{proposition}
  These are the defining relations for $M\bar\otimes N$ in the category of abelian groups.
\end{proposition}
We will simply take this proposition as fact.

Now, let
\begin{align*}
  Q &= M\times N\\
    &= \set{\left( m,n \right) | m\in M,n\in N}
\end{align*}
be the Cartesian product of $M$ and $N$ as sets. We will then take $\Z\left[ Q \right]$ to be the standard free $\Z$-module (i.e., free abelian group) on $Q$. That is, $\Z\left[ Q \right]$ is the set of formal linear combinations
\begin{align*}
  v &= \sum_{q\in Q} \lambda_q q,
\end{align*}
where $\lambda_q\in \Z$ and only finitely many coefficients are nonzero. By the universal property of free abelian groups, the map $\left( m,n \right)\mapsto m\otimes n$ descends to a unique homomorphism $\varphi\colon\Z\left[ Q \right]\rightarrow M\bar\otimes N$. Such a homomorphism is necessarily surjective as every element of $M\bar\otimes N$ is an integral linear combination of simple tensors, meaning that we have
\begin{align*}
  M\bar\otimes N &\cong \Z\left[ Q \right]/\ker\left( \varphi \right)
\end{align*}
as abelian groups.

Now, consider the subgroup of $\Z\left[ Q \right]$, which we denote $ \left\langle K \right\rangle $, that is generated by the following elements:
\begin{enumerate}[(I)]
  \item $\left( m_1 + m_2,n \right) - \left( m_1,n \right) - \left( m_2,n \right)$;
  \item $\left( m,n_1 + n_2 \right) - \left( m,n_1 \right) - \left( m,n_2 \right)$;
  \item $\left( \alpha m,n \right) - \left( m,\alpha n \right)$
\end{enumerate}
for $m_1,m_2,m\in M$, $n_1,n_2,n\in N$, and $\alpha\in R$. Then, from proposition that the relations (R1) through (R3) define $M\bar\otimes N$, it follows that $\left\langle K \right\rangle = \ker\left( \varphi \right)$. Thus, we may define the tensor product canonically as follows.
\begin{definition}
  Letting $M,N,Q,K$ be as above, we define
  \begin{align*}
    M\otimes N &\coloneq \Z\left[ Q \right]/\left\langle K \right\rangle, \label{def:tensor_product_modules}\tag{$\dag$}
  \end{align*}
  and define $m\otimes n = \left( m,n \right) + K$.
\end{definition}
So far, this has only given us an abelian group. We may ask how to define $\Z\left[ Q \right]/\left\langle K \right\rangle$ as an $R$-vector space, which naturally seems to be defined by
\begin{align*}
  r\left( \sum_{i=1}^{n}m_i\otimes n_i \right) &= \sum_{i=1}^{n}\left( rm_i \right)\otimes n_i \label{eq:scalar_multiplication_tensor_product}\tag{$\ast$}
\end{align*}
To show that the right-hand side of \eqref{eq:scalar_multiplication_tensor_product} is well-defined is a very difficult task. We will not do it here.

Now, we can actually quite easily generalize \eqref{def:tensor_product_modules} to modules over non-fields.
\begin{itemize}
  \item If $R$ is a commutative ring with $1$, and $M$ and $N$ are left $R$-modules, the definition in \eqref{def:tensor_product_modules} copies over exactly.
  \item If $R$ is non-commutative with $1$, then the definition in \eqref{def:tensor_product_modules} makes sense, but the scalar multiplication in \eqref{eq:scalar_multiplication_tensor_product} does \textit{not} hold.

    In fact, we need to change the assumptions for $M$ and $N$ as $R$-modules. In particular, we need $M$ to be a \textit{right} $R$-module, and $N$ to be a left $R$-module, and take the generators of type (III) for $K$ to be defined by
    \begin{description}[font=\normalfont]
      \item[(III')] $\left( mr,n \right)-\left( m,rn \right)$
    \end{description}
    for $m\in M$, $n\in N$, and $r\in R$. This gives the tensor product $M\otimes_{R}N$ an abelian group structure, but does not endow it with a $R$-module structure.
\end{itemize}
We may now consider some simple examples computing tensor products.
\begin{example}
  Let $R = \Z$. We will show that $\Z/n\Z\otimes_{R}\Q = 0$.

  As a general strategy, in order to show that a tensor product is the zero module, it suffices to show for every simple tensor. Observe that $0\otimes y = 0$ for any tensor product, since we may take
  \begin{align*}
    0\otimes y &= \left( 0 + 0 \right)\otimes y\\
               &= 0\otimes y + 0\otimes y.
  \end{align*}
  Therefore, we may write
  \begin{align*}
    \left[ a \right]\otimes b &= \left( n\left[ a \right] \right)\otimes \left( \frac{b}{n} \right)\\
                              &= \left[ na \right]\otimes \frac{b}{n}\\
                              &= 0\otimes \frac{b}{n}\\
                              &= 0.
  \end{align*}
\end{example}
We may now work towards understanding one of the defining properties of tensor products in general. This requires a discussion of a weakened version of $R$-bilinear maps.
\begin{definition}
  Let $R$ be a ring, $M$ a right $R$-module, $N$ a left $R$-module, and $L$ an abelian group written additively. A map $\varphi\colon M\times N \rightarrow L$ is called \textit{$R$-balanced} if
  \begin{description}[font=\normalfont]
    \item[(BM1)] $\varphi\left( m_1 + m_2,n \right) = \varphi\left( m_1,n \right) + \varphi\left( m_2,n \right)$
    \item[(BM2)] $\varphi\left( m,n_1 + n_2 \right) = \varphi\left( m,n_1 \right) + \varphi\left( m,n_2 \right)$
    \item[(BM3)] $\varphi\left( mr,n \right) = \varphi\left( m,rn \right)$
  \end{description}
  for all $r\in R$, $m,m_1,m_2\in M$, and $n,n_1,n_2\in N$.
\end{definition}
\begin{theorem}
  Let $R,M,N,L$ be as above. Let
  \begin{align*}
    \Omega &= \set{\Phi\colon M\otimes N\rightarrow L | \Phi\text{ a group homomorphism}}\\
    \Delta &= \set{\varphi\colon M\times N\rightarrow L | \varphi\text{ $R$-balanced}}.
  \end{align*}
  Define the map $J\colon \Omega\rightarrow \Delta$ by
  \begin{align*}
    \left( J\Phi \right)(m,n) &= \Phi(m\otimes n).
  \end{align*}
  Then, $J$ is bijective.
\end{theorem}
\begin{proof}
  We have that $J$ is injective since $J\Phi$ captures the value of $\Phi$ on simple tensors, and $\Phi$ is completely determined by its value on simple tensors since $\Phi$ is a group homomorphism, and elements of $M\otimes N$ are sums of simple tensors.

  To prove surjectivity, we recall that
  \begin{align*}
    M\otimes N &= \Z\left[ M\times N \right]/\left\langle K \right\rangle.
  \end{align*}
  Let $\varphi\colon M\times N\rightarrow L$ be an $R$-balanced map. By the universal property for free modules, there is a homomorphism $ \widetilde{\varphi}\colon \Z\left[ M\times N \right]\rightarrow L $ taking $\left( m,n \right)\mapsto \varphi\left( m,n \right)$.

  We only need to show now that $\widetilde{\varphi}$ kills the elements of $K$ that generate $\left\langle K \right\rangle$, but this follows from the fact that $\varphi$ is $R$-balanced. Therefore, we get an induced map
  \begin{align*}
    \Phi\colon M\otimes N&\rightarrow L\\
    m\otimes n &\mapsto \varphi\left( m,n \right),
  \end{align*}
  so we are done.
\end{proof}
\begin{definition}
	Let $R$ be a commutative ring, $M,N,L$ left $R$-modules. A map $\varphi\colon M\times N\rightarrow L$ is called $R$-bilinear if it satisfies (BM1), (BM2), and
	\begin{description}[font=\normalfont]
		\item[(BM3')] $\varphi \left( m,rn \right) = \varphi \left( rm, n  \right) = r \varphi \left( m,n \right)$
    \end{description}
\end{definition}
\begin{theorem}
    If $R,M,N,L$ are as above, then there exists a natural bijection between $\hom_{R} \left( M\otimes N, L\right)$ and $\hom_{R} \left( M\times N, L \right)$.
\end{theorem}
The proof is the same as the proof in the case of $R$-balanced maps, mutatis mutandis.
\begin{proposition}
	Let $R$ be a commutative ring, and $M,N$ free left $R$-modules with respective bases $X$ and $Y$. Then, $M\otimes N$ is a free module with basis
	\begin{align*}
		Z &= \set{x\otimes y | x\in X,y\in Y}.
	\end{align*}
\end{proposition}
\begin{proof}
    We have that $Z$ generates $M\otimes N$ as a $R$-module, so we only need to show that $Z$ is linearly independent.

    Let
    \begin{align*}
        v &= \sum_{i=1}^{t}r_ix_i\otimes y_i.
    \end{align*}
    Without loss of generality, we assume that $r_1\neq 0$. It is enough to find a homomorphism $\varphi\colon M\otimes N\rightarrow R$ such that $\varphi(v)\neq 0$.

    Toward this end, we construct an $R$-bilinear map, which we only need to specify on the basis. Define
    \begin{align*}
	\alpha\colon M&\rightarrow R\\
	x_i &\mapsto \begin{cases}
		0 & x_i\neq x_1 \\
		1 & x_i = x_1
        \end{cases}\\
    \beta\colon N&\rightarrow R\\
    y_i &\mapsto \begin{cases}
	0 & y_i\neq y_1\\
	1 & y_i = y_1
    \end{cases}.
    \end{align*}
    The map $\varphi\colon M\times N\rightarrow R$ given by $\varphi \left( x_i, y_i \right) = \alpha \left( x_i \right) \beta \left( y_i \right)$ is thus $R$-bilinear and induces a map on the tensor product that is nonzero at $v$. Thus, $v$ is not the zero vector.
\end{proof}
\end{document}
