\documentclass[10pt]{mypackage}

\usepackage{mlmodern}
%\usepackage{newpxtext,eulerpx,eucal}
%\renewcommand*{\mathbb}[1]{\varmathbb{#1}}

%\usepackage{homework}
\usepackage{notes}

%\usepackage[ backend=bibtex, style = alphabetic, sorting=ynt ]{biblatex}
%\addbibresource{  }

\usepackage{parskip}

\fancyhf{}
\fancyhead[R]{Avinash Iyer}
\fancyhead[L]{Algebra II: Class Notes}
\fancyfoot[C]{\thepage}

\setcounter{secnumdepth}{0}

\begin{document}
\RaggedRight
The primary text for Algebra II is Dummit and Foote's \textit{Abstract Algebra}, and will cover the following topics:
\begin{itemize}
  \item modules and advanced linear algebra;
  \item representation theory of finite groups;
  \item field theory and Galois theory.
\end{itemize}
\tableofcontents
\section{Modules and Advanced Linear Algebra}%
\subsection{Tensor Products of Modules}%
To motivate tensor products, we recall a basic fact from linear algebra. If we assume that $R$ is a field, and $M,N$ are finite-dimensional $R$-vector spaces, then the following equation necessarily holds:
\begin{align*}
  \dim\left( M\oplus N \right) &= \dim\left( M \right) + \dim\left( N \right).
\end{align*}
We want to construct a similar operation on vector spaces, $M\otimes N$, that satisfies
\begin{align*}
  \dim\left( M\otimes N \right) &= \dim\left( M \right)\dim\left( N \right).
\end{align*}
For now, we will label this by $M\bar{\otimes}N$, where we use the $\bar{\otimes}$ to refer to the fact that this is a temporary definition. Naively, we might seek to define $M\bar\otimes N$ as follows. If we let $\set{x_1,\dots,x_k}$ be a basis for $M$ and $\set{y_1,\dots,y_{\ell}}$ a basis for $N$, then we will define $M\bar\otimes N$ to be all the formal $R$-linear combinations over the basis
\begin{align*}
  B &= \set{x_i\otimes y_j | 1\leq i \leq k,1\leq j \leq \ell}.
\end{align*}
While this is technically correct --- as in, this does yield a vector space with
\begin{align*}
  \dim\left( M\bar\otimes N \right) &= \dim\left( M \right)\dim\left( N \right),
\end{align*}
the issue is that this definition is not canonical, in that it depends on chosen bases for $M$ and $N$. Furthermore, it is not clear how one may generalize from this definition to modules over arbitrary rings, which do not necessarily have bases. To resolve this issue, we will go about defining a construction that ``extends,'' in a sense, this definition of $M\bar\otimes N$.

To start, we define the simple tensor $m\otimes n$ for any $m\in M$ and $n\in N$. If we let
\begin{align*}
  m &= \sum_{i=1}^{k}\lambda_i x_i\\
  n &= \sum_{j=1}^{\ell}\mu_j y_j,
\end{align*}
then we will define
\begin{align*}
  m\otimes n &= \sum_{i=1}^{k}\sum_{j=1}^{\ell} \lambda_i\mu_j \left( x_i\otimes y_j \right).
\end{align*}
We observe that every element of $M\bar\otimes N$ is a sum (i.e., an \textit{integral} linear combination) of simple tensors, as by regrouping we may take
\begin{align*}
  \sum_{i=1}^{k}\sum_{j=1}^{\ell}\lambda_{ij}\left( x_i\otimes y_j \right) &= \sum_{i=1}^{k}\sum_{j=1}^{\ell} \left( \lambda_{ij}x_i \right)\otimes y_j.
\end{align*}
The simple tensors satisfy the following relations:
\begin{enumerate}[(R1)]
  \item $\left( m_1 + m_2 \right)\otimes n = m_1\otimes n + m_2\otimes n$;
  \item $m\otimes \left( n_1 + n_2 \right) = m\otimes n_1 + m\otimes n_2$;
  \item $\left( \alpha m \right)\otimes n = m\otimes \left( \alpha n \right)$
\end{enumerate}
for $m,m_1,m_2\in M$, $n,n_1,n_2\in N$, and $\alpha\in R$.
\begin{proposition}
  These are the defining relations for $M\bar\otimes N$ in the category of abelian groups.
\end{proposition}
We will simply take this proposition as fact.

Now, let
\begin{align*}
  Q &= M\times N\\
    &= \set{\left( m,n \right) | m\in M,n\in N}
\end{align*}
be the Cartesian product of $M$ and $N$ as sets. We will then take $\Z\left[ Q \right]$ to be the standard free $\Z$-module (i.e., free abelian group) on $Q$. That is, $\Z\left[ Q \right]$ is the set of formal linear combinations 
\begin{align*}
  v &= \sum_{q\in Q} \lambda_q q,
\end{align*}
where $\lambda_q\in \Z$ and only finitely many coefficients are nonzero. By the universal property of free abelian groups, the map $\left( m,n \right)\mapsto m\otimes n$ descends to a unique homomorphism $\varphi\colon\Z\left[ Q \right]\rightarrow M\bar\otimes N$. Such a homomorphism is necessarily surjective as every element of $M\bar\otimes N$ is an integral linear combination of simple tensors, meaning that we have
\begin{align*}
  M\bar\otimes N &\cong \Z\left[ Q \right]/\ker\left( \varphi \right)
\end{align*}
as abelian groups.

Now, consider the subgroup of $\Z\left[ Q \right]$, which we denote $K$, that is generated by the following elements:
\begin{enumerate}[(I)]
  \item $\left( m_1 + m_2,n \right) - \left( m_1,n \right) - \left( m_2,n \right)$;
  \item $\left( m,n_1 + n_2 \right) - \left( m,n_1 \right) - \left( m,n_2 \right)$;
  \item $\left( \alpha m,n \right) - \left( m,\alpha n \right)$
\end{enumerate}
for $m_1,m_2,m\in M$, $n_1,n_2,n\in N$, and $\alpha\in R$. Then, from proposition that the relations (R1) through (R3) define $M\bar\otimes N$, it follows that $K = \ker\left( \varphi \right)$. Thus, we may define the tensor product canonically as follows.
\begin{definition}
  Letting $M,N,Q,K$ be as above, we define
  \begin{align*}
    M\otimes N &\coloneq \Z\left[ Q \right]/K,
  \end{align*}
  and define $m\otimes n = \left( m,n \right) + K$.
\end{definition}

\end{document}
