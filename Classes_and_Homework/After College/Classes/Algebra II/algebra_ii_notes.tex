\documentclass[10pt]{mypackage}

\usepackage{mlmodern}
%\usepackage{newpxtext,eulerpx,eucal}
%\renewcommand*{\mathbb}[1]{\varmathbb{#1}}

%\usepackage{homework}
\usepackage{notes}

%\usepackage[ backend=bibtex, style = alphabetic, sorting=ynt ]{biblatex}
%\addbibresource{  }

\usepackage{parskip}

\fancyhf{}
\fancyhead[R]{Avinash Iyer}
\fancyhead[L]{Algebra II: Class Notes}
\fancyfoot[C]{\thepage}

\setcounter{secnumdepth}{0}

\begin{document}
\RaggedRight
The primary text for Algebra II is Dummit and Foote's \textit{Abstract Algebra}, and will cover the following topics:
\begin{itemize}
  \item modules and advanced linear algebra;
  \item representation theory of finite groups;
  \item field theory and Galois theory.
\end{itemize}
\tableofcontents
\section{Modules and Advanced Linear Algebra}%
\subsection{Tensor Products of Modules}%
To motivate tensor products, we recall a basic fact from linear algebra. If we assume that $R$ is a field, and $M,N$ are finite-dimensional $R$-vector spaces, then the following equation necessarily holds:
\begin{align*}
  \dim\left( M\oplus N \right) &= \dim\left( M \right) + \dim\left( N \right).
\end{align*}
We want to construct a similar operation on vector spaces, $M\otimes N$, that satisfies
\begin{align*}
  \dim\left( M\otimes N \right) &= \dim\left( M \right)\dim\left( N \right).
\end{align*}
For now, we will label this by $M\bar{\otimes}N$, where we use the $\bar{\otimes}$ to refer to the fact that this is a temporary definition. Naively, we might seek to define $M\bar\otimes N$ as follows. If we let $\set{x_1,\dots,x_k}$ be a basis for $M$ and $\set{y_1,\dots,y_{\ell}}$ a basis for $N$, then we will define $M\bar\otimes N$ to be all the formal $R$-linear combinations over the basis
\begin{align*}
  B &= \set{x_i\otimes y_j | 1\leq i \leq k,1\leq j \leq \ell}.
\end{align*}
\end{document}
