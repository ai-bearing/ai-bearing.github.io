\documentclass[10pt]{mypackage}

\usepackage{mlmodern}
%\usepackage{newpxtext,eulerpx,eucal}
%\renewcommand*{\mathbb}[1]{\varmathbb{#1}}

\usepackage{homework}
%\usepackage{notes}

%\usepackage[ backend=bibtex, style = alphabetic, sorting=ynt ]{biblatex}
%\addbibresource{  }

\usepackage{parskip}

\fancyhf{}
\fancyhead[R]{Avinash Iyer}
\fancyhead[L]{Algebra II: Homework 3}
\fancyfoot[C]{\thepage}

\setcounter{secnumdepth}{0}

\begin{document}
\RaggedRight
\begin{problem}[Problem 2]
  Let $R$ be a PID. For an $R$-module $M$, denote by $d(M)$ the minimal number of generators of $M$.
  \begin{enumerate}[(a)]
    \item Prove that if $M$ is a finitely generated $R$-module, and $N$ is a submodule of $M$, then $d(N)\leq d(M)$.
    \item Let $a\in R$ be a nonzero non-unit. Find (with proof) the number of submodules of $R/aR$ in terms of the prime decomposition of $a$.
  \end{enumerate}
\end{problem}
\begin{solution}\hfill
  \begin{enumerate}[(a)]
    \item Let $\set{v_1,\dots,v_n}$ be a minimal generating set for $M$. Via the surjection $R^n\rightarrow M$ taking $\left( r_1,\dots,r_n \right) \mapsto \sum_{i=1}^{n}r_iv_i$, we observe that $M\cong R^n/G$ for some submodule $G$ of $R^n$. Since $N$ is a submodule of $M$, it follows from the fourth isomorphism theorem that $N$ corresponds to a submodule of $R^n$ containing $G$, which we will call $N'$; since $N'$ is a submodule of a free module, it is free with rank $m\leq n$, and $N'$ surjects onto $N$ so that $d(N)\leq d\left( N' \right)\leq n = d(M)$.
    \item Without loss of generality, let $a = p_1^{d_1}\cdots p_t^{d_t}$ be the prime decomposition for $a$, where $d_i\in \N$. From the Chinese Remainder Theorem, we have
      \begin{align*}
        R/\left( a \right) &\cong R/\left( p_1^{d_1} \right)\oplus\cdots\oplus R/\left( p_t^{d_t} \right).
      \end{align*}
      Observe that any submodule of $R/\left( a \right)$ is in correspondence with an ideal containing $a$ (by the fourth isomorphism theorem). These ideals are precisely the ideals of products of prime powers that are less than or equal to $a$. Since, given $p_i^{d_i}$, there are $0,\dots,d_i$ potential options for the power of $p_i$, so that there are $\left( d_1 + 1 \right)\cdots \left( d_t + 1 \right)$ submodules in $R/\left( a \right)$.
  \end{enumerate}
\end{solution}
\begin{problem}[Problem 3]\hfill
  \begin{enumerate}[(a)]
    \item Let $R$ be a PID, $M$ a finitely generated $R$-module, and
      \begin{align*}
        M &= \left( \bigoplus_{i=1}^{\ell}R/\left( a_i \right) \right)\oplus R^s
      \end{align*}
      its invariant factor decomposition. Prove that $d(M) = m + s$.
    \item Again let $R$ be a PID. Let $F$ be a free $R$-module of rank $n$ with basis $e_1,\dots,e_n$, $N$ the submodule of $F$ generated by some elements $v_1,\dots,v_n\in F$, and let $A \in \Mat_n(F)$ be the matrix such that
      \begin{align*}
        \begin{pmatrix}v_1\\\vdots\\v_n\end{pmatrix} &= A \begin{pmatrix}e_1\\\vdots\\e_n\end{pmatrix}.
      \end{align*}
      Find a simple condition on the entries of $A$ which holds if and only if $d\left( F/N \right) = n$.
  \end{enumerate}
\end{problem}
\begin{solution}\hfill
  \begin{enumerate}[(a)]
    \item Let $p$ be a prime dividing $a_1$. Now, we observe that
      \begin{align*}
        M/pM &\cong \frac{\left( \bigoplus_{i=1}^{m}R/\left( a_i \right) \right)\oplus R^{s}}{\left( \bigoplus_{i=1}^{m}p\left( R/\left( a_i \right) \right) \right)\oplus \left( pR \right)^{s}}\\
             &\cong \left( R/\left( p \right) \right)^{m + s}.
      \end{align*}
      Now, since $R$ is a PID, any prime ideal is maximal, meaning that $d(M') = m + s$ is the dimension of $M' = \left( R/\left( p \right) \right)^{m + s}$. Since $M'$ is a quotient of $M$, it follows that $d(M)\geq m + s$.

      Yet, since the set $\set{e_i | 1\leq i \leq m + s}$ generates $M$ as an $R$-module, it follows that $d(M)\leq m + s$, so that $d(M) = m + s$.
    \item Observe that $F/N$ is a finitely generated module over $R$, and that the invariant factors for $M\cong F/N$ emerge from the Smith normal form for $A$ (as discussed in the proof of the classification in invariant factors form). Therefore, we may write
      \begin{align*}
        M &= \bigoplus_{i=1}^{\ell}R/\left( a_i \right) \oplus R^{k}.
      \end{align*}
      Now, if $k + \ell = n$, then it follows that none of the ideals $\left( a_i \right)$ are equal to $R$ (as else their quotient would be equal to $0$), implying that the Smith normal form of $A$ does not admit any units.

      Similarly, if the Smith normal form of $A$ does not admit any units, then we would have that there are $\ell$ such $\left( a_i \right)$ in the invariant factors decomposition of $M$, with the remaining $n - \ell$ as free factors, or that $d(M) = n$.
  \end{enumerate}
\end{solution}
\begin{problem}[Problem 4]
  Let $M$ be a module over the integral domain $R$.
  \begin{enumerate}[(a)]
    \item Suppose that $M$ has rank $n$, and that $x_1,\dots,x_n$ is any maximal set of linearly independent elements of $M$. Let $N = \left\langle x_1,\dots,x_n \right\rangle$ be the submodule generated by $x_1,\dots,x_n$. Prove that $N$ is isomorphic to $R^n$ and that the quotient $M/N$ is a torsion $R$-module.
    \item Prove conversely that if $M$ contains a submodule $N$ that is free of rank $n$ such that the quotient $M/N$ is a torsion $R$-module, then $M$ has rank $n$.
  \end{enumerate}
\end{problem}
\begin{solution}\hfill
  \begin{enumerate}[(a)]
    \item Let $\pi\colon R^n\rightarrow N$ be the projection onto $N$ given by $\left( r_1,\dots,r_n \right)\mapsto \sum_{i=1}^{n}r_ix_i$. We observe that this projection is surjective by the definition of $N$, and it is injective since $\set{x_1,\dots,x_n}$ are linearly independent (hence the only way for the sum to equal zero is for each of the $r_i$ to equal zero). Thus, $N$ is isomorphic to $R^n$.

      Now, suppose toward contradiction that $M/N$ is not torsion. That is, there is some nonzero $x + N\in M/N$ such that $rx + N\neq 0$ for all $0\neq r\in R$. This gives that $rx\notin N$ for all $0\neq r\in R$, so if we have
      \begin{align*}
        0 &= \sum_{i=1}^{n}r_ix_i + rx,
      \end{align*}
      then upon taking quotients, we have that $r = 0$, and since the set $\set{x_1,\dots,x_n}$ are linearly independent, we must have that each of the $r_i$ are equal to zero, which would imply that $x$ was independent of $\set{x_1,\dots,x_n}$, contradicting maximality. Therefore, $M/N$ is torsion.
    \item Suppose $M$ contains a submodule $N$ of rank $n$ such that $M/N$ is torsion. Suppose $\set{y_1,\dots,y_{n+1}}$ are any $n+1$ elements of $M$. From our work above, we see that an equivalent result is that for any $y\in M$, we have some $0\neq r\in R$ such that $ry$ can be written as a linear combination of the maximally linearly independent set $\set{x_1,\dots,x_n}$. For each $i$, we do this, yielding
      \begin{align*}
        r_iy_i &= \sum_{j=1}^{n}a_{ij}x_j.
      \end{align*}
      This gives that we may write the collection $\set{r_1y_1,\dots,r_{n+1}y_{n+1}}$ as a collection of $n+1$ vectors in $N\cong R^{n}$, meaning that we get a linear dependence relation
      \begin{align*}
        s_1r_1y_1 + \cdots + s_{n+1}r_{n+1}y_{n+1} &= 0
      \end{align*}
      with not all $s_i = 0$ since $R$ is an integral domain. Therefore, we have that the collection $\set{y_1,\dots,y_{n+1}}$ is linearly dependent, so $M$ necessarily has rank $n$.
  \end{enumerate}
\end{solution}
\begin{problem}[Problem 5]
  Let $R$ be a PID, $M$ a finitely generated free $R$-module, and $N$ a submodule of $M$. Prove that the following are equivalent:
  \begin{enumerate}[(i)]
    \item any basis of $N$ can be extended to a basis of $M$;
    \item some basis of $N$ can be extended to a basis of $M$;
    \item $M/N$ is free;
    \item $M/N$ is torsion-free.
  \end{enumerate}
\end{problem}
\begin{solution}
  We will show (i) implies (ii) implies (iii) implies (i), and separately show (iii) holds if and only if (iv) holds.

  First, the implication (i) implies (ii) follows from the fact that any submodule of a free module is free, meaning that it admits a basis, which by the assumption in (i) means said basis can be extended to one for $M$.

  Now, suppose there is a basis $\set{y_1,\dots,y_m}$ of $N$ that can be extended to a basis $\set{y_1,\dots,y_n}$ of $M$. Upon taking quotients, we observe that the set $\set{y_{m+1}+N,\dots,y_{n}+N}$ generates $M/N$ since the full basis generates $M$. Furthermore, we have
  \begin{align*}
    \sum_{k=m+1}^{n} a_k\left( y_k + N \right) &= \left( \sum_{k=m+1}^{n}a_ky_k \right) + N,
  \end{align*}
  which equals zero in $M/N$ if and only if the sum is contained in $N$, but that would contradict the linear independence of the set $\set{y_1,\dots,y_n}$, meaning that $M/N$ is free.

  Suppose now that $M/N$ is free with basis $\set{x_1 + N,\dots,x_{\ell} + N}$, and let $\set{y_1,\dots,y_m}$ be a basis for $N$. We claim that $\set{y_1,\dots,y_m,x_1,\dots,x_{\ell}}$ is a basis for $M$. To see that it is generating, observe that if $v\in M$ is any element, then we may write
  \begin{align*}
    v+N &= \left( \sum_{k=1}^{\ell} a_kx_k \right) + N,
  \end{align*}
  and
  \begin{align*}
    v-\left( \sum_{k=1}^{\ell}a_kx_k \right) &= \sum_{i=1}^{m}b_iy_i,
  \end{align*}
  so that
  \begin{align*}
    v &= \sum_{i=1}^{m}b_iy_i + \sum_{k=1}^{\ell} a_kx_k.
  \end{align*}
  For linear independence, we let
  \begin{align*}
    \sum_{i=1}^{n}b_iy_i + \sum_{k=1}^{\ell}a_kx_k &= 0.
  \end{align*}
  Taking quotients, we observe that we have
  \begin{align*}
    \left( \sum_{k=1}^{\ell}a_kx_k \right) + N &= 0,
  \end{align*}
  implying that the sum $\sum_{k=1}^{\ell}a_kx_k\in N$, but this can only happen if $a_1,\dots,a_k = 0$. Therefore, we get
  \begin{align*}
    \sum_{i=1}^{m}b_iy_i &= 0,
  \end{align*}
  meaning that $b_1,\dots,b_m = 0$ since the $y_i$ form a basis. Notice that this expression is well-defined since any two $x_k$ differ by an element of $N$, which may then be absorbed into the linear combination of the $y_i$. This gives (i).

  Finally, we let $P = M/N$. Since $P$ is a finitely generated module over $R$, it admits an invariant factors decomposition
  \begin{align*}
    P &= \bigoplus_{i=1}^{\ell}R/\left( a_i \right) \oplus R^{s}.
  \end{align*}
  Observe that $P$ has non-trivial invariant factors if and only if it has torsion, meaning that $P$ is free if and only if it is torsion-free.
\end{solution}
\begin{problem}[Problem 6]\hfill
  \begin{enumerate}[(i)]
    \item Let $V$ and $W$ be $R$-modules over a commutative ring. Show that there is a natural homomorphism $\varphi\colon V^{\ast}\otimes W\rightarrow \Hom_{R}\left( V,W \right)$ such that $\left( \varphi\left( f\otimes w \right) \right)\left( v \right) = f(v)w$.
    \item Assume that $W$ is a finitely generated free $R$-module. Prove that $\varphi$ is an isomorphism.
    \item Give examples showing that $\varphi$ need not be surjective if $W$ is either not free or $W$ is free but not finitely generated.
  \end{enumerate}
\end{problem}
\begin{solution}\hfill
  \begin{enumerate}[(i)]
    \item We consider the map $\phi\colon V^{\ast}\times W \rightarrow \Hom_{R}\left( V,W \right)$ given by $\phi\left( f,w\right)v = f(v)w$. We observe that $\phi$ is bilinear, as
      \begin{align*}
        \phi\left( f_1 + f_2,w \right)v &= \left( f_1 + f_2 \right)\left( v \right)w\\
                                        &= \left( f_1(v) + f_2(v) \right)w\\
                                        &= f_1(v)w + f_2(v)w\\
                                        &= \left( \phi\left( f_1,w \right) + \phi\left( f_2,w \right) \right)v\\
        \phi\left( f,w_1 + w_2 \right)v &= f\left( v \right)\left( w_1 + w_2 \right)\\
                                        &= \left( \phi\left( f,w_1 \right) + \phi\left( f,w_2 \right) \right)v.
      \end{align*}
      Therefore, $\phi$ induces a linear map $\varphi\colon V^{\ast}\otimes_{R}W\rightarrow \Hom_{R}\left( V,W \right)$.
    \item Let $W$ be a finitely-generated free $R$-module. Then, we have that
      \begin{align*}
        V^{\ast}\otimes_{R}W &\cong V^{\ast}\otimes_{R} \left( \bigoplus_{i=1}^{n}R \right)\\
                             &\cong \bigoplus_{i=1}^{n}V^{\ast}\otimes_{R}R.
      \end{align*}
      Now, we come to the question of what exactly $V^{\ast}\otimes_{R}R$ is. 
  \end{enumerate}
\end{solution}
\end{document}
