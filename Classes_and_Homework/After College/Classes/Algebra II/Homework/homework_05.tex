\documentclass[10pt]{mypackage}

\usepackage{mlmodern}
%\usepackage{newpxtext,eulerpx,eucal}
%\renewcommand*{\mathbb}[1]{\varmathbb{#1}}

\usepackage{homework}
%\usepackage{notes}

%\usepackage[ backend=bibtex, style = alphabetic, sorting=ynt ]{biblatex}
%\addbibresource{  }

\usepackage{parskip}

\fancyhf{}
\fancyhead[R]{Avinash Iyer}
\fancyhead[L]{Algebra II: Homework 5}
\fancyfoot[C]{\thepage}

\setcounter{secnumdepth}{0}

\begin{document}
\RaggedRight
\begin{problem}[Problem 1]
  Let $F$ be a field. Use JCF to prove that any square matrix $A\in \Mat_n(F)$ is similar to its transpose.
\end{problem}
\begin{solution}
  It suffices to prove this for one Jordan block, as we may then apply the same process for every other separate Jordan block to obtain our desired transformation in the general case. For this, we observe that if $J_d(\lambda)$ is a Jordan block, then it is the expression of a linear transformation $T\colon V\rightarrow V$ in terms of the ordered basis
  \begin{align*}
    \beta_1 &= \set{ \overline{\left( x-\lambda \right)}^{d-1},\dots, \overline{\left( x-\lambda \right)}, \overline{1}},
  \end{align*}
  when $V$ is given the structure of an $F[x]$ module with $T$ acting via $x.v = Tv$. Reversing the order of this basis gives 
  \begin{align*}
    \beta_2 &= \set{ \overline{1}, \overline{\left( x-\lambda \right)},\dots, \overline{\left( x-\lambda \right)}^{d-1} },
  \end{align*}
  and we observe that acting via $T$ gives a matrix representation with $\lambda$ along the diagonal and $1$ along the sub-diagonal, which is exactly $J_{d}(\lambda)^{T}$. Thus, via the change of basis matrix $ P = \left[ \id \right]_{\beta_1}^{\beta_2} $, we find that $J_{d}(\lambda) = P^{-1}J_{d}(\lambda)^{T} P$, so $J_d(\lambda)$ is similar to its transpose.
\end{solution}
\begin{problem}[Problem 2]
  Let $G$ be a group, $F$ a field, and $V$ a vector space over $F$. Prove that there is a natural bijection between linear representations of $G$ of the form $\left( \rho,V \right)$ and $F[G]$-module structures on $V$ that extend the given $F$-vector space structure on $V$.
\end{problem}
\begin{solution}
  Let $\left( \rho,V \right)$ be a linear representation of $G$. We will determine an $F[G]$-module structure on $V$ extending the given $F$-vector space structure on $V$ by letting the basis $\set{\delta_g}_{g\in G}\subseteq F[G]$ act on vectors in $V$ via
  \begin{align*}
    \delta_g\cdot v &= \rho(g)v,
  \end{align*}
  and extending linearly via the universal property of the free module $F[G]$. That this is a module over $F[G]$ follows from the fact that $\rho$ is a homomorphism between $G$ and $\GL(V)$, so for any $g,h\in G$ and $v,v_1,v_2\in V$, we have
  \begin{align*}
    \delta_g\cdot \left( v_1 + v_2 \right) &= \rho(g)\left( v_1 + v_2 \right)\\
                                           &= \rho(g)v_1 + \rho(g)v_2\\
                                           &= \delta_g\cdot v_1 + \delta_g\cdot v_2\\
    \left( \delta_g + \delta_h \right)\cdot v &= \left( \rho(g) + \rho(h) \right)v\\
                                              &= \rho(g)v + \rho(h)v\\
                                              &= \delta_g\cdot v + \delta_h\cdot v\\
    \left( \delta_{g}\delta_h \right)\cdot v &= \delta_{gh}\cdot v\\
                                             &= \rho\left( gh \right)v\\
                                             &= \rho(g)\rho(h)v\\
                                             &= \rho(g)\left( \rho(h)v \right)\\
                                             &= \delta_g\left( \delta_h\cdot v \right).
  \end{align*}
  Now, if we have an $F[G]$-module structure on $V$ extending the $F$-vector space structure on $V$, then we claim that by defining $\rho\colon G\rightarrow \GL(V)$ by
  \begin{align*}
    \rho(g)v &= \delta_g\cdot v,
  \end{align*}
  then this defines a representation of $G$. We observe that $\delta_e\cdot v = v$ by the definition of the $F[G]$-module structure extending the $F$-vector space structure, that
  \begin{align*}
    v &= \delta_e\cdot v\\
      &= \delta_g\delta_{g^{-1}}\cdot v \\
      &= \delta_g\cdot \left( \delta_{g^{-1}}\cdot v \right),
  \end{align*}
  meaning that this is in fact a map into $\GL(V)$. Finally, we observe that if $g,h\in G$, then
  \begin{align*}
    \rho\left( gh \right)v &= \delta_{gh}\cdot v\\
                           &= \delta_g\delta_h\cdot v\\
                           &= \delta_g\cdot \left( \delta_h\cdot v \right)\\
                           &= \delta_g\cdot \left( \rho(h)v \right)\\
                           &= \rho(g)\rho(h)v,
  \end{align*}
  and since this holds for all $v\in V$, it follows that $\rho\left( gh \right) = \rho(g)\rho(h)$, giving that this is a homomorphism. 
\end{solution}
\begin{problem}[Problem 3]
  This problem collects several related results, each of which may be referred to as Schur's Lemma.
  \begin{enumerate}[(a)]
    \item Let $R$ be a ring, $M$ and $N$ irreducible (left) $R$-modules, and $f\colon M\rightarrow N$ a homomorphism of $R$-modules. Prove that $f$ is either an isomorphism or the zero map.
    \item Let $R$ be a ring and $M$ an irreducible $R$-module. Prove that $\operatorname{end}_R(M)$ is a division ring.
    \item Let $G$ be a group, $g\in Z(G)$, and $\left( \rho,V \right)$ a linear representation over some field. Prove that for any $\lambda\in F$, the map $\rho(g)-\lambda I$ lies in $\operatorname{end}_{F[G]}(V)$.
    \item In the setting of (c), assume that $F$ is algebraically closed and $V$ is finite-dimensional and irreducible. Use (b) and (c) to prove that $\rho(g) = \lambda I$ for some $\lambda\in F$. In other words, if we are given a finite-dimensional irreducible representation of an algebraically closed field, then any central element must act as a scalar operator.
  \end{enumerate}
\end{problem}
\begin{solution}\hfill
  \begin{enumerate}[(a)]
    \item Suppose $f\colon M\rightarrow N$ is a nonzero homomorphism of irreducible $R$-modules $M$ and $N$. Then, $\ker\left( f \right)\leq M$ is a submodule, and since $\ker\left( f \right)$ is not equal to all of $M$, it follows that $\ker\left( f \right) = \set{0}$, so $f$ is injective. Similarly, $\img\left( f \right)\leq N$ is a nonzero submodule, so $\img\left( f \right) = N$, so $f$ is an isomorphism.
    \item Observe that if $f\in \operatorname{end}_R(M)$, then we know from (a) that $f$ is either the zero map or an isomorphism. In particular, every nonzero element of $\operatorname{end}_R(M)$ is invertible, so $\operatorname{end}_R(M)$ is a division ring.
    \item Suppose $\delta_h$ is any basis element of $F[G]$. We will show that $\left( \rho(g)-\lambda I \right)\left( \delta_h\cdot v \right) = \delta_h\cdot \left( \left( \rho(g)-\lambda I \right)v \right)$. We have
      \begin{align*}
        \left( \rho(g)-\lambda I \right)\left( \delta_h\cdot v \right) &= \left( \rho(g)-\lambda I \right)\left( \rho(h)v \right)\\
                                                                       &= \left( \rho(gh)-\lambda\rho(h) \right)v\\
                                                                       &= \left( \rho\left( hg \right)-\lambda \rho(h) \right)v\\
                                                                       &= \rho(h)\left( \rho(g)-\lambda I \right)v\\
                                                                       &= \delta_h\cdot \left( \left( \rho(g)-\lambda I \right)v \right).
      \end{align*}
      Thus, this is an $F[G]$-module endomorphism.
    \item If $\left( \rho,V \right)$ is an irreducible representation of $G$ over an algebraically closed field $F$, then if $g\in Z(G)$, for all $\lambda\in F$, we have $\rho(g)-\lambda I$ is in $\operatorname{end}_{F[G]}(V)$. Since $\operatorname{end}_{F[G]}(V)$ is a division ring by (b), it follows that for all $\lambda\in F$, either $\rho(g)-\lambda I$ is zero or it is invertible. From linear algebra, we know that every linear map on a finite-dimensional vector space over an algebraically closed field admits an eigenvalue, so there is $\lambda\in F$ such that $\rho(g)-\lambda I = 0$, so $\rho(g) = \lambda I$. 
  \end{enumerate}
\end{solution}
\begin{problem}[Problem 4]
  Let $R$ be a ring, and $M$ an $R$-module. We will say that $M$ has the complement property if, for every submodule $N$ of $M$, there exists a submodule $P$ such that $M = N\oplus P$. From a theorem in class, $M$ has the complement property if and only if $M$ is completely reducible.
  \begin{enumerate}[(a)]
    \item Suppose $M = P\oplus Q$ for some submodules $P$ and $Q$. Prove that if $N$ is any submodule containing $P$, then $N = P\oplus \left( N\cap Q \right)$.
    \item Deduce from (a) that if $M$ has the complement property, then so does any submodule of $M$.
    \item Now prove lemma $11.2$ from class, which asserts that if $M$ has the complement property, then any nonzero submodule $L$ of $M$ contains an irreducible submodule.
  \end{enumerate}
\end{problem}
\begin{solution}\hfill
  \begin{enumerate}[(a)]
    \item Since $M= P\oplus Q$, we have that $M/P\cong Q$ via the short exact sequence $0\rightarrow P\rightarrow M\rightarrow Q \rightarrow 0$. Furthermore, since $N$ contains $P$, the fourth isomorphism theorem gives that $N/P$ is a submodule of $M/P$ with $N/P\cong N\cap Q$. Therefore, via the short exact sequence $0\rightarrow P\rightarrow N\rightarrow N\cap Q\rightarrow 0$, we observe that $N\cong P\oplus N\cap Q$.
    \item Suppose $M$ has the complement property. If $N$ is any submodule of $M$, then either $N$ contains no nonzero submodules, so it has the complement property via $N = N\oplus 0$, or $N$ contains a nonzero submodule, which we will call $P$. Since $P$ is complemented in $M$, there is $Q$ such that $M = P\oplus Q$. Yet, this means that $N = P\oplus \left( N\cap Q \right)$, so $N$ has the complement property.
    \item Suppose $M$ has the complement property, and let $N\subseteq M$ be a submodule, with $x\in N$. Then, the submodule generated by $x$, $L = Rx$ also has the complement property by (b). Note that $Rx\cong R$ by forgetting the $x$, so any submodule of $Rx$ can be viewed as any submodule of $R$ viewed as an $R$-module. Since the submodules of $R$ are the ideals, then $R$ has a maximal ideal $P$; the corresponding submodule is $K = Px$, and does not contain $x$ as $P$ does not contain $1$. Since $L$ has the complement property, we have $L = K \oplus B$, where $B$ is another submodule of $B$. Thus, we have $L/K = B$, and we observe that $L/K = Rx/Px\cong R/P$. Since $R/P$ is a field, it follows that $R/P$ has no $R$-submodules(/ideals) other than $0$ and the whole space. Thus, $B$ has no submodules; since $x\notin K$, it follows that $x\in B$, so $B$ is an irreducible submodule of $L$, which is a submodule of $N$. Thus, every submodule of $M$ has an irreducible submodule.
  \end{enumerate}
\end{solution}
\begin{problem}[Problem 5]
  Let $n\in \N$, and let $[n]$ denote the set $\set{1,2,\dots,n}$. Let $S_n$ be the symmetric group over $[n]$. Let $F$ be any field, $\left( \rho,V \right)$ the permutation representation of $S_n$ over $F$ corresponding to the defining action of $S_n$ on $[n]$. If we think of $V = F^n$, then $\rho\colon S_n\rightarrow \GL_n(F)$ is given by $\rho\left(\sigma\right)\left( e_i \right) = e_{\sigma(i)}$.
  \begin{enumerate}[(a)]
    \item Let 
      \begin{align*}
        Z &= F\left( e_1 + \cdots + e_n \right)\\
        W &= \set{\left( x_1,\dots,x_n \right)\in F^n | \sum_{i=1}^{n}x_i = 0}.
      \end{align*}
      Prove that $Z$ and $W$ are both subrepresentations of $V$.
    \item Prove that $V = W \oplus Z$ if and only if $\operatorname{char}(F)$ does not divide $n$.
    \item Assume that $\operatorname{char}(F)$ does not divide $n!$. Prove that $W$ is an irreducible representation of $S_n$.
  \end{enumerate}
\end{problem}
\begin{solution}\hfill
  \begin{enumerate}[(a)]
    \item If $\sigma$ is any permutation, then we observe that
      \begin{align*}
        e_1 + \cdots + e_n &= e_{\sigma(1)} + \cdots + e_{\sigma(n)},
      \end{align*}
      so that $Z$ is invariant under the representation of $\rho$. Similarly, we have
      \begin{align*}
        \sum_{i=1}^{n}x_i &= \sum_{i=1}^{n}x_{\sigma(i)},
      \end{align*}
      so $W$ is invariant under the representation of $\rho$. Furthermore, both $Z$ and $W$ are subspaces, the former by definition and the latter since for any $\lambda\in F$, we have
      \begin{align*}
        \lambda\left( x_1,\dots,x_n \right) &= \left( \lambda x_1,\dots,\lambda x_n \right),
      \end{align*}
      so if $\left( x_1,\dots,x_n \right)\in W$, so too is $\left( \lambda x_1,\dots,\lambda x_n \right)$, and we can split sums to find that $W$ is a subspace.
    \item Consider the map
      \begin{align*}
        \varphi\colon V\rightarrow F
      \end{align*}
      given by
      \begin{align*}
        \varphi\left( \left( x_1,\dots,x_n \right) \right) &= x_1 + \cdots + x_n.
      \end{align*}
      If the characteristic of $F$ does not divide $n$, then we have that $\ker\left( \varphi \right) = W$, and $\img\left( \varphi \right) = F \cong Z$, so that $V = W\oplus Z$.

      Now, suppose the characteristic of $F$ divides $n$. In that case, we observe that $e_1 + e_2 + \cdots + e_n\mapsto 0$ under $\varphi$, meaning that $Z\subseteq \ker\left( \varphi \right)$, or that $Z\subseteq W$, meaning that the sum cannot be direct.
    \item 
  \end{enumerate}
\end{solution}
\end{document}
