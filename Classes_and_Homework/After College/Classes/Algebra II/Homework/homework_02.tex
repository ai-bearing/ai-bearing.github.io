\documentclass[10pt]{mypackage}

\usepackage{mlmodern}
%\usepackage{newpxtext,eulerpx,eucal}
%\renewcommand*{\mathbb}[1]{\varmathbb{#1}}

\usepackage{homework}
%\usepackage{notes}

%\usepackage[ backend=bibtex, style = alphabetic, sorting=ynt ]{biblatex}
%\addbibresource{  }

\usepackage{parskip}

\fancyhf{}
\fancyhead[R]{Avinash Iyer}
\fancyhead[L]{Algebra II: Homework 2}
\fancyfoot[C]{\thepage}

\setcounter{secnumdepth}{0}

\begin{document}
\RaggedRight
\begin{problem}[Problem 1]
  Let $R$ be a Euclidean domain, $n\geq 2$ an integer.
  \begin{enumerate}[(a)]
    \item Use the proof of the Smith Normal Form to show that every matrix $A\in \GL_n\left( R \right)$ can be written as a product of elementary matrices $E_{ij}\left( \lambda \right)$, flip matrices $F_{ij}$, and a diagonal matrix $D$.
    \item Now show that the flip matrices can be eliminated from the product in (a), and one can assume that $D=\operatorname{diag}\left( d,1,\dots,1 \right)$. That is, all diagonal entries of $D$ except possibly the $(1,1)$ entry are equal to $1$.
    \item Deduce from (b) that $\SL_n\left( R \right)$ is generated by the elementary matrices $E_{ij}\left( \lambda \right)$.
  \end{enumerate}
\end{problem}
\begin{solution}\hfill
  \begin{enumerate}[(a)]
    \item Observe that a square matrix is in Smith normal form if and only if it is a diagonal matrix of the form $D = \operatorname{diag}\left( d_1,\dots,d_m,0,\dots,0 \right)$ where $d_1 | d_2 | \cdots | d_m$. By the proof of the Smith normal form, we have that the matrix $UAV$ in Smith normal form is the product of three invertible matrices, so it is invertible, meaning that it is necessarily diagonal with $d_1,\dots,d_n\in R^{\times}$. Since the inverse of any $E_{ij}(\lambda)$ is another matrix of the form $E_{ij}(\lambda)$, and the inverse of $F_{ij}$ is $F_{ji}$, it follows that we may write any $A\in \GL_n\left( R \right)$ as
      \begin{align*}
        A &= U^{-1}DV^{-1},
      \end{align*}
      where $U^{-1}$ and $V^{-1}$ are collections of flips and $E_{ij}(\lambda)$ and $D$ is a diagonal matrix with $d_1,\dots,d_n\in R^{\times}$.
    \item 
  \end{enumerate}
\end{solution}
\end{document}
