\documentclass[10pt]{mypackage}

\usepackage{mlmodern}
%\usepackage{newpxtext,eulerpx,eucal}
%\renewcommand*{\mathbb}[1]{\varmathbb{#1}}

\usepackage{homework}
%\usepackage{notes}

%\usepackage[ backend=bibtex, style = alphabetic, sorting=ynt ]{biblatex}
%\addbibresource{  }

\usepackage{parskip}

\fancyhf{}
\fancyhead[R]{Avinash Iyer}
\fancyhead[L]{Algebra II: Homework 2}
\fancyfoot[C]{\thepage}

\setcounter{secnumdepth}{0}

\begin{document}
\RaggedRight
\begin{problem}[Problem 1]
  Let $R$ be a Euclidean domain, $n\geq 2$ an integer.
  \begin{enumerate}[(a)]
    \item Use the proof of the Smith Normal Form to show that every matrix $A\in \GL_n\left( R \right)$ can be written as a product of elementary matrices $E_{ij}\left( \lambda \right)$, flip matrices $F_{ij}$, and a diagonal matrix $D$.
    \item Now show that the flip matrices can be eliminated from the product in (a), and one can assume that $D=\operatorname{diag}\left( d,1,\dots,1 \right)$. That is, all diagonal entries of $D$ except possibly the $(1,1)$ entry are equal to $1$.
    \item Deduce from (b) that $\SL_n\left( R \right)$ is generated by the elementary matrices $E_{ij}\left( \lambda \right)$.
  \end{enumerate}
\end{problem}
\begin{solution}\hfill
  \begin{enumerate}[(a)]
    \item Observe that a square matrix is in Smith normal form if and only if it is a diagonal matrix of the form $D = \operatorname{diag}\left( d_1,\dots,d_m,0,\dots,0 \right)$ where $d_1 | d_2 | \cdots | d_m$. By the proof of the Smith normal form, we have that the matrix $UAV$ in Smith normal form is the product of three invertible matrices, so it is invertible, meaning that it is necessarily diagonal with $d_1,\dots,d_n\in R^{\times}$. Since the inverse of any $E_{ij}(\lambda)$ is another matrix of the form $E_{ij}(\lambda)$, and the inverse of $F_{ij}$ is $F_{ji}$, it follows that we may write any $A\in \GL_n\left( R \right)$ as
      \begin{align*}
        A &= U^{-1}DV^{-1},
      \end{align*}
      where $U^{-1}$ and $V^{-1}$ are collections of flips and $E_{ij}(\lambda)$ and $D$ is a diagonal matrix with $d_1,\dots,d_n\in R^{\times}$.
    \item 
  \end{enumerate}
\end{solution}
\begin{problem}[Problem 3]
  Let $R$ be a commutative ring with $1$.
  \begin{enumerate}[(a)]
    \item Let $C$ be an $R$-algebra, and $A,B\subseteq C$ $R$-subalgebras that commute with each other; that is, $ab = ba$ for any $a\in A$ and $b\in B$. Prove that there is an $R$-algebra homomorphism $\varphi\colon A\otimes B\rightarrow C$ such that $\varphi\left( a\otimes b \right) = ab$ for each $a\in A$ and $b\in B$.
    \item Prove that $\R\otimes_{\Z}\Z[i]\cong \C$ as rings.
    \item Now assume that $R$ is a field, and let $A$ be a finite-dimensional $R$-algebra. Prove that $A\otimes A$ cannot be a field unless $\dim(A) = 1$. 
  \end{enumerate}
\end{problem}
\begin{solution}\hfill
  \begin{enumerate}[(a)]
    \item Let $\phi\colon A\times B \rightarrow C$ be defined by $\left( a,b \right)\mapsto ab$. Then, $\phi$ is an $R$-bilinear map, so it induces a unique linear map on the tensor product $\varphi\colon A\otimes B \rightarrow C$. We claim that this map is compatible with the $R$-algebra structure of $A\otimes B$.

      To see this, observe that if $a_1,a_2\in A$ and $b_1,b_2\in B$, then
      \begin{align*}
        \varphi\left( \left( a_1\otimes b_1 \right)\left( a_2\otimes b_2 \right) \right) &= \varphi\left( a_1a_2\otimes b_1b_2 \right)\\
                                                                                         &= a_1a_2b_1b_2\\
                                                                                         &= a_1b_1a_2b_2\\
                                                                                         &= \varphi\left( a_1\otimes b_1 \right)\varphi\left( a_2\otimes b_2 \right).
      \end{align*}
      This gives our desired $R$-algebra homomorphism.
    \item We observe that both $\R$ and $\Z[i]$ are $\Z$-subalgebras of $\C$. Therefore, from above, we have a $\Z$-algebra homomorphism
      \begin{align*}
        \varphi\colon \R\otimes_{\Z}\Z[i] &\rightarrow \C\\
        t\otimes \left( a+bi \right) &\mapsto ta + tbi.
      \end{align*}
      To see that this map is injective, observe that $ta + tbi = 0$ if and only if $ta = 0$ and $tb = 0$, meaning either that $t = 0$ or $a,b = 0$; in either case, the corresponding element of the tensor product is the zero tensor. As for surjectivity, if we have $x + yi\in \C$, then we may find the element $x\otimes 1 + y\otimes i\in \R\otimes_{\Z}\Z[i]$ that maps to $x + yi$. Since this is a bijective $\Z$-algebra homomorphism, it follows that $\R\otimes_{\Z}\Z[i]\cong \C$ as $\Z$-algebras, hence as rings.
    \item 
  \end{enumerate}
\end{solution}
\end{document}
