\documentclass[10pt]{mypackage}

% sans serif font:
%\usepackage{cmbright}
%\usepackage{sfmath}
%\usepackage{bbold} %better blackboard bold

%\usepackage{homework}
\usepackage{notes}
\usepackage{newpxtext,eulerpx,eucal}
\renewcommand*{\mathbb}[1]{\varmathbb{#1}}

\fancyhf{}
\rhead{Avinash Iyer}
\lhead{Differential Topology: Notes and Review}

\setcounter{secnumdepth}{0}

\begin{document}
\RaggedRight
\section{Basic Properties}%
\begin{definition}
  A topological space $M$ is called a \textit{manifold} if it satisfies the following:
  \begin{itemize}
    \item $M$ is Hausdorff (points can be separated by open sets);
    \item $M$ is second countable (the basis for the topology of $M$ is countable);
    \item $M$ is locally Euclidean (every point in $M$ has a neighborhood homeomorphic to $\R^n$ for some $n$).
  \end{itemize}
  In particular, the third condition says that for every $p\in M$, there is $U\in \mathcal{O}_p$ and a homeomorphism $\varphi\colon U\rightarrow \R^n$. The value of $n$ is called the \textit{dimension} of the manifold $M$.
\end{definition}
\begin{definition}
  Let $M$ be an $n$-manifold. A \textit{chart} on $M$ is a pair $\left( U,\varphi \right)$ such that $U\subseteq M$ is open, $\varphi\colon U\rightarrow \R^n$ is a homeomorphism.\newline

  A family of charts $\mathcal{A} = \set{\left( U_i,\varphi_i \right)}_{i\in I}$ is known as an \textit{atlas} if
  \begin{align*}
    M &= \bigcup_{i\in I} U_i.
  \end{align*}
\end{definition}
To understand the smooth structure of a manifold, we consider a point $p\in M$ and two charts $\left( U,\varphi_U \right)$ and $\left( V,\varphi_V \right)$ such that $p\in U$ and $p\in V$. The functions $\varphi_U\colon U\rightarrow \R^n$ and $\varphi_V\colon V\rightarrow \R^n$ are homeomorphism, meaning that $\varphi_V\circ\varphi_U^{-1}\colon \varphi_{U}\left( U\cap V \right)^n\rightarrow \R^n$ defined on the (nonempty) $U\cap V$ is also a homeomorphism.\newline

In particular, we develop the smooth structure by making sure all such pairs $\varphi_{V}\circ\varphi_U^{-1}$ are \textit{diffeomorphisms}. To do this, we need to first develop the derivative in $\R^n$.
\begin{definition}
  Let $f\colon \R^n\rightarrow \R^m$ be a function. We say $f$ is \textit{differentiable} at $p\in \R^n$ if there is a linear map $L\in \Hom\left( \R^n,\R^m \right)$ such that
  \begin{align*}
    \frac{\norm{f\left( p+h \right)-f\left( p \right)-Lh}}{\norm{h}} &\rightarrow 0
  \end{align*}
  as $h\rightarrow 0$.\newline

  The \textit{derivative} of $f$ is the association $f\mapsto L$ for each $p\in \R^n$. We write $D_pf$ to denote this map. Note that we consider elements of $\Mat_{n}\left( \R \right)$ as points in $\R^{n^2}$ with the standard topology on $\R^{n^2}$.\newline

  A function $f$ is called a \textit{diffeomorphism} if it is continuously differentiable and has a continuously differentiable inverse.
\end{definition}
\begin{definition}
  If $\left( U,\varphi_U \right)$ and $\left( V,\varphi_V \right)$ are charts such that $U\cap V \neq \emptyset$, the function $\varphi_{V}\circ \varphi_{U}^{-1}\colon \R^n\rightarrow \R^n$ is known as the \textit{transition map} between $\varphi_U$ and $\varphi_V$.\newline

  A \textit{smooth structure} for $M$ is an atlas $\set{\left( U_i,\varphi_{i} \right)}_{i\in I}$ such that for all $i,j\in I$, the transition maps $\varphi_j\circ \varphi_i^{-1}\colon \R^n\rightarrow \R^m$ are diffeomorphisms where defined (if not defined, then it is vacuously so). If $M$ admits a smooth structure, then we call $M$ a smooth manifold.
\end{definition}
\begin{note}
  From now on, we use ``manifold'' to refer to smooth manifolds, and will say \textit{topological} manifolds if the manifold does not necessarily admit a smooth structure.
\end{note}
\begin{definition}
  A map $f\colon M\rightarrow N$ between manifolds is called \textit{smooth} if for any chart $\left( U,\varphi_U \right)$ in $M$ and corresponding chart $\left( V,\varphi_V \right)$ in $N$, the map $\varphi_V\circ f \circ \varphi_U^{-1}\colon \R^n\rightarrow \R^k$ is continuously differentiable.\newline

  The function $f$ is a \textit{diffeomorphism} if $f$ is a smooth bijection with smooth inverse, and we say the manifolds $M$ and $N$ are diffeomorphic if they admit a diffeomorphism.
\end{definition}
In order to replace manifolds with linear maps, we need to understand smooth maps on $\R^n$. The most important theorems in this regard are the inverse function theorem and the implicit function theorem.
\begin{theorem}[Inverse Function Theorem]
  Let $f\colon U\subseteq \R^n \rightarrow \R^n$ be a continuously differentiable function. If $D_{p}f$ is invertible as a linear map, then $f$ has a local, continuously differentiable inverse $f^{-1}\colon V\rightarrow W$, where $p\in W\subseteq U$ and $f(p)\in V\subseteq \R^n$.
\end{theorem}
The proof uses the contraction mapping theorem. Recall that if $X$ is a complete metric space, and $f\colon X\rightarrow X$ is a strict uniform contraction --- that is, there exists $0\leq \lambda < 1$ such that $d\left( f(x),f(y) \right) \leq \lambda d\left( x,y \right)$ for all $x,y\in X$ --- then $f$ has a unique fixed point.\newline

We begin with a technical lemma.
\begin{lemma}
  If $U\left( 0,r \right)\subseteq V$ for some $r > 0$ where $V$ is a normed vector space, $g\colon V\rightarrow V$ is a uniform contraction, and $f = \id + g$, then the following hold:
  \begin{itemize}
    \item $\left( 1-\lambda \right) \norm{x-y}\leq \norm{f(x)-f(y)}$ (in particular, $f$ is injective);
    \item if $g(0) = 0$, then
      \begin{align*}
        U\left( 0,\left( 1-\lambda \right)r \right) \subseteq f\left( U\left( 0,r \right) \right) \subseteq U\left( 0,\left( 1+\lambda \right)r \right).
      \end{align*}
  \end{itemize}
\end{lemma}
\begin{proof}[Proof of Lemma]
  To see the first item, we notice that by the triangle inequality,
  \begin{align*}
    \norm{x-y} - \norm{f(x)-f(y)} &\leq \norm{x-y} - \norm{x-y} + \norm{g(x)-g(y)}\\
                                  &\leq \lambda\norm{x-y},
  \end{align*}
  so $\left( 1-\lambda \right)\norm{x-y}\leq \norm{f(x)-f(y)}$, and $f$ is injective. Furthermore, we see that if $g(0) = 0$, then
  \begin{align*}
    f\left( U\left( 0,r \right) \right) &= U\left( 0,r \right) + g\left( U\left( 0,r \right) \right)\\
                                        &\subseteq U\left( 0,r \right) + \lambda U\left( 0,r \right)\\
                                        &= U\left( 0,\left( 1+\lambda \right)r \right).
  \end{align*}
  Finally, if $y\in U\left( 0,\left( 1-\lambda \right)r \right)$, then we want to find $x$ such that $y = f(x) = x + g(x)$; equivalently, we see that we want $x$ such that $x = y-g(x)$. Since the function $F(x) = y-g(x)$ is a translation of a uniform contraction, $F(x)$ is a contraction, so there is a fixed point, meaning $y\in f\left( U\left( 0,r \right) \right)$.
\end{proof}
\begin{note}
  We will use $\left\vert \cdot \right\vert$ to denote the norm on $\R^n$.
\end{note}
\begin{proof}[Proof of the Inverse Function Theorem]
  By using a series of affine maps --- first by translating $p$ to $0$, then translating $f(p)$ to $0$, then inverting $D_0f$ as per our assumption, we may safely assume that $p = f(p) =0$ and $D_0f = \operatorname{Id}$.\newline

  Set $g = f - \operatorname{Id}$. We will show that $g$ is a contraction in a sufficiently small ball. Fixing $x,y\in \R^n$, consider the map $\R\rightarrow \R^n$ given by $t \mapsto g\left( x + t\left( y-x \right) \right)$. Notice that by the Fundamental Theorem of Calculus,
  \begin{align*}
    \left\vert g(y)-g(x) \right\vert &\leq \left\vert y-x \right\vert \sup_{0\leq t \leq 1} \left\vert g'\left( x + t\left( y-x \right) \right) \right\vert.
  \end{align*}
  Furthermore, since $g'(0) = \mathbf{0}$ by the fact that $D_0f = \operatorname{Id}$ and $\left( \operatorname{Id} \right)' = \operatorname{Id}$, and since $f$ is continuously differentiable, there is $r > 0$ such that
  \begin{align*}
    \left\vert g(y)-g(x) \right\vert &\leq \frac{1}{2}\left\vert y-x \right\vert
  \end{align*}
  for all $x,y\in U\left( 0,r \right)$. Thus, $g$ is a strict contraction on $U\left( 0,r \right)$. By the previous lemma, we see that
  \begin{align*}
    U\left( 0,r/2 \right) &\subseteq f\left( U\left( 0,r \right) \right);
  \end{align*}
  by setting $U = U\left( 0,r \right) \cap f^{-1}\left( U\left( 0,r \right) \right)$, we see that the map $f|_{U}\colon U\rightarrow V \coloneq U\left( 0,r/2 \right)$ is a bijection. The inverse function $f^{-1}\colon V\rightarrow U$ thus exists.\newline

  Now, we let $h = f^{-1}$, $x\in U$, $y\in V$ such that $h(x) = y$, and $A = D_xf$. We will show that $A^{-1} = D_yh$, which is enough to show that $h$ is continuously differentiable, as we assume the map $x \mapsto D_xf$ is continuous, and inversion is continuous in $\GL_n\left(\R\right)$.\newline

  For sufficiently small vectors $s$ and $k$, since $f$ and $h$ are bijections, we have
  \begin{align*}
    h\left( y+k \right) = x+s,
  \end{align*}
  so
  \begin{align*}
    f\left( x+s \right)  &= y+k.
  \end{align*}
  Furthermore, by unraveling the definitions of $f = g + \operatorname{Id}$, $s$, and $k$, and the fact that $g$ is a uniform contraction on $U$, we get
  \begin{align*}
    \left\vert s-k \right\vert &= \left\vert \left( f(x+s) - f(x) \right) - s \right\vert\\
                               &= \left\vert \left( x+s + g(x+s) \right) - \left( x + g(x) \right) -s\right\vert\\
                               &= \left\vert g(x+s) - g(x) \right\vert\\
                               &\leq \frac{\left\vert s \right\vert}{2}.
  \end{align*}
  In particular, since
  \begin{align*}
    \left\vert s \right\vert &\leq \left\vert s-k \right\vert + \left\vert k \right\vert\\
                             &\leq \left\vert k \right\vert + \frac{\left\vert s \right\vert}{2},
  \end{align*}
  we see that $\left\vert s \right\vert/2 \leq \left\vert k \right\vert$. We calculate
  \begin{align*}
    \left\vert h\left( y+k \right)-h\left( y \right) - A^{-1}k \right\vert &= \left\vert x+s-x-A^{-1}\left( f\left( x+s \right)-f(x) \right) \right\vert\\
                                                                           &= \left\vert s - A^{-1}\left( f\left( x+s \right)-f\left( x \right) \right) \right\vert\\
                                                                           &\leq \norm{A^{-1}}_{\op} \left\vert As - f\left( x+s \right) - f\left( x \right) \right\vert.
  \end{align*}
  Thus, since $\left\vert s \right\vert/2 \leq \left\vert k \right\vert$,
  \begin{align*}
    \frac{\left\vert h\left( y+k \right)-h\left( y \right) - A^{-1}k \right\vert}{\left\vert k \right\vert} &\leq \frac{2\norm{A^{-1}}_{\op}\left\vert As-f\left( x+s \right)-f\left( x \right) \right\vert}{\left\vert s \right\vert}\\
                                                                                                            &\rightarrow 0,
  \end{align*}
  so $D_yh = A^{-1}$.
\end{proof}
\end{document}
