\documentclass[10pt]{mypackage}

% sans serif font:
%\usepackage{cmbright}
%\usepackage{sfmath}
%\usepackage{bbold} %better blackboard bold

%\usepackage{homework}
\usepackage{notes}
\usepackage{newpxtext,eulerpx,eucal}
\renewcommand*{\mathbb}[1]{\varmathbb{#1}}

\fancyhf{}
\rhead{Avinash Iyer}
\lhead{Differential Topology: Notes and Review}

\setcounter{secnumdepth}{0}

\begin{document}
\RaggedRight
\section{Basic Properties}%
\begin{definition}
  A topological space $M$ is called a \textit{manifold} if it satisfies the following:
  \begin{itemize}
    \item $M$ is Hausdorff (points can be separated by open sets);
    \item $M$ is second countable (the basis for the topology of $M$ is countable);
    \item $M$ is locally Euclidean (every point in $M$ has a neighborhood homeomorphic to $\R^n$ for some $n$).
  \end{itemize}
  In particular, the third condition says that for every $p\in M$, there is $U\in \mathcal{O}_p$ and a homeomorphism $\varphi\colon U\rightarrow \R^n$. The value of $n$ is called the \textit{dimension} of the manifold $M$.
\end{definition}
\begin{definition}
  Let $M$ be an $n$-manifold. A \textit{chart} on $M$ is a pair $\left( U,\varphi \right)$ such that $U\subseteq M$ is open, $\varphi\colon U\rightarrow \R^n$ is a homeomorphism.\newline

  A family of charts $\mathcal{A} = \set{\left( U_i,\varphi_i \right)}_{i\in I}$ is known as an \textit{atlas} if
  \begin{align*}
    M &= \bigcup_{i\in I} U_i.
  \end{align*}
\end{definition}
To understand the smooth structure of a manifold, we consider a point $p\in M$ and two charts $\left( U,\varphi_U \right)$ and $\left( V,\varphi_V \right)$ such that $p\in U$ and $p\in V$. The functions $\varphi_U\colon U\rightarrow \R^n$ and $\varphi_V\colon V\rightarrow \R^n$ are homeomorphism, meaning that $\varphi_V\circ\varphi_U^{-1}\colon \varphi_{U}\left( U\cap V \right)^n\rightarrow \R^n$ defined on the (nonempty) $U\cap V$ is also a homeomorphism.\newline

In particular, we develop the smooth structure by making sure all such pairs $\varphi_{V}\circ\varphi_U^{-1}$ are \textit{diffeomorphisms}. To do this, we need to first develop the derivative in $\R^n$.
\begin{definition}
  Let $f\colon \R^n\rightarrow \R^m$ be a function. We say $f$ is \textit{differentiable} at $p\in \R^n$ if there is a linear map $L\in \Hom\left( \R^n,\R^m \right)$ such that
  \begin{align*}
    \frac{\norm{f\left( p+h \right)-f\left( p \right)-Lh}}{\norm{h}} &\rightarrow 0
  \end{align*}
  as $h\rightarrow 0$.\newline

  The \textit{derivative} of $f$ is the association $f\mapsto L$ for each $p\in \R^n$.\newline

  A function $f$ is called a \textit{diffeomorphism} if it is continuously differentiable and has a continuously differentiable inverse.
\end{definition}
\begin{definition}
  If $\left( U,\varphi_U \right)$ and $\left( V,\varphi_V \right)$ are charts such that $U\cap V \neq \emptyset$, the function $\varphi_{V}\circ \varphi_{U}^{-1}\colon \R^n\rightarrow \R^n$ is known as the \textit{transition map} between $\varphi_U$ and $\varphi_V$.\newline

  A \textit{smooth structure} for $M$ is an atlas $\set{\left( U_i,\varphi_{i} \right)}_{i\in I}$ such that for all $i,j\in I$, the transition maps $\varphi_j\circ \varphi_i^{-1}\colon \R^n\rightarrow \R^m$ are diffeomorphisms where defined (if not defined, then it is vacuously so). If $M$ admits a smooth structure, then we call $M$ a smooth manifold.
\end{definition}
\begin{note}
  From now on, we use ``manifold'' to refer to smooth manifolds, and will say \textit{topological} manifolds if the manifold does not necessarily admit a smooth structure.
\end{note}
\begin{definition}
  A map $f\colon M\rightarrow N$ between manifolds is called \textit{smooth} if for any chart $\left( U,\varphi_U \right)$ in $M$ and corresponding chart $\left( V,\varphi_V \right)$ in $N$, the map $\varphi_V\circ f \circ \varphi_U^{-1}\colon \R^n\rightarrow \R^k$ is continuously differentiable.\newline

  The function $f$ is a \textit{diffeomorphism} if $f$ is a smooth bijection with smooth inverse, and we say the manifolds $M$ and $N$ are diffeomorphic if they admit a diffeomorphism.
\end{definition}
\end{document}
