\documentclass[10pt]{mypackage}

% sans serif font:
%\usepackage{cmbright}
%\usepackage{sfmath}
%\usepackage{bbold} %better blackboard bold

%\usepackage{homework}
\usepackage{notes}
\usepackage{newpxtext,eulerpx,eucal}
\renewcommand*{\mathbb}[1]{\varmathbb{#1}}

\fancyhf{}
\rhead{Avinash Iyer}
\lhead{Differential Topology: Notes and Review}

\setcounter{secnumdepth}{0}

\begin{document}
\RaggedRight
\section{Basic Properties}%
\begin{definition}
  A topological space $M$ is called a \textit{manifold} if it satisfies the following:
  \begin{itemize}
    \item $M$ is Hausdorff (points can be separated by open sets);
    \item $M$ is second countable (the basis for the topology of $M$ is countable);
    \item $M$ is locally Euclidean (every point in $M$ has a neighborhood homeomorphic to $\R^n$ for some $n$).
  \end{itemize}
  In particular, the third condition says that for every $p\in M$, there is $U\in \mathcal{O}_p$ and a homeomorphism $\varphi\colon U\rightarrow \R^n$. The value of $n$ is called the \textit{dimension} of the manifold $M$.
\end{definition}
\begin{definition}
  Let $M$ be an $n$-manifold. A \textit{chart} on $M$ is a pair $\left( U,\varphi \right)$ such that $U\subseteq M$ is open, $\varphi\colon U\rightarrow \R^n$ is a homeomorphism.\newline

  A family of charts $\mathcal{A} = \set{\left( U_i,\varphi_i \right)}_{i\in I}$ is known as an \textit{atlas} if
  \begin{align*}
    M &= \bigcup_{i\in I} U_i.
  \end{align*}
\end{definition}
To understand the smooth structure of a manifold, we consider a point $p\in M$ and two charts $\left( U,\varphi_U \right)$ and $\left( V,\varphi_V \right)$ such that $p\in U$ and $p\in V$. The functions $\varphi_U\colon U\rightarrow \R^n$ and $\varphi_V\colon V\rightarrow \R^n$ are homeomorphism, meaning that $\varphi_V\circ\varphi_U^{-1}\colon \varphi_{U}\left( U\cap V \right)^n\rightarrow \R^n$ defined on the (nonempty) $U\cap V$ is also a homeomorphism.\newline

In particular, we develop the smooth structure by making sure all such pairs $\varphi_{V}\circ\varphi_U^{-1}$ are \textit{diffeomorphisms}. To do this, we need to first develop the derivative in $\R^n$.
\begin{definition}
  Let $f\colon \R^n\rightarrow \R^m$ be a function. We say $f$ is \textit{differentiable} at $p\in \R^n$ if there is a linear map $L\in \Hom\left( \R^n,\R^m \right)$ such that
  \begin{align*}
    \frac{\norm{f\left( p+h \right)-f\left( p \right)-Lh}}{\norm{h}} &\rightarrow 0
  \end{align*}
  as $h\rightarrow 0$.\newline

  The \textit{derivative} of $f$ is the association $f\mapsto L$ for each $p\in \R^n$. We write $D_pf$ to denote this map. Note that we consider elements of $\Mat_{n}\left( \R \right)$ as points in $\R^{n^2}$ with the standard topology on $\R^{n^2}$.\newline

  A function $f$ is called a \textit{diffeomorphism} if it is continuously differentiable and has a continuously differentiable inverse.
\end{definition}
\begin{definition}
  If $\left( U,\varphi_U \right)$ and $\left( V,\varphi_V \right)$ are charts such that $U\cap V \neq \emptyset$, the function $\varphi_{V}\circ \varphi_{U}^{-1}\colon \R^n\rightarrow \R^n$ is known as the \textit{transition map} between $\varphi_U$ and $\varphi_V$.\newline

  A smooth structure for $M$ is an atlas $\set{\left( U_i,\varphi_i \right)}_{i\in I}$ such that for all $i,j$, the transition maps $\varphi_j\circ \varphi_i^{-1}\colon \R^n\rightarrow \R^n$ are diffeomorphisms where defined.\newline

  If $\set{\left( U_i,\varphi_i \right)}_{i\in I}$ is a \textit{maximal} smooth atlas --- i.e., any other smooth atlas that contains $\set{\left( U_i,\varphi_i \right)}_{i\in I}$ is equal to $\set{\left( U_i,\varphi_i \right)}_{i\in I}$ --- then we call $\set{\left( U_i,\varphi_i \right)}_{i\in I}$ a \textit{smooth structure} for $M$.
\end{definition}
\begin{note}
  From now on, we use ``manifold'' to refer to smooth manifolds, and will say \textit{topological} manifolds if the manifold does not necessarily admit a smooth structure.
\end{note}
\begin{definition}
  A map $f\colon M\rightarrow N$ between manifolds is called \textit{smooth} if for any chart $\left( U,\varphi_U \right)$ in $M$ and corresponding chart $\left( V,\varphi_V \right)$ in $N$, the map $\varphi_V\circ f \circ \varphi_U^{-1}\colon \R^n\rightarrow \R^k$ is continuously differentiable.\newline

  The function $f$ is a \textit{diffeomorphism} if $f$ is a smooth bijection with smooth inverse, and we say the manifolds $M$ and $N$ are diffeomorphic if they admit a diffeomorphism.
\end{definition}
\subsection{Examples}%
There are a couple special examples of (smooth) manifolds.
\begin{enumerate}[(i)]
  \item Open subsets of $\R^n$ are always manifolds.
  \item The general linear group, $\GL_n\left( \R \right)$ of $n\times n$ invertible matrices, viewed as a subset of $\Mat_{n}\left( \R \right)\cong \R^{n^2}$, is a manifold. Furthermore, it is an open subset of $\R^{n^2}$, as considering the map $\det\colon \Mat_{n}\left( \R \right)\rightarrow \R$ given by $A\mapsto \det\left( A \right)$, we see that $\GL_{n}\left( \R \right) = \det^{-1}\left( \R\setminus \set{0} \right)$.
  \item The special linear group, $\SL_n\left( \R \right)\subseteq \GL_n\left( \R \right)$, consisting of $n\times n$ matrices with determinant $1$, is also a smooth manifold. Furthermore, this manifold is a closed subset of $\R^{n^2}$, as it is equal to $\det^{-1}\left( \set{1} \right)$.
  \item The $n$-sphere, $S^{n}$, given by
    \begin{align*}
      S^{n} = \set{\left( x_0,\dots,x_n \right) | \sum_{i=0}^{n}x_i^2 = 1}
    \end{align*}
    is a manifold in $\R^{n}$. That it is a smooth manifold is quite a bit less obvious.\newline

    Now, in low dimensions, we know that $S^{2} \cong \hat{\C} = \C\cup\set{\infty}$, and that the continuously differentiable transformation $z\mapsto \frac{1}{z}$ takes the neighborhood basis of $\infty$ to deleted neighborhoods of $0$, and takes the neighborhood basis of $0$ to the neighborhood basis of $\infty$. This is our desired smooth structure.\newline

    In the case of the general $S^{n}$, we use two stereographic projections to construct our smooth structure. The first stereographic projection is via the north pole, $N_{p}$, and maps points on $S^{n}\setminus \set{N_{p}}$ bijectively to $\R^{n}$; this is a chart that is defined everywhere on $S^{n}$ except $N_{p}$. Similarly, we may use a stereographic projection originating from the south pole, $S_{p}$, so as to create another chart defined everywhere except $S_{p}$. These two stereographic projections are our desired smooth structure, as these two charts are all that is necessary to cover $S^{n}$.
  \item The real projective plane, consisting of lines through the origin in $\R^{n+1}$, can be expressed as
    \begin{align*}
      \R\mathbb{P}^{n} &= \left( \R^{n+1}\setminus \set{0} \right)/\R^{\times}.
    \end{align*}
    We will show that this is a manifold by constructing a family of charts mapping to $\R^{n}$.\newline

    Consider a point $\left( r_0,\dots,r_n \right)\in \R^{n+1}\setminus \set{0}$. If $r_0\neq 0$, then by dividing, we may associate this point's equivalence class in $\R\mathbb{P}^{n}$ to
    \begin{align*}
      \left( 1,r_1/r_0,\dots,r_n/r_0 \right) &\in \set{1}\times \R^{n},
    \end{align*}
    so we may associate all points of the form $\left[ \left( r_0,\dots,r_n \right) \right]$ with $r_0\neq 0$ with a chart $\left( U_{0},\varphi_{0} \right)$ that maps $\R\mathbb{P}^{n}$ to $\R^{n}$.\newline

    Similarly, we may define $U_{k}$ via
    \begin{align*}
      U_{k} &= \set{\left[ \left( r_0,\dots,r_n \right) \right] | r_k\neq 0}
    \end{align*}
    with corresponding chart
    \begin{align*}
      \varphi_{k}\colon U_{k}&\rightarrow \R^{n}\\
      \left[ \left( r_0,\dots,r_n \right) \right] &\mapsto \frac{1}{r_{k}} \left( r_0,\dots,\widehat{r_k},\dots,r_n \right),
    \end{align*}
    where $\widehat{r_k}$ denotes the exclusion of the $r_k$ coordinate. Varying $k$ from $0$ to $n$, we see that
    \begin{align*}
      \R\mathbb{P}^{n} &= \bigcup_{k=0}^{n}U_k,
    \end{align*}
    the chart functions $\varphi_{k}\colon U_k\rightarrow \R^{n}$ are homeomorphisms (as they are just division and projections). Furthermore, the transition maps $\varphi_{j}\circ\varphi_{i}^{-1}$ are coordinate-wise rational functions defined by
    \begin{align*}
      \left( u_1,\dots,u_n \right) &\mapsto \left( \frac{u_1}{u_i},\dots,\frac{1}{u_i},\dots,\frac{u_n}{u_i} \right),
    \end{align*}
    where the $\frac{1}{u_i}$ is at position $j$.
  \item We now turn to a very important example from algebraic geometry: the Grassmannian, $\operatorname{Gr}\left( k,n \right)$, consisting of all the $k$-dimensional subspaces of $\R^{n}$.\newline

    This is a $k\left( n-k \right)$-dimensional manifold; we need to understand what the smooth structure is. To do this, we let $ \iprod{\cdot}{\cdot} $ be an inner product on $\R^{n}$, and for any $V\in \operatorname{Gr}\left( k,n \right)$, we consider maps in $\Hom\left( V,V^{\perp} \right)$, where $V^{\perp}$ denotes the orthogonal complement of $V$.\newline

    Now, we see that if $W\in \operatorname{Gr}\left( k,n \right)$ is any other $k$-dimensional subspace, the orthogonal $P_{V}\colon \R^{n}\rightarrow V$ restricted to $W$ is a linear isomorphism if and only if $W\nsubseteq V^{\perp}$, or that $W\cap V^{\perp} = \set{0}$.\newline

    We see that if $W$ is such that $P_{V}|_{W}\colon W\rightarrow V$ is a linear isomorphism, the inverse $\left( P_{V}|_{W} \right)^{-1}\colon V\rightarrow W$ is well-defined; so, we may make a correspondence between $\Hom\left( V,V^{\perp} \right)$ and $\operatorname{Gr}\left( k,n \right)$ by noting that any such $T\in \Hom\left( V,V^{\perp} \right)$ has a corresponding graph $\left( v,T(v) \right)$, so we take $v\mapsto P_{V}|_{W}^{-1}\left( v \right)$, then project onto $V^{\perp}$ by taking $T\left( P_{V^{\perp}}\left( P_{V}|_{W}^{-1}(v) \right) \right) = T(v)$. We depict it as a diagram below.
    \begin{center}
      % https://tikzcd.yichuanshen.de/#N4Igdg9gJgpgziAXAbVABwnAlgFyxMJZABgBoBGAXVJADcBDAGwFcYkQA1EAX1PU1z5CKchWp0mrdgHUefEBmx4CRAEyli4hizaJOc-kqFqxNbVL0AdSwFt6OABYAjJ8ABK3AHrAw3AwoFlYWR1VS1JXU5vazQYACc0P25xGCgAc3giUAAzOIgbJDIQHAgkUQkddgAFAH1gDm4AHzrpL2AAWnIk+Vz8spoSpABmAfosRnYHCAgAaxAzCPZrfBx6f16CxBHi0sR1EEZ6JxhGKsDjPTisNIcceYqLEFr67py8ze3BxAAWGkPj07nFR6RgwbJ3BaVPTPDjRSyxBLcV4gDaFAa7X4HI4nM5GYEgK43CEPSIAFXuxzAUCQ7SGxF4bz6iCKX32lOpWyK5ki1ggCPsEDiYHoNhgwCwUCSlG4QA
      \begin{tikzcd}
                                                                                                               &                            & V                                                             \\
      V \arrow[r, "P_{V}|_{W}^{-1}"] \arrow[rrd, "T"', bend right] \arrow[rru, "\operatorname{id}", bend left] & W \arrow[r, "\iota", hook] & \mathbb{R}^{n} \arrow[u, "P_{V}"'] \arrow[d, "P_{V^{\perp}}"] \\
                                                                                                               &                            & V^{\perp}                                                    
      \end{tikzcd}
    \end{center}
    Defining $U_{V} = \set{W\in \operatorname{Gr}\left( k,n \right) | W\cap V^{\perp} = \set{0}}$, we may define the chart from $U_{V}$ onto $\Hom\left( V,V^{\perp} \right)$ by $\varphi_{V} = P_{V^{\perp}}\circ P_{V}|_{W}^{-1}$. The family $\set{\left( U_V,\varphi_V \right) | V\in \operatorname{Gr}\left( k,n \right)}$ is our smooth atlas.
\end{enumerate}

\subsection{Inverse and Implicit Function Theorems}%
In order to replace manifolds with linear maps, we need to understand smooth maps on $\R^n$. The most important theorems in this regard are the inverse function theorem and the implicit function theorem.
\begin{theorem}[Inverse Function Theorem]
  Let $f\colon U\subseteq \R^n \rightarrow \R^n$ be a continuously differentiable function. If $D_{p}f$ is invertible as a linear map, then $f$ has a local, continuously differentiable inverse $f^{-1}\colon V\rightarrow W$, where $p\in W\subseteq U$ and $f(p)\in V\subseteq \R^n$.
\end{theorem}
The proof uses the contraction mapping theorem. Recall that if $X$ is a complete metric space, and $f\colon X\rightarrow X$ is a strict uniform contraction --- that is, there exists $0\leq \lambda < 1$ such that $d\left( f(x),f(y) \right) \leq \lambda d\left( x,y \right)$ for all $x,y\in X$ --- then $f$ has a unique fixed point.\newline

We begin with a technical lemma.
\begin{lemma}
  If $U\left( 0,r \right)\subseteq V$ for some $r > 0$ where $V$ is a normed vector space, $g\colon V\rightarrow V$ is a uniform contraction, and $f = \id + g$, then the following hold:
  \begin{itemize}
    \item $\left( 1-\lambda \right) \norm{x-y}\leq \norm{f(x)-f(y)}$ (in particular, $f$ is injective);
    \item if $g(0) = 0$, then
      \begin{align*}
        U\left( 0,\left( 1-\lambda \right)r \right) \subseteq f\left( U\left( 0,r \right) \right) \subseteq U\left( 0,\left( 1+\lambda \right)r \right).
      \end{align*}
  \end{itemize}
\end{lemma}
\begin{proof}[Proof of Lemma]
  To see the first item, we notice that by the triangle inequality,
  \begin{align*}
    \norm{x-y} - \norm{f(x)-f(y)} &\leq \norm{x-y} - \norm{x-y} + \norm{g(x)-g(y)}\\
                                  &\leq \lambda\norm{x-y},
  \end{align*}
  so $\left( 1-\lambda \right)\norm{x-y}\leq \norm{f(x)-f(y)}$, and $f$ is injective. Furthermore, we see that if $g(0) = 0$, then
  \begin{align*}
    f\left( U\left( 0,r \right) \right) &= U\left( 0,r \right) + g\left( U\left( 0,r \right) \right)\\
                                        &\subseteq U\left( 0,r \right) + \lambda U\left( 0,r \right)\\
                                        &= U\left( 0,\left( 1+\lambda \right)r \right).
  \end{align*}
  Finally, if $y\in U\left( 0,\left( 1-\lambda \right)r \right)$, then we want to find $x$ such that $y = f(x) = x + g(x)$; equivalently, we see that we want $x$ such that $x = y-g(x)$. Since the function $F(x) = y-g(x)$ is a translation of a uniform contraction, $F(x)$ is a contraction, so there is a fixed point, meaning $y\in f\left( U\left( 0,r \right) \right)$.
\end{proof}
\begin{note}
  We will use $\left\vert \cdot \right\vert$ to denote the norm on $\R^n$.
\end{note}
\begin{proof}[Proof of the Inverse Function Theorem]
  By using a series of affine maps --- first by translating $p$ to $0$, then translating $f(p)$ to $0$, then inverting $D_0f$ as per our assumption, we may safely assume that $p = f(p) =0$ and $D_0f = \operatorname{Id}$.\newline

  Set $g = f - \operatorname{Id}$. We will show that $g$ is a contraction in a sufficiently small ball. Fixing $x,y\in \R^n$, consider the map $\R\rightarrow \R^n$ given by $t \mapsto g\left( x + t\left( y-x \right) \right)$. Notice that by the Fundamental Theorem of Calculus,
  \begin{align*}
    \left\vert g(y)-g(x) \right\vert &\leq \left\vert y-x \right\vert \sup_{0\leq t \leq 1} \left\vert g'\left( x + t\left( y-x \right) \right) \right\vert.
  \end{align*}
  Furthermore, since $g'(0) = \mathbf{0}$ by the fact that $D_0f = \operatorname{Id}$ and $\left( \operatorname{Id} \right)' = \operatorname{Id}$, and since $f$ is continuously differentiable, there is $r > 0$ such that
  \begin{align*}
    \left\vert g(y)-g(x) \right\vert &\leq \frac{1}{2}\left\vert y-x \right\vert
  \end{align*}
  for all $x,y\in U\left( 0,r \right)$. Thus, $g$ is a strict contraction on $U\left( 0,r \right)$. By the previous lemma, we see that
  \begin{align*}
    U\left( 0,r/2 \right) &\subseteq f\left( U\left( 0,r \right) \right);
  \end{align*}
  by setting $U = U\left( 0,r \right) \cap f^{-1}\left( U\left( 0,r \right) \right)$, we see that the map $f|_{U}\colon U\rightarrow V \coloneq U\left( 0,r/2 \right)$ is a bijection. The inverse function $f^{-1}\colon V\rightarrow U$ thus exists.\newline

  Now, we let $h = f^{-1}$, $x\in U$, $y\in V$ such that $h(x) = y$, and $A = D_xf$. We will show that $A^{-1} = D_yh$, which is enough to show that $h$ is continuously differentiable, as we assume the map $x \mapsto D_xf$ is continuous, and inversion is continuous in $\GL_n\left(\R\right)$.\newline

  For sufficiently small vectors $s$ and $k$, since $f$ and $h$ are bijections, we have
  \begin{align*}
    h\left( y+k \right) = x+s,
  \end{align*}
  so
  \begin{align*}
    f\left( x+s \right)  &= y+k.
  \end{align*}
  Furthermore, by unraveling the definitions of $f = g + \operatorname{Id}$, $s$, and $k$, and the fact that $g$ is a uniform contraction on $U$, we get
  \begin{align*}
    \left\vert s-k \right\vert &= \left\vert \left( f(x+s) - f(x) \right) - s \right\vert\\
                               &= \left\vert \left( x+s + g(x+s) \right) - \left( x + g(x) \right) -s\right\vert\\
                               &= \left\vert g(x+s) - g(x) \right\vert\\
                               &\leq \frac{\left\vert s \right\vert}{2}.
  \end{align*}
  In particular, since
  \begin{align*}
    \left\vert s \right\vert &\leq \left\vert s-k \right\vert + \left\vert k \right\vert\\
                             &\leq \left\vert k \right\vert + \frac{\left\vert s \right\vert}{2},
  \end{align*}
  we see that $\left\vert s \right\vert/2 \leq \left\vert k \right\vert$. We calculate
  \begin{align*}
    \left\vert h\left( y+k \right)-h\left( y \right) - A^{-1}k \right\vert &= \left\vert x+s-x-A^{-1}\left( f\left( x+s \right)-f(x) \right) \right\vert\\
                                                                           &= \left\vert s - A^{-1}\left( f\left( x+s \right)-f\left( x \right) \right) \right\vert\\
                                                                           &\leq \norm{A^{-1}}_{\op} \left\vert As - f\left( x+s \right) - f\left( x \right) \right\vert.
  \end{align*}
  Thus, since $\left\vert s \right\vert/2 \leq \left\vert k \right\vert$,
  \begin{align*}
    \frac{\left\vert h\left( y+k \right)-h\left( y \right) - A^{-1}k \right\vert}{\left\vert k \right\vert} &\leq \frac{2\norm{A^{-1}}_{\op}\left\vert As-f\left( x+s \right)-f\left( x \right) \right\vert}{\left\vert s \right\vert}\\
                                                                                                            &\rightarrow 0,
  \end{align*}
  so $D_yh = A^{-1}$.
\end{proof}
One of the primary uses of the inverse function theorem is to prove the implicit function theorem. 
\begin{theorem}[Implicit Function Theorem]
  Let $f\colon \R^{n}\times \R^{m}\rightarrow \R^{m}$ be continuously differentiable, and let $a\in \R^{n}$, $b\in \R^{m}$. Assume
  \begin{itemize}
    \item $f(a,b) = 0$;
    \item the map $y\mapsto f\left( a,y \right)$ defined on $\R^{m}\rightarrow \R^{m}$ is invertible in a neighborhood of $b$ --- i.e., $D_b\left( y\mapsto f\left( a,y \right) \right)$ as a linear map has rank $m$.
  \end{itemize}
  Then, there exists a continuously differentiable function $g\colon U\rightarrow V$, where $U\in \mathcal{O}_{a}$ and $V\in \mathcal{O}_b$ such that $f\left( x,g(x) \right) = 0$ on $U$.
\end{theorem}
Essentially, the theorem says that we can solve $f\left( x,y \right) = 0$ on a neighborhood of $\left( a,b \right)$ by a function only depending on $x$. This means that the projection $P_{\R^{n}}\left( \Gamma\left( f \right) \right)$ onto $\R^{n}$ of the graph of $f$ on $\R^{n}\times \R^{m}$ gives a local chart about $\left( a,b \right)$.
\begin{proof}[Proof of the Implicit Function Theorem]
  Define a function $F\colon \R^{n+m}\rightarrow \R^{n+m}$ by
  \begin{align*}
    F(x,y) &= \left( x,f\left( x,y \right) \right).
  \end{align*}
  Since $f$ is continuously differentiable, this function $F$ is also continuously differentiable, so we may define $U\in \mathcal{O}_a$, $V\in \mathcal{O}_b$, and $W\in \mathcal{O}_{F\left( a,b \right)}$ such that
  \begin{align*}
    F\colon U\times V \rightarrow W
  \end{align*}
  is continuously differentiable with continuously differentiable inverse (owing to the Inverse Function Theorem), so that $G = F^{-1} = \left( G_1,G_2 \right)$ is defined on $W$. We see that
  \begin{align*}
    \left( x,y \right) &= F\left( G_1\left( x,y \right),f\left( G_1\left( x,y \right),G_2\left( x,y \right) \right) \right),
  \end{align*}
  meaning that $G_1\left( x,y \right) = x$, and $y = f\left( x,G_2\left( x,y \right) \right)$. Since at $b$, $f\left( a,b \right) = 0$, we have that $g(x) = G_2\left( x,0 \right)$ is the desired function.
\end{proof}
\subsection{Constructing $C^{\infty}$ Maps on Manifolds}%
\begin{definition}
  A function $f\colon U\rightarrow \R$, where $U\subseteq \R^{n}$ is open, is called $C^{\infty}$ if the partial derivatives of all orders,
  \begin{align*}
    \pd{^{\left\vert \alpha \right\vert}f}{x_1^{\alpha_1}\cdots \partial x_n^{\alpha n}}
  \end{align*}
  are continuous. Here, $\alpha = \left( \alpha_1,\dots,\alpha_n \right)$ is a \textit{multi-index}, where the $\alpha_i$ are positive integers for each $i$, and $\left\vert \alpha \right\vert $ is defined by $ \left\vert \alpha \right\vert = \sum_{i=1}^{n}\alpha_i$.
\end{definition}
We are concerned now with constructing $C^{\infty}$ functions on $C^{\infty}$-manifolds.\footnote{A $C^{\infty}$ manifold is one where all the transition functions $\varphi_{j}\circ\varphi_{i}^{-1}\colon \varphi_{i}\left( U_i\cap U_j \right)\rightarrow \varphi_{j}\left( U_i\cap U_j \right)$ are $C^{\infty}$ functions.} In order to do this, we introduce the bump functions.
\begin{definition}
  The \textit{bump function} that is equal to $1$ on $B\left( 0,1 \right)$ and is zero outside $U\left( 0,2 \right)$ is given by
  \begin{align*}
    h(x) &= \begin{cases}
      e^{-1/x} & x > 0\\
      0 & x\leq 0
    \end{cases}\\
      b(x) &= \frac{h\left( 4-\left\vert x \right\vert^2 \right)}{ h\left( 4-\left\vert x \right\vert^2 \right) + h\left( \left\vert x \right\vert^2 - 1 \right) }.\label{eq:bump_function}\tag{$\ast$}
  \end{align*}
\end{definition}
\begin{lemma}
  Let $M$ be a $C^{\infty}$ manifold. Let $U\in \mathcal{O}_{p}$, and let $f\colon U\rightarrow \R$ be an arbitrary $C^{\infty}$ function defined on $U$.\newline

  Then, there exists $V\in \mathcal{O}_{p}$ with $ \overline{V}\subseteq U $, and a $C^{\infty}$ function $ \widetilde{f} $ defined on $M$ such that
  \begin{align*}
    \widetilde{f}(q) &= \begin{cases}
      f(q) & q\in V\\
      0 & q\notin U.
    \end{cases}
  \end{align*}
\end{lemma}
\begin{proof}
  Let $\left( W,\varphi \right)$ be a chart centered at $p$ with $\varphi(p) = 0$ and $U\left( 0,3 \right)\subseteq \varphi(W)$. Let $ \overline{b} = b\circ\varphi $, where $b$ is the bump function defined in \eqref{eq:bump_function}. Then, $ \overline{b} $ is a $C^{\infty}$ function on $W$, and is $0$ outside $ \varphi^{-1}\left( U\left( 0,2 \right) \right) \subseteq W $.\newline

  We define $ \overline{b} $ to be equal to zero on $W^{c}$. Thus, if we define $V = \varphi^{-1}\left( U\left( 0,1 \right) \right)$, then $V\in \mathcal{O}_p$, $ \overline{V}\subseteq U $, and $ \overline{b} $ is equal to $1$ on $V$. Letting
  \begin{align*}
    \widetilde{f}(q) &= \begin{cases}
      \overline{b}(q) f(q) & q\in W\\
      0 & q\notin W,
    \end{cases}
  \end{align*}
  we see that $ \widetilde{f} $ satisfies the required property.
\end{proof}
Given an atlas $\set{\left( U_i,\varphi_i \right)}$, we want to be able to ``glue'' functions together by using these charts. A fundamental construction for this purpose is known as a partition of unity.
\begin{definition}
  Let $X$ be a topological space.
  \begin{itemize}
    \item An open cover $\set{U_{i}}_{i\in I}$ is called \textit{locally finite} if, for every $x\in X$, there is some $V\in \mathcal{O}_x$ such that $V\cap U_{i} = \emptyset$ for all but finitely many $i$.
    \item Another open cover $\set{V_{j}}_{j\in J}$ is called a refinement of another open cover $\set{U_i}_{i\in I}$ if for all $j\in J$, there exists some $i\in I$ such that $V_j\subseteq U_i$.
    \item We say $X$ is \textit{paracompact} if, for any open cover of $X$, there is a locally finite refinement.
  \end{itemize}
\end{definition}
\begin{proposition}
  Let $M$ be a topological manifold. Then, for any open cover $\set{U_{i}}_{i\in I}$ of $M$, there is a countable, locally finite refinement $\set{V_k}_{k=1}^{\infty}$ with the $ \overline{V_k} $ compact. In particular, $M$ is paracompact.\newline

  Additionally, we may select the coordinate maps $\psi_k\colon V_k\rightarrow \R^{n}$ such that $\psi_k\left( V_k \right) = U\left( 0,3 \right)$, and $\set{\psi_k^{-1}\left( U\left( 0,1 \right) \right)}_{k=1}^{\infty}$ is an open cover of $M$.
\end{proposition}
\begin{solution}
  Since $M$ is a locally Euclidean and second countable, there is a countable basis of pre-compact open sets $\set{O_\ell}_{\ell=1}^{\infty}$. In particular, we may select an exhaustion of $M$ by pre-compact sets by defining
  \begin{align*}
    E_1 &= O_1\\
    E_{k} &= O_1\cup O_2\cup \cdots \cup O_{\ell_k},
  \end{align*}
  where $\ell_{k}$ is some sufficiently large index as follows. Since $ \overline{E_k} $ is compact, there is a sufficiently large $\ell$ such that $ \overline{E_k}\subseteq O_1\cup\cdots\cup O_{\ell}$. Defining $\ell_{k+1}$ to be the smallest index greater than $\ell_{k}$ that satisfy this property, we define
  \begin{align*}
    E_{k+1} &= O_1 \cup \cdots \cup O_{\ell_{k+1}}.
  \end{align*}
  For arbitrary $k$, each $ \overline{E_k} $ is compact, and $ \overline{E_k}\subsetneq E_{k+1} $, and $\bigcup_{k=1}^{\infty}E_k = M$. Note that if $M$ is compact, this process terminates in a finite number of steps.\newline

  Now, let $\set{U_{i}}_{i\in I}$ be an arbitrary open cover of $M$, and fix $k \geq 1$. For each $p\in \overline{E_k}\setminus E_{k-1}$, select $i_{p}$ such that $p\in U_{i_p}$, and select a chart $\left( V_p,\psi_p \right)$ about $p$ that satisfies $\psi_p(p) = 0$, $\psi_p\left( V_p \right) = U\left( 0,3 \right)$, and $V_{p} \subseteq U_{i_p}\cap \left( E_{k+1}\setminus \overline{E_{k-2}} \right)$, where we set $E_{-1} = E_{0} = \emptyset$. Finally, set $W_{p} = \psi_{p}^{-1}\left( U\left( 0,1 \right) \right)$.\newline

  Since $ \overline{E_k} \setminus E_{k-1} $ is compact, we may select a finite number of such $p$ such that the open sets $W_{p}$ cover $ \overline{E_k}\setminus E_{k-1} $. Applying this process to all $k$, and lining up the charts $\left( V_p,\psi_p \right)$ corresponding to the finite number of points $p$ chosen at each stage, we have the locally finite refinement of $\set{U_{i}}_{i\in I}$ with each $ \overline{V_k} $ compact, $ \psi_k\left( V_k \right) = U\left( 0,3 \right) $, and $\set{\psi_k^{-1}\left( U\left( 0,1 \right) \right)}$ an open cover of $M$.
\end{solution}
\begin{definition}
  Let $M$ be a $C^{\infty}$ manifold. A family $\set{f_k}_{k=1}^{\infty}$ of at most countably many $C^{\infty}$ functions on $M$ is called a \textit{partition of unity} on $M$ if it satisfies:
  \begin{itemize}
    \item for each $k$, $f_{k}\left( p \right) \geq 0$ for all $p\in M$, and the family $\set{\supp\left( f_k \right)}_{k=1}^{\infty}$ is locally finite;
    \item at all points $p$ on $M$, $\sum_{k=1}^{\infty}f_k(p) = 1$.
  \end{itemize}
  If $\set{\supp\left( f_k \right)}_{k=1}^{\infty}$ is a refinement of an open cover $\set{U_i}_{i\in I}$, then we say the partition of unity is \textit{subordinate} to the open cover.
\end{definition}
\begin{theorem}
  Let $M$ be a $C^{\infty}$ manifold, and let $\set{U_{i}}_{i\in I}$ be an open cover of $M$. Then, there exists a partition of unity $\set{f_k}_{k=1}^{\infty}$ that is subordinate to $\set{U_{i}}_{i\in I}$.
\end{theorem}
\begin{proof}
  Let $\set{V_k}_{k=1}^{\infty}$ be a locally finite refinement of $\set{U_i}_{i\in I}$ such that the charts $\left( V_k,\psi_k \right)$ have $\psi_k\left( V_k \right) = U\left( 0,3 \right)$.\newline

  For each $k$, using the bump function \eqref{eq:bump_function}, define
  \begin{align*}
    \widetilde{b_k}\left( q \right) &= \begin{cases}
      b\circ \psi_k\left( q \right) & q\in V_k\\
      0 & q\notin V_k.
    \end{cases}
  \end{align*}
  Then, $ \widetilde{b_k} $ is a $C^{\infty}$ function defined on $M$, and since $ \supp\left( \widetilde{b_k} \right)\subseteq V_k $, we may set
  \begin{align*}
    f &= \sum_{k=1}^{\infty} \widetilde{b_k}.
  \end{align*}
  The function $f$ is a $C^{\infty}$ function defined on the whole of $M$. If we let $W_k = \psi_k^{-1}\left( U\left( 0,1 \right) \right)$, then since $\set{W_k}_{k\geq 1}$ is an open cover of $M$, for any $q\in M$, there exists $j$ such that $ \widetilde{b_j}\left( q \right) = 1 $. Thus, $f$ never equals $0$, so we if we set
  \begin{align*}
    f_k &= \frac{\widetilde{b_k}}{f},
  \end{align*}
  the family $\set{f_k}_{k\geq 1}$ is a partition of unity subordinate to $\set{U_i}_{i\in I}$.
\end{proof}
\section{The Tangent Space}%
Smooth manifolds are able to be embedded into $\R^{n}$,\footnote{This is actually a very deep theorem.} so we start by considering them as such.
\begin{definition}
  If $f\colon M\rightarrow N$ is a smooth map between an $n$-dimensional manifold $M$ and a $k$-dimensional manifold $N$, the \textit{derivative} of $f$ at $p$, defined for charts $\left( U,\varphi \right)$ and $\left( V,\psi \right)$, where $f(U)\subseteq V$, is defined by
  \begin{align*}
    Df &= D\left( \psi^{-1}\circ F\circ \varphi \right),
  \end{align*}
  where $F\colon \R^{n}\rightarrow \R^{m}$ is a continuously differentiable map that maps $\varphi(U)$ into $\psi(V)$. We then say that $D_pf\colon T_{p}\left( M \right)\rightarrow T_{f(p)}\left( M \right)$ is the derivative map between \textit{tangent spaces} of $M$ and $N$.
\end{definition}
One of the issues with this strategy, though, is that embeddings may carry different properties (though at high enough dimensions, any two embeddings are diffeomorphic to each other). For instance, embeddings $S^{1}\hookrightarrow \R^{3}$ form the field of knot theory, which is a very intricate field.\newline

As a result, we want to be able to define tangent spaces, derivatives, and the like without having to refer to coordinates. In order to do this, we need to discuss germs of functions.
\begin{definition}
  Let $g\colon M\rightarrow N$ map $p\mapsto g(p)$. We define an equivalence relation on the space of functions $f\colon M\rightarrow N$ with $f(p) = g(p)$ by taking $f_1\sim f_2$ whenever $f_1 = f_2$ on some open neighborhood $A\in \mathcal{O}_p$. The equivalence class $\left[ g \right]_p$ is known as the \textit{germ} of $g$ at $p$.\newline

  We denote by $\mathcal{C}_{p}$ the space of germs of $C^{\infty}$ functions $f\colon M\rightarrow \R$ at $p$.
\end{definition}
\begin{remark}
  Often, books will use $\mathcal{O}_p$ to refer to the space of germs at $p$. We will use $\mathcal{C}_p$ for this purpose though, as we have defined $\mathcal{O}_p$ to refer to the system of open neighborhoods at $p$.
\end{remark}
\begin{proposition}
  The space $\mathcal{C}_{p}$ with the operations
  \begin{itemize}
    \item $\left[ g \right] + \left[ h \right] = \left[ g + h \right]$;
    \item $\alpha \left[ g \right] = \left[ \alpha g \right]$;
    \item $\left[ g \right]\cdot \left[ h \right] = \left[ g\cdot h \right]$
  \end{itemize}
  forms an algebra over $\R$.
\end{proposition}
\begin{definition}
  Let $W_p$ be the space of germs of smooth maps $\gamma\colon \R\rightarrow M$ that send $0 \mapsto p$. The \textit{tangent space} $T_pM$ is defined by $W_p/\sim$, where we define the equivalence relation $\left[ g_1 \right]\sim \left[ g \right]_2$ by
  \begin{align*}
    \left( \varphi\circ g_1 \right)'(0) &= \left( \varphi\circ g_2 \right)(0)
  \end{align*}
  for all $\varphi\in \mathcal{C}_p$.
\end{definition}
\section{Notations}%
\begin{itemize}
  \item A general normed space $V$ will have its norm denoted by $\norm{\cdot}$. If $V = \R^{n}$, then we denote the norm by $ \left\vert \cdot \right\vert $.
  \item We denote topological spaces by $\left( X,\tau \right)$.
  \item $ U\left( x,r \right) = \set{y\in V | \norm{x-y} < r}$.
  \item $ B\left( x,r \right) = \set{y\in V | \norm{x-y} \leq r}$.
  \item $ \mathcal{N}_p $: neighborhood system centered at $p\in X$.
  \item $ \mathcal{O}_p $: system of \textit{open} neighborhoods centered at $p\in X$.
  \item When we say a number $n$ is positive, we mean that $n\geq 0$. Similarly, a sequence $\left( a_n \right)_n$ is decreasing (increasing) if $a_n\geq a_{n+1}$ ($a_n\leq a_{n+1}$).
\end{itemize}
\end{document}
