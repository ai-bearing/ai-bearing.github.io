\documentclass[10pt]{mypackage}

% sans serif font:
%\usepackage{cmbright}
%\usepackage{sfmath}
%\usepackage{bbold} %better blackboard bold

%\usepackage{homework}
\usepackage{notes}
\usepackage{newpxtext,eulerpx,eucal}
\renewcommand*{\mathbb}[1]{\varmathbb{#1}}

\fancyhf{}
\rhead{Avinash Iyer}
\lhead{Differential Topology: Notes and Review}

\setcounter{secnumdepth}{0}

\begin{document}
\RaggedRight
This is a set of notes I am taking for my Differential Topology class. The references occasionally used include
\begin{itemize}
  \item \textit{Geometry of Differential Forms} by Morita;
  \item \textit{Topology from the Differentiable Viewpoint} by Milnor;
  \item \textit{Differential Topology} by Hirsch
  \item \textit{Introduction to Smooth Manifolds} by Lee;
  \item \textit{A Short Course in Differential Topology} by Dundas.
\end{itemize}
\tableofcontents
\section{Basic Properties}%
\begin{definition}
  A topological space $M$ is called a \textit{manifold} if it satisfies the following:
  \begin{itemize}
    \item $M$ is Hausdorff (points can be separated by open sets);
    \item $M$ is second countable (the basis for the topology of $M$ is countable);
    \item $M$ is locally Euclidean (every point in $M$ has a neighborhood homeomorphic to $\R^n$ for some $n$).
  \end{itemize}
  In particular, the third condition says that for every $p\in M$, there is $U\in \mathcal{O}_p$ and a homeomorphism $\varphi\colon U\rightarrow \R^n$. The value of $n$ is called the \textit{dimension} of the manifold $M$.
\end{definition}
\begin{definition}
  Let $M$ be an $n$-manifold. A \textit{chart} on $M$ is a pair $\left( U,\varphi \right)$ such that $U\subseteq M$ is open, $\varphi\colon U\rightarrow \R^n$ is a homeomorphism.\newline

  A family of charts $\mathcal{A} = \set{\left( U_i,\varphi_i \right)}_{i\in I}$ is known as an \textit{atlas} if
  \begin{align*}
    M &= \bigcup_{i\in I} U_i.
  \end{align*}
\end{definition}
To understand the smooth structure of a manifold, we consider a point $p\in M$ and two charts $\left( U,\varphi_U \right)$ and $\left( V,\varphi_V \right)$ such that $p\in U$ and $p\in V$. The functions $\varphi_U\colon U\rightarrow \R^n$ and $\varphi_V\colon V\rightarrow \R^n$ are homeomorphism, meaning that $\varphi_V\circ\varphi_U^{-1}\colon \varphi_{U}\left( U\cap V \right)^n\rightarrow \R^n$ defined on the (nonempty) $U\cap V$ is also a homeomorphism.\newline

In particular, we develop the smooth structure by making sure all such pairs $\varphi_{V}\circ\varphi_U^{-1}$ are \textit{diffeomorphisms}. To do this, we need to first develop the derivative in $\R^n$.
\begin{definition}
  Let $f\colon \R^n\rightarrow \R^m$ be a function. We say $f$ is \textit{differentiable} at $p\in \R^n$ if there is a linear map $L\in \Hom\left( \R^n,\R^m \right)$ such that
  \begin{align*}
    \frac{\left\vert f\left( p+h \right)-f\left( p \right)-Lh \right\vert}{\left\vert h \right\vert} &\rightarrow 0
  \end{align*}
  as $h\rightarrow 0$.\newline

  The \textit{derivative} of $f$ is the association $f\mapsto L$ for each $p\in \R^n$. We write $D_pf$ to denote this map. Note that we consider elements of $\Mat_{n}\left( \R \right)$ as points in $\R^{n^2}$ with the standard topology on $\R^{n^2}$.\newline

  A function $f$ is called a \textit{diffeomorphism} if it is (sufficiently) continuously differentiable and has a (similarly sufficiently) continuously differentiable inverse.
\end{definition}
\begin{definition}
  If $\left( U,\varphi_U \right)$ and $\left( V,\varphi_V \right)$ are charts such that $U\cap V \neq \emptyset$, the function $\varphi_{V}\circ \varphi_{U}^{-1}\colon \R^n\rightarrow \R^n$ is known as the \textit{transition map} between $\varphi_U$ and $\varphi_V$.\newline

  A smooth structure for $M$ is an atlas $\set{\left( U_i,\varphi_i \right)}_{i\in I}$ such that for all $i,j$, the transition maps $\varphi_j\circ \varphi_i^{-1}\colon \R^n\rightarrow \R^n$ are diffeomorphisms where defined.\newline

  If $\set{\left( U_i,\varphi_i \right)}_{i\in I}$ is a \textit{maximal} smooth atlas --- i.e., any other smooth atlas that contains $\set{\left( U_i,\varphi_i \right)}_{i\in I}$ is equal to $\set{\left( U_i,\varphi_i \right)}_{i\in I}$ --- then we call $\set{\left( U_i,\varphi_i \right)}_{i\in I}$ a \textit{smooth structure} for $M$.
\end{definition}
\begin{note}
  From now on, we use ``manifold'' to refer to smooth manifolds, and will say \textit{topological} manifolds if the manifold does not necessarily admit a smooth structure.
\end{note}
\begin{definition}
  A map $f\colon M\rightarrow N$ between manifolds is called \textit{smooth} if for any chart $\left( U,\varphi_U \right)$ in $M$ and corresponding chart $\left( V,\varphi_V \right)$ in $N$, the map $\varphi_V\circ f \circ \varphi_U^{-1}\colon \R^n\rightarrow \R^k$ is (sufficiently) continuously differentiable.\newline

  The function $f$ is a \textit{diffeomorphism} if $f$ is a smooth bijection with smooth inverse, and we say the manifolds $M$ and $N$ are diffeomorphic if they admit a diffeomorphism.
\end{definition}
\begin{remark}
  If $f\colon M\rightarrow N$ is smooth, then any representation of $f$ is smooth. To see this, if $\left( U,\varphi_1 \right)$ and $\left( U,\varphi_2 \right)$ are charts in $M$, with corresponding charts $\left( V,\psi_1 \right)$ and $\left( V,\psi_2 \right)$, then
  \begin{align*}
    \psi_{1}\circ f \circ \varphi_1^{-1} &= \left( \psi_1\circ \psi_2^{-1} \right)\circ \left( \psi_2\circ f \circ \varphi_2^{-1} \right)\circ \left( \varphi_2\circ \varphi_1^{-1} \right),
  \end{align*}
  where the transition maps $\psi_1\circ \psi_2^{-2}$ and $\varphi_2\circ \varphi_1^{-1}$ are smooth.
\end{remark}
\subsection{More on Smooth Maps}%
Generally speaking, we will refer to charts on a dimension $n$ smooth manifold by $\left( U,\varphi \right) = \left( U;x_1,\dots,x_n \right)$, where $x_i\colon U\rightarrow \R$ are the coordinates of $U$. Additionally, if $\left( \R^{n};e_1,\dots,e_n \right)$ are the identity chart on $\R^{n}$, and the $e_i$ are standard coordinates on $\R^{n}$, then the coordinate maps satisfy
\begin{align*}
  x_i &= e_i\circ \varphi.
\end{align*}
\begin{definition}
  Let $\left( U;x_1,\dots,x_n \right)$ be a chart on a manifold $M$ of dimension $n$. If $f\colon M\rightarrow \R$ is a $C^{\infty}$ function, we define the \textit{partial derivative} of $f$ with respect to $x_i$ at $p$ to be
  \begin{align*}
    \pd{f}{x_i}\left( p \right) &= \pd{\left( f\circ \varphi^{-1} \right)}{\partial e_i} \left( \varphi(p) \right).
  \end{align*}
\end{definition}
  In particular,
  \begin{align*}
    \pd{f}{x_i}\circ \varphi^{-1} &= \pd{\left( f\circ \varphi^{-1} \right)}{e_i}
  \end{align*}
  as functions on $\phi(U)$.
\begin{proposition}
  The coordinate functions $x_1,\dots,x_n$ satisfy $\pd{x_i}{x_j} = \delta_{ij}$, where $\delta_{ij}$ is the Kronecker delta.
\end{proposition}
\begin{proof}
  For any $p\in U$, we calculate
  \begin{align*}
    \pd{x_i}{x_j}(p) &= \pd{\left( x_i\circ \varphi^{-1} \right)}{e_j}\left( \varphi\left( p \right) \right)\\
                     &= \pd{\left( \left( e_i\circ \varphi \right) \right)}{e_j}\left( \varphi\left( p \right) \right)\\
                     &= \pd{e_i}{e_j} \left( \varphi\left( p \right) \right)\\
                     &= \delta_{ij}.
  \end{align*}
\end{proof}
\subsection{Examples}%
There are a couple special examples of (smooth) manifolds.
\begin{enumerate}[(i)]
  \item Open subsets of $\R^n$ are always manifolds.
  \item The general linear group, $\GL_n\left( \R \right)$ of $n\times n$ invertible matrices, viewed as a subset of $\Mat_{n}\left( \R \right)\cong \R^{n^2}$, is a manifold. Furthermore, it is an open subset of $\R^{n^2}$, as considering the map $\det\colon \Mat_{n}\left( \R \right)\rightarrow \R$ given by $A\mapsto \det\left( A \right)$, we see that $\GL_{n}\left( \R \right) = \det^{-1}\left( \R\setminus \set{0} \right)$.
  \item The special linear group, $\SL_n\left( \R \right)\subseteq \GL_n\left( \R \right)$, consisting of $n\times n$ matrices with determinant $1$, is also a smooth manifold. Furthermore, this manifold is a closed subset of $\R^{n^2}$, as it is equal to $\det^{-1}\left( \set{1} \right)$.
  \item The $n$-sphere, $S^{n}$, given by
    \begin{align*}
      S^{n} = \set{\left( x_0,\dots,x_n \right) | \sum_{i=0}^{n}x_i^2 = 1}
    \end{align*}
    is a manifold in $\R^{n}$. That it is a smooth manifold is quite a bit less obvious.\newline

    Now, in low dimensions, we know that $S^{2} \cong \hat{\C} = \C\cup\set{\infty}$, and that the continuously differentiable transformation $z\mapsto \frac{1}{z}$ takes the neighborhood basis of $\infty$ to deleted neighborhoods of $0$, and takes the neighborhood basis of $0$ to the neighborhood basis of $\infty$. This is our desired smooth structure.\newline

    In the case of the general $S^{n}$, we use two stereographic projections to construct our smooth structure. The first stereographic projection is via the north pole, $N_{p}$, and maps points on $S^{n}\setminus \set{N_{p}}$ bijectively to $\R^{n}$; this is a chart that is defined everywhere on $S^{n}$ except $N_{p}$. Similarly, we may use a stereographic projection originating from the south pole, $S_{p}$, so as to create another chart defined everywhere except $S_{p}$. These two stereographic projections are our desired smooth structure, as these two charts are all that is necessary to cover $S^{n}$.
  \item The real projective plane, consisting of lines through the origin in $\R^{n+1}$, can be expressed as
    \begin{align*}
      \R\mathbb{P}^{n} &= \left( \R^{n+1}\setminus \set{0} \right)/\R^{\times}.
    \end{align*}
    We will show that this is a manifold by constructing a family of charts mapping to $\R^{n}$.\newline

    Consider a point $\left( r_0,\dots,r_n \right)\in \R^{n+1}\setminus \set{0}$. If $r_0\neq 0$, then by dividing, we may associate this point's equivalence class in $\R\mathbb{P}^{n}$ to
    \begin{align*}
      \left( 1,r_1/r_0,\dots,r_n/r_0 \right) &\in \set{1}\times \R^{n},
    \end{align*}
    so we may associate all points of the form $\left[ \left( r_0,\dots,r_n \right) \right]$ with $r_0\neq 0$ with a chart $\left( U_{0},\varphi_{0} \right)$ that maps $\R\mathbb{P}^{n}$ to $\R^{n}$.\newline

    Similarly, we may define $U_{k}$ via
    \begin{align*}
      U_{k} &= \set{\left[ \left( r_0,\dots,r_n \right) \right] | r_k\neq 0}
    \end{align*}
    with corresponding chart
    \begin{align*}
      \varphi_{k}\colon U_{k}&\rightarrow \R^{n}\\
      \left[ \left( r_0,\dots,r_n \right) \right] &\mapsto \frac{1}{r_{k}} \left( r_0,\dots,\widehat{r_k},\dots,r_n \right),
    \end{align*}
    where $\widehat{r_k}$ denotes the exclusion of the $r_k$ coordinate. Varying $k$ from $0$ to $n$, we see that
    \begin{align*}
      \R\mathbb{P}^{n} &= \bigcup_{k=0}^{n}U_k,
    \end{align*}
    the chart functions $\varphi_{k}\colon U_k\rightarrow \R^{n}$ are homeomorphisms (as they are just division and projections). Furthermore, the transition maps $\varphi_{j}\circ\varphi_{i}^{-1}$ are coordinate-wise rational functions defined by
    \begin{align*}
      \left( u_1,\dots,u_n \right) &\mapsto \left( \frac{u_1}{u_i},\dots,\frac{1}{u_i},\dots,\frac{u_n}{u_i} \right),
    \end{align*}
    where the $\frac{1}{u_i}$ is at position $j$.
  \item We now turn to a very important example from algebraic geometry: the Grassmannian, $\operatorname{Gr}\left( k,n \right)$, consisting of all the $k$-dimensional subspaces of $\R^{n}$.\newline

    This is a $k\left( n-k \right)$-dimensional manifold; we need to understand what the smooth structure is. To do this, we let $ \iprod{\cdot}{\cdot} $ be an inner product on $\R^{n}$, and for any $V\in \operatorname{Gr}\left( k,n \right)$, we consider maps in $\Hom\left( V,V^{\perp} \right)$, where $V^{\perp}$ denotes the orthogonal complement of $V$.\newline

    Now, we see that if $W\in \operatorname{Gr}\left( k,n \right)$ is any other $k$-dimensional subspace, the orthogonal $P_{V}\colon \R^{n}\rightarrow V$ restricted to $W$ is a linear isomorphism if and only if $W\nsubseteq V^{\perp}$, or that $W\cap V^{\perp} = \set{0}$.\newline

    We see that if $W$ is such that $P_{V}|_{W}\colon W\rightarrow V$ is a linear isomorphism, the inverse $\left( P_{V}|_{W} \right)^{-1}\colon V\rightarrow W$ is well-defined; so, we may make a correspondence between $\Hom\left( V,V^{\perp} \right)$ and $\operatorname{Gr}\left( k,n \right)$ by noting that any such $T\in \Hom\left( V,V^{\perp} \right)$ has a corresponding graph $\left( v,T(v) \right)$, so we take $v\mapsto P_{V}|_{W}^{-1}\left( v \right)$, then project onto $V^{\perp}$ by taking $T\left( P_{V^{\perp}}\left( P_{V}|_{W}^{-1}(v) \right) \right) = T(v)$. We depict it as a diagram below.
    \begin{center}
      % https://tikzcd.yichuanshen.de/#N4Igdg9gJgpgziAXAbVABwnAlgFyxMJZABgBoBGAXVJADcBDAGwFcYkQA1EAX1PU1z5CKchWp0mrdgHUefEBmx4CRAEyli4hizaJOc-kqFqxNbVL0AdSwFt6OABYAjJ8ABK3AHrAw3AwoFlYWR1VS1JXU5vazQYACc0P25xGCgAc3giUAAzOIgbJDIQHAgkUQkddgAFAH1gDm4AHzrpL2AAWnIk+Vz8spoSpABmAfosRnYHCAgAaxAzCPZrfBx6f16CxBHi0sR1EEZ6JxhGKsDjPTisNIcceYqLEFr67py8ze3BxAAWGkPj07nFR6RgwbJ3BaVPTPDjRSyxBLcV4gDaFAa7X4HI4nM5GYEgK43CEPSIAFXuxzAUCQ7SGxF4bz6iCKX32lOpWyK5ki1ggCPsEDiYHoNhgwCwUCSlG4QA
      \begin{tikzcd}
                                                                                                               &                            & V                                                             \\
      V \arrow[r, "P_{V}|_{W}^{-1}"] \arrow[rrd, "T"', bend right] \arrow[rru, "\operatorname{id}", bend left] & W \arrow[r, "\iota", hook] & \mathbb{R}^{n} \arrow[u, "P_{V}"'] \arrow[d, "P_{V^{\perp}}"] \\
                                                                                                               &                            & V^{\perp}                                                    
      \end{tikzcd}
    \end{center}
    Defining $U_{V} = \set{W\in \operatorname{Gr}\left( k,n \right) | W\cap V^{\perp} = \set{0}}$, we may define the chart from $U_{V}$ onto $\Hom\left( V,V^{\perp} \right)$ by $\varphi_{V} = P_{V^{\perp}}\circ P_{V}|_{W}^{-1}$. The family $\set{\left( U_V,\varphi_V \right) | V\in \operatorname{Gr}\left( k,n \right)}$ is our smooth atlas.
\end{enumerate}

\subsection{Inverse and Implicit Function Theorems}%
In order to replace manifolds with linear maps, we need to understand smooth maps on $\R^n$. The most important theorems in this regard are the inverse function theorem and the implicit function theorem.
\begin{theorem}[Inverse Function Theorem]
  Let $f\colon U\subseteq \R^n \rightarrow \R^n$ be a continuously differentiable function. If $D_{p}f$ is invertible as a linear map, then $f$ has a local, continuously differentiable inverse $f^{-1}\colon V\rightarrow W$, where $p\in W\subseteq U$ and $f(p)\in V\subseteq \R^n$.\newline

  Additionally, $D_p\left( f^{-1} \right) $ is given by the inverse of the derivative's corresponding linear map evaluated at $p$, $ D_p\left( f^{-1} \right) = \left( D_pf\left( f^{-1}(p) \right) \right)^{-1}$.
\end{theorem}
The proof uses the contraction mapping theorem. Recall that if $X$ is a complete metric space, and $f\colon X\rightarrow X$ is a strict uniform contraction --- that is, there exists $0\leq \lambda < 1$ such that $d\left( f(x),f(y) \right) \leq \lambda d\left( x,y \right)$ for all $x,y\in X$ --- then $f$ has a unique fixed point.\newline

We begin with a technical lemma.
\begin{lemma}
  If $U\left( 0,r \right)\subseteq V$ for some $r > 0$ where $V$ is a normed vector space, $g\colon V\rightarrow V$ is a uniform contraction, and $f = \id + g$, then the following hold:
  \begin{itemize}
    \item $\left( 1-\lambda \right) \norm{x-y}\leq \norm{f(x)-f(y)}$ (in particular, $f$ is injective);
    \item if $g(0) = 0$, then
      \begin{align*}
        U\left( 0,\left( 1-\lambda \right)r \right) \subseteq f\left( U\left( 0,r \right) \right) \subseteq U\left( 0,\left( 1+\lambda \right)r \right).
      \end{align*}
  \end{itemize}
\end{lemma}
\begin{proof}[Proof of Lemma]
  To see the first item, we notice that by the triangle inequality,
  \begin{align*}
    \norm{x-y} - \norm{f(x)-f(y)} &\leq \norm{x-y} - \norm{x-y} + \norm{g(x)-g(y)}\\
                                  &\leq \lambda\norm{x-y},
  \end{align*}
  so $\left( 1-\lambda \right)\norm{x-y}\leq \norm{f(x)-f(y)}$, and $f$ is injective. Furthermore, we see that if $g(0) = 0$, then
  \begin{align*}
    f\left( U\left( 0,r \right) \right) &= U\left( 0,r \right) + g\left( U\left( 0,r \right) \right)\\
                                        &\subseteq U\left( 0,r \right) + \lambda U\left( 0,r \right)\\
                                        &= U\left( 0,\left( 1+\lambda \right)r \right).
  \end{align*}
  Finally, if $y\in U\left( 0,\left( 1-\lambda \right)r \right)$, then we want to find $x$ such that $y = f(x) = x + g(x)$; equivalently, we see that we want $x$ such that $x = y-g(x)$. Since the function $F(x) = y-g(x)$ is a translation of a uniform contraction, $F(x)$ is a contraction, so there is a fixed point, meaning $y\in f\left( U\left( 0,r \right) \right)$.
\end{proof}
\begin{note}
  We will use $\left\vert \cdot \right\vert$ to denote the norm on $\R^n$.
\end{note}
\begin{proof}[Proof of the Inverse Function Theorem]
  By using a series of affine maps --- first by translating $p$ to $0$, then translating $f(p)$ to $0$, then inverting $D_0f$ as per our assumption, we may safely assume that $p = f(p) =0$ and $D_0f = \operatorname{Id}$.\newline

  Set $g = f - \operatorname{Id}$. We will show that $g$ is a contraction in a sufficiently small ball. Fixing $x,y\in \R^n$, consider the map $\R\rightarrow \R^n$ given by $t \mapsto g\left( x + t\left( y-x \right) \right)$. Notice that by the Fundamental Theorem of Calculus,
  \begin{align*}
    \left\vert g(y)-g(x) \right\vert &\leq \left\vert y-x \right\vert \sup_{0\leq t \leq 1} \left\vert g'\left( x + t\left( y-x \right) \right) \right\vert.
  \end{align*}
  Furthermore, since $g'(0) = \mathbf{0}$ by the fact that $D_0f = \operatorname{Id}$ and $\left( \operatorname{Id} \right)' = \operatorname{Id}$, and since $f$ is continuously differentiable, there is $r > 0$ such that
  \begin{align*}
    \left\vert g(y)-g(x) \right\vert &\leq \frac{1}{2}\left\vert y-x \right\vert
  \end{align*}
  for all $x,y\in U\left( 0,r \right)$. Thus, $g$ is a strict contraction on $U\left( 0,r \right)$. By the previous lemma, we see that
  \begin{align*}
    U\left( 0,r/2 \right) &\subseteq f\left( U\left( 0,r \right) \right);
  \end{align*}
  by setting $U = U\left( 0,r \right) \cap f^{-1}\left( U\left( 0,r \right) \right)$, we see that the map $f|_{U}\colon U\rightarrow V \coloneq U\left( 0,r/2 \right)$ is a bijection. The inverse function $f^{-1}\colon V\rightarrow U$ thus exists.\newline

  Now, we let $h = f^{-1}$, $x\in U$, $y\in V$ such that $h(x) = y$, and $A = D_xf$. We will show that $A^{-1} = D_yh$, which is enough to show that $h$ is continuously differentiable, as we assume the map $x \mapsto D_xf$ is continuous, and inversion is continuous in $\GL_n\left(\R\right)$.\newline

  For sufficiently small vectors $s$ and $k$, since $f$ and $h$ are bijections, we have
  \begin{align*}
    h\left( y+k \right) = x+s,
  \end{align*}
  so
  \begin{align*}
    f\left( x+s \right)  &= y+k.
  \end{align*}
  Furthermore, by unraveling the definitions of $f = g + \operatorname{Id}$, $s$, and $k$, and the fact that $g$ is a uniform contraction on $U$, we get
  \begin{align*}
    \left\vert s-k \right\vert &= \left\vert \left( f(x+s) - f(x) \right) - s \right\vert\\
                               &= \left\vert \left( x+s + g(x+s) \right) - \left( x + g(x) \right) -s\right\vert\\
                               &= \left\vert g(x+s) - g(x) \right\vert\\
                               &\leq \frac{\left\vert s \right\vert}{2}.
  \end{align*}
  In particular, since
  \begin{align*}
    \left\vert s \right\vert &\leq \left\vert s-k \right\vert + \left\vert k \right\vert\\
                             &\leq \left\vert k \right\vert + \frac{\left\vert s \right\vert}{2},
  \end{align*}
  we see that $\left\vert s \right\vert/2 \leq \left\vert k \right\vert$. We calculate
  \begin{align*}
    \left\vert h\left( y+k \right)-h\left( y \right) - A^{-1}k \right\vert &= \left\vert x+s-x-A^{-1}\left( f\left( x+s \right)-f(x) \right) \right\vert\\
                                                                           &= \left\vert s - A^{-1}\left( f\left( x+s \right)-f\left( x \right) \right) \right\vert\\
                                                                           &\leq \norm{A^{-1}}_{\op} \left\vert As - f\left( x+s \right) - f\left( x \right) \right\vert.
  \end{align*}
  Thus, since $\left\vert s \right\vert/2 \leq \left\vert k \right\vert$,
  \begin{align*}
    \frac{\left\vert h\left( y+k \right)-h\left( y \right) - A^{-1}k \right\vert}{\left\vert k \right\vert} &\leq \frac{2\norm{A^{-1}}_{\op}\left\vert As-f\left( x+s \right)-f\left( x \right) \right\vert}{\left\vert s \right\vert}\\
                                                                                                            &\rightarrow 0,
  \end{align*}
  so $D_yh = A^{-1}$.
\end{proof}
One of the primary uses of the inverse function theorem is to prove the implicit function theorem. 
\begin{theorem}[Implicit Function Theorem]
  Let $f\colon \R^{n}\times \R^{m}\rightarrow \R^{m}$ be continuously differentiable, and let $a\in \R^{n}$, $b\in \R^{m}$. Assume
  \begin{itemize}
    \item $f(a,b) = 0$;
    \item the map $y\mapsto f\left( a,y \right)$ defined on $\R^{m}\rightarrow \R^{m}$ is invertible in a neighborhood of $b$ --- i.e., $D_b\left( y\mapsto f\left( a,y \right) \right)$ as a linear map has rank $m$.
  \end{itemize}
  Then, there exists a continuously differentiable function $g\colon U\rightarrow V$, where $U\in \mathcal{O}_{a}$ and $V\in \mathcal{O}_b$ such that $f\left( x,g(x) \right) = 0$ on $U$.
\end{theorem}
Essentially, the theorem says that we can solve $f\left( x,y \right) = 0$ on a neighborhood of $\left( a,b \right)$ by a function only depending on $x$. This means that about $\left( a,b \right)$ in the graph $\Gamma\left( f \right)$, there is a coordinate representation as an $m$-manifold given by $g$.
\begin{proof}[Proof of the Implicit Function Theorem]
  Define a function $F\colon \R^{n+m}\rightarrow \R^{n+m}$ by
  \begin{align*}
    F(x,y) &= \left( x,f\left( x,y \right) \right).
  \end{align*}
  Since $f$ is continuously differentiable, this function $F$ is also continuously differentiable, so we may define $U\in \mathcal{O}_a$, $V\in \mathcal{O}_b$, and $W\in \mathcal{O}_{F\left( a,b \right)}$ such that
  \begin{align*}
    F\colon U\times V \rightarrow W
  \end{align*}
  is continuously differentiable with continuously differentiable inverse (owing to the Inverse Function Theorem), so that $G = F^{-1} = \left( G_1,G_2 \right)$ is defined on $W$. We see that
  \begin{align*}
    \left( x,y \right) &= F\left( G_1\left( x,y \right),f\left( G_1\left( x,y \right),G_2\left( x,y \right) \right) \right),
  \end{align*}
  meaning that $G_1\left( x,y \right) = x$, and $y = f\left( x,G_2\left( x,y \right) \right)$. Since at $b$, $f\left( a,b \right) = 0$, we have that $g(x) = G_2\left( x,0 \right)$ is the desired function.
\end{proof}
\subsection{Constructing $C^{\infty}$ Maps on Manifolds}%
\begin{definition}
  A function $f\colon U\rightarrow \R$, where $U\subseteq \R^{n}$ is open, is called $C^{\infty}$ if the partial derivatives of all orders,
  \begin{align*}
    \pd{^{\left\vert \alpha \right\vert}f}{x_1^{\alpha_1}\cdots \partial x_n^{\alpha n}}
  \end{align*}
  are continuous. Here, $\alpha = \left( \alpha_1,\dots,\alpha_n \right)$ is a \textit{multi-index}, where the $\alpha_i$ are positive integers for each $i$, and $\left\vert \alpha \right\vert $ is defined by $ \left\vert \alpha \right\vert = \sum_{i=1}^{n}\alpha_i$.
\end{definition}
We are concerned now with constructing $C^{\infty}$ functions on $C^{\infty}$-manifolds.\footnote{A $C^{\infty}$ manifold is one where all the transition functions $\varphi_{j}\circ\varphi_{i}^{-1}\colon \varphi_{i}\left( U_i\cap U_j \right)\rightarrow \varphi_{j}\left( U_i\cap U_j \right)$ are $C^{\infty}$ functions.} In order to do this, we introduce the bump functions.
\begin{definition}
  The \textit{bump function} that is equal to $1$ on $B\left( 0,1 \right)$ and is zero outside $U\left( 0,2 \right)$ is given by
  \begin{align*}
    h(x) &= \begin{cases}
      e^{-1/x} & x > 0\\
      0 & x\leq 0
    \end{cases}\\
      b(x) &= \frac{h\left( 4-\left\vert x \right\vert^2 \right)}{ h\left( 4-\left\vert x \right\vert^2 \right) + h\left( \left\vert x \right\vert^2 - 1 \right) }.\label{eq:bump_function}\tag{$\ast$}
  \end{align*}
\end{definition}
\begin{lemma}
  Let $M$ be a $C^{\infty}$ manifold. Let $U\in \mathcal{O}_{p}$, and let $f\colon U\rightarrow \R$ be an arbitrary $C^{\infty}$ function defined on $U$.\newline

  Then, there exists $V\in \mathcal{O}_{p}$ with $ \overline{V}\subseteq U $, and a $C^{\infty}$ function $ \widetilde{f} $ defined on $M$ such that
  \begin{align*}
    \widetilde{f}(q) &= \begin{cases}
      f(q) & q\in V\\
      0 & q\notin U.
    \end{cases}
  \end{align*}
\end{lemma}
\begin{proof}
  Let $\left( W,\varphi \right)$ be a chart centered at $p$ with $\varphi(p) = 0$ and $U\left( 0,3 \right)\subseteq \varphi(W)$. Let $ \overline{b} = b\circ\varphi $, where $b$ is the bump function defined in \eqref{eq:bump_function}. Then, $ \overline{b} $ is a $C^{\infty}$ function on $W$, and is $0$ outside $ \varphi^{-1}\left( U\left( 0,2 \right) \right) \subseteq W $.\newline

  We define $ \overline{b} $ to be equal to zero on $W^{c}$. Thus, if we define $V = \varphi^{-1}\left( U\left( 0,1 \right) \right)$, then $V\in \mathcal{O}_p$, $ \overline{V}\subseteq U $, and $ \overline{b} $ is equal to $1$ on $V$. Letting
  \begin{align*}
    \widetilde{f}(q) &= \begin{cases}
      \overline{b}(q) f(q) & q\in W\\
      0 & q\notin W,
    \end{cases}
  \end{align*}
  we see that $ \widetilde{f} $ satisfies the required property.
\end{proof}
Given an atlas $\set{\left( U_i,\varphi_i \right)}$, we want to be able to ``glue'' functions together by using these charts. A fundamental construction for this purpose is known as a partition of unity.
\begin{definition}
  Let $X$ be a topological space.
  \begin{itemize}
    \item An open cover $\set{U_{i}}_{i\in I}$ is called \textit{locally finite} if, for every $x\in X$, there is some $V\in \mathcal{O}_x$ such that $V\cap U_{i} = \emptyset$ for all but finitely many $i$.
    \item Another open cover $\set{V_{j}}_{j\in J}$ is called a refinement of another open cover $\set{U_i}_{i\in I}$ if for all $j\in J$, there exists some $i\in I$ such that $V_j\subseteq U_i$.
    \item We say $X$ is \textit{paracompact} if, for any open cover of $X$, there is a locally finite refinement.
  \end{itemize}
\end{definition}
\begin{proposition}
  Let $M$ be a topological manifold. Then, for any open cover $\set{U_{i}}_{i\in I}$ of $M$, there is a countable, locally finite refinement $\set{V_k}_{k=1}^{\infty}$ with the $ \overline{V_k} $ compact. In particular, $M$ is paracompact.\newline

  Additionally, we may select the coordinate maps $\psi_k\colon V_k\rightarrow \R^{n}$ such that $\psi_k\left( V_k \right) = U\left( 0,3 \right)$, and $\set{\psi_k^{-1}\left( U\left( 0,1 \right) \right)}_{k=1}^{\infty}$ is an open cover of $M$.
\end{proposition}
\begin{proof}
  Since $M$ is a locally Euclidean and second countable, there is a countable basis of pre-compact open sets $\set{O_\ell}_{\ell=1}^{\infty}$. In particular, we may select an exhaustion of $M$ by pre-compact sets by defining
  \begin{align*}
    E_1 &= O_1\\
    E_{k} &= O_1\cup O_2\cup \cdots \cup O_{\ell_k},
  \end{align*}
  where $\ell_{k}$ is some sufficiently large index as follows. Since $ \overline{E_k} $ is compact, there is a sufficiently large $\ell$ such that $ \overline{E_k}\subseteq O_1\cup\cdots\cup O_{\ell}$. Defining $\ell_{k+1}$ to be the smallest index greater than $\ell_{k}$ that satisfy this property, we define
  \begin{align*}
    E_{k+1} &= O_1 \cup \cdots \cup O_{\ell_{k+1}}.
  \end{align*}
  For arbitrary $k$, each $ \overline{E_k} $ is compact, and $ \overline{E_k}\subsetneq E_{k+1} $, and $\bigcup_{k=1}^{\infty}E_k = M$. Note that if $M$ is compact, this process terminates in a finite number of steps.\newline

  Now, let $\set{U_{i}}_{i\in I}$ be an arbitrary open cover of $M$, and fix $k \geq 1$. For each $p\in \overline{E_k}\setminus E_{k-1}$, select $i_{p}$ such that $p\in U_{i_p}$, and select a chart $\left( V_p,\psi_p \right)$ about $p$ that satisfies $\psi_p(p) = 0$, $\psi_p\left( V_p \right) = U\left( 0,3 \right)$, and $V_{p} \subseteq U_{i_p}\cap \left( E_{k+1}\setminus \overline{E_{k-2}} \right)$, where we set $E_{-1} = E_{0} = \emptyset$. Finally, set $W_{p} = \psi_{p}^{-1}\left( U\left( 0,1 \right) \right)$.\newline

  Since $ \overline{E_k} \setminus E_{k-1} $ is compact, we may select a finite number of such $p$ such that the open sets $W_{p}$ cover $ \overline{E_k}\setminus E_{k-1} $. Applying this process to all $k$, and lining up the charts $\left( V_p,\psi_p \right)$ corresponding to the finite number of points $p$ chosen at each stage, we have the locally finite refinement of $\set{U_{i}}_{i\in I}$ with each $ \overline{V_k} $ compact, $ \psi_k\left( V_k \right) = U\left( 0,3 \right) $, and $\set{\psi_k^{-1}\left( U\left( 0,1 \right) \right)}$ an open cover of $M$.
\end{proof}
\begin{definition}
  Let $M$ be a $C^{\infty}$ manifold. A family $\set{f_k}_{k=1}^{\infty}$ of at most countably many $C^{\infty}$ functions on $M$ is called a \textit{partition of unity} on $M$ if it satisfies:
  \begin{itemize}
    \item for each $k$, $f_{k}\left( p \right) \geq 0$ for all $p\in M$, and the family $\set{\supp\left( f_k \right)}_{k=1}^{\infty}$ is locally finite;
    \item at all points $p$ on $M$, $\sum_{k=1}^{\infty}f_k(p) = 1$.
  \end{itemize}
  If $\set{\supp\left( f_k \right)}_{k=1}^{\infty}$ is a refinement of an open cover $\set{U_i}_{i\in I}$, then we say the partition of unity is \textit{subordinate} to the open cover.
\end{definition}
\begin{theorem}
  Let $M$ be a $C^{\infty}$ manifold, and let $\set{U_{i}}_{i\in I}$ be an open cover of $M$. Then, there exists a partition of unity $\set{f_k}_{k=1}^{\infty}$ that is subordinate to $\set{U_{i}}_{i\in I}$.
\end{theorem}
\begin{proof}
  Let $\set{V_k}_{k=1}^{\infty}$ be a locally finite refinement of $\set{U_i}_{i\in I}$ such that the charts $\left( V_k,\psi_k \right)$ have $\psi_k\left( V_k \right) = U\left( 0,3 \right)$.\newline

  For each $k$, using the bump function \eqref{eq:bump_function}, define
  \begin{align*}
    \widetilde{b_k}\left( q \right) &= \begin{cases}
      b\circ \psi_k\left( q \right) & q\in V_k\\
      0 & q\notin V_k.
    \end{cases}
  \end{align*}
  Then, $ \widetilde{b_k} $ is a $C^{\infty}$ function defined on $M$, and since $ \supp\left( \widetilde{b_k} \right)\subseteq V_k $, we may set
  \begin{align*}
    f &= \sum_{k=1}^{\infty} \widetilde{b_k}.
  \end{align*}
  The function $f$ is a $C^{\infty}$ function defined on the whole of $M$. If we let $W_k = \psi_k^{-1}\left( U\left( 0,1 \right) \right)$, then since $\set{W_k}_{k\geq 1}$ is an open cover of $M$, for any $q\in M$, there exists $j$ such that $ \widetilde{b_j}\left( q \right) = 1 $. Thus, $f$ never equals $0$, so we if we set
  \begin{align*}
    f_k &= \frac{\widetilde{b_k}}{f},
  \end{align*}
  the family $\set{f_k}_{k\geq 1}$ is a partition of unity subordinate to $\set{U_i}_{i\in I}$.
\end{proof}
\section{Tangent Space, Vector Fields, and Cotangent Space}%
Smooth manifolds are able to be embedded into some Euclidean space,\footnote{This is actually a very deep theorem.} so we start by considering them as such.
\begin{definition}
  If $f\colon M\rightarrow N$ is a smooth map between an $n$-dimensional manifold $M$ and a $k$-dimensional manifold $N$ that are embedded into some Euclidean space $\R^{\ell}$, the \textit{derivative} of $f$ at $p$, defined for charts $\left( U,\varphi \right)$ and $\left( V,\psi \right)$, where $f(U)\subseteq V$, is defined by
  \begin{align*}
    D_{p}f &= D_{p}\left( \psi^{-1}\circ F\circ \varphi \right),
  \end{align*}
  where $F\colon \R^{n}\rightarrow \R^{k}$ is defined to be a map such that $f = \psi^{-1}\circ F \circ \varphi$.
\end{definition}
\begin{remark}
  This definition is independent of the chart representation. To see this, notice that as we have embedded both $M$ and $N$ into Euclidean space, the maps $\varphi\colon U\rightarrow \R^{n}$ and $\psi\colon V\rightarrow \R^{k}$ are diffeomorphisms, hence their derivatives are invertible linear maps.
  \begin{center}
    % https://tikzcd.yichuanshen.de/#N4Igdg9gJgpgziAXAbVABwnAlgFyxMJZABgBpiBdUkANwEMAbAVxiRAFUQBfU9TXfIRQBmclVqMWbAGrdeIDNjwEiZYePrNWiEAB1dAWzo4AFgCMzwAEpcAesDBc5fJYKKj11TVJ36jpi2s7YABrJy5xGCgAc3giUAAzACcIAyQyEBwIJABGL0ltEATnIpS0xAyspAAmajMYMCgkAFphDIY6eoYABX5lIRAkrGiTHBB8rTZ9eiS0EywAfRyS5NT06irEWpB6xqQ26g6u3tcVHQYYBLGJnz1dGbnF6pWymo3sxFEdhqbPjO9CgAxBbPHiJV5bd77Oo-FoHCSTHTA5aHTowHp9Nw6IYjMZg0prRB5TIfL67X7wo7ok4CM4gC5XcYI276NDYJYvQnEzZk2GIVr-ApTXRsp5MqkY04DHGjbgULhAA
\begin{tikzcd}
U \arrow[rrr, "f"] \arrow[ddd, "\varphi_1"', bend right] \arrow[ddd, "\varphi_2", bend left] &  &  & V \arrow[ddd, "\psi_1", bend left] \arrow[ddd, "\psi_2"', bend right] \\
                                                                                             &  &  &                                                                       \\
                                                                                             &  &  &                                                                       \\
\mathbb{R}^{n} \arrow[rrr, "F_2", bend left] \arrow[rrr, "F_1"', bend right]                 &  &  & \mathbb{R}^{k}                                                       
\end{tikzcd}
  \end{center}
  Using some coordinate changes, we see that
  \begin{align*}
    F_1 &= \left( \psi_1\circ \psi_2^{-1} \right)\circ F_2\circ\left( \varphi_2\circ \varphi_1^{-1} \right)
  \end{align*}
  so by the chain rule,
  \begin{align*}
    Df &= D\left( \psi_1^{-1}\circ F_1\circ \varphi_1 \right)\\
       &= D\left( \psi_1 \right)^{-1}\circ D\left( F_1 \right)\circ D\left( \varphi_1 \right)\\
       &= D\left( \psi_1^{-1} \right)\circ D\left( \left( \psi_1\circ \psi_2^{-1} \right)\circ F_2\circ \left( \varphi_2\circ \varphi_1^{-1} \right) \right)\\
       &= D\left( \psi_1 \right)^{-1}\circ D\left( \psi_1 \right)\circ D\left( \psi_2^{-1}\circ F_2\circ \varphi_2 \right)\circ D\left( \varphi_1 \right)^{-1}\circ D\left( \varphi_1 \right)\\
       &= D\left( \psi_2^{-1}\circ F_2\circ \varphi_2 \right).
  \end{align*}
\end{remark}
Note here that the chain rule is being used in $\R^{\ell}$, which Dundas calls the ``flat chain rule,''\footnote{Flatness is always relative.} rather than the general case on a manifold.\newline

One of the issues with this strategy, though, is that embeddings may carry different properties (though at high enough dimensions, any two embeddings are diffeomorphic to each other). For instance, embeddings $S^{1}\hookrightarrow \R^{3}$ form the field of knot theory, which is a very intricate field.\newline

As a result, we want to be able to define tangent spaces, derivatives, and the like without having to refer to coordinates. In order to do this, we need to discuss germs of functions.
\begin{definition}
  Let $g\colon M\rightarrow N$ map $p\mapsto g(p)$. We define an equivalence relation on the space of functions $f\colon M\rightarrow N$ with $f(p) = g(p)$ by taking $f_1\sim f_2$ whenever $f_1 = f_2$ on some open neighborhood $A\in \mathcal{O}_p$. The equivalence class $\left[ g \right]_p$ is known as the \textit{germ} of $g$ at $p$.\newline

  We denote by $\mathcal{C}_{p}$ the space of germs of $C^{\infty}$ functions $f\colon M\rightarrow \R$ at $p$.
\end{definition}
\begin{remark}
  Often, books will use $\mathcal{O}_p$ to refer to the space of germs at $p$. We will use $\mathcal{C}_p$ for this purpose though, as we have defined $\mathcal{O}_p$ to refer to the system of open neighborhoods at $p$.
\end{remark}
\begin{proposition}
  The space $\mathcal{C}_{p}$ with the operations
  \begin{itemize}
    \item $\left[ g \right] + \left[ h \right] = \left[ g + h \right]$;
    \item $\alpha \left[ g \right] = \left[ \alpha g \right]$;
    \item $\left[ g \right]\cdot \left[ h \right] = \left[ g\cdot h \right]$
  \end{itemize}
  forms an algebra over $\R$.
\end{proposition}
\begin{definition}
  Let $W_p$ be the space of germs of smooth maps $\gamma\colon \R\rightarrow M$ that send $0 \mapsto p$. The \textit{tangent space} $T_pM$ is defined by $W_p/\sim$, where we define the equivalence relation $\left[ g_1 \right]\sim \left[ g_2 \right]$ for two germs at $p$  $g_1\colon M\rightarrow \R$ and $g_2\colon M\rightarrow \R$ by
  \begin{align*}
    \left( \varphi\circ g_1 \right)'(0) &= \left( \varphi\circ g_2 \right)(0)
  \end{align*}
  for all $\varphi\in \mathcal{C}_p$.
\end{definition}
\begin{definition}
  If $f\colon M\rightarrow N$ is a smooth map, we define the map $T_pf\colon T_pM\rightarrow T_{f(p)}N$ to act via
  \begin{align*}
    T_pf\left( \left[ \gamma \right] \right) &= \left[ \gamma\circ f \right]
  \end{align*}
  for all $\gamma\in W_{p}$.
\end{definition}
\begin{proposition}[Chain Rule]
  If $f\colon M\rightarrow N$ and $g\colon N\rightarrow L$ are smooth maps, then
  \begin{align*}
    T_{f(p)}g\circ T_pf &= T_{p}\left( g\circ f \right).
  \end{align*}
\end{proposition}
\begin{proof}
  If $\gamma\in W_{p}$, then
  \begin{align*}
    T_{f(p)}g\circ T_{p}f\left( \left[ \gamma \right] \right) &= T_{f(p)}g\left( \left[ f\circ \gamma \right] \right)\\
                                                              &= \left[ g\circ f \circ \gamma \right]\\
                                                              &= T_{p}\left( g\circ f \right)\left( \left[ \gamma \right] \right).
  \end{align*}
\end{proof}
A terribly kept secret is that this function $T_pf$ is actually the differential $D_pf$. This requires us to prove that this definition comports with the definition for the case of $M$ as an embedded manifold. We require a few basic propositions for this purpose whose proofs follow from the inverse function theorem and various definitions.
\begin{proposition}
  If $f\colon M\rightarrow N$ is a locally invertible smooth map about $p\in M$, then $T_{p}f\colon T_pM\rightarrow T_{f(p)}N$ is an isomorphism of vector spaces.
\end{proposition}
\begin{proposition}
  If $0\in \R^{k}$, then $T_0\R^{k}$ is represented by linear maps $t\mapsto \lambda t$ for some vector $\lambda\in \R^{k}$. Therefore, if $M$ is $k$-dimensional, $T_pM\cong \R^{k}\cong \Hom\left( \R,\R^{k} \right)$.
\end{proposition}
\begin{proposition}
  If $\varphi\colon U\rightarrow \R^{k}$ is a local diffeomorphism about $p$ such that $\varphi(p) = 0$, then if $f\in C^{\infty}\left( \R^{k} \right)$, $f\circ\varphi\in C^{\infty}\left( U \right)$, which induces an algebra homomorphism 
  \begin{align*}
    \varphi^{\ast}\colon \mathcal{C}_{0,\R^{k}}&\rightarrow \mathcal{C}_{p,M}\\
    f &\mapsto f\circ\varphi.
  \end{align*}
\end{proposition}
\begin{proposition}
  If $M$ and $N$ are embedded submanifolds of $\R^{n}$ with dimensions $m$ and $k$ respectively, and $f\colon M\rightarrow N$ is a smooth map, then
  \begin{align*}
    D_pf \equiv T_pf.
  \end{align*}
\end{proposition}
\begin{proof}
  Let $\left( U,\varphi \right)$ and $\left( V,\psi \right)$ be charts for $M$ and $N$ respectively with $p\in U$ and $f(p)\in V$. Then, we may consider the coordinate maps $\varphi\colon U\rightarrow \R^{m}$ and $\psi\colon V\rightarrow \R^{k}$ to be such that $p\mapsto 0$ and $f(p)\mapsto 0$ respectively.\newline

  Now, we see that $T_pf$ and $D_pf$ can be written as
  \begin{align*}
    T_pf &= T_{f(p)}\psi^{-1}\circ T_0F\circ T_{f(p)}\varphi\\
    D_pf &= D_{p}\left( \psi^{-1} \circ F \circ \varphi \right).
  \end{align*}
  Yet, since $T_0F = D_0F$, $T_{f(p)}\psi^{-1} = D_{f(p)}\psi^{-1}$, and $T_p\varphi = D_p\varphi$, the chain rule gives
  \begin{align*}
    T_pf &= T_p\left( \psi^{-1}\circ F\circ \varphi \right)\\
    &= D_{p}\left( \psi^{-1}\circ F \circ \varphi \right)\\
    &= D_pf,
  \end{align*}
  implying that $T_pf = D_pf$.
\end{proof}
Now that we have established that we can consider manifolds as either standalone entities or as submanifolds of $\R^{n}$, we now shift our focus to understanding what information the derivative map $D_pf\colon T_pM\rightarrow T_{f(p)}M$ gives us about the underlying topology of $M$ and $N$.
\begin{definition}
  Let $f\colon M\rightarrow N$ be a smooth map, and let $p\in M$. We say $p$ is a \textit{critical point} for $f$ if $D_pf$ does not have maximal rank.\newline

  If $D_pf$ has maximal rank, then we say $p$ is a \textit{regular point} of $f$.\newline

  We say $q\in N$ is a \textit{critical value} if $f^{-1}\left( \set{q} \right)$ contains a critical point for $f$. Else, we say $q$ is a \textit{regular value}.
\end{definition}
The study of critical points is actually very vital in understanding the underlying manifold's global topology. This is the field known as Morse theory, and we will discuss it later in the course.
\begin{definition}
  Let $f\colon M\rightarrow \R$ be a smooth function, with $M$ a manifold. We say $f$ is \textit{Morse} if all the critical points of $f$ are isolated, and the critical points are nondegenerate, in the sense that the Hessian matrix, given by
  \begin{align*}
    H_pf &= \left( \pd{^2f}{x_i\partial x_j}(p) \right)_{i,j=1}^{n}
  \end{align*}
  has nonzero determinant, where $\left( x_1,\dots,x_n \right)$ is a coordinate system about the critical point $p$.
\end{definition}
Morse functions allow us to calculate a quantity known as the index of the manifold at any given value of $\R$, thereby allowing us to reconstruct the manifold from the information that the functions give us.\newline

There are two important theorems related to critical points/values and regular points/values. 
\begin{theorem}[Sard's Theorem]
  Let $f\colon M\rightarrow N$ be a smooth map of manifolds. The set of critical values in $N$ is of measure zero (where measure is defined by the measure of the image under a coordinate map).
\end{theorem}
We will not prove Sard's Theorem now, but we will prove a very useful result that is often used in conjunction with Sard's Theorem.
\begin{theorem}[Regular Value Theorem]
  Let $f\colon M\rightarrow N$ be a smooth map of manifolds with dimensions $m\geq n$ respectively. If $q\subseteq N$ is a regular value, then $f^{-1}\left( \set{q} \right)\subseteq M$ is a submanifold of dimension $m-n$.
\end{theorem}
\begin{proof}
  Let $p\in f^{-1}\left( \set{q} \right)$, and let $\left( U,\varphi \right)$ be a chart of $p$ where $\varphi\colon U\rightarrow \R^{m}\cong T_pM$ are identified as such. Since $D_pf$ is of full rank, we know that $K = \ker\left( D_pf \right)$ is of dimension $m-n$, so that $K\cong \R^{m-n}$.\newline

  Let $L\colon \R^{m}\rightarrow \R^{m-n}\cong K$ be a projection, and define
  \begin{align*}
    F\colon \R^{m}\supseteq U &\rightarrow N\times \R^{m-n}\\
    x &\mapsto \left( f(x),L(x) \right).
  \end{align*}
  Notice then that $D_pF = \left( D_pf,L \right)$, where the latter comes about from the fact that $L$ is a linear map. Then, we have that $D_pf$ is of rank $n$, and $L$ is of rank $m-n$, meaning that $D_pF\colon \R^{m}\rightarrow \R^{m}$ is of full rank, hence is invertible on a neighborhood $V\times W\subseteq N\times \R^{m-n}$, where $W$ is a neighborhood of $0\in \R^{m-n}$, so we may identify $U\cong V\times W$.\newline

  By composing with the projection $\pi\colon N\times \R^{m-n}\rightarrow N$ given by $\left( q,W \right)\rightarrow q$, we have that $f = \pi\circ F$, so that $f^{-1}\left( \set{q} \right) = F^{-1}\circ \pi^{-1}\left( \set{q} \right)$, meaning that $f^{-1}\left( \set{q} \right)\cong \R^{m-n}$.
\end{proof}
One of the central uses of the regular value theorem is the fact that it allows us to prove a smooth version of Brouwer's Fixed Point Theorem for the general case of manifolds, known as the No Retraction Theorem.
\begin{theorem}[No Retraction Theorem]
  Let $M$ be a compact smooth $n$-dimensional manifold with boundary, and let $N = \partial M$ be the boundary. There does not exist any smooth surjective function $r\colon M\rightarrow N$ such that $r(x) = x$ for every $x\in N$.
\end{theorem}
\begin{proof}
  Suppose toward contradiction that there were such a retraction, which we call $r$. Let $X$ be the set of critical points for $r$ in $M$; then, by Sard's Theorem, $r(X)\subseteq N$ has measure zero, so there exists a regular value $y\in N$.\newline

  By the Regular Value Theorem, $r^{-1}\left( \set{y} \right)$ is a smooth $1$-dimensional manifold, as $N$ is a $n-1$ dimensional manifold, meaning that $r^{-1}\left( \set{y} \right)$ is either $S^{1}$ or an open interval. If $r^{-1}\left( \set{y} \right)$ is a circle, then $r^{-1}\left( \set{y} \right)$ necessarily must be contained in the interior of $M$, which would contradict the fact that $y\in \partial M$. Therefore, $r^{-1}\left( \set{y} \right)$ is an interval, and specifically is one that has both of its endpoints on $N$, as on the interior of $M$, such an interval must be identified to a $1$-dimensional subspace of $M$, so there is some $z\neq y\in N$ such that $z\in r^{-1}\left( \set{y} \right)$. Yet, that means that $r(z) = y \neq z$, which is a contradiction.
\end{proof}
\subsection{The Tangent Bundle}%
Recall that we defined the differential $D_pf$ via the action on the manifold about the point $p$. Unfortunately, the issue with this formulation is that it is purely local --- the main reason we study manifolds is that we want to be able to use local information about the function to obtain insights about the global topology of the manifold. We need a construction that allows us to collect information about all the differentials at points of $M$ together. This is the tangent bundle.
\begin{definition}
  Let $M$ be a manifold. The \textit{tangent bundle} $TM$ is the disjoint union
  \begin{align*}
    TM &= \coprod _{p\in M} T_pM.
  \end{align*}
\end{definition}
Now, if $M$ is a manifold of dimension $m$, then $TM$ is a manifold of dimension $2m$. To see this, observe that if $p\in \R^{m}$, then $T_p\R^{m}\cong \R^{m}$.\newline

Therefore, if at each point in $\R^{m}$, we assign a copy of the tangent space, we have that
\begin{align*}
  T\R^{m} &\cong \R^{m}\times \R^{m}.
\end{align*}
If $f\colon \R^{m}\rightarrow \R^{n}$ is smooth, we get the map $Tf\colon T\R^{m}\rightarrow T\R^{n}$ given by
\begin{align*}
  \left( x,v \right) \mapsto \left( f(x),D_xf(v) \right).
\end{align*}
Now, given a coordinate map $\varphi\colon M\supseteq U\rightarrow \R^{m}$, we may define
\begin{align*}
  T\varphi(U) &= \varphi(U)\times \R^{m}.
\end{align*}
Thus, if $\set{\left( U_i,\varphi_i \right)}_{i\in I}$ is an atlas for $M$, we have transition maps
\begin{align*}
  T\psi\left( U\cap V \right)&\rightarrow T\varphi\left( U\cap V \right)\\
  \left( x,v \right) &\mapsto \left( \varphi\circ\psi^{-1}\left( x \right),D_{x}\left( \varphi\circ\psi^{-1} \right)\left( v \right) \right).
\end{align*}
Thus, if $f\colon M\rightarrow N$ is a smooth map, it induces a differential map on the tangent bundles $Df\colon TM\rightarrow TN$.
\begin{remark}
  If $M$ and $N$ only have $C^1$ structures, it turns out that there is a compatible $C^{\infty}$ structure, meaning that we may safely assume that any $C^{1}$ manifold is $C^{\infty}$.
\end{remark}
\subsection{Vector Fields}%
\begin{definition}
  If $M$ is a manifold, then a \textit{vector field} on $M$ is a smooth right-inverse of the projection map
  \begin{align*}
    \pi\colon TM&\rightarrow M\\
    \left( x,v \right) &\mapsto x.
  \end{align*}
\end{definition}
When we consider vector fields on manifolds, some basic questions crop up. The most basic of them all is the following: does there exist a nowhere-vanishing vector field on $M$?
\begin{itemize}
  \item In any Euclidean space, we may take a constant nonzero vector as our assignment, so the answer is yes.
  \item For $S^{1}$, we can embed it into $\R^{2}$, then take the map $\left( x,y \right) \mapsto \left( \left( x,y \right),\left( -y,x \right) \right)$, which is the family of tangent vectors to the unit circle in $\R^{2}$.
  \item For $S^{2}$, the answer is no. This is the much-celebrated ``hairy ball theorem.''
  \item For $S^{3}$, the answer is yes. In fact, for $S^{(2n-1)}$ where $n$ is a natural number, the answer is yes, while for $S^{2n}$, the answer is no.
\end{itemize}
Now, if $M$ is a manifold, with $\left( U,\varphi \right)$ a chart on $M$ with $p\in U$, identification $\varphi\colon U\rightarrow \R^{n}$, and coordinate representation $\left( x_1,\dots,x_n \right)$. Since $T\R^{n} = \R^{n}\times \R^{n}$, we may write the local coordinates for $T_pM$ formally as $ \left( \pd{}{x_1},\dots, \pd{}{x_n} \right) $. In other words, we consider the $\pd{}{x_i}$ as a basis for the vector space $T_pM$.\newline

If $X$ is a vector field on $M$, we may express $X$ locally about $p$ formally as
\begin{align*}
  X &= \sum_{i=1}^{n} a_i(p) \pd{}{x_i},
\end{align*}
where each $a_i\colon U\rightarrow \R$ is a $C^{\infty}$ function.\newline

The first question we have is whether this is well-defined. To do this, we consider another chart, $\left( V,\psi \right)$ with $p\in V$, identification $\psi\colon V\rightarrow \R^{n}$, and coordinate representation $\left( y_1,\dots,y_n \right)$. Given the coordinate change $\psi\circ \varphi^{-1}$, we want to consider a corresponding coordinate change $ \left( \pd{}{x_1},\dots, \pd{}{x_n} \right) \mapsto \left( \pd{}{y_1},\dots, \pd{}{y_n} \right) $. Via the chain rule on $\R^{n}$, we find that this corresponding coordinate change is
\begin{align*}
  \pd{}{x_i} &= \sum_{j=1}^{n} \pd{y_j}{x_i}\pd{}{y_j},
\end{align*}
which emerges from applying the differential to $\psi\circ \varphi^{-1}$.\newline

The space of vector fields on $M$ has algebraic structure.
\begin{itemize}
  \item If $X$ and $Y$ are vector fields, then so is $\alpha X + \beta Y$ for all $\alpha,\beta\in \R$.
  \item If $f\in C^{\infty}\left( M \right)$, then $f\cdot X$ is also a vector field on $X$, which is locally represented by
    \begin{align*}
      f\cdot X &= \sum_{i=1}^{n} \left( fa_i \right) \pd{}{x_i}.
    \end{align*}
\end{itemize}
Furthermore, $X$ acts on the space $C^{\infty}\left( M \right)$ via differentiation. If $p\in U$ with local coordinates $\left( x_1,\dots,x_n \right)$, we have $X(f)$ locally defined by
\begin{align*}
  X(f)(p) \coloneq \sum_{i=1}^{n} a_i(p) \pd{f}{x_i}(p).
\end{align*}
This action has two properties:
\begin{itemize}
  \item Linearity: $X\left( \alpha f + \beta g \right) = \alpha X(f) + \beta X(g)$;
  \item Leibniz Rule: $X\left( fg \right) = f\cdot X(g) + g\cdot X(f)$.
\end{itemize}
More generally, we say that an $\R$-linear map with these two properties is a \textit{derivation}. It turns out that the span of the set of derivations on $M$ is actually equal to the space of vector fields of $M$.\newline

Finally, vector fields on $M$ admit an intrinsic multiplication.
\begin{definition}
  If $X$ and $Y$ are vector fields on $M$, the \textit{Lie Bracket} of $X$ and $Y$ is defined for any $f\in C^{\infty}\left( M \right)$ by
  \begin{align*}
    \left[ X,Y \right]\left( f \right) &= X\left(Y\left(f\right)\right) - Y\left( X\left( f \right) \right).
  \end{align*}
\end{definition}
\begin{proposition}
  Let $p\in M$ have local chart $\left( U,\varphi \right)$, where $\varphi = \left( x_1,\dots,x_n \right)$ is the coordinate map, and let $X$ and $Y$ be vector fields on $M$ with local representation on $\left( U,\varphi \right)$ given by
  \begin{align*}
    X &= \sum_{i=1}^{n} a_i \pd{}{x_i}\\
    Y &= \sum_{i=1}^{n} b_i \pd{}{x_i}.
  \end{align*}
  Then, $\left[ X,Y \right]$ has local representation given by
  \begin{align*}
    \left[ X,Y \right] &= \sum_{i=1}^{n}\sum_{j=1}^{n} \left( a_i \pd{b_i}{x_j} - b_i \pd{a_i}{x_j} \right) \pd{}{x_i}.
  \end{align*}
\end{proposition}
\begin{proposition}
  The Lie Bracket $\left[ X,Y \right]$ of vector fields on $M$ satisfies three properties:
  \begin{itemize}
    \item bilinearity: $\left[ \alpha X_1 + \beta X_2 ,Y \right] = \alpha\left[ X_1,Y \right] + \beta \left[ X_2,Y \right]$ and $\left[ X,\alpha Y_1 + \beta Y_2 \right] = \alpha \left[ X,Y_1 \right] + \beta \left[ X,Y_2 \right]$ for all $\alpha,\beta\in \R$;
    \item alternating: $\left[ X,Y \right] = -\left[ Y,X \right]$;
    \item Jacobi identity: $\left[ \left[ X,Y \right],Z \right] + \left[ \left[ Z,X \right],Y \right] + \left[ \left[ Y,Z \right],X \right] = 0$;
    \item extended bilinearity for $C^{\infty}\left( M \right)$: $\left[ fX,gY \right] = fg\left[ X,Y \right] + fX\left( g \right)Y - gY\left( f \right)X$.
  \end{itemize}
\end{proposition}
\subsection{Submanifolds of Dimension 1}%
\begin{definition}
  Let $X$ be a vector field on $M$, with $p\in M$. An \textit{integral curve} for $X$ through $p$ is a $C^{\infty}$ map $c\colon \R\rightarrow M$ such that $0\mapsto p$ and $Dc\colon T\R \rightarrow TM$ maps $ \pd{}{t} \mapsto X $, where $\left( t,\pd{}{t} \right)$ are the global coordinates for the tangent bundle $T\R$.\newline

  In local coordinates, we may express $c$ and $Dc$ via
  \begin{align*}
    c(t) &= \left( x_1(t),\dots,x_n(t) \right)\\
    Dc\left( \pd{}{t} \right) &= \sum_{i=1}^{n} \diff{x_i}{t} \pd{}{x_i}\\
                              &= \left( \diff{x_1}{t},\dots, \diff{x_n}{t} \right)
  \end{align*}
  when $p\in U\subseteq M$ is a chart with local coordinates $\left( x_1,\dots,x_n \right)$. We then say that $c$ is an integral curve if, for a local representation
  \begin{align*}
    X &= \sum_{i=1}^{n} a_i \pd{}{x_i},
  \end{align*}
  we have
  \begin{align*}
    Dc\left( \pd{}{t} \right) &= \sum_{i=1}^{n} \diff{x_i}{t} \pd{}{x_i},
    \intertext{ or that }
    \diff{x_i}{t} &= a_i\left( x_1(t),\dots,x_n(t) \right)
  \end{align*}
  for all $i$, and $\left( x_1(0),\dots,x_n(0) \right) = p$.
\end{definition}
One can imagine an integral curve as a ``flow'' following a vector field traced out by a particle.\footnote{This is where the connection between differential topology and partial differential equations begins to appear.}
\begin{theorem}
  Given a vector field $X$ on $M$, there is a unique integral curve passing through every $p\in M$.
\end{theorem}
In order to prove this theorem, we need to recall the Picard--Lindelöf theorem from ordinary differential equations that gives us a sufficient condition for existence and uniqueness of solutions to initial value problems.
\begin{theorem}[Picard--Lindelöf]
  Let $x\colon \R\rightarrow \R^{n}$ and $f\colon \R\times \R^{n}\rightarrow \R^{n}$ be defined by
  \begin{align*}
    \left( \diff{x_1}{t},\dots, \diff{x_n}{t} \right) &= \dot{x}\\
                                                      &= f\left( t, x\right),
  \end{align*}
  with $x(0) = x_0$. If $f$ is Lipschitz, then the initial value problem has a unique solution defined on $\left( -\ve,\ve \right)$ for a sufficiently small $\ve$.
\end{theorem}
Recall that $f\colon \R^{n}\rightarrow \R^{m}$ is Lipschitz if there is $L < \infty$ such that
\begin{align*}
  \sup_{x\neq y} \frac{\left\vert f\left( x \right) - f\left( y \right) \right\vert}{\left\vert x-y \right\vert } &\leq L.
\end{align*}
Notice that if $f$ is continuously differentiable and defined on a compact domain, this follows immediately from the extreme value theorem.\newline

The proof of the Picard--Lindelöf theorem follows from a technique known as Picard iteration. Essentially, we rewrite the differential equation in integral form
\begin{align*}
  x(t) &= x_0 + \int_{0}^{t} f\left( s,x(s) \right)\:ds,
\end{align*}
and start by setting
\begin{align*}
  x_0(t) &= x_0.
\end{align*}
Then, we inductively define
\begin{align*}
  x_n(t) &= x_0 + \int_{0}^{t} f\left( s,x_{n-1}(s) \right)\:ds.
\end{align*}
Defining the integral operator
\begin{align*}
  K\left(y(t)\right) &= x_0 + \int_{0}^{t} f\left( s,y(s) \right)\:ds,
\end{align*}
we essentially desire to show that $K$ has a fixed point for all continuous $y$ on a sufficiently small neighborhood of $0$. This will follow from showing that $K$ is a contraction in the supremum metric on this sufficiently small neighborhood of $0$ whenever $f$ is Lipschitz, which will allow us to use the contraction mapping theorem, as continuous functions on a compact set are complete under the supremum metric.
\begin{corollary}
  If $M$ is a manifold, and $X$ a vector field on $M$ with $p\in M$, then there exists an integral curve $c\colon \left( -\ve,\ve \right)\rightarrow M$ such that $c(0) = p$.
\end{corollary}
\begin{definition}
  We call the vector field $X$ \textit{complete} if every integral curve along every point of $p$ can be extended to all of $\R$.
\end{definition}
There are many vector fields on manifolds that aren't complete. For instance, if $M = \R^{2}\setminus \set{0}$, and $X = \pd{}{x_1}$, then an integral curve through $\left( 1,0 \right)$ cannot be extended to all of $\R$, as it would hit the missing point at the origin.
\subsection{Flows and Diffeomorphism Groups}%
Complete vector fields on manifolds enable us to create diffeomorphisms. Furthermore, it can be shown that if $M$ has dimension greater than or equal to $2$, then $\operatorname{diff}\left( M \right)$ is $k$-transitive, in that any $k$-tuple of distinct elements can be mapped to any other $k$-tuple of distinct elements.
\begin{definition}
  A \textit{flow} on $M$ is a one-parameter group of diffeomorphisms of $M$, defined by
  \begin{align*}
    \varphi\colon R &\rightarrow \operatorname{diff}\left( M \right)\\
    t &\mapsto \varphi_t,
  \end{align*}
  where $\varphi_t(p) \coloneq c_p(t)$ when $c_p$ is the integral curve through $p$.\newline

  In particular, $\img(\varphi)$ is the flow.
\end{definition}
\begin{proposition}
  If $M$ is a connected manifold with $\Dim\left( M \right) \geq 2$, then $\operatorname{diff}\left( M \right)$ is $k$-transitive.
\end{proposition}
\begin{remark}
  The reason this does not work if $M$ is a $1$-dimensional manifold is that $\R$ is linearly ordered and $S^{1}$ is cyclically ordered.
\end{remark}
\begin{proof}
  We start with the case of $k = 1$.\newline

  Let $p$ and $q$ be in the same chart, $\left( U,\varphi \right)$, where $\varphi = \left( x_1,\dots,x_n \right)$ is the coordinate map. By composing with a series of affine transformations of $\R^{n}$, we may assume that $\varphi(p)$ is the origin and $q$ is on the coordinate axis $x_1$. Furthermore, let $N$ be a compact subset of $\R^{n}$ such that $p,q\in N$.\newline

  Define $f$ to be a smooth bump function on $M$ such that $f$ is $1$ on $N$ and $0$ outside a neighborhood of $N$. Then, if $X = f \pd{}{x_1}$ on $U$ and zero outside, we observe that the integral curve through $p$ passes through $q$, and is a flow with $\varphi(t) = \psi_t(p) = q$ for some $t\in \R$.\newline

  Thus $\operatorname{diff}\left( M \right)$ acts transitively on points in the same chart. Meanwhile, if $p$ and $q$ are not in some chart, we find a finite length ``chain'' of intersecting charts that move from the chart at $p$ to the chart at $q$, then by composing a finite collection of diffeomorphisms, we find our desired diffeomorphism.
\end{proof}
\subsection{Differential Forms}%
Now that we have a (reasonable) understanding of the tangent spaces of a manifold, we now concern ourselves with the dual to the tangent space, which are known as the \textit{cotangent spaces}. Similar to how vector fields emerge from the tangent bundle and its projection onto $M$, the ``dual'' to vector fields, known as differential forms, emerge from the cotangent bundle..\newline

Note that if $T_pM$ is a tangent space that is isomorphic to $\R^{n}$, then the corresponding cotangent space, denoted $T^{\ast}_pM$ is isomorphic to $\left( \R^{n} \right)^{\ast} \coloneq \Hom\left( \R^{n},\R \right)$.\newline

Recall that the basis for $T_pM$ is given by $\left( \pd{}{x_1},\dots,\pd{}{x_n} \right)$ for local coordinates $\left( x_1,\dots,x_n \right)$. The corresponding dual basis for $T_p^{\ast}M$ is established from the dual basis for a vector space $V$. Recall that if $\left( v_1,\dots,v_n \right)$ is a basis for $V$, then $\left( v_1^{\ast},\dots,v_n^{\ast} \right)$ is a basis for $V^{\ast}$, where
\begin{align*}
  v_i^{\ast}\left( v_j \right) &= \delta_{i}^{j}\\
                               &\coloneq \begin{cases}
                                 1 & i = j\\
                                 0 & \text{else}.
                               \end{cases}
\end{align*}
The corresponding (formal) basis for $T_p^{\ast}M$ is given by $\left( dx_1,\dots,dx_n \right)$.\newline

Now, the tangent bundle for $M$ is given by
\begin{align*}
  TM &= \bigcup_{p\in M} T_pM,
\end{align*}
and similarly, the cotangent bundle for $M$ is given by
\begin{align*}
  T^{\ast}M &= \bigcup_{p\in M}T_p^{\ast}M.
\end{align*}
Next, we concern ourselves with the manifold structure of $T^{\ast}M$. To start, if $U\subseteq M$ is a chart, then $T^{\ast}U \coloneq U\times \left( \R^{n} \right)^{\ast}$. Now we concern ourselves with the transition maps.\newline

Recall from linear algebra that if $A\colon V\rightarrow W$ is a linear map, then $A^{\ast}\colon W^{\ast}\rightarrow V^{\ast}$ is the dual map (or transpose) given by $A^{\ast}\varphi = \varphi\circ A$. Similarly, recall that for two charts $\left( U,\varphi \right)$ and $\left( V,\psi \right)$, the tangent bundle admits the map
\begin{align*}
  T\left( \varphi\left( U\cap V \right) \right)&\rightarrow T\left( \psi\left( U\cap V \right) \right),
  \left( x,v \right) &\mapsto \left( \psi\circ \varphi^{-1}\left( x \right),D_x\left( \psi\circ \varphi^{-1} \right) \right).
\end{align*}
Thus, we may consider the transition map for the cotangent bundle $T^{\ast}\left( \psi\left( U\cap V \right) \right)\rightarrow T^{\ast}\left( \varphi\left( U\cap V \right) \right)$ to be the dualization of the transition map for the tangent bundle.
\begin{definition}
  A differential (1-)form on $M$ is a smooth section of the projection map $\pi^{\ast}\colon T^{\ast}M\rightarrow M$.\newline

  Locally, if $U\subseteq M$ has coordinates $\left( x_1,\dots,x_n \right)$, then $\left( dx_1,\dots,dx_n \right)$ are coordinates for the cotangent space over $U$, with forms denoted
  \begin{align*}
    \omega &= \sum_{i=1}^{n}f_i\:dx_i,
  \end{align*}
  where $f_i\in C^{\infty}\left( M \right)$ for all $i$.
\end{definition}
\subsubsection{Some Exterior Algebra}%
One of the most important use cases for differential forms is that they enable us to perform integration on manifolds. This requires a bit of review of exterior algebra.\newline

If $\R^{n}$ admits a basis $\left( v_1,\dots,v_n \right)$, we define the exterior algebra
\begin{align*}
  \Lambda\R^{n} &= \bigoplus_{i=1}^{n} \Lambda^{i}\R^{n},
\end{align*}
where
\begin{align*}
  \Lambda^{0}\left( \R^{n} \right) &\cong \R\\
  \Lambda^{1}\left( \R^{n} \right) &\cong \R^{n}\\
  \Lambda^{i}\left( \R^{n} \right) &= \Span\set{v_{i_1}\wedge\cdots\wedge v_{i_k} | i_1 < i_2 < \cdots < i_k,v_i\wedge v_i = 0}.
\end{align*}
Thus, we see that for $\sigma\in S_{k}$,
\begin{align*}
  v_{i_{\sigma(1)}}\wedge\cdots\wedge v_{i_{\sigma(k)}} &= \sgn\left( \sigma \right) v_{i_1}\wedge\cdots\wedge v_{i_k}.
\end{align*}
\section{Notations}%
\begin{itemize}
  \item A general normed space $V$ will have its norm denoted by $\norm{\cdot}$. If $V = \R^{n}$, then we denote the norm by $ \left\vert \cdot \right\vert $.
  \item We denote topological spaces by $\left( X,\tau \right)$.
  \item $ U\left( x,r \right) = \set{y\in V | \norm{x-y} < r}$.
  \item $ B\left( x,r \right) = \set{y\in V | \norm{x-y} \leq r}$.
  \item $ \mathcal{N}_p $: neighborhood system centered at $p\in X$.
  \item $ \mathcal{O}_p $: system of \textit{open} neighborhoods centered at $p\in X$.
  \item When we say a number $n$ is positive, we mean that $n\geq 0$. Similarly, a sequence $\left( a_n \right)_n$ is decreasing (increasing) if $a_n\geq a_{n+1}$ ($a_n\leq a_{n+1}$).
\end{itemize}
\end{document}
