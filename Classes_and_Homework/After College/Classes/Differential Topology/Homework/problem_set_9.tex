\documentclass[10pt]{mypackage}

% sans serif font:
%\usepackage{cmbright}
%\usepackage{sfmath}
%\usepackage{bbold} %better blackboard bold

\usepackage{homework}
%\usepackage{notes}
\usepackage{newpxtext,eulerpx,eucal}
\renewcommand*{\mathbb}[1]{\varmathbb{#1}}

\fancyhf{}
\fancyhead[R]{Avinash Iyer}
\fancyhead[L]{Differential Topology: Problem Set 9}
\fancyfoot[C]{\thepage}

\setcounter{secnumdepth}{0}

\begin{document}
\RaggedRight
\begin{problem}[Problem 1]
  A topological group is a group which is also a Hausdorff topological space where the group operations are continuous.\newline

  Recall the definition of the concatenation operation on the fundamental group. Now, let $G$ be a path-connected topological group, and let $\pi_1\left( G,e \right)$ be the fundamental group of $G$ with base point $e$. Use the Hilton--Eckmann argument to prove that the concatenation operation on the fundamental group is commutative.
\end{problem}
\begin{problem}[Problem 2]
  Let $M$ and $N$ be smooth, orientable, closed manifolds of the same dimension $n$, and let $f\colon M\rightarrow N$ be a smooth function. Show that $f$ induces a map $f^{\ast}\colon H^n_{\operatorname{DR}}\left( N \right)\rightarrow H^{n}_{\operatorname{DR}}\left( M \right)$ which is multiplication by an integer. This is called the degree of $f$ and is written $\deg(f)$.
\end{problem}
\begin{solution}
  Letting $\omega\in H^{n}_{\operatorname{DR}}\left( N \right)$ be a nonvanishing top-dimensional form, we observe that there are two operations we are able to do on $\omega$. We may normalize to take
  \begin{align*}
    \int_{N}^{} \omega &= 1.
  \end{align*}
  By the naturality of the de Rham isomorphism, it follows that there is some $\delta\in \R$ such that.
  \begin{align*}
    \int_{M}^{} f^{\ast}\omega &= \delta \int_{N}^{} \omega
  \end{align*}
  Our task now is to show that $\delta \in \Z$.\newline

  Toward this end, let $q$ be a regular value of $f$. We may use a smooth bump function to restrict $\omega$ to a small open neighborhood $V$ of $q$. It follows then that $f^{-1}\left( q \right) = \set{p_1,\dots,p_{\ell}}$ for some $\ell$, with corresponding disjoint open neighborhoods $U_1,\dots,U_{\ell}$ locally diffeomorphic to $V$, whence the support of $f^{\ast}\omega$ is contained in the union of $U_1,\dots,U_{\ell}$. If $f^{-1}\left( q \right) = \emptyset$, then
  \begin{align*}
    \int_{M}^{} f^{\ast}\omega &= \int_{\emptyset}^{} f^{\ast}\omega\\
                               &= \delta \int_{N}^{} \omega\\
                               &= 0,
  \end{align*}
  whence $\delta = 0$. If $f^{-1}\left( q \right)\neq \emptyset$, then we see that
  \begin{align*}
    \int_{M}^{} f^{\ast}\omega &= \sum_{k=1}^{\ell} \int_{U_k}^{} f^{\ast}\omega\\
                               &= r \int_{N}^{} \omega,
  \end{align*}
  for some particular integer $r$ (as, by the definition of the pullback, integration over the pullback sums over the same domain with sign changes). In particular, this means that in the general case, $\delta \in \Z$.
\end{solution}
\begin{problem}[Problem 3]
  Recall the definition of the degree of $f$ from one of the previous problem sets, counting the sums of signs of determinants of the derivative of $f$ over the preimage of a regular value of $f$. Prove that the two definitions of the degree agree.
\end{problem}
\begin{solution}
  As we have seen in the previous problem, given a regular value $q\in N$ with the same setup as above, we have
  \begin{align*}
    \int_{M}^{} f^{\ast}\omega &= \sum_{k=1}^{\ell} \int_{U_k}^{} f^{\ast}\omega.
  \end{align*}
  In the particular case of exactly one regular value, we observe that by the change of coordinates formula,
  \begin{align*}
    \int_{U}^{} f^{\ast}\omega &= \sgn\left( \det\left( D_pf \right) \right) \int_{V}^{} \omega.
  \end{align*}
  Therefore, in the general case, we have
  \begin{align*}
    \int_{M}^{} f^{\ast}\omega &= \sum_{k=1}^{\ell} \sgn\left( \det\left( D_{p_k}f \right) \right) \int_{U_k}^{} f^{\ast}\omega\\
                               &= \sum_{k=1}^{\ell}\sgn\left( \det\left( D_{p_k}f \right) \right) \int_{V}^{} \omega\\
                               &= \deg\left( f \right) \int_{V}^{} \omega\\
                               &= \deg\left( f \right)\int_{N}^{} \omega.
  \end{align*}
  In particular, this gives
  \begin{align*}
    \deg\left( f \right) &= \sum_{k=1}^{\ell}\sgn\left( \det\left( D_{p_k}f \right) \right).
  \end{align*}
\end{solution}
\begin{problem}[Problem 4]
  With the setup of the previous exercises, prove that if $\omega$ is an arbitrary $n$-form on $N$, then
  \begin{align*}
    \int_{M}^{} f^{\ast}\omega &= \deg(f) \int_{N}^{} \omega.
  \end{align*}
\end{problem}
\begin{solution}\hfill
  \begin{center}
% https://tikzcd.yichuanshen.de/#N4Igdg9gJgpgziAXAbVABwnAlgFyxMJZABgBpiBdUkANwEMAbAVxiRAAkA9YMAXwH1gAHSEQ0MAE50cECWDoBbGMAAiAJV68AFADkAlCF6l0mXPkIoAjOSq1GLNlx4Dho8VJlzFy9Zq0BZAyMTbDwCIjJLW3pmVkQQEQVpAAsAI1TgDUNjEAxQ8yJrKOoYh3jElPTM3kNbGCgAc3giUAAzCQgFJDIQGSRrO1i2Vu4ROjgcGuCQds6kACZqPsQAZhL7OIShFRgGHDpsto6uxB7lxZAGOlTdgAVTMIsQCSwG5JwQdaHyoSwwHEEOimOVmJwGyzWgzKWz+AOA-imFF4QA
\begin{tikzcd}
H^{n}_{\operatorname{DR}}(N) \arrow[r, "f^{\ast}"] \arrow[d, "\int_{N}"'] & H^{n}_{\operatorname{DR}}(M) \arrow[d, "\int_{M}"] \\
\mathbb{R} \arrow[r, "\Delta"]                                            & \mathbb{R}                                        
\end{tikzcd}
  \end{center}
  Since the de Rham isomorphism induces a natural transformation, commutativity gives
  \begin{align*}
    \int_{M}^{} f^{\ast}\omega &= \delta \int_{N}^{} \omega\\
                               &= \deg\left( f \right) \int_{N}^{}\omega.
  \end{align*}
\end{solution}
\end{document}
