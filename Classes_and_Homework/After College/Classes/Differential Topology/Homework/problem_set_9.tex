\documentclass[10pt]{mypackage}

% sans serif font:
%\usepackage{cmbright}
%\usepackage{sfmath}
%\usepackage{bbold} %better blackboard bold

\usepackage{homework}
%\usepackage{notes}
\usepackage{newpxtext,eulerpx,eucal}
\renewcommand*{\mathbb}[1]{\varmathbb{#1}}

\fancyhf{}
\fancyhead[R]{Avinash Iyer}
\fancyhead[L]{Differential Topology: Problem Set 9}
\fancyfoot[C]{\thepage}

\setcounter{secnumdepth}{0}

\begin{document}
\RaggedRight
\begin{problem}[Problem 1]
  A topological group is a group which is also a Hausdorff topological space where the group operations are continuous.\newline

  Recall the definition of the concatenation operation on the fundamental group. Now, let $G$ be a path-connected topological group, and let $\pi_1\left( G,e \right)$ be the fundamental group of $G$ with base point $e$. Use the Hilton--Eckmann argument to prove that the concatenation operation on the fundamental group is commutative.
\end{problem}
\begin{solution}
  Define two operations, $\ast$ and $\cdot$, on the homotopy-classes of functions $f\colon S^{1}\rightarrow \left( G,e \right)$, where $S^{1}\cong [0,1]/\left(\set{0}\sim \set{1}\right)$ given by
  \begin{align*}
    f\ast g &= \begin{cases}
      f\left( 2t \right) & 0 \leq t \leq 1/2\\
      g\left( 2t-1 \right) & 1/2\leq t\leq 1
    \end{cases}\\
      f\cdot g &= f(t)g(t),
  \end{align*}
  where the latter is multiplication within the group and the former is concatenation. We see that the identity map
  \begin{align*}
    \operatorname{id}\colon S^{1} &\rightarrow \left( G,e \right)\\
    t &\mapsto e
  \end{align*}
  is an identity for both $\ast$ and $\cdot$. Our task now is to show that the Hilton--Eckmann condition holds. That is, let $a,b,c,d\colon S^{1}\rightarrow \left( G,e \right)$ be continuous maps with base point $e$. Then,
  \begin{align*}
    \left( a\ast b \right)\cdot \left( c\ast d \right) &= \left( a\ast b \right)(t)\cdot \left( c\ast d \right)(t)\\
                                                       &= \begin{cases}
                                                         a(2t)c(2t) & 0\leq t \leq 1/2\\
                                                         b(2t-1)d(2t-1) & 1/2\leq t \leq 1
                                                       \end{cases}\\
                                                       &= \left( a\cdot c \right)\ast \left( b\cdot d \right),
  \end{align*}
  whence $\cdot = \ast$ and the concatenation operation is commutative.
\end{solution}
\begin{problem}[Problems 2--4]\hfill
  \begin{description}[font=\normalfont]
    \item[(2)] Let $M$ and $N$ be smooth, orientable, closed manifolds of the same dimension $n$, and let $f\colon M\rightarrow N$ be a smooth function. Show that $f$ induces a map $f^{\ast}\colon H^n_{\operatorname{DR}}\left( N \right)\rightarrow H^{n}_{\operatorname{DR}}\left( M \right)$ which is multiplication by an integer. This is called the degree of $f$ and is written $\deg(f)$.
    \item[(3)] Recall the definition of the degree of $f$ from one of the previous problem sets, counting the sums of signs of determinants of the derivative of $f$ over the preimage of a regular value of $f$. Prove that the two definitions of the degree agree.
    \item[(4)] With the setup of the previous exercises, prove that if $\omega$ is an arbitrary $n$-form on $N$, then
      \begin{align*}
        \int_{M}^{} f^{\ast}\omega &= \deg(f) \int_{N}^{} \omega.
      \end{align*}
  \end{description}
\end{problem}
\begin{solution}
  Letting $\omega\in H^{n}_{\operatorname{DR}}\left( N \right)$ be a nonvanishing top-dimensional form. By the naturality of the de Rham isomorphism, it follows that there is some $\delta\in \R$ such that.
  \begin{align*}
    \int_{M}^{} f^{\ast}\omega &= \delta \int_{N}^{} \omega
  \end{align*}
  Our task now is to show that $\delta \in \Z$. In particular, we will show that $\delta = \deg\left( f \right)$, where $\deg\left( f \right)$ is defined as before.\newline

  Toward this end, let $q$ be a regular value of $f$. We may use a smooth bump function to restrict $\omega$ to a small open neighborhood $V$ of $q$. It follows then that $f^{-1}\left( q \right) = \set{p_1,\dots,p_{\ell}}$ for some $\ell$, with corresponding disjoint open neighborhoods $U_1,\dots,U_{\ell}$ locally diffeomorphic to $V$, whence the support of $f^{\ast}\omega$ is contained in the union of $U_1,\dots,U_{\ell}$. If $f^{-1}\left( q \right) = \emptyset$, then
  \begin{align*}
    \int_{M}^{} f^{\ast}\omega &= \int_{\emptyset}^{} f^{\ast}\omega\\
                               &= \delta \int_{N}^{} \omega\\
                               &= 0,
  \end{align*}
  whence $\delta = 0$. If $f^{-1}\left( q \right)\neq \emptyset$, then we see that
  \begin{align*}
    \int_{M}^{} f^{\ast}\omega &= \sum_{k=1}^{\ell} \int_{U_k}^{} f^{\ast}\omega.
  \end{align*}
  Now, since $f$ is a local diffeomorphism on each of the $U_k$, it follows that
  \begin{align*}
    \int_{U_k}^{} f^{\ast}\omega &= \sgn\left( \det\left( D_{p_k}f \right) \right) \int_{V}^{} \omega\\
                                 &= \sgn\left( \det\left( D_{p_k}f \right) \right) \int_{N}^{} \omega.
  \end{align*}
  Therefore, we find that
  \begin{align*}
    \int_{M}^{} f^{\ast}\omega &= \sum_{k=1}^{\ell}\sgn\left( \det\left( D_{p_k}f \right) \right) \int_{N}^{} \omega\\
                               &= \deg\left( f \right) \int_{N}^{} \omega,
  \end{align*}
  giving that $\deg\left( f \right)$ as defined via cohomology and as defined via summation over neighborhoods of preimages of a regular value are equal to each other.
\end{solution}
\end{document}
