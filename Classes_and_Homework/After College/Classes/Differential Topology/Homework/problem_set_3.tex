\documentclass[10pt]{mypackage}

% sans serif font:
%\usepackage{cmbright}
%\usepackage{sfmath}
%\usepackage{bbold} %better blackboard bold

\usepackage{homework}
%\usepackage{notes}
\usepackage{newpxtext,eulerpx,eucal}
\renewcommand*{\mathbb}[1]{\varmathbb{#1}}

\fancyhf{}
\rhead{Avinash Iyer}
\lhead{Differential Topology: Problem Set 3}

\setcounter{secnumdepth}{0}

\begin{document}
\RaggedRight
\begin{problem}[Problem 1]
  Let $f\colon M\rightarrow N$ be a smooth map of manifolds. Prove that the graph of $f$ is a smooth submanifold of $M\times N$.
\end{problem}
\begin{solution}
  Let $\left( U,\varphi \right)$ be a chart on $M$ with $\varphi\left( U \right)\cong \R^{m}$, and $\left( V,\psi \right)$ a chart on $N$ with $\psi\left( V \right)\cong \R^{n}$ and $f\left( U \right)\subseteq V$.\newline

  Define a chart on $M\times N$ corresponding to $U\times V$, and notice that the graph of $f|_{U}$ is a subset of $U\times V$.
\end{solution}
\begin{problem}[Problem 2]
  Let $\operatorname{U}(n)$ be the set of unitary complex $n\times N$ matrices. Write $\operatorname{SU}\left( n \right)\leq \operatorname{U}(n)$ for the kernel of the determinant map.
  \begin{enumerate}[(a)]
    \item Show that $\operatorname{U}(1)$ is diffeomorphic to the circle, so that $\operatorname{SU}(1)$ is a point.
    \item Prove that $\operatorname{U}(n)$ is a smooth manifold.
    \item Prove that $\operatorname{SU}(2)$ is diffeomorphic to $S^{3}$, the three-sphere.
    \item Prove that $\operatorname{U}(2)$ is diffeomorphic to $S^{1}\times S^{3}$.
  \end{enumerate}
\end{problem}
\begin{solution}\hfill
  \begin{enumerate}[(a)]
    \item Since complex $1\times 1$ matrices are diffeomorphic to $\C$, we see that $x\in \operatorname{U}(1)$ if and only if $\left\vert x \right\vert^2 = 1$, meaning $\left\vert x \right\vert = 1$, so $x = e^{i\theta}$ for some $\theta$. In particular, this means that the assignment $x\mapsto e^{i\theta}$ gives a diffeomorphism between $S^{1}$ and $\operatorname{U}(1)$.
    \item Consider the self-map $f\colon \Mat_{n}\left( \C \right)\rightarrow \Mat_{n}\left( \C \right)\cong \C^{n^2}$ given by $f(A) = A^{\ast}A$. Note that this maps $\Mat_{n}\left( \C \right)$ to positive semi-definite matrices $\Mat_{n}\left( \C \right)^{+}$.\newline

      We want to calculate the derivative of $f$ by taking
      \begin{align*}
        f\left( A + H \right) - f\left( A \right) &= \left( A+H \right)^{\ast}\left( A+H \right) - A^{\ast}A\\
                                                  &= \left( A^{\ast} + H^{\ast} \right)\left( A + H \right) - A^{\ast}A\\
                                                  &= A^{\ast}A + H^{\ast}A + A^{\ast}H + H^{\ast}H - A^{\ast}A\\
                                                  &= H^{\ast}A + A^{\ast}H + H^{\ast}H.
      \end{align*}
      Dividing out by $\norm{H}_{\op}$, we find that $D_{A}(f) = A + A^{\ast}$. Now, since $I$ is of full rank, so too is $\frac{1}{2}I$, meaning that $D_{\frac{1}{2}I}(f) = I$, and thus $f$ has a locally defined inverse about $I$. In particular, this means that $f^{-1}\left( \set{I} \right)$ consists entirely of regular points, or that $I$ is a regular value for $f$. Thus, $\operatorname{U}(n)$ is a smooth manifold.
    \item We view $S^{3}$ as a subset of $\C^{2}$, so that $S^{3}$ consists of all $\left( z_1,z_2 \right)$ such that
      \begin{align*}
        \left\vert z_1 \right\vert^2 + \left\vert z_2 \right\vert^2 &= 1.
      \end{align*}
      We claim that the matrix
      \begin{align*}
        A_{z_1,z_2} &= \begin{pmatrix}z_1 & z_2 \\ - \overline{z_2} & \overline{z_1}\end{pmatrix}
      \end{align*}
      is an element of $\operatorname{SU}\left( 2 \right)$. Since it is uniquely determined by $z_1$ and $z_2$ in $S^{3}$, it follows that $\operatorname{SU}(2)$ is diffeomorphic to $S^{3}$.\newline

      To see this, observe that
      \begin{align*}
        \det\left( A \right) &= 1\\
        A^{\ast}A &= \begin{pmatrix} \overline{z_1} & -z_2\\ \overline{z_2} & z_1\end{pmatrix} \begin{pmatrix}z_1 & z_2 \\ - \overline{z_2} & \overline{z_1}\end{pmatrix}\\
                  &= \begin{pmatrix}\left\vert z_1 \right\vert^2 + \left\vert z_2 \right\vert^2 & z_2 \overline{z_1} - z_2 \overline{z_1}\\ z_1 \overline{z_2} - z_1 \overline{z_2} & \left\vert z_1 \right\vert^2 + \left\vert z_2 \right\vert^2\end{pmatrix}\\
                  &= \begin{pmatrix}1 & 0 \\ 0 & 1\end{pmatrix}.
      \end{align*}
      Therefore, $\operatorname{SU}(3)$ is diffeomorphic to $S^{3}$, with the diffeomorphism given by coordinate assignment.
    \item Observe that if $\left( z_1,z_2 \right) = z\in S^{3}\subseteq \C^{2}$, then if $a\in \operatorname{U}(2)$, we have $az\in S^{3}$. In particular, since unitary matrices are invertible, the operation of $a\in \operatorname{U}(2)$ on $z\in S^{3}$ by multiplication is a group action.
  \end{enumerate}
\end{solution}
\end{document}
