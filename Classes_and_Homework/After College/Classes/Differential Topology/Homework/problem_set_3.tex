\documentclass[10pt]{mypackage}

% sans serif font:
%\usepackage{cmbright}
%\usepackage{sfmath}
%\usepackage{bbold} %better blackboard bold

\usepackage{homework}
%\usepackage{notes}
\usepackage{newpxtext,eulerpx,eucal}
\renewcommand*{\mathbb}[1]{\varmathbb{#1}}

\fancyhf{}
\rhead{Avinash Iyer}
\lhead{Differential Topology: Problem Set 3}

\setcounter{secnumdepth}{0}

\begin{document}
\RaggedRight
\begin{problem}[Problem 1]
  Let $f\colon M\rightarrow N$ be a smooth map of manifolds. Prove that the graph of $f$ is a smooth submanifold of $M\times N$.
\end{problem}
\begin{solution}
  Let $\left( U,\varphi \right)$ be a chart on $M$ with $\varphi\left( U \right)\cong \R^{m}$, and $\left( V,\psi \right)$ a chart on $N$ with $\psi\left( V \right)\cong \R^{n}$ and $f\left( U \right)\subseteq V$.\newline

  Define a chart on $M\times N$ corresponding to $U\times V$, and notice that the graph of $f|_{U}$ is a subset of $U\times V$.
\end{solution}
\begin{problem}[Problem 2]
  Let $\operatorname{U}(n)$ be the set of unitary complex $n\times N$ matrices. Write $\operatorname{SU}\left( n \right)\leq \operatorname{U}(n)$ for the kernel of the determinant map.
  \begin{enumerate}[(a)]
    \item Show that $\operatorname{U}(1)$ is diffeomorphic to the circle, so that $\operatorname{SU}(1)$ is a point.
    \item Prove that $\operatorname{U}(n)$ is a smooth manifold.
    \item Prove that $\operatorname{SU}(2)$ is diffeomorphic to $S^{3}$, the three-sphere.
    \item Prove that $\operatorname{U}(2)$ is diffeomorphic to $S^{1}\times S^{3}$.
  \end{enumerate}
\end{problem}
\begin{solution}\hfill
  \begin{enumerate}[(a)]
    \item Since complex $1\times 1$ matrices are diffeomorphic to $\C$, we see that $x\in \operatorname{U}(1)$ if and only if $\left\vert x \right\vert^2 = 1$, meaning $\left\vert x \right\vert = 1$, so $x = e^{i\theta}$ for some $\theta$. In particular, this means that the assignment $x\mapsto e^{i\theta}$ gives a diffeomorphism between $S^{1}$ and $\operatorname{U}(1)$.
    \item Consider the self-map $f\colon \Mat_{n}\left( \C \right)\rightarrow \Mat_{n}\left( \C \right)\cong \C^{n^2}$ given by $f(A) = A^{\ast}A$. Note that this maps $\Mat_{n}\left( \C \right)$ to positive semi-definite matrices $\Mat_{n}\left( \C \right)^{+}$.\newline

      We want to calculate the derivative of $f$ by taking
      \begin{align*}
        f\left( A + H \right) - f\left( A \right) &= \left( A+H \right)^{\ast}\left( A+H \right) - A^{\ast}A\\
                                                  &= \left( A^{\ast} + H^{\ast} \right)\left( A + H \right) - A^{\ast}A\\
                                                  &= A^{\ast}A + H^{\ast}A + A^{\ast}H + H^{\ast}H - A^{\ast}A\\
                                                  &= H^{\ast}A + A^{\ast}H + H^{\ast}H.
      \end{align*}
      Dividing out by $\norm{H}_{\op}$, we find that $D_{A}(f) = A + A^{\ast}$. Now, since $I$ is of full rank, so too is $\frac{1}{2}I$, meaning that $D_{\frac{1}{2}I}(f) = I$, and thus $f$ has a locally defined inverse about $I$. In particular, this means that $f^{-1}\left( \set{I} \right)$ consists entirely of regular points, or that $I$ is a regular value for $f$. Thus, $\operatorname{U}(n)$ is a smooth manifold.
    \item We view $S^{3}$ as a subset of $\C^{2}$, so that $S^{3}$ consists of all $\left( z_1,z_2 \right)$ such that
      \begin{align*}
        \left\vert z_1 \right\vert^2 + \left\vert z_2 \right\vert^2 &= 1.
      \end{align*}
      We claim that the matrix
      \begin{align*}
        A_{z_1,z_2} &= \begin{pmatrix}z_1 & z_2 \\ - \overline{z_2} & \overline{z_1}\end{pmatrix}
      \end{align*}
      is an element of $\operatorname{SU}\left( 2 \right)$. Since it is uniquely determined by $z_1$ and $z_2$ in $S^{3}$, it follows that $\operatorname{SU}(2)$ is diffeomorphic to $S^{3}$.\newline

      To see this, observe that
      \begin{align*}
        \det\left( A \right) &= 1\\
        A^{\ast}A &= \begin{pmatrix} \overline{z_1} & -z_2\\ \overline{z_2} & z_1\end{pmatrix} \begin{pmatrix}z_1 & z_2 \\ - \overline{z_2} & \overline{z_1}\end{pmatrix}\\
                  &= \begin{pmatrix}\left\vert z_1 \right\vert^2 + \left\vert z_2 \right\vert^2 & z_2 \overline{z_1} - z_2 \overline{z_1}\\ z_1 \overline{z_2} - z_1 \overline{z_2} & \left\vert z_1 \right\vert^2 + \left\vert z_2 \right\vert^2\end{pmatrix}\\
                  &= \begin{pmatrix}1 & 0 \\ 0 & 1\end{pmatrix}.
      \end{align*}
      Therefore, $\operatorname{SU}(3)$ is diffeomorphic to $S^{3}$, with the diffeomorphism given by coordinate assignment.
    \item Observe that if $\left( z_1,z_2 \right) = z\in S^{3}\subseteq \C^{2}$, then if $a\in \operatorname{U}(2)$, we have $az\in S^{3}$. In particular, since unitary matrices are invertible, the operation of $a\in \operatorname{U}(2)$ on $z\in S^{3}$ by multiplication is a group action.\newline

      We observe now that the action of $\operatorname{U}(2)$ on $S^{3}\subseteq \C^{2}$ by matrix multiplication is transitive, since for any element $\left( w_1,w_2 \right)\in S^{3}$, the matrix
      \begin{align*}
        \begin{pmatrix}w_1 & - \overline{w_2} \\ w_2 & \overline{w_1}\end{pmatrix} \begin{pmatrix}1\\0\end{pmatrix} &= \begin{pmatrix}w_1\\w_2\end{pmatrix},
      \end{align*}
      and
      \begin{align*}
        \begin{pmatrix} \overline{w_1} & \overline{w_2} \\ -w_2 & w_1\end{pmatrix} \begin{pmatrix}w_1 & - \overline{w_2} \\ w_2 & \overline{w_1}\end{pmatrix} &= \begin{pmatrix}1 & 0 \\ 0 & 1\end{pmatrix}.
      \end{align*}
      Additionally, we observe that for any $\theta$,
      \begin{align*}
        \begin{pmatrix}1 & 0 \\ 0 & e^{i\theta}\end{pmatrix} \begin{pmatrix}1\\0\end{pmatrix} &= \begin{pmatrix}1\\0\end{pmatrix},
      \end{align*}
      meaning 
      \begin{align*}
        S^{3} &\cong \operatorname{U}(2)/ P,
      \end{align*}
      where $P$ consists of all matrices of the form
      \begin{align*}
        Q &= \begin{pmatrix}1 & 0 \\ 0 & e^{i\theta}\end{pmatrix}.
      \end{align*}
      We observe that $P$ is diffeomorphic to $S^{1}$ via a coordinate assignment, so $\operatorname{U}(2)\cong S^{3}\times S^{1}$.
  \end{enumerate}
\end{solution}
\begin{problem}[Problem 3]
  In this exercise, we will prove the Frobenius theorem.\newline

  Let $M$ be a smooth manifold of dimension $n$, and let $D$ be an $r$-dimensional distribution on $M$, where $r\leq n$. That is, $D$ picks out an $r$-dimensional $D_p$ of $T_pM$ for each $p\in M$. In other words, at every point, there are $r$ distinguished, linearly independent vector fields defined in a neighborhood of the point.\newline

  A submanifold $N\subseteq M$ is called an \textit{integral submanifold} for $D$ if $T_pN = D_p$ for each $p\in M$. We say $D$ is \textit{completely integrable} if an integral submanifold exists through every point. Integral curves of a vector field are integral submanifolds of a vector field.\newline

  We call a distribution that is closed under taking Lie brackets involutive. That is, for any vector fields $X,Y\in D$ (i.e., local $1$-distributions that lie in $D$), then $\left[ X,Y \right]\in D$.\newline

  The Frobenius Theorem says that a distribution $D$ on $M$ is completely integrable if and only if it is involutive.
  \begin{enumerate}[(a)]
    \item Show that if $D$ is a completely integrable distribution, then $D$ is involutive.
    \item We say vector fields $X$ and $Y$ commute if $ \left[ X,Y \right] = 0 $. For fixed vector fields $X$ and $Y$, write $\phi_t$ and $\psi_t$ for the corresponding flows. Show that the following are equivalent:
      \begin{enumerate}[(i)]
        \item $X$ and $Y$ commute;
        \item $Y$ is invariant under $\phi_t$;
        \item the flows $\phi_t$ and $\psi_t$ commute, so that $\psi_s\circ \phi_t = \phi_t\circ \psi_s$ for all $t$ and $s$ where defined.
      \end{enumerate}
    \item Assume $D$ is $r$-dimensional. Choose local coordinates $\set{x_1,\dots,x_n}$ near a point $p$ and $r$-linearly independent vector fields $Y_1,\dots,Y_r$ near $p$. Write $Y_i$ as
      \begin{align*}
        \sum_{j=1}^{n} a_{ij} \pd{}{x_j},
      \end{align*}
      and show that there is some permutation of the coordinates such that the $r\times r$ matrix $\left( a_{ij} \right)_{1\leq i,j\leq r}$ is invertible near $p$.
    \item Let $\left( b_{ij} \right)_{1\leq i,j\leq r}$ be the inverse of the smoothly varying family of matrices $\left( a_{ij} \right)_{1\leq i,j\leq r}$ from the previous part, and let $X_i = \sum_{j}b_{ij}Y_j$. Show that
      \begin{align*}
        X_i &= \pd{}{x_i} + \sum_{j > r} c_{ij} \pd{}{x_j}
      \end{align*}
      for some suitable smooth functions. Show that $X_1,\dots,X_r$ form a basis for $D$ at every point.
    \item Show that $\left[ X_i,X_j \right] = 0$ for $1\leq i,j\leq r$.
    \item Use the flows generated by $\set{X_1,\dots,X_n}$ to define a smooth map $\phi\colon V\rightarrow U$ where $V$ is a neighborhood of $0\in\R^{r}$ and $U$ is a neighborhood of $p\in M$.
    \item Choose coordinates $\set{t_1,\dots,t_r}$ on $\R^{r}$ such that $\phi_{\ast}\left( \pd{}{t_i} \right) = X_i$. Argue by shrinking $V$ and $U$ if necessarily that $V$ is a submanifold of $U$. Use the fact that the flows generated by $X_1,\dots,X_r$ commute to prove that at an arbitrary point $q\in \phi(V)$, we have $D_q = T_q\phi(V)$. Conclude that $\phi(V)$ locally defines an integral submanifold $N$ of the distribution $D$.
  \end{enumerate}
\end{problem}
\begin{solution}\hfill
  \begin{enumerate}[(a)]
    \item Let $p\in N\subseteq M$, and let $\left( U;x_1,\dots,x_r \right)$ be a chart in $N$ such that $ \pd{}{x_1},\dots, \pd{}{x_r} $ forms a basis for $T_qN$ for any $q\in U$. Letting $X_1,\dots,X_r$ be linearly independent vector fields for $D$, we observe that, for any $f\in C^{\infty}\left( N \right)$,
      \begin{align*}
        \left[ X_k,X_{\ell} \right](f) &= \sum_{i=1}^{r} \left( \sum_{j=1}^{r} a_{ki}(q) \pd{a_{\ell i}}{x_j}(q) - a_{\ell i}(q) \pd{a_{ki}}{x_j}(q)  \right) \pd{f}{x_i}(q)
      \end{align*}
      Observe that if we extend $ \pd{}{x_1},\dots, \pd{}{x_r} $ to a basis for $T_qM$, then via this computation, $ \pd{}{x_1},\dots, \pd{}{x_r} $ remains invariant under the calculation of the Lie bracket, so $D_p$ is involutive.
  \end{enumerate}
\end{solution}
\end{document}
