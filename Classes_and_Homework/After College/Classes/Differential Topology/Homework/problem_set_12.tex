\documentclass[10pt]{mypackage}

% sans serif font:
%\usepackage{cmbright}
%\usepackage{sfmath}
%\usepackage{bbold} %better blackboard bold

\usepackage{homework}
%\usepackage{notes}
\usepackage{newpxtext,eulerpx,eucal}
\renewcommand*{\mathbb}[1]{\varmathbb{#1}}

\fancyhf{}
\fancyhead[R]{Avinash Iyer}
\fancyhead[L]{Differential Topology: Problem Set 12}
\fancyfoot[C]{\thepage}

\setcounter{secnumdepth}{0}

\begin{document}
\RaggedRight
\begin{problem}[Problem 1]
  In this exercise, we prove another fundamental result in differential topology, called the tubular neighborhood theorem. Let $M$ be a compact smooth manifold with orientable boundary $N$. For simplicity, assume that $N$ is connected. The tubular neighborhood theorem asserts that $N$ admits a neighborhood in $M$ which is diffeomorphic to $N\times [0,1)$.
  \begin{enumerate}[(a)]
    \item Choose a Riemannian metric on $M$, and show that $N$ admits a nonvanishing vector field that is everywhere orthogonal to the tangent space of $N$. That is, a vector field $X$ such that for all $p\in N$, $g\left( X_p,T_pN \right) = 0$.
    \item Use the flow generated by $X$ to find the desired neighborhood.
  \end{enumerate}
\end{problem}
\begin{solution}\hfill
  \begin{enumerate}[(a)]
    \item If $p\in N$, then we observe that $T_pN < T_pM$ is a proper subspace with codimension $1$. Letting $\set{e_1,\dots,e_{n-1}}$ be an orthonormal basis for $T_pN$, then we may extend to a basis for $T_pM$ by taking a representative for a basis for $T_pM/T_pN$, and observing that such a vector necessarily has
      \begin{align*}
        g_p\left( e_n,e_k \right) &= 0
      \end{align*}
      for all $k=1,\dots,n-1$. By smoothly varying over all points $p\in N$, we get our desired everywhere nonvanishing vector field normal to $T_pN$.
    \item Let $\varphi_t$ be the one-parameter diffeomorphism group generated by $X$, where $\varphi_t\colon M\rightarrow M$ is such that $\varphi_0(p) = p$ for all $p\in N$. Then, $\varphi\colon \left( -\ve,\ve \right)\rightarrow \operatorname{diff}\left( M \right)$ restricted to $\left[ 0,\ve \right)$ gives our desired neighborhood in $M$ diffeomorphic to $N\times \left[ 0,1 \right)$.
  \end{enumerate}
\end{solution}
\begin{problem}[Problem Set 7, Problem 5]
  Suppose $G$ is a finite group acting freely on a manifold $M$ by diffeomorphisms.
  \begin{enumerate}[(a)]
    \item Show that $M/G$ is a manifold.
    \item Show that the de Rham cohomology of $M/G$ is isomorphic to the $G$-invariant cohomology of $M$.
  \end{enumerate}
\end{problem}
\begin{problem}[Problem Set 8, Problem 3]
  Compute the de Rham cohomology of $\R\mathbb{P}^{n}$.
\end{problem}
\begin{solution}
  We will use the result related to invariant cohomology to compute this.
\end{solution}
\begin{problem}[Problem Set 8, Problem 5]
  Use the Mayer--Vietoris sequence to prove the Künneth Formula: if $M$ and $N$ are smooth manifolds, then $H^{\ast}_{\operatorname{DR}}\left( M\times N \right)$ is the tensor product of $H^{\ast}_{\operatorname{DR}}\left( M \right)$ and $H^{\ast}_{\operatorname{DR}}\left( N \right)$. Specifically, in each degree $\ell$, we have
  \begin{align*}
    H^{\ell}_{\operatorname{DR}}\left( M\times N \right) &= \bigoplus_{i + j = \ell} H^{i}_{\operatorname{DR}}\left( M \right)\otimes H^{j}_{\operatorname{DR}}\left( N \right).
  \end{align*}
\end{problem}
\begin{solution}
  For the sake of being able to solve this problem, we focus on the case where $M$ and $N$ are closed smooth manifolds.\newline

  Let $V = M\times N$ be the product manifold for $M$ and $N$. If $\pi_1\colon V\rightarrow M$ and $\pi_2\colon V\rightarrow N$ are the projection maps on $M$ and $N$ respectively, we get the composed maps
  \begin{align*}
    \mathcal{A}^{k}\left( M \right)\times \mathcal{A}^{\ell}\left( N \right)&\rightarrow \mathcal{A}^{k+\ell}\left( V \right)
  \end{align*}
  given by $\left( \omega,\eta \right) \mapsto \pi_1^{\ast}\omega\wedge \pi_2^{\ast}\eta$. If $\omega$ and $\eta$ are closed forms, then we observe that
  \begin{align*}
    d\left( \pi_1^{\ast}\omega\wedge\pi_2^{\ast}\eta \right) &= d\pi_1^{\ast}\omega\wedge\pi_2^{\ast}\eta + \left( -1 \right)^{k}\pi_1^{\ast}\omega\wedge d\pi_2^{\ast}\eta\\
                                                             &= \pi_1^{\ast}\left( d\omega \right)\wedge \pi_2^{\ast}\eta + \left( -1 \right)^{k} \pi_1^{\ast}\omega\wedge \pi_2^{\ast}\left( d\eta \right)\\
                                                             &= 0.
  \end{align*}
  Furthermore, if we let $\omega' = \omega + d\tau$ and $\eta' = \eta + d\rho$, then we know from earlier work that $\pi_1^{\ast}\omega'\wedge\pi_2^{\ast}\eta'$ can be expressed as $\pi_1^{\ast}\omega\wedge \pi_2^{\ast}\eta + d\sigma$ for some form $\sigma$ by using the fact that $d$ and the pullback commute. Thus, it follows that the map descends to a map in cohomology, given by
  \begin{align*}
    H^k_{\operatorname{DR}}\left( M \right)\times H^{\ell}_{\operatorname{DR}}\left( N \right) &\rightarrow H^{k+ \ell}\left( M\times N \right)\\
    \left( \left[ \omega \right],\left[ \eta \right] \right) &\mapsto \left[ \pi_1^{\ast}\omega\wedge\pi_2^{\ast}\eta \right],
  \end{align*}
  whence via the universal property of tensor products and direct sums, we get the map
  \begin{align*}
    \psi\colon H^{\ast}_{\operatorname{DR}}\left( M \right)\otimes H^{\ast}_{\operatorname{DR}}\left( N \right)\rightarrow H^{\ast}\left( M\times N \right).
  \end{align*}
  Our goal now is to show that $\psi$ is indeed an isomorphism.\newline

  Toward this end, suppose we have two open sets in the good cover for $M$, given by $U_1$ and $U_2$. From the Mayer--Vietoris sequence, this yields the following exact sequence in cohomology for a fixed $k$, where $D_k$ denote the connecting homomorphisms from $H^{k}\left( U_1\cap V_2 \right)$ to $H^{k+1}\left( M \right)$.
  \begin{center}
    % https://tikzcd.yichuanshen.de/#N4Igdg9gJgpgziAXAbVABwnAlgFyxMJZAJgBoAGAXVJADcBDAGwFcYkQAJAPQGsB9YAB1BENDABO9HBHFh6AWxjAAIgCUAvuoAUAWQCUIdaXSZc+QigAsFanSat23YD3UDhoiVJlzFKjdoBVPgBGPXc0FjgAAicXNxExSWlZBSU1TS0g4gMjE2w8AiIANhsaBhY2RE4uZ1chBM9knzT-TJDhAGN6NCisnOMQDHzzInJSuwr2TqgIHARcwdMCi2QADnHyhyrp2fnbGCgAc3giUAAzcQh5JABmGmkkMYmtkGUBHgBaYPVDAYurx73CBIYJleyVEBYX7nS7XRCgkAPRBkZ4QgBW0JA-zhKKR1lR7DetUMlHUQA
    \begin{tikzcd}
    \cdots \arrow[rr, "D_{k-1}"] &  & H^k_{\operatorname{DR}}(M) \arrow[rr, "i"] &  & H^{k}_{\operatorname{DR}}(U_1)\oplus H^{k}_{\operatorname{DR}}(U_2) \arrow[rr, "j"] &  & H^{k}_{\operatorname{DR}}(U_1\cap U_2) \arrow[rr, "D_{k}"] &  & \cdots
    \end{tikzcd}
  \end{center}
  Since the tensor product preserves exact sequences, we observe that by taking the tensor product with $H^{\ell}_{\operatorname{DR}}\left( N \right)$, giving the following.
  \begin{center}
    \scriptsize
% https://tikzcd.yichuanshen.de/#N4Igdg9gJgpgziAXAbVABwnAlgFyxMJZABgBpiBdUkANwEMAbAVxiRAB12BjKCHBAL6l0mXPkIoAjOSq1GLNgAkAesADWAgPrBOENDABOdHBANg6AWxjAAIgCUBAgBQBZAJS68VuAAIVO9hgGBi0AvUNjU3MrWwdnADk3ECERbDwCIgAmGWp6ZlZEEH8NbV19IxMzS2t7RycAVU1JD3Y+LG8-VU4gkNLW8siqmNqElr1mX2LQsojK6Jq4hs1Msa94ToCe6f7ZqOrYusTk4RAMNPEiAGYcuXylVQ0lyU4uOjQfRpXPdvX-buDRsdUmIMigACw3PIKQovXj8ZKyGBQADm8CIoAAZgYIBYkGQQCYkNJbtCQDZtGoALSSARAkBYnFE6iExDZEkFEBYOkM3Gs5kQJDXdlsABW3OxvKFLIhwsK5LUCIEQA
\begin{tikzcd}
\cdots \arrow[r, "D_{k-1}"] & H^{k}_{\operatorname{DR}}(M)\otimes H^{\ell}_{\operatorname{DR}}(N) \arrow[r, "i"] & H^{k}_{\operatorname{DR}}(U_1)\otimes H^{\ell}_{\operatorname{DR}}(N)\oplus H^{k}_{\operatorname{DR}}(U_2)\otimes H^{\ell}_{\operatorname{DR}}(N) \arrow[r, "j"] & H^{k}(U_1\cap U_2)\otimes H^{\ell}(N) \arrow[r, "D_k"] & \cdots
\end{tikzcd}
  \end{center}
\end{solution}
\end{document}
