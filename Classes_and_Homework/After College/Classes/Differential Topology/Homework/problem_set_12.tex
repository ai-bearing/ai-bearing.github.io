\documentclass[10pt]{mypackage}

% sans serif font:
%\usepackage{cmbright}
%\usepackage{sfmath}
%\usepackage{bbold} %better blackboard bold

\usepackage{homework}
%\usepackage{notes}
\usepackage{newpxtext,eulerpx,eucal}
\renewcommand*{\mathbb}[1]{\varmathbb{#1}}

\fancyhf{}
\fancyhead[R]{Avinash Iyer}
\fancyhead[L]{Differential Topology: Problem Set 12}
\fancyfoot[C]{\thepage}

\setcounter{secnumdepth}{0}

\begin{document}
\RaggedRight
\begin{problem}[Problem 1]
  In this exercise, we prove another fundamental result in differential topology, called the tubular neighborhood theorem. Let $M$ be a compact smooth manifold with orientable boundary $N$. For simplicity, assume that $N$ is connected. The tubular neighborhood theorem asserts that $N$ admits a neighborhood in $M$ which is diffeomorphic to $N\times [0,1)$.
  \begin{enumerate}[(a)]
    \item Choose a Riemannian metric on $M$, and show that $N$ admits a nonvanishing vector field that is everywhere orthogonal to the tangent space of $N$. That is, a vector field $X$ such that for all $p\in N$, $g\left( X_p,T_pN \right) = 0$.
    \item Use the flow generated by $X$ to find the desired neighborhood.
  \end{enumerate}
\end{problem}
\begin{solution}\hfill
  \begin{enumerate}[(a)]
    \item If $p\in N$, then we observe that $T_pN < T_pM$ is a proper subspace with codimension $1$. Letting $\set{e_1,\dots,e_{n-1}}$ be an orthonormal basis for $T_pN$, then we may extend to a basis for $T_pM$ by taking a representative for a basis for $T_pM/T_pN$, and observing that such a vector necessarily has
      \begin{align*}
        g_p\left( e_n,e_k \right) &= 0
      \end{align*}
      for all $k=1,\dots,n-1$. By smoothly varying over all points $p\in N$, we get our desired everywhere nonvanishing vector field normal to $T_pN$.
    \item Let $\varphi_t$ be the one-parameter diffeomorphism group generated by $X$, where $\varphi_t\colon M\rightarrow M$ is such that $\varphi_0(p) = p$ for all $p\in N$. Then, $\varphi\colon \left( -\ve,\ve \right)\rightarrow \operatorname{diff}\left( M \right)$ restricted to $\left[ 0,\ve \right)$ gives our desired neighborhood in $M$ diffeomorphic to $N\times \left[ 0,1 \right)$.
  \end{enumerate}
\end{solution}
\begin{problem}[Problem Set 7, Problem 5]
  Suppose $G$ is a finite group acting freely on a manifold $M$ by diffeomorphisms.
  \begin{enumerate}[(a)]
    \item Show that $M/G$ is a manifold.
    \item Show that the de Rham cohomology of $M/G$ is isomorphic to the $G$-invariant cohomology of $M$.
  \end{enumerate}
\end{problem}
\begin{solution}\hfill
  \begin{enumerate}[(a)]
    \item Let $p\in M$, and let $U$ be a neighborhood of $p$. By shrinking $U$ if necessary, the fact that $G$ acts freely on $M$ implies that for all $1\neq g$, we have $g\cdot U \cap U = \emptyset$. In particular, this gives that the projection map $q\colon M\rightarrow M/G$ is a covering map. Thus, to find a chart about $\left[ p \right]\in M/G$, we consider the image $ \overline{U} \coloneq q(U)\subseteq M/G$. If $\varphi$ is the coordinate map for $U$, define $ \overline{\varphi}\colon \overline{U}\rightarrow \R^{n} $ by taking $ \overline{\varphi}\left( q(U) \right) = \varphi\circ q^{-1}\left( \overline{U} \right) $.\newline

      Let $ [w]\in \overline{U} \cap \overline{V}$. We have some $w_1\in U$ and $w_2\in V$ such that $q\left(w_1\right) = q\left(w_2\right) = [w]$; in particular, there is $g\in G$ such that $g\cdot w_1 = w_2$. We may define $U' = U\cap g^{-1}\cdot V$ and $V' = V\cap \left( g\cdot U \right)$, where $ \overline{U'}\cap\overline{V'} \subseteq \overline{U}\cap \overline{V}$ and $[w]\in \overline{U'}\cap \overline{V'}$. Furthermore, we see that $q\left( U'\cap V' \right) = \overline{U'}\cap \overline{V'}$, as any element in the latter is given by $[x]$, where $x_1\in U'$ and $x_2\in V'$ have $q\left(x_i\right) = [x]$, meaning that the element $k\cdot x_1 = x_2$ is uniquely determined as the action is free.\newline

      We observe now that the transition map $ \overline{\psi}\circ \overline{\varphi}^{-1}\colon \overline{\varphi}\left( \overline{U'}\cap \overline{V'} \right) \rightarrow \overline{\psi}\left( \overline{U'}\cap \overline{V'} \right) $ is then given, by the definition of these maps, by the transition map between $\varphi \left( U\cap \left( g^{-1}\cdot V \right) \right)$ to $\psi \left( \left( g\cdot U \right)\cap V \right)$. Therefore, $M/G$ is a manifold.
    \item Consider a closed form $\omega\in \mathcal{A}^{\ast}\left( M/G \right)$. The pullback $q^{\ast}\omega\in \mathcal{A}^{\ast}\left( M \right)$ is necessarily $G$-invariant. This descends to a map in cohomology 
      \begin{align*}
        q^{\ast}\colon H^{\ast}_{\operatorname{DR}}\left( M/G \right) \rightarrow H^{\ast}_{\operatorname{DR}}\left( M \right)^{G}.
      \end{align*}
      Our goal is to show that this map is injective and surjective. First, let $\left[ \omega \right]\in \ker\left( q^{\ast} \right)$. Then, $\omega$ is exact, meaning that $q^{\ast}\omega$ is exact, so there is some $\eta$ such that $q^{\ast}\omega = d\eta$. If we let
      \begin{align*}
        \xi &= \frac{1}{\left\vert G \right\vert}\sum_{g\in \left\vert G \right\vert} g^{\ast}\eta,
      \end{align*}
      then we see that $\xi$ is $G$-invariant and has $d\xi = \eta$. By the commutativity of pullback and $d$, there is then some $ \overline{\xi}\in H^{\ast}_{\operatorname{DR}}\left( M/G \right) $ such that $q^{\ast} \overline{\xi} = \xi$, meaning that $d\xi$ and $\omega$ are in the same cohomology class. Thus, $\ker\left( q^{\ast} \right) = \set{0}$.

      Now, to see surjectivity, let $\left[ \omega \right]\in H^{\ast}_{\operatorname{DR}}\left( M \right)$ be $G$-invariant. By using the same averaging process, we have a representative of the cohomology class that can be found by pullback of a closed form in $H^{\ast}_{\operatorname{DR}}\left( M/G \right)$, so that $H^{\ast}_{\operatorname{DR}}\left( M/G \right)\cong H^{\ast}_{\operatorname{DR}}\left( M \right)^{G}$.
  \end{enumerate}
\end{solution}
\begin{problem}[Problem Set 8, Problem 3]
  Compute the de Rham cohomology of $\R\mathbb{P}^{n}$.
\end{problem}
\begin{solution}
  We observe that the antipodal map, $x\mapsto -x$, is a finite free action on the manifold $S^{n}$, and is such that the orbit space is $ \mathbb{RP}^{n} $, given by $\Z/2\Z$.\newline

  We know that the cohomology for $S^{n}$ is given by $\R$ at $H^{0}$, and $\R$ at $H^{n}$, with $0$ everywhere else. The antipodal map is of degree $1$ if and only if $n$ is odd, which means that the antipodal map is thus a sign-preserving local diffeomorphism. In particular, this means that for an $n$-form $\omega\in \mathcal{A}^{n}\left( \mathbb{RP}^{n} \right)$, we have
  \begin{align*}
    \int_{ S^{n} }^{} q^{\ast}\omega &= \int_{ \mathbb{RP}^{n} }^{} \omega,
  \end{align*}
  so that the simplicial cohomology is identical. Thus, if $n$ is odd, then $ \mathbb{RP}^{n} $ and $S^{n}$ have the same cohomology.\newline

  Now, if $n$ is even, we know that the degree of the antipodal map is $-1$. Yet, this means that there are no invariant closed $n$-forms, meaning that the top-dimensional cohomology of $ \mathbb{RP}^{n} $ is $0$.
\end{solution}
\begin{problem}[Problem Set 8, Problem 5]
  Use the Mayer--Vietoris sequence to prove the Künneth Formula: if $M$ and $N$ are smooth manifolds, then $H^{\ast}_{\operatorname{DR}}\left( M\times N \right)$ is the tensor product of $H^{\ast}_{\operatorname{DR}}\left( M \right)$ and $H^{\ast}_{\operatorname{DR}}\left( N \right)$. Specifically, in each degree $\ell$, we have
  \begin{align*}
    H^{\ell}_{\operatorname{DR}}\left( M\times N \right) &= \bigoplus_{i + j = \ell} H^{i}_{\operatorname{DR}}\left( M \right)\otimes H^{j}_{\operatorname{DR}}\left( N \right).
  \end{align*}
\end{problem}
\begin{solution}
  For the sake of being able to solve this problem, we focus on the case where $M$ and $N$ are closed smooth manifolds.\newline

  Let $V = M\times N$ be the product manifold for $M$ and $N$. If $\pi_1\colon V\rightarrow M$ and $\pi_2\colon V\rightarrow N$ are the projection maps on $M$ and $N$ respectively, we get the composed maps
  \begin{align*}
    \mathcal{A}^{k}\left( M \right)\times \mathcal{A}^{\ell}\left( N \right)&\rightarrow \mathcal{A}^{k+\ell}\left( V \right)
  \end{align*}
  given by $\left( \omega,\eta \right) \mapsto \pi_1^{\ast}\omega\wedge \pi_2^{\ast}\eta$. If $\omega$ and $\eta$ are closed forms, then we observe that
  \begin{align*}
    d\left( \pi_1^{\ast}\omega\wedge\pi_2^{\ast}\eta \right) &= d\pi_1^{\ast}\omega\wedge\pi_2^{\ast}\eta + \left( -1 \right)^{k}\pi_1^{\ast}\omega\wedge d\pi_2^{\ast}\eta\\
                                                             &= \pi_1^{\ast}\left( d\omega \right)\wedge \pi_2^{\ast}\eta + \left( -1 \right)^{k} \pi_1^{\ast}\omega\wedge \pi_2^{\ast}\left( d\eta \right)\\
                                                             &= 0.
  \end{align*}
  Furthermore, if we let $\omega' = \omega + d\tau$ and $\eta' = \eta + d\rho$, then we know from earlier work that $\pi_1^{\ast}\omega'\wedge\pi_2^{\ast}\eta'$ can be expressed as $\pi_1^{\ast}\omega\wedge \pi_2^{\ast}\eta + d\sigma$ for some form $\sigma$ by using the fact that $d$ and the pullback commute. Thus, it follows that the map descends to a map in cohomology, given by
  \begin{align*}
    H^k_{\operatorname{DR}}\left( M \right)\times H^{\ell}_{\operatorname{DR}}\left( N \right) &\rightarrow H^{k+ \ell}\left( M\times N \right)\\
    \left( \left[ \omega \right],\left[ \eta \right] \right) &\mapsto \left[ \pi_1^{\ast}\omega\wedge\pi_2^{\ast}\eta \right],
  \end{align*}
  whence via the universal property of tensor products and direct sums, we get the map
  \begin{align*}
    \psi\colon H^{\ast}_{\operatorname{DR}}\left( M \right)\otimes H^{\ast}_{\operatorname{DR}}\left( N \right)\rightarrow H^{\ast}\left( M\times N \right).
  \end{align*}
  Our goal now is to show that $\psi$ is indeed an isomorphism.\newline

  Toward this end, suppose we have two open sets in the good cover for $M$, given by $U_1$ and $U_2$. From the Mayer--Vietoris sequence, this yields the following exact sequence in cohomology for a fixed $k$, where $D_k$ denote the connecting homomorphisms from $H^{k}\left( U_1\cap V_2 \right)$ to $H^{k+1}\left( M \right)$.
  \begin{center}
    % https://tikzcd.yichuanshen.de/#N4Igdg9gJgpgziAXAbVABwnAlgFyxMJZAJgBoAGAXVJADcBDAGwFcYkQAJAPQGsB9YAB1BENDABO9HBHFh6AWxjAAIgCUAvuoAUAWQCUIdaXSZc+QigAsFanSat23YD3UDhoiVJlzFKjdoBVPgBGPXc0FjgAAicXNxExSWlZBSU1TS0g4gMjE2w8AiIANhsaBhY2RE4uZ1chBM9knzT-TJDhAGN6NCisnOMQDHzzInJSuwr2TqgIHARcwdMCi2QADnHyhyrp2fnbGCgAc3giUAAzcQh5JABmGmkkMYmtkGUBHgBaYPVDAYurx73CBIYJleyVEBYX7nS7XRCgkAPRBkZ4QgBW0JA-zhKKR1lR7DetUMlHUQA
    \begin{tikzcd}
    \cdots \arrow[rr, "D_{k-1}"] &  & H^k_{\operatorname{DR}}(M) \arrow[rr, "i"] &  & H^{k}_{\operatorname{DR}}(U_1)\oplus H^{k}_{\operatorname{DR}}(U_2) \arrow[rr, "j"] &  & H^{k}_{\operatorname{DR}}(U_1\cap U_2) \arrow[rr, "D_{k}"] &  & \cdots
    \end{tikzcd}
  \end{center}
  Since the tensor product preserves exact sequences, we observe that by taking the tensor product with $H^{\ell}_{\operatorname{DR}}\left( N \right)$, giving the following.
  \begin{center}
    \scriptsize
% https://tikzcd.yichuanshen.de/#N4Igdg9gJgpgziAXAbVABwnAlgFyxMJZABgBpiBdUkANwEMAbAVxiRAB12BjKCHBAL6l0mXPkIoAjOSq1GLNgAkAesADWAgPrBOENDABOdHBANg6AWxjAAIgCUBAgBQBZAJS68VuAAIVO9hgGBi0AvUNjU3MrWwdnADk3ECERbDwCIgAmGWp6ZlZEEH8NbV19IxMzS2t7RycAVU1JD3Y+LG8-VU4gkNLW8siqmNqElr1mX2LQsojK6Jq4hs1Msa94ToCe6f7ZqOrYusTk4RAMNPEiAGYcuXylVQ0lyU4uOjQfRpXPdvX-buDRsdUmIMigACw3PIKQovXj8ZKyGBQADm8CIoAAZgYIBYkGQQCYkNJbtCQDZtGoALSSARAkBYnFE6iExDZEkFEBYOkM3Gs5kQJDXdlsABW3OxvKFLIhwsK5LUCIEQA
\begin{tikzcd}
\cdots \arrow[r, "D_{k-1}"] & H^{k}_{\operatorname{DR}}(M)\otimes H^{\ell}_{\operatorname{DR}}(N) \arrow[r, "i"] & H^{k}_{\operatorname{DR}}(U_1)\otimes H^{\ell}_{\operatorname{DR}}(N)\oplus H^{k}_{\operatorname{DR}}(U_2)\otimes H^{\ell}_{\operatorname{DR}}(N) \arrow[r, "j"] & H^{k}(U_1\cap U_2)\otimes H^{\ell}(N) \arrow[r, "D_k"] & \cdots
\end{tikzcd}
  \end{center}
  Taking direct sums with the same dimension, we obtain the following diagram.
  \begin{center}
% https://tikzcd.yichuanshen.de/#N4Igdg9gJgpgziAXAbVABwnAlgFyxMJZABgBoBGAXVJADcBDAGwFcYkQAdDgIywHMIaFnAD6wANYACANSSuMRo0kBeSVgC05AL6SAEgD0JWgBQBVEeS4BjemknmATAEouEPAFt4ew-MUmAck4gWqTomLj4hCjkFNR0TKzsBsAa2sZmFta29iLOXB5egcGhIBjYeAREZA5xDCxsiJw8-ILCYlKyvkqqWFrJ4iYAsi4cbliecN7AXQFBIWHlkUQxNTR1iY3JvcaD+eOFcyVlEZUoZADMtQkNTbwCQsyiEjJyHArdaib9JubkI2MTKYzYyBVwPSbfDJ5UYFCE+N5+EFOQ4LE5RZAxS5ra5JQzbX57QGg0bgqb43KEg7FVEVdFkAAsV3q7C4d1aj3aLy6Kk+kIJHBsdkc-1hQIRjFm1NK4Vpy1IjOxzM2eJMGUsAuywspkyK82li1OJFIAFYmRsmrQoG4EHrjrLoiazTcuJbrVK7UszqRiE6WRxXTgbUcZZ6Md7fY0XVbA8E4jAoHx4ERQAAzABOEHcSDIIBwECQMXiSqaaGwUvTmaQDhoeaQ50V5q4paw5YzWcQ9Jr+cQxobzo4zdblcQADYu0gAOx9v2DvUV9s52uIasgRj0bgKAAKIdOIDT-AAFjgQNPGgARIftldLztF80tmhrjeMbcGqJ7w-HudtpC3pdj1d1y3Hd333PgjxPO8bgAK0vJAAKXAAOR8gJfED2DAiDTxAC9v2HQsl3rQDn1fNF2EYGAU2PbDcJKec63HHtsJbPD217XNuynKD2Fg1jJ0YgBOGi4MQchF27Yg+NEgju20SgtCAA
\begin{tikzcd}
\vdots \arrow[d]                                                                                                              & \vdots \arrow[d]                                           \\
\bigoplus_{k + \ell = i-1} H^{k}(U_1\cap U_2)\otimes H^{\ell}(N) \arrow[r, "\psi"] \arrow[d, "D_{i-1}"']                            & H^{i-1}((U_1\cap U_2)\times N) \arrow[d, "D_{i-1}"]              \\
\bigoplus_{k + \ell = i}H^{k}(M)\otimes H^{\ell}(N) \arrow[r, "\psi"] \arrow[d, "i"']                                         & H^{i}(M\times N) \arrow[d, "i"]                            \\
\bigoplus_{k + \ell = i}(H^{k}(U_1)\otimes H^{\ell}(N)\oplus H^{k}(U_2)\otimes H^{\ell}(N)) \arrow[r, "\psi"] \arrow[d, "j"'] & H^{i}(U_1\times N)\oplus H^{i}(U_2\times N) \arrow[d, "j"] \\
\bigoplus_{k + \ell = i}H^{k}(U_1\cap U_2)\otimes H^{\ell}(N) \arrow[r, "\psi"] \arrow[d, "D_i"']                               & H^{i}((U_1\cap U_2)\times N) \arrow[d, "D_i"]                \\
\vdots                                                                                                                        & \vdots                                                    
\end{tikzcd}
  \end{center}
  Since $U_1$, $U_2$, and $U_1\cap U_2$ are contractible, under the good cover assumption, it follows from the Poincaré Lemma that the following subsection of the diagram is commutative, with $\psi$ necessarily an isomorphism in each of the columns.
  \begin{center}
    % https://tikzcd.yichuanshen.de/#N4Igdg9gJgpgziAXAbVABwnAlgFyxMJZABgBpiBdUkANwEMAbAVxiRAB12AjLAcwjTM4AfWABrAAQBqCZxgMGEgLwSsAXwAUACQB64zQFVhARgCUnCHgC28CbuByFmgHLn2AoXb1jDwgExullg2cF4O7PIMLqamIGqk6Ji4+IQoZMZUtIwsbJw8-IJMIuLSshEKyqpq9j4aRsacAMZ0aBJGARbWtvaOURqucQkgGNh4BETGpBnU9MysiCD26hp1Jk0tbf5uXaED8YmjKRPkmbM5C0u+Dew7Eq4WhaGXq36ct3uZMFC88ESgAGYAJwgViQZBAOAgSAAzDNsvMOOw0NhBgDgaDEJMIVDEH44XNckiUfsQECQWDqJCkFiGHQuPIAApJMapECAvgACxwIHx5xAACtUaT0TDKTi8VkCQtBWoKGogA
\begin{tikzcd}
\bigoplus_{k + \ell = i}(H^{k}(U_1)\otimes H^{\ell}(N)\oplus H^{k}(U_2)\otimes H^{\ell}(N)) \arrow[r, "\psi"] \arrow[d, "j"'] & H^{i}(U_1\times N)\oplus H^{i}(U_2\times N) \arrow[d, "j"] \\
\bigoplus_{k + \ell = i}H^{k}(U_1\cap U_2)\otimes H^{\ell}(N) \arrow[r, "\psi"]                                               & H^{i}((U_1\cap U_2)\times N)                              
\end{tikzcd}
  \end{center}
  Similarly, we have that the following diagram is commutative, following from the Mayer--Vietoris sequence.
  \begin{center}
    % https://tikzcd.yichuanshen.de/#N4Igdg9gJgpgziAXAbVABwnAlgFyxMJZABgBoBGAXVJADcBDAGwFcYkQAdDgIywHMIaFnAD6wANYACANSSuMRo0kBeSVgC+ACgASAPQlaAqiPIBKLhDwBbeJL3B5irQDlzHQcLv7xRkQCY3SywbOC8HDgVGF1NTEHVSdExcfEIUcgpqOiZWdnsNTWNyLmtbVwshZlC83z9i4NLY+MTsPAIiMmJMhhY2RE4efg9KsSlZRyVVDXsfTQBZQJKq-XHouISQDBaUonTOmm6cvuq5upDJVzjMmCg+eCJQADMAJwgrJDIQHAgkdKye9i4aGwa0eLzeiD8NC+SAAzPtsr1+kCsCCQM9XkhIZ9vogPox6NwFAAFJKtVIgJ78AAWOBA8P+fRRTTRYNhUJxvwOiIAVpd1EA
\begin{tikzcd}
\bigoplus_{k + \ell = i}H^{k}(M)\otimes H^{\ell}(N) \arrow[r, "\psi"] \arrow[d, "i"']                         & H^{i}(M\times N) \arrow[d, "j"]             \\
\bigoplus_{k + \ell = i}(H^{k}(U_1)\otimes H^{\ell}(N)\oplus H^{k}(U_2)\otimes H^{\ell}(N)) \arrow[r, "\psi"] & H^{i}(U_1\times N)\oplus H^{i}(U_2\times N)
\end{tikzcd}
  \end{center}
  Therefore, we only need to verify commutativity for the following square.
  \begin{center}
    % https://tikzcd.yichuanshen.de/#N4Igdg9gJgpgziAXAbVABwnAlgFyxMJZABgBpiBdUkANwEMAbAVxiRAB12AjLAcwjTM4AfWABrAAQBqCZxgMGEgLwSsAWgCMAXwkAJAHritACgCqwjZwDGdNBPMAmAJScIeALbw9huQpMA5JxAtUnRMXHxCFA1yKlpGFjYDYHVtYzMLa1t7YWdODy9A4NCQDGw8AiIyDTj6ZlZEDm4+ASFRSRlfRRUsLWSxEwBZF3Y3LE84b2AugKCQsPLIohia6jrExuTe40H88cK5uJgoXngiUAAzACcIdyQyEBwIJBj4+rZONGxiy5u7xAc1CeSAAzGsEg0ml8sD8QNdbvcgc8AdQGHQuPIAArhCpREBXPgACxwIHB70aABFRKktLD4f9XsDEGCQGiMQxsYtKo0GDALiSyRsQFSUppaVoKFogA
    \begin{tikzcd}
    \bigoplus_{k + \ell = i-1} H^{k}(U_1\cap U_2)\otimes H^{\ell}(N) \arrow[r, "\psi"] \arrow[d, "D_{i-1}"'] & H^{i-1}((U_1\cap U_2)\times N) \arrow[d, "D_{i-1}"] \\
    \bigoplus_{k + \ell = i}H^{k}(M)\otimes H^{\ell}(N) \arrow[r, "\psi"]                                    & H^{i}(M\times N)                                   
    \end{tikzcd}
  \end{center}
  First, from the Mayer--Vietoris sequence and the fact that the coboundary map in de Rham cohomology emerges from the exterior derivative, we have that the map $D_i$ is given by
  \begin{align*}
    D_{i-1}\left( \left[ \omega \right] \right) &= \begin{cases}
      \left[ d\left( -f_U\omega \right) \right]\\
      \left[ d\left( f_V\omega \right) \right]
    \end{cases}
  \end{align*}
  for any cohomology class representative $\omega$. Now, we observe that
  \begin{align*}
    \psi\left( D_{i-1}\left( \left[ \omega \right],\left[ \eta \right] \right) \right) &= \left[ \pi_1^{\ast}\left( D_{i-1}\left( \omega \right) \right)\wedge \pi_2^{\ast}\eta \right]\\
    D_{i-1}\left( \psi\left( \left[ \omega \right],\left[ \eta \right] \right) \right) &= \left[ D_{i-1}\left( \pi_1^{\ast}\omega\wedge\pi_2^{\ast}\eta \right) \right]
  \end{align*}
  In particular, since $\pi_1^{\ast}f_U$ and $\pi_1^{\ast}f_V$ form a partition of unity for $M\times F$, we have
  \begin{align*}
    \pi_1^{\ast}\left( D_{i-1}\left( \omega \right) \right)\wedge \pi_2^{\ast}\eta &= \pi_1^{\ast}\left( d\left( f_V\omega \right) \right)\wedge \pi_2^{\ast}\eta\\
                                                                                   &= d\left( \pi_1^{\ast}\left( f_V\omega \right) \right)\wedge\pi_2^{\ast}\eta\\
    D_{i-1}\left( \pi_1^{\ast}\omega\wedge\pi_2^{\ast}\eta \right) &= d\left( \pi_1^{\ast}f_{V}\pi_1^{\ast}\omega\wedge\pi_2^{\ast}\eta \right)\\
                                                                   &= d\left( \pi^{\ast}\left( f_V\omega \right) \right)\wedge \pi_2^{\ast}\eta.
  \end{align*}
  Since $\psi$ at each of $U$, $V$, and $U\cap V$ is an isomorphism, and the diagram commutes, the Five Lemma gives that $\psi$ at $M$ is an isomorphism.\newline

  For any finite good cover with more than $2$ elements, induction gives the desired result.
\end{solution}
%\begin{problem}[Problem Set 10, Problem 7]
%  Show that the invariant cohomology $H^{\ast}_{L}(G)$ is isomorphic to the de Rham cohomology of $G$. Conclude that the de Rham cohomology of an $n$-torus is isomorphic to the exterior algebra on $\R^{n}$.
%\end{problem}
\end{document}
