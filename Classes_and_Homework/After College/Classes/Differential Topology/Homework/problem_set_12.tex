\documentclass[10pt]{mypackage}

% sans serif font:
%\usepackage{cmbright}
%\usepackage{sfmath}
%\usepackage{bbold} %better blackboard bold

\usepackage{homework}
%\usepackage{notes}
\usepackage{newpxtext,eulerpx,eucal}
\renewcommand*{\mathbb}[1]{\varmathbb{#1}}

\fancyhf{}
\fancyhead[R]{Avinash Iyer}
\fancyhead[L]{Differential Topology: Problem Set 12}
\fancyfoot[C]{\thepage}

\setcounter{secnumdepth}{0}

\begin{document}
\RaggedRight
\begin{problem}[Problem 1]
  In this exercise, we prove another fundamental result in differential topology, called the tubular neighborhood theorem. Let $M$ be a compact smooth manifold with orientable boundary $N$. For simplicity, assume that $N$ is connected. The tubular neighborhood theorem asserts that $N$ admits a neighborhood in $M$ which is diffeomorphic to $N\times [0,1)$.
  \begin{enumerate}[(a)]
    \item Choose a Riemannian metric on $M$, and show that $N$ admits a nonvanishing vector field that is everywhere orthogonal to the tangent space of $N$. That is, a vector field $X$ such that for all $p\in N$, $g\left( X_p,T_pN \right) = 0$.
    \item Use the flow generated by $X$ to find the desired neighborhood.
  \end{enumerate}
\end{problem}
\begin{problem}[Problem Set 7, Problem 5]
  Suppose $G$ is a finite group acting freely on a manifold $M$ by diffeomorphisms.
  \begin{enumerate}[(a)]
    \item Show that $M/G$ is a manifold.
    \item Show that the de Rham cohomology of $M/G$ is isomorphic to the $G$-invariant cohomology of $M$.
  \end{enumerate}
\end{problem}
\begin{problem}[Problem Set 8, Problem 3]
  Compute the de Rham cohomology of $\R\mathbb{P}^{n}$.
\end{problem}

\end{document}
