\documentclass[10pt]{mypackage}

% sans serif font:
%\usepackage{cmbright}
%\usepackage{sfmath}
%\usepackage{bbold} %better blackboard bold

\usepackage{homework}
%\usepackage{notes}
\usepackage{newpxtext,eulerpx,eucal}
\renewcommand*{\mathbb}[1]{\varmathbb{#1}}

\fancyhf{}
\fancyhead[R]{Avinash Iyer}
\fancyhead[L]{Differential Topology: Problem Set 12}
\fancyfoot[C]{\thepage}

\setcounter{secnumdepth}{0}

\begin{document}
\RaggedRight
\begin{problem}[Problem 1]
  In this exercise, we prove another fundamental result in differential topology, called the tubular neighborhood theorem. Let $M$ be a compact smooth manifold with orientable boundary $N$. For simplicity, assume that $N$ is connected. The tubular neighborhood theorem asserts that $N$ admits a neighborhood in $M$ which is diffeomorphic to $N\times [0,1)$.
  \begin{enumerate}[(a)]
    \item Choose a Riemannian metric on $M$, and show that $N$ admits a nonvanishing vector field that is everywhere orthogonal to the tangent space of $N$. That is, a vector field $X$ such that for all $p\in N$, $g\left( X_p,T_pN \right) = 0$.
    \item Use the flow generated by $X$ to find the desired neighborhood.
  \end{enumerate}
\end{problem}
\begin{solution}\hfill
  \begin{enumerate}[(a)]
    \item If $p\in N$, then we observe that $T_pN < T_pM$ is a proper subspace with codimension $1$. Letting $\set{e_1,\dots,e_{n-1}}$ be an orthonormal basis for $T_pN$, then we may extend to a basis for $T_pM$ by taking a representative for a basis for $T_pM/T_pN$, and observing that such a vector necessarily has
      \begin{align*}
        g_p\left( e_n,e_k \right) &= 0
      \end{align*}
      for all $k=1,\dots,n-1$. By smoothly varying over all points $p\in N$, we get our desired everywhere nonvanishing vector field normal to $T_pN$.
    \item Let $\varphi_t$ be the one-parameter diffeomorphism group generated by $X$, where $\varphi_t\colon M\rightarrow M$ is such that $\varphi_0(p) = p$ for all $p\in N$. Then, $\varphi\colon \left( -\ve,\ve \right)\rightarrow \operatorname{diff}\left( M \right)$ restricted to $\left[ 0,\ve \right)$ gives our desired neighborhood in $M$ diffeomorphic to $N\times \left[ 0,1 \right)$.
  \end{enumerate}
\end{solution}
\begin{problem}[Problem Set 7, Problem 5]
  Suppose $G$ is a finite group acting freely on a manifold $M$ by diffeomorphisms.
  \begin{enumerate}[(a)]
    \item Show that $M/G$ is a manifold.
    \item Show that the de Rham cohomology of $M/G$ is isomorphic to the $G$-invariant cohomology of $M$.
  \end{enumerate}
\end{problem}
\begin{problem}[Problem Set 8, Problem 3]
  Compute the de Rham cohomology of $\R\mathbb{P}^{n}$.
\end{problem}
\begin{solution}
  We will use the result related to invariant cohomology to compute this.
\end{solution}
\begin{problem}[Problem Set 8, Problem 5]
  Use the Mayer--Vietoris sequence to prove the Künneth Formula: if $M$ and $N$ are smooth manifolds, then $H^{\ast}_{\operatorname{DR}}\left( M\times N \right)$ is the tensor product of $H^{\ast}_{\operatorname{DR}}\left( M \right)$ and $H^{\ast}_{\operatorname{DR}}\left( N \right)$. Specifically, in each degree $\ell$, we have
  \begin{align*}
    H^{\ell}_{\operatorname{DR}}\left( M\times N \right) &= \bigoplus_{i + j = \ell} H^{i}_{\operatorname{DR}}\left( M \right)\otimes H^{j}_{\operatorname{DR}}\left( N \right).
  \end{align*}
\end{problem}
\end{document}
