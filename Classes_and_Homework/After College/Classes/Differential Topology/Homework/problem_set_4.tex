\documentclass[10pt]{mypackage}

% sans serif font:
%\usepackage{cmbright}
%\usepackage{sfmath}
%\usepackage{bbold} %better blackboard bold

\usepackage{homework}
%\usepackage{notes}
\usepackage{newpxtext,eulerpx,eucal}
\renewcommand*{\mathbb}[1]{\varmathbb{#1}}

\fancyhf{}
\rhead{Avinash Iyer}
\lhead{Differential Topology: Problem Set 4}

\setcounter{secnumdepth}{0}

\begin{document}
\RaggedRight
\begin{problem}[Problem 1]
  Prove that if $f\colon M\rightarrow N$ is smooth, and $L$ is a $k$-codimensional submanifold of $N$ that is transverse to $f$, then $f^{-1}\left( L \right)$ is either empty or a submanifold of $M$ with codimension $k$.
\end{problem}
\begin{solution}
  If $L$ is not contained in $f(M)$, then $f^{-1}\left( L \right)$ is clearly empty. Therefore, we focus on the case where $f^{-1}\left( L \right)$ is not empty.\newline

  Let $L$ be transverse to $f$, $q\in L$, and $p\in M$ such that $f(p) = q$. We observe that $T_qL + D_pF\left( T_pM \right) = T_qN$, so any vector in $T_qN$ can be written (not necessarily uniquely) as an element of $D_pF\left( T_pM \right)$ and $T_qL$. Next, we observe that, if we take a coordinate chart for $q$ in $U$ such that $\varphi(U)\cong \R^{k}$, then by the Regular Value Theorem, we may select $\varphi$ such that $L\cap U = \varphi^{-1}\left( 0 \right)$. This follows from the assumption that $L$ has codimension $k$.\newline

  Now, if we can show that $0$ is a regular value for $\varphi\circ f$, then $\left( \varphi\circ f \right)^{-1}\left( 0 \right) = f^{-1}\left( L \right)\cap f^{-1}\left( U \right)$, meaning that $f^{-1}\left( L \right)$ is a submanifold of $M$ with codimension $k$. First, since $0$ is a regular value for $\varphi$, it follows that if $v\in T_0\R^{k}$, then there is some $w\in T_{q}N$ such that $D_{q}\varphi\left( w \right) = v$. Since $f$ is transverse to $L$, there is $w_1\in T_qL$ and $w_2\in T_pM$ such that $w = w_1 + D_{p}F\left( w_2 \right)$. We observe that, since $\varphi$ is constant on $L$, we have $D_q\varphi\left( w_1 \right) = 0$, so that
  \begin{align*}
    D_p\left( \varphi\circ f \right)\left( w_2 \right) &= D_q\varphi\circ D_pF\left( w_2 \right)\\
                                                       &= D_q\varphi\left( w_1 + D_pF\left( w_2 \right) \right)\\
                                                       &= D_q\varphi\left( w \right)\\
                                                       &= v,
  \end{align*}
  so $0$ is a regular value for $\varphi\circ F$.
\end{solution}
\begin{problem}[Problem 2]
  Let $ \operatorname{GL}_n\left( \R \right) $ denote the space of invertible $n\times n$ matrices over $\R$, let $\operatorname{SL}_n\left( \R \right)$ denote the matrices of determinant one, and let $ \operatorname{O}\left( n \right) $ be the orthogonal group.
  \begin{enumerate}[(a)]
    \item Prove that we may identify the tangent space of $ \operatorname{GL}_n\left( \R \right) $ at the identity with $n\times n$ matrices over $\R$.
    \item Prove that the tangent space of $ \operatorname{SL}_n\left( \R \right) $ at the identity consists of matrices of trace zero.
    \item Prove that the tangent space of $ \operatorname{O}(n) $ at the identity consists of skew-symmetric matrices. What is the dimension of $ \operatorname{O}(n) $?
    \item Show that $ \operatorname{SL}_n\left( \R \right) $ and $ \operatorname{O}(n) $ do not intersect transversely at the identity.
  \end{enumerate}
\end{problem}

\begin{solution}\hfill
  \begin{enumerate}[(a)]
    \item Let $A\in \Mat_{n}\left( \R \right)$, and consider a path through the identity given by $\gamma(t) = I + tA$. Since the determinant is a smooth function, and $\det\left( I \right) = 1$, we have that for a small $\ve > 0$ there is $\delta$, such that $ \left\vert \det\left( I + tA \right) - 1 \right\vert < \ve$ whenever $\left\vert t \right\vert < \delta$. In particular, this means that the tangent space at the identity of $ \operatorname{GL}_n\left( \R \right) $ consists of all matrices.
    \item We let $\gamma(t) = I + tA$ be a curve in $ \operatorname{SL}_n\left( \R \right) $, so that $ \gamma'(0) = A $ is an element of the tangent space of $ \operatorname{SL}_n\left( \R \right) $ at the identity. We observe that $\det\left( \gamma(t) \right) = 1$ for all (sufficiently small) $t$, so by chain rule, we find that
      \begin{align*}
        0 &= \diff{}{t}\biggr\vert_{t = 0} \det\left( \gamma(t) \right)\\
          &= D_{\gamma(0)}\det\left( \gamma'(0) \right)\\
          &= D_{I}\det\left( A \right).
      \end{align*}
      Therefore, we must evaluate what $\det'\left( I \right)\left( A \right)$ yields. Toward this end, we see that
      \begin{align*}
        D_I\det\left( A \right) &= \lim_{t\rightarrow 0} \frac{\det\left( I - tA \right) - 1}{t}\\
                                &= \lim_{t\rightarrow 0} \frac{t^{n}\det\left( \frac{1}{t}I - A \right)-1}{t}.
      \end{align*}
      Observe that the expression $ \det\left( \frac{1}{t}I - A \right) $ is the characteristic polynomial of $A$ in $\frac{1}{t}$. This means that the $\left( \frac{1}{t} \right)^{n-1}$ term is equal to $ \operatorname{tr}\left( A \right) $, so that $D_I\det\left( A \right) = \tr\left( A \right)$. Thus, we find that $A$ is traceless.
    \item If $ \gamma\left( t \right) = I + tA $ is a curve in $ \operatorname{O}(n) $, then then we have that
      \begin{align*}
        \left( I + tA \right)^{T}\left( I + tA \right) &= I\\
        I + t\left( A^{T} + A \right) + t^2\left( A^{T}A \right) &= I,
      \end{align*}
      meaning that by taking an equivalence class of this tangent curve, we have
      \begin{align*}
        I + t\left( A^{T} + A \right) &= I,
      \end{align*}
      so that $A^{T} = -A$.\newline

      We observe that the function $f\colon \Mat_{n}\left( \R \right)\rightarrow \Mat_{n}\left( \R \right)_{\operatorname{s.a.}}$, given by
      \begin{align*}
        f\left( A \right) &= A^{T}A,
      \end{align*}
      has $I_{n}$ as a regular value. To see this, observe that curves in $T_I\Mat_{n}\left( \R \right)_{\operatorname{s.a.}}$ are of the form $\gamma(t) = I + tK$, where $K$ is a self-adjoint(/symmetric) matrix. Similarly, $T_A\Mat_{n}\left( \R \right)$ is of the form $ \ve(t) = A + tB $, where $B\in \Mat_{n}\left( \R \right)$ and $t\in \R$. Both of these follow from the fact that $\Mat_{n}\left( \R \right)$ and $\Mat_{n}\left( \R \right)_{\operatorname{s.a.}}$ are isomorphic to Euclidean spaces. Therefore, we see that the image of $\delta(t)$ is of the form $ A^{T}A + t\left( A^{T}B + B^{T}A \right) $; if $A$ satisfies $A^{T}A = I$, we can put this in the form of $ I + tK $ by taking $\delta(t) = A + \frac{1}{2}tAK$. Therefore, by the Regular Value Theorem, the dimension of $ \operatorname{O}(n) $ is $n^2 - \frac{n\left( n-1 \right)}{2} = \frac{n\left( n+1 \right)}{2}$
    \item Since both skew-symmetric and traceless matrices have trace zero, it follows that the tangent spaces of $\operatorname{SL}_n\left( \R \right)$ and $ \operatorname{O}\left( n \right) $ cannot span the tangent space of $ \operatorname{GL}_n\left( \R \right) $, as there are matrices with nonzero trace.
  \end{enumerate}
\end{solution}
\begin{problem}[Problem 4]
  Let $D$ be a distribution on a smooth manifold of dimension $n$. We write $I(D)$ for the ideal of $D$, which consists of graded pieces $I^{k}\left( D \right)\subseteq \mathcal{A}^{k}\left( M \right)$, where $I^{k}\left( D \right)$ consists of forms $\omega$ such that $\omega\left( X_1,\dots,X_k \right) = 0$ for all $X_i\in D$, and
  \begin{align*}
    I(D) &= \bigoplus_{k=0}^{n} I^{k}\left( D \right).
  \end{align*}
  The Frobenius Theorem says that $D$ is involutive if and only if $I$ is \textit{differential} --- i.e., $d(I)\subseteq I$, where $d$ is the exterior derivative.
  \begin{enumerate}[(a)]
    \item Prove that $I(D)$ is an ideal --- i.e., if $\omega\in I(D)$ and $\eta$ is arbitrary, then $\omega \wedge \eta \in I(D)$.
    \item Prove that $I(D)$ is locally generated by $ s = n-r $ linearly independent $1$-forms $\omega_1,\dots,\omega_s$, in the sense that for every point $p\in M$, there is a neighborhood $U$ of $p$ such that for any $\omega\in I^{k}\left( D \right)$ with $k$ arbitrary, we may write
      \begin{align*}
        \omega &= \sum_{i=1}^{s} \theta_i\wedge \omega_i
      \end{align*}
      for suitable forms $\theta_1,\dots,\theta_s$.
    \item Prove that if $D$ is involutive, then for all $\omega\in I(D)$, we have $d\omega\in I(D)$.
    \item Use this to show that if $\omega$ is a $1$-form, and $X,Y$ are vector fields, then
      \begin{align*}
        d\omega\left( X,Y \right) &= \frac{1}{2}\left( X\left( \omega(Y) \right) - Y\left( \omega(X) \right) - \omega\left( \left[ X,Y \right] \right) \right).
      \end{align*}
      Conclude that if $\omega\in I^{1}\left( D \right)$, and $X,Y\in D$, then $\omega\left( \left[ X,Y \right] \right) = 0$. Thus, if $I$ is a differential ideal, then $D$ is involutive.
    \item Show that if $D$ is defined by the vanishing of linearly independent forms $\omega_1,\dots,\omega_s$ near a point $p$, then $D$ is involutive if and only if for each $i$ there are $1$-forms $\omega_{i,j}$ such that
      \begin{align*}
        d\omega_i &= \sum_{j=1}^{s}\omega_{i,j}\wedge \omega_j.
      \end{align*}
  \end{enumerate}
\end{problem}
\begin{solution}\hfill
  \begin{enumerate}[(a)]
    \item Write
      \begin{align*}
        \omega &= \alpha_1\wedge\cdots\wedge \alpha_k
      \end{align*}
      so that
      \begin{align*}
        \omega\left( X_1,\dots,X_k \right) &= \det\left( \left( \alpha_{i}\left( X_j \right) \right)_{i,j} \right)\\
                                           &= 0.
      \end{align*}
      for $X_1,\dots,X_k\in D$. Then, if
      \begin{align*}
        \eta &= \beta_1\wedge\cdots\wedge \beta_{\ell},
      \end{align*}
      we have the determinant of the block matrices
      \begin{align*}
        \omega\wedge\eta \left( X_1,\dots,X_{k},\dots,X_{k + \ell} \right) &= \det \begin{pmatrix}\alpha_{i}\left( X_j \right) & \alpha_{i}\left( X_{\ell + j} \right) \\ \beta_{i}\left( X_j \right) & \beta_i \left( X_{\ell + j} \right)\end{pmatrix}\\
                                                                           &= 0,
      \end{align*}
      so that $\omega\wedge\eta$ is contained in $I(D)$.
  \end{enumerate}
\end{solution}
\end{document}
