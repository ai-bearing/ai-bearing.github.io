\documentclass[10pt]{mypackage}

% sans serif font:
%\usepackage{cmbright}
%\usepackage{sfmath}
%\usepackage{bbold} %better blackboard bold

\usepackage{homework}
%\usepackage{notes}
\usepackage{newpxtext,eulerpx,eucal}
\renewcommand*{\mathbb}[1]{\varmathbb{#1}}

\fancyhf{}
\rhead{Avinash Iyer}
\lhead{Differential Topology: Problem Set 6}

\setcounter{secnumdepth}{0}

\begin{document}
\RaggedRight
\begin{problem}[Problem 1]
  Let $X$ and $Y$ be simplicial complexes homeomorphic to the $2$-sphere, $S^{2}$, and the torus $S^{1}\times S^{1}$. Compute the real simplicial homology and cohomology of $X$ and $Y$.
\end{problem}
\begin{solution}
  We fix the order $\left( v_0,v_1,v_2,v_3,v_4,v_5 \right)$ in the simplicial complex for $X$. We see that the $k$-chains are as follows:
  \begin{itemize}
    \item $C_k\left( X,\R \right) = 0$ for all $k\geq 3$;
    \item $C_2\left( X,\R \right) = \R\left\langle v_0v_1v_2,v_0v_1v_3,v_0v_2v_3,v_1v_2v_5,v_1v_3v_5,v_2v_3v_5 \right\rangle \cong \R^{6}$;
    \item $C_1\left( X,\R \right) = \R\left\langle v_0v_1,v_0v_2,v_0v_3,v_1v_2,v_1v_3,v_2v_3,v_1v_5,v_2v_5,v_3v_5 \right\rangle\cong \R^{9}$;
    \item $C_0\left( X,\R \right) = \R\left\langle v_0,v_1,v_2,v_3,v_4,v_5 \right\rangle \cong \R^{6}$.
  \end{itemize}
  We start by applying the boundary map to $C_1\left( X,\R \right)$, yielding
  \begin{align*}
    v_0v_1 &\mapsto v_1-v_0\\
    v_0v_2 &\mapsto v_2-v_0\\
    v_0v_3 &\mapsto v_3-v_0\\
    v_1v_2 &\mapsto v_2-v_1\\
    v_1v_3 &\mapsto v_3-v_1\\
    v_2v_3 &\mapsto v_3-v_2\\
    v_1v_5 &\mapsto v_5-v_1\\
    v_2v_5 &\mapsto v_5-v_2\\
    v_3v_5 &\mapsto v_5-v_3.
  \end{align*}
  Since this forms a basis for the kernel of the linear functional given by mapping all of the $v_i$ to $1$, it follows that $B_0\left( X,\R \right)\cong \R^{5}$, while $Z_0\left( X,\R \right)\cong \R^{6}$, yielding $H_0\left( X,\R \right)\cong \R$.\newline

  Similarly, since we may find the boundary map $\partial\colon C_2\left( X,\R \right)\rightarrow C_1\left( X,\R \right)$ that yields a subspace that is the kernel of a linear functional on $\R^9$ with codimension $4$, it follows that $H_2\left( X,\R \right)\cong \R$ as well.\newline

  Finally, we see that the image of the basis for $C_2\left( X,\R \right)$ yields a basis with six linearly independent vectors, while the kernel of $\partial$ on $C_1\left( X,\R \right)$ yields another basis with six linearly independent vectors, so that $H_1\left( X,\R \right) \cong 0$.
\end{solution}
\begin{problem}[Problem 2]
  Use the definition of de Rham cohomology to prove that $H^{0}_{\operatorname{DR}}\left( \R \right)\cong \R$ and all higher de Rham cohomology vector spaces are zero.
\end{problem}
\begin{solution}
  Evaluating $H^{0}_{\operatorname{DR}}$, we see that the functions whose derivatives are zero are the constants on $\R$, meaning the cochains $Z^{0}\left( \R \right) \cong \R$, while the coboundaries $B^{0}\left( \R \right)\cong 0$.\newline

  Since $\R$ has dimension $1$, it follows that $\Lambda^{k}\left( \R \right) \cong 0$ for all $k\geq 2$, so we only need to verify that $Z^{1}\left( \R \right)\cong B^{1}\left( \R \right)$. This follows from the fact that every $1$-form can be integrated to yield a $C^{\infty}$ function on $\R$, while every $1$-form evaluates to zero under the exterior derivative.
\end{solution}
\begin{problem}[Problem 3]
  Use the definition of de Rham cohomology to prove that $H^{\ast}_{\operatorname{DR}}\left( S^{1} \right)\cong \R$ in dimensions $0$ and $1$ and vanishes in all higher dimensions.
\end{problem}
\begin{solution}
  Since $S^{1}$ is a $1$-dimensional manifold, it follows that $H^{k}_{\operatorname{DR}}\left( S^{1} \right)\cong 0$ for all $k\geq 2$ since all $2$-forms vanish.\newline

  Similarly, since only the constants $S^{1}$ vanish, we have $H^{0}_{\operatorname{DR}}\left( S^{1} \right)\cong \R$. Finally, to understand $H^{1}_{\operatorname{DR}}\left( S^{1} \right)$, we observe that any exact form $d\omega$ maps to $\R$ by integrating,
  \begin{align*}
    f(\theta) &= \int_{0}^{\theta}d\omega,
  \end{align*}
  and such non-closed exact forms exist on $S^{1}$, so that $H^{1}_{\operatorname{DR}}\left( S^{1} \right)\cong \R$.
\end{solution}
\begin{problem}[Problem 4]
  Prove that if $M$ is a closed, connected manifold of dimension $n$ that is not orientable, then the $n$th simplicial homology satisfies $H_n\left( M,\R \right) = 0$.
\end{problem}
\begin{solution}
  Let $p\in M$; since $M$ is orientable, if we select an $n$-simplex with a vertex at $p$, we find that both $v_0v_1\cdots v_n$ and $v_1v_0\cdots v_n$ yield valid orientations for $T_pM$. Taking a boundary of two of these $n$-simplices, we find that if $\sigma_i$ and $\sigma_j$ are two such simplices in $M$, we may orient $\sigma_i$ such that $\partial$ yields a positive value on this boundary, so that $B_n\left( M,\R \right)\cong \R$. Thus, we find that $H_n\left( M,\R \right)\cong 0$.
\end{solution}
\begin{problem}[Problem 5]
  A smooth map $f\colon M\rightarrow n$ is called a submersion if it induces surjections on tangent spaces. Prove that if $M$ and $N$ are smooth manifolds and $A\subseteq N$ is a smooth submanifold, then $f$ is transverse to $A$.
\end{problem}
\begin{solution}
  Let $p\in f^{-1}\left( A \right)$. By the definition of the submersion, we have $T_{F(p)}N = D_pF\left( T_pM \right)$, meaning that $D_pF\left( T_pM \right) + T_{F(p)}A = T_{F(p)}N$.
\end{solution}
\begin{problem}[Problem 6]
  In this exercise, we will prove a version of the Transversality Theorem. Let $M$ and $N$ be smooth manifolds. The transversality theorem asserts that for all $1\leq r \leq \infty$, the set of $C^{r}$ maps $M\rightarrow N$ that are transverse to $A$ is dense in any of the natural topologies $C^{r}\left( M,N \right)$.\newline

  We will restrict our attention to manifolds embedded in Euclidean space and prove a slightly weaker version of the transversality theorem.
  \begin{enumerate}[(a)]
    \item Let $M$, $N$, and $A$ be as above, and let $Y$ be an arbitrary smooth manifold. Let $F\colon Y\times M\rightarrow N$ be a smooth map transverse to $A$. For each $y\in Y$, let $f_y\colon M\rightarrow N$ be defined by $F\left( y,\cdot \right)$, and let $\pi\colon Y\times M\rightarrow Y$ be the projection.\newline

      Prove that for every regular value $y\in Y$ of $\pi$, the map $f_y$ is transverse to $A$.
    \item Let $f\colon M\rightarrow \R^{n}$ be a smooth map, and let $A\subseteq \R^{n}$ be a smooth submanifold. Show that the set of $p\in \R^{n}$ for which $f_p(x) \coloneq f(x) + p$ is not transverse to $A$ has measure zero.
    \item Prove that if $M$ and $N$ are smooth submanifolds of $\R^n$, then for all $p\in \R^{n}$ outside a set of measure zero, the manifolds $M + p$ and $N$ intersect transversely.
    \item Prove that if $f\colon M\rightarrow N$ is smooth, and $A\subseteq N$ is a smooth submanifold, then $f$ is smoothly homotopic to a map that is transverse to $A$.
  \end{enumerate}
\end{problem}
\begin{solution}\hfill
  \begin{enumerate}[(a)]
    \item Let $p\in A$, and let $y$ be a regular value for $\pi$. Observe that, by the regular value theorem, we have that $\pi^{-1}\left( y \right) = \set{y}\times M$ is a smooth submanifold of $Y\times M$. It follows from the definition of the $f_y$ that $F\circ \pi^{-1}\left( y \right) \equiv f_y$.\newline

      Since $F$ is transverse to $A$, it follows that for any $\left( z,q \right)\in F^{-1}\left( p \right)$, we have
      \begin{align*}
        D_{(z,q)}F\left( T_{(z,q)}\left( Y\times M \right) \right) + T_pA &= T_pN.
      \end{align*}
      We have, by chain rule and the inverse function theorem (seeing as $y$ is a regular value of $\pi$),
      \begin{align*}
        D_qf_y &= D_q\left( F\circ\pi^{-1}(y) \right)\\
               &= D_{(y,q)}F\circ \left( D_{\pi^{-1}(y)}\pi \right)^{-1}(y)\\
               &= D_{(y,q)}F,
      \end{align*}
      so that
      \begin{align*}
        D_qf_y\left( T_{q}M \right) + T_pA &= D_{(y,q)}F\left( T_{(y,q)}\left( Y\times M \right) \right) + T_pA\\
                                           &= T_pN,
      \end{align*}
      meaning $f_y$ is transverse to $A$ for any regular value $y\in Y$ of $\pi$.
    \item If we let $Y \equiv \R^{n}$ in part (a), and let $F\colon \R^{n}\times M \rightarrow \R^{n}$ be defined by $F\left( p,x \right) = f(x) + p$, then we observe that for every regular value $p$ of $\pi$, that $f(x) + p$ is transverse to $A$. In particular, since the set of critical values has measure zero in $\R^{n}$, it follows that for almost every $p$, $f(x) + p$ is transverse to $A$.
    \item Since $N\subseteq \R^{n}$ is a smooth submanifold, then we may apply part (b) to $\iota\colon M\hookrightarrow \R^{n}\supseteq N$, whence $M+p$ and $N$ intersect transversely for almost every $p\in \R^{n}$.
    \item Since we treat $A\subseteq N\subseteq \R^{n}$ as a smooth submanifold, we know that the set of all $p$ for which $f_p(x) = f(x) + p$ is not transverse to $A$ is a set of measure zero; in particular, we may find a smooth homotopy from $f$ to $f_p$ where $f_p$ is a translate of $f$ that intersects $A$ and is transverse to $A$ (which exists by the fact that the set of all points where this does not hold is of measure zero). Thus, $f$ is smoothly homotopic to a map that is transverse to $A$.
  \end{enumerate}
\end{solution}
\end{document}
