\documentclass[10pt]{mypackage}

% sans serif font:
%\usepackage{cmbright}
%\usepackage{sfmath}
%\usepackage{bbold} %better blackboard bold

\usepackage{homework}
%\usepackage{notes}
\usepackage{newpxtext,eulerpx,eucal}
\renewcommand*{\mathbb}[1]{\varmathbb{#1}}

\fancyhf{}
\rhead{Avinash Iyer}
\lhead{Differential Topology: Problem Set 1}

\setcounter{secnumdepth}{0}

\begin{document}
\RaggedRight
\begin{problem}[Problem 1]
  Describe the topology of the Grassmanian $\operatorname{Gr}\left( k,n \right)$ in a uniform way, so that $\R\mathbb{P}^n$ becomes the special case of $\operatorname{Gr}\left( 1,n \right)$.
\end{problem}
\begin{solution}
We let elements of $\operatorname{Gr}\left( k,n \right)$ be defined as equivalence classes of linearly independent $k$-tuples of vectors in $\R^n$, where $\left( v_1,\dots,v_k \right) \sim \left( w_1,\dots,w_k \right)$ if $\Span \set{v_1,\dots,v_k} = \Span\set{w_1,\dots,w_k} $.\newline

By extending $\left( v_1,\dots,v_k \right)$ and $\left( w_1,\dots,w_k \right)$ to ordered bases $\mathcal{B}_1 = \left( v_1,\dots,v_n \right)$ and $\mathcal{B}_2 = \left( w_1,\dots,w_n \right)$, we see that these $k$-tuples are equivalent if and only if there is a change of basis transformation $Q$ with matrix representation
\begin{align*}
  Q &= \begin{pmatrix}A & 0 \\ 0 & B\end{pmatrix},
\end{align*}
where $A$ is a $k\times k$ invertible matrix, and $B$ is a $\left( n-k \right)\times \left( n-k \right)$ matrix. The subgroup of all such $Q\subseteq \GL_n\left( \R \right)$, which we call $P$, is the stabilizer of $\operatorname{Gr}\left( k,n \right)$ as we have defined it, so by the orbit-stabilizer theorem (seeing as $\GL_n\left( \R \right)$ acts transitively on all ordered bases of $\R^n$), we obtain $\operatorname{Gr}\left( k,n \right)\cong \GL_n\left( \R \right)/P$, where the latter coset space is given the quotient topology.\newline

Note that this definition comports with the definition of $\R\mathbb{P}^{n}$ as the space of one-dimensional subspaces, as the invertible $1\times 1$ matrices are precisely the nonzero scalars.
\end{solution}
\begin{problem}[Problem 2]
  Fix an inner product on $\R^n$. Show that the map $V\mapsto V^{\perp}$ induces a $C^{\infty}$ diffeomorphism $\operatorname{Gr}\left( k,n \right)\rightarrow \operatorname{Gr}\left( n-k,n \right)$.
\end{problem}
\begin{solution}
  We know that, since there is an inner product, we may express the smooth atlas of $\operatorname{Gr}\left( n,k \right)$ by $\set{\left( U_V,\varphi_V \right)}$, where
  \begin{align*}
    U_V &= \set{W\in \operatorname{Gr}\left( k,n \right) | W\cap V^{\perp} = 0},
  \end{align*}
  and $\varphi = P_{V^{\perp}} P_{V}|_{W}^{-1}$ is the sequence of projections. By pre-composing with the map $V\mapsto V^{\perp}$, we get the atlas $\set{\left( U_{V^{\perp}},\varphi_{V^{\perp}} \right)}$ for $\operatorname{Gr}\left( n-k,n \right)$ consisting of charts of the form
  \begin{align*}
    U_{V^{\perp}} &= \set{W\in \operatorname{Gr}\left( n-k,n \right) | W\cap V = 0}\\
    \varphi_{V^{\perp}} &= P_{V}P_{V^{\perp}}|_{W}^{-1},
  \end{align*}
  Since the maps $\varphi_{V}\circ \left( V\mapsto V^{\perp} \right)\circ \varphi_{V^{\perp}}^{-1}$ are a composition of smooth bijections with smooth inverses, we see that this is a $C^{\infty}$ diffeomorphism between $\operatorname{Gr}\left( k,n \right) \cong \operatorname{Gr}\left( n-k,n \right)$.
\end{solution}
\begin{problem}[Problem 3]
  Prove that a $C^{k}$ map which is a $C^{1}$ diffeomorphism is necessarily a $C^{k}$ diffeomorphism.
\end{problem}
\begin{solution}
  Let $f\colon \R^{n}\rightarrow \R^{n}$ be a $C^{k}$ map that is a $C^{1}$ diffeomorphism. In order to show that $f$ is a $C^{k}$ diffeomorphism, we need to show that $f^{-1}\colon \R^{n}\rightarrow \R^{n}$ exists and is of class $C^{k}$.\newline

  First, by the inverse function theorem, since $f$ is a $C^{1}$ diffeomorphism, we see that $f^{-1}\colon \R^{n}\rightarrow \R^{n}$ exists, is continuous, and is such that $D\left( f^{-1} \right)$ is continuous.\newline

  Now, we observe that the association $y\mapsto D_y\left( f^{-1} \right)$ can be written as
  \begin{align*}
    y \mapsto f^{-1}(y) \mapsto D_{y}f\left( f^{-1}(y) \right) \mapsto \left( D_yf\left( f^{-1}\left( y \right) \right) \right)^{-1} = D_y\left( f^{-1} \right),
  \end{align*}
  where we observe that $f^{-1}$ is of class $C^1$, the derivative $D_f$ is of class $C^{k-1}$, and matrix inversion is $C^{\infty}$; since $D\left( f^{-1} \right)$ is a composition of $C^{1}$ functions, $D\left( f^{-1} \right)$ is $C^{1}$, so $f^{-1}$ is $C^{2}$. Inductively, we see that $f^{-1}$ is also of class $C^{k}$, so $f$ is a $C^{k}$ diffeomorphism.
\end{solution}
\begin{problem}[Problem 4]
  Recall that a topological space is paracompact if every open cover admits a locally finite refinement. Prove that a connected, paracompact manifold of dimension one is either $\R$ or $S^{1}$, depending on whether it is compact or not.
\end{problem}
\begin{solution}
  Let $M$ be a connected, paracompact manifold with dimension $1$, and let $\set{\left( U_i,\varphi_i \right)}_{i\in I}$ be an atlas for $M$, where $\varphi_i$ are homeomorphisms between $U_i$ and $\R$.\newline

  Let $\set{V_{j}}_{j\in J}$ be a locally finite refinement of $\set{U_i}_{i\in I}$, where the restrictions $\psi_j \coloneq \varphi_i|_{V_j}$ are homeomorphisms to $O_{j}\subseteq \R$. We see that for any $p\in M$, since the family of $V_j$ with $p\in V_j$, which we call $\mathcal{V}_p = \set{V_j | p\in V_j}$, is finite, the intersection $\bigcap \mathcal{V}_p$ is open; similarly, the intersection $\bigcap \mathcal{O}_p\subseteq \R$ is open, where $\mathcal{O}_p = \set{\varphi\vert_{V_j}\left( V_j \right)\subseteq \R | V_j\in \mathcal{V}_p}$.\newline

  We see that $M = \bigcup_{p\in M} \bigcap \mathcal{V}_p$. Note that for any distinguished point $p_1$, the corresponding sets $\bigcap \mathcal{V}_{p_1}$ and $\bigcup_{p\neq p_1} \bigcap \mathcal{V}_{p}$ must have nonempty (open) intersection, by the assumption that $M$ is connected. Thus, the corresponding union $\bigcup_{p\in M}\bigcap \mathcal{O}_p$ is an open and connected subset of $\R$. We may similarly map $\bigcup_{p\in M} \bigcap \mathcal{O}_p$ into $\S^1$ by composing with the quotient map.\newline

  Now, if $M$ is compact, then $\bigcup_{p\in M}\bigcap \mathcal{V}_p$ covers $M$, so there is a finite subcover $M = \bigcup_{i=1}^{n}\bigcap \mathcal{V}_{p_i}$, so that $\bigcup_{i=1}^{n}\bigcap \mathcal{O}_{p_i}$ fully covers the corresponding range, meaning that, composing with the quotient map $\bigcup_{i=1}^{n} \bigcap \mathcal{O}_{p_i}$, we have that $M \cong S^{1}$. Similarly, if $M$ is non-compact, then $\bigcup_{p\in M}\bigcap\mathcal{O}_p$ is an open and connected subset of $\R$ that does not admit any finite subcover, hence it is homeomorphic to $\R$.
\end{solution}
\begin{problem}[Problem 5]
  In this problem, we prove a weak version of the Whitney Embedding Theorem.
  \begin{enumerate}[(a)]
    \item Find a $C^{\infty}$ function $\lambda$ on $\R^{n}$ with values in $[0,1]$ such that $\lambda$ takes the value $1$ on the closed ball $B\left( 0,1 \right)$, and vanishes outside the closed ball $B\left( 0,2 \right)$.
    \item Suppose $M$ is a compact $C^{k}$ manifold of dimension $n$. Find a $C^{k}$ atlas $\set{U_i,\varphi_i}_{i\in I}$ such that $\varphi_i\left( U_i \right)$ contains $B\left( 0,2 \right)$, and such that $M$ is covered by the union of $\varphi_i^{-1}\left( B\left( 0,1 \right) \right)^{\circ}$.
    \item Let $\lambda_i$ be defined by $\lambda\circ\varphi_i$ on $U_i$, and $0$ outside $U_i$. Let $f_i\colon M\rightarrow \R^n$ be defined by $\lambda_i\circ\varphi_i$ on $U_i$ and zero otherwise. Use these functions to embed $M$ as a submanifold of some Euclidean space.
  \end{enumerate}
\end{problem}
\begin{problem}[Problem 6]
  Use the ideas of the previous exercise to prove that a $C^{k}$ manifold admits a $C^{k}$ partition of unity subordinate to any locally finite cover.
\end{problem}
\begin{problem}[Problem 7]
  Let $X$ and $Y$ be topological spaces, and let $C\left( X,Y \right)$ be the set of continuous maps from $X$ to $Y$. Equip $C\left( X,Y \right)$ with the compact-open topology, where the basic open sets are
  \begin{align*}
    U_{K,V} &= \set{f | f(K)\subseteq V},
  \end{align*}
  where $K\subseteq X$ is compact and $V\subseteq Y$ is open.\newline

  If $Y$ is a metric space, and if $X$ is compact, prove that this topology is the same as the topology of uniform convergence.
\end{problem}
\begin{solution}
  Let $Y$ be a metric space and let $X$ be compact. We note that a neighborhood basis in the topology of uniform convergence on $C\left( X,Y \right)$ consists of sets of the form
  \begin{align*}
    U_{f,\ve} &= \set{g | \sup_{x\in X}d\left( f(x),g(x) \right) < \ve}.
  \end{align*}
  Similarly, a subbase for the compact open topology consists of sets of the form
  \begin{align*}
    U_{f,K,\ve} &= \set{g | \sup_{x\in K}d\left( f(x),g(x) \right) < \ve};
  \end{align*}
  the fact that $Y$ is a metric space allows us to take this refinement of the compact-open topology.\newline

  Thus, to prove that the compact-open topology and the topology of uniform convergence are equivalent, we show that any basis element of the topology of uniform convergence is contained in a basis element of the compact-open topology, and vice versa.\newline

  First, we see that almost by definition, if $K\subseteq X$ is any compact subset, then
  \begin{align*}
    U_{f,\ve} &\subseteq U_{f,K,\ve},
  \end{align*}
  as any function whose supremum distance is less than $\ve$ over $X$ must have that supremum distance hold over $K\subseteq X$.\newline

  Now, in the reverse direction, we fix $f$ and $\ve$. We wish to show that there is a finite family of subsets $U_{K_i,V_i}$ with $f\in U_{K_i,V_i}$ for each $i$, and their intersection lies in $U_{f,\ve}$. We see that every point $x\in X$ has a pre-compact open neighborhood $U_x$ such that $f\left( \overline{U_x} \right) \subseteq U\left( f(x),\ve/3 \right)$. The family $\set{x\in X}U_x$ is an open cover for $X$, so admits a finite subcover $\set{U_{x_i}}_{i=1}^{n}$. Since each $ \overline{U_{x_i}}_{i=1}^{n} $ is compact, and $f\in U_{ \overline{U_{x_i}},U\left( f\left( x_i \right),\ve/3 \right) }$ for each $i$, we see that
  \begin{align*}
    V &= \bigcap_{i=1}^{n} U_{ \overline{U_{x_i}},U\left( f\left( x_i \right),\ve/3 \right) }
  \end{align*}
  is an open subset in the compact-open topology on $C\left( X,Y \right)$ that contains $f$ and is contained in $U_{f,\ve }$, so the topologies are thus equal.
\end{solution}
\begin{problem}[Problem 8]
  Let $C^{k}\left( M,N \right)$ be the set of $C^{k}$ maps from $M$ to $N$. The compact-open topology on $C^{k}\left( M,N \right)$ is defined similarly. Let $f\in C^{k}\left( M,N \right)$, $\left( U,\varphi \right)$ and $\left( V,\psi \right)$ charts on $M$ and $N$, let $K\subseteq U$ be compact such that $f(K)\subseteq V$, and let $\ve > 0$. We obtain a basic neighborhood $N\left( f,U,\varphi,V,\psi,K,\ve \right)$ by looking at all the maps $g\in C^{k}\left( M,N \right)$ such that $g(K)\subseteq V$, and
  \begin{align*}
    \norm{D^{r}\left( \psi f \varphi^{-1} \right)\left( x \right) - D^{r}\left( \psi g \varphi^{-1} \right)\left( x \right)}_{\op} \leq \ve\label{eq:compact_open_manifolds}\tag{$\ast$}
  \end{align*}
  for all integers $0\leq r \leq k$.\newline

  The Whitney topology is slightly different. Let $\Phi = \set{\left( U_i,\varphi_i \right)}_{i\in I}$ be a locally finite atlas on $M$, let $K_i\subseteq U_i$ be compact for all $i$, let $\Psi$ be an atlas on $N$, and let $\set{\ve_i}_{i\in I}$ be a family of positive numbers. A basic neighborhood of $f\in C^{k}\left( M,N \right)$ in this topology is given by all $g$ such that $g\left( K_i \right)\subseteq V_i$ for all $i$, and
  \begin{align*}
    \norm{D^r\left( \psi_i f \varphi_i^{-1} \right)\left( x \right) - D^{r}\left( \psi_i g \varphi_i^{-1} \right)\left( x \right)}_{\op} \leq \ve_i\label{eq:whitney_manifolds}\tag{$\ast\ast$}
  \end{align*}
  for all $x\in \varphi_i\left( K_i \right)$ and all integers $0\leq r \leq k$.\newline

  For infinite values of $k$, we take the compact-open and Whitney topologies on $C^{\infty}\left( M,N \right)$ to be the union of these topologies via the inclusion $C^{\infty}\left( M,N \right)\subseteq C^{k}\left( M,N \right)$. Show the following:
  \begin{enumerate}[(a)]
    \item these basic neighborhoods actually give a basis for a topology in both cases;
    \item if $M$ is compact, these two topologies coincide;
    \item if $M$ is compact and has no boundary, then the $C^{k}$ diffeomorphisms from $M$ to $N$ are open in $C^{k}\left( M,N \right)$ in the Whitney topology.
  \end{enumerate}
\end{problem}
\begin{solution}\hfill
  \begin{enumerate}[(a)]
    \item Clearly, in both the compact open topology and the Whitney topology, the respective neighborhoods cover $C^{k}\left( M,N \right)$, so we only need to verify the condition that if $X_1,X_2\subseteq C^{k}\left( M,N \right)$ are open subsets such that $f\in X_1\cap X_2$, then there is $X_3\subseteq C^{k}\left( M,N \right)$ open such that $X_3\subseteq X_1\cap X_2$.\newline

      We start with the case of the compact-open topology. Let $f\in X_1\cap X_2$, where $X_1$ and $X_2$ are open in the compact-open topology. Since $f\in X_1$, there is a chart $\left( U_1,\varphi_1 \right)$ of $M$, a chart $\left( V_1,\psi_1 \right)$ of $N$, $K_1\subseteq U_1$ compact such that $f\left( K_1 \right)\subseteq V_1$, and $\ve_1 > 0$ such that \eqref{eq:compact_open_manifolds} holds and $N\left( f,U_1,\varphi_1,V_1,\psi_1,\ve_1 \right) \subseteq X_1$. Similarly, since $f\in X_2$, there are charts $\left( U_2,\varphi_2 \right)$ and $\left( V_2,\psi_2 \right)$ of $M$ and $N$ respectively, $K_2\subseteq U_2$ compact with $f\left( K_2 \right)\subseteq V_2$, and $\ve_2 > 0$ such that \eqref{eq:compact_open_manifolds} holds, and $N\left( f,U_2,\varphi_2,V_2,\psi_2,\ve_2 \right)\subseteq X_2$. Note that by the characterization, \eqref{eq:compact_open_manifolds} holds for the supremum over all $x\in \varphi_j\left( K_j \right)$ for $j = 1,2$.
  \end{enumerate}
\end{solution}
\end{document}
