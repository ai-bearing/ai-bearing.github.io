\documentclass[10pt]{mypackage}

% sans serif font:
%\usepackage{cmbright}
%\usepackage{sfmath}
%\usepackage{bbold} %better blackboard bold

\usepackage{homework}
%\usepackage{notes}
\usepackage{newpxtext,eulerpx,eucal}
\renewcommand*{\mathbb}[1]{\varmathbb{#1}}

\fancyhf{}
\rhead{Avinash Iyer}
\lhead{Differential Topology: Problem Set 1}

\setcounter{secnumdepth}{0}

\begin{document}
\RaggedRight
\begin{problem}[Problem 1]
  Describe the topology of the Grassmanian $\operatorname{Gr}\left( k,n \right)$ in a uniform way, so that $\R\mathbb{P}^n$ becomes the special case of $\operatorname{Gr}\left( 1,n \right)$.
\end{problem}
\begin{solution}
We let elements of $\operatorname{Gr}\left( k,n \right)$ be defined as equivalence classes of linearly independent $k$-tuples of vectors in $\R^n$, where $\left( v_1,\dots,v_k \right) \sim \left( w_1,\dots,w_k \right)$ if $\Span \set{v_1,\dots,v_k} = \Span\set{w_1,\dots,w_k} $.\newline

By extending $\left( v_1,\dots,v_k \right)$ and $\left( w_1,\dots,w_k \right)$ to ordered bases $\mathcal{B}_1 = \left( v_1,\dots,v_n \right)$ and $\mathcal{B}_2 = \left( w_1,\dots,w_n \right)$, we see that these $k$-tuples are equivalent if and only if there is a change of basis transformation $Q$ with matrix representation
\begin{align*}
  Q &= \begin{pmatrix}A & 0 \\ 0 & B\end{pmatrix},
\end{align*}
where $A$ is a $k\times k$ invertible matrix, and $B$ is a $\left( n-k \right)\times \left( n-k \right)$ matrix. The subgroup of all such $Q\subseteq \GL_n\left( \R \right)$, which we call $P$, is the stabilizer of $\operatorname{Gr}\left( k,n \right)$ as we have defined it, so by the orbit-stabilizer theorem (seeing as $\GL_n\left( \R \right)$ acts transitively on all ordered bases of $\R^n$), we obtain $\operatorname{Gr}\left( k,n \right)\cong \GL_n\left( \R \right)/P$, where the latter coset space is given the quotient topology.\newline

Note that this definition comports with the definition of $\R\mathbb{P}^{n}$ as the space of one-dimensional subspaces, as the invertible $1\times 1$ matrices are precisely the nonzero scalars.
\end{solution}
\begin{problem}[Problem 2]
  Fix an inner product on $\R^n$. Show that the map $V\mapsto V^{\perp}$ induces a $C^{\infty}$ diffeomorphism $\operatorname{Gr}\left( k,n \right)\rightarrow \operatorname{Gr}\left( n-k,n \right)$.
\end{problem}
\begin{solution}
  We know that, since there is an inner product, we may express the smooth atlas of $\operatorname{Gr}\left( n,k \right)$ by $\set{\left( U_V,\varphi_V \right)}$, where
  \begin{align*}
    U_V &= \set{W\in \operatorname{Gr}\left( k,n \right) | W\cap V^{\perp} = 0},
  \end{align*}
  and $\varphi = P_{V^{\perp}} P_{V}|_{W}^{-1}$ is the sequence of projections. By pre-composing with the map $V\mapsto V^{\perp}$, we get the atlas $\set{\left( U_{V^{\perp}},\varphi_{V^{\perp}} \right)}$ for $\operatorname{Gr}\left( n-k,n \right)$ consisting of charts of the form
  \begin{align*}
    U_{V^{\perp}} &= \set{W\in \operatorname{Gr}\left( n-k,n \right) | W\cap V = 0}\\
    \varphi_{V^{\perp}} &= P_{V}P_{V^{\perp}}|_{W}^{-1},
  \end{align*}
  Since the maps $\varphi_{V}\circ \left( V\mapsto V^{\perp} \right)\circ \varphi_{V^{\perp}}^{-1}$ are a composition of smooth bijections with smooth inverses, we see that this is a $C^{\infty}$ diffeomorphism between $\operatorname{Gr}\left( k,n \right) \cong \operatorname{Gr}\left( n-k,n \right)$.
\end{solution}
\begin{problem}[Problem 3]
  Prove that a $C^{k}$ map which is a $C^{1}$ diffeomorphism is necessarily a $C^{k}$ diffeomorphism.
\end{problem}
\begin{problem}[Problem 4]
  Recall that a topological space is paracompact if every open cover admits a locally finite refinement. Prove that a connected, paracompact manifold of dimension one is either $\R$ or $S^{1}$, depending on whether it is compact or not.
\end{problem}
\begin{solution}
  Let $M$ be a connected, paracompact manifold with dimension $1$, and let $\set{\left( U_i,\varphi_i \right)}_{i\in I}$ be an atlas for $M$, where $\varphi_i$ are homeomorphisms between $U_i$ and $\R$.\newline

  Let $\set{V_{j}}_{j\in J}$ be a locally finite refinement of $\set{U_i}_{i\in I}$, where the restrictions $\psi_j \coloneq \varphi_i|_{V_j}$ are homeomorphisms to $O_{j}\subseteq \R$. We see that for any $p\in M$, since the family of $V_j$ with $p\in V_j$, which we call $\mathcal{V}_p = \set{V_j | p\in V_j}$, is finite, the intersection $\bigcap \mathcal{V}_p$ is open; similarly, the intersection $\bigcap \mathcal{O}_p\subseteq \R$ is open, where $\mathcal{O}_p = \set{\varphi\vert_{V_j}\left( V_j \right)\subseteq \R | V_j\in \mathcal{V}_p}$.\newline

  We see that $M = \bigcup_{p\in M} \bigcap \mathcal{V}_p$. Note that for any distinguished point $p_1$, the corresponding sets $\bigcap \mathcal{V}_{p_1}$ and $\bigcup_{p\neq p_1} \bigcap \mathcal{V}_{p}$ must have nonempty (open) intersection, by the assumption that $M$ is connected. Thus, the corresponding union $\bigcup_{p\in M}\bigcap \mathcal{O}_p$ is an open and connected subset of $\R$. We may similarly map $\bigcup_{p\in M} \bigcap \mathcal{O}_p$ into $\S^1$ by composing with the quotient map.\newline

  Now, if $M$ is compact, then $\bigcup_{p\in M}\bigcap \mathcal{V}_p$ covers $M$, so there is a finite subcover $M = \bigcup_{i=1}^{n}\bigcap \mathcal{V}_{p_i}$, so that $\bigcup_{i=1}^{n}\bigcap \mathcal{O}_{p_i}$ fully covers the corresponding range, meaning that, composing with the quotient map $\bigcup_{i=1}^{n} \bigcap \mathcal{O}_{p_i}$, we have that $M \cong S^{1}$. Similarly, if $M$ is non-compact, then $\bigcup_{p\in M}\bigcap\mathcal{O}_p$ is an open and connected subset of $\R$ that does not admit any finite subcover, hence it is homeomorphic to $\R$.
\end{solution}
\begin{problem}[Problem 5]
  In this problem, we prove a weak version of the Whitney Embedding Theorem.
  \begin{enumerate}[(a)]
    \item Find a $C^{\infty}$ function $\lambda$ on $\R^{n}$ with values in $[0,1]$ such that $\lambda$ takes the value $1$ on the closed ball $B\left( 0,1 \right)$, and vanishes outside the closed ball $B\left( 0,2 \right)$.
    \item Suppose $M$ is a compact $C^{k}$ manifold of dimension $n$. Find a $C^{k}$ atlas $\set{U_i,\varphi_i}_{i\in I}$ such that $\varphi_i\left( U_i \right)$ contains $B\left( 0,2 \right)$, and such that $M$ is covered by the union of $\varphi_i^{-1}\left( B\left( 0,1 \right) \right)^{\circ}$.
    \item Let $\lambda_i$ be defined by $\lambda\circ\varphi_i$ on $U_i$, and $0$ outside $U_i$. Let $f_i\colon M\rightarrow \R^n$ be defined by $\lambda_i\circ\varphi_i$ on $U_i$ and zero otherwise. Use these functions to embed $M$ as a submanifold of some Euclidean space.
  \end{enumerate}
\end{problem}
\begin{problem}[Problem 6]
  Use the ideas of the previous exercise to prove that a $C^{k}$ manifold admits a $C^{k}$ partition of unity subordinate to any locally finite cover.
\end{problem}
\begin{problem}[Problem 7]
  Let $X$ and $Y$ be topological spaces, and let $C\left( X,Y \right)$ be the set of continuous maps from $X$ to $Y$. Equip $C\left( X,Y \right)$ with the compact-open topology, where the basic open sets are
  \begin{align*}
    U_{K,V} &= \set{f | f(K)\subseteq V},
  \end{align*}
  where $K\subseteq X$ is compact and $V\subseteq Y$ is open.\newline

  If $Y$ is a metric space, and if $X$ is compact, prove that this topology is the same as the topology of uniform convergence.
\end{problem}
\begin{solution}
  Let $Y$ be a metric space, and let $X$ be compact. Assume $X$ is nonempty. We will show that a net $\left( f_{\alpha} \right)_{\alpha}\rightarrow f$ in the compact-open topology if and only if $\left( f_{\alpha} \right)_{\alpha}\rightarrow f$ uniformly.\newline

  If $\left( f_{\alpha} \right)_{\alpha}\rightarrow f$ uniformly, then for all $\ve > 0$, there exists $\alpha_0$ such that for all $\alpha \geq \alpha_0$ and for all $x\in X$,
  \begin{align*}
    d\left( f_{\alpha}\left( x \right),f\left( x \right) \right) < \ve/2,
  \end{align*}
  meaning that, in particular, $f_{\alpha}\left( X \right)\subseteq U\left( f(x),\ve \right)$ for all $x\in X$, so by setting $V = U\left( f\left( x_0 \right),\ve \right)$ for a distinguished $x_0\in X$ and the compact subset $X\subseteq X$, we see that $\left( f_{\alpha} \right)_{\alpha}\subseteq U_{X,V}$ for all $\alpha \geq \alpha_0$, so $\left( f_{\alpha} \right)_{\alpha}\rightarrow f$ in the compact-open topology.\newline

  Now, let $\left( f_{\alpha} \right)_{\alpha}\rightarrow f$ in the compact open topology; by definition, this means that for all basic open neighborhoods $U_{K,V}$, there exists $\alpha_{K,V}$ such that for all $\alpha\geq \alpha_{K,V}$, we have $f_{\alpha}\in U_{K,V}$.\newline

  Let $\ve > 0$. Since $f$ is uniformly continuous, for all $x\in X$, there is a pre-compact open neighborhood $U_x$ of $x$ such that $f\left( U_x \right)\subseteq U\left( f(x), \ve/3\right)$; in particular, notice that $f\left( \overline{U_x} \right)\subseteq U\left( f(x),\ve/2 \right)$.\newline

  The sets $\set{U_x}_{x\in X}$ cover $X$, so since $X$ is compact, there are $x_1,\dots,x_n$ such that $U_{x_1},\dots,U_{x_n}$ cover $X$. We let $V_i = U\left( f\left( x_i \right),\ve/2 \right)$ be the corresponding open balls such that $f\left( \overline{U_{x_i}} \right)\subseteq V_i$, meaning there exist $\alpha_{U_{x_i},V_i}$ for $i=1,\dots,n$ such that for all $\alpha \geq \alpha_{ \overline{U_{x_i}},V_i}$, $f_{\alpha}\in U_{ \overline{U_{x_i}},V_{i}}$. Since the $\alpha$ are directed, there is $\alpha_0\geq \alpha_{ \overline{U_{x_i}},V_i}$ for each $i = 1,\dots,n$, meaning that for all $\alpha \geq \alpha_{0}$, we have $f_{\alpha}\in \bigcap_{i=1}^{n}U_{ \overline{U_{x_i}},V_i}$. In particular, we see that for all $\alpha \geq \alpha_0$, and for all $x\in X$, $d\left( f_{\alpha}\left( x \right),f\left( x \right) \right) < \ve/2$.
\end{solution}
\end{document}
