\documentclass[10pt]{mypackage}

% sans serif font:
%\usepackage{cmbright}
%\usepackage{sfmath}
%\usepackage{bbold} %better blackboard bold

\usepackage{homework}
%\usepackage{notes}
\usepackage{newpxtext,eulerpx,eucal}
\renewcommand*{\mathbb}[1]{\varmathbb{#1}}

\fancyhf{}
\rhead{Avinash Iyer}
\lhead{Differential Topology: Problem Set 1}

\setcounter{secnumdepth}{0}

\begin{document}
\RaggedRight
\begin{problem}[Problem 1]
  Describe the topology of the Grassmanian $\operatorname{Gr}\left( k,n \right)$ in a uniform way, so that $\R\mathbb{P}^n$ becomes the special case of $\operatorname{Gr}\left( 1,n \right)$.
\end{problem}
\begin{solution}
We let elements of $\operatorname{Gr}\left( k,n \right)$ be defined as equivalence classes of linearly independent $k$-tuples of vectors in $\R^n$, where $\left( v_1,\dots,v_k \right) \sim \left( w_1,\dots,w_k \right)$ if $\Span \set{v_1,\dots,v_k} = \Span\set{w_1,\dots,w_k} $.\newline

By extending $\left( v_1,\dots,v_k \right)$ and $\left( w_1,\dots,w_k \right)$ to ordered bases $\mathcal{B}_1 = \left( v_1,\dots,v_n \right)$ and $\mathcal{B}_2 = \left( w_1,\dots,w_n \right)$, we see that these $k$-tuples are equivalent if and only if there is an invertible linear transformation $Q$ with matrix representation
\begin{align*}
  Q &= \begin{pmatrix}A & H \\ 0 & B\end{pmatrix},
\end{align*}
where $A$ is a $k\times k$ invertible matrix, and $B$ is a $\left( n-k \right)\times \left( n-k \right)$ invertible matrix, so that 
\begin{align*}
    Q \left[ v_1,v_2,\dots,v_k,v_{k+1},\dots,v_n \right] = \left[ w_1,\dots,w_k,w^{\ast}_{k+1},\dots,w_n^{\ast} \right],
\end{align*}
where $\set{w_{k+1}^{\ast},\dots,w_n^{\ast}}$ is a basis for the $n-k$-dimensional complementary subspace. The subgroup of all such $Q\subseteq \GL_n\left( \R \right)$, which we call $P$, is the stabilizer of $\operatorname{Gr}\left( k,n \right)$ as we have defined it, so by the orbit-stabilizer theorem (seeing as $\GL_n\left( \R \right)$ acts transitively on all ordered bases of $\R^n$), we obtain $\operatorname{Gr}\left( k,n \right)\cong \GL_n\left( \R \right)/P$, where the latter coset space is given the quotient topology.\newline

This definition comports with the definition of $\R\mathbb{P}^{n}$ as the space of one-dimensional subspaces, as the invertible $1\times 1$ matrices are precisely the nonzero scalars, so the stabilizers in the case of $\operatorname{Gr}\left( 1,n \right)$ are the $1\times 1$ invertible block matrices $A$, or the nonzero scalars.
\end{solution}
\begin{problem}[Problem 2]
  Fix an inner product on $\R^n$. Show that the map $V\mapsto V^{\perp}$ induces a $C^{\infty}$ diffeomorphism $\operatorname{Gr}\left( k,n \right)\rightarrow \operatorname{Gr}\left( n-k,n \right)$.
\end{problem}
\begin{solution}
  Due to the inner product, we make the identification $v\mapsto v^{\ast}$ such that $v^{\ast}\left( w \right) = \iprod{v}{w}$. In particular, we have isomorphisms $V\cong V^{\ast}$ and $V^{\perp}\cong \left( V^{\perp} \right)^{\ast}$. Therefore, given an element $T\in \Hom\left( V,V^{\perp} \right)$, dualization gives the transpose map $T^{\ast}\in \Hom\left( \left( V^{\perp} \right)^{\ast},V^{\ast} \right)$.\newline

  Now, given any chart $\left( U_V,\varphi_V \right)$ in $\operatorname{Gr}\left( k,n \right)$, we identify $T\in \Hom\left( V,V^{\perp} \right)\cong U_{V}$ to $T^{\ast}\in \Hom\left( \left( V^{\perp} \right)^{\ast},V^{\ast} \right) \cong U_{V^{\perp}}$, and identify subspaces $W\in U_{V}$ with their annihilators 
  \begin{align*}
    W^{0} &= \set{w^{\ast}\in \left( \R^n \right)^{\ast} | w^{\ast}\left( v \right) = 0\text{ for all }v\in W},
  \end{align*}
  so that $W^{0}\cap V^{\ast} = 0$. Finally, we define $\varphi_{V^{\perp}}$ by
  \begin{align*}
    \varphi_{V^{\perp}} &= P_{V^{\ast}}\circ P_{\left( V^{\perp} \right)^{\ast}}|_{W^{0}}^{-1}.
  \end{align*}
  Since every $W\in \operatorname{Gr}\left( k,n \right)$ has a unique annihilator subspace $W^{0}\in \operatorname{Gr}\left( n-k,n \right)$, we have the series of bijective correspondences
  \begin{align*}
    \Hom\left( V,V^{\perp} \right) &\xleftrightarrow{\varphi_{V}} U_{V}\\
                                   &\xleftrightarrow{W\leftrightarrow W^{0}} U_{V^{\perp}}\\
                                   &\xleftrightarrow{\varphi_{V^{\perp}}} \Hom\left( \left( V^{\perp} \right)^{\ast},V^{\ast} \right)\\
                                   &\xleftrightarrow{ \iprod{\cdot}{\cdot} } \Hom\left( V^{\perp},V \right),
  \end{align*}
  meaning that this identification is a $C^{\infty}$ diffeomorphism.
\end{solution}
\begin{problem}[Problem 3]
  Prove that a $C^{k}$ map which is a $C^{1}$ diffeomorphism is necessarily a $C^{k}$ diffeomorphism.
\end{problem}
\begin{solution}
  Let $f\colon \R^{n}\rightarrow \R^{n}$ be a $C^{k}$ map that is a $C^{1}$ diffeomorphism. In order to show that $f$ is a $C^{k}$ diffeomorphism, we need to show that $f^{-1}\colon \R^{n}\rightarrow \R^{n}$ exists and is of class $C^{k}$.\newline

  First, by the inverse function theorem, since $f$ is a $C^{1}$ diffeomorphism, we see that $f^{-1}\colon \R^{n}\rightarrow \R^{n}$ exists, is continuous, and is such that $D\left( f^{-1} \right)$ is continuous.\newline

  Now, we observe that the association $y\mapsto D_y\left( f^{-1} \right)$ can be written as
  \begin{align*}
    y \mapsto f^{-1}(y) \mapsto D_{y}f\left( f^{-1}(y) \right) \mapsto \left( D_yf\left( f^{-1}\left( y \right) \right) \right)^{-1} = D_y\left( f^{-1} \right),
  \end{align*}
  where we observe that $f^{-1}$ is of class $C^1$, the derivative $D_y f$ is of class $C^{k-1}$, and matrix inversion is $C^{\infty}$; since $D\left( f^{-1} \right)$ is a composition of $C^{1}$ functions, $D\left( f^{-1} \right)$ is $C^{1}$, so $f^{-1}$ is $C^{2}$. Inductively, we see that $f^{-1}$ is also of class $C^{k}$, so $f$ is a $C^{k}$ diffeomorphism.
\end{solution}
\begin{problem}[Problem 4]
  Recall that a topological space is paracompact if every open cover admits a locally finite refinement. Prove that a connected, paracompact manifold of dimension one is either $\R$ or $S^{1}$, depending on whether it is compact or not.
\end{problem}
\begin{solution}
  Let $M$ be a connected, paracompact manifold of dimension $1$, and let $\mathcal{A} = \set{\left( U_i,\varphi_i \right)}_{i\in I}$ be a locally finite atlas, where without loss of generality, each of the $U_i$ are connected, and $\varphi_i\left( U_i \right) = (0,1)$. We will show that this atlas allows us to define a homeomorphism between $M$ and either $S^{1}$ or $\R$.\newline

  Consider two open sets, $U_1$ and $U_2$ with respective charts $\varphi_1$ and $\varphi_2$. Suppose $U_1\cap U_2$ admits one connected component, and assume that $U_1\triangle U_2\neq \emptyset$. We will show that this allows us to, in a sense, ``amalgamate'' their respective coordinate maps $\varphi_1$ and $\varphi_2$, so that we may reduce to the case of two subsets. Since $U_1\cap U_2$ is an open subset of $U_1$, the coordinate map $\varphi_1\colon U_1\rightarrow (0,1)$ restricts to an embedding $\varphi_1\colon U_1\cap U_2\rightarrow (0,1)$. Note that since $\varphi_{1}$ is continuous, there is at most one cluster point for $\varphi_1\left( U_1\cap U_2 \right)$ within $(0,1)$, seeing as $\varphi_1$ is not defined on $U_2\setminus U_1$. Thus, we may assume that $\varphi_{1}\left( U_1\cap U_2 \right) = (b,1)$ for some $b\in (0,1)$. Similarly, we may assume that $\varphi_2\left( U_1\cap U_2 \right) = (0,a)$, so on $U_1\cup U_2$, we may define $\varphi_{1,2}\colon U_1\cup U_2\rightarrow (0,1)$ by $\varphi_{1}\left( U_1\setminus U_2 \right) = \left(0,a/(a+1)\right]$ and $\varphi_{2}\left( U_2 \right) = \left(a/(a+1),1\right)$, which is our desired amalgamation.\newline

  By taking a countable basis for the topology of $M$, using the fact that $\set{\left( U_i \right)}_{i\in I}$ is locally finite, and amalgamating the charts via this process for the finitely many elements of $\set{U_i}_{i\in I}$ that intersect elements of this topological basis, we may assume that the atlas $\mathcal{A}' = \set{\left( V_k,\psi_k \right)}_{k\geq 1}$ is countable. There are then two cases.\newline

  If $M$ is compact, then $M$ is covered by finitely many of these charts, $\set{\left( V_j,\psi_j \right)}_{j=1}^{n}$, so by using the amalgamation process once again, we are left with two charts. Without loss of generality, we call them $\left( V_1,\psi_1 \right)$ and $\left( V_2,\psi_2 \right)$. Observe that $V_1\cap V_2$  \textit{must} have two connected components; if there is one connected component, we may use this amalgamation process one more time, yielding a homeomorphism between the compact manifold $M$ and the non-compact interval $\left( 0,1 \right)$, a contradiction, and if there are no connected components, then $M$ is disconnected. Thus, if we are able to develop a continuous bijection between $M$ and $S^{1}$, since $S^{1}$ is Hausdorff and $M$ is compact, we automatically find that $M \cong S^{1}$.\newline

  From earlier, we know that if $W_1$ and $W_2$ are the connected components of $V_1\cap V_2$, then we may take $\psi_1\left( W_1 \right) = (0,a)$ and $\psi_1\left( W_2 \right) = (b,1)$. Similarly, we may take $\psi_2\left( W_2 \right) = (0,c)$ and $\psi_2\left( W_2 \right) = (d,1)$. We define the continuous bijection $r\colon M\rightarrow S^{1}$ piecewise, by taking
  \begin{align*}
    r(x) &= \begin{cases}
      \left( \cos\left( \pi \psi_1(x) \right),\sin\left( \pi\psi_1(x) \right) \right) & x\in V_1\\
      \left( \cos\left( \frac{\pi}{d-c} \psi_2(x) + \pi \right),\sin\left( \frac{\pi}{d-c}\psi_2(x) + \pi \right) \right) & x\in V_2\setminus V_1.
    \end{cases}
  \end{align*}
  If $M$ is not compact, then via some rearrangement, cutting, and compositions, we may assume that $V_k\cap V_{k+1}$ has one connected component, and $V_k\cap V_{k+n}$ for $n\geq 2$ has no connected components, and that $\psi_k\left( V_k \right) = (k,k+2)$ for each $k$. Then, we define $r^{\ast}\colon M\rightarrow (1,\infty)$ by
  \begin{align*}
    r(x) &= \begin{cases}
      \psi_1(x) & x\in V_1\\
      \psi_k(x) & x\in V_{k}\setminus V_{k-1}.
    \end{cases}
  \end{align*}
  This is a homeomorphism, so by composing with a homeomorphism between $\left( 1,\infty \right)$ and $\R$, we find that $M$ is homeomorphic to $\R$.
\end{solution}
\begin{problem}[Problem 5]
  In this problem, we prove a weak version of the Whitney Embedding Theorem.
  \begin{enumerate}[(a)]
    \item Find a $C^{\infty}$ function $\lambda$ on $\R^{n}$ with values in $[0,1]$ such that $\lambda$ takes the value $1$ on the closed ball $B\left( 0,1 \right)$, and vanishes outside the closed ball $B\left( 0,2 \right)$.
    \item Suppose $M$ is a compact $C^{k}$ manifold of dimension $n$. Find a $C^{k}$ atlas $\set{\left( U_i,\varphi_i \right)}_{i\in I}$ such that $\varphi_i\left( U_i \right)$ contains $B\left( 0,2 \right)$, and such that $M$ is covered by the union of $\varphi_i^{-1}\left( B\left( 0,1 \right) \right)^{\circ}$.
    \item Let $\lambda_i$ be defined by $\lambda\circ\varphi_i$ on $U_i$, and $0$ outside $U_i$. Let $f_i\colon M\rightarrow \R^n$ be defined by $\lambda_i\cdot\varphi_i$ on $U_i$ and zero otherwise. Use these functions to embed $M$ as a submanifold of some Euclidean space.
  \end{enumerate}
\end{problem}
\begin{solution}\hfill
  \begin{enumerate}[(a)]
    \item Consider the function $H\colon \R\rightarrow \R$ given by
      \begin{align*}
        H(t) &= \begin{cases}
          e^{-1/t} & t > 0\\
          0 & t\leq 0,
        \end{cases}
      \end{align*}
      which is a $C^{\infty}$ function on $\R$, as $e^{-1/t}$ is $C^{\infty}$ for all $t > 0$, and the derivative is well-defined at $t = 0$. Next, we see that the function
      \begin{align*}
        G(t) &= \frac{H\left( 4-t^2 \right)}{H\left( 4-t^2 \right) + H\left( t^2 - 1 \right)}
      \end{align*}
      takes on the value $1$ whenever $-1\leq t \leq 1$ and is supported on $[-2,2]$. Furthermore, it is a $C^{\infty}$ function, as it is a rational function of $C^{\infty}$ functions where the denominator never takes the value $0$. Therefore, if we replace $t$ with $\left\vert x \right\vert$, when $x \in \R^{n}$, since the norm is a $C^{\infty}$ function, we obtain a $C^{\infty}$ function that is supported on $B\left( 0,2 \right)$ and is equal to $1$ on $B\left( 0,1 \right)$, given by
      \begin{align*}
        \lambda\left( x \right) &= \frac{H\left( 4-\left\vert x \right\vert^2 \right)}{H\left( 4-\left\vert x \right\vert^2 \right) + H\left( \left\vert x \right\vert^2 - 1 \right)}.
      \end{align*}
    \item Let $M$ be a compact $C^{k}$ manifold, and let $\set{\left( V_i,\psi_i \right)}_{i\in I}$ be a $C^{k}$ atlas for $M$, where  $\set{V_i}_{i\in I}$ is an open cover, the $\psi_i\colon V_i\rightarrow \R^{n}$ are homeomorphisms, and the $\psi_{j}\circ \psi_{i}^{-1}$ are $C^{k}$ diffeomorphisms.\newline

      Since $M$ is compact, we have a finite subcover $\set{V_j}_{j=1}^{n}$ and an exhaustion by compact subsets via
      \begin{align*}
        U_{j} &= \bigcup_{k=1}^{j} V_k\\
        M &= \bigcup_{j=1}^{n}U_j,
      \end{align*}
      where, without loss of generality, $ \overline{U_{j}} \subsetneq U_{j+1}$.\newline

      Now, for each $p\in \overline{U_{j}}\setminus U_{j-1}$ (define $U_{0} = U_{1} = \emptyset$), we may find $i_p$ with a corresponding $C^{k}$ chart $\left( V_{i_p},\psi_{p} \right)$ mapping $\psi_{p}\left( V_{i_p} \right) = \R^{n}$. Without loss of generality, $\psi_{p}\left( p \right) = 0$ (compose with a translation if not), and let $W_p = \psi_{p}^{-1}\left( U\left( 0,1 \right) \right)$.\newline

      Clearly, $B\left( 0,2 \right)\subseteq \psi_{i_p}\left( V_{i_p} \right)$, and by finitely enumerating the elements $p_{j_k}$ in $ \overline{U_{j}}\setminus U_{j-1} $, we have an open cover $\set{W_{p_{j_k}}}_{k=1}^{m} = \set{\psi_{p_{j_k}}^{-1}\left( U\left( 0,1 \right) \right)}_{k=1}^{m}$ of $M$, and $\set{\left( V_{i_{p{j_k}}},\psi_{p_{j_k}} \right)}_{k=1}^{m}$ are $ C^{k}$ charts such that $B\left( 0,2 \right)\subseteq \psi_{p_{j_k}}\left( V_{i_{p{j_k}}} \right)$.
    \item We rename the finite atlas from part (b), $\set{\left( V_{i_{p_{j_k}}},\psi_{p_{j_k}} \right)}_{k=1}^{m}$, to $\set{\left( V_k,\psi_{k} \right)}_{k=1}^{m}$. Note that the $W_k = \psi_k^{-1}\left( U\left( 0,1 \right) \right)$ is the open cover we use to define $m$. We may redefine each $W_k$ to be equal to its closure.

      Now, if $f_k = \lambda_k\cdot \psi_k$, then by setting $g_k = \left( f_k,\lambda_k \right)$, we find that for any $x\in W_k$, $g_k(x) = \left( \psi_k(x),1 \right)$, so $g_k\left( W_k \right) = \left( \psi_k\left(W_k\right),1 \right)$, and if $x\notin W_k$, then $g_k(x) = \left( \psi_k(x),0 \right)$. It is clear that $g \colon M\rightarrow \R^{m\left( n+1 \right)}$ given by $ g\equiv \left( g_1,\dots,g_m \right)$ is continuous. It remains to show that $g$ is injective. To see this, if $x\neq y$, there are two cases:
      \begin{itemize}
        \item if $x,y\in W_k$, then since $\psi_k\colon V_k\rightarrow \R^{n}$ is a bijection, we must have $g_k(x)\neq g_k(y)$;
        \item if $x\in W_k$ and $y\notin W_k$, then since $\lambda_k(x) = 1$ and $\lambda_k(y) = 0$, we must have $g_k(x) \neq g_k(y)$.
      \end{itemize}
      Since the $W_k$ cover $M$, we must have that $g$ is injective. Thus, $M\hookrightarrow \R^{m\left( n+1 \right)}$ given by $x\mapsto g(x)$ is our desired embedding.
  \end{enumerate}
\end{solution}
\begin{problem}[Problem 6]
  Use the ideas of the previous exercise to prove that a $C^{k}$ manifold admits a $C^{k}$ partition of unity subordinate to any locally finite cover.
\end{problem}
\begin{solution}
  Let $\set{U_i}_{i\in I}$ be a locally finite open cover of $M$, and let $\set{\left( U_i,\varphi_i \right)}_{i\in I}$ be the corresponding $C^{k}$ atlas for $M$ where $B\left( 0,2 \right)\subseteq \varphi_{i} \left( U_i \right)$, and $M$ is covered by $\varphi_{i}^{-1}\left( U\left( 0,1 \right) \right)$. Then, we may define
  \begin{align*}
    f_i &= \begin{cases}
      G\circ \varphi_i & \text{on $U_i$}\\
      0 & \text{on $U_i^{c}$},
    \end{cases}
  \end{align*}
  where
  \begin{align*}
    G(x) &= \frac{e^{\frac{1}{4-\left\vert x \right\vert^2}}}{e^{\frac{1}{4-\left\vert x \right\vert^2}} + e^{ \frac{1}{\left\vert x \right\vert^2 - 1} }}
  \end{align*}
  is a $C^{\infty}$ function supported on $B\left( 0,2 \right)$ and equal to $1$ on $U\left( 0,1 \right)$. Defining
  \begin{align*}
    f &= \sum_{i\in I} f_i,
  \end{align*}
  we see that $f\neq 0$ everywhere, as $M$ is covered by the family $\varphi_{i}^{-1}\left( U\left( 0,1 \right) \right)$, where $G$ is nonzero on $U\left( 0,1 \right)$, and since $\set{U_i}_{i\in I}$ is locally finite, $f$ is also $C^{k}$ as each $f_i$ is the composition of a $C^{k}$ diffeomorphism and a $C^{\infty}$ function. The functions
  \begin{align*}
    g_i &= \frac{f_i}{f}
  \end{align*}
  are thus $C^{k}$ functions, $0\leq g_i \leq 1$, and $\sum_{i\in I} g_i = 1$.
\end{solution}
\begin{problem}[Problem 7]
  Let $X$ and $Y$ be topological spaces, and let $C\left( X,Y \right)$ be the set of continuous maps from $X$ to $Y$. Equip $C\left( X,Y \right)$ with the compact-open topology, where the basic open sets are
  \begin{align*}
    U_{K,V} &= \set{f | f(K)\subseteq V},
  \end{align*}
  where $K\subseteq X$ is compact and $V\subseteq Y$ is open.\newline

  If $Y$ is a metric space, and if $X$ is compact, prove that this topology is the same as the topology of uniform convergence.
\end{problem}
\begin{solution}
  Let $Y$ be a metric space and let $X$ be compact. We note that a neighborhood basis in the topology of uniform convergence on $C\left( X,Y \right)$ consists of sets of the form
  \begin{align*}
    U_{f,\ve} &= \set{g | \sup_{x\in X}d\left( f(x),g(x) \right) < \ve}.
  \end{align*}
  Similarly, a neighborhood basis for the compact open topology consists of sets of the form
  \begin{align*}
    U_{f,K,\ve} &= \set{g | \sup_{x\in K}d\left( f(x),g(x) \right) < \ve};
  \end{align*}
  the fact that $Y$ is a metric space allows us to take this refinement of the compact-open topology.\newline

  Thus, to prove that the compact-open topology and the topology of uniform convergence are equivalent, we show that any basis element of the topology of uniform convergence is contained in a basis element of the compact-open topology, and vice versa.\newline

  First, we see that almost by definition, if $K\subseteq X$ is any compact subset, then
  \begin{align*}
    U_{f,\ve} &\subseteq U_{f,K,\ve},
  \end{align*}
  as any function whose supremum distance is less than $\ve$ over $X$ must have that supremum distance hold over $K\subseteq X$.\newline

  Now, in the reverse direction, we fix $f$ and $\ve$. We wish to show that there is a finite family of subsets $U_{K_i,V_i}$ with $f\in U_{K_i,V_i}$ for each $i$, and their intersection lies in $U_{f,\ve}$. We see that every point $x\in X$ has a pre-compact open neighborhood $U_x$ such that $f\left( \overline{U_x} \right) \subseteq U\left( f(x),\ve/3 \right)$, which follows from the fact that compact subsets of $Y$ are bounded. The family $\set{U_x | x\in X}$ is an open cover for $X$, so admits a finite subcover $\set{U_{x_i}}_{i=1}^{n}$. Since each $ \set{\overline{U_{x_i}}}_{i=1}^{n} $ is compact, and for each $i$, $f\in U_{ \overline{U_{x_i}},U\left( f\left( x_i \right),\ve/3 \right) }$, we see that
  \begin{align*}
    V &= \bigcap_{i=1}^{n} U_{ \overline{U_{x_i}},U\left( f\left( x_i \right),\ve/3 \right) }
  \end{align*}
  is a nonempty open subset in the compact-open topology on $C\left( X,Y \right)$ that contains $f$. Now, for any $g\in V$ and for any $x\in X$, we see that there is some $U_{x_j}$ such that $x\in U_{x_j}$, and since $g\in U_{ \overline{U_{x_j}},U\left( f\left( x_j \right),\ve/3 \right) }$, we have that
  \begin{align*}
    d\left( g(x),f(x) \right) &\leq d\left( g(x),f\left( x_j \right) \right) + d\left( f\left( x_j \right),f(x) \right)\\
                              &< \ve/3 + \ve/3\\
                              &< \ve,
  \end{align*}
  so $V\subseteq U_{f,\ve}$, meaning the topologies are equal.
\end{solution}
%\begin{problem}[Problem 8]
%  Let $C^{k}\left( M,N \right)$ be the set of $C^{k}$ maps from $M$ to $N$. The compact-open topology on $C^{k}\left( M,N \right)$ is defined similarly. Let $f\in C^{k}\left( M,N \right)$, $\left( U,\varphi \right)$ and $\left( V,\psi \right)$ charts on $M$ and $N$, let $K\subseteq U$ be compact such that $f(K)\subseteq V$, and let $\ve > 0$. We obtain a basic neighborhood $N\left( f,U,\varphi,V,\psi,K,\ve \right)$ by looking at all the maps $g\in C^{k}\left( M,N \right)$ such that $g(K)\subseteq V$, and
%  \begin{align*}
%    \norm{D^{r}\left( \psi f \varphi^{-1} \right)\left( x \right) - D^{r}\left( \psi g \varphi^{-1} \right)\left( x \right)}_{\op} \leq \ve\label{eq:compact_open_manifolds}\tag{$\ast$}
%  \end{align*}
%  for all integers $0\leq r \leq k$.\newline
%
%  The Whitney topology is slightly different. Let $\Phi = \set{\left( U_i,\varphi_i \right)}_{i\in I}$ be a locally finite atlas on $M$, let $K_i\subseteq U_i$ be compact for all $i$, let $\Psi$ be an atlas on $N$, and let $\set{\ve_i}_{i\in I}$ be a family of positive numbers. A basic neighborhood of $f\in C^{k}\left( M,N \right)$ in this topology is given by all $g$ such that $g\left( K_i \right)\subseteq V_i$ for all $i$, and
%  \begin{align*}
%    \norm{D^r\left( \psi_i f \varphi_i^{-1} \right)\left( x \right) - D^{r}\left( \psi_i g \varphi_i^{-1} \right)\left( x \right)}_{\op} \leq \ve_i\label{eq:whitney_manifolds}\tag{$\ast\ast$}
%  \end{align*}
%  for all $x\in \varphi_i\left( K_i \right)$ and all integers $0\leq r \leq k$.\newline
%
%  For infinite values of $k$, we take the compact-open and Whitney topologies on $C^{\infty}\left( M,N \right)$ to be the union of these topologies via the inclusion $C^{\infty}\left( M,N \right)\subseteq C^{k}\left( M,N \right)$. Show the following:
%  \begin{enumerate}[(a)]
%    \item these basic neighborhoods actually give a basis for a topology in both cases;
%    \item if $M$ is compact, these two topologies coincide;
%    \item if $M$ is compact and has no boundary, then the $C^{k}$ diffeomorphisms from $M$ to $N$ are open in $C^{k}\left( M,N \right)$ in the Whitney topology.
%  \end{enumerate}
%\end{problem}
%\begin{solution}\hfill
%  \begin{enumerate}[(a)]
%    \item Clearly, in both the compact open topology and the Whitney topology, the respective neighborhoods cover $C^{k}\left( M,N \right)$, so we only need to verify the condition that if $X_1,X_2\subseteq C^{k}\left( M,N \right)$ are open subsets such that $f\in X_1\cap X_2$, then there is $X_3\subseteq C^{k}\left( M,N \right)$ open such that $X_3\subseteq X_1\cap X_2$.\newline
%
%     We start with the case of the compact-open topology. Let $f\in X_1\cap X_2$, where $X_1$ and $X_2$ are open in the compact-open topology. Since $f\in X_1$, there is a chart $\left( U_1,\varphi_1 \right)$ of $M$, a chart $\left( V_1,\psi_1 \right)$ of $N$, $K_1\subseteq U_1$ compact such that $f\left( K_1 \right)\subseteq V_1$, and $\ve_1 > 0$ such that \eqref{eq:compact_open_manifolds} holds and $N\left( f,U_1,\varphi_1,V_1,\psi_1,\ve_1 \right) \subseteq X_1$. Similarly, since $f\in X_2$, there are charts $\left( U_2,\varphi_2 \right)$ and $\left( V_2,\psi_2 \right)$ of $M$ and $N$ respectively, $K_2\subseteq U_2$ compact with $f\left( K_2 \right)\subseteq V_2$, and $\ve_2 > 0$ such that \eqref{eq:compact_open_manifolds} holds, and $N\left( f,U_2,\varphi_2,V_2,\psi_2,\ve_2 \right)\subseteq X_2$. Note that by the characterization, \eqref{eq:compact_open_manifolds} holds for the supremum over all $x\in \varphi_j\left( K_j \right)$ for $j = 1,2$.
%  \end{enumerate}
%\end{solution}
\end{document}
