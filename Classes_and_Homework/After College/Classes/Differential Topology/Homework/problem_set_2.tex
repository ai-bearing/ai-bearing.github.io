\documentclass[10pt]{mypackage}

% sans serif font:
%\usepackage{cmbright}
%\usepackage{sfmath}
%\usepackage{bbold} %better blackboard bold

%\usepackage{homework}
%\usepackage{notes}
\usepackage{newpxtext,eulerpx,eucal}
\renewcommand*{\mathbb}[1]{\varmathbb{#1}}

\fancyhf{}
\rhead{Avinash Iyer}
\lhead{Differential Topology: Problem Set 2}

\setcounter{secnumdepth}{0}

\begin{document}
\RaggedRight
\begin{problem}[Problem 1]
  A subset $A\subseteq \R^{n}$ is said to have \textit{measure zero} if, for all $\ve > 0$, the set $A$ can be covered by open balls of total volume at most $\ve$. Prove that a countable subset of $\R^{n}$ has measure zero, and that the standard middle-thirds cantor set in $[0,1]\subseteq \R$ has measure zero.
\end{problem}
\begin{solution}
  Let $A$ be countable, and let $\set{a_k}_{k\geq 1}$ be an enumeration of the points in $A$. Let $\ve > 0$. Let $c_{n}$ be the constant dependent on $n$ such that the volume of $U\left( x,r \right) = c_nr^{n}$. For each $k$, define
  \begin{align*}
    r_k &= \left( \frac{1}{2^{k}c_n} \ve \right)^{1/n}.
  \end{align*}
  Then, we see that the family $\set{U\left( a_k,r_k \right)}_{k=1}^{\infty}$ has total volume no more than $\ve$, seeing as if all the open balls are disjoint, their union has total volume $\ve$. Thus, countable subsets of $\R^{n}$ have measure zero.\newline

  If $\mathcal{C}\subseteq [0,1]$ is the traditional middle-thirds Cantor set, then we calculate the measure of its complement by taking
  \begin{align*}
    \frac{1}{3}\sum_{k=1}^{\infty} \left( \frac{2}{3} \right)^{k} &= \frac{1}{3} \frac{1}{1-\left( \frac{2}{3} \right)}\\
                                                                  &= 1,
  \end{align*}
  meaning that the Cantor set has measure zero.
\end{solution}
\begin{problem}[Problem 2]
  Prove that if $A\subseteq U\subseteq \R^{n}$ has measure zero (with $U$ open), and $f\colon U\rightarrow \R^{n}$ is smooth, show that $f(A)$ has measure zero.
\end{problem}
\begin{problem}[Problem 5]
  Prove that $\SL_2\left( \R \right)$, the $2\times 2$ real matrices of determinant one, is diffeomorphic to $\R^{2}\times S^{1}$.
\end{problem}
\begin{solution}
  We consider the action of $\SL_{2}\left( \R \right)$ on the upper half-plane of $\C$, $ \mathbb{H} = \set{z | \im(z) > 0} $, given by
  \begin{align*}
    \begin{pmatrix}a & b \\ c & d\end{pmatrix}\colon Z\mapsto \frac{az + b}{cz + d}.
  \end{align*}
  First, we may verify that this is a group action, and specifically it is one that preserves $ \mathbb{H} $.
\end{solution}
\end{document}
