\documentclass[10pt]{mypackage}

% sans serif font:
%\usepackage{cmbright}
%\usepackage{sfmath}
%\usepackage{bbold} %better blackboard bold

\usepackage{homework}
%\usepackage{notes}
\usepackage{newpxtext,eulerpx,eucal}
\renewcommand*{\mathbb}[1]{\varmathbb{#1}}

\fancyhf{}
\rhead{Avinash Iyer}
\lhead{Differential Topology: Problem Set 2}

\setcounter{secnumdepth}{0}

\begin{document}
\RaggedRight
\begin{problem}[Problem 1]
  A subset $A\subseteq \R^{n}$ is said to have \textit{measure zero} if, for all $\ve > 0$, the set $A$ can be covered by open balls of total volume at most $\ve$. Prove that a countable subset of $\R^{n}$ has measure zero, and that the standard middle-thirds cantor set in $[0,1]\subseteq \R$ has measure zero.
\end{problem}
\begin{solution}
  Let $A$ be countable, and let $\set{a_k}_{k\geq 1}$ be an enumeration of the points in $A$. Let $\ve > 0$. Let $c_{n}$ be the constant dependent on $n$ such that the volume of $U\left( x,r \right) = c_nr^{n}$. For each $k$, define
  \begin{align*}
    r_k &= \left( \frac{1}{2^{k}c_n} \ve \right)^{1/n}.
  \end{align*}
  Then, we see that the family $\set{U\left( a_k,r_k \right)}_{k=1}^{\infty}$ has total volume no more than $\ve$, seeing as if all the open balls are disjoint, their union has total volume $\ve$. Thus, countable subsets of $\R^{n}$ have measure zero.\newline

  If $\mathcal{C}\subseteq [0,1]$ is the traditional middle-thirds Cantor set, then we calculate the measure of its complement by taking
  \begin{align*}
    \frac{1}{3}\sum_{k=1}^{\infty} \left( \frac{2}{3} \right)^{k} &= \frac{1}{3} \frac{1}{1-\left( \frac{2}{3} \right)}\\
                                                                  &= 1,
  \end{align*}
  meaning that the Cantor set has measure zero.
\end{solution}
\begin{problem}[Problem 2]
  Prove that if $A\subseteq U\subseteq \R^{n}$ has measure zero (with $U$ open), and $f\colon U\rightarrow \R^{n}$ is smooth, show that $f(A)$ has measure zero.
\end{problem}
\begin{problem}[Problem 5]
  Prove that $\SL_2\left( \R \right)$, the $2\times 2$ real matrices of determinant one, is diffeomorphic to $\R^{2}\times S^{1}$.
\end{problem}
\begin{solution}
  We consider the action of $\SL_{2}\left( \R \right)$ on the upper half-plane of $\C$, $ \mathbb{H} = \set{z | \im(z) > 0} $, given by
  \begin{align*}
    \begin{pmatrix}a & b \\ c & d\end{pmatrix}\colon z\mapsto \frac{az + b}{cz + d}.
  \end{align*}
  In particular, if $z = x + iy$ with $y > 0$, then
  \begin{align*}
    \begin{pmatrix}a & b \\ c & d\end{pmatrix} z &= \frac{\left( ax + b \right) + iay}{\left( cx + d \right) + icy}\\
                     &= \frac{1}{\left( cx + d \right)^2 + c^2y^2} \left( \left( \left( ax + b \right)\left( cx + d \right) + acy^2 \right) + i\left( acxy - acxy + ady - bcy \right) \right)\\
                     &= \frac{1}{\left( cx + d \right)^2 + c^2y^2}\left( \left( \left( ax + b \right)\left( cx + d \right) + acy^2 \right) + iy \right),
  \end{align*}
  In particular, this is a fractional linear transformation on $\C$ that is an automorphism of $ \mathbb{H} $, so by composing these fractional linear transformations, we can see that $\SL_2\left( \R \right)$ acting on $ \mathbb{H} $ via this map is a group action.\newline

  This action is transitive, since for any $x + iy\in \mathbb{H}$, we may map $i\mapsto x + iy$ by using the transformation
  \begin{align*}
    \frac{ai + b}{ci + d} &= i
    \intertext{which via multiplication and matching parts gives}
    a &= cx + dy\\
    b &= xd-yc
    \intertext{so by multiplying and back-substituting, we get}
    c^2 + d^2 &= \frac{1}{y}.
    \intertext{By guessing that $c = 0$, we get}
    d &= \frac{1}{\sqrt{y}}\\
    a &= \sqrt{y}\\
    b &= \frac{x}{\sqrt{y}}.
  \end{align*}
  Now, to understand the stabilizer of some $z\in \mathbb{H}$, we only need to understand the stabilizer of $i$. For this, we see that
  \begin{align*}
    \frac{ai + b}{ci + d} &= i\\
    ai + b &= di-c
    \intertext{so}
    a &= d\\
    b &= -c,
    \intertext{and by back-substituting into the determinant, we get}
    a^2 + c^2 &= 1,
  \end{align*}
  so the stabilizer of $i$ is all matrices of the form
  \begin{align*}
    R &= \begin{pmatrix}\cos\left( \theta \right) & -\sin\left( \theta \right)\\\sin\left( \theta \right) & \cos\left( \theta \right)\end{pmatrix}.
  \end{align*}
  Thus, by orbit-stabilizer, $ \mathbb{H}\cong \SL_2\left( \R \right)/P $, where $P$ is the group of rotation matrices and the action is left-multiplication. In particular, since every rotation matrix corresponds one-to-one with an element of $S^{1}\subseteq \C$, given by
  \begin{align*}
    \begin{pmatrix}\cos\left( \theta \right) & -\sin\left( \theta \right)\\ \sin\left( \theta \right) & \cos\left( \theta \right)\end{pmatrix} &\mapsto e^{i\theta},
  \end{align*}
  we find that $ \mathbb{H} \cong \SL_2\left( \R \right)/S^{1}$, or that $ \mathbb{H}\times S^{1}\cong \SL_2\left( \R \right) $. 
\end{solution}
\end{document}
