\documentclass[10pt]{mypackage}

% sans serif font:
%\usepackage{cmbright}
%\usepackage{sfmath}
%\usepackage{bbold} %better blackboard bold

\usepackage{homework}
%\usepackage{notes}
\usepackage{newpxtext,eulerpx,eucal}
\renewcommand*{\mathbb}[1]{\varmathbb{#1}}

\fancyhf{}
\rhead{Avinash Iyer}
\lhead{Differential Topology: Problem Set 6}

\setcounter{secnumdepth}{0}

\begin{document}
\RaggedRight
\begin{problem}[Problem 5]
  A smooth map $f\colon M\rightarrow n$ is called a submersion if it induces surjections on tangent spaces. Prove that if $M$ and $N$ are smooth manifolds and $A\subseteq N$ is a smooth submanifold, then $f$ is transverse to $A$.
\end{problem}
\begin{solution}
  Let $p\in f^{-1}\left( A \right)$. By the definition of the submersion, we have $T_{F(p)}N = D_pF\left( T_pM \right)$, meaning that $D_pF\left( T_pM \right) + T_{F(p)}A = T_{F(p)}N$.
\end{solution}

\end{document}
