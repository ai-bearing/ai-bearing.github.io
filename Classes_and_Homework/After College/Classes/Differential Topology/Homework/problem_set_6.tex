\documentclass[10pt]{mypackage}

% sans serif font:
%\usepackage{cmbright}
%\usepackage{sfmath}
%\usepackage{bbold} %better blackboard bold

\usepackage{homework}
%\usepackage{notes}
\usepackage{newpxtext,eulerpx,eucal}
\renewcommand*{\mathbb}[1]{\varmathbb{#1}}

\fancyhf{}
\rhead{Avinash Iyer}
\lhead{Differential Topology: Problem Set 6}

\setcounter{secnumdepth}{0}

\begin{document}
\RaggedRight
\begin{problem}[Problem 5]
  A smooth map $f\colon M\rightarrow n$ is called a submersion if it induces surjections on tangent spaces. Prove that if $M$ and $N$ are smooth manifolds and $A\subseteq N$ is a smooth submanifold, then $f$ is transverse to $A$.
\end{problem}
\begin{solution}
  Let $p\in f^{-1}\left( A \right)$. By the definition of the submersion, we have $T_{F(p)}N = D_pF\left( T_pM \right)$, meaning that $D_pF\left( T_pM \right) + T_{F(p)}A = T_{F(p)}N$.
\end{solution}
\begin{problem}[Problem 6]
  In this exercise, we will prove a version of the Transversality Theorem. Let $M$ and $N$ be smooth manifolds. The transversality theorem asserts that for all $1\leq r \leq \infty$, the set of $C^{r}$ maps $M\rightarrow N$ that are transverse to $A$ is dense in any of the natural topologies $C^{r}\left( M,N \right)$.\newline

  We will restrict our attention to manifolds embedded in Euclidean space and prove a slightly weaker version of the transversality theorem.
  \begin{enumerate}[(a)]
    \item Let $M$, $N$, and $A$ be as above, and let $Y$ be an arbitrary smooth manifold. Let $F\colon Y\times M\rightarrow N$ be a smooth map transverse to $A$. For each $y\in Y$, let $f_y\colon M\rightarrow N$ be defined by $F\left( y,\cdot \right)$, and let $\pi\colon Y\times M\rightarrow Y$ be the projection.\newline

      Prove that for every regular value $y\in Y$ of $\pi$, the map $f_y$ is transverse to $A$.
    \item Let $f\colon M\rightarrow \R^{n}$ be a smooth map, and let $A\subseteq \R^{n}$ be a smooth submanifold. Show that the set of $p\in \R^{n}$ for which $f_p(x) \coloneq f(x) + p$ is not transverse to $A$ has measure zero.
    \item Prove that if $M$ and $N$ are smooth submanifolds of $\R^n$, then for all $p\in \R^{n}$ outside a set of measure zero, the manifolds $M + p$ and $N$ intersect transversely.
    \item Prove that if $f\colon M\rightarrow N$ is smooth, and $A\subseteq N$ is a smooth submanifold, then $f$ is smoothly homotopic to a map that is transverse to $A$.
  \end{enumerate}
\end{problem}
\begin{solution}\hfill
  \begin{enumerate}[(a)]
    \item Let $p\in A$, and let $y$ be a regular value for $\pi$. Observe that, by the regular value theorem, we have that $\pi^{-1}\left( y \right) = \set{y}\times M$ is a smooth submanifold of $Y\times M$. It follows from the definition of the $f_y$ that $F\circ \pi^{-1}\left( y \right) \equiv f_y$.\newline

      Since $F$ is transverse to $A$, it follows that for any $\left( z,q \right)\in F^{-1}\left( p \right)$, we have
      \begin{align*}
        D_{(z,q)}F\left( T_{(z,q)}\left( Y\times M \right) \right) + T_pA &= T_pN.
      \end{align*}
      We have, by chain rule and the inverse function theorem (seeing as $y$ is a regular value of $\pi$),
      \begin{align*}
        D_qf_y &= D_q\left( F\circ\pi^{-1}(y) \right)\\
               &= D_{(y,q)}F\circ \left( D_{\pi^{-1}(y)}\pi \right)^{-1}(y)\\
               &= D_{(y,q)}F,
      \end{align*}
      so that
      \begin{align*}
        D_qf_y\left( T_{q}M \right) + T_pA &= D_{(y,q)}F\left( T_{(y,q)}\left( Y\times M \right) \right) + T_pA\\
                                           &= T_pN,
      \end{align*}
      meaning $f_y$ is transverse to $A$ for any regular value $y\in Y$ of $\pi$.
    \item If we let $Y \equiv \R^{n}$ in part (a), and let $F\colon \R^{n}\times M \rightarrow \R^{n}$ be defined by $F\left( p,x \right) = f(x) + p$, then we observe that for every regular value $p$ of $\pi$, that $f(x) + p$ is transverse to $A$. In particular, since the set of critical values has measure zero in $\R^{n}$, it follows that for almost every $p$, $f(x) + p$ is transverse to $A$.
  \end{enumerate}
\end{solution}
\end{document}
