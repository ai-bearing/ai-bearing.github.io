\documentclass[10pt]{mypackage}

% sans serif font:
%\usepackage{cmbright}
%\usepackage{sfmath}
%\usepackage{bbold} %better blackboard bold

\usepackage{homework}
%\usepackage{notes}
\usepackage{newpxtext,eulerpx,eucal}
\renewcommand*{\mathbb}[1]{\varmathbb{#1}}

\fancyhf{}
\fancyhead[R]{Avinash Iyer}
\fancyhead[L]{Differential Topology: Problem Set 10}
\fancyfoot[C]{\thepage}

\setcounter{secnumdepth}{0}

\begin{document}
\RaggedRight
\begin{problem}[Problem 1]
  Let $G$ be a Lie group, which is a topological group that is also a smooth manifold and where all group operations are smooth. For convenience, we will always assume that $G$ is path-connected. Prove that the tangent bundle $TG$ of $G$ is trivial --- i.e., $TG$ composes as a direct product.
\end{problem}
\begin{solution}
  From Cayley's Theorem, we know that $G$ acts on itself transitively by left-multiplication. That is, for any $g\in G$, there is a map $L_g\colon G\rightarrow G$ that takes $h\mapsto gh$. This is a diffeomorphism of smooth manifolds since $L_g$ is smooth and admits the smooth inverse $L_{g^{-1}}$. In particular, this means that
  \begin{align*}
    D_{e}\left( L_g \right)\colon T_{e}G \rightarrow T_gG
  \end{align*}
  is invertible as a linear map. Letting $T_eG\cong \R^{n}$ have a local basis $\mathcal{B}_e = \set{\pd{}{x_1},\dots, \pd{}{x_n}}$, we then observe that $D_e\left( L_g \right)$ then maps this basis to a basis for $T_gG$ since $D_e\left( L_g \right)$ is a linear isomorphism, meaning that
  \begin{align*}
    TG &= \bigsqcup_{g\in G} T_gG\\
       &= \bigsqcup_{g\in G} D_e\left( L_g \right)\left( T_eG \right)\\
       &\cong \bigsqcup_{g\in G} \R^{n}\\
       &\cong G\times \R^{n}.
  \end{align*}
  Thus, $TG$ is trivial.
\end{solution}
\begin{problem}[Problem 2]
  Note that a Lie group can act on itself by left or right multiplication. A vector field on $G$ is called \textit{left-invariant} if it is invariant under the differential of left multiplication $L_g$ for every $g\in G$. Prove that $T_eG$ can be identified with left invariant vector fields on $G$.
\end{problem}
\begin{solution}
  We observe that by definition, a left-invariant vector field $X$ is one where $g\cdot X = X$ for every $g\in G$. In particular, this means that for any vector field $X_e\in T_eG$, there is a corresponding left-invariant vector field on $G$ defined at each $g\in G$ by taking $X_g = D_e\left( L_{g} \right)\left( X_e \right)$; that such a vector field is left-invariant follows from the fact that $L_g$ is a diffeomorphism of $G$ onto itself. Thus, we get the correspondence between vector fields at $T_eG$ and left-invariant vector fields on $G$.
\end{solution}
\begin{problem}[Problem 3]
  Similar to invariant vector fields, invariant forms are ones for which $L_g^{\ast}\omega = \omega$. Prove that invariant forms are stable under taking $d$ and under contraction by a left-invariant vector field.
\end{problem}
\begin{solution}
  Let $\omega$ be left-invariant. Then, by definition of the pullback,
  \begin{align*}
    L_g^{\ast}\left( d\omega \right) &= d\left( L_g^{\ast}\omega \right)\\
                                     &= d\omega.
  \end{align*}
  Similarly, by definition of the contraction, if $ \pd{}{x_1},\dots,\pd{}{x_k} $ are a $k$-dimensional collection of vector fields, then
  \begin{align*}
    L_g^{\ast}\left( \iota_X\left( \omega \right) \right)\left( \pd{}{x_1},\dots,\pd{}{x_k} \right) &= L_g^{\ast}\left( \omega\left( X,\pd{}{x_1},\dots,\pd{}{x_k} \right) \right)\\
                                                                                                    &= \left( L_g^{\ast}\omega \right)\left( X,\pd{}{x_1},\dots,\pd{}{x_k} \right)\\
                                                                                                    &= \omega\left( X,\pd{}{x_1},\dots,\pd{}{x_k} \right)\\
                                                                                                    &= \iota_X\left( \omega \right)\left( \pd{}{x_1},\dots,\pd{}{x_k} \right).
  \end{align*}

\end{solution}
\end{document}
