\documentclass[10pt]{mypackage}

% sans serif font:
%\usepackage{cmbright}
%\usepackage{sfmath}
%\usepackage{bbold} %better blackboard bold

\usepackage{homework}
%\usepackage{notes}
\usepackage{newpxtext,eulerpx,eucal}
\renewcommand*{\mathbb}[1]{\varmathbb{#1}}

\fancyhf{}
\fancyhead[R]{Avinash Iyer}
\fancyhead[L]{Differential Topology: Problem Set 10}
\fancyfoot[C]{\thepage}

\setcounter{secnumdepth}{0}

\begin{document}
\RaggedRight
\begin{problem}[Problem 1]
  Let $G$ be a Lie group, which is a topological group that is also a smooth manifold and where all group operations are smooth. For convenience, we will always assume that $G$ is path-connected. Prove that the tangent bundle $TG$ of $G$ is trivial --- i.e., $TG$ composes as a direct product.
\end{problem}
\begin{solution}
  From Cayley's Theorem, we know that $G$ acts on itself transitively by left-multiplication. That is, for any $g\in G$, there is a map $L_g\colon G\rightarrow G$ that takes $h\mapsto gh$. This is a diffeomorphism of smooth manifolds since $L_g$ is smooth and admits the smooth inverse $L_{g^{-1}}$. In particular, this means that
  \begin{align*}
    D_{e}\left( L_g \right)\colon T_{e}G \rightarrow T_gG
  \end{align*}
  is invertible as a linear map. Letting $T_eG\cong \R^{n}$ have a local basis $\mathcal{B}_e = \set{\pd{}{x_1},\dots, \pd{}{x_n}}$, we then observe that $D_e\left( L_g \right)$ then maps this basis to a basis for $T_gG$ since $D_e\left( L_g \right)$ is a linear isomorphism, meaning that
  \begin{align*}
    TG &= \bigsqcup_{g\in G} T_gG\\
       &= \bigsqcup_{g\in G} D_e\left( L_g \right)\left( T_eG \right)\\
       &\cong \bigsqcup_{g\in G} \R^{n}\\
       &\cong G\times \R^{n}.
  \end{align*}
  Thus, $TG$ is trivial.
\end{solution}
\begin{problem}[Problem 2]
  Note that a Lie group can act on itself by left or right multiplication. A vector field on $G$ is called \textit{left-invariant} if it is invariant under the differential of left multiplication $L_g$ for every $g\in G$. Prove that $T_eG$ can be identified with left invariant vector fields on $G$.
\end{problem}
\begin{solution}
  We observe that by definition, a left-invariant vector field $X$ is one where $g\cdot X = X$ for every $g\in G$. In particular, this means that for any vector field $X_e\in T_eG$, there is a corresponding left-invariant vector field on $G$ defined at each $g\in G$ by taking $X_g = D_e\left( L_{g} \right)\left( X_e \right)$; that such a vector field is left-invariant follows from the fact that $L_g$ is a diffeomorphism of $G$ onto itself. Thus, we get the correspondence between vector fields at $T_eG$ and left-invariant vector fields on $G$.
\end{solution}
\begin{problem}[Problem 3]
  Similar to invariant vector fields, invariant forms are ones for which $L_g^{\ast}\omega = \omega$. Prove that invariant forms are stable under taking $d$ and under contraction by a left-invariant vector field.
\end{problem}
\begin{solution}
  Let $\omega$ be left-invariant. Then, by definition of the pullback,
  \begin{align*}
    L_g^{\ast}\left( d\omega \right) &= d\left( L_g^{\ast}\omega \right)\\
                                     &= d\omega.
  \end{align*}
  Similarly, by definition of the contraction, if $ \pd{}{x_1},\dots,\pd{}{x_k} $ are a $k$-dimensional collection of vector fields, then
  \begin{align*}
    L_g^{\ast}\left( \iota_X\left( \omega \right) \right)\left( \pd{}{x_1},\dots,\pd{}{x_k} \right) &= L_g^{\ast}\left( \omega\left( X,\pd{}{x_1},\dots,\pd{}{x_k} \right) \right)\\
                                                                                                    &= \left( L_g^{\ast}\omega \right)\left( X,\pd{}{x_1},\dots,\pd{}{x_k} \right)\\
                                                                                                    &= \omega\left( X,\pd{}{x_1},\dots,\pd{}{x_k} \right)\\
                                                                                                    &= \iota_X\left( \omega \right)\left( \pd{}{x_1},\dots,\pd{}{x_k} \right).
  \end{align*}
\end{solution}
\begin{problem}[Problem 4]
  Similar to left-invariant forms are right-invariant forms. Prove that a connected compact Lie group admits a bi-invariant volume form.
\end{problem}
\begin{solution}
  Let $G$ be a compact connected Lie group, and define
  \begin{align*}
    L\colon G\times G \rightarrow G\\
    R\colon G\times G \rightarrow G
  \end{align*}
  by
  \begin{align*}
    L\left( g,x \right) &= gx\\
    R\left( g,x \right) &= xg^{-1}.
  \end{align*}
  Letting $L_g \coloneq L\left( g,\cdot \right)$ and $R_g \coloneq R\left( g,\cdot \right)$, we observe that $L$ and $R$ define faithful, free actions of $G$ by regular permutations, with $R$ following from the fact that 
  \begin{align*}
    R_g\circ R_h(x) &= R_g\left( xh^{-1} \right)\\
                    &= xh^{-1}g^{-1}\\
                    &= x\left( gh \right)^{-1}\\
                    &= R_{gh}\left( x \right).
  \end{align*}
  For any $g\in G$, we may then define a linear map $T_g\colon T_eG\rightarrow T_eG$ by taking $x\mapsto \left( D_gR_g \right)\circ \left( D_eL_g \right)(x)$. We observe that $T_g$ transforms the left-invariant vector fields in $T_eG$ by right-multiplication, since if $X\in T_eG$, then
  \begin{align*}
    T_g\left( X \right) &= D_gR_g\left( \left( D_eL_g \right)\left( Y \right) \right)\\
                        &= \left( D_gR_g \right)\left( Y \right).
  \end{align*}
  Now, let $\nu\in \Lambda^{n}T_e^{\ast}G$ be a non-zero max-dimensional form at $T_eG$, and let $X_1,\dots,X_n$ be a basis for $T_eG$. We claim that $\omega\in \Lambda^{n}T_e^{\ast}G$ defined by
  \begin{align*}
    \omega\left( X_1,\dots,X_n \right)(e)\coloneq \nu\left( D_eR_g\left(X_1\right),\dots,D_eR_g(X_n) \right)\left( g^{-1} \right)
  \end{align*}
  is a right-invariant form. Towards this end, we observe that
  \begin{align*}
    R_h^{\ast}\omega\left( X_1,\dots,X_n \right)(e) &= R_h^{\ast}\nu\left( D_eR_g\left( X_1 \right),\dots,D_eR_g\left( X_n \right) \right)\left( g^{-1} \right)\\
                                                    &= \nu\left( D_{g^{-1}}R_h\circ D_eR_g\left( X_1 \right),\dots,D_{g^{-1}}R_h\circ D_eR_g\left( X_n \right) \right)\left( g^{-1}h^{-1} \right)\\
                                                    &= \nu\left( D_{e}\left( R_h\circ R_g \right)\left( X_1 \right),\dots,D_e\left( R_h\circ R_g \right)\left( X_n \right) \right)\left( \left( gh \right)^{-1} \right)\\
                                                    &= \omega\left( X_1,\dots,X_n \right)(e).
  \end{align*}
  Having constructed now a right-invariant form, we observe that $L_g^{\ast}\left( \omega \right)$ is a map from $\Lambda^{n}T_e^{\ast}G$ to $\Lambda^{n}T_e^{\ast}G$, so by the definition of the determinant, it follows that $L_g^{\ast}\left( \omega \right) = \det\left( L_g^{\ast} \right) \omega$. Since $L_g$ is a diffeomorphism, it follows that $\det\left( L_g^{\ast} \right) = \pm 1$, but since the family of $L_g$ also form a group, and $\det\colon \operatorname{end}\left( T_eG \right)\rightarrow \R$ is a group homomorphism, we must have that $\det\left( L_g^{\ast} \right) = 1$, so $\omega$ is in fact bi-invariant.
\end{solution}
\begin{problem}[Problem 5]
  Fix a bi-invariant volume form $\mu$ on a compact connected lie group $G$ whose total mass is $1$. For $\omega$ an arbitrary $k$-form on $G$, and $k$ left-invariant vector fields $X_1,\dots,X_k$, show that
  \begin{align*}
    \rho\left( \omega \right)\left( X_1,\dots,X_k \right) &= \int_{G}^{} L_g^{\ast}\omega\left( X_1,\dots,X_k \right)\:\mu
  \end{align*}
  furnishes a left-invariant form on $G$, and that $\rho\left( \omega \right) = \omega$ if $\omega$ was already left-invariant.
\end{problem}
\begin{solution}
  Let $p\in G$, and consider
  \begin{align*}
    L_p^{\ast}\rho\left( \omega \right)\left( X_1,\dots,X_k \right)\left( q \right) &= \rho\left( \omega \right)\left( D_pL_p\left( X_1 \right),\dots,D_pL_p\left( X_k \right) \right)\left( pq \right)\\
                                                                                    &= \int_{G}^{} L_g^{\ast}\omega\left( D_pL_p\left( X_1 \right)\left( pq \right),\dots,D_pL_p\left( X_k \right)\left( pq \right) \right)\:\mu\\
                                                                                    &= \int_{G}^{} L_p^{\ast}L_g^{\ast}\omega\left( X_1,\dots,X_k \right)(q)\:mu
                                                                                    \intertext{Using the change of coordinates by the diffeomorphism $L_{p^{-1}}$, we get}
                                                                                    &= \int_{G}^{} L_{p p^{-1} g}\left( X_1,\dots,X_k \right)(q)\:L_{p^{-1}}^{\ast}\mu
                                                                                    \intertext{and since $\mu$ is bi-invariant, using the fact that $L$ is a group action, we get}
                                                                                    &= \int_{G}^{} L_g^{\ast}\omega\left( X_1,\dots,X_k \right)(q)\:\mu\\
                                                                                    &= \rho\left( \omega \right)\left( X_1,\dots,X_k \right)\left( q \right).
  \end{align*}
  Since $q$ is arbitrary, we are done.
\end{solution}
\begin{problem}[Problem 6]
  Show that $\rho$ and $d$ commute.
\end{problem}
\begin{solution}
  We observe that
  \begin{align*}
    d\left( \rho\left( \omega \right) \right) &= d\left( \int_{G}^{} L_g^{\ast}\omega\:\mu \right)\\
                                              &= \int_{G}^{} dL_g^{\ast}\omega\:\mu\\
                                              &= \int_{G}^{} L_g^{\ast}\left( d\omega \right)\:\mu\\
                                              &= \rho\left( d\omega \right).
  \end{align*}
\end{solution}
\end{document}
