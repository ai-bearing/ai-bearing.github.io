
\documentclass[10pt]{mypackage}

% sans serif font:
%\usepackage{cmbright}
%\usepackage{sfmath}
%\usepackage{bbold} %better blackboard bold

\usepackage{homework}
%\usepackage{notes}
\usepackage{newpxtext,eulerpx,eucal}
\renewcommand*{\mathbb}[1]{\varmathbb{#1}}

\fancyhf{}
\fancyhead[R]{Avinash Iyer}
\fancyhead[L]{Differential Topology: Problem Set 7}
\fancyfoot[C]{\thepage}

\setcounter{secnumdepth}{0}

\begin{document}
\RaggedRight
\begin{problem}[Problem 2]
  Prove the claim from class that the open star cover of a simplicial complex is good.
\end{problem}
\begin{solution}
  Let $X$ be the simplicial complex for $M$. We start by observing that an open $n$-simplex contained in $\R^{n}$ is itself contractible, as it is convex, hence there is a straight-line homotopy to any point in its interior.\newline

  Now, if $v\in X$ is a vertex, then the open star for $v$ consists of finitely many open simplices that contain $v$, whence it is possible to contract via a straight-line homotopy to $v$. If $v$ is not a vertex of $X$, then $v$ is contained in some open $n$-simplex, so it is once again possible to straight-line homotopy to $v$.\newline

  Finally, we observe that any non-empty $\left( k+1 \right)$-fold intersection of open stars in $\mathcal{U}$ defines an open $k$-simplex of $X$, whence every point on the interior of the $k$-simplex can be contracted to the given point via the straight-line homotopy once again.\newline

  Thus, the open star cover of $M$ is good.
\end{solution}
\begin{problem}[Problem 3]
  Let $\omega$ be a closed $k$-form on a closed manifold $M$ of dimension $n$, and let $\eta$ be a closed $\left( n-k \right)$-form on $M$. Prove that if $\omega\wedge\eta$ is nonzero at every point of $M$, then $\omega$ is nonvanishing in $H^{k}_{\operatorname{DR}}\left( M \right)$.
\end{problem}
\begin{solution}
  Since $\omega\wedge\eta$ is nonvanishing, it follows from the definitions that $\Lambda^{n}T^{\ast}M$ admits a smooth nonvanishing section, meaning that $M$ admits an orientation. In particular, integration on $M$ is well-defined.\newline

  Thus, we may show that $\omega\wedge\eta$ is not exact by seeing that if there were some $\left( n-1 \right)$-dimensional form $\xi$ such that $d\xi = \omega\wedge\eta$, then
  \begin{align*}
    \int_{M}^{}d\xi &= \int_{\partial M}^{} \xi\\
                    &= 0,
  \end{align*}
  which would be a contradiction as $\omega\wedge\eta$ is nonvanishing. We observe then that, if $\omega = d\tau$ for some $\tau\in \mathcal{A}^{k-1}\left( M \right)$, then
  \begin{align*}
    \omega\wedge\eta &= d\tau\wedge\eta\\
                     &= d\tau\wedge\eta + \left( -1 \right)^{k-1} \tau\wedge d\eta\\
                     &= d\left( \tau\wedge\eta \right),
  \end{align*}
  which would be a contradiction.
\end{solution}
\begin{problem}[Problem 4]
  Compute the de Rham cohomology of $\R^{2}\setminus \set{0}$, and find representatives of all nontrivial classes.
\end{problem}
\begin{solution}
  We observe that $\R^{2}\setminus \set{0} \cong S^{1}\times \R$, so by the Poincaré lemma, we have
  \begin{align*}
    H^{\ast}_{\operatorname{DR}}\left( \R^{2}\setminus \set{0} \right) &\cong H^{\ast}_{\operatorname{DR}}\left( S^{1} \right)
    \intertext{or}
    H^{0}_{\operatorname{DR}}\left( \R^{2}\setminus\set{0} \right) &\cong \R\\
    H^{1}_{\operatorname{DR}}\left( \R^{2}\setminus\set{0} \right) &\cong \R\\
    H^{k}_{\operatorname{DR}}\left( \R^{2}\setminus\set{0} \right) &\cong 0\text{ for $k\geq 2$}.
  \end{align*}
  We know that a complete set of representatives for cohomology classes of $S^{1}$ are $1$ for $H^{0}$ and $d\theta$ for $H^{1}$. We know from the lemma that then, $d\theta$ corresponds to $\pi^{\ast}\left( d\theta \right)$, where $\pi\colon S^{1}\times \R\rightarrow S^{1}$ is the projection. Thus, we observe that $\set{1,\pi^{\ast}\left( d\theta \right)}$ is the complete set of representatives of cohomology classes for $H^{\ast}_{\operatorname{DR}}\left( \R^{2}\setminus \set{0} \right)$.
\end{solution}
\begin{problem}[Problem 5]
  Let $G$ be a finite group acting freely on a manifold $M$ by diffeomorphisms. Show that:
  \begin{itemize}
    \item $M/G$ is a manifold;
    \item the de Rham cohomology of $M/G$ is isomorphic to the $G$-invariant cohomology of $M$.
  \end{itemize}
\end{problem}
\begin{solution}\hfill
  \begin{enumerate}[(i)]
    \item We observe that the quotient map $\pi\colon M\rightarrow M/G$, taking $p\mapsto \left[ p \right]$ is a covering map. This follows from the fact that for any $p\in M$, there is a sufficiently small $U\subseteq M$ such that $g\cdot U\cap U = \emptyset$ for all $g\in G$ with $g\neq e$, as $G$ is finite and the action of $G$ on $M$ is free.\newline

      Let $\varphi\colon U\rightarrow \R^{n}$ be a coordinate map for $p\in U\subseteq M$, where $U$ is as above (where $g\cdot U \cap U = \emptyset$). An open neighborhood of $\left[ p \right]\in M/G$, where $M/G$ is endowed with the quotient topology, thus admits $\varphi^{\ast}\colon U^{\ast}\rightarrow \R^{n}$ by taking $\varphi^{\ast}\left( U^{\ast} \right) = \varphi\left( \pi^{-1}\left( U^{\ast} \right)\cap U \right)$. Therefore, $M/G$ admits a manifold structure.
  \end{enumerate}
\end{solution}
\begin{problem}[Problem 6]
  Let $U$ and $V$ be open subsets of a smooth manifold $M$, and let $W = U\cup V$. Write $i_U,i_V$ for the inclusions of $U$ and $V$ into $W$ respectively, and write $j_U,j_V$ for the inclusions of $U\cap V$ into $U$ and $V$ respectively. Show that the sequence 
  \begin{center}
% https://q.uiver.app/#q=WzAsNSxbMCwwLCIwIl0sWzIsMCwiXFxtYXRoY2Fse0F9XmsoVykiXSxbNCwwLCJcXG1hdGhjYWx7QX1ee2t9KFUpXFxvcGx1c1xcbWF0aGNhbHtBfV57a30oVikiXSxbNiwwLCJcXG1hdGhjYWx7QX1ee2t9KFVcXGNhcCBWKSJdLFs4LDAsIjAiXSxbMCwxXSxbMSwyLCJcXGxlZnQoaV9VLGlfVlxccmlnaHQpIl0sWzIsMywial9VXntcXGFzdH0tal9WXntcXGFzdH0iXSxbMyw0XV0=
\[\begin{tikzcd}
	0 && {\mathcal{A}^k(W)} && {\mathcal{A}^{k}(U)\oplus\mathcal{A}^{k}(V)} && {\mathcal{A}^{k}(U\cap V)} && 0
	\arrow[from=1-1, to=1-3]
	\arrow["{\left(i_U^{\ast},i_V^{\ast}\right)}", from=1-3, to=1-5]
	\arrow["{j_U^{\ast}-j_V^{\ast}}", from=1-5, to=1-7]
	\arrow[from=1-7, to=1-9]
\end{tikzcd}\]
  \end{center}
  is exact.
\end{problem}
\begin{solution}
  Exactness at $\mathcal{A}^{k}\left( W \right)$ follows from the fact that $\left( i_U^{\ast},i_V^{\ast} \right)$ is an inclusion map, hence has kernel $0$.\newline

  To verify that the sequence is exact at $\mathcal{A}^{k}\left( U \right)\oplus \mathcal{A}^{k}\left( V \right)$, we observe that if $\omega\in \mathcal{A}^{k}\left( W \right)$, then $\left( \omega|_{U},\omega|_{V} \right)$ yields zero when subjected to $j_U^{\ast}-j_V^{\ast}$ as $\omega$ when restricted to $U\cap V$ is equal to itself. Therefore, the sequence is exact at $\mathcal{A}^{k}\left( U \right)\oplus \mathcal{A}^{k}\left( V \right)$.\newline

  Finally, we let $\set{f_U,f_V}$ be a partition of unity for $W$ subordinate to $\set{U,V}$. If $\omega\in \mathcal{A}^{k}\left( U\cap V \right)$, we observe that $f_U\omega$ extends to $0$ on $V\setminus \left( U\cap V \right)$, whence $f_U\omega\in \mathcal{A}^{k}\left( V \right)$, and similarly for $f_V\omega\in \mathcal{A}^{k}\left( U \right)$. Therefore, $\left( f_V\omega,-f_U\omega \right)\in \mathcal{A}^{k}\left( U \right)\oplus \mathcal{A}^{k}\left( V \right)$ maps to $\omega\in \mathcal{A}^{k}\left( U\cap V \right)$, meaning $j_U^{\ast}-j_V^{\ast}$ is surjective, so the sequence is exact at $\mathcal{A}^{k}\left( U\cap V \right)$.
\end{solution}
\end{document}
