\documentclass[10pt]{mypackage}

% sans serif font:
%\usepackage{cmbright}
%\usepackage{sfmath}
%\usepackage{bbold} %better blackboard bold

\usepackage{homework}
%\usepackage{notes}
\usepackage{newpxtext,eulerpx,eucal}
\renewcommand*{\mathbb}[1]{\varmathbb{#1}}

\fancyhf{}
\rhead{Avinash Iyer}
\lhead{Differential Topology: Problem Set 5}

\setcounter{secnumdepth}{0}

\begin{document}
\RaggedRight
\begin{problem}[Problem 1]
  Prove that smooth homotopy and smooth isotopy are equivalence relations.
\end{problem}
\begin{solution}
  If $f\colon M\rightarrow N$ is a smooth map, then we can define a smooth homotopy $F\colon M\times [0,1]\rightarrow N$ by taking $F\left( \cdot,t \right) = f$. If $f$ is a diffeomorphism, then this is a smooth isotopy. Thus, this relation is reflexive.\newline

  The relation is symmetric since, if $f$ and $g$ are smoothly homotopic (isotopic), then $F^{\ast}\colon M\times [0,1]\rightarrow N$, given by $F^{\ast}\left( \cdot,t \right) = F\left( \cdot,1-t \right)$ is a composition of smooth maps, hence smooth.\newline

  The relation is transitive since, if $F\colon M\times [0,1]\rightarrow N$ is a homotopy (isotopy) from $f$ to $g$, and $G\colon M\times [0,1]\rightarrow N$ is a homotopy (isotopy) from $g$ to $h$, then we may find a homotopy from $f$ to $h$ by taking
  \begin{align*}
    H\left( \cdot,t \right) &= \begin{cases}
      F\left( \cdot,2t \right) & 0 \leq t \leq \frac{1}{2}\\
      G\left( \cdot,2t-1 \right) & \frac{1}{2}\leq t \leq 1.
    \end{cases}
  \end{align*}
  This is a smooth map since the derivatives of all orders for $F$ and $G$ agree at $t = \frac{1}{2}$.
\end{solution}
\begin{problem}[Problem 2]
  Prove that if $M$ is connected, then for all pairs $p$ and $q$ of points on $M$, there is a diffeomorphism $f$ of $M$ such that $f(p) = q$ and $f$ is isotopic to the identity.
\end{problem}
\begin{solution}
  We know that the diffeomorphism group, $ \operatorname{diff}\left( M \right) $, is transitive whenever $M$ is connected, so there is a diffeomorphism $f\colon M\rightarrow M$ such that $f(p) = q$. Now, if $p$ and $q$ are in the same Euclidean chart, $\left( U,\varphi \right)$, where $\varphi(p) = 0$ and $\varphi(q) = ax_1$, then we may find the desired isotopy to the identity by taking
  \begin{align*}
    F\colon M\times [0,1]\rightarrow M
  \end{align*}
  to be given by
  \begin{align*}
    F\left( \cdot,t \right) &= f_{t},
  \end{align*}
  where $f_{t}$ is a diffeomorphism such that $\varphi\circ f_t(p) = atx_1$.\newline

  Now, if $p$ and $q$ are not in the same chart, then since $M$ is connected, there is a finite chain of $k$ intersecting Euclidean charts that we may compose with each other such that we get our diffeomorphism between $p$ and $q$. Dividing $ \left[ 0,1 \right] $ into intervals of length $1/k$, we may then find isotopies from the identity to the diffeomorphism mapping $p$ to the $\ell$-th intersection point along in this chain as we showed for the case where both $p$ and $q$ are in the same chart. By chaining these isotopies together, we get the isotopy between $f$ and the identity.
\end{solution}
\begin{problem}[Problem 3]
  Suppose $M$ is compact and has no boundary, and that $M$ and $N$ have the same dimension. Let $f$ and $g$ be homotopic maps from $M$ to $N$. Suppose $p\in N$ is a regular value for both $f$ and $g$. Prove that $ \left\vert f^{-1}(p) \right\vert = \left\vert g^{-1}\left( p \right) \right\vert $ modulo $2$.
\end{problem}
\begin{solution}
  Let $F\colon M\times [0,1]\rightarrow N$ be a smooth homotopy with $F\left( \cdot,0 \right) = f$ and $F\left( \cdot,1 \right) = g$. If $p\in N$ is a regular value for $F$ (in addition to one for $f$ and $g$), it follows that $F^{-1}(p)$ is a $1$-manifold subset of $M\times [0,1]$, where $F^{-1}(p)\cap \left( M\times \set{0} \right) = f^{-1}(p)\times \set{0}$, and $F^{-1}(p)\cap \left( M\times \set{1} \right) = g^{-1}(p)\times \set{1}$. Since the boundary of $M\times \left[ 0,1 \right]$ must contain an even number of points (as every $1$-submanifold with boundary of $M\times [0,1]$ must have both of its boundary points touch the boundary of $M\times \left[ 0,1 \right]$, which are $0$ and $1$), we must have $\left\vert f^{-1}(p) \right\vert + \left\vert g^{-1}(p) \right\vert \equiv 0$ modulo $2$, so that $ \left\vert f^{-1}(p) \right\vert = \left\vert g^{-1}(p) \right\vert $.\newline

  Suppose $y$ is not a regular value for $F$. Since $M\times [0,1]$ is compact, and $F$ is continuous, it follows that, by Sard's Theorem, $y$ is part of a closed, measure-zero subset of $N$. In particular, for any neighborhood of $y$, there is a regular value for $F$ within this neighborhood. Next, we observe that, for a sufficiently small open neighborhood $V$ of $y$, the number of regular points mapping to $y$ does not change, as the map $x\mapsto \left\vert F^{-1}(x) \right\vert$ is continuous and discrete-valued (for the open subset of regular values for $F$). Thus, on $V$, we may find $q\in V$ such that $\left\vert F^{-1}(q) \right\vert$ is constant, and thus $\left\vert f^{-1}(y) \right\vert + \left\vert g^{-1}(y) \right\vert$ is even, hence are equal to each other modulo $2$.
\end{solution}
\begin{problem}[Problem 4]
  Prove that for $M,N,f$ as in the previous exercise, $\left\vert f^{-1}(p) \right\vert \equiv \left\vert f^{-1}(q) \right\vert$ modulo $2$ for all regular values $p$ and $q$ of $f$, using the previous exercises.
\end{problem}
\begin{solution}
  There is a diffeomorphism $\varphi\colon N\rightarrow N$ of $N$ such that $\varphi\left( p \right) = q$ and $\varphi$ is isotopic to the identity, as shown in the solution to Problem 2. In particular, this means that $\varphi\circ f\colon M\rightarrow N$ is homotopic to $f\colon M\rightarrow N$, meaning that $\left\vert f^{-1}(p) \right\vert = \left\vert \left( \varphi\circ f \right)^{-1}(q) \right\vert = \left\vert f^{-1}(q) \right\vert$, with the latter equality following from Problem 3.
\end{solution}
\end{document}
