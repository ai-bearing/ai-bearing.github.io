
\documentclass[10pt]{mypackage}

% sans serif font:
%\usepackage{cmbright}
%\usepackage{sfmath}
%\usepackage{bbold} %better blackboard bold

\usepackage{homework}
%\usepackage{notes}
\usepackage{newpxtext,eulerpx,eucal}
\renewcommand*{\mathbb}[1]{\varmathbb{#1}}

\fancyhf{}
\fancyhead[R]{Avinash Iyer}
\fancyhead[L]{Differential Topology: Problem Set 7}
\fancyfoot[C]{\thepage}

\setcounter{secnumdepth}{0}

\begin{document}
\RaggedRight
\begin{problem}[Problem 2]
  Prove the claim from class that the open star cover of a simplicial complex is good.
\end{problem}
\begin{problem}[Problem 4]
  Compute the de Rham cohomology of $\R^{2}\setminus \set{0}$, and find representatives of all nontrivial classes.
\end{problem}
\begin{solution}
  We observe that $\R^{2}\setminus \set{0} \cong S^{1}\times \R$, so by the Poincaré lemma, we have
  \begin{align*}
    H^{\ast}_{\operatorname{DR}}\left( \R^{2}\setminus \set{0} \right) &\cong H^{\ast}_{\operatorname{DR}}\left( S^{1} \right)
    \intertext{or}
    H^{0}_{\operatorname{DR}}\left( \R^{2}\setminus\set{0} \right) &\cong \R\\
    H^{1}_{\operatorname{DR}}\left( \R^{2}\setminus\set{0} \right) &\cong \R\\
    H^{k}_{\operatorname{DR}}\left( \R^{2}\setminus\set{0} \right) &\cong 0\text{ for $k\geq 2$}.
  \end{align*}
  We know that a complete set of representatives for cohomology classes of $S^{1}$ are $1$ for $H^{0}$ and $d\theta$ for $H^{1}$. We know from the lemma that then, $d\theta$ corresponds to $\pi^{\ast}\left( d\theta \right)$, where $\pi\colon S^{1}\times \R\rightarrow S^{1}$ is the projection. Thus, we observe that $\set{1,\pi^{\ast}\left( d\theta \right)}$ is the complete set of representatives of cohomology classes for $H^{\ast}_{\operatorname{DR}}\left( \R^{2}\setminus \set{0} \right)$.
\end{solution}
\begin{problem}[Problem 6]
  Let $U$ and $V$ be open subsets of a smooth manifold $M$, and let $W = U\cup V$. Write $i_U,i_V$ for the inclusions of $U$ and $V$ into $W$ respectively, and write $j_U,j_V$ for the inclusions of $U\cap V$ into $U$ and $V$ respectively. Show that the sequence 
  \begin{center}
% https://q.uiver.app/#q=WzAsNSxbMCwwLCIwIl0sWzIsMCwiXFxtYXRoY2Fse0F9XmsoVykiXSxbNCwwLCJcXG1hdGhjYWx7QX1ee2t9KFUpXFxvcGx1c1xcbWF0aGNhbHtBfV57a30oVikiXSxbNiwwLCJcXG1hdGhjYWx7QX1ee2t9KFVcXGNhcCBWKSJdLFs4LDAsIjAiXSxbMCwxXSxbMSwyLCJcXGxlZnQoaV9VLGlfVlxccmlnaHQpIl0sWzIsMywial9VXntcXGFzdH0tal9WXntcXGFzdH0iXSxbMyw0XV0=
\[\begin{tikzcd}
	0 && {\mathcal{A}^k(W)} && {\mathcal{A}^{k}(U)\oplus\mathcal{A}^{k}(V)} && {\mathcal{A}^{k}(U\cap V)} && 0
	\arrow[from=1-1, to=1-3]
	\arrow["{\left(i_U^{\ast},i_V^{\ast}\right)}", from=1-3, to=1-5]
	\arrow["{j_U^{\ast}-j_V^{\ast}}", from=1-5, to=1-7]
	\arrow[from=1-7, to=1-9]
\end{tikzcd}\]
  \end{center}
  is exact.
\end{problem}
\begin{solution}
  Exactness at $\mathcal{A}^{k}\left( W \right)$ follows from the fact that $\left( i_U^{\ast},i_V^{\ast} \right)$ is an inclusion map, hence has kernel $0$.\newline

  To verify that the sequence is exact at $\mathcal{A}^{k}\left( U \right)\oplus \mathcal{A}^{k}\left( V \right)$, we observe that if $\omega\in \mathcal{A}^{k}\left( W \right)$, then $\left( \omega|_{U},\omega|_{V} \right)$ yields zero when subjected to $j_U^{\ast}-j_V^{\ast}$ as $\omega$ when restricted to $U\cap V$ is equal to itself. Therefore, the sequence is exact at $\mathcal{A}^{k}\left( U \right)\oplus \mathcal{A}^{k}\left( V \right)$.\newline

  Finally, we let $\set{f_U,f_V}$ be a partition of unity for $W$ subordinate to $\set{U,V}$. If $\omega\in \mathcal{A}^{k}\left( U\cap V \right)$, we observe that $f_U\omega$ extends to $0$ on $V\setminus \left( U\cap V \right)$, whence $f_U\omega\in \mathcal{A}^{k}\left( V \right)$, and similarly for $f_V\omega\in \mathcal{A}^{k}\left( U \right)$. Therefore, $\left( f_V\omega,-f_U\omega \right)\in \mathcal{A}^{k}\left( U \right)\oplus \mathcal{A}^{k}\left( V \right)$ maps to $\omega\in \mathcal{A}^{k}\left( U\cap V \right)$, meaning $j_U^{\ast}-j_V^{\ast}$ is surjective, so the sequence is exact at $\mathcal{A}^{k}\left( U\cap V \right)$.
\end{solution}
\end{document}
